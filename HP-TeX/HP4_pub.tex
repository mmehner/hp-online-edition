\documentclass[11pt,twoside]{article}
\usepackage[papersize={17cm,24cm},
	centering, textwidth=12.5cm, textheight=17.9cm,
	asymmetric]{geometry} % so that the orgvnr always appear on the right side
%\linespread{1.1}
\sloppy

\usepackage{xcolor, xparse, xspace, pifont, datetime2}
\usepackage{enumitem}

\newcommand{\HP}{\textit{Haṭha\-pra\-dī\-pikā}\xspace}

\usepackage{fancyhdr}
	\renewcommand{\headrulewidth}{0pt}
	\fancyhead[EL]{\small\texteng{\thepage}}
	\fancyhead[ER]{\small\texteng{Critical Edition \& Translation}}
	\fancyhead[OL]{\small\texteng{HP\hindsection}}
	\fancyhead[OR]{\small\texteng{\thepage}}
	\fancyhead[C]{}
	\fancyfoot[C]{}
%	\pagestyle{fancy}

\fancypagestyle{firstpage}{%
	\fancyhead[OL]{\small\texteng{(\Today)}}
	\fancyhead[OR]{\small\texteng{\thepage}}
	\fancyhead[C]{}
	}

\fancypagestyle{HPed}{%
	\renewcommand{\headrulewidth}{0pt}
	\fancyhead[EL]{\small\texteng{\thepage}}
	\fancyhead[ER]{\small\texteng{[\rightmark\textendash}}
	\fancyhead[OL]{\small\texteng{\textendash\leftmark]}}
	\fancyhead[OR]{\small\texteng{\thepage}}
	\fancyhead[C]{\small\texteng{Critical Edition}}
	\fancyfoot[C]{}
	}
\pagestyle{HPed}

\usepackage{scrextend}
	\deffootnote[]{0em}{1em}{}
%	\deffootnote[2.1em]{0em}{1.5em}{\color{blue}\texteng{\textbf{Verse \thefootnotemark}}}
\usepackage{fnpos}
	\makeFNbottom
%\footnoterulefalse
\setlength{\footnotesep}{1em}
%\dimen\footins=??
%\renewcommand{\footnotesize}{\small}  % Wirkt auf App und Fn beides.

\usepackage[english]{babel}
\usepackage{babel-iast}
\babelfont[iast]{rm}[Renderer=Harfbuzz, Scale=1.1]{AdishilaSan}
\babeltags{dev = iast}
\babeltags{eng = english}
\usepackage{libertine}

\usepackage[teiexport=tidy,poetry=verse]{ekdosis}
\usepackage{sanskrit-poetry}
\usepackage{textgreek}

%%% Gr1,4b,6
\DeclareWitness{N3}{\texteng{\textalpha\textsubscript{1}}}{NGMPP B 62-20}[]
        \DeclareHand{N3ac}{N3}{\texteng{\textalpha\rlap{\textsubscript{1}}\textsuperscript{ac}}}[]
        \DeclareHand{N3pc}{N3}{\texteng{\textalpha\rlap{\textsubscript{1}}\textsuperscript{pc}}}[]
\DeclareWitness{J5}{\texteng{\textalpha\textsubscript{2}}}{Jodhpur 02235}[]
\DeclareWitness{G4}{\texteng{\textalpha\textsubscript{3}}}{GOML 18885}[]% Telugu script
\DeclareWitness{N24}{\texteng{\textalpha\textsubscript{4}}}{NGMPP G 190-16}[]
\DeclareWitness{Gr1r}{\texteng{\textAlpha *}}{Gr1 reconstructed}[]

\DeclareWitness{P11}{\texteng{\textbeta\textsubscript{1}}}{}[]
\DeclareWitness{C6}{\texteng{\textbeta\textsubscript{2}}}{Lalchand M-2089}[]

\DeclareWitness{V3}{\texteng{\textbeta\textsubscript{\textomega}}}{Sampurnananda Library Sarasvati Bhavan 29899}[]

%%% Gr2

\DeclareWitness{N23}{\texteng{\textgamma\textsubscript{1}}}{NGMPP G 25-2}[]
        \DeclareHand{N23ac}{N23}{\texteng{\textgamma\rlap{\textsubscript{1}}\textsuperscript{ac}}}[]
        \DeclareHand{N23pc}{N23}{\texteng{\textgamma\rlap{\textsubscript{1}}\textsuperscript{pc}}}[]
\DeclareWitness{J7}{\texteng{\textgamma\textsubscript{2}}}{Jodhpur 02241}[]
%\DeclareWitness{V6}{\texteng{V\textsubscript{6}}}{Sampurnananda Library Sarasvati Bhavan 29991}[]
\DeclareWitness{K1}{\texteng{K\textsubscript{1}}}{Raghunātha Temple Library 4383}[settlement=Jammu]
        \DeclareWitness{K1ac}{\texteng{K\rlap{\textsubscript{1}}\textsuperscript{ac}\space}}{}[]
        \DeclareWitness{K1pc}{\texteng{K\rlap{\textsubscript{1}}\textsuperscript{pc}\space}}{}[]


%%% Gr3

\DeclareWitness{V19}{\texteng{\textdelta\textsubscript{1}}}{Sampurnananda Library Sarasvati Bhavan 30069}[]
\DeclareWitness{K3}{\texteng{\textdelta\textsubscript{2}}}{Privat collection}
\DeclareWitness{C7}{\texteng{\textdelta\textsubscript{3}}}{Lalchand M-6494}[]
%\DeclareWitness{C1}{\texteng{C\textsubscript{1}}}{Lalchand M-2080}[]%L1 And C1 very close (and come from same region)
%\DeclareWitness{P23}{\texteng{P\textsubscript{23}}}{}[]
%\DeclareWitness{L1}{\texteng{L\textsubscript{1}}}{SOAS RE 43454}[settlement=Jammu]

\DeclareWitness{J6}{\texteng{\textdelta\textsubscript{\textomega}}}{Jodhpur 02237}[]
        \DeclareHand{J6ac}{J6}{\texteng{\textdelta\rlap{\textomega}\textsuperscript{ac}}}[]
        \DeclareHand{J6pc}{J6}{\texteng{\textdelta\rlap{\textomega}\textsuperscript{pc}}}[]

%%% Gr4c

\DeclareWitness{P15}{\texteng{\textepsilon\textsubscript{1}}}{}[]
\DeclareWitness{N19}{\texteng{\textepsilon\textsubscript{2}}}{NGMPP E-1528-1 / E-1527-7(4)}[]
\DeclareWitness{V15}{\texteng{\textepsilon\textsubscript{3}}}{Sampurnananda Library Sarasvati Bhavan 30051}[]
        \DeclareHand{V15ac}{V15}{\texteng{\textepsilon\rlap{\textsubscript{3}}\textsuperscript{ac}}}[]
        \DeclareHand{V15pc}{V15}{\texteng{\textepsilon\rlap{\textsubscript{3}}\textsuperscript{pc}}}[]
\DeclareWitness{J11}{\texteng{\textepsilon\textsubscript{4}}}{Jodhpur 23532}[]
        \DeclareHand{J11ac}{J11}{\texteng{\textepsilon\rlap{\textsubscript{4}}\textsuperscript{i.t.}}}[]
        \DeclareHand{J11pc}{J11}{\texteng{\textepsilon\rlap{\textsubscript{4}}\textsuperscript{mg.}}}[alternative reading written by the first hand in margin or interlinearly (J11)]
%\DeclareWitness{J14}{\texteng{\textepsilon\textsubscript{5}}}{Jodhpur 02239}[]

%\DeclareWitness{L2}{\texteng{L\textsubscript{2}}}{Wellcome Collection O.36]}
\DeclareWitness{M1}{\texteng{M\textsubscript{1}}}{P-5682/4}[]

\DeclareWitness{N26}{\texteng{\textepsilon\textsubscript{\textomega}}}{NGMPP}[]
%\DeclareWitness{V17}{\texteng{\textepsilon\textsubscript{\textomega 3}}}{Sampurnananda Library Sarasvati Bhavan 30053}[]

\DeclareWitness{V1}{\texteng{\texteta\textsubscript{1}}}{Sampurnananda Library Sarasvati Bhavan 30109}[]
        \DeclareHand{V1ac}{V1}{\texteng{\texteta\rlap{\textsubscript{1}}\textsuperscript{ac}}}[]
        \DeclareHand{V1pc}{V1}{\texteng{\texteta\rlap{\textsubscript{1}}\textsuperscript{pc}}}[]

%%% Gr4d

\DeclareWitness{J10}{\texteng{\texteta\textsubscript{2}}}{MSPP Jodhpur 2230}[]
        \DeclareHand{J10ac}{J10}{\texteng{\texteta\rlap{\textsubscript{2}}\textsuperscript{ac}}}[]
        \DeclareHand{J10pc}{J10}{\texteng{\texteta\rlap{\textsubscript{2}}\textsuperscript{pc}}}[]

\DeclareWitness{N9}{\texteng{\texteta\textsubscript{\textomega}}}{NGMPP A62-33}[]
%\DeclareWitness{J15}{\texteng{\textepsilon\textsubscript{\textomega 4}}}{Jodhpur 9732A}[]

%%%

\DeclareWitness{Jyo}{\texteng{\textchi}}{Brahmānanda's version}[]
%\DeclareWitness{Tue}{\texteng{Tü}}{Ma I 339}[]

\DeclareWitness{ceteri}{\texteng{cett.}}{ceteri}[]

%%% Group Sigla

\DeclareWitness{Gr1}{\texteng{\textAlpha}}{N3,J5,G4}

\DeclareWitness{Gr2}{\texteng{\textGamma}}{N23,J7}
%\DeclareWitness{Gr2}{\texteng{%
%	\textbeta\textsubscript{1}%
%	\textbeta\textsubscript{2}%
%	}}{N23,J7}
\DeclareWitness{Gr3a}{\texteng{\textDelta}}{V19,K3,C7}
\DeclareWitness{Gr4b}{\texteng{%
	\textbeta\textsubscript{1}%
	\textbeta\textsubscript{2}%
	}}{C6,P11}
\DeclareWitness{GrB}{\texteng{%
	\textbeta\textsubscript{1}%
	\textbeta\textsubscript{2}%
	\textbeta\textsubscript{\textomega}%
	}}{C6,P11,V3}
\DeclareWitness{Gr4c}{\texteng{\textEpsilon}}{P15,N19,V15}

% \DeclareWitness{Gr4d}{\texteng{%
	% \texteta\textsubscript{1}%
	% \texteta\textsubscript{2}%
	% }}{V1,J10}
\DeclareWitness{Gr6}{\texteng{\textOmega}}{V3,J6,N9,N26}




%%%%%%%%%%%%%%%%%%%% THE  MSS         %%%%%%%%%%%%%%%%%%%%%%%%%%%

%%% Versions
\DeclareWitness{Vu}{\selectlanguage{english}Vulg}{Vulgate, i.e. Brahmānanda's version}[]           
\DeclareWitness{X}{\selectlanguage{english}X}{TenChapter Version, Jodhpur 02228 and 02225 (ed. Lonavla)}[]
\DeclareWitness{Six}{\selectlanguage{english}Ṣ}{SixChapterVersion, ``6ChapterHPms'', fragment of enlarged text, Jodhpur}[]
% Mss. in Geographical Groups
%%%% Varanasi mss (Sampūrṇānanda mss). V1 is Important
\DeclareWitness{V1}{\selectlanguage{english}V\textsubscript{1}}{Sampurnananda Library Sarasvati Bhavan 30109}[]
        \DeclareHand{V1ac}{V1}{\selectlanguage{english}V\rlap{\textsubscript{1}}\textsuperscript{ac}}[] % added by MD
        \DeclareHand{V1pc}{V1}{\selectlanguage{english}V\rlap{\textsubscript{1}}\textsuperscript{pc}}[] % added by MD
\DeclareWitness{V2}{\selectlanguage{english}V\textsubscript{2}}{Sampurnananda Library Sarasvati Bhavan 29869}[]
\DeclareWitness{V3}{\selectlanguage{english}V\textsubscript{3}}{Sampurnananda Library Sarasvati Bhavan 29899}[]
\DeclareWitness{V4}{\selectlanguage{english}V\textsubscript{4}}{Sampurnananda Library Sarasvati Bhavan 29937}[]
\DeclareWitness{V5}{\selectlanguage{english}V\textsubscript{5}}{Sampurnananda Library Sarasvati Bhavan 29938}[]
\DeclareWitness{V6}{\selectlanguage{english}V\textsubscript{6}}{Sampurnananda Library Sarasvati Bhavan 29991}[]
\DeclareWitness{V8}{\selectlanguage{english}V\textsubscript{8}}{Sampurnananda Library Sarasvati Bhavan 30014}[]
\DeclareWitness{V11}{\selectlanguage{english}V\textsubscript{11}}{Sampurnananda Library Sarasvati Bhavan 30029}[]
\DeclareWitness{V12}{\selectlanguage{english}V\textsubscript{12}}{Sampurnananda Library Sarasvati Bhavan 30030}[]
\DeclareWitness{V13}{\selectlanguage{english}V\textsubscript{13}}{Sampurnananda Library Sarasvati Bhavan 30031}[]
\DeclareWitness{V14}{\selectlanguage{english}V\textsubscript{14}}{Sampurnananda Library Sarasvati Bhavan 30050}[]
\DeclareWitness{V15}{\selectlanguage{english}V\textsubscript{15}}{Sampurnananda Library Sarasvati Bhavan 30051}[]
\DeclareWitness{V15pc}{\selectlanguage{english}V\rlap{\textsubscript{15}}\textsuperscript{pc}\space}{}[]
\DeclareWitness{V16}{\selectlanguage{english}V\textsubscript{16}}{Sampurnananda Library Sarasvati Bhavan 30052}[]
\DeclareWitness{V17}{\selectlanguage{english}V\textsubscript{17}}{Sampurnananda Library Sarasvati Bhavan 30053}[] % added by MD
\DeclareWitness{V16pc}{\selectlanguage{english}V\rlap{\textsubscript{16}}\textsuperscript{pc}\space}{}[]
\DeclareWitness{V18}{\selectlanguage{english}V\textsubscript{18}}{Sampurnananda Library Sarasvati Bhavan 30064}[]
\DeclareWitness{V19}{\selectlanguage{english}V\textsubscript{19}}{Sampurnananda Library Sarasvati Bhavan 30069}[]
\DeclareWitness{V21}{\selectlanguage{english}V\textsubscript{21}}{Sampurnananda Library Sarasvati Bhavan 30104}[]
\DeclareWitness{V22}{\selectlanguage{english}V\textsubscript{22}}{Sampurnananda Library Sarasvati Bhavan 30110}[]
\DeclareWitness{V25}{\selectlanguage{english}V\textsubscript{25}}{Sampurnananda Library Sarasvati Bhavan 30122}[]
\DeclareWitness{V26}{\selectlanguage{english}V\textsubscript{26}}{Sampurnananda Library Sarasvati Bhavan 30123}[]
\DeclareWitness{V28}{\selectlanguage{english}V\textsubscript{28}}{Sampurnananda Library Sarasvati Bhavan 30136}[]
\DeclareWitness{W2}{\selectlanguage{english}W\textsubscript{2}}{Wai ??}[]
\DeclareWitness{W4}{\selectlanguage{english}W\textsubscript{4}}{Wai 399-6171}[]

%%%%%%%%%%%%%%%%%%%%%%%%%%%%%%%%%
%%% Jammu & Kaschmir
\DeclareWitness{K1}{\selectlanguage{english}K\textsubscript{1}}{Raghunātha Temple Library 4383}[settlement=Jammu]
        \DeclareWitness{K1ac}{\selectlanguage{english}K\rlap{\textsubscript{1}}\textsuperscript{ac}\space}{}[]
        \DeclareWitness{K1pc}{\selectlanguage{english}K\rlap{\textsubscript{1}}\textsuperscript{pc}\space}{}[]
\DeclareWitness{K3}{\selectlanguage{english}K\textsubscript{3}}{Privat collection}
\DeclareWitness{L1}{\selectlanguage{english}L\textsubscript{1}}{SOAS RE 43454}[settlement=Jammu]
% More details? Catalogue number? L1 And C1 very close (and come from same region)
%%%%%%%%%%%%%%%%%%%%%%%%%%%%%%%%
% Jodhpur
% J10 is important
\DeclareWitness{J10}{\selectlanguage{english}J\textsubscript{10}}{MSPP Jodhpur 2230}[]
        \DeclareHand{J10ac}{J10}{\selectlanguage{english}J\rlap{\textsubscript{10}}\textsuperscript{ac}}[] % modified by MD
        \DeclareHand{J10pc}{J10}{\selectlanguage{english}J\rlap{\textsubscript{10}}\textsuperscript{pc}}[] % modified by MD
\DeclareWitness{J1}{\selectlanguage{english}J\textsubscript{1}}{Jodhpur 02231}[]
\DeclareWitness{J2}{\selectlanguage{english}J\textsubscript{2}}{Jodhpur 02232}[]   
\DeclareWitness{J3}{\selectlanguage{english}J\textsubscript{3}}{Jodhpur 02233}[]
\DeclareWitness{J4}{\selectlanguage{english}J\textsubscript{4}}{Jodhpur 02234}[]
        \DeclareWitness{J4ac}{\selectlanguage{english}J\rlap{\textsubscript{4}}\textsuperscript{ac}\space}{MSPP Jodhpur 02234}[]
        \DeclareWitness{J4pc}{\selectlanguage{english}J\rlap{\textsubscript{4}}\textsuperscript{pc}\space}{MSPP Jodhpur 02234}[]
\DeclareWitness{J5}{\selectlanguage{english}J\textsubscript{5}}{Jodhpur 02235}[]  % 4 chapters, 34 jpgs,   long colophon, missing lines in the beginning.
\DeclareWitness{J6}{\selectlanguage{english}J\textsubscript{6}}{Jodhpur 02237}[]  % 4 chapters, 41 jpgs
%\DeclareWitness{J6ac}{\selectlanguage{english}J\rlap{\textsubscript{6}}\textsubscript{ac}}{Jodhpur 02237}[]  % 4 chapters, 49 jpgs,   1st folio: idaṃ gulābarāyasya
% tulasīrāmaśarmmaṇaḥ putrasya pustakaṃ ...        End: iti śrīsahajānandasantānacintāmaṇisvātmārāmaviracitāyāṃ ..
% saṃvat 1802   (more consistent text)
%\DeclareWitness{J6pc}{\selectlanguage{english}J\rlap{\textsubscript{6}}\textsubscript{pc}}{Jodhpur 02237}[] 
\DeclareWitness{J7}{\selectlanguage{english}J\textsubscript{7}}{Jodhpur 02241}[]  % 4 chapters, 41 jpgs
\DeclareWitness{J8}{\selectlanguage{english}J\textsubscript{8}}{Jodhpur 23709}[]  % 4 chapters,  87 jpgs.   saṃvat 1724
\DeclareHand{J8ac}{J8}{\selectlanguage{english}J\rlap{\textsubscript{8}}\textsuperscript{ac}}[]  % changed by MD
\DeclareHand{J8pc}{J8}{\selectlanguage{english}J\rlap{\textsubscript{8}}\textsuperscript{pc}}[]  % changed by MD
\DeclareWitness{J9}{\selectlanguage{english}J\textsubscript{9}}{Jodhpur 02224}[]  %  fragment, 20 jpgs.
\DeclareWitness{J11}{\selectlanguage{english}J\textsubscript{11}}{Jodhpur 23532}[]
        \DeclareHand{J11ac}{J11}{\selectlanguage{english}J\rlap{\textsubscript{11}}\textsuperscript{ac}}[] % added by MD
        \DeclareHand{J11pc}{J11}{\selectlanguage{english}J\rlap{\textsubscript{11}}\textsuperscript{pc}}[] % added by MD
\DeclareWitness{J12}{\selectlanguage{english}J\textsubscript{12}}{Jodhpur 18552}[] 
\DeclareWitness{J13}{\selectlanguage{english}J\textsubscript{13}}{Jodhpur 02229}[]  %  5 chapters, 93 jpgs.
\DeclareWitness{J14}{\selectlanguage{english}J\textsubscript{14}}{Jodhpur 02239}[]  %  4 chapters
\DeclareWitness{J15}{\selectlanguage{english}J\textsubscript{15}}{Jodhpur 9732A}[]
\DeclareWitness{J16}{\selectlanguage{english}J\textsubscript{16}}{Jodhpur 9732B}[]
\DeclareWitness{J17}{\selectlanguage{english}J\textsubscript{17}}{Jodhpur 3013}[]
% Haṭhapradīpikā with (non-Sanskrit) Bhāṣya RORI Jodhpur ACC.NO.18552
%  Haṭhapradīpikā with (non-Sanskrit) commentary, RORI Alwar 952, 4 chapters,  colophon of the comm:
% iti śrīlāhorīmiśravrajabhūṣanaviracitāyāṃ bhāvārthadīpikāyāṃ caturthodhyāya ..    
%  Haṭhapradīpikā (5 chapter) MSPP Jodhpur ACC.NO.02229/

%%%%%%%%%%        Bodleian, Oxford
\DeclareWitness{B1}{\selectlanguage{english}B\textsubscript{1}}{Bodleian Library No. d.457(8)}[settlement=Oxford]
\DeclareWitness{B2}{\selectlanguage{english}B\textsubscript{2}}{Bodleian Library No. d.458(1)}[settlement=Oxford]
\DeclareWitness{B3}{\selectlanguage{english}B\textsubscript{3}}{Bodleian Library No. d.458(9)}[settlement=Oxford]

%%%%%%%%%%%   Chandigarh
\DeclareWitness{C1}{\selectlanguage{english}C\textsubscript{1}}{Lalchand M-2080}[]%L1 And C1 very close (and come from same region)
\DeclareWitness{C2}{\selectlanguage{english}C\textsubscript{2}}{Lalchand M-6065}[]
\DeclareWitness{C3}{\selectlanguage{english}C\textsubscript{3}}{Lalchand M-1293}[]
\DeclareWitness{C4}{\selectlanguage{english}C\textsubscript{4}}{Lalchand M-2081}[]
\DeclareWitness{C4ac}{\selectlanguage{english}C\rlap{\textsubscript{4}}\textsuperscript{ac}\space}{}[]
\DeclareWitness{C4pc}{\selectlanguage{english}C\rlap{\textsubscript{4}}\textsuperscript{pc}\space}{}[]
\DeclareWitness{C5}{\selectlanguage{english}C\textsubscript{5}}{Lalchand M-2082}[]%doesn't have chapter 1
\DeclareWitness{C6}{\selectlanguage{english}C\textsubscript{6}}{Lalchand M-2089}[]
\DeclareWitness{C7}{\selectlanguage{english}C\textsubscript{7}}{Lalchand M-6494}[]
\DeclareWitness{C8}{\selectlanguage{english}C\textsubscript{8}}{Lalchand M-2091}[]
        \DeclareHand{C8ac}{C8}{\selectlanguage{english}C\rlap{\textsubscript{8}}\textsuperscript{ac}}[]
        \DeclareHand{C8pc}{C8}{\selectlanguage{english}C\rlap{\textsubscript{8}}\textsuperscript{pc}}[]
\DeclareWitness{C9}{\selectlanguage{english}C\textsubscript{9}}{Lalchand M-4530}[]


% %%%%%%%%%%        Nepalese
\DeclareWitness{N1}{\selectlanguage{english}N\textsubscript{1}}{NGMPP A1400-2}[]
\DeclareWitness{N2}{\selectlanguage{english}N\textsubscript{2}}{NGMPP B 39-19}[]
\DeclareWitness{N3}{\selectlanguage{english}N\textsubscript{3}}{NGMPP B 62-20}[]
\DeclareWitness{N5}{\selectlanguage{english}N\textsubscript{5}}{NGMPP A60-15 + A61-1}[]
\DeclareWitness{N4}{\selectlanguage{english}N\textsubscript{4}}{NGMPP A61-2}[]
\DeclareWitness{N6}{\selectlanguage{english}N\textsubscript{6}}{NGMPP A61-6}[]
\DeclareWitness{N9}{\selectlanguage{english}N\textsubscript{9}}{NGMPP A62-33}[]
\DeclareWitness{N10}{\selectlanguage{english}N\textsubscript{10}}{NGMPP A62-37}[]
\DeclareWitness{N11}{\selectlanguage{english}N\textsubscript{11}}{NGMPP A63-15}[]
\DeclareWitness{N12}{\selectlanguage{english}N\textsubscript{12}}{NGMPP A939-19}[]
\DeclareWitness{N13}{\selectlanguage{english}N\textsubscript{13}}{NGMPP A1378-18}[]
\DeclareWitness{N16}{\selectlanguage{english}N\textsubscript{16}}{NGMPP B39-20}[]
\DeclareWitness{N17}{\selectlanguage{english}N\textsubscript{17}}{NGMPP B 111-10}[]
\DeclareWitness{N18}{\selectlanguage{english}N\textsubscript{18}}{NGMPP E 929-3}[]
\DeclareWitness{N19}{\selectlanguage{english}N\textsubscript{19}}{NGMPP E-1528-1 / E-1527-7(4)}[]
\DeclareWitness{N20}{\selectlanguage{english}N\textsubscript{20}}{NGMPP E 2037-13 }[]
\DeclareWitness{N21}{\selectlanguage{english}N\textsubscript{21}}{NGMPP E 2097-31}[]
\DeclareWitness{N22}{\selectlanguage{english}N\textsubscript{22}}{NGMPP G 4-4}[]
\DeclareWitness{N23}{\selectlanguage{english}N\textsubscript{23}}{NGMPP G 25-2}[]
        \DeclareHand{N23ac}{N23}{\selectlanguage{english}N\rlap{\textsubscript{23}}\textsuperscript{ac}}[] % added by MD
        \DeclareHand{N23pc}{N23}{\selectlanguage{english}N\rlap{\textsubscript{23}}\textsuperscript{pc}}[] % added by MD
\DeclareWitness{N24}{\selectlanguage{english}N\textsubscript{24}}{NGMPP G 190-16}[]
\DeclareWitness{N24ac}{\selectlanguage{english}N\rlap{\textsubscript{24}}\textsuperscript{ac}\space}{}[]
\DeclareWitness{N24pc}{\selectlanguage{english}N\rlap{\textsubscript{24}}\textsuperscript{pc}\space}{}[]
\DeclareWitness{N26}{\selectlanguage{english}N\textsubscript{26}}{NGMPP T 24-3}[]

% %%%%%%%%%%        Pune

\DeclareWitness{P1}{\selectlanguage{english}P\textsubscript{1}}{Ānandāśrama S16-3-21}[]
\DeclareWitness{P2}{\selectlanguage{english}P\textsubscript{2}}{Ānandāśrama S16-2-20}[]
\DeclareWitness{P3}{\selectlanguage{english}P\textsubscript{3}}{BISM (79) 314}[]
\DeclareWitness{P4}{\selectlanguage{english}P\textsubscript{4}}{BISM (91) 191}[]
\DeclareWitness{P5}{\selectlanguage{english}P\textsubscript{5}}{BISM (29) 5790}[]
\DeclareWitness{P6}{\selectlanguage{english}P\textsubscript{6}}{BORI 263/1879-80}[]
\DeclareWitness{P7}{\selectlanguage{english}P\textsubscript{7}}{BORI 665/1883-84}[]
\DeclareWitness{P8}{\selectlanguage{english}P\textsubscript{8}}{BORI 316/1895-98}[]
\DeclareWitness{P9}{\selectlanguage{english}P\textsubscript{9}}{BORI 733-1891-95}[]
\DeclareWitness{P10}{\selectlanguage{english}P\textsubscript{10}}{BORI 222-1884-86}[]
\DeclareWitness{P11}{\selectlanguage{english}P\textsubscript{11}}{BORI 221-1882–83}[]
\DeclareWitness{P12}{\selectlanguage{english}P\textsubscript{12}}{Ānandāśrama S16-3-24}[]
\DeclareWitness{P13}{\selectlanguage{english}P\textsubscript{13}}{Ānandāśrama S16-2-22}[]
\DeclareWitness{P14}{\selectlanguage{english}P\textsubscript{14}}{Ānandāśrama S16-3-23}[]
\DeclareWitness{P15}{\selectlanguage{english}P\textsubscript{15}}{BISM (64) 919}[]
\DeclareWitness{P16}{\selectlanguage{english}P\textsubscript{16}}{BISM (64) 1115}[]
\DeclareWitness{P17}{\selectlanguage{english}P\textsubscript{17}}{BISM 620/1886-92}[]
\DeclareWitness{P18}{\selectlanguage{english}P\textsubscript{18}}{BORI 615/1887-91}[]
\DeclareWitness{P19}{\selectlanguage{english}P\textsubscript{19}}{BISM 46-39}[]
\DeclareWitness{P20}{\selectlanguage{english}P\textsubscript{20}}{BISM 39-273}[]
\DeclareWitness{P21}{\selectlanguage{english}P\textsubscript{21}}{BISM 37-743}[]
\DeclareWitness{P22}{\selectlanguage{english}P\textsubscript{22}}{BISM 37-729}[]
\DeclareWitness{P23}{\selectlanguage{english}P\textsubscript{23}}{BISM 33-60}[]
\DeclareWitness{P24}{\selectlanguage{english}P\textsubscript{24}}{BISM 29-5790}[]% =P5!
\DeclareWitness{P25}{\selectlanguage{english}P\textsubscript{25}}{BISM 29-3657}[]
\DeclareWitness{P26}{\selectlanguage{english}P\textsubscript{26}}{BISM 25-281}[]
\DeclareWitness{P27}{\selectlanguage{english}P\textsubscript{27}}{BISM 7-489}[]
\DeclareWitness{P28}{\selectlanguage{english}P\textsubscript{28}}{BORI 399-1895-1902}[]

%%%%%   Mysore
\DeclareWitness{M1}{\selectlanguage{english}M\textsubscript{1}}{P-5682/4}[]
%%%%%   Tübingen
\DeclareWitness{Tue}{\selectlanguage{english}Tü}{Ma I 339}[]
%%%%%%%%%%
\DeclareWitness{YC}{\selectlanguage{english}YC}{Yogacintāmaṇi}[]
\DeclareWitness{ceteri}{\selectlanguage{english}cett.}{ceteri}[]

%%%%%%%%%% Mss with Commentary
\DeclareWitness{A1}{\selectlanguage{english}A\textsubscript{1}}{Alwar 952}[]

\DeclareWitness{Jyo}{\selectlanguage{english}J\textsubscript{yo}}{Brahmānanda's version}[]

%%%%%%%%%%%%%%%%%%%%%%%%%%%%%%%%%%%%%%%%%%%
%List of all Sigla:
%A1,B1,B2,B3,C1,C2,C3,C4,C6,C7,C8,C9,J1,J2,J3,J4,J10,J13,J14,J15,J17,L1,M1,N3,N5,N6,N9,N10,N11,N12,N13,N16,N17,N19,N20,N21,N22,N23,N24,Tü,V1,V2,V3,V4,V5,V6,V8,V11,V19,V22,V26,Vu
%%%%%%%%%%%%%%%%%%%%%%%%%%%%%%%%%%%%%%%%%%%

\DeclareWitness{G4}{\selectlanguage{english}G\textsubscript{4}}{GOML D18885 (Bundle SD5051)}[]
\DeclareWitness{G5}{\selectlanguage{english}G\textsubscript{5}}{GOML R3841/ SR2190}[]
\DeclareWitness{G7}{\selectlanguage{english}G\textsubscript{7}}{GOML D4394}[]

\DeclareWitness{Ko}{\selectlanguage{english}K\textsubscript{o}}{Koba, Gujarat 55626}[]

% addition 2023-12-11 MD:
\TeXtoTEIPat{\begin {metre}[#1]}{<note type="metre" target="##1">}
\TeXtoTEIPat{\end {metre}}{</note>}
\TeXtoTEIPat{\texttheta}{θ}

% change 2023-12-05 mm
\TeXtoTEI{teimute}{} 

% changes/additions 2023-11-27 MM:
\TeXtoTEIPat{\medialink {#1}{#2}}{<ref target="resources/#2">#1</ref>}

% changes/additions 2023-10-25 MM:
% new Sigla
\TeXtoTEIPat{\textAlpha}{Α}
\TeXtoTEIPat{\textalpha}{α}
\TeXtoTEIPat{\textBeta}{Β}
\TeXtoTEIPat{\textbeta}{β}
\TeXtoTEIPat{\textGamma}{Γ}
\TeXtoTEIPat{\textgamma}{γ}
\TeXtoTEIPat{\textDelta}{Δ}
\TeXtoTEIPat{\textdelta}{δ}
\TeXtoTEIPat{\textEpsilon}{Ε}
\TeXtoTEIPat{\textepsilon}{ε}
\TeXtoTEIPat{\textEta}{Η}
\TeXtoTEIPat{\texteta}{η}
\TeXtoTEIPat{\textChi}{Χ}
\TeXtoTEIPat{\textchi}{χ}
\TeXtoTEIPat{\textOmega}{Ω}
\TeXtoTEIPat{\textomega}{ω}

%new environments
\TeXtoTEIPat{\begin {postmula}[#1]}{<note type="postmula" target="##1">}
  \TeXtoTEIPat{\end {postmula}}{</note>}
\TeXtoTEIPat{\begin {altava}[#1]}{<div type="altrec"><note type="avataranika" target="##1">} %%% changed 2023-12-05 mm
  \TeXtoTEIPat{\end {altava}}{</note></div>} %%% changed 2023-12-05 mm
\TeXtoTEIPat{\sgwit {#1}}{<note type="inlineref"><ref>#1</ref></note>}

% changes/additions 2023-10-12 MM:
\TeXtoTEIPat{\\.}{}

% changes/additions 2023-08-15 MD:
\TeXtoTEIPat{\lineom {#1}{#2}}{<note type="omission">#1 omitted in <ref>#2</ref></note>}
\TeXtoTEI{graus}{hi}[rend="grey"]
\TeXtoTEIPat{\startgray}{} %%% changed 2023-12-05 mm
\TeXtoTEIPat{\endgray}{} %%% changed 2023-12-05 mm



% additions/changes 2023-06-05 mm:
%\TeXtoTEIPat{\lineom {#1}}{<note type="omission">Line omitted in <ref>#1</ref></note>}
\TeXtoTEIPat{\NotIn {#1}}{<note type="omission">Stanza omitted in <ref>#1</ref></note>}

% additions 2023-04-16 MD:
\TeXtoTEIPat{\,}{ }

% additions 2023-04-13 mm:
\TeXtoTEIPat{\begin {versinnote}}{<lg>}
  \TeXtoTEIPat{\end {versinnote}}{</lg>}

% additions 2023-04-05 MD:
\TeXtoTEIPat{\begin {testimonia}[#1]}{<note type="testimonia" target="##1">}
  \TeXtoTEIPat{\end {testimonia}}{</note>}
\TeXtoTEI{devnote}{s}[xml:lang="sa-deva"]

% app in philcomm und testimonia %%% added MM 2023-12-02
\TeXtoTEI{var}{note}[type="appinnote"]


\TeXtoTEI{anm}{note}[type="memo"] %% change 2023-04-16 MD
\TeXtoTEI{Anm}{note}[type="memo"] %% change 2023-12-05 MM
\TeXtoTEIPat{\startverse}{} %%% marked for change 2023-04-13 mm
\TeXtoTEIPat{\endverse}{} %%% marked for change 2023-04-13 mm
\TeXtoTEIPat{\newpage}{}
\TeXtoTEIPat{\marma}{}
\TeXtoTEIPat{\marmas}{}
\TeXtoTEIPat{\vin}{} % added by MD 2023-11-14

%%% modify environments and commands
%%% TEI mapping
% additions/changes 2022-06-07 mm:
\TeXtoTEI{grau}{hi}[rend="grey"]
\TeXtoTEIPat{ \& }{ &amp; }

% additions/changes 2022-06-01 mm:
\TeXtoTEI{skp}{seg}[type="deva-ignore"]
\TeXtoTEI{skm}{seg}[type="ltn-ignore"]

\TeXtoTEIPat{\rlap {#1}}{#1}

% additions/changes 2022-04-06 mm:
%\TeXtoTEI{sgwit}{ref}
\TeXtoTEI{textdev}{s}[xml:lang="sa-deva"]
\TeXtoTEIPat{\begin {col}[#1]}{<div type="colophon" xml:id="#1"><p>}
  \TeXtoTEIPat{\end {col}}{</p></div>}
\TeXtoTEIPat{\begin {ava}[#1]}{<note type="avataranika" target="##1">}
  \TeXtoTEIPat{\end {ava}}{</note>}
												   
\TeXtoTEIPat{\outdent}{}
\TeXtoTEIPat{\startaltrecension}{} %%% changed 2023-12-05 mm
\TeXtoTEIPat{\endaltrecension}{} %%% changed 2023-12-05 mm
\TeXtoTEIPat{\startaltnormal}{} % added by MD 2023-11-14 %%% changed 2023-12-05 mm
\TeXtoTEIPat{\endaltnormal}{} % added by MD 2023-11-14 %%% changed 2023-12-05 mm
\TeXtoTEIPat{\begin {alttlg}[#1]}{<div type="altrec"><lg xml:id="#1">}
  \TeXtoTEIPat{\end {alttlg}}{</lg></div>}



% additions/changes 2022-03-12 mm:
\TeXtoTEIPat{\begin {tlg}[#1]}{<lg xml:id="#1">}
  \TeXtoTEIPat{\end {tlg}}{</lg>}

\TeXtoTEIPat{\begin {translation}[#1]}{<note type="translation" target="##1">}
  \TeXtoTEIPat{\end {translation}}{</note>}
\TeXtoTEIPat{\begin {philcomm}[#1]}{<note type="philcomm" target="##1">}
  \TeXtoTEIPat{\end {philcomm}}{</note>}
\TeXtoTEIPat{\begin {sources}[#1]}{<note type="sources" target="##1">}
  \TeXtoTEIPat{\end {sources}}{</note>}


\TeXtoTEIPat{\begin {marma}[#1]}{<note type="marma" target="##1">}
  \TeXtoTEIPat{\end {marma}}{</note>}

\TeXtoTEIPat{\begin {jyotsna}[#1]}{<note type="jyotsna" target="##1">}
  \TeXtoTEIPat{\end {jyotsna}}{</note>}

\EnvtoTEI{description}{list}
\EnvtoTEI{itemize}{list}
\TeXtoTEIPat{\item [#1]}{<label>#1</label>\item}

\TeXtoTEI{tl}{l}
\TeXtoTEI{myfn}{note}[type="myfn"]
\TeXtoTEIPat{\getsiglum {#1}}{<ref target="##1"/>}

\TeXtoTEI{SetLineation}{}
\TeXtoTEI{noindent}{}
\TeXtoTEI{subsection*}{}

\TeXtoTEI{rlap}{}

% end additions/changes
% \TeXtoTEIPat{\skp {#1}}{#1}
% \TeXtoTEIPat{\skm {#1}}{}

\TeXtoTEIPat{\begin {prose}}{<p>}
  \TeXtoTEIPat{\end {prose}}{</p>}

\TeXtoTEIPat{\begin {tlate}}{<p>}
  \TeXtoTEIPat{\end {tlate}}{</p>}

\TeXtoTEI{emph}{hi}
\TeXtoTEI{bigskip}{}
% \TeXtoTEI{/}{|}
\TeXtoTEI{tl}{l}
\TeXtoTEIPat{english}{}
%\TeXtoTEIPat{-}{ } %% change 2023-04-16 MD
%\TeXtoTEIPat{°}{} %% change 2023-04-16 MD
\TeXtoTEIPat{\textcolor {#1}{#2}}{<hi rend="#1">#2</hi>}

% \TeXtoTEIPat{\eyeskip}{}
% \TeXtoTEIPat{\aberratio}{}
% \TeXtoTEIPat{\ad}{}
\TeXtoTEIPat{\add}{<hi rend="italic">add.</hi>} %% change 2023-04-16 MD
% \TeXtoTEIPat{\ann}{}
\TeXtoTEIPat{\ante}{<hi rend="italic">ante</hi> } %% change 2023-04-16 MD
\TeXtoTEIPat{\post}{<hi rend="italic">post</hi> } %% change 2023-04-16 MD
% \TeXtoTEIPat{\codd}{}
% \TeXtoTEIPat{\conj }{}
% \TeXtoTEIPat{\contin}{}
% \TeXtoTEIPat{\corr}{}
% \TeXtoTEIPat{\del}{}
% \TeXtoTEIPat{\dub}{}
% \TeXtoTEIPat{\emend }{}
% \TeXtoTEIPat{\expl}{}
% \TeXtoTEIPat{\ȩxplicat}{}
% \TeXtoTEIPat{\fol}{}
% \TeXtoTEIPat{\gloss}{}
% \TeXtoTEIPat{\ins}{}
% \TeXtoTEIPat{\im}{}
% \TeXtoTEIPat{\inmargine}{}
% \TeXtoTEIPat{\intextu}{}
% \TeXtoTEIPat{\indist}{}
% \TeXtoTEIPat{\iteravit}{}
% \TeXtoTEIPat{\lectio}{}
% \TeXtoTEIPat{\leginequit}{}
% \TeXtoTEIPat{\legn}{}
% \TeXtoTEIPat{\illeg}{<hi rend="italic">illeg.</hi>}
\TeXtoTEIPat{\illeg}{<gap reason="illeg."/>} %%% change 2023-04-11 mm
% \TeXtoTEIPat{\om}{<hi rend="italic">om.</hi>}
\TeXtoTEIPat{\om}{<gap reason="om."/>} %%% change 2023-04-11 mm
% \TeXtoTEIPat{\primman}{}
% \TeXtoTEIPat{\prob}{}
% \TeXtoTEIPat{\rep}{}
% \TeXtoTEIPat{\sequentia}{}
% \TeXtoTEIPat{\supralineam}{}
% \TeXtoTEIPat{\interlineam}{}
\TeXtoTEIPat{\vl}{<hi rend="italic">v.l.</hi>}
% \TeXtoTEIPat{\vide}{}
% \TeXtoTEIPat{\videtur}{}
% \TeXtoTEIPat{\crux}{}
% \TeXtoTEIPat{\cruxxx}{}
\TeXtoTEIPat{\unm}{<hi rend="italic">unm.</hi>}


% List of Scholars
\DeclareScholar{nos}{nos}[
forename=HPP,
surname=Team]


% Nullify \selectlanguage in TEI as it has been used in
% \DeclareWitness but should be ignored in TEI.
\TeXtoTEI{selectlanguage}{}



\setlength\parindent{1em}
\SetLineation{lineation=none}
\poemlines{0}

\SetHooks{
	lemmastyle=\bfseries,
	refnumstyle=\selectlanguage{english}\color{blue}\bfseries, 
	appfontsize=\footnotesize
	}
\DeclareApparatus{default}[
	lang=english,
	sep = {] },
	delim=\hskip 0.75em,
	%	rule=none,
	]
\DeclareApparatus{anmkg}[
	notelang=english,
	sep = { },
	delim=\texteng{\ \textbullet\ \ },
%	rule=\relax
	rule=\rule{0.15\columnwidth}{0.4pt}
	]

\newcommand{\mydelim}{\xspace\textcolor{violet}{\textbullet}\ \ }
\newcommand{\mylem}[1]{\texteng{\textcolor{violet}{#1}}}
\setlength{\vrightskip}{-15pt}
\setlength{\vgap}{-3em} % default 1.5em
\verselinenumfont{\footnotesize\selectlanguage{english}\normalfont}

\newlength{\myoutdent}\setlength{\myoutdent}{2em}

\DeclareShorthand{emend}{\texteng{\emph{em.}}}{ego}
%\DeclareShorthand{conj}{\texteng{\emph{conj.}}}{ego}

%Define two commands: \skp ("sanskrit plus"), to be ignored by TeX in
%the edition text, but processed in the TEI output. Conversely, \skm
%("sanskrit minus") is to be processed in the edition text, but
%ignored if found in the apparatus criticus and in the TEI output:

\newif\ifinapparatus
\NewDocumentCommand{\skp}{m}{}
\NewDocumentCommand{\skm}{m}{\unless\ifinapparatus#1\fi}

\SetTEIxmlExport{autopar=false}

\newcommand{\versenr}{\ \themyvnum//}

\NewDocumentEnvironment{tlg}{O{}}{
	\def\hpvnum{\texteng{\thepoemline}}
	\markboth{\hpvnum}{\hpvnum}
	\setcounter{myvnum}{\value{poemline}}
	\begin{ekdverse}
	\Large}{\normalsize
	\end{ekdverse}
	%\smallskip
%  \stepcounter{myvnum}
}

\NewDocumentEnvironment{alttlg}{O{}}{
	\setvnum{\hindsection.\arabic{saved@poemline}*\arabic{poemline}}
	\def\hpvnum{\texteng{\hindsection.\arabic{saved@poemline}*\arabic{poemline}}}
	\markboth{\hpvnum}{\hpvnum}
	\setcounter{altvnum}{\value{poemline}}
	\begin{ekdverse}[type=altrecension]
	\color{gray}
	\Large}{\normalsize
	\end{ekdverse}
	%\smallskip
}

\NewDocumentCommand{\tl}{m}{#1}

\NewDocumentEnvironment{ava}{O{}}{
	\setvnum{prescript:}
	\begin{ekdverse}
	\hspace{-\myoutdent}
	\Large}{\normalsize
	\end{ekdverse}
	\smallskip
}

\NewDocumentEnvironment{altava}{O{}}{
	\setvnum{prescript:}
	\begin{ekdverse}[type=altrecension]
	\color{gray}
	\hspace{-\myoutdent}
	\Large}{\normalsize
	\end{ekdverse}
	\smallskip
}   

\NewDocumentEnvironment{postmula}{O{}}{
	\setvnum{postscript:}
	\smallskip
	\begin{ekdverse}
	\hspace{-\myoutdent}
	\Large}{\normalsize
	\end{ekdverse}
}

\NewDocumentEnvironment{altpostmula}{O{}}{
	\setvnum{postscript:}
	\smallskip
	\begin{ekdverse}[type=altrecension]
	\color{gray}
	\hspace{-\myoutdent}
	\Large}{\normalsize
	\end{ekdverse}
}

\NewDocumentEnvironment{col}{O{}}{
	\setvnum{colophon:}
	\medskip
	\begin{ekdverse}%
	\hspace{-2.5em}%
	\Large%
	}{\normalsize
	\end{ekdverse}
	%\smallskip
      }
      
\NewDocumentCommand{\tcommref}{m}{}
\NewDocumentCommand{\ttransref}{m}{}
\NewDocumentCommand{\tnocomm}{}{}


\def\startaltrecension{
	\setcounter{altvnum}{0}
	\stopvline
	\addtocounter{saved@poemline}{-1}
	\renewcommand{\versenr}{\ \themyvnum *{\small \arabic{poemline}}//}
%	\small
	}
	
\def\endaltrecension{
	\addtocounter{saved@poemline}{1}
	\startvline
	\setvnum{\hindsection.\arabic{poemline}}
	\renewcommand{\versenr}{\ \themyvnum//}
%	\normalsize
	}

\def\startaltnormal{
	\startaltrecension
	\setvnum{\hindsection.\arabic{saved@poemline}*\arabic{poemline}}
	}

\def\endaltnormal{\endaltrecension}

%%%%%%

\newcommand{\teionly}[1]{}
\newcommand{\teimute}[1]{#1}
\newcommand{\manuref}[1]{#1}
\newcounter{myvnum}\setcounter{myvnum}{0}
\newcounter{altvnum}\setcounter{altvnum}{0}
\newcounter{mynotenr}\setcounter{mynotenr}{0}
%\newcommand{\myfn}[1]{\footnote{\texteng{#1}}}

\newcommand{\myfn}[1]{%
	\setcounter{ekd@padanum}{0} % um Pāda-Nummer zu unterdrücken
	\stepcounter{mynotenr}%
	\linelabel{note\themynotenr}%
	\note[type=anmkg, labelb={note\themynotenr}]{#1}
	}

% \newcommand{\myfnx}[1]{%
	% \setcounter{ekd@padanum}{0} % um Pāda-Nummer zu unterdrücken
	% \stepcounter{mynotenr}%
	% \linelabel{note\themynotenr}%
	% \note[type=anmkg, labelb={note\themynotenr},num]{#1}
	% }

\renewcommand{\thefootnote}{\texteng{\arabic{footnote}}}
\newcommand{\devnote}[1]{{\small\textdev{#1}}}
\newcommand{\devtext}[1]{{\normalsize\textdev{#1}}}
%\newcommand{\vsn}[1]{{\footnotesize\texteng{#1}}}
\newcommand{\graus}[1]{\small\textcolor{gray}{#1}\normalsize} % partial altrecension
\newcommand{\grau}[1]{\textcolor{gray}{#1}} % partial altrecension
\newcommand{\Anm}[1]{\begin{ekdverse}
	\texteng{\footnotesize (#1)}
	\end{ekdverse}
	}

%\newcommand{\sgwit}[1]{{\footnotesize (\getsiglum{#1})}}
%\newcommand{\NotIn}[1]{\texteng{\footnotesize (om. \getsiglum{#1})}}
%\newcommand{\lineom}[2]{\texteng{\footnotesize (#1 om. \getsiglum{#2})}}
%\newcommand{\anm}[1]{\texteng{\footnotesize [#1]}}
\newcommand{\sgwit}[1]{}% Nur für Online version; Change TEI too!!
%\newcommand{\lineom}[2]{\myfn{#1 om. \getsiglum{#2}}}
\newcommand{\anm}[1]{\myfn{#1}}
%\newcommand{\unavbl}[1]{\marginpar{\scriptsize\texteng{−\,\getsiglum{#1}}}}
%\newcommand{\unavbl}[1]{\myfn{Folio lost in \getsiglum{#1}}}
\newcommand{\textapp}[1]{\texteng{\textsf{#1}}}
\newcommand{\unavbl}{\textapp{folio lost}}
\newcommand{\incl}{\textapp{included in}}
\newcommand{\only}{\textapp{only included in}}
\newcommand{\also}{\textapp{also included in}}
\newcommand{\excl}{\textapp{included in all except}}
\newcommand{\NotIn}{\om}
\newcommand{\expnr}[1]{\textcolor{magenta}{#1}}% X\kern 1pt

\def\om{\texteng{\emph{om.\@}}}% \kern-0.3ex
\def\illeg{\texteng{\emph{illeg.\@}}} 
\def\lost{\texteng{\emph{lost}}} 
\def\lacuna{\texteng{\emph{lac.\@}}}
\def\unm{\texteng{\emph{unm.\ }}}
\def\ante{\texteng{\normalfont\textapp{ante\ }}}
\def\add{\texteng{\normalfont\emph{add.\@}}}
\def\post{\texteng{\normalfont\textapp{post\ }}}
\def\antecorr{\texteng{\textsubscript{ac}}}
\def\postcorr{\texteng{\textsubscript{pc}}}
\def\marmas{\ }%\texteng{\textsuperscript{\#}}\ }
\def\marma{}%\texteng{\textsuperscript{\#}}}
\def\crux{\texteng{\textsuperscript{\textdagger}}}

%%%%%%% Commentary part

\usepackage{catchfilebetweentags}

\NewDocumentEnvironment{translation}{O{}}{%
	\selectlanguage{english}}{%
	\selectlanguage{iast}}
	
\NewDocumentEnvironment{sources}{O{}}{%
	\selectlanguage{english}%
	\begin{description}[leftmargin=1em, 
		topsep=0pt, parsep=0pt, partopsep=0pt,
		listparindent=0pt, labelwidth=1em, labelsep=0pt]
	\item[\ding{118}\ Sources]
	\item %
	}{\end{description}\selectlanguage{iast}}

\NewDocumentEnvironment{testimonia}{O{}}{%
	\selectlanguage{english}%
	\begin{description}[leftmargin=1em,
		topsep=0pt, parsep=0pt, partopsep=0pt,
		listparindent=0pt, labelwidth=1em, labelsep=0pt]
	\item[\ding{118}\ Testimonia]
	\item %
	}{\end{description}\selectlanguage{iast}}
	
\NewDocumentEnvironment{philcomm}{O{}}{%
	\selectlanguage{english}%
	\begin{description}[leftmargin=1em, 
		topsep=0pt, parsep=0pt, partopsep=0pt,
		listparindent=1.5em,
		labelwidth=1em, labelsep=0pt]
	\item[\ding{118}\ Commentary]\ %
	\newline
	}{\end{description}\selectlanguage{iast}}

\newenvironment{variants}{%
	\begin{description}[%
		leftmargin=4em,
		topsep=3.5pt,
		parsep=0pt,
	%	partopsep=0pt,
		listparindent=-1.5em,
		labelwidth=2.5em,
		labelsep=0pt]
	\item\scriptsize}{%
	\end{description}
	}
 
\newenvironment{versinnote}{%
	\setlength{\vindent}{0pt}
%	\poemlines{0}
	\vspace{4pt plus 2pt minus 2pt}
	\begin{ekdverse}
	\linespread{0.9}\normalsize\selectlanguage{iast}}{%
	\linespread{1}\selectlanguage{english}\end{ekdverse}
	\vspace{4pt plus 2pt minus 2pt}
%	\poemlines{1}
	\addtocounter{poemline}{-1}
	}

  \newenvironment{versinnoterm}{%
	\setlength{\vindent}{0pt}
	\vspace{1pt}
	\begin{ekdverse}
		\itshape}{%
		\rmfamily
	\end{ekdverse}
	\vspace{1pt}
	\addtocounter{poemline}{-1}
	}

\newenvironment{appinnote}{% still in use: 1.16, 1.30, 2.50, 2.77, 3.25, 3.34, 3.39*1, 4.9
	\setlength{\vindent}{0pt}
	\begin{ekdverse}
	\scriptsize\selectlanguage{english}}{%
	\selectlanguage{iast}\end{ekdverse}
	\vspace{3pt minus 1pt}
	\addtocounter{poemline}{-1}
}

%\newcommand{\vnumfix}{\addtocounter{poemline}{1}}
%\TeXtoTEIPat{\vnumfix}{}
\newcommand{\labelincomm}{\smallskip\newline\noindent}
%\TeXtoTEIPat{\labelincomm}{<lb/>} % >> PreambleComm.tex
%\newcommand{\tre}{\ }
%\TeXtoTEIPat{\tre}{}
\newcommand{\skx}[2]{#1} % sandhi between pādas
%\TeXtoTEIPat{\skx {#1}{#2}}{#2} % >> PreambleComm.tex
%\TeXtoTEIPat{\commcitecore}{}
%\TeXtoTEIPat{\commcite}{}
%\TeXtoTEIPat{\commciterange}{}
%\TeXtoTEIPat{\altcommcite}{}
%\TeXtoTEIPat{\avacite}{}
%\TeXtoTEIPat{\colcite}{}
%\TeXtoTEIPat{\trcite}{}

%\TeXtoTEIPat{\labelvnum}{}
%\TeXtoTEIPat{\commvnum}{}

\newcommand{\myvspace}{\vspace{-3pt plus 3pt minus 3pt}}
\newcommand{\commlabel}{\hfill\texteng{\raisebox{0pt}{\textbf{[\hindsection.\labelvnum]}}}\hfill}

\newcommand{\comminfn}{%
	\footnotetext{%
	\commlabel
	\ExecuteMetaData[\commfilename]{sc\commvnum}%
	\ExecuteMetaData[\commfilename]{ts\commvnum}%
	\ExecuteMetaData[\commfilename]{cm\commvnum}%
	}}
	
\newcommand{\commcitecore}{%
	\myvspace
	\begin{quote}%
	\ExecuteMetaData[\commfilename]{tr\commvnum}
	\texteng{(\labelvnum)}\comminfn
	\end{quote}}

\def\commfilename{HP\hindsection_comm.tex}
\newcommand{\commcite}{%
	\def\commvnum{\themyvnum}%
	\def\labelvnum{\themyvnum}%
	\commcitecore}

\newcommand{\commciterange}[2]{%
	\def\commvnum{#1}%
	\def\labelvnum{#2}%
	\commcitecore}
	
\newcommand{\altcommcite}{%
	\def\commvnum{\themyvnum-\thealtvnum}%
	\def\labelvnum{\themyvnum*\thealtvnum}%
	\myvspace
	\begin{quote}%
	\textcolor{gray}{%
	\ExecuteMetaData[\commfilename]{tr\commvnum}
	\texteng{(\labelvnum)}}\comminfn
	\end{quote}}

\newcommand{\avacite}[1]{%
	\bigskip%
	\setlength{\parindent}{1em} %\hspace*{0.5em} in HP4X
	\ExecuteMetaData[\commfilename]{tr#1}
	\vspace{-3pt}
	}

\newcommand{\trcite}[1]{
	\myvspace
	\begin{quote}
	\ExecuteMetaData[\commfilename]{tr#1}
	\texteng{(#1)}
	\end{quote}
	}

\newcommand{\alttrcite}{
	\def\commvnum{\themyvnum-\thealtvnum}%
	\def\labelvnum{\themyvnum*\thealtvnum}%
	\myvspace
	\begin{quote}
	\textcolor{gray}{\ExecuteMetaData[\commfilename]{tr\commvnum}
	\texteng{(\labelvnum)}}
	\end{quote}
	}

\newcommand{\colcite}{
	\medskip
	\noindent
	\ExecuteMetaData[\commfilename]{trcol}
	}


\newcommand{\closer}{\vspace{-1ex}}
\newcommand{\lb}{\par}
\newcommand{\mylb}{\smallskip\lb}
\newcommand{\sep}{\par}
% \TeXtoTEIPat{\sep}{<lb/>}% oder besser mit einem Trennzeichen in einer Zeile lassen?
\def\vl{\textit{v.l.}\xspace}
%\newcommand{\var}[1]{\texteng{\scriptsize #1}}
%\newcommand{\varsep}{\xspace\texteng{\textbullet}\xspace}

\def\sl#1{\emph{#1}}
\newcommand{\medialink}[2]{\textcolor{violet}{\underline{#1}}}
%\TeXtoTEIPat{\medialink {#1}{#2}}{<ref target="/images/#2">#1</ref>}
\usepackage{url}

\newcommand{\alphaOne}{\textalpha\textsubscript{1}}% N3
\newcommand{\alphaTwo}{\textalpha\textsubscript{2}}% J5
\newcommand{\alphaThree}{\textalpha\textsubscript{3}}% G4
\newcommand{\gammaOne}{\textgamma\textsubscript{1}}% N23
\newcommand{\gammaTwo}{\textgamma\textsubscript{2}}% J7
\newcommand{\deltaOne}{\textdelta\textsubscript{1}}% V19
\newcommand{\deltaTwo}{\textdelta\textsubscript{2}}% K3
\newcommand{\deltaThree}{\textdelta\textsubscript{3}}% C7
\newcommand{\deltaOmega}{\textdelta\textsubscript{\textomega}}% J6
\newcommand{\epsilonOne}{\textepsilon\textsubscript{1}}% G11
\newcommand{\epsilonTwo}{\textepsilon\textsubscript{2}}% G5
\newcommand{\zetaOne}{\textzeta\textsubscript{1}}% P15
\newcommand{\zetaTwo}{\textzeta\textsubscript{2}}% N19
\newcommand{\zetaThree}{\textzeta\textsubscript{3}}% V15
\newcommand{\zetaFour}{\textzeta\textsubscript{4}}% J11
\newcommand{\zetaOmega}{\textzeta\textsubscript{\textomega}}% N26
\newcommand{\etaOne}{\texteta\textsubscript{1}}% V1
\newcommand{\etaTwo}{\texteta\textsubscript{2}}% J10
\newcommand{\etaOmega}{\texteta\textsubscript{\textomega}}% E4
\newcommand{\piOne}{\textpi\textsubscript{1}}% P11
\newcommand{\piTwo}{\textpi\textsubscript{2}}% C6
\newcommand{\piOmega}{\textpi\textsubscript{\textomega}}% V3

\def\attr{\hbox{attrib.}\xspace}
\babelhyphenation{%
	Dattā-treya-yoga-śāstra
	Gorakṣa-śataka
	Go-rakṣa-nātha
	Haṭha-pra-dī-pikā
	Haṭha-ratnā-valī
	Haṭha-tattva-kaumudī
	Jāran-dhara
	Rāja-yoga
	Śām-bhavī
	Śāṃ-bhavī
	Śārṅga-dhara-pad-dhati
	Svātmā-rāma
	Śiva-saṃhitā
	Vasiṣṭha-saṃhitā
	Viveka-mārtaṇḍa
	Yukta-bhava-deva
	Yoga-cintā-maṇi
	Yoga-tattva-pra-kāśa
	Yoga-yājña-valkya
	}

\def\hindsection{4}

% Chp. 4 - N3,J5,G4; P11,C6,V3; N23,J7; V19,E2,C7(partly); G11,N19,V15; J10,Jyo

\renewcommand{\Anm}[1]{}
\newcommand{\orgx}[2]{#1}
\newcommand{\orgvnr}[1]{}
\newcommand{\myfnx}[1]{}

\begin{document}
\thispagestyle{firstpage}
\begin{center}
\section*{Chapter 4}
\end{center}
\bigskip
\begin{otherlanguage}{iast}
\begin{ekdosis}


\teimute{\setcounter{saved@poemline}{1}}
%<*vs1a>
\begin{ava}[hp04_001a]
\app{\lem[nolem]{}
	\rdg[wit={N3,J5,P11,C6}]{\only}}% G4 broken
atha samādhiḥ/ 
\end{ava}
%</vs1a>
\iffalse
\startaltnormal
%<*vs0-1>
\begin{alttlg}[hp04_000_1]
\tl{\app{\lem[nolem]{}
	\rdg[wit={N23,J7,V19,E2}]{\also}}%
\pada{\app{\lem[wit={ceteri}]{namaḥ}
	\rdg[wit={N23,E2,V3}]{oṃ namaḥ}% +K3,C7
	} śivāya gurave}
\pada{nādabindu%
	\app{\lem[wit={Gr2,Gr3,G5,J10,C6,Jyo}]{kalātmane}
		\rdg[wit={G11,N19,V15,P11,V3}]{layātmane}% +M3,F; °tmano? V3
		}/}\\+}
\tl{
\pada{\app{\lem[wit={ceteri}]{nirañjanapadaṃ}
		\rdg[wit={V3}]{nirañjanaṃ padaṃ}
		\rdg[wit={N23},alt={\om}]{\skp{\om}}}
	\app{\lem[wit={N23,J7,V19,E2,G11,V15,J10,Jyo}]{yāti}
		\rdg[wit={G5,N19,GrB}]{yānti}}}
\pada{\app{\lem[wit={J7,V19,G11,G5,N19,V15,GrB,Jyo}]{nityaṃ}
		\rdg[wit={N23}]{aharniśaṃ}
		\rdg[wit={J10}]{yato}
		\rdg[wit={E2}]{yatra}} % +K3,C7
	\app{\lem[wit={V19,G11,G5,V15,P11,V3,Jyo}]{yatra}
		\rdg[wit={Gr2,N19}]{yatna}% yatta? N23; +F
%		\rdg[wit={Jyo}]{tatra}
		\rdg[wit={C6}]{ca yat}
		\rdg[wit={J10}]{yogī}
		\rdg[wit={E2}]{nityaṃ}% +K3,C7
		}%
	\app{\lem[wit={Gr2,Gr3,V15,J10,V3,Jyo}]{parāyaṇaḥ}
		\rdg[wit={G11,G5,N19,P11,C6}]{parāyaṇāḥ}%##
		}//\versenr}%
	\orgx{}{\note[type=anmkg, labelb={note0},nonum,lem={*}]{The text and apparatus of the black-printed verses are identical to those of the older version. The grey-scaled verses are usually only found in manuscript groups \textepsilon, \textzeta, \texteta, \textpi, and \textchi. Notes about the omission or inclusion of these verses are only provided when there is a deviation from this pattern.
	}}
	\\!}
\end{alttlg}
%</vs0-1>

%<*vs0-2>
\begin{alttlg}[hp04_000_2]
\tl{\app{\lem[nolem]{}
	\rdg[wit={N23,J7,V19,E2}]{\also}}%
\pada{\app{\lem[wit={ceteri}]{athedānīṃ} % °dāniṃ V15
		\rdg[wit={V3}]{athodānī}
		\rdg[wit={N23}]{athekṣanīṃ}}
	pravakṣyāmi} % vakṣāmi N19,N23,V19,V3
\pada{samādhikrama%
	\app{\lem[wit={G11,G5,N19,V15,J10,GrB,Jyo},alt={°m uttamam}]{\skp{°}m uttamam}
		\rdg[wit={Gr2,Gr3}]{lakṣaṇam}}/}\\+}
\tl{
\pada{mṛtyughnaṃ % ghaṃ N23
	\app{\lem[wit={Gr2,E2,G11,G5,GrB}]{tu}
		\rdg[wit={N19,V15,J10,Jyo}]{ca}
		\rdg[wit={V19}]{su}} sukhopāyaṃ} % muṣopāpaṃ N23, mukho° N19
\pada{brahmānandakaraṃ 
	\app{\lem[wit={ceteri}]{param}
		\rdg[wit={G5}]{sadā}}//\versenr}\\!}% brahmanāṃda J7ac; para J10ac
\end{alttlg}
%</vs0-2>

%\startaltrecension
%\teimute{\small}
% \begin{alttlg}[hp04_000_3]
% \tl{%\vin
% \pada{\app{\lem[wit={G11,V15,Jyo}]{rājayogaḥ}
		% \rdg[wit={G5,N19,J10,C6}]{rājayoga}% +F
		% }
	% \app{\lem[wit={G11,G5,J10,C6,Jyo}]{samādhiś ca}% +G7
		% \rdg[wit={N19,V15}]{samādhiḥ syād}% samādhi V15; +F,N22,P6
		% }}
% \pada{\app{\lem[wit={ceteri}]{unmanī}
	% \rdg[wit={G11}]{py unmanī}} ca manonmanī/}\\+}
% \tl{%\vin
% \pada{\app{\lem[wit={V15,J10}]{amaraugho}
		% \rdg[wit={G11}]{amaraughā}
		% \rdg[wit={C6}]{amaraughi}
		% \rdg[wit={N19}]{avaraubhū}
		% \rdg[wit={Jyo}]{amaratvaṃ}
		% \rdg[wit={G5}]{aromaro}}
	% \app{\lem[wit={G11,N19,J10,C6,Jyo},alt={layas}]{laya\skp{s}}
		% \rdg[wit={V15}]{layes}
		% \rdg[wit={G5}]{yas tat}}%
	% \app{\lem[wit={G11,N19,V15,C6,Jyo},alt={tattvaṃ}]{\skm{s }tattvaṃ}
		% \rdg[wit={J10}]{tatra}
		% \rdg[wit={G5}]{tulyaḥ}}}
% \pada{\app{\lem[wit={G11,N19,V15,J10,Jyo}]{śūnyāśūnyaṃ} % °śūnya N19
		% \rdg[wit={C6}]{śūnyāc chūnyaṃ}
		% \rdg[wit={G5}]{śūnyāt śūnya}}
		% paraṃ padam//\versenr}\label{synonym3}
	% \sgwit{G11,G5,N19,V15,J10,C6,Jyo} \anm{\textleftarrow\ \ref{A1}}\\!}
% \end{alttlg}

% \begin{alttlg}[hp04_000_4]
% \tl{%\vin
% \pada{amanaskaṃ tathādvaitaṃ}
% \pada{nirālambaṃ nirañjanam/}\\+}
% \tl{%\vin
% \pada{jīvanmuktiś ca % jīvamuktiś N19, °muktaś G5
	% \app{\lem[wit={G11,G5,N19,J10,C6,Jyo}]{sahajaṃ}
% %		\rdg[wit={Jyo}]{sahajā}
		% \rdg[wit={V15},alt={\om}]{\skp{\om}}}}
% \pada{\app{\lem[wit={V15,C6}]{turyaṃ} % +J11
		% \rdg[wit={G5}]{tulyaṃ}
		% \rdg[wit={N19}]{turyai}
		% \rdg[wit={Jyo}]{turyā}
		% \rdg[wit={G11}]{turīyaṃ}
		% \rdg[wit={J10}]{muktiś}}
	% \app{\lem[wit={J10pc,Jyo}]{cety ekavācakāḥ}
		% \rdg[wit={J10ac},alt={°kaḥ}]{cety ekavācakaḥ}
		% \rdg[wit={G5,C6}]{cety ekavācakam}% +F; caitye° C6
		% \rdg[wit={V15}]{cittaikavācakam}% +N22,P6
		% \rdg[wit={N19}]{ciṃtaikavācakam}
		% \rdg[wit={G11}]{caikavācakaṃ}
		% }//\versenr}\label{synonym4}
% \sgwit{G11,G5,N19,V15,J10,C6,Jyo} \anm{\textleftarrow\ \ref{A2}}%
% \myfn{\getsiglum{C6} has these verses on synonyms both here and at \ref{A1}/\ref{A2}, but \getsiglum{P11} has them at the latter place only.}\\!}
% \end{alttlg}
%\endaltrecension
%\teimute{\normalsize}

%<*vs0-3>
\begin{alttlg}[hp04_000_3]
\tl{\app{\lem[nolem]{}
	\rdg[wit={N23,J7,V19,E2}]{\also}}%
\pada{salile saindhavaṃ
	\app{\lem[wit={ceteri},alt={yadvat}]{yadva\skp{t}}
		\rdg[wit={N19}]{tadvat}}t}
\pada{sāmyaṃ \app{\lem[wit={Gr2,Gr3,J10,C6,Jyo}]{bhajati}
		\rdg[wit={V3}]{bhajata}
		\rdg[wit={G11,G5,N19,V15}]{bhavati}% +F
		\rdg[wit={P11}]{ttadgati}} yogataḥ/}\\+} % °ta N19
\tl{
\pada{\app{\lem[wit={ceteri}]{tathā}
		\rdg[wit={V3}]{athā}
		\rdg[wit={J10}]{yathā}}%
	\app{\lem[wit={ceteri},alt={°tmamanasor}]{\skp{°}tmamanaso\skp{r}} % °saur N23
		\rdg[wit={J10}]{tmānamanor}}r aikyaṃ} % ma om. P7
\pada{samādhiḥ % °dhi P11, °dhir J10,Jyo
	\app{\lem[wit={ceteri}]{so}
		\rdg[wit={P11}]{sā}
		\rdg[wit={G11,G5,J10,Jyo}]{a°}}%
	\app{\lem[wit={ceteri}]{'bhidhīyate}
		\rdg[wit={N19}]{'bhidhīte}
		\rdg[wit={N23}]{vidhīyate}}//\versenr}\label{salile}\\!}
\end{alttlg}
%</vs0-3>
\Anm{\getsiglum{G11plus,Jyo} have \ref{visvarupa} \textit{yadā saṃkṣīyate prāṇo} here%
%\myfn{In the following, not all of the differences in the verse order of \getsiglum{GrB} and \getsiglum{Jyo} are noted.}% \getsiglum{GrB} follow the order of \getsiglum{Gr2} (or of \getsiglum{Gr3}?) in the beginning and the end (after 4.72). The middle part is a kind of mix of \getsiglum{Gr2} and \getsiglum{N19,V15}. The verse order of \getsiglum{Jyo} is similar to that of \getsiglum{N19,V15}, but with many small differences.
}%


%\startaltrecension
%\teimute{\small}
%<*vs0-4>
\begin{alttlg}[hp04_000_4]
\tl{\app{\lem[nolem]{}
 \rdg[wit={GrB}]{\NotIn}}%
\pada{\app{\lem[wit={N19,V15}]{yat samatvaṃ}
	\rdg[wit={G11,G5}]{tat samatvaṃ}
	\rdg[wit={J10,Jyo}]{tat samaṃ ca}% jat J10pc
	}
	dvayo%r
	\app{\lem[wit={G11,G5},alt={atra}]{\skm{r }atra}
	\rdg[wit={N19,V15}]{eva}
	\rdg[wit={J10,Jyo}]{aikyaṃ}
	}}
\pada{jīvātmaparamātmanoḥ/}\\+} % ātmā G11
\tl{%\vin
\pada{\app{\lem[wit={G5,N19,V15,J10}]{samastanaṣṭa}
	\rdg[wit={G11}]{samastaṃ naṣṭa}
	\rdg[wit={Jyo}]{pranaṣṭasarva}}%
\app{\lem[wit={G11,G5,V15}]{saṃkalpaḥ}
	\rdg[wit={N19,J10}]{saṃkalpa}
	\rdg[wit={Jyo}]{saṃkalpaṃ}}}
\pada{samādhiḥ so'bhidhīyate//\versenr} % sā G5
\label{yatsamatvam}%
%\sgwit{G11,G5,N19,V15,J10,Jyo}
\myfn{\getsiglum{J10} adds another similar verse here:
\vspace{2pt minus 1pt}\\
\devnote{karpūraṃ salile yadvat saindhavaṃ salile yathā/
tathātmamanasor aikyaṃ samādhiḥ so'bhidhīyate//} (cf. \ref{karpura}ab and \ref{salile}cd)}\\!}
\end{alttlg}
%</vs0-4>
%\endaltrecension
%\teimute{\normalsize}


%<*vs0-5>
\begin{alttlg}[hp04_000_5]
\tl{\app{\lem[nolem]{}
	\rdg[wit={N23,J7,V19,E2}]{\also}}%
\pada{rājayogasya
	\app{\lem[wit={ceteri}]{māhātmyaṃ}
		\rdg[wit={J7}]{māhatmyaṃ}
		\rdg[wit={V15}]{mahā}}}
\pada{ko vā jānāti tattvataḥ/}\\+} % ki N19; cā P11; jānaṃti N19
\tl{
\pada{\app{\lem[wit={ceteri},alt={jñānān}]{jñānā\skp{n}}
		\rdg[wit={V15,J10}]{jñāna}
		\rdg[wit={Jyo}]{jñānaṃ}
		\rdg[wit={V19}]{jñān}}%
	\app{\lem[wit={Gr2,E2,C6,Jyo},alt={muktiḥ}]{\skm{n }muktiḥ}% +K1
		\rdg[wit={V19,G11,G5,N19,V15,J10,P11,V3}]{mukti}}
	\app{\lem[wit={G11,G5}]{sthirā}% +M3
		\rdg[wit={N19,V3}]{sthite}% +F,K1
		\rdg[wit={P11}]{sthitai}% sthitā P6
		\rdg[wit={Gr2,E2,J10,C6,Jyo}]{sthitiḥ}
		\rdg[wit={V19}]{sthiti<<ḥ>>}
		\rdg[wit={V15}]{°s tato}}
	\app{\lem[wit={N19,V15,P11,C6,Jyo},alt={siddhir}]{siddhi\skp{r}}% +F,K1
		\rdg[wit={J10,V3}]{siddhi}
		\rdg[wit={Gr2,Gr3,G11,G5}]{siddhā}}r}
\pada{guru\app{\lem[wit={ceteri}]{vākyena} % gurur P11
		\rdg[wit={N23}]{vākyāt <<pra>>}}
	\app{\lem[wit={ceteri}]{labhyate}
		\rdg[wit={J10}]{sidhyati}}//\versenr}\\!}
\end{alttlg}
%</vs0-5>

%<*vs0-6>
\begin{alttlg}[hp04_000_6]
\tl{\app{\lem[nolem]{}
	\rdg[wit={N23,J7,V19,E2}]{\also}}%
\pada{durlabho viṣayatyāgo} % durlabha C6; viṣayāt* yogo N23
\pada{durlabhaṃ tattvadarśanam/}\\+} % labha V3
\tl{
\pada{durlabhā sahajāvasthā} % dullā G11, tu sahāvasthā G5
\pada{sadguroḥ karuṇāṃ vinā//\versenr}% guro P11,V3; karuṇā P11,N19
\label{durlabho}\\!}
\end{alttlg}
%</vs0-6>

\Anm{\getsiglum{G11} has \ref{kastha} \textit{kāṣṭhagoṣṭhīprapañcena} here}

\Anm{\getsiglum{G11,G5,N19,V15,J10} have \ref{yavan} \textit{yāvan naiva praviśati} here}

%<*vs0-7>
\begin{alttlg}[hp04_000_7]
\tl{\app{\lem[nolem]{}
	\rdg[wit={N23,J7,V19,E2}]{\also}}%
\pada{vividhair % vicitrair G5
	āsanaiḥ % āsanai P11, āsanaḥ V15
	kumbhai}% °bhai N23,C6
\pada{\app{\lem[wit={G5,C6,Jyo},alt={vicitraiḥ}]{\skm{r }vicitraiḥ}% +K3,M3,F
		\rdg[wit={Gr2,Gr3,N19,V15,J10,P11,V3}]{vicitra}
		\rdg[wit={G11}]{citraiś ca}}
	\app{\lem[wit={Gr3,G11,G5,J10,GrB,Jyo}]{karaṇair api}
		\rdg[wit={J7}]{karuṇair api}
		\rdg[wit={N23}]{kalaṇair api}
		\rdg[wit={N19,V15}]{karaṇair atha}}/}\\+}
\tl{
\pada{\app{\lem[wit={ceteri},alt={prabuddhāyām}]{prabuddhāyā\skp{m}}% pravudhā° V19
		\rdg[wit={N19}]{pradhadhāyām}}%
	\app{\lem[wit={ceteri},alt={ādi}]{\skm{m }ādi}
		\rdg[wit={V15}]{idaṃ}
		\rdg[wit={Jyo}]{mahā}}%
	\app{\lem[wit={ceteri}]{śaktau} % proktau J10ac
		\rdg[wit={N23}]{śaktiḥ}}}
\pada{prāṇaḥ śūnye % prāṇaṃ C6, prāṇa V15; sampra° G5
	\app{\lem[wit={N23,Gr3,G11,G5,J10,C6}]{vilīyate}
		\rdg[wit={J7}]{vidhīyate}
		\rdg[wit={N19,V15,P11,V3,Jyo}]{pralīyate}% +F
		}//\versenr}\\!}
\end{alttlg}
%</vs0-7>

%<*vs0-8>
\begin{alttlg}[hp04_000_8]
\tl{\app{\lem[nolem]{}
	\rdg[wit={N23,J7,V19,E2}]{\also}}%
\pada{\app{\lem[nolem]{\skp{pāda a}}
	\rdg[wit={C6},alt={\om}]{\skp{\om}}}%
\app{\lem[wit={ceteri}]{utpanna}
		\rdg[wit={V19}]{utpannā}
		\rdg[wit={N23}]{ut<<pan>>na}}%
	\app{\lem[wit={ceteri}]{śaktibodhasya}
		\rdg[wit={N23}]{śaktibodhaḥ syāt}
		\rdg[wit={V15}]{śaktibodhaś ca}}}
\pada{	\app{\lem[nolem]{\skp{pāda b}}
	\rdg[wit={C6},alt={\om}]{\skp{\om}}}%
\app{\lem[wit={ceteri}]{tyakta}
		\rdg[wit={N23}]{prakṣa}}%
	niḥśeṣakarmaṇaḥ/}\\+}
\tl{
\pada{\app{\lem[wit={ceteri}]{yoginaḥ}
	\rdg[wit={C6}]{yogināṃ}} sahajāvasthā}
\pada{svaya%m
	\app{\lem[wit={G11,G5,V15,J10,P11,V3},alt={eva prakāśate}]{\skm{m }eva prakāśate}% +K1; prakāśyate N22,P6
		\rdg[wit={N19}]{eva prakāśayet}
		\rdg[wit={Gr2,Gr3,C6,Jyo}]{eva prajāyate}
		}//\versenr}\\!}
\end{alttlg}
%</vs0-8>

%<*vs0-9>
\begin{alttlg}[hp04_000_9]
\tl{\app{\lem[nolem]{}
	\rdg[wit={N23,J7,V19,E2}]{\also}}%
\pada{suṣumṇā\app{\lem[wit={ceteri}]{vāhini}% °<<m>>ṇā N23
		\rdg[wit={N23,G5,N19,V3}]{vāhinī}
		\rdg[wit={V19}]{vāhi}}
	\app{\lem[wit={ceteri}]{prāṇe}
		\rdg[wit={V3}]{prāṇa}}}
\pada{\app{\lem[wit={G11,G5,V15,P11}]{śūnyaṃ}
		\rdg[wit={J10}]{śūnya}% śūnyaṃ J10pc
		\rdg[wit={Gr2,Gr3,C6,Jyo}]{śūnye}% +K1,N22,P6
		\rdg[wit={V3}]{śūne}
		\rdg[wit={N19}]{śūnyā}} 
	\app{\lem[wit={ceteri}]{viśati}
		\rdg[wit={P11}]{vasati}}
	\app{\lem[wit={G11,P11,V3,Jyo}]{mānase} % +M1,M3
		\rdg[wit={J10}]{mārutaḥ}
		\rdg[wit={Gr2,Gr3,G5,N19,V15,C6}]{mārute}}\marma/}\\+} % mārutai N23; +K1,N22,P6,G7
\tl{
\pada{\app{\lem[wit={Gr2,Gr3,G11}]{tathā}
		\rdg[wit={G5,N19,V15,J10,GrB,Jyo}]{tadā}}
	\app{\lem[wit={ceteri}]{samasta}
		\rdg[wit={J10,Jyo}]{sarvāṇi}}karmāṇi}
\pada{\app{\lem[wit={ceteri}]{nirmūlayati}
		\rdg[wit={V19,V15}]{nimūlayati}
		\rdg[wit={N23}]{nirmūlaṃ yāti}
		\rdg[wit={G5}]{nirmalaṃ yāti}} % °ḷayati V15
	\app{\lem[wit={G11,N19,J10,GrB}]{marmavit}% K1,M1,G7
		\rdg[wit={N23,G5,V15}]{karmavit}% P6,P22,G5,M3,F
		\rdg[wit={J7}]{karmakṛt}
		\rdg[wit={Gr3,Jyo}]{yogavit}
		}//\versenr}\\!}
\end{alttlg}
%</vs0-9>

%<*vs0-10>
\begin{alttlg}[hp04_000_10]
\tl{\app{\lem[nolem]{}
	\rdg[wit={N23,J7,V19,E2},postwit=\textapp{(pāda a and d only)}]{\also}}%
\pada{\app{\lem[wit={G5,V15,V3}]{amaraugha}% +F
		\rdg[wit={N19,P11}]{amarogha}% +M3?
		\rdg[wit={C6}]{amaraughi}
		\rdg[wit={J10,Jyo}]{amarāya}% +C2
		\rdg[wit={G11}]{amareśa}
		\rdg[wit={Gr2}]{amano nir°}
		\rdg[wit={Gr3}]{amalo nir°}} % verse om. K1,P6
	\app{\lem[wit={G11,G5,N19,V15,J10,GrB,Jyo}]{namas tubhyaṃ}
	\rdg[wit={Gr2}]{°manāḥ śūnyaṃ}
	\rdg[wit={Gr3}]{°malaḥ śūnyaṃ}}%
	}
\pada{\app{\lem[nolem]{\skp{pāda b}}
	\rdg[wit={Gr2,Gr3},alt={\om}]{\skp{\om}}}%
so'pi \app{\lem[wit={G11,N19,C6,V3,Jyo}]{kālas tvayā}% trayā V3
	\rdg[wit={P11}]{kālaṃ tvayā}
	\rdg[wit={V15}]{kāla tvayā}
	\rdg[wit={J10}]{kālantayā}
	\rdg[wit={G5}]{kālasya vā°}
	}
\app{\lem[wit={G11,N19,V15,J10,GrB}]{hataḥ}% hata N19
	\rdg[wit={G5}]{°hakaḥ}
	\rdg[wit={Jyo}]{jitaḥ}
	}/}\\+}
\tl{
\pada{\app{\lem[nolem]{\skp{pāda c}}
	\rdg[wit={Gr2,Gr3},alt={\om}]{\skp{\om}}}%
patitaṃ \app{\lem[wit={G11,G5,N19,V15,GrB,Jyo}]{vadane}
	\rdg[wit={J10}]{pavane}
	} yasya} % yasyā P11
\pada{jagad etac carācaram//\versenr} % yagad V19; carāccaraṃ J7, °care P11
	\\!}
\end{alttlg}
%</vs0-10>

%<*vs0-11>
\begin{alttlg}[hp04_000_11]
\tl{\app{\lem[nolem]{}
	\rdg[wit={N23,J7,V19,E2}]{\also}}%
\pada{citte
	\app{\lem[wit={ceteri},alt={samatvam}]{samatva\skp{m}}
		\rdg[wit={N19,V15}]{śamatvam}
		\rdg[wit={N23}]{samatyam}}m āpanne}
\pada{\app{\lem[wit={J7,Gr3,G11,G5,N19,Jyo}]{vāyau}
		\rdg[wit={V15}]{vāyo}
		\rdg[wit={N23,V3}]{vāyor}
		\rdg[wit={J10,C6}]{vāyur}
		\rdg[wit={P11}]{vāyu}}
	\app{\lem[wit={ceteri}]{vrajati}
	\rdg[wit={N23}]{javati}} madhyame/}\\+}
\tl{
\pada{\app{\lem[nolem]{\skp{pāda c}}
		\rdg[wit={Gr2},alt={\om}]{\skp{\om}}}%
	\app{\lem[wit={G11,G5,N19}]{tadāmaraugha}
		\rdg[wit={P11,V3}]{eṣāmaraugha}
		\rdg[wit={V15}]{tadāmaroḷi}
		\rdg[wit={Jyo}]{tadāmarolī}
		\rdg[wit={J10}]{tathāmarolī}
		\rdg[wit={C6}]{saivāmarolī}
		\rdg[wit={V19}]{eṣā naulīti}
		\rdg[wit={E2}]{eṣā naulī ca}}% +C7
	\app{\lem[wit={V19,G5,N19,J10,GrB,Jyo}]{vajrolī}% +G5,M3,K3,C7
		\rdg[wit={G11}]{vajrolīs}
		\rdg[wit={V15}]{vajrolis}
		\rdg[wit={E2}]{vajrī ca}}}
\pada{\app{\lem[nolem]{\skp{pāda d}}
		\rdg[wit={Gr2},alt={\om}]{\skp{\om}}}%
	\crux\app{\lem[wit={G11,G5,N19,V15}]{tadāśā jīvite'pi ca}% +M3
		\rdg[wit={GrB}]{sadā me bhimateti ca} % bhimate cita V3; timateti vaḥ P11
		\rdg[wit={Gr3}]{sadā cābhimateti ca}
		\rdg[wit={J10}]{sahajolī mato pi ca}
		\rdg[wit={Jyo}]{sahajolī prajāyate}}\crux//\versenr}
		\\!}
\end{alttlg}
%</vs0-11>
% G7 tadha manā vajroci tādhāśrājiṃtasya ca (*12 om.)
% M3 tato maśā .. vajraulī tadāśājīvitepi ca
% K1 eṣāmarolī vajrolī sadā abhinayāti ca
% P6 eṣāmaroli vajroli sadā abhinayaṃti ca (*12 om.)
% N22 eṣāmaroli vajroli sadā savitayiti ca (*12 om.)
% P11 eṣāmaraughavajrolī sadā me timateti vaḥ


%<*vs0-12>
\begin{alttlg}[hp04_000_12]
\tl{\app{\lem[nolem]{}
	\rdg[wit={N23,J7,V19,E2}]{\also}}%
\pada{jñānaṃ 
	\app{\lem[wit={ceteri}]{kuto}
		\rdg[wit={G11}]{tato}} manasi
	\app{\lem[wit={Gr2,Gr3,J10,GrB}]{jīvati devi yāvat} % jāvat V19, yā<<va>>t N23
		\rdg[wit={G11,N19}]{jīvati devi tāvat}
		\rdg[wit={G5}]{jīvati tepi tāvat}
		\rdg[wit={Jyo}]{saṃbhavatīha tāvat}
		\rdg[wit={V15}]{jīvati durvikalpe}}}\\+}
\tl{
\pada{\app{\lem[wit={ceteri}]{prāṇo'pi}
		\rdg[wit={G11,V15,C6}]{prāṇe pi}
		\rdg[wit={G5}]{prāṇeha}} jīvati mano
	\app{\lem[wit={ceteri}]{mriyate} % sriyate N23, mrīyate P7
		\rdg[wit={J7,V19}]{mṛyate}
		\rdg[wit={V15}]{miyata}
		\rdg[wit={G5}]{priyate}}
	\app{\lem[wit={ceteri}]{na}
		\rdg[wit={N19}]{ca}}
	\app{\lem[wit={GrB}]{tāvat}
		\rdg[wit={ceteri}]{yāvat}% +F,G5
		}/}\\+}
\tl{
\pada{\app{\lem[wit={ceteri}]{prāṇo}
		\rdg[wit={Gr3}]{prāṇaṃ}}
	\app{\lem[wit={ceteri}]{mano}
		\rdg[wit={G11,G5,N19}]{'pi ca}
		} dvayam idaṃ % dvayām N23
	\app{\lem[wit={ceteri}]{vilayaṃ}
		\rdg[wit={V15}]{na vilī°}}
	\app{\lem[wit={P11,C6}]{prayāti}% +K1,P6,F,source
		\rdg[wit={V3}]{prajāti}
		\rdg[wit={J10}]{na yāti}% +G7,K3
		\rdg[wit={N19}]{na yāvat}
		\rdg[wit={G5}]{na yattat}
		\rdg[wit={Gr3,Jyo}]{nayed yo}
		\rdg[wit={J7}]{naved yo}
		\rdg[wit={N23}]{jayed  yo}
		\rdg[wit={G11}]{nayet taṃ}% nayete M3; gate cen M1; nayeta? G3
		\rdg[wit={V15}]{°yate tra}}}\\+}
\tl{
\pada{mokṣaṃ
	\app{\lem[wit={ceteri}]{sa}
		\rdg[wit={V15}]{na}
		\rdg[wit={C6}]{ca}} gacchati % gacchatiti V19
	naro na kathaṃci%d % narā N23
	\app{\lem[wit={ceteri},alt={anyaḥ}]{\skm{d }anyaḥ}
		\rdg[wit={G5}]{anyam}
		\rdg[wit={J10}]{anyat}
		\rdg[wit={V3}]{anya}}//\versenr}
	\label{jnanam}\\!}
\end{alttlg}
%</vs0-12>

\Anm{\getsiglum{G11plus,N19,V15,J10,Jyo} have \ref{jnatva}--\ref{tatraika} \textit{jñātvā suṣumṇāsadbhedaṃ} here}


%<*vs0-13>
\begin{alttlg}[hp04_000_13]
\tl{\app{\lem[nolem]{}
	\rdg[wit={N23,J7,V19,E2}]{\also}}%
\pada{\app{\lem[wit={ceteri}]{rasasya}% +P6,M3
		\rdg[wit={J7,N19,V15}]{rasaś ca}} % +G7
	\app{\lem[wit={ceteri}]{manasaś caiva} % caivaṃ G11,J10
		\rdg[wit={V3}]{manaś caiva}
		\rdg[wit={N23}]{manasaiva caṃ°}}}
\pada{\app{\lem[wit={ceteri}]{cañcalatvaṃ}
		\rdg[wit={N23}]{°calatvaṃ ca}
		\rdg[wit={N19}]{vaṃcatvaṃ ca}} svabhāvataḥ/}\\+} % °ta P11
\tl{
\pada{\app{\lem[wit={G11,G5,V15}]{rasa}% +F
		\rdg[wit={N23,N19}]{rase}
		\rdg[wit={J7,Gr3,J10,GrB,Jyo}]{raso}
		}%
	\app{\lem[wit={G5,N19,V15}]{bandhe}
		\rdg[wit={G11}]{baddhe}% +G7, rasabandhe manobandhe M3,V15
		\rdg[wit={ceteri}]{baddho}% +F
		} 
	mano\app{\lem[wit={V15}]{bandhe}
		\rdg[wit={G11}]{baddhe}
		\rdg[wit={C6}]{baddho}
		\rdg[wit={ceteri}]{baddhaṃ} % varddhaṃ N23
		\rdg[wit={P11}]{baddhaḥ}% +F
		\rdg[wit={G5}]{dhatte}}}
\pada{\app{\lem[wit={ceteri}]{kiṃ}
		\rdg[wit={N19}]{tan}}
	na sidhyati bhūtale//\versenr}\\!} % siddhyaṃti V3
\end{alttlg}
%</vs0-13>
%<*vs0-14>
\begin{alttlg}[hp04_000_14]
\tl{\app{\lem[nolem]{}
	\rdg[wit={N23,J7,V19,E2}]{\also}}%
\pada{mūrchito % mūrchato E2
	\app{\lem[wit={Gr2,Gr3,N19,V15,C6,Jyo}]{harate}
		\rdg[wit={G11,G5,J10,P11,V3}]{harati}% +F
		}
	\app{\lem[wit={ceteri}]{vyādhiṃ}
		\rdg[wit={J10,V3}]{vyādhi}
		\rdg[wit={P11}]{vyādhin}
		\rdg[wit={G5,Jyo}]{vyādhīn}}}
\pada{mṛto
	\app{\lem[wit={ceteri}]{jīvayati}
		\rdg[wit={V15}]{jīvayate}}
		svayam/}\\+}
\tl{
\pada{baddhaḥ % badho P11, bandhaḥ G5
	\app{\lem[wit={ceteri}]{khecaratāṃ}
		\rdg[wit={V19}]{khacatāṃ}}
	\app{\lem[wit={ceteri}]{dhatte}
		\rdg[wit={N23,N19}]{dhartte}
		\rdg[wit={V3}]{yāti}}}
\pada{\app{\lem[wit={ceteri}]{raso vāyuś ca}
		\rdg[wit={V3}]{vāyuś ca}
		\rdg[wit={J10}]{sa jīveśvara}}
	\app{\lem[wit={Gr3,C6}]{bhairavi}
		\rdg[wit={Gr2,G11,G5,N19,V15}]{bhairavī}
		\rdg[wit={V3},post=\texteng{(tathā for missing raso)}]{bhairavī tathā}
		\rdg[wit={P11}]{tad dvayaṃ}
		\rdg[wit={J10}]{seśvaraḥ}
		\rdg[wit={Jyo}]{pārvati}
		}//\versenr}\\!}
\end{alttlg}
%</vs0-14>
\Anm{\getsiglum{G11,N19,V15,J10} have \ref{vayumargena} \textit{vāyumārge tv asaṃcāre} here}

\Anm{\getsiglum{G11,N19,V15,J10,Jyo} have \ref{manahsthairye} \textit{manaḥsthairye} here}
\endaltnormal
\fi


%<*vs1>
\begin{tlg}[hp04_001]
\tl{
\pada{\app{\lem[wit={ceteri}]{indriyāṇāṃ} % °nāṃ V19, °ṇā N3
		\rdg[wit={N19}]{indriyāṇi}} mano nātho} % nāthe G11
\pada{\app{\lem[wit={N3,J5,G11,GrB,Jyo}]{manonāthas tu}
		\rdg[wit={G4}]{manonāthasu}
		\rdg[wit={N19}]{manonāthaḥ su}
		\rdg[wit={N23,Gr3,V15,J10}]{manonāthaś ca}
		\rdg[wit={J7}]{manaso nātha}} mārutaḥ/}\\+} % mārute P11
\tl{
\pada{mārutasya layo
	\app{\lem[wit={ceteri},alt={nāthas/nāthaḥ/nātho}]{nātha\skp{s/nāthaḥ/nātho}}
		\rdg[wit={J7}]{nāthāḥ}}}%
\pada{\app{\lem[wit={N3,J5,G11,N19,V15,J10,V3},alt={taṃ nāthaṃ layam āśrayet}]{\skm{s }taṃ nāthaṃ layam āśrayet}% K1A; nātha N3
		\rdg[wit={G4}]{tan nātho laya\,+\,+\,+}
		\rdg[wit={Gr2,E2,C6,Jyo}]{sa layo nādam āśritaḥ}% K1B
		\rdg[wit={P11}]{laya nātha niraṃjanāṃ}% tasya nātho nirañjanaḥ F
		\rdg[wit={V19}]{layo dasamāśrayaḥ}}//\versenr}
		\orgvnr{1}\\!}
\end{tlg}
%</vs1>

\avacite{1a}
\commcite\newpage


\iffalse
\startaltnormal
%<*vs1-1>
\begin{alttlg}[hp04_001_1]
\tl{\app{\lem[nolem]{}
	\rdg[wit={N23,J7,V19,E2}]{\also}}%
\pada{\app{\lem[wit={G11,G5,V15,GrB,Jyo}]{so'yam evāstu}
		\rdg[wit={N19}]{soyamo vāstu}
		\rdg[wit={J10}]{svayam evāstu}% +K1,P6
		\rdg[wit={Gr2,Gr3}]{ayam eva tu}% +F
		} % evaṃ N23
	\app{\lem[wit={ceteri}]{mokṣākhyo} % °kṣyo N19, °syo P11, °khyā G5
		\rdg[wit={J10}]{vā mokṣaḥ}
		}}
\pada{\app{\lem[wit={G11,G5,V15,GrB,Jyo}]{māstu vāpi}% +P11; astu K1,P6
		\rdg[wit={N19}]{māstu kapi}
		\rdg[wit={J10}]{sosti vāpi}
		\rdg[wit={J7}]{'stu vāpi sa}
		\rdg[wit={V19}]{yas tu vāpi}% +K3,C7
		\rdg[wit={E2}]{yas tu vyāpi}
		\rdg[wit={N23}]{aya vāpi}}
	matāntare/}\\+} % matātare J7, matāṃbare P11, mātā° J10ac
\tl{
\pada{manaḥprāṇa% mana P11
	\app{\lem[wit={Gr2,G11,G5,V15,P11,C6}]{layānando} % layāṃnado N23; layo? V15
		\rdg[wit={N19}]{layānanda}
		\rdg[wit={V3}]{layāna} % 1 syllable too short
		\rdg[wit={Gr3}]{layo nādo}
		\rdg[wit={Jyo}]{laye kaścid}
		\rdg[wit={J10}]{°m apānaṃ vā}}}
\pada{\app{\lem[wit={G11,N19,V15,P11,C6}]{mayi}% +K1,M1,M3,P6
		\rdg[wit={V3}]{māpi}
		\rdg[wit={Gr2,Gr3}]{nāpi}
		\rdg[wit={G5}]{bhuvi}
		\rdg[wit={J10}]{layaḥ}
		\rdg[wit={Jyo}]{āna°}}
	\app{\lem[wit={ceteri},alt={kaścit/°cid}]{kaści\skp{t/°cid}}
		\rdg[wit={V19}]{kviṃcid}
		\rdg[wit={Jyo}]{°ndaḥ saṃ°}}%
	\app{\lem[wit={G11,N19,V15,J10,P11,C6,Jyo},alt={pravartate}]{\skm{t }pravartate}% °vattate G11
		\rdg[wit={V3}]{pravartate na}
		\rdg[wit={G5}]{pravartatām}
		\rdg[wit={N23}]{vibhedyate}
		\rdg[wit={J7,Gr3}]{vibhidyate}
		}//\versenr}\\!}
\end{alttlg}
%</vs1-1>
\endaltnormal
\fi


%<*vs2>
\begin{tlg}[hp04_002]
\tl{
\pada{\app{\lem[wit={Gr3,G11}]{praṇaṣṭocchvāsa}% +F
		\rdg[wit={J7,V15,J10,V3}]{pranaṣṭocchvāsa}
		\rdg[wit={P11}]{pranaṣṭosvāsa}
		\rdg[wit={N19}]{pranaṣṭauśvāsa}
		\rdg[wit={N23}]{prabhṛṣṭo\,\_\,sa}
		\rdg[wit={N3,Jyo}]{praṇaṣṭaśvāsa}% na Jyo ##?
		\rdg[wit={J5}]{praṇaṣṭabhyāsa}
		\rdg[wit={C6}]{pranaṣṭaḥ svā<<sa>>}}%
	\app{\lem[wit={N3,G11,V15,Jyo}]{niśvāsaḥ}
		\rdg[wit={J5,V3}]{niśvāsa}
		\rdg[wit={C6pc,N19,J10,P11}]{niḥśvāsaḥ} % niḥñcāsaḥ N19, °svāsaḥ P11
		\rdg[wit={C6ac,Gr3}]{niḥśvāsa}% niḥsvāsa V19
		\rdg[wit={J7}]{niśvāsāḥ}
		\rdg[wit={N23}]{niśvāsā}
		}}
\pada{\app{\lem[wit={ceteri}]{pradhvasta}% praddhasta N19; +G5,M3
		\rdg[wit={G11}]{prabhṛṣṭa}
		\rdg[wit={J10}]{pranaṣṭa}% °naṣṭaḥ V17, prā° N26.
		}%
	\app{\lem[wit={ceteri}]{viṣaya}
		\rdg[wit={G11}]{viṣayā}
		\rdg[wit={N19}]{viṣaga}}%
	\app{\lem[wit={N3,J5,V19,G11,J10,C6,V3,Jyo}]{grahaḥ}
		\rdg[wit={Gr2,E2}]{grahāḥ}
		\rdg[wit={P11}]{grataḥ}
		\rdg[wit={V15}]{jvaraḥ}
		\rdg[wit={N19}]{hvaraḥ}}/}\\+}
\tl{
\pada{\app{\lem[wit={G11,C6,V3,Jyo}]{niśceṣṭo}
		\rdg[wit={N3}]{niḥśceṣṭo}
		\rdg[wit={J5}]{niścaiṣṭo}
		\rdg[wit={Gr2,Gr3,V15}]{niśceṣṭā}
		\rdg[wit={P11}]{niḥśreṣṭo}
		\rdg[wit={N19}]{nidyeṣṭo}
		\rdg[wit={J10}]{niścalo}}
	\app{\lem[wit={N23,G11,N19,V15,J10,GrB,Jyo}]{nirvikāraś ca} % nirvikā<<ra>>ś N23, nivikāraś V15
		\rdg[wit={J7,Gr3}]{nirvikārāś ca}
		\rdg[wit={N3}]{nirvikāras tu}
		\rdg[wit={J5}]{nivikalpas tu}
		}}
\pada{\app{\lem[wit={ceteri}]{layo}
		\rdg[wit={V19}]{laye}
		\rdg[wit={Gr2,E2}]{layaṃ}}
	\app{\lem[wit={ceteri}]{jayati}
		\rdg[wit={Gr2,Gr3}]{yānti ca}}
	\app{\lem[wit={N3,J5,G11,N19,V15,GrB,Jyo}]{yoginām}% °naṃ N3
		\rdg[wit={Gr2,Gr3,J10}]{yoginaḥ}}//\versenr}
		\orgvnr{2}\\!}
\end{tlg}
%</vs2>
\commcite\newpage

%<*vs3>
\begin{tlg}[hp04_003]
\tl{\app{\lem[nolem]{}
	\rdg[wit={E2},alt={\om}]{\skp{\om}}}%
\pada{\app{\lem[wit={ceteri}]{ucchinna}
		\rdg[wit={N3,G11,V15}]{ucchinnaḥ}
		\rdg[wit={V19}]{ucchūna}% +K3,C7
		}%
	sarva\app{\lem[wit={ceteri}]{saṃkalpo}
		\rdg[wit={V19}]{saṃkalpe}}}
\pada{\app{\lem[wit={ceteri}]{niḥśeṣāśeṣa}% ni<<ḥ>>śeṣā° J10
		\rdg[wit={Gr2}]{niḥśeṣagata}
		\rdg[wit={J5,V3}]{niḥśeṣośeṣa}}%
	\app{\lem[wit={ceteri}]{ceṣṭitaḥ}
		\rdg[wit={C6}]{ceṣṭitam}
		%\rdg[wit={K3,C7}]{veṣṭitaḥ}
		\rdg[wit={V15}]{varjitaḥ}}/}\\+}
\tl{
\pada{\app{\lem[wit={N3,J5,V19,J10,V3,Jyo}]{svāvagamyo}
		\rdg[wit={G4,G11,P11}]{svāvagamya}
		\rdg[wit={C6}]{sovagamyo}
		\rdg[wit={N19}]{svāgamyo}
		\rdg[wit={V15}]{svānugamyo}
		\rdg[wit={Gr2}]{svāgate cā}
		}
		layaḥ ko'pi} % pi added in margin C7; laya kaupi P11
\pada{\app{\lem[wit={Gr1,G11,C6}]{jayatāṃ vāgagocaraḥ}% °ra J5
		\rdg[wit={N19}]{japatāṃ vāgagocara}
		\rdg[wit={V15}]{jāyatāṃ vāgagocaraḥ}
		\rdg[wit={P11}]{jāyatāṃ cāpi gaucaraḥ}
		\rdg[wit={J10,V3,Jyo}]{jāyate vāgagocaraḥ} % agocara V3
		\rdg[wit={Gr2,V19}]{manovācām agocaraḥ}% +K3,C7
		}//\versenr}
		\orgvnr{3}\\!}
\end{tlg}
%</vs3>
\commcite\newpage


%<*vs4>
\begin{tlg}[hp04_004]
\tl{\app{\lem[nolem]{}
	\rdg[wit={E2},alt={\om}]{\skp{\om}}}%
\pada{yatra % yabha N23, yanna N19
	\app{\lem[wit={ceteri},alt={dṛṣṭir}]{dṛṣṭi\skp{r}}
		\rdg[wit={N3,V15,J10}]{dṛṣṭi}
		\rdg[wit={C6}]{vṛṣṭir}
		}r layas tatra} % yayas? G11
\pada{bhūtendriya% bhute° N19
	\app{\lem[wit={N3,J5,G11,V15,V3}]{sanātanaḥ}
		\rdg[wit={P11}]{sanātana}
		\rdg[wit={N19}]{sanātanaṃ}
		\rdg[wit={Gr2,V19,J10,C6,Jyo}]{sanātanī}% +K3,C7
		}/}\\+}
\tl{
\pada{\app{\lem[wit={N3,Gr2,V19},alt={syāc chaktir/°tiḥ}]{syāc chakti\skp{r}}% °kti<<ḥ>> N23; +K3,C7
		\rdg[wit={J5}]{syāt saktir}
		\rdg[wit={G11,N19,J10,GrB,Jyo}]{sā śaktir}% śakti P11,G11; +F
		\rdg[wit={V15}]{sa śaktir}}%
	\app{\lem[wit={N3,J5,G11,J10,GrB,Jyo},alt={jīva}]{\skm{r }jīva}
		\rdg[wit={Gr2,V19}]{sarva}% +K3,C7
		\rdg[wit={N19,V15}]{bhāva}}%
	\app{\lem[wit={ceteri}]{bhūtānāṃ}
		\rdg[wit={N23}]{bhūtānī}
		\rdg[wit={N19}]{bhūnāṃ}}}
\pada{\app{\lem[wit={N3,G4,Gr2,J10,C6,V3},alt={dṛṣṭir}]{dṛṣṭi\skp{r}}
		\rdg[wit={J5,V19,G11,P11}]{dṛṣṭi}% +K3,C7
		\rdg[wit={N19,V15}]{dṛṣṭe}
		\rdg[wit={Jyo}]{dve a°}}%
	\app{\lem[wit={G11,N19,P11,V3},alt={lakṣ(y)e layaṃ gatā}]{\skm{r }lakṣye layaṃ gatā}% lakṣe P11,V3,N19
		\rdg[wit={J5}]{lakṣe la(!) gatā}
		\rdg[wit={N3}]{lakṣe layaṃ gatāḥ}
		\rdg[wit={G4}]{lakṣy[e] layaṃ gataḥ}
		\rdg[wit={J10,Jyo}]{lakṣye layaṃ gate}
		\rdg[wit={V15}]{lakṣaṃ layaṃ gatau}
		\rdg[wit={J7}]{lakṣe na saṃgatā}
		\rdg[wit={N23}]{lakṣana saṃgatā}
		\rdg[wit={V19}]{lakṣeṇa saṃgatā}%lakṣyeṇa K3,C7
		\rdg[wit={C6}]{gacchel layaṃ gate}}//\versenr}%
		%\myfnx{After this verse, \getsiglum{Jyo} has \ref{layo}.}
		\label{yatradrsti}
		\orgvnr{4}
		\\!}
\end{tlg}
%</vs4>
\commcite\newpage

%\Anm{\getsiglum{Jyo} has \ref{layo} \textit{layo laya iti} here}


%<*vs5>
\begin{tlg}[hp04_005]
\tl{
\pada{\app{\lem[nolem]{\skp{pāda a}}
		\rdg[wit={J5,V3},alt={\om}]{\skp{\om}}}%
	vedaśāstra\app{\lem[wit={N3,G4,G11,N19,V15,J10,P11,C6,Jyo}]{purāṇāni}
		\rdg[wit={N23}]{purāṇādyāḥ}
		\rdg[wit={J7}]{puraṇādyāḥ}
		\rdg[wit={E2}]{purāṇaughāḥ}% +K3,C7
		\rdg[wit={V19}]{purāṇaiś ca}}}
\pada{\app{\lem[nolem]{\skp{pāda b}}
		\rdg[wit={J5,V3},alt={\om}]{\skp{\om}}}%
	\app{\lem[wit={ceteri}]{sāmānya}% °yaṃ G11, °nyā E2
		\rdg[wit={C6}]{samāni}}% 
	\app{\lem[wit={ceteri}]{gaṇikā} % gatikā N26
		\rdg[wit={V19}]{gaṇivā}} iva/}
		\\+}
\tl{
\pada{\app{\lem[nolem]{\skp{pāda c}}
	\rdg[wit={V3},alt={\om}]{\skp{\om}}}%
\app{\lem[wit={ceteri}]{ekaiva}
		\rdg[wit={E2}]{idaṃ tu}
		} śāmbhavī % sāṃ° J10
	\app{\lem[wit={ceteri}]{mudrā}
		\rdg[wit={V15}]{māyā}
		\rdg[wit={J10}]{vidyā}}}
\pada{\app{\lem[nolem]{\skp{pāda d}}
	\rdg[wit={V3},alt={\om}]{\skp{\om}}}%
\app{\lem[wit={N3,J5,Gr2,P11,C6,Jyo}]{guptā kulavadhūr iva}% vadhū iva P7
		\rdg[wit={J10}]{gopyā kulavadhūr iva}
		\rdg[wit={G11,N19,V15},post=\texteng{(cf.\,\ref{gopita}d)}]{sarvatantreṣu gopitā}
		\rdg[wit={Gr3}]{sarvatantreṣu gopitā rakṣaṇīyā prayatnena guptā kulavadhūr iva}}//\versenr}\label{vedasastra}
		\orgvnr{5}\\!}
\end{tlg}
%</vs5>
\commcite\newpage

% atha śāmbhavī (Jyo)


%<*vs6>
\begin{tlg}[hp04_006]
\tl{\app{\lem[nolem]{}
	\rdg[wit={N19,V15},alt={\om}]{\skp{\om}}}% \anm{eye-skip?}
\pada{anta\app{\lem[wit={J5,C6ac,Gr2,J10,V3,Jyo},alt={lakṣ(y)aṃ}]{\skm{r }lakṣyaṃ} % antalakṣaṃ P7; °lakṣaṃ V3,N23,J10; °lakṣyaṃ Jyo,C6ac
		\rdg[wit={V19}]{lakṣā} % lakṣyā K3
		\rdg[wit={E2}]{lakṣyo}
		\rdg[wit={N3,C6pc,G11,P11}]{lakṣ(y)a}} % y C7,P11,C6
	\app{\lem[wit={ceteri},alt={bahir}]{bahi\skp{r}}% N3,J5,Gr2,Gr3,G11,GrB,Jyo
		\rdg[wit={J10}]{mano}}%
	\app{\lem[wit={ceteri},alt={dṛṣṭir}]{\skm{r }dṛṣṭi\skp{r}}
		\rdg[wit={J5,V19,G11,J10,V3}]{dṛṣṭi}}}%
\pada{\app{\lem[wit={ceteri},alt={nimeṣonmeṣa}]{\skm{r }nimeṣonmeṣa}% N3,J5,J7,Gr3,G11,J10,C6,V3,Jyo
		\rdg[wit={N23,P11}]{nirmiṣonmeṣa}% °ṣya N23
		}\app{\lem[wit={ceteri}]{varjitā}
		\rdg[wit={E2,P11}]{varjjitaḥ}}/}\\+}
\tl{
\pada{\app{\lem[wit={N3,G11,P11,C6,Jyo}]{eṣā sā}% sāṃ P11
		\rdg[wit={J5}]{eṣāsau}
		\rdg[wit={V3}]{eṣā hi}
		\rdg[wit={J10}]{eṣā tu}
		\rdg[wit={E2}]{eṣā vai}
		\rdg[wit={Gr2,V19}]{saiṣā tu}% +K3,C7
		}
		śāmbhavī mudrā} % sāṃbhavī J10
\pada{\app{\lem[wit={ceteri}]{sarvatantreṣu}% sarve J10 % N3,J5,Gr2,E2,G11,J10,GrB
		%\rdg[wit={K3,C7}]{sarvaśāstreṣu}
		\rdg[wit={V19}]{sarvatantreṣu śastreṣu}
		\rdg[wit={Jyo}]{vedaśāstreṣu}}
		gopitā//\versenr}\label{gopita}
		\orgvnr{6}\\!}
\end{tlg}
%</vs6>
\commcite\newpage
		%\myfn{The omission was probably caused by haplography. \getsiglum{L2} inserts the following passage after 4.20a: \devnote{eṣā tu śāṃbhavī vidyā gopā ca kulavadhūr iva/ aṃtarlakṣamanodṛṣṭi nimeṣonmeṣonmeṣa(!)varjito/}}\\!

%<*vs7>
\begin{tlg}[hp04_007]
\tl{
\pada{anta\app{\lem[wit={N3,Gr3,G11,J10,P11,C6,Jyo},alt={lakṣya}]{\skm{r}lakṣya}
		\rdg[wit={J5,Gr2,N19,V15,V3}]{lakṣa}}%
		vilīnacittapavano yogī
	\app{\lem[wit={ceteri}]{yadā}
		\rdg[wit={J10}]{yathā}
		\rdg[wit={Gr1,N19}]{sadā}% +F
		} vartate}\\+} % ##?
\tl{
\pada{\app{\lem[wit={ceteri}]{dṛṣṭyā}
		\rdg[wit={J10}]{dṛṣṭvā}
		\rdg[wit={P11}]{dṛṣyā}
		\rdg[wit={V3}]{dṛśyā}}
	niścala\app{\lem[wit={ceteri}]{tārayā}
		\rdg[wit={P11}]{tālayā}
		\rdg[wit={N23}]{tāra}}
	\app{\lem[wit={ceteri},alt={bahir}]{bahi\skp{r}}% bahidharaḥ! G11
		\rdg[wit={N23}]{hir}}%
	\app{\lem[wit={Gr1,G11,V15,J10,GrB,Jyo},alt={adhaḥ}]{\skm{r }adhaḥ}
		\rdg[wit={N19}]{adhraḥ}
		\rdg[wit={Gr2,Gr3}]{asau}}
	\app{\lem[wit={J5,Gr3,G11,N19,V15,Jyo}]{paśyann apaśyann api}
		\rdg[wit={N3}]{paśyann apaśyann ivā}% illeg. G4 ##?
		\rdg[wit={Gr2}]{paśyan na paśyaty api}
		\rdg[wit={J10}]{paśyann api}
		\rdg[wit={P11,V3}]{paśyan na paśyet tataḥ}% tata V3; paśyet tadā F
		\rdg[wit={C6}]{paśyen na paśyet tataḥ}}/}\\+}
\tl{
\pada{\app{\lem[wit={ceteri}]{mudreyaṃ}
		\rdg[wit={V15}]{mudre}} khalu % khaluṃ N23
	\app{\lem[wit={N3,J5,J10,P11,V3}]{khecarī} % = source; +K1,P6
		\rdg[wit={Gr2,Gr3,G11,N19,V15,C6,Jyo}]{śāmbhavī}% +F
		}
	\app{\lem[wit={ceteri}]{bhavati sā}% +G5,M3
		\rdg[wit={V3}]{bhavati}
		\rdg[wit={G11}]{°ti kathitā}}
	\app{\lem[wit={N3,J5,Gr3,N19,V15,V3},alt={yuṣmat}]{yuṣma\skp{t}}% +J10pc
		\rdg[wit={J7}]{<<yu>>ṣmat}
		\rdg[wit={J10}]{yuṣmān}
		\rdg[wit={N23}]{puṣpat}
		\rdg[wit={G11,C6}]{yasya}
		\rdg[wit={P11}]{yāsya}
		\rdg[wit={Jyo}]{labdhā}}tprasādā%d
	\app{\lem[wit={Gr2,V19,V15,J10ac,P11,V3},alt={guro}]{\skm{d }guro}% +K3,C7
		\rdg[wit={E2,G11,N19,J10pc,C6,Jyo}]{guroḥ}
		\rdg[wit={N3}]{gurau}
		\rdg[wit={J5}]{gure}}}\\+}
\tl{
\pada{\app{\lem[wit={ceteri}]{śūnyāśūnya}% °nyaṃ N19
	\rdg[wit={C6}]{śūnyāc chūnya}}%
	\app{\lem[wit={ceteri}]{vivarjitaṃ}
		\rdg[wit={N23}]{vivarjite}
		\rdg[wit={V19}]{vivarjiti}
		\rdg[wit={J5}]{vivarjito}
		\rdg[wit={Jyo}]{vilakṣaṇaṃ}}
	\app{\lem[wit={ceteri}]{sphurati}
		\rdg[wit={V19}]{spharati}}
	\app{\lem[wit={ceteri},alt={yat}]{ya\skp{t}}% pat N23
		\rdg[wit={V3}]{ya}
		\rdg[wit={V19}]{[pta]t}
		\rdg[wit={N3,Jyo}]{tat}% ##?
		\rdg[wit={J5}]{ttat}}t
	tattvaṃ \app{\lem[wit={ceteri}]{padaṃ}
		\rdg[wit={G11,N19},alt={\om}]{\skp{\om}}} śāmbhavam//\versenr}
		\label{antarlaksya}
		\orgvnr{7}\\!}
\end{tlg}
%</vs7>
\commcite\newpage


%<*vs8>
\begin{tlg}[hp04_008]
\tl{\app{\lem[nolem]{}
	\rdg[wit={N19,V15,J10},alt={\om}]{\skp{\om}}}% om. G3
\pada{śrīśāmbha\app{\lem[wit={N3,J7,Gr3,Jyo},alt={°vyāś ca khecaryā}]{\skp{°}vyāś ca khecaryā} % khecarayā J7, °caryyāḥ V19
		\rdg[wit={G11}]{°vāś ca khecaryā}
		\rdg[wit={N23}]{°vyāḥ khecaryā\,\_}
		\rdg[wit={GrB}]{°vyā(ḥ) khecaryāś ca} % vyā V3,P6
		\rdg[wit={J5}]{°vyā khecaryā}
		\rdg[wit={G4}]{°vavyā khecaryā}}}
\pada{\app{\lem[wit={P11}]{avasthāyām abhedatā}% +K1 (ana°)
		\rdg[wit={C6}]{hy avasthāyām abhedataḥ}
		\rdg[wit={N3,G11}]{avasthāyāṃ na bhedataḥ}% bhedatā M3
		\rdg[wit={G4}]{avasthāyā na bhedataḥ}% +G5?
		\rdg[wit={J5}]{avasthāyasya bhedataḥ}
		\rdg[wit={Jyo}]{avasthādhāmabhedataḥ}
		\rdg[wit={V3}]{avasthāyāṃ ca bhedatā}
		\rdg[wit={Gr2}]{avasthā ca na bhedataḥ\,(°naḥ \getsiglum{N23})}
		\rdg[wit={Gr3}]{avasthā balabhedataḥ}}\marma//\versenr}%
		\myfn{After this verse, \getsiglum{Jyo} has an additional line: \devnote{bhavec cittalayānandaḥ śūnye citsukharūpiṇi/}} % +M1,G8
		\orgvnr{8}\\!}
\end{tlg}
%</vs8>
\commcite\newpage

% J5 śobhavyā khecaryā avaschāyasya bhedataḥ
% G4 śrīśāṃbhavavyā khecaryā avasthāyā {{tavalabdhi}} na bhedataḥ
% K1a! śrīśāṃbhavyāś ca khecarya anasthāyām abhedatā
% K1b! śrīśāṃbhavyāḥ khecaryāś ca avasthā tu na bhedataḥ
% M1 śrīśāṃbhavyāś ca khecar.ā avasthāyām abh[e]dataḥ
% M3 śriśāṃbhavyāś ca khecaryā avastāyān na bhedatā
% G5 śriśāṃbhavyāś ca khecaryā avastāyā na bhedataḥ
% G7 śāṃbaryāś ca khecarya atastā naiva bhedataḥ
% P6 śrīśaṃbhavyā khecaryāś ca avasthā tu na bhedataḥ
% N22 śrīśaṃbhavyā khecaryāś ca avasthā <!> na bhedataḥ
% IFP: śriśāṃbhavyāś ca khecaryā avasthā sthānabhedataḥ


%<*vs9>
\begin{tlg}[hp04_009]
\tl{\app{\lem[nolem]{}
	\rdg[wit={Gr1,Gr2,V19,C7}]{\incl}%
 	\rdg[wit={Gr2,V19},alt=\textapp{found after \expnr{X4.42}}]{\skp{found after \expnr{X4.42}}}}%
\pada{\app{\lem[resp=emend]{pātāle yad viśati}
		\rdg[wit={N3,J5}]{pātāle yadvitaya}% °tayas G11
		\rdg[wit={G4}]{pātāḷe yadvita\,..}
		\rdg[wit={Gr2}]{pātālād yad viśati}
		\rdg[wit={V19,C7}]{pātālād vā viyati}
		}
	\app{\lem[wit={J5}]{suṣiraṃ}% sukhiraṃ K1,P6
		\rdg[wit={N3}]{suśiraṃ}
		\rdg[wit={N23}]{śikhiraṃ}
		\rdg[wit={J7}]{śikharaṃ}% +K3
		\rdg[wit={V19,C7}]{śikhare}
		} merumūle
	\app{\lem[wit={N3}]{tad asmin}% +G11
		\rdg[wit={J5}]{yadismi}
		\rdg[wit={N23}]{tasti}
		\rdg[wit={J7}]{tad asti}% +K1,P6
		\rdg[wit={V19}]{tadāstā}
		\rdg[wit={C7}]{tad āste}% +K3
		}}\\+}
\tl{
\pada{tattvaṃ caitat pravadati % pravahati C7
	\app{\lem[wit={N3,Gr2}]{sudhīs tan mukhaṃ}
		\rdg[wit={C7}]{sudhīḥ saṃmukhaṃ}% +K3
		\rdg[wit={J5}]{sudhī sanmukhaṃ}
		\rdg[wit={V19}]{susaṃmukhaṃ}} 
		nimnagānām/}\\+} % ninmaganāṃ V19
\tl{
\pada{candrā%t % caṃdrā N3
	\app{\lem[wit={Gr2},alt={sāraḥ}]{\skm{t }sāraḥ}% sāra K1, sāraṃ P6
		\rdg[wit={V19,C7}]{srāvaḥ}
		\rdg[wit={N3,J5}]{sāro}}
	\app{\lem[wit={N23,C7}]{sravati}% +K1,P6,G11,K3
		\rdg[wit={V19}]{śravati}
		\rdg[wit={J7}]{savati}
		\rdg[wit={N3}]{grasati}
		\rdg[wit={J5},alt={\om}]{\skp{\om}}}
	\app{\lem[wit={N3,J5,N23,V19,C7},alt={vapuṣas}]{vapuṣa\skp{s}}
		\rdg[wit={J7}]{puruṣas}}s
	tena mṛtyur narāṇāṃ}\\+} % mūtyur N23
\tl{
\pada{\app{\lem[wit={Gr1,J7,V19,C7},alt={taṃ badhnīyāt}]{taṃ badhnīyā\skp{t}}
		\rdg[wit={N23}]{tadvahmaṃpāt}}%
	\app{\lem[wit={N3,J5},alt={sukaraṇamṛdā}]{\skm{t }sukaraṇamṛdā}
		\rdg[wit={G4}]{sukaraṇāmudā}
		\rdg[wit={J7,C7}]{svakaraṇamṛdā}% sakaraṇamṛtā K1,P6
		\rdg[wit={N23}]{svakaraṇaimṛdā}
		\rdg[wit={V19}]{svakaraṇamṛjā}
		} % sukaraṇam amṛtaṃ G11 (unm.)
		nānyathā
	\app{\lem[wit={N3,J7,C7}]{kāyasiddhiḥ}% +K1,G11
		\rdg[wit={N23}]{kāyaḥ siddhiḥ}
		\rdg[wit={J5,G4,V19}]{kāryasiddhi(ḥ)}% V19 om. h; +P6
		}//\versenr}
	%\sgwit{Gr1,Gr2,V19,C7} % Not in E2!
	\anm{\textrightarrow\ \manuref{3.48*1}}\label{patala}
	\orgvnr{9}\\!}
	%\anm{\getsiglum{Gr2,V19,K3,C7}; the other mss have \ref{ardhogha} instead of this verse}
	%\NotIn{G11,N19,V15,J10,GrB,Jyo}
\end{tlg}
%</vs9>
\commcite\newpage
%\ \newpage


%<*vs10>
\begin{tlg}[hp04_010]
\tl{\app{\lem[nolem]{}
 \rdg[wit={Gr1}]{\only}}%}
\pada{yat kiṃcit sravate candrād}
\pada{amṛtaṃ divyarūpiṇaḥ/}\\+} % N24 has °rūpiṇaḥ too.
\tl{
\pada{tat sarvaṃ grasate sūryas}
\pada{tena piṇḍaṃ jarāyutam//\versenr} % piṇḍo jarāyutaḥ N24
%\sgwit{Gr1} 
\anm{\textrightarrow\ \manuref{3.77*1}}
\orgvnr{10}\\!} % almost the same J5; G4 damaged
\end{tlg}
%</vs10>
\commcite%\newpage


%<*vs11>
\begin{tlg}[hp04_011]
\tl{\app{\lem[nolem]{}
 \rdg[wit={Gr1}]{\only}}%}
\pada{tatrāsti karaṇaṃ divyaṃ}
\pada{sūryasya \app{\lem[resp=emend]{mukhabandhanam}
	\rdg[wit={N3,J5}]{paribandhanaṃ}
	\rdg[wit={G4},alt={\illeg}]{\skp{\illeg}}}/}\\+} % paripaṃthi ca N24
\tl{
\pada{gurūpadeśato jñeyaṃ}
\pada{na tu śāstrārthakoṭibhiḥ//\versenr}
%\sgwit{Gr1} 
\anm{\textrightarrow\ \manuref{3.77*2}}
\orgvnr{11}\\!} % almost the same J5; G4 damaged
\end{tlg}
%</vs11>
\commcite\newpage


%\Anm{The numbering of the last three verses will be changed to 4.9--11 later.}


\iffalse
\startaltrecension
%<*vs11-1>
\begin{alttlg}[hp04_011_1]
\tl{\app{\lem[nolem]{}
	\rdg[wit={V19,E2}]{\also}}%
\pada{\app{\lem[nolem]{\skp{pāda a}}
	\rdg[wit={N23,J7}]{\also}}%
\app{\lem[wit={J7,E2,V15,P11,Jyo}]{tāre}
		\rdg[wit={V19,V3}]{tāra}% E4
		%\rdg[wit={K3,C7}]{tāraṃ}
		\rdg[wit={C6}]{tārāṃ}
		\rdg[wit={J10}]{tārā}% +F
		\rdg[wit={N19}]{tāvad}
		\rdg[wit={N23}]{vāre}
		\rdg[wit={G11}]{kalāṃ}
		\rdg[wit={G5}]{kalā}
		}
	\app{\lem[wit={Gr2,E2,G11,G5,V15,C6,Jyo}]{jyotiṣi}% jotiṣi G5
		\rdg[wit={P11}]{jyotiṣīṃ}
		\rdg[wit={V3}]{jyotīṣa}
		\rdg[wit={V19}]{jyotiso}
		\rdg[wit={N19}]{yotiṣi}
		\rdg[wit={J10}]{jyotiṣu}
		}
	\app{\lem[wit={ceteri}]{saṃyojya}
		\rdg[wit={J10}]{saṃyojyā}
		\rdg[wit={N23}]{samojyaṃ}
		\rdg[wit={V19}]{jojya}}}
\pada{\app{\lem[nolem]{\skp{pāda b}}
	\rdg[wit={N23,J7}]{\also}}%
kiṃci%d
	\app{\lem[wit={G11,G5,V15,GrB,Jyo},alt={unnamayed}]{\skm{d }unnamaye\skp{d}} % °yet* V3; +M3,P6,N22
		\rdg[wit={N23,E2}]{uccālayed}
		\rdg[wit={J7}]{uccalayed}
		\rdg[wit={J10}]{uccārayed}% +K1b
		\rdg[wit={V19}]{uccācayed}
		\rdg[wit={N19}]{uṣṭānnama}}%
	\app{\lem[wit={ceteri},alt={bhruvau}]{\skm{d }bhruvau}% bhuvau? P11,J10
		\rdg[wit={N23}]{bhūvo<<ḥ>>}}/}\\+}
\tl{
\pada{\app{\lem[wit={E2,G11,G5,N19,V15,P11,V3},alt={pūrvayogasya mārgo'yam}]{pūrvayogasya mārgo'ya\skp{m}}% +K3,C7
		\rdg[wit={C6}]{pūrvayogasya mārgeṇa}
		\rdg[wit={J10}]{sūryayogasya mārge ca}
		\rdg[wit={V19}]{pūrvayogasya māhātmyam}
		\rdg[wit={Jyo}]{pūrvayogaṃ mano yuñjann}
		}}%
\pada{\app{\lem[wit={Gr3,G11,G5,N19,V15,P11,V3,Jyo},alt={unmanī}]{\skm{m }unmanī} % yaṃmunmanī N19
		\rdg[wit={C6}]{hy unmanī}
		\rdg[wit={J10}]{yunmanī}
		\rdg[wit={G11}]{kiṃcid un°}
		}%
	\app{\lem[wit={G11,P11,Jyo}]{kārakaḥ kṣaṇāt}
		\rdg[wit={C6}]{kārakakṣaṇāt}
		\rdg[wit={N19}]{kārakaṃ kṣaṇāt}
		\rdg[wit={V3}]{kāraṇaḥ kṣaṇāt}
		\rdg[wit={G5}]{kāraṇaṃ kṣaṇāt}
		\rdg[wit={Gr3,V15}]{karaṇaṃ kṣaṇāt}% +M3
		\rdg[wit={J10}]{kāralakṣaṇam}% +K1
%		\rdg[wit={G11}]{°namayet kṣaṇāt | unmanīkārakaḥ}
		}//\versenr}
	\\!} % om. +P6,N22
%\sgwit{Gr3,N19,V15,C8,G11,G5,N9,V17,J10,V3}
\end{alttlg}
%</vs11-1>

%<*vs11-2>
\begin{alttlg}[hp04_011_2]
\tl{\app{\lem[nolem]{}
	\rdg[wit={V19,E2}]{\also}}%
\pada{keci%d
	\app{\lem[wit={ceteri},alt={āgama}]{\skm{d }āgama}
		\rdg[wit={G11,G5}]{nigama}}%
	\app{\lem[wit={ceteri}]{jālena}
		\rdg[wit={J10}]{yogena}
		}}
\pada{keci%n % cecin N19, keccin V19
	\app{\lem[wit={N19,J10,P11,C6,Jyo},alt={nigama}]{\skm{n }nigama}% =source
		\rdg[wit={Gr3,V3}]{niyama}
		\rdg[wit={V15}]{nima}
		\rdg[wit={G11,G5}]{āgama}
		}%
	\app{\lem[wit={G11,G5,N19,J10,P11,C6,Jyo}]{saṃkulaiḥ}% saṃkulai P11
		\rdg[wit={V15,V3}]{saṃkule}
		\rdg[wit={E2}]{saṃkulāḥ}% +K3,C7
		\rdg[wit={V19}]{saṃkulā}
		}/}\\+}
\tl{
\pada{kecit tarkeṇa muhyanti} % ke<ci>nnarkkeṇa N19, tarkena E2
\pada{naiva jānanti tārakam//\versenr} % kārakam G5
%\myfn{Pādas ab and cd are transposed in \getsiglum{V19} and the correct order is indicated by a small number 1 and 2 above the hemistiches.}
\label{kecid}
\\!} % C6?
\end{alttlg}
%</vs11-2>
%\sgwit{Gr3,G11,G5,N19,V15,N9,V17,J10,GrB}


%\startaltnormal
%<*vs11-3>
\begin{alttlg}[hp04_011_3]
\tl{\app{\lem[nolem]{}
	\rdg[wit={N23,J7,V19},postwit=\texteng{(found after \ref{antarlaksya})}]{\also}}% Not in E2!
\pada{\app{\lem[wit={ceteri}]{ardhodghāṭita}
		\rdg[wit={P11}]{arddhoghāṭita}
		\rdg[wit={N23}]{arddhocchā[d]ita}
		\rdg[wit={Jyo}]{ardhonmīlita}}%
	\app{\lem[wit={V19,V15,Jyo}]{locanaḥ}
		\rdg[wit={Gr2,G11,G5,N19,J10,GrB}]{locana}}
	\app{\lem[wit={ceteri}]{sthira}
		\rdg[wit={N23}]{sthila}}manā
	nāsāgradatte% nāśā° N19, nāśādagra° V17
	\app{\lem[wit={ceteri},alt={°kṣaṇaḥ/-aś}]{\skp{°}kṣaṇaḥ\skp{/-aś}}% ś P11,C6,V19,J10
		\rdg[wit={N23,V3}]{°kṣaṇāś}% °āḥś V3
		\rdg[wit={N19}]{°kṣaṇaṃ}}}\\+}
\tl{
\pada{\app{\lem[wit={ceteri},alt={candrārkāv}]{candrārkā\skp{v}}
		\rdg[wit={V3}]{cāndrārkāv}
		\rdg[wit={J10}]{candrārkau}}%
	\app{\lem[wit={ceteri},alt={api}]{\skm{v }api}% N23,V19,G11,G5,V15,GrB,Jyo
		\rdg[wit={J7}]{avi}
		\rdg[wit={N19}]{aca}
		\rdg[wit={J10}]{ca vi°}} līnatā%m
	\app{\lem[wit={G11,Jyo},alt={upanayan}]{\skm{m }upanaya\skp{n}}% M1,M3,F
		\rdg[wit={G5}]{apanayan}
		\rdg[wit={Gr2,V19,N19,V15}]{upanayen}% upanaye V19,N19,P6; eva nayet G7; +K1
		\rdg[wit={GrB}]{upagatau}% = source
		\rdg[wit={J10}]{gatau}}%
	\app{\lem[wit={ceteri},alt={niṣpanda}]{\skm{n }niṣpanda} % nispanda K3,C7,Jyo, niṣyaṃda? N19
		\rdg[wit={P11}]{nirvyaṃda}
		\rdg[wit={G5}]{diṣyanda}
		\rdg[wit={J10}]{nikṣipya}}%
	\app{\lem[wit={G11,G5}]{bhāvāntare}% +C7
		\rdg[wit={N23,V19}]{bhāvo'ntare}% bhāvottaraḥ P6,F
		\rdg[wit={J7}]{bhāvotare}
		\rdg[wit={J10}]{bhāsoṃtare}
		\rdg[wit={V15}]{bāṣpaṃ tataḥ}
		\rdg[wit={N19}]{vāpyaṃ tataḥ}
		\rdg[wit={C6}]{rūpaṃ tataḥ}
		\rdg[wit={P11}]{rūpaṃ tanu}% ##
		\rdg[wit={V3}]{rūpatanu}
		\rdg[wit={Jyo}]{bhāvena yaḥ}}/}\\+}
\tl{
\pada{jyotī% jyoti P11, jyoṃtī J7, yotī N19, jyotī{{kha}} J10
	\app{\lem[wit={ceteri},alt={rūpam}]{rūpa\skp{m}}
		\rdg[wit={N19,V15}]{rūpa}
		\rdg[wit={J7}]{yatsyam}}%
	\app{\lem[wit={ceteri},alt={aśeṣa}]{\skm{m }aśeṣa}% aśaiṣa P11
		\rdg[wit={N19,V15}]{viśeṣa}}%
	\app{\lem[wit={ceteri}]{bāhyarahitaṃ}
		\rdg[wit={Jyo}]{bījam akhilaṃ}}
	\app{\lem[wit={ceteri}]{dedīpya}
		\rdg[wit={N23}]{devadīpya}}mānaṃ paraṃ}\\+} % puraṃ G5
\tl{
\pada{tattvaṃ % tatva V3
	\app{\lem[wit={ceteri},alt={tat}]{ta\skp{t}}
		\rdg[wit={J10}]{yac}}%
	\app{\lem[wit={Gr2,V19,Jyo},alt={padam eti}]{\skm{t }padam eti}% +K1,P6
		\rdg[wit={G11,G5,GrB}]{param eti}% +F
		\rdg[wit={N19,V15}]{param asti}% +HTK
		\rdg[wit={J10}]{carama}}
	\app{\lem[wit={ceteri}]{vastu}% +M1,M3,G7,K1,P6
		\rdg[wit={N23}]{vasta}
		\rdg[wit={P11,V3}]{yastu}
		\rdg[wit={C6}]{yat tu}} 
		paramaṃ % param avācyaṃ J7ac
	\app{\lem[wit={ceteri}]{vācyaṃ} % vācya V19
		\rdg[wit={N23}]{vāpyaṃ}} ki%m
	\app{\lem[wit={ceteri},alt={atrādhikam}]{\skm{m }atrādhikam}% °dikam G5
		\rdg[wit={N23}]{andrādhikaṃ}
		\rdg[wit={V19}]{atrāsanaṃ}}//\versenr}\label{ardhogha} 
	\\!}
\end{alttlg}
%</vs11-3>
%\endaltnormal


%\Anm{\getsiglum{N19,V15,J10,V3} have Vulg 4.42--65 about Khecarīsamādhi here}
\Anm{The following verses are not found in \getsiglum{Gr1,Gr2,Gr3}, but in \getsiglum{G11,N19,V15,J10,GrB,Jyo}}

%\startaltrecension
%\teimute{\small}
%<*vs11-4>
\begin{alttlg}[hp04_011_4]
\tl{
\pada{\app{\lem[wit={G11,G5,N19,V15,GrB,Jyo}]{divā na}
		\rdg[wit={J10}]{vāsare}} pūjayel liṅgaṃ}
\pada{\app{\lem[wit={N19,P11}]{rātrau naiva ca pūjayet}% +G7
		\rdg[wit={G11,C6,V3}]{rātrau naiva prapūjayet}
		\rdg[wit={G5,J10,Jyo}]{rātrau caiva na pūjayet} % pūyet J10; vāpi na M3
		\rdg[wit={V15}]{rātrau liṃgaṃ na pūjayet}}/}\\+}
\tl{
\pada{\app{\lem[wit={G11,G5,N19,V15,J10,GrB}]{satataṃ}
		\rdg[wit={Jyo}]{sarvadā}} pūjayel liṅgaṃ} % pū<<ja>>yel J10
\pada{\app{\lem[wit={Jyo}]{divārātrinirodhataḥ}
		\rdg[wit={G11,G5,P11,V3}]{divārātraṃ na pūjayet}% +M1,M3,F
		\rdg[wit={N19,V15,J10}]{divārātrau na pūjayet}
		\rdg[wit={C6}]{divārātrau ca pūjayet}% ca varjayet G3,G7
		}//\versenr}\\!}
\end{alttlg}
%</vs11-4>

%<*vs11-5a>
\begin{altava}[hp04_011_5a]
\app{\lem[nolem]{}
 \rdg[wit={P11,C6,Jyo}]{\only}}%
 atha \app{\lem[wit={C6,Jyo}]{khecarī}
	\rdg[wit={P11}]{khecarīsamādhiḥ}% +F
	}/ %\sgwit{P11,C6,Jyo}
\end{altava}
%</vs11-5a>
%<*vs11-5>
\begin{alttlg}[hp04_011_5]
\tl{\app{\lem[nolem]{}
	\rdg[wit={Jyo},alt={\om}]{\skp{\om}}}%
\pada{\app{\lem[wit={G11,G5,N19}]{suṣiro}% +F
		\rdg[wit={C6}]{sukhiro}
		\rdg[wit={P11}]{susthiro}
		\rdg[wit={J10,V3}]{sukhiraṃ}
		\rdg[wit={V15}]{dṛṅmukhaṃ}}
	jñāna\app{\lem[wit={G11,G5,N19,P11,C6}]{janakaḥ}% +F
		\rdg[wit={J10,V3}]{janakaṃ}
		\rdg[wit={V15}]{jaṃnakaṃ}
		}}
\pada{pañcasrotaḥ% srota G5
%\app{\lem[wit={G11,G5,V15,P11,C6}]{srotaḥ}
%	\rdg[wit={N19,J10,V3}]{śrotaḥ}}%
	\app{\lem[wit={G11,G5,N19,P11,C6}]{samanvitaḥ}% +F
		\rdg[wit={V3}]{samanvita}
		\rdg[wit={V15}]{samanvitaṃ}% J10pc
		\rdg[wit={J10}]{samanvite}}/}\\+}
\tl{
\pada{tiṣṭhate khecarī mudrā} % khacarī G11
\pada{\app{\lem[wit={J10}]{tasmin śūnye}% +F
		\rdg[wit={G11,G5,V15,P11,C6}]{tasmāc chūnye}
		\rdg[wit={N19}]{satyaṃ tatra}
		\rdg[wit={V3},alt={\om},post=\texteng{(eye-skip?)}]{\skp{\om}}} % cf. V3 resumes with tasmin!
	\app{\lem[wit={G11,G5,V15,J10,P11,C6}]{nirañjane}
		\rdg[wit={N19}]{na saṃśayaḥ}
		\rdg[wit={V3},alt={\om}]{\skp{\om}}}//\versenr}
	%\anm{=\,\manuref{3.48*1}}
	\myfn{\getsiglum{J10,V3,Jyo} have this verse after 3.48. 
	Here, \getsiglum{J10,V3} have it for the second time, while \getsiglum{Jyo} omits it.}
	\\!}
\end{alttlg}
%</vs11-5>
%\myfn{\getsiglum{L2} omits the 2nd half of this verse and the 1st half of the next verse (prob. by haplography) and adds them after the 1st half of 4.25*3.}

%<*vs11-6>
\begin{alttlg}[hp04_011_6]
\tl{
\pada{\app{\lem[nolem]{\skp{pāda a}}
	\rdg[wit={V3},alt={\om}]{\skp{\om}}}% % J6 has this omission too!!
savyadakṣiṇa\app{\lem[wit={V15,P11,C6}]{nāḍī}
		\rdg[wit={G11,G5,N19,J10,Jyo}]{nāḍi}
		}stho}
\pada{\app{\lem[nolem]{\skp{pāda b}}
	\rdg[wit={V3},alt={\om}]{\skp{\om}}}% % J6 has this omission too!!
\app{\lem[wit={G11,G5,N19,J10,P11,C6,Jyo}]{madhye}
		\rdg[wit={P11}]{madhyaṃ}
		\rdg[wit={V15}]{madhya}
		}
		\app{\lem[wit={G11,N19}]{calati}
		\rdg[wit={G5,P11,C6,Jyo}]{carati}%  <<carati>> C6
		\rdg[wit={V15}]{carita}
		\rdg[wit={J10}]{vahati}
		}
		mārutaḥ/}% ḥ om. N19
		\\+}
\tl{
\pada{\app{\lem[nolem]{\skp{pāda c}}
	\rdg[wit={V3},alt={\om}]{\skp{\om}}}% % J6 has this omission too!!
\app{\lem[wit={G11,G5,N19,V15,J10,P11,C6,Jyo}]{tiṣṭhate khecarī mudrā} % tiṣṭhajña? N19
		\rdg[wit={V3},alt={\om}]{\skp{\om}}}}
\pada{\app{\lem[wit={G11,G5,V15,GrB,Jyo}]{tasmin sthāne}
		\rdg[wit={N19}]{satyaṃ tatra}
		\rdg[wit={J10}]{tatra satyaṃ}}
		na saṃśayaḥ//\versenr}\\!}
\end{alttlg}
%</vs11-6>

%<*vs11-7>
\begin{alttlg}[hp04_011_7]
\tl{\app{\lem[nolem]{}
	\rdg[wit={J10,Jyo}]{\NotIn}}%
\pada{cittaṃ carati khe yasmāj} % citraṃ N19; cārati V15; yasmā V15,V3
\pada{jihvā carati
	\app{\lem[wit={G5,N19,GrB}]{khe gatā}% ga<tā> P11
		\rdg[wit={G11}]{khe yadā}
		\rdg[wit={V15}]{vegataḥ}}/}\\+}
\tl{
\pada{\app{\lem[wit={G11,V15,P11,V3}]{tenaiṣā}% P6
		\rdg[wit={C6}]{teneyaṃ}
		\rdg[wit={N19}]{tenaiva}
		\rdg[wit={G5}]{iyaṃ ca}} khecarī  % khecarīṃ N19
	\app{\lem[wit={G11,G5,N19,P11,V3}]{nāma}
		\rdg[wit={V15,C6}]{mudrā}}}
\pada{\app{\lem[wit={G11,G5,N19,P11,V3}]{mudrā}% mudra P11
		\rdg[wit={V15}]{satyaṃ}
		\rdg[wit={C6}]{sarva}} 
		siddhai%r  % siddhai N19
	\app{\lem[wit={G11,G5,N19,GrB},alt={namaskṛtā}]{\skm{r }namaskṛtā}% °tāḥ P11
		\rdg[wit={V15}]{nigadyate}}//\versenr}
	\anm{=\,\manuref{3.37}}\\!}
\end{alttlg}
%</vs11-7>
%<*vs11-8>
\begin{alttlg}[hp04_011_8]
\tl{\app{\lem[nolem]{}
	\rdg[wit={V15,J10}]{\NotIn}}%
\pada{iḍāpiṅgalayo%r  % idā N19
	\app{\lem[wit={G11,G5,N19,GrB},alt={yoge}]{\skm{r }yoge}
		\rdg[wit={Jyo}]{madhye}}}
\pada{\app{\lem[wit={G11,G5,C6,Jyo}]{śūnyaṃ}
		\rdg[wit={N19,P11}]{śūnye}% +F,P6
		\rdg[wit={V3}]{śūne}} % +J11ac
	\app{\lem[wit={G11,G5,N19,V3,Jyo}]{caivānilaṃ}
		\rdg[wit={P11,C6}]{caiva bilaṃ}}
	\app{\lem[wit={G11,G5,N19,P11,V3,Jyo}]{graset}% grasit P11
		\rdg[wit={C6}]{viśet}}/}\\+}
\tl{
\pada{\app{\lem[wit={G11,G5,N19,C6,V3,Jyo}]{tiṣṭhate}
		\rdg[wit={P11}]{tiṣṭhati}} khecarī mudrā} % khacarī G11
\pada{\app{\lem[wit={G11,G5,P11}]{tatra satyaṃ na saṃśayaḥ}% P6,F
		\rdg[wit={N19}]{satyaṃ tatra na saṃśayaḥ}
		\rdg[wit={C6,V3,Jyo}]{tatra satyaṃ punaḥ punaḥ}
		}//\versenr}
		\\!}
\end{alttlg}
%</vs11-8>
%<*vs11-9>
\begin{alttlg}[hp04_011_9]
\tl{
\pada{\app{\lem[wit={G11,G5,N19,J10}]{somasūryadvayo\skp{r}} % +G7,P6; sūryā J10
		\rdg[wit={V15}]{candrasūryadvayor}
		\rdg[wit={GrB,Jyo},alt={sūryācandramasor}]{sūryācandramasor}% sūrya C6, sūryāc P11, °masaur V3
		}r madhye}
\pada{\app{\lem[wit={N19,V15,C6,V3}]{nirālambe tale}% +G3
		\rdg[wit={G5,P11}]{nirālambatale}% °laṃbatare M3,F
		\rdg[wit={G11}]{nirālambe kale}
		\rdg[wit={J10}]{nirālambo'ntarā}%
		\rdg[wit={Jyo}]{nirālambāntare}}
		punaḥ/}\\+}
\tl{
\pada{saṃsthitā vyomacakre yā} % cakreṇa G11; °ścaikacakre yā G5; 
\pada{sā mudrā nāma khecarī//\versenr}\\!}
\end{alttlg}
%</vs11-9>
%<*vs11-10>
\begin{alttlg}[hp04_011_10]
\tl{\app{\lem[nolem]{}
	\rdg[wit={C6}]{\NotIn}}%
\pada{\app{\lem[wit={P11,V3}]{sā mayodbheditā vāmā}
		\rdg[wit={G11}]{sā māyodbhedikā vāmā}%  sā mayābhedatā M3, sā madhyodbheditā F
		\rdg[wit={G5}]{sā māyābhedito vāmā}
		\rdg[wit={N19}]{sā mayodve\,\_\,tā vāmā}
		\rdg[wit={V15}]{sā mayodve[dh]itā vāmā}
		\rdg[wit={J10}]{somayodbheditā dhāma}% °rbhidādhāmāmā P6
		\rdg[wit={Jyo}]{somād yatroditā dhārā}}}
\pada{\app{\lem[wit={N19,V15,P11,V3}]{sākṣāc ca}% +G7
		\rdg[wit={J10}]{sākṣād vai}
		\rdg[wit={G11,G5}]{sā sākṣāt}% +M3
		\rdg[wit={Jyo}]{sākṣāt sā}} śivavallabhā/}\\+}
\tl{
\pada{\app{\lem[wit={G11,G5,N19,V15,P11,V3},alt={pūrayen}]{pūraye\skp{n}}
		\rdg[wit={Jyo}]{pūrayed}
		\rdg[wit={J10}]{pūjayed}}%
	\app{\lem[wit={N19,V15,P11,V3},alt={mārutaṃ divyaṃ}]{\skm{n }mārutaṃ divyaṃ}% +G7
		\rdg[wit={G11}]{na tu tad divyaṃ}% + na tu tan divya M3
		\rdg[wit={G5}]{satataṃ divyaṃ}
		\rdg[wit={J10,Jyo}]{atulāṃ divyāṃ}}}
\pada{\app{\lem[wit={G11,G5,N19,V15,J10,P11,V3}]{suṣumṇā}
		\rdg[wit={Jyo}]{suṣumṇāṃ}}
	\app{\lem[wit={G11,G5,N19,V15,J10,P11,Jyo}]{paścime}%
		\rdg[wit={V3}]{paścimā}}
	mukhe//\versenr}
	\\!}
\end{alttlg}
%</vs11-10>

%<*vs11-11>
\begin{alttlg}[hp04_011_11]
\tl{
\pada{purastāc caiva pūryeta} % °stāvaiva P11, °stācaiva V3, t* caiva N19; °yetā V3
\pada{\app{\lem[wit={G11,G5,N19,V15,GrB,Jyo}]{niścitā}% °taṃ F, °cittā G5
		\rdg[wit={J10}]{niśritā}
		} khecarī bhavet/}\\+}
\tl{
\pada{\app{\lem[wit={G11,G5,N19,P11,C6},alt={abhyaset}]{abhyase\skp{t}}
		\rdg[wit={V3}]{abhyase}
		\rdg[wit={J10,Jyo}]{abhyastā}
		\rdg[wit={V15},alt={\om},post=\texteng{(eye-skip?)}]{\skp{\om}}}%
	\app{\lem[wit={G11,G5,N19,C6,V3},alt={khecarīmudrām}]{\skm{t }khecarīmudrā\skp{m}}
		\rdg[wit={P11}]{khecarīṃ mudrām}
		\rdg[wit={J10,Jyo}]{khecarīmudrā}
		\rdg[wit={V15},alt={\om}]{\skp{\om}}}}%
\pada{\app{\lem[wit={G11,G5,N19,J10,GrB},alt={unmanī}]{\skm{m }unmanī}
		\rdg[wit={Jyo}]{py unmanī}
		\rdg[wit={V15},alt={\om}]{\skp{\om}}}
	\app{\lem[wit={G11,G5,N19,J10,Jyo}]{saṃprajāyate}
		\rdg[wit={P11}]{sāṃdrajāyate}
		\rdg[wit={C6,V3}]{sā prajāyate}
		\rdg[wit={V15},alt={\om}]{\skp{\om}}}//\versenr}\\!}
\end{alttlg}
%</vs11-11>
%<*vs11-12>
\begin{alttlg}[hp04_011_12]
\tl{%\app{\lem[nolem]{}
	%\rdg[wit={Jyo},alt=\textapp{transposed with the next verse}]{\skp{transposed with the next verse}}}%
\pada{\app{\lem[wit={G11,N19,GrB,Jyo},alt={abhyaset}]{abhyase\skp{t}}
		\rdg[wit={V15}]{abhyasat}
		\rdg[wit={G5}]{abhyasya}
		\rdg[wit={J10}]{abhyaste}}%
	\app{\lem[wit={G11,G5,N19,V15,J10,GrB},alt={khecarī}]{\skm{t }khecarī}
		\rdg[wit={Jyo}]{khecarīṃ}}% +J11
	\app{\lem[wit={G11,V15,J10}]{mudrāṃ}
		\rdg[wit={G5,N19}]{mudrā}
		\rdg[wit={GrB,Jyo}]{tāvad}
		}}
\pada{\app{\lem[wit={G11,G5,N19,V15,J10}]{tāva\skp{t}}
		\rdg[wit={GrB,Jyo},alt={yāvat}]{yāvat}
		}t syā%d
	\app{\lem[wit={G11,N19,V15,C6,Jyo},alt={yoganidritaḥ}]{\skm{d }yoganidritaḥ}
		\rdg[wit={P11}]{yoganidritāḥ}
		\rdg[wit={J10}]{yoganidratāḥ}
		\rdg[wit={V3}]{yoganiṃdrataḥ}
		\rdg[wit={G5}]{coramudritā}}/}\\+}
\tl{
\pada{saṃprāptayoganidrasya} % niṃdrasya V3
\pada{kālo nāsti kadācana//\versenr}% canaṃ V3, canā G11
\\!}
\end{alttlg}
%</vs11-12>
%<*vs11-13>
\begin{alttlg}[hp04_011_13]
\tl{%\app{\lem[nolem]{}
	%\rdg[wit={Jyo},alt=\textapp{transposed with the previous verse}]{\skp{transposed with the previous verse}}}%
\pada{\app{\lem[nolem]{\skp{pāda a}}
	\rdg[wit={G11,G5},alt={\om}]{\skp{\om}}}% M3 omits too.
bhruvor madhye  % °vo P11, bhrūvaur N19
\app{\lem[wit={N19,V15,J10,C6,V3,Jyo}]{śiva}
	\rdg[wit={P11}]{bhavet}}sthānaṃ}
\pada{\app{\lem[nolem]{\skp{pāda b}}
	\rdg[wit={G11,G5},alt={\om}]{\skp{\om}}}% M3 omits too.
manas tatra vilīyate/}  % vilīyati P11
	\\+}
\tl{
\pada{jñātavyaṃ tat padaṃ turyaṃ} % talpadaṃ G11; tūryaṃ P11,V15
\pada{\app{\lem[wit={G11,G5,N19,J10,GrB,Jyo}]{tatra}
		\rdg[wit={V15}]{yatra}}
	\app{\lem[wit={G11,G5,V15,J10,GrB,Jyo}]{kālo}
		\rdg[wit={N19}]{kopi}} 
		na vidyate//\versenr}\\!}
\end{alttlg}
%</vs11-13>
%<*vs11-14>
\begin{alttlg}[hp04_011_14]
\tl{\app{\lem[nolem]{}
	\rdg[wit={Jyo}]{\NotIn}}%
\pada{candrasūryadvayor madhye}
\pada{\app{\lem[wit={G11,G5,V15,J10,GrB}]{mudrāṃ}
		\rdg[wit={N19}]{mudrā}}
	\app{\lem[wit={G5,V15,J10,GrB}]{dadyāc ca} % +M3; dadyā V3
		\rdg[wit={G11}]{dadyāt tu}
		\rdg[wit={N19}]{divyā ca}}
	\app{\lem[wit={G11,G5,V15,J10,C6}]{khecarīm}
		\rdg[wit={N19,V3}]{khecarī}
		\rdg[wit={P11}]{khecare}}/}\\+}
\tl{
\pada{\app{\lem[wit={G11,J10,C6}]{nirālambe}
		\rdg[wit={N19,V15,V3}]{nirālambaṃ}% +F
		\rdg[wit={P11}]{nirālambas}
		\rdg[wit={G5}]{nirālamba}}
	\app{\lem[wit={J10,C6}]{mahāśūnye}
		\rdg[wit={N19,V15}]{mahāśūnyaṃ}% +F
		\rdg[wit={G11}]{mahacchūnye}
		\rdg[wit={G5,V3}]{mahāśūnya}
		\rdg[wit={P11}]{tadā śūnya}}}
\pada{vyoma\app{\lem[wit={G11,G5,N19,J10,GrB}]{cakre}
		\rdg[wit={V15}]{cakraṃ}}
	\app{\lem[wit={G11,J10,C6,V3}]{vyavasthitām}
		\rdg[wit={N19,V15}]{vyavasthitaṃ}% +G7,F
		\rdg[wit={G5,P11}]{vyavasthitā}% P6; °taḥ M3?
		}//\versenr}\\!} % cf. 4.25*6
\end{alttlg}
%</vs11-14>
%<*vs11-15>
\begin{alttlg}[hp04_011_15]
\tl{
\pada{nirālambaṃ manaḥ kṛtvā} % manaṃ P11
\pada{na kiṃcid api cintayet/}\\+}
\tl{
\pada{sabāhyā\app{\lem[wit={G11,N19,V15,GrB,Jyo},alt={°bhyantare}]{\skp{°}bhyantare}
		\rdg[wit={G5,J10}]{bhyantaraṃ}} vyomni} % vyomani G11, vyoma G5
\pada{\app{\lem[wit={V15,J10,GrB,Jyo},alt={ghaṭavat}]{ghaṭava\skp{t}}
		\rdg[wit={N19}]{paṭavat}
		\rdg[wit={G11}]{aṭavat}
		\rdg[wit={G5}]{maghaṭat}}%
	\app{\lem[wit={G11,G5,N19,V15},alt={tiṣṭhate}]{\skm{t }tiṣṭhate}
		\rdg[wit={J10,GrB,Jyo}]{tiṣṭhati}} 
		dhruvam//\versenr}\\!}
\end{alttlg}
%</vs11-15>
%<*vs11-16>
\begin{alttlg}[hp04_011_16]
\tl{
\pada{bāhyavāyu%r
	\app{\lem[wit={J10,GrB,Jyo},alt={yathā}]{\skm{r }yathā}% P6
		\rdg[wit={G11,G5}]{tathā}% +M3
		\rdg[wit={N19,V15}]{yadā}}
	\app{\lem[wit={G11,V15,C6}]{līnaḥ}
		\rdg[wit={N19,P11}]{līna}
		\rdg[wit={V3}]{līnaṃ}
		\rdg[wit={G5}]{līnā}
		\rdg[wit={J10,Jyo}]{līnas}}}
\pada{\app{\lem[wit={G11,G5,P11,V3}]{khasya madhye}
		\rdg[wit={C6}]{khamadhye tu}
		\rdg[wit={V15}]{khamadhye ca}
		\rdg[wit={N19}]{khamadhya\,\_}
		\rdg[wit={J10}]{tathā madhye}
		\rdg[wit={Jyo}]{tathā madhyo}}
	 \app{\lem[wit={G11,G5,V15,J10,GrB,Jyo}]{na saṃśayaḥ}
		\rdg[wit={N19}]{\_\,\_\,sayaḥ}}/}\\+}
\tl{
\pada{\app{\lem[wit={G11,N19,V15,J10,GrB}]{svasthānaṃ gacchati prāṇaḥ} % gacchatiṃ G11; prāṇa N19,V3,F
	\rdg[wit={G5}]{saṃsthānaṃ gacchati prāṇaḥ}
	\rdg[wit={Jyo}]{svasthāne sthiratām eti}}}
\pada{\crux\app{\lem[wit={G11,G5,N19,V15,GrB}]{sūryāṅge manasā tathā}% sūryāgnir M1; manatā P6, mānasaṃ G7
		\rdg[wit={J10}]{sūryāṅge pavane tathā}% +P17
		\rdg[wit={Jyo}]{pavano manasā saha}}\crux//\versenr}\\!} % sveyāṅge F
\end{alttlg}
%</vs11-16>

%<*vs11-17>
\begin{alttlg}[hp04_011_17]
\tl{
\pada{evam abhyasyamānasya}
		%\rdg[wit={Jyo}]{abhyasyatas tasya}}}
\pada{\app{\lem[wit={G11,G5,J10,GrB,Jyo}]{vāyumārge}% vāyurmāge C6
		\rdg[wit={N19,V15}]{vāyor mārge}} % vāyomārgre N19, mārgaṃ J10
	\app{\lem[wit={C6,Jyo}]{divāniśam}% +M1,F
		\rdg[wit={P11}]{divā niśi}
		\rdg[wit={V3}]{divādisam}
		\rdg[wit={G11,G5,J10}]{sadāniśaṃ}
		\rdg[wit={N19,V15}]{sadānilaṃ}}/}\\+}
\tl{
\pada{\app{\lem[wit={G11,G5,N19,J10,GrB,Jyo}]{abhyāsāj jīryate} % abhyāsā V3; jāyate J10ac?
		\rdg[wit={V15}]{abhyāsāl līyate}}
		vāyur} % vāyu V3
\pada{mana\app{\lem[wit={G11,G5,N19,V15,J10},alt={tatra vilīyate}]{\skm{s }tatra vilīyate}% P6
		\rdg[wit={GrB,Jyo}]{tatraiva līyate}}//\versenr}\\!}% +F
\end{alttlg}
%</vs11-17>

%<*vs11-18>
\begin{alttlg}[hp04_011_18]
\tl{
\pada{\app{\lem[wit={N19,P11,V3},alt={amṛtaṃ plāvayed deham}]{amṛtaṃ plāvayed deha\skp{m}}
		\rdg[wit={G11,G5},postwit=\texteng{(amṛtā \getsiglum{G11})}]{amṛtāt plāvayed deham}% amṛtā G11; +F
		\rdg[wit={V15}]{amṛte plāvayed deham}
		\rdg[wit={C6}]{amṛtaṃ plavate \_\,\_}
		\rdg[wit={Jyo}]{amṛtaiḥ plāvayed deham}
		\rdg[wit={J10}]{ajaratvaṃ bhaved dehe}}}%m}
\pada{\app{\lem[wit={ceteri},alt={ā pādatala}]{\skm{m }ā pādatala}
		\rdg[wit={J10}]{apādapala}
		\rdg[wit={C6},alt={\lacuna}]{\skp{\lacuna}}}%
	\app{\lem[wit={G11,G5,V15,GrB,Jyo}]{mastakam} % dehaṃāpādattala°  apādapala J10
		\rdg[wit={J10}]{mastake}
		\rdg[wit={N19}]{mastakān}
		\rdg[wit={C6},alt={\lacuna}]{\skp{\lacuna}}}/}\\+}
\tl{
\pada{\app{\lem[wit={G11,G5,V3,Jyo}]{sidhyaty eva}% °ateva G11
		\rdg[wit={N19}]{siddhaty eva}
		\rdg[wit={V15}]{siddhyaty evaṃ}
		\rdg[wit={J10}]{sidhyate ca}
		\rdg[wit={C6}]{siddhadeho}
		\rdg[wit={P11}]{siddhideho}}
	\app{\lem[wit={G11,G5,N19,V3}]{sadā kāyo}
		\rdg[wit={C6,Jyo}]{mahākāyo}
		\rdg[wit={P11}]{mahākāryo}
		\rdg[wit={J10}]{mahāyogo}
		\rdg[wit={V15}]{tadā kāyo}% +F
		}}
\pada{mahābalaparākramaḥ//\versenr}\\!} % °kramī P11
\end{alttlg}
%</vs11-18>

% \begin{altpostmula}[hp04_011_18p]
% iti khecarī/ \sgwit{Jyo} % only in the edition
% \end{altpostmula}

%<*vs11-19a>
\begin{altava}[hp04_011_19a]
\app{\lem[nolem]{}
 \rdg[wit={N19,P11}]{\only}}%}
\app{\lem[wit={N19}]{atha}\rdg[wit={P11},alt={\om}]{\skp{\om}}}
\app{\lem[wit={P11}]{śāmbhavī}
	\rdg[wit={N19}]{śāṃbhavī śaktiḥ}}/ 
	%\sgwit{N19,P11}% +F
\end{altava}
%</vs11-19a>
%<*vs11-19>
\begin{alttlg}[hp04_011_19]
\tl{
\pada{śaktimadhye manaḥ kṛtvā} % mana P11
\pada{\app{\lem[wit={G11,G5,N19}]{śaktiṃ ca manamadhyagām}% P6
		\rdg[wit={V15}]{śaktiṃ ca svāṃtamadhyagām}
		\rdg[wit={Jyo}]{śaktiṃ mānasamadhyagām}
		\rdg[wit={J10}]{śaktiṃ manasi madhyataḥ}
		\rdg[wit={P11}]{sumadhyagaṃ}
		\rdg[wit={C6,V3}]{manaḥ śaktes tu madhyagam}% +F
		}/}\\+}
\tl{
\pada{manasā
	\app{\lem[wit={G11,G5,V15,J10,P11,C6,Jyo}]{mana ālokya}
		\rdg[wit={N19}]{mana ārokya}
		\rdg[wit={V3}]{manam ālokya}}}
\pada{\app{\lem[wit={G11,N19,V15,C6},alt={tad dhyāyet}]{tad dhyāye\skp{t}}
		\rdg[wit={G5}]{taṃ dhyāyet}
		\rdg[wit={P11}]{taṃ dhātaṃ}
		\rdg[wit={V3}]{vaddhyāyait}
		\rdg[wit={J10,Jyo}]{dhārayet}% dhāryate P6
		}t paramaṃ padam//\versenr}\\!}
\end{alttlg}
%</vs11-19>
%<*vs11-20>
\begin{alttlg}[hp04_011_20]
\tl{
\pada{\app{\lem[wit={G11,G5,N19,V15,J10,C6,V3,Jyo}]{khamadhye}
		\rdg[wit={P11}]{khaṃmadhye}% +F
		} kuru cātmāna}%m}
\pada{\app{\lem[wit={G11,G5,V15,V3,Jyo},alt={ātmamadhye}]{\skm{m }ātmamadhye} % J10pc
		\rdg[wit={N19,J10,P11,C6}]{ātmāmadhye}} % J10ac
		ca khaṃ kuru/}\\+}
\tl{
\pada{\app{\lem[wit={G11,C6,V3}]{ātmānaṃ}% +G7,??
		\rdg[wit={G5,N19,V15,J10,Jyo}]{sarvaṃ ca}% +M1,P6,G10,F
		\rdg[wit={P11}]{evaṃ kṛ°}}
	\app{\lem[wit={N19,V15,V3,Jyo}]{khamayaṃ kṛtvā}
		\rdg[wit={G11,G5,J10,C6}]{khaṃmayaṃ kṛtvā}% +F
		\rdg[wit={P11}]{°tvā tayoś cāpi}}}
\pada{na kiṃcid api cintayet//\versenr}\\!}
\end{alttlg}
%</vs11-20>
%<*vs11-21>
\begin{alttlg}[hp04_011_21]
\tl{\app{\lem[nolem]{}
	\rdg[wit={N19,V15,V3}]{\NotIn}}%
\pada{antaḥśūnyo bahiḥśūnyaḥ} % aṃtaśūnye °śūnyo J10; bahiśūnyaṃ P11
\pada{\app{\lem[wit={G11,G5,J10,P11,C6}]{śūnya}
	\rdg[wit={Jyo}]{śūnyaḥ}}kumbha ivāmbare/}\\+} % °baro P11
\tl{
\pada{\app{\lem[nolem]{\skp{pāda c}}
		\rdg[wit={G11},alt={\om}]{\skp{\om}}}%
	antaḥpūrṇo bahiḥpūrṇaḥ} % bahipūrṇa P11, pūrṇa<<ḥ>> J10
\pada{\app{\lem[nolem]{\skp{pāda d}}
		\rdg[wit={G11},alt={\om}]{\skp{\om}}}%
	\app{\lem[wit={G5,J10,P11,C6}]{pūrṇa}
		\rdg[wit={Jyo}]{pūrṇaḥ}}kumbha
	\app{\lem[wit={G5,J10,Jyo}]{ivārṇave}% ivādhare P6
		\rdg[wit={P11}]{ivāṃbare}% +N22
		\rdg[wit={C6}]{ivāmbudhau}}//\versenr}
	\\!} % G5 vollständig, M3 om.
\end{alttlg}
%</vs11-21>

%<*vs11-22>
\begin{alttlg}[hp04_011_22]
\tl{\app{\lem[nolem]{}
	\rdg[wit={N19,V15}]{\NotIn}}%
\pada{bāhyacintā na kartavyā} % °tavyo J10
\pada{tathaivāntara%
	\app{\lem[wit={G11,G5,J10,Jyo}]{cintanam}% P6/N22
		\rdg[wit={C6,V3}]{cintanā}% +F
		\rdg[wit={P11}]{ciṃtamān}}/}\\+}
\tl{
\pada{\app{\lem[wit={G11,G5,C6,Jyo}]{sarvacintāṃ parityajya}
	\rdg[wit={P11,V3}]{sarvacintā parityajya}
	\rdg[wit={J10}]{sarvacintā parityājyā}}}
\pada{na kiṃcid api cintayet//\versenr}
\\!} % M3 omits. too, but G11,G5 have it.
\end{alttlg}
%</vs11-22>

%<*vs11-23>
\begin{alttlg}[hp04_011_23]
\tl{
\app{\lem[nolem]{\skp{pāda a}}
	\rdg[wit={P11,C6},alt={\om}]{\skp{\om}}}%
\pada{saṃkalpamātra%
	\app{\lem[wit={G11,G5,N19,V15,J10,Jyo}]{kalanaiva}% ka<la>naiva G11
		\rdg[wit={V3}]{kalanaṃ ca}} jaga%t
	\app{\lem[wit={G11,G5,N19,V15,V3,Jyo},alt={samagraṃ}]{\skm{t }samagraṃ}
		\rdg[wit={J10}]{samastaṃ}}}\\+}
\tl{
\pada{\app{\lem[nolem]{\skp{pāda b}}
	\rdg[wit={P11,C6},alt={\om}]{\skp{\om}}}%
saṃkalpamātra% kal<p>a G11
	\app{\lem[wit={G11,G5,N19,V15,V3}]{kalanā hi}% karahāṇi P6,N22b
		\rdg[wit={J10,Jyo}]{kalanaiva}} % phalanaiva N22a
	mano\app{\lem[wit={G5,J10,Jyo}]{vilāsaḥ}% +P6,M3
		\rdg[wit={V3}]{vilāsā}
		\rdg[wit={G11}]{vivāsaḥ}
		\rdg[wit={N19}]{vilīnā}
		\rdg[wit={V15}]{valīnā}}/}\\+}
\tl{
\pada{\app{\lem[nolem]{\skp{pāda c}}
	\rdg[wit={C6},alt={\om}]{\skp{\om}}}%
\app{\lem[wit={G11}]{saṃkalpam etam ata}
		\rdg[wit={V15}]{saṃkalpamātramatam}% P6,N22
		\rdg[wit={N19},alt={°mātramata}]{saṃkalpamātramata}
		\rdg[wit={G5},alt={°mātramanam}]{saṃkalpamātramanam}
		\rdg[wit={Jyo},alt={°mātramatim}]{saṃkalpamātramatim}
		\rdg[wit={P11},alt={°mātrami[m]}]{saṃkalpamātrami[m]}
		\rdg[wit={V3},alt={°mātram idam}]{saṃkalpamātram idam}
		\rdg[wit={J10},alt={°mātrakalanaiva}]{saṃkalpamātrakalanaiva}
		}
	\app{\lem[wit={G11,G5,V15,P11,V3,Jyo}]{utsṛja}
		%\rdg[wit={Jyo}]{utsṛjya}
		\rdg[wit={N19}]{tsṛja}
		\rdg[wit={J10}]{vikṛtis tu}}
	\app{\lem[wit={G11,G5,N19,V15,P11,V3,Jyo}]{nirvikalpaṃ}
		\rdg[wit={J10}]{nityaṃ}}}\\+}
\tl{
\pada{\app{\lem[nolem]{\skp{pāda d}}
	\rdg[wit={C6},alt={\om}]{\skp{\om}}}%
\app{\lem[wit={G11,G5,N19,P11,V3,Jyo}]{āśritya} % āśrītya V3
		\rdg[wit={V15}]{āśrita}
		\rdg[wit={J10}]{saṃkalpa}}
	\app{\lem[wit={G11,J10,Jyo},alt={niścayam}]{niścaya\skp{m}} % P6; = South Ind. 4c
		\rdg[wit={G5,P11}]{niścalam}
		\rdg[wit={V3}]{niścalayam}
		\rdg[wit={N19,V15}]{niścitam}}%
	\app{\lem[wit={G11,G5,N19,V15,V3,Jyo},alt={avāpnuhi}]{\skm{m }avāpnuhi}
		\rdg[wit={J10}]{avāpnudhi}
		\rdg[wit={P11}]{anāpnuhi}}
	\app{\lem[wit={G11,J10,P11,V3,Jyo}]{rāma}% śāmya P6, ?? N22
		\rdg[wit={G5}]{kāma}
		\rdg[wit={V15}]{rāga}
		\rdg[wit={N19}]{roga}} 
		śāntim//\versenr}\\!} % śānti P11,V3
\end{alttlg}
%</vs11-23>

%<*vs11-24>
\begin{alttlg}[hp04_011_24]
\tl{\app{\lem[nolem]{}
	\rdg[wit={J10}]{\NotIn}}%
\pada{karpūra%m
	\app{\lem[wit={G11,G5,N19,V15,P11,V3,Jyo},alt={anale}]{\skm{m }anale}
		\rdg[wit={C6}]{anile}} yadvat}
\pada{saindhavaṃ salile yathā/}\\+}
\tl{
\pada{\app{\lem[wit={G11,G5,V15,GrB,Jyo}]{tathā}
		\rdg[wit={N19}]{yathā}}
	\app{\lem[wit={G11,G5,GrB,Jyo}]{saṃdhīyamānaṃ ca}
		\rdg[wit={N19,V15}]{saṃdīpamānaṃ ca}}}
\pada{mana\app{\lem[wit={G11,V15,C6,Jyo},alt={tattve}]{\skm{s }tattve}
		\rdg[wit={P11}]{tātva}
		\rdg[wit={V3}]{tatva}
		\rdg[wit={G5,N19}]{tatra}}
	\app{\lem[wit={G11,G5,N19,GrB,Jyo}]{vilīyate}
		\rdg[wit={V15}]{valīyate}}//\versenr}\label{karpura}
	\\!}
\end{alttlg}
%</vs11-24>
%<*vs11-25>
\begin{alttlg}[hp04_011_25]
\tl{
\pada{jñeyaṃ % jñoyaṃ N19
	\app{\lem[wit={G11,G5,P11,C6,Jyo}]{sarvaṃ pratītaṃ}
		\rdg[wit={N19,V15,V3}]{sarvapratītaṃ}
		\rdg[wit={J10}]{sarvam atītaṃ}} ca}
\pada{\app{\lem[wit={G11,N19,V15}]{tajjñānaṃ}
		\rdg[wit={G5}]{tat jñātaṃ}
		\rdg[wit={J10,Jyo}]{jñānaṃ ca}% jñāna J10; +F
		\rdg[wit={GrB}]{jñānaṃ tu} % P6
		} mana ucyate/}\\+} % ucyata P11
\tl{
\pada{jñānaṃ % + yaṃ! G11; but jñānaṃ G5,M3
	  jñeyaṃ % m. om. V3, jñoyaṃ N19
	\app{\lem[wit={G11,G5,N19,V15,GrB,Jyo}]{samaṃ naṣṭaṃ}
		\rdg[wit={J10}]{manaś caiva}}}
\pada{\app{\lem[wit={G11,G5,N19,V15,J10,C6,V3,Jyo}]{nānyaḥ}
		\rdg[wit={P11}]{mānyaḥ}}
	\app{\lem[wit={G11,G5,N19,J10,C6,Jyo}]{panthā}
		\rdg[wit={V15}]{paṃtha}
		\rdg[wit={P11}]{paṃthyā}
		\rdg[wit={V3}]{pathā}}
	\app{\lem[wit={G11,G5,V15,J10,C6,Jyo}]{dvitīyakaḥ}
		\rdg[wit={N19,P11}]{dvitīyakaṃ}
		\rdg[wit={V3}]{dvitiyaka}}//\versenr}\\!}
\end{alttlg}
%</vs11-25>
%<*vs11-26>
\begin{alttlg}[hp04_011_26]
\tl{
\pada{manodṛśyam idaṃ sarvaṃ} % the last 3 akṣaras damaged G11
\pada{yat kiṃcit sacarācaraṃ/}\\+}
\tl{
\pada{\app{\lem[wit={J10,Jyo}]{manaso hy unmanī}
		\rdg[wit={G11}]{manaso hy amanī}
		\rdg[wit={G5,V15,GrB}]{manasopy unmanī}% +M3
		\rdg[wit={N19}]{mano so 'py unmanī}
		}%
	\app{\lem[wit={V15,J10pc,V3},alt={°bhāve}]{\skp{°}bhāve}% +G7,F
		\rdg[wit={P11}]{bhāvai}
		\rdg[wit={G11,G5,C6}]{bhāvo}% +M3
		\rdg[wit={J10ac}]{bhāvavo}
		\rdg[wit={Jyo}]{bhāvād}
		\rdg[wit={N19},alt={\om}]{\skp{\om}}
		}}
\pada{\app{\lem[wit={V15,P11,C6}]{dvaitābhāvaṃ}% +M3,G7; bhāvaḥ G5
		\rdg[wit={G11}]{dvaitābhā\,+}
		\rdg[wit={G5}]{dvaitābhāvaḥ}
		\rdg[wit={V3}]{dvaitābhāva}
		\rdg[wit={N19}]{bhāvaṃ}
		\rdg[wit={J10,Jyo}]{dvaitaṃ naivo°}}
	\app{\lem[wit={G11,G5,V15,C6,V3}]{pracakṣate}% +Loc-Nom
		\rdg[wit={N19,P11}]{pracakṣyate}% +Nom-Nom
		\rdg[wit={J10,Jyo}]{°palabhyate}}//\versenr}\\!}
\end{alttlg}
%</vs11-26>

%<*vs11-27>
\begin{alttlg}[hp04_011_27]
\tl{
\pada{jñeyavastuparityāgād} % jñeyaṃ?  parī° N19; jñeyaṃ yas tu P11
\pada{vilayaṃ yāti % yāṃti P11, damaged G11
	\app{\lem[wit={G11,G5,V15,J10,GrB,Jyo}]{mānasam}
		\rdg[wit={N19}]{mārutaṃ}}/}\\+}
\tl{
\pada{\app{\lem[wit={G11,N19,V15,GrB}]{mānase}
		\rdg[wit={G5,J10,Jyo}]{manaso}}
	\app{\lem[wit={G11,G5,N19,V15,J10,P11,V3}]{vilayaṃ}
		\rdg[wit={C6,Jyo}]{vilaye}}
	\app{\lem[wit={G11,N19,V15,P11}]{yāte}
		\rdg[wit={G5}]{yāti}
		\rdg[wit={J10,C6,V3,Jyo}]{jāte}}}
\pada{kaivalya%m
	\app{\lem[wit={G11,G5,V15,GrB,Jyo},alt={avaśiṣyate}]{\skm{m }avaśiṣyate}
		\rdg[wit={N19}]{anasīṣyate}
		\rdg[wit={J10}]{api kalpate}}//\versenr}\\!}
\end{alttlg}
%</vs11-27>

%<*vs11-28>
\begin{alttlg}[hp04_011_28]
\tl{%\app{\lem[nolem]{}
	%\rdg[wit={Jyo},alt=\textapp{found between \ref{yatradrsti} and \ref{vedasastra}}]{\skp{found between xxx and xxx}}}%
\pada{layo laya iti prāhuḥ} % first 2 akṣaras damaged G11; layaṃ? C6; prāhur N19,V15
\pada{\app{\lem[wit={G11,G5,J10,GrB,Jyo}]{kīdṛśaṃ}
		\rdg[wit={N19,V15}]{īdṛśaṃ}} layalakṣaṇam/}\\+}
\tl{
\pada{apunarvāsano% apu<na>r° N19
	\app{\lem[wit={N19,J10,P11,C6,Jyo},alt={°tthānāl}]{\skp{°}tthānā\skp{l}}% °nāl P11,Jyo, °tthā .. l C6, nāt* N19, °nād J10
		\rdg[wit={G11,V15,V3}]{°tthānā}
		\rdg[wit={G5}]{°tthāna}}}% % +P7
\pada{\app{\lem[wit={G5,N19,V15,GrB,Jyo},alt={layo viṣaya}]{\skm{l }layo viṣaya}
		\rdg[wit={G11}]{yalo viṣaya}%
		\rdg[wit={J10}]{vṛttyayā viśva}
		}vismṛtiḥ//\versenr}% °smṛti N19,V3; °smati P11, niśrutiḥ G5
	\label{layo}
	\\!}
\end{alttlg}
%</vs11-28>
%<*vs11-29>
\begin{alttlg}[hp04_011_29]
\tl{\app{\lem[nolem]{}
 \rdg[wit={G11,G5,N19,V15,J10,P11,C6,V3,Jyo}]{\incl}}%
\pada{evaṃ nānāvidhopāyāḥ} % eva N19; ḥ om. P11,V3
\pada{samyaksvānu%
	\app{\lem[wit={G11,N19,J10,GrB,Jyo}]{bhavānvitāḥ} % ḥ om. V3; sānubhavanvitā P11
		\rdg[wit={G5}]{bhavānyuta}
		\rdg[wit={V15}]{bhavātmikāḥ}}/}\\+}
\tl{
\pada{samādhi\app{\lem[wit={G11,G5,N19,V15,P11,C6,Jyo}]{mārgāḥ}% ḥ om. P11
		\rdg[wit={J10}]{mārge}
		\rdg[wit={V3},alt={\illeg}]{\skp{\illeg}}} 
		kathitāḥ} % °tā V15
\pada{pūrvācāryair mahātmabhiḥ//\versenr}\\!}% °ryai P11,V3
%\myfn{After this verse, \getsiglum{P11,C6} have \devnote{iti viśrāntiḥ} and \getsiglum{N19,V15} \devnote{atha viśrāntiḥ}.}
\end{alttlg}
%</vs11-29>

%<*vs11-30a>
\begin{altava}[hp04_011_30a]
\app{\lem[nolem]{}
	\rdg[wit={J10,V3,Jyo}]{\NotIn}}%
	%\rdg[wit={G11,G5,N19,V15,P11,C6}]{\incl}}%}
\app{\lem[wit={G11,G5,N19,V15}]{atha}
	\rdg[wit={P11,C6}]{iti}% °ti P11; +F
	} viśrāntiḥ/ 
\end{altava}
%</vs11-30a>

%<*vs11-30>
\begin{alttlg}[hp04_011_30]
\tl{\app{\lem[nolem]{}
	\rdg[wit={J10}]{\NotIn}}%
\pada{\app{\lem[wit={G11,G5,V15,GrB,Jyo}]{suṣumṇāyai}
		\rdg[wit={N19}]{sukhayaiḥ}} kuṇḍalinyai}
\pada{sudhāyai candra% ceṃdra V3
	\app{\lem[wit={G11,G5}]{maṇḍale}% +M3
		\rdg[wit={N19,V15}]{maṇḍalāt}% +G7,G3
		\rdg[wit={GrB,Jyo}]{janmane}% +M1
		}/}\\+}
\tl{
\pada{manonmanyai namas tubhyaṃ} % °nye P11,G5
\pada{mahā\app{\lem[wit={G11,G5,N19,V15,P11,C6}]{śakti}
		\rdg[wit={V3}]{śakte}
		\rdg[wit={Jyo}]{śaktyai}% +F
		}%
	\app{\lem[wit={N19,V15,C6,V3,Jyo}]{cidātmane}
		\rdg[wit={P11}]{cidātmani}% +G7 
		\rdg[wit={G11}]{cidātmike}% °ātmake M3
		\rdg[wit={G5}]{cidātmine}}//\versenr}
	\\!}
\end{alttlg}
%</vs11-30>

%<*vs11-31>
\begin{alttlg}[hp04_011_31]
\tl{
\pada{\app{\lem[wit={G5,N19,V15,P11,Jyo}]{aśakya}% +G5
		\rdg[wit={G11,J10}]{aśakyaṃ}
		\rdg[wit={C6,V3}]{aśakta}}tattvabodhānāṃ} % ṃ om. G11
\pada{\app{\lem[wit={G11,G5,N19,V15,J10,C6,V3,Jyo},alt={mūḍhānām}]{mūḍhānā\skp{m}}
		\rdg[wit={P11}]{gūḍhānām}}%
	\app{\lem[wit={G11,G5,J10,GrB,Jyo},alt={api saṃmatam}]{\skm{m }api saṃmatam}
		\rdg[wit={V15}]{api saṃtataṃ}
		\rdg[wit={N19}]{atisaṃtataṃ}}/} \anm{cf. \ref{saukhya}ab}\\+}
\tl{
\pada{proktaṃ 
	\app{\lem[wit={N19,V15,P11,C6,V3,J10,Jyo}]{gorakṣa}
		\rdg[wit={G11,G5}]{śrīśaṃbhu}}nāthena}
\pada{nādopāsana%m % nāptepā° P11
	\app{\lem[wit={G11,G5,N19,V15,J10,P11,V3,Jyo},alt={ucyate}]{\skm{m }ucyate}
		\rdg[wit={C6}]{uttamam}}//\versenr}\\!}
\end{alttlg}
%</vs11-31>
\endaltrecension
\fi


%<*vs12>
\begin{tlg}[hp04_012]
\tl{
\pada{\app{\lem[wit={ceteri}]{śrīādināthena}% nāthona N23
	\rdg[wit={G11,G5}]{śrīśaṃbhunāthena}} 
	sapādakoṭi}-\\+}% koṭī G11
\tl{
\pada{\app{\lem[wit={ceteri}]{laya}
		\rdg[wit={N3,Gr2,N19}]{layaḥ}
		\rdg[wit={J5}]{laṣa}
		}prakārāḥ % prakār<<āḥ>> J10
		kathitā  % kathaṃ J10
	\app{\lem[wit={N3,J5,G11,G5,N19}]{jayante}% +F
		\rdg[wit={Gr2,E2,V15,J10,GrB,Jyo}]{jayanti} % jayati V17
		\rdg[wit={V19}]{yayaṃti}}/}\\+}
\tl{
\pada{nādānusandhānaka%m % nodānasaṃdhānasaṃdhānakam N19
	\app{\lem[wit={N3,G11,G5,P11,C6,Jyo},alt={ekam eva}]{\skm{m }ekam eva}% +F
		\rdg[wit={J5,V3}]{eva}
		\rdg[wit={N19,J10}]{eva nānyaṃ}
		\rdg[wit={V15}]{eva mānyaṃ}
		\rdg[wit={Gr2,Gr3}]{eva kāryaṃ}}}\\+} % °karmeva kārya N23
\tl{
\pada{\app{\lem[wit={ceteri}]{manyāmahe} % manyāṃmahe N19
	\rdg[wit={C6}]{gaṇyāmahe}}
	\app{\lem[wit={N3,N19,V15,P11,V3}]{mānyatamaṃ}% myanya° P11
		\rdg[wit={J5,Gr2,Gr3,G11,G5}]{nānyatamaṃ}% +F
		\rdg[wit={C6}]{nānyamataṃ}
		\rdg[wit={J10}]{tātarasaṃ}
		\rdg[wit={Jyo}]{mukhyatamaṃ}} 
		layānām//\versenr}\label{sapadakoti}
		\orgvnr{12}\\!} % layāṇāṃ J10
\end{tlg}
%</vs12>
\commcite\newpage


\Anm{\getsiglum{G11plus,N19,V15,J10,GrB} have \ref{sravanaputa} \textit{śravaṇamukhanayana} here}


%<*vs13>
\begin{tlg}[hp04_013]
\tl{%\app{\lem[nolem]{}
	%\rdg[wit={Jyo},alt=\textapp{found betw. X4.72 and X4.73}]{\skp{found betw. X4.72 and X4.73}}}%
\pada{	\app{\lem[nolem]{\skp{pāda a}}
	\rdg[wit={N23a,J7a},alt={\om}]{\skp{\om}}}%
\app{\lem[wit={ceteri}]{muktāsanasthito}% +K3,C7
	\rdg[wit={V19a,Jyo}]{muktāsane sthito} % or: °sāna? V19
	\rdg[wit={N23b}]{mudrāsanasthite}} yogī}
\pada{\app{\lem[nolem]{\skp{pāda b}}
	\rdg[wit={N23a,J7a},alt={\om}]{\skp{\om}}}%
mudrāṃ saṃdhāya śāmbhavīm/}% sadhāya? V19
% mudrā P11,N23b,V3,J10; śāṃbhavī Gr2b,P11,V3, sāṃbhavī J10
	\\+}
\tl{
\pada{śṛṇuyād dakṣiṇe karṇe} % śṛṇuyā J7a,V3, śuṇu° V19a, śruṇu° V19b,V15
\pada{\app{\lem[wit={ceteri},alt={nādam}]{nāda\skp{m}}
	\rdg[wit={C6}]{\_\,\_}}%
\app{\lem[resp=emend,alt={antaḥstham ekadhīḥ}]{\skm{m }antaḥstham ekadhīḥ}% 
	\rdg[wit={N3,G4,N23a,J7a,P11,Jyo}]{antastham ekadhīḥ}% aṃtta G4, ekadhī P11
	\rdg[wit={J5}]{atastham ekadhā}
	\rdg[wit={V19a}]{ekāntake sudhīḥ}
	\rdg[wit={E2a}]{ekāntike sudhīḥ}% +K3,C7
	\rdg[wit={N23b,J7b,V19b,E2b,G11,G5,N19,V15}]{antargataṃ sadā}
	\rdg[wit={C6}]{nādamataṃ sadā}
	\rdg[wit={J10,V3}]{antargataṃ mahat}
	}//\versenr}\myfn{%
	This verse is found twice in \getsiglum{Gr2,Gr3}: %
	first (a) after 4.12\,=\,X4.72, and second (b) after 4.36\,=\,X4.84.}
	\orgvnr{13}\\!}
\end{tlg}
%</vs13>
\commcite%\newpage



\Anm{\getsiglum{Jyo} has \ref{sravanaputa} \textit{śravaṇamukhanayana} here}


\Anm{\getsiglum{G11,N19,V15,J10} have the following 5 verses after \ref{yatrakutrapi}, and \getsiglum{GrB} after \ref{muktasana}}


%<*vs14>
\begin{tlg}[hp04_014]
\tl{
\pada{\app{\lem[wit={ceteri}]{kāṣṭhe}
		\rdg[wit={J7,Gr3,C6}]{kāṣṭhaiḥ}
		\rdg[wit={N23}]{kaṣṭaiḥ}}
	\app{\lem[wit={ceteri}]{pravartito}% pravartte J5
		\rdg[wit={V15,J10}]{pravartate}} vahniḥ} % vahni J5,P11,V3
\pada{\app{\lem[wit={ceteri}]{kāṣṭhena}
		\rdg[wit={N23}]{kaṣṭena}}
	\app{\lem[wit={ceteri}]{saha}
		\rdg[wit={V15}]{sa}}
	\app{\lem[wit={ceteri}]{śāmyati}
		\rdg[wit={N3,J5,V19,V3}]{sāmyati}
		\rdg[wit={V15}]{līyate}}/}\\+}
\tl{
\pada{\app{\lem[wit={ceteri}]{nāde}% nādena J5
		\rdg[wit={N23}]{nā}}
	\app{\lem[wit={ceteri}]{pravartitaṃ} %°taṃś N19
		\rdg[wit={V15}]{pravartite}
		\rdg[wit={J10}]{pravartate}}
	\app{\lem[wit={ceteri}]{cittaṃ}
		\rdg[wit={N23},alt={\om}]{\skp{\om}}}}
\pada{nādena saha līyate//\versenr}\label{kasthe}
\orgvnr{14}\\!}
\end{tlg}
%</vs14>
\commcite\newpage

%<*vs15>
\begin{tlg}[hp04_015]
\tl{\app{\lem[nolem]{}
	\rdg[wit={J10,Jyo},alt={\om}]{\skp{\om}}}%
\pada{\app{\lem[wit={ceteri}]{vismṛtya}
%		\rdg[wit={C7}]{nismṛtya}
		\rdg[wit={E2}]{niḥsṛtya}
		} sakalaṃ bāhyaṃ} % bāhya N19
\pada{\app{\lem[wit={N3,J5,J7,Gr3,V15,GrB}]{nāde}% +G5
		\rdg[wit={G11}]{nādo}
		\rdg[wit={N19}]{nāda}
		\rdg[wit={N23}]{na\,\_}}
	\app{\lem[wit={ceteri}]{dugdhāmbu}
		\rdg[wit={N23}]{gugyāṃbu}}va%n
	\app{\lem[wit={ceteri},alt={manaḥ}]{\skm{n }manaḥ}
		\rdg[wit={V3}]{mana}
		\rdg[wit={N23,Gr3}]{naraḥ}}/}\\+}
\tl{
\pada{\app{\lem[wit={G4,Gr2,E2,G11,N19,V15,C6}]{ekībhūyātha}
		\rdg[wit={J5}]{ekībhūyotha}
		\rdg[wit={P11}]{ekībhūyādya}
		\rdg[wit={V19}]{ekībhūyāya}
		\rdg[wit={V3}]{ekībhūyā}
		\rdg[wit={N3}]{ekībhūtvātha}}
	\app{\lem[wit={ceteri}]{sahasā}
		\rdg[wit={V3}]{sahasā ca}
		\rdg[wit={J5}]{manasā}}}
	% ekībhūyād atha saha cidā(page break)ekībhūyātha sahasā J7
\pada{\app{\lem[wit={cetwG4}]{cidākāśe}
		\rdg[wit={J5}]{cidāśe}
		\rdg[wit={N23}]{vidāktośe}
		\rdg[wit={J7}]{cidākaro}} 
\app{\lem[wit={ceteri}]{vilīyate}
	\rdg[wit={N3}]{valīyate}
	\rdg[wit={G4}]{na lipyate}}//\versenr}
	\orgvnr{15}\\!}
\end{tlg}
%</vs15>
\commcite%\newpage

%<*vs16>
\begin{tlg}[hp04_016]
\tl{\app{\lem[nolem]{}
	\rdg[wit={Jyo},alt={\om}]{\skp{\om}}}%
\pada{\app{\lem[wit={Gr3,G11,J10,P11}]{audāsīnya}
		\rdg[wit={V15}]{audāsinya}
		\rdg[wit={G4}]{audāśinya}
		\rdg[wit={C6}]{audāsīna}
		\rdg[wit={N23}]{odāsīnya}
		\rdg[wit={J7,V3}]{udāsīnya}
		\rdg[wit={J5}]{udāsinya}
		\rdg[wit={N3}]{udāsonya}
		\rdg[wit={N19}]{ṛdāsīnya}}paro bhūtvā}
\pada{sadābhyāsena saṃyamī/}\\+} % °bhyosena V3
\tl{
\pada{unmanī\app{\lem[wit={N3,Gr2,Gr3,P11,C6}]{karaṇaṃ}
		\rdg[wit={V3}]{karaṇa}
		\rdg[wit={J5}]{karaṇe}
		\rdg[wit={G11,N19,V15,J10}]{kārakaṃ}} sadyo} % sadyā N3
\pada{\app{\lem[wit={ceteri},alt={nādam}]{nāda\skp{m}}
		\rdg[wit={N19}]{bhāda}}%
	\app{\lem[wit={ceteri},alt={evāvadhārayet}]{\skm{m }evāvadhārayet}
		\rdg[wit={J5}]{evāvadhārayan}
		\rdg[wit={V15}]{eva sadābhyaset}}//\versenr}
	\orgvnr{16}\\!}
\end{tlg}
%</vs16>
\commcite\newpage

%<*vs17a>
\begin{ava}[hp04_017a]
\app{\lem[nolem]{}
	\rdg[wit={Jyo},alt={\om}]{\skp{\om}}}%
\app{\lem[wit={N3,N23,G11,P11},alt={kīdṛśam},post=\texteng{(ki° \getsiglum{N3})}]{kīdṛśa\skp{m}} % ki° N3
%		\rdg[wit={C7}]{kīdṛṣam}
		\rdg[wit={J5,J7}]{kīdṛśīm}
		\rdg[wit={C6,V3}]{kīdṛśyam}% +K3
		\rdg[wit={N19,J10}]{idṛśam}% +F
		\rdg[wit={V19}]{kim}
		\rdg[wit={E2,V15},alt={\om}]{\skp{\om}}}%
	\app{\lem[wit={N23,J7,V19,G11,J10,GrB},alt={audāsīnyam}]{\skm{m }audāsīnyam} % °śīnyaṃ V3
		\rdg[wit={N19,V15}]{audāsinyaṃ}
		\rdg[wit={N3}]{audasīnyaṃ}
		\rdg[wit={J5}]{audāsinyā}
		\rdg[wit={E2}]{athaudāsīnyam}}/ %\NotIn{Jyo}
\end{ava}
%</vs17a>
%<*vs17>
\begin{tlg}[hp04_017]
\tl{\app{\lem[nolem]{}
	\rdg[wit={Jyo},alt={\om}]{\skp{\om}}}%
\pada{\app{\lem[wit={ceteri}]{śīte} % sīte N19,V3
		\rdg[wit={V15}]{śīti}
		\rdg[wit={J5}]{śīta}
		\rdg[wit={J10}]{jñāte}}
	\app{\lem[wit={ceteri}]{kāle}
		\rdg[wit={J7}]{kāla}
		\rdg[wit={J10}]{kā}
		\rdg[wit={J5}]{rakṣa°}
		\rdg[wit={N3},alt={\om}]{\skp{\om}}}
	\app{\lem[wit={J10,V3}]{caupaṭī vā kuṭī vā}% =J6
		\rdg[wit={P11}]{copaṭī vā kuṭī vā}
		\rdg[wit={C6}]{cāpaṭī vā kuṭī vā}
		\rdg[wit={G11}]{dvaupaṭī vā kuṭī vā}% caupaṭī vā kuṭī vā? G5,G7,M3; dvau paṭau vā kuṭī vā F
		\rdg[wit={N3}]{caupaṭī vā paṭī vā}
		\rdg[wit={N19}]{copaṭī vā paṭī vā}
		\rdg[wit={J7,E2}]{cāpaṭī vā paṭī vā}% cāpaṭe E2ac
		\rdg[wit={V19}]{cāpaṭī vā paṭīkā}
		\rdg[wit={N23}]{cāpaṭī cāpaṭī vā}% +C7
		\rdg[wit={V15}]{paṭī vā}
		\rdg[wit={J5}]{°ṇe kathā vā paṭī vā}}}\\+}
\tl{
\pada{\app{\lem[wit={N3,J5,E2,G11,N19,P11,V3}]{pathyāhāre}
		\rdg[wit={J7,V15,J10,C6}]{pathyāhāro}% +K3,C7
		\rdg[wit={N23}]{yathāhārā}
		\rdg[wit={V19}]{<<mi>>thyāhāro}}
	\app{\lem[wit={ceteri}]{gopayo}% +K3
		\rdg[wit={V19}]{gopatho}
%		\rdg[wit={E2,C7}]{gomayo}% +F
		}
	\app{\lem[wit={ceteri}]{vā}
		\rdg[wit={J10}]{co}
		\rdg[wit={N23},alt={\om}]{\skp{\om}}}
	\app{\lem[wit={ceteri}]{payo vā}
		\rdg[wit={N23}]{<<payo>> vā}
		\rdg[wit={V19}]{patho vā}
		\rdg[wit={C6}]{°tha pānaṃ}}/}\\+}
\tl{
\pada{\app{\lem[wit={Gr1,G11,P11,V3}]{bhojye}% G11pc,G5
		\rdg[wit={V15,J10}]{bhojyaṃ}% +M3
		\rdg[wit={N19}]{bhojya}
		\rdg[wit={Gr2}]{bhakṣe}
		\rdg[wit={V19,C6}]{bhakṣyaṃ}
		\rdg[wit={E2}]{bhikṣye}}
	\app{\lem[wit={ceteri}]{bhikṣā} % bhīkṣā V15, bhikṣyā N3,P11
		\rdg[wit={J10}]{bhuktaṃ}}%
	\app{\lem[wit={ceteri},alt={vṛndam}]{vṛnda\skp{m}}% vṛndapā° E2
		\rdg[wit={P11}]{mṛdam}
		\rdg[wit={G11plus}]{kandam}
		\rdg[wit={J10}]{cānnam}}%
	\app{\lem[wit={Gr1,J7,Gr3,V15},alt={āraṇyakandaṃ}]{\skm{m }āraṇyakandaṃ}% +F
		\rdg[wit={N19,J10,V3},alt={°kaṃda}]{āraṇyakaṃda}
		\rdg[wit={P11},alt={°kaṃdā}]{āraṇyakaṃdā}
		\rdg[wit={N23}]{āramyakaṃdaṃ}
		\rdg[wit={G11plus}]{āraṇyakaṃ vā}% +N22,N12
		\rdg[wit={C6}]{āpaṇyakaṃ vā}}}\\+}
\tl{
\pada{\app{\lem[wit={N3,J7,Gr3,G11,P11}]{pāṇī droṇī}
		\rdg[wit={J5,V15,J10}]{pāṇi droṇī}
		\rdg[wit={G4}]{pāṇi droṇi}
		\rdg[wit={N19}]{pāṇī drāṇi}
		\rdg[wit={N23}]{pāṇīndrāṇī}
		\rdg[wit={C6}]{pāṇiṃ droṇe}
		\rdg[wit={V3}]{pāṇi}}
	\app{\lem[wit={N3,G4,G11,N19,V15,P11}]{kāpi vā}
		\rdg[wit={V3}]{kāpivāṃ}
		\rdg[wit={J10}]{kāthivā}
		\rdg[wit={J5}]{vā kapī}
		\rdg[wit={E2}]{karparā}% +K3,C7
		\rdg[wit={C6}]{karpaṭaṃ}
		\rdg[wit={J7}]{kāpaṭo}
		\rdg[wit={N23}]{khapaḍā}
		\rdg[wit={V19}]{kharparo}}
	\app{\lem[wit={J5,G4,G11,N19,P11}]{bhojyapātre}% patre P11
		\rdg[wit={N3,Gr3,V15,J10,V3}]{bhojyapātraṃ}% +F
		\rdg[wit={C6}]{bhojapatraṃ}
		\rdg[wit={N23}]{bhājapatraṃ}
		\rdg[wit={J7}]{bhūrjapatraṃ}}//\versenr}
	\orgvnr{17}\\!}
\end{tlg}
%</vs17>

\avacite{17a}
\commcite\newpage


%<*vs18>
\begin{tlg}[hp04_018]
\tl{\app{\lem[nolem]{}
	\rdg[wit={Jyo},alt={\om}]{\skp{\om}}}%
\pada{\app{\lem[wit={J7,Gr3,G11,N19}]{sarvacintāṃ}
		\rdg[wit={N3,J5,V15,J10,GrB}]{sarvacintā}
		\rdg[wit={N23},alt={\om}]{\skp{\om}}}
	\app{\lem[wit={J5,N19,V15,J10,P11,V3}]{samutsṛjya}% +M3
		\rdg[wit={G11}]{samṛtsṛjya}
		\rdg[wit={N3}]{samutyajya}
		\rdg[wit={J7,Gr3,C6}]{parityajya}% +G5
		\rdg[wit={N23},alt={\om}]{\skp{\om}}}}
\pada{sarva\app{\lem[wit={N3,G11,V15,GrB}]{ceṣṭāṃ}
		\rdg[wit={J5}]{ceṣṭā}
		\rdg[wit={J10}]{ceṣṭāś}
		\rdg[wit={N19}]{ceṣṭī}
		\rdg[wit={Gr2,Gr3}]{kāle}} ca sarvadā/}\\+}
\tl{
\pada{nādam % -m- is a hiatus bridge.
	evānu\app{\lem[wit={N3,P11,C6},alt={°saṃdhānān}]{\skp{°}saṃdhānā\skp{n}}
		\rdg[wit={V3}]{saṃdhānā}
		\rdg[wit={J5,G11,N19,V15,J10}]{saṃdadhyān} % °dhyā N19, saṃ<<da>>dhyān J10
		\rdg[wit={Gr2,Gr3}]{saṃdhatte}}}%n % dhartte N23
\pada{\app{\lem[wit={ceteri},alt={nāde}]{\skm{n }nāde}
		\rdg[wit={C6}]{devi}} cittaṃ vilīyate//\versenr} % vittaṃ vileyate P11
		\label{sarvacinta}
	\orgvnr{18}\\!}
\end{tlg}
%</vs18>
\commcite\newpage

%\Anm{\getsiglum{G11,N19,V15,J10} have \ref{sarvacinta2} and \ref{makaranda1}--64 after these verses.}


%<*vs19>
\begin{tlg}[hp04_019]
\tl{
\pada{ārambha%ś  % araṃbha° P7
	\app{\lem[wit={ceteri},alt={ca}]{\skm{ś }ca}
		\rdg[wit={V19}]{ca\,\_}}
	\app{\lem[wit={ceteri},alt={ghaṭaś}]{ghaṭa\skp{ś}}
		\rdg[wit={N23}]{gha\,\_\,ś}}%
	\app{\lem[wit={ceteri},alt={caiva}]{\skm{ś }caiva}
		\rdg[wit={J10}]{caivas}
		\rdg[wit={V19}]{ca}}}
\pada{tathā
	\app{\lem[wit={N3,G4,G11,N19,J10,GrB},alt={paricayas}]{paricaya\skp{s}}
		\rdg[wit={V15}]{paricas}
		\rdg[wit={J5,N23,Gr3,Jyo}]{paricayo}
		\rdg[wit={J7}]{pariyo}}%
	\app{\lem[wit={N3,V15,V3},alt={tathā}]{\skm{s }tathā}% =ŚS; trayaṃ N24; +F
		\rdg[wit={G4,G11,N19,J10,P11,C6}]{tataḥ}% +M3, smṛtaḥ G5
		\rdg[wit={V19}]{pi vā}
		\rdg[wit={J5,Gr2,E2,Jyo}]{'pi ca}}/}\\+}
\tl{
\pada{niṣpattiḥ % niṣpaṃti N23
	\app{\lem[wit={ceteri}]{sarvayogeṣu}
		\rdg[wit={E2}]{sarvayoge ca}
		\rdg[wit={GrB}]{ceti yogeṣu}
		}}
\pada{\app{\lem[wit={N3,G4}]{yogāvasthā bhavanti tāḥ}
		\rdg[wit={J5}]{yogāvasthā bhavanti te}
		\rdg[wit={Gr2,Gr3}]{yogāvasthā prakīrtitā}
		\rdg[wit={G11,N19,V15,J10,GrB,Jyo}]{syād avasthācatuṣṭayaṃ}% +F
		}\marma//\versenr}
		\orgvnr{19}\\!}
\end{tlg}
%</vs19>
\commcite\newpage


%<*vs20a>
\begin{ava}[hp04_020a]
\app{\lem[nolem]{}
	\rdg[wit={N3,J5,GrB},alt={\om}]{\skp{\om}}}%
\app{\lem[resp=emend]{tatrārambhāvasthā}
		\rdg[wit={G4,N19,V15}]{tatra ārambhaḥ}
		\rdg[wit={G11}]{tatrārambhaḥ}
		\rdg[wit={J10}]{tatra cārambhaḥ}
		\rdg[wit={N23,Jyo}]{athārambhāvasthā}% atha a° Jyo
		\rdg[wit={V19}]{athārambharakṣā}% +K3,C7
		\rdg[wit={E2}]{athārambhadīkṣā}
		\rdg[wit={J7}]{ārambhāvasthātha}
		%\rdg[wit={N3,J5,GrB},alt={\om}]{\skp{\om}}
		}/
\end{ava}
%</vs20a>
%<*vs20>
\begin{tlg}[hp04_020]
\tl{
\pada{brahma\app{\lem[wit={N3,Jyo},alt={granther}]{granthe\skp{r}}% +J11pc,G5,M3
		\rdg[wit={P11}]{granthe}
		\rdg[wit={E2}]{granthau}
		\rdg[wit={J7,V19,V15,V3}]{granthir}% grathir J7; +K3,C7
		\rdg[wit={N23,C6}]{granthi}
		\rdg[wit={J10}]{granthiṃ}
		\rdg[wit={J5}]{granthid}
		\rdg[wit={G11}]{gra\,+}
		\rdg[wit={N19}]{raṃdhre}
		}r bhave%d % bhavod C6
	\app{\lem[wit={N3,G11,C6,V3},alt={bhedād}]{\skm{d }bhedā\skp{d}}
		\rdg[wit={J5,P11}]{bhedā}
		\rdg[wit={Gr2,V19}]{bhinna}
		\rdg[wit={E2}]{bhinne}
		\rdg[wit={J10}]{bhinnā}
		\rdg[wit={V15}]{bhinnād}
		\rdg[wit={Jyo}]{bhedo hy}
		\rdg[wit={N19}]{bhed}}}%
\pada{\app{\lem[wit={ceteri},alt={ānandaḥ}]{\skm{d }ānandaḥ}
		\rdg[wit={J5,N23,C6}]{ānaṃda}
		\rdg[wit={P11}]{nanādaḥ}
		\rdg[wit={J10}]{nādaḥ}}
	śūnya\app{\lem[wit={ceteri}]{saṃbhavaḥ} % śūnyaṃ J10; °bhava N19, °bhavaṃ P7, °bhavā J5
		\rdg[wit={J10}]{samaṃbhavaḥ}}/}\\+}
\tl{
\pada{vicitra\app{\lem[wit={E2,G11}]{kvaṇako}
		\rdg[wit={N3}]{kvana˟ko}
		\rdg[wit={V15}]{kvaṇiko}
		\rdg[wit={N19,V3}]{kaṇako}
		\rdg[wit={J5}]{kanako}
		\rdg[wit={J10}]{kuṇako}
		\rdg[wit={C6}]{kuṇape}
		\rdg[wit={Jyo}]{°ḥ kvaṇako}
		\rdg[wit={P11}]{°ṣkāṇako}
		%\rdg[wit={K3,C7}]{kṣaṇike}
		\rdg[wit={V19}]{kṣike}
		\rdg[wit={Gr2}]{°s tatkṣaṇād}} % ta<<t>> N23
	\app{\lem[wit={ceteri}]{dehe}
		\rdg[wit={J5}]{deho}
		\rdg[wit={C6}]{caivā}}}% caiva C6ac
\pada{\app{\lem[wit={N3,J5,G11,N19,V15,J10,GrB,Jyo}]{'nāhataḥ śrūyate}% °hata P11
		\rdg[wit={Gr2}]{sarvataḥ śrūyate}
		\rdg[wit={Gr3}]{śrūyate (')nāhata}}
		dhvaniḥ//\versenr}
		\orgvnr{20}\\!} % dhvani J5,P11,V3,J7, ddhavaniḥ N19
\end{tlg}
%</vs20>

\avacite{20a}
\commcite\newpage

%<*vs21>
\begin{tlg}[hp04_021]
\tl{
\pada{\app{\lem[nolem]{\skp{pāda a}}
	\rdg[wit={Gr3,V3},alt={\om}]{\skp{\om}}}%
\app{\lem[wit={N3,J5,Gr2,P11,C6,Jyo}]{divyadehaś ca tejasvī} % tejasvā N3; +G5,M3
		\rdg[wit={G11}]{divyadehasya tejasvī}
		\rdg[wit={N19}]{ādityatejaś ca tejasvī}
		\rdg[wit={V15}]{tejasvī divyagandhaś ca}
		\rdg[wit={J10}]{divyagandho divyacakṣuś ca}
		\rdg[wit={Gr3,V3},alt={\om}]{\skp{\om}}}}
\pada{\app{\lem[nolem]{\skp{pāda b}}
	\rdg[wit={Gr3,V3},alt={\om}]{\skp{\om}}}%
\app{\lem[wit={N3,G4,Gr2,P11,C6,Jyo}]{divyagandhas tv arogavān} % gaṃdhās N3
		\rdg[wit={G11,N19}]{divyagandho py arogavān}% paro° N19
		\rdg[wit={V15}]{divyadeho py arogavān}
		\rdg[wit={J5}]{divyadeham arogavān}
		\rdg[wit={J10}]{tejasvī ārogavān}
		\rdg[wit={Gr3,V3},alt={\om}]{\skp{\om}}}/}
		\\+}
\tl{
\pada{\app{\lem[wit={ceteri}]{saṃpūrṇa}
		\rdg[wit={V15}]{saṃpūrṇe}}%
	\app{\lem[wit={Gr1,N19,P11,Jyo}]{hṛdayaḥ}
		\rdg[wit={J7,G11}]{hṛdaya}
		\rdg[wit={N23,V19,V15,J10,C6,V3}]{hṛdaye}% +J11pc
		%\rdg[wit={E2,C7}]{nilaye}
		}
	\app{\lem[wit={Gr1,N19,V15},alt={śūnye tv}]{śūnye\skp{ tv}}
		\rdg[wit={Gr2,Gr3,G11,J10,C6}]{śūnye}
		\rdg[wit={V3,Jyo}]{śūnya}% +G5,M3,F
		\rdg[wit={P11}]{śūra}}}
\pada{\app{\lem[wit={ceteri},alt={ārambhe}]{\skm{tv }ārambhe}
		\rdg[wit={V3}]{ārambha}
		\rdg[wit={J10}]{āraṃbho}}
	\app{\lem[wit={ceteri},alt={yogavān}]{yogavā\skp{n}}
		\rdg[wit={N23}]{bhogavān}}n bhavet//\versenr}
		\orgvnr{21}\\!}
\end{tlg}
%</vs21>
\commcite\newpage


%<*vs22a>
\begin{ava}[hp04_022a]
atha
\app{\lem[wit={ceteri}]{ghaṭāvasthā}% ghaṭavaṃsthā N19, ghaṃṭā° V17
		\rdg[wit={G4}]{khaṭavasthā}
		\rdg[wit={J5}]{ghaṭā arthaḥ}
		\rdg[wit={Gr3}]{ghaṭarakṣā}
		\rdg[wit={P11}]{ghaṭaḥ}}/
\end{ava}
%</vs22a>
%<*vs22>
\begin{tlg}[hp04_022]
\tl{
\pada{\app{\lem[wit={ceteri}]{dvitīyāyāṃ} % dviti° V3 % N3,Gr2,E2,G11,N19,V15pc,GrB,Jyo
		\rdg[wit={V19,V15ac}]{dvitīyā}
		\rdg[wit={J10}]{dvitīye}
		\rdg[wit={J5}]{dvitī}}
	\app{\lem[wit={N3,Gr2,Gr3,GrB,Jyo}]{ghaṭī}
		\rdg[wit={V15}]{ghaṃṭi}
		\rdg[wit={N19}]{ghaṭāṃ}
		\rdg[wit={J5}]{ghaṭikā}
		\rdg[wit={G11plus}]{sphuṭī}
		\rdg[wit={J10}]{bheda}}%
	\app{\lem[wit={ceteri}]{kṛtya}% N3,J5,Gr2,Gr3,G11,N19,GrB,Jyo
		\rdg[wit={V15}]{kṛtvā}
		\rdg[wit={J10}]{mukte tu}}}
\pada{vāyur bhavati madhyagaḥ/}\\+} % vāḥsur P7
%	\app{\lem[wit={ceteri}]{madhyagaḥ}
%		\rdg[wit={K3,C7}]{madhyamaḥ}}/}\\+}
\tl{
\pada{\app{\lem[wit={ceteri}]{dṛḍhāsano}
		\rdg[wit={J10}]{haṭhāsano}} bhaved yogī}
\pada{jñānī
	\app{\lem[wit={N3,J5,Gr2,V19,G11,N19,V15,Jyo}]{deva}
		\rdg[wit={V3}]{devaḥ}
		\rdg[wit={E2,J10,P11,C6}]{deha}}sama%s  % samās N3
	\app{\lem[wit={N3,J5,GrB,Jyo},alt={tadā}]{\skm{s }tadā}
		\rdg[wit={Gr2,Gr3,G11,N19,V15,J10}]{tathā}}//\versenr}
		\orgvnr{22}\\!}
\end{tlg}
%</vs22>

\avacite{22a}
\commcite\newpage


%<*vs23>
\begin{tlg}[hp04_023]
\tl{
\pada{viṣṇu\app{\lem[wit={N3,P11}]{granthes tadā}% +N24,F
		\rdg[wit={V3}]{granthis tadā}
		\rdg[wit={N19}]{granthe sadā}
		\rdg[wit={J5,J10}]{granthes tathā}% +G5
		\rdg[wit={G11}]{granthe tathā}
		\rdg[wit={C6}]{granther yadā}
		\rdg[wit={Gr2,Gr3,V15}]{granthir yadā}% <<graṃthi>> C7
		\rdg[wit={Jyo}]{granthes tato}}
	\app{\lem[wit={N3,G11,N19,J10,GrB,Jyo},alt={bhedāt}]{bhedā\skp{t}}
		\rdg[wit={J5}]{bhidā}
		\rdg[wit={Gr2,Gr3}]{bhinnaḥ} % vimugraṃthipadābhinna N23
		\rdg[wit={V15}]{bhinnā}}}%
\pada{\app{\lem[wit={ceteri},alt={paramānanda}]{\skm{t }paramānanda}
		\rdg[wit={N19}]{sadānandasya}}%
	\app{\lem[wit={ceteri}]{sūcakaḥ} % śū° N23, °caka V3
		\rdg[wit={V15}]{sūcakā<<ḥ>>}
		\rdg[wit={C6}]{kārakaḥ}}/}\\+}% °kāḥ V15pc
\tl{
\pada{\app{\lem[wit={Gr1,G11,P11,V3,Jyo}]{atiśūnye}% śunye N3
		\rdg[wit={Gr2,Gr3,V15,J10}]{atiśūnya}
		\rdg[wit={C6}]{aṃtyaśūnye}
		\rdg[wit={N19}]{api śūnyo}}
	\app{\lem[wit={N3,G4,GrB,Jyo}]{vimardaś ca}
		\rdg[wit={J5}]{vimardasya}
		\rdg[wit={N19}]{'saṃmardā}
		\rdg[wit={G11}]{visanmarde}% 'pi saṃmardo G5
		\rdg[wit={J10}]{visaṃmardo}
		\rdg[wit={Gr2,Gr3,V15}]{vibhedaś ca}}}
\pada{bherīśabda%s % sabdas V3
	\app{\lem[wit={N3,V15,GrB,Jyo},alt={tadā}]{\skm{s }tadā}
		\rdg[wit={G4,Gr2,Gr3,G11,N19,J10}]{tathā}% +N24
		\rdg[wit={J5}]{tatho}
		} bhavet//\versenr}
		\orgvnr{23}\\!}
\end{tlg}
%</vs23>
\commcite\newpage


%<*vs24a>
\begin{ava}[hp04_024a]
\app{\lem[wit={ceteri}]{atha}
		\rdg[wit={C6}]{tathā}
		\rdg[wit={E2},alt={\om}]{\skp{\om}}}
\app{\lem[wit={ceteri}]{paricayāvasthā}% <<yā>> N23
		\rdg[wit={N19,V15,P11}]{paricayaḥ}}/ % °caya P11,N19
\end{ava}
%</vs24a>
%<*vs24>
\begin{tlg}[hp04_024]
\tl{
\pada{\app{\lem[wit={N3,Gr3,G11,V15,GrB}]{tṛtīyāyāṃ tato bhittvā} % tṛtiyāyāṃ V3
		\rdg[wit={J5}]{tṛtīyāyāṃ tathā bhitvā}
		\rdg[wit={Gr2}]{karṇikāṃ tu tato bhittvā}
		\rdg[wit={N19}]{karttikāyāṃ tato bhittvā}
		\rdg[wit={J10}]{atha granthitrayaṃ bhittvā}
		\rdg[wit={Jyo}]{tṛtīyāyāṃ tu vijñeyo}}}
\pada{\app{\lem[nolem]{\skp{pāda b}}
	\rdg[wit={N3},alt={\unavbl}]{\skp{\unavbl}}}%
\app{\lem[wit={J5,G11,N19,Jyo}]{vihāyo}
		\rdg[wit={Gr2,V15}]{vihāya}
		\rdg[wit={P11}]{vikāryo}
		\rdg[wit={Gr3}]{vimalo}
		\rdg[wit={V3}]{vimāyo}
		\rdg[wit={C6}]{visphāro}
		\rdg[wit={J10}]{jāyate}}%
	\app{\lem[wit={J5,Gr2,N19,J10,GrB,Jyo}]{mardala}
		\rdg[wit={G11}]{maddala}
		\rdg[wit={Gr3}]{mandala} % maṃdala V19
		\rdg[wit={V15}]{mṛḍula}}%
	\app{\lem[wit={ceteri}]{dhvaniḥ}
		\rdg[wit={J7}]{dhvaniṃ}
		\rdg[wit={P11,V3}]{dhvani}}/}\marma\\+}
\tl{
\pada{\app{\lem[nolem]{\skp{pāda c}}
	\rdg[wit={N3},alt={\unavbl}]{\skp{\unavbl}}}%
\app{\lem[wit={ceteri}]{mahāśūnyaṃ}
		\rdg[wit={V15,P11}]{mahāśūnya}
		\rdg[wit={G11}]{mahāśūnyas}
		}
	\app{\lem[wit={J5,G11,GrB,Jyo}]{tadā}% +N24
		\rdg[wit={Gr2,N19}]{tathā}
		\rdg[wit={Gr3}]{tato}
		\rdg[wit={V15}]{tayā}
		\rdg[wit={J10}]{samā}}
	\app{\lem[wit={ceteri}]{yāti}
		\rdg[wit={J5}]{jāti}
		\rdg[wit={N19}]{jātiḥ}}}
\pada{\app{\lem[nolem]{\skp{pāda d}}
	\rdg[wit={N3},alt={\unavbl}]{\skp{\unavbl}}}%
\app{\lem[wit={ceteri}]{sarvasiddhi} % +P11
		\rdg[wit={V3}]{mahāsiddhi}
		\rdg[wit={C6}]{siddhisādha°}
		\rdg[wit={N19}]{sarva}}% sarva<siddhi> N19
	\app{\lem[wit={ceteri}]{samāśrayam}
		\rdg[wit={P11}]{samāśriyaṃ}
		\rdg[wit={J5}]{matāśrayāt}
		\rdg[wit={C6}]{kam āśrayaṃ}}//\versenr}\label{cittananda}
	%\anm{The folio which should contain pāda b--\ref{nadanu}d is lost in \getsiglum{N3}.}
	%\unavbl{N3}
	\orgvnr{24}\\!}
\end{tlg}
%</vs24>

\avacite{24a}
\commcite\newpage

%<*vs25>
\begin{tlg}[hp04_025]
\tl{\app{\lem[nolem]{}
	\rdg[wit={N3}]{\unavbl}}%
\pada{\app{\lem[wit={G4,Gr2,Gr3,G11,C6,Jyo}]{cittānandaṃ}% <<da>> C7; +N24
		\rdg[wit={J5,V15,V3}]{cidānaṃda}% cidānaṃdaṃ HR,F, cidātmānaṃ M3, cittamānaṃ G5 ##
		\rdg[wit={P11}]{vivarttānaṃdaṃ}
		\rdg[wit={J10}]{ciṃtāmanas}
		\rdg[wit={N19}]{virāmānaṃ}}
	\app{\lem[wit={ceteri}]{tato}
		\rdg[wit={Jyo}]{tadā}}
	\app{\lem[wit={ceteri}]{jitvā}% J5,G4,G11,N19,V15,J10,GrB,Jyo
		\rdg[wit={Gr2,Gr3}]{bhittvā}}}
\pada{sahajānanda%
	\app{\lem[wit={ceteri}]{saṃbhavaḥ}
		\rdg[wit={N19,P11}]{saṃbhava}}/}\\+}
\tl{
\pada{\app{\lem[wit={ceteri}]{doṣaduḥkha}
		\rdg[wit={P11}]{doṣaduḥkhaṃ}
		\rdg[wit={N23}]{dokhaduḥkhe}}%
	\app{\lem[wit={G4,G11,V15,J10,GrB}]{jarāmṛtyu}
		\rdg[wit={J5,N19}]{jarāmṛtyuḥ}
		\rdg[wit={Jyo}]{jarāvyādhi}
		\rdg[wit={Gr2,Gr3}]{kṣudhānidrā}}}%
\pada{\app{\lem[wit={J5,G4,G11,N19,V15,J10,P11,C6,Jyo}]{kṣudhānidrā}
		\rdg[wit={V3}]{kṣudhātṛṣā}
		\rdg[wit={Gr2,Gr3}]{jarāmṛtyu}}% mṛrtyu N23
	\app{\lem[wit={ceteri}]{vivarjitaḥ}
		\rdg[wit={C6},alt={°tāḥ}]{vivarjitāḥ}
		\rdg[wit={V3},alt={°taṃ}]{vivarjitaṃ}
		\rdg[wit={J10}]{tṛṣā tathā}}//\versenr}
	%\unavbl{N3}
	\orgvnr{25}\\!}
\end{tlg}
%</vs25>
\commcite\newpage


%<*vs26a>
\begin{ava}[hp04_026a]
\app{\lem[nolem]{}
	\rdg[wit={N3}]{\unavbl}
	\rdg[wit={Jyo},alt={\om}]{\skp{\om}}
	\rdg[wit={J5,J7,Gr3,GrB},alt=\textapp{found after \ref{rudra}b}]{\skp{found after the first line of \manuref{4.26}}}}%	
	atha \app{\lem[wit={Gr2,C6,V3}]{niṣpattyavasthā} % niḥpatya° N23, niṣpatya° J7
		\rdg[wit={J5}]{niḥpatti-avasthā}
		\rdg[wit={Gr3}]{niṣṭhāvasthā}
		\rdg[wit={G11,N19,V15,J10,P11}]{niṣpattiḥ}% niṣpatti P11, niṣpati N19, °tiḥ J10; +F ##
		}/
\end{ava}
%</vs26a>
%<*vs26>
\begin{tlg}[hp04_026]
\tl{\app{\lem[nolem]{}
	\rdg[wit={N3}]{\unavbl}}%
\pada{rudragranthiṃ % urddha J5; rudraṃ graṃ{{ga}}thi N23; grathiṃ J7, graṃthi C7,N19,P11,V3, grandhiṃ G11
	\app{\lem[wit={ceteri}]{tato}
		\rdg[wit={Jyo}]{yadā}}
	\app{\lem[wit={ceteri}]{bhittvā} % bhītvā J5,P7
		\rdg[wit={N19}]{bhūtvā}}}
\pada{\app{\lem[wit={ceteri}]{sarva}
		\rdg[wit={Jyo}]{śarva}
		\rdg[wit={P11}]{satva}}pīṭha% pīca P11, pīḍa G11
	\app{\lem[wit={ceteri}]{gato'nilaḥ}% +M3
		\rdg[wit={J7}]{gatonalaḥ}
		\rdg[wit={G11}]{gatānilaḥ}% °laṃ G5
		\rdg[wit={J5,V3}]{gatānila}}/}\\+}
\tl{
\pada{\app{\lem[wit={J5,J7,GrB,Jyo}]{niṣpattau}
		\rdg[wit={N19,V15}]{niṣpannau} % niḥṣpannau? V15
		\rdg[wit={G11,J10}]{niṣpanno} % niḥpanno J10
		\rdg[wit={N23}]{niṣpatto}
		\rdg[wit={Gr3}]{niṣṭhāto}}
	\app{\lem[wit={ceteri}]{vaiṇavaḥ śabdaḥ}% veṇavaḥ V17
		\rdg[wit={J7}]{vaiṇavaśabdaḥ}
		\rdg[wit={J5}]{vauṇāvat sado}
		\rdg[wit={N23}]{veṇacaśabdaṃ}}}
\pada{\app{\lem[wit={V15,Jyo}]{kvaṇadvīṇākvaṇo}% viṇā V15; +F
		\rdg[wit={G11}]{kvaṇan vīṇakvaṇo}% kvaṇan* gīta° G5
		\rdg[wit={N19}]{kaṇatvītakvaṇo}% sic!
		\rdg[wit={J7}]{kvaṇadvīṇotvaṇo}% < °vīṇolbaṇo?
		\rdg[wit={P11}]{kvaṇan vītaḥ kvaṇo} 
		\rdg[wit={C6}]{kvacid vīṇākvaṇo}
		\rdg[wit={V3}]{kvaṇatuvītakvaṇo}
		\rdg[wit={J10}]{kvaṇantenākvuṇo}
		\rdg[wit={Gr3}]{kvaṇadvīṇāsamo}% vīnā V19; kvaṇadvīvakvaṇo K1
		\rdg[wit={N23}]{karṇavīṇādgato}
		\rdg[wit={J5}]{kṛṇanityakṛṇo}}\marmas
	\app{\lem[wit={ceteri}]{bhavet}
	\rdg[wit={C6}]{°dayaḥ}}//\versenr}\label{rudra}
	%\unavbl{N3}
	\orgvnr{26}\\!}
\end{tlg}
%</vs26>

\avacite{26a}
\commcite\newpage

% MD: marking für die andere Recensions hier abgebrochen.

%<*vs27>
\begin{tlg}[hp04_027]
\tl{\app{\lem[nolem]{}
	\rdg[wit={N3}]{\unavbl}
	\rdg[wit={N19,V15},alt={\om}]{\skp{\om}}}%
\pada{ekībhūtaṃ % ekāṃ P11, eka° C6
	\app{\lem[wit={J5,G11,GrB,Jyo}]{tadā}
		\rdg[wit={G4,Gr2,Gr3,J10}]{tathā}} cittaṃ}
\pada{\app{\lem[nolem]{\skp{pāda b}}
	\rdg[wit={P11},alt={\om}]{\skp{\om}}}%
\app{\lem[wit={ceteri}]{rājayogā}
		\rdg[wit={J10}]{rājayoga}
		\rdg[wit={V3}]{rājayogo}}%
	\app{\lem[wit={J7,G11,V3},alt={°bhidhāyakam}]{\skp{°}bhidhāyakam}
		\rdg[wit={J5}]{vidhāyakaḥ}
		\rdg[wit={N23}]{bhidhāyanaṃ}
		\rdg[wit={G4,Gr3,J10,C6,Jyo}]{bhidhānakaṃ}% +F
		}\marma/}\\+} % +N24
\tl{
\pada{\app{\lem[nolem]{\skp{pāda c}}
	\rdg[wit={P11},alt={\om}]{\skp{\om}}}%
sṛṣṭisaṃhāra%
	\app{\lem[wit={ceteri}]{kartāsau}
		\rdg[wit={N23}]{karttasau}
		\rdg[wit={V3}]{karttāso}}}
\pada{\app{\lem[nolem]{\skp{pāda d}}
	\rdg[wit={P11},alt={\om}]{\skp{\om}}}%
yogīśvarasamo bhavet//\versenr}%
	%\myfnx{After this verse, \getsiglum{Jyo} has X4.117, 119, and 121ab.}
	\label{ekibhutam}
	%\unavbl{N3}
	\orgvnr{27}\\!}
%	\anm{\getsiglum{C7} in mg. sec. m.}
\end{tlg}
%</vs27>
\commcite%\newpage


\iffalse
%<*vs27-1a>
\begin{altava}[hp04_027_1a]
\app{\lem[nolem]{}
	\rdg[wit={G11,G5}]{\only}}%
	atha nādānusandhānam/
\end{altava}
%</vs27-1a>

\startaltrecension % B1 has this verse too.
%<*vs27-1>
\begin{alttlg}[hp04_027_1]
\tl{%\app{\lem[nolem]{}
	%\rdg[wit={GrB},alt=\textapp{found after \ref{astuva}}]{\skp{found after xxx}}}%
\pada{rājayoga\app{\lem[wit={G5,P11,C6}]{padaprāptau}% +C8,P7,M3,F
		\rdg[wit={G11}]{padaprāptā}
		\rdg[wit={N19}]{padaprāptaḥ}
		\rdg[wit={V3}]{padaṃ prāptaṃ}
		\rdg[wit={J10,Jyo}]{padaṃ prāptuṃ}
		\rdg[wit={V15}]{padaṃ prāpti}
		}}
\pada{\app{\lem[wit={G11,G5,N19,V15,J10,P11,C6,Jyo}]{sukhopāyo'lpa}
		\rdg[wit={V3}]{sukhopāyogya}}cetasām/}\\+} % cetasam P11
\tl{
\pada{sadyaḥpratyaya%
	\app{\lem[wit={N19,J10,C6,V3,Jyo}]{saṃdhāyī}
		\rdg[wit={V15,P11}]{saṃdhāyi}
		\rdg[wit={G11,G5}]{saṃdāyī}}}
\pada{\app{\lem[wit={G5,N19,V15,GrB,Jyo}]{jāyate}% +M3
		\rdg[wit={G11}]{līyate}
		\rdg[wit={J10}]{sevyate}}
	\app{\lem[wit={N19,C6,V3,Jyo}]{nādajo layaḥ} % laya V3
		\rdg[wit={J10,P11}]{nādayo layaḥ}
		\rdg[wit={G5}]{nādamūlayā}
		\rdg[wit={V15}]{nātra saṃśayaḥ}}//\versenr}
	\sgwit{G11,G5,N19,V15,J10,GrB,Jyo}
	\anm{cf. \ref{saukhya}cd}\\!}
\end{alttlg}
%</vs27-1>
\endaltrecension
\fi


\Anm{Verses \ref{astuva}--\ref{svastho} are found before \ref{drsti} in \getsiglum{G11,N19,V15,J10}}

%<*vs28>
\begin{tlg}[hp04_028]
\tl{\app{\lem[nolem]{}
	\rdg[wit={N3}]{\unavbl}
	\rdg[wit={P11},alt={\om}]{\skp{\om}}}%
\pada{astu vā
	\app{\lem[wit={J5,Gr2,Gr3,G11,N19,J10,C6,Jyo}]{māstu} % māsta N19
		\rdg[wit={V15,V3}]{mastu}
		} vā
	\app{\lem[wit={J5,E2,G11,N19,J10,C6,Jyo},alt={muktir}]{mukti\skp{r}}% +K3,C7
		\rdg[wit={V15}]{muktis}
		\rdg[wit={V3}]{muktiṃ}
		\rdg[wit={Gr2}]{śaktir}
		\rdg[wit={V19}]{kiṃcid}}}%
\pada{\app{\lem[wit={Gr3,G11,N19,C6,Jyo},alt={atraivākhaṇḍitaṃ}]{\skm{r }atraivākhaṇḍitaṃ}% °ṣaṃḍitaṃ N19
		\rdg[wit={J5,J7}]{atraiva khaṇḍitaṃ}% ṣaṃḍitaṃ
%		\rdg[wit={J5}]{atraiva ṣaṃḍitaṃ}
		\rdg[wit={J10}]{atra vākhaṇḍitaṃ}
		\rdg[wit={N23}]{ātrevikhaṇḍitaṃ}
%		\rdg[wit={N19}]{atraivāṣaṃḍitaṃ}
		\rdg[wit={V15,V3}]{tatraivākhaṇḍitaṃ}% +F
		}
	\app{\lem[wit={J5,J7,E2,G11,N19,V15,J10,V3}]{mahat}
		\rdg[wit={N23}]{marut}
		\rdg[wit={C6}]{manaḥ}
		\rdg[wit={V19}]{bhavet}
		\rdg[wit={Jyo}]{sukham}}/}\\+}
\tl{
\pada{\app{\lem[wit={J5,G11,N19,V15,C6}]{layāmṛtamayaṃ}% leyā° J5, staye° N24
		\rdg[wit={V3}]{layāmṛtalayaṃ}
		\rdg[wit={J7,Gr3}]{layāmṛtam idaṃ}% +F
		\rdg[wit={N23}]{layāmṛdaṃmitaṃ}
		\rdg[wit={J10}]{layāmṛtakaraṃ}
		\rdg[wit={Jyo}]{layodbhavam idaṃ}}
	% \app{\lem[wit={V19,E2,G11,V15,C6,V3,Jyo}]{saukhyaṃ}
		% \rdg[wit={N23}]{sokhyaṃ}
		% \rdg[wit={J5,J7,J10}]{sauṣyaṃ}
		% \rdg[wit={N19}]{saukṣaṃ}}
		saukhyaṃ}
\pada{\app{\lem[wit={ceteri}]{rājayogād avāpyate}
		\rdg[wit={J10}]{rājayogam avāpyate}
		\rdg[wit={V19},alt={\om}]{\skp{\om}}}//\versenr}%
%\myfn{After this verse, \getsiglum{N19} reads \devnote{460 saṃka} and verses \ref{hathamvina}--\ref{ajananta}. With this, the 346th folio of the ms has just been filled, the text of the \HP ends without a colophon and another text begins on the next folio.}
		\label{astuva}
	%\unavbl{N3}
	\orgvnr{28}\\!}
\end{tlg}
%</vs28>
\commcite\newpage
%	\anm{\getsiglum{N19,V15,J10} after 4.71?}

%\Anm{\getsiglum{V3} has 4.48*1--\ref{muktasana}, \ref{kasthe}--\ref{sarvacinta} etc. here}


%<*vs29>
\begin{tlg}[hp04_029]
\tl{\app{\lem[nolem]{}
	\rdg[wit={N3}]{\unavbl}
	\rdg[wit={Gr2,Gr3,G11plus,J10,V3,Jyo},alt={\om}]{\skp{\om}}}%
\pada{haṭhaṃ vinā rājayogo} % yogaḥ J11, yogaṃ G4
\pada{rājayogaṃ vinā haṭhaḥ/}\\+} % yoga J5,J11; haṭhaṃ P11
\tl{
\pada{na sidhyati tato yugmam}
\pada{ā niṣpatteḥ samabhyaset//\versenr}% +M1,J11; °ttiḥ P11, °tte C6
\anm{= 2.77}
\myfn{This verse is abbreviated with \textit{haṭhaṃ vinā rājayoga iti} in \getsiglum{N19,V15}.}
%\NotIn{Gr2,Gr3,G11plus,J10,V3,Jyo} 
%\anm{= \manuref{2.77}}
	%\unavbl{N3}
	\label{hathamvina}
	\orgvnr{29}\\!}

%\sgwit{J5,G4,N19,V15,P11,C6} 
\end{tlg}
%</vs29>
\commcite%\newpage

% \myfn{\getsiglum{N19} has then \devnote{460 saṃ ka} before its last verse \ref{ajananta}.}

%<*vs30>
\begin{tlg}[hp04_030]
%\teimute{\normalsize}\color{black}
\tl{\app{\lem[nolem]{}
	\rdg[wit={N3}]{\unavbl}
	\rdg[wit={Gr2,Gr3},alt={\om}]{\skp{\om}}}%
\pada{rājayogam ajānantaḥ} % °nataḥ V3, °naṃteḥ N19
\pada{kevalaṃ
	haṭha\app{\lem[wit={G11,V15,P11}]{karmaṭhāḥ}
		\rdg[wit={J5}]{karmaṭhaḥ}
		\rdg[wit={N19}]{karmacā}
		\rdg[wit={C6,V3}]{karmaṇā}
		\rdg[wit={J10}]{karmaṇaḥ}
		\rdg[wit={Jyo}]{karmiṇaḥ}}/}\\+}
\tl{
\pada{\app{\lem[wit={G11,P11,C6}]{ye tu tān karṣakān manye}
		\rdg[wit={N19,V15}]{ye tu tān karkaśān manye}
		\rdg[wit={J5}]{ye ca te kāmukān manne}
		\rdg[wit={J10}]{ye tuṃgān karmavasān manye}
		\rdg[wit={Jyo}]{etān abhyāsino manye}
		\rdg[wit={V3},alt={\lacuna}]{\skp{\lacuna}}}} % V3 gap
\pada{\app{\lem[wit={G11,N19,V15,P11,Jyo}]{prayāsaphalavarjitān}
		\rdg[wit={J10},alt={°varjitāḥ}]{prayāsaphalavarjitāḥ}
		\rdg[wit={J5}]{prayāsakalavarjitaḥ}
		\rdg[wit={C6}]{prāyaśaḥ phalavarjitāḥ}
		\rdg[wit={V3},alt={\lacuna}]{\skp{\lacuna}}}//\versenr}%
	\orgx{\note[type=anmkg,nonum,lem=\mylem{29--30},labelb={4-30-1}]{The \textgamma\ and \textdelta\ manuscripts have the following verse as a substitute for \ref{hathamvina}--\ref{ajananta}:
	\devnote{haṭhaṃ vinā rājayogaṃ rājayogaṃ vinā haṭham/
		ye vai caranti tān manye prayāsaphalavarjitān//}}
	}{\myfnx{\getsiglum{N19} ends with this verse.}}
	\label{ajananta}
	%\unavbl{N3}
	\orgvnr{30}\\!}
%		\sgwit{J5,G4,G11,N19,V15,J10,GrB,Jyo}
\end{tlg}
%</vs30>
\commcite\newpage


%\startaltrecension
%%<*vs30-1>
%\begin{alttlg}[hp04_030_1]
%\tl{\pada{\app{\lem[wit={Gr2,E2}]{haṭhaṃ vinā}
%		 \rdg[wit={V19},alt={\om}]{\skp{\om}}}
%	 \app{\lem[wit={J7,E2}]{rājayogaṃ}
%		 \rdg[wit={N23}]{rājayogo}
%		 \rdg[wit={V19},alt={\om}]{\skp{\om}}}}
%\pada{rājayogaṃ vinā
%	 \app{\lem[wit={J7,Gr3}]{haṭham}
%		 \rdg[wit={N23}]{haṭhaḥ}}/}\\+} % hathaḥ N23ac
%\tl{
%\pada{ye \app{\lem[wit={N23,Gr3}]{vai}
%		 \rdg[wit={J7}]{cai}}
%	 \app{\lem[wit={Gr3}]{caranti}
%		 \rdg[wit={Gr2}]{varaṃti}} tā%n
%	 \app{\lem[wit={N23,Gr3},alt={manye}]{\skm{n }manye}
%		 \rdg[wit={J7}]{madhye}}}
%\pada{prayāsa\app{\lem[wit={J7,Gr3}]{phala} % varjitāḥ J10
%		 \rdg[wit={N23}]{ptalevi}}varjitān//\versenr}
%		 \sgwit{Gr2,Gr3}
%	\myfn{This verse is found in \getsiglum{Gr2,Gr3} as a substitute for \ref{hathamvina}--\ref{ajananta}.}
%	\\!}
%\end{alttlg}
%%</vs30-1>
%\endaltrecension
%
%\altcommcite
%\newpage

%\Anm{\getsiglum{V3} has \ref{ajananta}--\ref{saukhya} after the Kālavañcana section}


%<*vs31>
\begin{tlg}[hp04_031]
\tl{\app{\lem[nolem]{}
	\rdg[wit={N3}]{\unavbl}
	\rdg[wit={E2},alt={\om}]{\skp{\om}}}%
\pada{\app{\lem[wit={ceteri}]{tattvaṃ}% +M3
		\rdg[wit={N23,G11,V3}]{tattva}}\marmas 
		bījaṃ
	\app{\lem[wit={V19,Jyo}]{haṭhaḥ}% +M3
		\rdg[wit={J5,Gr2,G11,V15,P11}]{haṭha}% +J11
		\rdg[wit={G4,J10,C6,V3}]{haṭhaṃ}% +F,K3,C7
		} kṣetra}%m 
\pada{\app{\lem[wit={Gr2,J10,C6,V3,Jyo},alt={audāsīnyaṃ}]{\skm{m }audāsīnyaṃ}% +K3,C7
		\rdg[wit={J5,V15,P11}]{audāsinyaṃ}
		\rdg[wit={G4}]{audāśinyaṃ}
		\rdg[wit={G11}]{audāsīnya}
		\rdg[wit={V19}]{<<sau>>dāmanyaṃ}}
	\app{\lem[wit={J5,V15,J10,P11,V3,Jyo}]{jalaṃ tribhiḥ}% tribhi V3
		\rdg[wit={G11}]{layaṃ tribhiḥ}
		\rdg[wit={G4,Gr2,V19,C6}]{jalaṃ smṛtam}
		}/}\\+}
\tl{
\pada{unmanīkalpalatikā}
\pada{sadya
	\app{\lem[wit={J5,V19,G11,V15,J10,C6,V3}]{evodbhaviṣyati}% yevo C6; K3,C7
		\rdg[wit={P11}]{evādbhaviṣyati}
		\rdg[wit={G4,Gr2}]{eva bhaviṣyati}% +F
		\rdg[wit={Jyo}]{eva pravartate}}//\versenr}
	%\unavbl{N3}
	\orgvnr{31}\\!}
\end{tlg}
%</vs31>
\commcite\newpage


%<*vs32>
\begin{tlg}[hp04_032]
\tl{\app{\lem[nolem]{}
	\rdg[wit={N3}]{\unavbl}
	%\rdg[wit={G11,N19,V15,J10,Jyo},alt=\textapp{found after xxx}]{\skp{found after xxx}}
	}%
\pada{\app{\lem[wit={J7,Gr3,G11,V15,V3,Jyo}]{rājayogaḥ}
		\rdg[wit={J5,N23,N19,J10,P11,C6a,C6b}]{rājayoga}
		}
	\app{\lem[wit={ceteri}]{samādhiś ca}
		\rdg[wit={P11}]{samādhīś cā}
		}}
\pada{\app{\lem[wit={ceteri}]{unmanī}
		\rdg[wit={J5,G11}]{py unmanī}
		\rdg[wit={C6b}]{hy unmanī}
		\rdg[wit={P11}]{nmatī}
		} ca manonmanī/}\\+} % va P11
\tl{
\pada{\app{\lem[wit={V15,J10}]{amaraugho}
		\rdg[wit={G11,C6b}]{amaraughā}
		\rdg[wit={C6a}]{amaraughi}
		\rdg[wit={N23}]{araughau}
		\rdg[wit={J7}]{amaraudhyai}
		\rdg[wit={V3}]{amarogho}
		\rdg[wit={J5,P11}]{amarodyo}
		\rdg[wit={Jyo}]{amaratvaṃ}
		\rdg[wit={N19}]{avaraubhū}
		\rdg[wit={Gr3}]{amaroly a°}
		}%
	\app{\lem[wit={J5,P11,C6b,V3}]{'pi cādvaitaṃ}% vādvaitaṃ P11
		\rdg[wit={J7}]{ghacāṃdrī ca}
		\rdg[wit={N23}]{ghatvīṃdrī ca}
		\rdg[wit={Gr3}]{°bhicāndrī ca}
		\rdg[wit={G11,N19,V15,J10,C6a,Jyo}]{layas tattvaṃ}% layes V15
		\rdg[wit={J10}]{layas tatra}
		}}
\pada{\app{\lem[wit={J5,Gr2,P11,C6b,V3}]{nirālambaṃ}
		\rdg[wit={Gr3}]{nirālambo}
		\rdg[wit={G11,N19,V15,J10,Jyo}]{śūnyāśūnyaṃ} % °śūnya N19
		\rdg[wit={C6a}]{śūnyācūnyaṃ}
		} 
	\app{\lem[wit={ceteri}]{nirañjanam}
		\rdg[wit={J5}]{nirāmayaṃ}
		\rdg[wit={G11,N19,V15,J10,C6a,Jyo}]{paraṃ padam}
		}//\versenr}\label{A1}
%	\sgwit{J5,G4,Gr2,Gr3,GrB} 
	\myfn{\getsiglum{C6} has this pair of verses twice: first (\getsiglum{C6a}) as \expnr{X4.3--4} of the expanded version, and second (\getsiglum{C6b}) as 4.32--33 of the older version.}
	%\unavbl{N3}
	\orgvnr{32}\\!}
\end{tlg}
%</vs32>
\commcite\newpage


%<*vs33>
\begin{tlg}[hp04_033]
\tl{\app{\lem[nolem]{}
	\rdg[wit={N3}]{\unavbl}
	%\rdg[wit={G11,N19,V15,J10,Jyo},alt=\textapp{found after xxx}]{\skp{found after xxx}}
	}% C6 twice
\pada{\app{\lem[wit={J7,V19,P11,C6b,V3}]{amanasko} % °skoṃ? V19
		\rdg[wit={J5}]{amarasko}
		\rdg[wit={N23}]{amanaskau}
		\rdg[wit={E2,G11,N19,V15,J10,C6a,Jyo}]{amanaskaṃ} % +K3,C7
		}
	\app{\lem[wit={P11,C6b,V3}]{layas tattvaṃ}
		\rdg[wit={J5}]{layas tatra}
		\rdg[wit={J7,Gr3}]{layaś caiva}
		\rdg[wit={N23}]{lyayāś caiva}
		\rdg[wit={G11,N19,V15,J10,C6a,Jyo}]{tathādvaitaṃ}
		}}
\pada{\app{\lem[wit={J5,J7,Gr3,P11}]{śūnyāśūnyaṃ}% °sūnyaṃ J5
		\rdg[wit={N23,V3}]{śūnyāśūnya}
		\rdg[wit={C6}]{śūnyāc chūnyaṃ}
		\rdg[wit={G11,N19,V15,J10,C6a,Jyo}]{nirālambaṃ}
		}
	\app{\lem[wit={J5,G4,P11,C6b,V3}]{paraṃ padam}
		\rdg[wit={N23,Gr3}]{parāparaṃ}
		\rdg[wit={J7}]{parāvaraṃ}
		\rdg[wit={G11,N19,V15,J10,C6a,Jyo}]{nirañjanam}
		}/}\\+}
\tl{
\pada{\app{\lem[wit={ceteri}]{jīvanmuktiś ca}
		\rdg[wit={G4}]{jīvanmuktaś ca}
		\rdg[wit={N23}]{jīvanmuktiḥ}} 
	\app{\lem[wit={ceteri}]{sahajaṃ} % ṃ om. N23
%		\rdg[wit={Jyo}]{sahajā}
		\rdg[wit={V15},alt={\om}]{\skp{\om}}
		}}
\pada{\app{\lem[wit={J5,G4,Gr2,E2,V15,P11,C6a}]{turyaṃ} % turya J5
		\rdg[wit={V19}]{turjaṃ}
		\rdg[wit={C6}]{turyāṃ}
		\rdg[wit={N19}]{turyai}
		\rdg[wit={Jyo}]{turyā}
		\rdg[wit={G11}]{turīyaṃ}
		\rdg[wit={V3}]{tuṣkaṃ}
		\rdg[wit={J10}]{muktiś}
		}
	\app{\lem[wit={J5,J7,Gr3,J10,P11,C6,Jyo}]{cety eka}
		\rdg[wit={G4}]{..\,ty eka}
%		\rdg[wit={C7}]{cety eva}
		\rdg[wit={N23}]{vatyaka}
		\rdg[wit={C6a}]{caityeka}
		\rdg[wit={V3}]{caiyeka}
		\rdg[wit={G11}]{caika}
		\rdg[wit={V15}]{cittaika}% +N22,P6
		\rdg[wit={N19}]{ciṃtaika}
}%
	\app{\lem[wit={N23,J10pc,Jyo}]{vācakāḥ}
		\rdg[wit={J5,J10ac}]{vācakaḥ}
		\rdg[wit={G4,Gr3,G11,N19,V15,P11,C6a,C6b,V3}]{vācakaṃ}% +N22,P6
		\rdg[wit={J7}]{vācakīṃ}
		}//\versenr}\label{A2}
%	\sgwit{J5,G4,Gr2,Gr3,GrB} 
	%\unavbl{N3}
	\orgvnr{33}\\!}
\end{tlg}
%</vs33>
\commcite\newpage


%<*vs34>
\begin{tlg}[hp04_034]
\tl{\app{\lem[nolem]{}
	\rdg[wit={E2,N19,V15,J10},alt={\om}]{\skp{\om}}
	\rdg[wit={N3}]{\unavbl}}%
\pada{\app{\lem[nolem]{\skp{pāda a}}
	\rdg[wit={Gr2},alt={\om}]{\skp{\om}}}%
\app{\lem[wit={J5,G11,P11,V3,Jyo}]{unmanyavāptaye}
		\rdg[wit={C6}]{unmanyā\,\_\,\_\,ye}
		\rdg[wit={V19}]{unmanyavāsayet}
		%\rdg[wit={K3,C7}]{unmanyā vāsayec}
		\rdg[wit={G4}]{unmanyaye}} śīghraṃ} % chīghraṃ V19,K3,C7
\pada{\app{\lem[nolem]{\skp{pāda b}}
	\rdg[wit={Gr2},alt={\om}]{\skp{\om}}}%
\app{\lem[wit={J5,G11,P11,C6}]{mārgau dvau}
		\rdg[wit={V3}]{mārgo dvau}
		\rdg[wit={G4}]{mārgā\,..}
		\rdg[wit={V19}]{dvau mārgau}% +K3,C7
		\rdg[wit={Jyo}]{bhrūdhyānaṃ}}
	\app{\lem[wit={J5,G11,V3}]{mama saṃmatau}
		\rdg[wit={G4}]{myama saṃ[m].\,+}
		\rdg[wit={C6}]{mamatau}
		\rdg[wit={V19,P11}]{samasaṃmatau}% +K3,C7
		\rdg[wit={Jyo}]{mama saṃmatam}}/}
	\\+}
	%\sgwit{J5,G4,V19,K3,C7,G11,GrB,Jyo}\\+}
\tl{
\pada{\app{\lem[nolem]{\skp{pāda c}}
	\rdg[wit={V19,Jyo},alt={\om}]{\skp{\om}}}%
tattvaṃ parama% padaṃ J5
	\app{\lem[wit={Gr2,G5,C6}]{saukhyaṃ} % +G5; 2 x ṃ om. N23
		\rdg[wit={J5}]{sākhyaṃ}
		\rdg[wit={G11,V3}]{sāṃkhyaṃ}% saṃkhyaṃ F
		\rdg[wit={P11}]{vāgraṃ}
		} vā} % ca J5
\pada{\app{\lem[nolem]{\skp{pāda d}}
	\rdg[wit={V19,Jyo},alt={\om}]{\skp{\om}}}%
nādopāsanam eva  % nadipā°? J7ac
	\app{\lem[wit={J5,Gr2,V3}]{ca}
		\rdg[wit={G11,G5,P11,C6}]{vā}% +F
		}//\versenr} % cā N24
	%\unavbl{N3}
	\orgvnr{34}\\!}
%	\sgwit{J5,G4,G11,Gr2,GrB}\\!}% G4 only supposedly
\end{tlg}
%</vs34>
\commcite%\newpage

%<*vs35>
\begin{tlg}[hp04_035]
\tl{\app{\lem[nolem]{}
	\rdg[wit={E2,N19,V15,J10,Jyo},alt={\om}]{\skp{\om}}
	\rdg[wit={N3}]{\unavbl}}%
\pada{\app{\lem[nolem]{\skp{pāda a}}
	\rdg[wit={V19},alt={\om}]{\skp{\om}}}%
\app{\lem[wit={N23,G5,C6}]{saukhya}
		\rdg[wit={J7}]{saukhyā}
		\rdg[wit={P11,V3}]{sāṃkhya}
		\rdg[wit={G11}]{sāṃkhyaṃ}
		\rdg[wit={J5}]{sākṣaṃ}}%
	\app{\lem[wit={J7,G11,C6,V3}]{praviṣṭa}
		\rdg[wit={J5}]{pravṛṣṭa}
		\rdg[wit={N23,P11}]{pratiṣṭha}}cittānāṃ} % cittānyu N23
\pada{\app{\lem[nolem]{\skp{pāda b}}
	\rdg[wit={V19},alt={\om}]{\skp{\om}}}%
mūḍhānām api saṃmatam/} % °taḥ P11, dṛḍhānām! G11
	\\+}
%	\sgwit{J5,G4,Gr2,G11,GrB}\\+} % G4 only supposedly
\tl{
\pada{\app{\lem[wit={J5,V19,C7,G11,GrB}]{sadya}% +K3
		\rdg[wit={Gr2}]{satyam}}%
	\app{\lem[wit={J5,Gr2,C7,G11,GrB}]{ānanda}% +K3
		\rdg[wit={V19}]{ādāya}}%
	\app{\lem[wit={G4,J7,V19,C7}]{saṃdhāyī}
		\rdg[wit={N23}]{saṃdhyāyī}
		\rdg[wit={G11,P11}]{saṃdāyī}% +K3 % saṃ√dā ist nicht gebräuchlich.
		\rdg[wit={V3}]{sadāyī}
		\rdg[wit={C6}]{saṃdāyi}
		\rdg[wit={J5}]{saṃdāï}}}
\pada{\app{\lem[wit={ceteri}]{jāyate}
		\rdg[wit={V19}]{jāvate}}
	\app{\lem[wit={G4,Gr2,V19,C7,G11,C6,V3}]{nādajo}% °jau N23; +K3
		\rdg[wit={P11}]{nādato}
		\rdg[wit={J5}]{natato}} layaḥ//\versenr} % laya V3
	\label{saukhya}
	%\unavbl{N3}
	\orgvnr{35}\\!}
%	\sgwit{J5,G4,Gr2,V19,K3,C7,G11,GrB}
\end{tlg}
%</vs35>
\commcite\newpage


\iffalse
\startaltrecension
%<*vs35-1>
\begin{alttlg}[hp04_035_1]
\tl{\app{\lem[nolem]{}
	\rdg[wit={G11,G5,V15,J10}]{\only}}%
\pada{ekaṃ sṛṣṭimayaṃ bījaṃ} % svasti° G11
\pada{ekā mudrā 
\app{\lem[wit={G11,G5,J10}]{ca}
	\rdg[wit={V15}]{tu}% +J11
	} khecarī/}\\+}
\tl{
\pada{eko devo \app{\lem[wit={V15,J10}]{nirālamba}% +J11
	\rdg[wit={G11,G5}]{nirālambo hy}}}
\pada{ekāvasthā manonmanī//\versenr} % °mani V15
\anm{=\,\manuref{3.48}}\\!} % Not in V3,G7, but in G11
\end{alttlg}
%</vs35-1>

%<*vs35-2>
\begin{alttlg}[hp04_035_2]
\tl{\app{\lem[nolem]{}
	\rdg[wit={N19}]{\NotIn}}%
\pada{śaṅkhadundubhi%
	\app{\lem[wit={G5,V15,J10,P11,Jyo}]{nādaṃ ca}
		\rdg[wit={G11,V3}]{nādaś ca}
		\rdg[wit={C6}]{nādāṃś ca}}}
\pada{na śṛṇoti kadācana/}\\+} % sṛṇoti V3,J10; canaḥ J11, canā G11
\tl{
\pada{\app{\lem[wit={G5,V15,J10,Jyo}]{kāṣṭhavaj jāyate}
		\rdg[wit={G11}]{kāṣṭhavaj jñāyate}
		\rdg[wit={C6}]{sthāṇuvad vartate}% =M1,F,G3
		\rdg[wit={P11}]{sthāṇuvarddhattayed}% °yej jogī P11
		\rdg[wit={V3}]{sthāṇu vardhate}}
	\app{\lem[wit={J10,Jyo}]{deha}
		\rdg[wit={V15}]{dehe}
		\rdg[wit={G11,G5}]{nādam}
		\rdg[wit={GrB}]{yogī hy}}} % =M1,G3
\pada{unmanyā\app{\lem[wit={G11,G5,V15,GrB,Jyo},alt={°vasthayā}]{\skp{°}vasthayā} % unmanyava° would be unmetrical. V15 has avagraha: unmanyā'va°
		\rdg[wit={J10}]{vasthāyāṃ}} dhruvam//\versenr} 
		%\sgwit{G11,G5,V15,J10,GrB,Jyo}
		\\!}
\end{alttlg}
%</vs35-2>

%<*vs35-3>
\begin{alttlg}[hp04_035_3]
\tl{\app{\lem[nolem]{}
	\rdg[wit={N19}]{\NotIn}}%
\pada{sarvāvasthāvinirmuktaḥ}
\pada{sarvacintā%
	\app{\lem[wit={G11,G5,V15,J10,P11,C6,Jyo}]{vivarjitaḥ} % jita P11
		\rdg[wit={V3}]{vivarjitaṃ}}/}\\+}
\tl{
\pada{\app{\lem[wit={G11,G5,V15,J10,Jyo},alt={mṛtavat}]{mṛtava\skp{t}}
		\rdg[wit={GrB}]{kāṣṭhavat}}% =M1,F
	\app{\lem[wit={G5,V15,J10,P11,C6,Jyo},alt={tiṣṭhate}]{\skm{t }tiṣṭhate}
		\rdg[wit={V3}]{tiṣṭhayed}
		\rdg[wit={G11}]{vartate}} yogī}
\pada{sa mukto nātra saṃśayaḥ//\versenr}\myfn{After this verse, \getsiglum{Jyo} has a verse which has no correspondence in the other manuscripts of the \HP:
\devnote{khādyate na ca kālena bādhyate na ca karmaṇā/
sādhyate na sa kenāpi yogī yuktaḥ samādhinā//} (4.108)}
%\sgwit{G11,G5,V15,J10,GrB,Jyo}
\\!}
\end{alttlg}
%</vs35-3>
%\Anm{\getsiglum{Jyo} has Vulg 4.108 \textit{khādyate na ca kālena}... here}


%<*vs35-4>
\begin{alttlg}[hp04_035_4]
\tl{\app{\lem[nolem]{}
	\rdg[wit={N19,J10}]{\NotIn}}%
\pada{na \app{\lem[wit={G11,G5,P11}]{hi jānāti}% +F
		\rdg[wit={V15,Jyo}]{vijānāti}
		\rdg[wit={V3}]{hi jānaṃti}} śītoṣṇaṃ}
\pada{\app{\lem[wit={G11,V15,P11,Jyo}]{na duḥkhaṃ na sukhaṃ}
		\rdg[wit={G5}]{na duḥkhaṃ sukham eva vā}
		\rdg[wit={V3}]{na ca duḥkhaṃ sukhaṃ}} tathā/}\\+}
\tl{
\pada{\app{\lem[wit={V15,Jyo}]{na mānaṃ nāpamānaṃ}
		\rdg[wit={G11,G5}]{na mānaṃ nāvamānaṃ}% +F
		\rdg[wit={P11}]{na mānaṃ cāpamānaṃ}
		\rdg[wit={V3}]{na ca mānāpamānaṃ}} ca}
\pada{yogī \app{\lem[wit={G11,P11,C6,Jyo}]{yuktaḥ}
		\rdg[wit={G5,V15}]{muktaḥ}
		\rdg[wit={V3}]{yukti}} samādhinā//\versenr}
		%\sgwit{G11,G5,V15,GrB,Jyo}
		\\!}
\end{alttlg}
%</vs35-4>

%<*vs35-5>
\begin{alttlg}[hp04_035_5]
\tl{\app{\lem[nolem]{}
	\rdg[wit={G11,G5,V3,Jyo}]{\only}}%
\pada{na gandhaṃ na rasaṃ rūpaṃ} % nāgandhaṃ G11, raso G5
\pada{\app{\lem[resp=emend]{na sparśanaṃ na ca śrutam}
	\rdg[wit={G11,G5,V3}]{na\,(\om\ \getsiglum{V3}) sparśaṃ na ca na śrutaṃ}% +G3; niśrutaṃ F
	\rdg[wit={Jyo}]{na ca sparśaṃ na niḥsvanam}% = VM
	}/}\\+}
\tl{
\pada{nātmānaṃ
	\app{\lem[wit={G11,G5,Jyo}]{na paraṃ}
	\rdg[wit={V3}]{paramaṃ}}
	vetti}
\pada{yogī
	\app{\lem[wit={G11,Jyo}]{yuktaḥ}
	\rdg[wit={G5}]{muktaḥ}
	\rdg[wit={V3}]{yukti}
	} samādhinā//\versenr} 
	%\sgwit{G11,G5,V3,Jyo} %\NotIn{N19,V15,J10}
	\\!}
\end{alttlg}
%</vs35-5>
%<*vs35-6>
\begin{alttlg}[hp04_035_6]
\tl{\app{\lem[nolem]{}
	\rdg[wit={G11,G5,V15,J10,Jyo}]{\only}}%
\pada{\app{\lem[resp=emend]{avedhyaḥ}
	\rdg[wit={V15,J10,Jyo}]{avadhyaḥ}% abodhya P6
	\rdg[wit={G11}]{adhyāpyāḥ}% avāpyaḥ M3, avyāpya G7; G3 does not have this verse?
	\rdg[wit={G5}]{adhyāpaḥ}} 
	sarva\app{\lem[wit={V15,J10,Jyo},alt={śastrāṇām}]{śastrāṇā\skp{m}}
	\rdg[wit={G11,G5}]{śāstrāṇām}}}%m
\pada{\app{\lem[wit={G11,G5,V15,J10},alt={avadhyaḥ}]{\skm{m }avadhyaḥ}% ā(!)bādhyaḥ P6
	\rdg[wit={Jyo}]{aśakyaḥ}} 
	sarva\app{\lem[wit={ceteri}]{dehinām}
	\rdg[wit={G11},alt={\om}]{\skp{\om}}}/}\\+}
\tl{
\pada{\app{\lem[wit={G5,V15,J10,Jyo}]{agrāhyo}
	\rdg[wit={G11},alt={\om}]{\skp{\om}}} 
\app{\lem[wit={V15,J10}]{mantratantrāṇāṃ} % tatrāṇāṃ V15
	\rdg[wit={G5,Jyo}]{mantrayantrāṇāṃ}% +M3; yantramantrāṇāṃ G7
	\rdg[wit={G11},alt={\om}]{\skp{\om}}}}
\pada{yogī \app{\lem[wit={J10,Jyo}]{yuktaḥ}% M3
	\rdg[wit={G5,V15}]{muktaḥ}% G5
	\rdg[wit={G11},alt={\om}]{\skp{\om}}} 
	\app{\lem[wit={G5,V15,J10,Jyo}]{samādhinā}
	\rdg[wit={G11}]{mādhinā}}//\versenr}
	%\sgwit{G11,V15,J10,Jyo}
	\\!} % NotIn{V3}
\end{alttlg}
%</vs35-6>
\Anm{\getsiglum{G11,V15,J10} have \ref{pravese} \textit{praveśe nirgame vāme} here}


%<*vs35-7>
\begin{alttlg}[hp04_035_7]
\tl{\app{\lem[nolem]{}
	\rdg[wit={N19}]{\NotIn}}%
\pada{\app{\lem[nolem]{\skp{pāda a}}
	\rdg[wit={G5},alt={\om}]{\skp{\om}}}%
cittaṃ na suptaṃ no jāgrat} % nna P11; jāgrati J10ac
\pada{\app{\lem[nolem]{\skp{pāda b}}
	\rdg[wit={G5},alt={\om}]{\skp{\om}}}%
\app{\lem[wit={G11}]{smṛtiman na}% +F; smṛtaṃ na ca nānyathā M3
		\rdg[wit={C6}]{smṛtyaman}
		\rdg[wit={V3}]{sṛtinannaṃ}
		\rdg[wit={V15}]{smṛtivarṇaṃ}
		\rdg[wit={P11}]{na smṛtir na}
		\rdg[wit={Jyo}]{smṛtivismṛ°}
		\rdg[wit={J10}]{spṛśati vastu}}
	\app{\lem[resp=emend]{na cānyathā}
		\rdg[wit={G11,V15,J10,P11,V3}]{ca nānyathā}
		\rdg[wit={C6}]{na nānyathā}
		\rdg[wit={Jyo}]{°tivarjitam}}/}
		\\+}
\tl{
\pada{\app{\lem[wit={G11,G5,V15,GrB}]{nāstam eti}
		\rdg[wit={J10}]{na vāstum eti}
		\rdg[wit={Jyo}]{na cāstam eti}}
	\app{\lem[wit={G11,G5,V15,J10,P11,C6}]{na codeti}
		\rdg[wit={V3}]{na cādeti}
		\rdg[wit={Jyo}]{nodeti}}}
\pada{\app{\lem[wit={G11,G5,V15,P11,C6,Jyo}]{yasyāsau}
		\rdg[wit={J10}]{yathāsau} % mukti J10ac
		\rdg[wit={V3},alt={\illeg}]{\skp{\illeg}}}
	\app{\lem[wit={G11,G5,V15,J10,P11,C6,Jyo}]{mukta eva saḥ} % mukti J10ac
		\rdg[wit={V3},alt={\illeg}]{\skp{\illeg}}}//\versenr} 
		%\sgwit{G11,G5,V15,J10,GrB,Jyo}
		\label{nasuptam}\\!} % yavaprasa in J8
\end{alttlg}
%</vs35-7>

%<*vs35-8>
\begin{alttlg}[hp04_035_8]
\tl{\app{\lem[nolem]{}
	\rdg[wit={N19,J10}]{\NotIn}}%
\pada{\app{\lem[wit={G11,V3,Jyo}]{svastho}% +M3?
		\rdg[wit={P11}]{svapno}
		\rdg[wit={G5}]{svapne}
		\rdg[wit={C6}]{supto}
		\rdg[wit={V15}]{svecchā}
		} jāgradavasthāyāṃ} % °sthīyāṃ P11; ṃ om. V3
\pada{\app{\lem[wit={G11,P11,V3,Jyo}]{suptavad yo}
		\rdg[wit={C6}]{suptavadhyo}
		\rdg[wit={V15}]{suptaḥ sadyo}
		\rdg[wit={G5}]{pūrvavad yo}}%
	\app{\lem[wit={G11,G5,V15,V3,Jyo}]{'vatiṣṭhate}
		\rdg[wit={P11,C6}]{vatiṣṭhati}}/}\\+} % °tiṣṭhati V15pc?
\tl{
\pada{\app{\lem[wit={V15,Jyo}]{niḥśvāsocchvāsa}
		\rdg[wit={G11,G5}]{niśvāsocchvāsa}% +F
		\rdg[wit={V3}]{niśvāsośvāsa}
		\rdg[wit={P11}]{nisvāsośvaḥsa}
		\rdg[wit={C6}]{niḥśvāsaśvāsa}}%
	\app{\lem[wit={V15,V3,Jyo}]{hīnaś ca}% source & testimonia
		\rdg[wit={G11,P11,C6}]{hīnas tu}% +F
		\rdg[wit={G5}]{hīnasya}
		}} 
\pada{\app{\lem[wit={G11,G5,V15,Jyo}]{niścitaṃ}
		\rdg[wit={V3}]{niścito}
		\rdg[wit={P11}]{niścitto}
		\rdg[wit={C6}]{niśceṣṭo}} mukta eva saḥ//\versenr} % sa V3
		%\sgwit{G11,G5,V15,GrB,Jyo}
		\label{svastho}\\!}
\end{alttlg}
%</vs35-8>
\endaltrecension
\fi


%\Anm{The following verses appear immediately after \ref{ekibhutam} in \getsiglum{N19,V15,J10} and after 4.42 in \getsiglum{GrB}}


%==================================
%<*vs36>
\begin{tlg}[hp04_036]
\tl{
\pada{\app{\lem[nolem]{\skp{pāda a}}
	\rdg[wit={N3},alt={\unavbl}]{\skp{\unavbl}}}%
nādānusandhānasamādhibhājāṃ}\\+} % dānā° V15, saṃdhāta P11, samādhinā J5
\tl{
\pada{\app{\lem[nolem]{\skp{pāda b}}
	\rdg[wit={N3},alt={\unavbl}]{\skp{\unavbl}}}%
yogīśvarāṇāṃ % <<yo>>gośvarāṇāṃ J7, °svarā° N23
	\app{\lem[wit={J5,J7,Gr3,G11,V15,C6,V3}]{hṛdaye prarūḍham}% prarūḍha V3
		\rdg[wit={N23,P11}]{hṛdayaprarūḍhaṃ}% [rū]ḍhaṃ N23, rūṇāṃ P11; +F
		\rdg[wit={N19,J10,Jyo}]{hṛdi vardhamānaṃ}}/}\\+}  % vaddha° N19, vaddhra°? N26
\tl{
\pada{\app{\lem[nolem]{\skp{pāda c}}
	\rdg[wit={N3},alt={\unavbl}]{\skp{\unavbl}}
	\rdg[wit={J5},alt={\om}]{\skp{\om}}}%
ānandam ekaṃ vacasā%m
	\app{\lem[wit={ceteri},alt={avācyaṃ}]{\skm{m }avācyaṃ} % yacasām V3
%		\rdg[wit={G11}]{avācaṃ}
		\rdg[wit={N19}]{avākyaṃ}
		\rdg[wit={C6,Jyo}]{agamyaṃ}% +G3
		}}\\+}
\tl{
\pada{\app{\lem[nolem]{\skp{pāda d}}
	\rdg[wit={J5},alt={\om}]{\skp{\om}}}%
\app{\lem[wit={ceteri}]{jānāti}
		\rdg[wit={P11,C6}]{jānāty a°}
		\rdg[wit={N19}]{jānaṃti}
		\rdg[wit={N3},alt= {\lost}]{\skp{\lost}}}
	\app{\lem[wit={J7,G11,N19,V15,J10,V3,Jyo}]{taṃ śrī}
		\rdg[wit={C6}]{°taḥ śrī}
		\rdg[wit={P11}]{°tītaṃ}
		\rdg[wit={N23}]{tatvaṃ śrī}
		\rdg[wit={Gr3}]{tattvaṃ}
		\rdg[wit={N3},alt= {\lost}]{\skp{\lost}}}%
	\app{\lem[wit={ceteri}]{gurunātha}
		\rdg[wit={Gr3}]{guṇanātha}
		\rdg[wit={N3}]{+\,+\,nātha}}
	\app{\lem[wit={N3,J7,Gr3,G11,V15,GrB}]{eva}
		\rdg[wit={N23}]{evaṃ}
		\rdg[wit={N19,Jyo}]{ekaḥ}% +F
		\rdg[wit={J10}]{ekaṃ}}//\versenr}
	\label{nadanu}
	%\unavbl{N3}
	\orgvnr{36}\\!} % N3 resumes with nātha eva
\end{tlg}
%</vs36>
\commcite%\newpage


%<*vs37>
\begin{tlg}[hp04_037]
\tl{%\app{\lem[nolem]{}
	%\rdg[wit={G11,N19,V15,J10},alt=\textapp{found after \ref{sarvacinta}}]{\skp{found after \manuref{4.18}}}%
	%\rdg[wit={Jyo},alt=\textapp{found \expnr{???}}]{\skp{found ???}}
	%\rdg[wit={GrB},alt=\textapp{transposed with the next one}]{\skp{transposed with the next one}}}%
\pada{sarvacintāṃ parityajya} % ciya J5, ciṃtā C6,V3, ciṃtāḥ J10
\pada{\app{\lem[wit={ceteri}]{sāvadhānena}
	\rdg[wit={N19,J10}]{sarvadānena}}
	cetasā/}\\+}
\tl{
\pada{\app{\lem[wit={ceteri}]{nāda evānusandheyo}% nāda yecānusaṃdhyeyo J5 % Gr1,Gr2,E2,G11,V15,GrB,Jyo
	\rdg[wit={N19}]{nādam evānusaṃdhe}% (yo omitted)
	\rdg[wit={V19,J10}]{nādam evānusaṃdhatte}% +K3
	}}
\pada{yoga\app{\lem[wit={ceteri},alt={sāmrājyam}]{sāmrājya\skp{m}}
	\rdg[wit={V19}]{samrājyam}
	\rdg[wit={C6}]{samrājam}
	}%
	\app{\lem[wit={N3,J7,Gr3,G11,V15,GrB,Jyo},alt={icchatā}]{\skm{m }icchatā}
		\rdg[wit={G4,N19}]{icchatāṃ}
		\rdg[wit={N23,J10}]{icchati}
		\rdg[wit={J5}]{iṣṭatā}
		}//\versenr}%
		\label{sarvacinta2}
		\orgvnr{37}\\!}
\end{tlg}
%</vs37>
\commcite\newpage


%<*vs38>
\begin{tlg}[hp04_038]
\tl{\app{\lem[nolem]{}
	\rdg[wit={J10},alt={\om}]{\skp{\om}}}%
\pada{\app{\lem[wit={ceteri}]{karṇau}
		\rdg[wit={N3,N23}]{karṇo}
		\rdg[wit={G4}]{karṇā}
		\rdg[wit={P11}]{karṇa}}
	\app{\lem[wit={ceteri}]{pidhāya}% N3,Gr2,E2,G11,N19,V15,GrB,Jyo
		\rdg[wit={G4}]{pidhāna}
		\rdg[wit={V19}]{pi}
		\rdg[wit={J5}]{nidhāya}}
	\app{\lem[wit={G4,G5,N19}]{tūlena}% 
		\rdg[wit={P11}]{tulyena}
		\rdg[wit={N3,J5,G11,V3}]{mūlena}% +F,G3
		\rdg[wit={Gr2}]{hastena}
		\rdg[wit={E2,C6,Jyo}]{hastābhyāṃ}
		\rdg[wit={V19}]{hastābhya[ṃ]}
		\rdg[wit={V15}]{śū\,\_\,na}}}
\pada{\app{\lem[wit={N3,J5,G11,N19,V15,Jyo}]{yaṃ}
		\rdg[wit={G4,Gr2,Gr3,C6}]{yaḥ}% +F
		\rdg[wit={P11}]{saṃ}
		\rdg[wit={V3}]{sa}} śṛṇoti % °tiṃ P11, śruṇoti V3
	\app{\lem[wit={N3,J5,Gr3,G11,N19,V15,GrB,Jyo}]{dhvaniṃ muniḥ} % dhvani V3, ddhvaniṃ N19; muni P11,V3
		\rdg[wit={N23}]{dhvaniṃ muniṃ}
		\rdg[wit={J7}]{munir dhvanim}
%		\rdg[wit={C7}]{dhvaniṃ dhvaniḥ}
		}/}\\+}
\tl{
\pada{\app{\lem[wit={ceteri}]{tatra cittaṃ} % citta N23
		\rdg[wit={J5,P11}]{tatra ciṃtāṃ}} % ṃ oṃ J5
	\app{\lem[wit={N3,J5,G5,GrB,Jyo}]{sthirī}
		\rdg[wit={Gr2,Gr3,N19,V15}]{sthiraṃ}
		\rdg[wit={G11}]{sthitaṃ}}kuryād}
\pada{yāva%t  % yāva Gr2,J5,V3
	\app{\lem[wit={ceteri},alt={sthirapadaṃ}]{\skm{t }sthirapadaṃ}% sira° P11
		\rdg[wit={V3}]{sthiparamaṃ}}
	\app{\lem[wit={ceteri}]{vrajet}% vraje N23
		\rdg[wit={N19,V15}]{bhavet}}//\versenr}
		\orgvnr{38}\\!}
\end{tlg}
%</vs38>
\commcite\newpage

%\Anm{\getsiglum{V3} has \ref{makaranda1}--\ref{makaranda6} here}


%========= common verses ==================
%<*vs39>
\begin{tlg}[hp04_039]
\tl{
\pada{abhyasyamāno %  abhyāsya° N23, °mano N3
	\app{\lem[wit={ceteri}]{nādo}% nādau J5
		\rdg[wit={N23}]{nātho}}%
	\app{\lem[wit={ceteri}]{'yaṃ}% +G5, ya G11
		\rdg[wit={C6}]{yo}}}
\pada{\app{\lem[wit={J7,G11,C6,Jyo}]{bāhyam āvṛṇute}
		\rdg[wit={P11}]{bāhyanāvṛṇute}
		\rdg[wit={N23}]{bāhyanā\,\_\,ṇute}
		\rdg[wit={N3}]{bāhyam āśṛṇu}
		\rdg[wit={V3}]{bāhyam āsṛṇate}
		\rdg[wit={J5}]{bāhyaṃ ca śṛṇute}
		\rdg[wit={N19}]{bāhyamānaśṛṇvate}
		\rdg[wit={J10}]{cānyam āśṛṇute}
		\rdg[wit={V19,V15}]{bāhyam āvartaye}% °yed +K3,C7
		\rdg[wit={E2}]{bāhyād āvartayed}\marma} 
	\app{\lem[wit={N3,J7,Gr3,V15,J10,Jyo}]{dhvanim}
		\rdg[wit={N23}]{dhvani}
		\rdg[wit={G11,G5,N19,GrB}]{dhvaniḥ}% ddhaniḥ N19; +F
		\rdg[wit={J5}]{dhvaniṃḥ}}/}\\+}
\tl{
\pada{\app{\lem[wit={ceteri},alt={pakṣād}]{pakṣā\skp{d}}% N3,J5,Gr2,E2,G5,N19,V15,GrB,Jyo
		\rdg[wit={G4,V19,G11,J10}]{paścād}}%
	\app{\lem[wit={N3,J5,J7,E2,J10,V3,Jyo},alt={vikṣepam akhilaṃ}]{\skm{d }vikṣepam akhilaṃ}% +C7
		\rdg[wit={N23}]{vikṣeyam akhilaṃ}
		\rdg[wit={V19}]{vikṣepam atulaṃ}
		\rdg[wit={G4}]{vikṣiptam a[nila]ṃ}
		\rdg[wit={G11,G5}]{vikṣiptam akhilaṃ}
		\rdg[wit={P11}]{vikṣyemanilaṃ}
		\rdg[wit={N19,V15}]{vipakṣam akhilaṃ}
		\rdg[wit={C6}]{vipakṣayed enaṃ}}}
\pada{\app{\lem[wit={ceteri}]{jitvā} % jītvā P7
		\rdg[wit={J10}]{jīvo}} yogī sukhī bhavet//\versenr}
		\orgvnr{39}\\!}
\end{tlg}
%</vs39>
\commcite%\newpage


%<*vs40>
\begin{tlg}[hp04_040]
\tl{
\pada{\app{\lem[wit={ceteri}]{śrūyate}% śūyate N3
		\rdg[wit={E2}]{jāyate}% +C7
		}
	\app{\lem[wit={ceteri}]{prathamābhyāse}
		\rdg[wit={V19}]{prathame bhyāse}
		\rdg[wit={N3}]{prathamābhyāso}}}
\pada{nādo nānāvidho mahān/}\\+} % nādau J5, nādā N23; vidhā N19; mahāt N19 
\tl{
\pada{\app{\lem[wit={ceteri}]{vardhamāne tato'bhyāse}
		\rdg[wit={V15,Jyo}]{tato'bhyāse vardhamāne}}}
\pada{śrūyate % srū˟yate N3, śrūyete P11
	\app{\lem[wit={N3,J5,Gr3,G11,J10,C6,V3}]{sūkṣmasūkṣmataḥ}
		\rdg[wit={J7,V15,Jyo}]{sūkṣmasūkṣmakaḥ} % 
		\rdg[wit={N23}]{sūjyasūjyakaḥ}
		\rdg[wit={N19,P11}]{sūkṣmataḥ}}//\versenr}
		\orgvnr{40}\\!} % śū° P11; ḥ om. N19; haplogr.
\end{tlg}
%</vs40>
\commcite\newpage


%<*vs41>
\begin{tlg}[hp04_041]
\tl{
\pada{ādau jaladhi% adau P7
	\app{\lem[wit={ceteri}]{jīmūta}% jāmūta J5
		\rdg[wit={N23,P11,V3}]{jīmūte}}}% °te J10pc
\pada{bherī% bhīrī K3
	\app{\lem[wit={G11,N19,V15,J10,P11}]{nirjhara}
		\rdg[wit={V19}]{nirjara}
		\rdg[wit={V3}]{nirbhara} % +P7
		\rdg[wit={C6}]{nigama}
		\rdg[wit={J5}]{nisara}
		\rdg[wit={N3}]{rsara} % unm.
		\rdg[wit={N23}]{sarāva}
		\rdg[wit={J7}]{śabdatu}
		\rdg[wit={E2}]{bhūrbhūra}% +C7
		\rdg[wit={Jyo}]{jharjhara}}%
	\app{\lem[wit={N19,C6,Jyo}]{saṃbhavāḥ}
		\rdg[wit={N3,J5,P11}]{saṃbhavā}
		\rdg[wit={Gr2,Gr3,G11,V15}]{saṃbhavaḥ}
		\rdg[wit={J10,V3}]{nisvanaḥ}}/}\\+} % niśvanaḥ
\tl{
\pada{madhye
	\app{\lem[wit={ceteri}]{mardala}% mardvala N23
		\rdg[wit={G11}]{maddala}
		\rdg[wit={E2}]{mandala}% +K3,C7
		}%
	\app{\lem[wit={N3,J5,G11,N19,V15,Jyo}]{śaṃkhotthā}% śaṃdho J5; °otchā N3
		\rdg[wit={C6pc,Gr3,G5,J10,P11,V3}]{śaṃkhottha}% saṃkho V3, śaṃ<<kho>>tha C6; °occha P11
		\rdg[wit={Gr2}]{śaṃkhotha}
		\rdg[wit={C6ac}]{śaṅkhottho}% +K3
		\rdg[wit={G4}]{śaṃkhoddhāḥ}}\marma}
\pada{ghaṇṭā\app{\lem[wit={J5,G4,J7,G11,N19,V15,C6,V3,Jyo}]{kāhala}
		\rdg[wit={N3,P11}]{kāhāla}
		\rdg[wit={N23}]{kāhla}
		\rdg[wit={Gr3}]{kalaha}
		\rdg[wit={J10}]{kolāha}}%
	\app{\lem[wit={N3,J5,GrB,Jyo},alt={°jās}]{\skp{°}jā\skp{s}}
		\rdg[wit={Gr2,Gr3,G11}]{jas}
		\rdg[wit={G4,N19,V15}]{kās}
		\rdg[wit={J10}]{las}}%
	\app{\lem[wit={ceteri},alt={tathā}]{\skm{s }tathā} % tatha J10
		\rdg[wit={C6}]{tataḥ}}//\versenr}
		\orgvnr{41}\\!}
\end{tlg}
%</vs41>
\commcite%\newpage


%<*vs42>
\begin{tlg}[hp04_042]
\tl{
\pada{\app{\lem[wit={ceteri}]{ante}% Gr1,J7,Gr3,G11,V15,GrB,Jyo
		\rdg[wit={N19,J10}]{anye}
		\rdg[wit={N23}]{avai}
		} tu kiṅkiṇī% kikiṇī N3, °nī V19, kiṃkaṇī J5
	\app{\lem[wit={N3,G11,N19,V15,J10,Jyo}]{vaṃśa}% vaṃśaṃ
		\rdg[wit={Gr2,Gr3,C6,V3}]{vṛnda}% vṛṃda -> <<śa>>bdaṃva?! J7
		\rdg[wit={G4}]{bṛṃdā}
		\rdg[wit={P11}]{vaṃda}
		\rdg[wit={J5}]{śabda}}}%
\pada{\app{\lem[wit={ceteri}]{vīṇā}% Gr1,Gr2,Gr3,G11,J10,GrB,Jyo
		\rdg[wit={N19,V15}]{nādā}
		}bhramara% bhrumara N23, bhrasara N19
	\app{\lem[wit={N3,G4,G11,N19,C6}]{nisvanāḥ}
		\rdg[wit={J10,V3}]{nisvanā} % nisvānā J10
		\rdg[wit={J7,V19}]{nisvanaḥ} % niśvanaḥ V19
		\rdg[wit={V15,Jyo}]{niḥsvanāḥ}
		\rdg[wit={J5}]{niḥśvanā}
		\rdg[wit={N23,E2,P11}]{niḥsvanaḥ}% nissvanaḥ K3
		}/}\\+}
\tl{
\pada{iti \app{\lem[wit={N3,J5,G11,N19,V15,J10,P11,C6,Jyo}]{nānāvidhā}
		\rdg[wit={Gr2,Gr3,V3}]{nānāvidho}}
	\app{\lem[wit={N3,J10,C6,Jyo}]{nādāḥ}
		\rdg[wit={J5,G11,V15,P11,V3}]{nādā}
		\rdg[wit={J7,Gr3}]{nādaḥ}
		\rdg[wit={N23}]{nādaṃ}
		\rdg[wit={N19}]{vādāḥ}}}
\pada{\app{\lem[wit={J5,G11,V15,J10,P11,C6,Jyo}]{śrūyante}
		\rdg[wit={N3,Gr2,Gr3,N19,V3}]{śrūyate}} % śṛyate N3
	\app{\lem[wit={ceteri}]{deha}% N3,J5,Gr2,Gr3,G11,GrB,Jyo
		\rdg[wit={N19,J10}]{yatra}
		\rdg[wit={V15}]{tatra}}%
	\app{\lem[wit={N3,J5,G11,N19,V15,J10,P11,V3}]{madhyataḥ}
		\rdg[wit={C6,Jyo}]{madhyagāḥ}
		\rdg[wit={Gr2,Gr3}]{madhyagaḥ}}//\versenr}
		\orgvnr{42}\\!}
\end{tlg}
%</vs42>
\commcite\newpage


%<*vs43>
\begin{tlg}[hp04_043]
\tl{
\pada{\app{\lem[wit={ceteri}]{mahati}
		\rdg[wit={J5}]{mahatiḥ}
		\rdg[wit={V15}]{mahatī}
		\rdg[wit={C6},alt={\om}]{\skp{\om}}}
	\app{\lem[wit={ceteri},alt={śrūyamāṇe/-māne}]{śrūyamāṇe}% °ṇo J5, °ne J10
		\rdg[wit={N23}]{{[ṇya]}yatamāne}}%
	\app{\lem[wit={ceteri}]{'pi}
		\rdg[wit={Gr2}]{ti}
		\rdg[wit={C6}]{pi nāde vai}}} % +P7
\pada{meghabhe%ry
	\app{\lem[wit={J5,Gr2,G5,N19,J10},alt={ādikadhvanau}]{\skm{ry}ādikadhvanau}% ddhūnau N19
		\rdg[wit={G11}]{ākadhvanau}
		\rdg[wit={Gr3,C6,V3,Jyo}]{ādike dhvanau}
		\rdg[wit={P11}]{ādike dhṛti}
		\rdg[wit={V15}]{ādike svane}
		\rdg[wit={N3}]{ādidaṃ dhvanau}}/}\\+}
\tl{
\pada{\app{\lem[wit={ceteri}]{tatra}% N3,J5,G11,N19,V15,J10,GrB,Jyo
		\rdg[wit={Gr2,Gr3}]{tataḥ}}
	\app{\lem[wit={ceteri},alt={sūkṣmāt}]{sūkṣmā\skp{t}}% sūkṣmā<<t>> J7
		\rdg[wit={J5,N19}]{sūkṣmā}
		\rdg[wit={P11}]{sūkṣmāṃ°}
		\rdg[wit={J10}]{sūkṣmaṃ}
		\rdg[wit={P11},alt={\om}]{\skp{\om}}}%
	\app{\lem[wit={ceteri},alt={sūkṣmataraṃ}]{\skm{t }sūkṣmataraṃ}% °tara N23
%		\rdg[wit={C7}]{sūkṣmatamaṃ}
		\rdg[wit={P11}]{°taraṃ nādaṃ}
		\rdg[wit={J10}]{nādam eva}}}
\pada{\app{\lem[wit={ceteri}]{nādam eva}
		\rdg[wit={J7}]{nādam evaṃ}
		\rdg[wit={J10}]{paritopi}}
	\app{\lem[wit={ceteri}]{parāmṛśet} % pasamṛśet N23
		\rdg[wit={V19}]{parāmṛṣet} % pasamṛṣet V19ac
		\rdg[wit={J5}]{parāmṛśaṃ}
		\rdg[wit={J7}]{samabhyaset}}//\versenr}
		\orgvnr{43}\\!}
\end{tlg}
%</vs43>
\commcite%\newpage


%<*vs44>
\begin{tlg}[hp04_044]
\tl{\app{\lem[nolem]{}
	\rdg[wit={E2},alt={\om}]{\skp{\om}}}%
\pada{\app{\lem[wit={ceteri},alt={ghanam}]{ghana\skp{m}}
		\rdg[wit={J10}]{dhvanam}}m utsṛjya vā 
	\app{\lem[wit={N3,G11,N19,V15,J10,GrB,Jyo}]{sūkṣme}
		\rdg[wit={J5,G4,Gr2,V19}]{sūkṣmaṃ}
		}} % sūkṣmo P7
\pada{sūkṣmam utsṛjya vā % sūkṣmasūtsṛjya N23; śū° P11
	\app{\lem[wit={Gr1,G11,N19,V15,P11,C6,Jyo}]{ghane}
		\rdg[wit={V3}]{ghanen}
		\rdg[wit={Gr2,V19}]{ghanam}% +K3,C7
		\rdg[wit={J10}]{dhune}}\marma/}\\+}
\tl{
%\pada{\app{\lem[wit={N3,J5,G11,GrB},alt={tau tyaktvā ... syād vā}]{%	dau F; tyaktrā J5, tyaktā P11, tyakvā V3; tu tyaktvā madh. + + + G4
%			tau tyaktvā
%			\app{\lem[wit={J5},alt={madhyame}]{madhyame}% +F
%				\rdg[wit={N3,G11,P11,V3},alt={madhyama}]{madhyama}
%				\rdg[wit={C6},alt={madhyama<<ḥ>>}]{madhyama<<ḥ>>}} 
%			\app{\lem[wit={N3,GrB},alt={syād vā}]{syād vā}% madhyame vāpi F
%				\rdg[wit={G11,G5},alt={syādau}]{syādau}
%				\rdg[wit={J5},alt={syātaṃstā}]{syātaṃstā}
%				}}% nested!
%		\rdg[wit={N19,V15}]{ramamāṇam api kṣipraṃ}
%		\rdg[wit={J10,Jyo}]{ramamāṇam api kṣiptaṃ}
%		\rdg[wit={Gr2,V19}]{paraṃ tatraiva nikṣipya}% +K3,C7
%		}}
\pada{% tu tyaktvā madh. + + + G4
	\app{\lem[wit={J5}]{tau tyaktvā madhyame}% dau F, tyaktrā J5
		\rdg[wit={N3,G11,P11,V3}]{tau tyaktvā madhyama}% tyaktā P11, tyakvā V3
		\rdg[wit={C6}]{tau tyaktvā madhyama<<ḥ>>} 
		\rdg[wit={Gr2,V19}]{paraṃ tatraiva}
		\rdg[wit={N19,V15,J10,Jyo}]{ramamāṇam api}
			} 
	\app{\lem[wit={N3,GrB}]{syād vā}% madhyame vāpi F
		\rdg[wit={G11,G5}]{syādau}
		\rdg[wit={J5}]{syātaṃstā}
		\rdg[wit={Gr2,V19}]{nikṣipya}% +K3,C7
		\rdg[wit={N19,V15}]{kṣipraṃ}
		\rdg[wit={J10,Jyo}]{kṣiptaṃ}
		}}
\pada{mano % manā? N23, manau V19
	\app{\lem[wit={ceteri}]{nānyatra}
		\rdg[wit={N19,V15,J10}]{nātra pra°}}
	\app{\lem[wit={ceteri}]{cālayet} % c<<ā>>layet J7
		\rdg[wit={J10}]{cālet}
		\rdg[wit={N23}]{vālayet}
		\rdg[wit={V3}]{cālayan}}//\versenr}
		\orgvnr{44}\\!}
\end{tlg}
%</vs44>
\commcite\newpage


%<*vs45>
\begin{tlg}[hp04_045]
\tl{
\pada{yatra kutrāpi vā nāde}
\pada{\app{\lem[wit={ceteri}]{lagati}
		\rdg[wit={N23}]{lagavi}
		\rdg[wit={P11}]{lagnaṃti}
		\rdg[wit={J10}]{galati}}
	\app{\lem[wit={ceteri}]{prathamaṃ}% °ma J5
		\rdg[wit={V19}]{prathame}}
	\app{\lem[wit={ceteri}]{manaḥ}
		\rdg[wit={N23}]{mataḥ}}/}\\+} %
\tl{
\pada{tatraiva
	\app{\lem[wit={N3,G11,V15,P11,C6},alt={tat}]{ta\skp{t}}% ta(l.br.)tat V15
		\rdg[wit={N19,V3}]{ta}
		\rdg[wit={J5}]{tā}
		\rdg[wit={J7,Gr3,Jyo}]{su°}% +F
		\rdg[wit={N23}]{stu}
		\rdg[wit={J10}]{niś°}}%
	\app{\lem[wit={ceteri},alt={sthirī}]{\skm{t }sthirī}% sthirā J5; +G5
		\rdg[wit={G11}]{sthiro}
		\rdg[wit={N19}]{śarī}
		\rdg[wit={J10}]{°calo}}%
	\app{\lem[wit={Gr1,G11,N19,V15,J10,GrB}]{bhūtvā}
		\rdg[wit={Jyo}]{bhūya}
		\rdg[wit={Gr2,Gr3}]{kuryāt}}} % kuryā J7
\pada{tena sārdhaṃ vilīyate//\versenr}%
%\myfnx{After this verse, \getsiglum{Jyo} has X4.100, 103, and 102.}
\label{yatrakutrapi}
\orgvnr{45}\\!} % vinīyate N3
\end{tlg}
%</vs45>
\commcite%\newpage

\Anm{\getsiglum{G11,N19,V15,J10} have \ref{kasthe}--\ref{sarvacinta} and \ref{sarvacinta2} here}% , and \getsiglum{V3} \ref{anahata}


%======== passage 4 (makaranda) ======

%<*vs46>
\begin{tlg}[hp04_046]
\tl{
\pada{makarandaṃ % ṃ om. J5,V15,J10
	\app{\lem[wit={ceteri},alt={piban}]{piba\skp{n}}
		\rdg[wit={J5}]{pived}
		\rdg[wit={N19}]{piven}}%
	\app{\lem[wit={Gr1,G11,V15,J10,GrB,Jyo},alt={bhṛṅgo}]{\skm{n }bhṛṅgo}% bhraṃgo P11
		\rdg[wit={Gr2,Gr3}]{bhṛṅgī}
		\rdg[wit={N19}]{śṛṃgo}}}
\pada{\app{\lem[wit={N3,G4,Gr3,G11,V3},alt={gandhān}]{gandhā\skp{n}}
		%\rdg[wit={K3,C7}]{gandhā}
		\rdg[wit={J7,N19,V15,J10,Jyo}]{gandhaṃ}
		\rdg[wit={J5,N23,C6}]{gandha}
		\rdg[wit={P11}]{gandho}}%
	\app{\lem[wit={ceteri},alt={nāpekṣate}]{\skm{n }nāpekṣate}% nā[p]i° G4; °kṣata J5
		\rdg[wit={N23}]{napekṣate}
		\rdg[wit={N19,J10}]{nopekṣate}}
	\app{\lem[wit={ceteri}]{yathā}
		\rdg[wit={N19}]{'nyathā}
		\rdg[wit={E2}]{yadā}}/}\\+}
\tl{
\pada{\app{\lem[wit={ceteri}]{nādāsaktaṃ}
		\rdg[wit={Gr2}]{nādasaktaṃ}} % śaktaṃ N19
	\app{\lem[wit={ceteri}]{tathā} % +P7
		\rdg[wit={C6}]{yathā}} cittaṃ}
\pada{viṣayā%n % viṣayā J5,J7, viṣayā<<n>> N23, °yāṃ V3,N19
	\app{\lem[wit={ceteri},alt={na hi}]{\skm{n }na hi}% nāhi P11
		\rdg[wit={V15}]{naiva}
%		\rdg[wit={C7}]{api}
		}
	\app{\lem[wit={N3,G11,N19,GrB,Jyo}]{kāṅkṣate}
		\rdg[wit={J5,Gr2,Gr3,V15,J10}]{kāṅkṣati}}//\versenr}
	\label{makaranda1}\orgvnr{46}\\!}
\end{tlg}
%</vs46>
\commcite\newpage


\Anm{\getsiglum{Gr2,Gr3} have \ref{nadakoti} \textit{nādakoṭisahasrāṇi} here}
 


%<*vs47>
\begin{tlg}[hp04_047]
\tl{\pada{\app{\lem[nolem]{\skp{pāda a}}
	\rdg[wit={Gr2,Gr3},alt={\om}]{\skp{\om}}}%
\app{\lem[wit={J5,G11,N19,V15,GrB,Jyo}]{baddhaṃ}
		\rdg[wit={J10}]{buddhaṃ}
		\rdg[wit={N3}]{baṃdhaṃ}}
	\app{\lem[wit={N3,J5,G11,P11,C6,Jyo}]{vimukta}
		\rdg[wit={N19}]{vimuktaṃ}
		\rdg[wit={V15,J10}]{viyuktaṃ}
		\rdg[wit={V3}]{timukta}
		}cāñcalyaṃ}
\pada{\app{\lem[nolem]{\skp{pāda b}}
	\rdg[wit={Gr2,Gr3},alt={\om}]{\skp{\om}}}%
nāda\app{\lem[wit={N3,J5,G11,N19,V15,J10,V3,Jyo}]{gandhaka}
		\rdg[wit={C6}]{gandhena}
		\rdg[wit={P11}]{gandhāya}}%
	\app{\lem[wit={N3,J5,G11,V15,C6,V3,Jyo}]{jāraṇāt}
		\rdg[wit={N19,J10,P11}]{jīraṇāt}}/} 
		\\+}
\tl{	%\sgwit{G11,N19,V15,J10,C6,V3,Jyo}
\pada{\app{\lem[nolem]{\skp{pāda c}}
	\rdg[wit={E2},alt={\om}]{\skp{\om}}}%
\app{\lem[wit={N3,J5,J7,V19,G11,N19,V15,J10,C6,Jyo}]{manaḥ}% +K3,C7
		\rdg[wit={P11,V3}]{mana}
		\rdg[wit={N23}]{vona}
		}%
	\app{\lem[wit={J5,G11,N19,J10,P11,C6,Jyo}]{pāradam āpnoti}
		\rdg[wit={V15}]{pārada āpnoti}
		\rdg[wit={V3}]{pāradham āpnoti}
		\rdg[wit={N3}]{pārajam āpnoti}
		\rdg[wit={J7,V19}]{pākam avāpnoti}% +K3,C7
		\rdg[wit={N23}]{cāvam avāpnoti}
		}}
\pada{\app{\lem[nolem]{\skp{pāda d}}
	\rdg[wit={E2},alt={\om}]{\skp{\om}}}%
\app{\lem[wit={ceteri}]{nirālambākhya}
%		\rdg[wit={C7},alt={°ākṣa}]{nirālambākṣa}
		\rdg[wit={P11},alt={°āsthya}]{nirālambāsthya}
		\rdg[wit={J5},alt={°aratha}]{nirālaṃbaratha}
		}%
	\app{\lem[wit={P11,V3}]{khoṭatām}
		\rdg[wit={N3,G11,G5,C6}]{ghoṭatāṃ}% ##
		\rdg[wit={J5}]{ghoṭatā}
		\rdg[wit={Gr2}]{ghoṭanam}
		\rdg[wit={N19}]{khoṭatī}
		\rdg[wit={V15}]{khoṭakaṃ}
		\rdg[wit={Jyo}]{khe'ṭanam} % +J11pc
		\rdg[wit={J10}]{khegataṃ}
		\rdg[wit={G4}]{gopitāṃ}
		\rdg[wit={V19}]{codanaṃ}
		}//\versenr}
		\orgvnr{47}\\!}
\end{tlg}
%</vs47>
\commcite%\newpage


\iffalse
\startaltrecension
%<*vs47-1>
\begin{alttlg}[hp04_047_1]
\tl{
\pada{\app{\lem[wit={G11,N19,V15,V3}]{baddhaḥ}
		\rdg[wit={C6}]{baddhas}
		\rdg[wit={J10}]{baddha}
		\rdg[wit={G5,Jyo}]{baddhaṃ}
		\rdg[wit={P11}]{baṃdhaḥ}}
	\app{\lem[wit={G11,G5,V3}]{sunādagandhena}
		\rdg[wit={N19}]{sunāde gandhena}
		\rdg[wit={P11}]{sunādavānpana}
		\rdg[wit={J10}]{sven nādagandhena}
		\rdg[wit={C6}]{tu nādagandhena}
		\rdg[wit={Jyo}]{tu nādabandhena}
		\rdg[wit={V15}]{suṃdhanādena}}}
\pada{\app{\lem[wit={G11,G5,N19,V15,J10,GrB}]{sadyaḥ}
		\rdg[wit={Jyo}]{manaḥ}}%
	\app{\lem[wit={G11,G5,N19,V15,J10,P11,C6,Jyo}]{saṃtyakta}
		\rdg[wit={V3}]{sa tyakta}}%
	\app{\lem[wit={G11,G5,N19,V15,J10,GrB}]{cāpalaḥ}
		\rdg[wit={Jyo}]{cāpalam}}/}\\+}
\tl{
\pada{prayāti
	\app{\lem[wit={G11}]{cetaḥsūtendraḥ}
		\rdg[wit={V3}]{cetaḥsuteṃdra}
		\rdg[wit={C6}]{cetaḥsūtrendre}
		\rdg[wit={G5}]{cetaḥśailendra}
		\rdg[wit={P11}]{cet sthūlendraḥ}
		\rdg[wit={V15}]{sūtacittendraḥ}
		\rdg[wit={N19}]{sūtaś citteṃdra}
		\rdg[wit={J10}]{svataś caikyaṃ iṃdra}
		\rdg[wit={Jyo}]{sutarāṃ sthairyaṃ}}}
\pada{\app{\lem[wit={G11,G5,N19,V15,P11,C6}]{pakṣachinna}
		\rdg[wit={J10}]{pacchacchinna}
		\rdg[wit={Jyo}]{chinnapakṣaḥ}
		\rdg[wit={V3},alt={\lacuna}]{\skp{\lacuna}}}
	\app{\lem[resp=emend]{iti prathām}% +F,M1
		\rdg[wit={G5}]{iti prathā}% +M3?
		\rdg[wit={P11}]{dṛti pṛthāṃ}
		\rdg[wit={C6}]{\_\,va patham}
		\rdg[wit={G11}]{iva prathāṃ}
		\rdg[wit={N19}]{iva prabhāṃ}
		\rdg[wit={V15}]{ivāprabhuḥ}
		\rdg[wit={J10}]{iva parvataḥ drumāḥ}
		\rdg[wit={Jyo}]{khago yathā}
		\rdg[wit={V3},alt={\lacuna}]{\skp{\lacuna}}}//\versenr}
	\sgwit{G11,N19,V15,J10,GrB,Jyo}\\!}
\end{alttlg}
%</vs47-1>
% ithi pradhā M3, iti prathāṃ M1, iva prathāṃ G11, iva dyāmaḥ G7
% cf. Rasendracūḍāmaṇi 16.52-54
%{pañcamo grāsaḥ}
%evaṃ ca pañcamo grāsaḥ pradātavyo'ṣṭamāṃśataḥ /
%sa pātrastho'gnisaṃtapto na gacchati kathañcana // Rcūm_16.52 //
%sa pakṣacchinna ity uktaḥ sa mukto'khiladurguṇaiḥ /
%so'yaṃ niṣevitaḥ sūtastrimāsaṃ rājikāmitaḥ // Rcūm_16.53 //
%viḍaṅgatriphalākṣaudraiḥ khe devaiḥ saha saṅgamam /
%ghrāṇamātreṇa sūtendraḥ sarvaroganikṛntanaḥ // Rcūm_16.54 //
%guṇā ete vinirdiṣṭā rasasya rasavādibhiḥ /
%sakalāste guṇāḥ satyā bhairaveṇa prakīrtitāḥ // Rcūm_16.55 //
% cf. also NWS pakṣaccheda
\endaltrecension
\fi


%<*vs48>
\begin{tlg}[hp04_048]
\tl{\app{\lem[nolem]{}
	\rdg[wit={G4},alt={\om}]{\skp{\om}}
	%\rdg[wit={G11},alt=\textapp{found after \ref{manomatta}}]{\skp{found after \manuref{4.49}}}
	}%% om. G5,M3
\pada{\app{\lem[wit={N3,J7,Gr3,V15,P11,C6},alt={nādaśravaṇataś cittam}]{nādaśravaṇataś citta\skp{m}} % cittaṃm V19
		\rdg[wit={N19}]{nādaḥ śravaṇataś cittam}
		\rdg[wit={V3}]{nādaḥ śravaṇataḥś citam}
		\rdg[wit={J5}]{nādaḥ śravaṇañ vittaṃm}
		\rdg[wit={G11}]{nadaśravaṇakṛc cittaṃ}
		\rdg[wit={N23}]{nādaśravaṇaś cittaṃ matam}
		\rdg[wit={J10}]{nādena praṇataṃ cittam}
		\rdg[wit={Jyo}]{nādaśravaṇataḥ kṣipram}
		}}%
\pada{\app{\lem[wit={N3,Gr2,E2,G11,GrB,Jyo},alt={antaraṅga}]{\skm{m }antaraṅga}
		\rdg[wit={J5}]{anataraṃga}
		\rdg[wit={N19,V15}]{aṃtaraṃgaṃ}
		\rdg[wit={J10}]{aṃtaraṃgā}
		\rdg[wit={V19}]{aṃtaraṃ sa}
		}%
	\app{\lem[wit={ceteri}]{bhujaṅgamaḥ} % ṃ om. V3
		\rdg[wit={J7,E2}]{turaṅgamaḥ}% +C7
		\rdg[wit={N23}]{turaṃgavaḥ}
		}/}\\+}
\tl{
\pada{\app{\lem[wit={Gr2,N19,V15,J10,P11,V3,Jyo}]{vismṛtya}
		\rdg[wit={N3,J5,G11,C6}]{saṃsmṛtya}% +G3 ??
		\rdg[wit={Gr3}]{viśūnyaṃ}}
	\app{\lem[wit={ceteri},alt={sarvam}]{sarva\skp{m}}
		\rdg[wit={N19,V15,J10}]{viśvam}}%
	\app{\lem[wit={N3,Jyo},alt={ekāgraḥ}]{\skm{m }ekāgraḥ}
		\rdg[wit={N23,Gr3,G11,J10,GrB}]{ekāgraṃ}% aikā° G11
		\rdg[wit={J5}]{(e)kāgra}% me om. J5
		\rdg[wit={J7}]{ekāgryaṃ}
		\rdg[wit={V15}]{evāgraḥ}
		\rdg[wit={N19}]{evāgra}}}
\pada{kutracin na hi dhāvati//\versenr} % naṃ hi P11
	\orgvnr{48}\\!}
\end{tlg}
%</vs48>
\commcite\newpage


%<*vs49>
\begin{tlg}[hp04_049]
\tl{
\pada{\app{\lem[wit={ceteri}]{manomatta}
		\rdg[wit={N23}]{manomantra}
		\rdg[wit={J10,V3}]{manonmatta}}gajendrasya} % maje° N19; °āsya V15
\pada{\app{\lem[wit={ceteri}]{viṣayodyāna}% °naṃ P11
		\rdg[wit={C6}]{viṣayoḍyā}
		\rdg[wit={J5}]{viṣayodhanu}
		\rdg[wit={V3}]{viṣayodhāma}
		\rdg[wit={G4}]{viṣayeṣudra}}%
	\app{\lem[wit={ceteri}]{cāriṇaḥ}
		\rdg[wit={P11}]{cāriṇaṃ}
		\rdg[wit={G4}]{cāraṇā[ḥ]}
		\rdg[wit={J5}]{vāriṇaḥ}
		\rdg[wit={N23}]{vāriṇaṃ}}/}\\+}
\tl{
\pada{\app{\lem[wit={N3,G4,Gr3,V3}]{niyāmana}
		\rdg[wit={G11,V15}]{niyāmane}
		\rdg[wit={Jyo}]{niyamane}
		\rdg[wit={J10}]{nīyamānaḥ}
		\rdg[wit={J5,P11,C6}]{niyamena}
		\rdg[wit={J7}]{niryāmana}
		\rdg[wit={N19}]{niryāsane}
		\rdg[wit={N23}]{niyamitra}
		}%
	\app{\lem[wit={ceteri}]{samartho'yaṃ}% samatho C6f
		\rdg[wit={G11}]{samartheyaṃ}
		}}
\pada{\app{\lem[wit={ceteri}]{ninādo}% N3,J5,Gr2,Gr3,G11,GrB
		\rdg[wit={N19,V15,J10,Jyo}]{nināda}}
%	\app{\lem[wit={J5,Gr2,G11,V15,J10,GrB,Jyo}]{niśitāṅkuśaḥ}
	\app{\lem[wit={ceteri}]{niśitāṅkuśaḥ} % niśinā° N23; °kuśa V3, °kuśaṃ J10, ṅ om. G11
		\rdg[wit={N19}]{niśatāṅkuḥ}
		\rdg[wit={Gr3}]{niścayāṅkuśaḥ}
		\rdg[wit={N3}]{niyatāṃkuśaḥ}}//\versenr}\label{manomatta}
		\orgvnr{49}\\!}
\end{tlg}
%</vs49>
\commcite\newpage


%<*vs50>
\begin{tlg}[hp04_050]
\tl{
\pada{\app{\lem[wit={ceteri}]{antaraṅga}
		\rdg[wit={V19}]{aṃtaraṃgaṃ}
		\rdg[wit={J10}]{nādoṃtaraṃ}}%
	\app{\lem[wit={G11,C6,V3},alt={°sya javino}]{\skp{°}sya javino}
		\rdg[wit={N3,J5}]{°sya javinaḥ}
		\rdg[wit={Jyo}]{°sya yamino}
		\rdg[wit={P11}]{°sya ca mano}
		\rdg[wit={Gr2,Gr3,N19,V15}]{turaṅgasya} % truraṃga N19
		\rdg[wit={J10}]{tu saṃgamya}}}
\pada{\app{\lem[wit={N19,V15,J10,GrB,Jyo}]{vājinaḥ}
		\rdg[wit={N3,J5}]{kariṇaḥ}
		\rdg[wit={G11}]{<<ga>>jasya}% G5 omits this hemistich.
		\rdg[wit={Gr2,Gr3}]{vijñānaṃ}} % °na N23
	\app{\lem[wit={N3,G11,Jyo}]{parighāyate}% +C8,P7
		\rdg[wit={P11}]{parighātayaḥ}
		\rdg[wit={C6}]{pariṣāyate}
		\rdg[wit={J5,Gr2,N19,J10,V3}]{paridhāyate}
		\rdg[wit={V15}]{paridhāvataḥ}
		%\rdg[wit={K3,C7}]{parimīyate}
		\rdg[wit={V19}]{parimeyate}
		\rdg[wit={E2}]{parameyate}}/}\\+}
\tl{
\pada{\app{\lem[nolem]{\skp{pāda c}}
	\rdg[wit={J10},alt={\om}]{\skp{\om}}}%
\app{\lem[wit={ceteri}]{nādopāstir ato}% N3,J5,Gr3,G11,N19,V15,GrB,Jyo
		\rdg[wit={Gr2}]{nādopāstivato}
%		\rdg[wit={C7}]{nādopāstimato}
%		\rdg[wit={V19}]{nādopāstiratir}% 1st occurrence
%		\rdg[wit={J10},alt={\om}]{\skp{\om}}
		} 
		nitya}%m % nityaṃm N3
\pada{\app{\lem[nolem]{\skp{pāda d}}
	\rdg[wit={J10},alt={\om}]{\skp{\om}}}%
\app{\lem[wit={N3,J5,V19a,P11,V3},alt={avadhāryāpi}]{\skm{m }avadhāryāpi}% +V19(1),C7(1)
		\rdg[wit={J7a}]{avadhāyāpi}
		\rdg[wit={N23a}]{anadhāyāpi}% 1st occurrence
		\rdg[wit={C6}]{avadhāryo pi}
		\rdg[wit={V15,Jyo}]{avadhāryā hi}
		\rdg[wit={N23b,J7b,E2b,G11}]{avagamyā hi}% avagamyo G5
		\rdg[wit={V19b}]{avagamya hi}
		\rdg[wit={N19}]{avagamyaṃ hi}
%		\rdg[wit={J10},alt={\om}]{\skp{\om}}
		}
	\app{\lem[wit={J5,GrB,Jyo}]{yoginā}
		\rdg[wit={N3,G11,N19,V15}]{yogināṃ}
		\rdg[wit={N23a,J7a,V19a}]{yoginaḥ}% +G5,K3,C7   1st
		\rdg[wit={N23b,J7b,V19b,E2b}]{yogibhiḥ}
%		\rdg[wit={J10},alt={\om}]{\skp{\om}}
		}//\versenr}%
\myfn{\orgx{\getsiglum{Gr2,Gr3} have a different verse order: 4.50cd (except \getsiglum{E2}) \textrightarrow\ 4.51 \textrightarrow\ \ref{anahata} \textrightarrow\ 4.50. }{}\getsiglum{Gr2,V19} have 4.50cd\,=\,X4.105cd cd twice. The first time (a), their reading of the last pāda is closer to the \textalpha\ reading \textit{avadhāryāpi yoginaḥ}, while the second time (b) it is \textit{avagamyā hi yogibhiḥ}, which is closer to the reading of the expanded version.}
		\label{IV95Vu}
		\orgvnr{50}\\!}
\end{tlg}
%</vs50>
\commcite%\newpage


\iffalse
\startaltrecension
%<*vs50-1>
\begin{alttlg}[hp04_050_1]
\tl{\app{\lem[nolem]{}
	\rdg[wit={N23,J7,V19,E2}]{\also}}%
\pada{\app{\lem[wit={Gr2,E2,G11,V15,P11,Jyo}]{nādo'ntaraṅga}
		\rdg[wit={C6,V3}]{nādotaraṅga}
		\rdg[wit={N19}]{nādāṃtaraṅga}
		\rdg[wit={V19}]{nādaturaṃga}
		\rdg[wit={J10},alt={\om}]{\skp{\om}}}%
	\app{\lem[wit={ceteri}]{sāraṅga}% sātaṅga? E2
%		\rdg[wit={C7}]{mātaṃga}
		\rdg[wit={J10},alt={\om}]{\skp{\om}}}}% sāraṅgaṃ V15
\pada{\app{\lem[wit={ceteri}]{bandhane}
		\rdg[wit={N23}]{baṃdhāna}
		\rdg[wit={V3}]{baṃdhana}
		\rdg[wit={J10},alt={\om}]{\skp{\om}}}
	\app{\lem[wit={ceteri}]{vāgurāyate}
		\rdg[wit={N23}]{yāgurāyate}
		\rdg[wit={J10},alt={\om}]{\skp{\om}}}/}\\+}
\tl{
\pada{\app{\lem[nolem]{\skp{pāda c}}
	\rdg[wit={G11plus},alt={\om}]{\skp{\om}}}%
\app{\lem[wit={ceteri}]{antaraṅga}
		\rdg[wit={N19,V15}]{antaraṅgaṃ}}%
	\app{\lem[wit={V15,Jyo}]{kuraṅgasya}
		\rdg[wit={Gr2,Gr3,N19,J10,GrB}]{turaṅgasya}
		%\rdg[wit={K3,C7}]{turaṅgasyā°}
		}}
\pada{\app{\lem[nolem]{\skp{pāda d}}
	\rdg[wit={G11plus},alt={\om}]{\skp{\om}}}%
\app{\lem[wit={N19,GrB}]{rodhe}
		\rdg[wit={J10}]{rogo}
		\rdg[wit={V15}]{nādo}
		\rdg[wit={Jyo}]{vadhe}
		\rdg[wit={N23}]{bāhye}
		\rdg[wit={J7}]{bodho}
		\rdg[wit={E2}]{°vabodhe}% +C7; °varodhe K3
		\rdg[wit={V19},alt={\lacuna}]{\skp{\lacuna}}}\marmas
	\app{\lem[wit={V15,Jyo}]{vyādhāyate}
		\rdg[wit={V3}]{vādhāyate}
		\rdg[wit={P11}]{vādyāyate}
		\rdg[wit={C6}]{pi pariṣā°}%  parighāyate P7
		\rdg[wit={N19}]{vā gāyate}
		\rdg[wit={J10}]{vā gīyate}
		\rdg[wit={Gr2}]{pi līyate}
		\rdg[wit={E2}]{līyate}% +K3,C7
		\rdg[wit={V19},alt={\lacuna}]{\skp{\lacuna}}}%
	\app{\lem[wit={ceteri}]{'pi ca}
		\rdg[wit={P11}]{ti ca}
		\rdg[wit={C6}]{°yate}
		\rdg[wit={V19},alt={\lacuna}]{\skp{\lacuna}}}//\versenr}%
		\myfn{In \getsiglum{G11}, the first hemistich is found between \ref{IV95Vu}ab and cd, and the second hemistich is omitted;
		In \getsiglum{GrB,Jyo}, the whole verse is found before \ref{IV95Vu};
		\getsiglum{J10} merges the two verses into one:
		\devnote{nādo'ntaraṃ tu saṃgamya vājinaḥ paridhāyate/
		antaraṅgaturaṃgasya rogo vā gīyate'pi ca//}}
		\label{makaranda6}\\!}
\end{alttlg}
%</vs50-1>
\endaltrecension
\fi


%
%% 4.064_1 (cf. 4.65cd)
%\pada{nādopāsti\app{\lem[wit={K3}]{r ato}
%		\rdg[wit={V19}]{ratir}
%		\rdg[wit={Gr2}]{vato}
%		\rdg[wit={C7}]{mato}} nityam}
%\pada{\app{\lem[wit={V19,C7}]{avadhāryāpi}
%		\rdg[wit={J7}]{avadhāyāpi}
%		\rdg[wit={N23}]{anadhāyāpi}
%		\rdg[wit={K3}]{avidhāryaṃ hi}} yoginaḥ/}\label{nadopasti}
%		\sgwit{Gr2,Gr3} \anm{≈ \ref{IV95Vu}cd}\\!}


%<*vs51>
\begin{tlg}[hp04_051]
\tl{
\pada{\app{\lem[nolem]{\skp{pāda a}}
	\rdg[wit={E2,N19,V15,J10},alt={\om}]{\skp{\om}}
	%\rdg[wit={G11,V3},alt=\textapp{found after \ref{saukhya}}]{\skp{found after \manuref{4.35}}}
	}%
\app{\lem[wit={N3,J5,P11,V3,Jyo},post=\texteng{(ādī \getsiglum{N3})}]{ghaṇṭādināda}
%	\rdg[wit={N3}]{ghaṃṭādīnāda}
	\rdg[wit={Gr2,V19,G11,C6}]{ghaṇṭānināda}}% ghaṃṭāki? V19
\app{\lem[wit={V3,Jyo}]{sakta}
	\rdg[wit={J5}]{śakti}
	\rdg[wit={N3}]{śaktaś ca}
	\rdg[wit={Gr2,V19,G11}]{saktasya}% sukta N23, °syaṃ? V19
	\rdg[wit={P11}]{sadaṃkatā}
	\rdg[wit={C6}]{kuliśa}}%
\app{\lem[wit={Jyo}]{stabdhāntaḥ}
	\rdg[wit={P11}]{stabdhyaṃtaḥ}
	\rdg[wit={J5}]{stadhvāṃta}
	\rdg[wit={N3}]{stavyāṃtaḥ}
	\rdg[wit={V3}]{statravadhātaḥ}
	\rdg[wit={G11}]{stabdhasyāntaḥ}
	\rdg[wit={N23}]{sabdāntaḥ}
	\rdg[wit={J7}]{śabdataḥ}
	\rdg[wit={V19}]{śuddhāntaḥ}%  °syaṃ? V19; +K3,C7
	\rdg[wit={C6}]{pradhvānta}}%
\app{\lem[wit={G11,P11,V3,Jyo}]{karaṇahariṇasya}
	\rdg[wit={N3}]{karaṇaṃ hariṇasya}
	\rdg[wit={J5}]{karaṇaṃ mṛgasya}
	\rdg[wit={C6}]{hariṇasya ca}
	\rdg[wit={J7,V19}]{karaṇasya ca}
	\rdg[wit={N23}]{karaṇasya na}}/} 
	\\+}
%	\sgwit{Gr1,Gr2,V19,K3,C7,G11,GrB,Jyo}
\tl{
\pada{\app{\lem[nolem]{\skp{pāda b}}
	\rdg[wit={Gr2,Gr3,N19,V15,J10},alt={\om}]{\skp{\om}}}%
	praharaṇa%m % karṇam! J5
\app{\lem[wit={G11,GrB},alt={atisukaraṃ}]{\skm{m }atisukaraṃ}
	\rdg[wit={N3}]{atisukasteraṃ}
	\rdg[wit={J5}]{avisukaraṇaṃ}
	\rdg[wit={Jyo}]{api sukaraṃ}}
\app{\lem[wit={N3,G11,P11,Jyo}]{syāc chara}% so P7
	\rdg[wit={C6}]{syāt sadṛ°}
	\rdg[wit={V3}]{syāra}
	\rdg[wit={J5}]{chara}}%
\app{\lem[wit={N3,G11,P11,V3}]{saṃdhātā}
	\rdg[wit={C6}]{°śaṃ dhātā}
	\rdg[wit={J5}]{saṃdhā}
	\rdg[wit={Jyo}]{saṃdhāna}} 
	pravīṇaś cet//\versenr} % vra° N3; °ṇāś G11
%	\sgwit{Gr1,G11,GrB,Jyo}%
	\orgvnr{51}\\!}
\end{tlg}
%</vs51>
\commcite\newpage
% P11
% ghaṃṭādinādasadaṃkatāstabdhāṃtaḥkaraṇahariṇasya/
% praharaṇam atisukaraṃ syāc charasaṃdhātā pravīṇaś cet//
% M1:
% ghaṃṭāninādasaktastabdhāṃtaḥkaraṇahariṇasya/
% praharaṇe sukaraṃ syā(c) [ch](ara)[sa]ṃdhānapravīṇaś cet// 4.103//
% P7:
% ghaṃṭādinādakuliśapradhvāṃtaharisasya ca/
% praharaṇam atisūkaraṃ syā<c> charasaṃghātā praviṇaś cet// 106/107 //
% C6:
% ghaṇṭāninādakuliśapradhvāṃtahariṇasya ca/
% praharaṇam atisukaraṃ syāt sadṛśaṃdhātā pravīnaś cet//
% N12:
% ghaṇṭādinākalādhvāṃtaḥkaraṇahariṇyaḥ syāt/
% praharaṇam atisukaraṃ syāc charasaṃdhānā pravīṇaś ca/


%<*vs52>
\begin{tlg}[hp04_052]
\tl{\app{\lem[nolem]{}
	\rdg[wit={G5,N19,V15,J10},alt={\om}]{\skp{\om}}}%
\pada{\app{\lem[wit={Gr1,Gr2,Gr3,G11,P11,V3,Jyo}]{anāhatasya śabdasya}% sabda° V3,N23
%		\rdg[wit={N23,V3}]{anāhatasya sabdasya}
		\rdg[wit={C6}]{anāhatas tu yaḥ śabdas}}}
\pada{\app{\lem[wit={J5,Gr2,Gr3,C6}]{tasya śabdasya yo dhvaniḥ}
		\rdg[wit={G11}]{tasya śabdasya yā dhvaniḥ}
		\rdg[wit={N3}]{tasya śabdasya ca dhvaniḥ}% +F
		\rdg[wit={G4}]{tasya yo dhvaniḥ}
		\rdg[wit={V3}]{śabdasyāṃtargato dhvaniḥ}
		\rdg[wit={P11}]{śabdasyāṃganabho dhvaniḥ}
		\rdg[wit={Jyo}]{dhvanir ya upalabhyate}}
		/}\\+}
\tl{
\pada{\app{\lem[wit={N3,Gr3,G11,P11,C6,Jyo},postwit=\texteng{\getsiglum{N23}\textsubscript{pc}},alt={dhvaner}]{dhvane\skp{r}}
		\rdg[wit={J5,G4,Gr2,V3}]{dhvanir}}r
		antargataṃ % °gata N23
	\app{\lem[wit={G4,N23,E2,G11},alt={jyotir}]{jyoti\skp{r}}% +G3
		\rdg[wit={J7,V19}]{jyoti} % jyotī P7
		\rdg[wit={N3,Jyo}]{jñeyaṃ}% +F ##
		\rdg[wit={P11,V3}]{geyaṃ}% geya P11
		\rdg[wit={J5,C6},alt={\om}]{\skp{\om}}}}%
\pada{\app{\lem[wit={Gr2,G11},alt={jyotirantar}]{\skm{r }jyotiranta\skp{r}}% +K3
		\rdg[wit={Gr3,C6}]{jyoterantar}
		\rdg[wit={J5}]{yotiraṃtar}
		\rdg[wit={G4}]{jyoti\,..\,.. }
		\rdg[wit={Jyo}]{jñeyasyāntar}% +F
		\rdg[wit={P11,V3}]{geyasyāntar}
		\rdg[wit={N3}]{yasyāṃtvaṃtar}
		}rgataṃ manaḥ/}\\+} % mana V3
\tl{
\pada{\app{\lem[wit={N3,J7,P11,V3}]{tan mano vilayaṃ}% lost G4
		\rdg[wit={G11}]{tan mano nilayaṃ}
		\rdg[wit={J5}]{tan maṃnaṃ vilayaṃ}
		\rdg[wit={N23,Gr3,C6}]{yan mano vilayaṃ}
		\rdg[wit={Jyo}]{manas tatra layaṃ}% +F
		}
	\app{\lem[wit={J5,N23,Gr3,G11,C6,V3,Jyo}]{yāti}
		\rdg[wit={N3,J7,P11}]{yāṃti}}}
\pada{tad viṣṇoḥ paramaṃ padam//\versenr} % viṣṇo N3,J5,C6
	\sgwit{Gr1,Gr2,Gr3,G11,GrB,Jyo}%
	\anm{cf. \expnr{X4.107}}
	\label{anahata}
	\orgvnr{52}\\!}
\end{tlg}
%</vs52>
\commcite\newpage


\iffalse
\startaltnormal
%<*vs52-1>
\begin{alttlg}[hp04_052_1]
\tl{\app{\lem[nolem]{}
	\rdg[wit={G11,G5,N19,V15,J10}]{\incl}}% not in Gr6,Jyo
\pada{anāhatadhvaner anta}%r anāhataḥ V17; anta<<r>> J10
\pada{\app{\lem[wit={N19,V15,J10},alt={°r jñeyaṃ yat}]{\skp{°}r jñeyaṃ ya\skp{t}}% gataṃ M3; +G7
	\rdg[wit={G11}]{°r geyaṃ yat}
	\rdg[wit={G5}]{°m āpnuyāt}}t
	sūkṣma\app{\lem[wit={N19,V15,J10}]{sūkṣmakam} % +M3; sūkṣmya° N19
	\rdg[wit={G11,G5}]{sūkṣmataḥ}}/}\\+}
\tl{
\pada{manas tatra layaṃ yāti}
\pada{tad viṣṇoḥ paramaṃ padam//\versenr} % viṣṇo
\label{anahata2}%\sgwit{G11,G5,N19,V15,J10}%
\anm{cf. \expnr{\ref{anahata}}}
\\!}
	%\myfn{\getsiglum{G11,G5,N19,V15,J10} have this verse as a substitute for \ref{anahata}; \getsiglum{G11} also contains 4.52 (see ??).}
%	-- Alt\,2 here and Alt\,1 after \ref{sadanada} (preceded by three additional lines: 
%  while \getsiglum{G5,M3} have the latter only.
	% \devnote{vindur bhidyati nādena sa nādaḥ khena bhidyate/
	% oṃkāradhvaninādena vāyus saṃharaṇāntikaṃ/
	% nirālaṃbaṃ samuddiśya yatra nādo layaṃ gataḥ//}) --, 
\end{alttlg}
%</vs52-1>
\endaltnormal
\fi


%<*vs53>
\begin{tlg}[hp04_053]
\tl{\app{\lem[nolem]{}
	\rdg[wit={E2},alt={\om}]{\skp{\om}}}%
\pada{\app{\lem[wit={ceteri},alt={tāvad ā°}]{tāvad ā}% tavad N23
		\rdg[wit={J10}]{bhāvanā°}}kāśasaṃkalpo} % kāsa J10; saḥkalpo J7
\pada{\app{\lem[wit={ceteri}]{yāvac chabdaḥ}% ḥ om. J5,P11,P7; N3,J5,Gr2,G11,V15,J10,GrB,Jyo
		\rdg[wit={V19}]{yāvad bandhaḥ}% +C7
		\rdg[wit={N19}]{yāvad vādhaḥ}} pravartate/}\\+}
\tl{
\pada{niḥśabdaṃ  % niśabdaṃ V19,N3,J5; °bdāṃ V17
	\app{\lem[wit={ceteri}]{tat paraṃ}
		\rdg[wit={N23}]{paramaṃ}} brahma}
\pada{\app{\lem[wit={ceteri}]{paramātmā}
		\rdg[wit={Jyo}]{paramātme°}} % paramatmā V19
	\app{\lem[wit={N3,J7,V3}]{samīryate} % for workshop!
		\rdg[wit={J5,N23,V19,P11}]{samīyate}% +G3; to be treated equally, be placed on a level with (Apte)
		\rdg[wit={C6}]{°yam īryate}
		\rdg[wit={G4}]{samīkṣate}
		\rdg[wit={N19,V15,J10}]{°numīyate}
		\rdg[wit={G11,Jyo}]{°ti gīyate}}//\versenr}
	\orgvnr{53}\\!}
\end{tlg}
%</vs53>
\commcite%\newpage


%<*vs54>
\begin{tlg}[hp04_054]
\tl{\app{\lem[nolem]{}
	\rdg[wit={E2,N19,V15,J10},alt={\om}]{\skp{\om}}}%
\pada{\app{\lem[wit={Gr1,Gr2,V19,G11,P11,C6,Jyo},alt={yat}]{ya\skp{t}}
		\rdg[wit={V3},alt={\om}]{\skp{\om}}}t kiṃci%n
	\app{\lem[wit={Gr1,G11,GrB,Jyo},alt={nāda}]{\skm{n }nāda}
		\rdg[wit={Gr2,V19}]{nāma}}rūpeṇa}
\pada{śrūyate śaktir eva sā/}\\+} % śū° N3
\tl{
\pada{\app{\lem[wit={N3,Gr2,G11,P11}]{yas tacchrotā}% +K3,C7
		\rdg[wit={C6}]{yas tatsrotā}
		\rdg[wit={V19}]{yat ta[cch]roto}
		\rdg[wit={V3}]{yac chrotā ca}
		\rdg[wit={J5}]{yasmin śrato}
		\rdg[wit={Jyo}]{yas tattvānto}} 
		nirākāraḥ} % °kāra P11,V3; °kārā J5
\pada{sa eva parameśvaraḥ//\versenr}
\orgvnr{54}\\!} % ḥ om. J5,V3; e<<va>> N23
\end{tlg}
%</vs54>
\commcite\newpage


%<*vs55>
\begin{tlg}[hp04_055]
\tl{%\app{\lem[nolem]{}
	%\rdg[wit={G11,N19,V15,J10,GrB,Jyo},alt=\textapp{found after \ref{sapadakoti}}]{\skp{found after xxx}}}%
\pada{śravaṇa\app{\lem[wit={N3,J5,G11,N19,V15,GrB}]{mukha}
		\rdg[wit={Gr2,Gr3,J10,Jyo}]{puṭa}}%
	\app{\lem[wit={ceteri}]{nayana} % cayana N23, naya<<na>> N3
		\rdg[wit={J10,Jyo}]{nayanayugala}}%
	\app{\lem[wit={ceteri}]{nāsā} % nāśā V19,N19,J5,V3,J10
		\rdg[wit={Jyo}]{ghrāṇa}}%
	\app{\lem[wit={J5,G11,N19,V15,P11,C6}]{nirodhanaṃ naiva kartavyam}% nirodhamaṃ N19
		\rdg[wit={N3}]{nirodhaṃ naiva kartavyaṃ}
		\rdg[wit={V3}]{nirodhanenaiva kartavyaṃ}
		\rdg[wit={Gr2,E2}]{mukhapuṭasaṃrodhanaṃ kāryam}
		\rdg[wit={V19}]{mukhapuṭarodhane kāryaṃ}
		\rdg[wit={J10}]{mukharodhanam eva kartavyaṃ}
		\rdg[wit={Jyo}]{mukhānāṃ nirodhanaṃ kāryam}}/}\\+}
\tl{
\pada{\app{\lem[wit={ceteri}]{śuddha}
		\rdg[wit={Gr2}]{śrīśuddha} % suddha N23
		\rdg[wit={V3},alt={\om}]{\skp{\om}}}%
	\app{\lem[wit={ceteri}]{suṣumṇā}% śu° P11,V15
		\rdg[wit={N23}]{suṣumū}}% 
	\app{\lem[wit={J7,Gr3,G11,Jyo}]{saraṇau}% śaraṇau V19,J10pc
		\rdg[wit={N19,V15,J10}]{śaraṇe}
		\rdg[wit={N3}]{tsaraṇaḥ}
		\rdg[wit={J5}]{śarada}
		\rdg[wit={G4}]{saraṇaiḥ}
		\rdg[wit={C6}]{tmaśaraṇaiḥ}
		\rdg[wit={P11}]{tmakārausaṃ}
		\rdg[wit={V3}]{maraṇai}
		\rdg[wit={N23}]{ṇau}}
	\app{\lem[wit={ceteri}]{sphuṭam amalaḥ}% amala J5,G4, amalaṃ V3; = Gr1,Gr2,Gr3,G11,J10,P11,V3,Jyo
		\rdg[wit={C6}]{saṃsphurad amalaḥ}
		\rdg[wit={V15}]{vimalaḥ saṃ°}
		\rdg[wit={N19}]{vimalaḥ}}
	śrūyate nādaḥ//\versenr} % śṛyate N3, śraya .. + + G11; nāda J5,P11,V3
	\label{sravanaputa}
	\orgvnr{55}\\!}
\end{tlg}
%</vs55>
\commcite%\newpage


\iffalse
\startaltrecension
%<*vs55-1>
\begin{alttlg}[hp04_055_1]
\tl{\app{\lem[nolem]{}
	\rdg[wit={Jyo}]{\NotIn}}%
\pada{\app{\lem[wit={V15,J10,C6,V3}]{nādaḥ}% +M3
		\rdg[wit={G11,G5,N19,P11}]{nāda}} % +F
		śaktir iti
	\app{\lem[wit={V15,J10}]{khyāto} % khyato J10ac
		\rdg[wit={G11,G5}]{khyātā}% +M3
		\rdg[wit={N19}]{kṣāto}
		\rdg[wit={P11}]{jñeyaṃ}
		\rdg[wit={C6}]{jñeyā}% +F
		\rdg[wit={V3}]{jñeya}}}
\pada{\app{\lem[wit={G11,G5,N19,V15,P11,V3}]{nādajñānaṃ}
		\rdg[wit={J10,C6}]{nādo jñānaṃ}} sadāśivaḥ/}\\+} % śiva P11,V3
\tl{
\pada{\app{\lem[wit={G11}]{jñeye jñāne ca naṣṭe tu}
		\rdg[wit={G5}]{jñeyajñāne ca naṣṭe ca}
		\rdg[wit={V3}]{jñeye jñāne vilineṃta}
		\rdg[wit={P11}]{jñeye jñāne vilīnīṃta}% jñeyajñāne hi līyete
		\rdg[wit={C6}]{jñeyo jñāne vilīne tu}% vilīyeta P7
		\rdg[wit={N19}]{nādajñāne ca neṣṭe tad}% +J11ac
		\rdg[wit={V15}]{nādajñāne vinaṣṭe ca tad}
		\rdg[wit={J10}]{nādajñānena naṣṭena}% +J11pc
		}}
\pada{\app{\lem[wit={G11,G5,V15},alt={unmany}]{unma\skp{ny}}% +J11
		\rdg[wit={N19}]{unmadhy}
		\rdg[wit={J10}]{hy unmany}
		\rdg[wit={GrB}]{sonmany}% +F
		}%
	\app{\lem[wit={G11,G5,J10,C6},alt={evāvaśiṣyate}]{\skm{ny }evāvaśiṣyate}
		\rdg[wit={N19}]{edhāvaśiṣyate}
		\rdg[wit={V3}]{avāvaśiṣyate}
		\rdg[wit={P11}]{enāvaśiṣyati}
		\rdg[wit={V15}]{eva śiṣyate}}//\versenr}
		%\sgwit{G11,G5,N19,V15,J10,GrB}
		\\!}
\end{alttlg}
%</vs55-1>

%<*vs55-2>
\begin{alttlg}[hp04_055_2]
\tl{\app{\lem[nolem]{}
	\rdg[wit={Jyo}]{\NotIn}}%
\pada{nādo yāvan manas tāva}%n
\pada{\app{\lem[wit={G11,G5,N19,J10,P11,V3},alt={nādānte tu}]{\skm{n }nādānte tu}
	\rdg[wit={V15}]{nādānte ca}
	\rdg[wit={C6}]{nādātīte}} manonmanī/}\\+}
\tl{
\pada{saśabdaṃ kathitaṃ vyoma} % na śabdaṃ G5
\pada{niḥśabdaṃ brahma 
	\app{\lem[wit={N19,V15,J10,GrB}]{kathyate}
	\rdg[wit={G11,G5}]{ucyate}% +F
	}//\versenr}
%\sgwit{G11,G5,N19,V15,J10,GrB}
\\!}
\end{alttlg}
%</vs55-2>

%<*vs55-3>
\begin{alttlg}[hp04_055_3]
\tl{
\pada{sadā nādānusandhānā}%t °bandhānāt G5
\pada{\app{\lem[wit={G11,G5,N19,V15,J10,GrB},alt={saṃkṣīṇe}]{\skm{t }saṃkṣīṇe}% sa P11
		\rdg[wit={Jyo}]{kṣīyante}}
	\app{\lem[wit={G11,G5,P11,C6}]{vāsanācaye}% +P7
		\rdg[wit={J10}]{vāsanodaye}% °dvaye F
		\rdg[wit={V3}]{vāsanāvayo}
		\rdg[wit={N19}]{vāsanākṣaye}
		\rdg[wit={V15}]{vāsanākṣaṇe}
		\rdg[wit={Jyo}]{pāpasaṃcayāḥ}}/}\\+}
\tl{
\pada{nirañjane
	\app{\lem[wit={G11,Jyo}]{vilīyete}% +F
		\rdg[wit={C6}]{vilīyeta}
		\rdg[wit={P11,V3}]{vilīyaṃte}
		\rdg[wit={V15,J10}]{ca līyete}
		\rdg[wit={G5,N19}]{ca līyeta}
		}}
\pada{\app{\lem[wit={G11}]{niścitaṃ manamārutau}% ##
		\rdg[wit={G5}]{niścitaṃ manamārute}
		\rdg[wit={N19}]{niścitta manamārutau}
		\rdg[wit={J10}]{niścitau manamārutau}% +F
		\rdg[wit={P11}]{niścitaṃ māruto manaḥ}
		\rdg[wit={V3}]{niścita māruto mana}
		\rdg[wit={V15,Jyo}]{niścitaṃ cittamārutau}
		\rdg[wit={C6}]{marutā niścitaṃ manaḥ}
		}//\versenr}
	%\sgwit{G11,G5,N19,V15,J10,GrB,Jyo}
	\label{sadanada}\\!}
\end{alttlg}
%</vs55-3>

%<*vs55-4>
\begin{alttlg}[hp04_055_4]
\tl{\app{\lem[nolem]{}
	\rdg[wit={N23,J7,V19,E2},postwit=\texteng{(found after \ref{makaranda1})}]{\also}
	\rdg[wit={Jyo}]{\NotIn}}%
\pada{nādakoṭisahasrāṇi} % sahasrāṇī J7, sahasrāni N23, °śrāṇi V3,V19
\pada{\app{\lem[wit={ceteri}]{bindu}
	\rdg[wit={C6}]{veda}}koṭiśatāni ca/}\\+} % saṭāni N23ac, satāni N23pc
\tl{
\pada{\app{\lem[wit={ceteri}]{sarve}
		\rdg[wit={N23}]{sarvaṃ}} tatra layaṃ
	\app{\lem[wit={ceteri}]{yānti}
		\rdg[wit={V19,C6}]{yāti}}}
\pada{yatra % ya ca P11
	\app{\lem[wit={ceteri}]{devo}% +G5,M3
		\rdg[wit={G11}]{deve}
		\rdg[wit={N19,V3}]{deva}}
	\app{\lem[wit={ceteri}]{nirañjanaḥ}% +G5,M3; °jana P11, janaṃ V3
		\rdg[wit={G11}]{nirañjane}
		}//\versenr}
	\label{nadakoti} %\sgwit{Gr2,Gr3,G11,N19,V15,J10,GrB}
	\\!}
\end{alttlg}
%</vs55-4>

%<*vs55-4p>
\begin{altpostmula}[hp04_055_4p]
\app{\lem[wit={G11,N19,V15,J10,P11,Jyo}]{iti nādānusandhānam}
\rdg[wit={G5,C6,V3},postwit=\texteng{(found between pāda ab and cd of the next verse \getsiglum{C6})}]{iti nādānusandhānavidhiḥ} % vidhi V3; +F
%\rdg[wit={N19}]{iti nādānusaṃdhānāṃ yathā vṛddho veti}
%\rdg[wit={V15},post=\texteng{(metrical!)}]{iti nādānusaṃdhānaṃ yathā vṛddhaiḥ prabhāṣitaṃ}
}//
\sgwit{G11,G5,N19,V15,J10,GrB,Jyo}
\end{altpostmula}
%</vs55-4p>
\Anm{\getsiglum{V3} has Kālajñāna, Videhamuktikathana, and Kālavañcana sections here}

%%%%%%%%%%%%%%%%%%%%%%%


%<*vs55-5a>
\begin{altava}[hp04_055_5a]
\app{\lem[nolem]{}
	\rdg[wit={G11,G5}]{\only}}%
atha rājayogaḥ/
\end{altava}
%</vs55-5a>
\Anm{\getsiglum{G11,G5} has 1.64 \textit{yuvā vṛddho 'tivṛddho vā} here.%
\myfn{\getsiglum{N19,V15} have the remnant of this verse. See the apparatus to the ending ``iti nādānusandhānam" above.}}

\Anm{\getsiglum{G11,N19,V15,J10} have \ref{kalavancaka} \textit{sarve layahaṭhābhyāsāḥ} and \ref{astuva}ff. \textit{astu vā māstu vā} here}

%<*vs55-5>
%\begin{alttlg}[hp04_055_5]
%\tl{\app{\lem[nolem]{}
%	\rdg[wit={Gr3}]{\incl}}
%\pada{sarve \app{\lem[wit={C6,V3}]{haṭhalayopāyā}
%	\rdg[wit={P11}]{haṭhalayā bhāvyā}}}
%\pada{rājayoga\app{\lem[wit={P11}]{padāvadhi}
%	\rdg[wit={C6}]{padāvadhiḥ}
%	\rdg[wit={V3}]{padāvadhiṃ}}/}\\+}
%\tl{
%\pada{rājayogapadaṃ prāpya}
%\pada{jāyate\app{\lem[wit={P11,C6}]{'sau}
%	\rdg[wit={V3}]{so}} nirañjanaḥ//\versenr}% °jana V3
%	\myfn{This verse is only included in \getsiglum{GrB} as a substitute for \expnr{X4.116}.}
%	\sgwit{GrB}
%	\\!}
%\end{alttlg}
%</vs55-5>
\endaltrecension
\fi


%<*vs56>
\begin{tlg}[hp04_056]
\tl{\app{\lem[nolem]{}
	\rdg[wit={N19,V15,J10,Jyo},alt={\om}]{\skp{\om}}
	%\rdg[wit={G11},alt=\textapp{found after \manuref{4.0*8}}]{\skp{found after xxx}}
	}%
\pada{\app{\lem[nolem]{\skp{pāda a}}
	\rdg[wit={J5},alt={\om}]{\skp{\om}}}%
\app{\lem[wit={N3,G4,Gr2,G11,GrB}]{kāṣṭha}
		\rdg[wit={Gr3}]{koṣṭha}}% ab lost J5; none N24
	\app{\lem[wit={Gr3,G11}]{goṣṭhī}
		\rdg[wit={N3,G4,J7}]{goṣṭhi}% +F
		\rdg[wit={N23,V3}]{goṣṭha}
		\rdg[wit={P11}]{mathnī}
		\rdg[wit={C6}]{mathnā}}%
	\app{\lem[wit={G11,V3}]{prapañcena}
		\rdg[wit={N3}]{prapaṃce}
		\rdg[wit={G4,Gr2,Gr3}]{prasaṅgena}% none J5,N24
		\rdg[wit={P11}]{pravacane}
		\rdg[wit={C6}]{pravartaṃ}}\marma}
\pada{\app{\lem[nolem]{\skp{pāda b}}
	\rdg[wit={J5},alt={\om}]{\skp{\om}}}%
\app{\lem[wit={N3,G4,G11,GrB}]{kiṃ sakhe śrūyatām idam}% śṛ° N3; °tam P11
		\rdg[wit={J7,Gr3}]{nādam antargataṃ śṛṇu} % śruṇu V19
		\rdg[wit={N23}]{nāgadaṃtaṃmatargataṃ sṛṇu}}/}
	\\+}
\tl{
\pada{purā matsyendra\app{\lem[wit={N3,J5,G11,GrB},alt={bodhārtham}]{bodhārtha\skp{m}}
		\rdg[wit={Gr2,Gr3}]{bodhāya}}}%
\pada{\app{\lem[wit={N3,J5,J7,Gr3,G11,P11,C6},alt={ādināthoditaṃ}]{\skm{m }ādināthoditaṃ}
		\rdg[wit={N23}]{ādināthotigaditaṃ}
		\rdg[wit={V3}]{ānināthodinaṃ}}
		vacaḥ//\versenr}\label{kastha}
	\orgvnr{56}\\!} % vaca N3,V3
\end{tlg}
%</vs56>
\commcite\newpage

% C6 4.110: kaṣṭamathnā pravartaṃ (1 syll. ami.) kiṃ sakhe śrūyatām idam/
% purā matsyendrabodhārtham ādināthoditaṃ vacaḥ//
% P7: kāṣṭamathnvāṃ pravartante kiṃ sakhe śrūyatām idam/
% purā matsyendrabodhārtham ādināthoditaṃ vacaḥ//
% C8: kāṣṭamanāpravartaṃte ... bodhāya
% P11: kāṣṭhamathnī pravacane ... bodhārtham


%<*vs57>
\begin{tlg}[hp04_057]
\tl{
\pada{yāvan naiva 
	\app{\lem[wit={ceteri}]{praviśati}
		\rdg[wit={N23}]{\_viśati}}
	\app{\lem[wit={ceteri},alt={caran}]{cara\skp{n}}
		\rdg[wit={J7}]{calan}
		\rdg[wit={N23}]{palan}
		\rdg[wit={N3}]{care}
		\rdg[wit={V3},alt={\om}]{\skp{\om}}}%n 
	\app{\lem[wit={ceteri},alt={māruto}]{\skm{n }māruto}
		\rdg[wit={N3}]{mārutaṃ}} % mādgato N23
	\app{\lem[wit={ceteri}]{madhya}
		\rdg[wit={V15}]{mādhya}}%
	\app{\lem[wit={N3,J5,J7,V19,G11,N19,J10,C6,Jyo}]{mārge}
		\rdg[wit={N23,P11}]{mārgo}
		\rdg[wit={E2,V15}]{mārgaṃ}% +C7
		\rdg[wit={V3}]{mārgā}}}\\+}
\tl{
\pada{yāva%d
	\app{\lem[wit={ceteri},alt={bindur}]{\skm{d }bindu\skp{r}} % bindu P11, vāyuḥ V19ac
		\rdg[wit={V15}]{bandho}
		\rdg[wit={N19}]{bandhaṃ}}%r 
	\app{\lem[wit={ceteri},alt={na bhavati}]{\skm{r }na bhavati}
		\rdg[wit={J10}]{bhavati na}}
	\app{\lem[wit={ceteri}]{dṛḍhaḥ} % dṛḍha V19
		\rdg[wit={N3,G11,P11}]{dṛḍhaṃ}% +F
		\rdg[wit={J5}]{sthiraḥ}}
	prāṇa\app{\lem[wit={Gr1,J7,G11,J10,GrB,Jyo}]{vāta}% prāṇa om. J5, prā<<ṇa>> V3
		\rdg[wit={N23,Gr3,V15}]{vātaḥ}
		\rdg[wit={N19}]{vātaṃ}}%
	\app{\lem[wit={Gr2,C6}]{prabaddhaḥ}
		\rdg[wit={G4}]{prabaddhaṃ}
		\rdg[wit={G11,V15,P11}]{prabandhaḥ}% ḥ om. P11
		\rdg[wit={Gr3,J10}]{prabuddhaḥ}
		\rdg[wit={N3}]{prabodhaḥ}
		\rdg[wit={V3}]{prabodhakaḥ}
		\rdg[wit={J5}]{prakṛddhaḥ}
		\rdg[wit={N19}]{na bandhanaḥ}
		\rdg[wit={Jyo}]{prabandhāt}% +F(twice),M1
		}/}\\+}
\tl{
\pada{\app{\lem[wit={N19,V15,P11,C6}]{yāvad vyomnā}
		\rdg[wit={N3,G4,G11}]{yāvad yomnā}
		\rdg[wit={J5}]{yād vyemnā}
		\rdg[wit={J7,Gr3,J10}]{yāvad vyomnaḥ}% dyo° J7; rather °nmaḥ J10
		\rdg[wit={N23}]{yāva\,\_\,mnaḥ}% yāva _ mnaḥ N23
		\rdg[wit={V3}]{yāvad byomna}
		\rdg[wit={Jyo}]{yāvad dhyāne}}
	\app{\lem[wit={ceteri}]{sahajasadṛśaṃ}% °sadṛsāṃ J5
		\rdg[wit={N23}]{sahajasaṃśaṃ}
		\rdg[wit={G11}]{sadṛśasahajā}% °sahajaṃ F
		} jāyate naiva
	\app{\lem[wit={ceteri}]{tattvaṃ}
		\rdg[wit={V15,J10,V3}]{cittaṃ}}}\\+}
\tl{
\pada{tāva\app{\lem[wit={ceteri},alt={sarvaṃ}]{\skm{t }sarvaṃ}
		\rdg[wit={G11}]{satvaṃ}
		\rdg[wit={J10,V3,Jyo}]{jñānaṃ}} vadati % vadaṃti V3
	\app{\lem[wit={N3,J5,J7,E2,N19,V15,J10,C6}]{yad idaṃ}% +C7
		\rdg[wit={V19,Jyo}]{tad idaṃ}% +K3
		\rdg[wit={N23,P11}]{yadi}% post=\texteng{(daṃ om. by haplography)}
		\rdg[wit={G11}]{yadi tat}% yadi taṃ F
		\rdg[wit={V3}]{satataṃ}}
	\app{\lem[wit={ceteri}]{dambha}
		\rdg[wit={G11,N19}]{ḍaṃbha}
		}mithyā%
	\app{\lem[wit={ceteri}]{pralāpaḥ}% °lāpa J5,N19, prallāpaḥ V3
		\rdg[wit={C6}]{pralābhaḥ}}//\versenr}%
	\myfn{In \getsiglum{Jyo}, this verse is found at the end of the chapter.}
	\label{yavan}		
	\orgvnr{57}\\!}
\end{tlg}
%</vs57>
\commcite\newpage

%========== Passage B ==========


\Anm{The following verses \ref{jnatva}--\ref{tatraika} are found immediately after \ref{jnanam} in \getsiglum{G11,N19,V15,J10,Jyo}}

%<*vs58>
\begin{tlg}[hp04_058]
\tl{
\pada{\app{\lem[wit={ceteri}]{jñātvā}
		\rdg[wit={C6}]{jitvā}
		\rdg[wit={V15}]{suṣu°}}
	\app{\lem[wit={N3,J5,J10,Jyo}]{suṣumṇāsadbhedaṃ}
		\rdg[wit={G11,GrB}]{suṣumṇāsaṃbhedaṃ}% +F
		\rdg[wit={N19}]{suṣumṇāṃ saśvedaṃ}
		\rdg[wit={J7,Gr3}]{suṣumṇābhedaṃ hi}% sukhu° V19
		\rdg[wit={N23}]{suṣu<<m>>ṇāṃmedehi}
		\rdg[wit={V15}]{°mnāṃtagataṃ mārgaṃ}}}
\pada{\app{\lem[wit={ceteri}]{kṛtvā vāyuṃ}% kṛtvān V3; vāyu V19, pāyu N19
		\rdg[wit={V15}]{vāyuṃ kṛtvā}
		\rdg[wit={J5}]{tvāpa vāyuṃ}} ca
	\app{\lem[wit={ceteri}]{madhyagam}
		\rdg[wit={P11}]{madhyamaḥ}}/}\\+}
\tl{
\pada{\app{\lem[wit={N3,V3}]{kṛtvāsāv aindave sthāne} % aidava V3
		\rdg[wit={P11}]{kṛtvāsav aidavai sthānair}
		\rdg[wit={J5}]{kṛtvā tām aidave tthāne}
		\rdg[wit={N23}]{nītvā tāv iṃdavasthāne}% kṛtvā tāv iṃdavasthāne K1
		\rdg[wit={J7}]{nītvā tāvad avasthāne}% kṛtvā tāvad vindusthairyaṃ B2
		\rdg[wit={Gr3}]{nītvā tām anavasthāne}
		\rdg[wit={G4}]{[dh]ṛ\,..\,[sāv a]ṃdra\,..\,[sthā]ne}% dhṛtvā taṃ caindavasthāne F
		\rdg[wit={C6}]{hṛtvā mamedaṃ ca sthānaṃ}
		\rdg[wit={G11}]{sthitvā sa vaindave sthāne}
		\rdg[wit={N19}]{sthitvā sāṃcaiṃdave sthāne}
		\rdg[wit={J10}]{sthitvā sadaiṃdave sthāne}
		\rdg[wit={Jyo}]{sthitvā sadaiva susthāne}
		\rdg[wit={V15}]{samāvasthā sthito yogī}}}
\pada{\app{\lem[wit={Gr1,G11,N19,GrB}]{ghrāṇa}
		\rdg[wit={Gr2,Gr3,V15,J10}]{prāṇa}% +N24
		\rdg[wit={Jyo}]{brahma}}%
	\app{\lem[wit={Gr1,J7,J10,C6,V3,Jyo}]{randhre}
		\rdg[wit={N23,Gr3,G11,N19,V15}]{randhraṃ}
		\rdg[wit={P11}]{randhra}}
	\app{\lem[wit={N3,G4,G11,N19,V15,J10,GrB,Jyo}]{nirodhayet}% +C7
		\rdg[wit={Gr2,Gr3}]{nirundhayet}
		\rdg[wit={J5}]{niyojayet}% +F
		}//\versenr}\label{jnatva}
		\orgvnr{58}\\!}
\end{tlg}
%</vs58>
\commcite\newpage
% V15 totally different: suṣumnāṃtagataṃ mārgaṃ vāyuṃ kṛtvā ca madhyagaṃ/ samāvasthā sthito yogī prāṇaraṃdhraṃ nirodhayet//
% K1 kṛtvā tāv iṃdavasthāne prāṇa°
% B2 kṛtvā tāvad bindusthairyaṃ prāṇa°
% J11 sthitvā sā caiṃdavasthāne ghrāṇa°


% between V3_4.175-176
%<*vs59a>
\begin{ava}[hp04_059a]
\app{\lem[nolem]{}
	\rdg[wit={Gr1,C6,V3}]{\only}}%
\app{\lem[wit={N3,G4,C6}]{tathā ca vasiṣṭhaḥ} % <<ḥ>> N3, vasiṣṭāḥ P7, ca vasiṣṭaḥ C6
		\rdg[wit={J5}]{tathā vaśiṣṭhavacanaṃ}
		\rdg[wit={V3}]{tatvāva || ☼ ||}/} %\sgwit{Gr1,C6,V3} % om. P11
\end{ava}
%</vs59a>
%<*vs59>
\begin{tlg}[hp04_059]
\tl{\app{\lem[nolem]{}
	\rdg[wit={G11,N19,V15,J10,Jyo},alt={\om}]{\skp{\om}}}%
\pada{iḍāyāṃ % idā° N23
	\app{\lem[wit={N3,J5,Gr2,Gr3,P11,C6}]{piṅgalāyāṃ ca}
		\rdg[wit={V3}]{piṅgalāyāṃśca}}}
\pada{carataś candrabhāskarau/}\\+}
\tl{
\pada{candras tāmasa ity uktaḥ}
\pada{sūryo
	\app{\lem[wit={N3,J5,J7,Gr3,GrB}]{rājasa}
		\rdg[wit={N23},post=\texteng{(end of the last available folio)}]{rā}} ucyate//\versenr}
		%\myfn{\getsiglum{N23} is lost after \textit{sūryo rā} in pāda d.}
		\orgvnr{59}\\!}
\end{tlg}
%</vs59>
		%\sgwit{N3,J5,Gr2,Gr3,GrB}


%<*vs60>
\begin{tlg}[hp04_060]
\tl{\app{\lem[nolem]{}
	\rdg[wit={N23}]{\unavbl}}%
\pada{\app{\lem[alt={tāv eva ... sakalaṃ}]{\skp{tāv eva ... sakalaṃ}}%wit={N3,J5,J7,V19,E2,GrB},
	\rdg[wit={G11}]{sūryaś candraḥ sadā dhatte}% om. G5; sūryaṃ candraṃ M3
	\rdg[wit={N19}]{sūryacandrau sadā dhatte}% dhattaḥ J11,E4,N26
	\rdg[wit={V15,Jyo}]{sūryācandramasau dhattaḥ}% +M1; sūryacandramasā yuktaṃ F
	\rdg[wit={J10}]{sūryācandramasau kṛtvā}
	}%
	\app{\lem[wit={N3,J5,J7,E2,P11,C6}]{tāv eva}% tau eva C6
		\rdg[wit={V19}]{tā eva}
		\rdg[wit={V3}]{tāṃve}} 
	\app{\lem[wit={N3,V19,E2,P11,V3}]{dhattaḥ}% dhataḥ N3, dharttaḥ P11
		\rdg[wit={J7}]{dattaḥ}% +K3,C7
		\rdg[wit={J5}]{dhanva}
		\rdg[wit={C6}]{vahataḥ}}
	\app{\lem[wit={N3,J5,J7,V19,E2,P11,V3}]{sakalaṃ}% ṃ om. V3
%		\rdg[wit={J5}]{śakalaṃ}
		\rdg[wit={C6}]{sarvaṃ}
		}} % nested!!
\pada{\app{\lem[nolem]{\skp{pāda b}}
	\rdg[wit={J10},alt={\om}]{\skp{\om}}}%
	\app{\lem[wit={J7,Gr3,G11,V15,P11,Jyo}]{kālaṃ}
		\rdg[wit={N3,J5,C6}]{kāla}
		\rdg[wit={N19}]{kālāṃ} % kālāṃśatri N19
		\rdg[wit={J10,V3},alt={\om}]{\skp{\om}}}
	\app{\lem[wit={G11,Jyo}]{rātriṃdivātmakam}
		\rdg[wit={N3,J5,J7,V15,P11,C6}]{rātridivātmakam}% °ka J5; +V6,F
		\rdg[wit={Gr3}]{rātrindinātmakaṃ}
		\rdg[wit={G4}]{rātriṃ divākaraṃ}
		\rdg[wit={V3}]{rātridivātmakaṃ yogavit}
		\rdg[wit={N19}]{°śa tridivātmakaṃ}
		}/}\\+}
\tl{
\pada{\app{\lem[nolem]{\skp{pāda c}}
	\rdg[wit={J10},alt={\om}]{\skp{\om}}}%
	\app{\lem[wit={N3,J7,Gr3,G11,V15,P11,Jyo}]{bhoktrī} % bho<<ktrī>> N3
		\rdg[wit={N19}]{bhoktī}
		\rdg[wit={V3}]{bhoktā}
		\rdg[wit={C6}]{bhoktṛ}
		\rdg[wit={J5}]{bhoktu}
		\rdg[wit={G4}]{[bho]gī}
		}
	suṣumṇā kālasya}
\pada{\app{\lem[nolem]{\skp{pāda d}}
	\rdg[wit={J10},alt={\om}]{\skp{\om}}}%
	\app{\lem[wit={ceteri},alt={guhyam etad}]{guhyam eta\skp{d}}
		\rdg[wit={V19}]{guptam etad}
		\rdg[wit={E2}]{sattvam etad}% +C7
		}%
	\app{\lem[wit={ceteri},alt={udāhṛtam}]{\skm{d }udāhṛtam}
	\rdg[wit={J5}]{udīritaṃ}}//\versenr}
	%\unavbl{N23}
	\orgvnr{60}\\!}
\end{tlg}
%</vs60>

\medskip
\avacite{59a}
\commciterange{59}{59}\teimute{\vspace{-2.5ex}}
\commciterange{60}{60}\newpage


%<*vs61a>
\begin{ava}[hp04_061a]
\app{\lem[nolem]{}
	\rdg[wit={N19,V15,J10,Jyo},alt={\om}]{\skp{\om}}
	%\rdg[wit={G11,G5},alt=\textapp{found after 3.93*7}]{\skp{found after 3.93*7}}
	}%
\app{\lem[wit={Gr1,V19,G5,C6,V3}]{tathā hi}% +K3,C7
	\rdg[wit={P11}]{tathāpi hi}
	\rdg[wit={J7,E2}]{tathā}
	\rdg[wit={G11}]{athā hi}}
\app{\lem[wit={N3,J5,Gr3}]{saubhadraṃ nāma}% rāma J5
	\rdg[wit={G11}]{sobhadrā nāma}
	\rdg[wit={J7}]{saubhadranāmā}
	\rdg[wit={G5}]{saubhadranāmaś ca}
	\rdg[wit={V3}]{saubhadreyaṃ nāma}
	\rdg[wit={C6}]{saubhadreyanāma}
	\rdg[wit={P11}]{saubhadreryān nāma}}
\app{\lem[wit={N3,Gr3,G11,G5,GrB}]{ślokacatuṣṭayam}% śloṣa P11
	\rdg[wit={J5}]{ślokam eva catuṣṭayaṃ}
	\rdg[wit={J7}]{ślokacatuṣṭayam āha}}/
%	\sgwit{N3,J5,J7,Gr3,GrB}
	\myfn{\getsiglum{G11,G5} have this set of verses as \manuref{3.93*7}ff. in a different order. Their readings are reported in the apparatus here.}
\end{ava}
%</vs61a>
%<*vs61>
\begin{tlg}[hp04_061]
\tl{\app{\lem[nolem]{}
	\rdg[wit={N23}]{\unavbl}
	\rdg[wit={N19,V15,J10,Jyo},alt={\om}]{\skp{\om}}
	%\rdg[wit={G11,G5},alt=\textapp{found after 3.93*7}]{\skp{found after 3.93*7}}
	}%
\pada{\app{\lem[wit={J5,J7,Gr3,G11,G5,GrB}]{ṣaṭcakraṃ}
	\rdg[wit={N3}]{ṣaḍraktaṃ}} ṣoḍaśādhāraṃ}
\pada{\app{\lem[wit={J7,Gr3,G11,G5,V3},alt={tridhā lakṣ(y)aṃ}]{tridhā lakṣyaṃ}% lakṣaṃ V3,J7,G11
		\rdg[wit={N3,J5}]{tridhā bhajyaṃ}% m. om. J5
%		\rdg[wit={C7}]{tridhā yuktaṃ}
		\rdg[wit={P11}]{tridhākṣa ca}
		\rdg[wit={C6}]{trilakṣyaṃ ca}} guṇatrayam/}\\+}
\tl{
\pada{\app{\lem[wit={N3,J5,G5,GrB}]{śeṣaṃ tu} % m vs n!
		\rdg[wit={J7,Gr3}]{śeṣas tu}% +F
		\rdg[wit={G11}]{śeṣaṃ tat}}
	\app{\lem[wit={ceteri}]{grantha}% ceteri = N3,J5,GrB,J7,V19,K3; gratha N3
		\rdg[wit={G11,C6}]{granthi}}% +C7
	\app{\lem[wit={N3,G11,G5,GrB}]{vistāraṃ}
		\rdg[wit={J5}]{vistāra}
		\rdg[wit={J7,Gr3}]{vistāras}% +F
		}}
\pada{\app{\lem[wit={N3,J5,J7,V19,G11,G5,P11,V3}]{trikūṭaṃ}% °kuṭaṃ G11
		\rdg[wit={C6}]{trikoṭi}
		\rdg[wit={E2}]{trirūpaṃ}% +K3,C7
		} paramaṃ padam//\versenr}
%	\sgwit{N3,J5,J7,Gr3,GrB}
%	\anm{\textrightarrow\ \manuref{3.93*7+2} \getsiglum{G11,G5}}
	%\unavbl{N23}
	\orgvnr{61}\\!}
\end{tlg}
%</vs61>

\avacite{61a}
\commcite\newpage


%<*vs62>
\begin{tlg}[hp04_062]
\tl{\app{\lem[nolem]{}
	\rdg[wit={N23}]{\unavbl}
	\rdg[wit={J7,N19,V15,J10,C6,Jyo},alt={\om}]{\skp{\om}}
	%\rdg[wit={G11,G5},alt=\textapp{found after 3.93*7}]{\skp{found after 3.93*7}}
	}%
\pada{kuṇḍalī kuṭilākārā} % kuṭulākāraṃ G4, kuḍalākārā J5
\pada{sarpavat parikīrtitā/}\\+} % sarvasthat G4, sarvavat G11
\tl{
\pada{sā śakti\app{\lem[wit={N3,J5,G11,G5,V3},alt={cālitā}]{\skm{ś }cālitā}
	\rdg[wit={P11}]{calitā}
	\rdg[wit={G4}]{cāri\,..}
	\rdg[wit={V19}]{kīlitā}% +K3
	\rdg[wit={E2}]{kelitā}% +C7
	} yena}
\pada{sa \app{\lem[wit={Gr3,G11,G5}]{mukto}% +G7 (eye-skip to the next verse?)
	\rdg[wit={N3,J5,P11,V3}]{yogī}} % +F ##
	nātra saṃśayaḥ//\versenr} % śaṃsayaḥ V19
%	\sgwit{Gr1,Gr3,P11,V3} 
%	\anm{\textrightarrow\ \manuref{3.93*7} \getsiglum{G11,G5}}
%	\unavbl{N23}
	\orgvnr{62}\\!} % +B2
\end{tlg}
%</vs62>
\commcite%\newpage
% V3 has a gap indicated before the following verse.


%<*vs63>
\begin{tlg}[hp04_063]
\tl{\app{\lem[nolem]{}
	\rdg[wit={N23}]{\unavbl}
	\rdg[wit={J7,Gr3,N19,V15,J10,Jyo},alt={\om}]{\skp{\om}}
	\rdg[wit={G11,G5},alt=\textapp{found after 3.93*7}]{\skp{found after 3.93*7}}}%
\pada{\app{\lem[wit={ceteri}]{yadā}\rdg[wit={G5}]{yathā}} 
	\app{\lem[wit={ceteri},alt={kūṭaṃ tri°}]{kūṭaṃ tri\skp{°}}
		\rdg[wit={C6}]{kūṭasti}}kūṭasthaṃ} % 
\pada{cittaṃ  % citta V3
	\app{\lem[wit={N3}]{citraṃ}
		\rdg[wit={J5}]{cittaṃ}
		\rdg[wit={GrB}]{tatra}% +F
		\rdg[wit={G11,G5}]{yatra}
		} 
	\app{\lem[wit={ceteri}]{nirantaram}
		\rdg[wit={G11,G5}]{nirajñanaṃ}}/}\\+}
\tl{
\pada{\app{\lem[wit={ceteri}]{kuṇḍalyās tu} % kuṃḍilyās P11
		\rdg[wit={G11}]{kuṇḍalyāpta}
		\rdg[wit={G5}]{kuṇḍalinyāḥ}}
	\app{\lem[wit={N3,J5,G11,G5,P11,V3},post=\texteng{(°na˟ \getsiglum{N3})}]{prayogeṇa}%  + + geṇa G4, °na G11
		\rdg[wit={C6}]{prabodhena}}}
\pada{sa mukto nātra saṃśayaḥ//\versenr} 
%	\anm{\textrightarrow\ \manuref{3.93*7+3} \getsiglum{G11,G5}}
%	\unavbl{N23}
	\orgvnr{63}\\!}
%	\sgwit{Gr1,GrB}\\!} % not in B2
\end{tlg}
%</vs63>
\commcite\newpage


%<*vs64>
\begin{tlg}[hp04_064]
\tl{\app{\lem[nolem]{}
	\rdg[wit={N23}]{\unavbl}}%
\pada{\app{\lem[wit={N3,J5,J7,Gr3,GrB,Jyo}]{dvāsaptatisahasrāṇi}
		\rdg[wit={G4,G11,G5,N19,V15},alt={dvisaptati°}]{dvisaptatisahasrāṇi}% +F
		\rdg[wit={J10},alt={\om}]{\skp{\om}}}} % °śrāṇi V3,V19
\pada{\app{\lem[wit={Gr1,J7,G11,G5,V15,GrB,Jyo},post=\texteng{(nāḍi° \getsiglum{J5,P11})}]{nāḍīdvārāṇi}% nāḍi J5,P11
		\rdg[wit={N19}]{nāḍīdvāre ca}
		\rdg[wit={E2}]{nāḍīnāṃ deha}% +K3,C7
		\rdg[wit={V19}]{nāḍīnāṃdeda}
		\rdg[wit={J10}]{datvā kārāpi}}\marmas
	\app{\lem[wit={ceteri}]{pañjare}
		\rdg[wit={N3}]{paṃkaje}
		\rdg[wit={G4}]{maṃjarī}}/}\\+}
\tl{
\pada{suṣumṇā śāmbhavī śaktiḥ} % sāṃbhavī J10; śakti V19,V3
\pada{\app{\lem[wit={N3,E2,G11,G5,N19,GrB,Jyo}]{śeṣās tv eva}% aiva P11; +C7
		\rdg[wit={J10}]{śeṣās tv evaṃ}
		\rdg[wit={J5}]{śeṣāsvevaṃ}
		\rdg[wit={J7,V19,V15}]{śeṣāś caiva}}
	\app{\lem[wit={ceteri}]{nirarthakāḥ} % nina° J10; °kā J5,V3
		\rdg[wit={N19}]{nivarttakāḥ}
		}//\versenr}
%	\myfn{\getsiglum{G11} has this verse in both Ch. 3 and 4.}
	%\anm{= \manuref{3.93*7+4} \getsiglum{G11,G5} (\getsiglum{G11} has this verse in both Ch. 3 and 4.)}
	%\unavbl{N23}
	\orgvnr{64}\\!}
\end{tlg}
%</vs64>
\commcite%\newpage


%<*vs65>
\begin{tlg}[hp04_065]
\tl{\app{\lem[nolem]{}
	\rdg[wit={N23}]{\unavbl}
	\rdg[wit={E2},alt={\om}]{\skp{\om}}}%
\pada{vāyuḥ % vāyu P11,V19
	\app{\lem[wit={N3,J5,G11,N19,V15,J10,C6,Jyo}]{paricito}
		\rdg[wit={V3}]{paricipta}
		\rdg[wit={J7}]{sa parito}
		\rdg[wit={V19}]{saṃparito}% +C7
		\rdg[wit={P11}]{parivṛtto}}
	\app{\lem[wit={N3,J7,V19,G11,N19,V15,P11,C6},alt={yatnād}]{yatnā\skp{d}}% yannād P11
%		\rdg[wit={C7}]{yadvad}
		\rdg[wit={J5,J10,Jyo}]{yasmād}% +F
		\rdg[wit={V3}]{nādād}}}%
\pada{\app{\lem[wit={V19,G11,N19,V15,J10,GrB,Jyo},alt={agninā}]{\skm{d }agninā} % agnidā N19
		\rdg[wit={J7}]{ṛgvinā}
		\rdg[wit={N3}]{yaṣṭinā}
		\rdg[wit={J5}]{yadasthā}} saha
	\app{\lem[wit={G11,Jyo}]{kuṇḍalīm}% +C7
		\rdg[wit={N3,J5,J7,V19,N19,V15,J10,GrB}]{kuṇḍalī}}/}\\+}
\tl{
\pada{\app{\lem[nolem]{\skp{pāda c}}
	\rdg[wit={J10},alt={\om}]{\skp{\om}}}%
	bodhayitvā suṣumṇāyāṃ} % °yā N3
\pada{\app{\lem[nolem]{\skp{pāda d}}
	\rdg[wit={J10},alt={\om}]{\skp{\om}}}%
\app{\lem[wit={ceteri},alt={praviśed}]{praviśe\skp{d}}% °viśad J5
		\rdg[wit={V3}]{praveśad}
		\rdg[wit={J10},alt={\om}]{\skp{\om}}}%
	\app{\lem[wit={G4,G11,V15,GrB,Jyo},alt={anirodhataḥ}]{\skm{d }anirodhataḥ} % °ta P11,V3; +M1
		\rdg[wit={N3,J5,J7,V19}]{avirodhataḥ} % aviśeṣataḥ V19ac; +C7
		\rdg[wit={N19}]{atirodhataḥ}
		\rdg[wit={J10},alt={\om}]{\skp{\om}}}//\versenr}
	\label{bodhayitva}
	%\unavbl{N23}
	\orgvnr{65}\\!}
\end{tlg}
%</vs65>
\commcite\newpage


%<*vs66>
\begin{tlg}[hp04_066]
\tl{\app{\lem[nolem]{}
	\rdg[wit={N23}]{\unavbl}}%
\pada{\app{\lem[nolem]{\skp{pāda a}}
	\rdg[wit={J10},alt={\om}]{\skp{\om}}}%
suṣumṇā\app{\lem[wit={G4,J7,E2,G11,C6,V3,Jyo}]{vāhini}
		\rdg[wit={N3,J5,N19,V15,P11}]{vāhinī}
		\rdg[wit={V19}]{hini}
		} 
		prāṇe} % prāṇo J5
\pada{\app{\lem[nolem]{\skp{pāda b}}
	\rdg[wit={J10},alt={\om}]{\skp{\om}}}%
\app{\lem[wit={G4,J7,Gr3,G11,V15,C6,V3,Jyo}]{sidhyaty eva}% °teva G11
		\rdg[wit={N3}]{siddhyety eva}
%		\rdg[wit={C7}]{siddhyatīva}
		\rdg[wit={N19,P11}]{siddhaty eva}
		\rdg[wit={J5}]{siddhity eva}
		} 
	manonmanī/}
	\\+}
\tl{
\pada{\app{\lem[wit={Gr1,J7,GrB}]{anyathā vividhā}% anātha J5
%		\rdg[wit={C7}]{anye ca vividhā}
		\rdg[wit={Gr3}]{anye ye vividhā}
		\rdg[wit={N19,V15}]{anyathā tv itare}% +G5
		\rdg[wit={Jyo}]{anyathā tv itarā}
		\rdg[wit={J10}]{atha cittāntare}
		\rdg[wit={G11}]{prāṇe suṣumnāṃ saṃ°}}%
	\app{\lem[wit={N3,E2,C6,Jyo}]{bhyāsāḥ}% bhyāsā<<ḥ>> C7
		\rdg[wit={G4,J7,V19,V3}]{bhyāsā}
		\rdg[wit={J5,N19,P11}]{bhyāsāt}% +G5
		\rdg[wit={V15,J10}]{bhyāsa}
		\rdg[wit={G11}]{°prāpte}}}
\pada{\app{\lem[wit={N3,J5,J7,G11,GrB,Jyo}]{prayāsāyaiva}% +C7
		\rdg[wit={E2}]{prayāsāyai}
		\rdg[wit={V19}]{prāyāsāś caiva}
		\rdg[wit={V15}]{prayāsā eva}
		\rdg[wit={N19}]{prayāsā eka}
		\rdg[wit={J10}]{pratyāśā jīva}}
	\app{\lem[wit={ceteri}]{yoginām}
		\rdg[wit={J5,J10,V3}]{yoginā}
		\rdg[wit={N19}]{yoginī}}//\versenr}
	%\unavbl{N23}
	\orgvnr{66}\\!}
\end{tlg}
%</vs66>
\commcite%\newpage


%<*vs67>
\begin{tlg}[hp04_067]
\tl{\app{\lem[nolem]{}
	\rdg[wit={N23}]{\unavbl}}%
\pada{pavano badhyate 
	\app{\lem[wit={ceteri}]{yena}\rdg[wit={J5}]{deva}}}
\pada{\app{\lem[wit={ceteri}]{manas tenaiva badhyate} % <va>dhyate N19
		\rdg[wit={J10}]{tenaiva badhyate manaḥ}}/}\\+}
\tl{
\pada{\app{\lem[nolem]{\skp{pāda c}}
	\rdg[wit={J5,J7,J10},alt={\om}]{\skp{\om}}}%
\app{\lem[wit={N3,G11,N19,V15,P11,V3,Jyo}]{manaś ca}
		\rdg[wit={Gr3}]{manas tu}% +F
		\rdg[wit={C6}]{manas tad}
		} 
		badhyate yena}
\pada{\app{\lem[nolem]{\skp{pāda d}}
	\rdg[wit={J5,J7,J10},alt={\om}]{\skp{\om}}}%
\app{\lem[wit={ceteri}]{pavanas tena} % <pa>vanas V19
		\rdg[wit={V3}]{pavanamana}
		} 
		badhyate//\versenr}
	%\unavbl{N23}
	\orgvnr{67}\\!}
\end{tlg}
%</vs67>
\commcite\newpage

%<*vs68>
\begin{tlg}[hp04_068]
\tl{\app{\lem[nolem]{}
	\rdg[wit={N23}]{\unavbl}
	\rdg[wit={V19},alt=\textapp{found after \ref{dugdha}}]{\skp{found after \manuref{4.70}}}}%
\pada{\app{\lem[wit={ceteri}]{hetu}
%		\rdg[wit={C7}]{deha}
		\rdg[wit={J5}]{heta}
		\rdg[wit={G4}]{eta}}%
	\app{\lem[wit={N3,G4,E2,J10,Jyo}]{dvayaṃ tu}% +C7
		\rdg[wit={J7,G11,P11,V3}]{dvayaṃ hi}
		\rdg[wit={V19,C6}]{dvayaṃ ca}
		\rdg[wit={N19,V15}]{dvayasya}
		\rdg[wit={J5}]{dvāv api}}
	\app{\lem[wit={ceteri}]{cittasya}
		\rdg[wit={J7,Gr3}]{manaso}}}
\pada{vāsanā ca samīraṇaḥ/}\\+} % ḥ om. V3; vāsanācca samīraṇa<<ḥ>> N3, vāsanaṃ ca samīraṇaṃ J5
\tl{
\pada{tayo\app{\lem[wit={ceteri},alt={vinaṣṭa ekasmin}]{\skm{r }vinaṣṭa ekasmi}% tayo J5,J7; ekasminn J7; ekasmi V19
		\rdg[wit={G11}]{vinaṣṭa etasmin}
		\rdg[wit={C6}]{vinaṣṭas tv ekaś ca hy}}}%n 
\pada{\app{\lem[wit={N3},alt={drutaṃ dvāv api naśyataḥ},post=\texteng{(druttaṃ)}]{\skm{n }drutaṃ dvāv api naśyataḥ}% +M3
%		\rdg[wit={N3}]{druttaṃ dvāv api naśyataḥ}
		\rdg[wit={G4}]{dhṛtaṃ dvāv api naśyataḥ}
		\rdg[wit={J5}]{dṛtaṃ vāvati nasyataḥ}
		\rdg[wit={G11}]{nṛtaṃ dvāv api naśyati}
		\rdg[wit={N19,V15,P11,V3,Jyo}]{tau dvāv api vinaśyataḥ} % °ta P11,V3, °tiḥ N19
		\rdg[wit={J7,E2,J10,C6}]{ubhāv api vinaśyataḥ}
		\rdg[wit={V19}]{svabhāvo pi vinaśyataḥ}}//\versenr}%
	%\unavbl{N23}
	\orgvnr{68}\\!}
\end{tlg}
%</vs68>
\commcite\newpage


%<*vs69>
\begin{tlg}[hp04_069]
\tl{\app{\lem[nolem]{}
	\rdg[wit={N23}]{\unavbl}
	\rdg[wit={V19},alt=\textapp{found after \ref{dugdha} together with the previous verse}]{\skp{found after \manuref{4.70} together with the previous verse}}}%
\pada{\app{\lem[nolem]{\skp{pāda a}}
	\rdg[wit={J10},alt={\om}]{\skp{\om}}}%
mano yatra \app{\lem[wit={ceteri}]{vilīyeta} % °līyena N19
		\rdg[wit={V3}]{vilīyate}
		}}
\pada{\app{\lem[nolem]{\skp{pāda b}}
	\rdg[wit={J10},alt={\om}]{\skp{\om}}}%
\app{\lem[wit={ceteri},alt={pavanas}]{pavana\skp{s}}
		\rdg[wit={G11,N19,V15}]{mārutas}
		}s tatra līyate%
\app{\lem[alt={\post līyate \add},nosep]{\skp{\post līyate \add}}
	\rdg[wit={V15}]{ekatra[m]iśritau}}/}\\+}
\tl{
\pada{\app{\lem[nolem]{\skp{pāda c}}
	\rdg[wit={J5,N19,V15},alt={\om}]{\skp{\om}}}%
\app{\lem[wit={N3,J7,C6,Jyo}]{pavano līyate yatra}
		\rdg[wit={Gr3}]{pavano yatra līyeta}
		\rdg[wit={P11,V3}]{pavano yatra līyate}% līyaṃte P11
		\rdg[wit={G11}]{māruto yatra līyeta}
		\rdg[wit={J10}]{yatraiva līyate vāyur}
		%\rdg[wit={J5,N19,V15},alt={\om}]{\skp{\om}}
		}}
\pada{\app{\lem[nolem]{\skp{pāda d}}
	\rdg[wit={J5,N19,V15},alt={\om}]{\skp{\om}}}%
mana\app{\lem[wit={N3,Gr3,G11,J10,GrB},alt={tatraiva līyate}]{\skm{s }tatraiva līyate}% +N2
		\rdg[wit={J7,Jyo}]{tatra vilīyate}% +V6
		%\rdg[wit={J5,N19,V15},alt={\om}]{\skp{\om}}
		}//\versenr}
	%\unavbl{N23}
	\orgvnr{69}\\!}
\end{tlg}
%</vs69>
\commcite%\newpage


%<*vs70>
\begin{tlg}[hp04_070]
\tl{\app{\lem[nolem]{}
	\rdg[wit={N23}]{\unavbl}}%
\pada{dugdhāmbuvat saṃmilitau % dugdhābu° N3; saṃmilaṃ J5, <<li>> C6, °tāv J10,Jyo, °to N19
	\app{\lem[wit={N3,J5,G11,N19,V15,GrB}]{sadaiva}
		\rdg[wit={G4}]{sadeva}
		\rdg[wit={J7,Gr3}]{tathaiva}
		\rdg[wit={J10,Jyo}]{ubhau tau}}}\\+}
\tl{
\pada{tulyakriyau % kriyo N19, tulyasya kriyā J5, tvalya C6
	\app{\lem[wit={ceteri}]{mānasamārutau} % māruto N19; 2 x mānasa J10
		\rdg[wit={G11,P11,C6}]{mārutamānasau}
		\rdg[wit={V3},alt={\illeg}]{\skp{\illeg}}}
	\app{\lem[wit={N3,G4,G11,N19,V15,J10,P11,Jyo}]{hi}
		\rdg[wit={J5,J7,Gr3,C6}]{ca}
		\rdg[wit={V3},alt={\illeg}]{\skp{\illeg}}
		}/}\\+}
\tl{
\pada{\app{\lem[wit={ceteri},alt={yāvan manas}]{yāvan mana\skp{s}}
		\rdg[wit={J10,Jyo}]{yato marut}}%
	\app{\lem[wit={ceteri},alt={tatra}]{\skm{s }tatra}
		\rdg[wit={J5}]{caiva}}
	\app{\lem[wit={ceteri},alt={marut}]{maru\skp{t}} % murut G11, marat V15
		\rdg[wit={J10,Jyo}]{manaḥ}
		\rdg[wit={C6}]{\_\,sat}}%
	\app{\lem[wit={ceteri},alt={pravṛttir}]{\skm{t}pravṛtti\skp{r}} % vṛtti J7
		\rdg[wit={C6}]{pravṛtta}% varti P7
		\rdg[wit={N19}]{pravṛddhitti}}-}\\+}
\tl{
\pada{\app{\lem[nolem]{}
	\rdg[wit={N19,V15},alt={\om}]{\skp{\om}}}%
	\app{\lem[wit={Gr1,J7,Gr3,G11,GrB},alt={yāvan}]{\skm{r }yāva\skp{n}}
		\rdg[wit={J10,Jyo}]{yato}
		}%
	\app{\lem[wit={N3,J5,J7,Gr3,G11,P11,C6},alt={maruc cāpi}]{\skm{n }maruc cāpi}
		\rdg[wit={V3}]{marut tatra}
		\rdg[wit={J10,Jyo}]{manas tatra}
		}
	\app{\lem[wit={N3pc,J7,E2,G11,C6,V3}]{manaḥ}
		\rdg[wit={N3ac,J5,V19,P11}]{mana}
		\rdg[wit={J10,Jyo}]{marut}% or: marun
		}%
	\app{\lem[wit={N3,J7,Gr3,G11,P11,V3,Jyo}]{pravṛttiḥ} % °vartiḥ P7
		\rdg[wit={C6}]{pravṛttaḥ}
		\rdg[wit={J5}]{pravittato}
		\rdg[wit={J10}]{nivṛttiḥ}
		}//\versenr}\label{dugdha}
	%\unavbl{N23}
	\orgvnr{70}\\!}
\end{tlg}
%</vs70>
\commcite\newpage


%<*vs71>
\begin{tlg}[hp04_071]
\tl{\app{\lem[nolem]{}
	\rdg[wit={N23}]{\unavbl}
	\rdg[wit={V19},alt=\textapp{ab and cd are transposed}]{\skp{ab and cd are transposed}}}%
\pada{\app{\lem[wit={ceteri}]{tatraika}% tatr<<aika>> N3
		\rdg[wit={N3ac}]{tatra}
		\rdg[wit={N19,V15}]{atraika}
		\rdg[wit={J10}]{ekasya}}nāśād aparasya % nā<śā>d J5,N19, nāsād J10
	\app{\lem[wit={N3,J5,J7,E2,N19,V15,J10,C6,Jyo},alt={nāśa(ḥ)}]{nāśa\skp{(ḥ)}}
		\rdg[wit={V3}]{nāśo}
		\rdg[wit={P11}]{nāśe}
		\rdg[wit={G11}]{nāśā}
		\rdg[wit={V19}]{nāśam}
%		\rdg[wit={J5,N19,V15}]{nāśaḥ}
%		\rdg[wit={J10}]{nāśas}
		}}\\+}
\tl{
\pada{\app{\lem[wit={N3,J5,J7,N19,P11,Jyo},alt={ekapravṛtter}]{ekapravṛtte\skp{r}}
		\rdg[wit={C6}]{ekapravṛtte}
		\rdg[wit={Gr3,G11,V15}]{ekapravṛttāv}
		\rdg[wit={V3}]{e\,..\,..\,..\,..}
		\rdg[wit={J10}]{tatraikavṛtter}}%
	\app{\lem[wit={ceteri},alt={aparapravṛttiḥ}]{\skm{r }aparapravṛttiḥ} % °ttir N19, °tt<<i>>ḥ N3; +G5
		\rdg[wit={C6}]{ca parapravṛttiḥ}
		\rdg[wit={J10}]{aparasya vṛttiḥ}
		\rdg[wit={G11}]{itarapravṛttiḥ}
		\rdg[wit={V3}]{..\,..\,..\,..\,..\,ttiḥ}}%
	\app{\lem[alt={\post pravṛttiḥ \add},nosep]{\skp{\post pravṛttiḥ \add}}
	\rdg[wit={V15},post=\texteng{(alternative reading for pāda a)}]{ekasya nā<śā>d aparasya nāśaḥ}}/}%
	\\+}
\tl{
\pada{\app{\lem[wit={N3,P11,Jyo},alt={adhvastayoś}]{adhvastayo\skp{ś}}
		\rdg[wit={J10,C6}]{adhastayoś}
		\rdg[wit={E2,V15}]{adhvastayor}
		\rdg[wit={J7}]{adhyastayor}
		\rdg[wit={V19}]{adhastayor}
		\rdg[wit={G11}]{adhvaścayoś}
		\rdg[wit={N19}]{addhastayoś}
		\rdg[wit={V3}]{atastayoś}
		\rdg[wit={J5}]{adhastasar}
		}%
	\app{\lem[wit={Gr1,G11,N19,J10,GrB,Jyo},alt={cendriya}]{\skm{ś }cendriya}% veddriya J5, caiṃdriya P11, cenniya G11, caidriya N19
		\rdg[wit={J7,Gr3,V15}]{indriya}}varga% 
	\app{\lem[wit={N3,G4},alt={buddhir}]{buddhi\skp{r}}
		\rdg[wit={V3}]{vudhir}
		\rdg[wit={J7,E2}]{vṛddhir}% +C7
		\rdg[wit={V19,G11,N19,V15,J10,Jyo}]{vṛttiḥ}% vṛttir V19, vṛtti V15
		\rdg[wit={P11}]{baṃdhir}
		\rdg[wit={J5,C6}]{śuddhir}
		}\marma-}\\+}
\tl{
\pada{\app{\lem[wit={N3,G4,Gr3,V15,GrB},alt={vidhvastayor}]{\skm{r }vidhvastayo\skp{r}}
		\rdg[wit={J5}]{adhastayor}
		\rdg[wit={J7}]{vivṛddhayor}% °yo J7
		\rdg[wit={G11}]{nidhvastayo}
		\rdg[wit={N19}]{addhvastayor}% +G5
		\rdg[wit={J10}]{vijñātayor}
		% vṛttiḥ|(gap for about one hemistich)raddhvasta° N19
		\rdg[wit={Jyo}]{pradhvastayor}}%
	\app{\lem[wit={ceteri},alt={mokṣapadasya}]{\skm{r }mokṣapadasya}
		\rdg[wit={J7},alt={°pradasya}]{mokṣapradasya}
%		\rdg[wit={C7},alt={°pathasya}]{mokṣapathasya}
		}
		siddhiḥ//\versenr}\label{tatraika}
	%\unavbl{N23}
	\orgvnr{71}\\!} % siddhi J5,P11,V19
\end{tlg}
%</vs71>
\commcite\newpage


%====== Passage B and yāvanna-stanza collated with more mss (V19,N19,J10)

%<*vs72>
\begin{tlg}[hp04_072]
%\Anm{This verse appears after 4.0*16 in \getsiglum{N19,V15,J10}}\\+}
\tl{\app{\lem[nolem]{}
	\rdg[wit={N23}]{\unavbl}
	\rdg[wit={Jyo},alt={\om}]{\skp{\om}}
	%\rdg[wit={G11,N19,V15,J10},alt=\textapp{found after \manuref{4.0*16}}]{\skp{found after xxx}}
	}%
	\pada{\app{\lem[wit={ceteri}]{vāyu}
		\rdg[wit={V19,V15}]{vāyur}}%
	\app{\lem[wit={G11}]{mārge tv asaṃcāre}
		\rdg[wit={V15}]{mārge py asaṃcāre}
		\rdg[wit={N19}]{mārge tha saṃcāre}
		\rdg[wit={J10}]{mārge ca saṃcāre}
		\rdg[wit={Gr1,J7,GrB}]{mārgeṇa saṃcāre}
		\rdg[wit={Gr3}]{mārgeṇa saṃcārī}
		}}
\pada{\app{\lem[wit={N3,J7,Gr3,V3}]{sakalāṃ}
		\rdg[wit={G4}]{sakalā}
		\rdg[wit={J5,G11,N19,V15,C6}]{sakalaṃ}
		\rdg[wit={J10}]{sa phalaṃ}
		\rdg[wit={P11}]{saṃkalpāt}}
	\app{\lem[wit={Gr1,G11,V15,J10,P11}]{labhate}
		\rdg[wit={N19,C6}]{labhyate}
		\rdg[wit={J7,Gr3}]{bhramate}
		\rdg[wit={V3}]{carate}}\marmas
	\app{\lem[wit={N3,G4,J7,Gr3,G11,P11}]{mahīm}
		\rdg[wit={C6,V3}]{mahī}
		\rdg[wit={J5}]{mahiḥ}
		\rdg[wit={N19,V15}]{mahaḥ}
		\rdg[wit={J10}]{mahān}}/}\\+}
\tl{
\pada{\app{\lem[wit={Gr1,Gr3,G11},post=\texteng{(tathā<<ṣṭa>> \getsiglum{N3})}]{tathāṣṭa}
		\rdg[wit={P11}]{aṣṭadhā}
		\rdg[wit={C6,V3}]{athāṣṭa}
		\rdg[wit={N19,V15,J10}]{tato'ṣṭa}
		\rdg[wit={J7}]{na tathā}
		}guṇam aiśvaryaṃ} % ṃ om. J7
\pada{\app{\lem[wit={N3,G4,J7,Gr3,GrB}]{satyaṃ satyaṃ varānane}
		\rdg[wit={G11,N19,V15,J10}]{satyam ity āha śaṃkaraḥ}
		\rdg[wit={J5}]{labhate sakalān varān}}//\versenr}
	\label{vayumargena}
	%\unavbl{N23}
	\orgvnr{72}\\!}
\end{tlg}
%</vs72>
\commcite\newpage


%<*vs73a>
\begin{ava}[hp04_073a]
\app{\lem[nolem]{}
	\rdg[wit={G11,N19,V15,J10,V3,Jyo},alt={\om}]{\skp{\om}}}%
\app{\lem[wit={N3,P11,C6}]{tathā}
	\rdg[wit={J5}]{tathā ca}
	\rdg[wit={G4}]{tathāha}
	\rdg[wit={J7,Gr3},alt={\om}]{\skp{\om}}} 
	viśvarūpācāryaḥ/ % °yāḥ J7
%	\NotIn{G11,N19,V15,J10,V3,Jyo}
%	\sgwit{Gr1,J7,Gr3,P11,C6}
\end{ava}
%</vs73a>
%<*vs73>
\begin{tlg}[hp04_073]
\tl{\app{\lem[nolem]{}
	\rdg[wit={N23}]{\unavbl}
		\rdg[wit={N19,V15,J10,V3},alt={\om}]{\skp{\om}}
		%\rdg[wit={G11,Jyo},alt=\textapp{found after \ref{salile}}]{\skp{found after \manuref{4.0*3}}}
		}%
	\pada{\app{\lem[wit={Gr1,Gr3,C6,Jyo}]{yadā saṃkṣīyate}
		\rdg[wit={J7,P11}]{yadā sa kṣīyate}
		\rdg[wit={G11},alt={\om}]{\skp{\om}}} 
		prāṇo} % =P7; prāṇaṃ J5, prāṇe C6
\pada{mānasaṃ \app{\lem[wit={Gr1,G11,P11,C6}]{ca vilīyate}% vīlī° J5; +C7
		\rdg[wit={J7,Jyo}]{ca pralīyate}
		\rdg[wit={V19}]{pravilīyate}
		\rdg[wit={E2}]{saṃpralīyate}}/}\\+}
\tl{
\pada{\app{\lem[wit={ceteri}]{tadā}
		\rdg[wit={G11}]{tayoḥ}} % tadvā N3ac
	\app{\lem[wit={ceteri}]{samarasatvaṃ}
		\rdg[wit={J5},post={\unm}]{samarasaikatvaṃ}} % = VM6! J5 is hypermetrical, since it has yat too.
	\app{\lem[wit={N3,J5,J7,E2,G11,C6},alt={yat}]{ya\skp{t}}% +F
		\rdg[wit={G4,V19}]{yaḥ}
%		\rdg[wit={C7}]{hi}
		\rdg[wit={P11,Jyo}]{ca}}}%
\pada{\app{\lem[wit={N3,G4,J7,V19,G11,C6},alt={samādhiḥ so'bhidhīyate}]{\skm{t }samādhiḥ so'bhidhīyate} % samādhi G4,V19
		\rdg[wit={P11}]{samādhī sau bhidhīyate}
		\rdg[wit={E2}]{samādhiḥ sābhidhīyate}
		\rdg[wit={Jyo}]{samādhir abhidhīyate}
		\rdg[wit={J5}]{samādhiś ca vilīyate}}//\versenr}
		\label{visvarupa} 
	%\unavbl{N23}
	\orgvnr{73}\\!}
\end{tlg}
%</vs73>

\avacite{73a}
\commcite\newpage


%<*vs74>
\begin{tlg}[hp04_074]
%\Anm{This verse appears before 4.13 in \getsiglum{G11,N19,V15,J10,Jyo}}
\tl{\app{\lem[nolem]{}
	\rdg[wit={N23}]{\unavbl}
	\rdg[wit={V3},alt={\om}]{\skp{\om}}
	%\rdg[wit={G11,N19,V15,J10,Jyo},alt=\textapp{found after \manuref{4.0*16}}]{\skp{found after \manuref{4.0*16}}}
	}%
\pada{\app{\lem[wit={N3pc,J7,Gr3,C6,Jyo}]{manaḥ}
		\rdg[wit={N3ac,J5,G4,G11,N19,V15,J10,P11}]{mana}}%
\app{\lem[wit={N3,J5,J7,G11,N19,J10,P11,C6,Jyo}]{sthairye}
		\rdg[wit={G4,V19}]{sthairya}
		\rdg[wit={E2}]{sthairyāt}
		\rdg[wit={V15}]{sthairyaḥ}}
	\app{\lem[wit={ceteri}]{sthiro}
		\rdg[wit={G4,G11,V15}]{sthito}} vāyus} % vāyaḥ G11
\pada{tato \app{\lem[wit={N3pc,G4,J7,E2,V15,Jyo}]{binduḥ}
		\rdg[wit={N3ac,J5,V19,G11,N19,J10,P11,C6}]{bindu}}
	\app{\lem[wit={ceteri}]{sthiro bhavet}
%		\rdg[wit={C7}]{sthito bhavet}
		\rdg[wit={G4}]{tato layaḥ}}/}\\+}
\tl{
\pada{\app{\lem[wit={ceteri}]{bindu}
		\rdg[wit={J7}]{binduḥ}}%
	\app{\lem[wit={N3,E2,P11,C6},alt={sthairyodayāt}]{sthairyodayā\skp{t}}% +C7
		\rdg[wit={G11}]{sthairyoyadāt}
		\rdg[wit={G4,N19}]{sthairyodayā}
		\rdg[wit={V15}]{sthairye dayā}
		\rdg[wit={J10}]{sthairyād dayā}
		\rdg[wit={J7}]{sthairyād athā}
		\rdg[wit={V19}]{sthairyād yathā}
		\rdg[wit={Jyo}]{sthairyāt sadā}
		\rdg[wit={J5}]{sthairyo sthiro}}%
	\app{\lem[wit={N3,P11},alt={putra}]{\skm{t }putra}
		\rdg[wit={C6}]{mūtra}
		\rdg[wit={G4}]{tatra}
		\rdg[wit={J7}]{panna}% pannaṃ V6,YCM; panna J7,N2
		\rdg[wit={G11plus}]{samyak}% +G5,M3
		\rdg[wit={E2,N19,V15}]{satyaṃ}% +Ten,C7,G3
		\rdg[wit={J10,Jyo}]{satvaṃ}
		\rdg[wit={J5}]{vāyu}
		\rdg[wit={V19},alt={\lacuna}]{\_\,\_\,\_}}}
\pada{piṇḍasthairyaṃ prajāyate//\versenr} % ddhaḍa J5; sthairya P11
	\label{manahsthairye}
	%\unavbl{N23}
	\orgvnr{74}\\!}
\end{tlg}
%</vs74>
\commcite\newpage


%%%%%%%%%%%%%%%%%%%%%%%
%<*vs75>
\begin{tlg}[hp04_075]
\tl{\app{\lem[nolem]{}
	\rdg[wit={N23}]{\unavbl}
	\rdg[wit={N19,Jyo},alt={\om}]{\skp{\om}}}%
\pada{dṛṣṭiḥ sthirā yasya % dṛṣṭi N3ac,J5,G4,P11,V19,V3,J10
	\app{\lem[wit={ceteri}]{vinaiva}% Gr1,G11,V15,J10,GrB
%		\rdg[wit={C7}]{vinā ca}
		\rdg[wit={J7,Gr3}]{vināpi}}
	\app{\lem[wit={N3,G4,V15,GrB},alt={dṛśyād}]{dṛśyā\skp{d}}% +G5
		\rdg[wit={J7,Gr3,G11,J10}]{dṛśyaṃ}
		\rdg[wit={J5}]{dṛśyavān}}-}\\+}
\tl{d
\pada{vāyuḥ sthiro yasya % vāyu N3ac,J5,G4,P11,J7,V19,G11,V15,J10
	\app{\lem[wit={ceteri}]{vinā prayatnāt} % prayatnataḥ J10ac
		\rdg[wit={J7}]{vināpi yatnaṃ}}/}\\+}
\tl{
\pada{cittaṃ sthiraṃ yasya
	\app{\lem[wit={N3pc,G4,G11,V15,C6,V3},alt={vināvalambāt}]{vināvalambā\skp{t}}% °laṃbā C6
		\rdg[wit={N3ac}]{vināvalambanāt}
		\rdg[wit={J5,Gr3}]{vināvalaṃbanaṃ}% +N2
		\rdg[wit={J10}]{vināvalaṃnaṃ}
		\rdg[wit={P11}]{vinā vilambāt}
%		\rdg[wit={C7}]{vinā balaṃ ca}
		\rdg[wit={J7}]{vinā prayatnāt}}-}\\+}
\tl{t
\pada{sa eva yogī % yeva J5,P11
	\app{\lem[wit={ceteri}]{sa guruḥ}
		\rdg[wit={J10}]{sadguruḥ}}
	\app{\lem[wit={ceteri}]{sa sevyaḥ} % sevya P11
		\rdg[wit={J7,V19}]{sa śiṣyaḥ}}//\versenr}
	\label{drsti}
	%\unavbl{N23}
	\orgvnr{75}\\!}
\end{tlg}
%</vs75>
\commcite\newpage

%%%%%%%%%%

%<*vs76>
\begin{tlg}[hp04_076]
\tl{\app{\lem[nolem]{}
	\rdg[wit={N23}]{\unavbl}
	\rdg[wit={N19,Jyo},alt={\om}]{\skp{\om}}
	%\rdg[wit={G11,V15,J10},alt=\textapp{found before \ref{nasuptam}}]{\skp{found before \manuref{4.35*7}}}
	}%
\pada{praveśe nirgame % praveśa G4; prathame niyame J5
	\app{\lem[wit={ceteri}]{vāme}
		\rdg[wit={G4}]{vāma}
		\rdg[wit={P11}]{vāpi}
		\rdg[wit={V15}]{cāpi}}}
\pada{dakṣiṇe \app{\lem[wit={Gr1,G11,P11}]{cordhvam apy adhaḥ}% adha G4
		\rdg[wit={C6}]{cordhvage'py adhaḥ}
%		\rdg[wit={C7}]{cordhvamadhyamaḥ}
		\rdg[wit={J7,Gr3}]{cordhvamadhyagaḥ} % cordha V19
		\rdg[wit={V15,J10}]{cordhvamadhyataḥ}
		\rdg[wit={V3}]{tanirodhataḥ}}/}\\+}
\tl{
\pada{\app{\lem[wit={ceteri}]{na yasya}
		\rdg[wit={C6}]{layasya}}
	\app{\lem[wit={ceteri}]{vāyur vahati}% vahari G4
		\rdg[wit={V3}]{vahate vāyu}}}
\pada{sa mukto nātra saṃśayaḥ//\versenr} % yukto? V19
	\label{pravese}
	%\unavbl{N23}
	\orgvnr{76}\\!}
\end{tlg}
%</vs76>
\commcite%\newpage


%<*vs77>
\begin{tlg}[hp04_077]
% J7 breaks off at \textit{rājayo} in pāda c%
\tl{\app{\lem[nolem]{}
	\rdg[wit={N23}]{\unavbl}
	%\rdg[wit={G11,N19,V15,J10},alt=\textapp{found after \ref{nadakoti}}]{\skp{found after \manuref{4.55*4}}}
	}%
\pada{sarve \app{\lem[wit={N3,J5,V15,J10,P11b,C6,V3,Jyo}]{haṭhalayopāyā}% °pāya J5
		\rdg[wit={G11}]{layahaṭhopāyā}
		\rdg[wit={N19}]{haṭhalayoyāgā}
		\rdg[wit={V19}]{haṭhālayābhyāsā}
		\rdg[wit={J7,E2}]{layahaṭhābhyāsā} % +V6,°haṭha°J7
		\rdg[wit={P11a}]{haṭhalayā bhāvyā}}}
\pada{\app{\lem[wit={N3,J5,J7,V19,E2,P11b,C6b,Jyo}]{rājayogasya siddhaye} % +J7
		\rdg[wit={G11,N19,V15,J10}]{rājayogāya kevalaṃ} % yogaya N19
		\rdg[wit={P11a}]{rājayogapadāvadhi} %
		\rdg[wit={C6a}]{°padāvadhiḥ} %
		\rdg[wit={V3a}]{°padāvadhiṃ}
		\rdg[wit={V3b}]{°phalāvadhi}
		}/}\\+}
\tl{
\pada{\app{\lem[wit={ceteri}]{rājayoga}
		\rdg[wit={G4}]{rajayogaṃ}
		\rdg[wit={E2}]{rājayoge}% +C7
		\rdg[wit={J7},post=\texteng{(then lost)}]{rājayo\skp{ (then lost)}}}%
	\app{\lem[wit={ceteri}]{samārūḍhaḥ}% rūḍha V15, rūḍhā V3
	\rdg[wit={P11a,C6a,V3a}]{padaṃ prāpya}
	\rdg[wit={J5}]{padaprāptaḥ}
	}}
\pada{\app{\lem[wit={ceteri}]{puruṣaḥ kālavañcakaḥ}
		\rdg[wit={P11a,C6a}]{jāyate'sau nirañjanaḥ}
		\rdg[wit={V3a}]{jāyate so nirañjana}
		}//\versenr}\label{kalavancaka}% °caka V3
\myfn{This verse appears twice in \getsiglum{P11,C6,V3}. The first instance (a) is as equivalent of X4.116, and the second (b) is as the semi-final verse of this chapter (4.77 in the α recension). Cf. Introduction, p. \ref{??}.}%
\orgx{\myfn{After this verse, \getsiglum{V19,C7} (not \getsiglum{E2}) have two additional verses:
\vspace{2pt minus 1pt}\\
\devnote{iḍā bhagavatī gaṅgā piṅgalā yamunā nadī/
vijñeyā taddvayor madhye suṣumṇā ca}\,(\getsiglum{V19}; \devnote{tu} \getsiglum{C7}) \devnote{sarasvatī//} (cf. 3.94*1)\\
\devnote{triveṇīsaṃgamo yatra tīrtharājaḥ sa ucyate/
tatra snānaṃ prakurvīta} (\getsiglum{V19}; \devnote{tasmiṃs tīrthavare snātvā} \getsiglum{C7}) \devnote{sarvapāpaiḥ pramucyate//}
}}{}
	%\unavbl{N23}
	\orgvnr{77}\\!} % ḥ om. V3
\end{tlg}
%</vs77>
\commcite\newpage


% \startaltrecension
% \begin{alttlg}[hp04_077_1]
% \tl{
% \pada{iḍā bhagavatī gaṅgā}
% \pada{piṅgalā
	% \app{\lem[wit={C7}]{yamunā}
	% \rdg[wit={V19}]{jamunā}} nadī/}\\+}
% \tl{
% \pada{\app{\lem[wit={C7}]{vijñeyā}
	% \rdg[wit={V19}]{vidheyā}} taddvayor madhye} % tadvayor V19
% \pada{suṣumṇā \app{\lem[wit={C7}]{tu} % so auch in L1
	% \rdg[wit={V19}]{ca}} sarasvatī//\versenr} 
	% \sgwit{V19,C7} \NotIn{E2} % E5 hat dies auch nicht
	% \anm{cf. 3.94*1}\\!}
% \end{alttlg}

% \begin{alttlg}[hp04_077_2]
% \tl{
% \pada{triveṇīsaṃgamo yatra}
% \pada{tīrtharājaḥ sa ucyate/}\\+}
% \tl{
% \pada{\app{\lem[wit={C7}]{tasmiṃs tīrthavare snātvā} % +L1,K2
	% \rdg[wit={V19}]{tatra snānaṃ prakurvīta}
	% }}
% \pada{sarvapāpaiḥ pramucyate//\versenr} 
	% \sgwit{V19,C7} \NotIn{E2}\\!}
% \end{alttlg}
% \endaltrecension


%<*vs78>
\begin{tlg}[hp04_078]
\tl{\app{\lem[nolem]{}
	\rdg[wit={N23,J7}]{\unavbl}
	\rdg[wit={N19,V15,J10,Jyo},alt={\om}]{\skp{\om}}}%
\pada{iti 
	\app{\lem[wit={Gr3,GrB}]{tu}
		\rdg[wit={N3}]{<<tu>>}
		\rdg[wit={J5,G11},alt={\om}]{\skp{\om}}% +F
		\rdg[wit={G4}]{śrī}}
	\app{\lem[wit={ceteri}]{sakalayoga}
		\rdg[wit={G11}]{sakalasuyoga}}śāstra% 
	\app{\lem[wit={N3pc,E2,C6}]{sindhoḥ}% +C7
		\rdg[wit={J5}]{sindhauḥ}
		\rdg[wit={V19}]{sindhau}
		\rdg[wit={G11}]{siddhoḥ}
		\rdg[wit={P11}]{siddheḥ}
		\rdg[wit={N3ac}]{siddhāḥ}
		\rdg[wit={V3}]{siddhyaiḥ}
		\rdg[wit={G4},alt={\om}]{\skp{\om}}}}\\+}
\tl{
\pada{\app{\lem[wit={N3,J5,Gr3,G11,P11,C6},alt={parimathitād}]{parimathitā\skp{d}}
		\rdg[wit={V3}]{paripaṭhitā}
		\rdg[wit={G4}]{mathitā pari}}%
	\app{\lem[wit={N3ac,J5,Gr3,G11},alt={avakṛṣṭa}]{\skm{d }avakṛṣṭa}
		\rdg[wit={N3pc,C6}]{avakṛṣya}
		\rdg[wit={P11}]{avakṛṣṇa}
%		\rdg[wit={C7}]{apakṛṣṭa}
		\rdg[wit={V3}]{kṛṣṭa}
		\rdg[wit={G4}]{sāra}}% \marma
	\app{\lem[wit={Gr1,E2,G11,C6,V3}]{sāra}% +C7
		\rdg[wit={P11}]{sārā}
		\rdg[wit={V19}]{sarva}}bhūtam/}\\+} % bhūtaḥ P11
\tl{
\pada{\app{\lem[wit={N3,G4,Gr3,V3}]{anubhavata}
		\rdg[wit={C6}]{anubhavatu}
		\rdg[wit={J5}]{anubhavān}
		\rdg[wit={G11,P11}]{anubhava}}
		haṭhāmṛtaṃ % <<ha>>ṭhā° V19, ayamṛtaṃ J5
	\app{\lem[wit={N3,G4,C7,V3}]{yamīndrā}% °rāḥ N3pc, yamiṃdrā N3ac, yamīndro F
		\rdg[wit={V19,G11,P11}]{yatīndrā}
		\rdg[wit={J5}]{yogīdrā}
		\rdg[wit={C6}]{mayedaṃ}
		\rdg[wit={E2}]{ya\,(text stopps here)}
		}}\\+}
\tl{
\pada{yadi bhavatā%m
	\app{\lem[wit={N3,J5,V19,C7,P11},alt={ajarāmaratvavāñchā}]{\skm{m }ajarāmaratvavāñchā}% ajjarā° P11
		\rdg[wit={C6},alt={°vāṃchāḥ}]{ajarāmaratvavāṃchāḥ}
		\rdg[wit={G4},alt={°vāṃcchāṃ}]{ajarāmaratvavāṃcchāṃ}
		\rdg[wit={G11}]{ajarāmṛtatvavāṃcha}
		\rdg[wit={V3}]{ajarājaraṃ tvaṃ vā}}//\versenr}%
	\myfnx{In place of this verse, the \texteta\ group manuscripts have the following as the final verse:
	\vspace{2pt minus 1pt}\\
	\devnote{vidyātīrthe jagati vibudhāḥ sādhavaḥ satyatīrthe, %
 	gaṅgātīrthe malinamanaso yogino jñānatīrthe/\\ %
	dhārātīrthe dharaṇipatayo dānatīrthe dhanāḍhyāḥ, %
	lajjātīrthe kulayuvatayaḥ pātakaṃ kṣālayanti//}}
	%\unavbl{N23,J7}
	\orgvnr{78}\\!}
\end{tlg}
%</vs78>
\commcite%\newpage
		% \sgwit{Gr1,V19,C7,G11,V3}


\iffalse
\startaltrecension
\begin{alttlg}[hp04_078_1]
%<*vs78-1>
\tl{\pada{vidyātīrthe \app{\lem[resp=emend]{jagati}\rdg[wit={J10}]{yagati}} vibudhāḥ sādhavaḥ satyatīrthe}\\+}
\tl{\pada{gaṅgātīrthe malinamanaso yogino jñānatīrthe/}\\+}
\tl{\pada{dhārātīrthe dharaṇipatayo dānatīrthe dhanāḍhyāḥ}\\+}
\tl{\pada{lajjātīrthe kulayuvatayaḥ pātakaṃ kṣālayanti//\versenr}
\sgwit{J10}\\!}
%</vs78-1>
\end{alttlg}
\endaltrecension
\fi


%<*vscol>
\begin{col}[hp04_col]
\app{\lem[nolem]{}
	\rdg[wit={N23,J7,E2}]{\unavbl}
	\rdg[wit={N19}]{\om}}%
iti \app{\lem[wit={N3,J5,C7,V15,J10,V3,Jyo}]{śrī}
	\rdg[wit={G4,V19,G11,P11,C6},alt={\om}]{\skp{\om}}}%
\app{\lem[alt={\post śrī \add},nosep]{\skp{\post śrī \add}}
	\rdg[wit={N3}]{sadguru}
	\rdg[wit={J5}]{madguru}
	\rdg[wit={G11,V15,Jyo}]{sahajānandasaṃtānacintāmaṇinā}}% saṃtāra G11
\app{\lem[wit={J5,C6,V3}]{svātmārāmayogīndra} % yogiṃdra V3
	\rdg[wit={N3}]{svātmārāmayogendra}
	\rdg[wit={Jyo}]{svātmārāmayogīndra}
	\rdg[wit={V15}]{svātmārāmayogīṃdreṇa}
	\rdg[wit={G4,J10}]{ātmārāmayogīṃdra}
	\rdg[wit={P11},post=\texteng{(sic!)}]{°yo°}
	\rdg[wit={V19,C7,G11},alt={\om}]{\skp{\om}}}%
\app{\lem[wit={ceteri}]{viracitāyāṃ}
	\rdg[wit={N3}]{pravaracitāyāṃ}
	\rdg[wit={V19,P11},alt={\om}]{\skp{\om}}} haṭhapradīpikāyāṃ
\app{\lem[alt={\ante caturtho° \add},nosep]{\skp{\ante caturtho° \add}}
	\rdg[wit={V15}]{nādopāsanaṃ nāma}
	\rdg[wit={Jyo}]{samādhilakṣaṇaṃ nāma}
	\rdg[wit={V3}]{siddhāntamuktāvalī nāma}}%
\app{\lem[wit={Gr1,G11,V15,GrB,Jyo}]{caturthopadeśaḥ}
	\rdg[wit={V19}]{caturtha upadeśaḥ}
	\rdg[wit={C7}]{caturtho\{\{dhyā\}\}yam upadeśaḥ}
	\rdg[wit={J10}]{caturthodhyāyaḥ}}//~4//%\sgwit{N3,V19,C7,G11,V15,J10,GrB}
%\myfn{%
%The colophon is found only in \getsiglum{Gr1,V19,C7,G11,V15,J10,GrB}. 
%\getsiglum{N23,J7,K3} have lost their last folios.
%\getsiglum{N19} has no colophon.
%\getsiglum{N19} has no colophon. Its last verse is \ref{ajananta}, which just fills fol. no. 346, and from the next folio another text begins. The colophon of \getsiglum{Jyo} reads: \devnote{iti śrīsvātmārāmayogīṃdraviracitāyāṃ haṭhayogapradīpikāyāṃ nāma caturtho'dhyāyaḥ} (Wai) or \devnote{iti śrīsajahānandasantānacintāmaṇisvātmārāmayogīṃdraviracitāyāṃ haṭhayogapradīpikāyāṃ samādhilakṣaṇaṃ nāma caturthopadeśaḥ samāptaḥ} (Tue)}
%\unavbl{N23,J7,E2}%\NotIn{N19}
\end{col}
%</vscol>

\colcite

% N3 śrīsahajasadguruātmārāmārpaṇa saṃpūrṇaṃ |
% P11 (immeadiately followed by another text)
% C6 oṃ || pustakam idaṃ jājīrāmavyāsasya || śrīrāmaḥ
% V3 subhaṃ būyāt @ idaṃ graṃthasaṃṣyā || 502 || @ @ samvat || 16 || 93 || varṣe || || || liṣitaṃ jamadagnigirisanyāsī svātmapaṭhanārthaṃ || oṃ namaḥ śivāya ||
% V19
% C7
% G11 śrī śivāya namaḥ || śrī śaṃkarācāryagurucaraṇāravindābhyāṃ namaḥ || gaṇapati svahasta likhitaṃ | śrī gurubhyo namaḥ || oṃ | śubhaṃ |
% J10 saṃvat || 16 || 83 ||


\end{ekdosis}
\end{otherlanguage}
\end{document}


%\vspace*{2cm}
%\vfill
\teimute{\small}
\begin{tabular}{l l l}
\multicolumn{3}{l}{\textbf{List of Sigla}} \\
\\
\getsiglum{N3} & N3 & one folio missing in Ch. 4 (\ref{cittananda}b--\ref{nadanu}d)\\
\getsiglum{J5} & J5 \\
\getsiglum{G4} & G4 & damaged; collated only when available\\
\getsiglum{N23} & N23 & incomplete; breaks off at \manuref{4.56d}\\
\getsiglum{J7} & J7 & incomplete; breaks off at \manuref{4.74b}\\
%\getsiglum{V6} & collated for 4.91--92 only\\
\getsiglum{V19} & V19 \\
\getsiglum{E2} & E2 \\
\getsiglum{C7} & C7 & partially collated, when \getsiglum{E2} is not available\\
\getsiglum{G11} & G11 \\
\getsiglum{G5} & G5 & collated for gray-scaled verses only \\
\getsiglum{N19} & N19 \\
\getsiglum{V15} & V15 \\
%\getsiglum{J11} & J11 & collated for \manuref{4.28} and \manuref{4.32*1--8} only\\
\getsiglum{J10} & J10 \\
\getsiglum{P11} & P11 & \\
\getsiglum{C6} & C6 \\
\getsiglum{V3} & V3 \\
\getsiglum{Jyo} & Jyo & Brahmānanda's version, based on the edition 1972 \\
\end{tabular}

\end{document}
