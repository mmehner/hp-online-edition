\documentclass[11pt,twoside]{article}
\usepackage[papersize={17cm,24cm},
	centering, textwidth=12.5cm, textheight=17.9cm,
	asymmetric]{geometry} % so that the orgvnr always appear on the right side
%\linespread{1.1}
\sloppy

\usepackage{xcolor, xparse, xspace, pifont, datetime2}
\usepackage{enumitem}

\newcommand{\HP}{\textit{Haṭha\-pra\-dī\-pikā}\xspace}

\usepackage{fancyhdr}
	\renewcommand{\headrulewidth}{0pt}
	\fancyhead[EL]{\small\texteng{\thepage}}
	\fancyhead[ER]{\small\texteng{Critical Edition \& Translation}}
	\fancyhead[OL]{\small\texteng{HP\hindsection}}
	\fancyhead[OR]{\small\texteng{\thepage}}
	\fancyhead[C]{}
	\fancyfoot[C]{}
%	\pagestyle{fancy}

\fancypagestyle{firstpage}{%
	\fancyhead[OL]{\small\texteng{(\Today)}}
	\fancyhead[OR]{\small\texteng{\thepage}}
	\fancyhead[C]{}
	}

\fancypagestyle{HPed}{%
	\renewcommand{\headrulewidth}{0pt}
	\fancyhead[EL]{\small\texteng{\thepage}}
	\fancyhead[ER]{\small\texteng{[\rightmark\textendash}}
	\fancyhead[OL]{\small\texteng{\textendash\leftmark]}}
	\fancyhead[OR]{\small\texteng{\thepage}}
	\fancyhead[C]{\small\texteng{Critical Edition}}
	\fancyfoot[C]{}
	}
\pagestyle{HPed}

\usepackage{scrextend}
	\deffootnote[]{0em}{1em}{}
%	\deffootnote[2.1em]{0em}{1.5em}{\color{blue}\texteng{\textbf{Verse \thefootnotemark}}}
\usepackage{fnpos}
	\makeFNbottom
%\footnoterulefalse
\setlength{\footnotesep}{1em}
%\dimen\footins=??
%\renewcommand{\footnotesize}{\small}  % Wirkt auf App und Fn beides.

\usepackage[english]{babel}
\usepackage{babel-iast}
\babelfont[iast]{rm}[Renderer=Harfbuzz, Scale=1.1]{AdishilaSan}
\babeltags{dev = iast}
\babeltags{eng = english}
\usepackage{libertine}

\usepackage[teiexport=tidy,poetry=verse]{ekdosis}
\usepackage{sanskrit-poetry}
\usepackage{textgreek}

%%% Gr1,4b,6
\DeclareWitness{N3}{\texteng{\textalpha\textsubscript{1}}}{NGMPP B 62-20}[]
        \DeclareHand{N3ac}{N3}{\texteng{\textalpha\rlap{\textsubscript{1}}\textsuperscript{ac}}}[]
        \DeclareHand{N3pc}{N3}{\texteng{\textalpha\rlap{\textsubscript{1}}\textsuperscript{pc}}}[]
\DeclareWitness{J5}{\texteng{\textalpha\textsubscript{2}}}{Jodhpur 02235}[]
\DeclareWitness{G4}{\texteng{\textalpha\textsubscript{3}}}{GOML 18885}[]% Telugu script
\DeclareWitness{N24}{\texteng{\textalpha\textsubscript{4}}}{NGMPP G 190-16}[]
\DeclareWitness{Gr1r}{\texteng{\textAlpha *}}{Gr1 reconstructed}[]

\DeclareWitness{P11}{\texteng{\textbeta\textsubscript{1}}}{}[]
\DeclareWitness{C6}{\texteng{\textbeta\textsubscript{2}}}{Lalchand M-2089}[]

\DeclareWitness{V3}{\texteng{\textbeta\textsubscript{\textomega}}}{Sampurnananda Library Sarasvati Bhavan 29899}[]

%%% Gr2

\DeclareWitness{N23}{\texteng{\textgamma\textsubscript{1}}}{NGMPP G 25-2}[]
        \DeclareHand{N23ac}{N23}{\texteng{\textgamma\rlap{\textsubscript{1}}\textsuperscript{ac}}}[]
        \DeclareHand{N23pc}{N23}{\texteng{\textgamma\rlap{\textsubscript{1}}\textsuperscript{pc}}}[]
\DeclareWitness{J7}{\texteng{\textgamma\textsubscript{2}}}{Jodhpur 02241}[]
%\DeclareWitness{V6}{\texteng{V\textsubscript{6}}}{Sampurnananda Library Sarasvati Bhavan 29991}[]
\DeclareWitness{K1}{\texteng{K\textsubscript{1}}}{Raghunātha Temple Library 4383}[settlement=Jammu]
        \DeclareWitness{K1ac}{\texteng{K\rlap{\textsubscript{1}}\textsuperscript{ac}\space}}{}[]
        \DeclareWitness{K1pc}{\texteng{K\rlap{\textsubscript{1}}\textsuperscript{pc}\space}}{}[]


%%% Gr3

\DeclareWitness{V19}{\texteng{\textdelta\textsubscript{1}}}{Sampurnananda Library Sarasvati Bhavan 30069}[]
\DeclareWitness{K3}{\texteng{\textdelta\textsubscript{2}}}{Privat collection}
\DeclareWitness{C7}{\texteng{\textdelta\textsubscript{3}}}{Lalchand M-6494}[]
%\DeclareWitness{C1}{\texteng{C\textsubscript{1}}}{Lalchand M-2080}[]%L1 And C1 very close (and come from same region)
%\DeclareWitness{P23}{\texteng{P\textsubscript{23}}}{}[]
%\DeclareWitness{L1}{\texteng{L\textsubscript{1}}}{SOAS RE 43454}[settlement=Jammu]

\DeclareWitness{J6}{\texteng{\textdelta\textsubscript{\textomega}}}{Jodhpur 02237}[]
        \DeclareHand{J6ac}{J6}{\texteng{\textdelta\rlap{\textomega}\textsuperscript{ac}}}[]
        \DeclareHand{J6pc}{J6}{\texteng{\textdelta\rlap{\textomega}\textsuperscript{pc}}}[]

%%% Gr4c

\DeclareWitness{P15}{\texteng{\textepsilon\textsubscript{1}}}{}[]
\DeclareWitness{N19}{\texteng{\textepsilon\textsubscript{2}}}{NGMPP E-1528-1 / E-1527-7(4)}[]
\DeclareWitness{V15}{\texteng{\textepsilon\textsubscript{3}}}{Sampurnananda Library Sarasvati Bhavan 30051}[]
        \DeclareHand{V15ac}{V15}{\texteng{\textepsilon\rlap{\textsubscript{3}}\textsuperscript{ac}}}[]
        \DeclareHand{V15pc}{V15}{\texteng{\textepsilon\rlap{\textsubscript{3}}\textsuperscript{pc}}}[]
\DeclareWitness{J11}{\texteng{\textepsilon\textsubscript{4}}}{Jodhpur 23532}[]
        \DeclareHand{J11ac}{J11}{\texteng{\textepsilon\rlap{\textsubscript{4}}\textsuperscript{i.t.}}}[]
        \DeclareHand{J11pc}{J11}{\texteng{\textepsilon\rlap{\textsubscript{4}}\textsuperscript{mg.}}}[alternative reading written by the first hand in margin or interlinearly (J11)]
%\DeclareWitness{J14}{\texteng{\textepsilon\textsubscript{5}}}{Jodhpur 02239}[]

%\DeclareWitness{L2}{\texteng{L\textsubscript{2}}}{Wellcome Collection O.36]}
\DeclareWitness{M1}{\texteng{M\textsubscript{1}}}{P-5682/4}[]

\DeclareWitness{N26}{\texteng{\textepsilon\textsubscript{\textomega}}}{NGMPP}[]
%\DeclareWitness{V17}{\texteng{\textepsilon\textsubscript{\textomega 3}}}{Sampurnananda Library Sarasvati Bhavan 30053}[]

\DeclareWitness{V1}{\texteng{\texteta\textsubscript{1}}}{Sampurnananda Library Sarasvati Bhavan 30109}[]
        \DeclareHand{V1ac}{V1}{\texteng{\texteta\rlap{\textsubscript{1}}\textsuperscript{ac}}}[]
        \DeclareHand{V1pc}{V1}{\texteng{\texteta\rlap{\textsubscript{1}}\textsuperscript{pc}}}[]

%%% Gr4d

\DeclareWitness{J10}{\texteng{\texteta\textsubscript{2}}}{MSPP Jodhpur 2230}[]
        \DeclareHand{J10ac}{J10}{\texteng{\texteta\rlap{\textsubscript{2}}\textsuperscript{ac}}}[]
        \DeclareHand{J10pc}{J10}{\texteng{\texteta\rlap{\textsubscript{2}}\textsuperscript{pc}}}[]

\DeclareWitness{N9}{\texteng{\texteta\textsubscript{\textomega}}}{NGMPP A62-33}[]
%\DeclareWitness{J15}{\texteng{\textepsilon\textsubscript{\textomega 4}}}{Jodhpur 9732A}[]

%%%

\DeclareWitness{Jyo}{\texteng{\textchi}}{Brahmānanda's version}[]
%\DeclareWitness{Tue}{\texteng{Tü}}{Ma I 339}[]

\DeclareWitness{ceteri}{\texteng{cett.}}{ceteri}[]

%%% Group Sigla

\DeclareWitness{Gr1}{\texteng{\textAlpha}}{N3,J5,G4}

\DeclareWitness{Gr2}{\texteng{\textGamma}}{N23,J7}
%\DeclareWitness{Gr2}{\texteng{%
%	\textbeta\textsubscript{1}%
%	\textbeta\textsubscript{2}%
%	}}{N23,J7}
\DeclareWitness{Gr3a}{\texteng{\textDelta}}{V19,K3,C7}
\DeclareWitness{Gr4b}{\texteng{%
	\textbeta\textsubscript{1}%
	\textbeta\textsubscript{2}%
	}}{C6,P11}
\DeclareWitness{GrB}{\texteng{%
	\textbeta\textsubscript{1}%
	\textbeta\textsubscript{2}%
	\textbeta\textsubscript{\textomega}%
	}}{C6,P11,V3}
\DeclareWitness{Gr4c}{\texteng{\textEpsilon}}{P15,N19,V15}

% \DeclareWitness{Gr4d}{\texteng{%
	% \texteta\textsubscript{1}%
	% \texteta\textsubscript{2}%
	% }}{V1,J10}
\DeclareWitness{Gr6}{\texteng{\textOmega}}{V3,J6,N9,N26}




%%%%%%%%%%%%%%%%%%%% THE  MSS         %%%%%%%%%%%%%%%%%%%%%%%%%%%

%%% Versions
\DeclareWitness{Vu}{\selectlanguage{english}Vulg}{Vulgate, i.e. Brahmānanda's version}[]           
\DeclareWitness{X}{\selectlanguage{english}X}{TenChapter Version, Jodhpur 02228 and 02225 (ed. Lonavla)}[]
\DeclareWitness{Six}{\selectlanguage{english}Ṣ}{SixChapterVersion, ``6ChapterHPms'', fragment of enlarged text, Jodhpur}[]
% Mss. in Geographical Groups
%%%% Varanasi mss (Sampūrṇānanda mss). V1 is Important
\DeclareWitness{V1}{\selectlanguage{english}V\textsubscript{1}}{Sampurnananda Library Sarasvati Bhavan 30109}[]
        \DeclareHand{V1ac}{V1}{\selectlanguage{english}V\rlap{\textsubscript{1}}\textsuperscript{ac}}[] % added by MD
        \DeclareHand{V1pc}{V1}{\selectlanguage{english}V\rlap{\textsubscript{1}}\textsuperscript{pc}}[] % added by MD
\DeclareWitness{V2}{\selectlanguage{english}V\textsubscript{2}}{Sampurnananda Library Sarasvati Bhavan 29869}[]
\DeclareWitness{V3}{\selectlanguage{english}V\textsubscript{3}}{Sampurnananda Library Sarasvati Bhavan 29899}[]
\DeclareWitness{V4}{\selectlanguage{english}V\textsubscript{4}}{Sampurnananda Library Sarasvati Bhavan 29937}[]
\DeclareWitness{V5}{\selectlanguage{english}V\textsubscript{5}}{Sampurnananda Library Sarasvati Bhavan 29938}[]
\DeclareWitness{V6}{\selectlanguage{english}V\textsubscript{6}}{Sampurnananda Library Sarasvati Bhavan 29991}[]
\DeclareWitness{V8}{\selectlanguage{english}V\textsubscript{8}}{Sampurnananda Library Sarasvati Bhavan 30014}[]
\DeclareWitness{V11}{\selectlanguage{english}V\textsubscript{11}}{Sampurnananda Library Sarasvati Bhavan 30029}[]
\DeclareWitness{V12}{\selectlanguage{english}V\textsubscript{12}}{Sampurnananda Library Sarasvati Bhavan 30030}[]
\DeclareWitness{V13}{\selectlanguage{english}V\textsubscript{13}}{Sampurnananda Library Sarasvati Bhavan 30031}[]
\DeclareWitness{V14}{\selectlanguage{english}V\textsubscript{14}}{Sampurnananda Library Sarasvati Bhavan 30050}[]
\DeclareWitness{V15}{\selectlanguage{english}V\textsubscript{15}}{Sampurnananda Library Sarasvati Bhavan 30051}[]
\DeclareWitness{V15pc}{\selectlanguage{english}V\rlap{\textsubscript{15}}\textsuperscript{pc}\space}{}[]
\DeclareWitness{V16}{\selectlanguage{english}V\textsubscript{16}}{Sampurnananda Library Sarasvati Bhavan 30052}[]
\DeclareWitness{V17}{\selectlanguage{english}V\textsubscript{17}}{Sampurnananda Library Sarasvati Bhavan 30053}[] % added by MD
\DeclareWitness{V16pc}{\selectlanguage{english}V\rlap{\textsubscript{16}}\textsuperscript{pc}\space}{}[]
\DeclareWitness{V18}{\selectlanguage{english}V\textsubscript{18}}{Sampurnananda Library Sarasvati Bhavan 30064}[]
\DeclareWitness{V19}{\selectlanguage{english}V\textsubscript{19}}{Sampurnananda Library Sarasvati Bhavan 30069}[]
\DeclareWitness{V21}{\selectlanguage{english}V\textsubscript{21}}{Sampurnananda Library Sarasvati Bhavan 30104}[]
\DeclareWitness{V22}{\selectlanguage{english}V\textsubscript{22}}{Sampurnananda Library Sarasvati Bhavan 30110}[]
\DeclareWitness{V25}{\selectlanguage{english}V\textsubscript{25}}{Sampurnananda Library Sarasvati Bhavan 30122}[]
\DeclareWitness{V26}{\selectlanguage{english}V\textsubscript{26}}{Sampurnananda Library Sarasvati Bhavan 30123}[]
\DeclareWitness{V28}{\selectlanguage{english}V\textsubscript{28}}{Sampurnananda Library Sarasvati Bhavan 30136}[]
\DeclareWitness{W2}{\selectlanguage{english}W\textsubscript{2}}{Wai ??}[]
\DeclareWitness{W4}{\selectlanguage{english}W\textsubscript{4}}{Wai 399-6171}[]

%%%%%%%%%%%%%%%%%%%%%%%%%%%%%%%%%
%%% Jammu & Kaschmir
\DeclareWitness{K1}{\selectlanguage{english}K\textsubscript{1}}{Raghunātha Temple Library 4383}[settlement=Jammu]
        \DeclareWitness{K1ac}{\selectlanguage{english}K\rlap{\textsubscript{1}}\textsuperscript{ac}\space}{}[]
        \DeclareWitness{K1pc}{\selectlanguage{english}K\rlap{\textsubscript{1}}\textsuperscript{pc}\space}{}[]
\DeclareWitness{K3}{\selectlanguage{english}K\textsubscript{3}}{Privat collection}
\DeclareWitness{L1}{\selectlanguage{english}L\textsubscript{1}}{SOAS RE 43454}[settlement=Jammu]
% More details? Catalogue number? L1 And C1 very close (and come from same region)
%%%%%%%%%%%%%%%%%%%%%%%%%%%%%%%%
% Jodhpur
% J10 is important
\DeclareWitness{J10}{\selectlanguage{english}J\textsubscript{10}}{MSPP Jodhpur 2230}[]
        \DeclareHand{J10ac}{J10}{\selectlanguage{english}J\rlap{\textsubscript{10}}\textsuperscript{ac}}[] % modified by MD
        \DeclareHand{J10pc}{J10}{\selectlanguage{english}J\rlap{\textsubscript{10}}\textsuperscript{pc}}[] % modified by MD
\DeclareWitness{J1}{\selectlanguage{english}J\textsubscript{1}}{Jodhpur 02231}[]
\DeclareWitness{J2}{\selectlanguage{english}J\textsubscript{2}}{Jodhpur 02232}[]   
\DeclareWitness{J3}{\selectlanguage{english}J\textsubscript{3}}{Jodhpur 02233}[]
\DeclareWitness{J4}{\selectlanguage{english}J\textsubscript{4}}{Jodhpur 02234}[]
        \DeclareWitness{J4ac}{\selectlanguage{english}J\rlap{\textsubscript{4}}\textsuperscript{ac}\space}{MSPP Jodhpur 02234}[]
        \DeclareWitness{J4pc}{\selectlanguage{english}J\rlap{\textsubscript{4}}\textsuperscript{pc}\space}{MSPP Jodhpur 02234}[]
\DeclareWitness{J5}{\selectlanguage{english}J\textsubscript{5}}{Jodhpur 02235}[]  % 4 chapters, 34 jpgs,   long colophon, missing lines in the beginning.
\DeclareWitness{J6}{\selectlanguage{english}J\textsubscript{6}}{Jodhpur 02237}[]  % 4 chapters, 41 jpgs
%\DeclareWitness{J6ac}{\selectlanguage{english}J\rlap{\textsubscript{6}}\textsubscript{ac}}{Jodhpur 02237}[]  % 4 chapters, 49 jpgs,   1st folio: idaṃ gulābarāyasya
% tulasīrāmaśarmmaṇaḥ putrasya pustakaṃ ...        End: iti śrīsahajānandasantānacintāmaṇisvātmārāmaviracitāyāṃ ..
% saṃvat 1802   (more consistent text)
%\DeclareWitness{J6pc}{\selectlanguage{english}J\rlap{\textsubscript{6}}\textsubscript{pc}}{Jodhpur 02237}[] 
\DeclareWitness{J7}{\selectlanguage{english}J\textsubscript{7}}{Jodhpur 02241}[]  % 4 chapters, 41 jpgs
\DeclareWitness{J8}{\selectlanguage{english}J\textsubscript{8}}{Jodhpur 23709}[]  % 4 chapters,  87 jpgs.   saṃvat 1724
\DeclareHand{J8ac}{J8}{\selectlanguage{english}J\rlap{\textsubscript{8}}\textsuperscript{ac}}[]  % changed by MD
\DeclareHand{J8pc}{J8}{\selectlanguage{english}J\rlap{\textsubscript{8}}\textsuperscript{pc}}[]  % changed by MD
\DeclareWitness{J9}{\selectlanguage{english}J\textsubscript{9}}{Jodhpur 02224}[]  %  fragment, 20 jpgs.
\DeclareWitness{J11}{\selectlanguage{english}J\textsubscript{11}}{Jodhpur 23532}[]
        \DeclareHand{J11ac}{J11}{\selectlanguage{english}J\rlap{\textsubscript{11}}\textsuperscript{ac}}[] % added by MD
        \DeclareHand{J11pc}{J11}{\selectlanguage{english}J\rlap{\textsubscript{11}}\textsuperscript{pc}}[] % added by MD
\DeclareWitness{J12}{\selectlanguage{english}J\textsubscript{12}}{Jodhpur 18552}[] 
\DeclareWitness{J13}{\selectlanguage{english}J\textsubscript{13}}{Jodhpur 02229}[]  %  5 chapters, 93 jpgs.
\DeclareWitness{J14}{\selectlanguage{english}J\textsubscript{14}}{Jodhpur 02239}[]  %  4 chapters
\DeclareWitness{J15}{\selectlanguage{english}J\textsubscript{15}}{Jodhpur 9732A}[]
\DeclareWitness{J16}{\selectlanguage{english}J\textsubscript{16}}{Jodhpur 9732B}[]
\DeclareWitness{J17}{\selectlanguage{english}J\textsubscript{17}}{Jodhpur 3013}[]
% Haṭhapradīpikā with (non-Sanskrit) Bhāṣya RORI Jodhpur ACC.NO.18552
%  Haṭhapradīpikā with (non-Sanskrit) commentary, RORI Alwar 952, 4 chapters,  colophon of the comm:
% iti śrīlāhorīmiśravrajabhūṣanaviracitāyāṃ bhāvārthadīpikāyāṃ caturthodhyāya ..    
%  Haṭhapradīpikā (5 chapter) MSPP Jodhpur ACC.NO.02229/

%%%%%%%%%%        Bodleian, Oxford
\DeclareWitness{B1}{\selectlanguage{english}B\textsubscript{1}}{Bodleian Library No. d.457(8)}[settlement=Oxford]
\DeclareWitness{B2}{\selectlanguage{english}B\textsubscript{2}}{Bodleian Library No. d.458(1)}[settlement=Oxford]
\DeclareWitness{B3}{\selectlanguage{english}B\textsubscript{3}}{Bodleian Library No. d.458(9)}[settlement=Oxford]

%%%%%%%%%%%   Chandigarh
\DeclareWitness{C1}{\selectlanguage{english}C\textsubscript{1}}{Lalchand M-2080}[]%L1 And C1 very close (and come from same region)
\DeclareWitness{C2}{\selectlanguage{english}C\textsubscript{2}}{Lalchand M-6065}[]
\DeclareWitness{C3}{\selectlanguage{english}C\textsubscript{3}}{Lalchand M-1293}[]
\DeclareWitness{C4}{\selectlanguage{english}C\textsubscript{4}}{Lalchand M-2081}[]
\DeclareWitness{C4ac}{\selectlanguage{english}C\rlap{\textsubscript{4}}\textsuperscript{ac}\space}{}[]
\DeclareWitness{C4pc}{\selectlanguage{english}C\rlap{\textsubscript{4}}\textsuperscript{pc}\space}{}[]
\DeclareWitness{C5}{\selectlanguage{english}C\textsubscript{5}}{Lalchand M-2082}[]%doesn't have chapter 1
\DeclareWitness{C6}{\selectlanguage{english}C\textsubscript{6}}{Lalchand M-2089}[]
\DeclareWitness{C7}{\selectlanguage{english}C\textsubscript{7}}{Lalchand M-6494}[]
\DeclareWitness{C8}{\selectlanguage{english}C\textsubscript{8}}{Lalchand M-2091}[]
        \DeclareHand{C8ac}{C8}{\selectlanguage{english}C\rlap{\textsubscript{8}}\textsuperscript{ac}}[]
        \DeclareHand{C8pc}{C8}{\selectlanguage{english}C\rlap{\textsubscript{8}}\textsuperscript{pc}}[]
\DeclareWitness{C9}{\selectlanguage{english}C\textsubscript{9}}{Lalchand M-4530}[]


% %%%%%%%%%%        Nepalese
\DeclareWitness{N1}{\selectlanguage{english}N\textsubscript{1}}{NGMPP A1400-2}[]
\DeclareWitness{N2}{\selectlanguage{english}N\textsubscript{2}}{NGMPP B 39-19}[]
\DeclareWitness{N3}{\selectlanguage{english}N\textsubscript{3}}{NGMPP B 62-20}[]
\DeclareWitness{N5}{\selectlanguage{english}N\textsubscript{5}}{NGMPP A60-15 + A61-1}[]
\DeclareWitness{N4}{\selectlanguage{english}N\textsubscript{4}}{NGMPP A61-2}[]
\DeclareWitness{N6}{\selectlanguage{english}N\textsubscript{6}}{NGMPP A61-6}[]
\DeclareWitness{N9}{\selectlanguage{english}N\textsubscript{9}}{NGMPP A62-33}[]
\DeclareWitness{N10}{\selectlanguage{english}N\textsubscript{10}}{NGMPP A62-37}[]
\DeclareWitness{N11}{\selectlanguage{english}N\textsubscript{11}}{NGMPP A63-15}[]
\DeclareWitness{N12}{\selectlanguage{english}N\textsubscript{12}}{NGMPP A939-19}[]
\DeclareWitness{N13}{\selectlanguage{english}N\textsubscript{13}}{NGMPP A1378-18}[]
\DeclareWitness{N16}{\selectlanguage{english}N\textsubscript{16}}{NGMPP B39-20}[]
\DeclareWitness{N17}{\selectlanguage{english}N\textsubscript{17}}{NGMPP B 111-10}[]
\DeclareWitness{N18}{\selectlanguage{english}N\textsubscript{18}}{NGMPP E 929-3}[]
\DeclareWitness{N19}{\selectlanguage{english}N\textsubscript{19}}{NGMPP E-1528-1 / E-1527-7(4)}[]
\DeclareWitness{N20}{\selectlanguage{english}N\textsubscript{20}}{NGMPP E 2037-13 }[]
\DeclareWitness{N21}{\selectlanguage{english}N\textsubscript{21}}{NGMPP E 2097-31}[]
\DeclareWitness{N22}{\selectlanguage{english}N\textsubscript{22}}{NGMPP G 4-4}[]
\DeclareWitness{N23}{\selectlanguage{english}N\textsubscript{23}}{NGMPP G 25-2}[]
        \DeclareHand{N23ac}{N23}{\selectlanguage{english}N\rlap{\textsubscript{23}}\textsuperscript{ac}}[] % added by MD
        \DeclareHand{N23pc}{N23}{\selectlanguage{english}N\rlap{\textsubscript{23}}\textsuperscript{pc}}[] % added by MD
\DeclareWitness{N24}{\selectlanguage{english}N\textsubscript{24}}{NGMPP G 190-16}[]
\DeclareWitness{N24ac}{\selectlanguage{english}N\rlap{\textsubscript{24}}\textsuperscript{ac}\space}{}[]
\DeclareWitness{N24pc}{\selectlanguage{english}N\rlap{\textsubscript{24}}\textsuperscript{pc}\space}{}[]
\DeclareWitness{N26}{\selectlanguage{english}N\textsubscript{26}}{NGMPP T 24-3}[]

% %%%%%%%%%%        Pune

\DeclareWitness{P1}{\selectlanguage{english}P\textsubscript{1}}{Ānandāśrama S16-3-21}[]
\DeclareWitness{P2}{\selectlanguage{english}P\textsubscript{2}}{Ānandāśrama S16-2-20}[]
\DeclareWitness{P3}{\selectlanguage{english}P\textsubscript{3}}{BISM (79) 314}[]
\DeclareWitness{P4}{\selectlanguage{english}P\textsubscript{4}}{BISM (91) 191}[]
\DeclareWitness{P5}{\selectlanguage{english}P\textsubscript{5}}{BISM (29) 5790}[]
\DeclareWitness{P6}{\selectlanguage{english}P\textsubscript{6}}{BORI 263/1879-80}[]
\DeclareWitness{P7}{\selectlanguage{english}P\textsubscript{7}}{BORI 665/1883-84}[]
\DeclareWitness{P8}{\selectlanguage{english}P\textsubscript{8}}{BORI 316/1895-98}[]
\DeclareWitness{P9}{\selectlanguage{english}P\textsubscript{9}}{BORI 733-1891-95}[]
\DeclareWitness{P10}{\selectlanguage{english}P\textsubscript{10}}{BORI 222-1884-86}[]
\DeclareWitness{P11}{\selectlanguage{english}P\textsubscript{11}}{BORI 221-1882–83}[]
\DeclareWitness{P12}{\selectlanguage{english}P\textsubscript{12}}{Ānandāśrama S16-3-24}[]
\DeclareWitness{P13}{\selectlanguage{english}P\textsubscript{13}}{Ānandāśrama S16-2-22}[]
\DeclareWitness{P14}{\selectlanguage{english}P\textsubscript{14}}{Ānandāśrama S16-3-23}[]
\DeclareWitness{P15}{\selectlanguage{english}P\textsubscript{15}}{BISM (64) 919}[]
\DeclareWitness{P16}{\selectlanguage{english}P\textsubscript{16}}{BISM (64) 1115}[]
\DeclareWitness{P17}{\selectlanguage{english}P\textsubscript{17}}{BISM 620/1886-92}[]
\DeclareWitness{P18}{\selectlanguage{english}P\textsubscript{18}}{BORI 615/1887-91}[]
\DeclareWitness{P19}{\selectlanguage{english}P\textsubscript{19}}{BISM 46-39}[]
\DeclareWitness{P20}{\selectlanguage{english}P\textsubscript{20}}{BISM 39-273}[]
\DeclareWitness{P21}{\selectlanguage{english}P\textsubscript{21}}{BISM 37-743}[]
\DeclareWitness{P22}{\selectlanguage{english}P\textsubscript{22}}{BISM 37-729}[]
\DeclareWitness{P23}{\selectlanguage{english}P\textsubscript{23}}{BISM 33-60}[]
\DeclareWitness{P24}{\selectlanguage{english}P\textsubscript{24}}{BISM 29-5790}[]% =P5!
\DeclareWitness{P25}{\selectlanguage{english}P\textsubscript{25}}{BISM 29-3657}[]
\DeclareWitness{P26}{\selectlanguage{english}P\textsubscript{26}}{BISM 25-281}[]
\DeclareWitness{P27}{\selectlanguage{english}P\textsubscript{27}}{BISM 7-489}[]
\DeclareWitness{P28}{\selectlanguage{english}P\textsubscript{28}}{BORI 399-1895-1902}[]

%%%%%   Mysore
\DeclareWitness{M1}{\selectlanguage{english}M\textsubscript{1}}{P-5682/4}[]
%%%%%   Tübingen
\DeclareWitness{Tue}{\selectlanguage{english}Tü}{Ma I 339}[]
%%%%%%%%%%
\DeclareWitness{YC}{\selectlanguage{english}YC}{Yogacintāmaṇi}[]
\DeclareWitness{ceteri}{\selectlanguage{english}cett.}{ceteri}[]

%%%%%%%%%% Mss with Commentary
\DeclareWitness{A1}{\selectlanguage{english}A\textsubscript{1}}{Alwar 952}[]

\DeclareWitness{Jyo}{\selectlanguage{english}J\textsubscript{yo}}{Brahmānanda's version}[]

%%%%%%%%%%%%%%%%%%%%%%%%%%%%%%%%%%%%%%%%%%%
%List of all Sigla:
%A1,B1,B2,B3,C1,C2,C3,C4,C6,C7,C8,C9,J1,J2,J3,J4,J10,J13,J14,J15,J17,L1,M1,N3,N5,N6,N9,N10,N11,N12,N13,N16,N17,N19,N20,N21,N22,N23,N24,Tü,V1,V2,V3,V4,V5,V6,V8,V11,V19,V22,V26,Vu
%%%%%%%%%%%%%%%%%%%%%%%%%%%%%%%%%%%%%%%%%%%

\DeclareWitness{G4}{\selectlanguage{english}G\textsubscript{4}}{GOML D18885 (Bundle SD5051)}[]
\DeclareWitness{G5}{\selectlanguage{english}G\textsubscript{5}}{GOML R3841/ SR2190}[]
\DeclareWitness{G7}{\selectlanguage{english}G\textsubscript{7}}{GOML D4394}[]

\DeclareWitness{Ko}{\selectlanguage{english}K\textsubscript{o}}{Koba, Gujarat 55626}[]

% addition 2023-12-11 MD:
\TeXtoTEIPat{\begin {metre}[#1]}{<note type="metre" target="##1">}
\TeXtoTEIPat{\end {metre}}{</note>}
\TeXtoTEIPat{\texttheta}{θ}

% change 2023-12-05 mm
\TeXtoTEI{teimute}{} 

% changes/additions 2023-11-27 MM:
\TeXtoTEIPat{\medialink {#1}{#2}}{<ref target="resources/#2">#1</ref>}

% changes/additions 2023-10-25 MM:
% new Sigla
\TeXtoTEIPat{\textAlpha}{Α}
\TeXtoTEIPat{\textalpha}{α}
\TeXtoTEIPat{\textBeta}{Β}
\TeXtoTEIPat{\textbeta}{β}
\TeXtoTEIPat{\textGamma}{Γ}
\TeXtoTEIPat{\textgamma}{γ}
\TeXtoTEIPat{\textDelta}{Δ}
\TeXtoTEIPat{\textdelta}{δ}
\TeXtoTEIPat{\textEpsilon}{Ε}
\TeXtoTEIPat{\textepsilon}{ε}
\TeXtoTEIPat{\textEta}{Η}
\TeXtoTEIPat{\texteta}{η}
\TeXtoTEIPat{\textChi}{Χ}
\TeXtoTEIPat{\textchi}{χ}
\TeXtoTEIPat{\textOmega}{Ω}
\TeXtoTEIPat{\textomega}{ω}

%new environments
\TeXtoTEIPat{\begin {postmula}[#1]}{<note type="postmula" target="##1">}
  \TeXtoTEIPat{\end {postmula}}{</note>}
\TeXtoTEIPat{\begin {altava}[#1]}{<div type="altrec"><note type="avataranika" target="##1">} %%% changed 2023-12-05 mm
  \TeXtoTEIPat{\end {altava}}{</note></div>} %%% changed 2023-12-05 mm
\TeXtoTEIPat{\sgwit {#1}}{<note type="inlineref"><ref>#1</ref></note>}

% changes/additions 2023-10-12 MM:
\TeXtoTEIPat{\\.}{}

% changes/additions 2023-08-15 MD:
\TeXtoTEIPat{\lineom {#1}{#2}}{<note type="omission">#1 omitted in <ref>#2</ref></note>}
\TeXtoTEI{graus}{hi}[rend="grey"]
\TeXtoTEIPat{\startgray}{} %%% changed 2023-12-05 mm
\TeXtoTEIPat{\endgray}{} %%% changed 2023-12-05 mm



% additions/changes 2023-06-05 mm:
%\TeXtoTEIPat{\lineom {#1}}{<note type="omission">Line omitted in <ref>#1</ref></note>}
\TeXtoTEIPat{\NotIn {#1}}{<note type="omission">Stanza omitted in <ref>#1</ref></note>}

% additions 2023-04-16 MD:
\TeXtoTEIPat{\,}{ }

% additions 2023-04-13 mm:
\TeXtoTEIPat{\begin {versinnote}}{<lg>}
  \TeXtoTEIPat{\end {versinnote}}{</lg>}

% additions 2023-04-05 MD:
\TeXtoTEIPat{\begin {testimonia}[#1]}{<note type="testimonia" target="##1">}
  \TeXtoTEIPat{\end {testimonia}}{</note>}
\TeXtoTEI{devnote}{s}[xml:lang="sa-deva"]

% app in philcomm und testimonia %%% added MM 2023-12-02
\TeXtoTEI{var}{note}[type="appinnote"]


\TeXtoTEI{anm}{note}[type="memo"] %% change 2023-04-16 MD
\TeXtoTEI{Anm}{note}[type="memo"] %% change 2023-12-05 MM
\TeXtoTEIPat{\startverse}{} %%% marked for change 2023-04-13 mm
\TeXtoTEIPat{\endverse}{} %%% marked for change 2023-04-13 mm
\TeXtoTEIPat{\newpage}{}
\TeXtoTEIPat{\marma}{}
\TeXtoTEIPat{\marmas}{}
\TeXtoTEIPat{\vin}{} % added by MD 2023-11-14

%%% modify environments and commands
%%% TEI mapping
% additions/changes 2022-06-07 mm:
\TeXtoTEI{grau}{hi}[rend="grey"]
\TeXtoTEIPat{ \& }{ &amp; }

% additions/changes 2022-06-01 mm:
\TeXtoTEI{skp}{seg}[type="deva-ignore"]
\TeXtoTEI{skm}{seg}[type="ltn-ignore"]

\TeXtoTEIPat{\rlap {#1}}{#1}

% additions/changes 2022-04-06 mm:
%\TeXtoTEI{sgwit}{ref}
\TeXtoTEI{textdev}{s}[xml:lang="sa-deva"]
\TeXtoTEIPat{\begin {col}[#1]}{<div type="colophon" xml:id="#1"><p>}
  \TeXtoTEIPat{\end {col}}{</p></div>}
\TeXtoTEIPat{\begin {ava}[#1]}{<note type="avataranika" target="##1">}
  \TeXtoTEIPat{\end {ava}}{</note>}
												   
\TeXtoTEIPat{\outdent}{}
\TeXtoTEIPat{\startaltrecension}{} %%% changed 2023-12-05 mm
\TeXtoTEIPat{\endaltrecension}{} %%% changed 2023-12-05 mm
\TeXtoTEIPat{\startaltnormal}{} % added by MD 2023-11-14 %%% changed 2023-12-05 mm
\TeXtoTEIPat{\endaltnormal}{} % added by MD 2023-11-14 %%% changed 2023-12-05 mm
\TeXtoTEIPat{\begin {alttlg}[#1]}{<div type="altrec"><lg xml:id="#1">}
  \TeXtoTEIPat{\end {alttlg}}{</lg></div>}



% additions/changes 2022-03-12 mm:
\TeXtoTEIPat{\begin {tlg}[#1]}{<lg xml:id="#1">}
  \TeXtoTEIPat{\end {tlg}}{</lg>}

\TeXtoTEIPat{\begin {translation}[#1]}{<note type="translation" target="##1">}
  \TeXtoTEIPat{\end {translation}}{</note>}
\TeXtoTEIPat{\begin {philcomm}[#1]}{<note type="philcomm" target="##1">}
  \TeXtoTEIPat{\end {philcomm}}{</note>}
\TeXtoTEIPat{\begin {sources}[#1]}{<note type="sources" target="##1">}
  \TeXtoTEIPat{\end {sources}}{</note>}


\TeXtoTEIPat{\begin {marma}[#1]}{<note type="marma" target="##1">}
  \TeXtoTEIPat{\end {marma}}{</note>}

\TeXtoTEIPat{\begin {jyotsna}[#1]}{<note type="jyotsna" target="##1">}
  \TeXtoTEIPat{\end {jyotsna}}{</note>}

\EnvtoTEI{description}{list}
\EnvtoTEI{itemize}{list}
\TeXtoTEIPat{\item [#1]}{<label>#1</label>\item}

\TeXtoTEI{tl}{l}
\TeXtoTEI{myfn}{note}[type="myfn"]
\TeXtoTEIPat{\getsiglum {#1}}{<ref target="##1"/>}

\TeXtoTEI{SetLineation}{}
\TeXtoTEI{noindent}{}
\TeXtoTEI{subsection*}{}

\TeXtoTEI{rlap}{}

% end additions/changes
% \TeXtoTEIPat{\skp {#1}}{#1}
% \TeXtoTEIPat{\skm {#1}}{}

\TeXtoTEIPat{\begin {prose}}{<p>}
  \TeXtoTEIPat{\end {prose}}{</p>}

\TeXtoTEIPat{\begin {tlate}}{<p>}
  \TeXtoTEIPat{\end {tlate}}{</p>}

\TeXtoTEI{emph}{hi}
\TeXtoTEI{bigskip}{}
% \TeXtoTEI{/}{|}
\TeXtoTEI{tl}{l}
\TeXtoTEIPat{english}{}
%\TeXtoTEIPat{-}{ } %% change 2023-04-16 MD
%\TeXtoTEIPat{°}{} %% change 2023-04-16 MD
\TeXtoTEIPat{\textcolor {#1}{#2}}{<hi rend="#1">#2</hi>}

% \TeXtoTEIPat{\eyeskip}{}
% \TeXtoTEIPat{\aberratio}{}
% \TeXtoTEIPat{\ad}{}
\TeXtoTEIPat{\add}{<hi rend="italic">add.</hi>} %% change 2023-04-16 MD
% \TeXtoTEIPat{\ann}{}
\TeXtoTEIPat{\ante}{<hi rend="italic">ante</hi> } %% change 2023-04-16 MD
\TeXtoTEIPat{\post}{<hi rend="italic">post</hi> } %% change 2023-04-16 MD
% \TeXtoTEIPat{\codd}{}
% \TeXtoTEIPat{\conj }{}
% \TeXtoTEIPat{\contin}{}
% \TeXtoTEIPat{\corr}{}
% \TeXtoTEIPat{\del}{}
% \TeXtoTEIPat{\dub}{}
% \TeXtoTEIPat{\emend }{}
% \TeXtoTEIPat{\expl}{}
% \TeXtoTEIPat{\ȩxplicat}{}
% \TeXtoTEIPat{\fol}{}
% \TeXtoTEIPat{\gloss}{}
% \TeXtoTEIPat{\ins}{}
% \TeXtoTEIPat{\im}{}
% \TeXtoTEIPat{\inmargine}{}
% \TeXtoTEIPat{\intextu}{}
% \TeXtoTEIPat{\indist}{}
% \TeXtoTEIPat{\iteravit}{}
% \TeXtoTEIPat{\lectio}{}
% \TeXtoTEIPat{\leginequit}{}
% \TeXtoTEIPat{\legn}{}
% \TeXtoTEIPat{\illeg}{<hi rend="italic">illeg.</hi>}
\TeXtoTEIPat{\illeg}{<gap reason="illeg."/>} %%% change 2023-04-11 mm
% \TeXtoTEIPat{\om}{<hi rend="italic">om.</hi>}
\TeXtoTEIPat{\om}{<gap reason="om."/>} %%% change 2023-04-11 mm
% \TeXtoTEIPat{\primman}{}
% \TeXtoTEIPat{\prob}{}
% \TeXtoTEIPat{\rep}{}
% \TeXtoTEIPat{\sequentia}{}
% \TeXtoTEIPat{\supralineam}{}
% \TeXtoTEIPat{\interlineam}{}
\TeXtoTEIPat{\vl}{<hi rend="italic">v.l.</hi>}
% \TeXtoTEIPat{\vide}{}
% \TeXtoTEIPat{\videtur}{}
% \TeXtoTEIPat{\crux}{}
% \TeXtoTEIPat{\cruxxx}{}
\TeXtoTEIPat{\unm}{<hi rend="italic">unm.</hi>}


% List of Scholars
\DeclareScholar{nos}{nos}[
forename=HPP,
surname=Team]


% Nullify \selectlanguage in TEI as it has been used in
% \DeclareWitness but should be ignored in TEI.
\TeXtoTEI{selectlanguage}{}



\setlength\parindent{1em}
\SetLineation{lineation=none}
\poemlines{0}

\SetHooks{
	lemmastyle=\bfseries,
	refnumstyle=\selectlanguage{english}\color{blue}\bfseries, 
	appfontsize=\footnotesize
	}
\DeclareApparatus{default}[
	lang=english,
	sep = {] },
	delim=\hskip 0.75em,
	%	rule=none,
	]
\DeclareApparatus{anmkg}[
	notelang=english,
	sep = { },
	delim=\texteng{\ \textbullet\ \ },
%	rule=\relax
	rule=\rule{0.15\columnwidth}{0.4pt}
	]

\newcommand{\mydelim}{\xspace\textcolor{violet}{\textbullet}\ \ }
\newcommand{\mylem}[1]{\texteng{\textcolor{violet}{#1}}}
\setlength{\vrightskip}{-15pt}
\setlength{\vgap}{-3em} % default 1.5em
\verselinenumfont{\footnotesize\selectlanguage{english}\normalfont}

\newlength{\myoutdent}\setlength{\myoutdent}{2em}

\DeclareShorthand{emend}{\texteng{\emph{em.}}}{ego}
%\DeclareShorthand{conj}{\texteng{\emph{conj.}}}{ego}

%Define two commands: \skp ("sanskrit plus"), to be ignored by TeX in
%the edition text, but processed in the TEI output. Conversely, \skm
%("sanskrit minus") is to be processed in the edition text, but
%ignored if found in the apparatus criticus and in the TEI output:

\newif\ifinapparatus
\NewDocumentCommand{\skp}{m}{}
\NewDocumentCommand{\skm}{m}{\unless\ifinapparatus#1\fi}

\SetTEIxmlExport{autopar=false}

\newcommand{\versenr}{\ \themyvnum//}

\NewDocumentEnvironment{tlg}{O{}}{
	\def\hpvnum{\texteng{\thepoemline}}
	\markboth{\hpvnum}{\hpvnum}
	\setcounter{myvnum}{\value{poemline}}
	\begin{ekdverse}
	\Large}{\normalsize
	\end{ekdverse}
	%\smallskip
%  \stepcounter{myvnum}
}

\NewDocumentEnvironment{alttlg}{O{}}{
	\setvnum{\hindsection.\arabic{saved@poemline}*\arabic{poemline}}
	\def\hpvnum{\texteng{\hindsection.\arabic{saved@poemline}*\arabic{poemline}}}
	\markboth{\hpvnum}{\hpvnum}
	\setcounter{altvnum}{\value{poemline}}
	\begin{ekdverse}[type=altrecension]
	\color{gray}
	\Large}{\normalsize
	\end{ekdverse}
	%\smallskip
}

\NewDocumentCommand{\tl}{m}{#1}

\NewDocumentEnvironment{ava}{O{}}{
	\setvnum{prescript:}
	\begin{ekdverse}
	\hspace{-\myoutdent}
	\Large}{\normalsize
	\end{ekdverse}
	\smallskip
}

\NewDocumentEnvironment{altava}{O{}}{
	\setvnum{prescript:}
	\begin{ekdverse}[type=altrecension]
	\color{gray}
	\hspace{-\myoutdent}
	\Large}{\normalsize
	\end{ekdverse}
	\smallskip
}   

\NewDocumentEnvironment{postmula}{O{}}{
	\setvnum{postscript:}
	\smallskip
	\begin{ekdverse}
	\hspace{-\myoutdent}
	\Large}{\normalsize
	\end{ekdverse}
}

\NewDocumentEnvironment{altpostmula}{O{}}{
	\setvnum{postscript:}
	\smallskip
	\begin{ekdverse}[type=altrecension]
	\color{gray}
	\hspace{-\myoutdent}
	\Large}{\normalsize
	\end{ekdverse}
}

\NewDocumentEnvironment{col}{O{}}{
	\setvnum{colophon:}
	\medskip
	\begin{ekdverse}%
	\hspace{-2.5em}%
	\Large%
	}{\normalsize
	\end{ekdverse}
	%\smallskip
      }
      
\NewDocumentCommand{\tcommref}{m}{}
\NewDocumentCommand{\ttransref}{m}{}
\NewDocumentCommand{\tnocomm}{}{}


\def\startaltrecension{
	\setcounter{altvnum}{0}
	\stopvline
	\addtocounter{saved@poemline}{-1}
	\renewcommand{\versenr}{\ \themyvnum *{\small \arabic{poemline}}//}
%	\small
	}
	
\def\endaltrecension{
	\addtocounter{saved@poemline}{1}
	\startvline
	\setvnum{\hindsection.\arabic{poemline}}
	\renewcommand{\versenr}{\ \themyvnum//}
%	\normalsize
	}

\def\startaltnormal{
	\startaltrecension
	\setvnum{\hindsection.\arabic{saved@poemline}*\arabic{poemline}}
	}

\def\endaltnormal{\endaltrecension}

%%%%%%

\newcommand{\teionly}[1]{}
\newcommand{\teimute}[1]{#1}
\newcommand{\manuref}[1]{#1}
\newcounter{myvnum}\setcounter{myvnum}{0}
\newcounter{altvnum}\setcounter{altvnum}{0}
\newcounter{mynotenr}\setcounter{mynotenr}{0}
%\newcommand{\myfn}[1]{\footnote{\texteng{#1}}}

\newcommand{\myfn}[1]{%
	\setcounter{ekd@padanum}{0} % um Pāda-Nummer zu unterdrücken
	\stepcounter{mynotenr}%
	\linelabel{note\themynotenr}%
	\note[type=anmkg, labelb={note\themynotenr}]{#1}
	}

% \newcommand{\myfnx}[1]{%
	% \setcounter{ekd@padanum}{0} % um Pāda-Nummer zu unterdrücken
	% \stepcounter{mynotenr}%
	% \linelabel{note\themynotenr}%
	% \note[type=anmkg, labelb={note\themynotenr},num]{#1}
	% }

\renewcommand{\thefootnote}{\texteng{\arabic{footnote}}}
\newcommand{\devnote}[1]{{\small\textdev{#1}}}
\newcommand{\devtext}[1]{{\normalsize\textdev{#1}}}
%\newcommand{\vsn}[1]{{\footnotesize\texteng{#1}}}
\newcommand{\graus}[1]{\small\textcolor{gray}{#1}\normalsize} % partial altrecension
\newcommand{\grau}[1]{\textcolor{gray}{#1}} % partial altrecension
\newcommand{\Anm}[1]{\begin{ekdverse}
	\texteng{\footnotesize (#1)}
	\end{ekdverse}
	}

%\newcommand{\sgwit}[1]{{\footnotesize (\getsiglum{#1})}}
%\newcommand{\NotIn}[1]{\texteng{\footnotesize (om. \getsiglum{#1})}}
%\newcommand{\lineom}[2]{\texteng{\footnotesize (#1 om. \getsiglum{#2})}}
%\newcommand{\anm}[1]{\texteng{\footnotesize [#1]}}
\newcommand{\sgwit}[1]{}% Nur für Online version; Change TEI too!!
%\newcommand{\lineom}[2]{\myfn{#1 om. \getsiglum{#2}}}
\newcommand{\anm}[1]{\myfn{#1}}
%\newcommand{\unavbl}[1]{\marginpar{\scriptsize\texteng{−\,\getsiglum{#1}}}}
%\newcommand{\unavbl}[1]{\myfn{Folio lost in \getsiglum{#1}}}
\newcommand{\textapp}[1]{\texteng{\textsf{#1}}}
\newcommand{\unavbl}{\textapp{folio lost}}
\newcommand{\incl}{\textapp{included in}}
\newcommand{\only}{\textapp{only included in}}
\newcommand{\also}{\textapp{also included in}}
\newcommand{\excl}{\textapp{included in all except}}
\newcommand{\NotIn}{\om}
\newcommand{\expnr}[1]{\textcolor{magenta}{#1}}% X\kern 1pt

\def\om{\texteng{\emph{om.\@}}}% \kern-0.3ex
\def\illeg{\texteng{\emph{illeg.\@}}} 
\def\lost{\texteng{\emph{lost}}} 
\def\lacuna{\texteng{\emph{lac.\@}}}
\def\unm{\texteng{\emph{unm.\ }}}
\def\ante{\texteng{\normalfont\textapp{ante\ }}}
\def\add{\texteng{\normalfont\emph{add.\@}}}
\def\post{\texteng{\normalfont\textapp{post\ }}}
\def\antecorr{\texteng{\textsubscript{ac}}}
\def\postcorr{\texteng{\textsubscript{pc}}}
\def\marmas{\ }%\texteng{\textsuperscript{\#}}\ }
\def\marma{}%\texteng{\textsuperscript{\#}}}
\def\crux{\texteng{\textsuperscript{\textdagger}}}

%%%%%%% Commentary part

\usepackage{catchfilebetweentags}

\NewDocumentEnvironment{translation}{O{}}{%
	\selectlanguage{english}}{%
	\selectlanguage{iast}}
	
\NewDocumentEnvironment{sources}{O{}}{%
	\selectlanguage{english}%
	\begin{description}[leftmargin=1em, 
		topsep=0pt, parsep=0pt, partopsep=0pt,
		listparindent=0pt, labelwidth=1em, labelsep=0pt]
	\item[\ding{118}\ Sources]
	\item %
	}{\end{description}\selectlanguage{iast}}

\NewDocumentEnvironment{testimonia}{O{}}{%
	\selectlanguage{english}%
	\begin{description}[leftmargin=1em,
		topsep=0pt, parsep=0pt, partopsep=0pt,
		listparindent=0pt, labelwidth=1em, labelsep=0pt]
	\item[\ding{118}\ Testimonia]
	\item %
	}{\end{description}\selectlanguage{iast}}
	
\NewDocumentEnvironment{philcomm}{O{}}{%
	\selectlanguage{english}%
	\begin{description}[leftmargin=1em, 
		topsep=0pt, parsep=0pt, partopsep=0pt,
		listparindent=1.5em,
		labelwidth=1em, labelsep=0pt]
	\item[\ding{118}\ Commentary]\ %
	\newline
	}{\end{description}\selectlanguage{iast}}

\newenvironment{variants}{%
	\begin{description}[%
		leftmargin=4em,
		topsep=3.5pt,
		parsep=0pt,
	%	partopsep=0pt,
		listparindent=-1.5em,
		labelwidth=2.5em,
		labelsep=0pt]
	\item\scriptsize}{%
	\end{description}
	}
 
\newenvironment{versinnote}{%
	\setlength{\vindent}{0pt}
%	\poemlines{0}
	\vspace{4pt plus 2pt minus 2pt}
	\begin{ekdverse}
	\linespread{0.9}\normalsize\selectlanguage{iast}}{%
	\linespread{1}\selectlanguage{english}\end{ekdverse}
	\vspace{4pt plus 2pt minus 2pt}
%	\poemlines{1}
	\addtocounter{poemline}{-1}
	}

  \newenvironment{versinnoterm}{%
	\setlength{\vindent}{0pt}
	\vspace{1pt}
	\begin{ekdverse}
		\itshape}{%
		\rmfamily
	\end{ekdverse}
	\vspace{1pt}
	\addtocounter{poemline}{-1}
	}

\newenvironment{appinnote}{% still in use: 1.16, 1.30, 2.50, 2.77, 3.25, 3.34, 3.39*1, 4.9
	\setlength{\vindent}{0pt}
	\begin{ekdverse}
	\scriptsize\selectlanguage{english}}{%
	\selectlanguage{iast}\end{ekdverse}
	\vspace{3pt minus 1pt}
	\addtocounter{poemline}{-1}
}

%\newcommand{\vnumfix}{\addtocounter{poemline}{1}}
%\TeXtoTEIPat{\vnumfix}{}
\newcommand{\labelincomm}{\smallskip\newline\noindent}
%\TeXtoTEIPat{\labelincomm}{<lb/>} % >> PreambleComm.tex
%\newcommand{\tre}{\ }
%\TeXtoTEIPat{\tre}{}
\newcommand{\skx}[2]{#1} % sandhi between pādas
%\TeXtoTEIPat{\skx {#1}{#2}}{#2} % >> PreambleComm.tex
%\TeXtoTEIPat{\commcitecore}{}
%\TeXtoTEIPat{\commcite}{}
%\TeXtoTEIPat{\commciterange}{}
%\TeXtoTEIPat{\altcommcite}{}
%\TeXtoTEIPat{\avacite}{}
%\TeXtoTEIPat{\colcite}{}
%\TeXtoTEIPat{\trcite}{}

%\TeXtoTEIPat{\labelvnum}{}
%\TeXtoTEIPat{\commvnum}{}

\newcommand{\myvspace}{\vspace{-3pt plus 3pt minus 3pt}}
\newcommand{\commlabel}{\hfill\texteng{\raisebox{0pt}{\textbf{[\hindsection.\labelvnum]}}}\hfill}

\newcommand{\comminfn}{%
	\footnotetext{%
	\commlabel
	\ExecuteMetaData[\commfilename]{sc\commvnum}%
	\ExecuteMetaData[\commfilename]{ts\commvnum}%
	\ExecuteMetaData[\commfilename]{cm\commvnum}%
	}}
	
\newcommand{\commcitecore}{%
	\myvspace
	\begin{quote}%
	\ExecuteMetaData[\commfilename]{tr\commvnum}
	\texteng{(\labelvnum)}\comminfn
	\end{quote}}

\def\commfilename{HP\hindsection_comm.tex}
\newcommand{\commcite}{%
	\def\commvnum{\themyvnum}%
	\def\labelvnum{\themyvnum}%
	\commcitecore}

\newcommand{\commciterange}[2]{%
	\def\commvnum{#1}%
	\def\labelvnum{#2}%
	\commcitecore}
	
\newcommand{\altcommcite}{%
	\def\commvnum{\themyvnum-\thealtvnum}%
	\def\labelvnum{\themyvnum*\thealtvnum}%
	\myvspace
	\begin{quote}%
	\textcolor{gray}{%
	\ExecuteMetaData[\commfilename]{tr\commvnum}
	\texteng{(\labelvnum)}}\comminfn
	\end{quote}}

\newcommand{\avacite}[1]{%
	\bigskip%
	\setlength{\parindent}{1em} %\hspace*{0.5em} in HP4X
	\ExecuteMetaData[\commfilename]{tr#1}
	\vspace{-3pt}
	}

\newcommand{\trcite}[1]{
	\myvspace
	\begin{quote}
	\ExecuteMetaData[\commfilename]{tr#1}
	\texteng{(#1)}
	\end{quote}
	}

\newcommand{\alttrcite}{
	\def\commvnum{\themyvnum-\thealtvnum}%
	\def\labelvnum{\themyvnum*\thealtvnum}%
	\myvspace
	\begin{quote}
	\textcolor{gray}{\ExecuteMetaData[\commfilename]{tr\commvnum}
	\texteng{(\labelvnum)}}
	\end{quote}
	}

\newcommand{\colcite}{
	\medskip
	\noindent
	\ExecuteMetaData[\commfilename]{trcol}
	}


\newcommand{\closer}{\vspace{-1ex}}
\newcommand{\lb}{\par}
\newcommand{\mylb}{\smallskip\lb}
\newcommand{\sep}{\par}
% \TeXtoTEIPat{\sep}{<lb/>}% oder besser mit einem Trennzeichen in einer Zeile lassen?
\def\vl{\textit{v.l.}\xspace}
%\newcommand{\var}[1]{\texteng{\scriptsize #1}}
%\newcommand{\varsep}{\xspace\texteng{\textbullet}\xspace}

\def\sl#1{\emph{#1}}
\newcommand{\medialink}[2]{\textcolor{violet}{\underline{#1}}}
%\TeXtoTEIPat{\medialink {#1}{#2}}{<ref target="/images/#2">#1</ref>}
\usepackage{url}

\newcommand{\alphaOne}{\textalpha\textsubscript{1}}% N3
\newcommand{\alphaTwo}{\textalpha\textsubscript{2}}% J5
\newcommand{\alphaThree}{\textalpha\textsubscript{3}}% G4
\newcommand{\gammaOne}{\textgamma\textsubscript{1}}% N23
\newcommand{\gammaTwo}{\textgamma\textsubscript{2}}% J7
\newcommand{\deltaOne}{\textdelta\textsubscript{1}}% V19
\newcommand{\deltaTwo}{\textdelta\textsubscript{2}}% K3
\newcommand{\deltaThree}{\textdelta\textsubscript{3}}% C7
\newcommand{\deltaOmega}{\textdelta\textsubscript{\textomega}}% J6
\newcommand{\epsilonOne}{\textepsilon\textsubscript{1}}% G11
\newcommand{\epsilonTwo}{\textepsilon\textsubscript{2}}% G5
\newcommand{\zetaOne}{\textzeta\textsubscript{1}}% P15
\newcommand{\zetaTwo}{\textzeta\textsubscript{2}}% N19
\newcommand{\zetaThree}{\textzeta\textsubscript{3}}% V15
\newcommand{\zetaFour}{\textzeta\textsubscript{4}}% J11
\newcommand{\zetaOmega}{\textzeta\textsubscript{\textomega}}% N26
\newcommand{\etaOne}{\texteta\textsubscript{1}}% V1
\newcommand{\etaTwo}{\texteta\textsubscript{2}}% J10
\newcommand{\etaOmega}{\texteta\textsubscript{\textomega}}% E4
\newcommand{\piOne}{\textpi\textsubscript{1}}% P11
\newcommand{\piTwo}{\textpi\textsubscript{2}}% C6
\newcommand{\piOmega}{\textpi\textsubscript{\textomega}}% V3

\def\attr{\hbox{attrib.}\xspace}
\babelhyphenation{%
	Dattā-treya-yoga-śāstra
	Gorakṣa-śataka
	Go-rakṣa-nātha
	Haṭha-pra-dī-pikā
	Haṭha-ratnā-valī
	Haṭha-tattva-kaumudī
	Jāran-dhara
	Rāja-yoga
	Śām-bhavī
	Śāṃ-bhavī
	Śārṅga-dhara-pad-dhati
	Svātmā-rāma
	Śiva-saṃhitā
	Vasiṣṭha-saṃhitā
	Viveka-mārtaṇḍa
	Yukta-bhava-deva
	Yoga-cintā-maṇi
	Yoga-tattva-pra-kāśa
	Yoga-yājña-valkya
	}

\def\hindsection{Kj}
\def\commfilename{Kalajnana_comm.tex}

\begin{document}
\thispagestyle{firstpage}
\begin{center}
\section*{Kālajñāna, Videhamukti, and Kālavañcana}
\end{center}
\bigskip
\begin{otherlanguage}{iast}
\begin{ekdosis}
% V3 folio 16v, l. 12, pdf 18 (upper half)
% J6 folio  20r l. 1, pdf 39.
% N9 folio 20r, l. 1, pdf 21 lower half (A62/33)


\begin{ava}[Kj01a]
atha kālajñānam/
\end{ava}

\teimute{\vspace{-1ex}}
\avacite{1a}
\bigskip

\begin{tlg}[Kj01]
\tl{
\pada{ariṣṭāni viśiṣṭāni}
\pada{tāni vakṣyāmi tvaṃ śṛṇu/}\\+}
\tl{
\pada{yeṣā%m
	\app{\lem[wit={J6},alt={ālokanān}]{\skm{m }ālokanā\skp{n}}
		\rdg[wit={E4}]{alokanān}
		\rdg[wit={V3}]{ālokyanāt}}n
	mṛtyuṃ}
\pada{nijaṃ jānāti yogavit//\versenr}\\!}
\end{tlg}
\commcite%\newpage


%==========
\begin{tlg}[Kj02]
\tl{
\pada{\app{\lem[wit={E4,J6}]{devamārgaṃ}
		\rdg[wit={V3}]{devamārgraṃ}}
	dhruvaṃ śukraṃ} % śukra E4
\pada{somacchāyā%m
	\app{\lem[wit={E4,J6pc},alt={arundhatīm}]{\skm{m }arundhatīm}
		\rdg[wit={J6ac}]{aruṃdhatiṃ}
		\rdg[wit={V3}]{aruṃdhatī}
	}/}\\+}
\tl{
\pada{yo na paśyen na jīvet sa}
\pada{\app{\lem[wit={E4,V3,J6ac},alt={naraḥ saṃvatsarāt}]{naraḥ saṃvatsarā\skp{t}}
		\rdg[wit={J6pc}]{naro vatsarāt}
		}t param//\versenr}\\!}
\end{tlg}
\commcite\newpage


%==========
\begin{tlg}[Kj03]
\tl{
\pada{araśmi bimbaṃ sūryasya} % arasmi viṃvaṃ V3
\pada{vahniṃ caivāṃśumālinam/}\\+} % āṃsu V3
\tl{
\pada{dṛṣṭvaikādaśa%
	\app{\lem[resp=emend]{māsebhyo}
		\rdg[wit={E4,V3}]{māsās tu}
		\rdg[wit={J6}]{māsāṃs tu}}}
\pada{naro nordhvaṃ sa jīvati//\versenr}\\!}
\end{tlg}
\commcite%\newpage


%==========
\begin{tlg}[Kj04]
\tl{
\pada{\app{\lem[resp=emend,alt={vamen}]{vame\skp{n}}
		\rdg[wit={E4,V3,J6}]{vātyāṃ}}% +N26
	\app{\lem[wit={E4,V3},alt={mūtra}]{\skm{n}mūtra}
		\rdg[wit={J6}]{mūtraṃ}
	}purīṣaṃ yaḥ}
\pada{suvarṇaṃ \app{\lem[wit={E4pc,J6}]{rajataṃ}
		\rdg[wit={E4ac,V3}]{rajasaṃ}}
	tathā/}\\+}
\tl{
\pada{pratyakṣam athavā svapne}
\pada{jīvitaṃ daśamāsikam//\versenr}\\!}
\end{tlg}
\commcite\newpage


%==========
\begin{tlg}[Kj05]
\tl{
\pada{dṛṣṭvā pretapiśācādīn}
\pada{gandharvanagarāṇi ca/}\\+}
\tl{
\pada{suvarṇa%
	\app{\lem[wit={J6},alt={varṇān}]{varṇā\skp{n}}%
		\rdg[wit={E4,V3}]{varṇāt}}%
	\app{\lem[wit={V3,J6},alt={vṛkṣāṃś}]{\skm{n }vṛkṣāṃ\skp{ś}}
		\rdg[wit={E4}]{vṛkṣāś}}ś ca}
\pada{nava\app{\lem[wit={J6},alt={māsān}]{māsā\skp{n}}
		\rdg[wit={E4,V3}]{māsāt}}n
	sa jīvati//\versenr}\\!}
\end{tlg}
\commcite%\newpage


%==========
\begin{tlg}[Kj06]
\tl{
\pada{sthūlaḥ kṛśaḥ kṛśaḥ sthūlo}
\pada{yo'kasmād eva jāyate/}\\+}
\tl{
\pada{prakṛtyāś ca \app{\lem[resp=emend]{nivarteta}
		\rdg[wit={E4,V3,J6}]{nivartaṃ ca}}}
\pada{tasyāyuś cāṣṭamāsikam//\versenr}\\!}
\end{tlg}
\commcite\newpage


%==========
\begin{tlg}[Kj07]
\tl{
\pada{\app{\lem[wit={N9,J6}]{khaṇḍaṃ}
		\rdg[wit={E4,V3}]{khaṇḍa}}
	yasya padaṃ
	\app{\lem[resp=emend]{pārṣṇau}
		\rdg[wit={E4,V3}]{pārṣṇe}
		\rdg[wit={J6}]{pārṣṇeḥ}
		\rdg[wit={N9}]{pārṣṇi}}}
\pada{pādasyāgre tathā bhavet/}\\+}
\tl{
\pada{pāṃśukardamayor madhye}
\pada{sapta\app{\lem[wit={N9,J6},alt={māsān}]{māsā\skp{n}}
		\rdg[wit={E4,V3}]{māsāt}}n
	sa jīvati//\versenr}\\!}
\end{tlg}
\commcite%\newpage


%==========
\begin{tlg}[Kj08]
\tl{
\pada{kapota\app{\lem[wit={N9,J6}]{gṛdhrau}
		\rdg[wit={E4,V3}]{gṛdhau}}
	\app{\lem[resp=emend]{kākolo}
		\rdg[wit={E4,V3,J6}]{kākolū}}}
\pada{\app{\lem[wit={E4,V3}]{vāyaso}
		\rdg[wit={J6}]{vāyasā}}
	vāpi mūrdhani/}\\+}
\tl{
\pada{\app{\lem[resp=emend]{kravyādo}
		\rdg[wit={E4,V3}]{kṛṣyādo}
		\rdg[wit={J6}]{\textit{om.}}} vā
	\app{\lem[wit={E4}]{khago}
		\rdg[wit={V3}]{svago}
		\rdg[wit={J6}]{\textit{om.}}}
	\app{\lem[resp=emend]{līnaḥ}
		\rdg[wit={E4,V3}]{līnaṃ}
		\rdg[wit={J6}]{\textit{om.}}}\myfn{\getsiglum{J6} omits \emph{pāda} 8c. As a correction \textit{spṛśanti yaṃ prātar eva} is added in the bottom margin.}}
\pada{ṣaṇmāsāyuḥpradarśakaḥ//\versenr}\\!}
\end{tlg}
\commcite\newpage


%==========
\begin{tlg}[Kj09]
\tl{
\pada{hanyate % haṃnyate E4
	kāka\app{\lem[wit={J6}]{śreṇībhiḥ}
		\rdg[wit={E4,V3}]{śreṇibhiḥ}}}
\pada{pāṃśu\app{\lem[wit={E4}]{varṣena}
		\rdg[wit={V3,J6}]{varṇena}}
	vā naraḥ/}\\+}
\tl{
\pada{svacchāyāṃ vānyathā dṛṣṭvā}
\pada{\app{\lem[wit={J6},alt={caturmāsān}]{caturmāsā\skp{n}}
		\rdg[wit={E4,V3}]{caturmāsāt}}n sa jīvati//\versenr}\\!}
\end{tlg}
\commcite%\newpage


%==========
\begin{tlg}[Kj10]
\tl{
\pada{\app{\lem[wit={E4}]{anabhre}
		\rdg[wit={V3}]{anabhro}
		\rdg[wit={J6}]{anabhrāṃ}}
	vidyutaṃ dṛṣṭvā}
\pada{dakṣiṇāṃ \app{\lem[resp=emend]{diśam āśritām}
		\rdg[wit={E4,V3}]{diśim āśritāṃ}
		\rdg[wit={J6}]{diśi āśritāḥ}}/}\\+}
\tl{
\pada{payasīndradhanu%r
	\app{\lem[wit={E4,J6},alt={vāpi}]{\skm{r }vāpi}
		\rdg[wit={V3}]{vāpiṃ}}}
\pada{jīvitaṃ dvitrimāsikam//\versenr}\\!}
\end{tlg}
\commcite\newpage


%==========

\begin{tlg}[Kj11]
\tl{
\pada{ghṛte taile
	tathā\app{\lem[wit={E4,V3,J6pc}]{darśe}
		\rdg[wit={J6ac}]{darthe}}}
\pada{toye \app{\lem[wit={J6},alt={vānātmanas}]{vānātmana\skp{s}}
		\rdg[wit={E4,V3}]{cānātmanas}}s tanum/}\\+}
\tl{
\pada{yaḥ paśyed aśiraskāṃ
	\app{\lem[wit={J6}]{ca}
		\rdg[wit={E4,V3}]{ti}}}
\pada{māsād ūrdhvaṃ na jīvati//\versenr}\\!}
\end{tlg}
\commcite%\newpage


%==========
\begin{tlg}[Kj12]
\tl{
\pada{\app{\lem[wit={V3,J6}]{yasyāsthi}
		\rdg[wit={E4}]{yasyāṃsthi}}sadṛśo gandho}
\pada{gātre śavasamo'pi vā/}\\+}
\tl{
\pada{tasyārdhamāsikaṃ jñeyaṃ}
\pada{yogino nṛpa \app{\lem[resp=emend]{jīvitam}
		\rdg[wit={E4,V3,J6}]{jīvati}}//\versenr}\\!}
\end{tlg}
\commcite\newpage


%==========
\begin{tlg}[Kj13]
\tl{
\pada{yasya vai \app{\lem[wit={E4,J6}]{snāta}%
		\rdg[wit={V3}]{śnāta}}mātrasya}
\pada{hṛtpādam avaśuṣyati/}\\+}
\tl{
\pada{pibataś ca jalaṃ śoṣo}
\pada{daśāhaṃ so'pi jīvati//\versenr}\\!}
\end{tlg}
\commcite%\newpage


%==========
\begin{tlg}[Kj14]
\tl{
\pada{sambhinno māruto yasya}
\pada{marmasthānāni kṛntati/}\\+}
\tl{
\pada{na \app{\lem[wit={E4,V3},alt={hṛṣyaty}]{hṛṣya\skp{ty}}
		\rdg[wit={J6}]{harṣaty}}ty
	ambu\app{\lem[resp=emend,alt={saṃsparśāt}]{saṃsparśā\skp{t}}% = V17,N26
		\rdg[wit={N9}]{saṃsparśā\,|}
		\rdg[wit={E4ac,V3}]{saṃspaśa}
		\rdg[wit={E4pc}]{saṃspṛśa}
		\rdg[wit={J6}]{saṃsparśe}}}%
\pada{\app{\lem[wit={J6},alt={tasya}]{\skm{t }tasya}
		\rdg[wit={V3}]{tisya}
		\rdg[wit={E4}]{°ti <<ta>>sya}}
	mṛtyur upasthitaḥ//\versenr}\\!}
\end{tlg}
\commcite\newpage


%==========
\begin{tlg}[Kj15]
\tl{
\pada{ṛkṣavānara\app{\lem[resp=emend]{yugya}
		\rdg[wit={E4,V3,J6}]{yugma}}%
	\app{\lem[wit={E4,V3}]{stho}
		\rdg[wit={J6}]{dhyo}}}
\pada{gāyan yo dakṣiṇāṃ diśam/}\\+}
\tl{
\pada{svapne prayāti tasyāpi}
\pada{mṛtyukāla upasthitaḥ//\versenr}\\!}
\end{tlg}
\commcite%\newpage


%==========
\begin{tlg}[Kj16]
\tl{
\pada{raktakṛṣṇāmbaradharā}
\pada{\app{\lem[resp=emend]{gāyantī ca} % =N26
		\rdg[wit={E4,V3,J6}]{gāyanti ca}}
	\app{\lem[resp=emend]{hasanty api}% =N26
		\rdg[wit={E4,V3,J6}]{hasanti ca}}/}\\+}
\tl{
\pada{\app{\lem[wit={J6}]{dakṣiṇāśāṃ}
		\rdg[wit={E4,V3}]{dakṣaṇāśāṃ}}
	\app{\lem[wit={E4,J6},alt={nayen}]{naye\skp{n}}
		\rdg[wit={V3}]{mayen}}n nārī}
\pada{svapne \app{\lem[wit={E4}]{so'pi na jīvati}
		\rdg[wit={V3,J6},alt={\om}]{\skp{\om}}
		}//\versenr}\\!}
\end{tlg}
\commcite\newpage


%==========

\begin{tlg}[Kj17]
\tl{
\pada{\app{\lem[nolem]{}
	\rdg[wit={V3,J6}]{\om}}%
	nagnaṃ kṣapaṇakaṃ svapne}
\pada{\app{\lem[wit={E4,V3}]{hasantaṃ}
		\rdg[wit={J6}]{hasati}}
	\app{\lem[resp=emend]{nṛtyatatparam}
		\rdg[wit={E4,V3},postwit=\texteng{(corrected to nṛtyatī in \getsiglum{V3})}]{nṛtyatatparāṃ}
		\rdg[wit={J6}]{nṛtyati}
		%\rdg[wit={V3}]{nṛtyatī}
		}/}\\+}
\tl{
\pada{\app{\lem[nolem]{}
	\rdg[wit={V3,J6}]{\om}}%
	\app{\lem[resp=emend]{ekaṃ} 
        \rdg[wit={E4}]{evaṃ}} 
    \app{\lem[resp=emend]{saṃvīkṣya}
		\rdg[wit={E4}]{vekṣa}}
	\app{\lem[resp=emend]{valgantaṃ}
		\rdg[wit={E4}]{valāṃtaṃ ca}}}
\pada{\app{\lem[nolem]{}
	\rdg[wit={V3,J6}]{\om}}%
	\app{\lem[resp=emend,alt={vidyān}]{vidyā\skp{n}}
		\rdg[wit={E4}]{viṃdyā}}%
	\app{\lem[resp=emend,alt={mṛtyum upasthitam}]{\skm{n }mṛtyum upasthitam}
		\rdg[wit={E4}]{mṛtyu upasthitāṃ}
		}//\versenr}\\!}
\end{tlg}
\commcite%\newpage


% V3  omits nagna ... upasthitaṃ because of eye skip svapne hasantaṃ nṛtya*tatparā* (deleted)
% J6 omits the whole verse (suggesting J6 descends from V3)
% E4 has this verse.

%==========
\begin{tlg}[Kj18]
\tl{
\pada{ā mastakatalād yas tu}
\pada{\app{\lem[wit={E4,V3}]{nimagnaṃ}
		\rdg[wit={J6}]{nimagnaḥ}}
	paṅkasāgare/}\\+}
\tl{
\pada{svapne \app{\lem[wit={J6}]{paśyaty athātmānaṃ}
		\rdg[wit={E4,V3}]{paśyan yathātmānaṃ}}}
\pada{yaḥ sadyo \app{\lem[wit={J6}]{mriyate}
		\rdg[wit={V3}]{mriyante}
		\rdg[wit={E4}]{mṛyate}} ca saḥ//\versenr}\\!}
\end{tlg}
\commcite\newpage


%==========
\begin{tlg}[Kj19]
\tl{
\pada{\app{\lem[wit={N9,J6pc},alt={keśāṅgārāṃs}]{keśāṅgārāṃ\skp{s}}
		\rdg[wit={E4,V3},alt={°rās}]{keśāṅgārās}
		\rdg[wit={J6ac},alt={°rāṃ}]{keśāṅgārāṃ}}s
	tathā \app{\lem[wit={N9,J6}]{bhasma}%
		\rdg[wit={E4,V3}]{bhasmā}}}%
\pada{\app{\lem[wit={N9,J6},alt={bhujaṅgān}]{bhujaṅgā\skp{n}}
		\rdg[wit={E4,V3}]{bhujaṅgā}}%
	\app{\lem[wit={J6},alt={nirjalāṃ}]{\skm{n }nirjalāṃ}
		\rdg[wit={E4,V3}]{nirjjalā}}
	\app{\lem[wit={J6}]{nadīm}
		\rdg[wit={E4,V3}]{nadī}}/}\\+}
\tl{
\pada{dṛṣṭvā svapne daśāhaṃ tu}
\pada{mṛtyu\app{\lem[wit={N9,J6},alt={ekādaśe}]{\skm{r }ekādaśe}
		\rdg[wit={E4,V3}]{ekādaśā}} dine//\versenr}\\!}
\end{tlg}
\commcite%\newpage


%==========
\begin{tlg}[Kj20]
\tl{
\pada{karālair vikaṭai rūkṣaiḥ}
\pada{puruṣair udyatāyudhaiḥ/}\\+} % V3 yudhai
\tl{
\pada{pāṣāṇais tāḍitaḥ svapne}
\pada{sadyomṛtyur bhave%n 
	\app{\lem[resp=emend,alt={naraḥ}]{\skm{n }naraḥ}
		\rdg[wit={E4,V3,J6}]{nṛṇām}}//\versenr}\\!}
\end{tlg}
\commcite\newpage


%==========
\begin{tlg}[Kj21]
\tl{
\pada{sūryodaye śivā yasya}
\pada{\app{\lem[wit={J6}]{krośantī}
		\rdg[wit={E4,V3}]{krośanti}}
	\app{\lem[wit={E4,V3}]{yāti}
		\rdg[wit={J6}]{yāṃti}} saṃmukham/}\\+}
\tl{
\pada{viparītaṃ parītaṃ vā}
\pada{sadyo\app{\lem[wit={J6}]{mṛtyur upasthitaḥ}
		\rdg[wit={E4,V3}]{mṛtyum upasthitam}}//\versenr}\\!}
\end{tlg}
\commcite%\newpage


%==========
\begin{tlg}[Kj22]
\tl{
\pada{yasya vai bhuktamātrasya}
\pada{hṛdayaṃ pīḍyate kṣudhā/}\\+}
\tl{
\pada{jāyate \app{\lem[wit={N9,J6}]{dantagharṣaś ca}
		\rdg[wit={E4,V3}]{dantasya gharṣaś}}}
\pada{\app{\lem[resp=emend]{sa gatāyur a}%
		\rdg[wit={E4,V3}]{ca gatāyur a}%
		\rdg[wit={J6}]{gatāyur na ca}}saṃśayaḥ//\versenr}\\!}
\end{tlg}
\commcite\newpage


%==========
\begin{tlg}[Kj23]
\tl{
\pada{dīpādigandhaṃ no vetti} % veti E4
\pada{svapne'py ahni tathā niśi/}\\+}
\tl{
\pada{nātmānaṃ paranetrasthaṃ}
\pada{\app{\lem[wit={E4,J6pc}]{vīkṣate}
		\rdg[wit={V3,J6ac}]{vīkṣyate}} na sa jīvati//\versenr}\\!}
\end{tlg}
\commcite%\newpage


%==========
\begin{tlg}[Kj24]
\tl{
\pada{śakrāyudhaṃ cārdharātre}
\pada{divā \app{\lem[wit={J6}]{grahagaṇaṃ}
		\rdg[wit={E4pc,V3}]{grahaggaṇaṃ}
		\rdg[wit={E4ac}]{graharggaṇaṃ}} tathā/}\\+}
\tl{
\pada{dṛṣṭvā manyeta saṃkṣīṇam}
\pada{ātmajīvitam ātmavān//\versenr}\\!}
\end{tlg}
\commcite%\newpage


%==========
\begin{tlg}[Kj25]
\tl{
\pada{nāsikā % nāśikā E4
	\app{\lem[wit={V3,J6},alt={vakratām}]{vakratā\skp{m}}
		\rdg[wit={E4}]{vaktratām}}m eti}
\pada{karṇayo%r
	\app{\lem[resp=emend,alt={namanonnatī}]{\skm{r }namanonnatī}
		\rdg[wit={E4,V3,J6}]{namanonnatā}}/}\\+}
\tl{
\pada{netraṃ ca vāmaṃ śravati}
\pada{yasya tasyāyur udgatam//\versenr}\\!}
\end{tlg}
\commcite\newpage


%==========
\begin{tlg}[Kj26]
\tl{
\pada{āraktatām eti mukhaṃ}
\pada{jihvā \app{\lem[resp=emend]{cāpy asitā}
		\rdg[wit={E4,V3,J6}]{cāsya sitā}} yadā/}\\+}
\tl{
\pada{tadā prājño vijānīyān}
\pada{mṛtyum ātmānam āgatam//\versenr}\\!}
\end{tlg}
\commcite%\newpage


%==========
\begin{tlg}[Kj27]
\tl{
\pada{yasya \app{\lem[wit={E4,J6}]{kṛṣṇā}
		\rdg[wit={V3}]{kṛṣṇāṃ}}
	\app{\lem[resp=emend]{kharā}
		\rdg[wit={E4,V3,J6}]{parā}} jihvā}
\pada{padmākāraṃ ca vai mukham/}\\+}
\tl{
\pada{\app{\lem[wit={J6}]{gaṇḍe}
		\rdg[wit={E4,V3}]{gaṇḍaṃ}} vā piṇḍikā raktā}
\pada{tad antaṃ tasya jīvitam//\versenr}\\!}
\end{tlg}
\commcite\newpage


%==========
\begin{tlg}[Kj28]
\tl{
\pada{jihvā mūle bhavet sthūlā}
\pada{\app{\lem[wit={E4}]{romodvṛtti}%
		\rdg[wit={V3}]{romahati}%
		\rdg[wit={J6}]{romahaṃti}}% °oddhṛti J15, °oddhati N26, romāhati V17
	samudgame/}\\+}
\tl{
\pada{maṇi%
	\app{\lem[wit={E4,J6}]{bandhaṃ}
		\rdg[wit={V3}]{bandha}}
	\app{\lem[wit={J6}]{vīkṣya}
		\rdg[wit={V3}]{vīkṣa}
		\rdg[wit={E4}]{vīkṣyate}} sthūlaṃ}
\pada{\app{\lem[wit={J6}]{mriyate}
		\rdg[wit={E4,V3}]{mṛyate}}
	sārdhavarṣataḥ//\versenr}\\!}
\end{tlg}
\commcite\newpage


%==========
\begin{tlg}[Kj29]
\tl{
\pada{śruti\app{\lem[wit={N9}]{dhvaṃsaṃ} % śrutidhvaṃsaṃ ve yas tu N9
		\rdg[wit={E4}]{dhvaṃ}% baṃdhvaṃ J15
		\rdg[wit={V3,J6}]{pathaṃ}% +N26; paṃthaṃ V17
		}
	vahed yas tu}
\pada{\app{\lem[wit={N9},alt={saptāhair}]{saptāhai\skp{r}}
		\rdg[wit={E4,V3,J6}]{staptāhair}}r gandhanāśanam/}\\+} %V3 ggandhanāsanam
\tl{
\pada{kṛṣṇatvaṃ dantajihvāyāṃ}
\pada{tripañcāhne dhruvaṃ  %J6 pañcāhe
	\app{\lem[wit={J6}]{mriyet}
		\rdg[wit={E4,V3}]{mṛyet}}//\versenr}\\!}
\end{tlg}
\commcite%\newpage


%==========
\begin{tlg}[Kj30]
\tl{
\pada{\app{\lem[wit={V3,J6}]{uṣṭra}
		\rdg[wit={E4}]{uṣṭrā}}%
	rāsabhayānena yaḥ}
\pada{svapne dakṣiṇāṃ diśam/}\\+}
\tl{
\pada{prayāti taṃ vijānīyāt}
\pada{sadyo\app{\lem[resp=emend]{mṛtyuṃ nareśvara}
		\rdg[wit={E4,V3,J6}]{mṛtyur bhaven nṛṇām}}//\versenr}\\!}
\end{tlg}
\commcite\newpage


%==========
\begin{tlg}[Kj31]
\tl{
\pada{\app{\lem[wit={E4,V3}]{pidhāya}
		\rdg[wit={J6}]{vidhāya}} karṇau nirghoṣaṃ}
\pada{na śṛṇoty ātmasambhavam/}\\+} % sṛṇoty E4
\tl{
\pada{na paśyec cakṣuṣor jyotir}
\pada{yaś ca so'pi na jīvati//\versenr}\\!}
\end{tlg}
\commcite%\newpage


%==========
\begin{tlg}[Kj32]
\tl{
\pada{\app{\lem[wit={J6}]{patato}
		\rdg[wit={E4,V3}]{patito}} yasya vai garte}
\pada{svapne dvāraṃ pidhīyate/}\\+}
\tl{
\pada{na \app{\lem[wit={J6}]{cottiṣṭhati}
		\rdg[wit={E4,V3}]{cotiṣṭhati}} yaḥ 
		\app{\lem[resp=emend,alt={śvabhrāt}]{śvabhrā\skp{t}}
		\rdg[wit={E4,V3,J6}]{svapnāt}}t}
\pada{tad antaṃ tasya jīvitam//\versenr}\\!}
\end{tlg}
\commcite\newpage


%==========
\begin{tlg}[Kj33]
\tl{
\pada{\app{\lem[wit={E4,V3}]{ūrdhvā}
		\rdg[wit={J6}]{ūrdhvaṃ}} ca dṛṣṭir na ca % nna E4,V3
	\app{\lem[resp=emend]{saṃpratiṣṭhā}
		\rdg[wit={E4,V3,J6}]{sampraviṣṭā}}}\\+}
\tl{
\pada{raktā punaḥ saṃparivartamānā/}\\+}
\tl{
\pada{mukhasya coṣmā \app{\lem[resp=emend]{śiśirā}
		\rdg[wit={J6pc}]{suṣirā}
		\rdg[wit={E4,V3,J6ac}]{sukhirā}} ca
	\app{\lem[wit={J6}]{nābhiḥ}
		\rdg[wit={E4,V3}]{nābhi}}}\\+}
\tl{
\pada{śaṃsanti puṃsām aparaṃ śarīram//\versenr}\\!}
\end{tlg}
\commcite%\newpage


%==========
\begin{tlg}[Kj34]
\tl{
\pada{svapne'gniṃ \app{\lem[wit={J6},alt={praviśed}]{praviśe\skp{d}}
		\rdg[wit={E4,V3}]{praveśed}}d yas tu}
\pada{na ca niṣkramate punaḥ/}\\+}
\tl{
\pada{jalapraveśād api vā} % V3[f.18r]
\pada{tadantaṃ tasya jīvitam//\versenr}\\!}
\end{tlg}
\commcite\newpage


%==========
\begin{tlg}[Kj35]
\tl{
\pada{yasyāpi hanyate dṛṣṭir}
\pada{bhūtai rātrāv atho divā/}\\+} % rātr<<ā>>v E4
\tl{
\pada{\app{\lem[wit={E4,V3,J6ac}]{sa}
		\rdg[wit={J6pc}]{taṃ}}
	\app{\lem[wit={N9}]{mṛtyuṃ}
		\rdg[wit={E4}]{mṛtyu}
		\rdg[wit={J6}]{mṛtyuṃḥ}
		\rdg[wit={V3}]{mṛtya}}
	\app{\lem[resp=emend]{saptarātrānte}
		\rdg[wit={J6pc,N9}]{saptame rātrau}
		\rdg[wit={E4ac,V3}]{saptame rātrāṃ}
		\rdg[wit={E4pc}]{saptamaṃ rātrāṃ}
		\rdg[wit={J6ac}]{saptame rātryaṃ}
        }} % saptarātryante saṃprāpnoti na saṃśayaḥ J15 (prob. a correction by the scribe)
\pada{\app{\lem[resp=emend,alt={pumān}]{pumā\skp{n}}
		\rdg[wit={E4,V3}]{teṣu}
		\rdg[wit={J6}]{sa ca}
        }n prāpno%ty
	\app{\lem[wit={J6},alt={asaṃśayam}]{\skm{ty }asaṃśayam}
		\rdg[wit={E4,V3}]{asaṃśayaḥ}}//\versenr}\\!}
\end{tlg}
\commcite%\newpage


%==========
\begin{tlg}[Kj36]
\tl{
\pada{svavastram amalaṃ śuklaṃ}
\pada{raktaṃ \app{\lem[resp=emend]{paśyaty athā}
        \rdg[wit={J6}]{paśyan athā}
		\rdg[wit={V3}]{paśyan tathā}
		\rdg[wit={E4}]{paśyet tathā}}sitam/}\\+}
\tl{
\pada{yaḥ pumān mṛtyum āsannaṃ}
\pada{tasyāpi hi vinirdiśet//\versenr}\\!}
\end{tlg}
\commcite\newpage


%==========
\begin{tlg}[Kj37]
\tl{
\pada{svabhāva\app{\lem[resp=emend]{viparītatvaṃ}
		\rdg[wit={E4,V3,J6}]{viparītaṃ ca}}}
\pada{prakṛteś ca \app{\lem[resp=emend]{viparyayaḥ}
		\rdg[wit={E4,V3,J6}]{viparyayam}}/}\\+} % viparjayaṃ E4
\tl{
\pada{kathayanti manuṣyāṇāṃ}
\pada{\app{\lem[wit={J6}]{samāsannau}
		\rdg[wit={E4,V3}]{samāsanau}} yamāntakau//\versenr}\\!}
\end{tlg}
\commcite%\newpage


%==========
\begin{tlg}[Kj38]
\tl{
\pada{lohadaṇḍa\app{\lem[wit={E4,J6}]{dharaṃ}
		\rdg[wit={V3}]{dharā}} hrasvaṃ}
\pada{kṛṣṇavastraparicchadam/}\\+}
\tl{
\pada{svapne prapaśyatas tasya}
\pada{trirātrān maraṇaṃ bhavet//\versenr}\\!}
\end{tlg}
\commcite\newpage


%==========
\begin{tlg}[Kj39]
\tl{
\pada{indriyāṇi na gṛhṇīyuḥ}
\pada{svakīyān viṣayān yadi/}\\+}
\tl{
\pada{māsānte maraṇaṃ tasya}
\pada{bhaviṣyati na saṃśayaḥ//\versenr}\\!}
\end{tlg}
\commcite%\newpage


%==========
\begin{tlg}[Kj40]
\tl{
\pada{darpaṇe
	\app{\lem[resp=emend,alt={svātmanaś}]{svātmana\skp{ś}}
		\rdg[wit={N9,J6}]{svātmanaḥ}
		\rdg[wit={E4,V3}]{svātmana}
		}%
	\app{\lem[resp=emend,alt={chāyām}]{\skm{ś }chāyā\skp{m}}
        \rdg[wit={N9}]{chāyā}% svātmanaḥ chāyā asyaṃ vā N9
		\rdg[wit={E4}]{sāyāṃ}
        \rdg[wit={V3,J6}]{kāyam}% +J15,6chp
		}}% 
\pada{\app{\lem[resp=emend,alt={apsu}]{\skm{m }apsu}
    \rdg[wit={V3,J6,E4}]{āsyaṃ}
    \rdg[wit={N9}]{asyaṃ}} vā yo na paśyati/}\\+}
\tl{
\pada{māsānte maraṇaṃ tasya}
\pada{bhaviṣyati na saṃśayaḥ//\versenr}\\!}
\end{tlg}
\commcite\newpage


%==========
\begin{tlg}[Kj41]
\tl{
\pada{uṣṇaṃ yasya śarīrārdham}
\pada{ardhaṃ cāpi ca śītalam/}\\+}
\tl{
\pada{karṇaśrutivināśo vā} % vināso E4
\pada{saptarātre mariṣyati//\versenr}\\!}
\end{tlg}
\commcite%\newpage


%==========
\begin{tlg}[Kj42]
\tl{
\pada{\app{\lem[wit={J6}]{yogināṃ}
		\rdg[wit={E4,V3}]{yoginā}}
	jñāna\app{\lem[wit={E4,J6},alt={viduṣām}]{viduṣā\skp{m}}
		\rdg[wit={V3}]{vidukhām}}m}%
\pada{anyeṣāṃ ca
	\app{\lem[wit={J6}]{mahātmanām}
		\rdg[wit={E4,V3}]{mahātmanam}}/}\\+}
\tl{
\pada{prāpte'ntakāle 
	\app{\lem[resp=emend,alt={puruṣais}]{puruṣai\skp{s}}
		\rdg[wit={E4,V3,J6}]{puruṣaṃ}}s}% J6pc kāla, puruṣas tu
\pada{tad vijñeyaṃ vicakṣaṇaiḥ//\versenr}\\!}
\end{tlg}
\commcite

\begin{postmula}[Kj42p]
iti kālajñānam//
\end{postmula}

\newpage
%==========
% V3 f. 28r, l. 7
% J6 f. 21v, l. 5, pdf 42.
% N9 (A62/33) f. 21v l. 4, pdf. 23 upper half
% E4 f. 49b4, pdf. 106
\begin{ava}[Kj43a]
atha videhamuktikathanam/
\end{ava}

\teimute{\vspace{-1ex}}
\avacite{43a}
\bigskip\smallskip

\begin{tlg}[Kj43]
\tl{
\pada{pūrvāhne vāparāhne vā}
\pada{madhyāhne vā dine
	\app{\lem[wit={V3,J6}]{kvacit}
		\rdg[wit={E4}]{dyuvit}}/}\\+}
\tl{
\pada{yatra vā
	rajanī\app{\lem[resp=emend]{bhāge}
		\rdg[wit={V3}]{bhavas}
		\rdg[wit={E4,J6}]{bhāvas}}}
\pada{tadāriṣṭaṃ nirīkṣayet//\versenr}\\!}
\end{tlg}

\commcite%\newpage


%==========
\begin{tlg}[Kj44]
\tl{
\pada{viniścityātmanaḥ kālaṃ}
\pada{bāhyābhyantaralakṣaṇaiḥ/}\\+}
\tl{
\pada{nyāsataḥ \app{\lem[wit={E4,V3,J6pc}]{sa pra}%
		\rdg[wit={J6ac}]{sapta}}}{sannātmā}
\pada{\app{\lem[wit={E4,J6}]{nirdvandvo}
		\rdg[wit={V3}]{nirdvando}} vijitendriyaḥ//\versenr}\\!}
\end{tlg}
\commcite\newpage


%==========
\begin{tlg}[Kj45]
\tl{
\pada{\app{\lem[resp=emend]{kurute}
		\rdg[wit={E4,V3,J6}]{kurvanti}}
	yuktakarmāṇi}
\pada{\app{\lem[wit={E4,V3,J6ac}]{nitya}%
		\rdg[wit={J6pc}]{tathā}}%
	\app{\lem[wit={J6}]{naimittikāni}
		\rdg[wit={E4,V3}]{naimityakāni}} ca/}\\+}
\tl{
\pada{yogena paramātmānaṃ}
\pada{guhāyāṃ prāpya \app{\lem[resp=emend]{cetasā}
		\rdg[wit={E4,V3,J6}]{cetasām}}//\versenr}\\!}
\end{tlg}
\commcite%\newpage


%==========
\begin{tlg}[Kj46]
\tl{
\pada{tārakeṇa yajen nityaṃ}
\pada{\app{\lem[wit={J6}]{jitāsuḥ}
		\rdg[wit={E4,V3}]{jitāsu}}
	\app{\lem[wit={J6}]{kāmavarjitaḥ}
		\rdg[wit={E4,V3}]{kāmavarjitam}}/}\\+}
\tl{
\pada{\app{\lem[wit={E4,J6},alt={japec}]{jape\skp{c}}
		\rdg[wit={V3}]{jayec}}c ca
	tārakaṃ brahma }%
\pada{niṣkāmaś cācyutapriyaḥ//\versenr}\\!}
\end{tlg}
\commcite\newpage


%==========
\begin{tlg}[Kj47]
\tl{
\pada{tasya bhāge \app{\lem[resp=emend]{tathaivāhno}
		\rdg[wit={E4}]{tathaivāhne}
		\rdg[wit={V3}]{tathaivahne}
		\rdg[wit={J6}]{tathaiva hi}}}
\pada{\app{\lem[wit={J6}]{yogaṃ}
		\rdg[wit={E4,V3}]{yoga}} yuñjīta yogavit/}\\+}
\tl{
\pada{videhamuktaye jñānī}
\pada{tyaktvā \app{\lem[wit={E4,V3}]{jananajaṃ}
		\rdg[wit={J6}]{ca janajaṃ}} bhayam//\versenr}\\!}
\end{tlg}
\commcite%\newpage


%==========
\begin{tlg}[Kj48]
\tl{
\pada{baddhapadmāsano dhīmān}
\pada{samasaṃsthānakandharaḥ/}\\+}
\tl{
\pada{\app{\lem[wit={J6}]{nirudhya}
		\rdg[wit={E4,V3}]{nirodhya}}
	\app{\lem[wit={J6}]{prāṇāpānau ca}
		\rdg[wit={E4,V3}]{prāṇapavanau}}}
\pada{\app{\lem[wit={E4,J6},alt={dantair}]{dantai\skp{r}}
		\rdg[wit={V3}]{rdantai}}%
	\app{\lem[wit={E4,V3},alt={dantān}]{\skm{r }dantā\skp{n}}
		\rdg[wit={J6}]{dantāṃś ca}}%
	\app{\lem[wit={E4,V3},alt={asaṃspṛśan}]{\skm{n }asaṃspṛśan} % °spṛsan V3
		\rdg[wit={J6}]{na saṃspṛśan}}//\versenr}\\!}
\end{tlg}
\commcite\newpage


%==========
\begin{tlg}[Kj49]
\tl{
\pada{buddhyā nirudhya dvārāṇi} % ddhārāṇi E4
\pada{\app{\lem[resp=emend]{nava}
		\rdg[wit={E4,V3,J6}]{na ca}}
	mīlitalocanaḥ/}\\+}
\tl{
\pada{oṃkāraṃ
	\app{\lem[wit={E4,V3}]{tu}
		\rdg[wit={J6}]{ca}} dhanuḥ kṛtvā}
\pada{guṇaṃ sattvaṃ niyojayet//\versenr}\\!}
\end{tlg}
\commcite%\newpage


%==========
\begin{tlg}[Kj50]
\tl{
\pada{tatrātmānaṃ
	\app{\lem[resp=emend]{śaraṃ so'pi}
		\rdg[wit={E4}]{ra\{\{ṃ\}\} so 'pi}
		\rdg[wit={V3,J6ac}]{ramaṇo 'pi}
		\rdg[wit={J6pc}]{lakṣyayitvā}}} %lakṣayitvā?
\pada{vṛto bhūtendriyādibhiḥ/}\\+}
\tl{
\pada{prāṇavāyumanaḥkṣepaiḥ}
\pada{\app{\lem[resp=emend]{kṣipto hṛtkamalasthitaḥ}
        \rdg[wit={E4}]{sthi kṣipto hṛtkamalasthite}
        \rdg[wit={V3,J6}]{saṃkṣipto hṛtkamalake}
		}//\versenr}\\!}
\end{tlg}
\commcite\newpage


%==========
\begin{tlg}[Kj51]
\tl{
\pada{\app{\lem[wit={N9,J6}]{daśama}%
		\rdg[wit={V3}]{daśaṃma}
		\rdg[wit={E4}]{daśa\{\{ṃ\}\}maṃ}}dvāramārgeṇa}
\pada{\app{\lem[wit={J6}]{lakṣyaṃ}
		\rdg[wit={E4,V3}]{bhakṣyaṃ}}
	prāpya tataḥ param/}\\+}
\tl{
\pada{ṣaṭ\app{\lem[resp=emend]{triṃśattattvasaṃyuktaḥ}
		\rdg[wit={E4,V3}]{triṃśatvāsamayutaḥ}
		\rdg[wit={J6ac}]{triṃśasamayutaś ca}
		\rdg[wit={J6pc}]{triṃśadbhiḥ saṃyutaś ca}}}
\pada{paramātmani līyate//\versenr}\\!}
\end{tlg}
\commcite%\newpage


%==========
\begin{tlg}[Kj52]
\tl{
\pada{tataḥ paramam ākāśam}
\pada{atīndriyam agocaram/}\\+}
\tl{
\pada{\app{\lem[resp=emend]{yad buddhyā}
		\rdg[wit={E4,V3,J6}]{yad buddhir}} naiva
	\app{\lem[resp=emend]{cākhyātuṃ}
		\rdg[wit={E4,V3,J6}]{paśyanti}}}
\pada{\app{\lem[wit={V3,J6}]{śakyate}
		\rdg[wit={E4}]{śakyaṃte}}
	\app{\lem[wit={E4,V3}]{na ca vastu tat}
		\rdg[wit={J6ac}]{na ca vastu taṃ}
		\rdg[wit={J6pc}]{na ca vastutaḥ}}//\versenr}\\!}
\end{tlg}
\commcite\newpage


%==========

\begin{ava}[Kj53a]
atha kālavañcanam/ 
\end{ava}
% V3 f. 28v, l. 3 pdf 20 upper half
% J6 f. 22r, l. 4, pdf 43.
% N9 f. 22r l. 2, pdf. 23 lower half

\teimute{\vspace{-1ex}}
\avacite{53a}
\bigskip\smallskip

\begin{tlg}[Kj53]
\tl{
\pada{jīvanmuktaḥ sadeho'haṃ}
\pada{vicarāmi jagattraye/}\\+} % jagatraye E4
\tl{
\pada{iti \app{\lem[wit={N9}]{sā jāyate}
		\rdg[wit={E4}]{sā yāyate}
		\rdg[wit={V3}]{sa jāyate}
		\rdg[wit={J6}]{saṃjāyate} % +V17,N26
        } vāñchā}
\pada{yoginas ta%n 
	\app{\lem[wit={E4,V3},alt={nibodha me}]{\skm{n }nibodha me}
		\rdg[wit={J6}]{nibodhata}}//\versenr}\\!}
\end{tlg}

\commcite%\newpage


%==========
\begin{tlg}[Kj54]
\tl{
\pada{śarīraṃ na
	\app{\lem[wit={N9}]{tyajaty eva}% +V17, evaṃ N26
		\rdg[wit={E4}]{tyajatyeca}
		\rdg[wit={V3}]{tyajate ca}
		\rdg[wit={J6}]{tyajati ca}
        }}
\pada{\app{\lem[resp=emend]{kālaḥ}
		\rdg[wit={E4,V3}]{kulaṃ}
		\rdg[wit={J6ac}]{kula}
		\rdg[wit={J6pc}]{manaḥ}}
	\app{\lem[wit={E4,V3}]{kasyāpi}
		\rdg[wit={J6}]{tasyāpi}}
	\app{\lem[wit={N9,J6}]{kutra}
		\rdg[wit={E4,V3}]{kvatra}}cit/}\\+}
\tl{
\pada{\app{\lem[wit={J6pc}]{ataḥ}
		\rdg[wit={E4,V3,J6ac}]{antaḥ}% aṃta N9
        }
	\app{\lem[wit={N9,J6}]{śarīra}
		\rdg[wit={E4,V3}]{śarīraṃ}}rakṣārthaṃ}
\pada{yatnaḥ kāryas tu yoginā//\versenr}\\!}
\end{tlg}
\commcite\newpage


%==========
\begin{tlg}[Kj55]
\tl{
\pada{yoginā satataṃ \app{\lem[wit={E4,V3},alt={yatnād}]{yatnā\skp{d}}
		\rdg[wit={J6}]{yatno}}}%
\pada{\app{\lem[wit={J6},alt={ariṣṭānāṃ}]{\skm{d }ariṣṭānāṃ}
		\rdg[wit={E4,V3}]{ariṣṭānaṃ}}
	\app{\lem[resp=emend]{vicāraṇā}
		\rdg[wit={E4,V3,J6}]{vicāraṇāt}}/}\\+}
\tl{
\pada{\app{\lem[resp=emend]{kartavyā}
		\rdg[wit={E4,V3,J6}]{kartavyo}} yena kālo'sau}
\pada{jñāto hanti \app{\lem[wit={E4,V3}]{cchalān na}
		\rdg[wit={J6}]{balān na}} tam//\versenr}\\!}
\end{tlg}
\commcite%\newpage


%==========
\begin{tlg}[Kj56]
\tl{
\pada{jñātvā ca kālaṃ \app{\lem[wit={E4,V3}]{taṃ samyag} % tacit change from -k to -g
		\rdg[wit={J6}]{samyak ca}}}
\pada{layasthānaṃ samāśritaḥ/}\\+}
\tl{
\pada{yuñjīta yogaṃ \app{\lem[resp=emend]{kālo'sya}
		\rdg[wit={E4,V3,J6}]{kālasya}}}
\pada{yathāsau jāyate\app{\lem[wit={J6pc}]{'phalaḥ}
		\rdg[wit={E4,V3,J6ac}]{'phalaṃ}}//\versenr}\\!}
\end{tlg}
\commcite\newpage


%==========
\begin{tlg}[Kj57]
\tl{
\pada{baddhasiddhāsano dehaṃ}
\pada{pūrayet prāṇavāyunā/}\\+}
\tl{
\pada{kṛtvā \app{\lem[wit={J6}]{daṇḍaṃ}
		\rdg[wit={E4,V3}]{daṇḍa}} sthiraṃ buddhyā}
\pada{\app{\lem[wit={E4,V3,J6pc}]{daśa}%
		\rdg[wit={J6ac}]{deśa}} dvārāṇi rundhayet//\versenr}\\!}
\end{tlg}
\commcite%\newpage


%==========
\begin{tlg}[Kj58]
\tl{
\pada{bandhaye%t 
	\app{\lem[wit={V3,J6},alt={khecarī}]{\skm{t }khecarī}
		\rdg[wit={E4}]{khecarīṃ}}mudrāṃ}
\pada{grīvāyāṃ ca jalandharam/}\\+}
\tl{
\pada{\app{\lem[resp=emend]{apāne mūlabandhaṃ}
		\rdg[wit={E4,V3,J6}]{apānaṃ mūlabandhe}} ca}
\pada{\app{\lem[wit={E4,V3}]{uḍḍiyāṇaṃ}
		\rdg[wit={J6}]{uḍḍiyānaṃ}}
	\app{\lem[wit={V3,J6}]{tathodare}
		\rdg[wit={E4}]{tathodaraṃ}}//\versenr}\\!}
\end{tlg}
\commcite\newpage


%==========
\begin{tlg}[Kj59]
\tl{
\pada{utthāpya \app{\lem[resp=emend]{bhujagīṃ}
		\rdg[wit={E4,V3}]{bhujaṃgī}
		\rdg[wit={J6}]{bhujagī}}
	\app{\lem[wit={E4,J6}]{śaktiṃ}
		\rdg[wit={V3}]{śakti}}}
\pada{mūlād ghātair 
	adhaḥ\app{\lem[wit={N9,J6}]{sthitām}
		\rdg[wit={E4,V3}]{sthitā}}/}\\+}
\tl{
\pada{suṣumnāntargatāṃ pañca}%
\pada{cakrāṇāṃ \app{\lem[wit={E4,J6}]{bhedinīṃ}
		\rdg[wit={V3}]{bhedinī}} śivām//\versenr}\\!}
\end{tlg}
\commcite%\newpage


%==========
\begin{tlg}[Kj60]
\tl{
\pada{\app{\lem[resp=emend]{jīvaṃ}
		\rdg[wit={E4,V3,J6}]{bandhaṃ}}
	\app{\lem[wit={E4,V3}]{hṛdyā}%
		\rdg[wit={J6}]{buddhyā}}śrayaṃ nītvā}
\pada{\app{\lem[resp=emend]{yāntīṃ}
		\rdg[wit={E4,V3,J6}]{yānti}}
	\app{\lem[wit={N9}]{buddhi}%
		\rdg[wit={E4,V3,J6}]{buddhiṃ}}%
	\app{\lem[resp=emend]{manoyutām}
		\rdg[wit={E4,V3,J6}]{manojitam}}/}\\+}
\tl{
\pada{sahasradalapadma\app{\lem[resp=emend]{sthe}
		\rdg[wit={E4}]{sthāṃ}
		\rdg[wit={V3,J6pc}]{sthā}
		\rdg[wit={J6ac}]{ścā}}}
\pada{\app{\lem[wit={E4pc}]{śive}
		\rdg[wit={J6}]{śivāṃ}
		\rdg[wit={E4ac,V3}]{śivo}} līnāṃ vicintayet//\versenr}\\!} % J6 f.22v
\end{tlg}
\commcite\newpage


%==========
\begin{tlg}[Kj61]
\tl{
\pada{\app{\lem[resp=emend]{tataḥ sudhākaro}
		\rdg[wit={E4,V3,J6ac}]{aśrudhārākaro}%
		\rdg[wit={J6pc}]{sudhādhārākaro}}%%
	\app{\lem[wit={E4,V3},alt={°dbhūtam}]{\skp{°}dbhūta\skp{m}}
		\rdg[wit={J6}]{dbhūtāṃm}}}%
\pada{\app{\lem[wit={E4,J6},alt={amṛtaṃ}]{\skm{m }amṛtaṃ}
		\rdg[wit={V3}]{amṛta}}
	\app{\lem[wit={E4,V3}]{tena}
		\rdg[wit={J6}]{yena}}
	\app{\lem[resp=emend]{mūlataḥ}
		\rdg[wit={E4,V3}]{mūlitam}
		\rdg[wit={J6}]{mūrcchitam}}/}\\+}
\tl{
\pada{\app{\lem[wit={E4,J6}]{siñcantīṃ}
		\rdg[wit={V3}]{siṃcaṃtā}} sakalaṃ dehaṃ}
\pada{\app{\lem[wit={E4,J6}]{plāvayantīṃ}
		\rdg[wit={V3}]{plāvayaṃtī}} 
	vicintayet//\versenr}\\!}
\end{tlg}
\commcite%\newpage


%==========
\begin{tlg}[Kj62]
\tl{
\pada{tayā sārdhaṃ tato yogī}
\pada{śivenaikātmatāṃ vrajet/}\\+}
\tl{
\pada{parānandamayo bhūtvā}
\pada{cidvṛttim api saṃtyajet//\versenr}\myfn{This verse has been omitted in the collated manuscripts but is found in J\textsubscript{15} of the \getsiglum{E4} group, the 10-chapter \emph{Haṭhapradīpikā} (10.26) and all of the reported testimonia.}\\!}
\end{tlg}
\commcite\newpage



%==========
\begin{tlg}[Kj63]
\tl{
\pada{tato\app{\lem[resp=emend,alt={'lakṣyam anābhāsam}]{'lakṣyam anābhāsa\skp{m}}
		\rdg[wit={E4,V3}]{lakṣamanābhyāsam}
		\rdg[wit={J6}]{lakṣyamano'bhyāsam}}m}
\pada{ahaṃbhāvavi\app{\lem[wit={E4,V3,J6pc}]{varjitam}
		\rdg[wit={J6ac}]{varjitaḥ}}/}\\+}
\tl{
\pada{sarvāṅgakalpanāhīnaṃ}
\pada{kathaṃ kālo nihanti tam//\versenr}\\!}
\end{tlg}
\commcite%\newpage


%==========
\begin{tlg}[Kj64]
\tl{
\pada{sa eva kālaḥ sa śivaḥ}
\pada{sa \app{\lem[resp=emend]{sarvaṃ nāpi}
		\rdg[wit={E4,V3}]{sarve nāpi}
		\rdg[wit={J6}]{sarveṇāpi}}
	kiñ cana/}\\+}
\tl{
\pada{kaḥ kena hanyate tatra}
\pada{mriyate \app{\lem[resp=emend]{nāpi}
		\rdg[wit={E4,V3}]{vāpi}
		\rdg[wit={J6}]{cāpi}}
	\app{\lem[wit={V3,J6}]{kaścana}
		\rdg[wit={E4}]{kaścanaḥ}}//\versenr}\\!}
\end{tlg}
\commcite\newpage


%==========
\begin{tlg}[Kj65]
\tl{
\pada{tato vyatīte samaye} % V3 [f.19r]
\pada{kālasya bhrānti\app{\lem[wit={J6}]{rūpiṇaḥ}
		\rdg[wit={E4,V3}]{rūpiṇā}}/}\\+}
\tl{
\pada{\app{\lem[wit={J6pc}]{yogī}
		\rdg[wit={V3,J6ac}]{yoga}
		\rdg[wit={E4}]{yogena}} suptotthita iva} % V3 [f.19r]
\pada{bodhaṃ yāti \app{\lem[wit={E4,J6}]{prabodhataḥ}
		\rdg[wit={V3}]{prabodhata}}//\versenr}\\!}
\end{tlg}
\commcite%\newpage


%==========
\begin{tlg}[Kj66]
\tl{
\pada{evaṃ siddho bhaved yogī}
\pada{\app{\lem[wit={V3,J6}]{vañcayitvā}
		\rdg[wit={E4}]{vaṃcayatvā}} vidhānataḥ/}\\+}
\tl{
\pada{kālaṃ kalitasaṃsāraṃ}
\pada{pauruṣeṇādbhutena hi//\versenr}\\!}
\end{tlg}
\commcite\newpage


%==========
\begin{tlg}[Kj67]
\tl{
\pada{tatra \app{\lem[wit={E4,J6}]{tribhuvane}
		\rdg[wit={V3}]{tribhavane}} yogī}
\pada{viharaty eka eva saḥ/}\\+}
\tl{
\pada{paśyan saṃsāra\app{\lem[wit={N9,J6}]{vaicitryaṃ}
		\rdg[wit={E4,V3}]{vaicitrīṃ}}}
\pada{svecchayā nirahaṃkṛtiḥ//\versenr}\\!}
\end{tlg}
\commcite%\newpage


%==========
\begin{tlg}[Kj68]
\tl{
\pada{yathārkaraśmisaṃyogā}%d
\pada{\app{\lem[wit={N9},alt={arkakānto}]{\skm{d }arkakānto}% sunstone?
		\rdg[wit={J6pc}]{arkakāco}
		\rdg[wit={E4,V3,J6ac}]{arkakaṇṭho}} % akuṃṭho vai N26
	hutāśanam/}\\+}
\tl{
\pada{āviṣkaroti naikaḥ san} %
\pada{dṛṣṭāntas tu sa yoginaḥ//\versenr}\\!}
\end{tlg}
\commcite%\newpage

\begin{postmula}[Kj68p]
iti kālavañcanam//
\end{postmula}

\avacite{68p}

\end{ekdosis}\end{otherlanguage}\end{document}
