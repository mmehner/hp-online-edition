\documentclass[11pt,twoside]{article}
\usepackage[papersize={17cm,24cm},
	centering, textwidth=12.5cm, textheight=17.9cm,
	asymmetric]{geometry} % so that the orgvnr always appear on the right side
%\linespread{1.1}
\sloppy

\usepackage{xcolor, xparse, xspace, pifont, datetime2}
\usepackage{enumitem}

\newcommand{\HP}{\textit{Haṭha\-pra\-dī\-pikā}\xspace}

\usepackage{fancyhdr}
	\renewcommand{\headrulewidth}{0pt}
	\fancyhead[EL]{\small\texteng{\thepage}}
	\fancyhead[ER]{\small\texteng{Critical Edition \& Translation}}
	\fancyhead[OL]{\small\texteng{HP\hindsection}}
	\fancyhead[OR]{\small\texteng{\thepage}}
	\fancyhead[C]{}
	\fancyfoot[C]{}
%	\pagestyle{fancy}

\fancypagestyle{firstpage}{%
	\fancyhead[OL]{\small\texteng{(\Today)}}
	\fancyhead[OR]{\small\texteng{\thepage}}
	\fancyhead[C]{}
	}

\fancypagestyle{HPed}{%
	\renewcommand{\headrulewidth}{0pt}
	\fancyhead[EL]{\small\texteng{\thepage}}
	\fancyhead[ER]{\small\texteng{[\rightmark\textendash}}
	\fancyhead[OL]{\small\texteng{\textendash\leftmark]}}
	\fancyhead[OR]{\small\texteng{\thepage}}
	\fancyhead[C]{\small\texteng{Critical Edition}}
	\fancyfoot[C]{}
	}
\pagestyle{HPed}

\usepackage{scrextend}
	\deffootnote[]{0em}{1em}{}
%	\deffootnote[2.1em]{0em}{1.5em}{\color{blue}\texteng{\textbf{Verse \thefootnotemark}}}
\usepackage{fnpos}
	\makeFNbottom
%\footnoterulefalse
\setlength{\footnotesep}{1em}
%\dimen\footins=??
%\renewcommand{\footnotesize}{\small}  % Wirkt auf App und Fn beides.

\usepackage[english]{babel}
\usepackage{babel-iast}
\babelfont[iast]{rm}[Renderer=Harfbuzz, Scale=1.1]{AdishilaSan}
\babeltags{dev = iast}
\babeltags{eng = english}
\usepackage{libertine}

\usepackage[teiexport=tidy,poetry=verse]{ekdosis}
\usepackage{sanskrit-poetry}
\usepackage{textgreek}

%%% Gr1,4b,6
\DeclareWitness{N3}{\texteng{\textalpha\textsubscript{1}}}{NGMPP B 62-20}[]
        \DeclareHand{N3ac}{N3}{\texteng{\textalpha\rlap{\textsubscript{1}}\textsuperscript{ac}}}[]
        \DeclareHand{N3pc}{N3}{\texteng{\textalpha\rlap{\textsubscript{1}}\textsuperscript{pc}}}[]
\DeclareWitness{J5}{\texteng{\textalpha\textsubscript{2}}}{Jodhpur 02235}[]
\DeclareWitness{G4}{\texteng{\textalpha\textsubscript{3}}}{GOML 18885}[]% Telugu script
\DeclareWitness{N24}{\texteng{\textalpha\textsubscript{4}}}{NGMPP G 190-16}[]
\DeclareWitness{Gr1r}{\texteng{\textAlpha *}}{Gr1 reconstructed}[]

\DeclareWitness{P11}{\texteng{\textbeta\textsubscript{1}}}{}[]
\DeclareWitness{C6}{\texteng{\textbeta\textsubscript{2}}}{Lalchand M-2089}[]

\DeclareWitness{V3}{\texteng{\textbeta\textsubscript{\textomega}}}{Sampurnananda Library Sarasvati Bhavan 29899}[]

%%% Gr2

\DeclareWitness{N23}{\texteng{\textgamma\textsubscript{1}}}{NGMPP G 25-2}[]
        \DeclareHand{N23ac}{N23}{\texteng{\textgamma\rlap{\textsubscript{1}}\textsuperscript{ac}}}[]
        \DeclareHand{N23pc}{N23}{\texteng{\textgamma\rlap{\textsubscript{1}}\textsuperscript{pc}}}[]
\DeclareWitness{J7}{\texteng{\textgamma\textsubscript{2}}}{Jodhpur 02241}[]
%\DeclareWitness{V6}{\texteng{V\textsubscript{6}}}{Sampurnananda Library Sarasvati Bhavan 29991}[]
\DeclareWitness{K1}{\texteng{K\textsubscript{1}}}{Raghunātha Temple Library 4383}[settlement=Jammu]
        \DeclareWitness{K1ac}{\texteng{K\rlap{\textsubscript{1}}\textsuperscript{ac}\space}}{}[]
        \DeclareWitness{K1pc}{\texteng{K\rlap{\textsubscript{1}}\textsuperscript{pc}\space}}{}[]


%%% Gr3

\DeclareWitness{V19}{\texteng{\textdelta\textsubscript{1}}}{Sampurnananda Library Sarasvati Bhavan 30069}[]
\DeclareWitness{K3}{\texteng{\textdelta\textsubscript{2}}}{Privat collection}
\DeclareWitness{C7}{\texteng{\textdelta\textsubscript{3}}}{Lalchand M-6494}[]
%\DeclareWitness{C1}{\texteng{C\textsubscript{1}}}{Lalchand M-2080}[]%L1 And C1 very close (and come from same region)
%\DeclareWitness{P23}{\texteng{P\textsubscript{23}}}{}[]
%\DeclareWitness{L1}{\texteng{L\textsubscript{1}}}{SOAS RE 43454}[settlement=Jammu]

\DeclareWitness{J6}{\texteng{\textdelta\textsubscript{\textomega}}}{Jodhpur 02237}[]
        \DeclareHand{J6ac}{J6}{\texteng{\textdelta\rlap{\textomega}\textsuperscript{ac}}}[]
        \DeclareHand{J6pc}{J6}{\texteng{\textdelta\rlap{\textomega}\textsuperscript{pc}}}[]

%%% Gr4c

\DeclareWitness{P15}{\texteng{\textepsilon\textsubscript{1}}}{}[]
\DeclareWitness{N19}{\texteng{\textepsilon\textsubscript{2}}}{NGMPP E-1528-1 / E-1527-7(4)}[]
\DeclareWitness{V15}{\texteng{\textepsilon\textsubscript{3}}}{Sampurnananda Library Sarasvati Bhavan 30051}[]
        \DeclareHand{V15ac}{V15}{\texteng{\textepsilon\rlap{\textsubscript{3}}\textsuperscript{ac}}}[]
        \DeclareHand{V15pc}{V15}{\texteng{\textepsilon\rlap{\textsubscript{3}}\textsuperscript{pc}}}[]
\DeclareWitness{J11}{\texteng{\textepsilon\textsubscript{4}}}{Jodhpur 23532}[]
        \DeclareHand{J11ac}{J11}{\texteng{\textepsilon\rlap{\textsubscript{4}}\textsuperscript{i.t.}}}[]
        \DeclareHand{J11pc}{J11}{\texteng{\textepsilon\rlap{\textsubscript{4}}\textsuperscript{mg.}}}[alternative reading written by the first hand in margin or interlinearly (J11)]
%\DeclareWitness{J14}{\texteng{\textepsilon\textsubscript{5}}}{Jodhpur 02239}[]

%\DeclareWitness{L2}{\texteng{L\textsubscript{2}}}{Wellcome Collection O.36]}
\DeclareWitness{M1}{\texteng{M\textsubscript{1}}}{P-5682/4}[]

\DeclareWitness{N26}{\texteng{\textepsilon\textsubscript{\textomega}}}{NGMPP}[]
%\DeclareWitness{V17}{\texteng{\textepsilon\textsubscript{\textomega 3}}}{Sampurnananda Library Sarasvati Bhavan 30053}[]

\DeclareWitness{V1}{\texteng{\texteta\textsubscript{1}}}{Sampurnananda Library Sarasvati Bhavan 30109}[]
        \DeclareHand{V1ac}{V1}{\texteng{\texteta\rlap{\textsubscript{1}}\textsuperscript{ac}}}[]
        \DeclareHand{V1pc}{V1}{\texteng{\texteta\rlap{\textsubscript{1}}\textsuperscript{pc}}}[]

%%% Gr4d

\DeclareWitness{J10}{\texteng{\texteta\textsubscript{2}}}{MSPP Jodhpur 2230}[]
        \DeclareHand{J10ac}{J10}{\texteng{\texteta\rlap{\textsubscript{2}}\textsuperscript{ac}}}[]
        \DeclareHand{J10pc}{J10}{\texteng{\texteta\rlap{\textsubscript{2}}\textsuperscript{pc}}}[]

\DeclareWitness{N9}{\texteng{\texteta\textsubscript{\textomega}}}{NGMPP A62-33}[]
%\DeclareWitness{J15}{\texteng{\textepsilon\textsubscript{\textomega 4}}}{Jodhpur 9732A}[]

%%%

\DeclareWitness{Jyo}{\texteng{\textchi}}{Brahmānanda's version}[]
%\DeclareWitness{Tue}{\texteng{Tü}}{Ma I 339}[]

\DeclareWitness{ceteri}{\texteng{cett.}}{ceteri}[]

%%% Group Sigla

\DeclareWitness{Gr1}{\texteng{\textAlpha}}{N3,J5,G4}

\DeclareWitness{Gr2}{\texteng{\textGamma}}{N23,J7}
%\DeclareWitness{Gr2}{\texteng{%
%	\textbeta\textsubscript{1}%
%	\textbeta\textsubscript{2}%
%	}}{N23,J7}
\DeclareWitness{Gr3a}{\texteng{\textDelta}}{V19,K3,C7}
\DeclareWitness{Gr4b}{\texteng{%
	\textbeta\textsubscript{1}%
	\textbeta\textsubscript{2}%
	}}{C6,P11}
\DeclareWitness{GrB}{\texteng{%
	\textbeta\textsubscript{1}%
	\textbeta\textsubscript{2}%
	\textbeta\textsubscript{\textomega}%
	}}{C6,P11,V3}
\DeclareWitness{Gr4c}{\texteng{\textEpsilon}}{P15,N19,V15}

% \DeclareWitness{Gr4d}{\texteng{%
	% \texteta\textsubscript{1}%
	% \texteta\textsubscript{2}%
	% }}{V1,J10}
\DeclareWitness{Gr6}{\texteng{\textOmega}}{V3,J6,N9,N26}




%%%%%%%%%%%%%%%%%%%% THE  MSS         %%%%%%%%%%%%%%%%%%%%%%%%%%%

%%% Versions
\DeclareWitness{Vu}{\selectlanguage{english}Vulg}{Vulgate, i.e. Brahmānanda's version}[]           
\DeclareWitness{X}{\selectlanguage{english}X}{TenChapter Version, Jodhpur 02228 and 02225 (ed. Lonavla)}[]
\DeclareWitness{Six}{\selectlanguage{english}Ṣ}{SixChapterVersion, ``6ChapterHPms'', fragment of enlarged text, Jodhpur}[]
% Mss. in Geographical Groups
%%%% Varanasi mss (Sampūrṇānanda mss). V1 is Important
\DeclareWitness{V1}{\selectlanguage{english}V\textsubscript{1}}{Sampurnananda Library Sarasvati Bhavan 30109}[]
        \DeclareHand{V1ac}{V1}{\selectlanguage{english}V\rlap{\textsubscript{1}}\textsuperscript{ac}}[] % added by MD
        \DeclareHand{V1pc}{V1}{\selectlanguage{english}V\rlap{\textsubscript{1}}\textsuperscript{pc}}[] % added by MD
\DeclareWitness{V2}{\selectlanguage{english}V\textsubscript{2}}{Sampurnananda Library Sarasvati Bhavan 29869}[]
\DeclareWitness{V3}{\selectlanguage{english}V\textsubscript{3}}{Sampurnananda Library Sarasvati Bhavan 29899}[]
\DeclareWitness{V4}{\selectlanguage{english}V\textsubscript{4}}{Sampurnananda Library Sarasvati Bhavan 29937}[]
\DeclareWitness{V5}{\selectlanguage{english}V\textsubscript{5}}{Sampurnananda Library Sarasvati Bhavan 29938}[]
\DeclareWitness{V6}{\selectlanguage{english}V\textsubscript{6}}{Sampurnananda Library Sarasvati Bhavan 29991}[]
\DeclareWitness{V8}{\selectlanguage{english}V\textsubscript{8}}{Sampurnananda Library Sarasvati Bhavan 30014}[]
\DeclareWitness{V11}{\selectlanguage{english}V\textsubscript{11}}{Sampurnananda Library Sarasvati Bhavan 30029}[]
\DeclareWitness{V12}{\selectlanguage{english}V\textsubscript{12}}{Sampurnananda Library Sarasvati Bhavan 30030}[]
\DeclareWitness{V13}{\selectlanguage{english}V\textsubscript{13}}{Sampurnananda Library Sarasvati Bhavan 30031}[]
\DeclareWitness{V14}{\selectlanguage{english}V\textsubscript{14}}{Sampurnananda Library Sarasvati Bhavan 30050}[]
\DeclareWitness{V15}{\selectlanguage{english}V\textsubscript{15}}{Sampurnananda Library Sarasvati Bhavan 30051}[]
\DeclareWitness{V15pc}{\selectlanguage{english}V\rlap{\textsubscript{15}}\textsuperscript{pc}\space}{}[]
\DeclareWitness{V16}{\selectlanguage{english}V\textsubscript{16}}{Sampurnananda Library Sarasvati Bhavan 30052}[]
\DeclareWitness{V17}{\selectlanguage{english}V\textsubscript{17}}{Sampurnananda Library Sarasvati Bhavan 30053}[] % added by MD
\DeclareWitness{V16pc}{\selectlanguage{english}V\rlap{\textsubscript{16}}\textsuperscript{pc}\space}{}[]
\DeclareWitness{V18}{\selectlanguage{english}V\textsubscript{18}}{Sampurnananda Library Sarasvati Bhavan 30064}[]
\DeclareWitness{V19}{\selectlanguage{english}V\textsubscript{19}}{Sampurnananda Library Sarasvati Bhavan 30069}[]
\DeclareWitness{V21}{\selectlanguage{english}V\textsubscript{21}}{Sampurnananda Library Sarasvati Bhavan 30104}[]
\DeclareWitness{V22}{\selectlanguage{english}V\textsubscript{22}}{Sampurnananda Library Sarasvati Bhavan 30110}[]
\DeclareWitness{V25}{\selectlanguage{english}V\textsubscript{25}}{Sampurnananda Library Sarasvati Bhavan 30122}[]
\DeclareWitness{V26}{\selectlanguage{english}V\textsubscript{26}}{Sampurnananda Library Sarasvati Bhavan 30123}[]
\DeclareWitness{V28}{\selectlanguage{english}V\textsubscript{28}}{Sampurnananda Library Sarasvati Bhavan 30136}[]
\DeclareWitness{W2}{\selectlanguage{english}W\textsubscript{2}}{Wai ??}[]
\DeclareWitness{W4}{\selectlanguage{english}W\textsubscript{4}}{Wai 399-6171}[]

%%%%%%%%%%%%%%%%%%%%%%%%%%%%%%%%%
%%% Jammu & Kaschmir
\DeclareWitness{K1}{\selectlanguage{english}K\textsubscript{1}}{Raghunātha Temple Library 4383}[settlement=Jammu]
        \DeclareWitness{K1ac}{\selectlanguage{english}K\rlap{\textsubscript{1}}\textsuperscript{ac}\space}{}[]
        \DeclareWitness{K1pc}{\selectlanguage{english}K\rlap{\textsubscript{1}}\textsuperscript{pc}\space}{}[]
\DeclareWitness{K3}{\selectlanguage{english}K\textsubscript{3}}{Privat collection}
\DeclareWitness{L1}{\selectlanguage{english}L\textsubscript{1}}{SOAS RE 43454}[settlement=Jammu]
% More details? Catalogue number? L1 And C1 very close (and come from same region)
%%%%%%%%%%%%%%%%%%%%%%%%%%%%%%%%
% Jodhpur
% J10 is important
\DeclareWitness{J10}{\selectlanguage{english}J\textsubscript{10}}{MSPP Jodhpur 2230}[]
        \DeclareHand{J10ac}{J10}{\selectlanguage{english}J\rlap{\textsubscript{10}}\textsuperscript{ac}}[] % modified by MD
        \DeclareHand{J10pc}{J10}{\selectlanguage{english}J\rlap{\textsubscript{10}}\textsuperscript{pc}}[] % modified by MD
\DeclareWitness{J1}{\selectlanguage{english}J\textsubscript{1}}{Jodhpur 02231}[]
\DeclareWitness{J2}{\selectlanguage{english}J\textsubscript{2}}{Jodhpur 02232}[]   
\DeclareWitness{J3}{\selectlanguage{english}J\textsubscript{3}}{Jodhpur 02233}[]
\DeclareWitness{J4}{\selectlanguage{english}J\textsubscript{4}}{Jodhpur 02234}[]
        \DeclareWitness{J4ac}{\selectlanguage{english}J\rlap{\textsubscript{4}}\textsuperscript{ac}\space}{MSPP Jodhpur 02234}[]
        \DeclareWitness{J4pc}{\selectlanguage{english}J\rlap{\textsubscript{4}}\textsuperscript{pc}\space}{MSPP Jodhpur 02234}[]
\DeclareWitness{J5}{\selectlanguage{english}J\textsubscript{5}}{Jodhpur 02235}[]  % 4 chapters, 34 jpgs,   long colophon, missing lines in the beginning.
\DeclareWitness{J6}{\selectlanguage{english}J\textsubscript{6}}{Jodhpur 02237}[]  % 4 chapters, 41 jpgs
%\DeclareWitness{J6ac}{\selectlanguage{english}J\rlap{\textsubscript{6}}\textsubscript{ac}}{Jodhpur 02237}[]  % 4 chapters, 49 jpgs,   1st folio: idaṃ gulābarāyasya
% tulasīrāmaśarmmaṇaḥ putrasya pustakaṃ ...        End: iti śrīsahajānandasantānacintāmaṇisvātmārāmaviracitāyāṃ ..
% saṃvat 1802   (more consistent text)
%\DeclareWitness{J6pc}{\selectlanguage{english}J\rlap{\textsubscript{6}}\textsubscript{pc}}{Jodhpur 02237}[] 
\DeclareWitness{J7}{\selectlanguage{english}J\textsubscript{7}}{Jodhpur 02241}[]  % 4 chapters, 41 jpgs
\DeclareWitness{J8}{\selectlanguage{english}J\textsubscript{8}}{Jodhpur 23709}[]  % 4 chapters,  87 jpgs.   saṃvat 1724
\DeclareHand{J8ac}{J8}{\selectlanguage{english}J\rlap{\textsubscript{8}}\textsuperscript{ac}}[]  % changed by MD
\DeclareHand{J8pc}{J8}{\selectlanguage{english}J\rlap{\textsubscript{8}}\textsuperscript{pc}}[]  % changed by MD
\DeclareWitness{J9}{\selectlanguage{english}J\textsubscript{9}}{Jodhpur 02224}[]  %  fragment, 20 jpgs.
\DeclareWitness{J11}{\selectlanguage{english}J\textsubscript{11}}{Jodhpur 23532}[]
        \DeclareHand{J11ac}{J11}{\selectlanguage{english}J\rlap{\textsubscript{11}}\textsuperscript{ac}}[] % added by MD
        \DeclareHand{J11pc}{J11}{\selectlanguage{english}J\rlap{\textsubscript{11}}\textsuperscript{pc}}[] % added by MD
\DeclareWitness{J12}{\selectlanguage{english}J\textsubscript{12}}{Jodhpur 18552}[] 
\DeclareWitness{J13}{\selectlanguage{english}J\textsubscript{13}}{Jodhpur 02229}[]  %  5 chapters, 93 jpgs.
\DeclareWitness{J14}{\selectlanguage{english}J\textsubscript{14}}{Jodhpur 02239}[]  %  4 chapters
\DeclareWitness{J15}{\selectlanguage{english}J\textsubscript{15}}{Jodhpur 9732A}[]
\DeclareWitness{J16}{\selectlanguage{english}J\textsubscript{16}}{Jodhpur 9732B}[]
\DeclareWitness{J17}{\selectlanguage{english}J\textsubscript{17}}{Jodhpur 3013}[]
% Haṭhapradīpikā with (non-Sanskrit) Bhāṣya RORI Jodhpur ACC.NO.18552
%  Haṭhapradīpikā with (non-Sanskrit) commentary, RORI Alwar 952, 4 chapters,  colophon of the comm:
% iti śrīlāhorīmiśravrajabhūṣanaviracitāyāṃ bhāvārthadīpikāyāṃ caturthodhyāya ..    
%  Haṭhapradīpikā (5 chapter) MSPP Jodhpur ACC.NO.02229/

%%%%%%%%%%        Bodleian, Oxford
\DeclareWitness{B1}{\selectlanguage{english}B\textsubscript{1}}{Bodleian Library No. d.457(8)}[settlement=Oxford]
\DeclareWitness{B2}{\selectlanguage{english}B\textsubscript{2}}{Bodleian Library No. d.458(1)}[settlement=Oxford]
\DeclareWitness{B3}{\selectlanguage{english}B\textsubscript{3}}{Bodleian Library No. d.458(9)}[settlement=Oxford]

%%%%%%%%%%%   Chandigarh
\DeclareWitness{C1}{\selectlanguage{english}C\textsubscript{1}}{Lalchand M-2080}[]%L1 And C1 very close (and come from same region)
\DeclareWitness{C2}{\selectlanguage{english}C\textsubscript{2}}{Lalchand M-6065}[]
\DeclareWitness{C3}{\selectlanguage{english}C\textsubscript{3}}{Lalchand M-1293}[]
\DeclareWitness{C4}{\selectlanguage{english}C\textsubscript{4}}{Lalchand M-2081}[]
\DeclareWitness{C4ac}{\selectlanguage{english}C\rlap{\textsubscript{4}}\textsuperscript{ac}\space}{}[]
\DeclareWitness{C4pc}{\selectlanguage{english}C\rlap{\textsubscript{4}}\textsuperscript{pc}\space}{}[]
\DeclareWitness{C5}{\selectlanguage{english}C\textsubscript{5}}{Lalchand M-2082}[]%doesn't have chapter 1
\DeclareWitness{C6}{\selectlanguage{english}C\textsubscript{6}}{Lalchand M-2089}[]
\DeclareWitness{C7}{\selectlanguage{english}C\textsubscript{7}}{Lalchand M-6494}[]
\DeclareWitness{C8}{\selectlanguage{english}C\textsubscript{8}}{Lalchand M-2091}[]
        \DeclareHand{C8ac}{C8}{\selectlanguage{english}C\rlap{\textsubscript{8}}\textsuperscript{ac}}[]
        \DeclareHand{C8pc}{C8}{\selectlanguage{english}C\rlap{\textsubscript{8}}\textsuperscript{pc}}[]
\DeclareWitness{C9}{\selectlanguage{english}C\textsubscript{9}}{Lalchand M-4530}[]


% %%%%%%%%%%        Nepalese
\DeclareWitness{N1}{\selectlanguage{english}N\textsubscript{1}}{NGMPP A1400-2}[]
\DeclareWitness{N2}{\selectlanguage{english}N\textsubscript{2}}{NGMPP B 39-19}[]
\DeclareWitness{N3}{\selectlanguage{english}N\textsubscript{3}}{NGMPP B 62-20}[]
\DeclareWitness{N5}{\selectlanguage{english}N\textsubscript{5}}{NGMPP A60-15 + A61-1}[]
\DeclareWitness{N4}{\selectlanguage{english}N\textsubscript{4}}{NGMPP A61-2}[]
\DeclareWitness{N6}{\selectlanguage{english}N\textsubscript{6}}{NGMPP A61-6}[]
\DeclareWitness{N9}{\selectlanguage{english}N\textsubscript{9}}{NGMPP A62-33}[]
\DeclareWitness{N10}{\selectlanguage{english}N\textsubscript{10}}{NGMPP A62-37}[]
\DeclareWitness{N11}{\selectlanguage{english}N\textsubscript{11}}{NGMPP A63-15}[]
\DeclareWitness{N12}{\selectlanguage{english}N\textsubscript{12}}{NGMPP A939-19}[]
\DeclareWitness{N13}{\selectlanguage{english}N\textsubscript{13}}{NGMPP A1378-18}[]
\DeclareWitness{N16}{\selectlanguage{english}N\textsubscript{16}}{NGMPP B39-20}[]
\DeclareWitness{N17}{\selectlanguage{english}N\textsubscript{17}}{NGMPP B 111-10}[]
\DeclareWitness{N18}{\selectlanguage{english}N\textsubscript{18}}{NGMPP E 929-3}[]
\DeclareWitness{N19}{\selectlanguage{english}N\textsubscript{19}}{NGMPP E-1528-1 / E-1527-7(4)}[]
\DeclareWitness{N20}{\selectlanguage{english}N\textsubscript{20}}{NGMPP E 2037-13 }[]
\DeclareWitness{N21}{\selectlanguage{english}N\textsubscript{21}}{NGMPP E 2097-31}[]
\DeclareWitness{N22}{\selectlanguage{english}N\textsubscript{22}}{NGMPP G 4-4}[]
\DeclareWitness{N23}{\selectlanguage{english}N\textsubscript{23}}{NGMPP G 25-2}[]
        \DeclareHand{N23ac}{N23}{\selectlanguage{english}N\rlap{\textsubscript{23}}\textsuperscript{ac}}[] % added by MD
        \DeclareHand{N23pc}{N23}{\selectlanguage{english}N\rlap{\textsubscript{23}}\textsuperscript{pc}}[] % added by MD
\DeclareWitness{N24}{\selectlanguage{english}N\textsubscript{24}}{NGMPP G 190-16}[]
\DeclareWitness{N24ac}{\selectlanguage{english}N\rlap{\textsubscript{24}}\textsuperscript{ac}\space}{}[]
\DeclareWitness{N24pc}{\selectlanguage{english}N\rlap{\textsubscript{24}}\textsuperscript{pc}\space}{}[]
\DeclareWitness{N26}{\selectlanguage{english}N\textsubscript{26}}{NGMPP T 24-3}[]

% %%%%%%%%%%        Pune

\DeclareWitness{P1}{\selectlanguage{english}P\textsubscript{1}}{Ānandāśrama S16-3-21}[]
\DeclareWitness{P2}{\selectlanguage{english}P\textsubscript{2}}{Ānandāśrama S16-2-20}[]
\DeclareWitness{P3}{\selectlanguage{english}P\textsubscript{3}}{BISM (79) 314}[]
\DeclareWitness{P4}{\selectlanguage{english}P\textsubscript{4}}{BISM (91) 191}[]
\DeclareWitness{P5}{\selectlanguage{english}P\textsubscript{5}}{BISM (29) 5790}[]
\DeclareWitness{P6}{\selectlanguage{english}P\textsubscript{6}}{BORI 263/1879-80}[]
\DeclareWitness{P7}{\selectlanguage{english}P\textsubscript{7}}{BORI 665/1883-84}[]
\DeclareWitness{P8}{\selectlanguage{english}P\textsubscript{8}}{BORI 316/1895-98}[]
\DeclareWitness{P9}{\selectlanguage{english}P\textsubscript{9}}{BORI 733-1891-95}[]
\DeclareWitness{P10}{\selectlanguage{english}P\textsubscript{10}}{BORI 222-1884-86}[]
\DeclareWitness{P11}{\selectlanguage{english}P\textsubscript{11}}{BORI 221-1882–83}[]
\DeclareWitness{P12}{\selectlanguage{english}P\textsubscript{12}}{Ānandāśrama S16-3-24}[]
\DeclareWitness{P13}{\selectlanguage{english}P\textsubscript{13}}{Ānandāśrama S16-2-22}[]
\DeclareWitness{P14}{\selectlanguage{english}P\textsubscript{14}}{Ānandāśrama S16-3-23}[]
\DeclareWitness{P15}{\selectlanguage{english}P\textsubscript{15}}{BISM (64) 919}[]
\DeclareWitness{P16}{\selectlanguage{english}P\textsubscript{16}}{BISM (64) 1115}[]
\DeclareWitness{P17}{\selectlanguage{english}P\textsubscript{17}}{BISM 620/1886-92}[]
\DeclareWitness{P18}{\selectlanguage{english}P\textsubscript{18}}{BORI 615/1887-91}[]
\DeclareWitness{P19}{\selectlanguage{english}P\textsubscript{19}}{BISM 46-39}[]
\DeclareWitness{P20}{\selectlanguage{english}P\textsubscript{20}}{BISM 39-273}[]
\DeclareWitness{P21}{\selectlanguage{english}P\textsubscript{21}}{BISM 37-743}[]
\DeclareWitness{P22}{\selectlanguage{english}P\textsubscript{22}}{BISM 37-729}[]
\DeclareWitness{P23}{\selectlanguage{english}P\textsubscript{23}}{BISM 33-60}[]
\DeclareWitness{P24}{\selectlanguage{english}P\textsubscript{24}}{BISM 29-5790}[]% =P5!
\DeclareWitness{P25}{\selectlanguage{english}P\textsubscript{25}}{BISM 29-3657}[]
\DeclareWitness{P26}{\selectlanguage{english}P\textsubscript{26}}{BISM 25-281}[]
\DeclareWitness{P27}{\selectlanguage{english}P\textsubscript{27}}{BISM 7-489}[]
\DeclareWitness{P28}{\selectlanguage{english}P\textsubscript{28}}{BORI 399-1895-1902}[]

%%%%%   Mysore
\DeclareWitness{M1}{\selectlanguage{english}M\textsubscript{1}}{P-5682/4}[]
%%%%%   Tübingen
\DeclareWitness{Tue}{\selectlanguage{english}Tü}{Ma I 339}[]
%%%%%%%%%%
\DeclareWitness{YC}{\selectlanguage{english}YC}{Yogacintāmaṇi}[]
\DeclareWitness{ceteri}{\selectlanguage{english}cett.}{ceteri}[]

%%%%%%%%%% Mss with Commentary
\DeclareWitness{A1}{\selectlanguage{english}A\textsubscript{1}}{Alwar 952}[]

\DeclareWitness{Jyo}{\selectlanguage{english}J\textsubscript{yo}}{Brahmānanda's version}[]

%%%%%%%%%%%%%%%%%%%%%%%%%%%%%%%%%%%%%%%%%%%
%List of all Sigla:
%A1,B1,B2,B3,C1,C2,C3,C4,C6,C7,C8,C9,J1,J2,J3,J4,J10,J13,J14,J15,J17,L1,M1,N3,N5,N6,N9,N10,N11,N12,N13,N16,N17,N19,N20,N21,N22,N23,N24,Tü,V1,V2,V3,V4,V5,V6,V8,V11,V19,V22,V26,Vu
%%%%%%%%%%%%%%%%%%%%%%%%%%%%%%%%%%%%%%%%%%%

\DeclareWitness{G4}{\selectlanguage{english}G\textsubscript{4}}{GOML D18885 (Bundle SD5051)}[]
\DeclareWitness{G5}{\selectlanguage{english}G\textsubscript{5}}{GOML R3841/ SR2190}[]
\DeclareWitness{G7}{\selectlanguage{english}G\textsubscript{7}}{GOML D4394}[]

\DeclareWitness{Ko}{\selectlanguage{english}K\textsubscript{o}}{Koba, Gujarat 55626}[]

% addition 2023-12-11 MD:
\TeXtoTEIPat{\begin {metre}[#1]}{<note type="metre" target="##1">}
\TeXtoTEIPat{\end {metre}}{</note>}
\TeXtoTEIPat{\texttheta}{θ}

% change 2023-12-05 mm
\TeXtoTEI{teimute}{} 

% changes/additions 2023-11-27 MM:
\TeXtoTEIPat{\medialink {#1}{#2}}{<ref target="resources/#2">#1</ref>}

% changes/additions 2023-10-25 MM:
% new Sigla
\TeXtoTEIPat{\textAlpha}{Α}
\TeXtoTEIPat{\textalpha}{α}
\TeXtoTEIPat{\textBeta}{Β}
\TeXtoTEIPat{\textbeta}{β}
\TeXtoTEIPat{\textGamma}{Γ}
\TeXtoTEIPat{\textgamma}{γ}
\TeXtoTEIPat{\textDelta}{Δ}
\TeXtoTEIPat{\textdelta}{δ}
\TeXtoTEIPat{\textEpsilon}{Ε}
\TeXtoTEIPat{\textepsilon}{ε}
\TeXtoTEIPat{\textEta}{Η}
\TeXtoTEIPat{\texteta}{η}
\TeXtoTEIPat{\textChi}{Χ}
\TeXtoTEIPat{\textchi}{χ}
\TeXtoTEIPat{\textOmega}{Ω}
\TeXtoTEIPat{\textomega}{ω}

%new environments
\TeXtoTEIPat{\begin {postmula}[#1]}{<note type="postmula" target="##1">}
  \TeXtoTEIPat{\end {postmula}}{</note>}
\TeXtoTEIPat{\begin {altava}[#1]}{<div type="altrec"><note type="avataranika" target="##1">} %%% changed 2023-12-05 mm
  \TeXtoTEIPat{\end {altava}}{</note></div>} %%% changed 2023-12-05 mm
\TeXtoTEIPat{\sgwit {#1}}{<note type="inlineref"><ref>#1</ref></note>}

% changes/additions 2023-10-12 MM:
\TeXtoTEIPat{\\.}{}

% changes/additions 2023-08-15 MD:
\TeXtoTEIPat{\lineom {#1}{#2}}{<note type="omission">#1 omitted in <ref>#2</ref></note>}
\TeXtoTEI{graus}{hi}[rend="grey"]
\TeXtoTEIPat{\startgray}{} %%% changed 2023-12-05 mm
\TeXtoTEIPat{\endgray}{} %%% changed 2023-12-05 mm



% additions/changes 2023-06-05 mm:
%\TeXtoTEIPat{\lineom {#1}}{<note type="omission">Line omitted in <ref>#1</ref></note>}
\TeXtoTEIPat{\NotIn {#1}}{<note type="omission">Stanza omitted in <ref>#1</ref></note>}

% additions 2023-04-16 MD:
\TeXtoTEIPat{\,}{ }

% additions 2023-04-13 mm:
\TeXtoTEIPat{\begin {versinnote}}{<lg>}
  \TeXtoTEIPat{\end {versinnote}}{</lg>}

% additions 2023-04-05 MD:
\TeXtoTEIPat{\begin {testimonia}[#1]}{<note type="testimonia" target="##1">}
  \TeXtoTEIPat{\end {testimonia}}{</note>}
\TeXtoTEI{devnote}{s}[xml:lang="sa-deva"]

% app in philcomm und testimonia %%% added MM 2023-12-02
\TeXtoTEI{var}{note}[type="appinnote"]


\TeXtoTEI{anm}{note}[type="memo"] %% change 2023-04-16 MD
\TeXtoTEI{Anm}{note}[type="memo"] %% change 2023-12-05 MM
\TeXtoTEIPat{\startverse}{} %%% marked for change 2023-04-13 mm
\TeXtoTEIPat{\endverse}{} %%% marked for change 2023-04-13 mm
\TeXtoTEIPat{\newpage}{}
\TeXtoTEIPat{\marma}{}
\TeXtoTEIPat{\marmas}{}
\TeXtoTEIPat{\vin}{} % added by MD 2023-11-14

%%% modify environments and commands
%%% TEI mapping
% additions/changes 2022-06-07 mm:
\TeXtoTEI{grau}{hi}[rend="grey"]
\TeXtoTEIPat{ \& }{ &amp; }

% additions/changes 2022-06-01 mm:
\TeXtoTEI{skp}{seg}[type="deva-ignore"]
\TeXtoTEI{skm}{seg}[type="ltn-ignore"]

\TeXtoTEIPat{\rlap {#1}}{#1}

% additions/changes 2022-04-06 mm:
%\TeXtoTEI{sgwit}{ref}
\TeXtoTEI{textdev}{s}[xml:lang="sa-deva"]
\TeXtoTEIPat{\begin {col}[#1]}{<div type="colophon" xml:id="#1"><p>}
  \TeXtoTEIPat{\end {col}}{</p></div>}
\TeXtoTEIPat{\begin {ava}[#1]}{<note type="avataranika" target="##1">}
  \TeXtoTEIPat{\end {ava}}{</note>}
												   
\TeXtoTEIPat{\outdent}{}
\TeXtoTEIPat{\startaltrecension}{} %%% changed 2023-12-05 mm
\TeXtoTEIPat{\endaltrecension}{} %%% changed 2023-12-05 mm
\TeXtoTEIPat{\startaltnormal}{} % added by MD 2023-11-14 %%% changed 2023-12-05 mm
\TeXtoTEIPat{\endaltnormal}{} % added by MD 2023-11-14 %%% changed 2023-12-05 mm
\TeXtoTEIPat{\begin {alttlg}[#1]}{<div type="altrec"><lg xml:id="#1">}
  \TeXtoTEIPat{\end {alttlg}}{</lg></div>}



% additions/changes 2022-03-12 mm:
\TeXtoTEIPat{\begin {tlg}[#1]}{<lg xml:id="#1">}
  \TeXtoTEIPat{\end {tlg}}{</lg>}

\TeXtoTEIPat{\begin {translation}[#1]}{<note type="translation" target="##1">}
  \TeXtoTEIPat{\end {translation}}{</note>}
\TeXtoTEIPat{\begin {philcomm}[#1]}{<note type="philcomm" target="##1">}
  \TeXtoTEIPat{\end {philcomm}}{</note>}
\TeXtoTEIPat{\begin {sources}[#1]}{<note type="sources" target="##1">}
  \TeXtoTEIPat{\end {sources}}{</note>}


\TeXtoTEIPat{\begin {marma}[#1]}{<note type="marma" target="##1">}
  \TeXtoTEIPat{\end {marma}}{</note>}

\TeXtoTEIPat{\begin {jyotsna}[#1]}{<note type="jyotsna" target="##1">}
  \TeXtoTEIPat{\end {jyotsna}}{</note>}

\EnvtoTEI{description}{list}
\EnvtoTEI{itemize}{list}
\TeXtoTEIPat{\item [#1]}{<label>#1</label>\item}

\TeXtoTEI{tl}{l}
\TeXtoTEI{myfn}{note}[type="myfn"]
\TeXtoTEIPat{\getsiglum {#1}}{<ref target="##1"/>}

\TeXtoTEI{SetLineation}{}
\TeXtoTEI{noindent}{}
\TeXtoTEI{subsection*}{}

\TeXtoTEI{rlap}{}

% end additions/changes
% \TeXtoTEIPat{\skp {#1}}{#1}
% \TeXtoTEIPat{\skm {#1}}{}

\TeXtoTEIPat{\begin {prose}}{<p>}
  \TeXtoTEIPat{\end {prose}}{</p>}

\TeXtoTEIPat{\begin {tlate}}{<p>}
  \TeXtoTEIPat{\end {tlate}}{</p>}

\TeXtoTEI{emph}{hi}
\TeXtoTEI{bigskip}{}
% \TeXtoTEI{/}{|}
\TeXtoTEI{tl}{l}
\TeXtoTEIPat{english}{}
%\TeXtoTEIPat{-}{ } %% change 2023-04-16 MD
%\TeXtoTEIPat{°}{} %% change 2023-04-16 MD
\TeXtoTEIPat{\textcolor {#1}{#2}}{<hi rend="#1">#2</hi>}

% \TeXtoTEIPat{\eyeskip}{}
% \TeXtoTEIPat{\aberratio}{}
% \TeXtoTEIPat{\ad}{}
\TeXtoTEIPat{\add}{<hi rend="italic">add.</hi>} %% change 2023-04-16 MD
% \TeXtoTEIPat{\ann}{}
\TeXtoTEIPat{\ante}{<hi rend="italic">ante</hi> } %% change 2023-04-16 MD
\TeXtoTEIPat{\post}{<hi rend="italic">post</hi> } %% change 2023-04-16 MD
% \TeXtoTEIPat{\codd}{}
% \TeXtoTEIPat{\conj }{}
% \TeXtoTEIPat{\contin}{}
% \TeXtoTEIPat{\corr}{}
% \TeXtoTEIPat{\del}{}
% \TeXtoTEIPat{\dub}{}
% \TeXtoTEIPat{\emend }{}
% \TeXtoTEIPat{\expl}{}
% \TeXtoTEIPat{\ȩxplicat}{}
% \TeXtoTEIPat{\fol}{}
% \TeXtoTEIPat{\gloss}{}
% \TeXtoTEIPat{\ins}{}
% \TeXtoTEIPat{\im}{}
% \TeXtoTEIPat{\inmargine}{}
% \TeXtoTEIPat{\intextu}{}
% \TeXtoTEIPat{\indist}{}
% \TeXtoTEIPat{\iteravit}{}
% \TeXtoTEIPat{\lectio}{}
% \TeXtoTEIPat{\leginequit}{}
% \TeXtoTEIPat{\legn}{}
% \TeXtoTEIPat{\illeg}{<hi rend="italic">illeg.</hi>}
\TeXtoTEIPat{\illeg}{<gap reason="illeg."/>} %%% change 2023-04-11 mm
% \TeXtoTEIPat{\om}{<hi rend="italic">om.</hi>}
\TeXtoTEIPat{\om}{<gap reason="om."/>} %%% change 2023-04-11 mm
% \TeXtoTEIPat{\primman}{}
% \TeXtoTEIPat{\prob}{}
% \TeXtoTEIPat{\rep}{}
% \TeXtoTEIPat{\sequentia}{}
% \TeXtoTEIPat{\supralineam}{}
% \TeXtoTEIPat{\interlineam}{}
\TeXtoTEIPat{\vl}{<hi rend="italic">v.l.</hi>}
% \TeXtoTEIPat{\vide}{}
% \TeXtoTEIPat{\videtur}{}
% \TeXtoTEIPat{\crux}{}
% \TeXtoTEIPat{\cruxxx}{}
\TeXtoTEIPat{\unm}{<hi rend="italic">unm.</hi>}


% List of Scholars
\DeclareScholar{nos}{nos}[
forename=HPP,
surname=Team]


% Nullify \selectlanguage in TEI as it has been used in
% \DeclareWitness but should be ignored in TEI.
\TeXtoTEI{selectlanguage}{}



\setlength\parindent{1em}
\SetLineation{lineation=none}
\poemlines{0}

\SetHooks{
	lemmastyle=\bfseries,
	refnumstyle=\selectlanguage{english}\color{blue}\bfseries, 
	appfontsize=\footnotesize
	}
\DeclareApparatus{default}[
	lang=english,
	sep = {] },
	delim=\hskip 0.75em,
	%	rule=none,
	]
\DeclareApparatus{anmkg}[
	notelang=english,
	sep = { },
	delim=\texteng{\ \textbullet\ \ },
%	rule=\relax
	rule=\rule{0.15\columnwidth}{0.4pt}
	]

\newcommand{\mydelim}{\xspace\textcolor{violet}{\textbullet}\ \ }
\newcommand{\mylem}[1]{\texteng{\textcolor{violet}{#1}}}
\setlength{\vrightskip}{-15pt}
\setlength{\vgap}{-3em} % default 1.5em
\verselinenumfont{\footnotesize\selectlanguage{english}\normalfont}

\newlength{\myoutdent}\setlength{\myoutdent}{2em}

\DeclareShorthand{emend}{\texteng{\emph{em.}}}{ego}
%\DeclareShorthand{conj}{\texteng{\emph{conj.}}}{ego}

%Define two commands: \skp ("sanskrit plus"), to be ignored by TeX in
%the edition text, but processed in the TEI output. Conversely, \skm
%("sanskrit minus") is to be processed in the edition text, but
%ignored if found in the apparatus criticus and in the TEI output:

\newif\ifinapparatus
\NewDocumentCommand{\skp}{m}{}
\NewDocumentCommand{\skm}{m}{\unless\ifinapparatus#1\fi}

\SetTEIxmlExport{autopar=false}

\newcommand{\versenr}{\ \themyvnum//}

\NewDocumentEnvironment{tlg}{O{}}{
	\def\hpvnum{\texteng{\thepoemline}}
	\markboth{\hpvnum}{\hpvnum}
	\setcounter{myvnum}{\value{poemline}}
	\begin{ekdverse}
	\Large}{\normalsize
	\end{ekdverse}
	%\smallskip
%  \stepcounter{myvnum}
}

\NewDocumentEnvironment{alttlg}{O{}}{
	\setvnum{\hindsection.\arabic{saved@poemline}*\arabic{poemline}}
	\def\hpvnum{\texteng{\hindsection.\arabic{saved@poemline}*\arabic{poemline}}}
	\markboth{\hpvnum}{\hpvnum}
	\setcounter{altvnum}{\value{poemline}}
	\begin{ekdverse}[type=altrecension]
	\color{gray}
	\Large}{\normalsize
	\end{ekdverse}
	%\smallskip
}

\NewDocumentCommand{\tl}{m}{#1}

\NewDocumentEnvironment{ava}{O{}}{
	\setvnum{prescript:}
	\begin{ekdverse}
	\hspace{-\myoutdent}
	\Large}{\normalsize
	\end{ekdverse}
	\smallskip
}

\NewDocumentEnvironment{altava}{O{}}{
	\setvnum{prescript:}
	\begin{ekdverse}[type=altrecension]
	\color{gray}
	\hspace{-\myoutdent}
	\Large}{\normalsize
	\end{ekdverse}
	\smallskip
}   

\NewDocumentEnvironment{postmula}{O{}}{
	\setvnum{postscript:}
	\smallskip
	\begin{ekdverse}
	\hspace{-\myoutdent}
	\Large}{\normalsize
	\end{ekdverse}
}

\NewDocumentEnvironment{altpostmula}{O{}}{
	\setvnum{postscript:}
	\smallskip
	\begin{ekdverse}[type=altrecension]
	\color{gray}
	\hspace{-\myoutdent}
	\Large}{\normalsize
	\end{ekdverse}
}

\NewDocumentEnvironment{col}{O{}}{
	\setvnum{colophon:}
	\medskip
	\begin{ekdverse}%
	\hspace{-2.5em}%
	\Large%
	}{\normalsize
	\end{ekdverse}
	%\smallskip
      }
      
\NewDocumentCommand{\tcommref}{m}{}
\NewDocumentCommand{\ttransref}{m}{}
\NewDocumentCommand{\tnocomm}{}{}


\def\startaltrecension{
	\setcounter{altvnum}{0}
	\stopvline
	\addtocounter{saved@poemline}{-1}
	\renewcommand{\versenr}{\ \themyvnum *{\small \arabic{poemline}}//}
%	\small
	}
	
\def\endaltrecension{
	\addtocounter{saved@poemline}{1}
	\startvline
	\setvnum{\hindsection.\arabic{poemline}}
	\renewcommand{\versenr}{\ \themyvnum//}
%	\normalsize
	}

\def\startaltnormal{
	\startaltrecension
	\setvnum{\hindsection.\arabic{saved@poemline}*\arabic{poemline}}
	}

\def\endaltnormal{\endaltrecension}

%%%%%%

\newcommand{\teionly}[1]{}
\newcommand{\teimute}[1]{#1}
\newcommand{\manuref}[1]{#1}
\newcounter{myvnum}\setcounter{myvnum}{0}
\newcounter{altvnum}\setcounter{altvnum}{0}
\newcounter{mynotenr}\setcounter{mynotenr}{0}
%\newcommand{\myfn}[1]{\footnote{\texteng{#1}}}

\newcommand{\myfn}[1]{%
	\setcounter{ekd@padanum}{0} % um Pāda-Nummer zu unterdrücken
	\stepcounter{mynotenr}%
	\linelabel{note\themynotenr}%
	\note[type=anmkg, labelb={note\themynotenr}]{#1}
	}

% \newcommand{\myfnx}[1]{%
	% \setcounter{ekd@padanum}{0} % um Pāda-Nummer zu unterdrücken
	% \stepcounter{mynotenr}%
	% \linelabel{note\themynotenr}%
	% \note[type=anmkg, labelb={note\themynotenr},num]{#1}
	% }

\renewcommand{\thefootnote}{\texteng{\arabic{footnote}}}
\newcommand{\devnote}[1]{{\small\textdev{#1}}}
\newcommand{\devtext}[1]{{\normalsize\textdev{#1}}}
%\newcommand{\vsn}[1]{{\footnotesize\texteng{#1}}}
\newcommand{\graus}[1]{\small\textcolor{gray}{#1}\normalsize} % partial altrecension
\newcommand{\grau}[1]{\textcolor{gray}{#1}} % partial altrecension
\newcommand{\Anm}[1]{\begin{ekdverse}
	\texteng{\footnotesize (#1)}
	\end{ekdverse}
	}

%\newcommand{\sgwit}[1]{{\footnotesize (\getsiglum{#1})}}
%\newcommand{\NotIn}[1]{\texteng{\footnotesize (om. \getsiglum{#1})}}
%\newcommand{\lineom}[2]{\texteng{\footnotesize (#1 om. \getsiglum{#2})}}
%\newcommand{\anm}[1]{\texteng{\footnotesize [#1]}}
\newcommand{\sgwit}[1]{}% Nur für Online version; Change TEI too!!
%\newcommand{\lineom}[2]{\myfn{#1 om. \getsiglum{#2}}}
\newcommand{\anm}[1]{\myfn{#1}}
%\newcommand{\unavbl}[1]{\marginpar{\scriptsize\texteng{−\,\getsiglum{#1}}}}
%\newcommand{\unavbl}[1]{\myfn{Folio lost in \getsiglum{#1}}}
\newcommand{\textapp}[1]{\texteng{\textsf{#1}}}
\newcommand{\unavbl}{\textapp{folio lost}}
\newcommand{\incl}{\textapp{included in}}
\newcommand{\only}{\textapp{only included in}}
\newcommand{\also}{\textapp{also included in}}
\newcommand{\excl}{\textapp{included in all except}}
\newcommand{\NotIn}{\om}
\newcommand{\expnr}[1]{\textcolor{magenta}{#1}}% X\kern 1pt

\def\om{\texteng{\emph{om.\@}}}% \kern-0.3ex
\def\illeg{\texteng{\emph{illeg.\@}}} 
\def\lost{\texteng{\emph{lost}}} 
\def\lacuna{\texteng{\emph{lac.\@}}}
\def\unm{\texteng{\emph{unm.\ }}}
\def\ante{\texteng{\normalfont\textapp{ante\ }}}
\def\add{\texteng{\normalfont\emph{add.\@}}}
\def\post{\texteng{\normalfont\textapp{post\ }}}
\def\antecorr{\texteng{\textsubscript{ac}}}
\def\postcorr{\texteng{\textsubscript{pc}}}
\def\marmas{\ }%\texteng{\textsuperscript{\#}}\ }
\def\marma{}%\texteng{\textsuperscript{\#}}}
\def\crux{\texteng{\textsuperscript{\textdagger}}}

%%%%%%% Commentary part

\usepackage{catchfilebetweentags}

\NewDocumentEnvironment{translation}{O{}}{%
	\selectlanguage{english}}{%
	\selectlanguage{iast}}
	
\NewDocumentEnvironment{sources}{O{}}{%
	\selectlanguage{english}%
	\begin{description}[leftmargin=1em, 
		topsep=0pt, parsep=0pt, partopsep=0pt,
		listparindent=0pt, labelwidth=1em, labelsep=0pt]
	\item[\ding{118}\ Sources]
	\item %
	}{\end{description}\selectlanguage{iast}}

\NewDocumentEnvironment{testimonia}{O{}}{%
	\selectlanguage{english}%
	\begin{description}[leftmargin=1em,
		topsep=0pt, parsep=0pt, partopsep=0pt,
		listparindent=0pt, labelwidth=1em, labelsep=0pt]
	\item[\ding{118}\ Testimonia]
	\item %
	}{\end{description}\selectlanguage{iast}}
	
\NewDocumentEnvironment{philcomm}{O{}}{%
	\selectlanguage{english}%
	\begin{description}[leftmargin=1em, 
		topsep=0pt, parsep=0pt, partopsep=0pt,
		listparindent=1.5em,
		labelwidth=1em, labelsep=0pt]
	\item[\ding{118}\ Commentary]\ %
	\newline
	}{\end{description}\selectlanguage{iast}}

\newenvironment{variants}{%
	\begin{description}[%
		leftmargin=4em,
		topsep=3.5pt,
		parsep=0pt,
	%	partopsep=0pt,
		listparindent=-1.5em,
		labelwidth=2.5em,
		labelsep=0pt]
	\item\scriptsize}{%
	\end{description}
	}
 
\newenvironment{versinnote}{%
	\setlength{\vindent}{0pt}
%	\poemlines{0}
	\vspace{4pt plus 2pt minus 2pt}
	\begin{ekdverse}
	\linespread{0.9}\normalsize\selectlanguage{iast}}{%
	\linespread{1}\selectlanguage{english}\end{ekdverse}
	\vspace{4pt plus 2pt minus 2pt}
%	\poemlines{1}
	\addtocounter{poemline}{-1}
	}

  \newenvironment{versinnoterm}{%
	\setlength{\vindent}{0pt}
	\vspace{1pt}
	\begin{ekdverse}
		\itshape}{%
		\rmfamily
	\end{ekdverse}
	\vspace{1pt}
	\addtocounter{poemline}{-1}
	}

\newenvironment{appinnote}{% still in use: 1.16, 1.30, 2.50, 2.77, 3.25, 3.34, 3.39*1, 4.9
	\setlength{\vindent}{0pt}
	\begin{ekdverse}
	\scriptsize\selectlanguage{english}}{%
	\selectlanguage{iast}\end{ekdverse}
	\vspace{3pt minus 1pt}
	\addtocounter{poemline}{-1}
}

%\newcommand{\vnumfix}{\addtocounter{poemline}{1}}
%\TeXtoTEIPat{\vnumfix}{}
\newcommand{\labelincomm}{\smallskip\newline\noindent}
%\TeXtoTEIPat{\labelincomm}{<lb/>} % >> PreambleComm.tex
%\newcommand{\tre}{\ }
%\TeXtoTEIPat{\tre}{}
\newcommand{\skx}[2]{#1} % sandhi between pādas
%\TeXtoTEIPat{\skx {#1}{#2}}{#2} % >> PreambleComm.tex
%\TeXtoTEIPat{\commcitecore}{}
%\TeXtoTEIPat{\commcite}{}
%\TeXtoTEIPat{\commciterange}{}
%\TeXtoTEIPat{\altcommcite}{}
%\TeXtoTEIPat{\avacite}{}
%\TeXtoTEIPat{\colcite}{}
%\TeXtoTEIPat{\trcite}{}

%\TeXtoTEIPat{\labelvnum}{}
%\TeXtoTEIPat{\commvnum}{}

\newcommand{\myvspace}{\vspace{-3pt plus 3pt minus 3pt}}
\newcommand{\commlabel}{\hfill\texteng{\raisebox{0pt}{\textbf{[\hindsection.\labelvnum]}}}\hfill}

\newcommand{\comminfn}{%
	\footnotetext{%
	\commlabel
	\ExecuteMetaData[\commfilename]{sc\commvnum}%
	\ExecuteMetaData[\commfilename]{ts\commvnum}%
	\ExecuteMetaData[\commfilename]{cm\commvnum}%
	}}
	
\newcommand{\commcitecore}{%
	\myvspace
	\begin{quote}%
	\ExecuteMetaData[\commfilename]{tr\commvnum}
	\texteng{(\labelvnum)}\comminfn
	\end{quote}}

\def\commfilename{HP\hindsection_comm.tex}
\newcommand{\commcite}{%
	\def\commvnum{\themyvnum}%
	\def\labelvnum{\themyvnum}%
	\commcitecore}

\newcommand{\commciterange}[2]{%
	\def\commvnum{#1}%
	\def\labelvnum{#2}%
	\commcitecore}
	
\newcommand{\altcommcite}{%
	\def\commvnum{\themyvnum-\thealtvnum}%
	\def\labelvnum{\themyvnum*\thealtvnum}%
	\myvspace
	\begin{quote}%
	\textcolor{gray}{%
	\ExecuteMetaData[\commfilename]{tr\commvnum}
	\texteng{(\labelvnum)}}\comminfn
	\end{quote}}

\newcommand{\avacite}[1]{%
	\bigskip%
	\setlength{\parindent}{1em} %\hspace*{0.5em} in HP4X
	\ExecuteMetaData[\commfilename]{tr#1}
	\vspace{-3pt}
	}

\newcommand{\trcite}[1]{
	\myvspace
	\begin{quote}
	\ExecuteMetaData[\commfilename]{tr#1}
	\texteng{(#1)}
	\end{quote}
	}

\newcommand{\alttrcite}{
	\def\commvnum{\themyvnum-\thealtvnum}%
	\def\labelvnum{\themyvnum*\thealtvnum}%
	\myvspace
	\begin{quote}
	\textcolor{gray}{\ExecuteMetaData[\commfilename]{tr\commvnum}
	\texteng{(\labelvnum)}}
	\end{quote}
	}

\newcommand{\colcite}{
	\medskip
	\noindent
	\ExecuteMetaData[\commfilename]{trcol}
	}


\newcommand{\closer}{\vspace{-1ex}}
\newcommand{\lb}{\par}
\newcommand{\mylb}{\smallskip\lb}
\newcommand{\sep}{\par}
% \TeXtoTEIPat{\sep}{<lb/>}% oder besser mit einem Trennzeichen in einer Zeile lassen?
\def\vl{\textit{v.l.}\xspace}
%\newcommand{\var}[1]{\texteng{\scriptsize #1}}
%\newcommand{\varsep}{\xspace\texteng{\textbullet}\xspace}

\def\sl#1{\emph{#1}}
\newcommand{\medialink}[2]{\textcolor{violet}{\underline{#1}}}
%\TeXtoTEIPat{\medialink {#1}{#2}}{<ref target="/images/#2">#1</ref>}
\usepackage{url}

\newcommand{\alphaOne}{\textalpha\textsubscript{1}}% N3
\newcommand{\alphaTwo}{\textalpha\textsubscript{2}}% J5
\newcommand{\alphaThree}{\textalpha\textsubscript{3}}% G4
\newcommand{\gammaOne}{\textgamma\textsubscript{1}}% N23
\newcommand{\gammaTwo}{\textgamma\textsubscript{2}}% J7
\newcommand{\deltaOne}{\textdelta\textsubscript{1}}% V19
\newcommand{\deltaTwo}{\textdelta\textsubscript{2}}% K3
\newcommand{\deltaThree}{\textdelta\textsubscript{3}}% C7
\newcommand{\deltaOmega}{\textdelta\textsubscript{\textomega}}% J6
\newcommand{\epsilonOne}{\textepsilon\textsubscript{1}}% G11
\newcommand{\epsilonTwo}{\textepsilon\textsubscript{2}}% G5
\newcommand{\zetaOne}{\textzeta\textsubscript{1}}% P15
\newcommand{\zetaTwo}{\textzeta\textsubscript{2}}% N19
\newcommand{\zetaThree}{\textzeta\textsubscript{3}}% V15
\newcommand{\zetaFour}{\textzeta\textsubscript{4}}% J11
\newcommand{\zetaOmega}{\textzeta\textsubscript{\textomega}}% N26
\newcommand{\etaOne}{\texteta\textsubscript{1}}% V1
\newcommand{\etaTwo}{\texteta\textsubscript{2}}% J10
\newcommand{\etaOmega}{\texteta\textsubscript{\textomega}}% E4
\newcommand{\piOne}{\textpi\textsubscript{1}}% P11
\newcommand{\piTwo}{\textpi\textsubscript{2}}% C6
\newcommand{\piOmega}{\textpi\textsubscript{\textomega}}% V3

\def\attr{\hbox{attrib.}\xspace}
\babelhyphenation{%
	Dattā-treya-yoga-śāstra
	Gorakṣa-śataka
	Go-rakṣa-nātha
	Haṭha-pra-dī-pikā
	Haṭha-ratnā-valī
	Haṭha-tattva-kaumudī
	Jāran-dhara
	Rāja-yoga
	Śām-bhavī
	Śāṃ-bhavī
	Śārṅga-dhara-pad-dhati
	Svātmā-rāma
	Śiva-saṃhitā
	Vasiṣṭha-saṃhitā
	Viveka-mārtaṇḍa
	Yukta-bhava-deva
	Yoga-cintā-maṇi
	Yoga-tattva-pra-kāśa
	Yoga-yājña-valkya
	}

\def\hindsection{1}

% Ms list: N3,J5,G4(if available),P11,C6,V3,N23,V19,E2,G11,P15,V15,V1,J10,Jyo
% deleted: C1,C8,V17,J10pc,N17,P28,W4,N19

\begin{document}
\thispagestyle{firstpage}
\begin{center}
\section*{Chapter 1}
\end{center}
\bigskip
\begin{otherlanguage}{iast}
\begin{ekdosis}

% N3 śrīgurusahajavināyakāya namaḥ/
% P11 śrīgaṇādhipataye namaḥ//
% V3 śrīgaṇeśāya namaḥ// śrīśadāśivāya namaḥ//
% N23 śrīgaṇeśāya namaḥ/
% J7
% V19 śrī .. geśvarāya namaḥ//
% E2 oṃ śrīgaṇādhipataye namaḥ oṃ
% G11 śrīgurubhyo namaḥ |
% P15 nama śrīgurubhyo namaḥ//
% V15 śrīgaṇeśāya namaḥ// oṃ namaḥ paramātmane viśvarūpatīrthāya gurave//
% V1 oṃ namaḥ//
% J10 śrīgaṇeśāya namaḥ//


\begin{tlg}[hp01_001]
\tl{\app{\lem[nolem]{}
	\rdg[wit={G11},alt={\om}]{\skp{\om}}}%
\pada{\app{\lem[wit={ceteri}]{śrīādināthāya}
	\rdg[wit={P15}]{anādināthāya}
	\rdg[wit={V1}]{ādīśanāthāya}} namo'stu tasmai}\\+}
\tl{
\pada{yenopadiṣṭā haṭhayogavidyā/}\\+}
\tl{
\pada{\app{\lem[wit={ceteri}]{virājate}% +P11
	\rdg[wit={C6,Jyo}]{vibhrājate}} pronnata%
\app{\lem[wit={ceteri},alt={rājasaudham}]{rājasaudha\skp{m}} % °saudhasa N23,
	\rdg[wit={N3}]{<<rāja>>saudham}
	\rdg[wit={G4,J10,V3,Jyo}]{rājayogam}}}-\\+}%
\tl{
\pada{\app{\lem[wit={ceteri},alt={āroḍhum}]{\skm{m }āroḍhu\skp{m}}
	\rdg[wit={P15}]{ārūḍham}}m icchor adhi% syllable "cchora" gap P15
\app{\lem[wit={ceteri}]{rohiṇīva}
	\rdg[wit={N23}]{rohiṇī ca}
	\rdg[wit={P11}]{rohiṇe ca}
	\rdg[wit={V3}]{roha eva}}//\versenr}
%	\NotIn{J5,G11}
	\\!}
\end{tlg}

\commcite\newpage
%\ \newpage

%1.2
\begin{tlg}[hp01_002]
\tl{\app{\lem[nolem]{}
	\rdg[wit={G11},alt={\om}]{\skp{\om}}}%
\pada{praṇamya śrīguruṃ nāthaṃ}
\pada{svātmārāmeṇa %°<<rā>>meṇa V19; ātmā F
\app{\lem[wit={ceteri}]{yoginā}
	\rdg[wit={V19,E2,V15}]{dhīmatā}}/}\\+}
\tl{
\pada{kevalaṃ rājayogāya}
\pada{\app{\lem[wit={ceteri}]{haṭhavidyo}
	\rdg[wit={J10}]{haṭhayogo}% +F
	}padiśyate//\versenr} % <<di>> V3; °yahiśyate J10ac
%	\NotIn{G11}
	\\!}
\end{tlg}

\trcite{2}%\newpage

%1.3
\begin{tlg}[hp01_003]
\tl{\app{\lem[nolem]{}
	\rdg[wit={G11},alt={\om}]{\skp{\om}}}%
\pada{\app{\lem[wit={ceteri}]{bhrāntyā}
	\rdg[wit={N3,P15}]{bhrāntā}
	\rdg[wit={V19,E2,V3}]{bhrāntvā}% °tvāṃ V3
	}
	bahu\app{\lem[wit={ceteri}]{matadhvānte} % thāṃte N3
	\rdg[wit={N23,C6}]{matadhvāntai}
	\rdg[wit={J5}]{mataṃ dhīmāt}
	\rdg[wit={P11}]{manaṃ dhāṃti}
	\rdg[wit={V1}]{mataṃ bhrāntaṃ}
	}}
\pada{rāja\app{\lem[wit={N3,G4,N23,P11},alt={mārgam}]{mārga\skp{m}}
	\rdg[wit={J5,V19,E2,G11,P15,V15,V1,J10,C6,V3,Jyo}]{yogam}
	}%
\app{\lem[wit={ceteri},alt={ajānatām}]{\skm{m }ajānatām}% °natī P11
	\rdg[wit={V19,E2,V15}]{ajānataḥ}}/}\\+}
\tl{
\pada{haṭhapradīpikāṃ  % °dipikāṃ N23
	\app{\lem[wit={ceteri}]{dhatte}
	\rdg[wit={P11,C6}]{datte}
	\rdg[wit={E2}]{kurve}}}
\pada{\app{\lem[wit={ceteri}]{svātmārāmaḥ}% $$
	\rdg[wit={P11,V3}]{svātmārāma}
	\rdg[wit={N23}]{svātmārame}}
\app{\lem[wit={ceteri}]{kṛpākaraḥ}
	\rdg[wit={J5}]{kṛtāpara}
	\rdg[wit={V19}]{kṛpāparaḥ}
	\rdg[wit={J10}]{kṣamākaraḥ}
	\rdg[wit={V3}]{prakāśyate}}//\versenr}
%	\NotIn{G11}
	\\!}
\end{tlg}

\commcite\newpage

%1.4
\begin{tlg}[hp01_004]
\tl{\app{\lem[nolem]{}
	\rdg[wit={G11},alt={\om}]{\skp{\om}}}%
\pada{\app{\lem[wit={ceteri}]{haṭhavidyāṃ}
	\rdg[wit={J5,N23,V1},alt={°vidyā}]{haṭhavidyā}
	\rdg[wit={N3},alt={°vidyo}]{haṭhavidyo}} hi
\app{\lem[wit={ceteri}]{matsyendra}% °draḥ P11
	\rdg[wit={N23}]{tachaṃdra}}}%
\pada{\app{\lem[wit={cetwG4}]{gorakṣādyā vijānate}% °dya C6
	\rdg[wit={N3}]{gorakṣādyā virājate}
	\rdg[wit={J5}]{gorakṣādiṣu rājate}}/}\\+}
\tl{
\pada{\app{\lem[wit={ceteri}]{svātmārāmo}
	\rdg[wit={G4}]{svātmārāmas}
	\rdg[wit={V19,E2,P15}]{ātmārāmo}}}%
\pada{\app{\lem[wit={ceteri}]{'thavā}
	\rdg[wit={G4}]{tathā}
	\rdg[wit={P15}]{mahā}} yogī
\app{\lem[wit={ceteri}]{jānīte}
	\rdg[wit={V3}]{jānaṃte}}
	ta\app{\lem[wit={ceteri},alt={prasādataḥ}]{\skm{t}prasādataḥ} % <<ta>>t V3
	\rdg[wit={J5}]{prasīdati}}//\versenr}
%	\NotIn{G11}
	\\!}
\end{tlg}

\commcite%\newpage

%1.5
\begin{tlg}[hp01_005]
\tl{\app{\lem[nolem]{}
	\rdg[wit={G11},alt={\om}]{\skp{\om}}}%
\pada{\app{\lem[wit={ceteri}]{śrīādinātha}% nāthā N23
	\rdg[wit={V1,J10}]{ādināthādi}}%
	matsyendra}% °draḥ P11
\pada{\app{\lem[wit={N3,N23,J10,C6,Jyo}]{śābarā}
	\rdg[wit={J5,V1}]{śabarā}
	\rdg[wit={V19,E2,P11}]{śāradā}
	\rdg[wit={V3}]{śāgarā}
	\rdg[wit={P15,V15}]{sāgarā}}nanda%
	\app{\lem[wit={ceteri}]{bhairavāḥ}% °vā P11,V3
	\rdg[wit={N3,J5}]{bhairavaḥ}}/}\\+}
\tl{
\pada{\app{\lem[wit={ceteri}]{cauraṅgī}% vau° P11
	\rdg[wit={J5,P15}]{coraṃgī}
	\rdg[wit={E2}]{caurāṅgī}
	\rdg[wit={G4}]{saraṃgi}% sāraṅgī F
	}%
\app{\lem[wit={ceteri}]{mīna}% nātha F
	\rdg[wit={P15}]{mena}
	\rdg[wit={G4}]{megha}
	\rdg[wit={V19}]{ṣīna}
	\rdg[wit={J5}]{pāna}}gorakṣa}%
\pada{\app{\lem[wit={ceteri}]{virūpākṣa}
	\rdg[wit={N3,V19}]{virūpākṣaḥ}
	\rdg[wit={P15}]{vairūpākṣa}}%
\app{\lem[wit={G4,N23,E2,G11,V15,V1,J10,C6,V3,Jyo}]{bileśayāḥ} % °yā V3
	\rdg[wit={P15}]{baleśayāḥ}
	\rdg[wit={P11}]{bileśyayaḥ}
	\rdg[wit={N3pc,J5,V19}]{savālikaḥ}
	\rdg[wit={N3ac}]{savālmikaḥ}}//\versenr}
%	\NotIn{G11}
	\\!}
\end{tlg}

\commcite\newpage

%\reversemarginpar
%1.6
\begin{tlg}[hp01_006]
\tl{\app{\lem[nolem]{}
	\rdg[wit={G11},alt={\om}]{\skp{\om}}}% +G5
\pada{\app{\lem[wit={ceteri}]{manthāna}
	\rdg[wit={E2}]{manthāra}
	\rdg[wit={Jyo}]{manthāno}}%
\app{\lem[wit={ceteri}]{bhairavo}
	\rdg[wit={N23}]{bhairavā}} yogī}
\pada{\app{\lem[wit={ceteri}]{siddha}
	\rdg[wit={J5}]{siddhi}
%	\rdg[wit={Jyo}]{siddhir}
	\rdg[wit={V1}]{suddha}
	\rdg[wit={J10,C6,Jyo}]{śuddha}% śraddha C6
	}%
\app{\lem[wit={cetwG4}]{buddhaś ca}% +P11
	\rdg[wit={N3,J5,C6,V3}]{buddhiś ca}% budhi V3
	}
\app{\lem[wit={V19,E2,Jyo}]{kanthaḍiḥ}% +L2
	\rdg[wit={N23}]{kanthaḍīḥ}
	\rdg[wit={J5,V15,GrB}]{kanthaḍī}
	\rdg[wit={N3,V1}]{kanthalī}
	\rdg[wit={P15}]{kanthaliḥ}
	\rdg[wit={G4}]{kaṃdaḷi}
	\rdg[wit={J10}]{kandalī}}/}\\+}
\tl{
\pada{\app{\lem[nolem]{\skp{pāda c}}
	\rdg[wit={V19,E2},alt={\om}]{\skp{\om}}}%
\app{\lem[resp=emend]{goraṇṭakaḥ}
	\rdg[wit={N3,J5}]{goraṃṭaka}
	\rdg[wit={G4}]{gorakṣakas}
	\rdg[wit={N23,P15,C6,Jyo}]{koraṇṭakaḥ}
	\rdg[wit={V15}]{koraṃḍakaḥ}
	\rdg[wit={P11}]{karaṃṭaka}
	\rdg[wit={V1,J10,V3}]{pauraṇṭakaḥ}
%	\rdg[wit={V19,E2},alt={\om}]{\skp{\om}}
	} % +C1
\app{\lem[wit={ceteri}]{surānandaḥ}
	\rdg[wit={J5,P11,V3}]{surānanda}
	\rdg[wit={G4}]{sadānaṃda}
	\rdg[wit={N3}]{kṣurānaṃda}
%	\rdg[wit={V19,E2},alt={\om}]{\skp{\om}}
	}}
\pada{\app{\lem[nolem]{\skp{pāda d}}
	\rdg[wit={V19,E2},alt={\om}]{\skp{\om}}}%
\app{\lem[wit={ceteri}]{siddhapādaś ca}
	\rdg[wit={V15}]{śrīpādaś caiva}
%	\rdg[wit={V19,E2},alt={\om}]{\skp{\om}}
	}
\app{\lem[wit={N23,V15,J10,Jyo}]{carpaṭiḥ}
	\rdg[wit={N3,P11,V3}]{carpaṭi}
	\rdg[wit={V1,C6}]{carpaṭī}
	\rdg[wit={G4}]{sarpaṭi}% sarpaṭī F
	\rdg[wit={J5}]{paryaṭī}
	\rdg[wit={P15}]{karpaṭiḥ}
%	\rdg[wit={V19,E2},alt={\om}]{\skp{\om}}
	}//\versenr} 
%	\lineom{cd}{V19,E2}
%	\NotIn{G11}
	\\!}
\end{tlg}

\commcite\newpage

%1.7
\begin{tlg}[hp01_007]
\tl{\app{\lem[nolem]{}
	\rdg[wit={G11},alt={\om}]{\skp{\om}}}% +G5
\pada{\app{\lem[wit={J5,N23,V1,C6}]{kaṇerī}
	\rdg[wit={N3}]{kaṇeri}
	\rdg[wit={P15}]{kaṇeriḥ} % kanerī L2,N19
	\rdg[wit={G4}]{ka[ṇ]e\,..}
	\rdg[wit={V15}]{kāṇeriḥ}
	\rdg[wit={J10,V3,Jyo}]{kānerī}
	\rdg[wit={V19}]{kariṇī}
	\rdg[wit={E2}]{karaṇī}% +C7
	\rdg[wit={P11}]{kāroṭiḥ}}
\app{\lem[wit={ceteri}]{pūjya}
	\rdg[wit={V1}]{pūrya}
	\rdg[wit={P15,J10}]{pūrva}
	\rdg[wit={J5}]{pūla}}pādaś ca}
\pada{\app{\lem[wit={ceteri}]{nityanātho}
	\rdg[wit={V19,E2}]{bilvanātho}
	\rdg[wit={J10}]{dhvaninātho}
	}
	nirañjanaḥ/}\\+}  % °naṃ V3
\tl{
\pada{kapālī bindunāthaś ca} % biṃduś ca J5
\pada{kāka\app{\lem[wit={N23,P15,V15,V1,C6,Jyo}]{caṇḍīśvarāhvayaḥ}% svarā N23,C6
	\rdg[wit={V19,E2,J10,V3}]{caṇḍīśvarādayaḥ}% svarā V19
	\rdg[wit={N3}]{caṃḍeśvaro gayaḥ}
	\rdg[wit={G4}]{caṃḍīśvaro gayaḥ}
	\rdg[wit={J5}]{caṃḍīśvaro gajaḥ}
	\rdg[wit={P11}]{caṃḍīsvaro mayaḥ} % mayaḥ M1,etc.
	}//\versenr}
%	\NotIn{G11}
	\\!}
\end{tlg}

\commcite\newpage


%1.8
\begin{tlg}[hp01_008]
\tl{\app{\lem[nolem]{}
	\rdg[wit={G11},alt={\om}]{\skp{\om}}}% +G5
\pada{\app{\lem[wit={G4}]{allama}% +G3,M1?
	\rdg[wit={N3}]{alama}
	\rdg[wit={P15,V1,J10,V3,Jyo}]{allamaḥ}
	\rdg[wit={J5}]{allumaḥ}
	\rdg[wit={N23}]{allasaḥ}
	\rdg[wit={P11}]{allasa}
	\rdg[wit={C6}]{alasaḥ}
	\rdg[wit={V15}]{amelleḥ}
	\rdg[wit={V19,E2}]{sukṣamaḥ}}prabhudevaś ca}
\pada{\app{\lem[wit={J5,V15,P11,V3,Jyo}]{ghoḍācolī}
	\rdg[wit={G4,N23}]{ghoḍācūlī}% cūli? G4
	\rdg[wit={N3}]{ghoḍāculī}
	\rdg[wit={C6}]{goḍācūlī}
	\rdg[wit={V19}]{ghoṭācolī}% +C7
	\rdg[wit={E2,V1,J10}]{ghorācolī}
	\rdg[wit={P15}]{\_\,gacolī}}
\app{\lem[wit={ceteri}]{ca} % cicaṃḍilaḥ G4
	\rdg[wit={G4}]{ci}
	\rdg[wit={V19}]{sa}
	\rdg[wit={E2}]{gha}} % +C7
\app{\lem[wit={N3,E2,Jyo}]{ṭiṇṭiṇiḥ}
	\rdg[wit={N23}]{tīṭiṇiḥ}
	\rdg[wit={V19}]{ṭiṃṭiniḥ}
	\rdg[wit={P11}]{ṭiṃṭiṇi}
	\rdg[wit={C6}]{ṭiṃṭaṇī<<ḥ>>}
	\rdg[wit={V1,J10}]{ṭiṇṭiṇī}
	\rdg[wit={J5}]{ṭīṃcaṇī}
	\rdg[wit={P15}]{caṃcaṇiḥ}
	\rdg[wit={V15}]{ciṃciṇīḥ}
	\rdg[wit={V3}]{ciṃciṇī}
	\rdg[wit={G4}]{caṃḍilaḥ}}/}\\+}
\tl{
\pada{\app{\lem[wit={P11},alt={bhālukir}]{bhāluki\skp{r}}
	\rdg[wit={N3}]{bhāluki}
	\rdg[wit={V1,J10,C6,Jyo}]{bhālukī}% Jyo-mss,N19
	\rdg[wit={N23}]{bhānukin}
%	\rdg[wit={Jyo}]{bhānukī}
	\rdg[wit={J5}]{tālukī}
	\rdg[wit={V19,E2}]{vālukir}
	\rdg[wit={G4}]{vāsuki}
	\rdg[wit={P15,V3}]{vāsukir}
	\rdg[wit={V15}]{vāsukīr}
	}%
\app{\lem[wit={G4,N23,V1,J10,V3},alt={nāgabodhaś ca}]{\skm{r }nāgabodhaś ca}% °bīdhaś ca? G4
	\rdg[wit={V19,E2}]{nāgarodhaś ca}
	\rdg[wit={P15,V15,P11}]{nāgabodhiś ca}
	\rdg[wit={C6,Jyo}]{nāgadevaś ca}
	\rdg[wit={N3}]{namioḍḍīśa}
	\rdg[wit={J5}]{nāma tuṃḍīśa}
%	\rdg[wit={Jyo}]{nāradevaś ca} % nāgadevaś Jyo-mss
	}}
\pada{\app{\lem[wit={G4,E2,J10,P11,V3,Jyo},alt={khaṇḍakāpālikas}]{khaṇḍakāpālika\skp{s}}% Jyo-mss
	\rdg[wit={N23,V1}]{khaṃḍaṃ kāpālikas}
	\rdg[wit={V15}]{khaṇḍaḥ kāpālikas}
	\rdg[wit={C6}]{khaṃḍikaḥpālikas}
	\rdg[wit={P15}]{kaṃḍhaḥ kāpālikas}
	\rdg[wit={N3,J5}]{siddhaḥ kāpālikas}
	\rdg[wit={V19}]{caṃḍīkāpālikas}
	}s tathā//\versenr}
%	\NotIn{G11}
	\\!}
\end{tlg}

\commcite\newpage

%1.9
\begin{tlg}[hp01_009]
\tl{\app{\lem[nolem]{}
	\rdg[wit={G11},alt={\om}]{\skp{\om}}}% +G5
\pada{ityādayo mahāsiddhā}
\pada{haṭhayoga\app{\lem[wit={ceteri}]{prabhāvataḥ}
	\rdg[wit={N23,GrB}]{prasādataḥ}}/}\\+}
\tl{
\pada{khaṇḍayitvā % khaṃḍaïtvā V19
	\app{\lem[wit={ceteri}]{kāla}
	\rdg[wit={N3}]{kāra}
	\rdg[wit={P11}]{kā}}daṇḍaṃ}
\pada{\app{\lem[wit={N3,J5,V19,V15,P11}]{brahmāṇḍeṣu}
	\rdg[wit={V3}]{brahmāṇḍe tu}
	\rdg[wit={G4,N23,E2,V1,J10,C6,Jyo}]{brahmāṇḍe vi°}% +N19
	\rdg[wit={P15}]{brahmāṇḍaṃ vi°}} caranti te//\versenr} % varaṃti P11
%	\NotIn{G11}
	\\!}
\end{tlg}

\commcite\newpage

%\newpage%\normalmarginpar
%1.10
\begin{tlg}[hp01_010]
\tl{
\pada{\app{\lem[wit={ceteri}]{saṃsāratāpa}
	\rdg[wit={V3}]{saṃsārāśrama}
	\rdg[wit={J10}]{saṃsāraśrama}
	\rdg[wit={C6,Jyo}]{aśeṣatāpa}}taptānāṃ}
\pada{\app{\lem[wit={ceteri}]{samāśraya}
	\rdg[wit={V1}]{samāśrayo}
	\rdg[wit={J10,V3}]{āśrayo'yaṃ}
	\rdg[wit={N3}]{samagrapra°}}%
\app{\lem[wit={N23,E2,P15,C6,Jyo}]{maṭho haṭhaḥ}
	\rdg[wit={J5,P11}]{maho haṭhaḥ}% ḥ om. P11
	\rdg[wit={G11}]{maṭho ṭhahaḥ}
	\rdg[wit={V19}]{mato haṭhaḥ}% +L2
	\rdg[wit={G4}]{mato\,(ṭho \emph{pc}) ha\,..}
	\rdg[wit={V15}]{mahāmaṭhaḥ}
	\rdg[wit={V1}]{haṭho maṭhaḥ}
	\rdg[wit={J10,V3}]{haṭho mataḥ}
	\rdg[wit={N3}]{°thamo haṭhaḥ}}/}\\+}
\tl{
\pada{\app{\lem[nolem]{\skp{pāda c}}
	\rdg[wit={Gr1},alt={\om}]{\skp{\om}}}%
\app{\lem[wit={ceteri}]{aśeṣa}% aśeṣe N23
	\rdg[wit={P15}]{samasta}
%	\rdg[wit={Gr1},alt={\om}]{\skp{\om}}
	}yoga%
\app{\lem[wit={N23,V19,E2,G11,P15,V15,J10},alt={jagatām}]{jagatā\skp{m}}
	\rdg[wit={V1}]{jagatīm}
	\rdg[wit={P11}]{jāyunām}
	\rdg[wit={C6,V3,Jyo}]{yuktānām}% +V1mg
%	\rdg[wit={Gr1},alt={\om}]{\skp{\om}}
	}}%
\pada{\app{\lem[nolem]{\skp{pāda d}}
	\rdg[wit={Gr1},alt={\om}]{\skp{\om}}}%
\app{\lem[wit={ceteri},alt={ādhāra}]{\skm{m }ādhāra}% ādhāre? V1
	\rdg[wit={V19,J10,V3}]{ādhāraḥ}
	\rdg[wit={E2}]{ādhārai}
%	\rdg[wit={Gr1},alt={\om}]{\skp{\om}}
	}%
\app{\lem[wit={ceteri}]{kamaṭho haṭhaḥ}% kamatho N23; ḥ om. P11
	\rdg[wit={V1}]{kahaṭho maṭhaḥ}
%	\rdg[wit={Gr1},alt={\om}]{\skp{\om}}
	}//\versenr}
%	\lineom{cd}{Gr1}
	\\!}
\end{tlg}

\commcite\newpage

%1.11
\begin{tlg}[hp01_011]
\tl{
\pada{haṭhavidyā paraṃ
\app{\lem[wit={ceteri}]{gopyā}
	\rdg[wit={J5,V3}]{gopyaṃ}
	\rdg[wit={P11}]{yogo}}
\app{\lem[wit={ceteri}]{yogināṃ}
	\rdg[wit={V3}]{yogīnāṃ}
	\rdg[wit={V19,E2,P15,Jyo}]{yoginā}}}
\pada{siddhi\app{\lem[wit={Gr1,N23,G11,V15,V1,J10,V3},alt={icchatām}]{\skm{m }icchatām}
	\rdg[wit={E2,P15,P11,C6,Jyo}]{icchatā}
	\rdg[wit={V19}]{icchitā}}/}\\+}
\tl{
\pada{bhaved vīryavatī guptā} % virya P11; gopyā J5
\pada{\app{\lem[wit={ceteri}]{nirvīryā} % °viryā N3, niviryā P11, nivīryā J5,C6
	\rdg[wit={V19}]{nirvījā}
	\rdg[wit={J10}]{nirvāryā}}
\app{\lem[wit={cetwG4}]{tu}
	\rdg[wit={N3,V19,V15}]{ca}
	\rdg[wit={P15}]{sā}} prakāśitā//\versenr}
	\\!}
\end{tlg}

\commcite\newpage

%1.12
\begin{tlg}[hp01_012]
\tl{
\pada{\app{\lem[wit={ceteri}]{surājye}
	\rdg[wit={E2}]{svarājye}
	\rdg[wit={P11}]{surāṣṭre}} dhārmike deśe}
\pada{subhikṣe % surbhikṣe C6,J10, °bhicche V19
\app{\lem[wit={ceteri}]{nirupadrave}
	\rdg[wit={P15}]{kṣemabhadrade}}/}%
	\\+}
\tl{
\pada{\app{\lem[wit={N3,G4,N23,V19,E2,P15,V15,P11,C6}]{ekānta}
	\rdg[wit={J5,G11,V1,J10,V3,Jyo}]{ekānte}}% 
	maṭhikāmadhye} % mahikā G11; sidhyai C6
\pada{sthātavyaṃ 
	haṭha\app{\lem[wit={N3,G4,G11}]{yoginām}
	\rdg[wit={N23,V19,E2,P15,V15,V1,J10,GrB,Jyo}]{yoginā}
	\rdg[wit={J5}]{yogibhiḥ}}//\versenr}%
	\myfn{\getsiglum{Jyo} has \devnote{dhanuḥpramāṇaparyantaṃ śilāgnijalavarjite} inserted between the two hemistiches.}%
	\\!}
\end{tlg}

\commcite\newpage

%1.13
\begin{tlg}[hp01_013]
\tl{
\pada{\app{\lem[wit={ceteri}]{alpadvāram arandhra}
	\rdg[wit={N23}]{ākalpadvā<<ra>>raṃdhra}
	\rdg[wit={P15}]{alpadvāram aruṃ<<dha>>}}garta%
\app{\lem[wit={N3,P15,V15,P11}]{piṭakaṃ}% ṭha N3
	\rdg[wit={G4}]{piṭanaṃ}
	\rdg[wit={C6}]{paṭikaṃ}
	\rdg[wit={N23}]{piṭhikaṃ}
	\rdg[wit={V19,E2ac}]{piṭharaṃ}
	\rdg[wit={E2pc}]{piṭhiraṃ}% +C7
	\rdg[wit={J5,G11}]{viṭakaṃ}
	\rdg[wit={J10,V3}]{viṭapaṃ}
	\rdg[wit={Jyo}]{vivaraṃ}
	\rdg[wit={V1}]{sahitaṃ}}
nātyucca\app{\lem[wit={ceteri}]{nīcā}
	\rdg[wit={N3pc}]{nītā}
	\rdg[wit={N23}]{naṃcā}
	\rdg[wit={V3}]{noccā}
	}%
\app{\lem[wit={ceteri},alt={°yataṃ}]{\skp{°}yataṃ}
	\rdg[wit={E2,J10}]{yitaṃ}
	\rdg[wit={V1,P11}]{yutaṃ}
	\rdg[wit={V15}]{vṛtaṃ}}}\\+}
\tl{
\pada{samyaggomaya% saṃgomaya N3, samyago° V3, samyak*go° P11,V1
\app{\lem[wit={ceteri}]{sāndra}
	\rdg[wit={J10,V3}]{sārdra}
	\rdg[wit={N3}]{sāpra}
	\rdg[wit={P15}]{lipta}}%
\app{\lem[wit={G4,N23,V19,E2,J10,C6,Jyo}]{liptam amalaṃ}% littam G4
	\rdg[wit={N3,J5,G11,V15,P11,V3}]{liptavimalaṃ}% +F
	\rdg[wit={V1}]{liptamavilaṃ}% abilaṃ V1
	\rdg[wit={P15}]{sāndravimalaṃ}}
\app{\lem[wit={ceteri}]{niḥśeṣa}% +G4; niḥśeśa˟ P11
	\rdg[wit={N3,J10}]{nirdoṣa}}%
\app{\lem[wit={G11,P15,V1,J10,V3}]{bādhojjhitam} % °jhitaṃ V3, bādhodgataṃ N19; +F
	\rdg[wit={G4}]{bodhojhitaṃ}
	\rdg[wit={P11}]{bādhārjjitaṃ}
	\rdg[wit={C6}]{vātojjhitaṃ}
	\rdg[wit={N23,V19,E2,V15,Jyo}]{jantūjjhitam}% jhi V19, jaṃbhūjhitaṃ N23
	\rdg[wit={N3}]{jyaṃtyūpsitaṃ}
	\rdg[wit={J5}]{jaṃtūṣṇitaṃ}
	}/}\\+}
\tl{
\pada{bāhye maṇḍapa% maṃṭapa G4
\app{\lem[wit={ceteri}]{vedikūpa}
	\rdg[wit={G11,P11}]{vedakopa}
	\rdg[wit={V15}]{kūpavedi}}%
\app{\lem[wit={N3,J5,N23,V15,GrB,Jyo}]{ruciraṃ}
	\rdg[wit={V19,E2,P15,V1,J10}]{racitaṃ}
	\rdg[wit={G11}]{ricitaṃ}}
	prākārasaṃveṣṭitaṃ}\\+}
\tl{
\pada{proktaṃ yogamaṭhasya lakṣaṇam idaṃ siddhair haṭhābhyāsibhiḥ//\versenr}
\\!}
% mahasya, siddhai, °bhyāsabhiḥ N3
\end{tlg}

\commcite\newpage
\ \newpage

%1.14
\begin{tlg}[hp01_014]
\tl{
\pada{evaṃvidhe maṭhe sthitvā} % viddhe C6; madhye J5, māva P11
\pada{sarvacintā\app{\lem[wit={ceteri}]{vivarjitaḥ}
	\rdg[wit={P15,P11}]{vivarjite}}/}\\+}
\tl{
\pada{\app{\lem[wit={ceteri}]{gurūpadiṣṭa} % gurupa N23,V3
	\rdg[wit={G11}]{gurūpadeṣṭa}
	\rdg[wit={C6}]{gurūpadeśa}}mārgeṇa}
\pada{\app{\lem[wit={ceteri}]{yogam eva}
	\rdg[wit={J5,P15,V1,J10}]{yogam evaṃ}
	\rdg[wit={V3}]{yogamārgaṃ}}
\app{\lem[wit={N3ac,G4,V19,E2,V15,P11,C6,Jyo}]{sadābhyaset}
	\rdg[wit={N3pc,J5,N23,G11,P15,V1,J10,V3}]{samabhyaset}% +F
	}//\versenr}
	\\!}
\end{tlg}

\commcite%\newpage

%1.15
\begin{tlg}[hp01_015]
\tl{
\pada{\app{\lem[wit={N23,E2,G11,V1,J10,C6,Jyo}]{atyāhāraḥ}
	\rdg[wit={J5,V15,P11}]{atyāhāra}
	\rdg[wit={N3}]{alpāhāro}
	\rdg[wit={V19}]{alpāhāra}
	\rdg[wit={V3}]{ātmāhāraḥ}
	\rdg[wit={P15}]{natyāhāsaḥ}
	\rdg[wit={G4}]{a[t].\,+\,+}}
\app{\lem[wit={ceteri}]{prayāsaś ca}
	\rdg[wit={V19}]{prayāsaś cā}
	\rdg[wit={J5}]{prayāsasya}
	\rdg[wit={N3,N23}]{pravāsaś ca}% +F
	\rdg[wit={G4},alt={\illeg}]{\skp{\illeg}}}}
\pada{prajalpo% °jālpā P11
\app{\lem[wit={ceteri},postwit=\texteng{(The \emph{avagraha} was added by the editors)}]{'niyama}% J10 corrects to niyamo, then deletes the inserted "o"
	\rdg[wit={Jyo}]{niyama}
	\rdg[wit={V15}]{'niyamā}% 
	\rdg[wit={P15}]{nama}}grahaḥ/}\\+} % grahaṃ G11
\tl{
\pada{\app{\lem[wit={ceteri}]{janasaṅgaś ca}
	\rdg[wit={C6}]{janasaṃgaṃ ca}
	\rdg[wit={P15}]{janasaṃkara}} laulyaṃ ca} % cālpolpaṃ P11, laulyaś ca J5
\pada{ṣaḍbhi\app{\lem[wit={ceteri},alt={yogo vinaśyati}]{\skm{r }yogo vinaśyati}% °syati V19
	\rdg[wit={N23}]{yogā vinaśyati}
	\rdg[wit={V1}]{yogaḥ praṇaśyati}
	\rdg[wit={J10}]{yogaś ca naśyati}
	}//\versenr}
	\\!}
\end{tlg}

\commcite\newpage


%1.16
\begin{tlg}[hp01_016]
\tl{
\pada{\app{\lem[wit={ceteri},alt={utsāhāt/n}]{utsāhā\skp{t/n}} % °h<<ā>><n> V15
	\rdg[wit={J5,V3}]{utsāha}
	\rdg[wit={G4}]{utsāho}}%
\app{\lem[wit={N23,V19,E2,Jyo},alt={sāhasād}]{\skm{t }sāhasā\skp{d}}
	\rdg[wit={N3,G11,P15,V15,V1,J10,P11,V3}]{niścayād}% ##
	\rdg[wit={J5}]{niścayā}
	\rdg[wit={G4}]{niścalo}
	\rdg[wit={C6}]{niyamād}}%
\app{\lem[wit={N23,V19,E2,G11,P15,V15,P11,C6,Jyo},alt={dhairyāt}]{\skm{d }dhairyā\skp{t}}
	\rdg[wit={N3,J5,V3}]{dhairyā}
	\rdg[wit={G4}]{dhairyaṃ}
	\rdg[wit={V1,J10}]{vairyāt}
	}}%t}
\pada{\app{\lem[wit={G11},alt={saṃtoṣāt tattvadarśanāt}]{\skm{t }saṃtoṣāt tattvadarśanāt}
	\rdg[wit={N3,P15,V15,V1,J10,P11,V3}]{tattvajñānāc ca darśanāt} % °jñānā ca V3, °jñā ca P11; +M1,M4
	\rdg[wit={J5}]{tattvajñānārthadarśanāt}
	\rdg[wit={G4}]{tattvajñānārthadarśanaṃ}
	\rdg[wit={N23,Jyo}]{tattvajñānāc ca niścayāt}
	\rdg[wit={V19,E2}]{tattvajñānād viniścayāt}
	\rdg[wit={C6}]{tattvajñānaviniścayāt}
	}/}\\+}
\tl{
\pada{\app{\lem[wit={ceteri}]{jana}
	\rdg[wit={N23}]{nija}}saṅga%
	\app{\lem[wit={ceteri},alt={parityāgāt}]{parityāgā\skp{t}}
	\rdg[wit={G4}]{parityāgāḥ}}%
	t}%
\pada{ṣaḍbhi\app{\lem[wit={ceteri}, alt={yogaḥ prasidhyati}]{\skm{r }yogaḥ prasidhyati}% yoga V3
	\rdg[wit={J5,V19}]{yogo prasidhyati}
	\rdg[wit={V1,J10}]{yogas tu sidhyati}
	}//\versenr}
	\\!}% °te J5
\end{tlg}

\commcite\newpage

\startaltrecension
\begin{alttlg}[hp01_016_1]
\tl{\app{\lem[nolem]{}
	\rdg[wit={Gr6}]{\incl}}%
\pada{ahiṃsā satyam asteyaṃ}
\pada{brahmacaryaṃ kṣamā dhṛtiḥ/}\\+
}\tl{
\pada{\app{\lem[resp=emend]{dayārjavaṃ}
	\rdg[wit={E4,V3}]{dayārjava}
	\rdg[wit={J6}]{devārcanaṃ}
	} 
	mitāhāraḥ} % ḥ om. E4
\pada{śaucaṃ caiva yamā daśa//\versenr}
\sgwit{Gr6}\\!
}\end{alttlg}

\teimute{\begin{quote}%
\textcolor{gray}{\ExecuteMetaData[\commfilename]{tr16-1}
\texteng{(16*1)}}
\end{quote}}

\begin{alttlg}[hp01_016_2]
\tl{\app{\lem[nolem]{}
	\rdg[wit={Gr6}]{\incl}}%
\pada{tapaḥ saṃtoṣam āstikyaṃ}
\pada{dānam īśvarapūjanam/}\\+
}\tl{
\pada{\app{\lem[wit={J6,E4}]{siddhānta}
	\rdg[wit={V3}]{siddhāntaṃ}}śravaṇaṃ caiva}
\pada{\app{\lem[wit={J6pc}]{hrīr matiś ca}
	\rdg[wit={J6ac,V3}]{hrī matiś ca}
	\rdg[wit={E4}]{hrī matī ca}
	} 
	japo hutam//\versenr} 
	\sgwit{Gr6}\\!
}\end{alttlg}
\endaltrecension

% \begin{alttlg}[hp01_016_3]
% \tl{
% \pada{niyamā daśa saṃproktā}
% \pada{yogaśāstraviśāradaiḥ//\versenr} \sgwit{V17}\\!
% }\end{alttlg}

\teimute{%
\def\commvnum{16-2}%
\def\labelvnum{16*1--2}%
\begin{quote}%
\textcolor{gray}{\ExecuteMetaData[\commfilename]{tr\commvnum}
\texteng{(16*2)}}%\footnotetext{\commlabel%
%\ExecuteMetaData[\commfilename]{sc\commvnum}%
%\ExecuteMetaData[\commfilename]{ts\commvnum}%
%\ExecuteMetaData[\commfilename]{cm\commvnum}%
%}
\end{quote}}
\newpage

%\begin{altava}[hp01_017a]
%\grau{\app{\lem[wit={P15}]{athāsanāni}
%	\rdg[wit={V15}]{<<atha>> āsanāni}}/
%	\sgwit{P15,V15}}
%\end{altava}

%1.17
\begin{tlg}[hp01_017]
\tl{\app{\lem[nolem]{}
	\rdg[wit={V19,E2,V15},alt=\textapp{transposed with the next verse}]{\skp{transposed with the next verse}}}%
\app{\lem[alt={\ante haṭhasya \add},nosep]{\skp{\ante haṭhasya \add}}
	\rdg[wit={P15}]{athāsanāni}
	\rdg[wit={V15}]{<<atha>> āsanāni}}%
\pada{haṭhasya prathamāṅgatvād}%
\pada{āsanaṃ pūrvam ucyate/}\\+}
\tl{
\pada{\app{\lem[wit={cetwG4},alt={tat kuryād}]{tat kuryā\skp{d}}% +G4
	\rdg[wit={J5}]{ta kuryād}
	\rdg[wit={N3}]{na kuryād}
	\rdg[wit={J10,Jyo}]{kuryāt tad}}%
\app{\lem[wit={J5,G4,N23,E2,G11,V1,P11,Jyo},alt={āsanaṃ sthairyam}]{\skm{d }āsanaṃ sthairya\skp{m}}
	\rdg[wit={N3,V19,P15,C6,V3}]{āsanasthairyam}
	\rdg[wit={V15}]{āsane sthairyaṃ} % °yaṃ a°(!) V15
	\rdg[wit={J10}]{āsanaṃ tasmād}}}%
\pada{m ārogyaṃ 
cāṅga\app{\lem[wit={N3,G4,G11,P15,V1,J10,P11,V3}]{pāṭavam}
	\rdg[wit={J5}]{pāṭave}
	\rdg[wit={N23,V19,E2,V15,C6,Jyo}]{lāghavaṃ}}//\versenr}%
%	\myfn{1.17--18 are transposed in \getsiglum{V19,E2,V15}.}
	\label{I17}
	\\!}
\end{tlg}

\commcite%\newpage

%1.18
\begin{tlg}[hp01_018]
\tl{\app{\lem[nolem]{}
	\rdg[wit={N3}]{\unavbl}
	\rdg[wit={V19,E2,V15},alt=\textapp{transposed with the previous verse}]{\skp{transposed with the previous verse}}}%
\pada{vasiṣṭhādyaiś ca munibhi}% vaśi° N23,V15
\pada{r matsyendrādyaiś ca yogibhiḥ/}\\+}% matsyendraiś G4; yogabhiḥ V3
\tl{
\pada{aṅgīkṛtāny āsanāni} % kṛtyāny P11
\pada{\app{\lem[wit={ceteri}]{kathyante}% kathyate J5
	\rdg[wit={V19,E2}]{vakṣyante}
	\rdg[wit={V1}]{likhyante}
	}
	kānicin mayā//\versenr}\linelabel{v18d}%
	\myfn{After this verse \getsiglum{G11} has two additional verses:
	\vspace{2pt minus 1pt}\\ 
	\devnote{svastikaṃ gomukhaṃ patmaṃ kūrmam uttānakūrmakaṃ/
	kukkuṭākhyaṃ ca matsyendraṃ dhanurāsanam eva ca//\\
	tathā paścimatāṇaṃ ca mayūraṃ siddhasaṃjñakaṃ/
	siṃhaṃ bhadraṃ baddhapatmam ityādyaṃ bahudhāsanaṃ//}}%
%\myfn{\getsiglum{N3} is not available for 1.18--28.}%
%	\unavbl{N3}
	\\!}
\end{tlg}

\commcite\newpage


%1.19
\begin{tlg}[hp01_019]
\tl{\app{\lem[nolem]{}
	\rdg[wit={N3}]{\unavbl}
	\rdg[wit={P15},alt=\textapp{found before \ref{I17}}]{\skp{found before 1.17}}}%
\pada{\app{\lem[wit={ceteri},alt={jānūrvor}]{jānūrvo\skp{r}} % jānu° V3
	\rdg[wit={V19}]{jaṃtūrvo}}%
\app{\lem[wit={ceteri},alt={antare}]{\skm{r }antare}
	\rdg[wit={N23,P15}]{antaraṃ}}
\app{\lem[wit={ceteri},alt={samyak}]{samya\skp{k}}
	\rdg[wit={G11}]{deśe}k}}
\pada{kṛtvā pādatale % <<ta>>le N23
\app{\lem[wit={ceteri}]{ubhe}
	\rdg[wit={N23}]{śubhe}}/}\\+}
\tl{
\pada{\app{\lem[wit={ceteri}]{ṛjukāyaḥ}% kāyuḥ P11
	\rdg[wit={J5,N23ac,V1}]{ṛjukāya}
	\rdg[wit={V3}]{ṛjuḥ kāya}}
\app{\lem[wit={ceteri}]{samāsīnaḥ}% °sīna J5
	\rdg[wit={V3}]{samāsīnaṃ}}}
\pada{svastikaṃ \app{\lem[wit={ceteri},alt={tat}]{ta\skp{t}}
	\rdg[wit={G4}]{taṃ}
	\rdg[wit={J5,N23}]{ca}}t pracakṣate//\versenr} % °vakṣate P11
%	\anm{before \ref{I17} \getsiglum{P15}}
%	\unavbl{N3}
	\\!}
\end{tlg}

\commcite\newpage

%1.20
\begin{tlg}[hp01_020]
\tl{\app{\lem[nolem]{}
	\rdg[wit={N3}]{\unavbl}
	\rdg[wit={P15},alt={\om}]{\skp{\om}}}%
\pada{savye
\app{\lem[wit={ceteri}]{dakṣiṇa}
	\rdg[wit={V3}]{dakṣaṇa}
	\rdg[wit={J5}]{dakṣa}}gulphaṃ tu}
\pada{\app{\lem[wit={ceteri}]{pṛṣṭha} % pṛṣṭe J5
	\rdg[wit={C6}]{pṛṣṭhi}
	}pārśve 
	\app{\lem[wit={ceteri}]{niyojayet}
	\rdg[wit={P11}]{tu yojayet}}/}\\+}
\tl{
\pada{dakṣiṇe\app{\lem[wit={ceteri}]{'pi}% +G4
	\rdg[wit={N23,E2}]{ca}
	\rdg[wit={V19}]{tu}% +C7
	} tathā
\app{\lem[wit={ceteri}]{savyaṃ} % BKhP
	\rdg[wit={V1}]{savye}
	\rdg[wit={G4}]{samyak}}}
\pada{\app{\lem[wit={ceteri}]{gomukhaṃ}
	\rdg[wit={E2}]{gomukhe}}  %+C1,C7
\app{\lem[wit={G4,V19,E2,G11,V15,V1,J10}]{gomukhaṃ yathā}
	\rdg[wit={V3}]{gomukhaṃ tathā}
	\rdg[wit={J5}]{gomukhaṃ bhavet}
	\rdg[wit={Jyo}]{gomukhākṛti}
	\rdg[wit={N23,P11,C6}]{gomukhākṛtiḥ}}//\versenr} % nomukhā° C6
%	\NotIn{P15}% +F
%	\unavbl{N3}
	\\!}
\end{tlg}

\commcite\newpage

%1.21
\begin{tlg}[hp01_021]
\tl{\app{\lem[nolem]{}
	\rdg[wit={N3}]{\unavbl}}%
\pada{\app{\lem[wit={ceteri}]{ekaṃ}
		\rdg[wit={V19,V15}]{eka}
		}
	pāda\app{\lem[wit={G4,V1},alt=athaikasmin]{\skm{m }athaikasmi\skp{n}}
	\rdg[wit={J5,N23,G11,P15,V15,J10,GrB,Jyo}]{tathaikasmin}
	\rdg[wit={V19,E2}]{yathaikasmin}
	}}%
\pada{\app{\lem[wit={V1,J10,V3},alt={vinyasyoruṇi saṃsthitam}]{\skm{n }vinyasyoruṇi saṃsthitam}% °ruṃni J10
	\rdg[wit={G11}]{vinyasyoruṇi saṃsthitaḥ}% +G5,G3
	\rdg[wit={G4}]{niveśyorūṇi saṃsthitam}
	\rdg[wit={N23,P11}]{vinyased ūruṇi sthitaṃ}%  ##
	\rdg[wit={C6}]{vinyased ūruṇi sthitaḥ}
	\rdg[wit={Jyo}]{vinyased ūruṇi sthiraṃ}
	\rdg[wit={J5}]{vinyaser upari sthitaṃ}
	\rdg[wit={V19,E2,N19,V15}]{vinyased ūru saṃsthitam}% +IFP; curu N19, uru V15
	\rdg[wit={P15}]{vinasyorasi saṃsthitaḥ}}
	/}\\+}
\tl{
\pada{\app{\lem[wit={ceteri}]{itarasmiṃs tathā}% °smin V3
	\rdg[wit={G4}]{itarasminn adhaś}}
\app{\lem[wit={ceteri}]{coruṃ}
	\rdg[wit={J5,P11}]{coru}% voru P11
	\rdg[wit={G4}]{corū}
	\rdg[wit={G11,C6}]{corau}
	\rdg[wit={P15}]{cairuṃ}% tathā cairuṃ or tatha cauruṃ
	\rdg[wit={E2}]{coktaṃ}
	\rdg[wit={V19}]{ce\,..}}}
\pada{\app{\lem[wit={ceteri},alt={vīrāsanam}]{vīrāsana\skp{m}}
	\rdg[wit={G11}]{patmāsanam}}% 
\app{\lem[wit={ceteri},alt={itīritam}]{\skm{m }itīritam} % īḍitaṃ C6
	\rdg[wit={J5,E2}]{iti smṛtam}% +F
	\rdg[wit={G4,G11}]{udāhṛtaṃ}}//\versenr}
%	\unavbl{N3}
	\\!}
\end{tlg}

\commcite\newpage


\begin{tlg}[hp01_022]
\tl{\app{\lem[nolem]{}
	\rdg[wit={N3}]{\unavbl}}%
\pada{gudaṃ
\app{\lem[wit={G4,J10}]{niṣpīḍya}% +G5,M3 "heftig drücken"
	\rdg[wit={G11}]{nipīḍya}
	\rdg[wit={J5,V19,E2,P11}]{niyamya}% +F
	\rdg[wit={N23,C6,Jyo}]{nirudhya}
	\rdg[wit={P15,V15,V1}]{nibadhya}% nabadhya P15
	\rdg[wit={V3}]{nibaddhi}
	} gulphābhyāṃ}
\pada{vyutkrameṇa % vut° P11
\app{\lem[wit={ceteri}]{samāhitaḥ}
	\rdg[wit={V3}]{samāhitaṃ}
	\rdg[wit={J5}]{samādhinā}}/}\\+}
\tl{
\pada{\app{\lem[wit={ceteri}]{kūrmāsanaṃ}
	\rdg[wit={N23,P11,C6}]{yogāsanaṃ}} bhaved eta}% bhaveddetad P11
\pada{\app{\lem[wit={ceteri},alt={iti}]{\skm{d }iti}% ita J5
	\rdg[wit={C6}]{sarve}} yogavido viduḥ//\versenr}
%	\unavbl{N3}
	\\!}
\end{tlg}

\commcite\newpage


\begin{tlg}[hp01_023]
\tl{\app{\lem[nolem]{}
	\rdg[wit={N3}]{\unavbl}}%
\pada{padmāsanaṃ
\app{\lem[wit={G4,N23,V19,G11,GrB}]{su}
	\rdg[wit={E2,P15,V15,V1,J10,Jyo}]{tu}% +C7
	\rdg[wit={J5}]{stu}}%
\app{\lem[wit={ceteri}]{saṃsthāpya}
	\rdg[wit={V19,E2}]{saṃyojya}}}
\pada{jānūrvor antare karau/}\\+} % jānuvor J5,V3; antaraṃ P15
\tl{
\pada{niveśya bhūmau saṃsthāpya} % niveśa P11, nivesya C6
\pada{\app{\lem[wit={ceteri}]{vyomasthaḥ}
	\rdg[wit={J5},alt={°sthā}]{vyomasthā}
	\rdg[wit={N23,P11,C6,Jyo},alt={°sthaṃ}]{vyomasthaṃ}
	}
\app{\lem[wit={G4,N23,V19,E2,G11,P15,V15,Jyo}]{kukkuṭāsanam}
	\rdg[wit={G4,G11}]{kukkuṭāsanaḥ}
	\rdg[wit={J5,V1,J10,P11,V3}]{kurk(k)uṭāsanam}% rkk V3
	\rdg[wit={C6}]{kukuṭāsanam}
	}//\versenr}
%	\unavbl{N3}
	\\!}
\end{tlg}

\commcite\newpage

%1.24
\begin{tlg}[hp01_024]
\tl{\app{\lem[nolem]{}
	\rdg[wit={N3}]{\unavbl}}%
\pada{\app{\lem[wit={G4,N23,V19,E2,G11,P15,V15,Jyo},alt={kukkuṭā°}]{kukkuṭā\skp{°}}
	\rdg[wit={J5,V1,J10,P11,V3}]{kurk(k)uṭā}% rkk P11,V3
	\rdg[wit={C6}]{kukuṭā}}sana% °sanaṃ V3 
\app{\lem[wit={ceteri}]{bandhastho} % sthau N23? scho V15, sthaḥ G11
	\rdg[wit={J10}]{madhyastho}
	\rdg[wit={V1}]{vat kṛtvā}}}\marmas 
\pada{dorbhyāṃ % ddorbhyāṃ J10ac
\app{\lem[wit={ceteri}]{saṃbadhya}
	\rdg[wit={J5,G4}]{saṃbaṃdha}
	\rdg[wit={N23}]{saṃveṣṭa}
	\rdg[wit={C6}]{saṃhṛtya}}
\app{\lem[wit={ceteri}]{kandharām}
	\rdg[wit={G4,V19,G11,V3}]{kandharam}
	\rdg[wit={P15}]{kandaraṃ}}/}\\+}
\tl{
\pada{\app{\lem[wit={J5,G11,P15,V15,V1,J10}]{śete}
	\rdg[wit={V3}]{śene}
	\rdg[wit={G4}]{sthite}
	\rdg[wit={N23,P11}]{sthitaḥ}% sthitaṃ F
	\rdg[wit={C6}]{sthitvā}
	\rdg[wit={V19,E2,Jyo}]{bhavet}}
\app{\lem[wit={ceteri},alt={kūrmavad}]{kūrmava\skp{d}}
	\rdg[wit={P11},alt={°cad}]{kūrmacad}
	\rdg[wit={V3},alt={°rad}]{kūrmarad}}%
\app{\lem[wit={J5,N23,J10,P11,V3,Jyo},alt={uttāna}]{\skm{d }uttāna}
	\rdg[wit={G4,V19,E2,G11,P15,V15,V1,C6}]{uttānam}}}
\pada{etad uttāna\app{\lem[wit={ceteri}]{kūrmakam}
	\rdg[wit={P15}]{pūrvakaṃ}}//\versenr}
%	\unavbl{N3}
	\\!}
\end{tlg}

\commcite\newpage


%1.25
\begin{tlg}[hp01_025]
\tl{\app{\lem[nolem]{}
	\rdg[wit={N3}]{\unavbl}}%
\pada{pādāṅguṣṭhau % °ṣṭhe G11
\app{\lem[wit={ceteri}]{tu}
	\rdg[wit={V19,E2}]{ca}}
\app{\lem[wit={ceteri}]{pāṇibhyāṃ}
	\rdg[wit={V1,J10}]{bāhubhyāṃ}}}
\pada{gṛhītvā % gṛhitvā V3, gra° G11
\app{\lem[wit={ceteri}]{śravaṇāvadhi}
	\rdg[wit={J5,P11,C6}]{śravaṇāvadhiḥ}
	\rdg[wit={V1}]{śravaṇāvadhiṃ}
	\rdg[wit={V3}]{śravaṇāvidhi}}/}\\+}
\tl{
\pada{dhanurā\app{\lem[wit={V19,E2},alt={ākarṣaṇaṃ kṛtvā}]{\skp{ā}karṣaṇaṃ kṛtvā} % = L1,N19; karkhaṇaṃ V19
	\rdg[wit={G4}]{ākarṣaṇaṃ kṛṣṭaṃ}
	\rdg[wit={V1}]{ākarṣaṇaḥ kṛṣṭaṃ}
	\rdg[wit={J5,G11,P15,V15,C6}]{ākarṣaṇākṛṣṭaṃ}% +G5,F
	\rdg[wit={N23,P11}]{ākarṣaṇāt kṛṣṭaṃ} % originally ākarṣavat?
	\rdg[wit={V3}]{ākarṣaṇāt kaṣṭaṃ}
	\rdg[wit={J10,Jyo}]{ākarṣaṇaṃ kuryād}
	}}
\pada{dhanurāsana%
\app{\lem[wit={J5,N23,V1,J10,GrB,Jyo},alt={ucyate}]{\skm{m }ucyate}
	\rdg[wit={G4,V19,E2,G11,P15,V15}]{īritam}}//\versenr}
%	\unavbl{N3}
	\\!}
\end{tlg}

\commcite\newpage

\newpage%\normalmarginpar
%1.26
\begin{tlg}[hp01_026]
\tl{\app{\lem[nolem]{}
	\rdg[wit={N3}]{\unavbl}}%
\pada{vāmorumūlārpita\app{\lem[wit={ceteri}]{dakṣapādaṃ}% +G7
	\rdg[wit={J5}]{dakṣapādau}
	\rdg[wit={G4}]{dakṣapādo}
	\rdg[wit={P11,C6}]{dakṣapāda}
	}}\\+}
\tl{
\pada{\app{\lem[wit={N23,V1,J10,Jyo},alt={jānor}]{jāno\skp{r}} % +C1
	\rdg[wit={J5,G4,V19,E2,G11,P15,V15,P11,C6}]{jānvor}
	\rdg[wit={V3}]{jānaur}
	}%
\app{\lem[wit={ceteri},alt={bahirveṣṭita}]{\skm{r }bahirveṣṭita}% bahiviṣṭita P11
	\rdg[wit={N23}]{bahi<<ḥsaṃ>>ṣṭhita}}%
\app{\lem[wit={J5,V19,E2,G11,C6}]{dakṣadoṣṇā}% +M3,G11
	\rdg[wit={P11}]{dakṣadorṣṇā}
	\rdg[wit={N23}]{dakṣadorbhyāṃ}
	\rdg[wit={V1,J10,V3,Jyo}]{vāmapādam}% dākṣapādaṃ V1ac (pc pr.m.)
	\rdg[wit={G4,V15}]{vāmadoṣṇā}% °doṣā? G4; +G7,F,G3
	\rdg[wit={P15}]{vāmadoṣṇi}}/}\\+}
\tl{
\pada{pragṛhya tiṣṭhe%t
	\app{\lem[wit={ceteri},alt={parivartitāṅgaḥ}]{\skm{t }parivartitāṅgaḥ}% °āṃmaḥ P15, °āṃgo G4, °āṃga P11
	\rdg[wit={J5}]{pariveṣṭitāṃga}
	\rdg[wit={V19,E2}]{parimarditāṅgaḥ}
	\rdg[wit={G11}]{paribharjitāṃgaḥ}}}\\+}
\tl{
\pada{\app{\lem[wit={ceteri}]{śrīmatsyanātho}% matsa P11
	\rdg[wit={V3}]{śrīmatsyadinātho}
	\rdg[wit={G4}]{matsyeṃdranātho}}ditam āsanaṃ syāt//\versenr} % āśanaṃ V3
%	\unavbl{N3}
	\\!}
\end{tlg}

\commcite\newpage


%1.27
\begin{tlg}[hp01_027]
\tl{\app{\lem[nolem]{}
	\rdg[wit={N3}]{\unavbl}}%
\pada{matsyendra\app{\lem[wit={ceteri}]{pīṭhaṃ}
	\rdg[wit={C6}]{pīṭho}
	\rdg[wit={V1}]{vīraṃ}
	\rdg[wit={P15}]{vīra}
	}
\app{\lem[wit={ceteri}]{jaṭhara}
	\rdg[wit={V3}]{jvalana}
	\rdg[wit={P15}]{vīra}
	}%
\app{\lem[wit={J5,G4,G11,P11}]{pracaṇḍa}
	\rdg[wit={P15,V15,V1}]{pracaṇḍaṃ}
	\rdg[wit={N23}]{pravuddhaṃ} % prabuddhaṃ L2
	\rdg[wit={V19,E2}]{pravuddhau}
	\rdg[wit={J10}]{pravṛddha}
	\rdg[wit={C6}]{prabodhaṃ}
	\rdg[wit={V3}]{pradiptaṃ}
	\rdg[wit={Jyo}]{pradīptiṃ}
	}}\marma-\\+}
\tl{
\pada{\app{\lem[wit={J5,G4,P15}]{picaṇḍa}
	\rdg[wit={G11,P11}]{vicaṇḍa}
	\rdg[wit={V1}]{viccaṇḍa}
	\rdg[wit={N23,V19,E2,J10,C6,V3,Jyo}]{pracaṇḍa}
	\rdg[wit={V15},postwit=\texteng{(khaṃḍaḷa \textapp{is inserted between} ruṅmaṇḍala \textapp{and} khaṇḍanāstram \textapp{instead})},alt={\om}]{\skp{\om}}
	}%
\app{\lem[wit={ceteri},alt={ruṅ-/rugmaṇḍala}]{ruṅmaṇḍala}% maṃḍaḷā V15
	\rdg[wit={V1}]{rūrmaṇḍala}
	\rdg[wit={N23}]{rugmaṃḍana}
	\rdg[wit={P15}]{ruk(!)māṃḍana}}%
\app{\lem[wit={ceteri}]{khaṇḍanāstram}% khaṃḍaḷakhaṃḍanāstraṃ V15
	\rdg[wit={N23}]{khaṇḍanāmaṃ}
	\rdg[wit={V19}]{khaṇḍalāsyam}
	\rdg[wit={E2}]{khaṇḍitāstram}}/}\\+}
\tl{
\pada{abhyāsataḥ kuṇḍalinīprabodhaṃ}\\+} % ābhy° N23; °bodha J5,N23,G11
\tl{
\pada{\app{\lem[wit={ceteri}]{daṇḍa}% +J10pc, daṃḍaṃ P11
	\rdg[wit={J10,Jyo}]{candra}% +C1
	\rdg[wit={V15}]{kāya}}\marma%
\app{\lem[wit={ceteri}]{sthiratvaṃ}
	\rdg[wit={V19}]{sthitatvaṃ}}
\app{\lem[wit={ceteri}]{ca dadāti}
	\rdg[wit={J5,N23ac}]{dadāti}
	\rdg[wit={N23pc,V19}]{pradadāti}
	\rdg[wit={E2}]{vidadhāti}
	\rdg[wit={G4}]{ca karoti}} puṃsām//\versenr}
%	\unavbl{N3}
	\\!}
\end{tlg}

\commcite\newpage


%1.28
\begin{tlg}[hp01_028]
\tl{\app{\lem[nolem]{}
	\rdg[wit={N3}]{\unavbl}}%
\pada{prasārya pādau bhuvi daṇḍarūpau}\\+} % tuvi V19
\tl{
\pada{\app{\lem[wit={ceteri}]{dorbhyāṃ}
	\rdg[wit={J10,V3}]{dvābhyāṃ}}
\app{\lem[wit={cetwG4}]{padāgra}% +G4; °graṃ V15, <<pa>>dāgra N23
	\rdg[wit={J5}]{padāgryau}
	\rdg[wit={J10,V3}]{karābhyāṃ}
	\rdg[wit={V19,E2}]{ca pāda}}%
	dvitayaṃ\marmas % dvitiyaṃ V3, dvitīyaṃ C1,E4
	gṛhītvā/}\\+} % gṛhitvā V3, gra° G11
\tl{
\pada{jānūparinyastalalāṭa% jā<<nū>> V19; nyasya C6
\app{\lem[wit={ceteri}]{deśo}% deso J5,N23
	\rdg[wit={C6ac,V3}]{deśe}% dese V3; +F
	\rdg[wit={G4,C6pc}]{deśaḥ}}}\\+} 
\tl{
\pada{\app{\lem[wit={J5,N23,G11,P15,V1,GrB,Jyo},alt={vased}]{vase\skp{d}}% <<vased i>>daṃ V3, vased <<i>>daṃ C6
	\rdg[wit={V19,E2}]{'bhyased}
	\rdg[wit={V15}]{bhaved}
	\rdg[wit={G4}]{paśyed}
	\rdg[wit={J10},alt={d \textapp{(two syllables omitted)}}]{d}}d idaṃ
	paścima\app{\lem[wit={J5,G4,G11}]{tāṇam āhuḥ}
	\rdg[wit={N23,V19,E2,J10,GrB,Jyo}]{tānam āhuḥ}
	\rdg[wit={P15,V1}]{tāṇabandhaḥ}
	\rdg[wit={V15}]{tānabandhaḥ}}//\versenr}
%	\unavbl{N3}
	\\!}
\end{tlg}

\commcite\newpage

%\normalmarginpar
\newpage
%1.29
\begin{tlg}[hp01_029]
\tl{
\pada{iti 
	paścima\app{\lem[wit={J5,G4,G11,P15,V1},alt={tāṇam}]{tāṇa\skp{m}}% tāṇaṇam G11
	\rdg[wit={N23,V19,E2,V15,J10,P11,C6,V3,Jyo}]{tānam}% tā<<na>>m V15
	\rdg[wit={N3}]{tāyām}}%
\app{\lem[wit={N3,E2,G11,V15,V1,C6,V3,Jyo},alt={āsanāgryaṃ}]{\skm{m }āsanāgryaṃ}% +C1
	\rdg[wit={P11}]{āsanāgraṃ}
	\rdg[wit={J10}]{āsanāśāgryaṃ}
	\rdg[wit={V19,P15}]{āsanākhyaṃ}
	\rdg[wit={J5,G4}]{āsanaṃ}
	\rdg[wit={N23}]{āyanaṃ}}}\\+}
\tl{
\pada{pavanaṃ
\app{\lem[wit={ceteri}]{paścima}% paścim<av>āhinaṃ G11
	\rdg[wit={V19}]{paścimā}}%
\app{\lem[wit={cetwG4}]{vāhinaṃ}
	\rdg[wit={N3,P15,V3}]{vāhanaṃ}
	\rdg[wit={J5}]{vahena}} % +C1
	karoti/}\\+}
\tl{
\pada{\app{\lem[wit={ceteri}]{udayaṃ}
	\rdg[wit={G4,P11}]{udaraṃ}} 
\app{\lem[wit={ceteri}]{jaṭharānalasya}% °laśca J5
	\rdg[wit={V19}]{jaṭharānilasya}} kuryā-}\\+}
\tl{
\pada{d udare
\app{\lem[wit={ceteri},alt={kārśyam}]{kārśya\skp{m}}% kāśyam J5, kārsyam V19
	\rdg[wit={N23}]{kāryam}
	\rdg[wit={V3}]{kṛśyam}}%
\app{\lem[wit={Gr1,P15,V15,V1pc,P11,C6,Jyo},alt={arogatāṃ}]{\skm{m }arogatāṃ}% °tā P11,C6; āro° J5
	\rdg[wit={N23}]{alogatāṃ}
	\rdg[wit={V19,V1ac,J10}]{arogitāṃ}
	\rdg[wit={E2}]{arogiṇaṃ}
	\rdg[wit={V3}]{arogyatāṃ}
	}
ca puṃsām//\versenr}\\!}% ca om. J5
\end{tlg}

\commcite\newpage


%1.30

\begin{tlg}[hp01_030]
\tl{
\pada{dharām avaṣṭabhya
\app{\lem[wit={G4,G11,P15}]{karasthalābhyāṃ}
	\rdg[wit={N3,J5,V15}]{karadvayābhyāṃ}% +F
	\rdg[wit={N23,V1,J10,GrB,Jyo}]{karadvayena} % dvayaina N23
	\rdg[wit={V19}]{punaḥ karābhyāṃ}
	\rdg[wit={E2}]{puraḥ karābhyāṃ}% +C7
	}}\\+}
\tl{
\pada{ta\app{\lem[wit={P15,V15,V1,J10,P11,C6,Jyo},alt={kūrpara}]{\skm{t }kūrpara}
	\rdg[wit={N3,N23,V3}]{kurpara}% .[ū]rpara G4
	\rdg[wit={J5}]{karpara}
	\rdg[wit={G11}]{korpara}
	\rdg[wit={V19,E2}]{kurpare}}%
	sthāpitanābhi%
\app{\lem[wit={ceteri}]{pārśvaḥ}
	\rdg[wit={J5,V19,E2,V3}]{pārśve}}/}\\+}
\tl{
\pada{\app{\lem[wit={ceteri}]{uccāsano}% udyāsano P11
	\rdg[wit={N3}]{uccāsanā}
	\rdg[wit={V15}]{uccāsane} %+C1,C7
	\rdg[wit={V19,E2}]{taccāsanaṃ}}
\app{\lem[wit={ceteri},alt={daṇḍavad}]{daṇḍava\skp{d}}
	\rdg[wit={G4}]{daṃḍa i°}}%
\app{\lem[wit={N23,V19,E2,G11,V15,V1,J10,P11,Jyo},alt={utthitaḥ khe}]{\skm{d }utthitaḥ khe} % usthita<<ḥ>> N23, uḥtchitaḥ khe G11
	\rdg[wit={P15}]{utthitaḥ khaṃ}
	\rdg[wit={J5}]{ucchitaḥ ṣe}
	\rdg[wit={N3}]{utthitaś cet}
	\rdg[wit={C6}]{ucchritaś ca}
	\rdg[wit={V3}]{uthitasya}
	\rdg[wit={G4}]{°votthitāṃgo}}}\\+}
\tl{
\pada{\app{\lem[wit={N3,G4,G11,V1,P11,V3},alt={mayūram}]{mayūra\skp{m}}% +F
	\rdg[wit={J5,N23,V19,E2,P15,V15,J10,C6,Jyo}]{māyūram}
	}m etat pravadanti
\app{\lem[wit={ceteri}]{pīṭham}
	\rdg[wit={J5}]{pāṭhaṃ}
	\rdg[wit={P11}]{pāṭhāṃ}
	\rdg[wit={V19,E2}]{santaḥ}}//\versenr}
	\\!}
\end{tlg}

\commcite\newpage
%\end{ekdosis}\end{otherlanguage}
%\end{document}

%1.31
\begin{tlg}[hp01_031]
\tl{
\pada{harati sakala%
\app{\lem[wit={ceteri}]{rogān āśu}
	\rdg[wit={V3}]{rogān asu}
	\rdg[wit={P15}]{rogān śvāsa}
	\rdg[wit={J5}]{rogāśca}
	\rdg[wit={J10}]{doṣān āśu}
	}
\app{\lem[wit={ceteri}]{gulmo}% gunmo° G11
	\rdg[wit={N23}]{gulpho}
	\rdg[wit={N3}]{gulphau}}darādī-}\\+}% ādin V3
\tl{
\pada{\app{\lem[wit={G4,V19,E2,G11,P15,V1,J10,Jyo},alt={abhibhavati ca}]{\skm{n }abhibhavati ca}
	\rdg[wit={N3,V15}]{abhibhavati}
	\rdg[wit={N23}]{abhavati ca}
	\rdg[wit={V3}]{abhavati}
	\rdg[wit={P11}]{na bhavati bhava}
	\rdg[wit={C6}]{na hi bhavati ca}% =F
	\rdg[wit={J5}]{(n)ibhibhavati vadi ca}}
	doṣān āsanaṃ śrīmayūram/}\\+} % doṣām P11, doṣam C6
\tl{
\pada{\app{\lem[wit={ceteri}]{bahukadaśanabhuktaṃ}% bahū P11
	\rdg[wit={G4}]{bahuḷam api ca bhuktaṃ}}
\app{\lem[wit={ceteri}]{bhasma}
	\rdg[wit={V19}]{tac ca}}
	kuryā\app{\lem[wit={ceteri}, alt={aśeṣaṃ}]{\skm{d }aśeṣaṃ}
	\rdg[wit={V1}]{aśeṣo}
	\rdg[wit={G11}]{vicitraṃ}% +F
	\rdg[wit={P11}]{avitraṃ}
	\rdg[wit={P15}]{iśutraṃ}
	\rdg[wit={C6},alt={\om}]{\skp{\om}}}}\\+}
\tl{
\pada{janayati
\app{\lem[wit={ceteri}]{jaṭharāgniṃ}
	\rdg[wit={P15}]{jaṭharāgraṃ}
	\rdg[wit={V3}]{vaḍavājñiṃ}}
\app{\lem[wit={ceteri},alt={jārayet}]{jāraye\skp{t}}
	\rdg[wit={N3}]{jīrayet}
	\rdg[wit={G4}]{jiryate}
	\rdg[wit={J10}]{jvālayet}
	}t kālakūṭam//\versenr} % kāra N23ac
	\\!}
\end{tlg}

\commcite\newpage

%1.32
\begin{tlg}[hp01_032]
\tl{
\pada{\app{\lem[wit={V19,E2,G11,P15,V15,V1,V3,Jyo}]{uttānaṃ}
	\rdg[wit={Gr1,N23,J10,P11,C6}]{uttāna}}
	śavavad bhūmau} % savad bhaumo J5
\pada{\app{\lem[wit={Gr1,N23,G11,P15,V1,GrB}]{śayanaṃ tu śavāsanam} % śayanāṃ G4; savā° N23; °samaṃ J5
	\rdg[wit={E2,V15,J10}]{śayanaṃ ca śavāsanam}
	\rdg[wit={Jyo}]{śayanaṃ tac chavāsanam}
	\rdg[wit={V19}]{śavāsanam idaṃ smṛtam}% +C7
	}/}\\+}
\tl{
\pada{\app{\lem[wit={J5,G4,N23,G11,P15,V1,J10,P11,C6}]{sarvāsana}% °āsa<<na>> J10
	\rdg[wit={V3}]{savāsana}
	\rdg[wit={N3,V19,E2,V15,Jyo}]{śavāsanaṃ}% +F
	}%
\app{\lem[wit={ceteri}]{śrānti}% srānti N23
	\rdg[wit={G11,V15}]{śrama}
	\rdg[wit={J5}]{gati}}haraṃ} % hāraṃ G4
\pada{citta% cita V3
\app{\lem[wit={ceteri}]{viśrānti}% srānti N23
	\rdg[wit={E2}]{vikrānti}}% 
\app{\lem[wit={N3,J5,N23,V19,G11,P15,GrB}]{sādhanam}
	\rdg[wit={G4,E2,V15,V1,J10,Jyo}]{kārakam}
	}//\versenr}\\!}
\end{tlg}

\commcite\newpage

%1.33
\begin{tlg}[hp01_033]
\tl{
\pada{\app{\lem[wit={ceteri}]{caturāśītyāsanāni}% sīty C6; tyatyā N3
	\rdg[wit={N23,G11,Jyo}]{caturaśītyāsanāni}
	\rdg[wit={P11}]{caturāśityāsanebhya}}}
\pada{\app{\lem[wit={ceteri}]{śivena kathitāni}% kathitāna V3
	\rdg[wit={P15}]{sarvāṇi kathitāni}
	\rdg[wit={C6}]{kathitāni śivena}
	\rdg[wit={P11},alt={\om}]{\skp{\om}}}
\app{\lem[wit={N3,G4,G11,P15,V15,V1,J10,C6,V3}]{tu}
	\rdg[wit={N23,V19,E2}]{vai}
	\rdg[wit={J5,Jyo}]{ca}
	\rdg[wit={P11},alt={\om}]{\skp{\om}}}/}\\+}
\tl{
\pada{\app{\lem[wit={ceteri}]{tebhyaś catuṣkam ādāya}
	\rdg[wit={P15}]{tebhyaś catu\,\_\,m ādāya}
	\rdg[wit={P11}]{caturāsanaṃ}}}
\pada{sārabhūtaṃ \app{\lem[wit={ceteri}]{bravīmy aham}
	\rdg[wit={P11},alt={\om}]{\skp{\om}}}//\versenr}
	\\!}
\end{tlg}

\commcite\newpage

%1.34
\begin{tlg}[hp01_034]
\tl{
\pada{siddhaṃ
\app{\lem[wit={ceteri}]{padmaṃ tathā}
	\rdg[wit={V3}]{padmaṃ yathā}
	\rdg[wit={G11}]{patmāsanaṃ}
	\rdg[wit={V19}]{bhadraṃ tathā}
	}
\app{\lem[wit={J5,G4,N23,G11,V1,J10,C6,V3,Jyo}]{siṃhaṃ}
	\rdg[wit={N3}]{sīhaṃ}
	\rdg[wit={P15,V15}]{saiṃhaṃ}
	\rdg[wit={P11}]{svasti}
	\rdg[wit={V19}]{padmaṃ}
	\rdg[wit={E2}]{bhadraṃ}}} %+C1,C7
\pada{\app{\lem[wit={ceteri}]{bhadraṃ}
	\rdg[wit={V19,E2}]{siṃhaṃ}}
\app{\lem[wit={N23,V19,E2,J10,P11,C6,Jyo}]{ceti}
	\rdg[wit={G4}]{ce\,..}
	\rdg[wit={N3,J5,V1}]{caiva}
	\rdg[wit={G11,P15,V3}]{caitac}
	\rdg[wit={V15}]{cātha}}
\app{\lem[wit={ceteri}]{catuṣṭayam}
	\rdg[wit={V15}]{catuṣkakaṃ}}/}\\+}
\tl{
\pada{śreṣṭhaṃ
\app{\lem[wit={N3,J5,N23,G11,GrB}]{tatrāpi ca sakhe}
	\rdg[wit={V1,J10,Jyo}]{tatrāpi ca sukhe}
	\rdg[wit={V15}]{tatrāpi ca sukhaṃ}
	\rdg[wit={G4}]{tatrāpi sarveṣāṃ}
	\rdg[wit={V19,E2}]{tathāpi bhadraṃ ca}
	\rdg[wit={P15}]{tatra viśeṣeṇa}}}
\pada{\app{\lem[wit={N3,J5,N23,G11,GrB}]{tiṣṭha}
	\rdg[wit={V15}]{tiṣṭhat}
	\rdg[wit={G4,V19,E2,V1,J10,Jyo}]{tiṣṭhet}
	\rdg[wit={P15}]{śreṣṭhaṃ}}
\app{\lem[wit={ceteri}]{siddhāsane}
	\rdg[wit={G4}]{siddhāsanaṃ}
	\rdg[wit={V19}]{siṃhāsane}
	\rdg[wit={P15}]{padmāsanaṃ}} 
\app{\lem[wit={ceteri}]{sadā}
	\rdg[wit={G4}]{tadā}}//\versenr} 
	\\!}
\end{tlg}

\commcite\newpage
% śreṣṭhaṃ tathāpi ca sakhe tiṣṭhet siddhā-
% tatheti vā/ 
% śreṣṭhaṃ tathāpi bhadraṃ ca tiṣṭhet siddhā-
% -sane sadā

\begin{ava}[hp01_035a]
\app{\lem[wit={N3,G4,J10,C6,Jyo}]{tatra siddhāsanam}
	\rdg[wit={J5,N23}]{atha siddhāsanam}
	\rdg[wit={V19,E2,G11,P15,V15,V1,P11},alt={\om}]{\skp{\om}}}/
%	\sgwit{Gr1,N23,J10,C6} 
%	\NotIn{V19,E2,G11,P15,V15,V1,P11}
\end{ava}


%1.35
\begin{tlg}[hp01_035]
\tl{
\pada{yoni\app{\lem[wit={G4,G11,P15,V15,V1,J10,V3,Jyo}]{sthānaka}
	\rdg[wit={N3,J5,N23,V19,E2,P11,C6}]{dvāraka}}% ##
\app{\lem[wit={ceteri},alt={°m aṅghrimūla}]{\skp{°}m aṅghrimūla}% aṃhri P11,P15,V1
	\rdg[wit={V19}]{m aṅghrimūlā}
	\rdg[wit={J10}]{mūlamaṅghri}}%
\app{\lem[wit={ceteri}]{ghaṭitaṃ}
	\rdg[wit={P15}]{puṭakaṃ}} kṛtvā
\app{\lem[wit={ceteri}]{dṛḍhaṃ}
	\rdg[wit={P15,V15}]{dhruvaṃ}} vinyase}- \\+}
\tl{
\pada{\app{\lem[wit={ceteri},alt={me(ṇ)ḍhre}]{\skm{n }meḍhre}% meṃḍhre P15,V15,Jyo
	\rdg[wit={J5,P11}]{meḍhraṃ}
	\rdg[wit={N23}]{medhre}
	\rdg[wit={V19}]{madhye}}
	pādam athaika%
\app{\lem[resp=emend,alt={ekahṛdayo},post=\texteng{(cf.\,VM)}]{\skm{m }ekahṛdayo}
	\rdg[wit={N3,N23,G11,P15}]{ekahṛdaye}
	\rdg[wit={G4}]{eka\,+\,+\,+}
	\rdg[wit={J5,V19,V15,J10,P11,Jyo}]{eva hṛdaye}% +K3; aṃva P11
	\rdg[wit={E2,C6,V3}]{eva niyataṃ}% +K3pc
	\rdg[wit={V1}]{āsyahṛdaye}}\marmas
\app{\lem[wit={ceteri}]{dhṛtvā}
	\rdg[wit={E2,C6,V3,Jyo}]{kṛtvā}} % +K3pc
\app{\lem[wit={ceteri}]{samaṃ}
	\rdg[wit={Jyo}]{hanuṃ}}
\app{\lem[wit={ceteri}]{vigraham}
	\rdg[wit={Jyo}]{susthiraṃ}}/}\\+}
\tl{
\pada{sthāṇuḥ % sthāṇu P11, sthānu V19
	saṃyamitendriyo'caladṛśā % V19,J10 write avagraha! °yatite° P11; daśā C6
\app{\lem[wit={N3,G11,V1,J10,GrB}]{paśyan}
	\rdg[wit={N23}]{paśyad}
	\rdg[wit={G4,V19,E2,P15,V15,Jyo}]{paśyed}
	\rdg[wit={J5}]{pārśve}}
	bhruvor antaraṃ} \\+} % bhṛvor N3
\tl{
\pada{\app{\lem[wit={J5,N23,G11,J10,GrB},alt={caitan}]{caita\skp{n}}
	\rdg[wit={N3}]{caitanya}
	\rdg[wit={G4}]{ceto}
	\rdg[wit={V19,E2,P15,V15,V1}]{etan}% °raṃmetan V19
	\rdg[wit={Jyo}]{hy etan}}n%
	mokṣakapāṭabheda% kavāṭa G4,E2,G11
\app{\lem[wit={ceteri}]{janakaṃ}
	\rdg[wit={J5,P15}]{jananaṃ}} 
	siddhāsanaṃ procyate//\versenr}\\!} % °ā<<sa>>naṃ J10
\end{tlg}

\avacite{35a}
\commcite\newpage\ \newpage

\newpage
\begin{ava}[hp01_036a]
\app{\lem[wit={N3,N23,G11,J10,C6,Jyo}]{matāntare tu}
	\rdg[wit={G4,P15,V15,V1,P11,V3}]{matāntare}
	\rdg[wit={J5}]{matāṃtaraṃ}
	\rdg[wit={N19}]{matsaṃtare}
	\rdg[wit={V19}]{matsyendraḥ\,| matāntaraṃ tu}
	\rdg[wit={E2}]{etan matsyendramataṃ matāntare tu}% +K3,C7
	}/
\end{ava}


%1.36
\begin{tlg}[hp01_036]
\tl{
\pada{meḍhrād upari 
% maiṃḍhrād P11, meṃḍhrād P15,V1, meḍhrādhaḥpari N23, meḍhrādraupari J5
\app{\lem[wit={N3,G4,P11,V3}]{nikṣipya}% +F
	\rdg[wit={J10}]{niḥkṣipya}
	\rdg[wit={J5,N23,V19,E2,G11,P15,V15,V1,C6,Jyo}]{vinyasya}% +J5; °syaṃ V19
	}}
\pada{\app{\lem[wit={Gr1,N23,G11,P15,J10,GrB}]{savya}
	\rdg[wit={V15,Jyo}]{savyaṃ}
	\rdg[wit={V1}]{savyaṃ tu}
	\rdg[wit={V19,E2}]{vāma}}gulphaṃ
	tathopari/}\\+}  % tato° N3; °darī J5, °pariḥ J10
\tl{
\pada{gulphāntaraṃ % °tare J5, gulphāṃtu nikṣipya sadā G4, °<<ta>>raṃ V19
\app{\lem[wit={ceteri}]{ca}
	\rdg[wit={V1}]{tu}}
\app{\lem[wit={ceteri}]{nikṣipya}
	\rdg[wit={J10}]{niḥkṣipya}
	\rdg[wit={J5}]{vikṣipya}
	\rdg[wit={V19}]{vinyasya}
	}}
\pada{siddhāsana\app{\lem[wit={ceteri},alt={idaṃ}]{\skm{m }idaṃ}
	\rdg[wit={E2}]{iti}} bhavet//\versenr}\\!}% procyate J5, itīritaṃ F
\end{tlg}

\avacite{36a}
\commcite%\newpage

\begin{postmula}[hp01_036p]
\app{\lem[nolem]{}
	\rdg[wit={V19,E2,V1,J10,V3},alt={\om}]{\skp{\om}}}%
\app{\lem[wit={N3,J5,N23,G11,P15,N19,V15,P11,C6,Jyo}]{pūrvoktam eva}
	\rdg[wit={G4}]{pūrvam evoktam etan}}
\app{\lem[wit={N19}]{matsaṃmatam} % cf. P6
	\rdg[wit={N3}]{matsamaṃtaṃ}
	\rdg[wit={G11}]{tatsaṃmataṃ}% mama saṃmataṃ G5
	\rdg[wit={P11}]{saṃmataṃ}
	\rdg[wit={G4,P15,V15}]{matsyamataṃ}
	\rdg[wit={J5,N23,C6,Jyo}]{matsyendramatam}}/
%	\sgwit{N3,G4,N23,G11,P15,N19,V15,P11,C6}
%	\NotIn{V19,E2,V1,J10,V3}% +F
\end{postmula}

\avacite{36p}\newpage

%1.37
\begin{tlg}[hp01_037]
\tl{
\pada{\app{\lem[wit={ceteri},alt={etat}]{eta\skp{t}}
	\rdg[wit={J5}]{esmin}
	\rdg[wit={V19}]{kecit}
	\rdg[wit={E2}]{iti}% +K3,C7
	}t siddhāsanaṃ prāhu}%
\pada{\app{\lem[wit={ceteri},alt={anye}]{\skm{r }anye}% +G4; anya? N23, (r)aṇe J5
	\rdg[wit={N3}]{anyathā}} vajrāsanaṃ viduḥ/}\\+}
\tl{
\pada{\app{\lem[wit={cetwG4}]{muktāsanaṃ}
	\rdg[wit={V19}]{muktvāsanaṃ}
	\rdg[wit={N3,J5}]{guptāsanaṃ}}
\app{\lem[wit={ceteri}]{vadanty eke}% vadaṃteke N3
	\rdg[wit={V19,E2,G11}]{vadanty anye}% +F
	}}
\pada{prāhu\app{\lem[wit={cetwG4},alt={guptāsanaṃ}]{\skm{r }guptāsanaṃ}
	\rdg[wit={N3,J5}]{muktāsanaṃ}} pare//\versenr}\\!}% prāhu P11
% from pare on lost N3
\end{tlg}

\commcite\newpage

%1.38
\begin{tlg}[hp01_038]
\tl{
\pada{\app{\lem[wit={ceteri}]{yameṣv iva}
	\rdg[wit={V1,J10}]{yameṣv eva}
	\rdg[wit={C6}]{yameṣu ca}}
\app{\lem[wit={ceteri},alt={mitāhāram}]{mitāhāra\skp{m}}
	\rdg[wit={V1,V3}]{mitāhāra}
	\rdg[wit={J10}]{mitāhāraḥ}}}%
\pada{\app{\lem[wit={V19,E2,V15,C6,Jyo},alt={ahiṃsāṃ}]{\skm{m }ahiṃsāṃ}
	\rdg[wit={J5,G4,G11,P15,V1,J10,P11,V3}]{ahiṃsā} % ahisā J5
	\rdg[wit={N23}]{nahiṃsāṃ}
	}
\app{\lem[wit={ceteri}]{niyameṣv iva}
	\rdg[wit={V1,C6}]{niyameṣu ca}}/}\marma\\+}
\tl{
\pada{mukhyaṃ sarvāsane% mukṣyaṃ C6; sarvaś ca teṣv J5
\app{\lem[wit={ceteri},alt={ekaṃ}]{\skm{ṣv }ekaṃ}
	\rdg[wit={E2,V15}]{eke}
	\rdg[wit={V19}]{evaṃ}
	\rdg[wit={J5}]{eva}
	}}
\pada{\app{\lem[wit={ceteri}]{siddhāḥ siddhāsanaṃ viduḥ}% siddhā V1ac,V3
	\rdg[wit={G4}]{siddhaṃ si[ddh]. viduḥ}
	\rdg[wit={C6}]{etat siddhāsanaṃ viduḥ}
	\rdg[wit={V19}]{siddhāsanam idaṃ viduḥ}% +K3,C7
	\rdg[wit={E2},alt={\om}]{\skp{\om}}% eye-skip
	}
	//\versenr}%\myfn{1.38--46 lost in \getsiglum{N3}}
%	\unavbl{N3}
	\\!}
\end{tlg}

\commcite\newpage

%1.39
\begin{tlg}[hp01_039]
\tl{\app{\lem[nolem]{}
	\rdg[wit={N3}]{\unavbl}}%
\pada{\app{\lem[wit={ceteri}]{caturāśītipīṭheṣu} % sīti J10
	\rdg[wit={N23,Jyo},alt={catura°}]{caturaśītipīṭheṣu}
	\rdg[wit={E2},alt={\om}]{\skp{\om}}
	}}
\pada{\app{\lem[wit={ceteri}]{siddham eva}
	\rdg[wit={V19,E2}]{siddhāsanaṃ}
	}
	\app{\lem[wit={G11}]{sadā bhajet}% +Z,F
	\rdg[wit={J5}]{sadā bhavet}
	\rdg[wit={G4}]{sadā paṭhet}
	\rdg[wit={N23,V19,E2,P15,V15,V1,J10,P11,C6,V3,Jyo}]{sadābhyaset}
	}/}\\+}
\tl{
\pada{\app{\lem[wit={ceteri}]{dvāsaptati}
	\rdg[wit={G4,V15}]{dvisaptati}}% +C8
\app{\lem[wit={ceteri}]{sahasreṣu} % V3 sahaśreṣu
	\rdg[wit={V15}]{sahasrāsu}
	\rdg[wit={Jyo}]{sahasrāṇāṃ}}}
\pada{\app{\lem[wit={G4,V19,E2,V15,J10,P11,C6},alt={suṣumṇām}]{suṣumṇā\skp{m}}% suṣunmām J10
	\rdg[wit={N23}]{sukhumṇām}
	\rdg[wit={J5,P15,V1}]{suṣumṇā}
	\rdg[wit={G11}]{mnām}
	\rdg[wit={V3,Jyo}]{nāḍīnāṃ}}%
\app{\lem[wit={G4,N23,V19,E2,G11,P15,V15,P11,C6},alt={iva nāḍiṣu}]{\skm{m }iva nāḍiṣu}
	\rdg[wit={V1}]{iva nāḍikā}
	\rdg[wit={J10}]{eva nāḍiṣu}
	\rdg[wit={J5}]{ca nāḍiṣu}
	\rdg[wit={V3,Jyo}]{malaśodhanam}% sodhanaṃ V3
	}\marma//\versenr}
%	\unavbl{N3}
	\\!}
\end{tlg}

\commcite\newpage

\newpage
%1.40
\begin{tlg}[hp01_040]
\tl{\app{\lem[nolem]{}
	\rdg[wit={N3}]{\unavbl}}%
\pada{\app{\lem[wit={ceteri}]{ātmadhyāyī}
	\rdg[wit={V15}]{ātmādhyāyī}
	\rdg[wit={J5}]{ātmadhyāna}}
\app{\lem[wit={ceteri}]{mitāhārī}
	\rdg[wit={V19,E2,P15}]{mitāhāro}}}
\pada{yāvad dvādaśavatsaram/}\\+} % yāvadvā° G4,V3,V19,G11,V15,V1, yatavādadvā° C6; dasa V19
\tl{
\pada{sadā siddhāsanā\app{\lem[wit={ceteri},alt={°bhyāsād}]{bhyāsā\skp{d}}
	\rdg[wit={J5,P15}]{°bhyāsā}
	\rdg[wit={G4}]{°bhyāse}
	\rdg[wit={V19}]{°bhyānād}}}%
\pada{\app{\lem[wit={ceteri},alt={yogī}]{\skm{d }yogī}
	\rdg[wit={G4}]{yoge}
	\rdg[wit={V15}]{yoga}}
\app{\lem[wit={cetwG4}]{niṣpattim āpnuyāt}% niḥ° V1
	\rdg[wit={J10},post={\unm}]{niṣpattim avāpnuyāt}
	\rdg[wit={J5,V3}]{siddhim avāpnuyāt}}/}\\+}
\tl{
\pada{\app{\lem[wit={J5,G4,N23,G11,P11,C6}]{śramadair bahubhiḥ}% śramalair G4
	\rdg [wit={V1}]{śramādau bahubhiḥ}
	\rdg[wit={P15}]{samastair bahubhiḥ}
	\rdg[wit={V15}]{samastabahubhiḥ}
	\rdg[wit={V19,E2}]{śramadairghyādibhiḥ}
	\rdg[wit={J10,V3}]{kim ādyair bahubhiḥ}
	\rdg[wit={Jyo}]{kim anyair bahubhiḥ}}
	pīṭhaiḥ} % ḥ om. V3
\pada{\app{\lem[wit={J5,N23,V19,E2,V15,C6},alt={kiṃ syāt}]{kiṃ syā\skp{t}}
	\rdg[wit={G4}]{kiṃ vā}% kim u G5
	\rdg[wit={G11,V1,J10,V3}]{sadā}% +F
	\rdg[wit={P11}]{kiṃ sadā}
	\rdg[wit={P15}]{yadā}
	\rdg[wit={Jyo}]{siddhe}}%	
\app{\lem[wit={ceteri},alt={siddhāsane sati}]{\skm{t }siddhāsane sati}
	\rdg[wit={V3}]{siddhāsane satya}
	\rdg[wit={N23,C6}]{siddhāsane sthite}
	\rdg[wit={P11}]{siddhisādhanaiḥ}% °sādhane F
	}\marma//\versenr}
%	\unavbl{N3}
	\\!}
\end{tlg}

\commcite%\newpage


%1.41
\begin{tlg}[hp01_041]
\tl{\app{\lem[nolem]{}
	\rdg[wit={N3}]{\unavbl}}%
\pada{\app{\lem[wit={ceteri}]{prāṇānile}
	\rdg[wit={V3}]{prāṇānale}}
\app{\lem[wit={J5,G4,N23,E2,G11,P15,V1,J10,P11}]{sāvadhānaṃ}
	\rdg[wit={V19,C6,V3,Jyo},alt={°ne}]{sāvadhāne} % +C1
	\rdg[wit={V15},alt={°no}]{sāvadhāno}}}
\pada{\app{\lem[wit={V19,E2,G11,V15,V1,J10,C6,Jyo}]{baddhe}
	\rdg[wit={N23}]{baddhvai}
	\rdg[wit={P15}]{baṃdhaḥ}
	\rdg[wit={P11}]{badha}
	\rdg[wit={V3}]{baṃdhe}
	\rdg[wit={G4}]{siddh.}
	\rdg[wit={J5}]{yuṃye}}
\app{\lem[wit={V19,E2,G11,V15,V1,J10,Jyo}]{kevalakumbhake}% +G5
	\rdg[wit={J5,G4,P15,P11,V3}]{kevalakumbhakaḥ}% +F (baddha°)
	\rdg[wit={N23,C6}]{kevalakumbhataḥ}}/}\\+}
\tl{
\pada{\app{\lem[wit={ceteri}]{utpadyate}% ulpadyate G11
	\rdg[wit={V1,C6,V3}]{utpadyaṃte}}
	nirāyāsā}%t} % °yāsā E2,K3, nirāgrāsāt J5, nirāyāt C6
\pada{\app{\lem[wit={ceteri},alt={svayam evonmanī}]{\skm{t }svayam evonmanī}
	\rdg[wit={C6},alt={\om},post=\texteng{(eye-skip)}]{\skp{\om\ (eye-skip)}}}  
	% evātmanī J5, evvonmanīṃ N23, evonmanā P11
	\app{\lem[wit={ceteri}]{yathā}% +G5
	\rdg[wit={G11}]{pathaḥ}
	\rdg[wit={V1}]{tathā}
	\rdg[wit={Jyo}]{kalā}
	\rdg[wit={C6},alt={\om}]{\skp{\om}}}//\versenr} % by eye-skip
%	\lineom{d}{C6}
%	\unavbl{N3}
	\\!}
\end{tlg}

\commcite\newpage

%1.42
\begin{tlg}[hp01_042]
\tl{\app{\lem[nolem]{}
	\rdg[wit={N3}]{\unavbl}}%
\pada{\app{\lem[nolem]{\skp{pāda a}}
	\rdg[wit={P15,N19,C6},alt={\om}]{\skp{\om}}}%
\app{\lem[wit={ceteri},alt={tathaika°}]{tathaika\skp{°}}% J5,G4,G11,V1,J10,P11,V3,Jyo
	\rdg[wit={N23,V19,E2,V15}]{athaika}
	}sminn eva
\app{\lem[wit={J5,G4,V19,E2,G11,P11,V3}]{dṛḍhaṃ}
	\rdg[wit={V15,V1,J10,Jyo}]{dṛḍhe}
	\rdg[wit={N23}]{dṛdhe}
	}}
\pada{\app{\lem[nolem]{\skp{pāda b}}
	\rdg[wit={P15,N19,C6},alt={\om}]{\skp{\om}}}%
\app{\lem[wit={ceteri}]{baddhe}
	\rdg[wit={N23,P11}]{baddha}
	}
\app{\lem[wit={ceteri}]{siddhāsane}
	\rdg[wit={V19}]{siṃhāsane}
	\rdg[wit={P11}]{padmāsana}
	}
\app{\lem[wit={ceteri}]{sadā}
	\rdg[wit={N23}]{tadā}
	\rdg[wit={Jyo}]{sati}
	}/\\+}}
\tl{
\pada{\app{\lem[nosep]{\skp{pāda c}}
	\rdg[wit={C6},alt={\om}]{\skp{\om}}}%
	bandhatrayam anāyāsāt}
\pada{svayam evopajāyate//\versenr}
%	\lineom{ab}{P15,N19,C6} % \lineom{c}{C6} 
%	\unavbl{N3}
	\\!}
\end{tlg}

\commcite%\newpage

%1.43
\begin{tlg}[hp01_043]
\tl{\app{\lem[nolem]{}
	\rdg[wit={N3}]{\unavbl}%
	\rdg[wit={V3},alt={\om}]{\skp{\om}}}%
\pada{\app{\lem[wit={ceteri}]{nāsanaṃ siddhasadṛśaṃ}
	\rdg[wit={P15}]{nāsanaṃ siddhasadanaṃ} % siṃdhu° P15ac
	\rdg[wit={V1,J10}]{na cāsanaṃ siddhasamaṃ}}}
\pada{na
\app{\lem[wit={ceteri}]{kumbhaḥ kevalopamaḥ}
	\rdg[wit={P11}]{kuṃbhako balopeta}
	\rdg[wit={V1,J10}]{kumbhasadṛśo'nilaḥ}}/}\\+}
% J10 has vs no. 43 here, but counts atha padmāsanam as 45.
\tl{
\pada{na khecarīsamā mudrā}
\pada{na
\app{\lem[wit={ceteri}]{nāda}
	\rdg[wit={V1,P11}]{nādaḥ}
	\rdg[wit={G4}]{nādāt}}sadṛśo layaḥ//\versenr}%
%	\unavbl{N3}%
%	\NotIn{V3}%
	\myfn{In \getsiglum{V3} this verse is omitted here, but found at the beginning of the Khecarī section in chapter 3 (\textrightarrow\ 3.31*1).}%	
	\\!}
\end{tlg}

\commcite\newpage


\newpage
\begin{ava}[hp01_044a]
\app{\lem[wit={ceteri}]{atha padmāsanam}
	\rdg[wit={V1}]{tathā padmāsanam}
	\rdg[wit={V19,P11}]{padmāsanam}
	\rdg[wit={P15},alt={\om}]{\skp{\om}}}/
\end{ava}


%1.44
\begin{tlg}[hp01_044]
\tl{\app{\lem[nolem]{}
	\rdg[wit={N3}]{\unavbl}}%
\pada{vāmorūpari
\app{\lem[wit={ceteri}]{dakṣiṇaṃ}
	\rdg[wit={J5,V3}]{dakṣaṇaṃ}
	\rdg[wit={V19}]{vidakṣiṇaṃ}}
\app{\lem[wit={ceteri}]{ca}
	\rdg[wit={N23,V19,E2,C6}]{hi}
	\rdg[wit={G4},alt={\om}]{\skp{\om}}}
	caraṇaṃ saṃsthāpya vāmaṃ % saṃsthāya C6
\app{\lem[wit={ceteri}]{tathā}
	\rdg[wit={V3}]{tato}}}\\+}
\tl{
\pada{\app{\lem[wit={J5,G4,G11,P15,V1,P11,V3}]{yāmyo}
	\rdg[wit={N23,V19,E2,V15,C6,Jyo}]{dakṣo}
	\rdg[wit={J10}]{jānvo}}rūpari
\app{\lem[wit={ceteri}]{paścimena vidhinā}
	\rdg[wit={V3}]{tasya bandhanavidhau}}
\app{\lem[wit={ceteri}]{dhṛtvā}
	\rdg[wit={J5}]{vṛttā}
	\rdg[wit={V3}]{pṛṣṭe}} karābhyāṃ dṛḍham/}\\+}
\tl{
\pada{\app{\lem[wit={ceteri}]{aṅguṣṭhau}
	\rdg[wit={N23}]{aṅguṣṭho}
	\rdg[wit={G11}]{aṅguṣṭhe}} 
\app{\lem[wit={ceteri}]{hṛdaye}
	\rdg[wit={G4,G11}]{hṛdayaṃ}} nidhāya
	cibukaṃ nāsāgra%m  % cabukaṃ V15, cubukaṃ G4, cibukaṃ {hṛdayan} G11
\app{\lem[wit={ceteri},alt={ālokayed}]{\skm{m }ālokaye\skp{d}}
	\rdg[wit={G11}]{ālokayan}}-}\\+}
\tl{
\pada{d etad % eta<<d>> N23
	vyādhi\app{\lem[resp=emend]{vighātakāri}
	\rdg[wit={J5}]{vighātakāra}% cf. add. verse after 2.35 in G4
	\rdg[wit={P11}]{vivātakāri}
	\rdg[wit={G4,G11,P15,V15,V1,J10,Jyo}]{vināśakāri}% +G5,G7,G11,M3,F
	\rdg[wit={C6}]{vināśakāya}
	\rdg[wit={V3}]{vināsanaṃ}
	\rdg[wit={N23}]{vināśam āśu}
	\rdg[wit={V19,E2}]{vikāranāśa°}}
\app{\lem[wit={ceteri}]{yamināṃ}% ṃ om. P15
	\rdg[wit={J5}]{ṇāmimaṃ}
	\rdg[wit={N23}]{janakaṃ}
	\rdg[wit={V19,E2}]{°nakaraṃ}}
	padmāsanaṃ procyate//\versenr}
%	\unavbl{N3}
	\\!}
\end{tlg}

\avacite{44a}
\commcite\newpage

\begin{ava}[hp01_045a]
\app{\lem[wit={J5,G4,N19,V1,P11,V3,Jyo}]{matāntare}
	\rdg[wit={N23}]{matāntaraṃ}
	\rdg[wit={V15}]{matāntara}
	\rdg[wit={E2,G11,P15,J10,C6}]{matāntare tu}% +K3
	\rdg[wit={V19}]{matabhede}}/
\end{ava}


%1.45
\begin{tlg}[hp01_045]
\tl{\app{\lem[nolem]{}
	\rdg[wit={N3}]{\unavbl}}%
\pada{uttānau caraṇau kṛtvā}
\pada{ūrusaṃsthau % ūrū G4,P15,V15,J10?, uru P11,V1, ura J5, kuru V3; saṃstho N23, saṃsthāpya yatna° G4
\app{\lem[wit={ceteri}]{prayatnataḥ}
	\rdg[wit={V19}]{vidhānataḥ}}/}\\+}
\tl{
\pada{ūrumadhye  % ūrū K3,P15,V15,J10, urū P11, uru V3, urau J5; madhyai N23
\app{\lem[wit={ceteri}]{tathottānau}% +J10pc; ttatho° P11, tathātānau C6
	\rdg[wit={J5,V19,J10}]{tathauttānau}}}
\pada{\app{\lem[wit={ceteri}]{pāṇī}% +J10pc; pāṇau G4, ghrāṇī J5
	\rdg[wit={J10,C6}]{pāṇiṃ}} kṛtvā
\app{\lem[wit={ceteri}]{tato dṛśau}
	\rdg[wit={V3}]{tato dṛśai}
	\rdg[wit={J5}]{tato dṛśe}
	\rdg[wit={V19,E2}]{tu tādṛśau}}//\versenr}
%	\unavbl{N3}
	\\!}
\end{tlg}

%\commcite\newpage

%1.46
\begin{tlg}[hp01_046]
\tl{\app{\lem[nolem]{}
	\rdg[wit={N3}]{\unavbl}}%
\pada{\app{\lem[wit={ceteri}]{nāsāgre}% nāśāgrye J5
	\rdg[wit={V19}]{nāsagre}}
\app{\lem[wit={ceteri},alt={vinyased}]{vinyase\skp{d}}
	\rdg[wit={V3}]{vinyasya}}%
\app{\lem[wit={J5,G4,N23,V19,E2,J10,P11,Jyo},alt={rāja}]{\skm{d }rāja}
	\rdg[wit={G11}]{rājan}
	\rdg[wit={P15}]{rājā}
	\rdg[wit={V15,V1}]{dṛṣṭiṃ}
	\rdg[wit={V3}]{dṛṣṭī}
	\rdg[wit={C6},alt={\lacuna}]{\skp{\lacuna}}}}%
\pada{danta\app{\lem[wit={ceteri}]{mūlaṃ}% śūlaṃ J5
	\rdg[wit={V19,E2,C6,Jyo}]{mūle}}
\app{\lem[wit={J5,G4,V19,E2,G11,V15,V1,J10,V3}]{ca}
	\rdg[wit={N23,P15,P11,C6,Jyo}]{tu}} jihvayā/}\\+}
\tl{
\pada{\app{\lem[wit={J5,N23,P15,J10}]{uttabhya}% utabhya P15
	\rdg[wit={V19,E2,V15,V1,GrB,Jyo}]{uttambhya} % uttaṃbhā P11
	\rdg[wit={G11}]{unnamya}}
	cibukaṃ % cubukaṃ G4,P11,V15
\app{\lem[wit={J5,G4,N23,V19,E2,G11,P15,P11,C6,Jyo},alt={vakṣasy}]{vakṣa\skp{sy}} % vakṣyasy C6
	\rdg[wit={V15}]{cakṣasy}
	\rdg[wit={V1}]{vakṣaṃ}
	\rdg[wit={J10,V3}]{vakṣa}}}%
\pada{\app{\lem[wit={V19,E2,G11},alt={āsthāpya}]{\skm{sy }āsthāpya}%
	\rdg[wit={N23,Jyo}]{utthāpya}
	\rdg[wit={C6}]{utthāya}
	\rdg[wit={P15}]{utthāyot}
	\rdg[wit={V15}]{otthāpya}% otthāya G5
	\rdg[wit={J5,P11}]{osthāpyot}% vakṣa-sthāpyo F
	\rdg[wit={V1,J10,V3}]{sthāpayet}
	\rdg[wit={G4}]{ākṛṣya}
}\marmas
	pavanaṃ śanaiḥ//\versenr}% ṃ om. P11; patanaṃ P15
%\myfn{Incomplete description. See the philological commentary.}
%	\unavbl{N3}
	\\!}
\end{tlg}

\avacite{45a}
\commciterange{46}{45--46}\newpage


%1.47
\begin{tlg}[hp01_047]
\tl{% N3 is available again.
\pada{idaṃ padmāsanaṃ
\app{\lem[wit={ceteri}]{proktaṃ}
	\rdg[wit={V19}]{praktaṃ}}}
	\pada{sarvavyādhivināśanam/}\\+} % °kaṃ J5
\tl{
\pada{durlabhaṃ	yena kenāpi} % durlabha V19
\pada{\app{\lem[wit={N3,N23,V19,E2,G11,V1,J10,P11,V3,Jyo}]{dhīmatā labhyate}
	\rdg[wit={J5}]{dhīmatā labhate}
	\rdg[wit={G4,C6}]{dhīmatāṃ labhyate}
	\rdg[wit={P15,V15}]{labhyate dhīmatā}}
	\app{\lem[wit={ceteri}]{bhuvi}
	\rdg[wit={J5}]{matiḥ}}//\versenr}\\!}
\end{tlg}

\commcite

\begin{postmula}[hp01_047p]
\app{\lem[wit={N3,J5,N23,G11,C6,Jyo},post=\texteng{(evā \getsiglum{N3})}]{paścād uktam eva} % evā N3
	\rdg[wit={V19,E2,V1,J10,V3}]{paścād uktaṃ}
	\rdg[wit={P15,N19,V15}]{idam api}
	\rdg[wit={P11},alt={\om}]{\skp{\om}}}
\app{\lem[wit={N3,N19}]{matsaṃmatam}
	\rdg[wit={G11}]{tatsaṃmataṃ}
	\rdg[wit={P15,V15,V1,J10,V3}]{matsyamatam}
	\rdg[wit={N23,V19,E2,C6,Jyo}]{matsyendramatam} % macchendra N23
	\rdg[wit={J5}]{matsiṃdraḥ}
	\rdg[wit={P11},alt={\om}]{\skp{\om}}}/
%	\NotIn{P11}
\end{postmula}

\avacite{47p}\newpage

%1.48
\begin{tlg}[hp01_048]
\tl{\app{\lem[nolem]{}
	\rdg[wit={P11},alt={\om}]{\skp{\om}}}%
\pada{\app{\lem[wit={ceteri}]{kṛtvā}
	\rdg[wit={V19,E2}]{dhṛtvā}}
	saṃpuṭitau % °to N23
\app{\lem[wit={ceteri}]{karau}
	\rdg[wit={N23},alt={\om}]{\skp{\om}}}
	dṛḍhataraṃ baddhvā % dṛḍhatarau C6
\app{\lem[wit={ceteri}]{tu}
	\rdg[wit={G4}]{tha}
	\rdg[wit={V19,E2}]{ca}
	} % = VM
	padmāsanaṃ}\\+}
\tl{
\pada{gāḍhaṃ vakṣasi % gāḍhāṃ N3, dṛḍhaṃ ca J5
\app{\lem[wit={ceteri}]{saṃnidhāya}% +J5
	\rdg[wit={V1,J10}]{saṃvidhāya}
	\rdg[wit={N3,G4}]{nidhāya}}
	cibukaṃ % cubukaṃ V15,G4, cibuṭaṃ C6
\app{\lem[wit={ceteri}]{dhyānaṃ}
	\rdg[wit={Jyo}]{dhyāyaṃś}% +G3
	}
\app{\lem[wit={ceteri},alt={ca tac}]{ca ta\skp{c}}
	\rdg[wit={V1}]{tataś}}%
\app{\lem[wit={ceteri},alt={cetasi}]{\skm{c }cetasi}
	\rdg[wit={J10}]{cepsitaṃ}}\marma/}\\+}
\tl{
\pada{vāraṃ vāram apānam ūrdhvam anilaṃ % vālaṃ˟(1) N3; urdham V3
\app{\lem[wit={G11},alt={proccālayan}]{proccālaya\skp{n}}% +F
	\rdg[wit={G4}]{proccāla\,+}
	\rdg[wit={N3}]{proccārayaṃn}
	\rdg[wit={J5}]{procāraran}
	\rdg[wit={N23,V15}]{proccārayet}
	\rdg[wit={V3}]{procārayet}
	\rdg[wit={V1}]{protsālayan}
	\rdg[wit={E2,P15,C6,Jyo}]{protsārayan}% +K3
	\rdg[wit={V19}]{protsārayet}% +C7
	\rdg[wit={J10}]{prollāsayan}
	}%
\app{\lem[wit={ceteri},alt={pūritaṃ}]{\skm{n }pūritaṃ}
	\rdg[wit={V19,V1}]{pūrayan}
	\rdg[wit={E2}]{pūrayet}% +K3,C7
	}}\marma\\+}
\tl{
\pada{\app{\lem[wit={N3,N23,G11,P15,V15,V1,C6},post=\texteng{(upeti \getsiglum{N3})}]{muñcan prāṇam upaiti} % upeti N3
	\rdg[wit={G4}]{muṃcet prāṇam upaiti}
	\rdg[wit={J10,V3}]{muñcat prāṇam upaiti}% +F
	\rdg[wit={J5}]{muccaṃta prāṇapuṃsaiti}
	\rdg[wit={Jyo}]{nyañcan prāṇam upaiti}
	\rdg[wit={V19,E2}]{prāṇaṃ muñcati yāti}
	}
	bodham atulaṃ śakti%
\app{\lem[wit={ceteri}]{prabhāvān naraḥ}% naraḥ om. J5
	\rdg[wit={J10}]{prabhāvād ataḥ}
	\rdg[wit={V19,E2}]{prabhāvodayāt}}\marma//\versenr} %\NotIn{P11}
	\\!}
\end{tlg}

\commcite\newpage


%1.49
\begin{tlg}[hp01_049]
\tl{
\pada{\app{\lem[wit={ceteri}]{padmāsana}% °sane J8
	\rdg[wit={G4,J10,Jyo}]{padmāsane}}sthito yogī}
\pada{nāḍī\app{\lem[wit={ceteri}]{dvāreṣu} % nāḍi N3,N23
	\rdg[wit={V3,Jyo}]{dvāreṇa}% +F,G3
	}
\app{\lem[wit={G4,G11,V15,J10}]{pūrayan}
	\rdg[wit={N3,J5,N23,V19,E2,G5,P15,V1,GrB}]{pūrayet}% +F,G5
	\rdg[wit={Jyo}]{pūritam}}/}\\+}
\tl{
\pada{\app{\lem[wit={ceteri}]{mārutaṃ}
	\rdg[wit={P15}]{māruto}}
\app{\lem[wit={J5}]{mārayed yas tu}
	\rdg[wit={V19,E2,V15,Jyo}]{dhārayed yas tu}
	\rdg[wit={N3,G4,G11,P15}]{mriyate yas tu}% +M3,F,G7
	\rdg[wit={G5}]{dhriyate yas tu}
	\rdg[wit={P11,C6}]{nayate yas tu}
	\rdg[wit={N23}]{niyataṃ yas tu} % nīyate yas tu C1
	\rdg[wit={V3}]{pīvyate yas tu}
	\rdg[wit={V1}]{pīyate yas tu}
	\rdg[wit={J10}]{yas tu pibati}
	}}\marmas
\pada{sa \app{\lem[wit={ceteri}]{mukto}
	\rdg[wit={J5}]{śakto}} nātra saṃśayaḥ//\versenr}\\!} % śaṃsayaḥ N23, saṃsayaḥ V19
\end{tlg}

\commcite\newpage


\begin{ava}[hp01_050a]
\app{\lem[wit={ceteri}]{atha siṃhāsanam}
	\rdg[wit={N3}]{atha sīṃhāna}
	\rdg[wit={P11}]{siṃhāsanaṃ}
	\rdg[wit={V19}]{siṃhāsana yathā}
	\rdg[wit={J5}]{atha siddhāsan(!)}}/
\end{ava}


%1.50
\begin{tlg}[hp01_050]
\tl{
\pada{gulphau
\app{\lem[wit={ceteri}]{ca}
	\rdg[wit={V15}]{tu}% +F
	} vṛṣaṇasyādhaḥ} % vṛṣa<ṇa>syādhaḥ N3, vṛṣaṇ<asy>ādhaḥ P15, vṛṣaṇaḥsyādvaḥ P11, °syādha V1, °svādha J5
\pada{\app{\lem[wit={ceteri}]{sīvanyāḥ} % siva° N23
	\rdg[wit={N3}]{sīvānyāḥ}
	\rdg[wit={P15,P11}]{sīvinyāḥ}
	\rdg[wit={V19,E2}]{sīmanyāḥ}}
	pārśvayoḥ kṣipet/}\label{VuI50}\\+} % prā° N23; śve V3; °yo N3,J5,G11
\tl{
\pada{\app{\lem[wit={ceteri}]{dakṣiṇe}
	\rdg[wit={G11,P11}]{dakṣiṇaṃ}
	\rdg[wit={V3}]{dakṣaṇe}
	\rdg[wit={J5}]{dakṣe}}
\app{savya\lem[wit={ceteri}]{gulphaṃ tu}
	\rdg[wit={V19,E2}]{gulphaṃ ca} % 'savya! K3
	\rdg[wit={P11}]{gulpheṣu}
	\rdg[wit={G11}]{gulphena}}}
\pada{\app{\lem[wit={ceteri}]{dakṣagulphaṃ}
	\rdg[wit={G11}]{dakṣiṇena}}
\app{\lem[wit={ceteri}]{tu}
	\rdg[wit={V19,E2,V1}]{ca}
	\rdg[wit={G11}]{ta°}}
\app{\lem[wit={ceteri}]{savyake}
	\rdg[wit={P11}]{savyakam}
	\rdg[wit={V19}]{guhyake}
	\rdg[wit={G11}]{°thetaraṃ}}//\versenr}\\!}
\end{tlg}

%\commcite\newpage

%1.51
\begin{tlg}[hp01_051]
\tl{
\pada{hastau
\app{\lem[wit={N23,GrB}]{ca jānvoḥ}% janvauḥ N23
	\rdg[wit={N3}]{ca jāhno}
	\rdg[wit={G4}]{ca jānu}
	\rdg[wit={J5}]{jānyo}
	\rdg[wit={V19,E2}]{jānvoś ca}
	\rdg[wit={G11}]{jānvoḥ su°}
	\rdg[wit={P15,V15,V1,Jyo}]{tu jānvoḥ}
	\rdg[wit={J10}]{tu jānunauḥ}
	}
\app{\lem[wit={ceteri}]{saṃsthāpya}
	\rdg[wit={J10}]{sthāpya}}}
\pada{\app{\lem[wit={N3,V15,V1,J10,P11,V3,Jyo}]{svāṅgulīḥ}
	\rdg[wit={G4,N23,G11,P15,C6}]{svāṅgulī}
	\rdg[wit={N3}]{svāṃguliṃḥ}
	\rdg[wit={J5}]{saṃgulī}
	\rdg[wit={V19}]{aṅgulīḥ}
	\rdg[wit={E2}]{aṅgulī}}
\app{\lem[wit={ceteri}]{saṃprasārya}
	\rdg[wit={N23}]{yaṃ prasārmya}} ca/}\\+}
\tl{
\pada{\app{\lem[wit={ceteri}]{vyātta}
	\rdg[wit={V3}]{vyāta}
	\rdg[wit={V19}]{vyālā}}%
\app{\lem[wit={ceteri}]{vaktro} % vaktrā? N23
	\rdg[wit={P15}]{vaktrau}
	\rdg[wit={V3}]{vakro}}
\app{\lem[wit={ceteri}]{nirīkṣeta}
	\rdg[wit={V3}]{nirīkṣet}
	\rdg[wit={J10}]{nirīkṣyeta}
	\rdg[wit={N23}]{nirīkṣeya}
	\rdg[wit={G4}]{nikṣipet}}}
\pada{\app{\lem[wit={ceteri}]{nāsāgraṃ} % nāśā N3, nāsāyaṃ P11, nāsyagraṃ J5
	\rdg[wit={N23,J10,V3}]{nāsāgra}
	\rdg[wit={V1}]{nāsāgre}}
\app{\lem[wit={V19,E2,G11,P15,V15,P11,Jyo}]{susamāhitaḥ}
	\rdg[wit={N23}]{stusamāhitaḥ}
	\rdg[wit={N3,J5}]{tu samāhitaḥ}
	\rdg[wit={G4,C6}]{susamāhitaṃ}
	\rdg[wit={V1,J10}]{nyastalocanaḥ}
	\rdg[wit={V3}]{nyastalocanaṃ}}\marma//\versenr}\\!}
\end{tlg}

%\commcite\newpage

%1.52
\begin{tlg}[hp01_052]
\tl{
\pada{siṃhāsanaṃ
	bhave\app{\lem[wit={ceteri},alt={etat}]{\skm{d }eta\skp{t}}% sabhavetat J5
	\rdg[wit={N23}]{evaṃ}}t}
\pada{pūjitaṃ % pujitaṃ V3, pūjītaṃ V15, pūjita J10
\app{\lem[wit={ceteri}]{yogibhiḥ sadā}% yogabhiḥ V3; saha G4
	\rdg[wit={V19,E2}]{munipuṅgavaiḥ}
	\rdg[wit={Jyo}]{yogipuṅgavaiḥ}}/}\\+}
\tl{
\pada{bandha\app{\lem[wit={ceteri}]{tritaya} % <<ba>>ndha N23
	\rdg[wit={V1}]{tṛtīya}
	\rdg[wit={N23,V19,E2,V15}]{trayasya}}%
\app{\lem[wit={ceteri}]{saṃdhānaṃ}
	\rdg[wit={P15}]{saṃdhāyi}}}
\pada{\app{\lem[wit={ceteri}]{kurute}
	\rdg[wit={P15}]{sevate}}
\app{\lem[wit={ceteri}]{cāsanottamam}
	\rdg[wit={N3,V1,P11},alt={vāsano°}]{vāsanottamam}
	\rdg[wit={J10},alt={sādhano°}]{sādhanottamam}}\marma//\versenr}\\!}
\end{tlg}

\newpage
\avacite{50a}
\commciterange{52}{50--52}\newpage


\begin{ava}[hp01_053a]
\app{\lem[wit={N23,E2,G11,V1,J10,C6,V3,Jyo}]{atha bhadrāsanam} % V3 om. ṃ
	\rdg[wit={G4}]{\textapp{found betw. 1.52ab and cd}}
	\rdg[wit={V19}]{atha bhadraṃ}
	\rdg[wit={N3,J5,P15,V15,P11},alt={\om}]{\skp{\om}}}/ 
%	\NotIn{N3,J5,P15,V15,P11}
\end{ava}

%1.53
\begin{tlg}[hp01_053]
\tl{
\pada{\app{\lem[nolem]{\skp{pāda a}}
	\rdg[wit={P15,V3},alt={\om}]{\skp{\om}}}%
gulphau \app{\lem[wit={ceteri}]{ca}
	\rdg[wit={J5,G4,G11}]{tu}% +F
	} vṛṣaṇasyādhaḥ} % ḥ om. V1
\pada{\app{\lem[nolem]{\skp{pāda b}}
	\rdg[wit={P15,V3},alt={\om}]{\skp{\om}}}%
\app{\lem[wit={N3,N23,G11,V15,V1,C6,Jyo}]{sīvanyāḥ}
	\rdg[wit={J5,J10}]{sīvanyā}
	\rdg[wit={P11}]{sīvinyāḥ}
	\rdg[wit={V19,E2}]{sīmanyāḥ}
	\rdg[wit={G4}]{sebhanyāḥ}
%	\rdg[wit={P15,V3},alt={\om}]{\skp{\om}}
	} 
	pārśvayoḥ kṣipet/} % pārsva V19, °yo G11
%	\anm{=\,\ref{VuI50}ab} 
%	\lineom{ab}{P15,V3}
	\\+}
%\tl{
%\grau{\pada{savyagulphaṃ tathā savye} % savyaṃ C6
%\pada{dakṣagulphaṃ 
%	\app{\lem[wit={N23,Jyo}]{tu}
%	\rdg[wit={C6}]{ca}} dakṣiṇe/}
%	\anm{cf. \ref{VuI50}cd}
%	\sgwit{N23,C6,Jyo}}\\+} % not in V19,E2,G11,P15,V15,P11!
\tl{
\pada{\app{\lem[wit={ceteri}]{pārśva}
	\rdg[wit={V19,E2,V15}]{pārśve}
	\rdg[wit={P15}]{pārśvau}}pādau ca
	pāṇibhyāṃ} % pāni° V19
\pada{dṛḍhaṃ
\app{\lem[wit={ceteri}]{baddhvā}% baṃdhvā P11, badhvā V3
	\rdg[wit={V19}]{baddhaṃ}}
\app{\lem[wit={G4,E2,V15,V1,J10,P11,C6,Jyo}]{suniścalam}
	\rdg[wit={N3,G11}]{suniścalaḥ}
	\rdg[wit={J5}]{suniścayam}
	\rdg[wit={V3}]{tsuniścalaṃ}
	\rdg[wit={N23}]{stuniścalaṃ}
	\rdg[wit={P15}]{tu niścalaṃ}% +F
	\rdg[wit={V19}]{suniścitaṃ}}//\versenr}\myfn{%
	Between 1.53ab and 53cd, \getsiglum{N23,C6,Jyo} have an additional line: 
	\devnote{savyagulphaṃ tathā savye dakṣagulphaṃ tu dakṣiṇe} (cf. \ref{VuI50}cd).} 
	\\!}
\end{tlg}

%1.54
\begin{tlg}[hp01_054]
\tl{
\pada{bhadrāsanaṃ bhaved etat} % eva J5
\pada{sarvavyādhi\app{\lem[wit={Gr1,G11,J10,P11,V3}]{viṣāpaham}
	\rdg[wit={N23,V19,E2,P15,V15,C6,Jyo}]{vināśanam}
	\rdg[wit={V1}]{\{\{vināśanaṃ\}\}viṣāpahaṃ}% corr. by the first hand
	}\marma/}\\+}
\tl{
\pada{gorakṣāsanam ity āhu}% gorikṣā° P15
\pada{r i\app{\lem[wit={ceteri},alt={idaṃ}]{\skp{i}daṃ}
	\rdg[wit={N23}]{evaṃ}}
	\app{\lem[wit={ceteri}]{vai siddhayoginaḥ}% siddhi J5
	\rdg[wit={P11}]{te siddhayoginaḥ}
	\rdg[wit={C6}]{siddhāś ca yoginaḥ}}//\versenr}\\!}
\end{tlg}

\avacite{53a}
\commciterange{54}{53--54}
\newpage


%1.55
\begin{tlg}[hp01_055]
\tl{\app{\lem[alt={\ante evam \add},nosep]{\skp{\ante evam \add}}
	\rdg[wit={N3},post=\texteng{(cf.\,1.56ab)}]{asanaṃ kuṃbhakaṃ citraṃ mudrākhyaṃ karaṇaṃ bhavet}}%
\pada{\app{\lem[wit={ceteri}]{evam āsana}
	\rdg[wit={P15}]{pavanāsana}}bandheṣu} % baddheṣu E2
\pada{\app{\lem[wit={ceteri}]{yogīndro}% yogiṃdro V3
	\rdg[wit={P15}]{yogeṃdro}}
\app{\lem[wit={Gr1,E2,G11,P15,V15,P11}]{vijitaśramaḥ}% nijitā° J5
	\rdg[wit={N23}]{vijitaḥ śramaḥ}
	\rdg[wit={V19}]{vijiteśramaḥ}
	\rdg[wit={V3}]{vijitaśramāṃ}
	\rdg[wit={V1,J10,C6,Jyo}]{vigataśramaḥ}% +F
	}/}\\+}
\tl{
\pada{\app{\lem[wit={N23,V19,E2,J10,P11,C6},alt={athābhyasen}]{athābhyase\skp{n}}
	\rdg[wit={J5,G11}]{athābhyāsen}
	\rdg[wit={G4}]{athābhy[ā]s.}% athābhyāsān F
	\rdg[wit={P15}]{athābhyase}
	\rdg[wit={V15}]{athābhyāse}
	\rdg[wit={V3}]{athābhyāsaṃ}
	\rdg[wit={V1}]{athabhyā\,.e}
	\rdg[wit={Jyo}]{abhyasen}
	\rdg[wit={N3}]{abhyāse}}% 
\app{\lem[wit={N3,G4,N23,V19,E2,G11,V15,V1,GrB},alt={nāḍi}]{\skm{n }nāḍi} % ra vipulā (with caesura after the 4th)
	\rdg[wit={J10},post={\unm}]{nāḍī} % ma vipulā (but no caesura after 5th)
	\rdg[wit={P15},post=\texteng{(with both vowel signs)}]{nāḍi/ḍī}
	\rdg[wit={J5}]{nā}% athābhyāsen nāśuddhiḥ syāt J5!
	\rdg[wit={Jyo}]{nāḍikā}}%
\app{\lem[wit={V19,E2,G11,V15,J10,P11,V3,Jyo}]{śuddhiṃ}% su° V3
	\rdg[wit={N23,C6}]{śuddhi}
	\rdg[wit={V1}]{śvaddhiṃ}
	\rdg[wit={N3,J5}]{śuddhiḥ syān}% G4 illeg.; śuddhir F
%	\rdg[wit={G4},alt={\illeg}]{\skp{\illeg}}
	\rdg[wit={P15}]{ṣu}}}
\pada{\app{\lem[wit={ceteri}]{mudrādi}% mudrādī F
	\rdg[wit={E2}]{mudrayā}% +K3,C7
	\rdg[wit={V19}]{subaddhvā}}%
pavana\app{\lem[wit={V19,E2,V1,J10,P11,C6,Jyo}]{kriyām} % pa<<va>>na N23
	\rdg[wit={N3,G11,P15,V15}]{kriyāḥ}
	\rdg[wit={J5,N23,V3}]{kriyā}% +F
	}//\versenr}\\!}
\end{tlg}

\commcite\newpage

% These verses are in the J10 branch, with V3 also. (but not in V1,V19,E2,C1)

\newpage
\startaltrecension
\begin{alttlg}[hp01_055_1]
\tl{\app{\lem[alt=\mylem{55*1--2},nosep,nonum]{}
	\rdg[wit={J10,C6,V3,Jyo},postwit=\textapp{(found after 1.64 \getsiglum{Jyo})}]{\incl}}%
\pada{kriyāyuktasya siddhiḥ syād}% siddhi V3
\pada{akriyasya kathaṃ bhavet/} \\+
}\tl{
\pada{na śāstrapāṭhamātreṇa}
\pada{yogasiddhiḥ prajāyate//\versenr} % siddhi V3
\sgwit{J10,C6,V3,Jyo}%
%\myfn{In \getsiglum{Jyo} this verse and the next one (without the 3rd line) are found after 1.64.}
\\!
}\end{alttlg}

\teimute{\begin{quote}%
\textcolor{gray}{\ExecuteMetaData[\commfilename]{tr55-1}
\texteng{(55*1)}}
\end{quote}}

\begin{alttlg}[hp01_055_2]
\tl{%\app{\lem[nolem]{}
	%\rdg[wit={J10,C6,V3,Jyo}]{\incl}
	%\rdg[wit={Jyo},alt=\textapp{the first two lines found after 1.64 together with the precious verse}]{\skp{the first two lines found after 1.64 together with the precious verse}}}%
\pada{na
\app{\lem[wit={J10,C6,Jyo}]{veṣadhāraṇaṃ}
	\rdg[wit={V3}]{veṣṭadhāriṇyo}}
\app{\lem[wit={J10,C6,Jyo}]{siddheḥ}
	\rdg[wit={V3}]{siddhi}}}
\pada{kāraṇaṃ
\app{\lem[wit={J10,C6,Jyo}]{na ca}
	\rdg[wit={V3}]{ca}}
\app{\lem[wit={C6,V3,Jyo}]{tatkathā}
	\rdg[wit={J10}]{tatkathāḥ}}/}\\+
}\tl{
\pada{kriyaiva kāraṇaṃ
\app{\lem[wit={J10,Jyo}]{siddheḥ}
	\rdg[wit={V3}]{siddhi}
	\rdg[wit={C6}]{siddhaṃ}}}
\pada{satya\app{\lem[wit={J10,Jyo},alt={etan}]{\skm{m }eta\skp{n}}
	\rdg[wit={C6}]{eva}
	\rdg[wit={V3}]{eva tat}}n na saṃśayaḥ/}\\+
}\tl{
\pada{\app{\lem[nolem]{\skp{pāda e}}
	\rdg[wit={Jyo},alt={\om}]{\skp{\om}}}%
śiśnodara%
\app{\lem[wit={J10}]{ratāyeha}
	\rdg[wit={V3}]{ratāyena}
	\rdg[wit={C6}]{ratāya\,..}}}
\pada{\app{\lem[nolem]{\skp{pāda f}}
	\rdg[wit={Jyo},alt={\om}]{\skp{\om}}}%
na
\app{\lem[resp=emend]{deyā}
	\rdg[wit={C6,V3}]{deyo}
	\rdg[wit={J10}]{dayo}}
\app{\lem[wit={J10,C6}]{veṣa}
	\rdg[wit={V3}]{viṣa}}%
\app{\lem[wit={J10,V3}]{dhāriṇaḥ}
	\rdg[wit={C6}]{dhāriṇe}}//\versenr} 
	\sgwit{J10,C6,V3,Jyo}\\!}
\end{alttlg}

\teimute{%
\def\commvnum{55-2}%
\def\labelvnum{55*1--2}%
\begin{quote}%
\textcolor{gray}{\ExecuteMetaData[\commfilename]{tr\commvnum}
\texteng{(55*2)}}\footnotetext{\commlabel%
\ExecuteMetaData[\commfilename]{sc\commvnum}%
\ExecuteMetaData[\commfilename]{ts\commvnum}%
\ExecuteMetaData[\commfilename]{cm\commvnum}%
}
\end{quote}}

%\commciterange{55-2}{55*1--2}
\newpage

\begin{alttlg}[hp01_055_3]
\tl{\app{\lem[nolem]{}
	\rdg[wit={V19,E2}]{\incl}}%
\pada{\app{\lem[wit={E2}]{mayi}% +K3,C7,C8
	\rdg[wit={V19}]{miyi}}
\app{\lem[wit={E2}]{bodhāmbudhau}
	\rdg[wit={V19}]{bodhoṃbudhau}}
	svacche} % L1
\pada{tuccho'yaṃ
	viśva\app{\lem[wit={V19}]{budbudaḥ}
	\rdg[wit={E2}]{budbudhaḥ}}/}\\+
}\tl{
\pada{\app{\lem[wit={V19}]{pralīna}
	\rdg[wit={E2}]{pralīne}} udito veti}
\pada{vikalpapaṭalaḥ kutaḥ//\versenr}
\sgwit{V19,E2}\\!
}\end{alttlg}

\altcommcite\newpage

\begin{alttlg}[hp01_055_4]
\tl{\app{\lem[nolem]{}
	\rdg[wit={V19,E2}]{\incl}}%
\pada{śruti\app{\lem[wit={E2}]{pratītiḥ}
		\rdg[wit={V19}]{prītaḥ}}
	svagurupratītiḥ}\\+
}\tl{
\pada{svātmapratītir manaso%
\app{\lem[resp=emend]{'pi rodhaḥ}% N11/J13/C8
	\rdg[wit={V19,E2}]{'pi bodhaḥ}}/}\\+ % L1/N5/V19
}\tl{
\pada{etāni sarvāṇi
\app{\lem[wit={V19}]{samuccitāni} % C1/N11/J13/V19
	\rdg[wit={E2}]{samuddhṛtāni}}}\\+ % +K3/C7/L1/N5
}\tl{
\pada{matāni dhīrair iha sādhanāni//\versenr} 
\sgwit{V19,E2}\\!
}\end{alttlg}
\endaltrecension

\altcommcite\newpage

%1.56
\begin{tlg}[hp01_056]
\tl{\app{\lem[nolem]{}
	\rdg[wit={V1},alt={\om}]{\skp{\om}}}%
\pada{āsanaṃ % āsana V3
\app{\lem[wit={ceteri}]{kumbhakaṃ}
	\rdg[wit={G11,J10,V3}]{kumbhakaś}}
\app{\lem[wit={ceteri}]{citraṃ}
	\rdg[wit={N23,P15}]{citra}
	\rdg[wit={G11}]{citro}}}\marmas
\pada{\app{\lem[wit={cetwG4}]{mudrākhyaṃ}
%	\rdg[wit={N3}]{mudrākhya} °aṃ N3ditto
	\rdg[wit={N23}]{mudrāśyaṃ}
	\rdg[wit={J5,J10,V3}]{mudrādi}}
\app{\lem[wit={ceteri}]{karaṇaṃ tathā}
	\rdg[wit={G4}]{karaṇādikaṃ}
	\rdg[wit={J10}]{karaṇāni ca}
	\rdg[wit={J5,V3}]{pavanakriyā}}\marma/}\\+}
\tl{
\pada{atha nādānusandhāna}%m% nādāna° P15
\pada{\app{\lem[wit={ceteri},alt={°m abhyāsā°}]{\skp{°}m abhyāsā}
	\rdg[wit={V19,C6}]{m abhyāsyā}
	\rdg[wit={V3}]{syābhyāsā}}%
\app{\lem[wit={J5,N23,G11,V15,J10,C6,V3,Jyo},alt={°nukramo haṭhe}]{\skp{°}nukramo haṭhe}
	\rdg[wit={N3,P11}]{nukramo haṭhaḥ}
	\rdg[wit={P15}]{nukramo haṭho}
	\rdg[wit={G4}]{dukrame haṭhe}
	\rdg[wit={V19,E2}]{nukrameṇa tu}}//\versenr}
%	\NotIn{V1}
	\\!}
\end{tlg}

\commcite%\newpage


%1.57
\begin{tlg}[hp01_057]
\tl{
\pada{brahmacārī
\app{\lem[wit={ceteri}]{mitāhārī}
	\rdg[wit={V19,E2,G11}]{mitāhāro}}}
\pada{\app{\lem[wit={Gr1,V19,E2,G11,P15,V1,P11}]{yogī} % +N19
	\rdg[wit={N23,V15,J10,C6,V3,Jyo}]{tyāgī}}
	yogaparāyaṇaḥ% °naḥ V19
	/}\\+}
\tl{
\pada{abdād ūrdhvaṃ % ṃ om. V3,J10
	bhave\app{\lem[wit={ceteri},alt={siddho}]{\skm{t }siddho}
	\rdg[wit={N3}]{siddhir}% +F
	\rdg[wit={J5,G4,P11}]{siddhi}
	\rdg[wit={J10}]{siddhīn}}}
\pada{nātra
\app{\lem[wit={ceteri}]{kāryā}
	\rdg[wit={N23ac,V3}]{kārya}
	\rdg[wit={C6}]{kāryo}}
\app{\lem[wit={ceteri}]{vicāraṇā}% °riṇā J5
	\rdg[wit={J10}]{vicāraṇāt}
	\rdg[wit={C6}]{vicāraṇe}}//\versenr}\\!}
\end{tlg}

\commcite\newpage

%1.58
\begin{tlg}[hp01_058]
\tl{
\pada{susnigdhamadhu\app{\lem[wit={G4,N23,V19,E2,G11,P15,V15,Jyo},alt={āhāraś}]{\skm{r}āhāra\skp{ś}}% +F
	\rdg[wit={N3,J5,P11,C6}]{āhāra}
	\rdg[wit={V3}]{āhāraṃ}
	\rdg[wit={V1}]{āhāra<<ḥ>>}
	\rdg[wit={J10}]{āhāraḥ}
	}}%
\pada{\app{\lem[wit={ceteri},alt={caturthāṃśa}]{\skm{ś }caturthāṃśa}% āṃsa J10
	\rdg[wit={N23}]{caturthāśā}}%
\app{\lem[wit={ceteri}]{vivarjitaḥ}
	\rdg[wit={P11}]{vivarjita}
	\rdg[wit={V3}]{vivarjitam}}/}\\+}
\tl{
\pada{bhujyate % bhū° N3, bhuṃ° V3
	\app{\lem[wit={ceteri}]{śivasaṃprītyai} % °tyaiḥ C6
	\rdg[wit={V1}]{śivasaṃpritya}
	\rdg[wit={J5}]{yama ca prokto}}}
\pada{\app{\lem[wit={ceteri}]{mitāhāraḥ}% °ra J5
	\rdg[wit={N23}]{mitāhārī}}
\app{\lem[wit={ceteri}]{sa ucyate}
	\rdg[wit={V1,V3}]{samucyate}}//\versenr}\\!}
\end{tlg}

\commcite\newpage


%1.59
\begin{tlg}[hp01_059]
\tl{
\pada{\app{\lem[wit={ceteri}]{kaṭvamla}% kaṭra? C6
	\rdg[wit={V1,V3,Jyo}]{kaṭvāmla}
	\rdg[wit={J5}]{kaṭkāṃmla}}%
\app{\lem[wit={ceteri}]{tīkṣṇa}
	\rdg[wit={G4,V19}]{tikta}% +F
	}lavaṇoṣṇa%
\app{\lem[wit={N3,G4,V19,E2,G11,P15,V1,J10,Jyo}]{harīta}
	\rdg[wit={J5,V15}]{hārīta}
	\rdg[wit={N23,C6,V3}]{harita}
	\rdg[wit={P11}]{hārahārīta}}%
\app{\lem[wit={ceteri}]{śāka}% sāka J10
	\rdg[wit={V1,V3}]{śākaṃ}}}-\\+}
\tl{
\pada{sauvīrataila%
%\myfn{\emph{sauvīra} is glossed as \emph{kāṃjī} in \getsiglum{J8}. Cf. Brahmānanda's comm.: \emph{sauvīraṃ kāñjikam}.}%
	\app{\lem[wit={ceteri}]{tila}
	\rdg[wit={V1},alt={\illeg}]{\skp{\illeg}}
	\rdg[wit={N23,C6},alt={\om}]{\skp{\om}}}%
\app{\lem[wit={ceteri}]{sarṣapa}
	\rdg[wit={P11}]{sarpiṣa}}%
\app{\lem[wit={V19,V15,J10,V3}]{matsyamadyam}% +K3,C7
	\rdg[wit={N3}]{machyamadyā}% ??
	\rdg[wit={E2}]{matsyamadhyam}
	\rdg[wit={P15}]{matsyamadyāḥ}
	\rdg[wit={V1}]{tsyamaghaṃ}
%	\rdg[wit={N23ac}]{madyama\,..\,n}
	\rdg[wit={J5}]{madyamatsā}
	\rdg[wit={G4,N23,C6,Jyo}]{madyamatsyān}% °ma .. n N23ac, °tsyāṃ G4 ##
	\rdg[wit={G11,P11}]{madyamatsyāḥ}% +G5,F; māṃtsyāḥ P11
	}/}\\+}
\tl{
\pada{\app{\lem[wit={P15,Jyo},post=\texteng{\emph{m.c.\@}}]{ājādi}% +F
	\rdg[wit={N23,V15,P11}]{ajādi}% +G5
	\rdg[wit={N3}]{ājīvi}
	\rdg[wit={J5}]{ajādhi}
	\rdg[wit={E2,G11,J10}]{ājāvi}
	\rdg[wit={V19,V1,V3}]{ajāvi}% +K3,C7,G7
	\rdg[wit={C6}]{ajavya}}%
\app{\lem[wit={ceteri}]{māṃsa}% māṃśa V3
	\rdg[wit={V1}]{māsaṃ}
	\rdg[wit={V15}]{māṃsaṃ}}dadhitakra%
\app{\lem[wit={N3,N23,E2,P15,V15,J10,C6,Jyo}]{kulattha}% Jyo: kulattha or kulittha
	\rdg[wit={G11}]{kulastha}
	\rdg[wit={V19,V3}]{kulatha}% kultha? V19
	\rdg[wit={P11}]{kalatha}
	\rdg[wit={V1}]{kulatthya}
	\rdg[wit={J5}]{kulittha}
	\rdg[wit={G4}]{kuluddha}
	}%
\app{\lem[wit={N3,J5,V19,E2,G11,V15,GrB,Jyo}]{kola} % lokola C6
	\rdg[wit={P15}]{kela}
	\rdg[wit={G4,J10}]{kodra}
	\rdg[wit={V1}]{koṣṇā}
	\rdg[wit={N23}]{kāla}}-}\\+}% kolyā?
%\myfn{\emph{kola} is glossed as \emph{bhaṭavāsa} in \getsiglum{J8}.}
\tl{
\pada{\app{\lem[wit={ceteri}]{piṇyāka}
%	\rdg[wit={J10}]{piṃṇyāka}
	\rdg[wit={P11}]{paṇyaka}
	\rdg[wit={J5}]{pinyāka}
	\rdg[wit={V3}]{pinnāka}
	\rdg[wit={N23},alt={\om}]{\skp{\om}}}%
	hiṅgu\app{\lem[wit={ceteri},alt={laśunā-/lasunādyam}]{laśunādya\skp{m}}% laśunā C7,V15; lasunā N23,V1,V3,J10,V19
	\rdg[wit={J5}]{lasanādyam}
	\rdg[wit={G11}]{liśunādyam}
	\rdg[wit={P15}]{laśanādyam}
	\rdg[wit={P11}]{laśunādy}}%
\app{\lem[wit={ceteri},alt={apathyam}]{\skm{m }apathya\skp{m}}
	\rdg[wit={P11}]{avarppam}}m āhuḥ//\versenr}\\!} % āhu N3
\end{tlg}

\commcite\newpage
\ \newpage%\ \newpage

%1.60
\begin{tlg}[hp01_060]
\tl{
\pada{bhojanam ahitaṃ
\app{\lem[wit={J5,N23,V19,G11,Jyo},alt={vidyāt}]{vidyā\skp{t}}
	\rdg[wit={P15}]{vidyā}
	\rdg[wit={N3,G4,E2,V15,V1,J10,GrB}]{viṃdyāt}}% viṃdyā V1
\app{\lem[wit={ceteri},alt={punar apy}]{\skm{t }punar a\skp{py}}
	\rdg[wit={G4,V19,E2,P15}]{punar}
	}%
\app{\lem[wit={N23,G11,P15,V15,V1,J10,Jyo},alt={uṣṇīkṛtaṃ}]{\skm{py }uṣṇīkṛtaṃ}
	\rdg[wit={G4}]{uṣṇikṛta}
	\rdg[wit={N3}]{uṣṇi}
	\rdg[wit={J5}]{ullīkṛtaṃ}
	\rdg[wit={P11}]{uṣṇuṃktaṃta}
	\rdg[wit={C6}]{asvīkṛtaṃ}
	\rdg[wit={V3}]{uśnakrataṃ}
	\rdg[wit={V19,E2}]{uṣṇībhūtam}
	}
\app{\lem[wit={ceteri}]{rūkṣam}
	\rdg[wit={V1}]{rūkṣa}
	\rdg[wit={V19,E2}]{apramitaṃ}}/}\\+}
\tl{
\pada{\app{\lem[resp=emend]{atilavaṇam amlapṛktaṃ}
	\rdg[wit={N23}]{atīlavaṇāmlapṛktaṃ}
	\rdg[wit={N3}]{atilavaṇādyaprantaṃ}
	\rdg[wit={J5}]{atilavaṇādyaptataptaṃ}
	\rdg[wit={G11}]{atilavaṇādyaṃ proktaṃ}
	\rdg[wit={Jyo}]{atilavaṇam amlayuktaṃ}
	\rdg[wit={C6}]{atilavaṇāmlayuktaṃ}
	\rdg[wit={V15,P11}]{atilavaṇādiyuktaṃ}
	\rdg[wit={P15}]{atilavaṇādiprayuktaṃ}
	\rdg[wit={G4}]{atilavaṇaka[ṭu]prayukta}
	\rdg[wit={V1}]{atilavaṇādyuṣṇataṃ}
	\rdg[wit={J10}]{atilavaṇaṃ tilapiṇḍa}
	\rdg[wit={V3}]{atilavaṇaṃ tilaṃ piṇḍa}
	\rdg[wit={V19}]{atilavaṇasavapalala}
	\rdg[wit={E2}]{atilavaṇāsavapalalaṃ}
	}\marmas
\app{\lem[wit={ceteri}]{kadaśana}
	\rdg[wit={G11,J10,V3}]{kadaśanaṃ}}%
\app{\lem[wit={ceteri}]{śākotkaṭaṃ}
	\rdg[wit={V3}]{śākātkaṭa}
	\rdg[wit={V1}]{śokātkaṭa}
	\rdg[wit={N23}]{śākokṣadaṃ}}
\app{\lem[wit={J5,N23,P15,V15,V1,C6}]{duṣṭam}
	\rdg[wit={P11}]{duṣṭī}
	\rdg[wit={N3}]{duṣṇaṃ}
	\rdg[wit={G4}]{dṛṣṭaṃ}
	\rdg[wit={G11}]{ruṣṭaṃ}
	\rdg[wit={V19,E2,Jyo}]{varjyam}
	\rdg[wit={J10}]{varjjaṃ}
	\rdg[wit={V3}]{varjitaṃ}}\marma//\versenr}
%	\anm{Upagīti}
	\\!}
\end{tlg}

\commcite\newpage



\begin{ava}[hp01_061a]
\app{\lem[nolem]{}
	\rdg[wit={V19,E2,P15,V15},alt={\om}]{\skp{\om}}
	\rdg[wit={G4},alt=\textapp{found between \manuref{1.61--62}}]{\skp{found between 1.61--62}}
	}%
\app{\lem[wit={G4,N23,GrB,Jyo}]{tathā hi}
	\rdg[wit={N3,J5,V1,J10}]{tathā}}
	gorakṣavacanam/% gorakha P11, gorakakṣa N23
\app{\lem[alt={\post gorakṣavacanam \add},nosep]{\skp{\post gorakṣavacanam \add}}
	\rdg[wit={G4}]{tailāmlalavaṇāni tiṃtriṇi ś[o]kaṃ śad.\,..\,nime}
	\rdg[wit={V1}]{tailāmlāloṇītīnikālikābhāi\,(?)}}%
	% G7 also has tailāmlalavaṇāniti
	\myfn{Before the head line, \getsiglum{Jyo} has an additional line:
	\devnote{vahnistrīpathisevānām ādau varjanam ācaret} (cf. HP \manuref{3.31}cd).}
%	\sgwit{Gr1,N23,V1,J10,GrB}
%	\NotIn{V19,E2,P15,V15}%
\end{ava}


%1.61
\begin{tlg}[hp01_061]
\tl{
\pada{varjaye%d varjaye N3,J5,P11,V3,N23,G11,V15,V1
\app{\lem[wit={ceteri},alt={durjana}]{\skm{d }durjana}
	\rdg[wit={C6}]{durjanaṃ}
	\rdg[wit={V1}]{tarjana}}%
\app{\lem[wit={N3,G4,P15,P11,Jyo}]{prāntaṃ}% +F
	\rdg[wit={G11,V1}]{prātaṃ}% vrātaṃ G11
	\rdg[wit={J5}]{prāptaṃ}
	\rdg[wit={C6}]{prāpte}
	\rdg[wit={N23,V19,E2,V15,J10}]{prītiṃ}
	\rdg[wit={V3}]{prīti}}}
\pada{\app{\lem[wit={ceteri}]{vahnistrī} % stri N3,G11
	\rdg[wit={V19}]{vastrī}}%
\app{\lem[wit={ceteri}]{patha}
	\rdg[wit={J5,V3}]{pathya}
	\rdg[wit={Jyo}]{pathi}
	\rdg[wit={V19}]{madhu}}sevanam/}\\+} % sevana N3
\tl{
\pada{\app{\lem[wit={ceteri}]{prātaḥ}
	\rdg[wit={N3,J5,V19,G11,V1,J10}]{prāta}
	}snānopavāsādi}% °śnāno V3
\pada{kāya\app{\lem[wit={N3,G4,N23,G11,P15,V15,C6,V3,Jyo}]{kleśavidhiṃ} % kāyā N3; kleśaṃ P15
	\rdg[wit={J5,P11}]{kleśavidhis}
	\rdg[wit={V19,V1,J10}]{kleśādikaṃ}% +K3,C7
	\rdg[wit={E2}]{kleśādhikaṃ}}
\app{\lem[wit={ceteri}]{tathā}
	\rdg[wit={V19}]{yathā}}//\versenr}%
	\\!}
\end{tlg}

\avacite{61a}
\commcite\newpage


%1.62
\begin{tlg}[hp01_062]
\tl{
\pada{\app{\lem[wit={ceteri}]{godhūma}
	\rdg[wit={V19}]{godhūmā}}śāli% śālī N23
\app{\lem[wit={ceteri}]{yava} % ya<<va>> V15
	\rdg[wit={J5,V19,V1,C6}]{java}}%
\app{\lem[wit={V19,E2,V15,V1,J10,P11,V3}]{ṣaṣṭika}% Apte
	\rdg[wit={G11,C6,Jyo}]{ṣāṣṭika}% +F
	\rdg[wit={N3}]{śāṣṭika}
	\rdg[wit={J5}]{śākdhikṛ(?)}
	\rdg[wit={N23}]{māṣikaṃ}
	\rdg[wit={G4}]{piṣṭika}
	\rdg[wit={P15}]{piṣṭaka}}%
\app{\lem[wit={ceteri}]{śobhanānnaṃ}
	\rdg[wit={N23,V3}]{śobhanānna}
	\rdg[wit={V1}]{śobhanānnānī}}}\\+}
\tl{
\pada{kṣīrājya\app{\lem[wit={N3,J5,N23,V19,E2,G11,P11,C6}]{maṇḍa}% +J5
	\rdg[wit={G4,P15,V15,J10,Jyo}]{khaṇḍa}
	\rdg[wit={V1,V3}]{ṣaṃḍa}}%
\app{\lem[wit={ceteri}]{navanīta}
	\rdg[wit={J5,V1}]{navanīti}
	\rdg[wit={N23}]{va<<na>>nīta}
	}%
\app{\lem[wit={ceteri}]{sitā}
	\rdg[wit={V1}]{śītā}
	\rdg[wit={P15}]{sudhā}}madhūni/}\\+} % °nī P11
\tl{
\pada{\app{\lem[wit={ceteri}]{śuṇṭhī}% suṃṭhī V1,V3, suṭhī V19
	\rdg[wit={G11}]{kuṇṭhī}}% 
\app{\lem[wit={ceteri}]{paṭolaka} % vaṭo° C7
	\rdg[wit={P15,V1}]{paṭolika}
	\rdg[wit={J10}]{paṭola}}%
%\myfn{\emph{paṭolaka} is glossed as \emph{palavala} in \getsiglum{J8}. Cf. Brahmānanda's comm.: \emph{paṭolakaphalaṃ paravara iti bhāṣāyāṃ prasiddhaṃ}.}%
\app{\lem[wit={GrB}]{phalādi ca}
	\rdg[wit={G4,N23,V19,E2,P15,Jyo}]{phalādika}
	\rdg[wit={N3}]{phalādi<<ka>>}
	\rdg[wit={J5}]{phalādi}
	\rdg[wit={V15}]{phalādiṣu}
	\rdg[wit={G11}]{phalāni ca}% +G5 (many South Ind. mss.)
	\rdg[wit={J10}]{phalakādi ca}
	\rdg[wit={V1}]{phipalādika}
	}
\app{\lem[wit={ceteri}]{pañcaśākaṃ} % ṃ oṃ? V15, <<paṃ>>cakaṃ N23
	\rdg[wit={N3}]{pacyaśākaṃ}
	\rdg[wit={J10}]{śākabhuktaṃ}}} \\+}
\tl{
\pada{\app{\lem[wit={ceteri}]{mudgādi}
	\rdg[wit={N3,V15,V3}]{mudgā}
	\rdg[wit={C6}]{mu\,\_\,di}}
\app{\lem[wit={ceteri},alt={divyam}]{divya\skp{m}}
	\rdg[wit={V19}]{cālpam}}m % +C7
	udakaṃ\marmas
\app{\lem[wit={ceteri}]{ca}
	\rdg[wit={P15}]{hri\,(?)}
	\rdg[wit={P11,V3},alt={\om}]{\skp{\om}}}
\app{\lem[wit={Gr1,G11,P15,J10,C6,V3,Jyo}]{yamīndra}% yamīdrya P15, yamidra N3, yaṃmīṃdra J5
	\rdg[wit={N23}]{yatīndra}
	\rdg[wit={P11}]{yavatīṃdra}
	\rdg[wit={V19,E2,V15,V1}]{munīndra}}pathyam\marma//\versenr}%
\myfn{After this verse, \getsiglum{N23} has the following verse:\\
%which is cited in the Jyotsnā: \devnote{tad uktaṃ vaidyake}---
	\vspace{2pt minus 1pt}
	\devnote{sarvaśākamacākṣuṣyaṃ cakṣapyaṃ} (\textit{recte} \devnote{cākṣuṣyaṃ}) \devnote{śākapañcakam/
	jīvantī vāstu matsyākṣī meghanādaḥ punarnavā//};\\
%	\getsiglum{V3} has instead:
%	\devnote{kṣīravarṇī (\recte °parṇī) ca jaivantī (\recte jīvantī) matsāṣī (\recte matsyākṣī) ca punarnavā/
%	meghanādīti pañcaite śākanāma(?) prakīrtitā(?)//\versenr}\\
	\getsiglum{V15,V3} have instead:
	\devnote{kṣīraparṇī ca jīvantī matsyākṣī ca punarnavā/
	meghanādaś ca pañcaite pañcaśākāḥ prakīrtitāḥ//} (\getsiglum{V15};\\  
	\devnote{kṣīravarṇī ca jaivantī matsāṣī ca punarnavā/
	meghanādīti pañcaite śākanāma prakīrtitā//} \getsiglum{V3})}
	\\!}
\end{tlg}

\commcite\newpage\ \newpage

% \startaltrecension
%
% \begin{alttlg}[hp01_062_1]
%   \tl{
%     \pada{\app{\lem[wit={V15}]{kṣīraparṇī}
%	 \rdg[wit={V3}]{kṣīravarṇī}} ca
%       \app{\lem[wit={V15}]{jīvantī}
%	   \rdg[wit={V3}]{jaivantī}}}
%     \pada{\app{\lem[wit={V15}]{matsyākṣī}
%	 \rdg[wit={V3}]{matsāṣī}} ca punarnavā}/\\+}
%   \tl{
%     \pada{\app{\lem[wit={V15}]{meghanādaś ca}
%	 \rdg[wit={V3}]{meghanādīti}}} 
%	 pañcaite
%     \pada{\app{\lem[wit={V3}]{śākanāma prakīrtitā} 
%	 \rdg[wit={V15ac}]{pañcaśākaḥ prakīrtitaḥ}
%	 \rdg[wit={V15pc}]{pañcaśākāḥ prakīrtitāḥ}
%	 }\,\texteng{\teimute{\small}(sic)}//\versenr}
%%	 \myfn{This verse is also quoted in the \emph{Jyotsnā}, preceded by the phrase \textit{taduktaṃ vaidyake}.} %
%	 \sgwit{V15,V3}
%	 \\!}% Not in N9,E4
% \end{alttlg}
%
% \begin{alttlg}[hp01_062_2]
% \tl{\pada{sarvaśākam acākṣuṣyaṃ}
% \pada{\app{\lem[resp=emend]{cākṣuṣyaṃ}\rdg[wit={N23}]{cakṣapyaṃ}}
% śākapañcakaṃ}/\\+}
% \tl{
% \pada{jīvantī vāstu matsyākṣī}
% \pada{meghanādaḥ punarnavā}//
%  \sgwit{N23}\\!}%
% \end{alttlg}
%
% \endaltrecension
%

%1.63
\begin{tlg}[hp01_063]
\tl{
\pada{\app{\lem[wit={N3,G4,G11,V15,V1}]{mṛṣṭaṃ}% +M1,M3
	\rdg[wit={J5,N23,P15,GrB}]{miṣṭaṃ}% +N19
	\rdg[wit={V19,J10}]{iṣṭaṃ}% iṣṭa? V19 
	\rdg[wit={Jyo}]{puṣṭaṃ}
	\rdg[wit={E2}]{uṣṇaṃ}% +K3,C7
	}\marmas
\app{\lem[wit={ceteri}]{sumadhuraṃ} % °ra P11
	\rdg[wit={N3}]{sumadhu}
	\rdg[wit={J5}]{samudhuraṃ}
	\rdg[wit={V19,V15,V3}]{samadhuraṃ}} % +C1
	snigdhaṃ gavyaṃ % gabaṃ P11
dhātu\app{\lem[wit={ceteri}]{prapoṣaṇam}
	\rdg[wit={V15}]{prapoṣakaṃ}}/}\\+}
\tl{
\pada{mano\app{\lem[wit={ceteri}]{'bhilaṣitaṃ}
%	\rdg[wit={N23,P11}]{bhilakhitaṃ}
	\rdg[wit={N3}]{bhiliṣitaṃ}
	\rdg[wit={J5}]{bhivāṃchitaṃ}
	\rdg[wit={G4}]{bhilakṣitaṃ}
	\rdg[wit={V1}]{bhilāṣitaṃ}
	}
\app{\lem[wit={ceteri}]{yogyaṃ}% +G4
	\rdg[wit={V3}]{yonyaṃ}
	\rdg[wit={P15,V1}]{bhojyaṃ}
	\rdg[wit={C6}]{divyaṃ}} yogī % yogi N3, yomī G11
\app{\lem[wit={ceteri},alt={bhojanam}]{bhojana\skp{m}}
	\rdg[wit={J10}]{bhojanasam}}m ācaret//\versenr}\\!}
\end{tlg}

\commcite%\newpage


%1.64
\begin{tlg}[hp01_064]
\tl{
\pada{yuvā 
\app{\lem[wit={ceteri}]{vṛddho'tivṛddho} % vṛdho N3, vṛddhā? N23,V3
%	\rdg[wit={K3,C7}]{vṛddho pi vṛddho}
	\rdg[wit={E2}]{vṛddho py avṛddho}
	\rdg[wit={G4,G11}]{bhavatu vṛddo}
	} vā}
\pada{vyādhito
\app{\lem[wit={ceteri}]{durbalo'pi vā} % durbalā? N23
	\rdg[wit={J10,C6}]{durbalas tathā}}/}\\+}
\tl{
\pada{abhyāsāt siddhim āpnoti}
\pada{\app{\lem[wit={ceteri}]{sarvayogeṣv atandritaḥ}
	\rdg[wit={V3}]{sarvayogeṣu taṃdritaḥ}
	\rdg[wit={J5}]{sarvayogeṣu taṃtritā}
	\rdg[wit={V1}]{sarvaṃ yogī yateṃdriyaḥ}}//\versenr}\\!}
\end{tlg}

\commcite\newpage


%1.65
\begin{tlg}[hp01_065]
\tl{
\pada{\app{\lem[wit={ceteri}]{pīṭhāni}
	\rdg[wit={V19,E2}]{pīṭhādi}}
\app{\lem[wit={ceteri},alt={kumbhakāś}]{kumbhakā\skp{ś}} % °kāḥ C7
	\rdg[wit={J5,P11}]{kuṃbhikāś}
	\rdg[wit={V15ac,J10,V3}]{kumbhakaś}}%
\app{\lem[wit={ceteri},alt={citrā}]{\skm{ś }citrā}
	\rdg[wit={G11}]{citro}
	\rdg[wit={J10}]{citraṃ}
	\rdg[wit={P11}]{citra}
	}}
\pada{\app{\lem[wit={ceteri}]{divyāni}
	\rdg[wit={V1,J10}]{mudrādi}}
	karaṇāni ca/}\\+}
\tl{
\pada{\app{\lem[wit={N3,G11,P15,V15,V1,C6}]{sarvo'pi ca}% +F
	\rdg[wit={J5}]{sarvā pi ca}
	\rdg[wit={P11}]{sarve pi ca}
	\rdg[wit={G4}]{sarvo pi}
	\rdg[wit={V19,E2}]{sarvo pi hi}
	\rdg[wit={N23}]{sarve py ayaṃ}
	\rdg[wit={J10,V3,Jyo}]{sarvāṇy api}}
haṭhā\app{\lem[wit={Gr1,N23,G11,V15,C6,V3}]{bhyāso}
	\rdg[wit={V1,J10,Jyo}]{bhyāse}
	\rdg[wit={V19,E2,P11}]{bhyāsād}% °bhyāsā F
	\rdg[wit={P15}]{bhyā}}}
\pada{rājayoga% rājya P15
\app{\lem[wit={N3,J5,P15,V15,V1pc,J10,P11,Jyo}]{phalāvadhi}
	\rdg[wit={G4,N23,C6,V3}]{phalāvadhiḥ}
	\rdg[wit={G11}]{pathāvadhiḥ}
	\rdg[wit={V1ac}]{yugāvadhi}
	\rdg[wit={V19,E2}]{prasiddhaye}}//\versenr}\\!}
% \myfn{%
	% \getsiglum{V17} inserts here:
	% \devnote{svastheṣu cittasamitāsanabandhayukte
	% prāṇaṃ prapūratritayaṃ ghaṭavat prayukte/
	% savyāpasavyakramavṛddhi yathoktam eva
	% mātrā ca dvādaśa punar daśadvādaśābde//}}
\end{tlg}

\commcite%\newpage

\begin{col}[hp01_col]
iti \app{\lem[wit={J5,G4,GrB}]{śrīsvātmārāma}
	\rdg[wit={N3,N23,P15,V15,V1}]{svātmārāma}
	\rdg[wit={J10}]{ātmārāma}
	\rdg[wit={Jyo}]{śrīsahajānadasaṃtānaciṃtāmaṇisvātmārāma}
	\rdg[wit={V19,E2,G11},alt={\om}]{\skp{\om}}}%
\app{\lem[wit={G4,N23,P15,V15,V1,J10,P11,C6,Jyo}]{yogīndra}
	\rdg[wit={V3}]{yogendra}
	\rdg[wit={N3}]{mahāyogeṃdra}
	\rdg[wit={J5,V19,E2,G11},alt={\om}]{\skp{\om}}}%
\app{\lem[wit={ceteri}]{viracitāyāṃ}
	\rdg[wit={V19,E2,G11},alt={\om}]{\skp{\om}}}
	haṭhapradīpikāyāṃ
\app{\lem[alt={\ante prathamo° \add},nosep]{\skp{\ante prathamo° \add}}
	\rdg[wit={E2,V15}]{āsanayogo nāma}
	\rdg[wit={Jyo}]{āsanavidhikathanaṃ nāma}}%
\app{\lem[wit={ceteri}]{prathamopadeśaḥ}% V3 om. ḥ
	\rdg[wit={V15,J10,C6}]{prathama upadeśaḥ}
	\rdg[wit={V1}]{prathamo'dhyāyaḥ}}// 1//
\end{col}
% MD: Colophon der neuen Edition des Jyo noch nicht überprüft!

\colcite

\end{ekdosis}\end{otherlanguage}
\end{document}

%\hrule
%\vfill
%\bigskip
\small
\newpage
\begin{tabular}{lll}
%\hspace{4em}\=\hspace{10cm}\=\kill
%\\
%\hline
\multicolumn{3}{l}{\textbf{List of Sigla}} \\
\\
\getsiglum{N3} & N3 & 2 folios missing in Ch. 1 (1.18--28, 38--46)\\
\getsiglum{J5} & J5 \\
\getsiglum{G4} & G4 & damaged; collated only when available\\
\getsiglum{N23} & N23 \\
%\getsiglum{J7} & J7 \\
\getsiglum{V19} & V19 \\
%\getsiglum{K3} & K3 & the first folios missing (1.1--33)\\
%\getsiglum{C7} & C7 & 2 folios missing in Ch. 1 (1.23c--29d, 41c--47b)\\
\getsiglum{E2} & E2 \\
\getsiglum{G11} & G11 \\
\getsiglum{P15} & P15 \\
\getsiglum{N19} & N19 & consulted sporadically\\% 1.20 instead of N19
\getsiglum{V15} & V15 & influenced by Gr3\\
\getsiglum{V1} & V1 \\
\getsiglum{J10} & J10 \\
\getsiglum{N26} & N26 & collated only for 1.62*1\\
\getsiglum{P11} & P11 \\
\getsiglum{C6} & C6 \\
\getsiglum{V3} & V3 \\
\getsiglum{Jyo} & Jyo &  Brahmānanda's version, based on the new edition by Jürgen Hanneder \\
\end{tabular}
\vfill
\end{document}
