\documentclass[10pt]{memoir}
\setstocksize{220mm}{155mm} 	        
\settrimmedsize{220mm}{155mm}{*}	
\settypeblocksize{170mm}{116mm}{*}	
\setlrmargins{18mm}{*}{*}
\setulmargins{*}{*}{1.2}
% \setlength{\headheight}{5pt}
\checkandfixthelayout[lines]
\linespread{1}
\setlength{\parskip}{0.3em}
\setlength\parindent{0pt}

\makepagestyle{HPed}
\makeoddhead{HPed}{\small{HP Transl. \& Comm.}}{}{\small{\today}}
\makeevenhead{HPed}{\small{HP Transl. \& Comm.}}{}{\small{\today}}
\makeoddfoot{HPed}{}{\small{\thepage}}{}
\makeevenfoot{HPed}{}{\small{\thepage}}{}

\usepackage[teiexport=tidy,poetry=verse]{ekdosis}
\usepackage{sanskrit-poetry,libertine,xcolor}
\usepackage[english]{babel}
\setlength{\vindent}{0pt}
\setvnum{}




%%%%%%%%%%%%%%%%%%%% THE  MSS         %%%%%%%%%%%%%%%%%%%%%%%%%%%

%%% Versions
\DeclareWitness{Vu}{\selectlanguage{english}Vulg}{Vulgate, i.e. Brahmānanda's version}[]           
\DeclareWitness{X}{\selectlanguage{english}X}{TenChapter Version, Jodhpur 02228 and 02225 (ed. Lonavla)}[]
\DeclareWitness{Six}{\selectlanguage{english}Ṣ}{SixChapterVersion, ``6ChapterHPms'', fragment of enlarged text, Jodhpur}[]
% Mss. in Geographical Groups
%%%% Varanasi mss (Sampūrṇānanda mss). V1 is Important
\DeclareWitness{V1}{\selectlanguage{english}V\textsubscript{1}}{Sampurnananda Library Sarasvati Bhavan 30109}[]
        \DeclareHand{V1ac}{V1}{\selectlanguage{english}V\rlap{\textsubscript{1}}\textsuperscript{ac}}[] % added by MD
        \DeclareHand{V1pc}{V1}{\selectlanguage{english}V\rlap{\textsubscript{1}}\textsuperscript{pc}}[] % added by MD
\DeclareWitness{V2}{\selectlanguage{english}V\textsubscript{2}}{Sampurnananda Library Sarasvati Bhavan 29869}[]
\DeclareWitness{V3}{\selectlanguage{english}V\textsubscript{3}}{Sampurnananda Library Sarasvati Bhavan 29899}[]
\DeclareWitness{V4}{\selectlanguage{english}V\textsubscript{4}}{Sampurnananda Library Sarasvati Bhavan 29937}[]
\DeclareWitness{V5}{\selectlanguage{english}V\textsubscript{5}}{Sampurnananda Library Sarasvati Bhavan 29938}[]
\DeclareWitness{V6}{\selectlanguage{english}V\textsubscript{6}}{Sampurnananda Library Sarasvati Bhavan 29991}[]
\DeclareWitness{V8}{\selectlanguage{english}V\textsubscript{8}}{Sampurnananda Library Sarasvati Bhavan 30014}[]
\DeclareWitness{V11}{\selectlanguage{english}V\textsubscript{11}}{Sampurnananda Library Sarasvati Bhavan 30029}[]
\DeclareWitness{V12}{\selectlanguage{english}V\textsubscript{12}}{Sampurnananda Library Sarasvati Bhavan 30030}[]
\DeclareWitness{V13}{\selectlanguage{english}V\textsubscript{13}}{Sampurnananda Library Sarasvati Bhavan 30031}[]
\DeclareWitness{V14}{\selectlanguage{english}V\textsubscript{14}}{Sampurnananda Library Sarasvati Bhavan 30050}[]
\DeclareWitness{V15}{\selectlanguage{english}V\textsubscript{15}}{Sampurnananda Library Sarasvati Bhavan 30051}[]
\DeclareWitness{V15pc}{\selectlanguage{english}V\rlap{\textsubscript{15}}\textsuperscript{pc}\space}{}[]
\DeclareWitness{V16}{\selectlanguage{english}V\textsubscript{16}}{Sampurnananda Library Sarasvati Bhavan 30052}[]
\DeclareWitness{V17}{\selectlanguage{english}V\textsubscript{17}}{Sampurnananda Library Sarasvati Bhavan 30053}[] % added by MD
\DeclareWitness{V16pc}{\selectlanguage{english}V\rlap{\textsubscript{16}}\textsuperscript{pc}\space}{}[]
\DeclareWitness{V18}{\selectlanguage{english}V\textsubscript{18}}{Sampurnananda Library Sarasvati Bhavan 30064}[]
\DeclareWitness{V19}{\selectlanguage{english}V\textsubscript{19}}{Sampurnananda Library Sarasvati Bhavan 30069}[]
\DeclareWitness{V21}{\selectlanguage{english}V\textsubscript{21}}{Sampurnananda Library Sarasvati Bhavan 30104}[]
\DeclareWitness{V22}{\selectlanguage{english}V\textsubscript{22}}{Sampurnananda Library Sarasvati Bhavan 30110}[]
\DeclareWitness{V25}{\selectlanguage{english}V\textsubscript{25}}{Sampurnananda Library Sarasvati Bhavan 30122}[]
\DeclareWitness{V26}{\selectlanguage{english}V\textsubscript{26}}{Sampurnananda Library Sarasvati Bhavan 30123}[]
\DeclareWitness{V28}{\selectlanguage{english}V\textsubscript{28}}{Sampurnananda Library Sarasvati Bhavan 30136}[]
\DeclareWitness{W2}{\selectlanguage{english}W\textsubscript{2}}{Wai ??}[]
\DeclareWitness{W4}{\selectlanguage{english}W\textsubscript{4}}{Wai 399-6171}[]

%%%%%%%%%%%%%%%%%%%%%%%%%%%%%%%%%
%%% Jammu & Kaschmir
\DeclareWitness{K1}{\selectlanguage{english}K\textsubscript{1}}{Raghunātha Temple Library 4383}[settlement=Jammu]
        \DeclareWitness{K1ac}{\selectlanguage{english}K\rlap{\textsubscript{1}}\textsuperscript{ac}\space}{}[]
        \DeclareWitness{K1pc}{\selectlanguage{english}K\rlap{\textsubscript{1}}\textsuperscript{pc}\space}{}[]
\DeclareWitness{K3}{\selectlanguage{english}K\textsubscript{3}}{Privat collection}
\DeclareWitness{L1}{\selectlanguage{english}L\textsubscript{1}}{SOAS RE 43454}[settlement=Jammu]
% More details? Catalogue number? L1 And C1 very close (and come from same region)
%%%%%%%%%%%%%%%%%%%%%%%%%%%%%%%%
% Jodhpur
% J10 is important
\DeclareWitness{J10}{\selectlanguage{english}J\textsubscript{10}}{MSPP Jodhpur 2230}[]
        \DeclareHand{J10ac}{J10}{\selectlanguage{english}J\rlap{\textsubscript{10}}\textsuperscript{ac}}[] % modified by MD
        \DeclareHand{J10pc}{J10}{\selectlanguage{english}J\rlap{\textsubscript{10}}\textsuperscript{pc}}[] % modified by MD
\DeclareWitness{J1}{\selectlanguage{english}J\textsubscript{1}}{Jodhpur 02231}[]
\DeclareWitness{J2}{\selectlanguage{english}J\textsubscript{2}}{Jodhpur 02232}[]   
\DeclareWitness{J3}{\selectlanguage{english}J\textsubscript{3}}{Jodhpur 02233}[]
\DeclareWitness{J4}{\selectlanguage{english}J\textsubscript{4}}{Jodhpur 02234}[]
        \DeclareWitness{J4ac}{\selectlanguage{english}J\rlap{\textsubscript{4}}\textsuperscript{ac}\space}{MSPP Jodhpur 02234}[]
        \DeclareWitness{J4pc}{\selectlanguage{english}J\rlap{\textsubscript{4}}\textsuperscript{pc}\space}{MSPP Jodhpur 02234}[]
\DeclareWitness{J5}{\selectlanguage{english}J\textsubscript{5}}{Jodhpur 02235}[]  % 4 chapters, 34 jpgs,   long colophon, missing lines in the beginning.
\DeclareWitness{J6}{\selectlanguage{english}J\textsubscript{6}}{Jodhpur 02237}[]  % 4 chapters, 41 jpgs
%\DeclareWitness{J6ac}{\selectlanguage{english}J\rlap{\textsubscript{6}}\textsubscript{ac}}{Jodhpur 02237}[]  % 4 chapters, 49 jpgs,   1st folio: idaṃ gulābarāyasya
% tulasīrāmaśarmmaṇaḥ putrasya pustakaṃ ...        End: iti śrīsahajānandasantānacintāmaṇisvātmārāmaviracitāyāṃ ..
% saṃvat 1802   (more consistent text)
%\DeclareWitness{J6pc}{\selectlanguage{english}J\rlap{\textsubscript{6}}\textsubscript{pc}}{Jodhpur 02237}[] 
\DeclareWitness{J7}{\selectlanguage{english}J\textsubscript{7}}{Jodhpur 02241}[]  % 4 chapters, 41 jpgs
\DeclareWitness{J8}{\selectlanguage{english}J\textsubscript{8}}{Jodhpur 23709}[]  % 4 chapters,  87 jpgs.   saṃvat 1724
\DeclareHand{J8ac}{J8}{\selectlanguage{english}J\rlap{\textsubscript{8}}\textsuperscript{ac}}[]  % changed by MD
\DeclareHand{J8pc}{J8}{\selectlanguage{english}J\rlap{\textsubscript{8}}\textsuperscript{pc}}[]  % changed by MD
\DeclareWitness{J9}{\selectlanguage{english}J\textsubscript{9}}{Jodhpur 02224}[]  %  fragment, 20 jpgs.
\DeclareWitness{J11}{\selectlanguage{english}J\textsubscript{11}}{Jodhpur 23532}[]
        \DeclareHand{J11ac}{J11}{\selectlanguage{english}J\rlap{\textsubscript{11}}\textsuperscript{ac}}[] % added by MD
        \DeclareHand{J11pc}{J11}{\selectlanguage{english}J\rlap{\textsubscript{11}}\textsuperscript{pc}}[] % added by MD
\DeclareWitness{J12}{\selectlanguage{english}J\textsubscript{12}}{Jodhpur 18552}[] 
\DeclareWitness{J13}{\selectlanguage{english}J\textsubscript{13}}{Jodhpur 02229}[]  %  5 chapters, 93 jpgs.
\DeclareWitness{J14}{\selectlanguage{english}J\textsubscript{14}}{Jodhpur 02239}[]  %  4 chapters
\DeclareWitness{J15}{\selectlanguage{english}J\textsubscript{15}}{Jodhpur 9732A}[]
\DeclareWitness{J16}{\selectlanguage{english}J\textsubscript{16}}{Jodhpur 9732B}[]
\DeclareWitness{J17}{\selectlanguage{english}J\textsubscript{17}}{Jodhpur 3013}[]
% Haṭhapradīpikā with (non-Sanskrit) Bhāṣya RORI Jodhpur ACC.NO.18552
%  Haṭhapradīpikā with (non-Sanskrit) commentary, RORI Alwar 952, 4 chapters,  colophon of the comm:
% iti śrīlāhorīmiśravrajabhūṣanaviracitāyāṃ bhāvārthadīpikāyāṃ caturthodhyāya ..    
%  Haṭhapradīpikā (5 chapter) MSPP Jodhpur ACC.NO.02229/

%%%%%%%%%%        Bodleian, Oxford
\DeclareWitness{B1}{\selectlanguage{english}B\textsubscript{1}}{Bodleian Library No. d.457(8)}[settlement=Oxford]
\DeclareWitness{B2}{\selectlanguage{english}B\textsubscript{2}}{Bodleian Library No. d.458(1)}[settlement=Oxford]
\DeclareWitness{B3}{\selectlanguage{english}B\textsubscript{3}}{Bodleian Library No. d.458(9)}[settlement=Oxford]

%%%%%%%%%%%   Chandigarh
\DeclareWitness{C1}{\selectlanguage{english}C\textsubscript{1}}{Lalchand M-2080}[]%L1 And C1 very close (and come from same region)
\DeclareWitness{C2}{\selectlanguage{english}C\textsubscript{2}}{Lalchand M-6065}[]
\DeclareWitness{C3}{\selectlanguage{english}C\textsubscript{3}}{Lalchand M-1293}[]
\DeclareWitness{C4}{\selectlanguage{english}C\textsubscript{4}}{Lalchand M-2081}[]
\DeclareWitness{C4ac}{\selectlanguage{english}C\rlap{\textsubscript{4}}\textsuperscript{ac}\space}{}[]
\DeclareWitness{C4pc}{\selectlanguage{english}C\rlap{\textsubscript{4}}\textsuperscript{pc}\space}{}[]
\DeclareWitness{C5}{\selectlanguage{english}C\textsubscript{5}}{Lalchand M-2082}[]%doesn't have chapter 1
\DeclareWitness{C6}{\selectlanguage{english}C\textsubscript{6}}{Lalchand M-2089}[]
\DeclareWitness{C7}{\selectlanguage{english}C\textsubscript{7}}{Lalchand M-6494}[]
\DeclareWitness{C8}{\selectlanguage{english}C\textsubscript{8}}{Lalchand M-2091}[]
        \DeclareHand{C8ac}{C8}{\selectlanguage{english}C\rlap{\textsubscript{8}}\textsuperscript{ac}}[]
        \DeclareHand{C8pc}{C8}{\selectlanguage{english}C\rlap{\textsubscript{8}}\textsuperscript{pc}}[]
\DeclareWitness{C9}{\selectlanguage{english}C\textsubscript{9}}{Lalchand M-4530}[]


% %%%%%%%%%%        Nepalese
\DeclareWitness{N1}{\selectlanguage{english}N\textsubscript{1}}{NGMPP A1400-2}[]
\DeclareWitness{N2}{\selectlanguage{english}N\textsubscript{2}}{NGMPP B 39-19}[]
\DeclareWitness{N3}{\selectlanguage{english}N\textsubscript{3}}{NGMPP B 62-20}[]
\DeclareWitness{N5}{\selectlanguage{english}N\textsubscript{5}}{NGMPP A60-15 + A61-1}[]
\DeclareWitness{N4}{\selectlanguage{english}N\textsubscript{4}}{NGMPP A61-2}[]
\DeclareWitness{N6}{\selectlanguage{english}N\textsubscript{6}}{NGMPP A61-6}[]
\DeclareWitness{N9}{\selectlanguage{english}N\textsubscript{9}}{NGMPP A62-33}[]
\DeclareWitness{N10}{\selectlanguage{english}N\textsubscript{10}}{NGMPP A62-37}[]
\DeclareWitness{N11}{\selectlanguage{english}N\textsubscript{11}}{NGMPP A63-15}[]
\DeclareWitness{N12}{\selectlanguage{english}N\textsubscript{12}}{NGMPP A939-19}[]
\DeclareWitness{N13}{\selectlanguage{english}N\textsubscript{13}}{NGMPP A1378-18}[]
\DeclareWitness{N16}{\selectlanguage{english}N\textsubscript{16}}{NGMPP B39-20}[]
\DeclareWitness{N17}{\selectlanguage{english}N\textsubscript{17}}{NGMPP B 111-10}[]
\DeclareWitness{N18}{\selectlanguage{english}N\textsubscript{18}}{NGMPP E 929-3}[]
\DeclareWitness{N19}{\selectlanguage{english}N\textsubscript{19}}{NGMPP E-1528-1 / E-1527-7(4)}[]
\DeclareWitness{N20}{\selectlanguage{english}N\textsubscript{20}}{NGMPP E 2037-13 }[]
\DeclareWitness{N21}{\selectlanguage{english}N\textsubscript{21}}{NGMPP E 2097-31}[]
\DeclareWitness{N22}{\selectlanguage{english}N\textsubscript{22}}{NGMPP G 4-4}[]
\DeclareWitness{N23}{\selectlanguage{english}N\textsubscript{23}}{NGMPP G 25-2}[]
        \DeclareHand{N23ac}{N23}{\selectlanguage{english}N\rlap{\textsubscript{23}}\textsuperscript{ac}}[] % added by MD
        \DeclareHand{N23pc}{N23}{\selectlanguage{english}N\rlap{\textsubscript{23}}\textsuperscript{pc}}[] % added by MD
\DeclareWitness{N24}{\selectlanguage{english}N\textsubscript{24}}{NGMPP G 190-16}[]
\DeclareWitness{N24ac}{\selectlanguage{english}N\rlap{\textsubscript{24}}\textsuperscript{ac}\space}{}[]
\DeclareWitness{N24pc}{\selectlanguage{english}N\rlap{\textsubscript{24}}\textsuperscript{pc}\space}{}[]
\DeclareWitness{N26}{\selectlanguage{english}N\textsubscript{26}}{NGMPP T 24-3}[]

% %%%%%%%%%%        Pune

\DeclareWitness{P1}{\selectlanguage{english}P\textsubscript{1}}{Ānandāśrama S16-3-21}[]
\DeclareWitness{P2}{\selectlanguage{english}P\textsubscript{2}}{Ānandāśrama S16-2-20}[]
\DeclareWitness{P3}{\selectlanguage{english}P\textsubscript{3}}{BISM (79) 314}[]
\DeclareWitness{P4}{\selectlanguage{english}P\textsubscript{4}}{BISM (91) 191}[]
\DeclareWitness{P5}{\selectlanguage{english}P\textsubscript{5}}{BISM (29) 5790}[]
\DeclareWitness{P6}{\selectlanguage{english}P\textsubscript{6}}{BORI 263/1879-80}[]
\DeclareWitness{P7}{\selectlanguage{english}P\textsubscript{7}}{BORI 665/1883-84}[]
\DeclareWitness{P8}{\selectlanguage{english}P\textsubscript{8}}{BORI 316/1895-98}[]
\DeclareWitness{P9}{\selectlanguage{english}P\textsubscript{9}}{BORI 733-1891-95}[]
\DeclareWitness{P10}{\selectlanguage{english}P\textsubscript{10}}{BORI 222-1884-86}[]
\DeclareWitness{P11}{\selectlanguage{english}P\textsubscript{11}}{BORI 221-1882–83}[]
\DeclareWitness{P12}{\selectlanguage{english}P\textsubscript{12}}{Ānandāśrama S16-3-24}[]
\DeclareWitness{P13}{\selectlanguage{english}P\textsubscript{13}}{Ānandāśrama S16-2-22}[]
\DeclareWitness{P14}{\selectlanguage{english}P\textsubscript{14}}{Ānandāśrama S16-3-23}[]
\DeclareWitness{P15}{\selectlanguage{english}P\textsubscript{15}}{BISM (64) 919}[]
\DeclareWitness{P16}{\selectlanguage{english}P\textsubscript{16}}{BISM (64) 1115}[]
\DeclareWitness{P17}{\selectlanguage{english}P\textsubscript{17}}{BISM 620/1886-92}[]
\DeclareWitness{P18}{\selectlanguage{english}P\textsubscript{18}}{BORI 615/1887-91}[]
\DeclareWitness{P19}{\selectlanguage{english}P\textsubscript{19}}{BISM 46-39}[]
\DeclareWitness{P20}{\selectlanguage{english}P\textsubscript{20}}{BISM 39-273}[]
\DeclareWitness{P21}{\selectlanguage{english}P\textsubscript{21}}{BISM 37-743}[]
\DeclareWitness{P22}{\selectlanguage{english}P\textsubscript{22}}{BISM 37-729}[]
\DeclareWitness{P23}{\selectlanguage{english}P\textsubscript{23}}{BISM 33-60}[]
\DeclareWitness{P24}{\selectlanguage{english}P\textsubscript{24}}{BISM 29-5790}[]% =P5!
\DeclareWitness{P25}{\selectlanguage{english}P\textsubscript{25}}{BISM 29-3657}[]
\DeclareWitness{P26}{\selectlanguage{english}P\textsubscript{26}}{BISM 25-281}[]
\DeclareWitness{P27}{\selectlanguage{english}P\textsubscript{27}}{BISM 7-489}[]
\DeclareWitness{P28}{\selectlanguage{english}P\textsubscript{28}}{BORI 399-1895-1902}[]

%%%%%   Mysore
\DeclareWitness{M1}{\selectlanguage{english}M\textsubscript{1}}{P-5682/4}[]
%%%%%   Tübingen
\DeclareWitness{Tue}{\selectlanguage{english}Tü}{Ma I 339}[]
%%%%%%%%%%
\DeclareWitness{YC}{\selectlanguage{english}YC}{Yogacintāmaṇi}[]
\DeclareWitness{ceteri}{\selectlanguage{english}cett.}{ceteri}[]

%%%%%%%%%% Mss with Commentary
\DeclareWitness{A1}{\selectlanguage{english}A\textsubscript{1}}{Alwar 952}[]

\DeclareWitness{Jyo}{\selectlanguage{english}J\textsubscript{yo}}{Brahmānanda's version}[]

%%%%%%%%%%%%%%%%%%%%%%%%%%%%%%%%%%%%%%%%%%%
%List of all Sigla:
%A1,B1,B2,B3,C1,C2,C3,C4,C6,C7,C8,C9,J1,J2,J3,J4,J10,J13,J14,J15,J17,L1,M1,N3,N5,N6,N9,N10,N11,N12,N13,N16,N17,N19,N20,N21,N22,N23,N24,Tü,V1,V2,V3,V4,V5,V6,V8,V11,V19,V22,V26,Vu
%%%%%%%%%%%%%%%%%%%%%%%%%%%%%%%%%%%%%%%%%%%

\DeclareWitness{G4}{\selectlanguage{english}G\textsubscript{4}}{GOML D18885 (Bundle SD5051)}[]
\DeclareWitness{G5}{\selectlanguage{english}G\textsubscript{5}}{GOML R3841/ SR2190}[]
\DeclareWitness{G7}{\selectlanguage{english}G\textsubscript{7}}{GOML D4394}[]

\DeclareWitness{Ko}{\selectlanguage{english}K\textsubscript{o}}{Koba, Gujarat 55626}[]

%
%%%%%                   Abbreviation for the printed apparatus,        xml interface needed
%%%%%                   (synonyms in same line)

% Macro for Editing Abbrevs.
%\def\om{\textrm{\footnotesize \textit{omitted in}\ }} %prints om. for omitted in apparatus
%\def\korr{\textrm{\footnotesize \textit{em.}\ }} %prints em. for emended in apparatus
%\def\conj{\textrm{\footnotesize \textit{conj.}\ }} %prints conj. for conjectured in apparatus


\def\eyeskip{\textrm{{ab.\,oc. }}}   
\def\aberratio{\textrm{{ab.\,oc. }}}
\def\ad{\textrm{{ad}}}   
\def\add{\textrm{{add.\ }}}
\def\ann{\textrm{{ann.\ }}}
\def\ante{\textrm{{ante }}}
\def\post{\textrm{{post }}}
%\def\ceteri{cett.\,}             % for simplifying the apparatus in print                  
\def\codd{\textrm{{codd.\ }}}   %  the same
\def\conj{\textrm{{coni.\ }}}  
\def\coni{\textrm{{coni.\ }}}
\def\contin{\textrm{{contin.\ }}}
\def\corr{\textrm{{corr.\ }}}
\def\del{\textrm{{del.\ }}}
\def\dub{\textrm{{ dub.\ }}}
\def\emend{\textrm{{emend.\ }}}
\def\expl{\textrm{{explic.\ }}}   
\def\explicat{\textrm{{explic.\ }}}
\def\fol{\textrm{{fol.\ }}}         
\def\foll{\textrm{{foll.\ }}}
\def\gloss{\textrm{{glossa ad }}}
\def\ins{\textrm{{ins.\ }}}          \def\inseruit{\textrm{{ins.\ }}}
\def\im{{\kern-.7pt\lower-1ex\hbox{\textrm{\tiny{\emph{i.m.}}}\kern0pt}}}
\def\inmargine{{\kern-.7pt\lower-.7ex\hbox{\textrm{\tiny{\emph{i.m.}}}\kern0pt}}}
\def\intextu{{\kern-.7pt\lower-.95ex\hbox{\textrm{\tiny{\emph{i.t.}}}\kern0pt}}}%\textrm{\scriptsize{i.t.\ }}}               
\def\indist{\textrm{{indis.\ }}}          \def\indis{\textrm{{indis.\ }}}
\def\iteravit{\textrm{{iter.\ }}}          \def\iter{\textrm{{iter.\ }}}  
\def\lectio{\textrm{{lect.\ }}}             \def\lec{\textrm{{lect.\ }}}
\def\leginequit{\textrm{{l.n. }}}         \def\legn{\textrm{{l.n. }}}         \def\illeg{\textrm{{l.n. }}}
\def\om{\textrm{{om. }}}
\def\primman{\textrm{{pr.m.}}}
\def\prob{\textrm{{prob.}}}
\def\rep{\textrm{{repetitio }}}
% \def\secundamanu{\textrm{\scriptsize{s.m.}}}
% \def\secm{{\kern-.6pt\lower-.91ex\hbox{\textrm{\tiny{\emph{s.m.}}}\kern0pt}}}%   \textrm{\scriptsize{s.m.}}}
\def\sequentia{\textrm{{seq.\,inv.\ }}}         \def\seqinv{\textrm{{seq.\,inv.\ }}} \def\order{\textrm{{seq.\,inv.\ }}}
\def\supralineam{{\kern-.7pt\lower-.91ex\hbox{\textrm{\tiny{\emph{s.l.}}}\kern0pt}}} %\textrm{\scriptsize{s.l.}}}
\def\interlineam{{\kern-.7pt\lower-.91ex\hbox{\textrm{\tiny{\emph{s.l.}}}\kern0pt}}}   %\textrm{\scriptsize{s.l.}}}
\def\vl{\textrm{v.l.}}   \def\varlec{\textrm{v.l.}} \def\varialectio{\textrm{v.l.}}
\def\vide{\textrm{{cf.\ }}}           \def\cf{\textrm{{cf.\ }}}
\def\videtur{\textrm{{vid.\,ut}}}
\def\crux{\textup{[\ldots]} }
\def\cruxx{\textup{[\ldots]}}
\def\unm{\textit{unm.}}        % unmetrical
%%%%%%%%%%%%%%%%%%%%%%%%%%%%%%%%%%%%



%%% Local Variables:
%%% mode: latex
%%% TeX-master: t
%%% End:

% addition 2023-12-11 MD:
\TeXtoTEIPat{\begin {metre}[#1]}{<note type="metre" target="##1">}
\TeXtoTEIPat{\end {metre}}{</note>}
\TeXtoTEIPat{\texttheta}{θ}

% change 2023-12-05 mm
\TeXtoTEI{teimute}{} 

% changes/additions 2023-11-27 MM:
\TeXtoTEIPat{\medialink {#1}{#2}}{<ref target="resources/#2">#1</ref>}

% changes/additions 2023-10-25 MM:
% new Sigla
\TeXtoTEIPat{\textAlpha}{Α}
\TeXtoTEIPat{\textalpha}{α}
\TeXtoTEIPat{\textBeta}{Β}
\TeXtoTEIPat{\textbeta}{β}
\TeXtoTEIPat{\textGamma}{Γ}
\TeXtoTEIPat{\textgamma}{γ}
\TeXtoTEIPat{\textDelta}{Δ}
\TeXtoTEIPat{\textdelta}{δ}
\TeXtoTEIPat{\textEpsilon}{Ε}
\TeXtoTEIPat{\textepsilon}{ε}
\TeXtoTEIPat{\textEta}{Η}
\TeXtoTEIPat{\texteta}{η}
\TeXtoTEIPat{\textChi}{Χ}
\TeXtoTEIPat{\textchi}{χ}
\TeXtoTEIPat{\textOmega}{Ω}
\TeXtoTEIPat{\textomega}{ω}

%new environments
\TeXtoTEIPat{\begin {postmula}[#1]}{<note type="postmula" target="##1">}
  \TeXtoTEIPat{\end {postmula}}{</note>}
\TeXtoTEIPat{\begin {altava}[#1]}{<div type="altrec"><note type="avataranika" target="##1">} %%% changed 2023-12-05 mm
  \TeXtoTEIPat{\end {altava}}{</note></div>} %%% changed 2023-12-05 mm
\TeXtoTEIPat{\sgwit {#1}}{<note type="inlineref"><ref>#1</ref></note>}

% changes/additions 2023-10-12 MM:
\TeXtoTEIPat{\\.}{}

% changes/additions 2023-08-15 MD:
\TeXtoTEIPat{\lineom {#1}{#2}}{<note type="omission">#1 omitted in <ref>#2</ref></note>}
\TeXtoTEI{graus}{hi}[rend="grey"]
\TeXtoTEIPat{\startgray}{} %%% changed 2023-12-05 mm
\TeXtoTEIPat{\endgray}{} %%% changed 2023-12-05 mm



% additions/changes 2023-06-05 mm:
%\TeXtoTEIPat{\lineom {#1}}{<note type="omission">Line omitted in <ref>#1</ref></note>}
\TeXtoTEIPat{\NotIn {#1}}{<note type="omission">Stanza omitted in <ref>#1</ref></note>}

% additions 2023-04-16 MD:
\TeXtoTEIPat{\,}{ }

% additions 2023-04-13 mm:
\TeXtoTEIPat{\begin {versinnote}}{<lg>}
  \TeXtoTEIPat{\end {versinnote}}{</lg>}

% additions 2023-04-05 MD:
\TeXtoTEIPat{\begin {testimonia}[#1]}{<note type="testimonia" target="##1">}
  \TeXtoTEIPat{\end {testimonia}}{</note>}
\TeXtoTEI{devnote}{s}[xml:lang="sa-deva"]

% app in philcomm und testimonia %%% added MM 2023-12-02
\TeXtoTEI{var}{note}[type="appinnote"]


\TeXtoTEI{anm}{note}[type="memo"] %% change 2023-04-16 MD
\TeXtoTEI{Anm}{note}[type="memo"] %% change 2023-12-05 MM
\TeXtoTEIPat{\startverse}{} %%% marked for change 2023-04-13 mm
\TeXtoTEIPat{\endverse}{} %%% marked for change 2023-04-13 mm
\TeXtoTEIPat{\newpage}{}
\TeXtoTEIPat{\marma}{}
\TeXtoTEIPat{\marmas}{}
\TeXtoTEIPat{\vin}{} % added by MD 2023-11-14

%%% modify environments and commands
%%% TEI mapping
% additions/changes 2022-06-07 mm:
\TeXtoTEI{grau}{hi}[rend="grey"]
\TeXtoTEIPat{ \& }{ &amp; }

% additions/changes 2022-06-01 mm:
\TeXtoTEI{skp}{seg}[type="deva-ignore"]
\TeXtoTEI{skm}{seg}[type="ltn-ignore"]

\TeXtoTEIPat{\rlap {#1}}{#1}

% additions/changes 2022-04-06 mm:
%\TeXtoTEI{sgwit}{ref}
\TeXtoTEI{textdev}{s}[xml:lang="sa-deva"]
\TeXtoTEIPat{\begin {col}[#1]}{<div type="colophon" xml:id="#1"><p>}
  \TeXtoTEIPat{\end {col}}{</p></div>}
\TeXtoTEIPat{\begin {ava}[#1]}{<note type="avataranika" target="##1">}
  \TeXtoTEIPat{\end {ava}}{</note>}
												   
\TeXtoTEIPat{\outdent}{}
\TeXtoTEIPat{\startaltrecension}{} %%% changed 2023-12-05 mm
\TeXtoTEIPat{\endaltrecension}{} %%% changed 2023-12-05 mm
\TeXtoTEIPat{\startaltnormal}{} % added by MD 2023-11-14 %%% changed 2023-12-05 mm
\TeXtoTEIPat{\endaltnormal}{} % added by MD 2023-11-14 %%% changed 2023-12-05 mm
\TeXtoTEIPat{\begin {alttlg}[#1]}{<div type="altrec"><lg xml:id="#1">}
  \TeXtoTEIPat{\end {alttlg}}{</lg></div>}



% additions/changes 2022-03-12 mm:
\TeXtoTEIPat{\begin {tlg}[#1]}{<lg xml:id="#1">}
  \TeXtoTEIPat{\end {tlg}}{</lg>}

\TeXtoTEIPat{\begin {translation}[#1]}{<note type="translation" target="##1">}
  \TeXtoTEIPat{\end {translation}}{</note>}
\TeXtoTEIPat{\begin {philcomm}[#1]}{<note type="philcomm" target="##1">}
  \TeXtoTEIPat{\end {philcomm}}{</note>}
\TeXtoTEIPat{\begin {sources}[#1]}{<note type="sources" target="##1">}
  \TeXtoTEIPat{\end {sources}}{</note>}


\TeXtoTEIPat{\begin {marma}[#1]}{<note type="marma" target="##1">}
  \TeXtoTEIPat{\end {marma}}{</note>}

\TeXtoTEIPat{\begin {jyotsna}[#1]}{<note type="jyotsna" target="##1">}
  \TeXtoTEIPat{\end {jyotsna}}{</note>}

\EnvtoTEI{description}{list}
\EnvtoTEI{itemize}{list}
\TeXtoTEIPat{\item [#1]}{<label>#1</label>\item}

\TeXtoTEI{tl}{l}
\TeXtoTEI{myfn}{note}[type="myfn"]
\TeXtoTEIPat{\getsiglum {#1}}{<ref target="##1"/>}

\TeXtoTEI{SetLineation}{}
\TeXtoTEI{noindent}{}
\TeXtoTEI{subsection*}{}

\TeXtoTEI{rlap}{}

% end additions/changes
% \TeXtoTEIPat{\skp {#1}}{#1}
% \TeXtoTEIPat{\skm {#1}}{}

\TeXtoTEIPat{\begin {prose}}{<p>}
  \TeXtoTEIPat{\end {prose}}{</p>}

\TeXtoTEIPat{\begin {tlate}}{<p>}
  \TeXtoTEIPat{\end {tlate}}{</p>}

\TeXtoTEI{emph}{hi}
\TeXtoTEI{bigskip}{}
% \TeXtoTEI{/}{|}
\TeXtoTEI{tl}{l}
\TeXtoTEIPat{english}{}
%\TeXtoTEIPat{-}{ } %% change 2023-04-16 MD
%\TeXtoTEIPat{°}{} %% change 2023-04-16 MD
\TeXtoTEIPat{\textcolor {#1}{#2}}{<hi rend="#1">#2</hi>}

% \TeXtoTEIPat{\eyeskip}{}
% \TeXtoTEIPat{\aberratio}{}
% \TeXtoTEIPat{\ad}{}
\TeXtoTEIPat{\add}{<hi rend="italic">add.</hi>} %% change 2023-04-16 MD
% \TeXtoTEIPat{\ann}{}
\TeXtoTEIPat{\ante}{<hi rend="italic">ante</hi> } %% change 2023-04-16 MD
\TeXtoTEIPat{\post}{<hi rend="italic">post</hi> } %% change 2023-04-16 MD
% \TeXtoTEIPat{\codd}{}
% \TeXtoTEIPat{\conj }{}
% \TeXtoTEIPat{\contin}{}
% \TeXtoTEIPat{\corr}{}
% \TeXtoTEIPat{\del}{}
% \TeXtoTEIPat{\dub}{}
% \TeXtoTEIPat{\emend }{}
% \TeXtoTEIPat{\expl}{}
% \TeXtoTEIPat{\ȩxplicat}{}
% \TeXtoTEIPat{\fol}{}
% \TeXtoTEIPat{\gloss}{}
% \TeXtoTEIPat{\ins}{}
% \TeXtoTEIPat{\im}{}
% \TeXtoTEIPat{\inmargine}{}
% \TeXtoTEIPat{\intextu}{}
% \TeXtoTEIPat{\indist}{}
% \TeXtoTEIPat{\iteravit}{}
% \TeXtoTEIPat{\lectio}{}
% \TeXtoTEIPat{\leginequit}{}
% \TeXtoTEIPat{\legn}{}
% \TeXtoTEIPat{\illeg}{<hi rend="italic">illeg.</hi>}
\TeXtoTEIPat{\illeg}{<gap reason="illeg."/>} %%% change 2023-04-11 mm
% \TeXtoTEIPat{\om}{<hi rend="italic">om.</hi>}
\TeXtoTEIPat{\om}{<gap reason="om."/>} %%% change 2023-04-11 mm
% \TeXtoTEIPat{\primman}{}
% \TeXtoTEIPat{\prob}{}
% \TeXtoTEIPat{\rep}{}
% \TeXtoTEIPat{\sequentia}{}
% \TeXtoTEIPat{\supralineam}{}
% \TeXtoTEIPat{\interlineam}{}
\TeXtoTEIPat{\vl}{<hi rend="italic">v.l.</hi>}
% \TeXtoTEIPat{\vide}{}
% \TeXtoTEIPat{\videtur}{}
% \TeXtoTEIPat{\crux}{}
% \TeXtoTEIPat{\cruxxx}{}
\TeXtoTEIPat{\unm}{<hi rend="italic">unm.</hi>}


% List of Scholars
\DeclareScholar{nos}{nos}[
forename=HPP,
surname=Team]


% Nullify \selectlanguage in TEI as it has been used in
% \DeclareWitness but should be ignored in TEI.
\TeXtoTEI{selectlanguage}{}


\SetTEIxmlExport{autopar=false}

%%%%%%%%%%%

\SetTEIxmlExport{autopar=false}
\NewDocumentEnvironment{translation}{O{}}{\textcolor{blue}{\textbf{Translation:}}}{}
\NewDocumentEnvironment{philcomm}{O{}}{
	\textcolor{blue}{\textbf{Commentary:}}}{}
\NewDocumentEnvironment{metre}{O{}}{
	\textcolor{blue}{\textbf{Metre:}}}{} % added MD 2023-12-11
\NewDocumentEnvironment{sources}{O{}}{
	\textcolor{blue}{\textbf{Sources:}}\linebreak}{}
\NewDocumentEnvironment{testimonia}{O{}}{
	\textcolor{blue}{\textbf{Testimonia:}}\linebreak}{}
\NewDocumentEnvironment{versinnote}{O{}}{\begin{ekdverse}}{\end{ekdverse}}
%\newcommand{\var}[1]{\footnotesize\textup{#1}}
\newcommand{\medialink}[2]{\textcolor{green}{\underline{#1}}}
%\TeXtoTEIPat{\medialink {#1}{#2}}{<ref target="/images/#2">#1</ref>}

\NewDocumentCommand{\tl}{m}{#1}

\def\vl{\textit{v.l.}}
\def\var#1{{\footnotesize #1}}
\def\sl#1{\emph{#1}}

%%%%%%%%%%%%

\usepackage{textgreek}

\newcommand{\alphaOne}{\textalpha\textsubscript{1}}% N3
\newcommand{\alphaTwo}{\textalpha\textsubscript{2}}% J5
\newcommand{\alphaThree}{\textalpha\textsubscript{3}}% G4
\newcommand{\betaOne}{\textbeta\textsubscript{1}}% P11
\newcommand{\betaTwo}{\textbeta\textsubscript{2}}% C6
\newcommand{\betaOmega}{\textbeta\textsubscript{\textomega}}% V3
\newcommand{\gammaOne}{\textgamma\textsubscript{1}}% N23
\newcommand{\gammaTwo}{\textgamma\textsubscript{2}}% J7
\newcommand{\deltaOne}{\textdelta\textsubscript{1}}% V19
\newcommand{\deltaTwo}{\textdelta\textsubscript{2}}% K3
\newcommand{\deltaThree}{\textdelta\textsubscript{3}}% C7
\newcommand{\deltaOmega}{\textdelta\textsubscript{\textomega}}% J6
\newcommand{\epsilonOne}{\textepsilon\textsubscript{1}}% P15
\newcommand{\epsilonTwo}{\textepsilon\textsubscript{2}}% N19
\newcommand{\epsilonThree}{\textepsilon\textsubscript{3}}% V15
\newcommand{\epsilonFour}{\textepsilon\textsubscript{4}}% J11
\newcommand{\epsilonOmega}{\textepsilon\textsubscript{\textomega}}% N26
\newcommand{\etaOne}{\texteta\textsubscript{1}}% V1
\newcommand{\etaTwo}{\texteta\textsubscript{2}}% J10
\newcommand{\etaOmega}{\texteta\textsubscript{\textomega}}% N9

%%%%%%%%%%%%%%

\babelhyphenation{%
	Dattā-treya-yoga-śāstra
	Gorakṣa-śataka
	Haṭha-pra-dī-pikā
	Hātha-ratnā-valī
	Svātmā-rāma
	Śiva-saṃhitā
	Vasiṣṭha-saṃhitā
	Viveka-mārtaṇḍa
	Yukta-bhava-deva
	Yoga-cintā-maṇi
	Yoga-yājña-valkya}

\begin{document}
\pagestyle{HPed}
\begin{ekdosis}
\SetLineation{lineation = none,}

%\chapter*{Translation \& philological commentary}
%%%%%%%%%%
\subsection*{3.1}
%<*tr1>
\begin{translation}[hp03_001]
Just as the lord of snakes is the foundation of the regions of the earth along with their mountains and forests, so Kuṇḍalinī is the foundation of all systems of yoga.
\end{translation}
%</tr1>

%<*sc1>
%\begin{sources}[hp03_001]
%\end{sources}
%</sc1>

%<*ts1>
\begin{testimonia}[hp03_001]
\emph{Haṭharatnāvalī} 2.124, \emph{Yogacintāmaṇi} f.\,71v (\attr HP) 
\begin{variants}
dhātrīṇāṃ~] dhātryās tu HRĀ\sep 
sarveṣāṃ~] aśeṣa HRĀ    
\end{variants}


%\emph{Haṭharatnāvalī} 2.124
%\begin{versinnote}
%\tl{saśailavanadhātryās tu yathādhāro 'hināyakaḥ/\\+}
%\tl{aśeṣayogatantrāṇāṃ tathādhāro hi kuṇḍalī//\\!}
%\end{versinnote}

%\emph{Yogacintāmaṇi} f.\,71v
%\begin{versinnote}
%\tl{haṭhapradīpikāyām—\\+}
%\tl{saśailavanadhātrīṇāṃ yathādhāro 'hināyakaḥ/\\+}
%\tl{sarveṣāṃ yogatantrāṇāṃ tathādhāro hi kuṇḍalī//\\!}
%\end{versinnote}

%\emph{Yuktabhavadeva} 7.170
%\begin{versinnote}
%\tl{tatra haṭhapradīpikāyām--\\+}
%\tl{saśailavanadhātrīṇāṃ yathādhāro 'hināyakaḥ/\\+}
%\tl{sarveṣāṃ yogatantrāṇāṃ tathādhāro hi kuṇḍalī//\\!}
%\end{versinnote}
\end{testimonia}
%</ts1>

%<*cm1>
\begin{philcomm}[hp03_001]
The plural \emph{dhātrīṇām} is hard to construe. Only this world has mountains and forests, but we want a plural for the comparison with \emph{tantrāṇām}. Brahmānanda (\emph{Jyotsnā} 3.1) understands \emph{dhātrīṇām} to refer to the different regions of the earth, even though the world (\emph{dhātrī}) is a single entity (\emph{dhātryā ekatve 'pi deśabhedād bhedam ādāya bahuvacanam}). A similar comment occurs in the \emph{Yogaprakāśikā} 5.1 (\emph{yathā samastadvīpādisahitapṛthvī ādhāraḥ phaṇīndras tathā samastayogādhāraḥ kuṇḍalīty āha saśaileti}). The author of the \textit{Haṭharatnāvalī} circumvented this issue by adopting the reading \textit{dhātryāḥ}.
\end{philcomm}
%</cm1>

%%%%%%%%%%
\subsection*{3.2}
%<*tr2>
\begin{translation}[hp03_002]
When the sleeping Kuṇḍalinī awakens through the favour of the guru, then all the lotuses are pierced, and the knots too, [\dots] 
\end{translation}
%</tr2>

%<*sc2>
\begin{sources}[hp03_002]
\emph{Śivasaṃhitā} 4.21
\mylb
% \begin{versinnote}
% \tl{suptā guruprasādena yadā jāgarti kuṇḍalī/\\+}
% \tl{tadā sarvāṇi padmāni bhidyante granthayo 'pi ca//\\!}
% \end{versinnote}
\end{sources}
%</sc2>

%<*ts2>
\begin{testimonia}[hp03_002]
\emph{Yogacintāmaṇi} f.\,71v (\attr HP), \emph{Yuktabhavadeva} 171 (\attr HP) 
\begin{variants}
    yadā jāgarti kuṇḍalī YBhD~] bodhitā sukhadā bhavet YCM
\end{variants}

% \emph{Yogacintāmaṇi} f.\,71v (\attr to the \emph{Haṭhapradīpikā})
% \begin{versinnote}
% \tl{suptā guruprasādena bodhitā sukhadā bhavet/\\+}
% \tl{tathā sarvāṇi padmāni bhidyante granthayo 'pi ca//\\!}
% \end{versinnote}

% \emph{Yuktabhavadeva} 171 (\attr to the \emph{Haṭhapradīpikā})
% \begin{versinnote}
% \tl{suptā guruprasādena yadā jāgarti kuṇḍalī/\\+}
% \tl{tadā sarvāṇi padmāni bhidyante granthayo 'pi ca//\\!}
% \end{versinnote}
\end{testimonia}
%</ts2>

%<*cm2>
\begin{philcomm}[hp03_002]
The usual meaning of \emph{jāgarti} would be “is wakeful” rather than “awakens”, which explains the variant \emph{bodhitā}.
\end{philcomm}
%</cm2>

%%%%%%%%%%
\subsection*{3.3}
%<*tr3>
\begin{translation}[hp03_003]
{[}\dots] the empty pathway becomes the royal highway for \emph{prāṇa}, the mind becomes free of support, and death is cheated.%??JM: translate the tadā-s?
\end{translation}
%</tr3>

%<*sc3>
%\begin{sources}[hp03_003]
%\end{sources}
%</sc3>

%<*ts3>
\begin{testimonia}[hp03_003]
\emph{Yogacintāmaṇi} f.\,72r (\attr HP)
\begin{variants}
\textbf{b:} tadā~] tathā YCM\sep 
\textbf{c:} tadā~] tathā YCM\sep 
\textbf{d:} tadā~] tathā YCM   
\end{variants}

% \begin{versinnote}
% \tl{prāṇasya śūnyapadavī tathā rājapathāyate/\\+}
% \tl{tathā cittaṃ nirālambaṃ tathā kālasya vañcanam//\\!}
% \end{versinnote}
\end{testimonia}
%</ts3>

%<*cm3>
%\begin{philcomm}[hp03_003]
%JM: take prāṇasya with śūnyapadavī?
%JB: tadā in 3b indicates that this verse should be read with 3.2 (i.e., [when kuṇḍalinī has awakened...] then, the empty pathway becomes the royal pathway for prāṇa)
%\end{philcomm}
%</cm3>

\begin{metre}[hp03_003]
Anuṣṭubh (a: na-vipulā)
\end{metre}

%%%%%%%%%%
\subsection*{3.4 heading}
%<*tr4a>
\begin{translation}[hp03_004a]
What is ``the empty pathway"?
\end{translation}
%</tr4a>

%<*cm4a>
% \begin{philcomm}[hp03_004a]
% \end{philcomm}
%</cm4a>

%%%%%%%%%%
\subsection*{3.4}
%<*tr4>
\begin{translation}[hp03_004]
Suṣumṇā, the empty pathway, the great path to the aperture of Brahman, the cremation ground, Śāmbhavī, and the middle path are synonyms.
\end{translation}
%
%</tr4>

%<*sc4>
\begin{sources}[hp03_004]
Cf.\,\emph{Amṛtasiddhi} 2.6
\begin{versinnote}
\tl{avadhūtīpadaṃ ke cic chmaśānaṃ ca mahāpatham/\\+}
\tl{ke cid vadanti ādhārāṃ suṣumnāṃ ca sarasvatīm//\\!}
\end{versinnote}

Cf.\,\emph{Dattātreyayogaśāstra} 109c–110b
\begin{versinnote}
\tl{mahāpathaṃ śmaśānaṃ ca suṣumnāpy ekam eva hi//\\+}
\tl{nāmnāṃ matāntare bhedaḥ phale bhedo na vidyate/\\!}
\end{versinnote}
\end{sources}
%</sc4>

%<*ts4>
\begin{testimonia}[hp03_004]
\emph{Yogacintāmaṇi} f.\,59r (\attr HP), \emph{Yuktabhavadeva} 7.172 (\attr HP)
\begin{variants}
suṣumṇā YCM~] prāṇasya YBhD\sep
brahmarandhra~] brahmarandhraṃ YCM, mahārandhraṃ YBhD\sep
śmaśānaṃ YBhD~] śmaśānī YCM\sep 
ekavācakāḥ YCM~] ekavācakam YBhD    
\end{variants}

% \begin{versinnote}
% \tl{haṭhapradīpikāyām—\\+}
% \tl{suṣumṇā śūnyapadavī brahmarandhraṃ mahāpatham/\\+}
% \tl{śmaśānī śāṃbhavī madhyamārgaś cety ekavācakā iti//\\!}
% \end{versinnote}

% \emph{Yuktabhavadeva} 7.172 (\attr to the \emph{Haṭhapradīpikā})
% \begin{versinnote}
% \tl{prāṇasya śūnyapadavī mahārandhraṃ mahāpatham/\\+}
% \tl{śmaśānaṃ śāmbhavī madhyamārgaś cety ekavācakam//\\!}
% \end{versinnote}
\end{testimonia}
%</ts4>

%<*cm4>
%\begin{philcomm}[hp03_004]
%In the context of \textit{suṣumṇā}, Svātmārāma seems to understand \emph{śāmbhavī} as the power (\emph{śakti}) of Śambhu (Cf.\,\emph{Haṭhapradīpikā} 4.78c \emph{suṣumṇā śāmbhavī śaktiḥ}). JM: I'm not convinced by śāmbhavī = power of Śambhu; I think it's more feminine.
%\end{philcomm}
%</cm4>

\begin{metre}[hp03_004]
Anuṣṭubh (a: na-vipulā)
\end{metre}

%%%%%%%%%%
\subsection*{3.5}
%<*tr5>
\begin{translation}[hp03_005]
Therefore, in order to do his utmost to awaken the goddess sleeping in front of the doorway of Brahman, [the yogi] should undertake the practice of \emph{mudrā}.
\end{translation}
%</tr5>

%<*sc5>
\begin{sources}[hp03_005]
\emph{Śivasaṃhitā} 4.22
\mylb
% \begin{versinnote}
% \tl{tasmāt sarvaprayatnena prabodhayitum īśvarīm/\\+}
% \tl{brahmadvāramukhe suptāṃ mudrābhyāsaṃ samācaret//\\!}
% \end{versinnote}
\end{sources}
%</sc5>

%<*ts5>
\begin{testimonia}[hp03_005]
\emph{Yogacintāmaṇi} f.\,59r (\attr HP), \emph{Yuktabhavadeva} 7.173 (\attr HP)
\begin{variants}
mudrābhyāsaṃ samācaret YBhD~] mudrābhyāsaparo bhavet YCM    
\end{variants}

% \begin{versinnote}
% \tl{tasmāt sarvaprayatnena prabodhayitum īśvarīm/\\+}
% \tl{brahmadvāramukhe suptāṃ mudrābhyāsaparo bhavet//\\!}
% \end{versinnote}

% \emph{Yuktabhavadeva} 7.173 (\attr to the \emph{Haṭhapradīpikā})
% \begin{versinnote}
% \tl{tasmāt sarvaprayatnena prabodhayitum īśvarīm/\\+}
% \tl{brahmadvāramukhe suptāṃ mudrābhyāsaṃ samācaret//\\!}
% \end{versinnote}
\end{testimonia}
%</ts5>

%<*cm5>
%\begin{philcomm}[hp03_005]
%Consider pathe for mukhe [JB: no support for pathe]
%\end{philcomm}
%</cm5>

%%%%%%%%%%
\subsection*{3.6}
%<*tr6>
\begin{translation}[hp03_006]
The great seal, the great lock, the great piercing, the sky-roving [seal], the \emph{uḍḍiyāṇa} [lock], the root lock, then [the lock] called \emph{jālandhara}, [\dots] 
\end{translation}
%</tr6>

%<*sc6>
\begin{sources}[hp03_006]
%Cf.\,Amṛtasiddhi 13.12
%\begin{versinnote}
%\tl{mahāmudrā mahābandho mahāvedhas tṛtīyakaḥ/\\!}
%\end{versinnote}

Cf. \emph{Śivasaṃhitā} 4.23
% \begin{variants}
% uḍḍiyānaṃ mūlabandhas tato jālandharābhidhaḥ~] jālandharo mūlabandho viparītakṛtis tathā ŚS    
% \end{variants}
\begin{versinnote}
\tl{mahāmudrā mahābandho mahāvedhaś ca khecarī/\\+}
\tl{jālandharo mūlabandho viparītakṛtis tathā//\\!}
\end{versinnote}
\end{sources}
%</sc6>

%<*ts6>
\begin{testimonia}[hp03_006]
\emph{Haṭharatnāvalī} 2.32, \emph{Yogacintāmaṇi} f.\,72r (\attr HP), \emph{Yuktabhavadeva} 7.174 (\attr HP)
\begin{variants}
mahāvedhaś ca khecarī YCM YBhD~] mahāvedhas tṛtīyakaḥ HRĀ\sep
uḍḍiyāṇaṃ mūlabandhas~] uḍḍiyānaṃ mūlabandho HRĀ YCM, uḍyānaṃ mūlabandhaś ca YBhD\sep
tato jālandharābhidhaḥ~] bandho jālandharābhidhaḥ HRĀ YCM, bandho jālandharas tathā YBhD
\end{variants}

% \begin{versinnote}
% \tl{mahāmudrā mahābandho mahāvedhas tṛtīyakaḥ/\\+}
% \tl{uḍḍiyānaṃ mūlabandho bandho jālandharābhidhaḥ//\\!}
% \end{versinnote}

% \emph{Yogacintāmaṇi} f.\,72r (\attr to the \emph{Haṭhapradīpikā})
% \begin{versinnote}
% \tl{mahāmudrā mahābandho mahāvedhaś ca khecarī/\\+}
% \tl{uḍḍiyānaṃ mūlabandho bandho jālandharābhidhaḥ//\\!}
% \end{versinnote}

% \emph{Yuktabhavadeva} 7.174 (\attr to the \emph{Haṭhapradīpikā})
% \begin{versinnote}
% \tl{mahāmudrā mahābandho mahāvedhaś ca khecarī/\\+}
% \tl{uḍyānaṃ mūlabandhaś ca bandho jālandharas tathā//\\!}
% \end{versinnote}
\end{testimonia}
%</ts6>


%<*cm6>
%\begin{philcomm}[hp03_006]
%JM: Do we translate mudrā or not?
%\end{philcomm}
%</cm6>

\begin{metre}[hp03_006]
Anuṣṭubh (c: ra-vipulā)
\end{metre}

%%%%%%%%%%
\subsection*{3.7}
%<*tr7>
\begin{translation}[hp03_007]
[\ldots] the bodily position called inverted, \emph{vajrolī} [and] the stimulation of the goddess: this group of ten \emph{mudrā}s and other [practices] destroys old age and death.
\end{translation}
%</tr7>

%<*sc7>
\begin{sources}[hp03_007]
Cf. \emph{Śivasaṃhitā} 4.24
\begin{versinnote}
\tl{uḍyānaṃ caiva vajrolī daśamaṃ śakticālanam/\\+}
\tl{idaṃ hi mudrādaśakaṃ mudrāṇām uttamottamam//\\!}
\end{versinnote}
\end{sources}
%</sc7>

%<*ts7>
\begin{testimonia}[hp03_007]
\emph{Haṭharatnāvalī} 2.33, \emph{Yogacintāmaṇi} f.\,72r (\attr HP), \emph{Yuktabhavadeva} 7.175 (\attr HP)
\begin{variants}
vajrolī HRĀ YBhD~] tathā vai YCM\sep
idaṃ mudrādidaśakaṃ~] sampradāyā khecarī sā HRĀ, etad dhi mudrānavakaṃ YCM, idaṃ hi mudrādaśakaṃ YBhD\sep 
jarāmaraṇanāśanam~] daśa mudrāḥ prakīrtitāḥ HRĀ, jarāmaraṇavarjitam YCM, mudrāṇām uttamomam YBhD 
\end{variants}

% \begin{versinnote}
% \tl{karaṇī viparītākhyā vajrolī śakticālanam/\\+}
% \tl{sampradāyā khecarī sā daśa mudrāḥ prakīrtitāḥ//\\!}
% \end{versinnote}

% \emph{Yogacintāmaṇi} f.\,72r (\attr to the \emph{Haṭhapradīpikā})
% \begin{versinnote}
% \tl{karaṇī viparītākhyā tathā vai śakticālanam/\\+}
% \tl{etad dhi mudrānavakaṃ jarāmaraṇavarjitam//\\!}
% \end{versinnote}

% \emph{Yuktabhavadeva} 7.175 (\attr to the \emph{Haṭhapradīpikā})
% \begin{versinnote}
% \tl{viparītakṛtiś caiva vajrolī śakticālanam/\\+}
% \tl{idaṃ hi mudrādaśakaṃ mudrāṇām uttamomam//\\!}
% \end{versinnote}

\end{testimonia}
%</ts7>

%<*cm7>
\begin{philcomm}[hp03_007]
The reading of \textalpha\, \emph{idaṃ mudrādi}, is not attested by the other manuscript groups but it makes sense in so far as `locks' (\emph{bandha}) and `actions' (\emph{karaṇa}) figure among the ten techniques taught in this chapter. It is also consistent with the reference to \emph{mudrādi} in 1.55. However, most manuscripts have readings, such as \emph{idaṃ hi mudrādaśakam}, that refer to the techniques of this chapter as only \emph{mudrā}s, and this is consistent with 3.104.

%JB: hi (in 3.7c) is supported only by the \textit{Jyotsnā}. Mss. have tu and ca.
\end{philcomm}
%</cm7>

\begin{metre}[hp03_007]
Anuṣṭubh (c: na-vipulā)
\end{metre}

%%%%%%%%%%
\subsection*{3.8}
%<*tr8>
\begin{translation}[hp03_008]
It has been taught by Śiva, is divine, bestows the eight supramundane powers, is beloved of all the Siddhas, is difficult for even the gods to obtain, [\dots] 
\end{translation}
%</tr8>

%<*sc8>
%\begin{sources}[hp03_008]
%\end{sources}
%</sc8>

%<*ts8>
\begin{testimonia}[hp03_008]
\emph{Yogacintāmaṇi} f.\,72r (\attr HP)
\begin{variants}
divyam~] samyag YBhD\sep
sarvasiddhānāṃ YCM~] sarvasiddhendra YBhD 
\end{variants}

% \begin{versinnote}
% \tl{ādināthoditaṃ divyam aṣṭaiśvaryapradāyakam/\\+}
% \tl{vallabhaṃ sarvasiddhānāṃ durlabhaṃ marutām api//\\!}
% \end{versinnote}

% \emph{Yuktabhavadeva} 7.176 (\attr to the \emph{Haṭhapradīpikā})
% \begin{versinnote}
% \tl{ādināthoditaṃ samyagaṣṭaiśvaryapradāyakam/\\+}
% \tl{vallabhaṃ sarvasiddhendradurlabhaṃ marutām api//\\!}
% \end{versinnote}
\end{testimonia}
%</ts8>

%<*cm8>
%\begin{philcomm}[hp03_008]
%\end{philcomm}
%</cm8>

%%%%%%%%%%
\subsection*{3.9}
%<*tr9>
\begin{translation}[hp03_009]
[\ldots] should be carefully kept secret like a casket of gems [and] must not be spoken of to anyone, like sex with a respectable woman.
\end{translation}
%</tr9>

%<*sc9>
%\begin{sources}[hp03_009]
%\end{sources}
%</sc9>

%<*ts9>
\begin{testimonia}[hp03_009]
\emph{Yogacintāmaṇi} f.\,72r (\attr HP), \emph{Yuktabhavadeva} 7.177 (\attr HP)
\begin{variants}
yathā ratnakaraṇḍakam YCM~] jarāmaraṇanāśanam YBhD    
\end{variants}

% \begin{versinnote}
% \tl{gopanīyaṃ prayatnena yathā ratnakaraṇḍakam/\\+}
% \tl{kasyacinnaiva vaktavyaṃ kulastrīsurataṃ yathā//\\!}
% \end{versinnote}

% \emph{Yuktabhavadeva} 7.177 (\attr to the \emph{Haṭhapradīpikā})
% \begin{versinnote}
% \tl{gopanīyaṃ prayatnena jarāmaraṇanāśanam/\\+}
% \tl{kasyacinnaiva vaktavyaṃ kulastrīsurataṃ yathā// \\!}
% \end{versinnote}
\end{testimonia}
%</ts9>

%<*cm9>
%\begin{philcomm}[hp03_009]
%\end{philcomm}
%</cm9>

%%%%%%%%%%
\subsection*{3.9*1}
%<*tr9-1>
\begin{translation}[hp03_009_1]
[Together with] \emph{amarolī} and \emph{sahajolī}, \emph{vajrolī} is considered to be threefold. I shall teach their characteristics and the details of how they should be performed.
\end{translation}
%</tr9-1>

%<*sc9-1>
\begin{sources}[hp03_009_1]
\emph{Dattātreyayogaśāstra} 31c–32b
\begin{variants}
vajrolīr~] vajrolī DYŚ, vajrolir DYŚ\vl\sep 
amarolī~] amaroliś DYŚ\vl, cāmarolī DYŚ\sep
sahajolī DYŚ~] sahajolis DYŚ\vl
\end{variants} 

% \begin{versinnote}
% \tl{vajrolī cāmarolī ca sahajolī tridhā matā/\\+}
% \tl{eteṣāṃ lakṣaṇaṃ vakṣye kartavyaṃ ca viśeṣataḥ//\\!}
% \end{versinnote}
% \begin{appinnote}
% \tl{vajrolī~] vajrolir \vl • cāmarolī~] amaroliś \vl • sahajolī~] sahajolis \vl \\!}
% \end{appinnote}
\end{sources}
%</sc9-1>


%<*cm9-1>
%\begin{philcomm}[hp03_009_1]
%\end{philcomm}
%</cm9-1>

%%%%%%%%%%
\subsection*{3.10 heading}
%<*tr10a>
\begin{translation}[hp03_010a]
Of these, the great seal (\emph{mahāmudrā}) [is now taught]: 
\end{translation}
%</tr10a>

%<*cm10a>
% \begin{philcomm}[hp03_010a]
% \end{philcomm}
%</cm10a>

%%%%%%%%%%
\subsection*{3.10}
%<*tr10>
\begin{translation}[hp03_010]
{}[The yogi] should press the perineum with the heel of the left foot, hold [the foot of] the extended leg with the hands and breathe in through the mouth.
\end{translation}
%</tr10>

%<*sc10>
\begin{sources}[hp03_010]
\emph{Amaraugha} 19
\begin{variants}
dhṛtvā A~] kṛtvā A\vl     
\end{variants}
% \begin{versinnote}
% \tl{pādamūlena vāmena yoniṃ saṃpīḍya dakṣiṇam/\\+}
% \tl{pādaṃ prasāritaṃ dhṛtvā karābhyāṃ pūrayen mukhe//\\!}
% \end{versinnote}
% \begin{appinnote}
% \tl{\textbf{c} dhṛtvā~] kṛtvā \vl \\!}
% \end{appinnote}

Cf.\,\emph{Amṛtasiddhi} 11.3
\begin{versinnote}
\tl{yoniṃ saṃpīḍya vāmena pādamūlena yatnataḥ/\\+}
\tl{savyaṃ prasāritaṃ pādaṃ karābhyāṃ dhārayed dṛḍham//\\!}
\end{versinnote}
\end{sources}
%</sc10>

%<*ts10>
\begin{testimonia}[hp03_010]
\emph{Haṭharatnāvalī} 2.37, \emph{Yogacintāmaṇi} ff. 72v–73r (\attr HP), \emph{Yuktabhavadeva} 7.178 (\attr HP)
\begin{variants}
dhṛtvā~] kṛtvā HRĀ YCM YBhD\sep 
pūrayen HRĀ YCM~] dhārayed YBhD\sep
mukhe HRĀ\vl~] mukham HRĀ YCM, dṛḍham YBhD
\end{variants}


% \begin{versinnote}
% \tl{pādamūlena vāmena yoniṃ sampīḍya dakṣiṇam/\\+}
% \tl{pādaṃ prasāritaṃ kṛtvā karābhyāṃ pūrayen mukham//\\!}
% \end{versinnote}
% \begin{appinnote}
% \tl{\textbf{d} mukham~] mukhe \vl \\!}
% \end{appinnote}

% \emph{Yogacintāmaṇi} ff. 72v–73r (\attr to the \emph{Haṭhapradīpikā})
% \begin{versinnote}
% \tl{pādamūlena vāmena yoniṃ saṃpīḍya dakṣiṇam/\\+}
% \tl{pādaṃ prasāritaṃ dhṛtvā karābhyāṃ pūrayen mukham//\\!}
% \end{versinnote}

% \emph{Yuktabhavadeva} 7.178 (\attr to the \emph{Haṭhapradīpikā})
% \begin{versinnote}
% \tl{pādamūlena vāmena yoniṃ sampīḍya dakṣiṇam/\\+}
% \tl{pādaṃ prasāritaṃ kṛtvā karābhyāṃ dhārayed dṛḍham//\\!}
% \end{versinnote}
\end{testimonia}
%</ts10>

%<*cm10>
%\begin{philcomm}
%\end{philcomm}
%</cm10>

%%%%%%%%%%
\subsection*{3.11}
%<*tr11>
\begin{translation}[hp03_011]
He should apply a lock to the throat and hold the breath in the upper [part of the body]. Just as a snake hit with a staff assumes the form of a staff, [\dots] 
\end{translation}
%</tr11>

%<*sc11>
\begin{sources}[hp03_011]
\emph{Amaraugha} 20
\mylb
% \begin{versinnote}
% \tl{kaṇṭhe bandhaṃ samāropya dhārayed vāyum ūrdhvataḥ/\\+}
% \tl{yathā daṇḍāhataḥ sarpo daṇḍākāraḥ prajāyate//\\!}
% \end{versinnote}
\end{sources}
%</sc11>

%<*ts11>
\begin{testimonia}[hp03_011]
\emph{Haṭharatnāvalī} 2.37cd–38ab, \emph{Yogacintāmaṇi} f.\,73r (\attr HP), \emph{Yuktabhavadeva} 7.179 (\attr HP)
\begin{variants}
dhārayed YCM YBhD~] pūrayed HRĀ    
\end{variants}


% \begin{versinnote}
% \tl{kaṇṭhe bandhaṃ samāropya pūrayed vāyum ūrdhvataḥ//\\+}
% \tl{yathā daṇḍāhataḥ sarpo daṇḍākāraḥ prajāyate/\\!}
% \end{versinnote}

% \emph{Yogacintāmaṇi} ff.\,73r (\attr to the \emph{Haṭhapradīpikā})
% \begin{versinnote}
% \tl{kaṇṭhe bandhaṃ samāropya dhārayed vāyum ūrdhvataḥ/\\+}
% \tl{yathā daṇḍāhataḥ sarpo daṇḍākāraḥ prajāyate//\\!}
% \end{versinnote}

% \emph{Yuktabhavadeva} 7.179 (\attr to the \emph{Haṭhapradīpikā})
% \begin{versinnote}
% \tl{kaṇṭhe bandhaṃ samāropya dhārayed vāyum ūrdhvataḥ/\\+}
% \tl{yathā daṇḍāhataḥ sarpo daṇḍākāraḥ prajāyate//\\!}
% \end{versinnote}
\end{testimonia}
%</ts11>

%<*cm11>
\begin{philcomm}[hp03_011]
The instruction to hold the breath upwards (\emph{ūrdhvataḥ}) is somewhat vague. In a commentarial passage on this verse in \emph{Yuktabhavadeva} 7.187, Bhavadevamiśra clarifies this by saying, `one should hold it higher than the heart' (\emph{hṛdayād ūrdhvato dhārayet}).
\end{philcomm}
%</cm11>

%%%%%%%%%%
\subsection*{3.12}
%<*tr12>
\begin{translation}[hp03_012]
[\dots] so the goddess Kuṇḍalinī suddenly becomes straight. Then she becomes still in the vessel with two halves.
\end{translation}
%</tr12>

%<*sc12>
\begin{sources}[hp03_012]
\emph{Amaraugha} 21
\mylb
% \begin{versinnote}
% \tl{ṛjvībhūtā tathā śaktiḥ kuṇḍalī sahasā bhavet/\\+}
% \tl{tadāsau maraṇāvasthā jāyate dvipuṭāśritā//\\!}
% \end{versinnote}
\end{sources}
%</sc12>

%<*ts12>
\begin{testimonia}[hp03_012]
\emph{Haṭharatnāvalī} 2.38cd–39ab, \emph{Yogacintāmaṇi} ff.\,72v–73r (\attr HP), \emph{Yuktabhavadeva} 7.180 (\attr HP)
\begin{variants}
sahasā YBhD~] sahajā HRĀ YCM\sep
tathāsau YCM YBhD~] tathā sā HRĀ\sep
maraṇāvasthā HRĀ YBhD~] maraṇāvasthāṃ YCM\sep
jāyate dvipuṭāśritā YBhD~] jāyate dvipuṭīsthitā HRĀ, harate dvipuṭāśrayām YCM (\emph{em}., dvipaṭā L, dvipadā N) 
\end{variants}


% \begin{versinnote}
% \tl{ṛjvībhūtā tathā śaktiḥ kuṇḍalī sahajā bhavet//\\+}
% \tl{tathā sā maraṇāvasthā jāyate dvipuṭīsthitā/\\!}
% \end{versinnote}

% \emph{Yogacintāmaṇi} ff. 72v–73r (\attr to the \emph{Haṭhapradīpikā}) % order of the hemistichs has been changed. I don't know if or where I should mention this.
% \begin{versinnote}
% \tl{ṛjvībhūtā tathā śaktiḥ kuṇḍalī sahajā bhavet/\\+}
% \tl{tathāsau maraṇāvasthāṃ harate dvipuṭāśrayām//\\!}
% \end{versinnote}
% \begin{appinnote}
% \tl{\var{\textbf{d} dvipuṭā~] em., dvipaṭā° L, dvipadā° N.} \\!}
% \end{appinnote}

% \emph{Yuktabhavadeva} 7.180 (\attr to the \emph{Haṭhapradīpikā})
% \begin{versinnote}
% \tl{ṛjvībhūtā tathā śaktiḥ kuṇḍalī sahasā bhavet/\\+}
% \tl{tadāsau maraṇāvasthā jāyate dvipuṭāśritā//\\!}
% \end{versinnote}
\end{testimonia}
%</ts12>

%<*cm12>
\begin{philcomm}[hp03_012]
This verse is taken from the \emph{Amaraugha}, which uses the alchemical imagery of the \emph{Amṛtasiddhi} to describe the stilling of Kuṇḍalinī in the central channel. Drawing on Hellwig 2009: 238–240, Mallinson and Szanto (2021: 21) note that “In alchemical texts \emph{māraṇa} (“killing”) involves heating a substance and thereby changing its state, usually through calcination or oxidation, so that it becomes inert. In the \emph{Amṛtasiddhi māraṇa} and other derivatives of the root \emph{mṛ}, “die”, are used to denote the stilling or stopping of either the breath or Bindu.” Thus when Kuṇḍalinī is said to be in the state of \emph{maraṇa} the meaning is that she is stilled. The \emph{dvipuṭa} or “vessel with two halves” in which this occurs is the same as the \emph{Amṛtasiddhi}’s \emph{saṃpuṭa}, which, drawing on Hellwig (2009: 342), Mallinson and Szanto (2021: 22) say “consists of two \emph{puṭa}s joined together to form a sealed crucible for heating reagents without evaporation”. In the yoga of the \emph{Amṛtasiddhi}, the bodily \emph{saṃpuṭa} is formed by applying locks at the top and bottom of the central channel, i.e.~constricting the perineal region and the throat. In the \emph{Haṭhapradīpikā} it is formed by pressing the perineum with the heel and constricting the throat.\lb

As Birch (2019: 971) notes, it is unlikely that later non-Buddhist authors understood \textit{maraṇāvasthā} and \emph{dvipuṭa} according to the alchemical metaphors of the \textit{Amṛtasiddhi}. Later commentators take \emph{dvipuṭa} as the two nostrils (e.g., \textit{Yuktabhavadeva} 7.187, \textit{dvināsāpuṭa}); the \textit{iḍā} and \textit{piṅgalā} channels (e.g., \textit{Jyotsnā} 3.27, \textit{puṭayor dvayam iḍāpiṅgalayor yugmam}); or the in and out flows of the breath (e.g., \textit{Yogaprakāśikā} 5.16–17, \textit{vāyor bahirnirgamanam antaḥpraveśa iti yat puṭadvayaṃ tam}). How these commentators understood \textit{asau māraṇāvasthā} is less clear. Brahmānanda seems to take it as the death of \textit{prāṇa}, or in other words, the absence of the breath, in the two nostrils (\textit{maraṇāvasthā jāyate kuṇḍalībodhe sati suṣumnāyāṃ praviṣṭe prāṇe dvayoḥ puṭayoḥ prāṇaviyogāt}).  Bhavadeva thought that Kuṇḍalinī, along with \textit{prāṇa} and \textit{apāna}, remains in the two nostrils while the breath is being held (\textit{evaṃ vāyudhāraṇāyāṃ kriyamāṇāyāṃ vyākulā bhūtā kuṇḍalinī apānaprāṇābhyāṃ saha nāsāpuṭadvayāśritā bhavati}). Others, such as Śivānanda and Bālakṛṣṇa, favour the reading \textit{tadā sā maraṇāvasthāṃ harate dvi\-puṭāśritām} (or \textit{dvipuṭāśrayām}), which is present in group \textgamma\ and \textdelta\ manuscripts of the \textit{Haṭhapradīpikā}. Bālakṛṣṇa understands this to mean that the great seal destroys death (\textit{maraṇāvasthāṃ harate mahāmudreti bhāvaḥ}) but it could also mean that the awakened Kuṇḍalinī destroys death, which is usually dependent on the in and out breaths.\lb

The form \textit{ṛjvībhūtā} is non-Pāṇinian (it should be \textit{ṛjūbhūtā}) but all witnesses have \textit{ṛjvī} or variants thereof. \alphaOne\ has \textit{ṛjvī bhūtvā} which is correct morphologically, but it does not make sense with \textit{bhavet} in the next \textit{pāda} and may be an attempt at correction by the scribe.
%JB: by the scribe (not by the redactor).

%JB: I think our translation should reflect a likely Śaiva interpretation rather than the alchemical metaphors in the Amṛtasiddhi (e.g., `Then, that state of death arises in the two nostrils'). JM: this would imply that we think that Svātmārāma decided to use the verse with this understanding, which seems unlikely to me. JB. Unlikely? All the evidence in commentaries on the HP and later compendiums, like the YCM and Yuktabhavadeva, indicates that the alchemical metaphors of the AS were not understood here. It's not even clear that the alchemical metaphors work with raising moving or raising kuṇḍalinī. In contrast to this, there's not one commentarial explanation or rewrite of the verse that reveals an alchemical meaning. We also have to bear in mind that there's no evidence that Svātmārāma and later commentators knew the AS. In the one case where there is some evidence (i.e., Śivānanda quotes it in the YCM), it doesn't seem to have helped the author understand the alchemical metaphor in this instance (as Śivānanda rewrote the verse to say that kuṇḍalinī destroys the state of death).JM:but the commentators etc. are all a lot later than the HP. I think it's better to give Svātmārāma the benefit of the doubt. I don't think he would have used the verse if he didn't understand it correctly — as you (JB) point out in the note the later attempts to understand maraṇa in a non-alchemical way are unconvincing.

%The meaning of "Śaiva" here and above is unclear to me. Doesn't it work to say  "later authors"? JB I've changed the note above to non-Buddhist.
\end{philcomm}
%</cm12>

%%%%%%%%%%
\subsection*{3.13}
%<*tr13>
\begin{translation}[hp03_013]
{[}The yogi] should then exhale very slowly, not quickly. This is the great seal revealed by the great Siddhas.
\end{translation}
%</tr13>

%<*sc13>
%\begin{sources}[hp03_013]
%\end{sources}
%</sc13>

%<*ts13>
\begin{testimonia}[hp03_013]
\emph{Yuktabhavadeva} 181 (\attr HP)
\begin{variants}
pradarśitā~] pradṛśyate YBhD 
\end{variants}

% \begin{versinnote}
% \tl{tataḥ śanaiḥ śanair eva recayen na tu vegataḥ/\\+}
% \tl{iyaṃ khalu mahāmudrā mahāsiddhaiḥ pradṛśyate//\\!}
% \end{versinnote}
\end{testimonia}
%</ts13>

%<*cm13>
%\begin{philcomm}[hp03_013]
%\end{philcomm}
%</cm13>

%%%%%%%%%%
\subsection*{3.14}
%<*tr14>
\begin{translation}[hp03_014]
Problems such as the great afflictions [and] death and so forth dissolve, and that is why the most wise call it the `great seal.'
\end{translation}
%</tr14>
% MD: Now I prefer jīryante to hīyante. The latter is suppoted only α1 and α2, may be an outlier. As an explanation of the name mahāmudrā, √jṝ seems to make more sense. The kleśa etc. decay after being confined (mudrita) or so. done

%<*sc14>
\begin{sources}[hp03_014]
\emph{Amaraugha} 22
\begin{variants}
mahākleśādayo doṣā A~] mahārogā mahākleśā A\vl\sep
jīryante A\vl~] bhidyante A
\end{variants}

% \begin{versinnote}
% \tl{mahākleśādayo doṣā bhidyante maraṇādayaḥ/\\+}
% \tl{mahāmudrāṃ tu tenaiva vadanti vibudhottamāḥ//\\!}
% \end{versinnote}
% \begin{appinnote}
% \tl{\var{\textbf{a} mahākleśādayo doṣā~] mahārogā mahākleśā \vl\ \ %
%     \textbf{b} bhidyante~] jīryante \vl} \\!}
% \end{appinnote}
\end{sources}
%</sc14>

%<*ts14>
\begin{testimonia}[hp03_014]
\emph{Yogacintāmaṇi} f.\,72v (\attr \emph{Skandapurāṇa}), \emph{Yuktabhavadeva} 7.182 (\attr HP)
\begin{variants}
mahākleśādayo~] mahākleśā yato YCM YBhD\sep
tenaiva YCM~] tām eva YBhD
\end{variants}

% \begin{versinnote}
% \tl{mahākleśā yato doṣā jīryante maraṇādayaḥ/\\+}
% \tl{mahāmudrāṃ ca tenaiva vadanti vibudhottamā iti//\\!}
% \end{versinnote}

% \emph{Yuktabhavadeva} 7.182 (\attr HP)
% \begin{versinnote}
% \tl{mahākleśā yato doṣā jīryante maraṇādayaḥ/\\+}
% \tl{mahāmudrāṃ ca tām eva vadanti vibudhottamāḥ//\\!}
% \end{versinnote}
\end{testimonia}
%</ts14>

%<*cm14>
\begin{philcomm}[hp03_014]
%Discuss the group 2 reading \emph{mahākleśā yato}. Also the V19 reading \emph{mahākleśā mahādoṣā}, which is similar to the \emph{Amaraughaprabodha}'s reading (\textit{mahārogā mahākleśā}). [JB: can't see a need for a note as alpha reading is close to the Amaraugha.]
%JM: note on assonance with the sounds of first line supporting the name?
This verse appears to be explaining the name of \emph{mahāmudrā} through assonance with \emph{mahā\-kleśādayo, doṣā} and \emph{maraṇādayaḥ} in the first line.
\end{philcomm}
%</cm14>

%%%%%%%%%%
\subsection*{3.15}
%<*tr15>
\begin{translation}[hp03_015]
After practising on the lunar side of the body, the yogi should then practise on the solar side. [The yogi] should finish practising the seal when the count is even.
\end{translation}
%</tr15>

%<*sc15>
\begin{sources}[hp03_015]
\emph{Vivekamārtaṇḍa} 60
\mylb
% \begin{versinnote}
% \tl{candrāṅge tu samabhyasya sūryāṅge punar abhyaset/\\+}
% \tl{yāvat tulyā bhavet saṅkhyā tato mudrāṃ visarjayet//\\!}
%\tl{\var{candrāṅge tu~] candrāṅgena \vl • sūryāṅge punar abhyaset~] sūryāṅgenābhyaset punaḥ \vl}\\!}
% \end{versinnote}
\end{sources}
%</sc15>

%<*ts15>
\begin{testimonia}[hp03_015]
\emph{Yuktabhavadeva} 7.183 (\attr HP)
\begin{variants}
candrāṅge tu~] candrāṅgena YBhD\sep
sūryāṅge punar abhyaset~] sūryāṅgenābhyaset tataḥ YBhD 
\end{variants}

% \begin{versinnote}
% \tl{candrāṅgena samabhyasya sūryāṅgenābhyaset tataḥ/\\+}
% \tl{yāvat tulyā bhavet saṃkhyā tato mudrāṃ visarjayet//\\!}
% \end{versinnote}
\end{testimonia}
%</ts15>

%<*cm15>
\begin{philcomm}[hp03_015]
The terms \emph{candrāṅga} and \emph{sūryāṅga} are unusual and not used in other yoga texts outside the context of \emph{mahāmudrā}. In \emph{Jyotsnā} 3.15, Brahmānanda glosses \emph{candrāṅga} as \emph{vāmāṅga} (`the left side of the body') and \emph{sūryāṅga} as \emph{dakṣāṅga} (`the right side of the body') and goes on to explain the sequence of practice as follows:

\begin{versinnote}  
\tl{atrāyaṃ kramaḥ/ ākuñcitavāmapādapārṣṇiṃ yonisthāne saṃyojya prasāritadakṣiṇapādāṅguṣṭham ākuñcitatarjanībhyāṃ\skx{\linebreak}{ }gṛhītvābhyāso vāmāṅge\skx{}{ }'bhyāsaḥ/ asminn abhyāse pūrito vāyur vāmāṅge tiṣṭhati/ ākuñcita\-dakṣa\-pāda\-pārṣṇiṃ yoni\skx{-}{}sthāne saṃyojya prasārita\-vāma\-pādā\-ṅguṣṭham ākuñcita\-tarjanībhyāṃ gṛhītvābhyāso dakṣāṅge\skx{}{ }'bhyāsaḥ/ asminn abhyāse\skx{\linebreak}{ }pūrito vāyur dakṣāṅge tiṣṭhati/\\!}
\end{versinnote}
\closer
\begin{quote}
This is the sequence in the [practice]. Joining the heel of the bent left leg with the region of the perineum and holding the big toe of the extended right leg with index fingers curled [around it] is the practice, that is, the practice on the left side of the body (\emph{vāmāṅga}). In this practice, the inhaled breath remains on the left side of the body. Joining the heel of the bent right leg with the region of the perineum and holding the big toe of the extended left leg with index fingers curled [around it] is the practice, that is, the practice on the right side of the body. In this practice, the inhaled breath remains on the right side of the body.
\end{quote}
\end{philcomm}
%</cm15>

%%%%%%%%%%
\subsection*{3.16}
%<*tr16>
\begin{translation}[hp03_016]
{}[For the yogi who practises thus] there is no wholesome or unwholesome [food], and all flavours without exception become flavourless. Even terrible poison, when consumed, is digested like nectar.
\end{translation}
%</tr16>

%<*sc16>
\begin{sources}[hp03_016]
\emph{Vivekamārtaṇḍa} 61
\begin{variants}
jīryate VM\vl~] jīryati VM, jāyate VM\vl    
\end{variants}
% \begin{versinnote}
% \tl{na hi pathyam apathyaṃ vā rasāḥ sārve 'pi nīrasāḥ/\\+}
% \tl{api bhuktaṃ viṣaṃ ghoraṃ pīyūṣam iva jīryati//\\!}
% \end{versinnote}
% \begin{appinnote}
% \tl{\textbf{d} jīryati~] jīryate \vl, jāyate \vl \\!}
% \end{appinnote}
\end{sources}
%</sc16>

%<*ts16>
\begin{testimonia}[hp03_016]
\emph{Haṭharatnāvalī} 2.40, \emph{Yogacintāmaṇi} f.\,73r (\attr HP), \emph{Yuktabhavadeva} 7.184 (\attr HP)
\begin{variants}
rasāḥ sarve 'pi nīrasāḥ HRĀ YCM~] sarasaṃ nirasaṃ ca vā YBhD\sep
jīryate HRĀ~] jīryati YCM YBhD    
\end{variants}

% \begin{versinnote}
% \tl{na hi pathyam apathyaṃ vā rasāḥ sarve 'pi nīrasāḥ/\\+}
% \tl{api bhuktaṃ viṣaṃ ghoraṃ pīyūṣam iva jīryate//\\!}
% \end{versinnote}
 
% \emph{Yogacintāmaṇi} f.\,73r (\attr to the \emph{Haṭhapradīpikā})
% \begin{versinnote}
% \tl{na hi pathyam apathyaṃ vā rasāḥ sarve 'pi nīrasāḥ/\\+}
% \tl{api bhuktaṃ viṣaṃ ghoraṃ pīyūṣam iva jīryati//\\!}
% \end{versinnote}

% \emph{Yuktabhavadeva} 7.184 (\attr to the \emph{Haṭhapradīpikā})
% \begin{versinnote}
% \tl{iha pathyam apathyaṃ vā sarasaṃ nirasaṃ ca vā/\\+}
% \tl{api bhuktaṃ viṣaṃ ghoraṃ pīyūṣam iva jīryati//\\!}
% \end{versinnote}
\end{testimonia}
%</ts16>

%<*cm16>
%\begin{philcomm}[hp03_016]
%In \emph{Jyotsnā} 3.16, Brahmānanda understands the second verse quarter to mean all flavours –pungent (\emph{kaṭu}), acid (\emph{amlā}), etc.– and stale (\emph{nīrasa}) [food] and takes it with the verb \emph{jīryati}.
%[JB: This note doesnt seem relevant to our translation]
%\end{philcomm}
%</cm16>

%%%%%%%%%%
\subsection*{3.17}
%<*tr17>
\begin{translation}[hp03_017]
Diseases such as consumption, skin afflictions, constipation, swelling and indigestion disappear for [the yogi] who practises the great seal.
\end{translation}
%</tr17>

%<*sc17>
\begin{sources}[hp03_017]
\emph{Vivekamārtaṇḍa} 62
\begin{variants}
purogamāḥ~] jvaravyathāḥ VM\sep
tasya doṣāḥ VM~] sarvarogāḥ VM\vl, rogās tasya VM\vl
\end{variants}

% \begin{versinnote}
% \tl{kṣayakuṣṭagudāvarttagulmājīrṇajvaravyathāḥ/\\+}
% \tl{tasya doṣāḥ kṣayaṃ yānti mahāmudrāṃ tu yo 'bhyaset//\\!}
% \end{versinnote}
% \begin{appinnote}
% \tl{\textbf{c} tasya doṣāḥ~] sarvarogāḥ \vl, rogās tasya \vl \\!}
% \end{appinnote}
\end{sources}
%</sc17>

%<*ts17>
\begin{testimonia}[hp03_017]
\emph{Haṭharatnāvalī} 2.41, \emph{Yogacintāmaṇi} f.\,73r (\attr HP), \emph{Yuktabhavadeva} 7.185 (\attr HP)
\begin{variants}
tasya doṣāḥ YCM~] doṣāḥ sarve HRĀ, tasya rogāḥ YBhD\sep
tu HRĀ YBhD~] ca YCM
\end{variants}

% \begin{versinnote}
% \tl{kṣayakuṣṭhagudāvartagulmājīrṇapurogamāḥ/\\+}
% \tl{doṣāḥ sarve kṣayaṃ yānti mahāmudrāṃ tu yo 'bhyaset//\\!}
% \end{versinnote}

% \emph{Yogacintāmaṇi} f.\,73r (\attr to the \emph{Haṭhapradīpikā})
% \begin{versinnote}
% \tl{kṣayakuṣṭhagudāvartagulmaplīhapurogamāḥ/\\+}
% \tl{tasya doṣāḥ kṣayaṃ yānti mahāmudrāṃ ca yo 'bhyaset// \\!}
% \end{versinnote}

% \emph{Yuktabhavadeva} 7.185 (\attr to the \emph{Haṭhapradīpikā})
% \begin{versinnote}
% \tl{kṣayakuṣṭhagudāvarttagulmājīrṇapurogamāḥ/\\+}
% \tl{tasya rogāḥ kṣayaṃ yānti mahāmudrāṃ tu yo'bhyaset//\\!}
% \end{versinnote}
\end{testimonia}
%</ts17>

%<*cm17>
%\begin{philcomm}[hp03_017]
%\end{philcomm}
%</cm17>

%%%%%%%%%%
\subsection*{3.18}
%<*tr18>
\begin{translation}[hp03_018]
This great seal which brings about the great \emph{siddhi} for men has been taught. It should be carefully kept secret [and] not be given to all and sundry.
\end{translation}
%</tr18>

%<*sc18>
\begin{sources}[hp03_018]
\emph{Vivekamārtaṇḍa} 63
\mylb
% \begin{versinnote}
% \tl{kathiteyaṃ mahāmudrā mahāsiddhikarī nṛṇām/\\+}
% \tl{gopanīyā prayatnena na deyā yasya kasya cit//\\!}
% \end{versinnote}
\end{sources}
%</sc18>

%<*ts18>
\begin{testimonia}[hp03_018]
\emph{Haṭharatnāvalī} 2.42, \emph{Yuktabhavadeva} 7.186 (\attr HP)
\begin{variants}
mahāsiddhikarī nṛṇām YBhD~] jarāmṛtyuvināśinī HRĀ   
\end{variants}
% \begin{versinnote}
% \tl{kathiteyaṃ mahāmudrā jarāmṛtyuvināśinī/\\+}
% \tl{gopanīyā prayatnena na deyā yasya kasya cit//\\!}
% \end{versinnote}

% \emph{Yuktabhavadeva} 7.186 (\attr to the \emph{Haṭhapradīpikā})
% \begin{versinnote}
% \tl{kathiteyaṃ mahāmudrā mahāsiddhikarī nṛṇām/\\+}
% \tl{gopanīyā prayatnena na deyā yasya kasya cit//\\!}
% \end{versinnote}
\end{testimonia}
%</ts18>

%<*cm18>
\begin{philcomm}[hp03_018]
Two readings of the second quarter are well-attested: \emph{jarāmṛtyuvināśinī} (\alphaOne\ etc.) and \emph{mahāsiddhikarī nṛṇām} (\alphaThree\ etc.). While the first reading is possible, the play on \emph{mahāsiddhi} and \emph{mahāmudrā} seems more likely original, as seen in the source text, the \emph{Vivekamārtaṇḍa} (without significant variants).\lb


%Manuscripts of three groups (\textalpha, \textgamma\ and \textpi), which are important stemmatically, have the reading \emph{jarāmṛtyuvināśinī} (and \alphaTwo\ has \emph{nṛṇāṃ mṛtyuvināśinī}) for the second verse quarter. While this reading is possible, the play on \emph{mahāsiddhi} and \emph{mahāmudrā} seems more likely original, as seen in the source text, the \emph{Vivekamārtaṇḍa} (without significant variants), and the \texteta\ and \textzeta\ groups. \lb %  check the stemma tree -- α1+α2 vs α3

In \emph{Jyotsnā} 3.18, Brahmānanda understands \emph{mahāsiddhi} as referring to `great \emph{siddhi}s,' but in other works it can mean liberation (Mallinson 2012).
\end{philcomm}
%</cm18>

%%%%%%%%%%
\subsection*{3.19 heading}
%<*tr19a>
\begin{translation}[hp03_019a]
Now, the great lock (\emph{mahābandha}):
\end{translation}
%</tr19a>

%<*cm19a>
% \begin{philcomm}[hp03_019a]
% \end{philcomm}
%</cm19a>

%%%%%%%%%%
\subsection*{3.19}
%<*tr19>
\begin{translation}[hp03_019]
{}[The yogi] should place the heel of the left foot on the perineal region. And he should put the right foot on the left thigh, [\dots]
\end{translation}
%</tr19>

%<*sc19>
\begin{sources}[hp03_019]
19ab = \emph{Dattātreyayogaśāstra} 132cd (in the section on \emph{mahāmudrā})
\mylb
% \begin{versinnote}
% \tl{pārṣṇiṃ vāmasya pādasya yonisthāne niyojayet//\\!}
% \end{versinnote}

19cd. Cf.\,\emph{Vivekamārtaṇḍa} 8a (not \emph{anuṣṭubh})
\begin{versinnote}
\tl{vāmorūpari dakṣiṇañ ca caraṇaṃ saṃsthāpya\\!}
\end{versinnote}
\end{sources}
%</sc19>

%<*ts19>
\begin{testimonia}[hp03_019]
\emph{Haṭharatnāvalī} 2.43, \emph{Yogalakṣanāvalī} f.\,31v (\attr HP), \emph{Yogacintāmaṇi} f.\,73r (\attr HP), \emph{Yuktabhavadeva} 7.190 (\attr HP)
\begin{variants}
pārṣṇiṃ vāmasya pādasya HRĀ YCM~] vāmāṅghripārṣṇibhāgena YLĀ, pārṣṇivāmasya pādasya YBhD\sep
yonisthāne niyojayet HRĀ YCM YBhD~]  yonisthānaṃ nipīḍayet YLĀ
\end{variants}


% \begin{versinnote}
% \tl{pārṣṇiṃ vāmasya pādasya yonisthāne niyojayet/\\+}
% \tl{vāmorūpari saṃsthāpya dakṣiṇaṃ caraṇaṃ tathā//\\!}
% \end{versinnote}

% \emph{Yogalakṣanāvalī} f.\,31v
% \begin{versinnote}
% \tl{haṭhapradīpikāyām—\\+}
% \tl{vāmāṅghripārṣṇibhāgena yonisthānaṃ nipīḍayet/\\+}
% \tl{vāmorupari saṃsthāpya dakṣiṇaṃ caraṇaṃ tathā//\\!}
% \end{versinnote}

% \emph{Yogacintāmaṇi} f.\,73r
% \begin{versinnote}
% \tl{haṭhapradīpikāyām—\\+}
% \tl{pārṣṇiṃ vāmasya pādasya yonisthāne niyojayet/\\+}
% \tl{vāmorūpari saṃsthāpya dakṣiṇaṃ caraṇaṃ tathā//\\!}
% \end{versinnote}

% \emph{Yuktabhavadeva} 7.190 (\attr to the \emph{Haṭhapradīpikā})
% \begin{versinnote}
% \tl{pārṣṇivāmasya pādasya yonisthāne niyojayet/\\+}
% \tl{vāmorupari saṃsthāpya dakṣiṇaṃ caraṇaṃ tathā//\\!}
% \end{versinnote}
\end{testimonia}
%</ts19>

%<*cm19>
\begin{philcomm}[hp03_019]
The seated position for \emph{mahābandha} described in this verse is not in the \emph{Amṛtasiddhi} (chapter 12) or \emph{Amaraugha} (25cd–27). The \emph{Amṛtasiddhi} instructs the same position for \emph{mahāmudrā} and \emph{mahābandha}, and the \emph{Amaraugha} does not comment on the posture of \emph{mahābandha}, implying that its posture is the same as \emph{mahāmudrā}.
\end{philcomm}
%</cm19>

%%%%%%%%%%
\subsection*{3.20}
%<*tr20>
\begin{translation}[hp03_020]
[\dots] inhale through the mouth, firmly put the chin on the chest, contract the perineum and fix the mind in the centre.
\end{translation}
%</tr20>

%<*sc20>
\begin{sources}[hp03_020]
\emph{Amaraugha} 24
\begin{variants}
hṛdaye cibukaṃ~] cibukaṃ hṛdaye A\sep
nibhṛtya A~] nibhṛtaṃ A\vl
\end{variants}
% \begin{versinnote}
% \tl{pūrayitvā mukhe vāyuṃ cibukaṃ hṛdaye dṛḍham/\\+}
% \tl{nibhṛtya yonim ākuñcya mano madhye niyojayet//\\!}
% \end{versinnote}
% \begin{appinnote}
% \tl{\textbf{c} nibhṛtya~] nibhṛtaṃ \vl \\!}
% \end{appinnote}
\end{sources}
%</sc20>

%<*ts20>
\begin{testimonia}[hp03_020]
\emph{Haṭharatnāvalī} 2.44, \emph{Yogalakṣanāvalī} f.\,31v (\attr HP), \emph{Yogacintāmaṇi} f.\,73v (\attr HP), \emph{Yuktabhavadeva} 7.191 (\attr HP) 
\begin{variants}
pūrayitvā mukhe YCM~] pūrayen mukhato HRĀ, pūrayitvā tato YLĀ, YBhD\sep
dṛḍham HRĀ YLĀ YBhD~] tathā YCM\sep
nibhṛtya HRĀ~] niṣpīḍya HRĀ\vl, niḥpīḍya YLĀ YCM, niḥkṣipya YBhD
\end{variants}

% \begin{versinnote}
% \tl{pūrayen mukhato vāyuṃ hṛdaye cibukaṃ dṛḍham/\\+}
% \tl{nibhṛtya yonim ākuñcya mano madhye niyojayet//\\!}
% \end{versinnote}
% \begin{appinnote}
% \tl{\textbf{c} nibhṛtya~] niṣpīḍya \vl \\!}
% \end{appinnote}

% \emph{Yogalakṣanāvalī} f.\,31v (\attr to the \emph{Haṭhapradīpikā})
% \begin{versinnote}
% \tl{pūrayitvā tato vāyuṃ hṛdaye cibukaṃ dṛḍham/\\+}
% \tl{niḥpīḍya yonim ākuṃcya mano madhye niyojayet//\\!}
% \end{versinnote}

% \emph{Yogacintāmaṇi} f.\,73v (\attr to the \emph{Haṭhapradīpikā})
% \begin{versinnote}
% \tl{pūrayitvā mukhe vāyuṃ hṛdaye cibukaṃ tathā/\\+}
% \tl{niḥpīḍya yonim ākuñcya mano madhye niyojayet// \\!}
% \end{versinnote}

% \emph{Yuktabhavadeva} 7.190 (\attr to the \emph{Haṭhapradīpikā})
% \begin{versinnote}
% \tl{pūrayitvā tato vāyuṃ hṛdaye cibukaṃ dṛḍham/\\+}
% \tl{niḥkṣipya yonim ākuñcya mano madhye niyojayet//\\!}
% \end{versinnote}
\end{testimonia}
%</ts20>

%<*cm20>
\begin{philcomm}[hp03_020]
The referent of \emph{madhye} is uncertain. The verse is derived from the \emph{Amaraugha}, and the \emph{Amṛtasiddhi} makes no mention of a place to focus the mind in its treatment of \emph{mahābandha} (it does however instruct the yogi to place the mind at the \emph{catuṣpatha} in its teachings on \emph{mahāmudrā}). Bhavadevamiśra (7.196), Brahmānanda (3.20) and Bālakṛṣṇa (5.24) take it to mean the central channel. It could also plausibly mean the region between the chest and perineum, or perhaps the place between the eyebrows. At 3.23 this practice is said to make the mind reach Kedāra, which is sometimes located between the eyebrows (see Mallinson 2007: 214 n.\,285; Birch 2019: 967 n.\,57).
\end{philcomm}
%</cm20>

%%%%%%%%%%
\subsection*{3.21}
% recayec ca śanair eva mahābandho 'yam ucyate/
%<*tr21>
\begin{translation}[hp03_021]
And he should exhale very gently. This is called the Great Lock.
\end{translation}
%</tr21>

%<*ts21>
\begin{testimonia}[hp03_021]
\emph{Haṭharatnāvalī} 2.44cd
\begin{variants}
eva~] evaṃ HRĀ    
\end{variants}
% \begin{versinnote}
% \tl{recayec ca śanair evaṃ mahābandho 'yam ucyate//\\!}
% \end{versinnote}
\end{testimonia}
%</ts21>

%<*cm21>
\begin{philcomm}[hp03_021]
This portion of the description of \emph{mahābandha} has undergone various revisions in the manuscript groups. The shortest version appears in \alphaOne\ and \alphaTwo, as well as the \textdelta\ group and the \emph{Haṭha\-ratnāvalī}, all of which omit verses 3.21*2 and 3.22*1cd. We have adopted the \textalpha\ version, as it seems that 3.21--22 were reworked in a major redaction of the whole work (for further discussion, see the introductory chapter by Mitsuyo Demoto).\lb%?? add intro page ref?
%
% JB check note. I've reluctantly had a go at rewriting this note (as I think JM has lots of other things to do ;). I've commented out the note to 3.21*1 as it no longer seems necessary now that this verse is grey-scaled. JM: "it seems that 3.21--22 were reworked in other manuscript groups" > "it seems that 3.21--22 were reworked and added to in other manuscript groups"?
% omission results in the somewhat odd juxtaposition of two lines stating that this is \emph{mahābandha}. 

%MD: 3.21 is not really omitted [JM:  it's not in alpha is it? MD: It seems to be a confusion due to the different numberings. According to the lastest version, 3.21*2 (matam atra) and 3.22*1cd (kālapāśa) are not included in the Alpha recension.]. In these manuscripts the first line reads "recayec ca śanair evaṃ mahābandho 'yam ucyate" and the second line comes later as in the citation in HR. Only the delta group omits 3.21cd.    
\end{philcomm}
%</cm21>
%%%%%%%%%%
\subsection*{3.21*1}
% dhārayitvā yathāśaktyā recayed anilaṃ śanaiḥ/
%savyāṅge ca samabhyasya dakṣāṅge ca samabhyaset//
%<*tr21-1>
\begin{translation}[hp03_021_1]
[The yogi] should hold the breath as long as possible and exhale slowly. And having practised it on the left side, he should practise it on the right side. 
\end{translation}
%</tr21-1>

%<*sc21-1>
\begin{sources}[hp03_021_1]
\emph{Dattātreyayogaśāstra} 62cd (\emph{padmāsana}), 134cd (\emph{mahāmudrā})
\begin{variants}
yathāśaktyā~] yathāśakti DYŚ\sep
anilaṃ~] iḍayā DYŚ\sep
savyāṅge ca~] vāmāṅgena DYŚ\sep
dakṣiṇāṅge ca~] dakṣiṇāṅgena DYŚ
\end{variants}

% \begin{versinnote}
% \tl{dhārayitvā yathāśakti recayed iḍayā śanaiḥ//\\+}
% \tl{\texteng{...}\\+}
% \tl{vāmāṅgena samabhyasya dakṣiṇāṅgena cābhyaset//\\!}
% \end{versinnote}
\end{sources}
%</sc21-1>

%<*ts21-1>
\begin{testimonia}[hp03_021_1]
\emph{Yuktabhavadeva} 7.192ab (\attr HP)
\begin{variants}
yathāśaktyā~] yathāśakti YBhD\sep
śanaiḥ~] sudhīḥ YBhD
\end{variants}

% \begin{versinnote}
% \tl{dhārayitvā yathāśakti recayed anilaṃ sudhīḥ/\\!}
% \end{versinnote}
\end{testimonia}
%</ts21-1>

%<*cm21-1>
%\begin{philcomm}[hp03_021_1] 
%Verse 3.21*1 is in the \textgamma\ group and some manuscripts of the \texteta\ and \textpi\ groups. It is a composite verse consisting of two lines from the \emph{Dattātreyayogaśāstra} (62cd, 134cd). The first line is taken from that text's description of \emph{padmāsana} and may have been included by Svātmārāma in his description of \emph{mahābandha} to make it clear that the breath is held for as long as possible after the inhalation. 
%MD: Or by the revisor. This component is not mentioned in the Amaraugha either.
%The second line is from a passage on \emph{mahāmudrā}, and it echoes a similar statement on \emph{mahāmudrā} in \emph{Haṭhapradīpikā} 3.15. It also occurs in the \textalpha\ group after 3.22.\lb
%\end{philcomm}
%</cm21-1>

%%%%%%%%%%
\subsection*{3.21*2}
%<*tr21-2>
\begin{translation}[hp03_021_2]
With regard to this [practice] some are of the opinion that [the yogi] should leave out the throat lock, saying that he should lift up the opening at the uvula with the tongue instead.
\end{translation}
%</tr21-2>

%<*sc21-2>
\begin{sources}[hp03_021_2]
Cf.\,\emph{Vivekamārtaṇḍa} 126ab
\begin{versinnote}
\tl{saṃpīḍya rasanāgreṇa rājadantabilaṃ mahat/\\!}
\end{versinnote}
Cf.\,\emph{Dattātreyayogaśāstra} 36
\begin{versinnote}
\tl{nāsāgre vinyased rājadantamūlaṃ ca jihvayā/\\+}
\tl{uttabhya cibukaṃ vakṣasy āsthāpya pavanaṃ śanaiḥ//\\!}
\end{versinnote}
\end{sources}
%</sc21-2>

%<*ts21-2>
\begin{testimonia}[hp03_021_2]
\emph{Yogacintāmaṇi} f.\,73r (\attr Īśvara), \emph{Yuktabhavadeva} 7.192cd (\attr HP)% not in the \emph{Yogacintāmaṇi} spreadsheet, but in Muktabodha etext (in a passage attributed to īśvara)
\begin{variants}
visarjayet~] vivarjayet YCM\sep
rājadantabilaṃ~] rājadantadvayaṃ YCM YBhD\sep
jihvayottambhayed YCM~] jihvayonnamayed YBhD
\end{variants}


% \begin{versinnote}
% \tl{matam atra tu keṣāṃ cit kaṇṭhabandhaṃ vivarjayet/\\+}
% \tl{rājadantadvayaṃ tatra jihvayottambhayed iti//\\!}
% \end{versinnote}

% \emph{Yuktabhavadeva} 7.192cd (\attr to the \emph{Haṭhapradīpikā})
% \begin{versinnote}
% \tl{rājadantadvayaṃ tatra jihvayonnamayed iti//\\!}
% \end{versinnote}
\end{testimonia}
%</ts21-2>

%<*cm21-2>
\begin{philcomm}[hp03_021_2]% rewrite [MD: This verse is now regarded as an interpolation. Explain why it was added?]
This verse expresses an alternative to the application of the chin lock in \emph{mahā\-bandha} mentioned in 3.20. It is found in all groups except \alphaOne\ and \alphaTwo, and the \textdelta\ group. %At present we cannot be certain if this verse was written by Svātmārāma because it does not reflect textual teachings on \emph{mahābandha} in so far as no other text known to us advocates the use of the tongue rather than the chin lock in \emph{mahābandha}. If the verse was composed by Svātmārāma, it was omitted early in the transmission by someone who did not agree with the alternative teaching.
\lb

We do not find the idea of lifting up the \emph{rājadantabila} with the tongue in other works, but \emph{Vivekamārtaṇḍa} 126ab instructs the yogi to press it with the tip of the tongue and \emph{Dattātreyayogaśāstra} 36 (found at \emph{Haṭhapradīpikā} 1.46) instructs the yogi in \emph{padmāsana} to lift up the `root of the uvula' (\emph{rājadantamūla}) with the tongue.\lb

This verse occurs in the \emph{Yogacintāmaṇi} in the middle of a quotation attributed to Īśvara. The other verses of the quoted passage are found in the \emph{Śivasaṃhitā} (4.37–42), but the verse in question is not reported in the critical edition of the \emph{Śivasaṃhitā} (2009). The verse is absent in another passage on \emph{mahābandha} that the author of the \emph{Yogacintāmaṇi} cites and attributes to the \emph{Haṭhapradīpikā}.\lb
\end{philcomm}
%</cm21-2>

%%%%%%%%%%
\subsection*{3.22}
% amuṃ yogī mahābandhaṃ mahāsiddhipradāyakam/
% savyāṅge ca samabhyasya dakṣāṅge ca samabhyaset//
%<*tr22>
\begin{translation}[hp03_022]
After practising this Great Lock, which bestows great success, on the left side of the body, the yogi should practise it on the right side of the body.
\end{translation}
%</tr22>

%<*ts22>
\begin{testimonia}[hp03_022]
\emph{Haṭharatnāvalī} 2.45
\begin{variants}
amuṃ yogī mahābandhaṃ mahāsiddhipradāyakam~] ayaṃ yogo mahābandhas sarvasiddhi\-pra\-dā\-ya\-kaḥ HRĀ    
\end{variants}


% \begin{versinnote}
% \tl{ayaṃ yogo mahābandhas sarvasiddhipradāyakaḥ/\\+} 
% \tl{savyāṅge ca samabhyasya dakṣiṇāṅge samabhyaset//\\!}
% %MD: The Berlin ms reads: ayaṃ yo + + + bandhaṃ sarvasiddhipradāyakaṃ.
% % Prāṇatoṣiṇī has: amuṃ yogī mahābandhaṃ mahāsiddhipradāyakam
% \end{versinnote}    
\end{testimonia}
%</ts22>

%<*cm22>
%\begin{philcomm}[hp03_022]
    
%\end{philcomm}
%</cm22>

%%%%%%%%%%
\subsection*{3.22*1}
%<*tr22-1>
\begin{translation}[hp03_022_1]
This is truly the great lock: it bestows the great \emph{siddhi} [and] is adept at loosening the great bond (\emph{mahābandha}) that is the noose of time.
\end{translation}
%</tr22-1>%in edition add "is" before "included" after 22*1 [MD: done]

%<*sc22-1>
%\begin{sources}[hp03_022_1]
%\end{sources}
%</sc22-1>

%<*ts22-1>
\begin{testimonia}[hp03_022_1]
\emph{Yogacintāmaṇi} f.\,73v (\attr HP), \emph{Yuktabhavadeva} 7.193 (\attr HP)
\mylb
% \begin{versinnote}
% \tl{ayaṃ khalu mahābandho mahāsiddhipradāyakaḥ/\\+}
% \tl{kālapāśamahābandhavimocanavicakṣaṇaḥ//\\!}
% \end{versinnote}

% \emph{Yuktabhavadeva} 7.193 (\attr to the \emph{Haṭhapradīpikā})
% \begin{versinnote}
% \tl{ayaṃ khalu mahābandho mahāsiddhipradāyakaḥ/\\+}
% \tl{kālapāśamahābandhavimocanavicakṣaṇaḥ//\\!}
% \end{versinnote}
\end{testimonia}
%</ts22-1>

%<*cm22-1>
%\begin{philcomm}[hp03_022_1]
%J5 reading for this passage is different from alpha1 so should be reported.
%MD: Read ayaṃ yogo mahābandho instead of ayaṃ khalu ...? Alpha has yoga (N3, acc.)/yogī (J5)/yogo (G4).
%JM: Alpha yogamahābandhaṃ works. Adopt alpha one alternative verse?
%(this was above) Adopt yogamahābandhaṃ in 3.23a (and report alpha readings). JM: khalu seems ok though. elsewhere it introduces a nirukti like this, and yoga is odd. I say go with khalu.
%\end{philcomm}
%</cm22-1>

%%%%%%%%%%
\subsection*{3.23}
%<*tr23>
\begin{translation}[hp03_023]
And this [lock] initiates an upward flow in all the channels. It brings about a confluence at the Triveṇī [and] causes the mind to reach Kedāra.
\end{translation}
%</tr23>

%<*sc23>
\begin{sources}[hp03_023]
\emph{Amaraugha} 25
\mylb
% \begin{versinnote}
% \tl{ayaṃ ca sarvanāḍīnām ūrdhvaṃgativibodhakaḥ/\\+}
% \tl{triveṇīsaṅgame dhatte kedāraṃ prāpayen manaḥ//\\!}
% \end{versinnote}
\end{sources}
%</sc23>

%<*ts23>
\begin{testimonia}[hp03_023]
\emph{Haṭharatnāvalī} 2.46, \emph{Yogacintāmaṇi} f.\,73v (\attr HP), \emph{Yuktabhavadeva} 7.94 (\attr HP)
\begin{variants}
ca HRĀ~] tu YCM, hi YBhD\sep
ūrdhvaṃgativibodhakaḥ YBhD~] ūrdhvagativibodhakaḥ HRĀ, ūrdhvaṃgamanarodhakaḥ YCM
\end{variants}



% \begin{versinnote}
% \tl{ayaṃ ca sarvanāḍīnām ūrdhvagativibodhakaḥ/\\+}
% \tl{triveṇīsaṅgamaṃ dhatte kedāraṃ prāpayen manaḥ//\\!}
% \end{versinnote}

% \emph{Yogacintāmaṇi} f.\,73v (\attr to the \emph{Haṭhapradīpikā})
% \begin{versinnote}
% \tl{ayaṃ tu sarvanāḍīnām ūrdhvaṃgamanarodhakaḥ/\\+}
% \tl{triveṇīsaṃgamaṃ dhatte kedāraṃ prāpayen manaḥ//\\!}
% \end{versinnote}

% \emph{Yuktabhavadeva} 7.94 (\attr to the \emph{Haṭhapradīpikā})
% \begin{versinnote}
% \tl{ayaṃ hi sarvanāḍīnām ūrdhvaṃgativibodhakaḥ/\\+}
% \tl{triveṇīsaṃgamaṃ dhatte kedāraṃ prāpayen manaḥ//\\!}
% \end{versinnote}
\end{testimonia}
%</ts23>

%<*cm23>
\begin{philcomm}[hp03_023]
The reading \emph{ūrdhvaṃgativibodhakaḥ} (`initiates an upward flow') is found in the source (\emph{Amaraugha} 25) and all \emph{Haṭhapradīpikā} witnesses except the \emph{Jyotsnā} (where the line is found earlier). While the \emph{Amaraughaprabodha} has the reading \emph{ūrdhvaṃgativiśodhanaḥ} (`purification of the upward flow'), the \emph{Amṛtasiddhi} (12.14) states that the chin-lock prevents the upward flow (\emph{ūrdhvaṃgatinirodhakaḥ}) in all the channels. The \textit{Amaraugha} is referring to the idea (likely accepted by Svātmārāma) that the root lock creates an upward flow in all the channels that prevents the elements and essences of the body from escaping (cf.\,\emph{Amṛtasiddhi} 12.8–10).\lb

Triveṇī and Kedāra are pilgrimage sites, the former at Prayāga where the Gaṅgā, Yamunā and subtle Sarasvatī meet, the latter in the Himālaya, near the source of the Gaṅgā. The bodily \emph{triveṇī} is located in the navel or heart by earlier Śaiva works (Birch 2019: 967). Here it may be the same as the \emph{trikūṭa} and located between the eyebrows (Mallinson 2007: 209 n.\,259). Brahmānanda, who does not identify a location for Triveṇī, understands Kedāra to be between the eyebrows. In the \emph{Khecarīvidyā} it is located on the back of the head above the nape of the neck (Mallinson 2007: 214 n.\,285). For other references on the location of Kedāra, see Birch 2019: 967 n.\,57.
\end{philcomm}
%</cm23>

%%%%%%%%%%
\subsection*{3.24}
%<*tr24>
\begin{translation}[hp03_024]
Like a beautiful and charming woman without a man, the great seal and the great lock are barren without the great piercing.
\end{translation}
%</tr24>

%<*sc24>
\begin{sources}[hp03_024]
\emph{Amaraugha} 26
% \begin{versinnote}
% \tl{rūpalāvaṇyasampannā yathā strī puruṣaṃ vinā/\\+}
% \tl{mahāmudrāmahābandhau niṣphalau vedhavarjitau //\\!}
% \end{versinnote}
\begin{variants}
bandhau niṣphalau vedhavarjitau A~] bandho niṣphalo vedhavarjitaḥ A\vl    
\end{variants}
% \begin{appinnote}
% \tl{\textbf{cd} °bandhau niṣphalau vedhavarjitau~] °bandho niṣphalo vedhavarjitaḥ \vl \\!}
% \end{appinnote}

Cf.\,\emph{Śivasaṃhitā} 4.47
\begin{versinnote}
\tl{mahāmudrāmahābandhau niṣphalau vedhavarjitau/\\+}
\tl{tasmād yogī prayatnena karoti tritayaṃ kramāt//\\!}
\end{versinnote}

Cf.\,\emph{Amṛtasiddhi} 13.3
\begin{versinnote}
\tl{guṇarūpavatī nārī niṣphalā puruṣaṃ vinā/\\+}
\tl{mahāmudrāmahābandhau vinā vedhena niṣphalau//\\!}
\end{versinnote}
\end{sources}
%</sc24>

%<*ts24>
\begin{testimonia}[hp03_024]
\emph{Haṭharatnāvalī} 2.47, \emph{Yogacintāmaṇi} f.\,73v (\attr HP), \emph{Yuktabhavadeva} 7.197cd, 197ab (\attr HP)
\mylb

% \begin{versinnote}
% \tl{rūpalāvaṇyasampannā yathā strī puruṣaṃ vinā/\\+}
% \tl{mahāmudrāmahābandhau niṣphalau vedhavarjitau//\\!}
% \end{versinnote}
% \begin{appinnote}
% \tl{\textbf{c} °mahābandhau~] mahābandho \vl \\!}
% \end{appinnote}

% \emph{Yogacintāmaṇi} f.\,73v (\attr to the \emph{Haṭhapradīpikā})
% \begin{versinnote}
% \tl{atha mahāvedhaḥ—\\+}
% \tl{rūpalāvaṇyasaṃpannā yathā strī puruṣaṃ vinā/\\+}
% \tl{mahāmudrāmahābandhau niṣphalau vedhavarjitau//\\!}
% \end{versinnote}

% \emph{Yuktabhavadeva} 7.197 (\attr to the \emph{Haṭhapradīpikā})
% \begin{versinnote}
% \tl{atha mahāvedhaḥ -\\+}
% \tl{mahāmudrāmahābandhau niṣphalau vedavarjitau/\\+}
% \tl{rūpalāvaṇyasampannā yathā strī puruṣaṃ vinā//\\!}
% \end{versinnote}
\end{testimonia}
%</ts24>

%<*cm24>
\begin{philcomm}[hp03_024]
This verse, which is from the \emph{Amaraugha} and similar to verses in the \emph{Amṛtasiddhi} and \emph{Śivasaṃhitā}, is stating that the great seal, lock and piercing should be practised together. This can be done as a sequence as shown in \skx{the video available in the digital edition.}{\medialink{this video}{the_three_mudrās.mp4} (© Mark Robberds).}
%  JB should we put the link to the video in the printed edition or change the last sentence to say that a video demonstrating this practice is in the digital edition? -- MD: We can use the macro \skx{text für the printed edition}{text für the digital edition} to have a different output for each edition.
% NJL: The book will last a very long time. Statistically, the link won't survive for very long. I would suggest to exclusively show it in the digital edition.  
\end{philcomm}
%</cm24>

%%%%%%%%%%
\subsection*{3.25 heading}
%<*tr25a>
\begin{translation}[hp03_025a]
Now the great piercing (\emph{mahāvedha}):
\end{translation}
%</tr25a>

%<*cm25a>
% \begin{philcomm}[hp03_025a]
% \end{philcomm}
%</cm25a>

%%%%%%%%%%
\subsection*{3.25}
%<*tr25>
\begin{translation}[hp03_025]
While in the great lock, the yogi should inhale, focus his mind and firmly block the flow of the bodily winds by means of the throat seal.
\end{translation}
%</tr25>

%<*sc25>
\begin{sources}[hp03_025]
Cf. \emph{Amaraugha} 27
\begin{versinnote}
\tl{punar āsphālayet kaṭyāṃ susthiraṃ kaṇṭhamudrayā/\\+}
\tl{vāyūnāṃ gatim ārudhya kṛtvā pūrakakumbhakau//\\!}
\end{versinnote}
\begin{appinnote}
\tl{\textbf{c} ārudhya~] āvṛtya, āśritya \vl \\!}
\end{appinnote}

Cf.\,\emph{Śivasaṃhitā}  4.43
\begin{versinnote}
\tl{mahābandhasthito yogī kukṣim āpurya vāyunā/\\+}
\tl{sphicau saṃtāpayed dhīmān vedho 'yaṃ kīrtito mayā//\\!}
\end{versinnote}
\end{sources}
%</sc25>

%<*ts25>
\begin{testimonia}[hp03_025]
\emph{Haṭharatnāvalī} 2.48, \emph{Yogacintāmaṇi} f.\,73v (\attr HP), \emph{Yuktabhavadeva} 7.198 (\attr HP)
\begin{variants}
mahābandha HRĀ YBhD~] mahābandhe YCM\sep
ekadhīḥ HRĀ~] ekadhā YCM YBhD\sep
āvṛtya YCM YBhD~] ākṛṣya HRĀ   
\end{variants}

% \begin{versinnote}
% \tl{mahābandhasthito yogī kṛtvā pūrakam ekadhīḥ/\\+}
% \tl{vāyūnāṃ gatim ākṛṣya nibhṛtaṃ kaṇṭhamudrayā//\\!}
% \end{versinnote}
% \begin{appinnote}
% \tl{\textbf{b} ekadhīḥ~] ekadhā \vl \\!}
% \end{appinnote}

% \emph{Yogacintāmaṇi} f.\,73v (\attr to the \emph{Haṭhapradīpikā})
% \begin{versinnote}
% \tl{mahābandhe sthito yogī kṛtvā pūrakam ekadhā/\\+}
% \tl{vāyūnāṃ gatim āvṛtya nibhṛtaṃ kaṇṭhamudrayā//\\!}
% \end{versinnote}

% \emph{Yuktabhavadeva} 7.198 (\attr to the \emph{Haṭhapradīpikā})
% \begin{versinnote}
% \tl{mahābandhasthito yogī kṛtvā pūrakam ekadhā/\\+}
% \tl{vāyunāṃ gatim āvṛtya nibhṛtaṃ kaṇṭhamudrayā//\\!}
% \end{versinnote}
\end{testimonia}
%</ts25>

%<*cm25>
\begin{philcomm}[hp03_025]
We are not certain of the meaning here of \emph{nibhṛtam}, which is found in all of the collated witnesses and important testimonia. We have understood it as an adverb with the meaning `firmly' rather than the more usual but here inappropriate `secretly'. In the \emph{Jyotsnā} (3.26), Brahmānanda glosses it as \emph{niścalam} (`immovably').

\end{philcomm}
%</cm25>

%%%%%%%%%%
\subsection*{3.26}
%<*tr26>
\begin{translation}[hp03_026]
With hands even on the ground, he should gently tap the buttocks [on the ground]. The breath enters the vessel of two halves and quickly flashes forth.
\end{translation}
%</tr26>

%<*sc26>
\begin{sources}[hp03_026]
\emph{Amaraugha} 28ab, 29cd, \emph{Dattātreyayogaśāstra} 136ab
\begin{variants}
samahastayugo A~] mahābandhasthito DYŚ\sep
sphicau saṃtāḍayec chanaiḥ DYŚ~] samapādayugas tathā A\sep
samākramya A\vl~] samākṛṣya A
\end{variants}

% \begin{versinnote}
% \tl{samahastayugo bhūmau samapādayugas tathā/\\+}
% \tl{[vedhayet kramayogena catuṣpīṭhaṃ tu vāyunā//\\+}
% \tl{āsphālayen mahāmeruṃ vāyuvajrāgnikoṭibhiḥ/]\\+}
% \tl{puṭadvayaṃ samākṛṣya vāyuḥ sphurati satvaram//\\!}
% \end{versinnote}
% \begin{appinnote}
% \tl{\textbf{d} samākṛṣya~] samākramya \vl  \\!}
% \end{appinnote}

% \emph{Dattātreyayogaśāstra} 136ab
% \begin{versinnote}
% \tl{mahābandhasthito bhūmau sphicau santāḍayec chanaiḥ/\\!}
% \end{versinnote}
\end{sources}
%</sc26>

%<*ts26>
\begin{testimonia}[hp03_026]
\emph{Haṭharatnāvalī} 2.51ab, \emph{Yogacintāmaṇi} f.\,73v (\attr HP), \emph{Yuktabhavedeva} 7.199 (\attr HP)
\begin{variants}
sama HRĀ YCM~] nyasta YBhD\sep
puṭadvayaṃ YCM~] jaṃghādvayaṃ YBhD\sep
samākramya YCM~] samākṛṣya YBhD\sep
satvaram YBhD~] madhyagaḥ YCM
\end{variants}


% \begin{versinnote}
% \tl{samahastayugo bhūmau sphicau saṃtāḍayec chanaiḥ/\\+}
% \tl{ayam eva mahāvedhaḥ siddhido 'bhyāsato bhavet \\!}
% \end{versinnote}

% \emph{Yogacintāmaṇi} f.\,73v (\attr to the \emph{Haṭhapradīpikā})
% \begin{versinnote}
% \tl{samahastayugo bhūmau sphijau saṃtāḍayec chanaiḥ/\\+}
% \tl{puṭadvayaṃ samākramya vāyuḥ sphurati madhyagaḥ//\\!}
% \end{versinnote}

% \emph{Yuktabhavedeva} 7.199 (\attr to the \emph{Haṭhapradīpikā})
% \begin{versinnote}
% \tl{nyastahastayugo bhūmau sphicau santāḍayec chanaiḥ/\\+}
% \tl{jaṃghādvayaṃ samākṛṣya vāyuḥ spharati satvaram//\\!}
% \end{versinnote}
\end{testimonia}
%</ts26>

%<*cm26>
\begin{philcomm}[hp03_026]
The term \emph{puṭadvaya} is referring back to the \emph{dvipuṭa} mentioned in verse 3.12. On the alchemical meaning and other interpretations by later commentators, see the note to 3.12.\lb

%The line added to this verse in the \textgamma\ group is taken from the \emph{Śivasaṃhitā} (4.42ab) in the section on \emph{mahābandha}.%put this in apparatus? [MD: done]
\end{philcomm}
%</cm26>

%%%%%%%%%%
\subsection*{3.27}
%<*tr27>
\begin{translation}[hp03_027]
And the union of the moon, sun and fire arises, which leads to immortality. The state of a dead person has arisen, so where is the fear of death?
\end{translation}
%</tr27>

%<*sc27>
\begin{sources}[hp03_027]
\emph{Amaraugha} 30
\begin{variants}
saṃbandho~] saṃbandhaṃ A, saṃbandhāj A\vl, saṃbandhā A\vl\sep
jāyate cāmṛtāya~] jānīyād amṛtāya A 
\end{variants}

% \begin{versinnote}
% \tl{somasūryāgnisaṃbandhaṃ jānīyād amṛtāya vai/\\+}
% \tl{mṛtāvasthā samutpannā tato mṛtyubhayaṃ kutaḥ//\\!}
% \end{versinnote}
% \begin{appinnote}
% \tl{\textbf{a} °saṃbandhaṃ~] °saṃbandhāj, °saṃbandhā \vl \\!}
% \end{appinnote}
\end{sources}
%</sc27>

%<*ts27>
\begin{testimonia}[hp03_027]
\emph{Yogacintāmaṇi} f.\,73v (\attr HP), \emph{Yuktabhavadeva} 7.200 (\attr HP)
\begin{variants}
saṃbandho YBhD~] sandhānaṃ YCM\sep
cāmṛtāya vai~] cāmṛtāyate YCM, cāmṛtāya ca YBhD\sep
samutpannā YCM\vl YBhD~] samutpannaṃ YCM\sep
mṛtyubhayaṃ kutaḥ YCM~] vāyuṃ virecayet YBhD
\end{variants}

% \begin{versinnote}
% \tl{somasūryāgnisandhānaṃ jāyate cāmṛtāyate/\\+}
% \tl{mṛtāvasthāsamutpannaṃ tato mṛtyubhayaṃ kutaḥ//\\!}
% \end{versinnote}
% \begin{appinnote}
% \tl{\textbf{c} °samutpannaṃ~] °samutpannā \vl \\!}
% \end{appinnote}

% \emph{Yuktabhavadeva} 7.200 (\attr to the \emph{Haṭhapradīpikā})
% \begin{versinnote}
% \tl{somasūryāgnisambandho jāyate cāmṛtāya ca/\\+}
% \tl{samutpannā mṛtāvasthā tato vāyuṃ virecayet//\\!}
% \end{versinnote}
\end{testimonia}
%</ts27>

%<*cm27>
\begin{philcomm}[hp03_027]
The \textalpha\ and \textgamma\ groups have \emph{saṃbandhāj}, which is possible but somewhat awkward (i.e.~`because of the union of the moon, sun and fire, the state of a dead person, which has arisen, leads to immortality'). The adopted reading \emph{saṃbandho}, which is supported by \zetaTwo, \emph{Yuktabhavadeva} 7.200 and \emph{Jyotsnā} 3.28, makes better sense but its meaning is not as clear as the formulation in the source text (i.e.~\emph{Amaraugha} 30).\lb

The compound \emph{mṛtāvasthā} (`the state of death') likely refers to a dead person in the sense that the yogi appears as if dead when the moon, sun and fire have united. In the version found in \emph{Jyotsnā} 3.28 and \zetaThree, \etaOne, \etaTwo\ and the \emph{Yuktabhavadeva}, the final verse quarter has been rewritten to say that the yogi then exhales the breath (\emph{tato vāyuṃ virecayet}). This implies that \emph{mṛtāvasthā} is a breath retention (\emph{kumbhaka}), which is apparent in Brahmānanda's explanation:
%?? MD: Is not mṛtāvasthā same as maraṇāvasthā in 3.12c?
% JH: read saṃbandhāj and translate: the death state that arises from the union of moon, sun and fire, leads to immortality. (i.e. still alchemical metaphor, as previous verse)
\begin{versinnote}
\tl{mṛtasya prāṇaviyuktasyāvasthā mṛtāvasthā samutpannā bhavati, iḍāpiṅgalayoḥ prāṇasañcārābhāvāt/ tatas tadanantaraṃ vāyuṃ virecayen nāsikāpuṭābhyāṃ śanais tyajet/\\!}
\end{versinnote}
\closer
\begin{quote}
The state of death that has arisen is the state of one who has died, [that is,] of one who is devoid of the breath because of the absence of movement of \emph{prāṇa} in the \emph{iḍā} and \emph{piṅgalā} channels. Then, immediately after that [state], [the yogi] exhales the breath, [that is,] he gradually releases it through the nostrils.
\end{quote}
\end{philcomm}
%</cm27>

%%%%%%%%%%
\subsection*{3.28}
%<*tr28>
\begin{translation}[hp03_028]
Through practice, this great piercing bestows the great \emph{siddhi} [and] cures wrinkles, grey hair, and trembling. It is used by the best practitioners.
\end{translation}
%</tr28>

%<*sc28>
%\begin{sources}[hp03_028]
%\end{sources}
%</sc28>

%<*ts28>
\begin{testimonia}[hp03_028]
Cf. \emph{Haṭharatnāvalī} 2.51cd
\begin{versinnote}
\tl{ayam eva mahāvedhaḥ siddhido 'bhyāsato bhavet//\\!}
\end{versinnote}

\emph{Yogacintāmaṇi} f.\,73v (\attr HP), \emph{Yuktabhavadeva} 7.201 (\attr HP)
\begin{variants}
abhyāsān YCM~] abhyasto YBhD    
\end{variants}


% \begin{versinnote}
% \tl{mahāvedho 'yam abhyāsān mahāsiddhipradāyakaḥ/\\+}
% \tl{valīpalitavepaghnaḥ sevyate sādhakottamaiḥ//\\!}
% \end{versinnote}

% \emph{Yuktabhavadeva} 7.201 (\attr to the \emph{Haṭhapradīpikā})
% \begin{versinnote}
% \tl{mahāvedho 'yam abhyasto mahāsiddhipradāyakaḥ/\\+}
% \tl{valīpalitavepaghnaḥ sevyate sādhakottamaiḥ//\\!}
% \end{versinnote}

\end{testimonia}
%</ts28>

%<*cm28>
\begin{philcomm}[hp03_028]
In the third verse quarter, the reading \emph{vedhaghnaḥ} (`cures wounds'?) found in \textalpha\ and the other groups, with the exception of \textdelta\ (\emph{vegaghnaḥ}), is odd as it does not seem related to the other two symptoms of old age (i.e.~wrinkles and grey hair) that this \emph{mudrā} can cure. Adopting Brahmānanda’s otherwise unattested reading, we have understood \emph{vegaghnaḥ} to be a mistake for \emph{vepaghnaḥ}, `cures trembling', 
%which occurs in some manuscripts in the \textpi\ (\getsiglum{J1}, \getsiglum{N12}), \textzeta\ (\getsiglum{J14}) and \textdelta\ (\getsiglum{J3}, \getsiglum{N16}, \getsiglum{N18}) groups, 
\ which occurs in some manuscripts in the \deltaOmega\ (\getsiglum{J1}, \getsiglum{J3}, \getsiglum{N16}, \getsiglum{N18}) and \textpi\ (\getsiglum{N12} contaminated) and \textzeta\ (\getsiglum{J14} contaminated) groups, % by MD
as well as the \emph{Yogacintāmaṇi}, \emph{Yuktabhavadeva}, and \emph{Jyotsnā}.
%The collocation \emph{valīpalitaroga} that we have adopted, found in \getsiglum{C8}, is quite common (e.g. \emph{Śāradātilaka} \emph{6.104, Ānandakanda} 1.7.13, .1.7.31).
%JB: I think we should adopt vepa on the strength of the marmasthāna entry (\item[valīpalitavepaghnaḥ] Ba5, J1, J3, J6, J14, N4a, N11, N12, N16, N18), and the fact that it's closer to vedha (alpha reading) than roga (which looks like an outlier in the marmasthāna data). 
% J1,J3,J6,N16,N18=6a, J14=4c, N11 =Gr2, N12 =4b,

%?? MD: Shouldn't we mention the possibility of interpreting “cures wrinkles and gray hair by [the alchemical process] vedha,” since the two preceding verses play with alchemical imagery?
% i.e., valīpalitavedhaghnaḥ -> vedhavalīpalitaghnaḥ

\end{philcomm}
%</cm28>

%%%%%%%%%%
\subsection*{3.29}
%<*tr29>
\begin{translation}[hp03_029]
This triad is a great secret that destroys old and death, increases [the body's] fire and bestows the powers beginning with minimisation.
\end{translation}
%</tr29>

%<*sc29>
\begin{sources}[hp03_029]
\emph{Amaraugha} 31
\mylb
% \begin{versinnote}
% \tl{etat trayaṃ mahāguhyaṃ jarāmṛtyuvināśanam/\\+}
% \tl{vahnivṛddhikarañ caiva aṇimādiguṇapradam//\\!}
% \end{versinnote}
% \begin{appinnote}
% \tl{\textbf{a} °guhyaṃ~] °guṇyaṃ \vl \\!}
% \end{appinnote}
\end{sources}
%</sc29>

%<*ts29>
\begin{testimonia}[hp03_029]
\emph{Haṭharatnāvalī} 2.52, \emph{Yogacintāmaṇi} f.\,73v (\attr HP), \emph{Yuktabhavadeva} 7.204 (\attr HP)
\begin{variants}
etat trayaṃ HRĀ YCM~] bandhatrayaṃ YBhD\sep 
mahāguhyaṃ HRĀ YBhD~] mahāguptaṃ YCM\sep
caiva YCM YBhD~] caiva hy HRĀ
\end{variants}


% \begin{versinnote}
% \tl{etat trayaṃ mahāguhyaṃ jarāmṛtyuvināśanam/\\+}
% \tl{vahnivṛddhikaraṃ caiva hy aṇimādiguṇapradam//\\!}
% \end{versinnote}

% \emph{Yogacintāmaṇi} f.\,73v (\attr to the \emph{Haṭhapradīpikā})
% \begin{versinnote}
% \tl{etat trayaṃ mahāguptaṃ jalāmṛtyuvināśanam/\\+}
% \tl{vahnivṛddhikaraṃ caiva aṇimādiguṇapradam//\\!}
% \end{versinnote}

% \emph{Yuktabhavadeva} 7.204 (\attr to the \emph{Haṭhapradīpikā})
% \begin{versinnote}
% \tl{bandhatrayaṃ mahāguhyaṃ jarāmṛtyuvināśanam/\\+}
% \tl{vahnivṛddhikaraṃ caiva aṇimādiguṇapradam//\\!}
% \end{versinnote}
\end{testimonia}
%</ts29>

%<*cm29>
%\begin{philcomm}[hp03_029]
%\end{philcomm}
%</cm29>

%%%%%%%%%%
\subsection*{3.30}
%<*tr30>
\begin{translation}[hp03_030]
It is practised eight times a day, every three hours. It always produces a wealth of merit and destroys an ocean of demerit.
\end{translation}
%</tr30>

%<*sc30>
\begin{sources}[hp03_030]
\emph{Amaraugha} 32
\begin{variants}
caitad A\vl~] caiva A\sep
saṃbhāra~] sañcaya A
\end{variants}
% \begin{versinnote}
% \tl{aṣṭadhā kriyate caiva yāme yāme dine dine/\\+}
% \tl{puṇyasañcayasambhāvi pāpaughabhiduraṃ sadā//\\!}
% \end{versinnote}
% \begin{appinnote}
% \tl{\textbf{a} caiva~] caitad \vl \\!}
% \end{appinnote}
\end{sources}
%</sc30>

%<*ts30>
\begin{testimonia}[hp03_030]
\emph{Haṭharatnāvalī} 2.49, \emph{Yogacintāmaṇi} f.\,73v (\attr HP), \emph{Yuktabhavadeva} 7.205 (\attr HP)
\begin{variants}
saṃbhāra YCM YBhD~] saṅghāta HRĀ\sep
caitat HRĀ~] caiva YCM YBhD\sep
saṃbhāvi~] sandhāyī HRĀ YCM YBhD
\end{variants}

% \begin{versinnote}
% \tl{aṣṭadhā kriyate caitat yāme yāme dine dine/\\+}
% \tl{puṇyasaṅghātasandhāyī pāpaughabhiduraḥ sadā//\\!}
% \end{versinnote}

% \emph{Yogacintāmaṇi} f.\,73v (\attr to the \emph{Haṭhapradīpikā})
% \begin{versinnote}
% \tl{aṣṭadhā kriyate caiva yāme yāme dine dine/\\+}
% \tl{puṇyasaṃbhārasandhāyi pāpaughabhiduraṃ sadā//\\!}
% \end{versinnote}

% \emph{Yuktabhavadeva} 7.205 (\attr to the \emph{Haṭhapradīpikā})
% \begin{versinnote}
% \tl{aṣṭadhā kriyate caiva yāme yāme dine dine/\\+}
% \tl{puṇyasambhārasandhāyi pāpaughabhiduraṃ sadā//\\!}
% \end{versinnote}
\end{testimonia}
%</ts30>

%<*cm30>
%\begin{philcomm}[hp03_030]
%JB: It seems to make better sense to read this as saying that the yogi practises the three seals `eight times, every three hours, every day.' This would mean the triad is practised sixty-four times in total a day (commesurate with the eighty kumbhakas per day). not accepted at the workshop by Haru and Lubomir. (In a post reading session discussion, Haru thinks the interpretation of 64 times a day is possible)
%\end{philcomm}
%</cm30>

%%%%%%%%%%
\subsection*{3.31}
%<*tr31>
\begin{translation}[hp03_031]
It is only for those who have received proper instruction. It is taught that they should [do it] a little in the first stage of the practice [and] that at the beginning they should avoid frequenting fire, women and roads.
\end{translation}
%</tr31>

%<*sc31>
\begin{sources}[hp03_031]
\emph{Amaraugha} 33
\begin{variants}
ādiśet~] ācaret A
\end{variants}

% \begin{versinnote}
% \tl{samyakśikṣāvatām eva svalpaṃ prathamasādhane/\\+}
% \tl{vahnistrīpathasevānām ādau varjanam ācaret//\\!}
% \end{versinnote}
\end{sources}
%</sc31>

%<*ts31>
\begin{testimonia}[hp03_031]
\emph{Yogacintāmaṇi} f.\,73v (ab only) (\attr HP), \emph{Yuktabhavadeva} 7.205 (\attr HP)
\begin{variants}
sādhane YCM~] sādhanam YBhD\sep 
ādiśet~] ācaret YBhD
\end{variants}


% to the \emph{Haṭhapradīpikā})
% \begin{versinnote}
% \tl{samyakśikṣāvatām eva svalpaṃ prathamasādhane//\\!}
% \end{versinnote}

% \emph{Yuktabhavadeva} 7.205 (\attr to the \emph{Haṭhapradīpikā})
% \begin{versinnote}
% \tl{samyak śikṣāvatām evaṃ svalpaṃ prathamasādhanam/\\+}
% \tl{vahnistrīpathisevānām ādau varjanam ācaret//\\!}
% \end{versinnote}

\end{testimonia}
%</ts31>

%<*cm31>
%\begin{philcomm}[hp03_031]
%\end{philcomm}
%</cm31>

%%%%%%%%%%
\subsection*{3.32 heading}
%<*tr32a>
\begin{translation}[hp03_032a]
Now, the sky-roving [seal] (\emph{khecarī}):
\end{translation}
%</tr32a>

%<*cm32a>
% \begin{philcomm}[hp03_032a]
% \end{philcomm}
%</cm32a>

%%%%%%%%%%
\subsection*{3.32}
%<*tr32>
\begin{translation}[hp03_032]
By cutting, moving, and milking, [the yogi] should gradually lengthen the tongue until it touches the middle of the brows. Then the sky-roving [seal] is perfected.% in ed break up yāvad bhrū°, conjunct is hard to read?
\end{translation}
%</tr32>

%<*sc32>
%\begin{sources}[hp03_032]
%\end{sources}
%</sc32>

%<*ts32>
\begin{testimonia}[hp03_032]
\emph{Haṭharatnāvalī} 2.141 (\attr HP), \emph{Yogacintāmaṇi} f.\,74r (\attr HP), \emph{Haṭhatattvakaumudī} 14.18 (\attr HP) 
\begin{variants}
kalāṃ krameṇa pravardhayet~] kalāṃ krameṇa vardhayet HRĀ, jihvāṃ saṃvardhayet YCM, krameṇa jihvāṃ pravardhayet HTK\sep
sā yāvad bhrūmadhyaṃ spṛśati HTK~] yāvad iyaṃ bhrūmadhye spṛśati HRĀ, sā yāti yāvad bhrūmadhyaṃ spṛśati hi HRĀ\vl, yāvad iyaṃ bhrūmadhyaṃ spṛśati YCM\sep
tadā khecarīsiddhiḥ YCM HTK~] tadānīṃ khecarīsiddhiḥ HRĀ
\end{variants}

% \begin{versinnote}
% \tl{haṭhapradīpikāyām//\\+}
% \tl{chedanacālanadohaiḥ kalāṃ krameṇa vardhayet tāvat/\\+}
% \tl{yāvad iyaṃ bhrūmadhye spṛśati tadānīṃ khecarīsiddhiḥ//\\!}
% \end{versinnote}
% \begin{appinnote}
% \tl{yāvad iyaṃ bhrūmadhye spṛśati~] sā yāti yāvad bhrūmadhyaṃ spṛśati hi \vl \\!}
% \end{appinnote}

% \emph{Yogacintāmaṇi} f.\,74r
% \begin{versinnote}
% \tl{haṭhapradīpikāyām—\\+}
% \tl{chedanacālanadohair jihvāṃ saṃvardhayet tāvat/\\+}
% \tl{yāvad iyaṃ bhrūmadhyaṃ spṛśati tadā khecarīsiddhiḥ//\\!}
% \end{versinnote}

% \emph{Haṭhatattvakaumudī} 14.18
% \begin{versinnote}
% \tl{haṭhapradīpikāmate tu –\\+}
% \tl{chedanacālanadohaiḥ krameṇa jihvāṃ pravardhayet tāvat/\\+}
% \tl{sā yāvad bhrūmadhyaṃ spṛśati tadā khecarīsiddhiḥ// \\!}
% \end{versinnote}
\end{testimonia}
%</ts32>

%<*cm32>
\begin{philcomm}[hp03_032]
%There are no \emph{āryā} or \emph{gīti} metres elsewhere in the text, but there are verses  in \emph{upagīti} metre (e.g. 1.60). We have therefore adopted variants that fit the \emph{upagīti} metre. Perhaps this is authorial, as we don't know a source for it, and we don't know a source for 1.60. % JB: this is a weak reason given the readings of alpha!

Various versions of this verse have been transmitted in \emph{upagīti} (\textgamma, \textdelta), \emph{gīti} (\textepsilon), \emph{āryā} (\etaTwo) and \emph{anuṣṭubh} (\piOmega) metres. We have adopted a version close to \alphaTwo, which has a slight metrical fault:
\begin{versinnote}
\tl{chedanacālanaṃ dohaiḥ kalā kramaṇa pravardhayet tāvat/\\+}
\tl{sā yāvad bhrūmadhyaṃ visati tadānī khecarīsiddhiḥ//\\!}
\end{versinnote}
% alpha2
% chedanacālanaṃ dohaiḥ kalā kramaṇa pravardhaye tāvat
% sā yāvad bhrumadhyaṃ visati tadānī khecarīsiddhiḥ
The emendation of \emph{tadānīṃ} to \emph{tadā} renders the verse an \emph{āryā}. The word \emph{krameṇa} is well attested by manuscripts of the \textalpha, \textzeta, \texteta\ and \textpi\ groups, which all have unmetrical or corrupted versions.\lb 
% krameṇa is well attested
% close to alpha 2
% damaged in alpha 1: [damaged]ḥ kalāḥ krameṇa varddayet/ sā yāva[d] bhṛmadhyaṃ viśa tadānīṃ khecarīsiddhiḥ// but it looks like someone has unsuccessfully tried to make it anuṣṭubh. Perhaps, suggests that a longer metre was original.

It is possible that this verse was originally composed in the \emph{upagīti} metre, as three other verses in the \emph{Haṭhapradīpikā} (i.e., 1.60, 4.51 and 4.55), likely composed by the author, are in this metre. If this were the case, the verse may have read \emph{kramād vardhayet}, which would scan correctly as \emph{upagīti}. However, this reading is not attested by the manuscripts or testimonia that we have consulted.\lb   
% MD: Could we add a note about another possible reconstruction of the first line, "chedanacālanadohaiḥ, kalāṃ kramād vardhayet tāvat", which fits the upagīti metre? I believe that this verse, along with 1.40, 4.51 and 4.55, was composed in upagīti by the author himself. The redactor of the Gamma and Delta recensions must have thought so too, so that they reworked the verse in the upagīti metre, changing the text in another way. These four verses have been transmitted more incorrectly than the others, as this rather uncommon metre was not recognised by the scribes. 
%?? @Jim include this verse as a possible authorial verse in your intro?

The meaning of \emph{kalā} as `tongue' is not attested in any Sanskrit dictionary but \emph{kalā} occurs in the sense of the tongue in a subsequent verse of this chapter (cf.\,3.34a) and it is glossed by Brahmānanda with \emph{jihvā} in \emph{Jyotsnā} 3.33 and 3.37.
%
\end{philcomm}
%</cm32>

\begin{metre}[hp03_032]
Āryā
\end{metre}

%%%%%%%%%%%%%%%%%%%%%%%
\subsection*{3.32*1}
%<*tr32-1>
\begin{translation}[hp03_032_1]

He should take a very sharp, well-oiled and clean blade resembling a leaf of the Snuhī plant and then cut away a hair’s breadth [of the frenum] with it.
%
\end{translation}
%</tr32-1>

%<*sc32-1>
\begin{sources}[hp03_032_1]
\emph{Khecarīvidyā} 1.46

\begin{variants}
samucchidet~] samucchinet KhV
\end{variants}
\end{sources}
%</sc32-1>

%<*ts32-1>
\begin{testimonia}[hp03_032_1]
\emph{Haṭharatnāvalī} 2.136
\begin{variants}
samādāya tatas tena~] samanāyāṃ tu jihvāyāṃ HRĀ\sep   
\end{variants}
\end{testimonia}
%</ts32-1>

%<*cm32-1>
\begin{philcomm}[hp03_032_1]
%?? JB how about the following note on snuhī.
The term \emph{snuhī} can refer to several species of \emph{Euphorbia}, which is generally known as spurge. Two common species are \emph{Euphorbia antiquorum} Linn. (`triangular' or `oleander' spurge) or \emph{neriifolia} Linn. The latter was probably not used in the practice of \emph{khecarīmudrā} as  Nadkarni (1926: 349) describes it as a `leafless shrub.' Although all varieties of \emph{Euphorbia} are poisonous, the sap, roots and bark have been used in medicines since the time of Caraka and Suśruta (Singh and Chunekar 1999: 459).
\end{philcomm}
%</cm32-1>
%%%%%

\subsection*{3.32*2}
%<*tr32-2>
\begin{translation}[hp03_032_2]
After cutting, he should rub [the cut] with a powder of rock-salt and \emph{pathyā}. After seven days he should again cut away a hair’s breadth.
%??  JB: translate pathyā as myrobalan (it's very common and so I don't think we need a note on it).

\end{translation}
%</tr32-2>

%<*sc32-2>
\begin{sources}[hp03_032_2]
\emph{Khecarīvidyā} 1.47
% ?? em ed. to match source ( = Jyotsnā) °pathyābhyāṃ cūrṇitābhyāṃ: JB agree.
\end{sources}
%</sc32-2>

%<*ts32-2>
%\begin{testimonia}[hp03_032_2]%??there are lots of empty sources and testimonia here now, remove?
%\end{testimonia} 
%</ts32-2>

%<*cm32-2>
% \begin{philcomm}[hp03_032_2]
% \end{philcomm}
%</cm32-2>
%%%%%

\subsection*{3.32*3}
%<*tr32-3>
\begin{translation}[hp03_032_3]
[The yogi], constantly applying himself, should thus practise gradually for six months. After six months the binding tendon at the base of the tongue is destroyed.
%??JB: this is one occasion where nitya might be rendered regularly? I can't beleive I just typed that ;) 
% ??JB: change edition to śirābandhaṃ (eta, omega pc). Correct spelling is sirā but I see that śirā is common elsewhere for this verse.
\end{translation}
%</tr32-3>

%<*sc32-3>
\begin{sources}[hp03_032_3]
\emph{Khecarīvidyā} 1.48
% \begin{variants}  
% \end{variants}
\end{sources}
%</sc32-3>

%<*ts32-3>
% \begin{testimonia}[hp03_032_3]
% \end{testimonia}
%</ts32-3>

%<*cm32-3>
% \begin{philcomm}[hp03_032_3]
% \end{philcomm}
%</cm32-3>
%%%%%

\subsection*{3.32*4}
%<*tr32-4>
\begin{translation}[hp03_032_4]
Then, knowing the rules of time and limit, the yogi should gradually pull upwards the tip of the tongue having wrapped it in cloth.
%?? perhaps we should translate vāgīśvarīdhāma? (tip [of the tongue], the abode of the goddess of speech, ...)
\end{translation}
%</tr32-4>

%<*sc32-4>
\begin{sources}[hp03_032_4]
\emph{Khecarīvidyā} 1.49
%compound °dhāmaśiro in edition [MD: done]
\begin{variants}  
\end{variants}
\end{sources}
%</sc32-4>

%<*ts32-4>
% \begin{testimonia}[hp03_032_4]
% \end{testimonia}
%</ts32-4>

%<*cm32-4>
% \begin{philcomm}[hp03_032_4]
% \end{philcomm}
%</cm32-4>
%%%%%

\subsection*{3.32*5}
%<*tr32-5>
\begin{translation}[hp03_032_5]
The characteristics of the wrapped cloth are taught to be that it is one handspan in length, four fingers wide, soft [and] white.

%?? JB: This verse = HP 3.95*2. i think we have a convention for referring the reader to an identical verse elsewhere in the text. This would also avoid repeating sources, testimony, etc. There we translate it as, "It is said that the characteristics of the cloth for wrapping around [the tongue] are that it is a handspan long, four fingerbreadths wide, soft and white." 
\end{translation}
%</tr32-5>

%<*sc32-5>
\begin{sources}[hp03_032_5]
\emph{Yogabīja} 92
\begin{variants}  
\end{variants}
\end{sources}
%</sc32-5>

%<*ts32-5>
% \begin{testimonia}[hp03_032_5]
% \end{testimonia}
%</ts32-5>

%<*cm32-5>
% \begin{philcomm}[hp03_032_5]
% \end{philcomm}
%</cm32-5>
%%%%%

\subsection*{3.32*6}
%<*tr32-6>
\begin{translation}[hp03_032_6]
Then, in six months, after repeated drawing out [of the tongue], my dear, it increases [in length to reach upwards] between the eyebrows, obliquely to the ears,
%?? JB: mātreṇa? Then, after repeated drawing out [of the tongue] for only six months, my dear, it extends as far as the middle of brow, across to the ears ...
%?? JB: emend punaḥ in pāda b to nitya (vis-à-vis KhV)? It looks like it's been translated and makes better sense.
\end{translation}
%</tr32-6>

%<*sc32-6>
\begin{sources}[hp03_032_6]
\emph{Khecarīvidyā} 1.50
\begin{variants}  
punaḥ~] nitya KhV\sep
vardheta~] sābhyeti KhV
\end{variants}
\end{sources}
%</sc32-6>

%<*ts32-6>
% \begin{testimonia}[hp03_032_6]
% \end{testimonia}
%</ts32-6>

%<*cm32-6>
% \begin{philcomm}[hp03_032_6]
% ?? KhV ed has nityasaṃkarṣaṇāt which works better. Adopt? JB: Sorry, just seen these comments. yes, I agree.
% ?? KhV ed has sābhyeti for vardheta, adopt? Jb: maybe rendering vardheta as 'extends' is okay?
% \end{philcomm}
%</cm32-6>
%%%%%

\subsection*{3.32*7}
%<*tr32-7>
\begin{translation}[hp03_032_7]
and downwards it is gradually made to reach the base of the chin. Upwards it easily reaches the hairline [and] sideways the temples, my dear.
\end{translation}
%</tr32-7>

%<*sc32-7>
\begin{sources}[hp03_032_7]
\emph{Khecarīvidyā} 1.51ab + 1.52ab
\begin{variants}  
adhastāc~] adhaś ca KhV
\end{variants}
\end{sources}
%</sc32-7>

%<*ts32-7>
% \begin{testimonia}[hp03_032_7]
% \end{testimonia}
%</ts32-7>

%<*cm32-7>
% \begin{philcomm}[hp03_032_7]
% % em. krośād ūrdhvaṃ ca to keśāntam ūrdhvaṃ [MD: done]
% ?? JB: em. saṃkhyā° to śaṅkhā°
% \end{philcomm}
%</cm32-7>

\begin{metre}[hp03_032_7]
Anuṣṭubh (c: bha-vipulā)
\end{metre}
%%%%%

\subsection*{3.32*8}
%<*tr32-8>
\begin{translation}[hp03_032_8]
And then, after the second year, o goddess, it easily covers the top of the aperture of Brahman, o she who is worshipped by the gods.
%?? JB: by tiṣṭhet, are we supposed to understand that the tongue can remain there? (that's pretty incredible)
%?? JB: paramavandite? I don't find this vocative elsewhere. Highly venerated? Combine with devi? (i.e., O most venerable goddess). Or emend back to tiṣṭhaty amaravandite.
\end{translation}
%</tr32-8>

%<*sc32-8>
\begin{sources}[hp03_032_8]
ab: cf. \emph{Khecarīvidyā} 1.51cd; cd: \emph{Khecarīvidyā} 1.53ab 
\begin{variants}  
saṃvatsarāṇāṃ tu tṛtīyād~] saṃvatsarāṇāṃ tu tṛtīyād KhV\sep
tiṣṭhet paramavandite~] tiṣṭhaty amaravandite KhV
\end{variants}
\end{sources}
%</sc32-8>

%<*ts32-8>
% \begin{testimonia}[hp03_032_8]
% \end{testimonia}
%</ts32-8>

%<*cm32-8>
% \begin{philcomm}[hp03_032_8]
% % em. to dvitīyāc [MD: done]
% \end{philcomm}
%</cm32-8>
%%%%%

\subsection*{3.32*9}
%<*tr32-9>
\begin{translation}[hp03_032_9]
In the manner described by his guru, [every day] for seven days the knower of \emph{ātman} should rub the base of his palate and remove all impurity.
\end{translation}
%</tr32-9>

%<*sc32-9>
\begin{sources}[hp03_032_9]
\emph{Khecarīvidyā} 1.45
\begin{variants}  
svatālumūlaṃ saṃghṛṣya~] tālumūlaṃ samutkṛṣya KhV
\end{variants}
\end{sources}
%</sc32-9>

%<*ts32-9>
% \begin{testimonia}[hp03_032_9]
% \end{testimonia}
%</ts32-9>

%<*cm32-9>
% \begin{philcomm}[hp03_032_9]
% % em ātmani to ātmavit (as in KhV ed) [MD: done]
% \end{philcomm}
%</cm32-9>

\begin{metre}[hp03_032_9]
Anuṣṭubh (a: ma-vipulā)
\end{metre}

%%%%%

\subsection*{3.32*10}
%<*tr32-10>
\begin{translation}[hp03_032_10]
After rubbing there with the tip of his finger, he should insert his tongue. Very gradually it breaks the great adamantine doorway out of the head.
% ?? JB: mastakāc ca?  HP 10 Ch has mastakastha (probably a patch)
\end{translation}
%</tr32-10>

%<*sc32-10>
\begin{sources}[hp03_032_10]
ab: \emph{Khecarīvidyā} 1.56cd\sep
cd: cf. \emph{Khecarīvidyā} 1.33cd
\begin{versinnote}
% \tl{samyak śikṣāvatām evaṃ svalpaṃ prathamasādhanam/\\+}
 \tl{mastakākhyā mahācaṇḍā śikhivahnikavajrabhṛt//\\!}
\end{versinnote}
\end{sources}
%</sc32-10>

%<*ts32-10>
% \begin{testimonia}[hp03_032_10]
% \end{testimonia}
%</ts32-10>

%<*cm32-10>
% \begin{philcomm}[hp03_032_10]
% \end{philcomm}
%</cm32-10>

\begin{metre}[hp03_032_10]
Anuṣṭubh (c: ra-vipulā)
\end{metre}
%%%%%

\subsection*{3.32*11}
%<*tr32-11>
\begin{translation}[hp03_032_11]
The yogi should practise the Vidyā that is extremely hard to obtain joined with the previously described seed syllable [and] its six limbs with it divided according to the six cakras.
\end{translation}
%</tr32-11>
% ?? NJL: Vidyā? 
%?? JB: vyākhyātām? [The yogi] should practise the [previously] mentioned Vidyā ... It is difficult to translate with the same syntax. Maybe, ignore the Sanskrit syntax and translate ab and cd separately (like the KhV)? Maybe add a note on what the vidyā and its seed syllables in KhV, or a reference to the relevant note(s) your book. 


%<*sc32-11>
\begin{sources}[hp03_032_11]
Cf. \emph{Khecarīvidyā} 1.34
\begin{versinnote}
 \tl{pūrvabījayutā vidyā vyākhyātā hy atidurlabhā/\\+}
 \tl{ṣaḍaṅgavidyāṃ vakṣyāmi tayā ṣaṭsvarabhinnayā//\\!}
\end{versinnote}
\end{sources}

%</sc32-11>

%<*ts32-11>
% \begin{testimonia}[hp03_032_11]
% \end{testimonia}
%</ts32-11>

%<*cm32-11>
\begin{philcomm}[hp03_032_11]
This verse is a reworking (or corruption) of \emph{Khecarīvidyā} 1.34 which is difficult to make sense of.%?? Ah just seen this. Agree!
\end{philcomm}
%</cm32-11>

\begin{metre}[hp03_032_11]
Anuṣṭubh (c: ma-vipulā)
\end{metre}
%%%%%

\subsection*{3.32*12}
%<*tr32-12>
\begin{translation}[hp03_032_12]
The mind which moves in the inert and eternally blessed void is the Khecarī [mind], the bringer of union with Śivahood [and] the remover of the suffering of existence.
\end{translation}
%?? JB: nirastasakalakriyākrame? (...void in which the course of all activities has ceased). Wouldn't śāśvatodaya mean something like 'eternally arising'? And just wondering if it necessary to supply [mind]?
%</tr32-12>

%<*sc32-12>
% \begin{sources}[hp03_032_12]
% \end{sources}
%</sc32-12>

% %<*ts32-12>
% \begin{testimonia}[hp03_032_12]
% \end{testimonia}
%</ts32-12>

%<*cm32-12>
% \begin{philcomm}[hp03_032_12]
% \end{philcomm}
%</cm32-12>

\begin{metre}[hp03_032_12]
Rathoddhatā
\end{metre}
%%%%%

\subsection*{3.32*13}
%<*tr32-13>
\begin{translation}[hp03_032_13]
It is to be brought about very gradually, through practice, o beautiful lady. The body of him who strives [for it] all at once is destroyed.
\end{translation}
%</tr32-13>

%<*sc32-13>
\begin{sources}[hp03_032_13]
\emph{Khecarīvidyā} 1.54
\begin{variants}  
krameṇaiva~] śanair eva KhV\sep
varavarṇini~] yugapan na hi KhV\sep
yatate~] yaś caret KhV
\end{variants}
\end{sources}
%</sc32-13>
%1.54ab śanair eva prakartavyam abhyāsaṃ yugapan na hi
%1.54cd yugapad yaś caret tasya śarīraṃ vilayaṃ vrajet

%<*ts32-13>
% \begin{testimonia}[hp03_032_13]
% \end{testimonia}
%</ts32-13>

%<*cm32-13>
% \begin{philcomm}[hp03_032_13]
% \end{philcomm}
%</cm32-13>
%%%%%

\subsection*{3.32*14}
%<*tr32-14>
\begin{translation}[hp03_032_14]
Therefore [its] practice is to be done very gradually, not all at once, my dear. After practising in this way for three years, one is sure to enter the door of Brahman.
\end{translation}
%</tr32-14>

%<*sc32-14>
\begin{sources}[hp03_032_14]
\emph{Khecarīvidyā} 1.55ab + 1.57ab
\begin{variants}  
kāryo 'bhyāso na yugapat priye~] kāryam abhyāsaṃ varavarṇini KhV\sep
viśet dhruvam~] praviśyati KhV
\end{variants}
\end{sources}
%</sc32-14>

%<*ts32-14>
% \begin{testimonia}[hp03_032_14]
% \end{testimonia}
%</ts32-14>

%<*cm32-14>
% \begin{philcomm}[hp03_032_14]
% \end{philcomm}
%</cm32-14>
%%%%%

\subsection*{3.32*15}
%<*tr32-15>
\begin{translation}[hp03_032_15]
He who pierces the six \emph{cakra}s, wakes up the snake-goddess situated at the Base, breaks through the great triad of knots which are like a defensive wall for the rear channel, then leads the breath to the opening in the head, churns that \emph{liṅga} sufficiently with his mind and drinks [the \emph{amṛta}] dripping from the orb of the moon, is liberated, Śiva himself in person.
\end{translation}
%?? JB [The yogi] who pierces
%?? JB: protthāpya would mean raise, no? I'm not sure that 'rear channel' will convey to the reader that the central channel is intended. I don't understand what tat in pāda c is supposed to convey. Does it have a referent? Maybe it made sense in the source work of this verse (note?)
%</tr32-15>

%<*sc32-15>
% \begin{sources}[hp03_032_15]
% \end{sources}
%</sc32-15>

%<*ts32-15>
% \begin{testimonia}[hp03_032_15]
% \end{testimonia}
%</ts32-15>

%<*cm32-15>
% \begin{philcomm}[hp03_032_15]
% %change sat to ṣaṭ [MD: done]
% \end{philcomm}
%</cm32-15>

\begin{metre}[hp03_032_15]
Śārdūlavikrīḍita
\end{metre}
%%%%%

\subsection*{3.32*16}
%<*tr32-16>
\begin{translation}[hp03_032_16]
If a man has his tongue constantly up [in the aperture above the palate] and drinks the stream of \emph{amṛta} with its seven flows, which is delicious, cool, removes trouble and danger [and] wards off hunger and thirst, then steadiness of the body arises, \crux death, disease and misfortune disappear ... death turns around and goes away\crux.
\end{translation}
%?? JB: do we know what the saptadhārā is? AŚ has vaktradhārā (drinks the stream of nectar flowing [down] to his mouth?).
%?? JB: I don't understand śītalāṃgaṃ (whose essential part is cool?)
%?? maybe read bata mahāmṛtyurogā dravante instead of mṛtapathā mṛtyurogād bhavanti. At least the former makes some sense (not sure about bata though). Can't see a remedy for prasarati sakalaṃ yāti kālaṃ bhramitvā. I see the 10-ch HP has kālo, which provides a subject. But what to do with sakalaṃ and the AŚ's harati viṣajarāṃ?
%</tr32-16>

%<*sc32-16>
\begin{sources}[hp03_032_16]
\emph{Amaraughaśāsana} 3.1--2
% nṛtyan nityordhvajihvo yadi pibati pumān vaktradhārāmṛtaughaṃ 
% susvādaṃ śītalāṅgaṃ duritabhayaharaṃ kṣutpipāsāvināśi |
% piṇḍasthairyaṃ yad asmād bhavati bata mahāmṛtyurogā dravante
% daurbhāgyaṃ yāti nāśaṃ harati viṣajarāṃ yāti kāle bhramitvā || 
\end{sources}
%</sc32-16>

%<*ts32-16>
% \begin{testimonia}[hp03_032_16]
% \end{testimonia}
%</ts32-16>

%<*cm32-16>
\begin{philcomm}[hp03_032_16]
The second half of this verse, which is found in the published 10-chapter \emph{Haṭhapradīpikā} (5.51) is corrupt.
\end{philcomm}
%</cm32-16>

\begin{metre}[hp03_032_16]
Sragdharā
\end{metre}
%%%%%

\subsection*{3.32*17}
%<*tr32-17>
\begin{translation}[hp03_032_17]
[If the taste is] sharp it removes disease; bitter it gets rid of skin problems; and [if it is] like ghee the yogi is sure to attain immortality.
%?? cf. Jnānasāra 2.10cd: ghṛtāsvādūpamānāś ca hy amaratvaṃ na saṃśayaḥ
\end{translation}
%</tr32-17>

%<*sc32-17>
% \begin{sources}[hp03_032_17]
% \end{sources}
%</sc32-17>

%<*ts32-17>
% \begin{testimonia}[hp03_032_17]
% \end{testimonia}
%</ts32-17>

%<*cm32-17>
% \begin{philcomm}[hp03_032_17]
% \end{philcomm}
%</cm32-17>
%%%%%

\subsection*{3.32*18}
%<*tr32-18>
\begin{translation}[hp03_032_18]
And [if it is] like honey, he can recite lots of scriptures. \crux Sweetmeats and sugary morsels, lots of cooked food\crux.
%?? JB: ab is parallel to Jñānasāra 2.10cd madhusvādūpamānāś ca śāstrotgīraṇatā bhavet || 
%?? JB: cf. JS 2.11:  mṛṣṭānikhaṇḍakadyāni laḍukāśokavartikāḥ| evaṃ vārāhy aneke ca kāmadevo vyavasthitaḥ (I don't understand the syntax (aiśa?) but perhaps the idea is that if the nectar is similar to any of these foods, one exits as a Kāmadeva). Maybe the second hemistich of the JS dropped out of the HP transmission (note?).

\end{translation}
%</tr32-18>

%<*sc32-18>
\begin{sources}[hp03_032_18]
\end{sources}
%</sc32-18>

%<*ts32-18>
\begin{testimonia}[hp03_032_18]
\end{testimonia}
%</ts32-18>

%<*cm32-18>
\begin{philcomm}[hp03_032_18]
%Em. pādāni to khādyāni, crux second line in edition. [MD. done]
\end{philcomm}
%</cm32-18>
%%%%%

\subsection*{3.32*19}
%<*tr32-19>
\begin{translation}[hp03_032_19]
He enjoys himself constantly for an age of the gods, is sure to be exalted, and attains identity with Brahman, like a silk worm making a cocoon.
%?? JB [Brahman]? tanmaya
\end{translation}
%</tr32-19>

%<*sc32-19>
% \begin{sources}[hp03_032_19]
% \end{sources}
%</sc32-19>

%<*ts32-19>
% \begin{testimonia}[hp03_032_19]
% \end{testimonia}
%</ts32-19>

%<*cm32-19>
% \begin{philcomm}[hp03_032_19]
% \end{philcomm}
%</cm32-19>

%%%%%%%%%%
\subsection*{3.33}
%<*tr33>
\begin{translation}[hp03_033]
When the tongue is turned back and inserted into the cavity of the skull and the gaze is between the brows, the sky-roving seal arises.
\end{translation}
%</tr33>

% NJL: We translated the names of the \emph{mudrā}s elsewhere. Shouldn't we do the same in the above case? In 3.6 we have the sky-roving [seal]. 

%<*sc33>
\begin{sources}[hp03_033]
\emph{Vivekamārtaṇḍa} 47
\mylb
% \begin{versinnote}
% \tl{kapālakuhare jihvā praviṣṭā viparītagā/\\+}
% \tl{bhruvor antargatā dṛṣṭir mudrā bhavati khecarī//\\!}
% \end{versinnote}
\end{sources}
%</sc33>

%<*ts33>
\begin{testimonia}[hp03_033]
\emph{Haṭharatnāvalī} 2.138 (\attr Dattātreya), \emph{Yogacintāmaṇi} f.\,75r (\attr \emph{Skandapurāṇa}), \emph{Yuktabhavadeva} 7.207 (\attr HP)
\mylb

% \begin{versinnote}
% \tl{dattātreyas tu//\\+}
% \tl{kapālakuhare jihvā praviṣṭā viparītagā/\\+}
% \tl{bhruvor antargatā dṛṣṭir mudrā bhavati khecarī//\\!}
% \end{versinnote}

% \emph{Yogacintāmaṇi} f.\,75r
% \begin{versinnote}
% \tl{skandapurāṇe—\\+}
% \tl{kapālakuhare jihvā praviṣṭā viparītagā/\\+}
% \tl{bhruvor antargatā dṛṣṭir mudrā bhavati khecarī//\\!}
% \end{versinnote}

% \emph{Yuktabhavadeva} 7.207 (\attr to the \emph{Haṭhapradīpikā})
% \begin{versinnote}
% \tl{atha khecarī -\\+}
% \tl{kapālakuhare jihvā praviṣṭā viparītagā/\\+}
% \tl{bhruvor antargatā dṛṣṭir mudrā bhavati khecarī//\\!}
% \end{versinnote}
\end{testimonia}
%</ts33>

%<*cm33>
%\begin{philcomm}[hp03_033]
%\end{philcomm}
%</cm33>

%%%%%%%%%%
\subsection*{3.34}
%<*tr34>
\begin{translation}[hp03_034]
If the yogi turns back the tongue and remains [like that] for half an instant, he is instantly freed from disease, death, old age and the like.
\end{translation}
%</tr34>

%<*sc34>
\begin{sources}[hp03_034]
\emph{Śivasaṃhitā} 3.91
\begin{variants}
kalāṃ parāṅmukhīṃ kṛtvā~] rasanām ūrdhvagāṃ kṛtvā ŚS 
\end{variants}

Cf. \emph{Jñānasāra} 2.6 %??JB are we treating this as a source? JS 2.3–6 seems to be close to some HP verses. Also JS 2.12–15 are close to HP 2.54–56
\begin{versinnote}
\tl{rasanām ūrdhvagāṃ kṛtvā kṣaṇārdhaṃ yadi tiṣṭhati/\\+}
\tl{kṣaṇena mucyate yogī vyādhibhis tu jarādibhiḥ//\\!}
\end{versinnote}
\begin{appinnote}
\tl{kṣaṇārdhaṃ \emph{em.}~] kṣaṇādhvaṃ \emph{codex.}\\!}
\end{appinnote}
% \begin{versinnote}
% \tl{rasanām ūrdhvagāṃ kṛtvā kṣaṇārdhaṃ yadi tiṣṭhati/\\+}
% \tl{kṣaṇena mucyate yogī vyādhimṛtyujarādibhiḥ//\\!}
% \end{versinnote}
\end{sources}
%</sc34>

%<*ts34>
\begin{testimonia}[hp03_034]
\emph{Yuktabhavadeva} 7.209 (\attr HP)\lb

Cf. \emph{Yogacintāmaṇi} f.\,74r (\attr HP)
\begin{versinnote}
\tl{kalāṃ parāṅmukhīṃ kṛtvā tripathe parivartayet/\\+}
\tl{sā bhavet khecarī mudrā vyomacakraṃ tad ucyate//\\+}
\tl{rasanām ūrdhvagāṃ kṛtvā kṣaṇārdhaṃ yadi tiṣṭhati/\\+}
\tl{viṣayair mucyate yogī vyādhimṛtyujarādibhiḥ//\\!}
\end{versinnote}

% \emph{Yuktabhavadeva} 7.209 (\attr to the \emph{Haṭhapradīpikā})
% \begin{versinnote}
% \tl{jihvāṃ parāṅmukhīṃ kṛtvā kṣaṇārddhaṃ yadi tiṣṭhati/\\+}
% \tl{kṣaṇena mucyate yogī vyādhimṛtyujarādibhiḥ//\\!}
% \end{versinnote}
\end{testimonia}
%</ts34>

%<*cm34>
\begin{philcomm}[hp03_034]
Although 3.34 is absent in \alphaOne, it is in \alphaTwo\ and \alphaThree, and also the \textgamma\ and \texteta\ groups. Other manuscripts have an additional line that gives \emph{vyomacakra} as an alternative name for \emph{khecarīmudrā} (see, for example, the verses of the \emph{Yogacintāmaṇi} cited in the testimonia). This alternative name does not occur in any of the source texts known to have been used by Svātmārāma, but it may have been inspired by the name \emph{nabhomudrā}, which is what the \emph{Vivekamārtaṇḍa} calls \emph{khecarīmudrā} (\emph{Vivekamārtaṇḍa} 40).   
\end{philcomm}
%</cm34>

%%%%%%%%%%
\subsection*{3.34*1--2}
%<*tr34-2>
\begin{translation}[hp03_034_2] % for 3.34*1-2
[The yogi] should roll back the tongue and turn it onto [the junction of] the three pathways. This is \emph{khecarīmudrā}, [also] called the `cakra of space.' If the yogi turns the tongue upwards and keeps it there for half a moment, he is instantly freed from disease, death, old age and the like.
\end{translation} %?? JB: these verses have not been translated. I've bashed out a draft. Please check.
%</tr34-2>

%%%%%%%%%%
\subsection*{3.35}
%<*tr35>
\begin{translation}[hp03_035]
For the yogi who knows \emph{khecarīmudrā} there is no disease, death, sleep, hunger, thirst or fainting.
\end{translation}
%</tr35>

%<*sc35>
\begin{sources}[hp03_035]
\emph{Vivekamārtaṇḍa} 48
\begin{variants}
tasya VM\vl~] tandrā VM
\end{variants}

% \begin{versinnote}
% \tl{na rogo maraṇaṃ tandrā na nidrā na kṣudhā tṛṣā/\\+}
% \tl{na ca mūrchā bhavet tasya yo mudrāṃ vetti khecarīm//\\!}
% \end{versinnote}
% \begin{appinnote}
% \tl{\textbf{a} tandrā~] tasya \vl \\!}
% \end{appinnote}
\end{sources}
%</sc35>

%<*ts35>
\begin{testimonia}[hp03_035]
\emph{Haṭharatnāvalī} 2.139 (\attr Dattātreya), \emph{Yogacintāmaṇi} f.\,75v (\attr Dattātreya), \emph{Yuktabhavadeva} 7.210 (\attr HP)
\begin{variants}
kṣudhā tṛṣā HRĀ YCM~] tṛṣā kṣudhā YBhD     
\end{variants}

% \begin{versinnote}
% \tl{na rogo maraṇaṃ tasya na nidrā na kṣudhā tṛṣā/\\+}
% \tl{na ca mūrcchā bhavet tasya yo mudrāṃ vetti khecarīm//\\!}
% \end{versinnote}

% \emph{Yogacintāmaṇi} f.\,75v (\attr to Dattātreya)
% \begin{versinnote}
% \tl{na rogo maraṇaṃ tasya na nidrā na kṣudhā tṛṣā/\\+}
% \tl{na ca mūrcchā bhavet tasya yo mudrāṃ vetti khecarīm//\\!}
% \end{versinnote}

% \emph{Yuktabhavadeva} 7.210 (\attr to the \emph{Haṭhapradīpikā})
% \begin{versinnote}
% \tl{na rogo maraṇaṃ tasya na nidrā na tṛṣā kṣudhā/\\+}
% \tl{na ca mūrcchā bhavet tasya yo mudrāṃ vetti khecarīm//\\!}
% \end{versinnote}
\end{testimonia}
%</ts35>

%<*cm35>
%\begin{philcomm}[hp03_035]
%\end{philcomm}
%</cm35>

%%%%%%%%%%
\subsection*{3.36}
%<*tr36>
\begin{translation}[hp03_036]
[The yogi] who knows \emph{khecarīmudrā} is neither afflicted by disease, nor tainted by action, nor tormented by death.
\end{translation}
%</tr36>

%<*sc36>
\begin{sources}[hp03_036]
\emph{Vivekamārtaṇḍa} 49
\begin{variants}
na ca lipyati~] lipyate na ca VM\sep
ca kālena~] sa kālena VM
\end{variants}
% \begin{versinnote}
% \tl{pīḍyate na sa rogeṇa lipyate na ca karmaṇā/\\+}
% \tl{bādhyate na sa kālena yo mudrāṃ vetti khecarīṃ//\\!}
% %JB: Retain variants for lipyate na ca as alpha reads na ca lipyata/lipyati
% %MD(2024-5-14): Perhaps we should read na ca lipyati karmaṇā as epic Skt.
% \end{versinnote}
\end{sources}
%</sc36>

%<*ts36>
\begin{testimonia}[hp03_036]
\emph{Haṭharatnāvalī} 2.140 (\attr Dattātreya), \emph{Yogacintāmaṇi} f.\,75v (\attr \emph{Skandapurāṇa}), \emph{Yuktabhavadeva} 7.211 (\attr HP)
\begin{variants}
na ca lipyati~] lipyate na ca HRĀ, na ca lipyeta YCM, lipyate na sa YBhD\sep
ca kālena HRĀ~] sa kālena YCM YBhD\sep
yo mudrāṃ vetti khecarīm HRĀ YCM~] yasya mudrāsti khecarī YBhD
\end{variants}


% \begin{versinnote}
% \tl{pīḍyate na sa rogeṇa lipyate na ca karmaṇā/\\+}
% \tl{bādhyate na ca kālena yo mudrāṃ vetti khecarīm//\\!}
% \end{versinnote}

% \emph{Yogacintāmaṇi} f.\,75v (\attr to the \emph{Skandapurāṇa})
% \begin{versinnote}
% \tl{piḍyate na sa rogeṇa na ca lipyeta karmaṇā/\\+}
% \tl{bādhyate na sa kālena yo mudrāṃ vetti khecarīm//\\!}
% \end{versinnote}

% \emph{Yuktabhavadeva} 7.211 (\attr to the \emph{Haṭhapradīpikā})
% \begin{versinnote}
% \tl{pīḍyate na sa rogeṇa lipyate na sa karmaṇā/\\+}
% \tl{bādhyate na sa kālena yasya mudrāsti khecarī//\\!}
% \end{versinnote}
\end{testimonia}
%</ts36>

%<*cm36>
\begin{philcomm}[hp03_036]
The unusual passive form \emph{lipyati}, which is found in \alphaTwo\ and has been adopted, is widely attested in epic Sanskrit.%JM if lipyati is only in alpha two, should we not consider lipyate na ca? -- MD: π1 has it too, and α1 supports it with na ca lipyata. By the way, isn't it better to write "unusual passive form"?
\end{philcomm}
%</cm36>

%%%%%%%%%%
\subsection*{3.37}
%<*tr37>
\begin{translation}[hp03_037]
Because the mind moves (\emph{carati}) in the ether (\emph{khe}) and the tongue moves (\emph{carati}) in the cavity (\emph{khe}), this seal is called sky-rover [and] is worshipped by the Siddhas.%?? JM: there's no and: I think it's "The mind moves (\sl{carati}) in space (\sl{khe}), because the tongue moves in space, which is why this \sl{mudrā} is called \sl{khecarī}."
\end{translation}
%</tr37>

%<*sc37>
\begin{sources}[hp03_037]
\emph{Vivekamārtaṇḍa} 50
\begin{variants}
tenaiṣā~] tenaiva VM, teneyaṃ VM\vl\sep 
nāma mudrā VM~] mudrā sarva VM\vl
\end{variants}

% \begin{versinnote}
% \tl{cittaṃ carati khe yasmāj jihvā carati khe gatā/\\+}
% \tl{tenaiṣā  khecarī nāma mudrā siddhair namaskṛtā//\\!}
% \end{versinnote}
% \begin{appinnote}
% \tl{\textbf{c} tenaiṣā~] tenaiva, teneyaṃ \vl\varsep\textbf{cd} nāma mudrā~] mudrā sarva° \vl \\!}
% \end{appinnote}
\end{sources}
%</sc37>

%<*ts37>
\begin{testimonia}[hp03_037]
\emph{Yogacintāmaṇi} (\attr Skandapurāṇa), \emph{Yuktabhavadeva} 7.212 (\attr HP)
\begin{variants}
gatā YCM~] yataḥ YBhD\sep
tenaiṣā YCM~] teneyaṃ YBhD\sep
nāma mudrā YCM~] mudrā sarva YBhD\sep
namaskṛtā YBhD~] niṣevitā YCM 
\end{variants}


% \begin{versinnote}
% \tl{cittaṃ carati khe yasmāj jihvā carati khe gatā/\\+}
% \tl{tenaiṣā khecarī nāma mudrā siddhair niṣevitā//\\!}
% \end{versinnote}

% \emph{Yuktabhavadeva} 7.212 (\attr to the \emph{Haṭhapradīpikā})
% \begin{versinnote}
% \tl{cittaṃ carati khe yasmāj jihvā carati khe yataḥ/\\+}
% \tl{teneyaṃ khecarī mudrā sarvasiddhair namaskṛtā//\\!}
% \end{versinnote}
\end{testimonia}
%</ts37>

%<*cm37>
%\begin{philcomm}[hp03_037]
%Judit: this mudrā is worshipped by the siddhas as the one called khecarī. (the idea being that this verse is explaining the name, khe + carī.
%\end{philcomm}
%</cm37>

%%%%%%%%%%
\subsection*{3.38}
%<*tr38>
\begin{translation}[hp03_038]
The yogi who has sealed the cavity above the uvula with \emph{khecarī} does not lose his semen [even if] embraced by an amorous woman.
\end{translation}
%</tr38>

%<*sc38>
\begin{sources}[hp03_038]
\emph{Vivekamārtaṇḍa} 51
\begin{variants}
tasya na VM\vl~] na tasya VM\sep
kāminyāśleṣitasya VM\vl~] kāminyāliṅgitasya VM
\end{variants}


% \begin{versinnote}
% \tl{khecaryā mudritaṃ yena vivaraṃ lambikordhvataḥ/\\+}
% \tl{na tasya kṣarate binduḥ kāminyāliṅgitasya ca//\\!}
% \end{versinnote}
% \begin{appinnote}
% \tl{\textbf{c} na tasya kṣarate binduḥ~] binduḥ kṣarati no tasya, tasya na kṣarate binduḥ \vl \\+}
% \tl{\textbf{d} °āliṃgitasya~] °āśleṣitasya \vl \\!}
% \end{appinnote}
\end{sources}
%</sc38>

%<*ts38>
\begin{testimonia}[hp03_038]
\emph{Yogacintāmaṇi} f.\,74v (\attr HP), \emph{Yuktabhavadeva} 7.213 (\attr HP)
\begin{variants}
tasya na~] na tasya YCM YBhD\sep 
kāminyāśleṣitasya~] kāminyāliṅgitasya YCM YBhD
\end{variants}

% \begin{versinnote}
% \tl{khecaryā mudritaṃ yena vivaraṃ lambikordhvataḥ/\\+}
% \tl{na tasya kṣarate binduḥ kāminyāliṅgitasya ca//\\!}
% \end{versinnote}

% \emph{Yuktabhavadeva} 7.213 (\attr to the \emph{Haṭhapradīpikā})
% \begin{versinnote}
% \tl{khecaryā mudritaṃ yena vivaraṃ lambikordhvataḥ/\\+}
% \tl{na tasya kṣarate binduḥ kāminyāliṃgitasya ca//\\!}
% \end{versinnote}
\end{testimonia}
%</ts38>

%<*cm38>
%\begin{philcomm}[hp03_038]
%\end{philcomm}
%</cm38>

%%%%%%%%%%
\subsection*{3.39}
%<*tr39>
\begin{translation}[hp03_039]
Even when semen has moved [down] and reached the region of the perineum, it moves upwards having been blocked by the perineal seal (\emph{yonimudrā}) and struck by the goddess [Kuṇḍalinī].
\end{translation}
%</tr39>

%%  NJL: be consistent and translate \emph{yonimudrā}: perineal seal??

%<*sc39>
\begin{sources}[hp03_039]
\emph{Vivekamārtaṇḍa} 53
\begin{variants}
saṃprāpto yonimaṇḍalam~] saṃprāptaś ca hutāśanam VM\sep
vrajaty VM\vl~] gacchaty VM\sep 
hataḥ VM~] tanaṃ VM\vl, hṛtas VM\vl, kṛte VM\vl, kṛtaḥ VM\vl, tadā VM\vl, gatā VM\vl\sep
nibaddho VM~] niruddho VM\vl
\end{variants}

% \begin{versinnote}
% \tl{calito 'pi yadā binduḥ saṃprāptaś ca hutāśanam/\\+}
% \tl{gacchaty ūrdhvaṃ hataḥ śaktyā nibaddho yonimudrayā//\\!}
% \end{versinnote}
% \begin{appinnote}
% \tl{\textbf{53c} gacchaty~] vrajaty \vl\ • hataḥ~] tanaṃ, hṛtas, kṛte, kṛtaḥ, tadā, gatā \vl \\+}
% \tl{\textbf{53d} nibaddho~] niruddho \vl \\!}
% \end{appinnote}

Cf.\,\emph{Śivasaṃhitā} 4.82
\begin{versinnote}
\tl{svakaṃ binduṃ ca saṃbodhya liṅgacālanam ācaret/\\+}
\tl{daivāc calati ced ūrdhvaṃ nibaddho yonimudrayā//\\!}
\end{versinnote}
\end{sources}
%</sc39>

%<*ts39>
\begin{testimonia}[hp03_039]
\emph{Yogacintāmaṇi} f.\,74v (\attr HP), \emph{Yuktabhavadeva} 7.278 (\attr HP)
\begin{variants}
saṃprāpto yonimaṇḍalam~] saṃprāptaś ca hutāśanam YCM, samprāpte 'pi hutāśanam YBhD\sep
hataḥ śaktyā~] hi tacchaktyā YCM, haṭhaḥ śaktyā YBhD\sep
nibaddho YBhD~] niruddho YCM
\end{variants}

% \begin{versinnote}
% \tl{calito 'pi yadā binduḥ saṃprāptaś ca hutāśanam/\\+}
% \tl{vrajaty ūrdhvaṃ hi tacchaktyā niruddho yonimudrayā//\\!}
% \end{versinnote}

% \emph{Yuktabhavadeva} 7.278 (\attr to the \emph{Haṭhapradīpikā})
% \begin{versinnote}
% \tl{calito'pi mahābinduḥ samprāpte'pi hutāśanam/\\+}
% \tl{vrajaty ūrdhvaṃ haṭhaḥ śaktyā nibaddho yonimudrayā//\\!}
% \end{versinnote}
\end{testimonia}
%</ts39>

%<*cm39>
\begin{philcomm}[hp03_039]
% JB Perhaps we should adopt \emph{yonimaṇḍalam} in 3.39b? It is attested by the alpha group (N3 and J5) as well as V1 and V15. It makes good sense as `region of perineum', as understood by Brahmānanda in \emph{Jyotsnā} 3.43 (\emph{yonimaṇḍalam yonisthānam}).

The third quarter of this verse has been subjected to much rewriting. Most of the collated manuscripts, including \alphaTwo\ (\alphaOne\ is illegible here and \alphaThree\ is missing this verse quarter), have the reading \emph{haṭhāt śaktyā}. This reading only makes sense if one infers that \emph{śaktyā} is referring to \emph{khecarīmudrā}, so that the second line means `blocked by \emph{yonimudrā}, semen goes up forcefully by the power [of \emph{khecarīmudrā}].' The reading \emph{hi tacchaktyā} of \deltaTwo\ (and the \emph{Yogacintāmaṇi}), appears to be an attempt to render more clearly the meaning `by the power of \emph{khecarī}.' Such an interpretation suggests that \emph{yonimudrā} blocks \emph{bindu}'s downward course and \emph{khecarī} causes it to go upwards forcefully.\lb

The oldest manuscript of the \emph{Vivekamārtaṇḍa} (ms.~no.~4110) has \emph{hataḥ śaktyā}, which is attested by three \emph{Haṭhapradīpikā} manuscripts on lower branches of the stemma (i.e.~\getsiglum{Ba}, % MD: this ms does not belong to any group
\getsiglum{C2} and \getsiglum{P4}). The participle \emph{hataḥ} makes sense of the instrumental \emph{śaktyā}, rendering the meaning `struck by Kuṇḍalinī.' \lb

Alternatively, the word \emph{hataḥ} (as well as the other variants \emph{kṛtaḥ, kṛte}, and even \emph{haṭhāt}) may derive from \emph{hṛtaḥ}, which is attested by manuscripts of the \emph{Jyotsnā}. The reading \emph{hṛtaḥ śaktyā} renders the verse as saying that semen goes up, carried by Kuṇḍalinī.\lb


%Adopt hataḥ śaktyā, which is supported by C2,  and the good manuscripts of the \emph{Vivekamārtaṇḍa}.
%[MD: Text changed. C2 reads haṭ(h)aḥ, J5 haṭhāt saktyā, G4 damaged]
%
% JB haṭhāt is very well attested (J5, J7, Gr4c, V1, C6, V3,J10). Perhaps, we should reconsider this. Since the context is khecarīmudrā, it makes sense to infer it with śaktyā and perhaps haṭhāt fits with Svātmārāma's agenda to foreground Haṭhayoga?

%But Judit/Philipp think hṛtaḥ is better and explains all the variations (kṛtaḥ, kṛte, etc. and even haṭha). So, Kuṇḍalinī carries semen upwards.

%Adopt nibaddho (J5) "bound/stopped by yonimudrā". [MD: Text changed.]

% write a note on yonimudrā
In \emph{Jyotsnā} 3.43, Brahmānanda explains \emph{yonimudrā} as essentially the contraction of the penis (\emph{yonimudrayā meḍhrākuñcanarūpayā}). He may have had in mind the practice of contracting and drawing the urethra upwards, which is described below in the section on \emph{vajrolimudrā} (\emph{Haṭhapradīpikā} 3.82). The author of the \emph{Yogaprakāśikā} (5.66) states that \emph{yonimudrā} is well known in treatises on mantra (\emph{yonimudrayeti mantraśāstraprasiddhayety arthaḥ}/ \emph{prasidhyayety} ed.). This is consistent with the \emph{Śivasaṃhitā}'s discussion of \emph{yonimudrā} (4.2, 5.12), where it is described as activating the perineum (\emph{yoni}) by contracting it in order to bring about success in mantra repetition. Later compendiums on yoga reiterate the role of \emph{yonimudrā} in mantra practice (e.g., the \emph{Yogacintāmaṇi} f.\,65r, citing the \emph{Pārameśvaratantra}, and \emph{Haṭhatattvakaumudī} 33.12). The \emph{Haṭhayogasaṃhitā} (43–48) teaches a different version of \emph{yonimudrā} in its repertoire of twenty-five \emph{mudrā}s. In this work, \emph{yonimudrā} is supposed to awaken Kuṇḍalinī and involves sitting in \emph{siddhāsana}, blocking the ears, eyes, nose and mouth with the thumbs, index, middle and ring fingers respectively, uniting \emph{prāṇa} and \emph{apāna}, meditating on the six \emph{cakra}s, and repeating the mantra \emph{huṃ haṃsa}. \lb
\end{philcomm}
%</cm39>


%%%%%%%%%%
\subsection*{3.39*1}

%<*tr39-1>
\begin{translation}[hp03_039_1]
\end{translation}
%</tr39-1>

%<*cm39-1>
\begin{philcomm}[hp03_039_1]
Manuscripts of the main groups, including \textalpha, \textgamma\ and \textepsilon, contain an additional line after 3.39 that is largely incoherent, aside from indicating that the tongue is in the cavity of the skull and that there is a \emph{mudrā} for uniting the \emph{kalā}s (\emph{kapālakuhare jihvā kalāsandhānamudrayā}). This line likely derives from a marginal note, the first half of which was probably explaining \emph{ūrdhvajihvaḥ} in the next verse. The compound \emph{kalāsaṃdhānamudrayā} may have been added as some form of dittography or as a gloss on \emph{yonimudrā}, which is not described elsewhere in the text. In a slightly modified form, this line appears in a verse in the six-chapter version of the \emph{Haṭhapradīpikā} (f. 112r–112v):

\begin{versinnote}
\tl{kapālakuhare jihvā kalāsaṃdhānavarjitā/\\+}
\tl{brahmarandhragatā nityaṃ tasya siddhir na dūrataḥ//\\!}
\end{versinnote}
\begin{appinnote}
\tl{nityaṃ \emph{em.}~] nityāṃ \emph{codex} \ • \ siddhir \emph{em.}~] siddhi \emph{codex}\\!}
\end{appinnote}

%RW: C2 includes greyscale 3.39 and adds another pāda; then skips 3.40 to get to 3.41 (MD: old numbering)
%The additional pāda is: tasmād idaṃ prakuvīrta nitya yuktuḥ samāhitaḥ
%It is difficult to understand the first hemistich without the second (and the second is not attested in any HP witnesses).
%Judit: kalāsaṃdhānamudrayā may have been added as some form of dittography
%JB: maybe this line was a marginal note attempting to explain yonimudrā, which is not described elsewhere in the text (?).
\end{philcomm}
%</cm39-1>

%%%%%%%%%%
\subsection*{3.40}
%<*tr40>
\begin{translation}[hp03_040]
The knower of yoga who remains with the tongue upwards and drinks Soma certainly conquers death in half a month.% JM: translate literally as Soma?
\end{translation}
%</tr40>

%<*sc40>
\begin{sources}[hp03_040]
\emph{Vivekamārtaṇḍa} 125
\begin{variants}
sthito bhūtvā VM~] tato bhūtvā VM\vl, sthiraṃ kṛtvā VM\vl, sthirāṃ kṛtvā VM\vl   
\end{variants}
% \begin{versinnote}
% \tl{ūrdhvajihvaḥ sthito bhūtvā somapānaṃ karoti yaḥ/\\+}
% \tl{māsārdhena na sandeho mṛtyuṃ jayati yogavit//\\!}
% \end{versinnote}
% \begin{appinnote}
% \tl{ūrdhvajihvaḥ sthito bhūtvā~] ūrdhvajihvas tato bhūtvā, ūrdhvaṃ jihvāṃ sthiraṃ kṛtvā, ūrdhvāṃ jihvā sthirāṃ kṛtvā \vl \\!}
% \end{appinnote}
\end{sources}
%</sc40>

%<*ts40>
\begin{testimonia}[hp03_040]
\emph{Yogacintāmaṇi} f.\,75v (\attr \emph{Skandapurāṇa}), \emph{Yuktabhavadeva} 7.215 (\attr HP)
\begin{variants}
sthito bhūtvā~] sthiro bhutvā YCM, sa medhāvī YBhD    
\end{variants}

% \begin{versinnote}
% \tl{ūrdhvajihvaḥ sthiro bhutvā somapānaṃ karoti yaḥ/\\+}
% \tl{māsārdheṇa na saṃdeho mṛtyuṃ jayati yogavid//\\!}
% \end{versinnote}

% \emph{Yuktabhavadeva} 7.215 (\attr to the \emph{Haṭhapradīpikā})
% \begin{versinnote}
% \tl{ūrdhvajihvaḥ sa medhāvī somapānaṃ karoti yaḥ/\\+}
% \tl{māsārddhena na sandeho mṛtyuṃ jayati yogavit//\\!}
% \end{versinnote}
\end{testimonia}
%</ts40>

%<*cm40>
%\begin{philcomm}[hp03_040]
%Alpha and \emph{Vivekamārtaṇḍa} read sthito. Adopt
%MD: Done.
%\end{philcomm}
%</cm40>

%%%%%%%%%%
\subsection*{3.41}
%<*tr41>
\begin{translation}[hp03_041]
Poison does not enter the yogi whose body is constantly filled by [nectar from] the digits of the moon, even if he is bitten by Takṣaka.
\end{translation}
%</tr41>

%<*sc41>
\begin{sources}[hp03_041]
\emph{Vivekamārtaṇḍa} 130
\begin{variants}
sarpati VM\vl~] pīḍayet VM, pīḍyate VM\vl, bādhyate VM\vl   
\end{variants}

% \begin{versinnote}
% \tl{nityaṃ somakalāpūrṇaṃ śarīraṃ yasya yoginaḥ/\\+}
% \tl{takṣakenāpi daṣṭasya viṣaṃ tasya na pīḍayet//\\!}
% \end{versinnote}
% \begin{appinnote}
% \tl{\textbf{d} pīḍayet~] pīḍyate, sarpati, bādhyate \vl \\!}
% \end{appinnote}
\end{sources}
%</sc41>

%<*ts41>
\begin{testimonia}[hp03_041]
\emph{Yogacintāmaṇi} f.\,75v (\attr \emph{Skandapurāṇa}), \emph{Yuktabhavadeva} 7.216 (\attr HP)
\begin{variants}
tasya na YBhD~] taṃ na ca YCM     
\end{variants}

% \begin{versinnote}
% \tl{nityaṃ somakalāpūrṇaṃ śarīraṃ yasya yoginaḥ/\\+}
% \tl{takṣakenāpi daṣṭasya viṣaṃ taṃ na ca sarpati//\\!}
% \end{versinnote}

% \emph{Yuktabhavadeva} 7.216 (\attr to the \emph{Haṭhapradīpikā})
% \begin{versinnote}
% \tl{nityaṃ somakalāpūrṇaṃ śarīraṃ yasya yoginaḥ/\\+}
% \tl{takṣakenāpi daṣṭasya viṣaṃ tasya na sarpati//\\!}
% \end{versinnote}
\end{testimonia}
%</ts41>

%<*cm41>
\begin{philcomm}[hp03_041]
In the context of poison, \emph{takṣaka} refers to one of the three kings of the snakes (\emph{nāga}), the other two being Śeṣa and Vāsuki (Mani 1975: 782–783).\lb % Purāṇic Encyclopaedia: A Comprehensive Dictionary with Special Reference to the Epic and Purāṇic Literature. Vettam Mani. Delhi: Motilal Banarsidass.

%The verb \emph{sarpati} is well attested by the manuscripts of the \emph{Haṭhapradīpikā} and testimonia. It can take an object, which in this case is the yogi's body.
%JB is the second paragraph necessary? The apparatus shows that sarpati is well attested, and MW dictionary indicates that sarpati can be transitive.  
%At least two alpha mss have this verse. Therefore, keep it.
\end{philcomm}
%</cm41>

%%%%%%%%%%
\subsection*{3.42}
%<*tr42>
\begin{translation}[hp03_042]
Just as fire does not leave its fuel nor light a wick in oil, so the embodied person does not leave a body filled by the [nectar from] digits of the moon.%digits of the moon? maybe tr. as nectar here.
\end{translation}
%</tr42>

%<*sc42>
\begin{sources}[hp03_042]
\emph{Vivekamārtaṇḍa} 131
\begin{variants}
vartiṃ ca VM] vartīva VM\vl, vartti ca VM\vl    
\end{variants}

% \begin{versinnote}
% \tl{indhanāni yathā vahnis tailavartiṃ ca dīpakaḥ/\\+}
% \tl{tathā somakalāpūrṇaṃ dehī dehaṃ na muñcati//\\!}
% \end{versinnote}
% \begin{appinnote}
% \tl{\textbf{b} °vartiṃ ca~] °vartīva, °vartti ca \vl \\!}
% \end{appinnote}
\end{sources}
%</sc42>

%<*ts42>
\begin{testimonia}[hp03_042]
\emph{Yogacintāmaṇi} f.\,74v (\attr HP)
\begin{variants}
tailavartīṃ~] tailavartī YCM\sep
tathā~] nityaṃ YCM
\end{variants}

% \begin{versinnote}
% \tl{indhanāni yathā vahnis tailavartī ca dīpakaḥ/\\+}
% \tl{nityaṃ somakalāpūrṇaṃ dehī dehaṃ na muñcati//\\!}
% \end{versinnote}
\end{testimonia}
%</ts42>

%<*cm42>
%\begin{philcomm}[hp03_042]
%3.42.1 is not found elsewhere (?). 3.42.2 is in the 6ch HP, but nowhere else.
%\end{philcomm}
%</cm42>

%%%%%%%%%%
\subsection*{3.43}
%<*tr43>
\begin{translation}[hp03_043]
I consider he who regularly eats cow flesh and drinks the liquor of the gods to be of good family. Others are destroyers of the family.%?? JM: lineage for family? family sounds a bit odd to me.
\end{translation}
%</tr43>

%<*sc43>
%\begin{sources}[hp03_043]
%\end{sources}
%</sc43>

%<*ts43>
\begin{testimonia}[hp03_043]
\emph{Haṭharatnāvalī} 2.158, \emph{Yogacintāmaṇi} f.\,74v (\attr HP)
\begin{variants}
itare kulaghātakāḥ~] anye tu kulaghātakāḥ HRĀ, netarān kulaghātakān YCM    
\end{variants}

% \begin{versinnote}
% \tl{gomāṃsaṃ bhakṣayen nityaṃ pibed amaravāruṇīṃ/\\+}
% \tl{kulīnaṃ tam ahaṃ manye anye tu kulaghātakāḥ//\\!}
% \end{versinnote}

% \emph{Yogacintāmaṇi} f.\,74v (\attr to the \emph{Haṭhapradīpikā})
% \begin{versinnote}
% \tl{gomāṃsaṃ bhakṣayen nityaṃ pibed amaravāruṇīm/\\+}
% \tl{kulīnaṃ tam ahaṃ manye netarān kulaghātakān//\\!}
% \end{versinnote}


\end{testimonia}
%</ts43>

%<*cm43>
%\begin{philcomm}[hp03_043]
%Brahmānanda identifies this (and next two verses?) as compositions of Svātmārāma, presumably because of \emph{manye}. [JB: I don't think this note is worth mentioning as it is not relevant to an editorial decision]
%\end{philcomm}
%</cm43>

%%%%%%%%%%
\subsection*{3.44}
%<*tr44>
\begin{translation}[hp03_044]
By the word `cow' is meant the tongue, for its insertion into the palate is the eating of cow's flesh, which destroys great sin.
\end{translation}
%</tr44>

%<*sc44>
%\begin{sources}[hp03_044]
%\end{sources}
%</sc44>

%<*ts44>
\begin{testimonia}[hp03_044]
\emph{Haṭharatnāvalī} 2.157, \emph{Yogacintāmaṇi} ff.\,74v–75r (\attr HP)

% \begin{versinnote}
% \tl{gośabdenoditā jihvā tatpraveśo hi tāluni/\\+}
% \tl{gomāṃsabhakṣaṇaṃ tat tu mahāpātakanāśanaṃ//\\!}
% \end{versinnote}

% \emph{Yogacintāmaṇi} f.\,74v–75r (\attr to the \emph{Haṭhapradīpikā})
% \begin{versinnote}
% \tl{gośabdenoditā jihvā tatpraveśo hi tāluni/\\+}
% \tl{gomāṃsabhakṣaṇaṃ tat tu mahāpātakanāśanam//\\!}
% \end{versinnote}

\end{testimonia}
%</ts44>

%<*cm44>
%\begin{philcomm}[hp03_044]
%44cd is in N3 and J5, so retain.
%\end{philcomm}
%</cm44>

%%%%%%%%%%
\subsection*{3.45}
%<*tr45>
\begin{translation}[hp03_045]
The essence produced by the fire caused by the insertion of the tongue which flows from the moon is the liquor of the gods.
\end{translation}
%</tr45>

%<*sc45>
%\begin{sources}[hp03_045]
%\end{sources}
%</sc45>

%<*ts45>
\begin{testimonia}[hp03_045]
\emph{Haṭharatnāvalī} 2.159, \emph{Yogacintāmaṇi} f.\,75r (\attr HP)
\begin{variants}
vahninotpāditaḥ YCM~] vahninotthāpitā HRĀ    
\end{variants}

% \begin{versinnote}
% \tl{jihvāpraveśasaṃbhūtavahninotthāpitā khalu/\\+}
% \tl{candrāt sravati yaḥ sāraḥ sā syād amaravāruṇī//\\!}
% \end{versinnote}

% \emph{Yogacintāmaṇi} f.\,75r (\attr to the \emph{Haṭhapradīpikā})
% \begin{versinnote}
% \tl{jihvāpraveśasaṃbhūtavahninotpāditaḥ khalu/\\+}
% % \tl{candrāt sravati yaḥ sāraḥ sā syād amaravāruṇī//\\!}
% \end{versinnote}

\end{testimonia}
%</ts45>

%<*cm45>
%\begin{philcomm}[hp03_045]
%\end{philcomm}
%</cm45>

%%%%%%%%%%
\subsection*{3.46}
%<*tr46>
\begin{translation}[hp03_046]% ed: line break after haṭhād, not haṭhā, and piben -- MD: I have changed pibet to pibe-n.
With his face turned upwards and his tongue fixed in the aperture [of the skull], [the yogi] should visualise as the supreme \emph{śakti} [the nectar] that is forcibly obtained from the breath having dripped from the head into the sixteen petals of the lotus. And the yogi who drinks the gushing nectar, the pure fluid [surging] from the [moon's] digits in waves, is free of disease, has a body as soft as lotus fibre, and lives a long time.
\end{translation}
%</tr46>

%<*sc46>
\begin{sources}[hp03_046]
\emph{Vivekamārtaṇḍa} 118
\begin{variants}
padmapattra~] patrapadma VM\sep
niyamya VM~] nidhāya VM\vl, vidhāya VM\vl\sep 
cintayet VM~] cālayet VM\vl\sep
utkallolakalājalaṃ VM~] utkallolakalākalaṃ VM\vl, utkallolajalākulaṃ VM\vl, utkallolajalāmṛtaṃ VM\vl, tat kallola\-kalā\-jalaṃ VM\vl, tat kallolajalākulaṃ VM\vl\sep
ca vimalaṃ VM~] suvimalaṃ VM\vl\sep
dhārāmṛtaṃ~] jīvākulaṃ VM\vl, jihvākulaṃ VM\vl, dhārājalaṃ VM\vl\sep
vapur VM\vl~] tanur VM
\end{variants}

% \begin{versinnote}
% \tl{mūrdhnaḥ ṣoḍaśapatrapadmagalitaṃ prāṇād avāptaṃ haṭhād \\+}
% \tl{ūrdhvāsyo rasanāṃ niyamya vivare śaktiṃ parāṃ cintayet/\\+}
% \tl{utkallolakalājalaṃ ca vimalaṃ dhārāmayaṃ yaḥ piben \\+}
% \tl{nirdoṣaḥ sa mṛṇālakomalatanur yogī ciraṃ jīvati//\\!}
% \end{versinnote}
% \begin{appinnote}
% \tl{\textbf{118b} niyamya~] nidhāya, vidhāya \vl • cintayet~] cālayet \vl\ \ \textbf{118c} utkallolakalājalaṃ~] ca vimalaṃ dhārāmayaṃ~] (from HP); utkallolakalākalaṃ, utkallolajalākulaṃ, utkallolajalāmṛtaṃ, tat kallolakalājalaṃ, tat kallolajalākulaṃ \vl • ca vimalaṃ~] suvimalaṃ \vl • dhārāmayaṃ~] (from HP); jīvākulaṃ, jihvākulaṃ, dhārājalaṃ \vl\ \  \textbf{118d} tanur~] vapur \vl \\!}
% \end{appinnote}
\end{sources}
%</sc46>

%<*ts46>
\begin{testimonia}[hp03_046]
Cf. \emph{Haṭharatnāvalī} 2.150 
\begin{versinnote}
\tl{utkallolakalāmṛtaṃ ca vimalaṃ dhārāmṛtaṃ yaḥ pibet/\\+}
\tl{nirdoṣaḥ sa mṛnālakomalatanur yogī ciraṃ jīvati//\\!}
\end{versinnote}
% \begin{appinnote}
% \tl{utkallola°~] tatkallola° \vl, °tanur~] °vapur \vl \\!}
% \end{appinnote}
\emph{Yogacintāmaṇi} f.\,75r (\attr HP), \emph{Yuktabhavadeva} 7.217 (\attr Gorakṣanātha), \emph{Haṭhatattvakaumudī} 14.24 (\attr HP)
\begin{variants}
mūrdhnaḥ YCM YBhD~] ūrdhvaṃ HTK, mūrdhvaṃ HTK\vl\sep
padmapattra YCM~] patrapadma YBhD HTK\sep
vivare YCM YBhD~] kuhare HTK\sep
cintayet YBhD~] cintayan YCM YBhD\vl HTK\sep
utkallola HTK~] tatkallola YCM YBhD\sep
ca vimalaṃ YCM~] suvimalaṃ YBhD HTK\sep
dhārāmṛtaṃ YBhD HTK~] jihvākulaṃ YCM\sep
vapur HTK~] tanur YCM YBhD\sep
ciraṃ YCM HTK~] paraṃ YBhD
\end{variants}

% \emph{Yogacintāmaṇi} f.\,75r (\attr to the \emph{Haṭhapradīpikā})
% \begin{versinnote}
% \tl{mūrdhnaḥ ṣoḍaśapadmapatragalitaṃ prāṇād avāptaṃ haṭhād\\+}
% \tl{ūrdhvāsyo rasanāṃ niyamya vivare śaktiṃ parāṃ cintayan/\\+}
% \tl{tatkallolakalājalaṃ ca vimalaṃ jihvākulaṃ yaḥ piben\\+}
% \tl{nirdoṣaḥ sa mṛṇālakomalatanur yogī ciraṃ jīvati//\\!}
% \end{versinnote}

% \emph{Yuktabhavadeva} 7.217 (\attr to Gorakṣanātha)
% \begin{versinnote}
% \tl{mūrdhnaḥ ṣoḍaśapatrapadmagalitaṃ prāṇād avāptaṃ haṭhāt \\+}
% \tl{ūrdhvāsyo rasanāṃ niyamya vivare śaktiṃ parāṃ cintayet/\\+}
% \tl{tatkallolakalājalaṃ suvimalaṃ dhārāmṛtaṃ yaḥ pibet\\+}
% \tl{nirddoṣaḥ sa mṛṇālakomalatanur yogī paraṃ jīvati//\\!}
% \end{versinnote}
% \begin{appinnote}
% \tl{\textbf{b} cintayet~] cintayan \vl \\!}
% \end{appinnote}

% \emph{Haṭhatattvakaumudī} 14.24 (\attr to the \emph{Haṭhapradīpikā})
% \begin{versinnote}
% \tl{ūrdhvaṃ ṣoḍaśapatrapadmagalitaṃ prāṇād avāptaṃ haṭhād \\+}
% \tl{ūrdhvāsyo rasanāṃ niyamya kuhare śaktiṃ parāṃ cintayan/\\+}
% \tl{utkallolakalājalaṃ suvimalaṃ dhārāmṛtaṃ yaḥ piben \\+}
% \tl{nirddoṣaḥ sa mṛṇālakomalavapur yogī ciraṃ jīvati//\\!}
% \end{versinnote}
% \begin{appinnote}
% \tl{ūrdhvaṃ~] mūrdhvaṃ \vl \\!}
% \end{appinnote}
\end{testimonia}
%</ts46>

%<*cm46>
\begin{philcomm}[hp03_046]
The meaning of \emph{prāṇāt} (`from the breath') in the first verse quarter is not easy to understand without the context of this verse in the source text, the \emph{Vivekamārtaṇḍa}. In the verse preceding this one in the \emph{Vivekamārtaṇḍa} (117), the breath, on reaching the “great lotus”, is said to turn into nectar (\emph{amṛta}). In \emph{Jyotsnā} 3.51, Brahmānanda notes that there is a variant \emph{prāṇaiḥ} (`by means of the breaths'), which is easier to understand than \emph{prāṇāt}. He nonetheless accepts \emph{prāṇāt} and understands it as being the means by which the nectar is obtained (\emph{prāṇāt sādhanabhūtād avāptam}). He also understands the sixteen-petalled lotus to be the lotus in the throat, into which the nectar drips.
%Adopt patrapadma. % MD: The important mss and also VM-T have padmapattra. Perhaps we should adopt this. We can dissolve the compound into ṣoḍaśāni padmapattrāṇi, not necessarily ṣoḍaśānāṃ padmānāṃ pattrāṇi.
\end{philcomm}
%</cm46>

\begin{metre}[hp03_046]
Śārdūlavikrīḍita 
\end{metre}

%%%%%%%%%%
\subsection*{3.47}
%<*tr47>
\begin{translation}[hp03_047]
If the tongue, while oozing nectar and constantly kissing the tip of the uvula, is salty, pungent, like  milk or the same as honey and ghee, diseases are eliminated for [the yogi], he stops ageing, can recite treatises and scriptures, attains immortality together with the eightfold powers, and attracts Siddha women.
\end{translation}
%</tr47>

%<*sc47>
\begin{sources}[hp03_047]
\emph{Vivekamārtaṇḍa} 128
\begin{variants}
rasasyandinī VM~] rasaspandanī VM\vl, rasāsvādinī VM\vl\sep
jarāntakaraṇaṃ VM\vl~] jaropaśamanaṃ VM, jarāpaharaṇaṃ VM\vl\sep
odīraṇaṃ VM~] odgīraṇaṃ VM\vl, occāraṇaṃ VM\vl\sep
guṇavat~] guṇitaṃ VM
\end{variants}

% \begin{versinnote}
% \tl{cumbantī yadi lambikāgram aniśaṃ jihvā rasasyandinī\\+}
% \tl{sakṣārā kaṭukātha dugdhasadṛśā madhvājyatulyāthavā /\\+}
% \tl{vyādhīnāṃ haraṇaṃ jaropaśamanaṃ śāstrāgamodīraṇaṃ\\+}
% \tl{tasya syād amaratvaṃ aṣṭaguṇitaṃ siddhāṅgānākarṣaṇam//\\!}
% \end{versinnote}
% \begin{appinnote}
% \tl{\textbf{128a} rasasyandanī~] rasaspandanī VTG, rasāsvādinī A \ \textbf{128c} jaropaśamanaṃ~] AGBGL; jarāpaharaṇaṃ V, jarāntakaranaṃ TGP • °odīraṇaṃ~] VA; °odgīraṇaṃ TGBGL, °occāraṇaṃ GP \\!}
% \end{appinnote}
%
\end{sources}
%</sc47>

%<*ts47>
\begin{testimonia}[hp03_047]
\emph{Yogacintāmaṇi} f.\,75r (\attr HP), \emph{Yuktabhavadeva} 7.218 (\attr Gorakṣanātha), \emph{Haṭhatattvakaumudī} 14.25 (\attr HP)
\begin{variants}
aniśaṃ YCM YBhD~] anilaṃ HTK\sep
kaṭukātha~] kaṭukāmla YCM YBhD HTK\sep
sadṛśā~] sadṛśaṃ YCM, sadṛśī YBhD, sadṛśāṃ HTK\sep
tulyāthavā YBhD~] tulyaṃ yadā YCM, tulyā tathā HTK\sep
jarāntakaraṇaṃ YCM HTK~] jarāmbutaraṇaṃ YBhD\sep
odgīraṇaṃ YBhD HTK~] oddhāraṇaṃ YCM\sep
amaratvam YBhD HTK~] iha siddhir YCM\sep
guṇavat HTK~] guṇitā YCM, guṇitaṃ YBhD
\end{variants}

% \begin{versinnote}
% \tl{cumbantī yadi lambikāgram aniśaṃ jihvā rasasyandinī\\+}
% \tl{sakṣārā kaṭukāmladugdhasadṛśaṃ madhvājyatulyaṃ yadā/\\+}
% \tl{vyādhīnāṃ haraṇaṃ jarāntakaraṇaṃ śāstrāgamoddhāraṇaṃ\\+}
% \tl{tasya syād iha siddhir aṣṭaguṇitā siddhāṅgaṇākarṣaṇam//\\!}
% \end{versinnote}

% \emph{Yuktabhavadeva} 7.218 (\attr Gorakṣanātha)
% \begin{versinnote}
% \tl{cumbantī yadi lambikāgram aniśaṃ jihvā rasasyandinī \\+}
% \tl{sakṣārā kaṭukāmladugdhasadṛśī madhvājyatulyāthavā/\\+}
% \tl{vyādhīnāṃ haraṇaṃ jarāmbutaraṇaṃ śāstrāgamodgīraṇaṃ\\+}
% \tl{tasya syād amaratvam aṣṭaguṇitaṃ siddhāṅganākarṣaṇam//\\!}
% \end{versinnote}

% \emph{Haṭhatattvakaumudī} 14.25 (\attr to the \emph{Haṭhapradīpikā})
% \begin{versinnote}
% \tl{cumbantī yadi lambikāgram anilaṃ jihvā rasasyandinī\\+}
% \tl{sakṣārā kaṭukāmladugdhasadṛśāṃ madhvājyatulyā tathā/\\+}
% \tl{vyādhīnāṃ haraṇaṃ jarāntakaraṇaṃ śāstrāgamodgīraṇaṃ\\+}
% \tl{tasya syād amaratvam aṣṭaguṇavat siddhāṅganākarṣaṇam//\\!}
% \end{versinnote}
\end{testimonia}
%</ts47>

%<*cm47>
%\begin{philcomm}[hp03_047]
%MD(2024-5-14): adopt sadṛśā? (alpha and VM)
%\end{philcomm}
%</cm47>

\begin{metre}[hp03_047]
Śārdūlavikrīḍita 
\end{metre}

%%%%%%%%%%
\subsection*{3.48}
%<*tr48>
\begin{translation}[hp03_048]
There is one seed [syllable], which contains creation, one \emph{mudrā}, \emph{khecarī}, one god, the unsupported, and one state, beyond mind.
\end{translation}
%</tr48>

%<*sc48>
\begin{sources}[hp03_048]
Cf. \emph{Timirodghāṭana} 5.14c–15b (NGMPP A35/3)
\begin{versinnote}
\tl{eka[ṃ] sṛṣṭimayaṃ bījaṃ ek[ā] mudrā tu khecarī/\\+}
\tl{dvāv etau jñāyate yena so pi śāntapade sthitam//\\!}
\end{versinnote}

Cf. Quotation by Jayaratha \emph{ad} \emph{Tantrāloka} 32.63, introduced with \emph{yad āgamaḥ}
\begin{versinnote}
\tl{ekaṃ sṛṣṭimayaṃ bījam ekā mudrā ca khecarī/\\+}
\tl{dvāv ekaṃ yo vijānāti sa vai pūjyaḥ kulāgame//\\!}
\end{versinnote}
\end{sources}
%</sc48>

%<*ts48>
\begin{testimonia}[hp03_048]
\emph{Haṭharatnāvalī} 4.28, \emph{Yogacintāmaṇi} f.\,75r (\attr HP), \emph{Yuktabhavadeva} 7.219 (\attr Gorakṣanātha) 

% \begin{versinnote}
% \tl{ekaṃ sṛṣṭimayaṃ bījam ekā mudrā ca khecarī/\\+}
% \tl{eko devo nirālambaḥ ekāvasthā manonmanī //\\!}
% \end{versinnote}

% \emph{Yogacintāmaṇi} f.\,75r (\attr to the \emph{Haṭhapradīpikā})
% \begin{versinnote}
% \tl{ekaṃ sṛṣṭimayaṃ bījam ekā mudrā ca khecarī/\\+}
% \tl{eko devo nirālamba ekāvasthā manonmanī//\\!}
% \end{versinnote}

% \emph{Yuktabhavadeva} 7.219 (\attr to Gorakṣanātha)
% \begin{versinnote}
% \tl{ekaṃ sṛṣṭimayaṃ bījaṃ ekā mudrā ca khecarī/\\+}
% \tl{eko devo nirālamba ekāvasthā manonmanī//\\!}
% \end{versinnote}
\end{testimonia}
%</ts48>

%<*cm48>
%\begin{philcomm}[hp03_048]
%\end{philcomm}
%</cm48>

%%%%%%%%%%
\subsection*{3.48*1}% font problem in top line of apparatus -- MD: Intentionally. But perhaps better to use italic. JM: I then saw that you've been doing it elsewhere and I think it works well.
%<*tr48-1>
% \begin{translation}[hp03_048_1]
% That which enters the aperture into the underworld exists at the base of Meru. The wise [yogi] says that is the truth, the source of [all] rivers. The essence of the body flows from the moon. Because of that people die. [The yogi] should block it with the clay of the excellent [\emph{khecarī}] technique. Bodily perfection [arises] no other way.
% \end{translation}% greyscale in ed.
%</tr48-1>

%<*cm48-1>
\begin{philcomm}[hp03_048_1]
For the translation and testimonia, as well as an explanation of the various places and versions of this verse in the text, see \manuref{4.9}. 
\end{philcomm}
%</cm48-1>


\begin{metre}[hp03_048_1]
Mandākrāntā 
\end{metre}

%%%%%%%%%%
\subsection*{3.49 heading}
%<*tr49a>
\begin{translation}[hp03_049a]
The root lock (\emph{mūlabandha}):
\end{translation}
%</tr49a>

%<*cm49a>
% \begin{philcomm}[hp03_049a]
% \end{philcomm}
%</cm49a>

%%%%%%%%%%
\subsection*{3.49}
%<*tr49>
\begin{translation}[hp03_049]
When [the yogi] presses the perineum with part of the heel, clenches the anus and draws up \emph{apāna}, it is called the root lock.
\end{translation}
%</tr49>
% JB: the back part of the heel presses the perineum. It would be better to say, 'with part of the heel.'
% MD: α1 and π-mss have 'yam iṣyate in place of 'yam ucyate.

%<*sc49>
\begin{sources}[hp03_049]
\emph{Vivekamārtaṇḍa} 42
\begin{variants}
iṣyate~] ucyate VM     
\end{variants}

% \begin{versinnote}
% \tl{pārṣṇibhāgena saṃpīḍya yonim ākuñcayed gudam/\\+}
% \tl{apānam ūrdhvam ākṛṣya mūlabandho 'yam ucyate//\\!}
% \end{versinnote}
\end{sources}
%</sc49>

%<*ts49>
\begin{testimonia}[hp03_049]
\emph{Haṭharatnāvalī} 2.58, \emph{Yogacintāmaṇi} f.\,76r (\attr HP)
\begin{variants}
iṣyate~] ucyate HRĀ YCM     
\end{variants}

% \begin{versinnote}
% \tl{pārṣṇibhāgena sampīḍya yonim ākuñcayed gudaṃ/\\+}
% \tl{apānam ūrdhvam ākuñcya mūlabandho 'yam ucyate//\\!}
% \end{versinnote}

% \emph{Yogacintāmaṇi} f.\,76r (\attr to the \emph{Haṭhapradīpikā})
% \begin{versinnote}
% \tl{pārṣṇibhāgena saṃpīḍya yonim ākuñcayed gudam/\\+}
% \tl{apānam ūrdhvam ākṛṣya mūlabandho 'yam ucyate//\\!}
% \end{versinnote}

\end{testimonia}
%</ts49>

%<*cm49>
%\begin{philcomm}[hp03_049]
%\end{philcomm}
%</cm49>

%%%%%%%%%%
\subsection*{3.50}
%<*tr50>
\begin{translation}[hp03_050]
It forces the downward-moving \emph{apāna} breath to move upwards by contraction [of the anus]. Yogis call that the root lock.
\end{translation}
%</tr50>
% JB: Perhaps it is better to take MB as the subject of 51ab as MB is the subject of the previous verse? i.e., It forces the downward-moving \emph{apāna} breath to move upwards because of the contraction.
%  NJL: translate mūlabandha?

%<*sc50>
\begin{sources}[hp03_050]
\emph{Gorakṣaśataka} 53
\begin{variants}
adhogatiṃ GŚ~] adhogataṃ GŚ\vl\sep 
ākuñcanena taṃ GŚ~] ākuñcane ca tat GŚ\vl\sep
mūlabandhaṃ hi yoginaḥ~] mūlabandhaṃ tu yoginaḥ GŚ, mūlabandho yam ucyate GŚ\vl
\end{variants}

% \begin{versinnote}
% \tl{adhogatim apānaṃ vai ūrdhvagaṃ kurute balāt/\\+}
% \tl{ākuñcanena taṃ prāhur mūlabandhaṃ tu yoginaḥ//\\!}
% \end{versinnote}
% \begin{appinnote}
% \tl{\textbf{53a} °gatiṃ~] TU; °gataṃ G \ \textbf{53c} ākuñcanena taṃ~] GU ākuñcane ca tat T \\+}
% \tl{\textbf{53d} mūlabandhaṃ tu yoginaḥ~] T; mūlabandho yam ucyate GU \\!}
% \end{appinnote}
\end{sources}
%</sc50>

%<*ts50>
\begin{testimonia}[hp03_050]
\emph{Haṭharatnāvalī} 2.59, \emph{Yogacintāmaṇi} f.\,76r (\attr HP)
\begin{variants}
adhogatim HRĀ~] adhogatam YCM\sep 
vai ūrdhvagaṃ HRĀ~] ca tad ūrdhvaṃ YCM\sep
balāt HRĀ~] haṭhāt YCM\sep
hi HRĀ~] tu YCM
\end{variants}

% \begin{versinnote}
% \tl{adhogatim apānaṃ vai ūrdhvagaṃ kurute balāt/\\+}
% \tl{ākuñcanena taṃ prāhur mūlabandhaṃ hi yoginaḥ//\\!}
% \end{versinnote}

% \emph{Yogacintāmaṇi} f.\,76r (\attr to the \emph{Haṭhapradīpikā})
% \begin{versinnote}
% \tl{adhogatam apānaṃ ca tad ūrdhvaṃ kurute haṭhāt/\\+}
% \tl{ākuñcanena taṃ prāhur mūlabandhaṃ tu yoginaḥ//\\!}
% \end{versinnote}
\end{testimonia}
%</ts50>

%<*cm50>
%\begin{philcomm}[hp03_050]
% MD: pāda d: hi is better attested than tu. 
%\end{philcomm}
%</cm50>

%%%%%%%%%%
\subsection*{3.51}
%<*tr51>
\begin{translation}[hp03_051]
[The yogi] should press his anus with his heel and forcefully contract the [\emph{apāna}] wind over and over again so that the breath goes upwards.
\end{translation}
%</tr51>

%<*sc51>
\begin{sources}[hp03_051]
\emph{Dattātreyayogaśāstra} 144
\begin{variants}
vāyum DYŚ\vl~] yonim (\emph{em.})     % MD: Jyotsnā has vāyum!
\end{variants}
% \begin{versinnote}
% \tl{gudaṃ pārṣṇyā tu saṃpīḍya yonim ākuñcayed balāt/ \\+}
% \tl{vāraṃ vāraṃ yathā cordhvaṃ samāyāti samīraṇaḥ//\\!}
% \end{versinnote}
% \begin{appinnote}
% \tl{\textbf{b} yonim~] from \emph{Jyotsnā}, vāyum \emph{codd.} \\!}
% \end{appinnote}

Cf.\,\emph{Śārṅgadharapaddhati} 4416
\mylb
% \begin{versinnote}
% \tl{gudaṃ pārṣṇyā tu sampīḍya vāyum ākuñcayed balāt/\\+}
% \tl{vāraṃ vāraṃ yathā cordhvaṃ samāyāti samīraṇaḥ//\\!}
% \end{versinnote}
\end{sources}
%</sc51>

%<*ts51>
\begin{testimonia}[hp03_051]
\emph{Haṭharatnāvalī} 2.60, \emph{Yogacintāmaṇi} f.\,76r (\attr \emph{Yogabīja}), \emph{Yogabīja} 103 (south-Indian recension)
\begin{variants}
tu YCM, YB~] ca HRĀ    
\end{variants}

% \begin{versinnote}
% \tl{gudaṃ pārṣṇyā ca saṃpīḍya vāyum ākuñcayed balāt/\\+}
% \tl{vāraṃ vāraṃ yathā cordhvaṃ samāyāti samīraṇaḥ//\\!}
% \end{versinnote} 

% \emph{Yogacintāmaṇi} f.76r (\attr to the \emph{Yogabīja})
% \begin{versinnote}
% \tl{gudaṃ pārṣṇyā tu saṃpīḍya vāyum ākuñcayed balāt/\\+}
% \tl{vāraṃ vāraṃ tathā cordhvaṃ samāyāti samīraṇaḥ//\\!}
% \end{versinnote}

% \emph{Yogabīja} 103 (south-Indian recension)
% \begin{versinnote}
% \tl{gudaṃ pārṣṇyā tu saṃpīḍya vāyum ākuñcayed balāt/\\+}
% \tl{vāraṃ vāraṃ yathā cordhvaṃ samāyāti samīraṇaḥ//\\!}
% \end{versinnote}

\end{testimonia}
%</ts51>

%<*cm51>
\begin{philcomm}[hp03_051]
The instruction to `contract the wind' (\emph{vāyum ākuñcayet}) is odd (especially with \emph{samīraṇaḥ} in the fourth \emph{pāda}) and not found in other texts. Mallin\-son has adopted \emph{yonim} for \emph{vāyum} (cf.\,\emph{Haṭhapradīpikā} 3.49b) in his edition of this verse in its source text, the \emph{Dattātreyayogaśāstra}, which is not found in the manuscripts of that text but is in the \emph{Haṭhapradīpikā}’s \textdelta \ manuscripts. \lb %?? MD: perhaps corrupted from pāyum?

In the context of the root lock, \emph{vāyum ākuñcayet} can be understood as an instruction to contract \emph{apāna\-vāyu}, which is mentioned in the previous verse (3.50a). Instructions to contract \emph{apāna\-vāyu} are found in other yoga texts, such as \emph{Yoga\-tārā\-valī} 7b (\emph{ākuñcanaiḥ śaśvad apānavāyoḥ}), \emph{Śiva\-saṃhitā} 4.84cd (\emph{apāna\-vāyum ākuñcya balād...}), \emph{Śiva\-yoga\-pradīpikā} 2.53ab (\emph{athordhva\-madhya\-sthira\-bandha\-nābhyām ākuñcanād ūrdhvam apānavāyoḥ}) and \emph{Yukta\-bhava\-deva} 7.297 (\emph{ādhāra\-kamale suptāṃ cālayet kuṇḍalīṃ dṛḍhām}/ \emph{apāna\-vāyum ākṛṣya balād ākuñcya buddhimān}). In \emph{Jyotsnā} 3.63, Brahmānanda understands \emph{vāyu} in this verse as \emph{apāna} when he says that `one should contract the wind, \emph{apāna}' (\emph{vāyum apānam ākuñcayed}), which he explains as, `one should pull it by contractions of the anus' (\emph{gudasyā\-kuñcanenā\-karṣayet}). In the same vein, Bhava\-deva\-miśra glosses `pulling \emph{apāna}' (\emph{apānā\-karṣaṇam}) as `contracting the anus' (\emph{gudā\-kuñcanam}), when commenting on `having pulled \emph{apāna\-vāyu} and forcefully contracted it ...' (\emph{apāna\-vāyum ākṛṣya balād ākuñcya...}) in \emph{Yukta\-bhava\-deva} 297 and 301.  
\end{philcomm} %
%</cm51>

%%%%%%%%%%
\subsection*{3.52}
%<*tr52>
\begin{translation}[hp03_052]
When \emph{prāṇa} and \emph{apāna} [and] \emph{nāda} and \emph{bindu} become united by means of the root lock they are sure to bestow complete success in yoga.
\end{translation}
%</tr52>

%<*sc52>
\begin{sources}[hp03_052]
\emph{Dattātreyayogaśāstra} 145
\begin{variants}
gatvā yogasya DYŚ~] gacchato yoga DYŚ\vl\sep 
yacchato DYŚ~] gacchato DYŚ\vl, kurute DYŚ\vl, gachate DYŚ\vl
\end{variants}

% \begin{versinnote}
% \tl{prāṇāpānau nādabindū mūlabandhena caikatām/\\+}
% \tl{gatvā yogasya saṃsiddhiṃ yacchato nātra saṃśayaḥ//\\!}
% \end{versinnote}
% \begin{appinnote}
% \tl{\textbf{145c} gatvā yogasya saṃ°~] gacchato yogasaṃ° M1A \\+}
% \tl{\textbf{145d} yacchato~] gacchato M1, kurute AM2, gachate \textpi \\!}
% \end{appinnote}

%\emph{Śārṅgadharapaddhati} 4417
%\begin{versinnote}
%\tl{prāṇāpānau nādabindū mūlabandhena caikatām/\\+}
%\tl{gatvā yogasya saṃsiddhiṃ gacchatau nātra saṃśayaḥ//\\!}
%\end{versinnote}
\end{sources}
%</sc52>

%<*ts52>
\begin{testimonia}[hp03_052]
\emph{Haṭharatnāvalī} 2.61, \emph{Yogacintāmaṇi} f.\,76r (\attr \emph{Yogabīja})
\begin{variants}
gatvā yogasya saṃsiddhiṃ YCM~] gatau tadā yogasiddhiṃ HRĀ\sep
yacchato nātra~] gacchato nātra YCM, prāpnoty eva na HRĀ 
\end{variants}

% \begin{versinnote}
% \tl{prāṇāpānau nādabindū mūlabandhena caikatām/\\+}
% \tl{gatau tadā yogasiddhiṃ prāpnoty eva na saṃśayaḥ//\\!}
% \end{versinnote}

% \emph{Yogacintāmaṇi} f.76r (\attr to the \emph{Yogabīja})
% \begin{versinnote}
% \tl{prāṇāpānau nādabindū mūlabandhena caikatām/\\+}
% \tl{gatvā yogasya saṃsiddhiṃ gacchato nātra saṃśayaḥ//\\!}
% \end{versinnote}

\end{testimonia}
%</ts52>

%<*cm52>
\begin{philcomm}[hp03_052]
Since the term \emph{nāda} usually means `internal sound' in Haṭha and Rājayoga texts, it is possible that \emph{bindu} here was understood by some to have the tantric connotations of sonic and visual foci (Mallinson 2007: 219 n.\,325) or two levels of sonic emanation in \emph{mantroccāra}, where \emph{nāda} is an unvoiced sound and \emph{bindu} is the slightly coarser sound of inner murmuring (see \emph{Tāntrikā\-bhidhānakośa} vol.~3, 2013: 278–279). However, there is a passage in the \emph{Amaraugha} (10–12) where \emph{nāda} and \emph{bindu} are paired and it is clear that \emph{bindu} means generative fluid. 
\end{philcomm}
%</cm52>


\begin{metre}[hp03_052]
Anuṣṭubh (a: ra-vipulā)
\end{metre}

%%%%%%%%%%
\subsection*{3.53}
%<*tr53>
\begin{translation}[hp03_053]
\textit{Prāṇa} and \textit{apāna} unite, urine and faeces diminish, [and] even an old man becomes young as a result of the continuous application of the root lock.
\end{translation}
%</tr53>

%<*sc53>
\begin{sources}[hp03_053]
\emph{Vivekamārtaṇḍa} 41
\mylb
% \begin{versinnote}
% \tl{apānaprāṇayor aikyaṃ kṣayo mūtrapurīṣayoḥ/\\+}
% \tl{yuvā bhavati vṛddho 'pi satataṃ mūlabandhanāt//\\!}
% \end{versinnote}

%\emph{Śārṅgadharapaddhati} 4418
%\begin{versinnote}
%\tl{apānaprāṇayor aikyaṃ kṣayo mūtrapurīṣayoḥ/\\+}
%\tl{yuvā bhavati vṛddho 'pi satataṃ mūlabandhanāt//\\!}
%\end{versinnote}
\end{sources}
%</sc53>

%<*ts53>
\begin{testimonia}[hp03_053]
\emph{Haṭharatnāvalī} 2.62
\mylb
% \begin{versinnote}
% \tl{apānaprāṇayor aikyaṃ kṣayo mūtrapurīṣayoḥ/\\+}
% \tl{yuvā bhavati vṛddho 'pi satataṃ mūlabandhanāt//\\!}
% \end{versinnote}
\end{testimonia}
%</ts53>

%<*cm53>
\begin{philcomm}[hp03_053]
The diminishing of urine and faeces as a result of success in yoga is mentioned in the \emph{Amanaska} (1.50c) and \emph{Dattātreyayogaśāstra} (80a).
%The term \emph{kṣaya} can mean either the end or diminishment of something. In the context of urine and faeces (\emph{mūtrapurīṣa}), their diminishment is the likely meaning intended as there are references in other texts to the successful practice of a yoga technique resulting in a reduced amount of urine and faeces, such as \emph{Amanaska} 1.50c (\emph{svalpamūtrapurīṣatvaṃ}) and \emph{Dattātreyayogaśāstra} 80a (\emph{alpamūtrapurīṣaḥ}), which were both known to Svātmārāma. In \emph{Jyotsnā} 3.65, Brahmānanda understands \emph{kṣaya} in this verse as `decline' (\emph{kṣayaḥ patanam}).%JM: I don't recall discussing this, but I think patanam here means excretion. Brahmānanda has  mūtrapurīṣayoḥ saṃcitayoḥ kṣayaḥ patanam. He's surely wrong in his understanding of the pāda. I have changed the note to just give the parallels.
\end{philcomm}
%</cm53>

%%%%%%%%%%
\subsection*{3.54}
%<*tr54>
\begin{translation}[hp03_054]
When \emph{apāna} has turned upwards and reached the orb of fire, then the flame of the fire, fanned by the wind, grows tall.
\end{translation}
%</tr54>

%<*sc54>
\begin{sources}[hp03_054]
\emph{Gorakṣaśataka} 54
\begin{variants}
maṇḍalam GŚ (\emph{em.} from HP)~] maṇḍale GŚ\vl\sep    
tadānalaśikhā~] tato 'nalaśikhā GŚ
\end{variants}

% \begin{versinnote}
% \tl{apāne cordhvage jāte saṃprāpte vahnimaṇḍalam/\\+}
% \tl{tato 'nalaśikhā dīrghā vardhate vāyunāhatā//\\!}
% \end{versinnote}
% \begin{appinnote}
% \tl{\textbf{54a} °maṇḍalam~] \emph{from Haṭhapradīpikā witnesses}, maṇḍale \emph{codd.} \\!}
% \end{appinnote}
\end{sources}
%</sc54>

%<*ts54>
\begin{testimonia}[hp03_054]
\emph{Haṭharatnāvalī} 2.63, \emph{Yogacintāmaṇi} f.\,76r (\attr \emph{Yogabīja})
\begin{variants}
saṃprāpte YCM~] prayāte HRĀ\sep 
maṇḍalam~] maṇḍale HRĀ YCM\sep
tadānalaśikhā dīrghā~] tathānalaśikhādīptir HRĀ, tathānalaśikhā dīrghā YCM\sep
vardhate vāyunāhatā YCM~] vāyunā preritā yathā HRĀ
\end{variants}

% \begin{versinnote}
% \tl{apāne cordhvage jāte prayāte vahnimaṇḍale/\\+}
% \tl{tathānalaśikhādīptir vāyunā preritā yathā//\\!}
% \end{versinnote}

% \emph{Yogacintāmaṇi} f.\,76r (\attr to the \emph{Yogabīja})
% \begin{versinnote}
% \tl{apāne cordhvage jāte saṃprāpte vahnimaṇḍale/\\+}
% \tl{tathānalaśikhā dīrghā vardhate vāyunāhatā//\\!}
% \end{versinnote}

\end{testimonia}
%</ts54>

%<*cm54>
%\begin{philcomm}[hp03_054]
%\end{philcomm}
%</cm54>

%%%%%%%%%%
\subsection*{3.55}
%<*tr55>
\begin{translation}[hp03_055]
As a result, fire and \textit{apāna} reach \textit{prāṇa}, which is hot by nature. Then [\emph{prāṇa}] makes the fire in the body extremely hot.
\end{translation}
%</tr55>

% MD: tathā or tadā in pāda d?
%<*sc55>
\begin{sources}[hp03_055]
\emph{Gorakṣaśataka} 55
\begin{variants}
pradīptas tu~] pradīptena GŚ\sep
dehajas GŚ~] dehagas GŚ\vl\sep
tadā~] tathā GŚ
\end{variants}

% \begin{versinnote}
% \tl{tato yātau vahnyapānau prāṇam uṣṇasvarūpakam/\\+}
% \tl{tenātyantapradīptena jvalano dehajas tathā//\\!}
% \end{versinnote}
% \begin{appinnote}
% \tl{\textbf{d} dehajas~] T; dehagas GU \\!}
% \end{appinnote}
\end{sources}
%</sc55>

%<*ts55>
\begin{testimonia}[hp03_055]
\emph{Haṭharatnāvalī} 2.64, \emph{Yogacintāmaṇi} f.\,76r–76v (\attr \emph{Yogabīja})
\begin{variants}
tato yātau YCM~] yātāyātau HRĀ\sep 
prāṇam uṣṇasvarūpakam~] mūlarūpasvarūpakau HRĀ, prāṇam uktasvarūpakau YCM\sep
tenātyanta YCM~] tenābhyantaḥ HRĀ\sep
tadā~] tathā HRĀ YCM
\end{variants}

% \begin{versinnote}
% \tl{yātāyātau vahnyapānau mūlarūpasvarūpakau/\\+}
% \tl{tenābhyantaḥ pradīptas tu jvalano dehajas tathā// 2.64//\\!}
% \end{versinnote}

% \emph{Yogacintāmaṇi} f.\,76r–76v (\attr to the \emph{Yogabīja})
% \begin{versinnote}
% \tl{tato yātau vahnyapānau prāṇam uktasvarūpakau/\\+}
% \tl{tenātyantapradīptas tu jvalano dehajas tathā//\\!}
% \end{versinnote}

\end{testimonia}
%</ts55>

%<*cm55>
\begin{philcomm}[hp03_055]
The second verse quarter has been rewritten in \alphaOne\ and \alphaTwo\ as \emph{prāṇamūla\-svarūpakam}, which is similar to \etaOne\ (\emph{prāṇamūlasvarūpakau}) and the \emph{Haṭharatnāvalī} (\emph{mūlarūpasva\-rūpakau}). It is likely that \emph{mūla} arose as a misreading of \emph{uṣṇa}. References to \emph{prāṇa} being hot by nature (and \emph{apāna} being cold) occur in other works, such as the \emph{Mokṣopāya} (6.85.111–112) and \emph{Haṭhatattvakaumudī} (4.14, 41.2). The commentators Bālakṛṣṇa (\emph{Yogaprakāśikā} 5.85) and Brahmānanda (\emph{Jyotsnā} 3.67) accept the idea that \emph{prāṇa} is hot by nature.
%MD: intentional rewriting or simple misreading?

\end{philcomm}
%</cm55>

\begin{metre}[hp03_055]
Anuṣṭubh (a: ra-vipulā)
\end{metre}

%%%%%%%%%%
\subsection*{3.56}
%<*tr56>
\begin{translation}[hp03_056]
Heated by that [blaze], the sleeping Kuṇḍalinī wakes up. Like a snake struck by a stick, she hisses and becomes straight.
\end{translation}
%</tr56>

%<*sc56>
\begin{sources}[hp03_056]
\emph{Gorakṣaśataka} 56
\mylb
% \begin{versinnote}
% \tl{tena kuṇḍalinī suptā saṃtaptā saṃprabudhyate/\\+}
% \tl{daṇḍāhatā bhujaṃgīva niśvasya ṛjutāṃ vrajet//\\!}
% \end{versinnote}
\end{sources}
%</sc56>

%<*ts56>
\begin{testimonia}[hp03_056]
\emph{Haṭharatnāvalī} 2.65ab, \emph{Yogacintāmaṇi} f.\,76v (\attr \emph{Yogabīja})
\begin{variants}
saṃtaptā~] satataṃ YCM\sep 
saṃprabudhyate~] saṃprabodhyate YCM\sep
daṇḍāhatā HRĀ~] daṇḍāhata YCM\sep
niśvasya HRĀ\vl~] niścitaṃ HRĀ YCM\sep
vrajet YCM~] iyāt HRĀ 
\end{variants}


% \begin{versinnote}
% \tl{daṇḍāhatā bhujaṅgīva niścitaṃ ṛjutām iyāt/\\!}
% \end{versinnote}
% \begin{appinnote}
% \tl{niścitaṃ~] niśvasya T,P,t1 \\!}
% \end{appinnote}

% \emph{Yogacintāmaṇi} f.\,76v (\attr to the \emph{Yogabīja})
% \begin{versinnote}
% \tl{tena kuṇḍalinī suptā satataṃ saṃprabodhyate/\\+}
% \tl{daṇḍāhatabhujaṅgīva niścitam ṛjutāṃ vrajet//\\!}
% \end{versinnote}

\end{testimonia}
%</ts56>

%<*cm56>
%\begin{philcomm}[hp03_056]
%\end{philcomm}
%</cm56>

%%%%%%%%%%
\subsection*{3.57}
%<*tr57>
\begin{translation}[hp03_057]
Then, like [a snake] that has entered a hole, she goes into the channel of Brahman. So yogis should regularly practise the root lock.
\end{translation}
%</tr57>

%<*sc57>
\begin{sources}[hp03_057]
\emph{Gorakṣaśataka} 57
\begin{variants}
bilaṃ~] bile GŚ, bila GŚ\vl, bilaṃ GŚ\vl\sep
praviṣṭeva~] praviṣṭe ca tato GŚ, praveśato yatra GŚ\vl
\end{variants}

% \begin{versinnote}
% \tl{bile praviṣṭe ca tato brahmanāḍyantaraṃ vrajet/\\+}
% \tl{tasmān nityaṃ mūlabandhaḥ kartavyo yogibhiḥ sadā//\\!}
% \end{versinnote}
% \begin{appinnote}
% \tl{57a bile~] bil*e*T, bila° G2U, bilaṃ G1 • °praviṣṭe ca tato~] T; °praveśato yatra GU \\!}
% \end{appinnote}
\end{sources}
%</sc57>

%<*ts57>
\begin{testimonia}[hp03_057]
\emph{Haṭharatnāvalī} 2.65c–f, \emph{Yogacintāmaṇi} f.\,76v (\attr \emph{Yogabīja})

% \begin{versinnote}
% \tl{bilaṃ praviṣṭeva tato brahmanāḍyantaraṃ vrajet/\\+}
% \tl{tasmān nityaṃ mūlabandhaḥ kartavyo yogibhiḥ sadā//\\!}
% \end{versinnote}

% \emph{Yogacintāmaṇi} f.\,76v (\attr to the \emph{Yogabīja})
% \begin{versinnote}
% \tl{bilaṃ praviṣṭeva tathā brahmanāḍyantaraṃ vrajet/\\+}
% \tl{tasmān nityaṃ mūlabandhaḥ kartavyo yogipuṅgavaiḥ//\\!}
% \end{versinnote}
% \begin{appinnote}
% \tl{\textbf{a} praviṣṭeva~] praviṣṭaiva \vl \\!}
% \end{appinnote}

\end{testimonia}
%</ts57>

%<*cm57>
%\begin{philcomm}[hp03_057]
%\end{philcomm}
%</cm57>

\begin{metre}[hp03_057]
Anuṣṭubh (a: bha-vipulā; c: ra-vipulā)
\end{metre}

%%%%%%%%%%
\subsection*{3.58 heading}
%<*tr58a>
\begin{translation}[hp03_058a]
Now the Uḍḍīyaṇa lock:
\end{translation}
%</tr58a>

%<*cm58a>
% \begin{philcomm}[hp03_058a]
% \end{philcomm}
%</cm58a>

%%%%%%%%%%
\subsection*{3.58}
%<*tr58>
\begin{translation}[hp03_058]
Yogis say that this [lock] is called Uḍḍīyaṇa because the breath flies up (\emph{uḍḍīyate}) into Suṣumṇā when bound by it.
\end{translation}
%?? JB: I'm not sure whether the referent of the second "it" is clear. "yogis say that this [lock] is called Uḍḍīyaṇa because the breath flies up (\emph{uḍḍīyate}) into the Suṣumṇā when bound by it."
%</tr58>

%<*sc58>
\begin{sources}[hp03_058]
\emph{Gorakṣaśataka} 58c–59b
\begin{variants}
baddho GŚ (\emph{em.} from HP)~] vajro GŚ\vl, bandho GŚ\vl\sep
yataḥ GŚ~] tataḥ GŚ\vl    
\end{variants}

% \begin{versinnote}
% \tl{baddho yena suṣumṇāyāṃ prāṇas tūḍḍīyate yataḥ/\\+}
% \tl{tasmād uḍḍīyaṇākhyo 'yaṃ yogibhiḥ samudāhṛtaḥ//\\!}
% \end{versinnote}
% \begin{appinnote}
% \tl{\textbf{58c} baddho~] \emph{em. from HP}; vajro G, bandho TU \ \textbf{58d} yataḥ~] TU; tataḥ G \\!}
% \end{appinnote}
\end{sources}
%</sc58>

%<*ts58>
\begin{testimonia}[hp03_058]
\emph{Haṭharatnāvalī} 2.53, \emph{Yogacintāmaṇi} f.\,76v (\attr \emph{Yogabīja})

% \begin{versinnote}
% \tl{baddho yena suṣumnāyāṃ prāṇas tūḍḍiyate yataḥ/\\+}
% \tl{tasmād uḍḍiyānākhyo 'yaṃ yogibhiḥ samudāhṛtaḥ//\\!}
% \end{versinnote}

% \emph{Yogacintāmaṇi} f.\,76v (\attr to the \emph{Yogabīja})
% \begin{versinnote}
% \tl{baddho yena suṣumṇāyāṃ prāṇas tūḍḍīyate yataḥ/\\+}
% \tl{tasmād uḍḍiyānākhyo 'yaṃ yogibhiḥ samudāhṛtaḥ//\\!}
% \end{versinnote}

\end{testimonia}
%</ts58>

%<*cm58>
%\begin{philcomm}[hp03_058]
%\end{philcomm}
%</cm58>

%%%%%%%%%%
\subsection*{3.59}
%<*tr59>
\begin{translation}[hp03_059]
Because the great bird tirelessly flies up (\emph{uḍyāṇaṃ kurute}), [this lock] is [called] `flying up' (\emph{uḍḍīyāṇam}). In it, the [root] lock is applied.
\end{translation}
% JB the referent of the first "it" seems to be the "great bird". ... that is [called] ... ?
%</tr59>

%<*sc59>
\begin{sources}[hp03_059]
\emph{Vivekamārtaṇḍa} 43
\begin{variants}
aviśrāntaṃ (\emph{em.})~] aviśrānta GŚ\vl, aviśrānto GŚ\vl    
\end{variants}

% NJL: I think perhaps the sources variants needs revision. GŚ?

% \begin{versinnote}
% \tl{uḍyāṇaṃ kurute yasmād aviśrāntaṃ mahākhagaḥ/\\+}
% \tl{uḍḍiyānaṃ tad eva syāt tatra bandho vidhīyate//\\!}
% \end{versinnote}
% \begin{appinnote}
% \tl{\textbf{43b} aviśrāntaṃ~] \emph{em.~from HP}; aviśrānta VTU, aviśrānto G \\!}
% \end{appinnote}
\end{sources}
%</sc59>

%<*ts59>
\begin{testimonia}[hp03_059]
\emph{Haṭharatnāvalī} 2.54, \emph{Yogacintāmaṇi} f.\,76v (\attr \emph{Yogabīja})
\begin{variants}
uḍyāṇaṃ~] uḍḍīnaṃ HRĀ YCM\sep 
tatra bandho vidhīyate~] tatra bandho 'bhidhīyate HRĀ, mūlabandho 'bhidhīyate YCM
\end{variants}


% \begin{versinnote}
% \tl{uḍḍīnaṃ kurute yasmād aviśrāntaṃ mahākhagaḥ/\\+}
% \tl{uḍḍiyānaṃ tad eva syāt tatra bandho 'bhidhīyate//\\!}
% \end{versinnote}

% \emph{Yogacintāmaṇi} f.\,76v (\attr to the \emph{Yogabīja})
% \begin{versinnote}
% \tl{uḍḍīnaṃ kurute yasmād aviśrāntaṃ mahākhagaḥ/\\+}
% \tl{uḍḍiyānaṇ tad eva syān mūlabandho 'bhidhīyate//\\!}
% \end{versinnote}

%Skandapurāṇa 4.41.47
%\begin{versinnote}
%\tl{uḍḍīnaṃ kurute yasmād ahorātraṃ mahākhagaḥ/\\+}
%\tl{uḍḍīyānaṃ tataḥ proktaṃ tatra bandho vidhīyate//\\!}
%\end{versinnote}
\end{testimonia}
%</ts59>

%<*cm59>
\begin{philcomm}[hp03_059]
In the \emph{Vivekamārtaṇḍa} this verse is preceded by a passage on \emph{mūlabandha}, so the likely meaning of the fourth verse quarter is that (\emph{mūla})\emph{bandha} is to be performed in this practice. The \textdelta\ group of the \emph{Haṭhapradīpikā} witnesses has `the root [lock] is applied' (\emph{mūlaṃ vidhīyate}), which appears to be an attempt to clarify the meaning of the original verse.\lb

The word \emph{uḍyāṇaṃ} in the first \emph{pāda} is attested in manuscripts of the \emph{Vivekamārtaṇḍa}, the source of this verse, and two manuscripts of the \textalpha\ group. This spelling is attested in several vernacular works (e.g., \emph{Aṣṭāṅgayoga} of Caraṇadāsa 144, 275, 307–8, \emph{Jogapradīpyaka} 588, 635, 686 etc.), as well as other Sanskrit yoga texts (e.g., \emph{Yogatārāvalī} 6, \emph{Dhyānabindūpaniṣat} 75 etc.). Other witnesses, including the \emph{Jyotsnā}, have \emph{uḍḍīnaṃ}, the usual form for the verbal noun from \emph{uḍ-ḍī}.


%There is considerable variation among the witnesses with regard to the spelling of \emph{uḍḍīnaṃ} in \emph{pāda} a. We have adopted \emph{uḍḍīnaṃ} because it is the correct form for the verbal noun from \emph{uḍ-ḍī}, as widely attested elsewhere. 
\end{philcomm}
%</cm59>

% MD: Or is uḍyāṇaṃ a wrong Sanskritization of the vernacular uḍḍāṇaṃ (= Skt. uḍḍayana)?

% uḍyāna is attested in the Aṣṭāṅgayoga of Caraṇadāsa 144, 275, 307–8
% Jogapradīpyaka: uḍyāṇa:  588, 635, 686
% Yogatattvabindu V: uḍyānapīṭha°
%%%%%%%%%%
\subsection*{3.60}
%<*tr60>
\begin{translation}[hp03_060]
 [The yogi] should perform a rearward and upward stretching of the navel into the abdomen. That is the \emph{uḍḍīyāṇa} lock, a lion to the elephant of death.
\end{translation}
%</tr60>

%<*sc60>
\begin{sources}[hp03_060]
\emph{Vivekamārtaṇḍa} 44, \emph{Śivasaṃhitā} 4.73
\begin{variants}
paścimaṃ SŚ~] paścime VM\sep
ca VM~] tu ŚS\sep
uḍḍīyāṇo hy asau~] uḍḍīyāno hy asau VM, uḍyānākhyo 'tra ŚS
\end{variants}
% \begin{versinnote}
% \tl{udare paścime tāṇaṃ nābher ūrdhvaṃ ca kārayet/\\+}
% \tl{uḍḍīyāno hy asau bandho mṛtyumātaṅgakesarī//\\!}
% \end{versinnote}
% tāna is not in M-W so prob comes from Marathi. Molesworth:  ताण tāṇa (p. 374)
% ताण tāṇa m (तन S) The state of being stretched or strained; stretchedness or strain (as of a rope, cloth &c.) v दे, बस, भर. 2 fig. Intense anger, a rage, a passion. v ये. Ex. वर्माची गोष्ट काढतांच कसा ताण आला. 3...

% \emph{Śivasaṃhitā} 4.73
% \be
% \begin{versinnote}
% \tl{udare paṣcimaṃ tānaṃ nābher ūrdhvaṃ tu kārayet/\\+}
% \tl{uḍyānākhyo'tra bandho'yaṃ mṛtyumātaṅgakesarī//\\!}
% \end{versinnote}
\end{sources}
%</sc60>

%<*ts60>
\begin{testimonia}[hp03_060]
\emph{Haṭharatnāvalī} 2.55, \emph{Yogacintāmaṇi} f.\,76v (\attr \emph{Yogabīja})
\begin{variants}
tāṇaṃ~] tānaṃ HRĀ YCM\sep
ca kārayet~] ca dhārayet HRĀ, samācaret YCM\sep
uḍḍīyāṇo~] uḍḍīyāno HRĀ YCM    
\end{variants}

% \begin{versinnote}
% \tl{udare paścimaṃ tānaṃ nābher ūrdhvaṃ ca dhārayet/ \\+}
% \tl{uḍḍiyāno hy asau bandho mṛtyumātaṅgakesarī// \\!}
% \end{versinnote}

% \emph{Yogacintāmaṇi} f.\,76v (\attr to the \emph{Yogabīja})
% \begin{versinnote}
% \tl{udare paścimaṃ tānaṃ nābher ūrdhvaṃ samācaret/\\+}
% \tl{uḍḍiyāno hy asau bandho mṛtyumātaṅgakesarī//\\!}
% \end{versinnote}

%Skandapurāṇa 4.41.48
%\begin{versinnote}
%\tl{jaṭhare paścimaṃ tānaṃ nābher ūrdhvaṃ ca dhārayet/\\+}
%\tl{uḍḍīyāno hy ayaṃ bandho mṛtyor api bhayaṃ tyajet// \\!}
%\end{versinnote}
\end{testimonia}
%</ts60>

%<*cm60>
\begin{philcomm}[hp03_060]
The spelling \sl{tāṇa} (where many witnesses have \emph{tāna}) reflects vernacular pronunciation (see e.g.,~Molesworth 1857 s.v.~\emph{tāṇa}). 
\end{philcomm}
%</cm60>


%%%%%%%%%%
\subsection*{3.61}
%<*tr61>
\begin{translation}[hp03_061]   
Uḍḍiyāṇa is easy, but it is always taught by a guru. Even an old person becomes young if they practise it tirelessly.%
\end{translation}
%</tr61>
%?? MD: "One should always practise uḍḍiyāṇa in its original form, as taught by a guru"?

%<*sc61>
\begin{sources}[hp03_061]
\emph{Dattātreyayogaśāstra} 141c–142b
\begin{variants}
guruṇā DYŚ~] guṇaughāt DYŚ\vl    
\end{variants}

% \begin{versinnote}
% \tl{uḍḍiyāṇaṃ tu sahajaṃ guruṇā kathitaṃ sadā/\\+}
% \tl{abhyased astatandras tu vṛddho 'pi taruṇo bhavet//\\!}
% \end{versinnote}
% \begin{appinnote}
% \tl{\textbf{141d} guruṇā~] guṇaughāt \pi \\!}
% \end{appinnote}
\end{sources}
%</sc61>

%<*ts61>
\begin{testimonia}[hp03_061]
\emph{Yogacintāmaṇi} f.\,76v (\attr \emph{Yogabīja})
\begin{variants}
sadā~] yathā YCM\sep
abhyased astatandras~] abhyaset tad atandras YCM\sep
taruṇo bhavet~] taruṇāyate YCM
\end{variants}

Cf.~\emph{Haṭharatnāvalī} 2.56
\begin{versinnote}
\tl{guruṇā sahajaṃ proktaṃ vṛddho 'pi taruṇo bhavet/\\+}
\tl{bāhyoḍyāṇaṃ ca kurute bāhyālaṅkāravardhanam//\\!}
\end{versinnote}

% \emph{Yogacintāmaṇi} f.\,76v (\attr to the \emph{Yogabīja})
% \begin{versinnote}
% \tl{uḍḍiyānaṃ tu sahajaṃ guruṇā kathitaṃ yathā/\\+}
% \tl{abhyaset tad atandras tu vṛddho 'pi taruṇāyate//\\!}
% \end{versinnote}

\end{testimonia}
%</ts61>

%<*cm61>
\begin{philcomm}[hp03_061]
%At the end of the second \emph{pāda} \emph{sadā} is attested by witnesses of \textalpha, \texteta, \textzeta\ and \delta \ groups, as well as the \emph{Dattātreyayogaśāstra}.  %all but the gamma/delta groups; JM: but do we need to mention this?


We have understood the first line to mean that even though the basics of the practice of \emph{uḍḍiyāṇa} are easy, it still needs to be taught by the guru. Witnesses of the \textgamma\ and \textdelta\ groups read \emph{yathā} for \emph{sadā}, % = gamma/delta
perhaps as a deliberate substitution of the more difficult \emph{sadā}, making the verse mean that \emph{uḍḍiyāṇa} is easy in the way that is taught by the guru.
%The reading \emph{astatandraḥ} is probably the original one in the \emph{Dattātreyayogaśāstra} and is attested in some manuscripts of the \emph{Haṭhapradīpikā}. However, it is not strongly supported by the witnesses of the important groups and so may have changed to something simpler, such as \emph{tad atandras tu}.
%  JB: collate J5 reading astatandras tu, so that last note is not necessary
\end{philcomm}
%</cm61>

\begin{metre}[hp03_061]
Anuṣṭubh (a: na-vipulā)
\end{metre}

%%%%%%%%%%
\subsection*{3.62}
%<*tr62>
\begin{translation}[hp03_062]
[The yogi] should carefully stretch [the region of the abdomen] above and below the navel. If he practises [like this] for six months, he is sure to conquer death.
\end{translation}
%</tr62>

%<*sc62>
\begin{sources}[hp03_062]
\emph{Dattātreyayogaśāstra} 142c–143b
\begin{variants}
adhaś cāpi DYŚ\vl~] ataḥ paścāt DYŚ\sep
tāṇaṃ~] tānaṃ DYŚ\sep
abhyasan~] abhyasen DYŚ
\end{variants}

% \begin{versinnote}
% \tl{nābher ūrdhvam ataḥ paścāt tānaṃ kuryāt prayatnataḥ// 142//\\+}
% \tl{ṣaṇmāsam abhyasen mṛtyuṃ jayaty eva na saṃśayaḥ/\\!}
% \end{versinnote}
% \begin{appinnote}
% \tl{\textbf{142c} ataḥ paścāt~] PT, adhaś cāpi \emph{cett.} \\!}
% \end{appinnote}

Cf.\,\emph{Śivasaṃhitā} 4.72
\begin{versinnote}
\tl{nābher ūrdhvam adhaś cāpi tānaṃ paścimam ācaret/\\+}
\tl{uḍyānabandha eṣaḥ syāt sarvaduḥkhaughanāśanaḥ//\\!}
\end{versinnote}
\end{sources}
%</sc62>

%<*ts62>
\begin{testimonia}[hp03_062]
\emph{Haṭharatnāvalī} 2.57, \emph{Yogacintāmaṇi} f.\,76v (\attr \emph{Yogabīja})
\begin{variants}
adhaś cāpi~] adho vāpi HRĀ YCM\sep
tāṇaṃ~] tānaṃ HRĀ YCM\sep
abhyasan~] abhyasen HRĀ YCM
\end{variants}

% \begin{versinnote}
% \tl{nābher ūrdhvam adho vāpi tānaṃ kuryāt prayatnataḥ/\\+}
% \tl{ṣaṇmāsam abhyasen mṛtyuṃ jayaty eva na saṃśayaḥ//\\!}
% \end{versinnote}

% \emph{Yogacintāmaṇi} f.\,76v (\attr to the \emph{Yogabīja})
% \begin{versinnote}
% \tl{nābher ūrdhvam adho vāpi tānaṃ kuryāt prayatnataḥ/\\+}
% \tl{ṣaṇmāsam abhyasen mṛtyuṃ jayaty eva na saṃśayaḥ//\\!}
% \end{versinnote}

Cf.\,\emph{Yuktabhavadeva} f.\,76v (\attr to the \emph{Śivayoga})
\begin{versinnote}
\tl{nābher ūrdhvam adhaś cāpi tānaṃ nirbharam ācaret/\\+}
\tl{uḍḍiyāno hy ayaṃ bandhaḥ sarvaduḥkhaughanāśanaḥ//\\!}
\end{versinnote}

\end{testimonia}
%</ts62>

%<*cm62>
%\begin{philcomm}[hp03_062]
%\end{philcomm}
%</cm62>


%%%%%%%%%%
\subsection*{3.63}
%<*tr63>
\begin{translation}[hp03_063]
Sitting in \emph{vajrāsana}, [the yogi] should hold his feet firmly with his hands near the region of the ankles and press the bulb (\emph{kanda}) there.
\end{translation}
%</tr63>

%<*sc63>
\begin{sources}[hp03_063]
\emph{Gorakṣaśataka} 59c–60b
\begin{variants}
pādau GŚ\vl~] jānū (\emph{em.}), jānu GŚ\vl, prādau GŚ\vl    
\end{variants}

% \begin{versinnote}
% \tl{sati vajrāsane jānū karābhyāṃ dhārayed dṛḍham//\\+}
% \tl{gulphadeśasamīpe ca kandaṃ tatra prapīḍayet/\\!}
% \end{versinnote}
% \begin{appinnote}
% \tl{\textbf{59c} jānū~] em.; pādau GU, jānu T, prādau V \\!}
% \end{appinnote}
\end{sources}
%</sc63>

%<*ts63>
\begin{testimonia}[hp03_063]
\emph{Yogacintāmaṇi} f.\,76v (\attr HP)
\begin{variants}
kandaṃ tatra~] udaraṃ tat YCM    
\end{variants}

% \begin{versinnote}
% \tl{sati vajrāsane pādau karābhyāṃ dhārayed dṛḍham/\\+}
% \tl{gulphadeśasamīpe ca udaraṃ tat prapīḍayet//\\!}
% \end{versinnote}

% Cf.\,\emph{Yuktabhavadeva} 7.224 (commenting on \emph{uḍḍiyāṇabandha})
% \begin{versinnote}
% \tl{dṛḍham āsanaṃ baddhvā gulphadeśasamīpe karābhyāṃ pādau datvā nābhisamīpasthaṃ kandaṃ pīḍayann udare paścimatāṇaṃ tathā kuryād yathā vāyuḥ kukṣisandhiṃ na gacchaty evam uḍḍiyānabandho jarāmṛtyuvināśanaḥ sampadyate//\\!}
% \end{versinnote}

% Cf. \emph{Haṭhasaṅketacandrikā} f.\,36r
% \begin{versinnote}
% \tl{tathā coktaṃ haṭhapradīpikāyām/\\+}
% \tl{sati vajrāsane pādau karābhyāṃ dharayed dṛḍhaṃ/\\+}
% \tl{gulphadeśasamīpe ca kandaṃ tatra nipīḍayet//... \\+}
% \tl{siddhāsane sthitvā hastābhyāṃ pādau gulphapradeśasamīpe dṛḍhaṃ dhṛtvā tunde nālotthāna[ṃ] sādhu vidhāya samāhitamanasā sudṛḍhamūlabandhajālandharabandhavatābhyāsinā sādhakena recakādau kuṃbhakānte udare paścimatāne kriyamāṇe nitarāṃ tadā tatra nābhikandanipīḍane paścimatānena sati nābhikandotthānāḍaya urdhvamukhā vikasitā viralā vimalā asaṃhatā vāyugrahasamarthā bhavanti tadā sakuṃbhitaḥ prāṇavāyuḥ śanaiḥ[//]\\!}
% \end{versinnote}
\end{testimonia}
%</ts63>

%<*cm63>
\begin{philcomm}[hp03_063]
In the \emph{Haṭhapradīpikā} this verse seems to instruct the yogi to press the bulb (\emph{kanda}) with the feet while holding them with the hands. The adopted reading of the source text, the \emph{Gorakṣaśataka}, which is only found in witness T (and there in the singular \emph{jānu}, which has been emended to the dual \emph{jānū} in Mallinson's edition) says that it is the knees that are to be held, which would still allow for the feet to press the bulb. In the \emph{Yuktabhavadeva} (7.224), Bhavadevamiśra says that the bulb is near the navel and the legs are held near the ankles, suggesting that the bulb is pressed by using the hands to pull the feet into the lower abdomen.\lb\smallskip

\emph{Yuktabhavadeva} 7.224 (commenting on \emph{uḍḍiyāṇabandha})
\begin{versinnote}
\tl{dṛḍham āsanaṃ baddhvā gulphadeśasamīpe karābhyāṃ pādau datvā nābhisamīpasthaṃ kandaṃ pīḍayann udare paścimatāṇaṃ tathā kuryād yathā vāyuḥ kukṣisandhiṃ na gacchaty evam uḍḍiyānabandho jarāmṛtyuvināśanaḥ sampadyate//\\!}
\end{versinnote}

Holding the ankles with the hands and pressing the \emph{kanda} with the feet is also the view of Brahmānanda (\emph{Jyotsnā} 3.114), who follows the \emph{Yogayājñavalkya} (4.14, 4.16) in thinking that the place of the \emph{kanda} is nine fingerbreadths above the middle of the body, which is two fingerbreadths above the anus (\emph{Jyotsnā} 3.113).\lb 

In the \emph{Haṭhasaṅketacandrikā} (f.\,36r), Sundaradeva explains that the yogi presses the \emph{kanda} in the navel by performing \emph{uḍḍiyāṇabandha}, along with the root and chin locks, at the end of \emph{kumbhaka} and the beginning of exhalation. It is thus the backward stretch in the abdomen (\emph{udare paścimatāna}) that presses the \emph{kanda} in the navel.\lb\smallskip

\emph{Haṭhasaṅketacandrikā} f.\,36r
\begin{versinnote}
\tl{tathā coktaṃ haṭhapradīpikāyām/\\+}
\tl{\ \ \ sati vajrāsane pādau karābhyāṃ dharayed dṛḍhaṃ/\\+}
\tl{\ \ \ gulphadeśasamīpe ca kandaṃ tatra nipīḍayet// \texteng{\dots}\\+}
\tl{siddhāsane sthitvā hastābhyāṃ pādau gulphapradeśasamīpe dṛḍhaṃ dhṛtvā tunde nālotthāna[ṃ] sādhu vidhāya samāhita\skx{-}{}manasā sudṛḍhamūlabandhajālandharabandhavatābhyāsinā sādhakena recakādau kuṃbhakānte udare paścimatāne kriyamāṇe nitarāṃ tadā tatra nābhikandanipīḍane paścimatānena sati nābhikandotthānāḍaya urdhvamukhā vikasitā viralā vimalā asaṃhatā vāyugrahasamarthā bhavanti tadā sakuṃbhitaḥ prāṇavāyuḥ śanaiḥ[//]\\!}
\end{versinnote}

%M3 and G5 (gr 4c)
\end{philcomm}
%</cm63>


%%%%%%%%%%
\subsection*{3.64}
%<*tr64>
\begin{translation}[hp03_064]
[The yogi] should very gently stretch back his abdomen, chest and throat in such a way that the breath does not come into contact with the stomach.
\end{translation}%
%</tr64>

%<*sc64>
\begin{sources}[hp03_064]
\emph{Gorakṣaśataka} 60c–61b
\begin{variants}
tāṇam~] tānam GŚ    
\end{variants}
% \begin{versinnote}
% \tl{paścimaṃ tānam udare kārayed dhṛdaye gale/\\+}
% \tl{śanaiḥ śanair yathā prāṇas tundasaṃdhiṃ na gacchati//\\!}
% \end{versinnote}
\end{sources}
%</sc64>

%<*ts64>
\begin{testimonia}[hp03_064]
\emph{Yogacintāmaṇi} f.\,76v (\attr HP), \emph{Haṭhsaṅketacandrikā}, f.\,36r–36v (\attr HP)
\begin{variants}
paścimaṃ YCM~] paścime HSC\sep
tāṇam~] tānam YCM HSC\sep
kārayed dhṛdaye gale~] kārayec cibukaṃ hṛdi YCM, ku[r]yac ca cibukaṃ hṛdi HSC\sep
tundasaṃdhiṃ~] tundasiddhiṃ YCM, kandasaṃdhi[ṃ] HSC
\end{variants}

% \begin{versinnote}
% \tl{paścimaṃ tānam udare kārayec cibukaṃ hṛdi/\\+}
% \tl{śanaiḥ śanair yathā prāṇas tundasiddhiṃ na gacchati//\\!}
% \end{versinnote}

% Cf. \emph{Haṭhsaṅketacandrikā}, f.\,36r–36v (\attr to the \emph{Haṭhapradīpikā})
% \begin{versinnote}
% \tl{paścime tānam udare ku[r]yac c[a] cibukaṃ hṛdi/\\+}
% \tl{śanaiḥ śanair yathā prāṇaḥ kandasaṃdhi[ṃ] nigacchati//\\+}
% \tl{yathābhyāsānurūpaṃ kandasaṃdhiṃ kandagatanāḍisaṃghavimalavivaraprānteṣu nigachati nitarāṃ gacchati tathā tathā kuṃbhakavṛddhiḥ sukharūpodbhavati sādhakasya yenoḍḍīyānena bandhena vāyuḥ proḍḍīyāste brahmanāḍyāṃ yato 'sau uḍḍīyānākhyaḥ smṛto bandha ārthaḥ sevyas tasmād yogibhiḥ siddhasevyaḥ//\\!}
% \end{versinnote}

% Cf. \emph{Yogaprakāśikā} 5.96
% \begin{versinnote}
% \tl{paścimaṃ tānam udare kārayec cibukaṃ hṛdi/\\+}
% \tl{śanaiḥ śanairyathā prāṇaḥ skandhasaṅge na gacchati//\\+}
% \tl{uktalakṣaṇe vajrāsane baddhe satī gulphadeśasamīpena meḍhraṃ vā prapīḍayet cibukaṃ hṛdi kṛtvā prāṇasyordhvasañcalanaṃ kārayet tena prāṇ[a]ḥ skandhasandhiṃ gacchatīty arthaḥ// \\!}
% \end{versinnote}
\end{testimonia}
%</ts64>

%<*cm64>
\begin{philcomm}[hp03_064]
In 3.64b, the reading \emph{gale} (`in the throat') is very well attested by manuscripts of the source text, the \emph{Gorakṣaśataka}, and the \emph{Hathapradīpikā} (including all three \textalpha\ witnesses). Its meaning is not entirely clear to us as the `backward stretch' (\emph{paścimaṃ tānam}) usually occurs above and below the navel when the \emph{uḍḍiyāṇa} lock is applied, as stated above in verse 3.60. We have not seen the neck mentioned in this regard in any other premodern work, and the absence of \emph{ca} suggests that \emph{gale} may be a corruption. The alternative reading \emph{cibukaṃ hṛdi} in manuscripts of the \emph{Hathapradīpikā} on lower branches of the stemma and in the testimonia is a reference to \emph{jālan\-dhara\-bandha} and appears to be a patch. 
% śanaiḥ śanaiḥ "gradually" for ca?
%In 3.64, the mention of the stretch in the chest (\emph{hṛdaya}) may also be consistent with 3.60 in so far as `above the navel' might include the lower region of the chest. 

\ It is possible that the practice of \textit{uḍḍiyāṇa} might affect the throat, as reported by Dr M.\,M.\,Gore (2005: 144). Drawing on x-ray experiments on \emph{uḍḍiyāṇabandha} conducted at the Kaivalyadhama Yoga Institute, the article mentions a sub-atmospheric (negative) pressure in visceral cavities, such as the oesophagus and stomach, as a physiological effect of applying \emph{uḍḍiyāṇa}.  
\end{philcomm}
%</cm64>
% `Anatomy and Physiology of Yogic Practices,' Dr. Makarand Madhukar Gore. New Delhi: Shri Jainendra Press. 2005.
% Remove crux marks in the edition as gale is not gibberish
% Is the KDham experiment relevant? It reports an effect of uddiyana but the text is saying to perform a stretch of the throat, ie the action not the result. JB: the negative pressure in the oesophagus might explain why a 'stretch' is felt in the throat (i.e., the idea of stretching the throat by doing uḍḍiyāṇa is plausible).   

\begin{metre}[hp03_064]
Anuṣṭubh (a: na-vipulā)
\end{metre}

%%%%%%%%%%
\subsection*{3.65}
%<*tr65>
\begin{translation}[hp03_065]
\emph{Uḍḍiyāṇa} is the best of all the locks. When the \emph{uḍḍiyāṇa} lock is firm, liberation becomes easy.
\end{translation}
%</tr65>

%<*sc65>
%\begin{sources}[hp03_065]
%\end{sources}
%</sc65>

%<*ts65>
\begin{testimonia}[hp03_065]
\emph{Yogacintāmaṇi} f.\,76v (\attr HP)
\begin{variants}
uttamo~] hy uttamo YCM\sep
uḍḍiyāṇakaḥ~] uḍḍiyānakaḥ YCM\sep
uḍḍīyāṇe~] uḍḍiyāne YCM\sep
muktiḥ svābhāvikī~] mūlaḥ svābhāviko YCM
\end{variants}

% \begin{versinnote}
% \tl{sarveṣām eva bandhānāṃ hy uttamo hy uḍḍiyānakaḥ/\\+}
% \tl{uḍḍiyāne dṛḍhe bandhe mūlaḥ svābhāviko bhavet//\\!}
% \end{versinnote}

% \emph{Yogaprakāśikā} 5.97
% \begin{versinnote}
% \tl{sarveṣām eva bandhānām uttamo hy uḍḍiyāṇakaḥ/\\+}
% \tl{uḍḍiyāṇe dṛḍhe bandhe mūlaṃ svābhāvikaṃ bhavet// \\+}
% \tl{uḍḍīyānabandham upasamharati sarveṣām iti// mūlam iti mūlabandho 'nāyāsena sidhyatīty arthaḥ//\\!}
% \end{versinnote}
\end{testimonia}
%</ts65>

%<*cm65>
%\begin{philcomm}[hp03_065]
%The reading \textit{mukti} is very well attested (alpha group, etc.) and often occurs with \textit{svābhāvikī}.
%The reading \textit{mūla} also makes sense as it elucidates on the relationship between these two locks.
% uḍḍiyāne kṛte bandhe mūlaṃ svābhāvikaṃ bhavet (Gr2,Gr3)
% uḍḍiyāṇe dṛḍhe bandhe muktiḥ svābhāvikī bhavet
%[JB: no issue now as alpha supports the readings we have adopted]
%\end{philcomm}
%</cm65>


%%%%%%%%%%
\subsection*{3.66 heading}
%<*tr66a>
\begin{translation}[hp03_066a]
Now the \sl{jālandhara} lock:
\end{translation}
%</tr66a>

%<*cm66a>
% \begin{philcomm}[hp03_066a]
% \end{philcomm}
%</cm66a>

%%%%%%%%%%
\subsection*{3.66}
%<*tr66>
\begin{translation}[hp03_066]
{}[The yogi] should contract the throat and firmly place the chin on the chest. This is the lock called \sl{jālandhara}. It prevents loss of the nectar of immortality.
%%?? JB and MD we should reconsider the sthāpayed dṛḍham icchayā reading: perhaps, we could understand it as, "Having contracted the throat against the chest, [the yogi] should hold it firmly, as he likes." The dṛḍham icchayā reading is in the ŚDP, so it was circulating before the HP was composed.
\end{translation}
%</tr66>

%<*sc66>
\begin{sources}[hp03_066]%apparatus, delete "inserted" from "Before this verse, ε1 has Gorakṣaśataka 61cd–62ab inserted." [MD: done]
\emph{Dattātreyayogaśāstra} 138
\begin{variants}
sthāpayec cibukaṃ dṛḍham DYŚ~] sthāpayed dṛḍhayā dhiyā DYŚ\vl, sthāpayed dṛḍham icchayā DYŚ\vl\sep
bandho jālandharākhyo 'yaṃ~] jālandharo bandha eṣa DYŚ    
\end{variants}

% \begin{versinnote}
% \tl{kaṇṭham ākuñcya hṛdaye sthāpayec cibukaṃ dṛḍham/\\+}
% \tl{jālandharo bandha eṣa amṛtāvyayakārakaḥ//\\!}
% \end{versinnote}
% \begin{appinnote}
% \tl{\textbf{138b} sthāpayec cibukaṃ dṛḍham] HRPTPT; sthāpayed dṛḍhayā dhiyā YTU, sthāpayed dṛḍham icchayā \emph{cett.} \\!}
% \end{appinnote}
\end{sources}
%</sc66>

%<*ts66>
\begin{testimonia}[hp03_066]
\emph{Haṭharatnāvalī} 2.66, \emph{Yogacintāmaṇi} f.\,77r (\attr \emph{Yogabīja})
\begin{variants}
cibukaṃ dṛḍham HRĀ~] dṛḍham icchayā YCM\sep
amṛtāvyayakārakaḥ YCM~] jarāmṛtyuvināśakaḥ HRĀ    
\end{variants}

% \begin{versinnote}
% \tl{kaṇṭham ākuñcya hṛdaye sthāpayec cibukaṃ dṛḍham/\\+}
% \tl{bandho jālandharākhyo 'yaṃ jarāmṛtyuvināśakaḥ//\\!}
% \end{versinnote}

% \emph{Yogacintāmaṇi} f.\,77r (\attr to the \emph{Yogabīja})
% \begin{versinnote}
% \tl{kaṇṭham ākuñcya hṛdaye sthāpayed dṛḍham icchayā/\\+}
% \tl{bandho jālandharākhyo 'yam amṛtāvyayakārakaḥ//\\!}
% \end{versinnote}

% \emph{Yogabīja} 109 (south Indian recension)
% \begin{versinnote}
% \tl{kaṇṭham ākuñcya hṛdaye sthāpayed dṛḍham icchayā/\\+}
% \tl{bandho jālandharākhyo 'yam amṛtāvyayakārakaḥ//\\!}
% \end{versinnote}
% \begin{appinnote}
% \tl{\textbf{d} amṛtāvyaya°] amṛtavyaya° \vl \\!}
% \end{appinnote}

\end{testimonia}
%</ts66>

%<*cm66>
\begin{philcomm}[hp03_066]
Manuscripts of the \textalpha, \textdelta, \texteta\ and \textpi\ groups have \emph{sthāpayed dṛḍham icchayā} (`one should place it firmly as desired') in the second verse quarter, which is also well-attested in the transmission of the source text, the \emph{Dattātreyayogaśāstra}. This reading seems secondary because, in a subsequent verse (3.68), contracting the throat is the main feature of \emph{jālandharabandha}, so it seems contradictory to say that it may be done `as one likes' in 3.66b. %The word \emph{icchayā} may have crept in to this verse because someone wanted to make this practice optional in light of 3.21*2, % MD: 3.22, now 3.21*2, is not  in the α mss, so this argument is not appropriate.
%or it might be a corruption of \emph{hṛdaye sthāpayed dṛḍhaṃ niścayam} which is found in some other manuscripts.
\end{philcomm}
%</cm66>

\begin{metre}[hp03_066]
Anuṣṭubh (a: na-vipulā)
\end{metre}

%%%%%%%%%%
\subsection*{3.67}
%<*tr67>
\begin{translation}[hp03_067]
Because it binds all the channels in which the liquid from the void flows down, it is [called] the \sl{jālandhara} lock. It gets rid of all problems in the throat.%
\end{translation}
%</tr67>

%<*sc67>
\begin{sources}[hp03_067]
\emph{Vivekamārtaṇḍa} 45
\mylb
% \begin{versinnote}
% \tl{badhnāti hi śirājālam adhogāminabhojalam/\\+}
% \tl{tato jālandharo bandhaḥ kaṇṭhaduḥkhaughanāśanaḥ//\\!}
% \end{versinnote}
\end{sources}
%</sc67>

%<*ts67>
\begin{testimonia}[hp03_067]
\emph{Haṭharatnāvalī} 2.66ef–2.67ab, \emph{Yogacintāmaṇi} f.\,77r (\attr HP), \emph{Yuktabhavadeva} 7.230 (\attr \emph{Śivayoga})
\begin{variants}
badhnāti hi HRĀ YBhD~] badhnātīha YCM\sep
adhogāminabhojalam YCM~] nādho yāti nabhojalam HRĀ YBhD\sep
bandhaḥ HRĀ YBhD~] proktaḥ YCM\sep
kaṇṭhaduḥkhaughanāśanaḥ YBhD~] kaṇṭhasaṅkocane kṛte HRĀ, kaṇṭhe duḥkhaughanāśanaḥ YCM
\end{variants}

% \begin{versinnote}
% \tl{badhnāti hi śirājālaṃ nādho yāti nabhojalam/\\+}
% \tl{tato jālandharo bandhaḥ kaṇṭhasaṅkocane kṛte //\\!}
% \end{versinnote}

% \emph{Yogacintāmaṇi} f.\,77r (\attr to the \emph{Haṭhapradīpikā})
% \begin{versinnote}
% \tl{badhnātīha śirājālam adhogāminabhojalam/\\+}
% \tl{tato jālandharaḥ proktaḥ kaṇṭhe duḥkhaughanāśanaḥ//\\!}
% \end{versinnote}

% \emph{Yuktabhavadeva} 7.230 (\attr to the \emph{Śivayoga})
% \begin{versinnote}
% \tl{badhnāti hi śirājālaṃ nādho yāti nabhojalam/\\+}
% \tl{tato jālandharo bandhaḥ kaṇṭhaduḥkhaughanāśanaḥ//\\!}
% \end{versinnote}

%Skandapurāṇa 4.41.149
%\begin{versinnote}
%\tl{badhnāti hi śirājālam adhogāmi nabhojalam/\\+}
%\tl{eṣa jālandharo bandhaḥ kaṇṭhe duḥkhaughanāśanaḥ// \\!}
%\end{versinnote}
\end{testimonia}
%</ts67>

%<*cm67>
%\begin{philcomm}[hp03_067]
%Judit: Read adhogāminabhojalam in which the fluid of the void flows down.
% JB: I've tweaked the translation to reflect Judit's suggestion.
%\end{philcomm}
%</cm67>


%%%%%%%%%%
\subsection*{3.68}
%<*tr68>
\begin{translation}[hp03_068]
When the \sl{jālandhara} lock is performed, its defining feature being the contraction of the throat, nectar does not fall in the fire and the breath does not escape.
\end{translation}
%</tr68>

%<*sc68>
\begin{sources}[hp03_068]
\emph{Vivekamārtaṇḍa} 46
\begin{variants}
pradhāvati VM~] prakupyati VM\vl     
\end{variants}
% \begin{versinnote}
% \tl{jālandhare kṛte bandhe kaṇṭhasaṃkocalakṣaṇe/\\+}
% \tl{na pīyūṣaṃ pataty agnau na ca vāyuḥ pradhāvati//\\!}
% \end{versinnote}
% \begin{appinnote}
% \tl{\textbf{d} pradhāvati~] ; prakupyati AGT \\!}
% \end{appinnote}
\end{sources}
%</sc68>

%<*ts68>
\begin{testimonia}[hp03_068]
\emph{Yogacintāmaṇi} f.\,77v (\attr HP), \emph{Yuktabhavadeva} 7.231 (\attr \emph{Śivayoga})
\begin{variants}
pradhāvati~] prakupyati YCM YBhD   
\end{variants}

% \begin{versinnote}
% \tl{jālandhare kṛte bandhe kaṇṭhasaṃkocalakṣaṇe/\\+}
% \tl{na pīyūṣaṃ pataty agnau na ca vāyuḥ prakupyati//\\!}
% \end{versinnote}

% \emph{Yuktabhavadeva} 7.231 (\attr to the \emph{Śivayoga})
% \begin{versinnote}
% \tl{jālandhare kṛte bandhe kaṇṭhasaṃkocalakṣaṇe/\\+}
% \tl{na pīyūṣaṃ pataty agnau na ca vāyuḥ prakupyati//\\!}
% \end{versinnote}
\end{testimonia}
%</ts68>

%<*cm68>
%\begin{philcomm}[hp03_068]
%\end{philcomm}
%</cm68>


%%%%%%%%%%
\subsection*{3.69}
%<*tr69>
\begin{translation}[hp03_069]
By contracting the throat, [the yogi] firmly blocks the two channels. This should be known as the middle cakra, which binds [the mind to] the sixteen supports [in the body].
\end{translation}
%</tr69>

%<*sc69>
\begin{sources}[hp03_069]
\emph{Jñānasāra} 2.4 %?? JB is this a new possible source? Check ṣoḍaśārārdha in other ms.? 
\begin{variants}
saṃkocanenaiva~] saṅkocanaṃ kṛtvā JS\sep 
dṛḍham~] dhruvam JS\sep
jñeyaṃ~] bhadre JS\sep
%ṣoḍaśādhāra~] ṣoḍaśārārdha JS % JB:Someone proficient in Śāradā might check this reading in the ms? Done
\end{variants}

% kaṇṭhasaṅkocanaṃ kṛtvā dve nāḍyau stambhayed dhruvam |
% madhyacakramidaṃ bhadre ṣoḍaśārārdhabandhanam || 4 ||
\end{sources}
%</sc69>

%<*ts69>
\begin{testimonia}[hp03_069]
% \emph{Yogaviṣaya} 19ab % probably not a source text as it seems to be a fragment
% \begin{versinnote}
% \tl{kaṇṭhasaṃkocanaṃ kṛtvā dve nāḍyau stambhayed dṛḍham/\\+}
% \tl{rasanāpīḍyamānās tu ṣoḍaśaś cordhvagāminī//\\!}
% \end{versinnote}
%  JB: As I mention in my intro chapter, the YV is a fragment of the Yogadīpikā, a recent work that we don't report elsewhere. So I've deleted this entry. 

\emph{Yogacintāmaṇi} f.\,77v (\attr HP)
\lb
% \begin{versinnote}
% \tl{madhyacakram idaṃ jñeyaṃ ṣoḍaśādhārabandhanam/\\!}
% \end{versinnote}

% \emph{Yogakarṇikā} 85 (ab only)
% \begin{versinnote}
% \tl{kaṇṭhasaṅkocanenaiva dve nāḍye kumbhayed dṛḍham/\\!}
% \end{versinnote}

% \emph{Yogasarasaṅgraha} p.58
% \begin{versinnote}
% \tl{kaṇṭhasaṃkocane nanaiva dvināḍyo stambhayed dṛḍham/\\+}
% \tl{ayaṃ bandho mayā proktaḥ ṣoḍaśādhārabandhanam//\\!}
% \end{versinnote}

Cf.~\emph{Haṭhayogasaṃhitā} p.\,23
\begin{versinnote}
\tl{kaṇṭhasaṅkocanaṃ kṛtvā cibukaṃ hṛdaye nyaset/\\+}
\tl{jālandhare kṛte bandhe ṣoḍaśādhārabandhanam//\\!}
\end{versinnote}
\end{testimonia}
%</ts69>

%<*cm69>
\begin{philcomm}[hp03_069]
The import of the second line of this verse is obscure to us. In \emph{Jyotsnā} 3.73, Brahmānanda says that the middle cakra (\emph{madhyacakra}) is \emph{viśuddha} cakra. The main reason for this appears to be that this cakra is located in the throat and the \sl{jālandhara} lock is a contraction of the throat. However, he also seems to connect the \emph{viśuddha} cakra to the sixteen supports (\emph{ṣoḍaśādhāra}) at the end of this verse, perhaps because this cakra has sixteen petals (as mentioned in 3.46). On the meaning of \emph{ādhāra} in yogic contexts, see entry no.\,3 in the \emph{Tāntrikābhidhānakośa} vol.\,1 2000: 191.%ref format? s.v. ādhāra? [MD: fixed]
% ??JB In the Jñānasāra, madhyacakra is between vyomacakra (JS 2.3 ~ HP 3.34*1) and mūlacakram (JS 2.5). The context of this verse in the JS is khecarī, but it seems to have been included here becuse of kaṇṭhasaṅkocanaṃ kṛtvā (i.e, Jālandhara). Perhaps, the JS is the source of this verse as it has the passage with the triad of these cakras (vyoma, cakra, mūla), and it may be the source of the verse on vyomacakra, which has not made it into our edition. Rewrite note?

%Jyotsnā: kaṇṭhasaṃkocaneneti. dṛḍham gāḍhaṃ kaṇṭhasaṃkocanenaiva kaṇṭhasaṃkocanamātreṇa dve nāḍyau iḍāpiṅgale stambhayed bandhayet. ayaṃ jālandhara iti kartṛpadādhyāhāraḥ. idaṃ kaṇṭhasthāne sthitaṃ viśuddhākhyaṃ cakram madhyacakram madhyamaṃ cakram jñeyam/ kīdṛśam ? ṣoḍaśādhārabandhanam ṣoḍaśasaṅkhyākā ye ādhārā aṅguṣṭhādhārādibrahmarandhrāntās teṣāṃ bandhanaṃ bandhanaprakāram/
%Yogaprakāśikā: uktalakṣaṇaṃ nāḍīstambhanaṃ madhyacakrasaṃjñakaṃ bhavati// siddhasiddhāntapaddhatau pratipāditānāṃ ṣoḍaśādhārāṇāṃ nirodhane sādhanaṃ ca//
\end{philcomm}
%</cm69>


%%%%%%%%%%
\subsection*{3.70}
%<*tr70>
\begin{translation}[hp03_070]
This triad of locks is the best [and] has been practised by the great Siddhas. Yogis know it to be a method of all systems of Haṭha.
\end{translation}
%</tr70>

%<*sc70>
%\begin{sources}[hp03_070]
%\end{sources}
%</sc70>

%<*ts70>
\begin{testimonia}[hp03_070]
\emph{Haṭharatnāvalī} 2.68, \emph{Yogacintāmaṇi} f.\,77v (\attr HP), \emph{Haṭhatattvakaumudī} 15.24
\begin{variants}
bandhatrayam idaṃ HRĀ YCM~] idaṃ bandhatrayaṃ HTK\sep
mahāsiddhair niṣevitam~] mahāsiddhaiś ca sevitam HRĀ, mahāsiddhaniṣevitam YCM, marujjayasusiddhadam HTK\sep
haṭha YCM~] yoga HRĀ HTK\sep
sādhanaṃ HRĀ HTK~] sādhane YCM\sep
yogino viduḥ HRĀ HTK~] yoginām iti YCM
\end{variants}

% \begin{versinnote}
% \tl{bandhatrayam idaṃ śreṣṭhaṃ mahāsiddhaiś ca sevitam/\\+}
% \tl{sarveṣāṃ yogatantrāṇāṃ sādhanaṃ yogino viduḥ//\\!}
% \end{versinnote}

% \emph{Yogacintāmaṇi} f.\,77v (\attr to the \emph{Haṭhapradīpikā})
% \begin{versinnote}
% \tl{bandhatrayam idaṃ śreṣṭhaṃ mahāsiddhaniṣevitam/\\+}
% \tl{sarveṣāṃ haṭhatantrāṇāṃ sādhane yoginām iti//\\!}
% \end{versinnote}

% \emph{Haṭhatattvakaumudī} 15.24
% \begin{versinnote}
% \tl{idaṃ bandhatrayaṃ śreṣṭhaṃ marujjayasusiddhadam/\\+}
% \tl{sarveṣāṃ yogatantrāṇāṃ sādhanaṃ yogino viduḥ//\\!}
% \end{versinnote}

%Yogakarṇikā 85
%\begin{versinnote}
%\tl{bandhatrayam idaṃ guhyaṃ mahāsiddhiniṣevitam// \\!}
%\end{versinnote}
\end{testimonia}
%</ts70>

%<*cm70>
%\begin{philcomm}[hp03_070]
%\end{philcomm}
%</cm70>


%%%%%%%%%%
\subsection*{3.70*1}
%<*tr70-1>
\begin{translation}[hp03_070_1]
By immediately contracting the lower [part of the body] (i.e.~by the root lock) when the neck has been contracted (i.e.~by the \emph{jālandhara} lock) and by stretching the abdomen backwards in the middle [of the body] (i.e.~by the \emph{uḍḍiyāṇa} lock), the breath enters the channel of Brahman.
\end{translation}
%</tr70-1>
% greyscaled as alpha

%<*sc70-1>
\begin{sources}[hp03_070_1]
\emph{Gorakṣaśataka} 63 (see 2.46)
\end{sources}
%</sc70-1>

%<*ts70-1>
\begin{testimonia}[hp03_070_1]
\emph{Haṭharatnāvalī} 2.8, \emph{Yogacintāmaṇi} f.\,80r, \emph{Yuktabhavadeva} 7.95 and \emph{Haṭhatattvakaumudī} 15.25 (see 2.46).

\end{testimonia}
%</ts70-1>

%<*cm70-1>
%\begin{philcomm}[hp03_070_1]
%In the second chapter, Group 2 has the reading \emph{kuñcanenāśu kaṇṭhasaṃkocane kṛte}.
%\end{philcomm}
%</cm70-1>


%%%%%%%%%%
\subsection*{3.71}
%<*tr71>
\begin{translation}[hp03_071]
[The yogi] should contract the place of the root and do the \emph{uḍḍiyāṇa} [lock]. He should [then] block the Iḍā and Piṅgalā [channels] and make [the breath] flow in the rear pathway.
\end{translation}
%</tr71>

%<*sc71>
%\begin{sources}[hp03_071]
%\end{sources}
%</sc71>

%<*ts71>
\begin{testimonia}[hp03_071]
\emph{Haṭharatnāvalī} 2.70, \emph{Yogacintāmaṇi} f.\,79v (\attr HP)
\begin{variants}
samākuñcya HRĀ~] samākṛṣya YCM\sep
uḍḍīyāṇaṃ~] uḍḍiyānaṃ HRĀ, YCM\sep 
paścimaṃ pathaṃ HRĀ~] paścime pathi YCM
\end{variants}

% \begin{versinnote}
% \tl{mūlasthānaṃ samākuñcya uḍḍiyānaṃ tu kārayet/\\+}
% \tl{iḍāṃ ca piṅgalāṃ baddhvā vāhayet paścimaṃ pathaṃ//\\!}
% \end{versinnote}

% \emph{Yogacintāmaṇi} f.\,79v (\attr to the \emph{Haṭhapradīpikā})
% \begin{versinnote}
% \tl{mūlasthānaṃ samākṛṣya uḍḍiyānaṃ tu kārayet/\\+}
% \tl{iḍāṃ ca piṅgalāṃ baddhvā vāhayet paścime pathi//\\!}
% \end{versinnote}

% \emph{Haṭhatattvakaumudī} 15.25–27
% \begin{versinnote}
% \tl{mūlasthānaṃ samākuñcya uḍyānaṃ tu kārayet/\\+}
% \tl{iḍāṃ ca piṃgalāṃ baddhvā vāhayet paścimāpathaṃ//\\+}
% \tl{mūlasthānaṃ mārgasaṃkocanaṃ vidhāya uḍyānam udarasaṃkocanaṃ tataḥ/ iḍāṃ piṃgalāṃ baddhvā kaṇṭhasaṃkocanena paścimapathaṃ pṛṣṭhavaṃśamārge pavanaṃ vāhayet kuryāt//\\!}
% \end{versinnote}
\end{testimonia}
%</ts71>


%<*cm71>
%\begin{philcomm}[hp03_071]
%No known source?
%\end{philcomm}
%</cm71>


%%%%%%%%%%
\subsection*{3.72}
%<*tr72>
\begin{translation}[hp03_072]
By this method alone, the breath attains dissolution. Then death does not arise nor old age, disease and the like.
\end{translation}
%</tr72>

%<*sc72>
%\begin{sources}[hp03_072]
%\end{sources}
%</sc72>

%<*ts72>
\begin{testimonia}[hp03_072]
\emph{Haṭharatnāvalī} 2.71, \emph{Yogacintāmaṇi} f.\,79v (\attr HP)
\begin{variants}
sevayet YCM~] prayāti HRĀ\sep
pavano layam HRĀ~] pavanālayam YCM
\end{variants}

% \begin{versinnote}
% \tl{anenaiva vidhānena prayāti pavano layam/\\+}
% \tl{tato na jāyate mṛtyur jarārogādikaṃ tathā// \\!}
% \end{versinnote}

% \emph{Yogacintāmaṇi} f.\,79v (\attr to the \emph{Haṭhapradīpikā})
% \begin{versinnote}
% \tl{anenaiva vidhānena sevayet pavanālayam/\\+}
% \tl{tato na jāyate mṛtyur jarārogādikaṃ tathā//\\!}
% \end{versinnote}

\end{testimonia}
%</ts72>

%<*cm72>
\begin{philcomm}[hp03_072]
The \textalpha, \textpi\ and delta groups have the reading \emph{sevayet pavanālayam} as the second verse quarter of this verse. It renders the meaning, `by this method alone, one should honour the abode of the breath.' As far as we know, the compound \emph{pavanālaya} does not occur in other yoga texts. The similar compound \emph{prāṇālaya}) is mentioned in other yoga texts, such as the \emph{Yogayājñavalkya} (4.52–53), but it refers to the locations in the body where \emph{prāṇa} resides, as opposed to the other bodily winds.
% JB: reconsider the alpha reading, which is well attested in other groups? Maybe \emph{sevayet pavanālayam} is just some general laudatory statement?
%?? MD: We need anyway a commentary. sevayet pavano layam is a mix of two recensions: we should adopt either prayāti pavano layam or sevayet pavaṇālayaṃ (alpha reading)
%Judit: ādikam is okay.
\end{philcomm}
%</cm72>

% MD 2024-8: I would again suggest that we consider the reading "jarārogādi kā kathā (not to mention age, sickness, etc.)". In terms of content, it fits better than "jarārogādikaṃ tathā". The only problem is that "kā kathā" is usually constructed with loc., gen. or prati with acc. Perhaps "jarārogādi" can be treated like "jarārogādi prati"?
% Cf. Yogabīja 67ab: vyādhayas tasya naśyanti chedaghātādikās tathā/
% °ghātādikās tathā~] \Sone\Sfour\Sfive\YCM\YSU, 
% -ghātādi kā kathā \Etwo\Nthree\Nfour\Nn, (vyādhayas tasya naśyanti chedaghātādi kā kathā)
% -dhyānādi+ kathaṃ \Eone, 
% -yātādi kā kathā \Ptwo, 
% -khātādikās tathā \Stwo\Sthree, 
% -khātādi kā kathā \Sy, 
% -ghātādikāḥ kathā \None\Ntwo\Nfive, 
% -ghātādikā vyathāḥ \Pthree, 



%%%%%%%%%%
\subsection*{3.73 heading}
%<*tr73a>
\begin{translation}[hp03_073a]
Now, the inverted bodily position:
\end{translation}
%</tr73a>

%<*cm73a>
% \begin{philcomm}[hp03_073a]
% \end{philcomm}
%</cm73a>

%%%%%%%%%%
\subsection*{3.73}
%<*tr73>
\begin{translation}[hp03_073]
The navel is up, the palate down; the sun up, the moon down: the bodily position called “inverted” is obtained through the teaching of a guru.
\end{translation}
%</tr73>

%<*sc73>
\begin{sources}[hp03_073]
\emph{Vivekamārtaṇḍa} 115
\mylb
% \begin{versinnote}
% \tl{ūrdhvaṃ nābhir adhas tālur ūrdhvaṃ bhānur adhaḥ śaśī/\\+}
% \tl{karaṇī viparītākhyā guruvākyena labhyate//\\!}
% \end{versinnote}
\end{sources}
%</sc73>

%<*ts73>
\begin{testimonia}[hp03_073]
\emph{Haṭharatnāvalī} 2.74, \emph{Yogacintāmaṇi} f.\,73r (\attr HP), \emph{Yuktabhavadeva} 7.236 (\attr Gorakṣanātha)
\begin{variants}
ūrdhvaṃ nābhir HRĀ YBhD~] ūrdhvanābhir YCM\sep 
ūrdhvaṃ bhānur HRĀ YBhD~] ūrdhvabhānur YCM\sep
guruvākyena labhyate HRĀ~] sarvavyādhivināśinī YCM, guruvaktreṇa gamyate YBhD
\end{variants}
% \begin{versinnote}
% \tl{ūrdhvaṃ nābhir adhas tālur ūrdhvaṃ bhānur adhaḥ śaśī/\\+}
% \tl{karaṇī viparītākhyā guruvākyena labhyate//\\!}
% \end{versinnote}

% \emph{Yogacintāmaṇi} f.\,73r (\attr to the \emph{Haṭhapradīpikā})
% \begin{versinnote}
% \tl{ūrdhvanābhir adhastālur ūrdhvabhānur adhaḥ śaśī/\\+}
% \tl{karaṇī viparītākhyā sarvavyādhivināśinī//\\!}
% \end{versinnote}

% \emph{Yuktabhavadeva} 7.236 (\attr to Gorakṣanātha)
% \begin{versinnote}
% \tl{ūrdhvaṃ nābhir adhas tālur ūrdhvaṃ bhānur adhaḥ śaśī/\\+}
% \tl{karaṇī viparītākhyā guruvaktreṇa gamyate //\\!}
% \end{versinnote}
\end{testimonia}
%</ts73>

%<*cm73>
%\begin{philcomm}[hp03_073]
%\end{philcomm}
%</cm73>


%%%%%%%%%%
\subsection*{3.74}
%<*tr74>
\begin{translation}[hp03_074]
The bodily position called “inverted" destroys all diseases. For [the yogi] who regularly engages in [its] practice, it increases the digestive fire.
\end{translation}
%</tr74>

%<*sc74>
\begin{sources}[hp03_074]
\emph{Dattātreyayogaśāstra} 146
\begin{variants}
karaṇī~] karaṇaṃ DYŚ\sep
vināśinī~] vināśanam DYŚ\sep
jaṭharāgnivivardhanī~] jaṭharāgnir vivardhate DYŚ
\end{variants}

% \begin{versinnote}
% \tl{karaṇaṃ viparītākhyaṃ sarvavyādhivināśanam/\\+}
% \tl{nityam abhyāsayuktasya jaṭharāgnir vivardhate//\\!}
% \end{versinnote}
\end{sources}
%</sc74>

%<*ts74>
\begin{testimonia}[hp03_074]
\emph{Haṭharatnāvalī} 2.75, \emph{Yogacintāmaṇi} f.\,78r (cd only) (\attr Dattātreya)
\begin{variants}
vivardhanī~] vivardhinī HRĀ, vivardhanam YCM    
\end{variants}
% \begin{versinnote}
% \tl{karaṇī viparītākhyā sarvavyādhivināśinī/\\+}
% \tl{nityam abhyāsayuktasya jaṭharāgnivivardinī //\\!}
% \end{versinnote}

% \emph{Yogacintāmaṇi} f.\,78r (\attr to Dattātreya)
% \begin{versinnote}
% \tl{nityam abhyāsayuktasya jaṭharāgnivivardhanam//\\!}
% \end{versinnote}
\end{testimonia}
%</ts74>

%<*cm74>
%\begin{philcomm}[hp03_074]
%\end{philcomm}
%</cm74>


%%%%%%%%%%
\subsection*{3.75}
%<*tr75>
\begin{translation}[hp03_075]
A lot of food should be provided for the practitioner. If the practitioner eats little, fire will quickly consume his body.
\end{translation}
%</tr75>

%<*sc75>
\begin{sources}[hp03_075]
\emph{Dattātreyayogaśāstra} 147
\begin{variants}
sādhakasya tu~] sāṃkṛte dhruvam DYŚ
\end{variants}
% \begin{versinnote}
% \tl{āhāro bahulas tasya saṃpādyaḥ sāṃkṛte dhruvam/\\+}
% \tl{alpāhāro yadi bhaved agnir dehaṃ dahet kṣaṇāt//\\!}
% \end{versinnote}
\end{sources}
%</sc75>

%<*ts75>
\begin{testimonia}[hp03_075]

\emph{Haṭharatnāvalī} 2.76, \emph{Yogacintāmaṇi} f.\,78r (\attr Dattātreya)
\begin{variants}
sādhakasya tu~] sādhakena vai HRĀ, sāṃkṛte dhruvam YCM\sep
agnir deham~] deham agnir HRĀ, agnir dāhaṃ YCM\sep
dahet kṣaṇāt~] dahet kramāt HRĀ, karoti vai YCM 
\end{variants}

% \begin{versinnote}
% \tl{āhāro bahulas tasya sampādyaḥ sādhakena vai/\\+}
% \tl{alpāhāro yadi bhaved deham agnir dahet kramāt//\\!}
% \end{versinnote}

% \emph{Yogacintāmaṇi} f.\,78r (\attr to Dattātreya)
% \begin{versinnote}
% \tl{āhāro bahulas tasya saṃpādyaḥ sāṃkṛte dhruvam/\\+}
% \tl{alpāhāro yadi bhaved agnir dāhaṃ karoti vai//\\!}
% \end{versinnote}

Cf.\,\emph{Yuktabhavadeva} 7.238
\begin{versinnote}
\tl{asyāṃ kriyamāṇāyāṃ sādhakasya bhakṣyaṃ bahulaṃ sampādyam anyathā pravṛddho jāṭharānalo dhātuṃ dahatīti//\\!}
\end{versinnote}
\end{testimonia}
%</ts75>

%<*cm75>
\begin{philcomm}[hp03_075]
Svātmārāma has removed the vocative from the \emph{Dattātreyayogaśāstra}, changing \emph{sāṃkṛte dhruvam} to \emph{sādhakasya tu}.
\end{philcomm}
%</cm75>

\begin{metre}[hp03_075]
Anuṣṭubh (c: na-vipulā)
\end{metre}

%%%%%%%%%%
\subsection*{3.76}
%<*tr76>
\begin{translation}[hp03_076]
On the first day [the yogi] should keep his head down and his feet up for a short while, and he should [then] practise for a little longer each day.
\end{translation}
%</tr76>

%<*sc76>
\begin{sources}[hp03_076]
\emph{Dattātreyayogaśāstra} 148c–149b
\begin{variants}
    adhaḥśirāś DYŚ~] adhaḥśiraś DYŚ\vl
\end{variants}
% \begin{versinnote}
% \tl{adhaḥśirāś cordhvapādaḥ kṣaṇaṃ syāt prathame dine//\\+}
% \tl{kṣaṇāc ca kiṃ cid adhikam abhyasec ca dine dine/\\!}
% \end{versinnote}
\end{sources}
%</sc76>

%<*ts76>
\begin{testimonia}[hp03_076]
\emph{Haṭharatnāvalī} 2.77, \emph{Yogacintāmaṇi} f.\,78r (\attr Dattātreya)
\begin{variants}
śirāś YCM~] śiraś HRĀ\sep
pādaḥ HRĀ\vl\ YCM~] pādau HRĀ    
\end{variants}

% \begin{versinnote}
% \tl{adhaḥ śiraś cordhvapādau kṣaṇaṃ syāt prathame dine/\\+}
% \tl{kṣaṇāc ca kiñ cid adhikam abhyasec ca dine dine//\\!}
% \end{versinnote}
% \begin{appinnote}
% \tl{\textbf{a} cordhvapādau~] cordhvapādaḥ \vl \\!}
% \end{appinnote}

% \emph{Yogacintāmaṇi} f.\,78r (\attr to Dattātreya)
% \begin{versinnote}
% \tl{adhaḥśirāś cordhvapādaḥ kṣaṇaṃ syāt prathame dine/\\+}
% \tl{kṣaṇāc ca kiñ cid adhikam abhyasec ca dine dine//\\!}
% \end{versinnote}

Cf.\,\emph{Yuktabhavadeva} 7.237
\begin{versinnote}
\tl{sa ca prathamadine kṣaṇamātraṃ vidheyā dvitīyadine/ kiñcidadhikaṃ kālam evaṃ yāmaparyantaṃ vidheyā/\\!}
\end{versinnote}
\end{testimonia}
%</ts76>

%<*cm76>
\begin{philcomm}[hp03_076]
It appears that the masculine stem form \emph{śira} was widely understood in the \emph{Haṭhapradīpikā}'s transmission instead of the more common \emph{śiras}, which would be rendered \emph{śirāḥ} at the end of a \emph{bahuvrīhi} compound.
\end{philcomm}
%</cm76>
%MD: Shall we adopt the reading adhaḥśiraś which is widely attested than the overcorrect °śirāś?

\begin{metre}[hp03_076]
Anuṣṭubh (a: ra-vipulā; c: na-vipulā)
\end{metre}

%%%%%%%%%%
\subsection*{3.77}
%<*tr77>
\begin{translation}[hp03_077]
After six months grey hair and wrinkles disappear. [The yogi] who regularly practises for three hours conquers death.
\end{translation}
%</tr77>

%<*sc77>
\begin{sources}[hp03_077]
\emph{Dattātreyayogaśāstra} 149c–150b
\begin{variants}
māsordhvaṃ na DYŚ~] māsāṃ hi na DYŚ\vl, māsāc ca na DYŚ\vl\sep 
tu~] hi DYŚ\sep
kālajit DYŚ~] yogavit DYŚ\vl
\end{variants}

% \begin{versinnote}
% \tl{valiś ca palitaṃ caiva ṣaṇmāsordhvaṃ na dṛśyate//\\+}
% \tl{yāmamātraṃ hi yo nityam abhyaset sa tu kālajit/\\!}
% \end{versinnote}
% \begin{appinnote}
% \tl{\textbf{149d} °māsordhvaṃ na~] °māsāṃ hi na M1, °māsāc ca na AM2, °māsārdhān na YTU, °māsān na tu HR, °māsārddhena \emph{Yogacintāmaṇi} \ \textbf{150b} kālajit~] yogavit πDYŚPT \\!}
% \end{appinnote}
\end{sources}
%</sc77>

%<*ts77>
\begin{testimonia}[hp03_077]
\emph{Haṭharatnāvalī} 2.78, \emph{Yogacintāmaṇi} f.\,78r (\attr Dattātreya), \emph{Yuktabhavadeva} 7.238 (\attr Gorakṣanātha)
\begin{variants}
valiś ca YCM~] valitaṃ HRĀ YBhD\sep
māsordhvaṃ na~] māsān na tu HRĀ, māsārdhe na YCM, māsārdhān na YBhD
\end{variants}

% \begin{versinnote}
% \tl{valitaṃ palitaṃ caiva ṣaṇmāsān na tu dṛśyate/\\+}
% \tl{yāmamātraṃ tu yo nityam abhyaset sa tu kālajit// \\!}
% \end{versinnote}

% \emph{Yogacintāmaṇi} f.\,78r (\attr to Dattātreya)
% \begin{versinnote}
% \tl{valiś ca palitaṃ caiva ṣaṇmāsārdhe na dṛśyate/\\+}
% \tl{yāmamātraṃ tu yo nityam abhyaset sa tu kālajit//\\!}
% \end{versinnote}

% \emph{Yuktabhavadeva} 7.238 (\attr to Gorakṣanātha)
% \begin{versinnote}
% \tl{valitaṃ palitaṃ caiva ṣaṇmāsārdhān na dṛśyate/\\+}
% \tl{yāmamātraṃ tu yo nityam abhyaset sa tu kālajit//\\!}
% \end{versinnote}

Cf.\,\emph{Haṭhatattvakaumudī} 14.3
\begin{versinnote}
\tl{ūrdhvapādo hy adhomastakaḥ syāt kṣaṇaṃ\\+}
\tl{vāsare 'thādime 'bhyāsaṃ vṛddhyā dhayet/\\+}
\tl{evam abhyāsato yāmamātraṃ sadā \\+}
\tl{mṛtyujit syāj jarājic ca ṣaṇmāsataḥ//\\!}
\end{versinnote}
\end{testimonia}
%</ts77>

%<*cm77>
\begin{philcomm}[hp03_077]
We have adopted the reading \emph{ṣaṇmāsordhvaṃ} in the second verse quarter. It is attested by manuscripts of the \emph{Dattātreyayogaśāstra} (the source text) and the \emph{Jyotsnā} (3.82). It makes good sense and explains the rather odd readings in \textalpha\ and other manuscripts, \emph{ṣaṇmāsār\-dhān}, \emph{ṣaṇmāsārdhaṃ} and \emph{ṣaṇmāsārdhe}. The \textgamma\ and \textdelta\ groups have a different verb as well, \emph{ṣaṇmāsārdhena naśyati}. %The original reading was likely \emph{ṣaṇmāsordhvaṃ na dṛśyate} because the compound \emph{ṣaṇmāsārdha} (`half of six months') is very strange and \emph{dṛśyate} is better attested. JM: I percented the last sentence as it seems unnecessary to me.
\end{philcomm}
%</cm77>

%%%%%%%%%%
\subsection*{3.77*1}
%<*tr77-1>
\begin{translation}[hp03_077_1]
The sun devours whatever nectar flows from the divine moon. As a result, the body is afflicted by old age.
\end{translation}
%</tr77-1>

%<*cm77-2>
\begin{philcomm}[hp03_077_1]
%See \manuref{4.10}, where this verse is also found.\lb
See 4.10--11, where these verses are also found.\lb

The \textalpha\ group do not have 3.77*1 and 3.77*2 in the third chapter (but rather in the fourth) and other manuscripts omit them as well (notably the \texteta\ group). For a discussion of these verses, see the introduction (add reference??).
\end{philcomm}
%</cm77-2>

%%%%%%%%%%
\subsection*{3.77*2}
%<*tr77-2>
\begin{translation}[hp03_077_2]
There is a divine bodily position for this, which blocks the mouth of the sun. It is to be known from the teaching of a guru and not through countless scriptural teachings.%JM: countless scriptural teachings? interpretations seems a bit much.
\end{translation}
%</tr77-2>
% greyscaled

%%%%%%%%%%
\subsection*{3.78 heading}
%<*tr78a>
\begin{translation}[hp03_078a]
Now \emph{vajrolī}:
\end{translation}
%</tr78a>

%<*cm78a>
% \begin{philcomm}[hp03_078a]
% \end{philcomm}
%</cm78a>

%%%%%%%%%%
\subsection*{3.78}
%<*tr78>
\begin{translation}[hp03_078]
Even if he behaves as he wishes without [following] the observances taught in yoga, the [yogi] who knows \emph{vajrolī} is worthy of success.
\end{translation}
%</tr78>
% MD: yogokta- or yogoktair?

%<*sc78>
\begin{sources}[hp03_078]
\emph{Dattātreyayogaśāstra} 152
\begin{variants}
yogoktair~] yogokta DYŚ\sep
vajrolīṃ~] vajroliṃ DYŚ\sep
bhājanam~] bhājanaḥ DYŚ, mān bhavet DYŚ\vl
\end{variants}

% \begin{versinnote}
% \tl{svecchayā varttamāno 'pi yogoktaniyamair vinā/\\+}
% \tl{vajroliṃ yo vijānāti sa yogī siddhibhājanaḥ//\\!}
% \end{versinnote}
% \begin{appinnote}
% \tl{\textbf{152d} °bhājanaḥ~] °mān bhavet M1AM2, °bhājanam YTU \\!}
% \end{appinnote}

Cf.\,\emph{Śivasaṃhitā} 4.79
\begin{versinnote}
\tl{svecchayā vartamāno 'pi yogoktaniyamair vinā/\\+}
\tl{mukto bhaved gṛhastho 'pi vajrolyabhyāsayogataḥ//\\!}
\end{versinnote}
\end{sources}
%</sc78>

%<*ts78>
\begin{testimonia}[hp03_078]

Cf. \emph{Haṭharatnāvalī} 2.79 (on \emph{viparītakaraṇī})
\begin{versinnote}
\tl{svasthaṃ yo vartamāno 'pi yogoktair niyamair vinā/\\+}
\tl{karaṇī viparītākhyā śrīnivāsena lakṣitā//\\!}
\end{versinnote}

Cf. \emph{Yogalakṣaṇāvalī} f.\,31r
\begin{versinnote}
\tl{svecchayā vartamāno 'pi yogoktaniyamair vinā/\\+}
\tl{vajrolyabhyāsayogena yogī siddhim avāpnuyāt//\\!}
\end{versinnote}

% \emph{Haṭhayogasaṃhitā} p.\,38
% \begin{versinnote}
% \tl{svecchayā varttamāno 'pi yogoktair niyamair vinā/\\+}
% \tl{vajrolīṃ yo vijānāti sa yogī siddhibhājanam// \\!}
% \end{versinnote}

Cf.\,\emph{Yuktabhavadeva} 7.240 (\attr to Gorakṣanātha)
\begin{versinnote}
\tl{vajrolīṃ kathayiṣyāmi gopitāṃ sarvayogibhiḥ/\\+}
\tl{tyaktayogoktaniyamā yayā sidhyanti yoginaḥ// \\!}
\end{versinnote}


% Cf.\,\emph{Haṭhatattvakaumudī} 16.3
% \begin{versinnote}
% \tl{svecchayā varttamāno 'pi yogoditaiḥ\\+}
% \tl{sadvidhānair vinā sādhakaḥ sābalaḥ/\\+}
% \tl{mucyate 'sau suvajrolikābhyāsataḥ\\+}
% \tl{sarvasiddhyāspadaṃ yāti bhūmaṇḍale//\\!}
% \end{versinnote}

%Vajroliyoga 1
%\begin{versinnote}
%\tl{svecchayā vartamāno 'pi yogoktair niyamair vinā/\\+}
%\tl{vajrolī yo vijānāti sa yogī siddhibhājanaṃ//\\!}
%\end{versinnote}

\end{testimonia}
%</ts78>

%<*cm78>
\begin{philcomm}[hp03_078]
In manuscripts of the delta group, the \emph{vajrolī} section is placed at the end of the work and the following comment is inserted at this place in the third chapter:

\begin{versinnote}
\tl{atratyā vajrolī granthānte likhitā/ kramaprāptāpy atra tyaktā/ asādhāraṇaprāṇyanuṣṭheyatvāt tasyāḥ/\\!}
\end{versinnote}
\closer
\begin{quote}
\emph{Vajrolī}, which is [usually] here, has been copied at the end of the text. Even though it comes here, it has been left out because it is to be practised [only] by special individuals.
\end{quote}

\end{philcomm}
%</cm78>


%%%%%%%%%%
\subsection*{3.79}
%<*tr79>
\begin{translation}[hp03_079]
I shall teach you two substances [needed] for it which are hard for just anyone to obtain. One is milk and the second is an obedient woman.
\end{translation}
%</tr79>

%<*sc79>
\begin{sources}[hp03_079]
\emph{Dattātreyayogaśāstra} 153ab-154ab
\begin{variants}
yasya kasya cit~] yena kena cit DYŚ\sep
tu~] ca DYŚ 
\end{variants}

% \begin{versinnote}
% \tl{tatra vastudvayaṃ vakṣye durlabhaṃ yena kena cit/\\+}
% \tl{kṣīraṃ caikaṃ dvitīyaṃ ca nārī ca vaśavartinī/\\!}
% \end{versinnote}
\end{sources}
%</sc79>

%<*ts79>
\begin{testimonia}[hp03_079]
\emph{Yuktabhavadeva} 7.241 (\attr Gorakṣanātha), \emph{Haṭhayogasaṃhitā} p.\,39
\begin{variants}
tatra HYS~] atra YBhD\sep 
vakṣye HYS~] manye YBhD\sep
caikaṃ HYS~] ekaṃ YBhD\sep
ca HYS~] sva YBhD
\end{variants}


% \begin{versinnote}
% \tl{atra vastudvayaṃ manye durlabhaṃ yasya kasyacit/\\+}
% \tl{kṣīram ekaṃ dvitīyaṃ tu nārī svavaśavarttinī// \\!}
% \end{versinnote}

% \emph{Haṭhayogasaṃhitā} p.\,39
% \begin{versinnote}
% \tl{tatra vastudvayaṃ vakṣye durlabhaṃ yasya kasya cit/\\+}
% \tl{kṣīraṃ caikaṃ dvitīyaṃ tu nārī ca vaśavarttinī// \\!}
% \end{versinnote}

%Vajroliyoga 2
%\begin{versinnote}
%\tl{tatra vastudvayaṃ vakṣye durlabhaṃ yasya kasya cit/\\+}
%\tl{kṣīraṃ caikaṃ dvitīyaṃ tu nārī ca vaśavartinī//\\!}
%\end{versinnote}
\end{testimonia}
%</ts79>

%<*cm79>
\begin{philcomm}[hp03_079]
On the possible referents of \emph{kṣīra}, see Mallinson 2024 on \emph{Dattātreyayogaśāstra} 154. According to Brahmānanda (\emph{Jyotsnā} 3.84), the compound \emph{vaśavartinī}, which we have translated as `an obedient woman,' could be a wife (\emph{vaśavartinī svādhīnā nārī vanitā}). In 3.82, the reading \emph{bhāryābhage} in \etaOne, \etaTwo\ and \piOmega\ supports Brahmānanda's view that the woman is the yogi's wife. 
%Is kṣīra semen? Or a sap that is hard to find? Difficult to say since nothing more is said about kṣīra.
%In DYŚ it's kṣīra and aṅgirasa, so maybe this is addressed to male and female practitioners, who can get one or the other. But what about the celibate yogi?
%LO: in Bengali texts vastu can mean sexual fluid.
% JB should we mention that 3.79–82ab are missing from Alpha Three. Also, Alpha Three is missing 3.83ab, 3.85cd
\end{philcomm}
%</cm79>


%%%%%%%%%%
\subsection*{3.80}
%<*tr80>
\begin{translation}[hp03_080]
[The yogi] should gently practise a full upward contraction through the urethra. Either a man or a woman may obtain success in \emph{vajrolī}.
\end{translation}
%</tr80>

%<*sc80>
%\begin{sources}[hp03_080]
%\end{sources}
%</sc80>

%<*ts80>
\begin{testimonia}[hp03_080]
\emph{Haṭhayogasaṃhitā} 53 (p.\,39)
\begin{variants}
puruṣo vāpi nārī vā~] puruṣo 'py athavā nārī HYS    
\end{variants}

Cf. \emph{Haṭhatattvakaumudī} 16.4
\begin{versinnote}
\tl{apānamārgataḥ samyag ūrdhvakuñcanam abhyaset/\\+}
\tl{puruṣo vāpi nārī vā vajrolīsiddhibhājanam//\\+}
\tl{apānamārgato gudadeśena ūrdhvam upari kuñcanaṃ saṃkocanam ūrdhvam ākarṣaṇaṃ vā abhyaset// iti//\\!}
\end{versinnote}

% \emph{Haṭhayogasaṃhitā} 53 (p. 39)
% \begin{versinnote}
% \tl{mehanena śanaiḥ samyag ūrdhvākuñcanam abhyaset/\\+}
% \tl{puruṣo 'py athavā nārī vajrolīsiddhim āpnuyāt//\\!}
% \end{versinnote}

%Vajroliyoga 3
%\begin{versinnote}
%\tl{mehanena śanaiḥ samyag ūrdhvākuñcanam abhyaset/\\+}
%\tl{puruṣo vāpi nārī vā vajrolīsiddhim āpnuyāt//\\!}
%\end{versinnote}

% \emph{Yogaprakāśikā} 118ab
% \begin{versinnote}
% \tl{mehanena śanaiḥ samyag ūrdhvaṃ kuñcanam abhyaset/\\+}
% \tl{ūrdhvaṃ yathā syāt tathā bindor ākarṣaṇaṃ meḍhreṇābhyased ity arthaḥ//\\!}
% \end{versinnote}
\end{testimonia}
%</ts80>

%<*cm80>
\begin{philcomm}[hp03_080]
 In the \sl{Haṭhatattvakaumudī}, Sundaradeva states that this upward contraction of the urethra, which is the method by which fluids are drawn up it, is done in the region of \emph{apānavāyu} and the anus. Brahmānanda states that this practice is done immediately after sex.
\end{philcomm}
%</cm80>

%%%%%%%%%%
\subsection*{3.81}
%<*tr81>
\begin{translation}[hp03_081]
Using a hollow stalk of bamboo grass, [the yogi] should carefully [and] very gently blow into the opening of the penis in order to make air move [into the urethra].
\end{translation}
%</tr81>

%<*sc81>
\begin{sources}[hp03_081]
\emph{Dattātreyayogaśāstra} 165
\begin{variants}
yatnataḥ~] tatas tu DYŚ
\end{variants}
% \begin{versinnote}
% \tl{tatas tu śaranālena phūtkāraṃ vajrakandare/\\+}
% \tl{śanaiḥ śanaiḥ prakurvīta vāyusaṃcārakāraṇāt//\\!}
% \end{versinnote}
\end{sources}
%</sc81>

%<*ts81>
\begin{testimonia}[hp03_081]
\emph{Haṭharatnāvalī} 2.86–2.87 (\attr HP), \emph{Haṭhasaṅketacandrikā} f.\,39r (\attr HP)
\begin{variants}
yatnataḥ HRĀ~] yantritaḥ HSC   
\end{variants}

% \begin{versinnote}
% \tl{haṭhapradīpikākāras tu\\+}
% \tl{yatnataḥ śaranālena phūtkāraṃ vajrakandare/\\+}
% \tl{śanaiḥ śanaiḥ prakurvīta vāyusaṃcārakāraṇāt//\\!}
% \end{versinnote}

% \emph{Haṭhasaṅketacandrikā} (f. 39r)
% \begin{versinnote}
% \tl{taduktaṃ haṭhapradīpikāyāṃ \\+}
% \tl{yantritaḥ śaranālena phūtkāraṃ vajrakandare/\\+}
% \tl{śanaiḥ śanaiḥ prakurv[ī]ta vāyusaṃcārakāraṇād iti//\\+}
% \tl{asyārthaḥ//\\+}
% \tl{ṣoḍaśāṃgulamānāṃ 16 tu prakuryād vaṃśanālikāṃ sūkṣmā'g[r]amūlāntāṃ li[ṃ]gachi[dra]mukhe datvā svāsye 'nu tanmukhaṃ dhṛtvā phūtkāram ante syāḥ k[u]ryād bāḍhaṃ muhur muhuḥ pratyahaṃ tena vivṛtaṃ liṅgadvāraṃ kramād bhavet[/] \\+}
% \tl{tato †nālyānayāto† yam alpaṃ phūtkārato 'ntare[/]\\+}%nālpānn apāno
% \tl{liṅgara[n]dhreṇa gṛhṇīyāt kramavṛddhyā susādhakaḥ[/]\\+}
% \tl{liṃgachidre 'tha vivṛte kṣīrākṛṣṭiṃ tato bhajed iti[/] \\+}
% \tl{vajrakandare liṅgachidre[//]\\!}
% \end{versinnote}

% \emph{Yogaprakāśikā} 118cd–ef
% \begin{versinnote}
% \tl{yatnataḥ śaranālena phūtkāraṃ vajrakandare/\\+}
% \tl{śanaiḥ śanaiḥ prakurvīta vāyusañcārakāraṇāt//\\+}
% \tl{śareti meḍhranālenety arthaḥ// vāyusañcārakāraṇam iti bindor ākarṣaṇaṃ kāraṇam ity arthaḥ \\!}
% \end{versinnote}

Cf.\,\emph{Yuktabhavadeva} 7.248cd--249ab
\begin{versinnote}
\tl{rasanālena phūtkāraṃ vāyoḥ sañcārakāraṇāt// \\+}
\tl{kuryāt śanaiḥ śanair yogī yāvac chaktiḥ prajāyate/\\!}
\end{versinnote}
\end{testimonia}
%</ts81>

%<*cm81>
%\begin{philcomm}[hp03_081]
%\end{philcomm}
%</cm81>


%%%%%%%%%%
\subsection*{3.82}
%<*tr82>
\begin{translation}[hp03_082]
With practice, [the yogi] may draw up semen which is falling into a woman’s vagina. And [even] if his own semen has moved [down], he may draw it upwards and retain it.
\end{translation}
%</tr82>

%<*sc82>
\begin{sources}[hp03_082]
\emph{Dattātreyayogaśāstra} 166
\begin{variants}
nāryā bhage patadbindum~] tadbhage patitaṃ bindum DYŚ\sep
svakaṃ~] tathā DYŚ
\end{variants}

% \begin{versinnote}
% \tl{tadbhage patitaṃ bindum abhyāsenordhvam āharet/ \\+}
% \tl{calitaṃ ca tathā bindum ūrdhvam ākṛṣya rakṣayet//\\!}
% \end{versinnote}
% there are differences bw the DYS and HP in the 1st and 3rd pādas, but the HP text is solid
\end{sources}
%</sc82>

%<*ts82>
\begin{testimonia}[hp03_082]
\emph{Haṭharatnāvalī} 2.96cd--2.97ab, \emph{Haṭhayogasaṃhitā} p.\,39
\begin{variants}
nāryā bhage~] nāryā bhagāt HRĀ, nārībhage HYS\sep
svakaṃ~] nijaṃ HRĀ HYS
\end{variants}

% \begin{versinnote}
% \tl{nāryā bhagāt patadbindum abhyāsenordhvam āharet//\\+}
% \tl{calitaṃ ca nijaṃ bindum ūrdhvam ākṛṣya rakṣayet/\\!}
% \end{versinnote}

% \emph{Haṭhayogasaṃhitā} p.\,39
% \begin{versinnote}
% \tl{nārībhage pated bindum abhyāsenordhvam āharet/\\+}
% \tl{calitaṃ ca nijaṃ bindum ūrdhvam ākṛṣya rakṣayet//\\!}
% \end{versinnote}

% \emph{Yogaprakāśikā} 5.120
% \begin{versinnote}
% \tl{nāryā bhage patadbindum abhyāsenordhvam āharet/\\+}
% \tl{calitaṃ ca svayaṃ bindum ūrdhvam ākṛṣya rakṣayet//\\+}
% \tl{nārīsaṃyoge bindupatanaṃ syād ity āśaṅkya nirasyati nāryā iti// patato bindor ūrdhvam āhared āhīyamāṇaṃ svayaṃ calitaṃ bindum ākṛṣyety anvayaḥ// \\!}
% \end{versinnote}

Cf.\,\emph{Haṭhasaṅketacandrikā} f.\,39r
\begin{versinnote}
\tl{apānam ākuñcya tato 'balenordhvaṃ dugdham ākṛṣṭividhikrameṇa/\\+}
\tl{samabhyasen niścalam alpam alpaṃ bhage patadbindum athārdhvam āharet//\\!}
\end{versinnote}

% Cf.\,\emph{Yuktabhavadeva} 7.249cd, 259
% \begin{versinnote}
% \tl{tato maithunakāle tu patadbinduṃ samunnayet//\\+}
% \tl{\texteng{[...]} patadbindum apānena huṃ huṃkārasahitena balād ūrdhvam ākṛṣya kiñcit kālaṃ vilambya ramet punaḥ// yadā tu na dhārayituṃ śakyate tadā bahiḥskhalitena bindunā saha prasvedenāṅgaṃ marddayet//\\!}
% \end{versinnote}

\end{testimonia}
%</ts82>

%<*cm82>
%\begin{philcomm}[hp03_082]
%The reading \emph{bhāryābhage} in V1, V3 and J10 affirms Brahmānanda's view that the women is the yogi's wife. [JB: I've moved this comment to our commentary on 3.83.
%\end{philcomm}
%</cm82>

%%%%%%%%%%
\subsection*{3.83}
%<*tr83>
\begin{translation}[hp03_083]
[If] the knower of yoga preserves his semen thus, he conquers death. Death arises through the loss of semen and life from retaining semen.
\end{translation}
%</tr83>

%<*sc83>
\begin{sources}[hp03_083]
\emph{Dattātreyayogaśāstra} 167
\begin{variants}
tu rakṣayed~] ca rakṣito DYŚ\sep
yogavit~] tattvataḥ DYŚ\sep
jīvitaṃ~] jīvanaṃ DYŚ
\end{variants}

% \begin{versinnote}
% \tl{evaṃ ca rakṣito bindur mṛtyuṃ jayati tattvataḥ/ \\+}
% \tl{maraṇaṃ bindupātena jīvanaṃ bindudhāraṇāt//\\!}
% \end{versinnote}

Cf.\,\emph{Amṛtasiddhi} 3.87cd
\begin{versinnote}
\tl{maraṇaṃ bindupātena jīvanaṃ bindudhāraṇāt//\\!}
\end{versinnote}

\end{sources}
%</sc83>

%<*ts83>
\begin{testimonia}[hp03_083]
\emph{Haṭhratnāvalī} 2.97cd-2.98ab, \emph{Yuktabhavadeva} 252cd-253ab, \emph{Haṭhayogasaṃhitā} p.\,39
\begin{variants}
evaṃ tu rakṣayed binduṃ~] evaṃ saṃrakṣayed binduṃ HRĀ HYS, evaṃ bindau sthire jāte YBhD\sep
yogavit HRĀ HYS~] sarvathā YBhD\sep
jīvitaṃ HRĀ~] jīvanaṃ YBhD HYS
\end{variants}

% \begin{versinnote}
% \tl{evaṃ saṃrakṣayed binduṃ mṛtyuṃ jayati yogavit//\\+}
% \tl{maraṇaṃ bindupātena jīvitaṃ bindudhāraṇāt/\\!}
% \end{versinnote}

% \emph{Yuktabhavadeva} 252cd-253ab
% \begin{versinnote}
% \tl{evaṃ bindau sthire jāte mṛtyuṃ jayati sarvathā//\\+}
% \tl{maraṇaṃ bindupātena jīvanaṃ bindudhāraṇāt/\\!}
% \end{versinnote}

% \emph{Haṭhayogasaṃhitā} p.\,39
% \begin{versinnote}
% \tl{evaṃ saṃrakṣayed binduṃ mṛtyuṃ jayati yogavit/\\+}
% \tl{maraṇaṃ bindupātena jīvanaṃ bindudhāraṇāt// \\!}
% \end{versinnote}

%Vajroliyoga 22
%\begin{versinnote}
%\tl{evaṃ bindau sthire yāte mṛtyuṃ jayati sarvathā/\\+}
%\tl{maraṇaṃ bindupātena jīvanaṃ bindudhāraṇāt//\\!}
%\end{versinnote}
\end{testimonia}
%</ts83>

%<*cm83>
%\begin{philcomm}[hp03_083]
%\end{philcomm}
%</cm83>



%%%%%%%%%%
\subsection*{3.84}
%<*tr84>
\begin{translation}[hp03_084]
As a result of the retention of semen, the yogi's body becomes fragrant. As long as semen is steady in the body then why fear death?
\end{translation}
%</tr84>

%<*sc84>
\begin{sources}[hp03_084]
\emph{Dattātreyayogaśāstra} 86cd (ab only), \emph{Vivekamārtaṇḍa} 52ad (cd only)
\begin{variants}
sugandhi yogino dehaṃ~] yogino 'ṅge sugandhaḥ syāt DYŚ\sep
jāyate~] satataṃ DYŚ\sep
sthiro~] sthito VM
\end{variants}

% \begin{versinnote}
% \tl{yogino 'ṅge sugandhaḥ syāt satataṃ bindudhāraṇāt//\\!}
% \end{versinnote}

% \emph{Vivekamārtaṇḍa} 52ad (cd only)
% \begin{versinnote}
% \tl{yāvad binduḥ sthito dehe tāvad mṛtyubhayaṃ kutaḥ/\\!}
% \end{versinnote}
\end{sources}
%</sc84>

%<*ts84>
\begin{testimonia}[hp03_084]
\emph{Haṭharatnāvalī} 2.112ab, \emph{Haṭhayogasaṃhitā} p.\,39
\begin{variants}
sugandhi~] sugandhir HRĀ, sugandho HYS\sep
dehaṃ~] dehe HRĀ HYS\sep
mṛtyubhayaṃ~] kālabhayaṃ HYS
\end{variants}

% \begin{versinnote}
% \tl{sugandhir yogino dehe jāyate bindudhāraṇāt//\\!}
% \end{versinnote}

% \emph{Haṭhayogasaṃhitā} p.\,39
% \begin{versinnote}
% \tl{sugandho yogino dehe jāyate bindudhāraṇāt/\\+}
% \tl{yāvad binduḥ sthiro dehe tāvat kālabhayaṃ kutaḥ// \\!}
% \end{versinnote}

Cf. \emph{Haṭhatattvakaumudī} 16.10
\begin{versinnote}
\tl{tathā coktaṃ granthāntare –\\+}
\tl{calitaṃ tu svakaṃ bindum ūrdhvam ākuñcya rakṣayet/\\+}
\tl{sugandho yogināṃ dehe jāyate bindudhāraṇād// iti//\\!}
\end{versinnote}
\end{testimonia}
%</ts84>

%<*cm84>
\begin{philcomm}[hp03_084]
The omission of 3.84ab in the \texteta\ group and \deltaThree\ is likely to be the result of haplography (\emph{bindu\-dhāraṇāt} is repeated).\lb
% MD: I have deleted "in \alphaThree", because the damaged part has enough space for 3.83cd and 3.84ab.

%sthito makes better sense (adopt?), but J5 has sthiro (missing in N3) nad G4 has kṣīro  (closer to sthiro)
%MD: Does 4.88 speak in favor of sthiro? Or is this bindu something different?
% JB: it looks like \alphaTwo and other mss are preserving a version of 3.84 where deha is neuter (I've come across this in the early recension of the Yogabīja.) maybe we should consider it: J5 sugandhi yogino dehaṃ jāyate bindudhāraṇāt. The J5 reading means that we dont have to emend to °gandhir

The readings \emph{mṛtyubhayaṃ} (\alphaThree, \textdelta, \textepsilon, \texteta, \textpi) and \emph{kālabhayaṃ} (\alphaTwo, \textgamma) are well attested by the main manuscript groups, but \emph{mṛtyubhayaṃ} is in the important witnesses of the source text, the \emph{Vivekamārtaṇḍa}.
\end{philcomm}
%</cm84>

%%%%%%%%%%
\subsection*{3.85}
%<*tr85>
\begin{translation}[hp03_085]
In men semen is dependent on the mind and life is dependent on semen, so semen and the mind should be carefully guarded.
\end{translation}
%</tr85>

%<*sc85>
%\begin{sources}[hp03_085]
%\begin{versinnote}
%\end{versinnote}
%\end{sources}
%</sc85>

%<*ts85>
\begin{testimonia}[hp03_085]
\emph{Haṭharatnāvalī} 2.98
\begin{variants}
manāyattaṃ~] cittāyattaṃ HRĀ YCM\sep
manaś caiva HRĀ~] ca śukraṃ ca YCM
\end{variants}

% \begin{versinnote}
% \tl{cittāyattaṃ nṛṇāṃ śukraṃ śukrāyattaṃ ca jīvitam/\\+}
% \tl{tasmāc chukraṃ manaś caiva rakṣanīyaṃ prayatnataḥ//\\!}
% \end{versinnote}

% \emph{Yogacintāmaṇi} f.\,74v (\attr to the \emph{Haṭhapradīpikā})
% \begin{versinnote}
% \tl{cittāyattaṃ nṛṇāṃ śukraṃ śukrāyattaṃ ca jīvitam/\\+}
% \tl{tasmāc cittaṃ ca śukraṃ ca rakṣaṇīyaṃ prayatnataḥ//\\!}
% \end{versinnote}

%Vajroliyoga 4
%\begin{versinnote}
%\tl{cittāyattaṃ nṛṇāṃ śukraṃ śukrāyattaṃ tu jīvitam/\\+}
%\tl{tasmāc chukraṃ manaś caiva rakṣaṇīyaṃ prayatnataḥ//\\!}
%\end{versinnote}
\end{testimonia}
%</ts85>

%<*cm85>
\begin{philcomm}[hp03_085]
Both \alphaTwo\ and \alphaThree\ indicate that \textit{manas} instead of \textit{citta} was the reading of the initial compound. Therefore, we have conjectured \textit{manāyattaṃ}, assuming double \textit{sandhi} from \emph{manas-āyattam}. 
\end{philcomm}
%</cm85>
%MD (2024-06-14): The original reading may have been mana-āyattaṃ as supported by α3 (anā°). manomayaṃ (α2), mano'dhīnaṃ (B) and cittāyattaṃ (the others) seem to have  resulted from an attempt to avoid the middle Indic form mana or the double Saṃdhi. And we have manas in Pāda c, not citta! Cf. also 4.8*22b.

%%%%%%%%%%
\subsection*{3.86}
%<*tr86>
\begin{translation}[hp03_086]
In this way a [the yogi] may also hold on to [both] the menses of a menstruating woman and his own semen. He who has mastered yoga through correct practice may draw up [both] through the urethra.
\end{translation}
%</tr86>

% MD: pāda b: striyā must be a wrong emendation of gamma group. If the best attested bījaṃ does not work, shall we read svīyaṃ (π1)? Maybe, eva svīyaṃ > evamvījaṃ > evaṃ bījaṃ?
% pāda d: abhyāsayogavān is better attested than °yogataḥ.

%<*sc86>
%\begin{sources}[hp03_086]
%\end{sources}
%</sc86>

%<*ts86>
\begin{testimonia}[hp03_086]
\emph{Haṭharatnāvalī} 2.100cd (ab only)
\begin{variants}
svīyaṃ~] rajo HRĀ, bijaṃ HYS\sep
yogavān~] yogavit HYS
\end{variants}

% \begin{versinnote}
% \tl{ṛtumatyā rajo 'py evaṃ rajo binduṃ ca rakṣayet//\\!}
% \end{versinnote}

% \emph{Haṭhayogasaṃhitā} p.39
% \begin{versinnote}
% \tl{ṛtumatyā rajo 'py evaṃ bijaṃ binduṃ ca rakṣayet/\\+}
% \tl{meḍhreṇākarṣayed ūrdhvaṃ samyagabhyāsayogavit// 59// \\!}
% \end{versinnote}
\end{testimonia}
%</ts86>

%<*cm86>
%\begin{philcomm}[hp03_086]
%No known source. 
%We have understood the reading of \emph{ṛtumatyā} as qualifying \emph{striyāḥ} (i.e.,`a menstruating woman'). Alternatively, the term \emph{ṛtumati} could mean a post-pubescent woman.
%\end{philcomm}
%</cm86>



%%%%%%%%%%
\subsection*{3.86*1}
%<*tr86-1>
\begin{translation}[hp03_086_1]
This yoga succeeds for those who have merit, are fortunate, abide in truth, and are without jealousy, not for those who are jealous.
\end{translation}
%</tr86-1>
%greyscaled

%<*sc86-1>
\begin{sources}[hp03_086_1]
\emph{Dattātreyayogaśāstra} 176
\begin{variants}
matsara~] mātsarya DYŚ
\end{variants}
% \begin{versinnote}
% \tl{ayaṃ yogaḥ puṇyavatāṃ dhanyānāṃ tattvaśalinām/\\+}
% \tl{nirmatsarāṇāṃ sidhyeta na tu mātsaryaśālinām//\\!}
% \end{versinnote}
\end{sources}
%</sc86-1>

%<*ts86-1>
\begin{testimonia}[hp03_086_1]
\emph{Haṭharatnāvalī} 2.110, \emph{Haṭhayogasaṃhitā} pp. 40–41
\begin{variants}
dhanyānāṃ HRĀ~] dhīrāṇāṃ HYS\sep
śālinām HRĀ~] darśinām HYS
\end{variants}

% \begin{versinnote}
% \tl{ayaṃ yogaḥ puṇyavatāṃ dhanyānāṃ tattvaśālinām/\\+}
% \tl{nirmatsarāṇāṃ sidhyeta na tu matsaraśālinām//\\!}
% \end{versinnote}

% \emph{Haṭhayogasaṃhitā} pp. 40-41
% \begin{versinnote}
% \tl{ayaṃ yāgaḥ puṇyavatāṃ dhīrāṇāṃ tattvadarśinām/ \\+}
% \tl{nirmatsarāṇāṃ sidhyeta na tu mātsaryaśālinām// \\!}
% \end{versinnote}

%Vajroliyoga 27
%\begin{versinnote}
%\tl{ayaṃ yogaḥ puṇyavatāṃ siddhe saṃsāriṇāṃ na hi/\\+}
%\tl{amunāṃ siddhim āpnoti yogād yogaḥ pravartate//\\!}
%\end{versinnote}
\end{testimonia}
%</ts86-1>

%<*cm86-1>
% \begin{philcomm}[hp03_086_1]
% This verse is omitted in \alphaTwo\ and \alphaThree\ (and the folio on which it would be found is missing in \alphaOne).
% \end{philcomm}
%</cm86-1>

\begin{metre}[hp03_086_1]
Anuṣṭubh (a: bha-vipulā; c: ma-vipulā)
\end{metre}

%%%%%%%%%%
\subsection*{3.87 heading}
%<*tr87a>
\begin{translation}[hp03_087a]
Now \emph{sahajolī}:
\end{translation}
%JB: Here we have translated atha sahajolī but it is greyscaled in the edition. Should we translate it here? [MD: No. I've just moved the header (also that for amaroli) in the apparatus.]
%</tr87a>
% atha sahajolī/
%<*cm87a>
% \begin{philcomm}[hp03_087a]
% \end{philcomm}
%</cm87a>

%%%%%%%%%%
\subsection*{3.87}
%<*tr87>
\begin{translation}[hp03_087]
\emph{Sahajolī} and \emph{amarolī} are varieties of \emph{vajrolī}. 
\end{translation}
%</tr87>

%<*sc87>
\begin{sources}[hp03_087]
Cf. \emph{Dattātreyayogaśāstra} 31cd
\begin{versinnote}
\tl{vajrolir amaroliś ca sahajolis tridhā matā/\\!}
\end{versinnote}

Cf. \emph{Śivasaṃhitā} 4.95ab
\begin{versinnote}
\tl{sahajolyamarolī ca vajrolyā bhedato bhavet/\\!}
\end{versinnote}
\end{sources}
%</sc87>

%<*ts87>
\begin{testimonia}[hp03_087]
\emph{Haṭharatnāvalī} 2.113cd, \emph{Haṭhayogasaṃhitā} p.\,40
\begin{variants}
sahajolī cāmarolī HRĀ~] sahajoliś cāmarolir HYS\sep
eva bhedataḥ HRĀ~] bheda eva te HYS 
\end{variants}

% \begin{versinnote}
% \tl{atha sahajoliḥ -\\+}
% \tl{sahajolī cāmarolī vajrolyā eva bhedataḥ//\\!}
% \end{versinnote}

% \emph{Haṭhayogasamhitā} p.40
% \begin{versinnote}
% \tl{sahajoliś cāmarolir vajrolyā bheda eva te/\\!}
% \end{versinnote}
\end{testimonia}
%</ts87>

%<*cm87>
\begin{philcomm}[hp03_087]
These two \emph{pāda}s appear to stand apart and function as a heading introducing the practices of \emph{sahajolī} and \emph{amarolī}, which are described in the verses that follow. Some manuscript groups other than \textalpha\ and \textpi\ insert separate headings for \emph{sahajolī} and \emph{amarolī}. However, since 3.87 introduces these practices, these additional headings are redundant and unlikely to be original.

%The \textalpha\ and \textpi\ groups omit the headings for \emph{sahajolī} and \emph{amarolī}. Since 3.87 introduces these practices, the headings are probably not original.
% MD: The headings are not printed in the edition any more.
\end{philcomm}
%</cm87>

\begin{metre}[hp03_087]
Anuṣṭubh (a: ra-vipulā)
\end{metre}

%%%%%%%%%%
\subsection*{3.88}
%<*tr88>
\begin{translation}[hp03_088]
After intercourse using \emph{vajrolī}, the woman and man should put ash made from burnt cow dung in water [and] smear their bodies [with it...]
\end{translation}
%</tr88>
% greyscaled 

%<*sc88>
\begin{sources}[hp03_088]
\emph{Dattātreyayogaśāstra} 182
\begin{variants}
jaleṣu bhasma nikṣipya~] tajjale bhasma saṃkṣipya DYŚ, tajjale bhasmasāt kṣipya DYŚ\vl, tajjale bhasma saddravyaṃ DYŚ\vl\sep 
puṃsoḥ svāṅga~] puṃsor aṅga DYŚ
\end{variants}

% \begin{versinnote}
% \tl{tajjale bhasma saṃkṣipya dagdhagomayasaṃbhavam/\\+}
% \tl{vajrolīmaithunād ūrdhvaṃ strīpuṃsor aṅgalepanam//\\!}
% \end{versinnote}
% \begin{appinnote}
% \tl{\textbf{182a} tajjale bhasma saṃkṣipya~] M2; tajjale bhasmasāt kṣipya M1, tajjale bhasma saddravyaṃ A \\!}
% \end{appinnote}

\end{sources}
%</sc88>

%<*ts88>
\begin{testimonia}[hp03_088]
\emph{Haṭharatnāvalī} 2.114, \emph{Haṭhayogasamhitā} p.40
\begin{variants}
jaleṣu bhasma~] jale subhasma HRĀ HYS\sep
puṃsoḥ svāṅga HYS~] puṃsoś cāṅga HRĀ
\end{variants}

% \begin{versinnote}
% \tl{jale subhasma nikṣipya dagdhagomayasaṃbhavam/\\+}
% \tl{vajrolīmaithunād ūrdhvaṃ strīpuṃsoś cāṅgalepanam//\\!}
% \end{versinnote}

% \emph{Haṭhayogasamhitā} p.40
% \begin{versinnote}
% \tl{jale subhasma nikṣipya dagdhagomayasambhavam//\\+}
% \tl{vajrolī maithunād ūrdhvaṃ strīpuṃsoḥ svāṅgalepanam/ \\!}
% \end{versinnote}
\end{testimonia}
%</ts88>

%<*cm88>
\begin{philcomm}[hp03_088]
Some manuscripts, including \alphaTwo\ and \alphaThree\ (missing in \alphaOne), omit 3.88ab. We have included it because in the \emph{Dattātreyayogaśāstra}, the source of this verse, 3.88ab specifies the substance mentioned in 3.88cd that the man and woman are supposed to rub into their bodies after sexual intercourse.\lb

In the \emph{Dattātreyayogaśāstra}’s teaching on \emph{sahajolī} (163 and 181–183) a rag is used to wipe up the residue of a mixture of semen and sweat that has been rubbed into the body, and then soaked in a paste of water and ash before being rubbed over the body.\lb

The awkward plural \emph{jaleṣu} in 3.88a was probably the result of Svātmārāma removing the pronoun from the compound \emph{tajjale} in the \emph{Dattātreyayogaśāstra}'s verse because it has no referent in the \emph{Haṭhapradīpikā}'s compilation.
\end{philcomm}
%</cm88>

%%%%%%%%%%
\subsection*{3.89}
%<*tr89>
\begin{translation}[hp03_089]
[...] while sitting at complete ease, having just finished intercourse. This is called \emph{sahajolī}. It is always to be trusted by yogis. 
\end{translation}
%</tr89>

%<*sc89>
\begin{sources}[hp03_089]
\emph{Dattātreyayogaśāstra} 183
\begin{variants}
sahajolīr iyaṃ proktā~] sahajolī ca saṃproktā DYŚ    
\end{variants}
% \begin{versinnote}
% \tl{āsīnayoḥ sukhenaiva muktavyāparayoḥ kṣaṇam/\\+}
% \tl{sahajolī ca saṃproktā śraddheyā yogibhiḥ sadā//\\!}
% \end{versinnote}
\end{sources}
%</sc89>

%<*ts89>
\begin{testimonia}[hp03_089]
\emph{Haṭharatnāvalī} 2.115, \emph{Haṭhayogasaṃhitā} p.\,40
\begin{variants}
kṣaṇam HRĀ~] kṣaṇāt HYS\sep
sahajolīr iyaṃ~] sahajolir iyaṃ HRĀ HYS\sep
śraddheyā HYS~] kartavyā HRĀ
\end{variants}
% \begin{versinnote}
% \tl{āsīnayoḥ sukhenaiva muktavyāpārayoḥ kṣaṇam/\\+}
% \tl{sahajolir iyaṃ proktā kartavyā yogibhiḥ sadā//\\!}
% \end{versinnote}

% \emph{Haṭhayogasaṃhitā} p.\,40
% \begin{versinnote}
% \tl{āsīnayoḥ sukhenaiva muktavyāpārayoḥ kṣaṇāt//\\+}
% \tl{sahajolir iyaṃ proktā śraddheyā yogibhiḥ sadā \\!}
% \end{versinnote}
\end{testimonia}
%</ts89>

%<*cm89>
\begin{philcomm}[hp03_089]
We have understood the \emph{repha} in \emph{sahajolīr iyam} as a hiatus bridge. Elsewhere the nominative of this name is found only as \emph{sahajolī} or \emph{sahajoliḥ}.
\end{philcomm}
%</cm89>


%%%%%%%%%%
\subsection*{3.89*1}
%<*tr89-1>
\begin{translation}[hp03_089_1]
This auspicious yoga bestows liberation even when pleasure has been enjoyed.
\end{translation}
%</tr89-1>
%greyscaled

%<*sc89-1>
%\begin{sources}[hp03_089_1]
%\end{sources}
%</sc89-1>

%<*ts89-1>
\begin{testimonia}[hp03_089_1]
\emph{Haṭhayogasaṃhitā} p.\,40
\begin{variants}
yogo~] yogī HYS\sep
bhoge bhukte~] bhogayukto HYS
\end{variants}

% \begin{versinnote}
% \tl{ayaṃ śubhakaro yogī bhogayukto'pi muktidaḥ//\\!}
% \end{versinnote}
\end{testimonia}
%</ts89-1>

%<*cm89-1>
\begin{philcomm}[hp03_089_1]
This line is absent in \alphaTwo,\ \alphaThree\ and \gammaOne\ (missing in \alphaOne). It may have been adapted from \emph{Dattātreyayogaśāstra} 179cd (\emph{tasmād ayaṃ vakṣyamāṇo bhoge bhukte ’pi muktidaḥ}). Cf. 3.93cd.
\end{philcomm}
%</cm89-1>


%%%%%%%%%%
\subsection*{3.90 heading}
%<*tr90a>
\begin{translation}[hp03_090a]
Now \emph{amarolī}:
\end{translation}
%  JB atha amarolī is greyscaled
%</tr90a>

%<*cm90a>
% \begin{philcomm}[hp03_090a]
% \end{philcomm}
%</cm90a>

%%%%%%%%%%
\subsection*{3.90}
%<*tr90>
\begin{translation}[hp03_090]
Leaving out the first flow because of its excessive heat and the last flow because it is worthless, the cool middle flow of urine is used by Kāpālikas of the Khaṇḍa school.
\end{translation}
%</tr90>

%<*sc90>
% \begin{sources}[hp03_090]
% \end{sources}
%</sc90>

%<*ts90>
\begin{testimonia}[hp03_090]
\emph{Haṭharatnāvalī} 2.116, \emph{Haṭhatattvakaumudī} 16.17, \emph{Haṭhayogasaṃhitā} p.\,41
\begin{variants}
pittolbaṇatvāt HRĀ\vl HTK HYS~] vihāya nityāṃ HRĀ\sep
prathamāṃ ca dhārāṃ HRĀ HTK~] prathamāmbudhārāṃ HYS\sep
vihāya niḥsāratayāntyadhārām HRĀ HTK~] niṣevyate śītalamadhyadhārā HYS\sep
niṣevyate śītalamadhyadhārā HTK~] niṣevyate śītalamadhyadhārāṃ HRĀ, vihāya niḥsāratayāntya\-dhārāṃ HYS\sep
kāpālikaiḥ khaṇḍamatair amaryāḥ~] kāpālikaiḥ khaṇḍamatair anarghyām HRĀ, kāpālikaiḥ khaṇḍamate 'marolī HTK, kāpālike khaṇḍamate 'marolī HYS
\end{variants}

% \begin{versinnote}
% \tl{athāmarolī\\+}
% \tl{vihāya nityāṃ prathamāṃ ca dhārāṃ \\+}
% \tl{vihāya niḥsāratayāntyadhārām/\\+}
% \tl{niṣevyate śītalamadhyadhārāṃ \\+}
% \tl{kāpālikaiḥ khaṇḍamatair anarghyām//\\!}
% \end{versinnote}
% \begin{appinnote}
% \tl{\var{\textbf{a} vihāya nityāṃ~] pittolbaṇatvāt \vl\ %
% \ \textbf{d} anarghyām~] anarghyā}\\!}
% \end{appinnote}

% \emph{Haṭhatattvakaumudī} 16.17
% \begin{versinnote}
% \tl{athāmarolī –\\+}
% \tl{pittolbaṇatvāt prathamāṃ ca dhārāṃ \\+}
% \tl{vihāya niḥsāratayāntyadhārām/\\+}
% \tl{niṣevyate śītamadhyadhārā \\+}
% \tl{kāpālikaiḥ khaṇḍamate 'marolī//\\!}
% \end{versinnote}

% \emph{Haṭhayogasaṃhitā} p.\,41
% \begin{versinnote}
% \tl{pittolvaṇatvāt prathamāmbudhārāṃ \\+}
% \tl{niṣevyate śītalamadhyadhārā/\\+}
% \tl{vihāya niḥsāratayāntyadhārāṃ \\+}
% \tl{kāpālike khaṇḍamate 'marolī//\\!}
% \end{versinnote}
\end{testimonia}
%</ts90>

%<*cm90>
\begin{philcomm}[hp03_090]
We understand `Kāpālikas of the Khaṇḍa school' (\emph{kāpālikair khaṇḍamataiḥ}) to be referring to followers of the Khaṇḍakāpālika who is mentioned in the list of siddhas given at 1.5–9, \emph{pace} Marcinkowska-Rosół and Sellmer (2021: 105–108) who understand \emph{khaṇḍamataiḥ} to mean `whose doctrine is defective'. 

%Maria Marcinkowska-Rosół & Sven Sellmer. “Notes on some difficult passages of the Haṭhapradīpikā”. Zeitschrift der Deutschen Morgenländischen Gesellschaft 171 (2021), pp. 101–21
\end{philcomm}
%</cm90>

\begin{metre}[hp03_090]
Upajāti
\end{metre}

%%%%%%%%%%
\subsection*{3.91}
%<*tr91>
\begin{translation}[hp03_091]
[The yogi] who regularly imbibes urine, taking it by the nose every day, practises \emph{vajrolī} thus. This is called \emph{amarolī}.
\end{translation}
%</tr91>

%<*sc91>
\begin{sources}[hp03_091]
\emph{Dattātreyayogaśāstra} 180c–181b
\begin{variants}
abhyased evam~] abhyasec ceyam DYŚ (\emph{em.}), abhyasec chrayam DYŚ\vl, abhyaset yeyam DYŚ\vl, abhyasec caivam DYŚ\vl 
\end{variants}

% \begin{versinnote}
% \tl{amarīṃ yaḥ piben nityaṃ nasyaṃ kurvan dine dine//\\+}
% \tl{vajrolīm abhyasec ceyam amarolīti kathyate/\\!}
% \end{versinnote}
% \begin{appinnote}
% \tl{\textbf{181a} abhyaset ceyam] \emph{em.}; abhyasec chrayam M1, abhyaset yeyam A, abhyasec caivam M2 \\!}
% \end{appinnote}
\end{sources}
%</sc91>

%<*ts91>
\begin{testimonia}[hp03_091]
\emph{Haṭharatnāvalī} 2.117, \emph{Haṭhayogasaṃhitā} 65 (p.41)
\begin{variants}
kuryād HRĀ~] kurvan HYS\sep
abhyased evam~] abhyasen nityam HRĀ, abhyaset samyag HYS
\end{variants}

% \begin{versinnote}
% \tl{amarīṃ yaḥ piben nityaṃ nasyaṃ kuryād dine dine /\\+}
% \tl{vajrolīm abhyasen nityam amarolīti kathyate//\\!}
% \end{versinnote}

% \emph{Haṭhayogasaṃhitā} 65 (p.41)
% \begin{versinnote}
% \tl{amarīṃ yaḥ piben nityaṃ nasyaṃ kurvan dine dine/\\+}
% \tl{vajrolīm abhyaset samyag amarolīti kathyate//\\!}
% \end{versinnote}
\end{testimonia}
%</ts91>

%<*cm91>
%\begin{philcomm}[hp03_091]
%\end{philcomm}
%</cm91>


%%%%%%%%%%
\subsection*{3.91*1}
%<*tr91-1>
\begin{translation}[hp03_091_1]
If a woman draws up the semen of a man through skillfulness in the correct practice and retains her menses by means of \emph{vajrolī}, it is she who is a [true] yoginī.
\end{translation}
%</tr91-1>
%greyscaled

%<*sc91-1>
\begin{sources}[hp03_091_1]
\emph{Dattātreyayogaśāstra} 169cd (cd only)
\mylb
% \begin{versinnote}
% \tl{yadi nārī rajo rakṣed vajrolyā sā hi yoginī//\\!}
% \end{versinnote}
\end{sources}
%</sc91-1>

%<*ts91-1>
\begin{testimonia}[hp03_091_1]
\emph{Haṭhayogasaṃhitā} p.\,41
\begin{variants}
sā hi~] sā 'pi HYS
\end{variants}

% \begin{versinnote}
% \tl{puṃso binduṃ samākuñcya samyagabhyāsapāṭavāt/\\+}
% \tl{yadi nārī rajo rakṣed vajrolyā sā'pi yoginī//\\!}
% \end{versinnote}
\end{testimonia}
%</ts91-1>

%<*cm91-1>
\begin{philcomm}[hp03_091_1]
In the first verse quarter, the gerund \emph{samākṛṣya} (\textgamma) has been adopted, instead of the better-attested \emph{samākuñcya}, as it yields a more appropriate sense and is used similarly to \emph{ākṛṣya} in 3.82.\lb

Verses 3.91*1–3 have been greyscaled because they are absent in \alphaThree\ (and missing in  \alphaOne). They appear to have been borrowed from the \emph{Dattātreyayogaśāstra}'s section on \emph{vajrolī}, perhaps with the intention of supplementing 3.92–93 by providing additional details on how a woman practises \emph{vajrolī}. The verses are present in \alphaTwo\ after verse 3.86ab where the verse quarter \emph{vajrolyā saha yoginī} occurs twice (also at 3.92b), which suggests that the version of \emph{vajrolī} in \alphaTwo\ has been subject to further revision. The fact that 3.91*1–3 are in groups \textgamma, \texteta\ and \textpi\ indicates that they were added early in the transmission of the \emph{Haṭhapradīpikā}.     
% J5
% ṛtumatyā rajo 'py evaṃ vīryaṃ binduṃ ca rakṣayet/
% yadi nārī rajo rakṣet vajrolyā saha yoginī// (note the difference with our current 3.89c)
% tasyāḥ kiñ cid rajo nāśaṃ na gacchati na saṃśayaḥ/
% tasyāḥ śarīre nādas tu bindu[t]ām eva gacchati/
% sa bindus tad rajaś caiva ekībhūtaṃ svadehagau/ 
% vajrolyābhyāsayogena sarvasiddhiḥ prakurvati [prajāyate]//
% sahajolī cāmarolī [ca?] vajrolyā eva bhedataḥ
% 3.93
\end{philcomm}
%</cm91-1>

%%%%%%%%%%
\subsection*{3.91*2}
%<*tr91-2>
\begin{translation}[hp03_091_2]
Assuredly none of her menses is lost. The \emph{nāda} in her body turns into \emph{bindu}.
\end{translation}
%</tr91-2>

%<*sc91-2>
\begin{sources}[hp03_091_2]
\emph{Dattātreyayogaśāstra} 174
\begin{variants}
tasyāḥ kiñ cid~] tasyās tadā DYŚ   
\end{variants}

% \begin{versinnote}
% \tl{tasyās tadā rajo nāśaṃ na gacchati na saṃśayaḥ/ \\+}
% \tl{tasyāḥ śarīre nādas tu bindutām eva gacchati// 174//\\!}
% \end{versinnote}
\end{sources}
%</sc91-2>

%<*ts91-2>
\begin{testimonia}[hp03_091_2]
\emph{Haṭharatnāvalī} 2.108ab (cd only), \emph{Haṭhayogasaṃhitā} pp. 41--42
\begin{variants}
nādas tu HRĀ~] nādaś ca HYS   
\end{variants}

% \begin{versinnote}
% \tl{tasyāḥ śarīre nādas tu bindutām eva gacchati/\\!}
% \end{versinnote}

% \emph{Haṭhayogasaṃhitā} pp. 41--42
% \begin{versinnote}
% \tl{tasyāḥ kiñ cid rajo nāśaṃ na gacchati na saṃśayaḥ/\\+}
% \tl{tasyāḥ śarīre nādaś ca bindutām eva gacchati//\\!}
% \end{versinnote}
\end{testimonia}
%</ts91-2>

%<*cm91-2>
\begin{philcomm}[hp03_091_2]
On why this verse is in greyscale, see the note to 3.91*1. %

On \emph{nāda} and \emph{bindu} see the note to 3.52. 
%Brahmānanda identifies nāda with rajas. LO: bindu is source of everything, nāda is going back to its source.
%
\end{philcomm}
%</cm91-2>

\begin{metre}[hp03_091_2]
Anuṣṭubh (c: ma-vipulā)
\end{metre}

%%%%%%%%%%
\subsection*{3.91*3}
%<*tr91-3>
\begin{translation}[hp03_091_3]
The \emph{bindu} and \emph{rajas}, which are produced in her own body, become one through \emph{vajrolī} and bring about complete perfection by means of practice.%
\end{translation}
%</tr91-3>

%<*sc91-3>
\begin{sources}[hp03_091_3]
\emph{Dattātreyayogaśāstra} 175
\begin{variants}
sarvasiddhiṃ prakurvataḥ~] sarvasiddhiḥ prajāyate DYŚ
\end{variants}

% \begin{versinnote}
% \tl{sa bindus tad rajaś caiva ekībhūya svadehagau/ \\+}
% \tl{vajrolyābhyāsayogena sarvasiddhiḥ prajāyate// 175//\\!}
% \end{versinnote}
\end{sources}
%</sc91-3>

%<*ts91-3>
\begin{testimonia}[hp03_091_3]
\emph{Haṭharatnāvalī} 2.108cd--109ab
\begin{variants}
svadehajau HRĀ~] svadehagau HYS\sep
sarvasiddhiṃ prakurvataḥ~] yogasiddhiḥ kare sthitā HRĀ, sarvasiddhiṃ prayacchataḥ HYS 
\end{variants}

% \begin{versinnote}
% \tl{sa bindus tad rajaś caiva ekīkṛtya svadehajau//\\+}
% \tl{vajrolyabhyāsayogena yogasiddhiḥ kare sthitā/\\!}
% \end{versinnote}

% \emph{Haṭhayogasaṃhitā} p.\,42
% \begin{versinnote}
% \tl{sa bindus tad rajaś caiva ekībhūya svadehagau/\\+}
% \tl{vajrolyabhyāsayogena sarvasiddhiṃ prayacchataḥ// \\!}
% \end{versinnote}
\end{testimonia}
%</ts91-3>

%<*cm91-3>
\begin{philcomm}[hp03_091_3]
On why this verse is in greyscale, see the note to 3.91*1.
\end{philcomm}
%</cm91-3>

%%%%%%%%%%
\subsection*{3.92}
%<*tr92>
\begin{translation}[hp03_092]
It is she who preserves her menses by means of the upward contraction who is the [true] yoginī. She knows the past and the future, and is sure to become a sky-rover (\emph{khecarī}).
\end{translation}
%</tr92>

%<*sc92>
\begin{sources}[hp03_092]
\emph{Dattātreyayogaśāstra} 170ab (cd only)
\begin{variants}
ca~] vā DYŚ    
\end{variants}

% \begin{versinnote}
% \tl{atītānāgataṃ vetti khecarī vā bhaved dhruvam/\\!}
% \end{versinnote}
\end{sources}
%</sc92>

%<*ts92>
\begin{testimonia}[hp03_092]
\emph{Haṭhayogasaṃhitā} p.\,42
\begin{variants}
ākuñcanenordhvaṃ~] ākuñcanād ūrdhvaṃ HYS\sep
atītānāgatajñānaṃ khecarī~] atītānāgataṃ vetti khecarī HYS
\end{variants}

% \begin{versinnote}
% \tl{rakṣed ākuñcanād ūrdhvaṃ yā rajaḥ sā hi yoginī/ \\+}
% \tl{atītānāgatajñānaṃ khecarī ca bhaved dhruvam//\\!}
% \end{versinnote}

%Haṭhapradīpikā 5.140--141ab
%\begin{versinnote}
%\tl{mehanākuñcanād ūrdhvaṃ rajasāpi ca yoginī/\\+}
%\tl{atītānāgataṃ vetti khecarī ca bhaved dhruvam//\\+}
%\tl{dehasiddhiṃ ca labhate vajrolyabhyāsayogataḥ/\\!}
%\end{versinnote}
\end{testimonia}
%</ts92>

%<*cm92>
%\begin{philcomm}[hp03_092]
%\end{philcomm}
%</cm92>

\begin{metre}[hp03_092]
Anuṣṭubh (c: na-vipulā)
\end{metre}

%%%%%%%%%%
\subsection*{3.93}
%<*tr93>
\begin{translation}[hp03_093]
And she attains perfection of the body as a result of the practice of \emph{vajrolī}. This auspicious yoga bestows liberation even when pleasure has been enjoyed.
\end{translation}
%</tr93>
%Therefore this yoga succeeds only for those who have merit.

%<*sc93>
\begin{sources}[hp03_093]
\emph{Dattātreyayogaśāstra} 179
\begin{variants}
ayaṃ śubhakaro yogo~] tasmād ayaṃ vakṣyamāṇo DYŚ\sep
bhoge bhukte 'pi muktidaḥ DYŚ (\emph{conj.})~] bhoge bhukte tv abhuktidaḥ DYŚ\vl, bhogo yogaś ca muktidaḥ DYŚ\vl 
\end{variants}

% \begin{versinnote}
% \tl{dehasiddhiṃ ca labhate vajrolyabhyāsayogataḥ/\\+}
% \tl{tasmād ayaṃ vakṣyamāṇo bhoge bhukte 'pi muktidaḥ//\\!}
% \end{versinnote}
% \begin{appinnote}
% \tl{\textbf{179d} bhoge bhukte ’pi muktidaḥ~] conj.; bhoge bhukte tv abhuktidaḥ M1, bhogo yogaś ca muktidaḥ AM2 \\!}
% \end{appinnote}
\end{sources}
%</sc93>

%<*ts93>
\begin{testimonia}[hp03_093]
Cf. \emph{Haṭharatnāvalī} 2.111
\begin{versinnote}
\tl{sarveṣām eva yogānām ayaṃ yogaḥ śubhaṅkaraḥ/\\+}
\tl{tasmād ayaṃ variṣṭho 'sau bhuktimuktiphalapradaḥ//\\!}
\end{versinnote}

\emph{Haṭhayogasaṃhitā} p.\,42
\begin{variants}
śubhakaro~] puṇyakaro HYS    
\end{variants}

% \begin{versinnote}
% \tl{dehasiddhiṃ ca labhate vajrolyabhyāsayogataḥ/\\+}
% \tl{ayaṃ puṇyakaro yogo bhoge bhukte 'pi muktidaḥ//\\!}
% \end{versinnote}

\end{testimonia}
%</ts93>

%<*cm93>
%\begin{philcomm}[hp03_093]
%tasmād ayaṃ sādhakāya hemistich is not in Gr2 and seems like an unnecessary repetition (that may have occurred to change a 6 pāda verse into two verses).
%
% MD: 3.101*1ab and 3.101cd by the old numbering are now merged as 3.93*cd.
%\end{philcomm}
%</cm93>

\begin{metre}[hp03_093]
Anuṣṭubh (a: ra-vipulā)
\end{metre}

%%%%%%%%%%
\subsection*{3.93*1}
%<*tr93-1>
% \begin{translation}
% \end{translation}
%</tr93-1>

%<*cm93-1>
\begin{philcomm}[hp03_093_1]
This verse is not in \textalpha\ and \textgamma, and seems like an unnecessary repetition of 3.100.
\end{philcomm}
%</cm93-1>

%%%%%%%%%%
\subsection*{3.93*2 heading}
%<*tr93-2a>
\begin{translation}[hp03_093_2a]
Now the Stimulation of the Goddess (\emph{śakticālanam}):
\end{translation}
%</tr93-2a>

%<*cm93-2a>
% \begin{philcomm}[hp03_093_2a]
% \end{philcomm}
%</cm93-2a>

%%%%%%%%%%
\subsection*{3.93*2}
%<*tr93-2>
\begin{translation}[hp03_093_2]
She whose body is bent (\emph{kuṭilāṅgī}), she who is coiled (\emph{kuṇḍalinī}), the female snake (\emph{bhujaṅgī}), the power (\emph{śakti}), the goddess (\emph{īśvarī}), she who is coiled (\emph{kuṇḍalī}) and Arundhatī: these words are synonyms.
\end{translation}
%</tr93-2>
 
%<*sc93-2>
%\begin{sources}[hp03_093_2]
%\end{sources}
%</sc93-2>

%<*ts93-2>
\begin{testimonia}[hp03_093_2]

Cf. \emph{Haṭharatnāvalī} 2.125–127
\begin{versinnote}
\tl{phaṇī kuṇḍalinī nāgī cakrī vakrī sarasvatī/\\+}
\tl{lalanā rasanā kṣatrī lalāṭī śaktiḥ śaṃkhinī//\\+}
\tl{rajvī bhujaṅgī śeṣā ca kuṇḍalī sarpiṇī maṇiḥ/\\+}
\tl{ādhāraśaktiḥ kuṭilā karālī prāṇavāhinī//\\+}
\tl{aṣṭavakrā ṣaḍādhārā vyāpinī kalanādharā//\\+}
\tl{kurīty evaṃ ca vikhyātāḥ śabdāḥ paryāyavācakāḥ//\\!}
\end{versinnote}

\emph{Yogacintāmaṇi} f.\,78v (\attr \emph{Haṭhayoga}), \emph{Yuktabhavadeva} 7.300 (\attr \emph{Śivayoga})
\begin{variants}
kuṭilāṅgī YBhD~] kuṇḍalāṅgī YCM\sep
kuṇḍaly arundhatī YBhD~] kuṭilārundhatī YCM\sep
ceti~] devī YCM YBhD
\end{variants}

% \begin{versinnote}
% \tl{kuṇḍalāṅgī kuṇḍalinī bhujaṅgī śaktir īśvarī/\\+}
% \tl{kuṭilārundhatī devī śabdāḥ paryāyavācakāḥ//\\!}
% \end{versinnote}

% \emph{Yuktabhavadeva} 7.300 (\attr to the \emph{Śivayoga})
% \begin{versinnote}
% \tl{kuṭilāṃgī kuṇḍalinī bhujaṅgī śaktir īśvarī/\\+}
% \tl{kuṇḍaly arundhatī devī śabdāḥ paryāyavācakāḥ//\\!}
% \end{versinnote}

%Haṃsavilāsa (p. 46)
%\begin{versinnote}
%\tl{āgame nāmāntarāṇi ca/\\+}
%\tl{kuṭilāṅgī kuṇḍalinī bhujaṅgī śaktir īśvarī/\\+}
%\tl{kuṇḍalyā ceti ruddhanti śabdāḥ paryāyavācakāḥ//\\!}
%\end{versinnote}

%Prāṇatoṣiṇī Part 6 (possiby attr.~to the Dattātreyasaṃhitā) %line 3723
%\begin{versinnote}
%\tl{kuṇḍalāṅgī kuṇḍalinī bhujaṅgī śaktirīśvarī/ \\+}
%\tl{kuṇḍalyarundhatī devī śabdāḥ paryāyavācakāḥ/\\!}
%\end{versinnote}
\end{testimonia}
%</ts93-2>

%<*cm93-2>
\begin{philcomm}[hp03_093_2]
%3.4 and 4.4 have cety ekavācakāḥ, but devī paryāyavācakāḥ is well attested among the HP manuscripts (Groups 2 and 3) and the testimonia.
The \alphaThree\ manuscript has a significantly shorter and more coherent version of \emph{śakticālana}. It omits six introductory verses, of which three are from the \emph{Vivekamārtaṇḍa} or one of its longer recensions and three have no known source, including one that contains a list of synonyms for \emph{kuṇḍalinī}. This section is missing in \alphaOne\ (3.83–3.96) and \alphaTwo\ adds these verses (except 3.93*6) after 3.96, which suggests that they have been inserted from elsewhere. Generally speaking, it appears that some redactors have taken the section on \emph{śakticālana} in the \emph{Haṭhapradīpikā} as an opportunity to add material on \emph{kuṇḍalinī}, in particular her location, shape, and soteriological importance. 

\end{philcomm}
%</cm93-2>

\begin{metre}[hp03_093_2]
Anuṣṭubh (a: bha-vipulā)
\end{metre}

%%%%%%%%%%
\subsection*{3.93*3}
%<*tr93-3>
\begin{translation}[hp03_093_3]
Just as one might use a key to force open a double door, so the yogi breaks open the door to liberation with Kuṇḍalinī.
\end{translation}
%</tr93-3>

%<*sc93-3>
\begin{sources}[hp03_093_3]
\emph{Vivekamārtaṇḍa} 35
\mylb
% \begin{versinnote}
% \tl{udghāṭayet kapāṭaṃ tu yathā kuñcikayā haṭhāt/\\+}
% \tl{kuṇḍalinyā tathā yogī mokṣadvāraṃ vibhedayet//\\!}
% \end{versinnote}
\end{sources}
%</sc93-3>

%<*ts93-3>
\begin{testimonia}[hp03_093_3]
\emph{Yogacintāmaṇi} f.\,78v (\attr \emph{Haṭhayoga}), Haṭhasaṅketacandrikā f.\,110r (\attr HP) 

% \begin{versinnote}
% \tl{haṭhayoge\\+}
% \tl{udghāṭayet kapāṭaṃ tu yathā kuñcikayā haṭhāt/\\+}
% \tl{kuṇḍalinyā tathā yogī mokṣadvāraṃ vibhedayet//\\!}
% \end{versinnote}

% Haṭhasaṅketacandrikā f.\,110r
% \begin{versinnote}
% \tl{tathā coktaṃ haṭhapradīpikāyāṃ/\\+}
% \tl{udghāṭayet kapāṭaṃ tu yathā kuñcikayā haṭhāt/\\+}
% \tl{kuṇḍalinyā tathā yogī mokṣadvāraṃ vibhedayet//\\!}
% \end{versinnote}
\end{testimonia}
%</ts93-3>

%<*cm93-3>
%\begin{philcomm}[hp03_093_3]
%SS: von Hinuber 1992 booklet has something relevant to this. Can provide.
%\end{philcomm}
%</cm93-3>


%%%%%%%%%%
\subsection*{3.93*4}
%<*tr93-4>
\begin{translation}[hp03_093_4]
The supreme goddess sleeps with her mouth covering the opening of the pathway by which the perfect place of Brahman is reached.%JM wholesome > perfect?
\end{translation}
%</tr93-4>

%<*sc93-4>
\begin{sources}[hp03_093_4]
\emph{Vivekamārtaṇḍa} 33
\mylb
% \begin{versinnote}
% \tl{yena mārgeṇa gantavyaṃ brahmasthānaṃ nirāmayam/\\+}
% \tl{mukhenācchādya taddvāraṃ prasuptā parameśvarī//\\!}
% \end{versinnote}
\end{sources}
%</sc93-4>

%<*ts93-4>
\begin{testimonia}[hp03_093_4]
\emph{Yogacintāmaṇi} f.\,78v (\attr \emph{Haṭhayoga}), \emph{Haṭhasaṅketacandrikā} f.\,110r (\attr HP)
\begin{variants}
mārgeṇa HSC~] dvāreṇa YCM   
\end{variants}

% \begin{versinnote}
% \tl{yena dvāreṇa gantavyaṃ brahmasthānaṃ nirāmayam/\\+}
% \tl{mukhenācchādya taddvāraṃ prasuptā parameśvarī//\\!}
% \end{versinnote}

% \emph{Haṭhasaṅketacandrikā} f.\,110r (\attr to the \emph{Haṭhapradīpikā})
% \begin{versinnote}
% \tl{yena mārgeṇa gaṃtavyaṃ brahmasthānaṃ nirāmayaṃ/\\+}
% \tl{mukhenācchādya taddvāraṃ prasuptā parameśvarī//\\!}
% \end{versinnote}
\end{testimonia}
%</ts93-4>

%<*cm93-4>
%\begin{philcomm}[hp03_093_4]
%\end{philcomm}
%</cm93-4>


%%%%%%%%%%
\subsection*{3.93*5}
%<*tr93-5>
\begin{translation}[hp03_093_5]
The coiled goddess, who sleeps above the bulb [in the abdomen], leads to liberation for yogis and bondage for the deluded. He who knows her knows yoga.
\end{translation}
%</tr93-5>

%<*sc93-5>
\begin{sources}[hp03_093_5]
\emph{Vivekamārtaṇḍa} 39
\begin{variants}
suptā mokṣāya yoginām VM~] aṣṭadhā kuṃḍalīkṛtā VM\vl, aṣṭadhā kuṇḍalākṛtiḥ VM\vl, śubha\-mokṣā\-pradāyinī VM\vl, śubhā mokṣapradāyinī VM\vl, aṣṭadhā kuṭilīkṛtā VM\vl    
\end{variants}

% \begin{versinnote}
% \tl{kandordhvaṃ kuṇḍalī śaktir suptā mokṣāya yoginām/\\+}
% \tl{bandhanāya ca mūḍhānāṃ yas tāṃ vetti sa yogavit//\\!}
% \end{versinnote}
% \begin{appinnote}
% \tl{\textbf{b} suptā mokṣāya yoginām~] VTvlH; aṣṭadhā kuṃḍalīkṛtā A, aṣṭadhā kuṇḍalākṛtiḥ GLGPk, śubhamokṣāpradāyinī GB, śubhā mokṣapradāyinī GP, aṣṭadhā kuṭilīkṛtā T \\!}
% \end{appinnote}
\end{sources}
%</sc93-5>

%<*ts93-5>
\begin{testimonia}[hp03_093_5]
\emph{Yogacintāmaṇi} f.\,78v (\attr \emph{Haṭhayoga}), \emph{Haṭhasaṅketacandrikā} f.\,110r (\attr HP)
\begin{variants}
kandordhvaṃ HSC~] kandordhve YCM\sep
śaktiḥ suptā HSC~] śaktir buddhā YCM
\end{variants}

% \begin{versinnote}
% \tl{kandordhve kuṇḍalī śaktir buddhā mokṣāya yoginām/\\+}
% \tl{bandhanāya ca mūḍhānāṃ yas tām vetti sa yogavit//\\!}
% \end{versinnote}

% \emph{Haṭhasaṅketacandrikā} f.\,110r (\attr to the \emph{Haṭhapradīpikā})
% \begin{versinnote}
% \tl{kandordhvaṃ kuḍalī śakti suptā mokṣāya yogināṃ/\\+}
% \tl{bandhanāya ca mūḍhānāṃ yas taṃ vetti sa yogavit//\\!}
% \end{versinnote}
\end{testimonia}
%</ts93-5>

%<*cm93-5>
%\begin{philcomm}[hp03_093_5]
%\end{philcomm}
%</cm93-5>


%%%%%%%%%%
\subsection*{3.93*6}
%<*tr93-6>
\begin{translation}[hp03_093_6]
[Just as] the coiled serpent Ananta (\emph{śeṣa\-kuṇḍalī}) is the foundation of the oceans, mountains and islands, so Kuṇḍalinī is the foundation of all systems of yoga.
\end{translation}
%</tr93-6>

%<*sc93-6>
%\begin{sources}[hp03_093_6]
%\end{sources}
%</sc93-6>

%<*ts93-6>
\begin{testimonia}[hp03_093_6]
Cf. \emph{Haṭharatnāvalī} 2.124 (see HP 3.1)
\mylb
% \begin{versinnote}
% \tl{saśailavanadhātryās tu yathādhāro 'hināyakaḥ/\\+}
% \tl{aśeṣayogatantrāṇāṃ tathādhāro hi kuṇḍalī//\\!}
% \end{versinnote} cited as testimonium for HP 3.1

\emph{Yogacintāmaṇi} f.\,78v (\attr \emph{Haṭhayoga})
\mylb

% \begin{versinnote}
% \tl{ambhodhiśailadvīpānām ādhāraḥ śeṣakuṇḍalī/\\+}
% \tl{aśeṣayogatantrāṇām ādhāraḥ kuṇḍalī tathā//\\!}
% \end{versinnote}
\end{testimonia}
%</ts93-6>

%<*cm93-6>
\begin{philcomm}[hp03_093_6]
%No known source. Authorial? 
This verse is similar to 3.1.
\end{philcomm}
%</cm93-6>

\begin{metre}[hp03_093_6]
Anuṣṭubh (a: ma-vipulā)
\end{metre}

%%%%%%%%%%
\subsection*{3.93*7}
%<*tr93-7>
\begin{translation}[hp03_093_7]
Kuṇḍalinī is said to have a curved shape like a snake. The person who makes that goddess move is sure to be liberated.
\end{translation}
%</tr93-7>

%<*sc93-7>
%\begin{sources}[hp03_093_7]
%\end{sources}
%</sc93-7>

%<*ts93-7>
\begin{testimonia}[hp03_093_7]
\emph{Yogacintāmaṇi} f.\,78v–79r (\attr \emph{Haṭhayoga})

% \begin{versinnote}
% \tl{kuṇḍalī kuṭilākārā sarpavat parikīrtitā/\\+}
% \tl{sā śaktiś cālitā yena sa mukto nātra saṃśayaḥ//\\!}
% \end{versinnote}
\end{testimonia}
%</ts93-7>

%<*cm93-7>
%\begin{philcomm}[hp03_093_7]
%No known source.
%\end{philcomm}
%</cm93-7>


%%%%%%%%%%
\subsection*{3.94}
%<*tr94>
\begin{translation}[hp03_094]
Between the Gaṅgā and Yamunā is the wretched young widow. [The yogi] should forcefully take [her]. That is the supreme state of Viṣṇu.
\end{translation}
%</tr94>

%<*sc94>
\begin{sources}[hp03_094]
Cf.\,\emph{Śivasaṃhitā} 5.169
\begin{versinnote}
\tl{gaṅgāyamunayor madhye vahaty eṣā sarasvatī/\\+}
\tl{tāsāṃ tu saṃgame snātvā dhanyo yāti parāṃ gatim//\\!}
\end{versinnote}
\end{sources}
%</sc94>

%<*ts94>
\begin{testimonia}[hp03_094]
\emph{Yogacintāmaṇi} f.\,79r (\attr \emph{Haṭhayoga})
\mylb
% \begin{versinnote}
% \tl{gaṅgāyamunayor madhye bālaraṇḍā tapasvinī/\\+}
% \tl{balātkāreṇa gṛhṇīyāt tad viṣṇoḥ paramaṃ padam//\\!}
% \end{versinnote}

%Haṃsavilāsa (p. 46)
%\begin{versinnote}
%\tl{gaṅgāyamunayor madhye bālaraṇḍāṃ tapasvinīm/\\+}
%\tl{balātkāreṇa gṛhṇīyāt tad viṣṇoḥ paramaṃ padam//\\!}
%\end{versinnote}

% Prāṇatoṣiṇī Part 6 (\attr to the \emph{Dattātreyasaṃhitā})
% \begin{versinnote}
% \tl{gaṅgāyamunayor madhye bālāraṇḍā tapasvinī/ \\+}
% \tl{balād ākṛṣya gṛhnīyāt tad viṣṇoḥ paramaṃ padam//\\!}
% \end{versinnote}

%Bodhasāra 39.34, 12.3 (check, has commentary too):

\end{testimonia}
%</ts94>

%<*cm94>
\begin{philcomm}[hp03_094]
%No known source. 
The referent of \emph{bālaraṇḍā tapasvinī} here is unclear. In some manuscripts this verse is followed by one (3.94*1) in which \emph{bālaraṇḍā} is identified as \emph{sarasvatī}, which in the context of \emph{śakticālana} could refer to the tongue. She could also be Kuṇḍalinī, who in 3.93*5 is located at the navel, which is said to be the location of Viṣṇu (e.g.~\emph{Dhyāna\-bindū\-paniṣat} 28–30). In his commentary on this verse in the \emph{Bodhasāra} (1906: 137), Divākara says that the seizing of Kuṇḍalinī itself is the highest state of Viṣṇu (\,... \emph{bālaraṇḍāṃ ... gṛhṇīyād vaśīkuryāt tat tasyā vaśīkaraṇam eva viṣṇor vyāpana\-lakṣa\-nasya paramā\-tmanaḥ paramaṃ kevalaṃ ... padaṃ svarūpaṃ jñeyam}).
% Narahari N Divākara Divākara Dayānanda Sarasvatī. Bodhasāraḥ = Bodhsār : A Treatise on Vedanta. Arthadīptyā Sahitaḥ. Benares: Vidya Vilas Pr; 1906.

\end{philcomm}
%</cm94>

%%%%%%%%%%
\subsection*{3.94*1}
%<*tr94-1>
\begin{translation}[hp03_094_1]
The Blessed Gaṅgā is the Iḍā [channel], the river Yamunā is Piṅgalā, [and] the young widow, the Sarasvatī, is between Iḍā and Piṅgalā.
\end{translation}
% JB I assume we have to translate this. I've just done a quick draft. Please check. I havent used diacritics for the English names of the rivers. I've also added the MBh verse as source cf., as it seems important (please check the code). JM: I think I'd prefer Gaṅgā etc. (which we already have in the previous verse).  And I think turn it around: "The Blessed Gaṅgā is the Iḍā [channel], the river Yamunā is the Piṅgalā [channel] and the young widow..." as it's explaining what's referred to in the previous verse.
%</tr94-1>

%<*cm94-1>
\begin{philcomm}[hp03_094_1]
Verse 3.94*1, which has no known source, simply identifies the technical terms in 3.94, namely, \emph{gaṅgā}, \emph{yamunā} and \emph{bālaraṇḍā} as \emph{iḍā, piṅgalā} and \emph{sarasvatī}. The verse appears to be missing in \alphaThree, and a different version of it occurs in \alphaTwo. The relevant folio is missing in \alphaOne. Nonetheless, the verse is absent in the most reliable manuscripts of the \textepsilon, \textgamma, \textdelta, and \textpi\ groups, and so it is likely that it came into the text as a marginal note early in the transmission. %
\end{philcomm}
%</cm94-1>
% MD: α1 is unavailable. JB: okay, I've mentioned this.
% α3 is damaged, so it is uncertain, but most probably it didn't have the verse. There is not enough space for it.
% α2 has this verse in a correct position, but its second half is different from that of the other recensions:
% iḍā bhagavatī gaṅgā piṅgalā yamunā nadī/
% tayor madhye prayāgaṃ tu yas taṃ veda sa vedavit//
\begin{sources}[hp03_094_1]
Cf. \emph{Mahābhārata}, \emph{Bhīṣmaparvan} 40.78 (supplementary verses 3A.41–42)
\begin{versinnote}
\tl{iḍā bhagavatī gaṅgā piṅgalā yamunā nadī/\\+}
\tl{tayor madhye tṛtīyā tu tat prayāgam anusmaret//\\!}
\end{versinnote}    
\end{sources}


%%%%%%%%%%
\subsection*{3.95}
%<*tr95>
\begin{translation}[hp03_095]
Seizing her tail, the fearless [yogi] wakes the sleeping serpent. She shakes off sleep and is forced to stand up straight.
\end{translation}
%</tr95>

%<*sc95>
%\begin{sources}[hp03_095]
%\end{sources}
%</sc95>

%<*ts95>
\begin{testimonia}[hp03_095]
\emph{Haṭharatnāvalī} 2.118, \emph{Yogacintāmaṇi} f.\,79r (\attr HP)
\begin{variants}
pucchaṃ YCM~] pucche HRĀ\sep
abhīḥ HRĀ~] abhi YCM
\end{variants}

% \begin{versinnote}
% \tl{pucche pragṛhya bhujagīṃ suptām udbodhayed abhīḥ /\\+}
% \tl{nidrāṃ vihāya sā ṛjvī ūrdhvam uttiṣṭhate haṭhāt//\\!}
% \end{versinnote}

% \emph{Yogacintāmaṇi} f.\,79r (\attr to the \emph{Haṭhayoga})
% \begin{versinnote}
% \tl{pucchaṃ pragṛhya bhujagīṃ suptām udbodhayed abhi/\\+}
% \tl{nidrāṃ vihāya sā ṛjvī ūrdhvam uttiṣṭhate haṭhāt//\\!}
% \end{versinnote}

\end{testimonia}
%</ts95>

%<*cm95>
%\begin{philcomm}[hp03_095]
%No known source.
%\end{philcomm}
%</cm95>

\begin{metre}[hp03_095]
Anuṣṭubh (a: na-vipulā)
\end{metre}

%%%%%%%%%%
\subsection*{3.95*1}
%<*tr95-1>
\begin{translation}[hp03_095_1]
The yogi should breathe in through the solar channel, take hold of the open-mouthed hooded [serpent] by wrapping a cloth around [her] and move her sideways for an hour and a half in the morning and evening.
\end{translation}
%</tr95-1>
%greyscaled

%<*sc95-1>
%\begin{sources}[hp03_095_1]
%\end{sources}
%</sc95-1>

%<*ts95-1>
\begin{testimonia}[hp03_095_1]
\emph{Yogacintāmaṇi} f.\,79r (\attr \emph{Haṭhayoga}), \emph{Yogalakṣnāvalī} f.30v
\begin{variants}
pravistṛtāsyaiva phaṇāvatī sā~] paristhitā caiva phaṇāvatī sā YCM, vajrāsanasthā bhujagī pragṛhya YLĀ\sep
sūryāt YLĀ~] sauryā YCM\sep
paridhānayuktyā~] paridhānamuktā YCM, paridhānayuktā YLĀ\sep
pragṛhya tiryak paricālanīyā~] pragṛhya niryāti vicālitā sā YCM, pāyuṃ samākuṃcya ca cālanīyā YLĀ
\end{variants}

% \begin{versinnote}
% \tl{paristhitā caiva phaṇāvatī sā\\+}
% \tl{prātaś ca sāyaṃ praharārdhamātram/\\+}
% \tl{prapūrya sauryā paridhānamuktā\\+}
% \tl{pragṛhya niryāti vicālitā sā//\\!}
% \end{versinnote}

% \emph{Yogalakṣnāvalī} f.30v
% \begin{versinnote}
% \tl{vajrāsanasthā bhujagī pragṛhya\\+}
% \tl{prātaś ca sāyaṃ praharārdhamātram/\\+}
% \tl{prapūrya sūryāt paridhānayuktā\\+} 
% \tl{pāyuṃ samākuṃcya ca cālanīyā//\\!}
% \end{versinnote}

Cf. \emph{Haṭhasaṅketacandrikā} f.\,110v–111r (\attr HP)
\begin{versinnote}
\tl{tadvidhim āha/\\+}
\tl{\ \ \ paristhitā caiva phaṇāvatī sā\\+}
\tl{\ \ \ prātaś ca sāyaṃ praharārdhamātraṃ/\\+}
\tl{\ \ \ prapūrya sūryāt paridhānayuktā\\+}
\tl{\ \ \ pragṛhya tīrthāt paricālanīyā//\\+}
\tl{paridhān[a]yukteti dvādaśāṅgulapramitasitasūkṣmacaturaṅgulavisṛtaśuddhavastrakhaṇḍena dṛḍhaṃ veṣṭatā sā prasiddhā [ph]aṇāvatī suṣumṇātmakā arundhatī jihvaiva kuṇḍalinī// uktaṃ ca//\\+}
\tl{\ \ \ arundhatī bhavej jihvā dhruvo nāsāgramaṇḍalam iti//\\+}
\tl{tāṃ jihvāṃ laṃbikāyogenordhvaṃ tālvantarbhrūmadhyadeśe vihitāṃ tatas tīrthād bhrūmadhyāt pragṛhya adhaḥ kṛtvā tasyā gurūpadiṣṭavartmanā cālanaṃ vidheyam iti saṃketaḥ[/] cālanaṃ tu khecarī mudrā sādhanavad vidheyaṃ[/] tīrthaṃ bhrūmadhyaḥ[/] \\!}
\end{versinnote}

% \emph{Haṭhatattvakaumudī} 44.5
% \begin{versinnote}
% \tl{paristhitasyeha phaṇāvatī sā\\+}
% \tl{prātas tu sāyaṃ praharārdhamātram/\\+}
% \tl{prapūrya sūryāt paridhānayuktā\\+} 
% \tl{pragṛhya niryāt paricālanīyā//\\!}
% \end{versinnote}
\end{testimonia}
%</ts95-1>

%<*cm95-1>
\begin{philcomm}[hp03_095_1]
Verses 3.95*1–2 are absent in \alphaTwo\ and \alphaThree. They introduce the idea of awakening \emph{kụṇḍalinī} by moving the tongue with a cloth, which is a practice called \emph{sarasvatīcālana} in the \emph{Gorakṣaśataka} (16–25). These verses do not have a known source and are somewhat obscure unless one is aware of the more coherent explanation of this practice in the \emph{Gorakṣaśataka}. In his \emph{Haṭhasaṅketacandrikā} (see testimonia), Sundaradeva makes sense of this verse by equating the tongue with Kuṇḍalinī. This enables him to understand the reference to the cloth (\emph{paridhāna}) as the technique of wrapping the tongue in a cloth and milking it (i.e.~\emph{sarasvatīcālana} in the \emph{Gorakṣaśataka}). This interpretation also makes sense of the next verse in the \emph{Haṭhapradīpikā} (3.11), which describes the cloth.

%GŚ 22 says to move the tongue for two muhūrtas
%adopt:
%paristhitā caiva phaṇāvatī sā
%prātaś ca sāyaṃ praharārdhamātram
%prapūrya sūryāt paridhānayuktyā
%pragṛhya nityaṃ paricālanīyā
\end{philcomm}
%</cm95-1>

\begin{metre}[hp03_095_1]
Upajāti
\end{metre}

%%%%%%%%%%
\subsection*{3.95*2}
%<*tr95-2>
\begin{translation}[hp03_095_2]
It is said that the characteristics of the cloth for wrapping around [the tongue] are that it is a handspan long, four fingerbreadths wide, soft and white.
\end{translation}
%</tr95-2>
%greyscaled

%<*sc95-2>
\begin{sources}[hp03_095_2]
Cf.\,\emph{Gorakṣaśataka} 20cd
\begin{versinnote}
\tl{dvādaśāṅguladairghyaṃ cāmbaraṃ caturaṅgulam\\!}
\end{versinnote}
\end{sources}
%</sc95-2>

%<*ts95-2>
\begin{testimonia}[hp03_095_2]
\emph{Yogabīja} 81 (South Indian recension), \emph{Yogacintāmaṇi} f.\,74r (\attr \emph{Yogabīja} in the context of \emph{khecarīmudrā}), \emph{Haṭhayogasaṃhitā} p.\,44
\begin{variants}
dairghyaṃ YB~] dīrgha YCM, dīrghaṃ\sep
vistāraṃ YCM~] vistāre YB HYS\sep
proktaṃ YB YCM~] sūkṣmaṃ HYS\sep
āmbara YB HYS~] ādhāra YCM
\end{variants}

% \begin{versinnote}
% \tl{vitastipramitaṃ dairghyaṃ vistāre caturaṅgulam/\\+}
% \tl{mṛdulaṃ dhavalaṃ proktaṃ veṣṭanāmbaralakṣaṇam //\\!}
% \end{versinnote}

% \emph{Yogacintāmaṇi} f.\,74r (\attr to the \emph{Yogabīja} in the context of \emph{khecarīmudrā})
% \begin{versinnote}
% \tl{yogabīje—\\+}
% \tl{vitastipramitaṃ dīrgha[ṃ] vistāraṃ caturaṅgulam/\\+}
% \tl{mṛdulaṃ dhavalaṃ proktaṃ veṣṭanādhāralakṣaṇam//\\!}
% \end{versinnote}

% \emph{Haṭhayogasaṃhitā} p.\,44
% \begin{versinnote}
% \tl{vitastipramitaṃ dīrghaṃ vistāre caturaṅgulam/\\+}
% \tl{mṛdulaṃ dhavalaṃ sūkṣmaṃ veṣṭanāmbaralakṣaṇam//\\!}
% \end{versinnote}
\end{testimonia}
%</ts95-2>

%<*cm95-2>
\begin{philcomm}[hp03_095_2]
This verse was likely added to explain `by the method of the cloth' (\emph{paridhānayuktyā}) in the previous verse. One would expect to read \emph{caturaṅgulavistāram} in the second verse quarter, and the current reading is probably a result of the metre.
\end{philcomm}
%</cm95-2>


%%%%%%%%%%
\subsection*{3.96}
%<*tr96>
\begin{translation}[hp03_096]
Sitting in \emph{vajrāsana}, the yogī should stimulate Kuṇḍalinī and immediately afterwards perform \emph{bhastrī}. He quickly awakens Kuṇḍalinī.
\end{translation}
%</tr96>

%<*sc96>
%\begin{sources}[hp03_096]
%\end{sources}
%</sc96>

%<*ts96>
\begin{testimonia}[hp03_096]
\emph{Yogabīja} 111 (South Indian recension), \emph{Yogacintāmaṇi} f.\,79r (\attr \emph{Haṭhayoga}), \emph{Haṭhasaṅketacandrikā} f.\,111r
\begin{variants}
kuryād YB~] sūryād YCM HSC   
\end{variants}

% \begin{versinnote}
% \tl{vajrāsanasthito yogī cālayitvā tu kuṇḍalīm/\\+}
% \tl{kuryād anantaraṃ bhastrāṃ kuṇḍalīm āśu bodhayet//\\!}
% \end{versinnote}

% \emph{Yogacintāmaṇi} f.\,79r (\attr to the \emph{Haṭhayoga})
% \begin{versinnote}
% \tl{vajrāsanasthito yogī cālayitvā tu kuṇḍalīm/\\+}
% \tl{sūryād anantaraṃ bhastrā kuṇḍalīm āśu bodhayet//\\!}
% \end{versinnote}

% \emph{Haṭhasaṅketacandrikā} f.\,111r
% \begin{versinnote}
% \tl{vajrāsanasthito yogī cālayitvā tu kuṃḍalīṃ/\\+}
% \tl{sūryād anantaraṃ bhastrī kuṃḍalīm āśu bodhayet//\\!}
% \end{versinnote}
\end{testimonia}
%</ts96>

%<*cm96>
\begin{philcomm}[hp03_096]
\emph{Bhastrī} or \emph{bhastrikā kumbhaka} is taught at 2.60–68.The reference to \emph{vajrāsana} may be pointing to the practice of \emph{uḍḍiyāṇa}, which was described earlier in the chapter and is supposed to awaken Kuṇḍalinī. The contraction of the sun mentioned in the next verse supports this.
\end{philcomm}
%</cm96>


%%%%%%%%%%
\subsection*{3.97}
%<*tr97>
\begin{translation}[hp03_097]
[The yogi] should contract the sun and then stimulate Kuṇḍalinī. Even if he were in the jaws of death, why would he fear death? 
\end{translation}
%</tr97>

%<*sc97>
%\begin{sources}[hp03_097]
%\end{sources}
%</sc97>

%<*ts97>
\begin{testimonia}[hp03_097]
\emph{Yogabīja} 83 (South Indian recension), \emph{Yogacintāmaṇi} f.\,79r (\attr \emph{Haṭhayoga})
\begin{variants}
vaktra YCM~] vartma YB    
\end{variants}

% \begin{versinnote}
% \tl{bhānor ākuñcanaṃ kuryāt kuṇḍalīṃ cālayet tataḥ/ \\+}
% \tl{mṛtyuvartmagatasyāpi tasya mṛtyubhayaṃ kutaḥ//\\!}
% % mṛtyuvaktra° is supported by the North Indian recension and printed editions
% \end{versinnote}
% \begin{appinnote}
% \tl{\textbf{c} mṛtyuvartma°~] mṛtyuvaktra°,  mṛtyuvajra°, mṛtyupadma° \vl \\!}
% \end{appinnote}

% \emph{Yogacintāmaṇi} f.\,79r (\attr to the \emph{Haṭhayoga})
% \begin{versinnote}
% \tl{bhānor ākuñcanaṃ kuryāt kuṇḍalīṃ cālayet tataḥ/\\+}
% \tl{mṛtyuvaktragatasyāpi tasya mṛtyubhayaṃ kutaḥ//\\!}
% \end{versinnote}

Cf. \emph{Haṭhasaṅketacandrikā} f.\,111r
\begin{versinnote}
\tl{bhānor ākuñcanaṃ kuryāt kuṃḍalīṃ cālayet svataḥ/\\+}
\tl{mṛtyuvakragatasyāpi tasya mṛtyu[bha]yaṃ kutaḥ//\\+}
\tl{asyārthaḥ sūryanāḍyākarṣaṇena vahneḥ prācuryaṃ tasmāj jvalanatejasā apānavāyvākarṣaṇena vā kuṇḍalyābodho bhavati[/] tasya śakticālanakṛtābhyāsasya mṛtyubhayaṃ kuta iti[//]\\!}
\end{versinnote}
\end{testimonia}
%</ts97>

%<*cm97>
\begin{philcomm}[hp03_097]
In \emph{Jyotsnā} 3.116, Brahmānanda understands \emph{bhānor ākuñcanaṃ} as a contraction of the navel. In contrast, Sundaradeva, in his \emph{Haṭha\-saṃketa\-candrikā} (see testimonia), interprets it as drawing \emph{prāṇa} through the sun channel (\emph{sūrya\-nāḍyā\-karṣaṇa}), thereby intensifying the bodily fire.
\end{philcomm}
%</cm97>


%%%%%%%%%%
\subsection*{3.97*1}
%<*tr97-1>
\begin{translation}[hp03_097_1]
%Through the breath that is flowing through the path of the right nostril the [the yogi's] breath is greatly extended. 
When \textit{prāṇa} has been greatly extended as a result of breath flowing through the path of the right nostril, [the yogi], with his body already immortal, filled with the nectar of the moon from the uvula in that way,
% JB: It looks like we have translated ghaṇṭikāyāḥ pathaḥ whereas the pdf edition has ghaṇṭikāyās tathā ("and with [his] body already immortal, filled with the nectar of the moon from the uvula")
%[his] body is immortal, filled with the nectar of the moon from the uvula for the first time. JM: maybe translate tathā as "in that way" i.e. referring to the method, because "and" doesn't work with the locative absolute. 
\ sprinkling [with nectar] the network of channels at the aperture at [the centre of] the brow that have been subjugated by the mighty fire of time, makes his body completely new again like the trunk of a withered tree [when it regenerates].
\end{translation}
%</tr97-1>
%greyscaled
% Suggested translation with proposed reading °pavanāt, etc
% The vital force is greatly extended because of the breath flowing through the path of the right nostril. The [yogi's] body is immortal, already filled with the nectar of the moon through the path from the uvula.  

%<*sc97-1>
\begin{sources}[hp03_097_1]
\emph{Amaraughaśāsana} 6.1–2 (sic; a single \emph{śārdūlavikrīḍita} verse is numbered thus)

\begin{variants}
dakṣiṇa~] paścima AŚ\sep 
vāhi~] vāha AŚ\sep
prāṇo~] prāṇe AŚ\sep
candrābhaḥparipūritāmṛtatanuḥ~] candrāmbupratisāraṇāṃ sukṛtinaḥ AŚ\sep
kāyās tathaḥ~] kāyāḥ pathaḥ AŚ\sep
vaśagān bhrūrandhra~] vaśagaṃ bhūtvā sa AŚ \sep 
nāḍīgaṇān~] nāḍīśataṃ AŚ, nāḍīgataṃ AŚ\vl, nāḍīgaṇaṃ AŚ\vl\sep
tat kāyaṃ~] tat kāryaṃ AŚ
\end{variants}

% \begin{versinnote}
% \tl{nāsāpaścimamārgavāhapavanāt prāṇe 'tidīrghīkṛte\\+}
% \tl{candrāmbupratisāraṇāṃ sukṛtinaḥ prāg ghaṇṭikāyāḥ pathaḥ/\\+}
% \tl{siñcan kālaviśālavahnivaśagaṃ bhūtvā sa nāḍīśataṃ \\+}
% \tl{tat kāryaṃ kurute punar navatanuṃ jīrṇadrumaskandhavat//\\+}
% \tl{pratisāraṇānantaraṃ śaṅkhasāraṇā kathyate\\!}
% \end{versinnote}
% \begin{appinnote}
% \tl{nāḍīśataṃ~] nāḍīgataṃ, nāḍīgaṇaṃ \vl \\!}
% \end{appinnote}
\end{sources}
%</sc97-1>

%<*ts97-1>
\begin{testimonia}[hp03_097_1]
\emph{Yogacintāmaṇi} f.\,79r (\attr \emph{Haṭhayoga})
\begin{variants}
pavanāt~] pavano YCM\sep
prāṇo~] ghrāṇe YCM\sep
kṛte~] kṛtaḥ YCM\sep
kāyās tathaḥ~] kāyās tataḥ YCM\sep
siñcan~] bhindan YCM\sep
tat kāyaṃ~] taṃ kāyaṃ YCM
\end{variants}


% \emph{Yogacintāmaṇi} f.\,79r (\attr to the \emph{Haṭhayoga})
% \begin{versinnote}
% \tl{nāsādakṣiṇamārgavāhipavano ghrāṇe 'tidīrghīkṛtaḥ\\+}
% \tl{candrābhaḥparipūritāmṛtatanuḥ prāg ghaṇṭikāyās tataḥ/\\+}
% \tl{bhindan kālaviśālavahnivaśagān bhrūrandhranāḍīgaṇān \\+}
% \tl{taṃ kāyaṃ kurute punar navataraṃ jīrṇadrumaskandhavat//\\!}
% \end{versinnote}

% \emph{Haṭhasaṅketacandrikā} f.\,111r–111v
% \begin{versinnote}
% \tl{nāsādakṣiṇamārgavāhipavano ghrāṇe tidī[r]ghīkṛtaś\\+}
% \tl{caṃdrāṃ'bhaḥparipūritā'mṛtatanuḥ prāgh ghaṃṭikāyās tataḥ[/]\\+}
% \tl{bhindan kālaviśālavahnivaśagān bh[r]ū[ran]dhranāḍīgaṇāṃs \\+}
% \tl{taṃ kāyaṃ kurute punar navataraṃ jīrṇadrumaskandhavat[//]\\+}
% \tl{dakṣiṇe ghrāṇe nāsikāyām atidī[r]ghīkṛtaś ciraṃ kuṃbhakīkṛtaḥ prāk prathamaṃ caṃdrāṃ'bhaḥparipūritā'mṛtatanuḥ sādhakaḥ kartā pūrvam iḍayā dhṛtakuṃbhakenātisukhakarasudhopamena saṃtṛptiṃ samupagataḥ satatas tadanantaraṃ sūryanāḍyā cirāyā kalitaṃ kuṃbhauṣṇyātīkṣṇatarakuṃ tenaivauṣṇasvabhāvād ghaṇṭikāyāṃ jatruṇaḥ pañcamy arthaḥ tatsaṃbadhikān ity arthaḥ[/] evaṃvidhān kālabījān bh[r]ūrandhragataśirāpuñjān bhindan saṃchedayan svaṃ nijaśarīraṃ punar navataraṃ kuruta iti asyārthaḥ {/}\\!}
% \end{versinnote}

%Yogasārasaṃgraha has this too, check
\end{testimonia}
%</ts97-1>

%<*cm97-1>
\begin{philcomm}[hp03_097_1]
Verse 3.97*1 is absent in the \textalpha\ group. It is very close to a verse in the \textit{Amaraughaśāsana}, which is likely to be its source, although the date of the \textit{Amaraughaśāsana} is yet to be firmly established. This verse's import of rejuvenating the body by flooding it with nectar is not directly connected with those proceeding it. Its meaning is not clear in the published edtion of the \textit{Amaraugha\-śāsana} and we are not confident of the readings adopted in our edition nor the meaning of the verse.

%Adopt °pavano prāṇe 'tidīrghīkṛte
%Adopt siñcan from AŚ, or (JT) understand bhindan to mean cutting off channels that are under the influence of time
%Adopt tat pāda 4 and understand as tasmāt
%If yogi is subject taṃ doesn't work, adopt tat
%Simile of tree coming back to life after being burned or withering, through nectar passing through the channels
\end{philcomm}
%</cm97-1>

\begin{metre}[hp03_097_1]
Śārdūlavikrīḍita
\end{metre}

%%%%%%%%%%
\subsection*{3.97*2}
%<*tr97-2>
\begin{translation}[hp03_097_2]
After stimulating Kuṇḍalinī, the yogi should perform \emph{bhastrī} in particular. The god of death is afraid of the ascetic who regularly practises in this way.
\end{translation}
%</tr97-2>
%greyscaled

%<*sc97-2>
%\begin{sources}[hp03_097_2]
%\end{sources}
%</sc97-2>

%<*ts97-2>
\begin{testimonia}[hp03_097_2]
\emph{Yogacintāmaṇi} f.\,79r (\attr \emph{Haṭhayoga}), \emph{Haṭhasaṅketacandrikā} f.\,111v–112r
\begin{variants}
bhastrīṃ HSC~] bhastrāṃ YCM\sep
śaṅkate yamaḥ YCM~] śaṃphate manaḥ HSC
\end{variants}

% \begin{versinnote}
% \tl{kuṇḍalīṃ cālayitvā tu kuryād bhastrāṃ viśeṣataḥ/\\+}
% \tl{evamabhyāsato nityaṃ yaminaḥ śaṅkate yamaḥ//\\!}
% \end{versinnote}

% \emph{Haṭhasaṅketacandrikā} f.\,111v–112r
% \begin{versinnote}
% \tl{kuṇḍalīṃ cālayitvā tu kuryād bhastrīṃ viśeṣataḥ/\\+}
% \tl{evamabhyāsato nityaṃ yaminaḥ śaṃphate manaḥ//\\!}
%% JB: śaṃphate manaḥ doesnt make sense but perhaps we should record this verse as testimonia because it supports the note.
% \end{versinnote}
\end{testimonia}
%</ts97-2>

%<*cm97-2>
\begin{philcomm}[hp03_097_2]
Verses 3.97*2–3 are absent in the \textalpha\ group and have no known source. They elaborate further on the instruction to practise \emph{bhastrī kumbhaka} in 3.96. The practice of other \emph{kumbhaka}s for moving Kuṇḍalinī is mentioned in 3.97*3. \lb

The reading \emph{abhyāsato} in 3.97*2c, which is attested by manuscripts of the \textgamma\ group and the testimonia, is possible but seems to be a corruption of the participle, \emph{abhyasato}.%JM: MD note below, should we write abhyasato in the note?
% MD: abhyasyato seems to be overcorrect in this text in which always abhyaset is used instead of abhyasyet. The participle abhyasato (cf. 3.62 abhyasan, though this might be wrong for abhyasen) or the adverb abhyāsato (cf. 1.27c) may be more probable.

\end{philcomm}
%</cm97-2>


%%%%%%%%%%
\subsection*{3.97*3}
%<*tr97-3>
\begin{translation}[hp03_097_3]
Then [the yogi] should practise \emph{sūryabheda}, \emph{ujjāyī} and also \emph{śītalī}. Where is the god of death for the ascetic engaged in the practice in this way?
\end{translation}
%</tr97-3>
%greyscaled

%<*sc97-3>
%\begin{sources}[hp03_097_3]
%\end{sources}
%</sc97-3>

%<*ts97-3>
\begin{testimonia}[hp03_097_3]
\emph{Yogacintāmaṇi} f.\,79r (\attr \emph{Haṭhayoga})
\begin{variants}
cāpi YCM~] vāpi HSC\sep
śamano~] yamas tu YCM HSC
\end{variants}

% \begin{versinnote}
% \tl{tadābhyaset sūryabhedam ujjāyīṃ cāpi śītalīm/\\+}
% \tl{evam abhyāsayuktasya yamas tu yaminaḥ kutaḥ//\\!}
% \end{versinnote}

% \emph{Haṭhasaṅketacandrikā} f.\,112r
% \begin{versinnote}
% \tl{tadābhyaset sūryabhedam ujjāyīṃ vāpi śītalīm/\\+}
% \tl{evamabhyāsayuktasya yamas tu yaminaḥ kutaḥ//\\!}
% \end{versinnote}
\end{testimonia}
%</ts97-3>

%<*cm97-3>
\begin{philcomm}[hp03_097_3]
On why this verse is in greyscale, see the note to 3.97*2.

%No known source. the readings \emph{cāpi} and \emph{vāpi} are both possible, but the former is better attested.

%MD: adopt śamano in Pāda d? śamana = Yama.
\end{philcomm}
%</cm97-3>

\begin{metre}[hp03_097_3]
Anuṣṭubh (a: ra-vipulā)
\end{metre}

%%%%%%%%%%
\subsection*{3.98}
%<*tr98>
\begin{translation}[hp03_098]
As a result of fearlessly moving [Kuṇḍalinī] for one hour 36 minutes (two \emph{muhūrta}s), Suṣumṇā at Kuṇḍalinī is drawn up slightly.
\end{translation} %?? JB moving or stimulating [Kuṇḍalinī]? (in 3.7, 3.93, etc. we translate cālanam as stimulating... i think I prefer moving as stimulating seems a bit ambiguous). JM: agreed, it's tricky as sometimes moving seems better, sometimes stimulating. It's also complicated by the different understandings of what's actually going on here. If it means moving the tongue with a cloth, then moving is better, but if it means getting Kuṇḍalinī to move, the stimulating is usually better. Maybe a note to this effect when it's first used?
%</tr98>

%<*sc98>
\begin{sources}[hp03_098]
\emph{Gorakṣaśataka} 22c–23b
\begin{variants}
nirbhayaṃ~] nirbhayaś GŚ, nirbharaś GŚ\vl, nirbhayāc GŚ\vl\sep
cālanād asau~] cālayed imām GŚ\sep
ākṛṣyate GŚ~] ākarṣayet GŚ\vl\sep
suṣumnā kuṇḍalīgatā GŚ] suṣumnāṃ kuṇḍalīgatāṃ GŚ\vl, suṣumnā kuṇḍalīyutā GŚ\vl
\end{variants}

% \begin{versinnote}
% \tl{muhūrtadvayaparyantaṃ nirbhayaś cālayed imām/\\+}
% \tl{ūrdhvam ākṛṣyate kiṃcit suṣumnā kuṇḍalīgatā//\\!}
% \end{versinnote}
% \begin{appinnote}
% \tl{\textbf{22c} nirbhayaś~] YL ; nirbharaś T, nirbhayāc GU \\+}
% \tl{\textbf{23a} ākṛṣyate~] YL ; ākarṣayet TGU \\+}
% \tl{\textbf{23b} suṣumnā kuṇḍalīgatā~] TG; suṣumnāṃ kuṇḍalīgatāṃ U,  suṣumnā kuṇḍalīyutā YL \\!}
% \end{appinnote}

Cf.\,\emph{Śivasamhitā} 4.109
\begin{versinnote}
\tl{gurūpadeśavidhinā tasya mṛtyubhayaṃ kutaḥ/\\+}
\tl{muhūrtadvayaparyantaṃ vidhinā śakticālanam//\\!}
\end{versinnote}

\end{sources}
%</sc98>

%<*ts98>
\begin{testimonia}[hp03_098]
\emph{Haṭharatnāvalī} 2.121, \emph{Yogacintāmaṇi} f.\,79r (ab only) (\attr  \emph{Haṭhayoga})

\begin{variants}
nirbhayaṃ~] nirbhītaś HRĀ, nirbhayaś YLĀ, nirbharaṃ YCM\sep
cālanād asau~] cālayed asau HRĀ, cālayed imām YLĀ, dhi vai YCM\sep
ūrdhvam ākṛṣyate HRĀ~] ākṛṣya tau HRĀ\vl\sep
kuṇḍalī gatā HRĀ\vl~] suṣumṇāṃ kuṇḍalīgatām HRĀ, suṣumnā kuṇḍalīyutā YLĀ
\end{variants}

% \begin{versinnote}
% \tl{muhūrtadvayaparyantaṃ nirbhītaś cālayed asau/\\+}
% \tl{ūrdhvam ākṛṣyate kiṃ cit suṣumṇāṃ kuṇḍalīgatām/\\+}
% \tl{ṣaṇmāsāc cālanenaiva śaktis tasyordhvagā bhavet//\\!}
% \end{versinnote}
% \begin{appinnote}
% \tl{ākṛṣyate] ākṛṣya tau P, T, t1. kuṇḍalīgatām~] kuṇḍalī gatā P,T,t1 \\!}
% \end{appinnote}

% \emph{Yogalakṣaṇāvalī} f.\,31r (\attr to the \emph{Gorakṣaśata})
% \begin{versinnote}
% \tl{muhūrtadvayaparyantaṃ nirbhayaś cālayed imām/\\+}
% \tl{ūrdhvam ākṛṣyate kiṃcit suṣumnā kuṇḍalīyutā//\\!}
% \end{versinnote}

% \emph{Yogacintāmaṇi} f.\,79r (\attr to the \emph{Haṭhayoga})
% \begin{versinnote}
% \tl{muhūrtadvayaparyantaṃ nirbharaṃ cālanād dhi vai/\\!}
% \end{versinnote}

% \emph{Haṭhasaṅketacandrikā} f.\,112r
% \begin{versinnote}
% \tl{muhūrtadvayaparyantaṃ nirbharaṃ cālanād asau/\\+}
% \tl{ūrdhvam ākṛṣyate kiṃcit suṣumnākuṇḍalīgatam//\\+}
% \tl{ku[ṇ]ḍalyās tadānīṃ gatam upari yātaṃ kiṃ cit svalpamātra abhyāsasadṛśam ity arthaḥ[/] akṛṣyate uccaiḥ karoti[/] \\!}
% \end{versinnote}
\end{testimonia}
%</ts98>

%<*cm98>
\begin{philcomm}[hp03_098]
%?? Add reference to Jim's introduction
As noted in our introduction, Svātmārāma appears not to have understood the practice of \emph{śakticālana} in the same way as his primary source text for its description, the \emph{Gorakṣaśataka}, in which a cloth is wrapped around the tongue so that it can be repeatedly pulled, thereby lifting up the base of the central channel. He does not include the \emph{Gorakṣaśataka} verses which mention the tongue or the cloth (but some later recensions of the \emph{Haṭhapradīpikā} do introduce them). Verses 98 and 99 suggest that he understood the practice to involve repeated contraction of the region of the sun at the lower end of the central channel. The result is the same, namely that Kuṇḍalinī is awakened and uncoils herself, thereby allowing Prāṇa to enter the central channel.
\end{philcomm}
%</cm98>


%%%%%%%%%%
\subsection*{3.99}
%<*tr99>
\begin{translation}[hp03_099]
Extracted from Suṣumṇā by this [practice], Kuṇḍalinī leaves it. As a result of this, \emph{prāṇa} automatically enters Suṣumṇā.
\end{translation}
%</tr99>

%<*sc99>
\begin{sources}[hp03_099]
\emph{Gorakṣaśataka} 23c-24b
\begin{variants}
suṣumṇāyāḥ samuddhṛtā~] suṣumnāyā mukhaṃ dhruvam GŚ
\end{variants}
% \begin{versinnote}
% \tl{tena kuṇḍalinī tasyāḥ suṣumnāyā mukhaṃ dhruvam/\\+}
% \tl{jahāti tasmāt prāno'yaṃ suṣumnāṃ vrajati svataḥ//\\!}
% \end{versinnote}
\end{sources}
%</sc99>

%<*ts99>
\begin{testimonia}[hp03_099]
\emph{Haṭhatattvakaumudī} 44.25
\begin{variants}
suṣumṇāyāḥ samuddhṛtā~] suṣumṇāyā mukhaṃ dhruvam HTK    
\end{variants}

% \begin{versinnote}
% \tl{tadā kuṇḍalinī tasyāḥ suṣumṇāyā mukhaṃ dhruvam/\\+}
% \tl{jahāti tasmāt prāṇo 'yaṃ suṣumṇāṃ vrajati svataḥ// \\!}
% \end{versinnote}

% \emph{Haṭhasaṅketacandrikā} f.\,112r
% \begin{versinnote}
% \tl{tena proktaśakticālanena vidhinā kuṇḍalinī tasyāḥ suṣumṇāyāḥ samuddhṛtā jahāti tasmāt prāṇo [']yaṃ suṣumṇāṃ vrajati svataḥ[/] suṣumṇāyā antaḥ kiṃ cit tatka[r]tṛkordhvākarṣaṇena samyag ūrdhvavihitā yadā kuṇḍalī bhūry antaḥ praviṣṭety arthaḥ[/]   tadāyaṃ va[hny]āpānamanobhiḥ sārdhaṃ vijitaḥ kuṃḍalīpadaṃ prāptaḥ prāṇavāyuḥ svataḥ svasmāt pārthivarājasavikāraśoṣam iti śeṣaḥ[/] jahāti kuṃḍalībodhe suṣumṇāṃtaḥ pātaprabhāvād vigatāśeṣabāhyavāhaprasaṃ[ga] iti bhāvaḥ// tasmād dhetoḥ suṣumṇaṃ gacchatīti kevalakuṃbhako bhavatīty arthaḥ//\\!}
% \end{versinnote}
\end{testimonia}
%</ts99>

%<*cm99>
%\begin{philcomm}[hp03_099]
%\end{philcomm}
%</cm99>

\begin{metre}[hp03_099]
Anuṣṭubh (c: ma-vipulā)
\end{metre}

%%%%%%%%%%
\subsection*{3.100}
%<*tr100>
\begin{translation}[hp03_100]
Therefore [the yogi] should regularly make Arundhatī move, she who contains speech. By making her move the yogi is freed from diseases.
\end{translation}% ??JB By moving or stimulating her (cf. translation of śakticālana in 3.7, 3.93, etc.)
%</tr100>

%<*sc100>
\begin{sources}[hp03_100]
\emph{Gorakṣaśataka} 26cd–27ab
\begin{variants}
garbhām arundhatīm~] garbhāṃ sarasvatīm GŚ\sep
tasyāḥ~] yasyāḥ GŚ
\end{variants}
% \begin{versinnote}
% \tl{tasmāt saṃcālayen nityaṃ śabdagarbhāṃ sarasvatīm/\\+}
% \tl{yasyāḥ saṃcālanenaiva yogī rogaiḥ pramucyate//\\!}
% \end{versinnote}
\end{sources}
%</sc100>

%<*ts100>
\begin{testimonia}[hp03_100]
\emph{Yogalakṣaṇāvalī} f.\,31r  (\attr \emph{Gorakṣaśata}), \emph{Haṭhasaṅketacandrikā} f.\,112r (\attr HP)
\begin{variants}
garbhām arundhatīm HSC~] garbhāṃ sarasvatī YLĀ\sep
tasyāḥ~] asyāḥ YLĀ, yasyāḥ HSC\sep
cālanenaiva~] cālanenaivaṃ YLĀ, cālanenāśu HSC\sep
yogī rogaiḥ pramucyate HSC~] rogā naśyaṃti niścitaṃ YLĀ
\end{variants}

Cf.\,\emph{Haṭharatnāvalī} 2.122
\begin{versinnote}
\tl{sūryeṇa pūrayed vāyuṃ sarasvatyās tu cālayet/\\+}
\tl{śabdagarbhācālanena yogī rogaiḥ pramucyate//\\!}
\end{versinnote}

% \emph{Yogalakṣaṇāvalī} f.\,31r  (\attr to the \emph{Gorakṣaśata})
% \begin{versinnote}
% \tl{tasmāt saṃcālayen nityaṃ śabdagarbhāṃ sarasvatī/\\+}
% \tl{asyāḥ saṃcālanenaivaṃ rogā naśyaṃti niścitaṃ//\\!}
% \end{versinnote}

% \emph{Haṭhasaṅketacandrikā} f.\,112r (\attr to the \emph{Haṭhapradīpikā})
% \begin{versinnote}
% \tl{tasmāt saṃcālayen nityaṃ śabdagarbhām arundhatīm//\\+}
% \tl{yasyāḥ saṃcālanenāśu yogī rogaiḥ pramucyate//\\!}
% \end{versinnote}

%Yogasārasaṅgraha (p.59)
%\begin{versinnote}
%\tl{tasmāt saṃcālayen nityaṃ śaṃbhugarbhām aruṃdhatīm/\\+}
%\tl{yasyāḥ sañcālanenāśu yogī rogaiḥ pramucyate//\\!}
%\end{versinnote}
\end{testimonia}
%</ts100>

%<*cm100>
\begin{philcomm}[hp03_100]
%comment on śabdagarbhām aruṃdhatīṃ.
In the \emph{Gorakṣaśataka} (26cd–27ab), the source text, this verse occurs in a passage on \emph{saras\-vatī\-cālana}, which is the practice of moving the tongue (i.e.~\emph{sarasvatī}) by wrapping a cloth around it and tugging it in order to raise the lower end of the Suṣumṇā. In the \emph{Gorakṣaśataka}, Sarasvatī is said to be another name for Arundhatī and, since the tongue is instrumental for speech and \emph{saras\-vatī} is the name of a Goddess identified with speech (\emph{vāc}), the \emph{Gorakṣaśataka}'s reading of \emph{śabda\-garbhāṃ sarasvatīm} makes good sense. However, it seems that Svātmārāma has changed 3.100b to read \emph{śabdagarbhām arundhatīm} and has understood \emph{arundhatī} as Kuṇḍalinī. This is affirmed in longer versions of the \emph{Haṭhapradīpikā} (3.93*2) that contain a verse on synonyms of Kuṇḍalinī, which include Arundhatī. We are yet to find Arundhatī equated with Kuṇḍalinī in a text composed before the \emph{Haṭhapradīpikā} but this identification is found in subsequent compendiums and commentaries (e.g.~\emph{Yogacintāmaṇi} f.\,78v, \emph{Yuktabhavadeva} 7.300, \emph{Jyotsnā} 104, 119, \emph{Yogaprakāśikā} 5.166).
\end{philcomm}
%</cm100>



%%%%%%%%%%
\subsection*{3.101}
%<*tr101>
\begin{translation}[hp03_101]
The yogi who has made Kuṇḍalinī move is worthy of success. There is no point in speaking at length about this. He easily conquers death.
\end{translation}
%</tr101>

%<*sc101>
%\begin{sources}[hp03_101]
%\end{sources}
%</sc101>

%<*ts101>
\begin{testimonia}[hp03_101]
\emph{Haṭharatnāvalī} 2.123, \emph{Yogalakṣaṇāvalī} (f.\,31r)  (\attr \emph{Gorakṣaśata}), \emph{Haṭhasaṅketacandrikā} f.\,112r (\attr HP)
\begin{variants}
sa yogī siddhibhājanam HSC~] sa yogī siddhibhājanaḥ HRĀ, śabdagarbhā tv aruṃdhatī YLĀ\sep
kālaṃ jayati līlayā HSC~] mṛtyuṃ jayati līlayā HRĀ, tasya kālabhayaṃ na hi YLĀ
\end{variants}

%\begin{versinnote}
%\tl{yena saṃcālitā śaktiḥ sa yogī siddhibhājanaḥ/\\+}
%\tl{kim atra bahunoktena mṛtyuṃ jayati līlayā// 2.123//\\!}
%\end{versinnote}
%
%\emph{Yogalakṣaṇāvalī} (f. 31r)  (\attr Gorakṣaśata)
%\begin{versinnote}
%\tl{yena saṃcālitā śaktiḥ śabdagarbhā tv aruṃdhatī/\\+}
%\tl{kim atra bahunoktena tasya kālabhayaṃ na hi//\\!}
%\end{versinnote}

%\emph{Haṭhasaṅketacandrikā} f.\,112r (\attr to the \emph{Haṭhapradīpikā})
%\begin{versinnote}
%\tl{yena saṃcālitā śaktiḥ sa yogī siddhibhājanaṃ/\\+}
%\tl{kim atra bahunoktena kālaṃ jayati līlayā//\\!}
%\end{versinnote}
\end{testimonia}
%</ts101>

%<*cm101>
%\begin{philcomm}[hp03_101]
%siddhibhājana is also possible.
%\end{philcomm}
%</cm101>


%%%%%%%%%%
\subsection*{3.101*1}
%<*tr101-1>
\begin{translation}[hp03_101_1]
For the yogi who observes celibacy, always eats a healthy and measured diet, and practises with Kuṇḍalinī, success is seen after forty days (\emph{maṇḍalāt}).
\end{translation}
%</tr101-1>
%greyscaled

%<*sc101-1>
%\begin{sources}[hp03_101_1]
%\end{sources}
%</sc101-1>

%<*ts101-1>
\begin{testimonia}[hp03_101_1]
\emph{Yogacintāmaṇi} f.\,79r (\attr \emph{Haṭhayoga}), \emph{Haṭhasaṅketacandrikā}  f.\,112r--112v (\attr HP)
\begin{variants}
vratasyaiva~] ratasyaiva YCM, jatasyaiva HSC\sep
hitamitāśinaḥ YCM~] hitamitāṃ śanaiḥ HSC
\end{variants}

%\begin{versinnote}
%\tl{brahmacaryaratasyaiva nityaṃ hitamitāśinaḥ/\\+}
%\tl{maṇḍalād dṛśyate siddhiḥ kuṇḍalyabhyāsayoginaḥ//\\!}
%\end{versinnote}

Cf. \emph{Yogalakṣaṇāvalī} (f.\,31r)  (\attr \emph{Gorakṣaśataka})
\begin{versinnote}
\tl{brahmacaryavratasyaiva kuṃḍalyabhyāsayoginaḥ//\\+}
\tl{maṇḍalād dṛśyate siddhir iti yogavido viduḥ//\\!}
\end{versinnote}

%\emph{Haṭhasaṅketacandrikā}  f.\,112r--112v (\attr to the \emph{Haṭhapradīpikā})
%\begin{versinnote}
%\tl{brahmacaryajatasyaiva nityaṃ hitamitāṃ śanaiḥ/\\+}
%\tl{maṇḍalād dṛśyate siddhiḥ kuṇḍalyabhyāsayoginaḥ//\\!}
%\end{versinnote}
\end{testimonia}
%</ts101-1>

%<*cm101-1>
\begin{philcomm}[hp03_101_1]
Verse 3.101*1 has been omitted by \alphaOne\ and \alphaThree. It is in \alphaTwo\ at the end of a block of verses (3.93*2–5, 93*7) that is excluded by \alphaThree. This block appears after 3.96 and appears to have been inserted from elsewhere. 3.101*1 has no known source and appears to have been added as a general laudatory statement on the benefits of practising with Kuṇḍalinī.\lb

The meaning of \emph{maṇḍalād} in 3.101*1c is not clear. Brahmānanda understands it as a period of time (i.e.~forty days) but we are yet to find this attested elsewhere.
\end{philcomm}
%</cm101-1>


%%%%%%%%%%
\subsection*{3.102}
%<*tr102>
\begin{translation}[hp03_102]
The yogi should mix with ash the fluid of the moon emitted as a result of the practice. Wearing that [mixture] on the head bestows divine sight.
\end{translation}
%</tr102>

%<*sc102>
%\begin{sources}[hp03_102]
%\end{sources}
%</sc102>

%<*ts102>
\begin{testimonia}[hp03_102]
\emph{Yogalakṣaṇāvalī} f.\,31r  (\attr \emph{Gorakṣaśataka})
\begin{variants}
tūttamāṅge~] cottamāṃge YLĀ
\end{variants}

%\begin{versinnote}
%\tl{abhyāsaniḥsṛtāṃ cāndrīṃ vibhūtyā saha miśrayet/\\+}
%\tl{taddhāraṇaṃ cottamāṃge divyadṛṣṭipradāyakaṃ//\\!}
%\end{versinnote}

Cf. \emph{Haṭhasaṅketacandrikā} f.\,112v (\attr HP)
\begin{versinnote}
\tl{abhyāsaniḥsṛtāṃ cāndrīṃ vibhūtyā saha miśrayet[/]\\+}
\tl{taddhāraṇaṃ tūttamāṃge divyadṛṣṭipradāyakaṃ[//] 19\\+}
\tl{cāndrīṃ lalāṭacandrān niḥsṛtāṃ abhyāse śramajātāṃ gharmadhārāṃ tāṃ vibhūtyā vimiśrayet/ tām uttamāṃge śirasi dhārayed asau sādhakasya divyadṛṣṭipradā bhravatīty arthaḥ[/]\\!}
\end{versinnote}

Cf. \emph{Haṭhayogasaṃhitā} p.\,41 (on \emph{amarolī})
\begin{versinnote}
\tl{abhyāsān niḥsṛtāṃ cāndrīṃ vibhūtyā saha miśrayet/\\+}
\tl{dhārayed uttamāṅgeṣu divyadṛṣṭiḥ prajāyate// \\!}
\end{versinnote}
\end{testimonia}
%</ts102>

%<*cm102>
\begin{philcomm}[hp03_102]
In the witnesses of the earliest reconstructable recensions of the \emph{Haṭhapradīpikā}, including the \textalpha\ manuscripts, this verse occurs in the section on \emph{śakticālana}. This is also the case in the \emph{Yoga\-lakṣaṇāvalī}, \emph{Haṭhasaṅketacandrikā} and the longer recensions of the \emph{Haṭhapradīpikā} with six and ten chapters. However, in the context of \emph{śakticālana}, the referent of \emph{cāndrī} is unclear. It appears to be understood as some sort of lunar fluid. In his \emph{Haṭhasaṅketacandrikā} (see the testimonia), Sundaradeva defines it as a flow of perspiration (\emph{gharmadhārā}) that arises from exertion in the practice and is emitted from the moon in forehead (\emph{lalāṭacandra}). In the \emph{Yogaprakāśikā}, Bālakṛṣṇa glosses it simply as nectar (\emph{sudhā}) (5.182). In the \emph{Haṭhayogasaṃhitā} (p.\,41) and \emph{Jyotsnā} (3.98), this verse is in the section on \emph{amarolī}, which provides a clear referent of \emph{cāndrī} as the cool middle flow of urine (see 3.96 [3.90 in our edition]).  

%Verse 120 is more appropriate for finishing the śakticālana section.
%Note Brahmānanda's interpretation of uttamāṅgeṣu, but usually uttamāṅga refers to the head.
\end{philcomm}
%</cm102>

\begin{metre}[hp03_102]
Anuṣṭubh (c: ra-vipulā)
\end{metre}

%%%%%%%%%%
\subsection*{3.102*1}
%<*tr102-1>
\begin{translation}[hp03_102_1]
For purifying the seventy-two thousand channels, there is no method of cleansing without the practice of Kuṇḍalinī.
\end{translation}
%</tr102-1>
%greyscaled

%<*sc102-1>
%\begin{sources}[hp03_102_1]
%\end{sources}
%</sc102-1>

%<*ts102-1>
\begin{testimonia}[hp03_102_1]
\emph{Yogacintāmaṇi} f.\,79v (\attr \emph{Haṭhayoga}), \emph{Yogalakṣaṇāvalī} f.\,31r  (\attr \emph{Gorakṣaśataka})
\begin{variants}
malaśodhane~] malaśodhanam YCM, api śodhanaṃ YLĀ\sep
kutaḥ prakṣālanopāyaḥ YCM~] asatkalpaṃ smṛtaṃ siddhaiḥ YLĀ\sep
abhyasanād ṛte YLĀ~] abhyāsato vinā YCM
\end{variants}

%\begin{versinnote}
%\tl{dvisaptatisahasrāṇāṃ nāḍīnāṃ malaśodhanam/\\+}
%\tl{kutaḥ prakṣālanopāyaḥ kuṇḍalyabhyāsato vinā//\\!}
%\end{versinnote}

%\emph{Yogalakṣaṇāvalī} f.\,31r  (\attr to Gorakṣaśataka)
%\begin{versinnote}
%\tl{dvisaptatisahasrāṇāṃ nāḍīnām api śodhanaṃ/\\+}
%\tl{asatkalpaṃ smṛtaṃ siddhaiḥ kuṃḍalyabhyasanād ṛte//\\!}
%\end{versinnote}

%\emph{Upāsanāsārasaṅgraha} (p.\,36) % IFP Transcript T1095
%\begin{versinnote}
%\tl{dvāsaptatisahasrāṇāṃ nāḍīnāṃ malaśodhane/ \\+}
%\tl{kutaḥ prakṣālanopāyaḥ kuṇḍalyabhyasanād ṛte/\\!}
%\end{versinnote}

%\emph{Haṭhasaṅketacandrikā} f.\,112v (\attr to the \emph{Haṭhapradīpikā})
%\begin{versinnote}
%\tl{dvisaptatisahastrāṇāṃ nāḍīnāṃ malaśodhanaṃ/\\+}
%\tl{kutaḥ prakṣālanopāyaḥ kuṃḍalyābhyāsanād ṛte//\\!}
%\end{versinnote}
\end{testimonia}
%</ts102-1>

%<*cm102-1>
\begin{philcomm}[hp03_102_1]
This verse is omitted by the \textalpha\ group. Its claim is not entirely consistent with the role of the \emph{ṣaṭkarma}, which are cleansing techniques that do not require Kuṇḍalinī. It was likely added to the original text as a further laudatory statement on the practice of Kuṇḍalinī.% JM: It's an odd statement, no? Kuṇḍalinī is not mentioned in the ṣaṭkarma teachings. Note?
\end{philcomm}
%</cm102-1>


%%%%%%%%%%
\subsection*{3.102*1 ending}
% iti śakticālanaṃ/
%<*tr102-1p>
\begin{translation}[hp03_102_1p]
\end{translation}
%</tr102-1p>

%<*cm102-1p>
% \begin{philcomm}[hp03_102_1p]
% \end{philcomm}
%</cm102-1p>

%%%%%%%%%%
\subsection*{3.103}
%<*tr103>
\begin{translation}[hp03_103]
Thus have the ten \emph{mudrā}s been taught by Śiva Ādinātha. Each of them can bestow liberation for those who observe the rules.
\end{translation}
%</tr103>

%<*sc103>
%\begin{sources}[hp03_103]
%\end{sources}
%</sc103>

%<*ts103>
\begin{testimonia}[hp03_103]
\emph{Haṭharatnāvalī} 2.35, \emph{Yogacintāmaṇi} f.\,79v (\attr granthāntara)
\begin{variants}
tāsu yamināṃ YCM~] mukhyā syān HRĀ
\end{variants}

%\begin{versinnote}
%\tl{iti mudrā daśa proktā ādināthena śambhunā/\\+}
%\tl{ekaikā tāsu mukhyā syān mahāsiddhipradāyinī//\\!}
%\end{versinnote}
%
%\emph{Yogacintāmaṇi} f.\,79v (\attr to granthāntara)
%\begin{versinnote}
%\tl{iti mudrā nava proktā ādināthena śambhunā/\\+}
%\tl{ekaikā tāsu yamināṃ mahāsiddhipradāyinī//\\!}
%\end{versinnote}

\end{testimonia}
%</ts103>

%<*cm103>
\begin{philcomm}[hp03_103]
Manuscripts of the \textepsilon, \textzeta, \texteta, and \textpi\ groups have a different reading for the second line; `each \emph{mudrā} is capable of bringing about all powers' (\emph{karaṇe sarvasiddhīnāṃ ekaikāpi kṣamaiva sā}).% MD: karaṇe of ε may be the better reading: "capable of bringing about all siddhis"
\end{philcomm}
%</cm103>


%%%%%%%%%%
\subsection*{3.104}
%<*tr104>
\begin{translation}[hp03_104]
Without a king the earth is not resplendent, without the moon the night does not sparkle, without Rājayoga even the wonderful [practice of] \emph{mudrā} does not shine.%JM: get rid of the rājate-s? Or include all the rājayogas? I prefer getting rid. 
% ??JM: Also I think the api is not concessive and just = "and". And "shine" doesn't work; "succeed"? "reign supreme"?
\end{translation}
%</tr104>

%<*sc104>
%\begin{sources}[hp03_104]

%\end{sources}
%</sc104>

%<*ts104>
\begin{testimonia}[hp03_104]
\emph{Haṭharatnāvalī} 1.16
\mylb
%\begin{versinnote}
%\tl{rājayogaṃ vinā pṛthvī rājayogaṃ vinā niśā/\\+}
%\tl{rājayogaṃ vinā mudrā vicitrāpi na rājate//\\!}
%\end{versinnote}

%Yogasārasaṅgraha (p.\,59)
%\begin{versinnote}
%\tl{rājayogaṃ vinā pṛthvī rājayogaṃ vinā niśi/\\+}
%\tl{rājayogaṃ vinā mudrā vicitrāpi na rājate//\\!}
%\end{versinnote}
\end{testimonia}
%</ts104>

%<*cm104>
\begin{philcomm}[hp03_104]
The \emph{Yogaprakāśikā} (5.186) interprets the similes in this verse as we have translated them, `Just as without a king [and] moon, the earth and night do not shine...' (\emph{yathā mahīpālaṃ candramasaṃ vinā pṛthvīniśe na rājete}...). However, in \emph{Jyotsnā} 3.126, Brahmānanda interprets the earth (\emph{pṛthvī}) as \emph{āsana} because both are connected by the quality of steadiness (\emph{sthairyaguṇayogāt}), and the night (\emph{niśā}) as breath retention (\emph{kumbhaka}) because both are characterised by the absence of movement of people and wind (\emph{prāṇasañcārābhāvalakṣaṇaḥ}). Brahmānanda's interpretation seems somewhat far-fetched. 
\end{philcomm}
%</cm104>

\begin{metre}[hp03_104]
Anuṣṭubh (c: na-vipulā)
\end{metre}

%%%%%%%%%%
\subsection*{3.105}
%<*tr105>
\begin{translation}[hp03_105]
[The yogi] should carry out all breath practice with his mind engaged. The wise man must not let his attention wander.
\end{translation}
%</tr105>

%<*sc105>
%\begin{sources}[hp03_105]
%\end{sources}
%</sc105>

%<*ts105>
\begin{testimonia}[hp03_105]
\emph{Haṭhasaṅketacandrikā} f.\,92v (\attr HP)
\begin{variants}
 mārutasya vidhiṃ sarvaṃ~] mārutābhyasanaṃ kiṃ cin HSC\sep
 samabhyaset~] samācaret HSC
\end{variants}

%\begin{versinnote}
%\tl{mārutābhyasanaṃ kiṃ cin manoyuktaṃ samācaret/\\+}
%\tl{itaratra na kartavyā manovṛttir manīṣiṇā//\\!}
%\end{versinnote}
\end{testimonia}
%</ts105>

%<*cm105>
%\begin{philcomm}[hp03_105]
%\end{philcomm}
%</cm105>


%%%%%%%%%%
\subsection*{3.106}
%<*tr106>
\begin{translation}[hp03_106]
By means of a yogi's steady practice through postures, breath retentions and \emph{mudrā}s, his central channel, even though untraversed, becomes straight.
%By means of postures, breath retentions and \emph{mudrā}s, the central channel, even though untraversed, becomes straight through yogis' firm practice.%??JM: "Through yogis’ steady practice by means of postures, breath retentions and \emph{mudrā}s, their central channels, even though untraversed, become straight." JB: At the last meeting, we reorganised the translation and went with the singular. Jim, please take a look. 
\end{translation}
%</tr106>
%

%<*sc106>
%\begin{sources}[hp03_106]
%\end{sources}
%</sc106>

%<*ts106>
\begin{testimonia}[hp03_106]
\emph{Upāsanāsārasaṅgraha} p.\,36 % IFP Transcript T1095
\begin{variants}
 khilāpi~] iyaṃ tu USS
\end{variants}

%\begin{versinnote}
%\tl{iyaṃ tu madhyamā nāḍī dṛḍhābhyāsena yoginām /\\+}
%\tl{āsanaprāṇasaṃyāmamudrābhiḥ saralā bhavet //\\!}
%\end{versinnote}
\end{testimonia}
%</ts106>

%<*cm106>
\begin{philcomm}[hp03_106]
The reading of \emph{khilāpi} in the first verse quarter is unusual but well attested by the witnesses including the \textalpha\ group of manuscripts. In the \emph{Abhidhāna\-cintāmaṇi} (940), \emph{khila} is defined as something uncultivated such as field (\emph{kṣetrādyaprahataṃ khilam}) or, as the \emph{Amarapadavivṛti} (2.1.5) puts it, `not marked by a plough' (\emph{lāṅgalena na likhitam iti khilam}). In the context of \emph{Haṭhapradīpikā} 3.106, qualifying \emph{suṣumnā} with \emph{khila} implies that the central channel has yet to be cultivated (i.e.~traversed).


%khila seems odd (could it mean defective or unperfected here?), but the alternative readings are not helpful (except iyaṃ tu, which seems like a patch).
%Philipp: Apte: khila can mean blocked or a wedge
%Haru: khila - untraversed (aprahata), common in kośas.
%Group 6 has kuṭilā for khilāpi. Unlikely to be original but makes sense.

\end{philcomm}
%</cm106>


%%%%%%%%%%
\subsection*{3.107}
%<*tr107>
\begin{translation}[hp03_107]
And for those who are tireless in their dedication, the peerless \emph{mudrā} of Śiva bestows great perfection, like the ocean of Rājayoga.
\end{translation}
%</tr107>
%JM rewrite note [JM: do we need to? We have followed alpha and translated accordingly. The other readings are hard to make sense of.] to say we have followed alpha1 and 3 but aren't sure (i.e. accepting MD note here). MD: Wouldn't the alpha reading rājayoga-samudra-vat work? "Like the ocean of rājayoga (as the ocean yields jewels?), Rudrāṇī bestows great perfection on those who are tireless..." I am not fully convinced myself, but the present reading is not nice either. And I have a strong feeling that the verse consists of only one sentence.
%%
%Jü: (the alphaone option)
%With the slightly emended reading of \alphaone in the second Pāda (\emph{rājayogasamudbhavāt}) the
%verse could be translated as:

%When rājayoga (i.e.\ samādhi) arises in the practice of those that are tireless,
%the supreme rudrāṇī (i.e.\ śāmbhavī) mudrā invariably yields perfection.

%This interpretation implicitly assumes that Haṭhayoga, which includes the ten mudrās, evolves in
%the course of intensive practice into an experience of Rājayoga or samādhi. When that happens the
%practice to be followed from then on is śāmbhavīmudrā. This interpretation presupposes a model not
%unlike the one used in the later interpretation of the Pātañjalayoga that one specific stage, after
%a prolonged practice, automatically leads to the next stage. In his \emph{Jyotsnā} (2.12)
%Brahmānanda quotes the \emph{Skandapurāṇa} for this idea.\footnote{See Jürgen Hanneder: Brahmānanda
%on Yoga, p.\,58f.} The problem with this interpretation is, the HP being rather unspecific, there is  
%no corroborating evidence within the text itself. And there is verse 4.10, which makes the 
%śāmbhavī mudrā a basis for nādopāsana. With this Svātmārama departs from all older models of 
%interpretation of this practice, Tantric or Rājayogic. 


%<*sc107>
%\begin{sources}[hp03_107]
%\end{sources}
%</sc107>

%<*ts107>
%\begin{testimonia}[hp03_107]
%\end{testimonia}
%</ts107>

%<*cm107>
\begin{philcomm}[hp03_107]
%Consider J5 emendation rājayogasamudbhavā ('produced from Rājayoga')
%samudraka : rājayoga has a mudrā, is sealed, has the seal of approval,
We understand \emph{rudrāṇī mudrā} to be a synonym of \emph{śāmbhavī mudrā}, which is taught in the next chapter.

\end{philcomm}
%</cm107>
% α1:   upāsane vinidrāṇāṃ rājayogasamudravat/
%       rudrāṇī cāparā mudrā bhadrāṃ siddhiṃ prayacchati//
% IFP:  upāsane vinidrāṇāṃ rājayogasamudragāṃ/
%       rudrāṇī vā parāṃ mudrāṃ bhadrāṃ siddhiṃ prayacchati//
% G11:  abhyāseṣu vinidrāṇām anudrutasamādhiṣu/
%       rudrāṇīva parā mudrā bhadrāṃ siddhiṃ prayacchati//

%%%%%%%%%%
\subsection*{3.107*1}
%<*tr107-1>
\begin{translation}[hp03_107_1]
May [the yogi] who offers the traditional teaching of the \emph{mudrā}s be the guru, the master. He is none but the Lord himself.
\end{translation}
%</tr107-1>
%greyscaled

%<*sc107-1>
%\begin{sources}[hp03_107_1]
%\end{sources}
%</sc107-1>

%<*ts107-1>
\begin{testimonia}[hp03_107_1]
\emph{Upāsanāsārasaṅgraha} p.\,40 % IFP Transcript T1095
\begin{variants}
 evāstu guruḥ~] eva śrīguruḥ USS
\end{variants}

%\begin{versinnote}
%\tl{upadeśaṃ hi mudrāṇāṃ yo datte sāṃpradāyikam/\\+}
%\tl{sa eva śrīguruḥ svāmī sākṣād īśvara eva saḥ//\\!}
%\end{versinnote}
\end{testimonia}
%</ts107-1>

%<*cm107-1>
\begin{philcomm}[hp03_107_1]
Verses 3.107*1–2 have no known source and are absent in the \textalpha\ group. It is likely both were added to the original text as further praise of those practising the haṭhayogic \emph{mudrā}s.
\end{philcomm}
%</cm107-1>


%%%%%%%%%%
\subsection*{3.107*2}
%<*tr107-2>
\begin{translation}[hp03_107_2]
The yogi who has become intent on that [guru's] teaching and practises a \textit{mudrā} with a focused mind cheats death with the powers beginning with minimisation.
\end{translation}
%</tr107-2>
%greyscaled

%<*sc107-2>
%\begin{sources}[hp03_107_2]
%\end{sources}
%</sc107-2>

%<*ts107-2>
%\begin{testimonia}[hp03_107_2]
%\end{testimonia}
%</ts107-2>

%<*cm107-2>
\begin{philcomm}[hp03_107_2]
No version of this verse is entirely satisfactory. See the note on 3.107*1 for why it is in greyscale.
\end{philcomm}
%</cm107-2>

%===============
\subsection*{colophon}
%<*trcol>
\begin{translation}[hp03_col]
Thus ends the third chapter in the \emph{Haṭhapradīpikā} composed by the glorious lord among yogis Svātmārāma.    
\end{translation}
%</trcol>

\end{ekdosis}
\end{document}