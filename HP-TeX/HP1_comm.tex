\documentclass[10pt]{memoir}
\setstocksize{220mm}{155mm} 	        
\settrimmedsize{220mm}{155mm}{*}	
\settypeblocksize{170mm}{116mm}{*}	
\setlrmargins{18mm}{*}{*}
\setulmargins{*}{*}{1.2}
% \setlength{\headheight}{5pt}
\checkandfixthelayout[lines]
\linespread{1}
\setlength{\parskip}{0.3em}
\setlength\parindent{0pt}

\makepagestyle{HPed}
\makeoddhead{HPed}{\small{HP Transl. \& Comm.}}{}{\small{\today}}
\makeevenhead{HPed}{\small{HP Transl. \& Comm.}}{}{\small{\today}}
\makeoddfoot{HPed}{}{\small{\thepage}}{}
\makeevenfoot{HPed}{}{\small{\thepage}}{}

\usepackage[teiexport=tidy,poetry=verse]{ekdosis}
\usepackage{sanskrit-poetry,libertine,xcolor}
\usepackage[english]{babel}
\setlength{\vindent}{0pt}
\setvnum{}




%%%%%%%%%%%%%%%%%%%% THE  MSS         %%%%%%%%%%%%%%%%%%%%%%%%%%%

%%% Versions
\DeclareWitness{Vu}{\selectlanguage{english}Vulg}{Vulgate, i.e. Brahmānanda's version}[]           
\DeclareWitness{X}{\selectlanguage{english}X}{TenChapter Version, Jodhpur 02228 and 02225 (ed. Lonavla)}[]
\DeclareWitness{Six}{\selectlanguage{english}Ṣ}{SixChapterVersion, ``6ChapterHPms'', fragment of enlarged text, Jodhpur}[]
% Mss. in Geographical Groups
%%%% Varanasi mss (Sampūrṇānanda mss). V1 is Important
\DeclareWitness{V1}{\selectlanguage{english}V\textsubscript{1}}{Sampurnananda Library Sarasvati Bhavan 30109}[]
        \DeclareHand{V1ac}{V1}{\selectlanguage{english}V\rlap{\textsubscript{1}}\textsuperscript{ac}}[] % added by MD
        \DeclareHand{V1pc}{V1}{\selectlanguage{english}V\rlap{\textsubscript{1}}\textsuperscript{pc}}[] % added by MD
\DeclareWitness{V2}{\selectlanguage{english}V\textsubscript{2}}{Sampurnananda Library Sarasvati Bhavan 29869}[]
\DeclareWitness{V3}{\selectlanguage{english}V\textsubscript{3}}{Sampurnananda Library Sarasvati Bhavan 29899}[]
\DeclareWitness{V4}{\selectlanguage{english}V\textsubscript{4}}{Sampurnananda Library Sarasvati Bhavan 29937}[]
\DeclareWitness{V5}{\selectlanguage{english}V\textsubscript{5}}{Sampurnananda Library Sarasvati Bhavan 29938}[]
\DeclareWitness{V6}{\selectlanguage{english}V\textsubscript{6}}{Sampurnananda Library Sarasvati Bhavan 29991}[]
\DeclareWitness{V8}{\selectlanguage{english}V\textsubscript{8}}{Sampurnananda Library Sarasvati Bhavan 30014}[]
\DeclareWitness{V11}{\selectlanguage{english}V\textsubscript{11}}{Sampurnananda Library Sarasvati Bhavan 30029}[]
\DeclareWitness{V12}{\selectlanguage{english}V\textsubscript{12}}{Sampurnananda Library Sarasvati Bhavan 30030}[]
\DeclareWitness{V13}{\selectlanguage{english}V\textsubscript{13}}{Sampurnananda Library Sarasvati Bhavan 30031}[]
\DeclareWitness{V14}{\selectlanguage{english}V\textsubscript{14}}{Sampurnananda Library Sarasvati Bhavan 30050}[]
\DeclareWitness{V15}{\selectlanguage{english}V\textsubscript{15}}{Sampurnananda Library Sarasvati Bhavan 30051}[]
\DeclareWitness{V15pc}{\selectlanguage{english}V\rlap{\textsubscript{15}}\textsuperscript{pc}\space}{}[]
\DeclareWitness{V16}{\selectlanguage{english}V\textsubscript{16}}{Sampurnananda Library Sarasvati Bhavan 30052}[]
\DeclareWitness{V17}{\selectlanguage{english}V\textsubscript{17}}{Sampurnananda Library Sarasvati Bhavan 30053}[] % added by MD
\DeclareWitness{V16pc}{\selectlanguage{english}V\rlap{\textsubscript{16}}\textsuperscript{pc}\space}{}[]
\DeclareWitness{V18}{\selectlanguage{english}V\textsubscript{18}}{Sampurnananda Library Sarasvati Bhavan 30064}[]
\DeclareWitness{V19}{\selectlanguage{english}V\textsubscript{19}}{Sampurnananda Library Sarasvati Bhavan 30069}[]
\DeclareWitness{V21}{\selectlanguage{english}V\textsubscript{21}}{Sampurnananda Library Sarasvati Bhavan 30104}[]
\DeclareWitness{V22}{\selectlanguage{english}V\textsubscript{22}}{Sampurnananda Library Sarasvati Bhavan 30110}[]
\DeclareWitness{V25}{\selectlanguage{english}V\textsubscript{25}}{Sampurnananda Library Sarasvati Bhavan 30122}[]
\DeclareWitness{V26}{\selectlanguage{english}V\textsubscript{26}}{Sampurnananda Library Sarasvati Bhavan 30123}[]
\DeclareWitness{V28}{\selectlanguage{english}V\textsubscript{28}}{Sampurnananda Library Sarasvati Bhavan 30136}[]
\DeclareWitness{W2}{\selectlanguage{english}W\textsubscript{2}}{Wai ??}[]
\DeclareWitness{W4}{\selectlanguage{english}W\textsubscript{4}}{Wai 399-6171}[]

%%%%%%%%%%%%%%%%%%%%%%%%%%%%%%%%%
%%% Jammu & Kaschmir
\DeclareWitness{K1}{\selectlanguage{english}K\textsubscript{1}}{Raghunātha Temple Library 4383}[settlement=Jammu]
        \DeclareWitness{K1ac}{\selectlanguage{english}K\rlap{\textsubscript{1}}\textsuperscript{ac}\space}{}[]
        \DeclareWitness{K1pc}{\selectlanguage{english}K\rlap{\textsubscript{1}}\textsuperscript{pc}\space}{}[]
\DeclareWitness{K3}{\selectlanguage{english}K\textsubscript{3}}{Privat collection}
\DeclareWitness{L1}{\selectlanguage{english}L\textsubscript{1}}{SOAS RE 43454}[settlement=Jammu]
% More details? Catalogue number? L1 And C1 very close (and come from same region)
%%%%%%%%%%%%%%%%%%%%%%%%%%%%%%%%
% Jodhpur
% J10 is important
\DeclareWitness{J10}{\selectlanguage{english}J\textsubscript{10}}{MSPP Jodhpur 2230}[]
        \DeclareHand{J10ac}{J10}{\selectlanguage{english}J\rlap{\textsubscript{10}}\textsuperscript{ac}}[] % modified by MD
        \DeclareHand{J10pc}{J10}{\selectlanguage{english}J\rlap{\textsubscript{10}}\textsuperscript{pc}}[] % modified by MD
\DeclareWitness{J1}{\selectlanguage{english}J\textsubscript{1}}{Jodhpur 02231}[]
\DeclareWitness{J2}{\selectlanguage{english}J\textsubscript{2}}{Jodhpur 02232}[]   
\DeclareWitness{J3}{\selectlanguage{english}J\textsubscript{3}}{Jodhpur 02233}[]
\DeclareWitness{J4}{\selectlanguage{english}J\textsubscript{4}}{Jodhpur 02234}[]
        \DeclareWitness{J4ac}{\selectlanguage{english}J\rlap{\textsubscript{4}}\textsuperscript{ac}\space}{MSPP Jodhpur 02234}[]
        \DeclareWitness{J4pc}{\selectlanguage{english}J\rlap{\textsubscript{4}}\textsuperscript{pc}\space}{MSPP Jodhpur 02234}[]
\DeclareWitness{J5}{\selectlanguage{english}J\textsubscript{5}}{Jodhpur 02235}[]  % 4 chapters, 34 jpgs,   long colophon, missing lines in the beginning.
\DeclareWitness{J6}{\selectlanguage{english}J\textsubscript{6}}{Jodhpur 02237}[]  % 4 chapters, 41 jpgs
%\DeclareWitness{J6ac}{\selectlanguage{english}J\rlap{\textsubscript{6}}\textsubscript{ac}}{Jodhpur 02237}[]  % 4 chapters, 49 jpgs,   1st folio: idaṃ gulābarāyasya
% tulasīrāmaśarmmaṇaḥ putrasya pustakaṃ ...        End: iti śrīsahajānandasantānacintāmaṇisvātmārāmaviracitāyāṃ ..
% saṃvat 1802   (more consistent text)
%\DeclareWitness{J6pc}{\selectlanguage{english}J\rlap{\textsubscript{6}}\textsubscript{pc}}{Jodhpur 02237}[] 
\DeclareWitness{J7}{\selectlanguage{english}J\textsubscript{7}}{Jodhpur 02241}[]  % 4 chapters, 41 jpgs
\DeclareWitness{J8}{\selectlanguage{english}J\textsubscript{8}}{Jodhpur 23709}[]  % 4 chapters,  87 jpgs.   saṃvat 1724
\DeclareHand{J8ac}{J8}{\selectlanguage{english}J\rlap{\textsubscript{8}}\textsuperscript{ac}}[]  % changed by MD
\DeclareHand{J8pc}{J8}{\selectlanguage{english}J\rlap{\textsubscript{8}}\textsuperscript{pc}}[]  % changed by MD
\DeclareWitness{J9}{\selectlanguage{english}J\textsubscript{9}}{Jodhpur 02224}[]  %  fragment, 20 jpgs.
\DeclareWitness{J11}{\selectlanguage{english}J\textsubscript{11}}{Jodhpur 23532}[]
        \DeclareHand{J11ac}{J11}{\selectlanguage{english}J\rlap{\textsubscript{11}}\textsuperscript{ac}}[] % added by MD
        \DeclareHand{J11pc}{J11}{\selectlanguage{english}J\rlap{\textsubscript{11}}\textsuperscript{pc}}[] % added by MD
\DeclareWitness{J12}{\selectlanguage{english}J\textsubscript{12}}{Jodhpur 18552}[] 
\DeclareWitness{J13}{\selectlanguage{english}J\textsubscript{13}}{Jodhpur 02229}[]  %  5 chapters, 93 jpgs.
\DeclareWitness{J14}{\selectlanguage{english}J\textsubscript{14}}{Jodhpur 02239}[]  %  4 chapters
\DeclareWitness{J15}{\selectlanguage{english}J\textsubscript{15}}{Jodhpur 9732A}[]
\DeclareWitness{J16}{\selectlanguage{english}J\textsubscript{16}}{Jodhpur 9732B}[]
\DeclareWitness{J17}{\selectlanguage{english}J\textsubscript{17}}{Jodhpur 3013}[]
% Haṭhapradīpikā with (non-Sanskrit) Bhāṣya RORI Jodhpur ACC.NO.18552
%  Haṭhapradīpikā with (non-Sanskrit) commentary, RORI Alwar 952, 4 chapters,  colophon of the comm:
% iti śrīlāhorīmiśravrajabhūṣanaviracitāyāṃ bhāvārthadīpikāyāṃ caturthodhyāya ..    
%  Haṭhapradīpikā (5 chapter) MSPP Jodhpur ACC.NO.02229/

%%%%%%%%%%        Bodleian, Oxford
\DeclareWitness{B1}{\selectlanguage{english}B\textsubscript{1}}{Bodleian Library No. d.457(8)}[settlement=Oxford]
\DeclareWitness{B2}{\selectlanguage{english}B\textsubscript{2}}{Bodleian Library No. d.458(1)}[settlement=Oxford]
\DeclareWitness{B3}{\selectlanguage{english}B\textsubscript{3}}{Bodleian Library No. d.458(9)}[settlement=Oxford]

%%%%%%%%%%%   Chandigarh
\DeclareWitness{C1}{\selectlanguage{english}C\textsubscript{1}}{Lalchand M-2080}[]%L1 And C1 very close (and come from same region)
\DeclareWitness{C2}{\selectlanguage{english}C\textsubscript{2}}{Lalchand M-6065}[]
\DeclareWitness{C3}{\selectlanguage{english}C\textsubscript{3}}{Lalchand M-1293}[]
\DeclareWitness{C4}{\selectlanguage{english}C\textsubscript{4}}{Lalchand M-2081}[]
\DeclareWitness{C4ac}{\selectlanguage{english}C\rlap{\textsubscript{4}}\textsuperscript{ac}\space}{}[]
\DeclareWitness{C4pc}{\selectlanguage{english}C\rlap{\textsubscript{4}}\textsuperscript{pc}\space}{}[]
\DeclareWitness{C5}{\selectlanguage{english}C\textsubscript{5}}{Lalchand M-2082}[]%doesn't have chapter 1
\DeclareWitness{C6}{\selectlanguage{english}C\textsubscript{6}}{Lalchand M-2089}[]
\DeclareWitness{C7}{\selectlanguage{english}C\textsubscript{7}}{Lalchand M-6494}[]
\DeclareWitness{C8}{\selectlanguage{english}C\textsubscript{8}}{Lalchand M-2091}[]
        \DeclareHand{C8ac}{C8}{\selectlanguage{english}C\rlap{\textsubscript{8}}\textsuperscript{ac}}[]
        \DeclareHand{C8pc}{C8}{\selectlanguage{english}C\rlap{\textsubscript{8}}\textsuperscript{pc}}[]
\DeclareWitness{C9}{\selectlanguage{english}C\textsubscript{9}}{Lalchand M-4530}[]


% %%%%%%%%%%        Nepalese
\DeclareWitness{N1}{\selectlanguage{english}N\textsubscript{1}}{NGMPP A1400-2}[]
\DeclareWitness{N2}{\selectlanguage{english}N\textsubscript{2}}{NGMPP B 39-19}[]
\DeclareWitness{N3}{\selectlanguage{english}N\textsubscript{3}}{NGMPP B 62-20}[]
\DeclareWitness{N5}{\selectlanguage{english}N\textsubscript{5}}{NGMPP A60-15 + A61-1}[]
\DeclareWitness{N4}{\selectlanguage{english}N\textsubscript{4}}{NGMPP A61-2}[]
\DeclareWitness{N6}{\selectlanguage{english}N\textsubscript{6}}{NGMPP A61-6}[]
\DeclareWitness{N9}{\selectlanguage{english}N\textsubscript{9}}{NGMPP A62-33}[]
\DeclareWitness{N10}{\selectlanguage{english}N\textsubscript{10}}{NGMPP A62-37}[]
\DeclareWitness{N11}{\selectlanguage{english}N\textsubscript{11}}{NGMPP A63-15}[]
\DeclareWitness{N12}{\selectlanguage{english}N\textsubscript{12}}{NGMPP A939-19}[]
\DeclareWitness{N13}{\selectlanguage{english}N\textsubscript{13}}{NGMPP A1378-18}[]
\DeclareWitness{N16}{\selectlanguage{english}N\textsubscript{16}}{NGMPP B39-20}[]
\DeclareWitness{N17}{\selectlanguage{english}N\textsubscript{17}}{NGMPP B 111-10}[]
\DeclareWitness{N18}{\selectlanguage{english}N\textsubscript{18}}{NGMPP E 929-3}[]
\DeclareWitness{N19}{\selectlanguage{english}N\textsubscript{19}}{NGMPP E-1528-1 / E-1527-7(4)}[]
\DeclareWitness{N20}{\selectlanguage{english}N\textsubscript{20}}{NGMPP E 2037-13 }[]
\DeclareWitness{N21}{\selectlanguage{english}N\textsubscript{21}}{NGMPP E 2097-31}[]
\DeclareWitness{N22}{\selectlanguage{english}N\textsubscript{22}}{NGMPP G 4-4}[]
\DeclareWitness{N23}{\selectlanguage{english}N\textsubscript{23}}{NGMPP G 25-2}[]
        \DeclareHand{N23ac}{N23}{\selectlanguage{english}N\rlap{\textsubscript{23}}\textsuperscript{ac}}[] % added by MD
        \DeclareHand{N23pc}{N23}{\selectlanguage{english}N\rlap{\textsubscript{23}}\textsuperscript{pc}}[] % added by MD
\DeclareWitness{N24}{\selectlanguage{english}N\textsubscript{24}}{NGMPP G 190-16}[]
\DeclareWitness{N24ac}{\selectlanguage{english}N\rlap{\textsubscript{24}}\textsuperscript{ac}\space}{}[]
\DeclareWitness{N24pc}{\selectlanguage{english}N\rlap{\textsubscript{24}}\textsuperscript{pc}\space}{}[]
\DeclareWitness{N26}{\selectlanguage{english}N\textsubscript{26}}{NGMPP T 24-3}[]

% %%%%%%%%%%        Pune

\DeclareWitness{P1}{\selectlanguage{english}P\textsubscript{1}}{Ānandāśrama S16-3-21}[]
\DeclareWitness{P2}{\selectlanguage{english}P\textsubscript{2}}{Ānandāśrama S16-2-20}[]
\DeclareWitness{P3}{\selectlanguage{english}P\textsubscript{3}}{BISM (79) 314}[]
\DeclareWitness{P4}{\selectlanguage{english}P\textsubscript{4}}{BISM (91) 191}[]
\DeclareWitness{P5}{\selectlanguage{english}P\textsubscript{5}}{BISM (29) 5790}[]
\DeclareWitness{P6}{\selectlanguage{english}P\textsubscript{6}}{BORI 263/1879-80}[]
\DeclareWitness{P7}{\selectlanguage{english}P\textsubscript{7}}{BORI 665/1883-84}[]
\DeclareWitness{P8}{\selectlanguage{english}P\textsubscript{8}}{BORI 316/1895-98}[]
\DeclareWitness{P9}{\selectlanguage{english}P\textsubscript{9}}{BORI 733-1891-95}[]
\DeclareWitness{P10}{\selectlanguage{english}P\textsubscript{10}}{BORI 222-1884-86}[]
\DeclareWitness{P11}{\selectlanguage{english}P\textsubscript{11}}{BORI 221-1882–83}[]
\DeclareWitness{P12}{\selectlanguage{english}P\textsubscript{12}}{Ānandāśrama S16-3-24}[]
\DeclareWitness{P13}{\selectlanguage{english}P\textsubscript{13}}{Ānandāśrama S16-2-22}[]
\DeclareWitness{P14}{\selectlanguage{english}P\textsubscript{14}}{Ānandāśrama S16-3-23}[]
\DeclareWitness{P15}{\selectlanguage{english}P\textsubscript{15}}{BISM (64) 919}[]
\DeclareWitness{P16}{\selectlanguage{english}P\textsubscript{16}}{BISM (64) 1115}[]
\DeclareWitness{P17}{\selectlanguage{english}P\textsubscript{17}}{BISM 620/1886-92}[]
\DeclareWitness{P18}{\selectlanguage{english}P\textsubscript{18}}{BORI 615/1887-91}[]
\DeclareWitness{P19}{\selectlanguage{english}P\textsubscript{19}}{BISM 46-39}[]
\DeclareWitness{P20}{\selectlanguage{english}P\textsubscript{20}}{BISM 39-273}[]
\DeclareWitness{P21}{\selectlanguage{english}P\textsubscript{21}}{BISM 37-743}[]
\DeclareWitness{P22}{\selectlanguage{english}P\textsubscript{22}}{BISM 37-729}[]
\DeclareWitness{P23}{\selectlanguage{english}P\textsubscript{23}}{BISM 33-60}[]
\DeclareWitness{P24}{\selectlanguage{english}P\textsubscript{24}}{BISM 29-5790}[]% =P5!
\DeclareWitness{P25}{\selectlanguage{english}P\textsubscript{25}}{BISM 29-3657}[]
\DeclareWitness{P26}{\selectlanguage{english}P\textsubscript{26}}{BISM 25-281}[]
\DeclareWitness{P27}{\selectlanguage{english}P\textsubscript{27}}{BISM 7-489}[]
\DeclareWitness{P28}{\selectlanguage{english}P\textsubscript{28}}{BORI 399-1895-1902}[]

%%%%%   Mysore
\DeclareWitness{M1}{\selectlanguage{english}M\textsubscript{1}}{P-5682/4}[]
%%%%%   Tübingen
\DeclareWitness{Tue}{\selectlanguage{english}Tü}{Ma I 339}[]
%%%%%%%%%%
\DeclareWitness{YC}{\selectlanguage{english}YC}{Yogacintāmaṇi}[]
\DeclareWitness{ceteri}{\selectlanguage{english}cett.}{ceteri}[]

%%%%%%%%%% Mss with Commentary
\DeclareWitness{A1}{\selectlanguage{english}A\textsubscript{1}}{Alwar 952}[]

\DeclareWitness{Jyo}{\selectlanguage{english}J\textsubscript{yo}}{Brahmānanda's version}[]

%%%%%%%%%%%%%%%%%%%%%%%%%%%%%%%%%%%%%%%%%%%
%List of all Sigla:
%A1,B1,B2,B3,C1,C2,C3,C4,C6,C7,C8,C9,J1,J2,J3,J4,J10,J13,J14,J15,J17,L1,M1,N3,N5,N6,N9,N10,N11,N12,N13,N16,N17,N19,N20,N21,N22,N23,N24,Tü,V1,V2,V3,V4,V5,V6,V8,V11,V19,V22,V26,Vu
%%%%%%%%%%%%%%%%%%%%%%%%%%%%%%%%%%%%%%%%%%%

\DeclareWitness{G4}{\selectlanguage{english}G\textsubscript{4}}{GOML D18885 (Bundle SD5051)}[]
\DeclareWitness{G5}{\selectlanguage{english}G\textsubscript{5}}{GOML R3841/ SR2190}[]
\DeclareWitness{G7}{\selectlanguage{english}G\textsubscript{7}}{GOML D4394}[]

\DeclareWitness{Ko}{\selectlanguage{english}K\textsubscript{o}}{Koba, Gujarat 55626}[]

%
%%%%%                   Abbreviation for the printed apparatus,        xml interface needed
%%%%%                   (synonyms in same line)

% Macro for Editing Abbrevs.
%\def\om{\textrm{\footnotesize \textit{omitted in}\ }} %prints om. for omitted in apparatus
%\def\korr{\textrm{\footnotesize \textit{em.}\ }} %prints em. for emended in apparatus
%\def\conj{\textrm{\footnotesize \textit{conj.}\ }} %prints conj. for conjectured in apparatus


\def\eyeskip{\textrm{{ab.\,oc. }}}   
\def\aberratio{\textrm{{ab.\,oc. }}}
\def\ad{\textrm{{ad}}}   
\def\add{\textrm{{add.\ }}}
\def\ann{\textrm{{ann.\ }}}
\def\ante{\textrm{{ante }}}
\def\post{\textrm{{post }}}
%\def\ceteri{cett.\,}             % for simplifying the apparatus in print                  
\def\codd{\textrm{{codd.\ }}}   %  the same
\def\conj{\textrm{{coni.\ }}}  
\def\coni{\textrm{{coni.\ }}}
\def\contin{\textrm{{contin.\ }}}
\def\corr{\textrm{{corr.\ }}}
\def\del{\textrm{{del.\ }}}
\def\dub{\textrm{{ dub.\ }}}
\def\emend{\textrm{{emend.\ }}}
\def\expl{\textrm{{explic.\ }}}   
\def\explicat{\textrm{{explic.\ }}}
\def\fol{\textrm{{fol.\ }}}         
\def\foll{\textrm{{foll.\ }}}
\def\gloss{\textrm{{glossa ad }}}
\def\ins{\textrm{{ins.\ }}}          \def\inseruit{\textrm{{ins.\ }}}
\def\im{{\kern-.7pt\lower-1ex\hbox{\textrm{\tiny{\emph{i.m.}}}\kern0pt}}}
\def\inmargine{{\kern-.7pt\lower-.7ex\hbox{\textrm{\tiny{\emph{i.m.}}}\kern0pt}}}
\def\intextu{{\kern-.7pt\lower-.95ex\hbox{\textrm{\tiny{\emph{i.t.}}}\kern0pt}}}%\textrm{\scriptsize{i.t.\ }}}               
\def\indist{\textrm{{indis.\ }}}          \def\indis{\textrm{{indis.\ }}}
\def\iteravit{\textrm{{iter.\ }}}          \def\iter{\textrm{{iter.\ }}}  
\def\lectio{\textrm{{lect.\ }}}             \def\lec{\textrm{{lect.\ }}}
\def\leginequit{\textrm{{l.n. }}}         \def\legn{\textrm{{l.n. }}}         \def\illeg{\textrm{{l.n. }}}
\def\om{\textrm{{om. }}}
\def\primman{\textrm{{pr.m.}}}
\def\prob{\textrm{{prob.}}}
\def\rep{\textrm{{repetitio }}}
% \def\secundamanu{\textrm{\scriptsize{s.m.}}}
% \def\secm{{\kern-.6pt\lower-.91ex\hbox{\textrm{\tiny{\emph{s.m.}}}\kern0pt}}}%   \textrm{\scriptsize{s.m.}}}
\def\sequentia{\textrm{{seq.\,inv.\ }}}         \def\seqinv{\textrm{{seq.\,inv.\ }}} \def\order{\textrm{{seq.\,inv.\ }}}
\def\supralineam{{\kern-.7pt\lower-.91ex\hbox{\textrm{\tiny{\emph{s.l.}}}\kern0pt}}} %\textrm{\scriptsize{s.l.}}}
\def\interlineam{{\kern-.7pt\lower-.91ex\hbox{\textrm{\tiny{\emph{s.l.}}}\kern0pt}}}   %\textrm{\scriptsize{s.l.}}}
\def\vl{\textrm{v.l.}}   \def\varlec{\textrm{v.l.}} \def\varialectio{\textrm{v.l.}}
\def\vide{\textrm{{cf.\ }}}           \def\cf{\textrm{{cf.\ }}}
\def\videtur{\textrm{{vid.\,ut}}}
\def\crux{\textup{[\ldots]} }
\def\cruxx{\textup{[\ldots]}}
\def\unm{\textit{unm.}}        % unmetrical
%%%%%%%%%%%%%%%%%%%%%%%%%%%%%%%%%%%%



%%% Local Variables:
%%% mode: latex
%%% TeX-master: t
%%% End:

% addition 2023-12-11 MD:
\TeXtoTEIPat{\begin {metre}[#1]}{<note type="metre" target="##1">}
\TeXtoTEIPat{\end {metre}}{</note>}
\TeXtoTEIPat{\texttheta}{θ}

% change 2023-12-05 mm
\TeXtoTEI{teimute}{} 

% changes/additions 2023-11-27 MM:
\TeXtoTEIPat{\medialink {#1}{#2}}{<ref target="resources/#2">#1</ref>}

% changes/additions 2023-10-25 MM:
% new Sigla
\TeXtoTEIPat{\textAlpha}{Α}
\TeXtoTEIPat{\textalpha}{α}
\TeXtoTEIPat{\textBeta}{Β}
\TeXtoTEIPat{\textbeta}{β}
\TeXtoTEIPat{\textGamma}{Γ}
\TeXtoTEIPat{\textgamma}{γ}
\TeXtoTEIPat{\textDelta}{Δ}
\TeXtoTEIPat{\textdelta}{δ}
\TeXtoTEIPat{\textEpsilon}{Ε}
\TeXtoTEIPat{\textepsilon}{ε}
\TeXtoTEIPat{\textEta}{Η}
\TeXtoTEIPat{\texteta}{η}
\TeXtoTEIPat{\textChi}{Χ}
\TeXtoTEIPat{\textchi}{χ}
\TeXtoTEIPat{\textOmega}{Ω}
\TeXtoTEIPat{\textomega}{ω}

%new environments
\TeXtoTEIPat{\begin {postmula}[#1]}{<note type="postmula" target="##1">}
  \TeXtoTEIPat{\end {postmula}}{</note>}
\TeXtoTEIPat{\begin {altava}[#1]}{<div type="altrec"><note type="avataranika" target="##1">} %%% changed 2023-12-05 mm
  \TeXtoTEIPat{\end {altava}}{</note></div>} %%% changed 2023-12-05 mm
\TeXtoTEIPat{\sgwit {#1}}{<note type="inlineref"><ref>#1</ref></note>}

% changes/additions 2023-10-12 MM:
\TeXtoTEIPat{\\.}{}

% changes/additions 2023-08-15 MD:
\TeXtoTEIPat{\lineom {#1}{#2}}{<note type="omission">#1 omitted in <ref>#2</ref></note>}
\TeXtoTEI{graus}{hi}[rend="grey"]
\TeXtoTEIPat{\startgray}{} %%% changed 2023-12-05 mm
\TeXtoTEIPat{\endgray}{} %%% changed 2023-12-05 mm



% additions/changes 2023-06-05 mm:
%\TeXtoTEIPat{\lineom {#1}}{<note type="omission">Line omitted in <ref>#1</ref></note>}
\TeXtoTEIPat{\NotIn {#1}}{<note type="omission">Stanza omitted in <ref>#1</ref></note>}

% additions 2023-04-16 MD:
\TeXtoTEIPat{\,}{ }

% additions 2023-04-13 mm:
\TeXtoTEIPat{\begin {versinnote}}{<lg>}
  \TeXtoTEIPat{\end {versinnote}}{</lg>}

% additions 2023-04-05 MD:
\TeXtoTEIPat{\begin {testimonia}[#1]}{<note type="testimonia" target="##1">}
  \TeXtoTEIPat{\end {testimonia}}{</note>}
\TeXtoTEI{devnote}{s}[xml:lang="sa-deva"]

% app in philcomm und testimonia %%% added MM 2023-12-02
\TeXtoTEI{var}{note}[type="appinnote"]


\TeXtoTEI{anm}{note}[type="memo"] %% change 2023-04-16 MD
\TeXtoTEI{Anm}{note}[type="memo"] %% change 2023-12-05 MM
\TeXtoTEIPat{\startverse}{} %%% marked for change 2023-04-13 mm
\TeXtoTEIPat{\endverse}{} %%% marked for change 2023-04-13 mm
\TeXtoTEIPat{\newpage}{}
\TeXtoTEIPat{\marma}{}
\TeXtoTEIPat{\marmas}{}
\TeXtoTEIPat{\vin}{} % added by MD 2023-11-14

%%% modify environments and commands
%%% TEI mapping
% additions/changes 2022-06-07 mm:
\TeXtoTEI{grau}{hi}[rend="grey"]
\TeXtoTEIPat{ \& }{ &amp; }

% additions/changes 2022-06-01 mm:
\TeXtoTEI{skp}{seg}[type="deva-ignore"]
\TeXtoTEI{skm}{seg}[type="ltn-ignore"]

\TeXtoTEIPat{\rlap {#1}}{#1}

% additions/changes 2022-04-06 mm:
%\TeXtoTEI{sgwit}{ref}
\TeXtoTEI{textdev}{s}[xml:lang="sa-deva"]
\TeXtoTEIPat{\begin {col}[#1]}{<div type="colophon" xml:id="#1"><p>}
  \TeXtoTEIPat{\end {col}}{</p></div>}
\TeXtoTEIPat{\begin {ava}[#1]}{<note type="avataranika" target="##1">}
  \TeXtoTEIPat{\end {ava}}{</note>}
												   
\TeXtoTEIPat{\outdent}{}
\TeXtoTEIPat{\startaltrecension}{} %%% changed 2023-12-05 mm
\TeXtoTEIPat{\endaltrecension}{} %%% changed 2023-12-05 mm
\TeXtoTEIPat{\startaltnormal}{} % added by MD 2023-11-14 %%% changed 2023-12-05 mm
\TeXtoTEIPat{\endaltnormal}{} % added by MD 2023-11-14 %%% changed 2023-12-05 mm
\TeXtoTEIPat{\begin {alttlg}[#1]}{<div type="altrec"><lg xml:id="#1">}
  \TeXtoTEIPat{\end {alttlg}}{</lg></div>}



% additions/changes 2022-03-12 mm:
\TeXtoTEIPat{\begin {tlg}[#1]}{<lg xml:id="#1">}
  \TeXtoTEIPat{\end {tlg}}{</lg>}

\TeXtoTEIPat{\begin {translation}[#1]}{<note type="translation" target="##1">}
  \TeXtoTEIPat{\end {translation}}{</note>}
\TeXtoTEIPat{\begin {philcomm}[#1]}{<note type="philcomm" target="##1">}
  \TeXtoTEIPat{\end {philcomm}}{</note>}
\TeXtoTEIPat{\begin {sources}[#1]}{<note type="sources" target="##1">}
  \TeXtoTEIPat{\end {sources}}{</note>}


\TeXtoTEIPat{\begin {marma}[#1]}{<note type="marma" target="##1">}
  \TeXtoTEIPat{\end {marma}}{</note>}

\TeXtoTEIPat{\begin {jyotsna}[#1]}{<note type="jyotsna" target="##1">}
  \TeXtoTEIPat{\end {jyotsna}}{</note>}

\EnvtoTEI{description}{list}
\EnvtoTEI{itemize}{list}
\TeXtoTEIPat{\item [#1]}{<label>#1</label>\item}

\TeXtoTEI{tl}{l}
\TeXtoTEI{myfn}{note}[type="myfn"]
\TeXtoTEIPat{\getsiglum {#1}}{<ref target="##1"/>}

\TeXtoTEI{SetLineation}{}
\TeXtoTEI{noindent}{}
\TeXtoTEI{subsection*}{}

\TeXtoTEI{rlap}{}

% end additions/changes
% \TeXtoTEIPat{\skp {#1}}{#1}
% \TeXtoTEIPat{\skm {#1}}{}

\TeXtoTEIPat{\begin {prose}}{<p>}
  \TeXtoTEIPat{\end {prose}}{</p>}

\TeXtoTEIPat{\begin {tlate}}{<p>}
  \TeXtoTEIPat{\end {tlate}}{</p>}

\TeXtoTEI{emph}{hi}
\TeXtoTEI{bigskip}{}
% \TeXtoTEI{/}{|}
\TeXtoTEI{tl}{l}
\TeXtoTEIPat{english}{}
%\TeXtoTEIPat{-}{ } %% change 2023-04-16 MD
%\TeXtoTEIPat{°}{} %% change 2023-04-16 MD
\TeXtoTEIPat{\textcolor {#1}{#2}}{<hi rend="#1">#2</hi>}

% \TeXtoTEIPat{\eyeskip}{}
% \TeXtoTEIPat{\aberratio}{}
% \TeXtoTEIPat{\ad}{}
\TeXtoTEIPat{\add}{<hi rend="italic">add.</hi>} %% change 2023-04-16 MD
% \TeXtoTEIPat{\ann}{}
\TeXtoTEIPat{\ante}{<hi rend="italic">ante</hi> } %% change 2023-04-16 MD
\TeXtoTEIPat{\post}{<hi rend="italic">post</hi> } %% change 2023-04-16 MD
% \TeXtoTEIPat{\codd}{}
% \TeXtoTEIPat{\conj }{}
% \TeXtoTEIPat{\contin}{}
% \TeXtoTEIPat{\corr}{}
% \TeXtoTEIPat{\del}{}
% \TeXtoTEIPat{\dub}{}
% \TeXtoTEIPat{\emend }{}
% \TeXtoTEIPat{\expl}{}
% \TeXtoTEIPat{\ȩxplicat}{}
% \TeXtoTEIPat{\fol}{}
% \TeXtoTEIPat{\gloss}{}
% \TeXtoTEIPat{\ins}{}
% \TeXtoTEIPat{\im}{}
% \TeXtoTEIPat{\inmargine}{}
% \TeXtoTEIPat{\intextu}{}
% \TeXtoTEIPat{\indist}{}
% \TeXtoTEIPat{\iteravit}{}
% \TeXtoTEIPat{\lectio}{}
% \TeXtoTEIPat{\leginequit}{}
% \TeXtoTEIPat{\legn}{}
% \TeXtoTEIPat{\illeg}{<hi rend="italic">illeg.</hi>}
\TeXtoTEIPat{\illeg}{<gap reason="illeg."/>} %%% change 2023-04-11 mm
% \TeXtoTEIPat{\om}{<hi rend="italic">om.</hi>}
\TeXtoTEIPat{\om}{<gap reason="om."/>} %%% change 2023-04-11 mm
% \TeXtoTEIPat{\primman}{}
% \TeXtoTEIPat{\prob}{}
% \TeXtoTEIPat{\rep}{}
% \TeXtoTEIPat{\sequentia}{}
% \TeXtoTEIPat{\supralineam}{}
% \TeXtoTEIPat{\interlineam}{}
\TeXtoTEIPat{\vl}{<hi rend="italic">v.l.</hi>}
% \TeXtoTEIPat{\vide}{}
% \TeXtoTEIPat{\videtur}{}
% \TeXtoTEIPat{\crux}{}
% \TeXtoTEIPat{\cruxxx}{}
\TeXtoTEIPat{\unm}{<hi rend="italic">unm.</hi>}


% List of Scholars
\DeclareScholar{nos}{nos}[
forename=HPP,
surname=Team]


% Nullify \selectlanguage in TEI as it has been used in
% \DeclareWitness but should be ignored in TEI.
\TeXtoTEI{selectlanguage}{}


\SetTEIxmlExport{autopar=false}

%%%%%%%%%%%

\SetTEIxmlExport{autopar=false}
\NewDocumentEnvironment{translation}{O{}}{\textcolor{blue}{\textbf{Translation:}}}{}
\NewDocumentEnvironment{philcomm}{O{}}{
	\textcolor{blue}{\textbf{Commentary:}}}{}
\NewDocumentEnvironment{metre}{O{}}{
	\textcolor{blue}{\textbf{Metre:}}}{} % added MD 2023-12-11
\NewDocumentEnvironment{sources}{O{}}{
	\textcolor{blue}{\textbf{Sources:}}\linebreak}{}
\NewDocumentEnvironment{testimonia}{O{}}{
	\textcolor{blue}{\textbf{Testimonia:}}\linebreak}{}
\NewDocumentEnvironment{versinnote}{O{}}{\begin{ekdverse}}{\end{ekdverse}}
%\newcommand{\var}[1]{\footnotesize\textup{#1}}
\newcommand{\medialink}[2]{\textcolor{green}{\underline{#1}}}
%\TeXtoTEIPat{\medialink {#1}{#2}}{<ref target="/images/#2">#1</ref>}

\NewDocumentCommand{\tl}{m}{#1}

\def\vl{\textit{v.l.}}
\def\var#1{{\footnotesize #1}}
\def\sl#1{\emph{#1}}

%%%%%%%%%%%%

\usepackage{textgreek}

\newcommand{\alphaOne}{\textalpha\textsubscript{1}}% N3
\newcommand{\alphaTwo}{\textalpha\textsubscript{2}}% J5
\newcommand{\alphaThree}{\textalpha\textsubscript{3}}% G4
\newcommand{\betaOne}{\textbeta\textsubscript{1}}% P11
\newcommand{\betaTwo}{\textbeta\textsubscript{2}}% C6
\newcommand{\betaOmega}{\textbeta\textsubscript{\textomega}}% V3
\newcommand{\gammaOne}{\textgamma\textsubscript{1}}% N23
\newcommand{\gammaTwo}{\textgamma\textsubscript{2}}% J7
\newcommand{\deltaOne}{\textdelta\textsubscript{1}}% V19
\newcommand{\deltaTwo}{\textdelta\textsubscript{2}}% K3
\newcommand{\deltaThree}{\textdelta\textsubscript{3}}% C7
\newcommand{\deltaOmega}{\textdelta\textsubscript{\textomega}}% J6
\newcommand{\epsilonOne}{\textepsilon\textsubscript{1}}% P15
\newcommand{\epsilonTwo}{\textepsilon\textsubscript{2}}% N19
\newcommand{\epsilonThree}{\textepsilon\textsubscript{3}}% V15
\newcommand{\epsilonFour}{\textepsilon\textsubscript{4}}% J11
\newcommand{\epsilonOmega}{\textepsilon\textsubscript{\textomega}}% N26
\newcommand{\etaOne}{\texteta\textsubscript{1}}% V1
\newcommand{\etaTwo}{\texteta\textsubscript{2}}% J10
\newcommand{\etaOmega}{\texteta\textsubscript{\textomega}}% N9

%%%%%%%%%%%%%%

\babelhyphenation{%
	Dattā-treya-yoga-śāstra
	Gorakṣa-śataka
	Haṭha-pra-dī-pikā
	Hātha-ratnā-valī
	Svātmā-rāma
	Śiva-saṃhitā
	Vasiṣṭha-saṃhitā
	Viveka-mārtaṇḍa
	Yukta-bhava-deva
	Yoga-cintā-maṇi
	Yoga-yājña-valkya}

%\def\Sone{{\scriptsize{\'S\lower.4ex\hbox{\textrm{\tiny{1}}}}}}
%\def\Soneac{{\scriptsize{\'S\lower.4ex\hbox{\textrm{\tiny{1}}}\kern-2.2pt\lower-1ex\hbox{\textrm{\tiny{ac}}\kern1pt}}}}
%\def\Sonepc{{\scriptsize{\'S\lower.4ex\hbox{\textrm{\tiny{1}}}\kern-2.2pt\lower-1ex\hbox{\textrm{\tiny{pc}}\kern.7pt}}}}
\begin{document}
\pagestyle{HPed}
\begin{ekdosis}
\SetLineation{lineation = none,}

%\chapter*{Translation \& philological commentary}

%%%%%%%%%%
\subsection*{1.1}
%<*tr1>
\begin{translation}[hp01_001]
Homage to the glorious Ādinātha who taught the science of Haṭhayoga which is like a splendid stairway for one who wants to climb to the lofty royal terrace.%JM: would much prefer "science" to "system" (vidyā doesn't mean system). I changed "ladder" to "stairway" (a bit more poetic). I'd also prefer "which is like a splendid stairway for one who wants to climb to the lofty royal terrace."
%JB: I much prefer Jim's alternative translation (i.e.~"which is like...). But I think we should avoid translating vidyā as `science' because it implies that there was some antecedent to the scientific method in premodern yoga traditions. Also, in light of the current politics around yoga in India, "science of Haṭha" fosters the rather absurd view that ancient Indian knowledge systems anticipated scientific discoveries of the 20th century.  JM: But "system" is wrong. JB: doctrine is better than science.

\end{translation}
%</tr1>

%<*ts1>
\begin{testimonia}[hp01_001]
\emph{Yogasārasaṅgraha} p.\,54, \emph{Gheraṇḍasaṃhitā} 1.1
\begin{variants}
śrīādināthāya namo ’stu tasmai~] sadādināthāya namo 'stu tubhyaṃ YSS, ādīśvarāya praṇamāmi tasmai GhS
\end{variants}

Cf.~\emph{Bhāvanāpuruṣottama} pp.\,99--100 
\begin{versinnote}
\tl{rājayogābhidhānasaudhādhirohaṇādhirohiṇī haṭhavidyā//\\!}
\end{versinnote}
%JB I've added this. Please check

%\emph{Yogasārasaṅgraha} p.\,54,
%
%\begin{versinnote}
%\tl{sadādināthāya namo 'stu tubhyaṃ\\+}
%\tl{yenopadiṣṭā haṭhayogavidyā/\\+}
%\tl{virājate pronnatarājayogam\\+} 
%\tl{āroḍhum icchor adhirohiṇīva//\\!}
%\end{versinnote}

%\emph{Gheraṇḍasaṃhitā} 1.1
%
%\begin{versinnote}
%\tl{ādīśvarāya praṇamāmi tasmai\\+} 
%\tl{yenopadiṣṭā haṭhayogavidyā/\\+}
%\tl{virājate pronnatarājayogam\\+} 
%\tl{āroḍhum icchor adhirohiṇīva//\\!}
%\end{versinnote}

\end{testimonia}
%</ts1>

%<*cm1>
\begin{philcomm}[hp01_001]

In his commentary on \emph{Haṭhapradīpikā} 1.1 Brahmānanda first states that the author Svātmārāma starts appropriately with a \emph{maṅgala}, a verse of adoration addressing Ādinātha, i.e.\ Śiva, but in accordance with his non-sectarian approach leaves room for a second interpretation of the word as Viṣṇu. A sentence later this is contradicted by his explanantion that Ādinātha, who first taught Yoga, taught it to Pārvatī, which limits the scope to Śiva. \lb

%add comment about yoga replacing mārga and saudha here and in the next two verses.
The reading \emph{rājasaudham} in the third verse quarter is well attested by the manuscripts, including \alphaOne. However, the most common reading in manuscripts on the lower branches of the stemma is \emph{rājayogam}, which was accepted by Brahmānanda in \emph{Jyotsnā} 1.1 (see below). It appears that some scribes have made a concerted effort to replace words like \emph{saudha}, \emph{vidyā} and \emph{mārga} with \emph{yoga} in the opening verses of the text (see also 1.2d and 1.3b), even at the expense of the poetic imagery. \lb

In light of the variants, which have led to our critical edition, Brahmānanda's choices and interpretation of the second half cannot be upheld. This is Brahmānanda's version and the relevant portion of his commentary:

\begin{versinnote}
\tl{śrīādināthāya namo'stu tasmai\\+} 
\tl{yenopadiṣṭā haṭhayogavidyā/\\+}
\tl{vibhrājate pronnatarājayogam\\+} 
\tl{āroḍhum icchor adhirohiṇīva// \texteng{\footnotesize 1.1}//\\!}
\end{versinnote}

Our choice of \emph{virājate} in the third verse quarter reflects the relevant manuscripts and rhymes with the following \emph{rāja-}. The difference in meaning is negligable.\lb

The main problem in Brahmānanda's interpretation is his choice of \emph{-rājayogaṃ} over \emph{-rāja\-sau\-dhaṃ}, which he explains as follows:

\begin{versinnote}
\tl{rājayogaś ca sarvavṛttinirodhalakṣaṇo'samprajñātayogaḥ. tam icchor mumukṣor adhirohiṇīva adhi\-ruhyate'nayety adhi\skx{-}{}rohiṇī niḥśreṇīva vibhrājate viśeṣeṇa bhrājate śobhate. yathā pronnatasaudham āroḍhum icchor adhirohiṇy anāyāsena saudha\-prāpi\-kā bhavati evaṃ haṭhapradīpikāpi pronnatarājayogam āroḍhum icchor anā\-yāsena rājayogaprāpikā bhavatīti upamālaṅkāraḥ.  indravajrākhyaṃ vṛttam.\\!}
\end{versinnote}
\closer
\begin{quote}
Rājayoga is the yoga without cognition (\emph{asaṃprajñāta}) defined [by Patañjali] as the stopping of all movements of the mind. To a liberation seeker desiring this, [the science of Haṭhayoga] shines like a ladder. Just as a ladder leads someone desiring to ascend to a lofty terrace easily to the terrace, in the same way also the \emph{Haṭhapradīpikā} leads someone desiring the lofty Rājayoga easily to Rājayoga. [In this verse] the trope is a comparison. The metre is Indravajrā.

% Rājayoga is the yoga without cognition,\footnote{This must refer to the \emph{asaṃprajñātasamādhi} of the \emph{Yogasūtra} via the equation \emph{yogaḥ samādhiḥ} in the \emph{Bhāṣya} on \emph{Yogasūtra}  1.1. condense to "i.e.~the \emph{asaṃprajnātasmādhi} of the \emph{Pātañjalayogaśāstra}.} defined as the stopping of all movements of the mind. To a liberation seeker desiring this, [the science of Haṭhayoga] shines like a ladder.\footnote{The commentary adds etymological explanations: ``\emph{adhirohiṇī} means that by which one ascends, i.e.\ a ladder'', and a synonym for ``shines'', which are not translated here.} Just as a ladder leads someone desiring to ascend to a lofty terrace easily to the terrace, in the same way also the \emph{haṭhapradīpikā} leads someone desiring the lofty Rājayoga easily to Rājayoga. [In this verse] the trope is a comparison. The metre is Indravajrā.
\end{quote}

The interpretation makes good sense: Haṭhayoga leads effortlessly to Rājayoga, just as a ladder leads one to a high terrace. And the comparison as outlined by Brahmānanda has all the elements deemed necessary by Sanskritic poetology:

\begin{enumerate}[topsep=4pt,labelsep=0.5em]
\item Particle expressing a comparison (\emph{upamāvācaka}): \emph{iva} 
\item Standard of comparison (\emph{upamāna}): ladder leading to the terrace (\emph{saudhaprāpikā adhirohiṇī}) 
\item Thing that is compared (\emph{upameya}): \emph{Haṭhapradīpikā}
that leads to Rājayoga (\emph{rājayoga\-prāpikā haṭhapradīpikā}) 
\item Common quality  (\emph{samānadharma}): Effortlessness (\emph{anāyāsena})
\end{enumerate}

According to classical poetology a comparison containing all four elements is termed a ``full comparison'' (\emph{pūrṇopamā}), whereas an elision of one or more elements is called a ``deficient comparison'' (\emph{luptopamā}). One element that cannot be omitted without losing the comparison is the \emph{upamāna}. In the verse as given by Brahmānanda this would be ``the ladder''. The implication is spelt out by Brahmānanda in his commmentary: ``Just as a ladder leads someone desiring to ascend to a lofty terrace easily to the terrace, in the same way also the \emph{haṭhapradīpikā} leads someone desiring the lofty Rājayoga easily to Rājayoga.'' However, his text version spells out only the side of the \emph{upameya}, that is, ``Haṭhayoga leads to Rājayoga'', but leaves the \emph{upamāna} open to interpretation. His mention of the word \emph{saudha} in the commentary suggests that this was one of the other options he found in manuscripts—this reading is very well attested—but was rejected by him. In this way Brahmānanda makes sure that the text states the obvious, but at the same time it loses part of the comparison, and it loses its dynamics, which is brought out in the reading \emph{rājasaudha} preferred with good manuscript evidence in our critical edition. We think that Brahmānanda was eager to state at the outset the primacy of Rājayoga, and to this end sacrificed the more poetically elegant original reading that mentions the royal mansion as a metaphor for the ``royal yoga''.

In comparing the science of Haṭhayoga to a stairway, Svātmārāma was perhaps consciously emulating his favourite source text, the \emph{Vivekamārtaṇḍa}, which opens with \emph{etad vimuktisopānam}, “this is the stairway to complete liberation”. 

% JM: now saudha is preferred for stemmatic reasons too. I suggest the following shortened form of the note: 
% JB: I prefer the longer note for the digital edition and the shorter one (proposed by Jim) for the printed edition. How many readers will have access to Hanneder 2020 and read the whole article? The longer note nicely summarises a longer discussion in the article.
%The \textalpha\ witnesses read \emph{virājate pronnatarājayogam} while several others have either \emph{vibhrājate} or \emph{°rājasaudham} or both. As explained in Hanneder 2020, \emph{virājate} is stylistically preferable and the comparison between the science of \emph{haṭhayoga} and a stairway works only with \emph{°rājasaudham}.

%The reading \emph{rājasaudha} is preferable for poetical reasons, as explained in Hanneder 2020, p. 128–130. Also, the \emph{Jyotsnā} explains that the \emph{alaṃkāra} here is a comparison (\emph{upamā}), which consists of four elements: (1) a particle expressing the comparison JM: what's happened here? ALso, do we need to repeat what Jürgen has already written elsewhere?
%The Sanskrit poeticians explain that when some elements of a comparison remains unexpressed we get an incomplete (\emph{lupta}) comparison. Often words like \emph{iva} or \emph{yathā} are missing or the common property, but if we read \emph{rājayoga} we lose the \emph{upamāna} (i.e.~\emph{rājasaudha}). One part of the comparison should not be missing. But from the perspective of \emph{alaṃkāraśāstra} the verse has a problem that has surely caused the dilemma: it either leaves the \emph{upameya} or the \emph{upamāna} incomplete. Brahmānanda has mentioned both ? \emph{yathā adhirohiṇī saudhaprāpikā bhavati evaṃ haṭhadīpikā rājayogaprāpikā bhavati}, but had to supply \emph{rājasaudha} because his text reads \emph{rājayoga}. But to have the \emph{upameya} in the text is odd. Even Brahmānanda could only know of the image (\emph{rājasaudha}) from the alternative reading he did not accept. However, it seems likely that the author would have included the \emph{upamāna} in the text, as in our critical text, so that the reader would know that the upper terrace of the palace is an image for \emph{rājayoga}. It is further likely that the substitution of \emph{rājayoga} for \emph{rājasaudha} was the result of the tendency to insert the word \emph{yoga} in the opening verses of the text wherever possible, even where it does not fit, as can be seen in 1.2d (\emph{haṭhayogopadiśyate}) and 1.3b (\emph{rājayogam ajānatām}). In the latter case the poetical image has also been lost.

%The metre of 1.1 is Indravajrā.

% JB Oct 11: It might be helpful to translate Brahmānanda's comment for the non-Sanskrit reader: yathā pronnatasaudham āroḍhum icchor adhirohiṇy anāyāsena saudhaprāpikā bhavati evaṃ haṭhadīpikāpi pronnatarājayogam āroḍhum icchor anāyāsena rājayogaprāpikā bhavatīti upamālaṅkāraḥ
% JB: ``Just like one, who wishes to climb a lofty terrace, reaches the terrace without effort by a staircase, in the same way, one, who wishes to climb the lofty royal Yoga, reaches the royal Yoga by means of the Haṭhapradīpikā. This is the (poetic figure called) \emph{comparison}."

% JB: [...] but had to supply \emph{rājasaudha} because his text reads \emph{rājayoga}.
% JB: Even Brahmānanda could only know of the image of the upper terrace of the palace (\emph{rājasaudha})
\end{philcomm}
%</cm1>

\begin{metre}[hp01_001]
Upajāti
\end{metre}
%%%%%%%%%%
\subsection*{1.2}
%<*tr2>
\begin{translation}[hp01_002]
After bowing to the glorious guru, the Lord, the yogi Svātmārāma teaches the system of Haṭhayoga solely for [attaining] Rājayoga.
\end{translation}
%</tr2>

%<*cm2>
%\begin{philcomm}[hp01_002]
%The reading \emph{yogopadiśyate} (\getsiglum{J7,J10,N17,W4}, etc.) is only possible if one accepts that double \emph{sandhi} is a feature of the style of composition, which it is not. This reading appears to have resulted from an attempt to replace the word \emph{vidyā} with \emph{yoga} in the opening verses of the text.
%\end{philcomm}
%</cm2>

%%%%%%%%%%
\subsection*{1.3}
%<*tr3>
\begin{translation}[hp01_003]
 For those who cannot find the royal highway because they are lost in the darkness of many doctrines, the compassionate Svātmārāma holds the Light on Haṭha.
\end{translation}
%</tr3>

%<*ts3>
\begin{testimonia}[hp01_003]
\emph{Haṭharatnāvalī} 1.4
\begin{variants}
rājamārgam HRĀ\vl~] rājayogam HRĀ\sep
haṭhapradīpikāṃ dhatte~] kevalaṃ rājayogāya HRĀ\sep
svātmārāmaḥ kṛpākaraḥ~] haṭhavidyopadiśyate HRĀ
\end{variants}

%\emph{Haṭharatnāvalī} 1.4
%
%\begin{versinnote}
%\tl{bhrāntyā bahumatadhvānte rājayogam ajānatām/\\+}
%\tl{kevalaṃ rājayogāya haṭhavidyopadiśyate//\\!}
%\end{versinnote}
%\begin{appinnote}
%\tl{\textbf{b} rājayogam~] rājamārgam P,T,t1 \\!}
%\end{appinnote}

\end{testimonia}
%</ts3>

%<*cm3>
\begin{philcomm}[hp01_003] 
Most witnesses (including \alphaTwo) have \emph{rājayogam ajānatām} (`for those ignorant of Rājayoga') in 1.3b. The reading \emph{rājamārgam ajānatām} (\alphaOne\ and \alphaThree), “for those unable to find the royal highway”, is more appropriate to the metaphor of being lost in darkness. \lb

%In 1.3d \emph{kṛpākaraḥ} is attested by the most important groups, including \textalpha, of the available manuscripts. The readings \emph{prakāśyate} and \emph{kṣamākaraḥ} are attested by some manuscripts in lower branches of the stemma. Since the context is the author helping yogis who have strayed from the royal path, \emph{kṛpākaraḥ} makes good sense. 

As Brahmānanda notes, the compound \emph{kṛpākaraḥ} can be understood as one who is compassionate (\emph{kṛpā} + \emph{kara}) or one who is a mine (i.e.~a rich source) of compassion (\emph{kṛpā} + \emph{ākara}). In the Devanāgarī transmission, the \emph{kṣa} of \emph{kṣamākaraḥ} probably arose as a mistake for \emph{kṛ}.       
\end{philcomm}
%</cm3>

%%%%%%%%%%
\subsection*{1.4}
%<*tr4>
\begin{translation}[hp01_004]
For Matsyendra, Gorakṣa, and other [perfected yogis] discovered the science of Haṭha, and the yogi Svātmārāma knows it through their favour.
\end{translation}
%</tr4>

%<*ts4>
\begin{testimonia}[hp01_004]
\emph{Haṭharatnāvalī} 1.3
\begin{variants}
matsyendragorakṣādyā~] gorakṣamatsyendrādyā HRĀ\sep
svātmārāmo' thavā yogī~] ātmārāmo 'pi jānīte HRĀ\sep
jānīte tat prasādataḥ~] śrīnivāsas tathā svayam HRĀ
\end{variants}

%\emph{Haṭharatnāvalī} 1.3
%
%\begin{versinnote}
%\tl{haṭhavidyāṃ hi gorakṣamatsyendrādyā vijānate/\\+}
%\tl{ātmārāmo 'pi jānīte śrīnivāsas tathā svayam// \\!}
%\end{versinnote}

\end{testimonia}
%</ts4>

%<*cm4>
\begin{philcomm}[hp01_004]   
The word \emph{athavā} (‘or’) is well attested but difficult to construe here. Brahmānanda understands it as a conjunction (\emph{athavāśabdaḥ samuccaye}), and this is how we have interpreted it. The variant \emph{mahāyogī} in \zetaOne\ and other manuscripts (\getsiglum{G5,J4,J11,Ko}) is probably an attempt to remove the difficulty of understanding \emph{athavā}. One could emend to \emph{tathā} in light of the attested reading \emph{yathā} (\getsiglum{C7}) but this would be a bold intervention given the weight of evidence supporting \emph{’thavā}.   
\end{philcomm}
%</cm4>

%%%%%%%%%%
\subsection*{1.5}
%<*tr5>
\begin{translation}[hp01_005]
The glorious Ādinātha, Matsyendra, Śābara, Ānandabhairava, Cauraṅgī, Mīna, Gorakṣa, Virūpākṣa, Bileśaya,
\end{translation}
%</tr5>

%<*ts5>
\begin{testimonia}[hp01_005]
\emph{Haṭharatnāvalī} 1.80
\begin{variants}
cauraṅgi~] śāraṅgī HRĀ
\end{variants}

%\emph{Haṭharatnāvalī} 1.80
%
%\begin{versinnote}
%\tl{śrīādināthamatsyendraśābarānandabhairavāḥ/\\+}
%\tl{śāraṅgīmīnagorakṣavirūpākṣabileśayāḥ// \\!}
%\end{versinnote}

\end{testimonia}
%</ts5>

%<*cm5>
\begin{philcomm}[hp01_005]     
In Śaiva texts which predate the Haṭha corpus, Mīnanātha and Matsyendra are one and the same, but they are differentiated in later Tibetan and Indian lists of siddhas (Mallinson 2019: 273 n.\,35). \lb

Two manuscripts of the \textalpha\ and \textdelta\ groups have the variant reading \emph{virūpākṣaḥ savālikaḥ} (\getsiglum{J5,V19}) for \emph{virūpākṣabileśayāḥ}. In another \textalpha\ manuscript, \getsiglum{N3}, \emph{savālikaḥ} was corrected to \emph{savālmikaḥ}, perhaps in an effort to restore a name similar to Vālmīki, the celebrated author of the \textit{Rāmāyaṇa}.
\end{philcomm}
%</cm5>

%%%%%%%%%%
\subsection*{1.6}
%<*tr6>
\begin{translation}[hp01_006]
Manthānabhairava Yogī, Siddhabuddha, and Kanthaḍi, Goraṇṭaka, Surānanda, Siddhapāda and Carpaṭi,
\end{translation}
%</tr6>

%<*ts6>
\begin{testimonia}[hp01_006]
\emph{Haṭharatnāvalī} 1.81, \emph{Mugdhāvabodhinī} 1.7.8
\begin{variants}
kanthaḍiḥ~] kandalī HRĀ, kanthaḍī MA\sep
goraṇṭakaḥ~] koraṇṭakaḥ MA, korandakaḥ HRĀ, gonandaka HRĀ\vl\sep
carpaṭiḥ~] carpaṭī HRĀ MA %JM:Changed Caturbhuja Miśra's \emph{Mugdhāvabodhinī} 1.7.8 on the \emph{Rasahṛdayatantra} to Mugdhāvabodhinī
\end{variants}

%\emph{Haṭharatnāvalī} 1.81
%
%\begin{versinnote}
%\tl{manthānabhairavo yogī siddhabuddhaś ca kandalī/\\+}
%\tl{korandakaḥ surānandaḥ siddhipādaś ca carpaṭī//\\!}
%\end{versinnote}
%\begin{appinnote}
%\tl{\textbf{c} korandakaḥ~] gonandaka P,T,J,n1,n4 \\!}
%\end{appinnote}

%Caturbhuja Miśra's \emph{Mugdhāvabodhinī} (1.7.8) on the \emph{Rasahṛdayatantra}
%
%\begin{versinnote}
%\tl{manthānabhairavo yogī siddhabuddhaś ca kanthaḍī/\\+}
%\tl{koraṇṭakaḥ surānandaḥ siddhapādaś ca carpaṭī//\\!}
%\end{versinnote}

\end{testimonia}
%</ts6>

%<*cm6>
\begin{philcomm}[hp01_006]        
The \textalpha\ manuscripts have \emph{goraṇṭaka}, and several other manuscript groups have \emph{pauraṇṭaka}.  We are yet to find the name \emph{goraṇṭaka} in other Sanskrit texts but it may be a Sanskrit rendering of \emph{Goraṇṭakuḍu}, which is the name of a disciple of Gorakṣanātha in the \emph{Navanāthacaritramu} (Jones 2017: 194 n.\,3). The spelling \emph{koraṇṭaka} is attested in the \emph{Haṭhābhyāsa\-paddhati}, and it is reasonably well attested by manuscripts of the \emph{Haṭhapradīpikā}, as well as those of the \emph{Haṭharatnāvalī} (which also has \emph{gonandaka}).\lb

The compound \emph{siddhapāda} could be a respectful affix, but it seems unlikely here because then the name would cross the \emph{pāda} break.

\end{philcomm}
%</cm6>

%%%%%%%%%%
\subsection*{1.7}
%<*tr7>
\begin{translation}[hp01_007]
Kaṇerī, Pūjyapāda, Nityanātha, Nirañjana, Kapālī, Bindunātha, and the one named Kākacaṇḍīśvara.%JM: I think we should adopt the well-attested Kaṇerī here. I have changed the tr., ed. needs changing.
\end{translation}
%</tr7>


%<*ts7>
\begin{testimonia}[hp01_007]
\emph{Haṭharatnāvalī} 1.82, \emph{Mugdhāvabodhinī} 1.7.9 %NJL: I don't have the MA e-text. Is this inferred reference correct? 
\begin{variants}
kāṇerī MA~] karoṭiḥ HRĀ\sep
kākacaṇḍīśvarāhvayaḥ HRĀ~] kākacaṇḍīśvaro gajaḥ MA
\end{variants}

%\emph{Haṭharatnāvalī} 1.82
%
%\begin{versinnote}
%\tl{karoṭiḥ pūjyapādaś ca nityanātho nirañjanaḥ/\\+}
%\tl{kapālī bindunāthaś ca kākacaṇḍīśvarāhvayaḥ// \\!}
%\end{versinnote}
%
%Caturbhuja Misra's \emph{Mugdhāvabodhinī} on the \emph{Rasahṛdayatantra}
%
%\emph{Mugdhāvabodhinī} 1.7.9
%
%\begin{versinnote}
%\tl{kaṇerī pūjyapādaś ca nityanātho nirañjanaḥ/\\+}
%\tl{kapālī bindunāthaś ca kākacaṇḍīśvaro gajaḥ/\\!}
%\end{versinnote}
%
\end{testimonia}
%</ts7>

%<*cm7>
\begin{philcomm}[hp01_007]   
It is possible that \emph{pūjyapāda} is a respectful affix to the name Kāṇerī. The variant \emph{dhvaninātha} may have resulted from a transposition of the first two syllables of \emph{nityanātha}. \lb 

The \textalpha\ group supports \emph{kākacaṇḍīśvaro gayaḥ} but we have not been able to find evidence for a Siddha called Gaya.
\end{philcomm}
%</cm7>

%%%%%%%%%%
\subsection*{1.8}
%<*tr8>
\begin{translation}[hp01_008]
Allamaprabhudeva, Ghoḍācolī, Ṭiṇṭiṇi, Bhāluki and Nāgabodha and Khaṇḍa\-kāpālika.
\end{translation}
%</tr8>

%<*ts8>
\begin{testimonia}[hp01_008]
\emph{Haṭharatnāvalī} 1.83, \emph{Mugdhāvabodhinī} 1.7.10 %NJL: I don't have the MA e-text. Is the inferred reference of MA correct? Yes.

\begin{variants}
  allamaḥ HRĀ~] allama HRĀ\vl, āllamaḥ MA\sep 
  ghoḍācolī MA~] naiṭacūṭiś HRĀ\sep
  tiṇṭiṇiḥ HRĀ~] ṭhiṇṭhinī MA\sep
  nāgabodhaś HRĀ~] nāgadevaś MA\sep
  khaṇḍa HRĀ~] khaṇḍī MA
\end{variants}
%JB: I don't think we have been using °. JM: there are lots of instances of it in all sections, including a few in the main apparatus.

%\emph{Haṭharatnāvalī} 1.83
%
%\begin{versinnote}
%\tl{allamaḥ prabhudevaś ca naiṭacūṭiś ca ṭiṇṭiṇiḥ/\\+}
%\tl{bhālukir nāgabodhaś ca khaṇḍakāpālikas tathā//\\!}
%\end{versinnote}
%\begin{appinnote}
%\tl{\textbf{a} allamaḥ prabhudevaś~] allamaprabhudevaś P,T,t1 \\!}
%\end{appinnote}
%
%Caturbhuja Misra's \emph{Mugdhāvabodhinī} on the \emph{Rasahṛdayatantra}
%
%\begin{versinnote}
%\tl{āllamaḥ prabhudevaś ca ghoḍācolī ca ṭhiṇṭhinī/\\+}
%\tl{bhālukir nāgadevaś ca khaṇḍī kāpālikas tathā//\\!}
%\end{versinnote}
%
\end{testimonia}
%</ts8>

%<*cm8>
\begin{philcomm}[hp01_008]  
The name Allamaprabhudeva (sometimes Allama Prabhu Deva or Allama Prabhudeva in secondary literature) is frequently transmitted as \emph{allamaḥ prabhu\-devaḥ} (Cf.~\alphaTwo), as though it were two names. However, manuscripts \alphaOne, \alphaThree\ and others (e.g.~\getsiglum{V3,V8,V13,V16,V22,N24,N26,Jyo}) do not have the \emph{visarga} and write it as one name (i.e.~\emph{allamaprabhudevaś ca}). This is also the case in some manuscripts of the \emph{Haṭharatnāvalī} (P,T,t1 in Gharote 2009: 35 n.\,8). \lb

The names Nāgabodha, Nāgabodhi, Nāradeva, Nāgadeva all seem possible in 1.8c. The reading \emph{nāgabodhaś ca} is attested across several primary groups of manuscripts. \lb

The \alphaOne\ and \alphaTwo\ reading of \emph{siddhaḥ kāpālikas} is an exception among the manuscripts and seems too vague to be referring to someone within a lineage. Khaṇḍa\-kāpālika is well attested by the remaining manuscripts (including \alphaThree) and this name appears in other texts, e.g.~Vajrapāṇi’s \emph{Laghutantraṭīkā}, p.\,45, where Khaṇḍakāpālika is the first of the 24 Vīras (\emph{vīrāḥ khaṇḍakāpālikādayaś caturviṃśatiḥ}). It is likely to refer to an ascetic who carries a broken skull. \emph{Matsyendrasaṃhitā} 33.2 mentions a practice for which one needs a \emph{khaṇḍakapāla} and in the \emph{Saṃvaramaṇḍala} of the \emph{Niṣpannayogāvalī}, p.26, Vajravārāhī is \emph{kapālakhaṇḍakṛtakaṭibhūṣaṇā}. The compound \emph{khaṇḍakāpālika} is found at \emph{Kathāsaritsāgara} 18.2.6, but there \sl{khaṇḍa} is being used in a derogatory sense (18.2.15 refers to the same character as a \sl{duṣṭakāpālika}).
%It may be a derogatory name for a Kāpālika, coined perhaps by an outsider and connoting something like a defective Kāpālika in the sense of a ‘part-time’ Kāpālika. Alternatively, it could simply refer to one who uses a broken skull as a bowl. 
% Reference to the KSS completed and text changed: before: Examples include \emph{Kathāsaritsāgara} 121.5 ff. (check).  Why "part-time".
\end{philcomm}
%</cm8>

%%%%%%%%%%
\subsection*{1.9}
%<*tr9>
\begin{translation}[hp01_009]
These and other great adepts used the power of haṭhayoga to smash the rod of death and [so] are roaming the worlds.
\end{translation}
%</tr9>

%<*ts9>
\begin{testimonia}[hp01_009]
\emph{Haṭharatnāvalī} 1.84, \emph{Mugdhāvabodhinī} 1.7.11, \emph{Haṭhatattvakaumudī} 17.24

% \begin{variants}
%   ityādayo mahāsiddhā MA~] ityādayo mahāsiddhāḥ HRĀ, ūrdhvaṃretaḥprabhāvena HTK\sep
%   mahāsiddhā MA~] mahāsiddhāḥ HRĀ\sep
%   haṭhayogaprabhāvataḥ~] haṭhayogaprasādataḥ HRĀ, rasabhogaprasādataḥ MA, sanakādyā maharṣayaḥ HTK\sep
%   brahmāṇḍeṣu~] brahmāṇḍe HRĀ, trilokyāṃ MA, yathecchaṃ HTK\sep
%   caranti te~] vicaranti te HRĀ MA, viharanti te HTK
% \end{variants}


%JB: I think this is correct. mahāsiddhāḥ in HRĀ is a sandhi error that is not worth reporting
\begin{variants}
  ityādayo mahāsiddhā MA HRĀ~] ūrdhvaṃretaḥprabhāvena HTK\sep
  haṭhayogaprabhāvataḥ~] haṭhayogaprasādataḥ HRĀ, rasabhogaprasādataḥ MA, sanakādyā maharṣayaḥ HTK\sep
  brahmāṇḍeṣu caranti~] brahmāṇḍe vicaranti HRĀ, trilokyāṃ vicaranti MA, yathecchaṃ viharanti HTK\sep
\end{variants}
%
%\emph{Haṭharatnāvalī} 1.84  
%
%\begin{versinnote}
%\tl{ityādayo mahāsiddhāḥ haṭhayogaprasādataḥ/\\+}
%\tl{khaṇḍayitvā kāladaṇḍaṃ brahmāṇḍe vicaranti te//\\!}
%\end{versinnote}
%
%Caturbhuja Misra's \emph{Mugdhāvabodhinī} on the \emph{Rasahṛdayatantra}
%
%\begin{versinnote}
%\tl{ityādayo mahāsiddhā rasabhogaprasādataḥ/\\+}
%\tl{khaṇḍayitvā kāladaṇḍaṃ trilokyāṃ vicaranti te/\\!}
%\end{versinnote}
%
%\emph{Haṭhatattvakaumudī} 17.24
%
%\begin{versinnote}
%\tl{ūrdhvaṃretaḥprabhāvena sanakādyā maharṣayaḥ/\\+}
%\tl{khaṇḍayitvā kāladaṇḍaṃ yathecchaṃ viharanti te// 24//\\!}
%\end{versinnote}

\end{testimonia}
%</ts9>

%<*cm9>
\begin{philcomm}[hp01_009]        
The reference to \emph{brahmāṇḍa} (‘the world’) implies that these great Siddhas have attained liberation-in-life (\emph{jīvanmukti}) and physical immortality.  
\end{philcomm}
%</cm9>

\begin{metre}[hp01_009]
Anuṣṭubh (c: ra-vipulā)
\end{metre}

%%%%%%%%%%
\subsection*{1.10}
%<*tr10>
\begin{translation}[hp01_010]
Haṭha is a hut of refuge for those who are burnt by the scorching torment of transmigration. Haṭha is the tortoise that supports the worlds of all yogas.
\end{translation}
%</tr10>

%<*ts10>
\begin{testimonia}[hp01_010]
\emph{Yogasārasaṅgraha}, p.\,53.

\begin{variants}
  samāśrayamaṭho~] samāśrayahaṭho YSS
\end{variants}

%\emph{Yogasārasaṅgraha}, p.\,53.
%\begin{versinnote}
%\tl{saṃsāratāpataptānāṃ samāśrayahaṭho haṭhaḥ/\\+}
%\tl{aśeṣayogajagatām ādhārakamaṭho haṭhaḥ//\\!}  
%\end{versinnote}

\end{testimonia}
%</ts10>

%<*cm10>
\begin{philcomm}[hp01_010] 
%The compound \emph{saṃsāratāpa°} is well attested and found elsewhere (e.g.~\emph{Viṣṇupurāṇa} 6.7.62, \emph{Agnipurāṇa} 371.1, \emph{Haṭhatattvakaumudī} 38.92, \emph{Haṭhābhyāsapaddhati} ms. 46/440, f.~1v). 

%The reading of \emph{samāśrayo} in \etaOne\ is metrically faulty. 
The \textalpha\ group omits the second line of this verse, but this was probably the result of eyeskip caused by the repetition of \emph{maṭho haṭhaḥ}. Both \emph{jagatām} and \emph{yuktānām} are well attested by the collated manuscripts. We have adopted \emph{jagatāṃ} because it makes good sense with  \emph{ādhārakamaṭhah} in light of the cosmological notion that the worlds are supported by a tortoise. This reading may not have been understood by some and was changed in other witnesses to \emph{yuktānām}, which was adopted by Brahmānanda in \emph{Jyotsnā} 1.10.       
\end{philcomm}
%</cm10>

\begin{metre}[hp01_010]
Anuṣṭubh (c: na-vipulā)
\end{metre}

%%%%%%%%%%
\subsection*{1.11}
%<*tr11>
\begin{translation}[hp01_011]
The science of Haṭha should be kept completely secret by yogis who want success. It becomes potent when kept secret but impotent when revealed.
\end{translation}
%</tr11>

%<*sc11>
%\begin{sources}[hp01_011]
%\end{sources}
%</sc11>

%<*ts11>
\begin{testimonia}[hp01_011]
  \emph{Śivasaṃhitā} 5.254, \emph{Yogacintāmaṇi} f.\,141r (\attr HP)

  \begin{variants}
    haṭhavidyā ŚS~] haṭhavidyā YCM\sep
    yogināṃ~] yoginā ŚS, YCM\sep
    icchatām~] icchatā ŚS, YCM\sep
    tu~ YCM~] ca ŚS\sep
    \end{variants}


   
% \emph{Śivasaṃhitā} 5.254,
%
%\begin{versinnote}
%\tl{haṭhavidyā paraṃ gopyā yoginā siddhim icchatā/\\+}
%\tl{bhaved vīryavatī guptā nirvīryā ca prakāśitā//\\!}
%\end{versinnote}
%\begin{appinnote}
%\tl{haṭhavidyā ... icchatā~] \emph{om.} III–XII, XIV \\!}
%\end{appinnote}
%
%\emph{Yogacintāmaṇi} f.\,141r %(NGMCP 1-1337, reel no: B 39/5)
%
%\begin{versinnote}
%\tl{tathā haṭhapradīpikāyām—\\+}
%\tl{haṭhavidyā paraṃ gopyā yoginā siddhim icchatā/\\+}
%\tl{bhaved vīryavatī guptā nirvīryā tu prakāśiteti//\\!}
%\end{versinnote}

%BKhP 10v4
\end{testimonia}
%</ts11>

%<*cm11>
\begin{philcomm}[hp01_011]        
Either the singular or plural of \emph{yogin} could be read here. The singular is well attested among the testimonia, but the manuscript transmission favours the plural.  
\end{philcomm}
%</cm11>

%%%%%%%%%%
\subsection*{1.12}
%<*tr12>
\begin{translation}[hp01_012]
The Haṭhayogi should live in an isolated hut in a well-ruled, righteous land which has plenty of food and is free from upheaval.
\end{translation}
%</tr12>
%MD: pāda d: Adopt the plural form °yogināṃ (reading of α1,α3,ε1) instead of °yoginā in line with the previous verse (HP1.11b yogināṃ siddhim icchatām)? JM: yes, good to me, so change tr. to Haṭhayogis...

%<*sc12>
%\begin{sources}[hp01_012]
%\end{sources}
%</sc12>

%<*ts12>
\begin{testimonia}[hp01_012]
\emph{Haṭharatnāvalī} 1.66, \emph{Yogacintāmaṇi} f.\,54r (\attr HP)

  \begin{variants}
    surājye YCM~] surāṣṭe HRĀ\sep
    ekānta HRĀ~] ekānte YCM
   \end{variants}

%\emph{Haṭharatnāvalī} 1.66
%
%\begin{versinnote}
%\tl{surāṣṭre dhārmike deśe subhikṣe nirupadrave/\\+}
%\tl{ekāntamaṭhikāmadhye sthātavyaṃ haṭhayoginā//\\!}
%\end{versinnote}

%\emph{Yogacintāmaṇi} f.\,54r
%
%\begin{versinnote}
%\tl{haṭhapradīpikāyām—\\+}
%\tl{surājye dhārmike deśe subhikṣe nirupadrave/\\+}
%\tl{ekānte maṭhikāmadhye sthātavyaṃ haṭhayoginā//\\!}
%\end{versinnote}

\end{testimonia}
%</ts12>

%<*cm12>
\begin{philcomm}[hp01_012]  
The term \emph{maṭhikā} occurs in narrative literature and yoga texts in the sense of a small hut. For example, in the \textit{Kathāsaritsāgara} (12.9.14, 29–30), \emph{maṭhikā} refers to the small hut built in a cremation ground by a young Brahmin who makes as his bed the ashes of the dead girl he had hoped to marry. In several other stories (\textit{Kathāsaritsāgara} 6.6.132, 10.5.89, 12.25.35), \emph{maṭhikā} is the term used for the hut of an ascetic. In an elaborate description of the huts (\emph{maṭhikā}) used for Haṭhayoga, the author of the \emph{Haṭhābhyāsapaddhati} states that the dimensions of the hut are four ``\emph{hasta}s" high and wide (there are various definitions of the term: 18 inches according to the Larger Petrograd Dictionary and Monier-Williams, 48 inches according to the Smaller Petrograd Dictionary; the NWS lists even more variations, but the stipulation that the hut is the length of a bow found in the \emph{Gorakṣaśataka} (on which see below) suggests that 18 inches was meant). The hut can be made of various materials, such as red earth, ashes, plaster and so on (Birch and Singleton 2019: 17–18).\lb

In the \emph{Jyotsnā} and printed editions of the \emph{Haṭhapradīpikā}, including one by Digambara and Kokaje (1970: 6), this verse has the additional line, \emph{dhanuḥ\-pramāṇa\-paryantaṃ śilāgnijalavarjite}. This line derives from the \emph{Gorakṣaśataka} (32cd), which has \emph{paryante} instead of \emph{paryantaṃ}. It stipulates that the hut should be built in a place measuring up to a bow length and free from rocks, fire and water. None of the early manuscripts has this line, which suggests that it was added at a later time. Nonetheless, it appears in over a dozen manuscripts that were consulted for this edition. These manuscripts are not close to an early hyp\-archetype of the text.
\end{philcomm}
%</cm12>

%%%%%%%%%%
\subsection*{1.13}
%<*tr13>
\begin{translation}[hp01_013]
Having a small door and no cracks, holes or bumps, neither too high nor too low in size, thickly smeared with cow dung in the proper way, clean, free from all annoyances, pleasing on the outside with a verandah, altar and well, surrounded by a wall: these are the characteristics of the yoga hut as taught by the adept practitioners of Haṭha.
%JM changed tr from : It has a small door and is without cracks, holes and bumps. It is neither too high nor too low in extent and is thickly smeared with cow dung in the proper way. It is clean, free from all annoyances, pleasing on the outside with a verandah, altar and well, surrounded by a wall: these are the characteristics of the yoga hut as taught by the adept practitioners of Haṭha. 
%JB is the "with" at the beginning necessary? JM: I think so, it gets the bahuvrīhi. Could be "Having a small door..."
\end{translation}
%</tr13>

%<*sc13>
\begin{sources}[hp01_013]
Cf.~\emph{Dattātreyayogaśāstra} 54cd–57

\begin{versinnote}
\tl{suśobhanaṃ maṭhaṃ kuryāt sūkṣmadvāraṃ tu nirvraṇaṃ//\\+}
\tl{suṣṭhu liptaṃ gomayena sudhayā vā prayatnataḥ/\\+}
\tl{matkuṇair maśakair bhūtair varjitaṃ ca prayatnataḥ//\\+}
\tl{dine dine susammṛṣṭaṃ sammārjanyā hy atandritaḥ/\\+}
\tl{vāsitaṃ ca sugandhena dhūpitaṃ guggulādibhiḥ//\\+}
\tl{malamūtrādibhir vargair aṣṭādaśabhir eva ca/\\+}
\tl{varjitaṃ dvārasampannaṃ vastrāvaraṇam eva vā//\\!}
\end{versinnote}

\end{sources}
%</sc13>

%<*ts13>
\begin{testimonia}[hp01_013]
\emph{Haṭharatnāvalī} 1.67, \emph{Yogacintāmaṇi} 54r (\attr HP)

  \begin{variants}
    piṭakaṃ HRĀ\vl~] piṭharaṃ HRĀ, ghaṭitaṃ YCM\sep
    mamalaṃ~] mavilaṃ YCM\vl, vimalaṃ HRĀ YCM\sep
    nātyuccanīcāyataṃ HRĀ~] nāpy uccanīcāyitam YCM\sep
    bādhojjhitam HRĀ~] jantūjjhitam YCM\sep
    vedikūparuciraṃ HRĀ~] kūpavediracitaṃ YCM
   \end{variants}

%\emph{Haṭharatnāvalī} 1.67
%
%\begin{versinnote}
%\tl{alpadvāram arandhragartapiṭharaṃ nātyuccanīcāyataṃ\\+}
%\tl{samyaggomayasāndraliptavimalaṃ niḥśeṣabādhojjhitaṃ/\\+}
%\tl{bāhye maṇḍapavedikūparuciraṃ prākārasaṃveṣṭitam\\+}
%\tl{proktaṃ yogamaṭhasya lakṣaṇam idaṃ siddhair haṭhābhyāsibhiḥ//\\!}
%\end{versinnote}
%\begin{appinnote}
%\tl{\textbf{a} °piṭharaṃ~] piṭakaṃ J,n2, peṭakaṃ N \\!}
%\end{appinnote}
%
%\emph{Yogacintāmaṇi} 54r (\attr HP)
%
%\begin{versinnote}
%\tl{alpadvāram arandhragartaghaṭitaṃ nāpy uccanīcāyitam/\\+}
%\tl{samyaggomayasāndraliptavimalaṃ niḥśeṣajantūjjhitam/\\+}
%\tl{bāhye maṇḍapakūpavediracitaṃ prākārasaṃveṣṭitam/\\+}
%\tl{proktaṃ yogamaṭhasya lakṣaṇam idaṃ siddhair haṭhābhyāsibhiḥ//\\!}
%\end{versinnote}
%\begin{appinnote}
%\tl{\textbf{b} °vimalaṃ~] L, mavilaṃ N \\!}
%\end{appinnote}

%\emph{Suśrutasaṃhitā} 6.17.67:
%
%\begin{versinnote}
%\tl{gṛhe nirābādhe\\!}
%\end{versinnote}
%NJL: I am not sure what I am supposed to do with quote of the Suśruta... Since it is already mentioned in the philological commentary, can we omit it here? JB: maybe leave it out. It's mentioned in the philcomm.
%BKhP 107v3
\end{testimonia}
%</ts13>

%<*cm13>
\begin{philcomm}[hp01_013]  
The syntax of this verse is problematic. One would expect the features of the hut, which are listed in the first three quarters of the verse, to be in the nominative case. Then, the words \emph{idaṃ lakṣaṇaṃ} in the fourth quarter would refer back to them. However, the compounds in the first three verse-quarters appear to qualify \emph{lakṣaṇa} as though they were adjectives, and this seems to have been the way the verse was composed. \lb    

The manuscripts preserve many different readings at the end of the compound beginning with \emph{arandhragarta}. We have adopted \emph{piṭaka}, which usually means ``a basket'' but can also mean ``a boil or blister," because it is well attested and might here refer to bumps on the floors or walls that would make them uneven. Another possibility is \emph{piṭharaṃ}, which can have the sense of potsherds and would here mean that the hut should be free of rubbish on the floor. One would expect a word for a defect in a hut that is similar to, but not the same as, cracks (\emph{randhra}) and holes (\emph{garta}). For this reason, the reading \emph{vivaraṃ} looks like a patch, as its meaning does not add anything to \emph{randhragarta}. The reading \emph{viṭapaṃ} (`the young branch of a tree or creeper') attested in some manuscripts of the \emph{Haṭhapradīpikā} is difficult to construe in this context unless it was intended to refer to creepers or branches that might invade or encroach upon the hut.\lb

% (MD: Stemmatically, \emph{piṭaka} ``a basket; a boil" is well attested. Is it perhaps possible to interpret the word as ``a bump"? \emph{a-randhra-garta-piṭaka} would then mean that the floor or walls of the hut are not uneven.) JM: yes, sounds good to me.
%JB Okay I've written the note to reflect this. 
% MD 2024-8: amalaṃ or vimalaṃ? amalaṃ best

Manuscripts of several groups, namely \textepsilon, \texteta, and \textpi, have \emph{bādhojjitaṃ}, whereas \textgamma, \textdelta\ and the \emph{Yogacintāmaṇi} have the more easily understood reading of \emph{jantūjjhitaṃ} (‘free from creatures’). The \textalpha\ group is split on this, with \alphaThree\ (\emph{bodhojhitaṃ}) closer to \emph{bādhojjhitaṃ} and \alphaOne\ (\emph{jyaṃtyūpsitaṃ}) and \alphaTwo\ (\emph{jaṃtūṣṇitaṃ}) closer to \emph{jantūjjhitaṃ}. We have adopted the more unusual reading of \emph{bādhojjhitaṃ} with the support of a similar description of a hut in \emph{Suśrutasaṃhitā} 6.17.67 (\emph{gṛhe nirābādhe}). 

\end{philcomm}
%</cm13>

\begin{metre}[hp01_013]
Śārdūlavikrīḍita
\end{metre}

%%%%%%%%%%
\subsection*{1.14}
%<*tr14>
\begin{translation}[hp01_014]
Staying in such a hut, free from all worry, [the yogi] should regularly practise yoga in exactly the way taught by his guru.
\end{translation}
%</tr14>

%<*sc14>
\begin{sources}[hp01_014]
Cf.~\emph{Amanaska} 2.15

\begin{versinnote}
\tl{evaṃvidhaṃ guruṃ labdhvā sarvacintāvivarjitaḥ/\\+}
\tl{sthitvā manohare deśe yogam eva samabhyaset//\\!}
\end{versinnote}

\end{sources}
%</sc14>

%<*ts14>
\begin{testimonia}[hp01_014]
\emph{Haṭharatnāvalī} 1.68, \emph{Yogacintāmaṇi} f.\,54r (\attr HP)
%\mylb
%\begin{versinnote}
%\tl{evaṃvidhe maṭhe sthitvā sarvacintāvivarjitaḥ/\\+}
%\tl{gurūpadiṣṭamārgeṇa yogam eva sadābhyaset//\\!}
%\end{versinnote}

%\begin{versinnote}
%\tl{evaṃvidhe maṭhe sthitvā sarvacintāvivarjitaḥ/\\+}
%\tl{gurūpadiṣṭamārgeṇa yogam eva sadābhyaset//\\!}
%\end{versinnote}

\end{testimonia}
%</ts14>

%<*cm14>
%\begin{philcomm}[hp01_014]        
%\end{philcomm}
%</cm14>
% MD 2024-8: samabhyaset or sadābhyaset?

%%%%%%%%%%
\subsection*{1.15}
%<*tr15>
\begin{translation}[hp01_015]
Overeating, exertion, idle chatter, not sticking to observances, socialising and sensuality: through [these] six, yoga is lost.
\end{translation}
%</tr15>

%<*ts15>
\begin{testimonia}[hp01_015]
\emph{Haṭharatnāvalī} 1.77, \emph{Yogacintāmaṇi} f.\,48v (\attr HP), \emph{Yuktabhavadeva} 4.25 (\attr \emph{Śivayoga})

\begin{variants}
  'niyamagrahaḥ HRĀ YCM~] niyamāgrahaḥ HRĀ\vl YBhD\sep
  janasaṅgaś ca YCM YBhD~] janasaṅgaṃ ca HRĀ\sep
  yogo vinaśyati YBhD HRĀ~] yogaḥ praṇaśyati YCM\sep
\end{variants}
%JB: yogo/yogaḥ is just a sandhi issue.
%\emph{Haṭharatnāvalī} 1.77
%
%\begin{versinnote}
%\tl{atyāhāraḥ prayāsaś ca prajalpo niyamagrahaḥ/\\+}
%\tl{janasaṅgaṃ ca laulyaṃ ca ṣaḍbhir yogo vinaśyati//\\!}
%\end{versinnote}
%\begin{appinnote}
%\tl{\textbf{b} niyamagrahaḥ~] niyamāgrahaḥ N,J \\!}
%\end{appinnote}
%
%\emph{Yogacintāmaṇi} f.\,48v (\attr HP)
%\begin{versinnote}
%\tl{atyāhāraḥ prayāsaś ca prajalpo niyamagrahaḥ/\\+}
%\tl{janasaṅgaś ca laulyaṃ ca ṣaḍbhir yogaḥ praṇaśyati//\\!}
%\end{versinnote}
%
%\emph{Yuktabhavadeva} 4.25 (attr. \emph{Śivayoga})
%
%\begin{versinnote}
%\tl{atyāhāraḥ prayāsaś ca prajalpo niyamāgrahaḥ/\\+}
%\tl{janasaṃgaś ca laulyaṃ ca ṣaḍbhir yogo vinaśyati//\\!}
%\end{versinnote}

Cf.~\emph{Jyotsnā} 1.15% put this and the following in the commentary?

\begin{versinnote}
\tl{śītodakena prātaḥsnānanaktabhojanaphalāhārādirūpaniyamasya grahaṇaṃ niyamagrahaḥ/\\!}
\end{versinnote}
%
\emph{Yogaprakāśikā} 1.48
%
\begin{versinnote}
\tl{niyamāgrahaḥ vakṣyamāṇaniyamāparipālanaṃ\\!}
\end{versinnote}
%
\end{testimonia}
%</ts15>

%<*cm15>
\begin{philcomm}[hp01_015]        
Since many scribes do not use an \emph{avagraha}, we cannot be sure whether to understand \emph{niyamagrahaḥ} in \emph{pāda} b as having a negative prefix. Although \emph{yama} and \emph{niyama} are not included in the \emph{Haṭhapradīpikā} as auxiliaries of Haṭhayoga, verse 2.14 implies that some \emph{niyama}s are necessary at least in the early stages of establishing a practice. Furthermore, verse 3.78 suggests that a yogi who does not practice \emph{niyama} might obtain success in yoga through the practice of \textit{vajrolī}. Thirty of the manuscripts consulted for this critical edition contain additional verses on ten \emph{yama} and ten \emph{niyama}, which are usually inserted after 1.16 and derive from either the \emph{Śāradātilakatantra} (25.7–8) or the \emph{Vasiṣṭhasaṃhitā} (1.38, 1.53). %JB 2.14 doesnt mention niyama (and more correctly only implies mitāhāra). Perhaps we should rephrase this as, "...2.14 implies that some rules (niyama) were necessary at least..."
%In the \emph{Jyotsnā}, Brahmānanda reads \emph{niyamāgraha} and takes it as though \emph{āgraha} was implied, which yields the meaning of ‘over-insistence on rules’, and he relates it to extreme ascetic practice.
\end{philcomm} 
%</cm15>

%%%%%%%%%%
\subsection*{1.16}
%<*tr16>
\begin{translation}[hp01_016]
Zeal, courage, resolve, contentment, realisation of the truth, and avoiding contact with people: through [these] six, yoga is successful.
\end{translation}
%</tr16>
%JN [virtues] seems unnecessary. delete? (we have not supplied anything in the previous verse, which has similar syntax) 
% JB; change to 'through [these] six, ...' (to be consistent with the previous verse?) And should we reflect the different syntax of this verse (i.e.~the ablatives)?

%<*sc16>
\begin{sources}[hp01_016]
\emph{Dharmaputrikā} 38cd–39ab, \emph{Śivadharmottara} 10 (W\,122r), \emph{Jñānārṇava} 20.1, \emph{Yogabindu} 411 (by Haribhadra)
%Narharināth ed.? Yes
%Yogī Naraharinātha, Śivadharma PaśupatimatamŚivadharmamahāśāstram Paśupatināthadarśanam.Kathmandu 1998, pp. 679 onwards

\begin{variants}
  utsāhāt~] utsāhān JA YB ŚDhU, utsāho DhP\sep
  sāhasād~] niścayo DhP, niścayād JA YB ŚDhU\sep
  dhairyāt JA YB ŚDhU~] dhairyaṃ DhP\sep
  saṃtoṣāt JA YB ŚDhU~] santoṣas DhP\sep
  darśanāt YB ŚDhU~] darśanam DhP, niścayāt JA\sep
  janasaṅgaparityāgāt~] muner janapadatyāgād JA YB ŚDhU, kratūnāṃ copasaṃhāraḥ DhP\sep
  ṣaḍbhir yogaḥ prasidhyati JA YB ŚDhU~] ṣaṭsādhanam iti smṛtam DhP
  \end{variants}

%\emph{Dharmaputrikā} 38cd–39ab  
%\begin{versinnote}
%\tl{utsāho niścayo dhairyaṃ santoṣas tattvadarśanam/\\+}
%\tl{kratūnāṃ copasaṃhāraḥ ṣaṭsādhanam iti smṛtam/\\!}
%\end{versinnote}
%
%\emph{Śivadharmottara} 10 (W 122r):%JM: what ed or ms is this?
%
%\begin{versinnote}
%\tl{utsāhān niścayād dhairyāt santoṣāt tattvadarśanāt/\\+}
%\tl{muner janapadatyāgād ṣaḍbhir yogaḥ prasiddhyati/\\!}
%\end{versinnote}
%
%\emph{Jñānārṇava} 20.1
%
%\begin{versinnote}
%\tl{utsāhān niścayād dhairyāt saṃtoṣāt tattvaniścayāt/\\+}
%\tl{muner janapadatyāgāt ṣaḍbhir yogaḥ prasidhyati//\\!}
%\end{versinnote}
%
%\emph{Yogabindu} 411 (by Haribhadra)
%
%\begin{versinnote}
%\tl{utsāhān niścayād dhairyāt saṃtoṣāt tattvadarśanāt/\\+}
%\tl{muner janapadatyāgāt ṣaḍbhir yogaḥ prasidhyati//\\!}
%\end{versinnote}
%
\end{sources}
%</sc16>

%<*ts16>
\begin{testimonia}[hp01_016]
\emph{Yogacintāmaṇi} f.\,49r (\attr HP)

\begin{variants}
 saṃtoṣāt tattvadarśanāt~] tatvajñānād viniścayāt YCM
\end{variants}

Cf.~\emph{Haṭharatnāvalī} 1.78:
 
\begin{versinnote}
\tl{utsāhān niścayād dhairyāt tattvajñānārthadarśanāt/\\+}
\tl{bindusthairyān mitāhārāj janasaṅgavivarjanāt//\\+}
\tl{nidrātyāgāj jitaśvāsāt pīṭhasthairyād anālasāt/\\+}
\tl{gurvācāryaprasādāc ca ebhir yogas tu sidhyati//\\!} 
\end{versinnote}
\begin{appinnote}
\tl{\textbf{a} niścayād~] niścalād- P,T \\!}
\end{appinnote}

%emph{Yogacintāmaṇi} f.\,49r (\attr HP)
%
%\begin{versinnote}
%\tl{utsāhāt sāhasād dhairyāt tatvajñānād viniścayāt/\\+}
%\tl{janasaṅgaparityāgāt ṣaḍbhir yogaḥ prasidhyati//\\!}
%\end{versinnote}
%

\end{testimonia}
%</ts16>

%<*cm16>
\begin{philcomm}[hp01_016]    
\textalpha\ and several other groups of manuscripts have \emph{tattvajñānārthadarśanāt} (as found in \emph{Bhagavadgītā} 13.11), \emph{tattvajñānāc ca darśanāt} or something very similar in the second \emph{pāda} of the verse, but \emph{darśanāt} (\alphaOne) by itself is problematic: a vision of what? The early sources of this verse, in particular the \emph{Śivadharmottara}, indicate that the second verse quarter read as \emph{santoṣāt tattvadarśanāt}, which makes much better sense of the word \emph{darśanāt} so we have adopted that reading. \lb
%However it seems likely that before the time of Svātmārāma other versions of this verse were circulating in which \emph{santoṣāt} was not found, \emph{niścayāt} had moved from the first to second verse quarter, \emph{tattvadarśanāt} had become \emph{tattvajñānāt} and \emph{sāhasāt} was introduced. 

It should also be noted that the word \emph{tattva} could have a more specific meaning in the \emph{Haṭhapradīpikā} (4.32–33) as Svātmā\-rāma states that it is a synonym of \emph{samādhi}. In other yoga texts, it can sometimes refer to the practices of yoga (e.g. \emph{tritattva} in \emph{Amṛtasiddhi} 13.12, 14.2--3) or, more generally, to the highest reality or truth (e.g. \emph{Amanaska} 1.2, 1.20–21, 2.17 etc.).  
\end{philcomm}
%</cm16>

% MD 2024-8: Are we really remaining with the reading sāhasād of gamma/delta? The compound tattvajñānārthadarśana occurs in Bhagavadgītā 13.11. Does it not work either? In BhG it is a single thing and we need two.

%%%%%%%%%%
\subsection*{1.16*1--2}
%<*tr16-1>
\begin{translation}[hp01_016_1]
The ten observances are non-violence, truthfulness, not stealing, celibacy, patience, resolve, compassion, sincerity, moderate eating, and cleanliness.
\end{translation}
%</tr16-1>
%<*tr16-2>
\begin{translation}[hp01_016_2]
[The ten rules are] asceticism, contentment, piety, charity, worship of God, listening to doctrinal teachings, compunction, contemplation, mantra recitation, and making offerings into a fire .
\end{translation}
%</tr16-2>

%<*sc16-2>
% \begin{sources}[hp01_016_2]% for both
% \end{sources}
%</sc16-2>

%<*ts16-2>
% \begin{testimonia}[hp01_016_2]
% \end{testimonia}
%</ts16-2>

%<*cm16-2>
% \begin{philcomm}[hp01_016_2]
% \end{philcomm}
%</cm16-2>

%%%%%%%%%%
% \subsection*{1.17 heading}
% athāsanāni/
%<*tr17a>
% \begin{translation}[hp01_017a]
% \end{translation}
%</tr17a>

%<*cm17a>
% \begin{philcomm}[hp01_017a]
% \end{philcomm}
%</cm17a>

%%%%%%%%%%
\subsection*{1.17}
%<*tr17>
\begin{translation}[hp01_017]
Because it is the first auxiliary of Haṭha, \emph{āsana} is taught first. This type of \emph{āsana} brings about steadiness, good health and physical fitness.
\end{translation}
%</tr17>

%<*ts17>
\begin{testimonia}[hp01_017]
  \emph{Haṭharatnāvalī} 3.5, \emph{Yogacintāmaṇi} f.\,84r (\attr HP)

  \begin{variants}
    pūrvam ucyate YCM~] darśyate mayā HRĀ\sep
    āsanaṃ HRĀ~] āsana YCM\sep
    pāṭavam HRĀ~] lāghavaṃ YCM
    \end{variants}
  
%\emph{Haṭharatnāvalī} 3.5
%    
%\begin{versinnote}
%\tl{haṭhasya prathamāṅgatvād āsanaṃ darśyate mayā/\\+}
%\tl{tat kuryād āsanaṃ sthairyam ārogyaṃ cāṅgapāṭavam// \\!}
%\end{versinnote}
%
%\emph{Yogacintāmaṇi} 84r (\attr HP)
%
%\begin{versinnote}
%\tl{haṭhasya prathamāṅgatvād āsanaṃ pūrvam ucyate/\\+}
%\tl{tat kuryād āsanasthairyam ārogyaṃ cāṅgalāghavaṃ//\\!}
%\end{versinnote}

\end{testimonia}
%</ts17>

%<*cm17>
\begin{philcomm}[hp01_017]        
The reading \emph{aṅgapāṭavam} is attested among many of the early manuscripts, including the \textalpha\ group. Although this compound rarely appears in other yoga texts, a similar term, \emph{śarīrapāṭava}, occurs in the \emph{Śivasaṃhitā} (2.35) as one of the benefits bestowed by digestive fire (\emph{vaiśvānarāgni}), which indicates that the word \emph{pāṭava} was used in relation to the body and the benefits of yoga. The compound \emph{aṅgapāṭava} seems to imply the optimal functioning of the body. The variant reading, \emph{aṅgalāghava} (‘lightness of the limbs’ or ‘dexterity’) is more common in yoga texts and similar formulations occur even in works known to Svātmārāma, such as the \emph{Dattātreyayogaśāstra} (\emph{śarīralaghutā}) and the \textit{Amanaska} ([...] \emph{laghutvaṃ ca śarīrasyopajāyate}). It appears that the less common term \emph{aṅgapāṭavam} was changed to the more widely used notion of \emph{aṅgalāghava}, perhaps early in the transmission, as the latter is attested by manuscripts in several early groups (i.e.~\textgamma, \textdelta, and \textpi).
\end{philcomm}
%</cm17>

%%%%%%%%%%
\subsection*{1.18}
%<*tr18>
\begin{translation}[hp01_018]
I shall now teach some of the postures which have been accepted by sages  such as Vasiṣṭha and yogis such as Matsyendra.
\end{translation}
% JB: shouldn't it be kāni cin in the edition? JM: what, with a space? Mitsuyo isn't a fan of spaces ;-)
%</tr18>

%<*ts18>
\begin{testimonia}[hp01_018]
  \emph{Haṭharatnāvalī} 3.6, \emph{Yogacintāmaṇi} f.\,84r (\attr HP)

\begin{variants}
    kathyante~] lakṣyante HRĀ, vakṣyante YCM
\end{variants}
  
%\emph{Haṭharatnāvalī} 3.6
%
%\begin{versinnote}
%\tl{vasiṣṭhādyaiś ca munibhir matsyendrādyaiś ca yogibhiḥ//\\+}
%\tl{aṃgīkṛtāny āsanāni lakṣyante kāni cin mayā//\\!}
%\end{versinnote}
%
%\emph{Yogacintāmaṇi} 84r
%\begin{versinnote}
%\tl{haṭhapradīpikāyām—\\+}
%\tl{vasiṣṭhādyaiś ca munibhir matsyendrādyaiś ca yogibhiḥ/\\+}
%\tl{aṅgīkṛtāny āsanāni vakṣyante kāni cin mayā//\\!}
%\end{versinnote}
%
\end{testimonia}
%</ts18>

%<*cm18>
\begin{philcomm}[hp01_018]        
On the historical implications of these two traditions of postural practice in early Haṭhayoga, see Mallinson 2016: 119--122 and Birch 2018a: 45–46.
\end{philcomm}
%</cm18>

\begin{metre}[hp01_018]
Anuṣṭubh (a: na-vipulā; c: ra-vipulā)
\end{metre}

%%%%%%%%%%
\subsection*{1.19}
%<*tr19>
\begin{translation}[hp01_019]
Placing the soles of both feet well between the knees and thighs [and] sitting up with the body straight: they call that the auspicious pose (\emph{svastikāsana}).%JM: correctly is tricky. It could mean thoroughly, completely — I've looked at some other usages of samyak and that understanding seems to work well quite often, as it does here. E.g. VS 3.37ab hastābhyāṃ bandhayet samyak karṇādikaraṇāni ca, VM 58cd rasānāṃ śoṣanaṃ samyak mahāmudrābhidhīyate. Or Brahmānanda ad 1.35 susthiram samyak sthiraṃ
% JB thoroughly doesn't make sense here. And is it possible to 'completely' place the feet between the knees and thighs?
\end{translation}
%</tr19>

%<*sc19>
\begin{sources}[hp01_019]
  \emph{Śāradātilaka} 25.12, \emph{Vasiṣṭhasaṃhitā} 1.68, \emph{Yogayājñavalkya} 3.3

  \begin{variants}
    antare ŚT YY~] antaraṃ VS\sep
    ṛjukāyaḥ YY~] ṛjukāyo ŚT ṛjukāyas VS\sep
    samāsīnaḥ~] sukhāsīnaḥ YY, tathāsīnaḥ VS, viśed yogī ST
  \end{variants}
%
%\emph{Śāradātilaka} 25.12
%
%\begin{versinnote}
%\tl{jānūrvor antare samyak kṛtvā pādatale ubhe/\\+}
%\tl{ṛjukāyo viśed yogī svastikaṃ tat pracakṣate//\\!}
%\end{versinnote}
%
%\emph{Vasiṣṭhasaṃhitā} 1.68
%
%\begin{versinnote}
%\tl{jānūrvor antaraṃ samyak kṛtvā pādatale ubhe/\\+}
%\tl{ṛjukāyas tathāsīnaḥ svastikaṃ tat pracakṣate//\\!}
%\end{versinnote}
%
%\emph{Yogayājñavalkya} 3.3
%
%\begin{versinnote}
%\tl{jānūrvor antare samyak kṛtvā pādatale ubhe\\+}
%\tl{ṛjukāyaḥ sukhāsīnaḥ svastikaṃ tat pracakṣate\\!}
%\end{versinnote}
%
\end{sources}
%</sc19>

%<*ts19>
\begin{testimonia}[hp01_019]
\emph{Haṭharatnāvalī} 3.52, \emph{Yogacintāmaṇi} f.\,83v (\attr Yājñavalkya)

\begin{variants}
  antare YCM~] antaraṃ HRĀ\sep
  pādatale YCM~] padatale HRĀ\sep
  ṛjukāyaḥ YCM~] ṛjukāya HRĀ
\end{variants}
%
%\emph{Haṭharatnāvalī} 3.52
%
%\begin{versinnote}
%\tl{atha svastikāsanam---\\+}
%\tl{jānūrvor antaraṃ samyak kṛtvā padatale ubhe//\\+}
%\tl{ṛjukāyasamāsīnaḥ svastikaṃ tat pracakṣate// \\!}
%\end{versinnote}
%
%\emph{Yogacintāmaṇi} f.\,83v (attr. Yājñavalkya)
%
%\begin{versinnote}
%\tl{yājñavalkyaḥ—\\+}
%\tl{jānūrvor antare samyak kṛtvā pādatale ubhe/\\+}
%\tl{ṛjukāyaḥ samāsīnaḥ svastikaṃ tat pracakṣate//\\!}
%\end{versinnote}

\end{testimonia}
%</ts19>

%<*cm19>
\begin{philcomm}[hp01_019]        
One might wonder how the soles of the feet could be placed between the knees and thighs. Brahmānanda explains that the region of the shank near the knee should be understood by the word ‘knee’ in this verse (\emph{atra jānuśabdena jānusaṃnihito jaṅghāpradeśo grāhyaḥ jānusaṃnihito jaṅghāpradeśaḥ}). This is consistent with the earliest known description of \emph{svastikāsana}, found in the \emph{Pātañjala\-yoga\-śāstra\-vivaraṇa} (2.46), which states that the big toe of one foot is tucked in between the shank and thigh of the other so it is not seen (\emph{dakṣiṇaṃ pādāṅguṣṭhaṃ savyenorujaṅghena parigṛhyādṛśyaṃ kṛtvā tathā savyaṃ pādāṅguṣṭhaṃ dakṣiṇenorujaṅghenādṛśyaṃ parigṛhya yathā ca pārṣṇibhyāṃ vṛṣaṇayor apīḍaṇaṃ tathā yenāste tat svastikam āsanam}). For a discussion of \emph{svastikāsana} in the Pātañjalayoga tradition, see Maas 2018: 68–69. The descriptions of \emph{svastikāsana} in early Śaiva Tantras do not mention the inserting of the toes between the knees and thighs (see Goodall 2004: 348–350, fn. 371).
\end{philcomm}
%</cm19>

%%%%%%%%%%
\subsection*{1.20}
%<*tr20>
\begin{translation}[hp01_020]
[The yogi] should place his right heel on the left side of the [lower] back, and the left [heel] on the right [side], in the same way. [This is] the cow-faced [pose] (\emph{gomukhāsana}), which [looks] like a cow's face.
\end{translation}
%</tr20>

%<*sc20>
\begin{sources}[hp01_020]
\emph{Vasiṣṭhasaṃhitā} 1.70, \emph{Yogayājñavalkya} 3.5cd–3.6ab

\begin{variants}
  niyojayet~] niveśayet VS YY\sep
  gomukhaṃ gomukhaṃ yathā YY~] gomukhaṃ tat pracakṣate VS
\end{variants}

Cf.~\emph{Ahirbudhnyasaṃhitā} 31.45cd–46

\begin{versinnote}
\tl{ubhayor gulphayoḥ kṛtvā pṛṣṭhapārśvāv ubhāv api//\\+}
\tl{vyutkrameṇātha pāṇibhyāṃ vinyastābhyāṃ vigṛhya ca/\\+}
\tl{pṛṣṭhagābhyāṃ padāṅguṣṭhāv etad gomukham ucyate// \\!}
\end{versinnote}
%
%\emph{Vasiṣṭhasaṃhitā} 1.70
%
%\begin{versinnote}
%\tl{savye dakṣiṇagulphaṃ tu pṛṣṭhapārśve niveśayet/\\+}
%\tl{dakṣiṇe 'pi tathā savyaṃ gomukhaṃ tat pracakṣate//\\!}
%\end{versinnote}
%
%\emph{Yogayājñavalkya} 3.5cd–3.6ab
%
%\begin{versinnote}
%\tl{savye dakṣiṇagulphaṃ tu pṛṣṭhapārśve niveśayet\\+}
%\tl{dakṣiṇe 'pi tathā savyaṃ gomukhaṃ gomukhaṃ yathā\\!}
%\end{versinnote}
%
\end{sources}
%</sc20>

%<*ts20>
\begin{testimonia}[hp01_020]
\emph{Haṭharatnāvalī} 3.53, \emph{Yogacintāmaṇi} f.\,83v (\attr Yājñavalkya)

\begin{variants}
  niyojayet HRĀ~] niveśayet YCM\sep
  gomukhaṃ yathā YCM~] gomukhāsanam HRĀ
\end{variants}
  
%\emph{Haṭharatnāvalī} 3.53
%
%\begin{versinnote}
%\tl{atha gomukhāsanam---\\+}
%\tl{savye dakṣiṇagulphaṃ tu pṛṣṭhapārśve niyojayet//\\+}
%\tl{dakṣiṇe 'pi tathā savyaṃ gomukhaṃ gomukhāsanam//\\!}
%\end{versinnote}
%
%\emph{Yogacintāmaṇi} f.\,83v (attr. Yājñavalkya)
%
%\begin{versinnote}
%\tl{savye dakṣiṇagulphaṃ tu pṛṣṭhapārśve niveśayet/\\+}
%\tl{dakṣiṇe 'pi tathā savyaṃ gomukhaṃ gomukhaṃ yathā//\\!}
%\end{versinnote}
\end{testimonia}
%</ts20>


%<*cm20>
\begin{philcomm}[hp01_020]        
This posture first appears in some Vaiṣṇava \emph{Saṃhitā}s that predate the \emph{Haṭhapradīpikā}, including the \emph{Ahirbudhnyasaṃhitā} and the \emph{Vasiṣṭhasaṃhitā}, which is likely to have been the source of this verse. The position of the ankles is the same in all the source texts. The \emph{Ahirbudhnyasaṃhitā} adds that the hands are crossed behind the back and hold the big toes. For illustrations of six possible positions of the arms and hands, see Gharote, Jha, Devnath, Sakhalkar 2006: 111–113.
\end{philcomm}
%</cm20>

%%%%%%%%%%
\subsection*{1.21}
%<*tr21>
\begin{translation}[hp01_021]
 Fixing one foot on one thigh and placing the [other] thigh on the other foot is called the hero pose (\emph{vīrāsana}).
\end{translation}
%</tr21>

%<*sc21>
\begin{sources}[hp01_021]
\emph{Vasiṣṭhasaṃhitā} 1.72, \emph{Yogayājñavalkya} 3.8

\begin{variants}
  vinyasyoruṇi YY~] vinyasyorau VS\sep
  tathā coruṃ YY~] tathaivoruṃ VS\sep
  itīritam VS~] udāhṛtam YY
\end{variants}
  
%\emph{Vasiṣṭhasaṃhitā} 1.72
%
%\begin{versinnote}
%\tl{ekaṃ pādam athaikasmin vinyasyorau ca saṃsthitam/\\+}
%\tl{itarasmiṃs tathaivoruṃ vīrāsanam itīritam//\\!}
%\end{versinnote}

Cf.~\emph{Śāradātilakatantra} 25.15cd–16ab

\begin{versinnote}
\tl{ekaṃ pādam adhaḥ kṛtvā vinyasyorau tathetaram//\\+}
\tl{ṛjukāyo viśed yogī vīrāsanam itīritam/\\!}
\end{versinnote}

%\emph{Yogayājñavalkya} 3.8
%
%\begin{versinnote}
%\tl{ekaṃ pādam athaikasmin vinyasyoruṇi saṃsthitam/\\+}
%\tl{itarasmiṃs tathā coruṃ vīrāsanam udāhṛtam//\\!}
%\end{versinnote}
%
\end{sources}
%</sc21>

%<*ts21>
\begin{testimonia}[hp01_021]
\emph{Haṭharatnāvalī} 3.54, \emph{Yogacintāmaṇi} f.\,83v (\attr Yājñavalkya)

\begin{variants}
  vinyasyoruṇi saṃsthitam~] vinyasyoruṇi saṃsthitaḥ YCM, vinyased ūruṇi sthiram HRĀ, vinyased ūruṇi sthitam HRĀ\vl\sep
  itīritam HRĀ~] udāhṛtam YCM
  \end{variants}
  
%\emph{Haṭharatnāvalī} 3.54
%
%\begin{versinnote}
%\tl{atha vīrāsanam---\\+}
%\tl{ekaṃ pādam athaikasmin vinyased ūruṇi sthiram//\\+}
%\tl{itarasmiṃs tathā coruṃ vīrāsanam itīritam//\\!} 
%\end{versinnote}
%\begin{appinnote}
%\tl{\textbf{b} sthiram~] sthitam T \\!}
%\end{appinnote}
%
%\emph{Yogacintāmaṇi} f.\,83v (attr. Yājñavalkya)
%
%\begin{versinnote}
%\tl{ekaṃ pādam athaikasmin vinyasyoruṇi saṃsthitaḥ/\\+}
%\tl{itarasmiṃs tathā coruṃ vīrāsanam udāhṛtam//\\!}
%\end{versinnote}
%
\end{testimonia}
%</ts21>


%<*cm21>
\begin{philcomm}[hp01_021]   
Although most witnesses have \emph{tathā} in 1.21a, the word \emph{atha} has been accepted because it is attested by \alphaThree\ and \etaOne, the sources and the testimonia. It appears to be verse filler here rather than indicating a temporal sequence of actions. Svātmārāma borrowed the verse on \emph{vīrāsana} from the \emph{Vasiṣṭhasaṃhitā}, the redactor of which appears to have adapted its first line from a description of this posture in the \emph{Śāradātilakatantra}. This would explain the rather strange syntax of the \emph{Vasiṣṭhasaṃhitā}’s version, in which \emph{adhaḥ kṛtvā} was changed to \emph{athaikasmin}, and \emph{tathetaram} became \emph{ca saṃsthitam}. It seems that \emph{saṃsthitaṃ} must be understood with \emph{ūruṃ} in the third \emph{pāda} in the sense of \emph{saṃsthāpya} (i.e.~‘having placed’).\lb

Different versions of \emph{vīrāsana} are found in earlier tantric and yogic works, such as the \emph{Kiraṇatantra}, Hemacandra’s \emph{Yogaśāstra} and commentaries on the \emph{Pātañjalayogaśāstra}. For a discussion of some of these, see Maas 2018: 66–68.
\end{philcomm}
%</cm21>

%%%%%%%%%%
\subsection*{1.22}
%<*tr22>
\begin{translation}[hp01_022]
Knowers of yoga know that the tortoise pose (\emph{kūrmāsana}) arises by carefully pressing the anus with the ankles crossed. 
\end{translation}
%</tr22>

%<*sc22>
\begin{sources}[hp01_022]
\emph{Vasiṣṭhasaṃhitā} 1.80, \emph{Ahirbudhnyasaṃhitā} 31.35

\begin{variants}
  niṣpīḍya~] nipīḍya ABS, nirudhya VS
  kūrmāsanaṃ bhaved etad VS~] etat kūrmāsanaṃ proktaṃ ABS
  iti yogavido viduḥ VS~] yogasiddhikaraṃ param ABS
  \end{variants}

%\emph{Vasiṣṭhasaṃhitā} 1.80
%
%\begin{versinnote}
%\tl{gudaṃ nirudhya gulphābhyāṃ vyutkrameṇa samāhitaḥ/\\+}
%\tl{kūrmāsanaṃ bhaved etad iti yogavido viduḥ//\\!}
%\end{versinnote}
%
%Cf.~\emph{Ahirbudhnyasaṃhitā} 31.35
%
%\begin{versinnote}
%\tl{gudaṃ nipīḍya gulphābhyāṃ vyutkrameṇa samāhitaḥ/\\+}
%\tl{etat kūrmāsanaṃ proktaṃ yogasiddhikaraṃ param// \\!}
%\end{versinnote}
%
\end{sources}
%</sc22>

\begin{testimonia}[hp01_022]
  \emph{Yogacintāmaṇi} f.\,84r (\attr HP), \emph{Yuktabhavadeva} 6.15 (\attr HP)

  \begin{variants}
    niṣpīḍya~] niyamya YCM YBhD\sep
    \end{variants}
 
%\emph{Yogacintāmaṇi} f.\,84r (\attr HP)
%
%\begin{versinnote}
%\tl{gudaṃ niyamya gulphābhyaṃ vyutkrameṇa samāhitaḥ/\\+}
%\tl{kūrmāsanaṃ bhaved etad iti yogavido viduḥ//\\!}
%\end{versinnote}
%
%\emph{Yuktabhavadeva} 6.15
%
%\begin{versinnote}
%\tl{haṭhapradīpikāyām\\+}
%\tl{gudaṃ niyamya gulphābhyāṃ vyutkrameṇa samāhitaḥ/\\+}
%\tl{kūrmāsanaṃ bhaved etad iti yogavido viduḥ// iti kūrmāsanam//\\!}
%\end{versinnote}
%
\end{testimonia}
%</ts22>

%<*cm22>
\begin{philcomm}[hp01_022]   
In the first quarter of the verse, the witnesses are split between \emph{nirudhya} (‘having blocked’), \emph{nibadhya} (‘having bound’), \emph{niyamya} (‘having restrained’) and \emph{niṣpīḍya} (‘having pressed’). The source, the \emph{Vasiṣṭhasaṃhitā}, and two manuscripts of the \textgamma\ and \textpi\ groups support \emph{nirudhya} whereas \alphaTwo\ and the testimonia support \emph{niyamya} and \alphaThree\ has \emph{niṣpīḍya}. While \emph{nirudhya} makes sense here (i.e.~`having blocked or closed the anus...'), we have adopted \emph{niṣpīḍya} because it is better attested among the \textalpha, \textepsilon\ and \texteta\ groups.\lb

The word \emph{vyutkrameṇa} describes the position of the ankles. Its basic meaning is ‘against the normal direction.’ In \emph{āsana} descriptions it usually means ‘crossed’ (see e.g.~\emph{Vasiṣṭhasaṃhitā} 1.71), which is how we have understood it here. It could also mean ‘turned out’: if the yogi is in a kneeling-type position, turning the feet out would bring the ankles together, blocking the perineal area. See \emph{Yoga Mīmāṃsā}, vol. 8, no. 2 (1965: 29–30) for a discussion of \emph{vyutkrameṇa} and the position of the ankles in \emph{kūrmāsana}, and vol. 8, no. 2, figures 3–6 for photographs of a practitioner performing this \emph{āsana}. 
\end{philcomm}
%</cm22>

%%%%%%%%%%
\subsection*{1.23}
%<*tr23>
\begin{translation}[hp01_023]
[The yogi] correctly assumes the lotus pose, inserts the hands between the knees and thighs, places [the hands] on the ground, and remains in the air. This is the wild cock pose (\emph{kukkuṭāsana}).
\end{translation}
%</tr23>

%% JH "the lotus pose" for "lotus pose"
% JB sometimes we translate as lotus pose, sometimes as \emph{padmāsana}. Should we make this consistent?

%<*sc23>
\begin{sources}[hp01_023]
  
\emph{Vasiṣṭhasaṃhitā} 1.78, cf. ~\emph{Ahirbudhnyasaṃhitā} 31.38

\begin{variants}
  susaṃsthāpya~] samāsthāya VS, adhiṣṭhāya ABS\sep
  jānūrvor antare karau VS~] jānvantaraviniḥsṛtau ABS\sep
  niveśya bhūmau saṃsthāpya VS\vl~] bhūmau niveśya saṃsthāpya VS, karau bhūmau niveśyaitad ABS\sep
  vyomasthaḥ~] vyomasthaṃ VS ABS
  \end{variants}
  
%\emph{Vasiṣṭhasaṃhitā} 1.78
%
%\begin{versinnote}
%\tl{padmāsanaṃ samāsthāya jānūrvor antare karau/\\+}
%\tl{bhūmau niveśya saṃsthāpya vyomasthaṃ kukkuṭāsanam//\\!}
%\end{versinnote}
%\begin{appinnote}
%\tl{\textbf{c} niveśya bhūmau – mss. la, va, śa \\!}
%\end{appinnote}
%
%Cf.~\emph{Ahirbudhnyasaṃhitā} 31.38
%
%\begin{versinnote}
%\tl{kukkuṭāsanam\\+}
%\tl{padmāsanam adhiṣṭhāya jānvantaraviniḥsṛtau/\\+}
%\tl{karau bhūmau niveśyaitad vyomasthaṃ kukkuṭāsanam// \\!}
%\end{versinnote}
%
\end{sources}
%</sc23>

%<*ts23>
\begin{testimonia}[hp01_023]
\emph{Haṭharatnāvalī} 3.73, \emph{Yogacintāmaṇi} f.\,84r (\attr HP), 
\emph{Yuktabhavadeva} 6.16 (\attr HP)

\begin{variants}
  susaṃsthāpya HRĀ~] tu saṃyojya YCM YBhD\sep
  vyomasthaḥ HRĀ~] vyomasthaṃ YCM YBhD
\end{variants}

%\emph{Haṭharatnāvalī} 3.73
%
%\begin{versinnote}
%\tl{atha kukkuṭāsanam---\\+}
%\tl{padmāsanaṃ susaṃsthāpya jānūrvor antare karau/\\+}
%\tl{niveśya bhūmau saṃsthāpya vyomasthaḥ kukkuṭāsanam//\\!}
%\end{versinnote}
%
%\emph{Yogacintāmaṇi} f.\,84r (\attr HP)
%
%\begin{versinnote}
%\tl{padmāsanaṃ tu saṃyojya jānūrvor antare karau/\\+}
%\tl{niveśya bhūmau saṃsthāpya vyomasthaṃ kukkuṭāsanam//\\!}
%\end{versinnote}
%
%\emph{Yuktabhavadeva} 6.16 (\attr HP)
%
%\begin{versinnote}
%\tl{padmāsanaṃ tu saṃyojya jānūrvor antare karau/\\+}
%\tl{niveśya bhūmau saṃsthāpya vyomasthaṃ kukkuṭāsanam//\\+}
%\tl{iti kukkuṭāsanam//\\!}
%\end{versinnote}

\end{testimonia}
%</ts23>

%<*cm23>
\begin{philcomm}[hp01_023]
The names \emph{kurkuṭa} and \emph{kurkkuṭa} in some manuscripts are variant spellings of \emph{kukkuṭa} attested in the \emph{Pañcatantra} (MW).
\end{philcomm}
%</cm23>

\begin{metre}[hp01_023]
Anuṣṭubh (c: ma-vipulā)
\end{metre}

%%%%%%%%%%
\subsection*{1.24}
%<*tr24>
\begin{translation}[hp01_024]
While in the wild cock pose, [the yogi] binds the neck with the hands and lies [on his back] upturned like a tortoise. This is the upturned tortoise (\emph{uttānakūrmaka}).
\end{translation}
%</tr24>

%<*ts24>
\begin{testimonia}[hp01_024]  
  \emph{Haṭharatnāvalī} 3.74, \emph{Yogacintāmaṇi} f.\,84r (\attr HP), \emph{Yuktabhavadeva} 6.17 (\attr HP)

  \begin{variants}
    saṃbadhya YCM~] sambadhya HRĀ YBhD\sep
    uttāna etad~] uttānam etad HRĀ YCM YBhD
    \end{variants}
  
%\emph{Haṭharatnāvalī} 3.74
%
%\begin{versinnote}
%\tl{kukkuṭāsanabandhastho dorbhyāṃ sambadhya kandharām//\\+}
%\tl{śete kūrmavad uttānam etad uttānakūrmakam// 74//\\!}
%\end{versinnote}
%
%\emph{Yogacintāmaṇi} f.\,84r (\attr HP)
%
%\begin{versinnote}
%\tl{kukkuṭāsanabandhastho dorbhyāṃ saṃbadhya kandharām/\\+}
%\tl{bhavet kūrmavad uttānam etad uttānakūrmakam//\\!}
%\end{versinnote}
%
%\emph{Yuktabhavadeva} 6.17 (\attr HP)
%
%\begin{versinnote}
%\tl{kukkuṭāsanabandhastho dorbhyāṃ sambadhya kandharām/\\+}
%\tl{śete kūrmavad uttānam etad uttānakūrmakam//\\+}
%\tl{iti uttānakūrmāsanam//\\!}
%\end{versinnote}
%
\end{testimonia}
%</ts24>

%<*cm24>
\begin{philcomm}[hp01_024]
The oldest dated manuscript, \etaOne, has \emph{kukkuṭāsanavat kṛtvā}, which is a simpler alternative to the widely attested reading \emph{kukkuṭāsanabandhasthaḥ} (including \alphaTwo\ and \alphaThree), which we have accepted. Since there is no known source for this verse other than the \emph{Haṭhapradīpikā}, it appears that the reading of \etaOne\ was an isolated attempt to simplify the syntax.
\end{philcomm}
%</cm24>

%%%%%%%%%%
\subsection*{1.25}
%<*tr25>
\begin{translation}[hp01_025]
Clasping the big toes with the hands and performing the action of drawing a bow as far as the ear is called the bow pose (\emph{dhanurāsana}).
\end{translation}
%</tr25>

%<*ts25>
\begin{testimonia}[hp01_025]
  \emph{Haṭharatnāvalī} 3.51, \emph{Yogacintāmaṇi} f.\,84r (\attr HP), \emph{Yuktabhavadeva} 6.18 (\attr HP)

  \begin{variants}
    tu HRĀ YBhD~] ca YCM\sep
    ākarṣaṇaṃ kṛtvā HRĀ YCM YBhD~] ākarṣaṇākṛṣṭaṃ HRĀ\vl\sep 
    ucyate HRĀ~] īritam YCM YBhD
    \end{variants}

%\emph{Haṭharatnāvalī} 3.51
%
%\begin{versinnote}
%\tl{atha dhanurāsanam---\\+}
%\tl{pādāṅguṣṭhau tu pāṇibhyāṃ gṛhītvā śravaṇāvadhi//\\+}
%\tl{dhanurākarṣaṇaṃ kṛtvā dhanurāsanam ucyate//\\!}
%\end{versinnote}
%\begin{appinnote}
%\tl{\textbf{c} ākarṣaṇaṃ kṛtvā~] ākarṣaṇākṛṣṭaṃ P,T,t1 \\!}
%\end{appinnote}
%
%\emph{Yogacintāmaṇi} f.\,84r (\attr HP)
%
%\begin{versinnote}
%\tl{pādāṅguṣṭhau ca pāṇibhyāṃ gṛhītvā śravaṇāvadhi/\\+}
%\tl{dhanurākarṣaṇaṃ kṛtvā dhanurāsanam īritam//\\!}
%\end{versinnote}
%
%\emph{Yuktabhavadeva} 6.18 (\attr HP)
%
%\begin{versinnote}
%\tl{pādāṅguṣṭhau tu pāṇibhyāṃ gṛhītvā śravaṇāvadhi/\\+}
%\tl{dhanurākarṣaṇaṃ kṛtvā dhanurāsanam īritam//\\+}
%\tl{iti dhanurāsanam//\\!}
%\end{versinnote}
%
Cf.~\emph{Haṭhayogasaṃhitā} p.\,21

\begin{versinnote}
\tl{dhanurāsanam/\\+}
\tl{prasārya pādau bhuvi daṇḍarūpau karau ca pṛṣṭhe dhṛtapādayugmau/\\+}
\tl{kṛtvā dhanustulyavivarttitāṅgaṃ nigadyate vai dhanurāsanaṃ tat// 25//\\!}
\end{versinnote}

\end{testimonia}
%</ts25>

%<*cm25>
\begin{philcomm}[hp01_025]
We have adopted the reading \emph{dhanurākarṣaṇaṃ kṛtvā}, which is in the \textdelta\ group, as well as the principal testimonia (i.e.~the \emph{Yogacintāmaṇi} and \emph{Haṭharatnāvalī}), because it fits the overall syntax of the verse, unlike the readings with \emph{kṛṣṭaṃ} (for \emph{kṛtvā}) that are found in the early manuscripts. It is curious that \emph{kṛṣṭaṃ} is so well attested because it seems redundant with \emph{ākarṣaṇaṃ}. The following reading in Godāvaramiśra's \emph{Yogacintāmaṇi} (f. 40r) appears to be an attempt to make sense of \emph{kṛṣṭaṃ}: \emph{dhanurākarṣavat kṛṣṭaṃ dhanurāsanam ucyate}.\lb
% MD: The well attested reading is dhanurākarṣaṇākṛṣṭaṃ as one word. The commentary should be slightly modified.

A different version of \emph{dhanurāsana} is described in the \emph{Haṭhayogasaṃhitā}. On the two versions of \emph{dhanurāsana}, see Hargreaves and Birch 2017.\lb
%https://www.theluminescent.org/2017/11/dhanurasana-two-versions-of-bow-pose.html

One manuscript of the \emph{Haṭhapradīpikā} (ms.~no.~30051, f.\,2v), which was consulted but not collated for this edition, has a scribal comment stating that \emph{dhanurāsana} should be done continuously (\emph{anavarata}) on the left and right sides (\emph{tatra ekam dhanurākarṣaṇāsanam āsanaṃ savyāpasavyapādahastābhyām [abhy]ased anavaratam}). This would make \emph{dhanurāsana} a dynamic practice as shown in \medialink{this video}{dhanurasana.mp4}. 
\end{philcomm}
%</cm25>

%%%%%%%%%%
\subsection*{1.26}
%<*tr26>
\begin{translation}[hp01_026]
[The yogi] should hold the right foot, which is placed at the base of the left thigh, with the [hand of] the right arm, which is wrapped around the outside of the knee, and remain [like that] with his body twisted. This posture was taught by the revered Matsyanātha.
\end{translation}
%</tr26>

%<*ts26>
\begin{testimonia}[hp01_026]
\emph{Haṭharatnāvalī} 3.57, \emph{Yogacintāmaṇi} f.\,84r (\attr HP), \emph{Yuktabhavadeva} 6.19 (\attr HP)

\begin{variants}
  dakṣapādaṃ HRĀ\vl YCM YBhD~] dakṣapādo HRĀ, dakṣapādau HRĀ\vl\sep
  jānor~] jānvor HRĀ YCM YBhD\sep
  tiṣṭhet HRĀ YCM~] tiṣṭhan YBhD
\end{variants}
  
%\emph{Haṭharatnāvalī} 3.57
%
%\begin{versinnote}
%\tl{atha matsyendrāsanam---\\+}
%\tl{vāmorumūlārpitadakṣapādo jānvor bahirveṣṭitadakṣadoṣṇā/\\+}
%\tl{pragṛhya tiṣṭhet parivartitāṅgaḥ śrīmatsyanāthoditam āsanaṃ syāt//\\!}
%\end{versinnote}
%\begin{appinnote}
%\tl{\textbf{a} °dakṣapādo~] °dakṣapādaṃ P, °dakṣapādau t1 \\!}
%\end{appinnote}
%
%\emph{Yogacintāmaṇi} f.\,84r (\attr HP)
%
%\begin{versinnote}
%\tl{vāmorumūlārpitadakṣapādaṃ
%jānvor bahirveṣṭitadakṣadoṣṇā/\\+}
%\tl{pragṛhya tiṣṭhet parivartitāṅgaḥ
%śrīmatsyanāthoditam āsanaṃ syāt//\\!}
%\end{versinnote}
%
%\emph{Yuktabhavadeva} 6.19 (\attr HP)
%
%\begin{versinnote}
%\tl{vāmorumūlārpitadakṣapādaṃ
%jānvor bahirveṣṭitadakṣadoṣṇā/\\+}
%\tl{pragṛhya tiṣṭhan parivartitāṅgaḥ
%śrīmatsyanāthoditam āsanaṃ syāt//\\!}
%\end{versinnote}
%
\end{testimonia}
%</ts26>


%<*cm26>
\begin{philcomm}[hp01_026]
In the second verse quarter, most of the manuscript groups have a compound with \emph{doṣṇā} at the end, as seen also in the \emph{Yogacintāmaṇi}, \emph{Haṭharatnāvalī} and \emph{Yuktabhavadeva}. The instrumental ending (`with the hand') works well with the gerund (\emph{pragṛhya}) in the third verse quarter and the object (\emph{dakṣpādaṃ}) in the first quarter. This reading indicates that the right foot is held by the hand of the arm that is wrapped around the outside of the left leg, which would be the right hand (\emph{dakṣadoṣṇā}) rather than the left (\emph{vāmadoṣṇā}), as shown in \medialink{Figure 1}{matsyendrasana1.jpeg}.\lb

\alphaThree\ and most manuscripts of the \textit{Haṭharatnāvalī} have \emph{dakṣapādo} in the first \emph{pāda}. This reading yields the same meaning as the adopted one if read with \emph{vāmapādaṃ} in the second. However, \alphaThree\ and manuscripts of the \textit{Haṭharatnāvalī} read \emph{vāmadoṣṇā}, which is not good because it leaves the gerund without an object.\lb

The version of this verse in \emph{Jyotsnā} (1.26), which is supported by some manuscripts in two important groups, \textpi\ and \texteta, has two objects of the gerund, namely the left and right feet, without an instrumental or conjunctive particle. In his commentarial remarks, Brahmānanda proposes that the left foot is grasped by the right hand and the right foot by the left foot, as seen in \medialink{Figure 2}{matsyendrasana2.jpg}.
\end{philcomm}
%</cm26>

\begin{metre}[hp01_026]
Upajāti
\end{metre}

%%%%%%%%%%
\subsection*{1.27}
%<*tr27>
\begin{translation}[hp01_027]
Matsyendra's seat is a missile for destroying the many chronic and painful diseases of the stomach; through practice it brings about in men the awakening of Kuṇḍalinī and steadiness of the spine.
\end{translation}
%</tr27>

%<*ts27>
\begin{testimonia}[hp01_027]
\emph{Haṭharatnāvalī} 3.58, \emph{Yuktabhavadeva} 6.20 (\attr HP), \emph{Haṭhatattvakaumudī} 7.8, 

\begin{variants}
  jaṭharapracaṇḍa HTK~] jaṭharapradīptaṃ HRĀ, jaṭharapravṛddha YCM, jaṭharapravṛddhiṃ YCM\vl, jaṭharaprabuddhaṃ YBhD\sep
  picaṇḍaruṅmaṇḍala~] pracaṇḍaruṅmaṇḍala YCM YBhD, pracaṇḍarugmaṇḍala HRĀ, ruṅmaṇḍalakhaṇḍana HTK\sep
  abhyāsataḥ HRĀ YCM HTK~] abhyasataṃ YBhD\sep
  daṇḍa HRĀ HTK YBhD~] daṇḍe YCM\sep
  ca dadāti HRĀ HTK YBhD~] pradadāti YCM
\end{variants}

% 
%\emph{Haṭharatnāvalī} 3.58%
%
%\begin{versinnote}
%\tl{matsyendrapīṭhaṃ jaṭharapradīptaṃ\\+}
%\tl{pracaṇḍarugmaṇḍalakhaṇḍanāstram/\\+}
%\tl{abhyāsataḥ kuṇḍalinīprabodhaṃ\\+}
%\tl{daṇḍasthiratvaṃ ca dadāti puṃsām//\\!}
%\end{versinnote}
%\begin{appinnote}
%\tl{\textbf{a} °pradīptaṃ~] pravṛttaṃ T,t1 °pravṛttaḥ N,n1,n3,J \\!}
%\end{appinnote}
%
%\emph{Yogacintāmaṇi} f.\,84r (\attr HP)
%
%\begin{versinnote}
%\tl{matsyendrapīṭhaṃ jaṭharapravṛddha-\\+}
%\tl{pracaṇḍaruṅmaṇḍalakhaṇḍanāstram/\\+}
%\tl{abhyāsataḥ kuṇḍalinīprabodhaṃ\\+}
%\tl{daṇḍe sthiratvaṃ pradadāti puṃsām//\\!}
%\end{versinnote}
%\begin{appinnote}
%\tl{\textbf{a} °pravṛddha~] N : °pravṛddhiṃ L \\!}
%\end{appinnote}
%
%\emph{Haṭhatattvakaumudī} 7.8
%
%\begin{versinnote}
%\tl{matsyendrapīṭhaṃ jaṭharapracaṇḍa-\\+}
%\tl{ruṅmaṇḍalakhaṇḍanakhaṇḍanāstram/\\+}
%\tl{abhyāsataḥ kuṇḍalinīprabodhaṃ\\+}
%\tl{daṇḍasthiratvaṃ ca dadāti puṃsām//\\!}
%\end{versinnote}
%
%\emph{Yuktabhavadeva} 6.20 (\attr HP)
%
%\begin{versinnote}
%\tl{matsyendrapīṭhaṃ jaṭharaprabuddhaṃ\\+}
%\tl{pracaṇḍaruṅmaṇḍalakhaṇḍanāstram/\\+}
%\tl{abhyasataṃ kuṇḍalinīprabodhaṃ\\+}
%\tl{daṇḍasthiratvaṃ ca dadāti puṃsām//\\!}
%\end{versinnote}

\end{testimonia}
%</ts27>

%<*cm27>
\begin{philcomm}[hp01_027]
Here we understand \emph{jaṭhara}, which usually refers to the stomach, to mean “tough” or “persistent” (cf.~\emph{Amarakośa} 3.3.740 \emph{jaṭharaḥ kaṭhine'pi syād adhastād api cādharaḥ}).\lb
%We have emended the reading \emph{jaṭhara} found in \emph{pāda} a in almost all witnesses to \emph{jaraṭha} on the understanding that \emph{jaraṭha} and the reading \emph{picaṇḍa} found in the alpha witnesses were original but obscure to many scribes, resulting in various corruptions including the change to \emph{jaṭhara}, which, like \emph{picaṇḍa}, means “stomach”.\lb
%The readings for the compound beginning with \emph{jaraṭha} diverge significantly and include \emph{jaṭharapravṛddha°}, \emph{jaṭharaprabuddha°}, \emph{jaṭharapradīpta°} and \emph{jaṭharapracaṇḍa°}. As descriptive compounds, none of these makes good sense in regard to Matsyendra's seat. Since the stomach or abdomen (\emph{jaṭhara}) is the first member of this compound, it seems more likely that it qualifies the terrible diseases (\emph{pracaṇḍarug}) that are mentioned in the next verse quarter, as suggested by the reading \emph{jaṭharapravṛddha°}, which is attested by \etaTwo\ and the \textit{Yogacintāmaṇi} and suggested by \gammaOne and \deltaOne (\emph{jaṭharapravuddha°}).\lb
% MD 2024-8: The best-attested reading is now jaṭhara-pracaṇḍa-picaṇḍa-ruṅ-maṇḍala°. Perhaps, jaṭhara is an adjective here like jaraṭha: "a troop of tough/persistent (jaṭhara/jaraṭha) and terrible abdominal (picaṇḍa) diseases"? >>>>>JM: yes, let's keep jaṭhara but understand it to mean tough, which is apparently possible. Then no need for a note, or rather just a short one to say we understand jaṭhara to mean tough/persistent (found in Amarakośa (3.3.740)  jaṭharaḥ kaṭhine'pi syādadhastādapi cā'dharaḥ\\

In 1.27d, the compound \emph{daṇḍasthiratvaṃ} (`steadiness of the spine') is attested by all the important manuscript groups and testimonia, so it was likely original. However, the \emph{Jyotsnā} (1.27d) has \emph{candrasthiratvaṃ} (`steadiness of the moon'), and this reading is well-attested in many manuscripts that are lower on the stemma. Brahmānanda understands steadiness here as `the absence of flow' (\emph{sthiratvaṃ kṣaraṇābhāvaṃ}), a reference to the moon retaining its nectar.
\end{philcomm}
%</cm27>

\begin{metre}[hp01_027]
Upajāti
\end{metre}

%%%%%%%%%%
\subsection*{1.28}
%<*tr28>
\begin{translation}[hp01_028]
[The yogi] should stretch out the legs on the ground [as straight] as a stick, hold the toes of both feet with the hands, and remain with the forehead placed on the knees. They call this the back-stretch (\emph{paścimatānam}).
\end{translation}
%</tr28>

%<*sc28>
\begin{sources}[hp01_028]
Cf.~\emph{Śivasaṃhitā} 3.108

\begin{versinnote}
\tl{prasārya caraṇadvandvaṃ parasparasusaṃyutam/\\+}
\tl{svapāṇibhyāṃ dṛḍhaṃ dhṛtvā jānūpari śiro nyaset//\\!}
\end{versinnote}

\end{sources}
%</sc28>

%<*ts28>
\begin{testimonia}[hp01_028]
  \emph{Haṭharatnāvalī} 3.66, \emph{Yogacintāmaṇi} f.\,84r (\attr HP), \emph{Yuktabhavadeva} 6.22 (\attr HP)

  \begin{variants}
    dorbhyāṃ HRĀ YBhD~] dvābhyāṃ HRĀ\vl, YCM\sep
    padāgradvitayaṃ HRĀ~] ca pādadvitayaṃ YCM YBhD, karābhyāṃ dvitayaṃ HRĀ\vl\sep
    lalāṭadeśo HRĀ YCM~] lalāṭapaṭṭo YBhD\sep
    vased idaṃ HRĀ~] 'bhyased idaṃ YCM, nyased idaṃ YBhD
    \end{variants}
  
%\emph{Haṭharatnāvalī} 3.66
%
%\begin{versinnote}
%\tl{atha paścimatānāsanam---\\+}
%\tl{prasārya pādau bhuvi daṇḍarūpau\\+}
%\tl{dorbhyāṃ padāgradvitayaṃ gṛhītvā/\\+}
%\tl{jānūpari nyastalalāṭadeśo\\+}
%\tl{vased idaṃ paścimatānam āhuḥ//\\!}
%\end{versinnote}
%\begin{appinnote}
%\tl{\textbf{b} dorbhyāṃ padāgradvitayaṃ~] dvābhyāṃ karābhyāṃ dvitayaṃ n1,n3 \\!}
%\end{appinnote}
%
%\emph{Yogacintāmaṇi} f.\,84r (\attr HP)
%
%\begin{versinnote}
%\tl{prasārya pādau bhuvi daṇḍarūpau\\+}
%\tl{dvābhyāṃ ca pādadvitayaṃ gṛhītvā/\\+}
%\tl{jānūpari nyastalalāṭadeśo\\+}
%\tl{'bhyased idaṃ paścimatānam āhuḥ//\\!}
%\end{versinnote}
%
%\emph{Yuktabhavadeva} 6.22 (\attr HP)
%
%\begin{versinnote}
%\tl{prasārya pādau bhuvi daṇḍarūpau\\+} 
%\tl{dorbhyāṃ ca pādadvitayaṃ gṛhītvā/\\+}
%\tl{jānūpari nyastalalāṭapaṭṭo\\+} 
%\tl{nyased idaṃ paścimatānam āhuḥ//\\!}
%\end{versinnote}

\end{testimonia}
%</ts28>

%<*cm28>
\begin{philcomm}[hp01_028]
The reading \emph{dorbhyāṃ padāgradvitayaṃ} is well attested but is somewhat strange because \emph{dos} usually means `the arm' rather than the hands. The variant \emph{dvābhyāṃ karā\-bhyāṃ dvitayaṃ}, “with both hands”, appears to be an attempt to remove \emph{dorbhyāṃ}, but it introduces the problem of the toes not being mentioned.
\end{philcomm}
%</cm28>

\begin{metre}[hp01_028]
Upajāti
\end{metre}

%%%%%%%%%%
\subsection*{1.29}
%<*tr29>
\begin{translation}[hp01_029]
Foremost among \emph{āsana}s, the back-stretch thus makes the breath flow to the back [of the body] (i.e.~in the central channel), increases the digestive fire, makes the belly thin and prevents diseases in men.
\end{translation}
%JB rear? ... makes the breath flow to the back [of the body] (i.e.~in the central channel),..
%</tr29>

%<*sc29>
\begin{sources}[hp01_029]
Cf.~\emph{Śivasaṃhitā} 3.109–110

\begin{versinnote}
\tl{āsanāgryam idaṃ proktaṃ jaṭharānaladīpanam/\\+}
\tl{dehāvasādaharaṇaṃ paścimottānasaṃjñakam//\\+}
\tl{ya etad āsanaṃ śreṣṭhaṃ pratyahaṃ sādhayet sudhīḥ/\\+}
\tl{vāyuḥ paścimamārgeṇa tasya saṃcarati dhruvam//\\!}
\end{versinnote}

\end{sources}
%</sc29>

%<*ts29>
\begin{testimonia}[hp01_029]
\emph{Haṭharatnāvalī} 3.67, \emph{Yogacintāmaṇi} f.\,84r (\attr HP)

\begin{variants}
  arogatāṃ HRĀ~] arogitāṃ YCM
  \end{variants}
  
%\emph{Haṭharatnāvalī} 3.67
%
%\begin{versinnote}
%\tl{iti paścimatānam āsanāgryaṃ\\+} 
%\tl{pavanaṃ paścimavāhinaṃ karoti/\\+}
%\tl{udayaṃ jaṭharānalasya kuryād\\+} 
%\tl{udare kārśyam arogatāṃ ca puṃsām//\\!}
%\end{versinnote}
%
%\emph{Yogacintāmaṇi} f.\,84r (\attr HP)
%
%\begin{versinnote}
%\tl{iti paścimatānam āsanāgryaṃ\\+}
%\tl{pavanaṃ paścimavāhinaṃ karoti/\\+}
%\tl{udayaṃ jaṭharānalasya kuryād\\+}
%\tl{udare kārśyam arogitāṃ ca puṃsām//\\!}
%\end{versinnote}

\end{testimonia}
%</ts29>


%<*cm29>
\begin{philcomm}[hp01_029]
 The use of the word \emph{paścima} to mean the central channel is found at \emph{Yogabīja} 95 (\emph{paścimamārgataḥ}), 108 (\emph{paścime pathi}), 117 (\emph{paścimadvāramārgeṇa}) and 121 (\emph{paścimaṃ}). Cf.\,the usages of \emph{paścima\-mārga} in \sl{Dattātreyayogaśāstra} 140 and \emph{Śivasaṃhitā} 3.110 (from which this verse is likely to be derived). Brahmānanda understands \emph{paścima} as referring to the Suṣumṇā (\emph{Jyotsnā} 1.29): \emph{paścimavāhinaṃ paścimena paścimamārgeṇa suṣumṇāmārgeṇa vahatīti paścimavāhī}.
\end{philcomm}
%</cm29>

\begin{metre}[hp01_029]
Śiśulīlā 
\end{metre}

%%%%%%%%%%
\subsection*{1.30}
%<*tr30>
\begin{translation}[hp01_030]
Supporting oneself on the ground with both palms, the elbows placed on either side of the navel, lifted up into the air in a raised posture [as straight] as a stick: they call this posture the peacock.
%Maybe we should adopt karadvayābhyāṃ in 30a. It is supported by alpha and explains many other variants ending in abhyāṃ, and perhaps its strangeness suggests it was original. JM: yes, but  karasthalābhyāṃ of alpha3 is even better and reflects the °talābhyāṃ of the parallels

%MD 2024-8: Could we adopt the reading mayūram instead of māyūram im Pāda d? Yes.
% Add comment on karadvayābhyāṃ and °sthalābhyāṃ?
\end{translation}
%</tr30>

%<*sc30>
\begin{sources}[hp01_030]%JM: this is a lot of indirect sources. I suggest having only the VS and a reference to Mallinson 2014: 227 n.9, which gives all these.
% Cf.~\emph{Vimānārcanākalpa} 96

% \begin{versinnote}
% \tl{karatale bhūmau saṃsthāpya kūrparau nābhipārśvayor nyasya nataśirāḥ (unnataśirāḥ) pādau ḍaṇḍavad vyomni saṃsthito mayūrāsanam iti//\\!}
% \end{versinnote}

% Cf.~\emph{Pādmasaṃhitā} (\emph{yogapāda}) 1.21c–22d:

% \begin{versinnote}
% \tl{avaṣṭabhya dharāṃ samyak talābhyāṃ hastayor dvayoḥ//\\+}
% \tl{kūrparau nābhipārśve ca sthāpayitvā mayūravat/\\+}
% \tl{samunnamya śiraḥpādau mayūrāsanam iṣyate//\\!}
% \end{versinnote}

% Cf.~\emph{Ahirbudhnyasaṃhitā} 31.36–37

% \begin{versinnote}
% \tl{mayūrāsanam\\+}
% \tl{niveśya kūrparau samyaṅ nābhimaṇḍalapārśvayoḥ/\\+}
% \tl{avaṣṭabhya bhuvaṃ pāṇitalābhyāṃ vyomni daṇḍavat//\\!}
% \end{versinnote}
Cf.~\emph{Vasiṣṭhasaṃhitā} 1.76–77

\begin{versinnote}
\tl{avaṣṭabhya dharāṃ samyak talābhyāṃ ca karadvayam/\\+}
\tl{hastayoḥ kūrparau cāpi sthāpayan nābhipārśvayoḥ//\\+}
\tl{samunnataśiraḥpādo daṇḍavad vyomni saṃsthitaḥ/\\+}
\tl{mayūrāsanam etad dhi sarvapāpavināśanam//\\!}
\end{versinnote}
\begin{appinnote}
\tl{\textbf{76b} ca karadvayam~] karayor dvayoḥ \\!}
\end{appinnote}
% \emph{Yogayājñavalkya} 3.15–16

% \begin{versinnote}
% \tl{avaṣṭabhya dharāṃ samyak talābhyāṃ tu karadvayoḥ/\\+}
% \tl{hastayoḥ kūrparau cāpi sthāpayan nābhipārśvayoḥ//\\+}
% \tl{samunnataśiraḥpādo daṇḍavad vyomni saṃsthitaḥ/\\+}
% \tl{mayūrāsanam etat tu sarvapāpapraṇāśanam//\\!}
% \end{versinnote}

\end{sources}
%</sc30>

%<*ts30>
\begin{testimonia}[hp01_030]
  \emph{Haṭharatnāvalī} 3.42, \emph{Yogacintāmaṇi} f.\,84r (\attr HP)

  \begin{variants}
    karasthalābhyāṃ~] karadvayena HRĀ, punaḥ karābhyāṃ YCM\sep
    kūrpara~] kūrpare HRĀ YCM\sep
    uccāsano HRĀ~] tadāsane YCM\sep
    pīṭham HRĀ~] santaḥ YCM\sep
    mayūram HRĀ YCM~] māyūram HRĀ\vl
    \end{variants}
  
%\emph{Haṭharatnāvalī} 3.42
%
%\begin{versinnote}
%\tl{atha mayūram\\+}
%\tl{dharām avaṣṭabhya karadvayena\\+}
%\tl{tatkūrpare sthāpitanābhipārśvaḥ/\\+}
%\tl{uccāsano daṇḍavad utthitaḥ khe\\+}
%\tl{mayūram etat pravadanti pīṭham//\\!}
%\end{versinnote}
%\begin{appinnote}
%\tl{\textbf{d} mayūram~] cett., māyūram P, T, t1 \\!}
%\end{appinnote}
%
%\emph{Yogacintāmaṇi} f.\,84r (\attr HP)
%
%\begin{versinnote}
%\tl{dharām avaṣṭabhya punaḥ karābhyāṃ\\+}
%\tl{tatkūrpare sthāpitanābhipārśvaḥ/\\+}
%\tl{tadāsane daṇḍavad utthitaḥ khe\\+}
%\tl{mayūram etat pravadanti santaḥ//\\!}
%\end{versinnote}

\end{testimonia}
%</ts30>

%<*cm30>
\begin{philcomm}[hp01_030]
There is no direct source of this verse, but it has the same elements as two verses in the \emph{Vasiṣṭhasaṃhitā} (1.76–77), which are themselves derived from earlier Vaiṣṇava sources (see Mallinson 2014: 227 n.\,9). The compound \emph{uccāsanaḥ} in the third verse quarter seems to approximate in a somewhat vague way the \emph{Vasiṣṭhasaṃhitā}’s reading \emph{samunnataśiraḥpādaḥ}. \lb
 
In the second verse quarter, the pronoun in \emph{tatkūrpare} refers to the two hands. This is stated more explicitly (i.e.~\emph{hastayoḥ kūrparau}) in \emph{Vasiṣṭhasaṃhitā} 1.76c and \emph{Yogayājñavalkya} 3.15c.
\end{philcomm}
%</cm30>

\begin{metre}[hp01_030]
Upajāti
\end{metre}

%%%%%%%%%%
\subsection*{1.31}
%<*tr31>
\begin{translation}[hp01_031]
The glorious peacock posture uickly cures all illnesses, beginning with bloating and abdominal diseases, and overcomes humoral imbalances. It reduces to ash food which is bad or has been eaten to excess, kindles the digestive fire and causes strong poison to be digested.
\end{translation}
%JB: gulmodara appears to always means gulma and udararoga in medical texts: cf. gulmodarayoḥ, gulmodareṣu, etc. PV Sharma translates gulmodara as gulma and udara in both the Suśruta and Caraka. Srikantamurty translated gulmodara as abdominal tumours and enlargment of the abdomen in the Aṣṭāṅgahṛdaya. Moreover, ayurvedic kośas do not define gulmodara as a disease. And in HP 1.31 sakalaroga might just be the equivalent of sarvapāpavināśana in the VS and sarvavyādhinivāraṇa in the Ahirbudhnya's description of mayūra. So perhaps the best translation is "... quickly cures all illnesses, beginning with bloating and abdominal diseases... JM: yes
%</tr31>
% MD: In 1.60 kadaśanaśākōtkaṭa was translated as  "an excess of leafy vegetables that are hard to chew". Perhaps it is better to interpret kadaśana in a similar sense. Indigestive food?

%<*ts31>
\begin{testimonia}[hp01_031]
  \emph{Haṭharatnāvalī} 3.43, \emph{Yogacintāmaṇi} f.\,84r (\attr HP)

  \begin{variants}
    aśeṣam YCM~] vicitram HRĀ\sep
    jārayet YCM~] jīryate HRĀ
    \end{variants}
  
%\emph{Haṭharatnāvalī} 3.43
%
%\begin{versinnote}
%\tl{harati sakalarogān āśu gulmodarādīn\\+}
%\tl{abhibhavati ca doṣān āsanaṃ śrīmayūram//\\+}
%\tl{bahukadaśanabhuktaṃ bhasma kuryād vicitram\\+}
%\tl{janayati jaṭharāgniṃ jīryate kālakūṭam//\\!}
%\end{versinnote}
%
%\emph{Yogacintāmaṇi} f.\,84r (\attr HP)
%
%\begin{versinnote}
%\tl{harati sakalarogān āśu gulmodarādīn\\+}
%\tl{abhibhavati ca doṣān āsanaṃ śrīmayūram/\\+}
%\tl{bahukadaśanabhuktaṃ bhasma kuryād aśeṣam\\+}
%\tl{janayati jaṭharāgniṃ jārayet kālakūṭam//\\!}
%\end{versinnote}
%

\end{testimonia}
%</ts31>

%<*cm31>
%\begin{philcomm}[hp01_031]
%\end{philcomm}
%</cm31>

\begin{metre}[hp01_031]
Mālinī 
\end{metre}

%%%%%%%%%%
\subsection*{1.32}
%<*tr32>
\begin{translation}[hp01_032]
Lying face up on the ground like a corpse is the corpse posture. It removes fatigue [caused by practising] any \emph{āsana} and calms the mind.
\end{translation}
%JB: fatigue caused by all kinds of \emph{āsana}s or caused by any of the \emph{āsana}s. cf. sarva-karman all kinds of works, sarva-kāma all kinds of desires, etc.
%</tr32>

%<*sc32>
\begin{sources}[hp01_032]
Cf.~\emph{Dattātreyayogaśāstra} 24cd

\begin{versinnote}
\tl{uttānaśavavad bhūmau śayanaṃ coktam uttamam//\\!}
\end{versinnote}

\end{sources}
%</sc32>

%<*ts32>
\begin{testimonia}[hp01_032]
  \emph{Yogacintāmaṇi} f.\,84r (\attr HP), \emph{Yuktabhavadeva} 6.21 (\attr HP)
  \begin{variants}
    śayanaṃ tu śavāsanam YBhD~] śavāsanam idaṃ smṛtam YCM\sep
    sarvāsana~] śavāsanaṃ YCM YBhD\sep
    sādhanam YCM~] kārakam YBhD
    \end{variants}
  
%\emph{Yogacintāmaṇi} f.\,84r (\attr HP)
%
%\begin{versinnote}
%\tl{uttānaṃ śavavad bhūmau śavāsanam idaṃ smṛtam/\\+}
%\tl{śavāsanaṃ śrāntiharaṃ cittaviśrāntisādhanam//\\!}
%\end{versinnote}
%
%
%\emph{Yuktabhavadeva} 6.21 (\attr HP)
%
%\begin{versinnote}
%\tl{uttānaṃ śavavad bhūmau śayanaṃ tu śavāsanam/\\+}
%\tl{śavāsanaṃ śrāntiharaṃ cittaviśrāntikārakam//\\+}
%\tl{iti śavāsanam//\\!}
%\end{versinnote}

Cf.~\emph{Haṭharatnāvalī} 3.76

\begin{versinnote}
\tl{athāntimaṃ śavāsanam\\+}
\tl{prasārya hastapādau ca viśrāntyā śayanaṃ tathā/\\+}
\tl{sarvāsanaśramaharaṃ śayitaṃ tu śavāsanam//\\!}
\end{versinnote}

Cf.~\emph{Haṭhatattvakaumudī} 7.12

\begin{versinnote}
\tl{śavāsanaṃ hṛtkupitavātagranthivibhedakam/\\+}
\tl{sarvāsanaśrāntijit hṛtśramaghnaṃ yogisaukhyadam//\\!}
\end{versinnote}

\end{testimonia}
%</ts32>

\begin{metre}[hp01_032]
Anuṣṭubh (c: bha-vipulā)
\end{metre}

%%%%%%%%%%
\subsection*{1.33}
%<*tr33>
\begin{translation}[hp01_033]
Śiva has taught eighty-four \emph{āsana}s. I shall take the four best from them and describe them.
\end{translation}
%</tr33>

%<*sc33>
\begin{sources}[hp01_033]
\emph{Śivasaṃhitā} 3.96

\begin{variants}
  śivena kathitāni tu~] santi nānāvidhāni ca ŚS\sep
  sārabhūtaṃ~] mayoktāni ŚS
  \end{variants}
  
%\emph{Śivasaṃhitā} 3.96
%
%\begin{versinnote}
%\tl{caturaśīty āsanāni santi nānāvidhāni ca/\\+}
%\tl{tebhyaś catuṣkam ādāya mayoktāni bravīmy aham//\\!}
%\end{versinnote}
%
Cf.~\emph{Vivekamārtaṇḍa} 5

\begin{versinnote}
\tl{caturāśītilakṣānām ekaikaṃ samudāhṛtaṃ/\\+}
\tl{ataḥ śivena pīṭhānāṃ ṣoḍaśonaṃ śataṃ kṛtam//\\!}
\end{versinnote}

\end{sources}
%</sc33>

%<*ts33>
\begin{testimonia}[hp01_033]
\emph{Haṭharatnāvalī} 3.23, \emph{Yogacintāmaṇi} f.\,84v (\attr HP)

\begin{variants}
  tu HRĀ~] vai YCM
\end{variants}
  
%\emph{Haṭharatnāvalī} 3.23
%
%\begin{versinnote}
%\tl{caturaśīty āsanāni śivena kathitāni tu/\\+}
%\tl{tebhyaś catuṣkam ādāya sārabhūtaṃ bravīmy aham//\\!}
%\end{versinnote}
%
%\emph{Yogacintāmaṇi} f.\,84v
%
%\begin{versinnote}
%\tl{haṭhapradīpikāyām—\\+}
%\tl{caturaśīty āsanāni śivena kathitāni vai/\\+}
%\tl{tebhyaś catuṣkam ādāya sārabhūtaṃ bravīmy aham//\\!}
%\end{versinnote}

\end{testimonia}
%</ts33>

%<*cm33>
\begin{philcomm}[hp01_033]
The word \emph{tu} is often used to introduce a new posture, but in this case seems to be a verse filler.\lb
 
In the first and third verse quarters, Svātmārāma appears to have rewritten \emph{Śivasaṃhitā} 3.96 to include the information that it was Śiva (\emph{śivena}) who taught the eighty-four \emph{āsana}s, whereas in the source Śiva is himself speaking. Svātmārāma also changes the meaning of the second half of the verse, as the \emph{Śivasaṃhitā} states that Śiva picked out the four best postures and taught them, whereas in the \emph{Haṭhapradīpikā} it reads as though Svātmārāma himself is responsible for picking out the four best postures and teaching them. There are other instances in the \emph{Haṭhapradīpikā} where Svātmārāma borrows a verse with a first person verb (e.g.~3.43). However, in this instance, he may have intended to indicate that he chose the four postures coming after this verse (i.e.~\emph{siddha}, \emph{padma}, \emph{siṃha} and \emph{bhadra}) because the \emph{Śivasaṃhitā} follows 3.96 with teachings on the postures called \emph{siddha}, \emph{padma}, \emph{paścimottāna} and \emph{svastika}. The vocative here (\emph{sakhe}) is also found in \emph{Haṭhapradīpikā} 4.56, which is also likely to be an authorial verse.

%Another possibility is that Svātmārāma borrowed 1.33–1.34 from an unknown source that contained a dialogue that was different to that of \emph{Śivasaṃhitā}, as indicated by \emph{sakhe} in 1.34.
\end{philcomm}%JM: I don't think the last sentence is needed. There's another sakhe in 4.56 which is also very likely to be an authorial verse.
%</cm33>

\begin{metre}[hp01_033]
Anuṣṭubh (a: ra-vipulā)
\end{metre}

%%%%%%%%%%
\subsection*{1.34}
%<*tr34>
\begin{translation}[hp01_034]
The adept, lotus, lion and auspicious pose: these four are the best and, among those, always sit in the adept’s pose, my dear.
\end{translation}
%</tr34>

%<*sc34>
%\begin{sources}[hp01_034]
%--
%\end{sources}
%</sc34>

%<*ts34>
\begin{testimonia}[hp01_034]
  \emph{Haṭharatnāvalī} 3.24, \emph{Yogacintāmaṇi} f.\,84v (\attr HP)

  \begin{variants}
    siṃhaṃ HRĀ~] bhadraṃ YCM\sep
    bhadraṃ HRĀ~] siṃhaṃ YCM\sep
    ca HRĀ~] vai YCM\sep
    sakhe HRĀ\vl~] tathā HRĀ, satve HRĀ\vl, padmam YCM\sep
    tiṣṭha~] tiṣṭhet HRĀ YCM
    \end{variants}
  
%\emph{Haṭharatnāvalī} 3.24
%
%\begin{versinnote}
%\tl{siddhaṃ padmaṃ tathā siṃhaṃ bhadraṃ ceti catuṣṭayam/\\+}
%\tl{śreṣṭhaṃ tatrāpi ca tathā tiṣṭhet siddhāsane sadā//\\!}
%\end{versinnote}
%\begin{appinnote}
%\tl{\textbf{c} tathā~] satve P, sakhe T,t1 \\!}
%\end{appinnote}
%
%\emph{Yogacintāmaṇi} f.\,84v (\attr HP)
%
%\begin{versinnote}
%\tl{siddhaṃ padmaṃ tathā bhadraṃ siṃhaṃ ceti catuṣṭayam/\\+}
%\tl{śreṣṭhaṃ tatrāpi vai padmaṃ tiṣṭhet siddhāsane sadā//\\!}
%\end{versinnote}

\end{testimonia}
%</ts34>

%<*cm34>
\begin{philcomm}[hp01_034]
It is likely that the original version of this verse contained the vocative with the imperative form of the verb (\emph{sakhe tiṣṭha}). There are other instances where Svātmārāma included a verse with the vocative (e.g.~4.7c, 4.56b, 4.72d, 4.74c) as though the text were a dialogue. % MD: I have deleted the instances 4.0*14a and *16d since they are secondary.
Other versions of this verse are transmitted by some manuscripts of the \emph{Haṭhapradīpikā}, in which the vocative and imperative verb have been removed. In these cases, \emph{sukhe} and \emph{sukham} are difficult to construe because the context suggests that the intended meaning was that one should always sit in \emph{siddhāsana} (as opposed to the other three \emph{āsana}s), rather than the prescription to always sit in a comfortable \emph{siddhāsana}.   
\end{philcomm}
%</cm34>

\begin{metre}[hp01_034]
Anuṣṭubh (c: na-vipulā)
\end{metre}

%%%%%%%%%%
\subsection*{1.35 heading}
%<*tr35a>
\begin{translation}[hp01_035a]
Now, the adept's pose (\emph{siddhāsana}).
\end{translation}
%</tr35a>

%<*cm35a>
% \begin{philcomm}[hp01_035a]
% \end{philcomm}
%</cm35a>

%%%%%%%%%%
\subsection*{1.35}
%<*tr35>
\begin{translation}[hp01_035]
[The yogi] should put the heel at the perineum, firmly place the [other] foot on the penis, focus the mind, hold the body erect and [remain] motionless, his senses restrained, gazing between the brows with his eyes unmoving. This, which breaks open the door to liberation, is called the adept’s pose.
\end{translation}
%</tr35>

%<*sc35>
\begin{sources}[hp01_035]
  \emph{Vivekamārtaṇḍa} 7

  \begin{variants}
    athaikam ekahṛdayo VM~] athaikadeśahṛdayo VM\vl, athaikam eva niyataṃ VM\vl, athaikam eva niṣataṃ VM\vl, athaikam eva hṛdayaṃ VM\vl, athaikam ekahṛdayaḥ VM\vl\sep
    dhṛtvā VM~] kṛtvā VM\vl\sep
    paśyan VM~] paśyed VM\vl, paśyad VM\vl   
    \end{variants}
  
%\emph{Vivekamārtaṇḍa} 7
%
%\begin{versinnote}
%\tl{yonisthānakam aṅghrimūlaghaṭitaṃ kṛtvā dṛḍhaṃ vinyasen\\+}
%\tl{meḍhre pādam athaikam ekahṛdayo dhṛtvā samaṃ vigraham/\\+}
%\tl{sthāṇuḥ saṃyamitendriyo 'caladṛśā paśyan bhruvor antaraṃ\\+}
%\tl{caitan mokṣakapāṭabhedajanakaṃ siddhāsanaṃ procyate//\\!}
%\end{versinnote}
%\begin{appinnote}
%\tl{7a °mūlaghaṭitaṃ~] °mulaghaṭanaṃ A • vinyasen~] GHT; vinyase VA, vinyaset Y \\+}
%\tl{\textbf{7b} athaikaṃ ekahṛdayo~] T; athaikadeśahṛdayo V, athaikam eva niyataṃ AGBGPk, athaikam eva niṣataṃ GL, athaikam eva hṛdayaṃ GP, athaikam ekahṛdayaḥ Y • dhṛtvā~] VGBT; kṛtvā GLGPGPkY %
%\ \textbf{7c} paśyan~] VAGHSTvl; paśyed T paśyad Y \\!}
%\end{appinnote}
%
\end{sources}
%</sc35>

%<*ts35>
\begin{testimonia}[hp01_035]
  \emph{Haṭharatnāvalī} 3.25, \emph{Yogacintāmaṇi} f.\,84v–85r (\attr \emph{Pavanayogasaṅgraha})

  \begin{variants}
    yonisthānakam HRĀ YCM~] yonidvārakam HRĀ\vl\sep
    ekahṛdayo~] ekahṛdayaḥ YCM, eva niyataṃ HRĀ eva hṛdaye HRĀ\vl\sep
    dhṛtvā~] kṛtvā HRĀ YCM\sep
    paśyan HRĀ~] paśyed YCM\sep
    caitan HRĀ~] etan YCM\sep
    kapāṭa HRĀ YCM~] kavāṭa HRĀ\vl\sep
    janakaṃ HRĀ~] nakaraṃ YCM
    \end{variants}
 
%\emph{Haṭharatnāvalī} 3.25
%
%\begin{versinnote}
%\tl{tatra siddhāsanam\\+}
%\tl{yonisthānakam aṅghrimūlaghaṭitaṃ kṛtvā dṛḍhaṃ vinyasen\\+}
%\tl{meḍhre pādam athaikam eva niyataṃ kṛtvā samaṃ vigraham/\\+}
%\tl{sthāṇuḥ saṃyamitendriyo 'caladṛśā paśyan bhruvor antaraṃ\\+}
%\tl{caitan mokṣakapāṭabhedajanakaṃ siddhāsanaṃ procyate//\\!}
%\end{versinnote}
%\begin{appinnote}
%\tl{\textbf{a} yonisthānakam~] yonidvārakam P,T. \ %
%\textbf{b} niyataṃ~] hṛdaye T,t1,n2. \\+}
%\tl{\textbf{d} °kapāṭa°~] °kavāṭa° P,T,t1\\!}
%\end{appinnote}
%
%\emph{Yogacintāmaṇi} f.\,84v–85r (attr.~\emph{Pavanayogasaṅgraha})
%
%\begin{versinnote}
%\tl{pavanayogasaṃgrahe—\\+}
%\tl{yonisthānakam aṅghrimūlaghaṭitaṃ kṛtvā dṛḍhaṃ vinyasen\\+}
%\tl{meḍhre pādam athaikam ekahṛdayaḥ kṛtvā samaṃ vigraham/\\+}
%\tl{sthāṇuḥ saṃyamitendriyo 'caladṛśā paśyed bhruvor antaraṃ tv\\+}
%\tl{etan mokṣakapāṭabhedanakaraṃ siddhāsanaṃ procyate//\\!}
%\end{versinnote}

\end{testimonia}
%</ts35>

%<*cm35>
\begin{philcomm}[hp01_035]
In the second verse quarter, the adopted reading \emph{ekahṛdayo} is supported by two manuscripts of the \emph{Haṭhapradīpikā} (\getsiglum{J2,M1}) and is close to the \alphaOne\ reading \emph{ekahṛdaye}. It is also attested by the six-chapter \emph{Viveka\-mārtaṇḍa} and the \emph{Yoga\-cintāmaṇi}, which attributes this verse to an unkown work called the \emph{Pavanayogasaṅgraha}. In this case, \emph{ekahṛdayaḥ} appears to describe the yogi as having his mind focused on one thing. There are many variations of this verse quarter in the \emph{Haṭhapradīpikā} manuscripts, as well as in the manuscripts of the sources and testimonia. Most of the collated witnesses have \emph{athaikam eva hṛdaye dhṛtvā}, which is close to the adopted reading. Other readings allude here to the practice of the Jālandhara lock, in which the chin is placed on the chest. This is most clearly seen in the \emph{Jyotsnā}'s version, \emph{hṛdaye kṛtvā hanuṃ susthiram} (`having put the jaw firmly on the chest'). The other well-attested reading, \emph{athaikaṃ eva niyataṃ}, was an attempt to fix the problem of \emph{hṛdaye} by replacing it with \emph{niyatam}, which must be read with \emph{meḍhre pādam athaikaṃ} (`having fixed one foot on the penis'), but \emph{niyatam} is redundant here because of \emph{vinyaset} in the first verse quarter.     

%The compound \emph{āsyahṛdaye} is supported by other manuscripts, such as \getsiglum{N10,P1,P6,J16} and the variants \emph{asyahṛdaye} and \emph{asyahṛdayaṃ} also occur (e.g.~\getsiglum{J7,J12,A1,V4,V18,V16,P8,P9}). 

%MD 2024-5-14: adopt caitan (alpha, pi, and most of the mss of VM) in Pāda d? JM: yes
\end{philcomm}
%</cm35>

\begin{metre}[hp01_035]
Śārdūlavikrīḍita 
\end{metre}

%%%%%%%%%%
\subsection*{1.36 heading}
%<*tr36a>
\begin{translation}[hp01_036a]
However, in another school [\emph{siddhāsana} is taught as follows]:
\end{translation}
%</tr36a>

%<*cm36a>
% \begin{philcomm}[hp01_036a]
% \end{philcomm}
%</cm36a>

%%%%%%%%%%
\subsection*{1.36}
%<*tr36>
\begin{translation}[hp01_036]
Place the left heel on the penis and put the other heel on top: this is the adept's pose (\emph{siddhāsana}).
\end{translation}
%</tr36>
%MD: Better to adopt nikṣipya in both pāda a and c? JM: yes

%<*sc36>
\begin{sources}[hp01_036]
  \emph{Vasiṣṭhasaṃhitā} 1.81, \emph{Yogayājñavalkya} 3.14

  \begin{variants}
    ca nikṣipya YY~] vinikṣipya VS\sep
    bhavet~] smṛtam VS, tu vā YY
    \end{variants}

%\emph{Vasiṣṭhasaṃhitā} 1.81
%
%\begin{versinnote}
%\tl{meḍhrād upari nikṣipya savyaṃ gulphaṃ tathopari/\\+}
%\tl{gulphāntaraṃ vinikṣipya muktāsanam idaṃ smṛtam//\\!}
%\end{versinnote}
%
%\emph{Yogayājñavalkya} 3.14
%
%\begin{versinnote}
%\tl{meḍhrād upari nikṣipya savyaṃ gulphaṃ tathopari/\\+}
%\tl{gulphāntaraṃ ca nikṣipya muktāsanam idaṃ tu vā//\\!}
%\end{versinnote}

\end{sources}
%</sc36>

%<*ts36>
\begin{testimonia}[hp01_036]
\emph{Haṭharatnāvalī} 3.26, \emph{Yogacintāmaṇi} f.\,85r (\attr \emph{Pavanayogasaṅgraha})

\begin{variants}
  nikṣipya~] niḥkṣipya HRĀ, vinyasya YCM\sep
  ca niḥkṣipya HRĀ~] tu vinyasya YCM\sep
  siddhāsanam idaṃ bhavet YCM~] siddhāḥ siddhāsanaṃ viduḥ HRĀ
  \end{variants}
  
%\emph{Haṭharatnāvalī} 3.26
%
%\begin{versinnote}
%\tl{matāntare tu\\+}
%\tl{meḍhrād upari niḥkṣipya savyaṃ gulphaṃ tathopari/\\+}
%\tl{gulphāntaraṃ ca niḥkṣipya siddhāḥ siddhāsanaṃ viduḥ//\\!}
%\end{versinnote}
%
%\emph{Yogacintāmaṇi} f.\,85r (attr. \emph{Pavanayogasaṅgraha})
%
%\begin{versinnote}
%\tl{tathā/\\+}
%\tl{meḍhrād upari vinyasya savyaṃ gulphaṃ tathopari/\\+}
%\tl{gulphāntaraṃ tu vinyasya siddhāsanam idaṃ bhavet//\\!}
%\end{versinnote}

\end{testimonia}
%</ts36>

%<*cm36>
\begin{philcomm}[hp01_036]
Svātmārāma's introductory and following remarks to verse 1.36 indicate that he preferred the \emph{siddhāsana} of the \emph{Vivekamārtaṇḍa} over the version taught as \emph{muktāsana} in the \emph{Vasiṣṭhasaṃhitā} and \emph{Yogayājñavalkya}.
\end{philcomm}
%</cm36>
%%%%%%%%%%
\subsection*{1.36 ending}
%<*tr36p>
\begin{translation}[hp01_036p]
Only the first teaching [on \emph{siddhāsana}] is accepted by me.
\end{translation}
%</tr36p>

%<*cm36p>
% \begin{philcomm}[hp01_036p]
% \end{philcomm}
%</cm36p>

%%%%%%%%%%
\subsection*{1.37}
%<*tr37>
\begin{translation}[hp01_037]
Some call this the adept's pose (\emph{siddhāsana}), others know it as the thunderbolt pose (\emph{vajrāsana}), a few say it is the pose of the liberated (\emph{muktāsana}) and some call it the secret pose (\emph{guptāsana}).
\end{translation}
%</tr37>

%<*sc37>
%\begin{sources}[hp01_037]
%--
%\end{sources}
%</sc37>

%<*ts37>
\begin{testimonia}[hp01_037]
\emph{Haṭharatnāvalī} 3.27
\mylb
% \begin{variants}
%   anye vajrāsanaṃ viduḥ HRĀ~] padmāsanam atho viduḥ YCM\sep
%   muktāsanaṃ HRĀ~] guptāsanaṃ YCM\sep
%   guptāsanaṃ HRĀ~] vajrāsanaṃ YCM
%   \end{variants}

  
%\emph{Haṭharatnāvalī} 3.27
%
%\begin{versinnote}
%\tl{etat siddhāsanaṃ prāhur anye vajrāsanaṃ viduḥ/\\+}
%\tl{muktāsanaṃ vadanty eke prāhur guptāsanaṃ pare//\\!}
%\end{versinnote}

Cf.~\emph{Yogacintāmaṇi} f.\,85r (\attr \emph{Pavanayogasaṅgraha})
% NJL: Is it OK to deal with the additional hemistich like this? JM: unnecessary IMO
\begin{versinnote}
\tl{etat siddhāsanaṃ prāhuḥ padmāsanam atho viduḥ/\\+}
\tl{guptāsanaṃ vadanty eke prāhur vajrāsanaṃ pare/\\+}
\tl{ke cin muktāsanaṃ prāhur idam āsanam uttamam//\\!}
\end{versinnote}

Cf.\,the Telugu \emph{Śivayogasāramu} by Kolani Ganapatideva (date 14th c.)

\begin{versinnote}
\tl{siddāsanambunu, gondaru vajrāsanambaniyu/\\+}
\tl{gondaru muktāsanambaniyu, gondadu gulbāsanam//\\!}
\end{versinnote}

and a Telugu verse by the poet Pingali Surana (active 16th c.)

\begin{versinnote}
\tl{kondaru siddāsanamani\\+}
\tl{kondaru vajrāsanamani koniyādudurī\\+}
\tl{pondaga dīnini mariyoka\\+}
\tl{kondaru guptāsamanu kondru mahātmā\\!}
\end{versinnote}

The last two references are taken from Reddy 1982: 41–42.

\end{testimonia}
%</ts37>

%%%%%%%%%%
\subsection*{1.38}
%<*tr38>
\begin{translation}[hp01_038]
Like measured diet amongst rules and non-violence amongst observances, the adepts know \emph{siddhāsana} to be the single most important of all postures.
\end{translation}
%</tr38>

%<*sc38>
\begin{sources}[hp01_038]
Cf.~\emph{Dattātreyayogaśāstra} 33

\begin{versinnote}
\tl{laghvāhāras tu teṣv eko mukhyo bhavati nāpare/\\+}
\tl{ahiṃsā niyameṣv eko mukhyo bhavati nāpare// 33//\\!}
\end{versinnote}

\end{sources}
%</sc38>

%<*ts38>
\begin{testimonia}[hp01_038]
\emph{Yogacintāmaṇi} f.\,85r (\attr HP)

\begin{variants}
  yameṣv iva~] niyameṣu YCM\sep
  mitāhāram~] mitāharo YCM\sep
  ahiṃsāṃ niyameṣv iva~] yathāhiṃsā yameṣv iva YCM\sep
  ekaṃ~] evaṃ YCM\sep
  siddhāḥ siddhāsanaṃ viduḥ~] siddhāsanaṃ idaṃ viduḥ YCM
  \end{variants}

%\emph{Yogacintāmaṇi} f.\,85r (\attr HP)
%\begin{versinnote}
%\tl{niyameṣu mitāharo yathāhiṃsā yameṣv iva/\\+}
%\tl{mukhyaṃ sarvāsaneṣv evaṃ siddhāsanaṃ idaṃ viduḥ/\\!}
%\end{versinnote}

\end{testimonia}
%</ts38>

%<*cm38>
%\begin{philcomm}[hp01_038]
%\emph{iva} or \emph{eva}? \emph{iva} does work — like \emph{siddhāsana}, \emph{mitāhāra} and \emph{ahiṃsā} are the best, but %for it to work properly \emph{mitāhāra} and \emph{ahiṃsā} should be accusative. V19 has acc + \emph{iva}, which seems best, %especially with \emph{siddhāḥ viduḥ}, but this might be a correction as V19 often corrects. However, one old KDham BORI (?) ms %(pha, 1695 CE) has it, as does Jyotsnā, so adopt.
%In pāda d V19 has \emph{siddhāsanam idaṃ viduḥ}, but the reading of all other mss is preferable.
% JB: The reading we have adopted is well attested by the new collation of manuscripts. Therefore, I've ignored the above %comments. 

%Clearly based on DYŚ 33, which includes \emph{ekaṃ} and \emph{mukhya}.
%\end{philcomm}
%</cm38>

%%%%%%%%%%
\subsection*{1.39}
%<*tr39>
\begin{translation}[hp01_039]
Of the eighty-four postures, it is \emph{siddhāsana} that one should always prefer; in the same way that of the 72,000 channels [one should prefer] Suṣumṇā.
\end{translation}
%</tr39>

% MD 2024-8: Perhaps "siddham eva sadā bhajet" (ε-mss,IFP) is better than "samabhyaset". α2 sadā bhavet and α3 sadā paṭhet support the former.

%<*sc39>
%\begin{sources}[hp01_039]
%\end{sources}
%</sc39>

%<*ts39>
\begin{testimonia}[hp01_039]
\emph{Yogacintāmaṇi} f.\,85r (\attr HP), \emph{Yogasārasaṅgraha} p.\,9 (\attr \emph{Yogasāramañjarī})

\begin{variants}
  siddham eva YSS~] siddhāsanaṃ YCM\sep
  sadā bhajet~] samabhyaset YCM YSS
  \end{variants}

%\emph{Yogacintāmaṇi} f.\,85r (\attr HP)
%
%\begin{versinnote}
%\tl{caturaśītipīṭheṣu siddhāsanaṃ samabhyaset/\\+}
%\tl{dvāsaptatisahasreṣu suṣumṇām iva nāḍiṣu//\\!}
%\end{versinnote}
%
%\emph{Yogasārasaṅgraha} p.9 (attr. \emph{Yogasāramañjarī})
%
%\begin{versinnote}
%\tl{caturāśītapīṭheṣu siddham eva samabhyaset/\\+}
%\tl{dvisaptatisahasreṣu suṣumnām iva nāḍiṣu//\\!}
%\end{versinnote}
%
%% NJL: I think the latter two verses are probably not worth mentioning, since their relationship to HP 1.39 seems indirect at best. I cannot collate them in a meaningful way and they do not at all contribute to reconstrucing the archetype. JM: agreed. JB: I think we included them because of the note below, but perhaps we could just mention them as references there 

%(i.e., ... by changing \emph{suṣumnām iva nāḍiṣu} to \emph{nāḍīnāṃ malaśodhanam}, which occurs in the \emph{Jyotsnā} (1.39), \emph{Kumbhakapaddhati} (120), and \emph{Haṭhayoga} (quoted in the \emph{Yogacintāmaṇi}, f.\,79r). This variant reading is probably a patch...)    
%
%Cf.~\emph{Yogacintāmaṇi} f.\,79r (\attr \emph{Haṭhayoga})
%
%\begin{versinnote}
%\tl{maṇḍalā dṛśyate siddhiḥ kuṇḍalyabhyāsayoginaḥ/\\+}
%\tl{dvisaptatisahasrāṇāṃ nāḍīnāṃ malaśodhanam//\\!}
%\end{versinnote}
%
%Cf.~\emph{Kumbhakapaddhati} 120 (on the effects of practising \emph{kumbhaka})
%
%\begin{versinnote}
%\tl{dvāsaptati sahāsrāṇāṃ nāḍīnāṃ malaśodhanam/\\+}
%\tl{yatheṣṭaṃ dhāraṇaṃ vāyor vikārābhāva eva ca//\\!}
%\end{versinnote}

\end{testimonia}
%</ts39>

%<*cm39>
\begin{philcomm}[hp01_039]
Most witnesses have \emph{suṣumṇām} as the object of the verb \emph{abhyaset}, which is odd. %  
% MD: we have finally adopated the reading "bhajet".
%This reading is well attested by manuscripts of the \emph{Haṭhapradīpikā} and is also found in the \emph{Yogacintāmaṇi}, which attributes it to the \emph{Yogasāramañjarī}. Perhaps, the second line was added somewhat haphazardly by Svātmārāma, and then 
%
Others have tried to make sense of this by changing \emph{suṣumnām iva nāḍiṣu} to \emph{nāḍīnāṃ malaśodhanam}, which occurs in the \emph{Jyotsnā} (1.39), \emph{Kumbhakapaddhati} (120), and \emph{Haṭhayoga} (quoted in the \emph{Yogacintāmaṇi}, f.\,79r). This variant reading is probably a patch
and is probably a patch as no other texts say that \emph{siddhāsana} clears the channels. It could have been borrowed from verse 3.102*1ab, which has the same half-verse and was added to the section on \textit{śakticālana}.
%However, the idea of purifying the channels can be found in other contexts (e.g.~\emph{Kumbhakapaddhati} 120) and may hark back to an earlier notion of flushing (\emph{cālana}) the channels (e.g.~\emph{Amṛtasiddhi} 11.6).% somewhat haphazardly? I don't %understand the second part of the note: purifying the channels is found all over the place.
%JB: The point is: no other texts say that \emph{siddhāsana} clears the channels. 
% MD: the same half verse with nāḍīnāṃ malaśodhanam/e appears as 3.102*1ab. Better to mention?

%Good example of early contamination.
%[\emph{nāḍiṣu} is better supported (J10ac,V19,J17).]
\end{philcomm}
%</cm39>

%%%%%%%%%%
\subsection*{1.40}
%<*tr40>
\begin{translation}[hp01_040]
By meditating upon the self, restricting the diet and regularly practising \emph{siddhāsana} for twelve years, the yogi attains \emph{niṣpatti}, [the final stage of yoga]. %
What’s the point of the [other] many tiring postures when there is \emph{siddhāsana}?
\end{translation}
% JB: this is the first reference to \emph{niṣpatti} and the only reference in ch 1. Should we help the reader here (who may not be familiar with ch 4 when they read this)? e.g., ... attains \emph{niṣpatti}, [the final stage of yoga].  
%</tr40>

%<*sc40>
%\begin{sources}[hp01_040]
%--
%\end{sources}
%</sc40>

%<*ts40>
\begin{testimonia}[hp01_040]
  \emph{Yogacintāmaṇi} f.\,85r (\attr HP), \emph{Yogasārasaṅgraha} p.\,9 (\attr \emph{Yogasāramañjarī})

  \begin{variants}
    ātmadhyāyī YCM~] ātmadhyāyo YSS\sep
    yogī YCM~] yoga YSS\sep
    kiṃ syāt YCM~] alaṃ YSS
    \end{variants}
  
%\emph{Yogacintāmaṇi} f.\,85r (\attr HP)
%
%\begin{versinnote}
%\tl{ātmadhyāyī mitāhārī yāvad dvādaśavatsaram/\\+}
%\tl{sadā siddhāsanābhyāsād yogī niṣpattim āpnuyāt/\\+}
%\tl{śramadair bahubhiḥ pīṭhaiḥ kiṃ syāt siddhāsane sati//\\!}
%\end{versinnote}
%
%\emph{Yogasārasaṅgraha} p.\,9 (attr. \emph{Yogasāramañjarī})
%
%\begin{versinnote}
%\tl{ātmadhyāyo mitāhārī yāvad dvādaśavatsaram/\\+}
%\tl{sadā siddhāsanābhyāsād yoganiṣpattim āpnuyāt//\\+}
%\tl{śramadair bahubhiḥ pīṭhair alaṃ siddhāsane sati/\\!}
%\end{versinnote}

\end{testimonia}
%</ts40>

%<*cm40>
\begin{philcomm}[hp01_040]
% Only possible variant is \emph{mitāhāro} in V19.
% J8 might be correction of J10’s unmetrical reading.

%[\emph{sadāsiddhāsanābhyāsād}? Or maybe read \emph{sadā} with \emph{avāpnuyāt}]

% V1 has:

%\begin{versinnote}
%\tl{śramādau bahubhiḥ pīṭhais sadā siddhāsane sati/\\+}
%\tl{prāṇānile sāvadhānaṃ baddhe kevalakumbhake// 1.41//\\!}
%\end{versinnote}

%Mixing up both versions of the verse — contamination already?

%V19 is found in Yogacintāmaṇi: \emph{śramadair bahubhịh pīṭhaiḥ kiṃ syāt siddhāsane sati}; JM this seems best to me.

The notion of \emph{āsana}s causing fatigue (\emph{śrama}) was mentioned earlier in the verse on the corpse pose (1.32).
\end{philcomm}
%</cm40>


%%%%%%%%%%
\subsection*{1.41}
%<*tr41>
\begin{translation}[hp01_041]
Just as the [state] beyond mind (\emph{unmanī}) arises automatically, without effort, when the \emph{prāṇa} breath has been carefully stopped in \emph{kevalakumbhaka}, [...]
\end{translation}
%</tr41>

%<*sc41>
%\begin{sources}[hp01_041]
%\end{sources}
%</sc41>

%<*ts41>
\begin{testimonia}[hp01_041]
  \emph{Yogacintāmaṇi} f.\,85r (\attr HP)

  \begin{variants}
    sāvadhānaṃ~] sāvadhāne YCM\sep
    utpadyate~] utpatsyate YCM
    \end{variants}

%\emph{Yogacintāmaṇi} f.\,85r (\attr HP)
%
%\begin{versinnote}
%\tl{prāṇānile sāvadhāne baddhe kevalakumbhake/\\+}
%\tl{utpatsyate nirāyāsāt svayam evonmanī yathā//\\!}
%\end{versinnote}
%
\end{testimonia}
%</ts41>

%<*cm41>
%\begin{philcomm}[hp01_041]
%--
%\end{philcomm}
%</cm41>

\begin{metre}[hp01_041]
Anuṣṭubh (a: ra-vipulā)
\end{metre}

%%%%%%%%%%
\subsection*{1.42}
%<*tr42>
\begin{translation}[hp01_042]
[...] so too the three locks (\emph{bandha}) arise automatically without effort, every time \emph{siddhāsana} alone is firmly adopted.
\end{translation}
%</tr42>
% adopt dṛḍhaṃ 
%% JB : only when the adept's pose is firmly bound.

%<*sc42>
%\begin{sources}[hp01_042]
%--
%\end{sources}
%</sc42>

%<*ts42>
\begin{testimonia}[hp01_042]
  \emph{Yogacintāmaṇi} f.\,85r (\attr HP), \emph{Yogasārasaṃgraha} p.\,9 
  
  \begin{variants}
tathaikasminn YSS~] athaikasminn YCM\sep
dṛḍhaṃ YCM~] baddhe YSS\sep
evopayāyate ~YCM] evopadhiyate YSS

\end{variants}

%\emph{Yogacintāmaṇi} f.\,85r (\attr HP)
%
%\begin{versinnote}
%\tl{athaikasminn eva dṛḍhaṃ baddhe siddhāsane sadā/\\+}
%\tl{bandhatrayam anāyāsāt svayam evopajāyate//\\!}
%\end{versinnote}

% NJL: YSS reference missing. Add?

%YSS
%\begin{versinnote}
%\tl{tathaikasminn eva baddhe siddhāsane sadā/\\+}
%\tl{granthitrayam anāyāsāt svayam evopabhidyate/\\!}
%\end{versinnote}

\end{testimonia}
%</ts42>

%<*cm42>
\begin{philcomm}[hp01_042]
It seems likely that \emph{dṛḍhaṃ} (rather than \emph{dṛḍhe}) was originally intended in 1.42a because \emph{dṛḍhataraṃ}, which is not ambiguous, is used in 1.48a to qualify how \emph{padmāsana} should be adopted, and \emph{dṛḍhaṃ} complements \emph{sāvadhānaṃ} in 1.41a.
% J5 and G4 have dṛḍhaṃ (alpha 2 and 3), + P11 (alpha and pi)
\end{philcomm}
%</cm42>

\begin{metre}[hp01_042]
Anuṣṭubh (a: bha-vipulā)
\end{metre}

%%%%%%%%%%
\subsection*{1.43}
%<*tr43>
\begin{translation}[hp01_043]
There is no posture like \emph{siddhāsana}, no breath-retention like \emph{kevala}, no seal like \emph{khecarī}, [and] no [means for the] dissolution [of mind] like the internal sound (\emph{nāda}).
\end{translation}
%</tr43>

%<*sc43>
\begin{sources}[hp01_043]
  \emph{Śivasaṃhitā} 5.47

  \begin{variants}
    na kumbhaḥ~]  na kumbha ŚS\sep
    kevalopamaḥ~] sadṛśaṃ balam ŚS
    \end{variants}
  
%\emph{Śivasaṃhitā} 5.47
%
%\begin{versinnote}
%\tl{nāsanaṃ siddhasadṛśaṃ na kumbhasadṛśaṃ balam/\\+}
%\tl{na khecarīsamā mudrā na nādasadṛśo layaḥ//\\!}
%\end{versinnote}

\end{sources}
%</sc43>

%<*ts43>
\begin{testimonia}[hp01_043]
\emph{Haṭharatnāvalī} 3.29, \emph{Yogacintāmaṇi} f.\,75r (\attr HP)

\begin{variants}
  kumbhaḥ kevalopamaḥ HRĀ YCM~] kumbhasadṛśo 'nilaḥ HRĀ\vl
  \end{variants}

%\emph{Haṭharatnāvalī} 3.29
%
%\begin{versinnote}
%\tl{nāsanaṃ siddhasadṛśaṃ na kumbhaḥ kevalopamaḥ//\\+}
%\tl{na khecarīsamā mudrā na nādasadṛśo layaḥ//\\!}
%\end{versinnote}
%\begin{appinnote}
%\tl{\textbf{b} kumbhaḥ kevalopamaḥ~] kumbhasadṛśo 'nilaḥ N,n1,n2,n3,J\\!}
%\end{appinnote}
%
%\emph{Yogacintāmaṇi} f.\,75r (\attr HP)
%
%\begin{versinnote}
%\tl{nāsanaṃ siddhasadṛśaṃ na kumbhaḥ kevalopamaḥ/\\+}
%\tl{na khecarīsamā mudrā na nādasadṛśo layaḥ//\\!}
%\end{versinnote}
%
\end{testimonia}
%</ts43>

%<*cm43>
%\begin{philcomm}[hp01_043]
%The reading \emph{na kumbhasadṛśo 'nilaḥ} (`no breath like a retention') is the lectio difficilior and attested by two early witnesses (\etaOne\ and \etaTwo) and is possibly original. However, the \textalpha\ manuscripts and several other important witness groups have the adopted reading \emph{kumbhaḥ kevalopamaḥ}, as well as the \emph{Yogacintāmaṇi} and some manuscripts of the \emph{Haṭharatnāvalī}, suggesting that this reading, which makes much better sense, was in the transmission at an early stage.    
% Śivasaṃhitā has \emph{kumbhasadṛśaṃ balam}.
% MD: I think kevalakumbha is indispensable after the comparison in 1.41-42. It seems more likely to me that this is the original text of Svātmārāma and that the eta group was influenced by the source text.
%\end{philcomm}
%</cm43>

\begin{metre}[hp01_043]
Anuṣṭubh (a: na-vipulā)
\end{metre}

%%%%%%%%%%
\subsection*{1.44 heading}
%<*tr44a>
\begin{translation}[hp01_044a]
Now the lotus pose (\emph{padmāsana}).
\end{translation}
%</tr44a>

%<*cm44a>
% \begin{philcomm}[hp01_044a]
% \end{philcomm}
%</cm44a>

%%%%%%%%%%
\subsection*{1.44}
%<*tr44>
\begin{translation}[hp01_044]
Place the right foot on the left thigh, and the left on the right thigh, firmly hold the big toes with the hands crossed behind the back, put the chin on the chest and gaze at the tip of the nose. This, which destroys diseases for those who are disciplined, is called the lotus pose.
\end{translation}
%</tr44>

%<*sc44>
\begin{sources}[hp01_044]
  \emph{Vivekamārtaṇḍa} 8

  \begin{variants}
    vighātakāri yamināṃ~] vikārahāri yamināṃ VM, vighātahāri yamināṃ VM\vl, vikāra\-nāśanakaraṃ VM\vl, vikārakaṃdadamanaṇ VM\vl, vināśakāri yamināṃ VM\vl
  \end{variants}
  
%\emph{Vivekamārtaṇḍa} 8
%
%\begin{versinnote}
%\tl{vāmorūpari dakṣiṇañ ca caraṇaṃ saṃsthāpya vāmaṃ tathā\\+}
%\tl{yāmyorūpari paścimena vidhinā dhṛtvā karābhyāṃ dṛḍham/\\+}
%\tl{aṅguṣṭhau hṛdaye nidhāya cibukaṃ nāsāgram ālokayed\\+}
%\tl{etad vyādhivikārahāri yamināṃ padmāsanaṃ procyate// 8//\\!}
%\end{versinnote}
%\begin{appinnote}
%\tl{\textbf{8d} °vikārahāri yamināṃ~] VAT; °vikāranāśanakaraṃ GPk Y, °vikārakaṃdadamanaṇ GB, °vināśakāri yamināṃ GL GP, °vighātahāri yamināṃ Tvl \\!}
%\end{appinnote}
%
\end{sources}
%</sc44>

%<*ts44>
\begin{testimonia}[hp01_044]
\emph{Haṭharatnāvalī} 3.34, \emph{Yogacintāmaṇi} f.\,85v (\attr \emph{Haṭhayoga})

\begin{variants}
  ca HRĀ~] hi YCM\sep
  yāmyorūpari HRĀ~] dakṣorūpari YCM\sep
  vighātakāri yamināṃ~] vināśakāri yamināṃ HRĀ, vikāranāśanakaraṃ YCM
  \end{variants}

%\emph{Haṭharatnāvalī} 3.34
%
%\begin{versinnote}
%\tl{vāmorūpari dakṣiṇaṃ ca caraṇaṃ saṃsthāpya vāmaṃ tathā\\+}
%\tl{yāmyorūpari paścimena vidhinā dhṛtvā karābhyāṃ dṛḍham/\\+}
%\tl{aṅguṣṭhau hṛdaye nidhāya cibukaṃ nāsāgram ālokayed\\+}
%\tl{etad vyādhivināśakāri yamināṃ padmāsanaṃ procyate//\\!}
%\end{versinnote}
%
%\emph{Yogacintāmaṇi} f.\,85v (attr. \emph{Haṭhayoga})
%
%\begin{versinnote}
%\tl{haṭhayoge—\\+}
%\tl{vāmorūpari dakṣiṇaṃ hi caraṇaṃ saṃsthāpya vāmaṃ tathā\\+}
%\tl{dakṣorūpari paścimena vidhinā dhṛtvā karābhyāṃ dṛḍham/\\+}
%\tl{aṅguṣṭhau hṛdaye nidhāya civukaṃ nāsāgram ālokayet\\+}
%\tl{etad vyādhivikāranāśanakaraṃ padmāsanaṃ procyate//\\!}
%\end{versinnote}
%
\end{testimonia}
%</ts44>


\begin{metre}[hp01_044]
Śārdūlavikrīḍita 
\end{metre}

%%%%%%%%%%
\subsection*{1.45 heading}
%<*tr45a>
\begin{translation}[hp01_045a]
However, in another school [lotus pose is taught as follows]:
\end{translation}
%</tr45a>

%<*cm45a>
% \begin{philcomm}[hp01_045a]
% \end{philcomm}
%</cm45a>

%%%%%%%%%%
\subsection*{1.45--46}
%<*tr46>
\begin{translation}[hp01_046]
Carefully put the upturned feet on the thighs and the upturned hands in the middle of the thighs, fix the eyes on the tip of the nose, raise the root of the uvula with the tongue, place the chin on the chest, gently [draw in] the breath [...].
\end{translation}
%</tr46>

%<*sc46>
\begin{sources}[hp01_046]
% \emph{Dattātreyayogaśāstra} 35–37, cf.~\emph{Śivasaṃhitā} 3.102–104

% \begin{variants}
%   %rājadantamūlaṃ DYŚ~] dṛṣṭiṃ rājadantaṃ ŚS\sep
%   %uttabhya DYŚ~] uttambhya ŚS, uttabya DYS\vl, yuttamā° DYS\vl, uttama DYS\vl, uttamā DYS\vl\sep
%   %cibukam DYŚ ŚS~] cubukaṃ DYŚ\vl, °bhyāṃ ca cu° DYŚ\vl\sep
%   vakṣasy DYŚ~] vakṣe ŚS DYŚ\vl, vakṣaḥ DYŚ\vl, vakṣya DYŚ\vl, °bukaṃ DYŚ\vl\sep
%   āsthāpya DYŚ~] saṃsthāpya ŚS, DYŚ\vl, āsthāya DYŚ\vl, sthāpayet DYŚ\vl, vakṣastha° DYŚ\vl 
% \end{variants}

% NJL: Why did we provide the next verses of the sources here additionally (yathāśaktyā...)? They are not in the HP. So no need to provide, right? I commented them out. JM: I think we need the final verse in the sources to show the reasons for complication with the syntax in them which leads to problems in the HP as we describe in the note. So the final verse from the sources needs giving in full but without variants.

\emph{Dattātreyayogaśāstra} 35–37 

\begin{versinnote}
\tl{uttānau caraṇau kṛtvā ūrusaṃsthau prayatnataḥ/\\+}
\tl{ūrumadhye tathottānau pāṇī kṛtvā tato dṛśau//\\+}
\tl{nāsāgre vinyased rājadantamūlaṃ ca jihvayā/\\+}
\tl{uttabhya cibukaṃ vakṣasy āsthāpya pavanaṃ śanaiḥ//\\+}
\tl{yathāśaktyā samākṛṣya pūrayed udaraṃ śanaiḥ/\\+}
\tl{yathāśaktyaiva paścāt tu recayet pavanaṃ śanaiḥ//\\!}
\end{versinnote}
\begin{appinnote}
\tl{vakṣasy DYŚ~] vakṣe ŚS DYŚ\vl, vakṣaḥ DYŚ\vl, vakṣya DYŚ\vl\sep
  āsthāpya DYŚ~] saṃsthāpya ŚS, DYŚ\vl, āsthāya DYŚ\vl, sthāpayet DYŚ\vl, vakṣastha DYŚ\vl \\!}
\end{appinnote}

Cf.~\emph{Śivasaṃhitā} 3.102–104

\begin{versinnote}
\tl{uttānau caraṇau kṛtvā ūrusaṃsthau prayatnataḥ/\\+}
\tl{ūrumadhye tathottānau pāṇī kṛtvā tu tādṛśau//\\+}
\tl{nāsāgre vinyased dṛṣṭiṃ rājadantaṃ ca jihvayā/\\+}
\tl{uttambhya cibukaṃ vakṣe saṃsthāpya pavanaṃ śanaiḥ//\\+}
\tl{yathāśaktyā samākṛṣya pūrayed udaraṃ śanaiḥ/\\+}
\tl{yathāśaktyaiva paścāt tu recayed anirodhataḥ//\\!}
\end{versinnote}

\end{sources}
%</sc46>

%<*ts46>
\begin{testimonia}[hp01_046]
  \emph{Haṭharatnāvalī} 3.36–3.37, \emph{Yogacintāmaṇi} f.\,85v

  \begin{variants}
    ūrusaṃsthau prayatnataḥ YCM~] ūrvoḥ saṃsthāpya yatnataḥ HRĀ\sep
    ca HRĀ ~] tu YCM\sep
    vakṣasyāsthāpya ~] vakṣaḥ saṃsthāpya HRĀ, vakṣasy utthāpya YCM
    
    \end{variants}
  
%\emph{Haṭharatnāvalī} 3.36–3.37
%
%\begin{versinnote}
%\tl{dattātreyo 'pi\\+}
%\tl{uttānau caraṇau kṛtvā ūrvoḥ saṃsthāpya yatnataḥ/\\+}
%\tl{ūrumadhye tathottānau pāṇī kṛtvā tato dṛśau//\\+}
%\tl{nāsāgre vinyased rājadantamūlaṃ ca jihvayā/\\+}
%\tl{uttabhya cibukaṃ vakṣaḥ saṃsthāpya pavanaṃ śanaiḥ//\\!}
%\end{versinnote}

%\emph{Yogacintāmaṇi} f.\,85v

%\begin{versinnote}
%\tl{dattātreyaḥ—\\+}
%\tl{uttānau caraṇau kṛtvā ūrusaṃsthau prayatnataḥ/\\+}
%\tl{ūrumadhye tathottānau pāṇī kṛtvā tato dṛśau//\\+}
%\tl{nāsāgre vinyased rājadantamūlaṃ tu jihvayā/\\+}
%\tl{uttabhya cibukaṃ vakṣasy utthāpya pavanaṃ śanaiḥ//\\+}
%\tl{yathāśaktyā samākṛṣya pūrayed udaraṃ śanaiḥ/\\+}
%\tl{yathāśaktyaiva paścāt tu recayet pavanaṃ śanaiḥ//\\!}
%\end{versinnote}
    
\end{testimonia}
%</ts46>

%<*cm46>
\begin{philcomm}[hp01_046]
%Both \emph{uttabhya} and \emph{uttambhya} are well attested
%The witnesses split predictably along the two main branches of the stemma. 
%\emph{vakṣasyāsthāpya} is a marmasthāna

%(MD 2024-5-14): It is remarkable that α2 and π1 have osthāpyot and ζ3 otthāpya. Perhaps o = ava as in the Middle Indic? avasthāpya would make the perfect sense.

The syntax of this verse as we have presented it is incomplete: at its end \emph{pavanaṃ śanaiḥ}, `the breath gradually', is left hanging. In the source text, the \emph{Dattātreyayogaśāstra}, the following verse completes the syntax with \emph{pūrayed}, “one should inhale”. Either Svātmārāma chose to leave the verse hanging (the following verse in the \emph{Dattātreyayogaśāstra} adds nothing about the form of the posture, which is the topic here) or the verse that completes the syntax fell out, perhaps because of a scribal error that happened early in the transmission. In the \emph{Dattātreyayogaśāstra} verses 36 and 37 both end with \emph{pavanaṃ śanaiḥ}, the repetition of which may have caused an eyeskip.\lb

The manuscript readings with \emph{vakṣa sthāpayet} (\piOmega\ and \etaTwo) or something similar (\etaOne) do not offer a solution to the incomplete syntax and do not indicate that Svātmārāma rewrote \emph{Dattātreyayogaśāstra} 36 so that he could omit \emph{Dattātreyayogaśāstra} 37. (The readings \emph{osthāpyot} (\alphaTwo\ and \piOne) and \emph{otthāpya} (\zetaThree) are surprising and are perhaps Middle Indic forms in which Sanskrit \emph{ava-} becomes \emph{o-}.) In the absence of evidence that Svātmārāma included \emph{Dattātreyayogaśāstra} 37 or wrote a coherent version of \emph{Haṭhapradīpikā} 1.46, we have made sense of \emph{pavanaṃ śanaiḥ} by adding “[draw in]” in our translation.\lb

Brahmānanda's comment on the statement, `having raised the root of the uvula with the tongue' (\emph{rājadantamūlaṃ ca jihvayā uttabhya}) in 1.46 is worth noting. In the context of Haṭhayoga, one would assume this statement to be referring to a type of  \emph{khecarīmudrā}, in which the tongue lifts the root of the uvula, here called the `royal tooth' (\emph{rājadanta}, on the meaning of which see Mallinson 2007: 209 n. 258). However, Brahmānanda understands it differently (synonyms omitted for clarity): 
\begin{versinnote}
\tl{rājadantānāṃ daṃṣṭrāṇāṃ savyadakṣiṇabhāge sthitānāṃ mūle ubhe mūlasthāne jihvayā uttambhya ūrdhvaṃ stambhayitvā/ gurumukhād avagantavyo'yaṃ jihvābandhaḥ/\\!}
\end{versinnote}\closer
\begin{quote}
Pushing against both roots of the front teeth on the left and right with the tongue [\ldots] — this fixation of the tongue has to be understood from the mouth of the teacher.
\end{quote}

Brahmānanda appears to have had in mind a probably older rule for meditation postures, according to which the tongue rests near the front teeth. One example of this is in \emph{Svacchandatantra} (4.365f.), which teaches a meditation pose called \emph{divyaṃ karaṇam}, in which the tongue is to rest at the tip of the teeth (\emph{dantāgre jihvām ādāya}). Other Tantric texts have this or similar rules, in which the tongue is supposed to rest either on the teeth or the palate, early examples being the \emph{Mṛgendrāgama}, \textit{yogapāda} 19 (\emph{dantāgre jihvām ādāya}) and \emph{Mataṅgapārameśvaratantra}, \textit{yogapāda} 2.27 (\emph{tālumadhyagatenaiva jihvāgreṇa}). Placing the tongue where it does not disturb the meditation seems quite appropriate for a `normal' meditative practice (The rule of placing the tongue at the palate is also found in \emph{Īśānaśivagurudevapaddhati} 18.120: \emph{tāluke jihvāṃ saṃyojya kiñcidvivṛtavaktro dantair dantān asaṃspṛśan ṛjukāyaḥ}. For similar references in tantric and other works see Mallinson 2007: 17–24.).\lb

When the context is haṭhayogic physiology, placing the tongue at the uvula, which is the source of `nectar', is more appropriate. Confusingly, yogic terminology includes many names for the uvula, and among these especially the term \emph{rājadanta} may give rise to confusion, since, as we have seen, the tongue might also in some yoga systems be placed at the front teeth.\lb

Furthermore, the haṭhayogic \emph{khecarīmudrā} has been described in manifold ways. Usually the tongue is said to be inserted into the cavity above the palate but in some cases it is placed at the uvula. Thus the tenth-century \emph{Mokṣopāya} (V.55.14c) says that the tongue rests at the `source of the palate' (\emph{tālumūlatalālagnajihvā}) and the commentary, the \emph{Saṃsārataraṇi}, on the parallel passage in \emph{Laghuyogavāsiṣṭha} V.6.155, which reads \emph{tālumūlāntarālagnajihvā}, explains that this means that the tongue is to be placed in the middle of the two regions of the palate, and that this is the \emph{nabhomudrā}, alias `\emph{khecarī}' (\emph{tālumūlāntarālagnajihvamūlaḥ tālumūlayoḥ kākudamūladeśayoḥ āntare lagnam ālagnam jihvāmūlam yasyety anena nabhomudrā darśitā}/ \emph{yā hi khecarı̄ty ucyate}).\lb

A little later in the \emph{Mokṣopāya} (V.78.24ab) it is made clear that one should reach the uvula, `at the root of the palate' (\emph{tālumūlagatāṃ yatnāj jihvayākramya ghaṇṭikām}). In view of this background we must conclude that the author of the \emph{Jyotsnā} was probably not aware of the yogic meaning of \emph{rājadanta} and has tried his best to make sense of the passage, echoing the idea of the two roots of the palate (although his text is not talking about the palate), but then referring to the instruction of the teacher for practical details, probably noticing that his literal interpretation is somewhat opaque. In addition to his commentary on 1.46 (translated above), Brahmānanda’s comments on \emph{rājadantasthajihvāyām} at 3.22 indicate that he thought the \emph{rājadanta} refers to the front teeth (\emph{kutaḥ}?\emph{yato dantānāṃ rājāno rājadantā rājadanteṣu tiṣṭhatīti rājadantasthāḥ, rājadantasthā cāsau jihvā ca tasyāṃ rājadantasthajihvāyāṃ bandhaḥ, taduparibhāgasya sambandhaḥ śastaḥ}).
\end{philcomm}
%</cm46>

%%%%%%%%%%
\subsection*{1.47}
%<*tr47>
\begin{translation}[hp01_047]
This is called the lotus pose [and] it cures all diseases. It is difficult for just anyone to accomplish; it is accomplished by a wise person [here] on earth.
\end{translation}
%</tr47>

%<*sc47>
\begin{sources}[hp01_047]
\emph{Dattātreyayogaśāstra} 38, \emph{Śivasaṃhitā} 3.105
  \begin{variants}
    proktaṃ DYŚ\vl ŚS~] nāma DYŚ\sep
    dhīmatā DYŚ ŚS~] dhīmatāṃ DYŚ\vl\sep
    bhuvi DYŚ~] param ŚS, yadi DYŚ\vl, hi vai DYŚ\vl
    \end{variants}
  
%\emph{Dattātreyayogaśāstra} 38
%
%\begin{versinnote}
%\tl{idaṃ padmāsanaṃ nāma sarvavyādhivināśanam/\\+}
%\tl{durlabhaṃ yena kenāpi dhīmatā labhyate bhuvi//\\!}
%\end{versinnote}
%\begin{appinnote}
%\tl{\textbf{38a} nāma~] P; ∗ma T, proktaṃ cett. \ \textbf{38d} dhīmatā~] dhīmatāṃ A • bhuvi~] yadi M1, hi vai A \\!}
%\end{appinnote}
%
%\emph{Śivasaṃhitā} 3.105
%
%\begin{versinnote}
%\tl{idaṃ padmāsanaṃ proktaṃ sarvavyādhivināśanam/\\+}
%\tl{durlabhaṃ yena kenāpi dhīmatā labhyate param//\\!}
%\end{versinnote}

\end{sources}
%</sc47>

%<*ts47>
\begin{testimonia}[hp01_047]
  \emph{Haṭharatnāvalī} 3.38, \emph{Yogacintāmaṇi} f.\,85v (\attr Dattātreya)

%\emph{Haṭharatnāvalī} 3.38
%
%\begin{versinnote}
%\tl{idaṃ padmāsanaṃ proktaṃ sarvavyādhivināśanam/\\+}
%\tl{durlabhaṃ yena kenāpi dhīmatā labhyate bhuvi//\\!}
%\end{versinnote}
%
%\emph{Yogacintāmaṇi} f.\,85v (attr. Dattātreya)
%
%\begin{versinnote}
%\tl{idaṃ padmāsanaṃ proktaṃ sarvavyādhivināśanam/\\+}
%\tl{durlabhaṃ yena kenāpi dhīmatā labhyate bhuvi//\\!}
%\end{versinnote}

\end{testimonia}
%</ts47>

%<*cm47>
\begin{philcomm}[hp01_047]
In this context, the word \emph{durlabham} is somewhat ambiguous as to whether the posture is hard to perform or hard to acquire (the more usual meaning). In commenting on \emph{durlabham} in \emph{Jyotsnā} 2.74, Brahmānanda glosses it as \emph{duṣprāpam}, which means `difficult to attain' and `inaccessible.'\lb 

The comment added to this verse by Svātmārāma indicates that he prefers the second version of \emph{padmāsana}, which derives from the \emph{Dattātreyayogaśāstra}, rather than the one he has borrowed from the \emph{Vivekamārtaṇḍa}. 
% Maybe add some remarks by commentators on how they understand it.
\end{philcomm}
%</cm47>

%%%%%%%%%%
\subsection*{1.47 ending}
%<*tr47p>
\begin{translation}[hp01_047p]
Only the second teaching [on the lotus pose] is approved by me.  
\end{translation}
%</tr47p>


%%%%%%%%%%
\subsection*{1.48}
%<*tr48>
\begin{translation}[hp01_048]
%Next is taught the doctrine of Matsya.

A man should put his hands together in a bowl shape, very firmly assume \emph{padmāsana}, place the chin tight on the chest and meditation in the mind. Raising the \emph{apāna} breath over and over again [and] releasing the inhaled \emph{prāṇa}, he attains unequalled knowledge through the power of the goddess [Kuṇḍalinī].
\end{translation}
%</tr48>

%<*sc48>
\begin{sources}[hp01_048]
  \emph{Vivekamārtaṇḍa} 36

  \begin{variants}
    baddhvā tu~] baddhvātha VM\sep
    dhyānaṃ VM\vl~] dhyāyaṃś VM\sep
    cetasi VM~] cetasaṃ VM\vl\sep
    proccālayan VM~] prodvārayaṃ VM\vl, proccālayet VM\vl, proccārayet VM\vl, prollāsayet VM\vl, proccārayan VM\vl\sep
    muñcan prāṇam upaiti bodhamatulaṃ śaktiprabhāvān naraḥ VM~] prāṇaṃ muṃcati bodham eti śanakaiḥ proktaprabhāvād ataḥ VM\vl, pāṇaṃ muṃcati bodham eti śanakaiḥ śaktiḥ prabhāvād ataḥ VM\vl, prāṇaṃ muṃcati yāti bodham amalaṃ śaktipradhānoditaḥ VM\vl, muṃcan prāṇam upaiti bodham atulaṃ śaktiprabhāvād ataḥ VM\vl, muñcan prāṇam upaiti bodham atulaṃ śaktiprabodhān naraḥ VM\vl, prāṇaṃ muñcati bodhameti śanakaiḥ śaktiprabodhān naraḥ VM\vl, muñcan prāṇam upaiti bodham akhilāṃ śaktiṃ prabhāvād ataḥ VM\vl
    \end{variants}

% NJL: The last entry of the apparatus seems a bit overwhelming. Keep it or filter some variants? JM: one to discuss
    
%\emph{Vivekamārtaṇḍa} 36
%
%\begin{versinnote}
%\tl{kṛtvā saṃpuṭitau karau dṛḍhataraṃ baddhvātha padmāsanaṃ\\+}
%\tl{gāḍhaṃ vakṣasi sannidhāya cibukaṃ dhyāyaṃś ca tac cetasi/\\+}
%\tl{vāraṃ vāram apānam ūrdhvam anilaṃ proccālayan pūritaṃ\\+}
%\tl{muñcan prāṇam upaiti bodham atulaṃ śaktiprabhāvān naraḥ//\\!}
%\end{versinnote}
%\begin{appinnote}
%\tl{\textbf{36b} dhyāyaṃś~] TH; dhyānaṃ VAGU •  °cetasi~] °cetasaṃ A \ %
%\textbf{36c} proccālayan~] ∗∗ T; pro cc ālayan V, prodvārayaṃ A, proccālayet GB, proccārayet GLGPk, prollāsayet GP, proccārayan U \ %
%\textbf{36d} muñcan prāṇamupaiti bodhamatulaṃ śaktiprabhāvān naraḥ~] U; prāṇaṃ muṃcati bodham eti śanakaiḥ proktaprabhāvād ataḥ V, pāṇaṃ muṃcati bodham eti śanakaiḥ śaktiḥ prabhāvād ataḥ A, prāṇaṃ muṃcati yāti bodham amalaṃ śaktipradhānoditaḥ GB, muṃcan prāṇam upaiti bodham atulaṃ śaktiprabhāvād ataḥ GLGP, muñcan prāṇam upaiti bodham atulaṃ śaktiprabodhān naraḥ GPk, prāṇaṃ muñcati bodhameti śanakaiḥ śaktiprabodhān naraḥ T, muñcan prāṇam upaiti bodham akhilāṃ śaktiṃ prabhāvād ataḥ Tvl \\!}
%\end{appinnote}

\end{sources}
%</sc48>

%<*ts48>
\begin{testimonia}[hp01_048]
  \emph{Haṭharatnāvalī} 3.39, \emph{Yogacintāmaṇi} f.\,79v

  \begin{variants}
     baddhvā tu HRĀ~] badhvā ca YCM\sep
     proccālayan~] proccālayat HRĀ\vl, proccārayet HRĀ, protsārayet YCM\sep
     pūritam HRĀ~] pūrayet YCM\sep
     muñcan prāṇam upaiti bodham~] muñcat prāṇam upaiti HRĀ, prāṇaṃ muñcati bodham eti YCM\sep
     atulaṃ śakteḥ prabhāvān naraḥ HRĀ~] niyataṃ śaktiprabodhodayāt YCM
    \end{variants}
  
%\emph{Haṭharatnāvalī} 3.39
%
%\begin{versinnote}
%\tl{kṛtvā saṃpuṭitau karau dṛḍhataraṃ baddhvā tu padmāsanam\\+}
%\tl{gāḍhaṃ vakṣasi sannidhāya cibukaṃ dhyānaṃ ca tac cetasi/\\+}
%\tl{vāraṃ vāram apānam ūrdhvam anilaṃ proccārayet pūritam\\+}
%\tl{muñcat prāṇam upaiti bodham atulaṃ śakteḥ prabhāvān naraḥ//\\!}
%\end{versinnote}
%\begin{appinnote}
%\tl{\textbf{c} proccārayet~] proccālayat P,T,t1,n2\\!}
%\end{appinnote}
%
%\emph{Yogacintāmaṇi} f.\,79v
%
%\begin{versinnote}
%\tl{tathā ca granthāntare—\\+}
%\tl{kṛtvā saṃpuṭitau karau dṛḍhataraṃ badhvā ca padmāsanam\\+}
%\tl{gāḍhaṃ vakṣasi saṃnidhāya cibukaṃ dhyānaṃ ca tac cetasi/\\+}
%\tl{vāraṃ vāram apānam ūrdhvam anilaṃ protsārayet pūrayet\\+}
%\tl{prāṇaṃ muñcati bodham eti niyataṃ śaktiprabodhodayāt//\\!}
%\end{versinnote}

\end{testimonia}
%</ts48>

%<*cm48>
\begin{philcomm}[hp01_048]
%The reading of \emph{matsyamataṃ} in the subheading is unusual because elsewhere Matsyendra is not referred to as simply Matsya. Furthermore, this verse is from the \emph{Vivekamārtaṇḍa} (36), and the teachings of the \emph{Vivekamārtaṇḍa} are attributed to Gorakṣa, not Matsyendra. However, in the extended recension of it known as the  \emph{Gorakṣaśataka},  \emph{Gorakṣasaṃhitā}, etc. the second verse includes homage to Mīna, i.e. Matsyendra.

The text at end of the second verse quarter is uncertain. Later witnesses, including Brahmānanda, have \emph{dhyāyaṃś ca} but none of the early ones has this reading. We are taking \emph{dhyānaṃ} with \emph{sannidhāya}, but this renders \emph{tat} problematic because it has no clear referent. In the source text, the \emph{Vivekamārtaṇḍa}, \emph{tat} appears to refer to the \emph{mokṣadvāra} broken by \emph{kuṇḍalinī}, which is mentioned in the previous verse. \lb

The two participles \emph{proccālayan} and \emph{muñcan} imply that the two things are happening at the same time, which is surprising but perhaps possible.

\end{philcomm}
%</cm48>

\begin{metre}[hp01_048]
Śārdūlavikrīḍita 
\end{metre}

%%%%%%%%%%
\subsection*{1.49}
%<*tr49>
\begin{translation}[hp01_049]
%The yogi who kills the breath by sitting in \emph{padmāsana} and filling himself up through the openings of the channels is liberated. In this there is no doubt.
The yogi who dies while sitting in lotus pose and filling himself up with air through the openings of the channels is liberated. In this there is no doubt.
\end{translation}
%</tr49>

%<*sc49>
%\begin{sources}[hp01_049]
%--
%\end{sources}
%</sc49>

%<*ts49>
\begin{testimonia}[hp01_049]
  \emph{Haṭharatnāvalī} 3.40, \emph{Yogacintāmaṇi} f.\,85v (\attr Dattātreya), \emph{Dhyānabindūpaniṣat} 70

  \begin{variants}
    padmāsanasthito YCM~] padmāsane sthito HRĀ DhBU\sep
    pūrayan YCM DhBU~] pūrayet HRĀ\sep
    mārutaṃ YCM DhBU~] pūritaṃ HRĀ\sep
    mārayed~] dhārayed YCM, dhriyate HRĀ, kumbhayan YCM DhBU 
    \end{variants}
  
%\emph{Haṭharatnāvalī} 3.40
%
%\begin{versinnote}
%\tl{padmāsane sthito yogī nāḍīdvāreṣu pūrayet/\\+}
%\tl{pūritaṃ dhriyate yas tu sa mukto nātra saṃśayaḥ//\\!}
%\end{versinnote}
%
%\emph{Yogacintāmaṇi} f.\,85v (attr. Dattātreya)
%
%\begin{versinnote}
%\tl{padmāsanasthito yogī nāḍīdvāreṣu pūrayan/\\+}
%\tl{mārutaṃ dhārayed yas tu sa mukto nātra saṃśayaḥ//\\!}
%\end{versinnote}
%
%\emph{Dhyānabindūpaniṣat} 70
%
%\begin{versinnote}
%\tl{padmāsanasthito yogī nāḍīdvāreṣu pūrayan/\\+}
%\tl{mārutaṃ kumbhayan yas tu sa mukto nātra saṃśayaḥ//\\!}
%\end{versinnote}

\end{testimonia}
%</ts49>

%<*cm49>
\begin{philcomm}[hp01_049]
It is not unusual to read forms from \emph{pṝ} with the locative as seen in this verse (cf.\,\emph{pūrayen mukhe} in \emph{Amaraugha} 21d) but it is not clear what is meant here. \emph{Siddha\-siddhānta\-paddhati} 1.66 identifies the ten \emph{dvāra}s at which the ten principle channels terminate, and adds that all other channels terminate at the pores of the skin. Perhaps the primary \emph{nāḍīdvāra}s are intended here, and the yogi is to fill himself up with air through each of them. Alternatively, the meaning could be that by filling the \emph{dvāra}s with air, the yogi is blocking them, thus enabling the stilling of the breath as described in some secondary readings in the second half of the verse. This appears to be how this verse is understood at \emph{Haṭhatattvakaumudī} 17.30 (cf.~\emph{Yogatattvopaniṣad} 141–142).\lb

The \emph{Jyotsnā} (1.49) has \emph{nāḍīdvāreṇa} instead of \emph{nāḍīdvāreṣu}, which Brahmānanda understands to mean “by way of the opening of the central channel (\emph{suṣumnāmārgeṇa})”. This yields the idea of filling up the central channel (as opposed to other channels), which is described in the \emph{Yogabīja} (94–95).\lb 

%\gammaOne's reading of \emph{niyatam} (instead of \emph{dhārayed}) in the third verse quarter explains the passive verbs in other witnesses. The passive verbs do not make sense with \emph{yas tu}. The passive verbs meaning to take in the breath (e.g.~\emph{pīyate}) may have been adopted to remove the reference to holding the breath because a breath retention is not mentioned in the previous verse describing \emph{padmāsana} (only inhalation and exhalation). \lb

In order to make sense of this verse we have understood \emph{mārutaṃ} to be read with the preceding line, as the object of \emph{pūrayan}. Such enjambement is rare in \emph{haṭha} texts (other than the \emph{Dattātreyayogaśāstra}) and is absent from the verses which we have identified as authorial. We have not identified a source text for this verse but the enjambement and its unusual teaching suggest that there may be one. The unusual enjambement accounts for many of the variants in the second line, with \alphaTwo \ reading \emph{mārutaṃ mārayed yas tu}, invoking an alchemical metaphor for stilling the breath as found in the \emph{Amṛtasiddhi} (see Mallinson 2025). The reading \emph{dhārayet} found in some later witnesses may have resulted from confusion over the meaning of \emph{mṛ} in this context.  % JM to change note: In \emph{pāda} c, \alphaOne\ and \alphaThree\ read \emph{mārutaṃ mriyate yas tu}, which does not make sense with the correlative clause \emph{sa mukto na saṃśayaḥ} in \sl{pāda} d. Nor too does \emph{māruto mriyate yas tu} as found in \zetaOne. We have thus adopted the reading of \alphaTwo, \emph{mārutaṃ mārayed yas tu}, which we are usually reluctant to do when it reads against \alphaOne \ and \alphaThree \ as it often patches problematic readings. \alphaTwo’s \emph{mārayet} is supported by the reading \emph{dhārayet} found in some later witnesses, which may have resulted from confusion over the meaning of \emph{mṛ} in this context.
% JM to Mitsuyo: is my explanation about alphatwo above ok? Or do those other readings not support it? [MD: The reading is unique to J5. Not supported by any other ms.]
%but J5 is contaminated (so the Gr3 reading dhārayet is more reliable).
% ed needs changing to mārutaṃ mārayed yas tu, and pūrayan
% some  mss from the full collation do have yasya
%MD 2024-8: the reading of α1 is "pūrayet/ mārutaṃ (not māruto) mriyate yas tu".
%JM: how about we adopt mārutaṃ mārayed yas tu (then no need for an emendation)? I rather like pūrayan too (with a causal sense: by filling/blocking the openings of the channels...). Agreed, but note alternatives. Go for pūrayan, but explain.
 
\end{philcomm}
%</cm49>

%%%%%%%%%%
\subsection*{1.50 heading}
%<*tr50a>
\begin{translation}[hp01_050a]
Now, the lion’s pose (\emph{siṃhāsana}).
\end{translation}
%</tr50a>

%<*cm50a>
% \begin{philcomm}[hp01_050a]
% \end{philcomm}
%</cm50a>

%%%%%%%%%%
\subsection*{1.50--52}
%<*tr52>
\begin{translation}[hp01_052]
[The yogi] should put both ankles at the sides of the perineal seam below the scrotum. He should place the left ankle on the right, the right ankle on the left and both hands on the knees, spread his fingers, open his mouth and gaze in deep concentration at the tip of his nose. This is the lion’s pose, which is always honoured by yogis. It causes the three locks to arise together and is the best of [all] postures.
\end{translation}
% JB: should we consider combining verses like this in other instances where we use [/dots] in our translations that run over two or more verses?
%</tr52>

%<*sc52>
\begin{sources}[hp01_052]
\emph{Vasiṣṭhasaṃhitā} 1.73–1.75ab (50–52ab only), \emph{Yogayājñavalkya} 3.9–3.11ab (50–52ab only), \emph{Sūtasaṃhitā} 15.7–8 (50–51 only)

\begin{variants}
  dakṣiṇe~] dakṣiṇaṃ VS YY SS\sep
  savyagulphaṃ tu~] savyagulphena VS YY SS\sep
  dakṣagulphaṃ tu savyake~] dakṣiṇenetaretaram VS, dakṣiṇena tathetaram YY, vāmaṃ dakṣiṇagulphataḥ SS\sep
  hastau ca jānvoḥ YY SS~] hastau jānau ca VS\sep
  svāṅgulīḥ saṃprasārya ca~] svāṅgulīś ca prasārya ca VS YY SS\sep
  vyāttavaktro VS YY~] nāsāgraṃ ca SS\sep
  nirīkṣeta VS SS~] nirīkṣet YY\sep
  nāsāgraṃ susamāhitaḥ VS YY~] bhavet siṃhāsanaṃ hi tat SS\sep
  %siṃhāsanaṃ bhaved etat VS YY~] \emph{om.}~SS\sep
  %pūjitaṃ yogibhiḥ sadā VS YY~] \emph{om.}~SS\sep
  %bandhatritayasaṃdhānaṃ kurute cāsanottamam~] \emph{om.}~VS YY SS
\end{variants}
% JB: I have been indicating omissions this way: \\emph{Sūtasaṃhitā} 15.7–8 (50–51 only) rather than reproduce the HP's pādas in the apparatus.
% JB: and should \emph{om.}~be \emph{om.}?

%\emph{Vasiṣṭhasaṃhitā} 1.73–1.75ab

%\begin{versinnote}
%\tl{gulphau ca vṛṣaṇasyādhaḥ sīvanyāḥ pārśvayoḥ kṣipet/\\+}
%\tl{dakṣiṇaṃ savyagulphena dakṣiṇenetaretaram//\\+}
%\tl{hastau jānau ca saṃsthāpya svāṅgulīś ca prasārya ca/\\+}
%\tl{vyāttavaktro nirīkṣeta nāsāgraṃ susamāhitaḥ//\\+}
%\tl{siṃhāsanaṃ bhaved etat pūjitaṃ yogibhiḥ sadā/\\!}
%\end{versinnote}

%\emph{Yogayājñavalkya} 3.9–3.11ab
%
%\begin{versinnote}
%\tl{gulpau ca vṛṣaṇasyādhaḥ sīvanyāḥ pārśvayoḥ kṣipet/\\+}
%\tl{dakṣiṇaṃ savyagulphena dakṣiṇena tathetaram//\\+}
%\tl{hastau ca jānvoḥ saṃsthāpya svāṅgulīś ca prasārya ca/\\+}
%\tl{vyāttavaktro nirīkṣet nāsagraṃ susamāhitaḥ//\\+}
%\tl{siṃhāsanaṃ bhaved etat pūjitaṃ yogibhiḥ sadā/\\!}
%\end{versinnote}

%\emph{Sūtasaṃhitā} 15.7–8
%
%\begin{versinnote}
%\tl{gulphau ca vṛṣaṇasyādhaḥ sīvanyāḥ pārśvayoḥ kṣipet/\\+}
%\tl{dakṣiṇaṃ savyagulphena vāmaṃ dakṣiṇagulphataḥ//\\+}
%\tl{hastau ca jānvoḥ saṃsthāpya svāṅgulīś ca prasārya ca/\\+}
%\tl{nāsāgraṃ ca nirīkṣeta bhavet siṃhāsanaṃ hi tat//\\!}
%\end{versinnote}

\end{sources}
%</sc52>

%<*ts52>
\begin{testimonia}[hp01_052]
  \emph{Haṭharatnāvalī} 3.31–3.33, \emph{Yogacintāmaṇi} f.\,83v (\attr Yājñavalkya)

  \begin{variants}
    dakṣiṇe HRĀ~] dakṣiṇaṃ YCM\sep
    savyagulphaṃ tu~] savyagulphaṃ ca HRĀ, savyagulphena YCM\sep
    dakṣagulphaṃ tu savyake~] dakṣiṇe tu tathetaram HRĀ, dakṣiṇena tathetaram YCM\sep
    ca jānvoḥ saṃsthāpya~] tu jānvoḥ saṃsthāpya HRĀ, jānūpari sthāpya YCM\sep
    susamāhitaḥ YCM~] tu samāhitaḥ HRĀ\sep
    pūjitaṃ YCM~] sevitaṃ HRĀ\sep
    bandhatritayasaṃsthānaṃ kurute cāsanottamam HRĀ~] \emph{om.}~YCM
    \end{variants}
 
%\emph{Haṭharatnāvalī} 3.31–3.33
%
%\begin{versinnote}
%\tl{atha siṃhāsanam\\+}
%\tl{gulphau ca vṛṣaṇasyādhaḥ sīvanyāḥ pārśvayoḥ kṣipet/\\+}
%\tl{dakṣiṇe savyagulphaṃ ca dakṣiṇe tu tathetaram//\\+}
%\tl{hastau tu jānvoḥ saṃsthāpya svāṅgulīḥ samprasārya ca/\\+}
%\tl{vyāttavaktro nirīkṣeta nāsāgraṃ tu samāhitaḥ//\\+}
%\tl{siṃhāsanaṃ bhaved etat sevitaṃ yogibhiḥ sadā/\\+}
%\tl{bandhatritayasaṃsthānaṃ kurute cāsanottamam//\\!}
%\end{versinnote}

%\emph{Yogacintāmaṇi} f.\,83v (attr. Yājñavalkya)
%
%\begin{versinnote}
%\tl{gulphau ca vṛṣaṇasyādhaḥ sīvanyāḥ pārśvayoḥ kṣipet/\\+}
%\tl{dakṣiṇaṃ savyagulphena dakṣiṇena tathetaram//\\+}
%\tl{hastau jānūpari sthāpya svāṅgulīḥ saṃprasārya ca/\\+}
%\tl{vyāttavaktro nirīkṣeta nāsāgraṃ susamāhitaḥ/\\+}
%\tl{siṃhāsanaṃ bhaved etat pūjitaṃ yogibhiḥ sadā/\\!}
%\end{versinnote}

\end{testimonia}
%</ts52>

%<*cm52>
\begin{philcomm}[hp01_052]
% comment on nyastalocanaḥ and susamāhitaḥ. The latter is well attested by the sources and testimonia, and is in the V19 and N23 branch. However, most of the old mss. have nyastalocanaḥ. Maybe comment on the meaning of sandhāna, which seems to mean simply that the three locks arise together (Cf.~a similar comment on siddhāsana in 1.42). 

Spreading the fingers and keeping the mouth wide open mimic a lion, and this is depicted in some iconography of Yoganarasiṃha (for example, Yoga Narasimha, Vishnu's Man-Lion Incarnation, Samuel Eilenberg Collection, Bequest of Samuel Eilenberg, 1998, Accession Number: 2000.284.4. \url{https://www.metmuseum.org/art/collection/search/39251}).\lb
%

%The \emph{Yogacintāmaṇi} attributes its citation of these verses on \emph{siṃhāsana} to Yājñavalkya. Its citation does not include \emph{Haṭhapradīpikā} 1.52cd, which affirms that 1.52cd is not from the \emph{Yogayājñavalkya} (or \emph{Vasiṣṭhasaṃhitā}). 
As far as we are aware, there is no source for the line (1.52cd) mentioning the three locks, so it may have been composed by Svātmārāma or borrowed from a lost work. 
\end{philcomm}
%</cm52>

\begin{metre}[hp01_051]
Anuṣṭubh (a: ma-vipulā)
\end{metre}

%%%%%%%%%%
\subsection*{1.53 heading}
%<*tr53a>
\begin{translation}[hp01_053a]
Now, the friendly pose (\emph{bhadrāsana}).
\end{translation}
%</tr53a>

%<*cm53a>
% \begin{philcomm}[hp01_053a]
% \end{philcomm}
%</cm53a>

%%%%%%%%%%
\subsection*{1.53--54}
%<*tr54>
\begin{translation}[hp01_054]
[The yogi] should put both ankles at the sides of the perineal seam below the scrotum.  By firmly and very steadily holding the sides of the feet with the hands, the friendly pose arises, which cures all diseases and poisons. Yogis of the Siddha tradition call it Gorakṣa's pose (\emph{gorakṣāsana}).
\end{translation}
%</tr54>

%<*sc54>
\begin{sources}[hp01_054]
  \emph{Vasiṣṭhasaṃhitā} 1.79 (only 53–54ab), \emph{Yogayājñavalkya} 3.11cd--3.12ab (only 53–54ab)

  \begin{variants}
    kṣipet YY~] kṣipan VS\sep
    %gorakṣāsanam ity āhur idaṃ vai siddhayoginaḥ~] \emph{om.}~VS YY
    \end{variants}
    %JB omission?
  
%\emph{Vasiṣṭhasaṃhitā} 1.79
%
%\begin{versinnote}
%\tl{gulphau ca vṛṣaṇasyādhaḥ sīvanyāḥ pārśvayoḥ kṣipan/\\+}
%\tl{pārśvapādau ca pāṇibhyāṃ dṛḍhaṃ baddhvā suniścalam/\\+}
%\tl{bhadrāsanaṃ bhaved etat sarvavyādhiviṣāpaham//\\!}
%\end{versinnote}
%
%\emph{Yogayājñavalkya} 3.11cd--3.12ab
%
%\begin{versinnote}
%\tl{gulphau ca vṛṣaṇasyādhaḥ sīvanyāḥ pārśvayoḥ kṣipet\\+}
%\tl{pārśvapādau ca pāṇibhyāṃ dṛḍhaṃ baddhvā suniścalam\\+}
%\tl{bhadrāsanaṃ bhaved etat sarvavyādhiviṣāpaham\\!}
%\end{versinnote}

\end{sources}
%</sc54>

%<*ts54>
\begin{testimonia}[hp01_054]
  \emph{Haṭharatnāvalī} 3.30 (only 53–54ab), \emph{Yogacintāmaṇi} f.\,83v  (\attr Yājñavalkya) (only 53–54ab)

  %\begin{variants}
    %atha bhadrāsanam HRĀ~] \emph{om.}~YCM\sep
    %gorakṣāsanam ity āhur idaṃ vai siddhayoginaḥ~] \emph{om.}~HRĀ YCM
   %\end{variants}
    % JB omission?
    
%\emph{Haṭharatnāvalī} 3.30 
%
%\begin{versinnote}
%\tl{atha bhadrāsanam\\+}
%\tl{gulphau ca vṛṣaṇasyādhaḥ sīvanyāḥ pārśvayoḥ kṣipet/\\+}
%\tl{pārśvapādau ca pāṇibhyāṃ dṛḍhaṃ baddhvā suniścalam//\\+}
%\tl{bhadrāsanaṃ bhaved etat sarvavyādhiviṣāpaham//\\!}
%\end{versinnote}
%
%\emph{Yogacintāmaṇi} f.\,83v (citing Yājñavalkya)
%
%\begin{versinnote}
%\tl{gulphau ca vṛṣaṇasyādhaḥ sīvanyāḥ pārśvayoḥ kṣipet/\\+}
%\tl{pārśvapādau ca pāṇibhyāṃ dṛḍhaṃ badhvā suniścalaḥ/\\+}
%\tl{bhadrāsanaṃ bhaved etat sarvavyādhiviṣāpaham/\\!}
%\end{versinnote}

\end{testimonia}
%</ts54>

%<*cm54>
\begin{philcomm}[hp01_054]
We have understood \emph{pārśvapāda} as a \emph{ekadeśitatpuruṣa} meaning the side of the foot, like  \emph{agrapāda}, the toes.\lb

Manuscripts of two early groups, \textgamma\ and \textpi, as well as the \emph{Jyotsnā} (1.53), include an additional line specifying that the left ankle is placed on the left side and the right ankle on the right (\emph{savya\-gulphaṃ tathā savye dakṣagulphaṃ tu dakṣiṇe}). This line appears to have been added to make it clear that the ankles are not crossed in \emph{bhadrāsana}, unlike the previous pose, \emph{siṃhāsana}.

\end{philcomm}
%</cm54>

%%%%%%%%%%
% \subsection*{1.54 endding}
%<*tr54p>
% \begin{translation}[hp01_054p]
% % ity āsanāni
% \end{translation}
%</tr54p>

%<*cm54p>
% \begin{philcomm}[hp01_054p]
% \end{philcomm}
%</cm54p>

%%%%%%%%%%
\subsection*{1.55}
%<*tr55>
\begin{translation}[hp01_055]
When the great yogi does not tire from adopting the \emph{āsana}s in this way, he should  practise the breath techniques with seals and so forth, from which purification of the channels arises.% Write to Haru to ask about nāḍiśuddhiṃ as a b/v
\end{translation}
%</tr55>

%<*sc55>
%\begin{sources}[hp01_055]
%\end{sources}
%</sc55>

%<*ts55>
\begin{testimonia}[hp01_055]
\emph{Yogacintāmaṇi} f.\,85v (\attr Dattātreya), \emph{Haṭhasaṅketacandrikā} f.\,23r

\begin{variants}
  bandheṣu YCM~] bandhastho HSC\sep
  vijitaśramaḥ YCM~] vigataśramaḥ HSC\sep
  athābhyasen nāḍiśuddhiṃ~] athābhyasen nāḍiśuddhiṃ HSC (\emph{em.}), athābhyasen nāhiśuddhi HSC, abhyasen nāḍiśuddhiṃ ca YCM\sep
  mudrādi HSC~] mudrayā YCM
  \end{variants}


% \emph{Gorakṣaśataka} 59c–60b
% \begin{variants}
% pādau GŚ\vl~] jānū (\emph{em.}), jānu GŚ\vl, prādau GŚ\vl    
% \end{variants}
  
%\emph{Yogacintāmaṇi} f.\,85v (attr. Dattātreya)
%
%\begin{versinnote}
%\tl{evam āsanabandheṣu yogīndro vijitaśramaḥ/\\+}
%\tl{abhyasen nāḍiśuddhiṃ ca mudrayā pavanakriyām// iti//\\!}
%\end{versinnote}
%
%\emph{Haṭhasaṅketacandrikā} f.\,23r
%
%\begin{versinnote}
%\tl{evam āsanabandhastho yogīndro vigataśramaḥ/\\+}
%\tl{athābhyasen nāḍiśuddhiṃ mudrādipavanakriyām//\\!}
%\end{versinnote}
%\begin{appinnote}
%\tl{\textbf{c} nāḍiśuddhiṃ~] \emph{em.}, nāhiśuddhi ms. no. 2244\\!}
%\end{appinnote}

\end{testimonia}
%</ts55>

%<*cm55>
\begin{philcomm}[hp01_055]
The second line can be interpreted in different ways. One possibility is to understand \emph{nāḍiśuddhiṃ} as a \emph{bahuvrīhi} qualifying \emph{mudrādipavanakriyām} in the sense that the yogi should practise the breathing techniques by way of the relevant \emph{mudrā}s and locks (taught in the third chapter), from which purification of the channels arise. Alternatively, one could separate \emph{mudrādi} from \emph{pavanakriyām} and understand three different techniques here, namely, the practice of purifying the channels (perhaps by the alternative nostril method mentioned at the beginning of the second chapter), the \emph{mudrā}s and the breathing techniques of \emph{prāṇāyāma}. The absence of a conjunctive particle, such as \emph{ca}, makes the second interpretation less likely. The version of this verse in the \emph{Yogacintāmaṇi} (cited in the testimonia) was changed to make it clear that \emph{nāḍiśuddhi} and \emph{pavanakriyā} with \emph{mudrā}s are two distinct things. 

%  
\end{philcomm}
%</cm55>

\begin{metre}[hp01_055]
Anuṣṭubh (c: ra-vipulā)
\end{metre}

%%%%%%%%%%
\subsection*{1.55*1--2}
%<*tr55-1>
\begin{translation}[hp01_055_1]
Success arises for one engaged in practice. How can it arise for one who has no practice? Success in yoga does not arise by merely reading scriptures.
\end{translation}
% JB; just a thought: should we grayscale our translation of grayscaled verses? JM: good idea, some people won't look at the Sanskrit
%</tr55-1>

%<*tr55-2>
\begin{translation}[hp01_055_2]
Wearing a robe does not bring about success, nor does talking [about yoga]. Practice alone is the cause of success. This is true, there is no doubt. In this system, [the practice] should not be given to one who wears robes and is devoted to sex and food.
\end{translation}
%</tr55-2>

%<*sc55-2>
\begin{sources}[hp01_055_2]
  \emph{Dattātreyayogaśāstra} 42cd–43ab, 46–47ab

  \begin{variants}
    yogasiddhiḥ~] kā cit siddhiḥ DYŚ\sep
    etan na saṃśayaḥ~] eva tu sāṃkṛte DYŚ\sep
    śiśnodararatāyeha~] śiśnodarārthaṃ yogasya DYŚ\sep
    na deyā~] kathayā DYŚ
    \end{variants}

 
%\emph{Dattātreyayogaśāstra} 42cd–43ab, 46–47ab 
%
%\begin{versinnote}
%%\tl{kriyāyuktasya siddhiḥ syād akriyasya kathaṃ bhavet// 42//\\+}
%%\tl{na śāstrapāṭhamātreṇa kā cit siddhiḥ prajāyate/\smallskip\\+}
%%\tl{na veṣadhāraṇaṃ siddheḥ kāraṇaṃ na ca tatkathā/\\+}
%%\tl{kriyaiva kāraṇaṃ siddheḥ satyam eva tu sāṃkṛte// 46//\\+}
%%\tl{śiśnodarārthaṃ yogasya kathayā veṣadhāriṇaḥ/\\+}
%\tl{anuṣṭhānavihīnās tu vañcayanti janān kila// 47//\\!}
%\end{versinnote}
%
\end{sources}
%</sc55-2>

%<*ts55-2>
%\begin{testimonia}[hp01_055_2]
%\end{testimonia}
%</ts55-2>

%<*cm55-2>
\begin{philcomm}[hp01_055_2]
1.55*1–2 are omitted from the \textalpha, \textgamma, \textdelta\ and \textzeta\ groups, so it is likely these verses were not in the earliest versions of the \emph{Haṭhapradīpikā}. In fact, it appears that both were added (perhaps initially as marginal notes) to elaborate on the word \emph{kriyā} in 1.55d. Both verses are similar to verses from the \emph{Dattātreyayogaśāstra} (cited as the source). However, only the first half of \emph{Dattātreyayogaśāstra} 47 is given in these later versions of the \emph{Haṭhapradīpikā}, resulting in a near-nonsensical line. Also, the syntax of 1.55.2ef is corrupt. One has to emend to \emph{deyā} to make sense of it. These verses (except 1.55.2ef) appear in the \emph{Jyotsnā} (1.65–66), but towards the end of chapter one. 
\end{philcomm}
%</cm55-2>


%%%%%%%%%%
\subsection*{1.55*3}
%<*tr55-3>
\begin{translation}[hp01_055_3]
Did this empty bubble we call the universe dissolve or arise in me, the pure ocean of awakening? Where does [this] veil of doubt come from?
\end{translation}
%</tr55-3>

%<*sc55-3>
\begin{sources}[hp01_055_3]
\emph{Tattvaratnāvalī} 24
% MD: perhaps better to print the verse, also for the testimonia. JM: agreed
  % \begin{variants}
  %   mayi bodhāmbhodhau~] bodhāmbhodhau mayi TR\sep
  %   svacche~] svacchaṃ TR\sep
  %   tuccho' yaṃ~] tac chāyam TR\sep
  %   °budbudaḥ~] °buddhayaḥ TR\sep
  %   pralīna udito veti~] udito vā pralīno vā TR\sep
  %   vikalpapaṭalaḥ kutaḥ~] na vikalpāya kalpate TR 
  %   \end{variants}

\begin{versinnote}
\tl{bodhāmbhodhau mayi svacchaṃ tac chāyam viśvabuddhayaḥ/\\+}
\tl{udito vā pralīno vā na vikalpāya kalpate//\\!}
\end{versinnote}

\end{sources}
%</sc55-3>

%<*ts55-3>
\begin{testimonia}[hp01_055_3]
  % \emph{Vārāhītantra} Ed. p.\,158

  % \begin{variants}
  %   mayi bodhāmbudhau~] mayi bodhībudho VT\sep
  %   svacche~] svasthe VT\sep
  %   tuccho' yaṃ~] tucho yaṃ VT\sep
  %   pralīna udito veti~] malīna udito vetti VT\sep
  %   vikalpapaṭlāḥ kutaḥ~] vikalpāvasaraḥ kutaḥ VT
  %   \end{variants}
  
\emph{Vārāhītantra} p.\,158

\begin{versinnote}
\tl{mayi bodhībudho svasthe tucho yaṃ viśvabudbudaḥ/\\+}
\tl{malīna udito vetti vikalpāvasaraḥ kutaḥ//\\!}
\end{versinnote}

% Cf.~\emph{Haṭhapradīpikā} (10 chapters) 3.7

% \begin{versinnote}
% \tl{śiśnodararatāya hi na deyaṃ veṣadhāriṇe//\\+}
% \tl{mayi bodhyaṃ buddhau svacche tad dheyaṃ viśvabudbudam//\\!}
% \end{versinnote}

% Cf.~\emph{Yogaprakāśikā} 3.7

% \begin{versinnote}
% \tl{“śiśnodararatāyaitan na deya" etat yogajñānam etena śiśnodararatas tyājyo nanv etanmate tyājyapadārtho 'pra\skx{-}{}siddha iti śaṃkāṃ nirasyati mayi iti svacche bodhasvarūpasamudre budbudatulyasya viśvasya heyatvād iti bhāvaḥ\\!}
% \end{versinnote}

\end{testimonia}
%</ts55-3>

%<*cm55-3>
\begin{philcomm}[hp01_055_3]
Verse 1.55.3 is only found in manuscripts of the \textdelta\ group. It is very difficult to find a reason why it should be inserted here. It is apparently a \emph{muktaka} that would befit an accomplished mystical poem more than an instructional manual, like the \emph{Haṭhapradīpikā}, even here, in what appears as a sort of miscellaneous section at the end of a chapter. In this verse, the lyrical subject wonders about why the mind is still able to doubt, despite its insight into the nature of reality. The reader might wonder how this illusionist verse could be understood to fit this yoga text. We can only speculate that perhaps the scribe of the hyparchetype of the \textdelta\ manuscripts was fond of it.\lb

The source is, as far as we can say, the \emph{Śāntiśataka} of the Kashmirian poet Sillana or Silhaṇa. The manuscripts of the \emph{Svātmopalabdhiśataka} give the name as \emph{Sillana}, the mostly Bengali manuscripts of the \emph{Śāntiśataka} read \emph{Śilhaṇa}, as does Aufrecht in his \emph{Catalogus Catalogorum}, 1891 (for further details see Hanneder 2026). Sillana cannot be dated with any certainty but predates the \emph{Haṭhapradīpikā} by a few centuries. The edition of the \emph{Śāntiśataka} – where a hundred original verses had to be identified – places the verse in question into an appendix of doubtful stanzas (see Schönfeld 1910: 90 [A9]). However, the editor did not provide a compelling reason to regard it as unoriginal except only the fact that it is not transmitted in all manuscripts. What prevents further investigation of the matter is the lack of Kashmirian manuscripts for the \emph{Śāntiśataka} and its compilatory character: one quarter of the material is identical with Bhartṛhari's \emph{Vairāgyaśataka}. A still superficial glance at Sillana's \emph{Svātmopalabdhiśataka} gives the impression that our verse would fit there, but not so much in the \emph{Śāntiśataka}. Perhaps its first citation is in Advayavajra’s \emph{Tattva\-ratnāvalī} (24). While these are only preliminary observations the verse is likely not original to the \emph{Haṭhapradīpikā}.
\end{philcomm}
%</cm55-3>

%%%%%%%%%%
\subsection*{1.55*4}
%<*tr55-4>
\begin{translation}[hp01_055_4]
Realisation from scripture, realisation from one's own guru, realisation from oneself, and the cessation of mind; all these methods have been combined and taught by the wise in this tradition.%
\end{translation}
%</tr55-4>

%<*sc55-4>
%\begin{sources}[hp01_055_4]
%\end{sources}
%</sc55-4>

%<*ts55-4>
\begin{testimonia}[hp01_055_4]
  \emph{Yogacintāmaṇi} f.\,48v (\attr HP)

  \begin{variants}
    manaso' pi rodhaḥ~] manaso nirodhaḥ YCM    
    \end{variants}
  
%\emph{Yogacintāmaṇi} f.\,48v
%
%\begin{versinnote}
%\tl{haṭhapradīpikāyām--\\+}
%\tl{śrutipratītiḥ svagurupratītiḥ\\+}
%\tl{svātmapratītir manaso nirodhaḥ/\\+}
%\tl{etāni sarvāṇi samuccitāni\\+}
%\tl{matāni dhīrair iha sādhanāni//\\!}
%\end{versinnote}
%
\end{testimonia}
%</ts55-4>

%<*cm55-4>
\begin{philcomm}[hp01_055_4]
Verse 1.55.4 is in some of the \textdelta\ manuscripts and is quoted in \emph{Yogacintāmaṇi} with attribution to the \emph{Haṭhapradīpikā}.\lb

The reading in the \emph{Yogacintāmaṇi} `cessation of mind' (\emph{manaso nirodhaḥ}) is better than \emph{manaso 'pi bodhaḥ} (the \textdelta\ reading) in a yogic context.
\end{philcomm}
%</cm55-4>

\begin{metre}[hp01_055_4]
Upajāti
\end{metre}

%%%%%%%%%%
\subsection*{1.56}
%<*tr56>
\begin{translation}[hp01_056]
%The various \emph{āsana}s, breath retention, bodily techniques (\emph{karaṇa}) called seals (\emph{mudrā}), and then concentrating on the internal resonance is the sequence of practice in Haṭha.%JM: 
Posture, manifold breath retention, the bodily technique called seal, then concentration on the internal sound is the sequence of practice in Haṭha.
\end{translation}
%</tr56>

%<*sc56>
%\begin{sources}[hp01_056]
%\end{sources}
%</sc56>

%<*ts56>
\begin{testimonia}[hp01_056]
    \emph{Yogacintāmaṇi} f.\,111v (\attr HP)

  \begin{variants}
    °nukramo haṭhe~] °nukrameṇa ca YCM %@MD
    \end{variants}

    %\emph{Yogacintāmaṇi} f.\,111v
%
%\begin{versinnote}
%\tl{haṭhapradīpikāyām--\\+}
%\tl{āsanaṃ kumbhakaṃ citraṃ mudrākhyaṃ karaṇaṃ tathā/\\+}
%\tl{atha nādānusandhānam abhyāsānukrameṇa ca//\\!}
%\end{versinnote}
%
\end{testimonia}
%</ts56>

%<*cm56>
\begin{philcomm}[hp01_056]
This verse was omitted from \etaOne, the oldest dated manuscript. The omission is probably deliberate as that manuscript does not have chapter four, which teaches \emph{nādānusan\-dhāna} and the numbering in \etaOne\ indicates that its exemplar included the verse.\lb

%The term \emph{kumbhaka} is almost always masculine [ neuter also in 2.53, 2.68 and the heading before 2.72 (MD)] but appears in this verse as a neuter in the majority of manuscripts of the important groups.\lb

%Marmasthāna: not clear whether to adopt \emph{citro} or \emph{citraṃ}, or V19’s \emph{mudrākhyaṃ karaṇaṃ tathā} or the others’ \emph{mudrādikaraṇāni ca}.

This verse is similar to 1.65, which has \emph{mudrādikaraṇāni ca}, so perhaps it was through confusion with 1.65 that the same reading is found in some witnesses of 1.56. It seems that each of the four auxiliaries (\emph{aṅga}) of Haṭhayoga is being referred to in the singular (hence \emph{āsanaṃ}), whereas in 1.65 the plural is used (i.e.~\emph{pīṭhāni}) which is why we have adopted the reading \emph{citraṃ} [...] \emph{karaṇaṃ tathā}.

%Yes, V19 reading probably best.
%
%\begin{versinnote}
%\tl{āsanaṃ kumbhakaś citraṃ mudrākhyaṃ karaṇaṃ tathā/\\+}
%\tl{atha nādānusandhānam abhyāsānukramo haṭhe//\\!}
%\end{versinnote}
\end{philcomm}
%</cm56>

%%%%%%%%%%
\subsection*{1.57}
%<*tr57>
\begin{translation}[hp01_057]
Celibate, having a measured diet and devoted to yoga, the yogi becomes an adept after a year. No doubt about this should be entertained.
\end{translation}
%</tr57>

%<*sc57>
\begin{sources}[hp01_057]
\emph{Vivekamārtaṇḍa} 38

\begin{variants}
  yogī~] tyāgī VM\vl
  \end{variants}
  
%\emph{Vivekamārtaṇḍa} 38
%
%\begin{versinnote}
%\tl{brahmacārī mitāhārī yogī yogaparāyaṇaḥ/\\+}
%\tl{abdād ūrdhvaṃ bhavet siddho nātra kāryā vicāraṇā//\\!}
%\end{versinnote}
%\begin{appinnote}
%\tl{\textbf{b} yogī~] VT; tyāgī AG\\!}
%\end{appinnote}

\end{sources}
%</sc57>

%<*ts57>
\begin{testimonia}[hp01_057]
\emph{Haṭharatnāvalī} 3.28, \emph{Yogacintāmaṇi} f.\,111v (\attr HP)

\begin{variants}
  yogī HRĀ\vl~] tyāgī HRĀ YCM
  \end{variants}
  
%\emph{Haṭharatnāvalī} 3.28
%
%\begin{versinnote}
%\tl{brahmacārī mitāhārī tyāgī yogaparāyaṇaḥ//\\+}
%\tl{abdād ūrdhvaṃ bhavet siddho nātra kāryā vicāraṇā//\\!}
%\end{versinnote}
%\begin{appinnote}
%\tl{\textbf{b} tyāgī~] yogī P,T,t1 \\!}
%\end{appinnote}
%
%\emph{Yogacintāmaṇi} f.\,111v (\attr HP)
%
%\begin{versinnote}
%\tl{brahmacārī mitāhārī tyāgī yogaparāyaṇaḥ/\\+}
%\tl{abdād ūrdhvaṃ bhavet siddho nātra kāryā vicāraṇā//\\!}
%\end{versinnote}

\end{testimonia}
%</ts57>

%<*cm57>
\begin{philcomm}[hp01_057]
The readings \emph{tyāgī} and \emph{yogī} are both well attested in \emph{Haṭhapradīpikā} 1.57b. The confusion between the two appears to have started early in the transmission of the \emph{Vivekamārtaṇḍa}. The occurrence of \emph{tyāgī} in \emph{Vivekamārtaṇḍa} 37 may be a dittographical type of mistake because the word \emph{tyāgī} is in the previous line of that work. But it is more difficult to determine whether Svātmārāma used a manuscript of the \emph{Vivekamārtaṇḍa} with \emph{tyāgī} or \emph{yogī} in verse 37. Since the best \textalpha\ manuscript has \emph{yogī}, as well as \etaOne\ and many others, we have tentatively adopted it bearing in mind that it was changed early in the transmission of the \emph{Haṭhapradīpikā}, most likely by a scribe who knew the reading of \emph{tyāgī} in a manuscript of the \emph{Vivekamārtaṇḍa}. 
\end{philcomm}
%</cm57>

%%%%%%%%%%
\subsection*{1.58}
%<*tr58>
\begin{translation}[hp01_058]
When very unctuous and sweet food is eaten for love of Śiva, leaving a quarter [of the stomach] empty, that is called a measured diet.
\end{translation}%JM:we very rarely cite the Sanskrit in the translation and when we do (only āsanas in ch.1) we use stem forms. Here we have both. I suggest deleting both. (I  changed restricted to measured for consistency with 1.38.)
%</tr58>

%<*sc58>
\begin{sources}[hp01_058]
\emph{Gorakṣaśataka} 12cd–13ab
\mylb

%\emph{Gorakṣaśataka} 12cd–13ab
%
%\begin{versinnote}
%\tl{susnigdhamadhurāhāraś caturthāṃśavivarjitaḥ//\\+}
%\tl{bhujyate śivasaṃprītyai mitāhāraḥ sa ucyate/\\!}
%\end{versinnote}

\end{sources}
%</sc58>

%<*ts58>
\begin{testimonia}[hp01_058]
  \emph{Yuktabhavadeva} 4.16 (\attr HP), \emph{Yogacūḍāmaṇyupaniṣat} 43

  \begin{variants}
    samprītyai YBhD~] saṃprītyā YCU\sep
    bhujyate YBhD~] bhuñjate YCU\sep
    mitāhāraḥ YBhD~] mitāhārī YCU
    \end{variants}

%
%% 
%\emph{Yuktabhavadeva} 4.16 (\attr HP)
%
%\begin{versinnote}
%\tl{tad uktaṃ haṭhapradīpikāyām--\\+}
%\tl{susnigdhamadhurāhārāś caturthāṃśavivarjitaḥ/\\+}
%\tl{bhujyate śivasamprītyai mitāhāraḥ sa ucyate//\\!}
%\end{versinnote}
%
%\emph{Yogacūḍāmaṇyupaniṣat} 43
%
%\begin{versinnote}
%\tl{susnigdhamadhurāhāraś caturthāṃśavivarjitaḥ/\\+}
%\tl{bhuñjate śivasaṃprītyā mitāhārī sa ucyate//\\!}
%\end{versinnote}

\end{testimonia}
%</ts58>

%<*cm58>
\begin{philcomm}[hp01_058]
This verse probably derives from the ‘original’ \emph{Gorakṣaśataka} (12c–13b). It is also found, but reworked to be about the \emph{mitāhārī}, in Nowotny’s \emph{Gorakṣaśataka} (55), which is an extended recension of the \emph{Vivekamārtaṇḍa}. \lb

The expression `lacking a fourth part' (\emph{caturthāṃśavivarjitaḥ}) is somewhat vague but probably refers to the idea of leaving a quarter of one's stomach empty, which is stated more clearly in the \emph{Dharmaputrikā} 1.51–52:

\begin{versinnote}
\tl{ṣaḍrasopetasusnigdhasvādusāndrasugandhinā/\\+}
\tl{udarasyārdhabhāgan tu bhojanena prapūrayet//\\+} 
\tl{pānīyena caturbhāgaṃ taccheṣaṃ śūnyam iṣyate/\\+}
\tl{vāyoḥ saṃcāradānārtham āhāraniyamaḥ smṛtaḥ//\\!} 
\end{versinnote}
% saṃcāradānārtham
% vāyoḥ saṃcāraṇārthaṃ tu  Madhusudana on BhG 6.17 (in a similar verse)
\closer
\begin{quote}
One should fill half of the stomach with food that has the six flavours and is very unctuous, delicious, rich, and fragrant. One quarter should be filled with water, and the remaining quarter should be left empty. This rule of diet is prescribed for producing the [proper] movement of vital air.
\end{quote}

And, as noted by Brahmānanda in \emph{Jyotsnā} 1.58, this idea also occurs in the \emph{Aṣṭāṅgahṛdayasaṃhitā} (\emph{Sūtrasthāna} 8.46cd–47ab): 

\begin{versinnote}
\tl{annena kukṣer dvāv aṃśau pānenaikaṃ prapūrayet/\\+}
\tl{āśrayaṃ pavanādīnāṃ caturtham avaśeṣayet//\\!}% 
\end{versinnote} 
\closer
\begin{quote}
One should fill two parts of the belly with food and one with water. One should leave the fourth part empty as a place for the [bodily] winds and so forth. 
\end{quote}

\end{philcomm}
%</cm58>

%%%%%%%%%%
\subsection*{1.59}
%<*tr59>
\begin{translation}[hp01_059]
 Pungent, sour, bitter, salty and hot foods, horseradish, sour gruel, [sesame] oil, sesame and mustard seeds, fish and intoxicating drink, flesh of goats and so forth, curds, diluted buttermilk, poor man's pulse, jujube fruit, the leftover paste of oily seeds, asafoetida, garlic and the like: they say that [such food] is unwholesome.
\end{translation}
%</tr59>

%<*sc59>
\begin{sources}[hp01_059]
Cf.~DYŚ 70ab
\begin{versinnote}
\tl{lavaṇaṃ sarṣapaś cāmlam uṣṇaṃ rūkṣaṃ ca tīkṣṇakam/\\!}
\end{versinnote}
\end{sources}
%</sc59>

%<*ts59>
\begin{testimonia}[hp01_059]
  \emph{Haṭharatnāvalī} 1.72, \emph{Yogacintāmaṇi} f.\,54v (\attr HP), \emph{Yuktabhavadeva} 4.18 (\attr HP), \emph{Haṭhatattvakaumudī}

  \begin{variants}
    tīkṣṇa HRĀ YCM HTK~] tikta YBhD\sep
    haritaśāka YCM HTK YBhD~] harītaśākaṃ HRĀ\sep
    matsyamadyam HRĀ YCM HTK~] madyamatsyāḥ YBhD\sep
    ajādimāṃsa HRĀ YCM~] ajāvimāṃsa YBhD HTK\sep
    kola YCM YBhD HTK~] kodra HRĀ    
    \end{variants}

  
%\emph{Haṭharatnāvalī} 1.72
%
%\begin{versinnote}
%\tl{kaṭvamlatīkṣṇalavaṇoṣṇaharītaśākaṃ\\+}
%\tl{sauvīratailatilasarṣapamatsyamadyam/\\+}
%\tl{ajādimāṃsadadhitakrakulatthakodra-\\+}
%\tl{piṇyākahiṅgulaśunādyam apathyam āhuḥ//\\!}
%\end{versinnote}
%
%\emph{Yogacintāmaṇi} f.\,54v
%
%\begin{versinnote}
%\tl{haṭhapradīpikāyām--\\+}
%\tl{kaṭvamlatīkṣṇalavaṇoṣṇaharītaśāka-\\+}
%\tl{sauvīratailatilasarṣapamatsyamadyam/\\+}
%\tl{ajādimāṃsadadhitakrakulatthakola-\\+}
%\tl{piṇyākahiṅgulaśunādyam apathyam āhuḥ//\\!}
%\end{versinnote}
%
%\emph{Yuktabhavadeva} 4.18 (\attr HP)
%\begin{versinnote}
%\tl{kaṭvamlatiktalavaṇoṣṇaharitaśāka-\\+}
%\tl{sauvīratailatilasarṣapamadyamatsyāḥ/\\+}
%\tl{ajāvimāṃsadadhitakrakulatthakola-\\+}
%\tl{piṇyākahiṅgulaśunādyam apathyam āhuḥ//\\!}
%\end{versinnote}
%
%\emph{Haṭhatattvakaumudī}
%
%\begin{versinnote}
%\tl{atha varjyāni –\\+}
%\tl{kaṭvamlatīkṣṇalavaṇoṣṇaharitaśāka-\\+}
%\tl{sauvīratailatilasarṣapamatsyamadyam//\\+}
%\tl{ajāvimāṃsadadhitakrakulatthakola-\\+}
%\tl{piṇyākahiṃgulaśunādyam apathyam āhuḥ// 28//\\!}
%\end{versinnote}
%
\end{testimonia}
%</ts59>

%<*cm59>
\begin{philcomm}[hp01_059]
%The compound \emph{kaṭvamla°} (1.59a) is better than \emph{kaṭvāmla°} and it is well attested by manuscripts of the \emph{Haṭhapradīpikā}, as well as in lists of tastes and types of foods in other texts.

On the meaning of \emph{uṣṇa} (1.59a) in relation to food,  Meulenbeld writes (1974: 254 fn. 13):
\begin{quote}
Cakra mentions as a variant: \emph{katvamlalavaṇakṣāra} (pungent, acid, saline and caustic). Cakra remarks that the term `hot' (\emph{uṣṇa}) denotes hot on touch when it occurs the first time, and hot with regard to potency when it occurs for the second time.
\end{quote}

The compound \emph{harītaśāka} in 1.59a is spelt \emph{haritaśāka} in other works. The spelling \emph{harīta} was probably adopted for metrical reasons. In some Nighaṇṭus, \emph{haritaśāka} is glossed as horseradish (\emph{śigru}).
\begin{quote}
\emph{Rājanighaṇṭu} 7.26

\begin{versinnote}
\tl{śigrur haritaśākaś ca śākapattraḥ supattrakaḥ/\\!}
\end{versinnote}
\emph{Sauśrutanighaṇṭu} 75ab
\begin{versinnote}
\tl{śigruko haritaśākaś ca mato vai mūlapatrakaḥ/\\!}
\end{versinnote}
\end{quote}

Brahmānanda’s understanding of \emph{harītaśāka} as \emph{pattraśāka} is probably wrong if \emph{pattraśāka} was intended as ‘leafy vegetables.’ But he may have used the term \emph{pattraśāka} in the sense of horseradish (\emph{śigru}) as the dictionary notes that \emph{pattraśāka} is probably equivalent to \emph{śākapattra}, which is mentioned in \emph{Rājanighaṇṭu} 7.26 (above).\lb

The term \emph{sauvīra} (1.59b) probably means sour gruel. Brahmānanda glosses  \emph{sauvīra} as \emph{kāñjika}, which is `fermented rice water.' On \emph{sauvīra}, Meulenbeld (1974: 516–517) says '\emph{sauvīra} is sour gruel made from barley and wheat.' The process of making it is described in the \emph{Suśrutasaṃhitā} (1.44.35--40ab) as follows:

\begin{quote}
Roots of trivṛt etc., the first group (vidārigandhādi), mahat pañcamūla, mūrvā and śārṅgaṣṭā, and also of snuhī, haimavatī, triphalā, ativiṣā and vacā -- these are taken and divided into two parts out of which one is decocted and the other is powdered; now, crushed barley grains are impregnated with the above decoction several times, dried and then slightly fried. Taking three parts of this and one part of the above powder are put in a jar and mixed with their (of trivṛt etc.) cold decoction and fermented properly. This is known as sauvīraka. (trans. Sharma 2018 (vol.1): 406)
\end{quote}

However, according to some Nighaṇṭus, \emph{sauvīra} can also mean stibnite (an ingredient in some \emph{añjana}s and medicines). For example, in the \emph{Rājanighaṇṭu} (13.86):


\begin{versinnote}
\tl{añjanaṃ yāmunaṃ kṛṣṇaṃ nādeyaṃ mecakaṃ tathā\\+}
\tl{srotojaṃ dṛkpradaṃ nīlaṃ sauvīraṃ ca suvīrajam//\\!}
\end{versinnote}

Note also that the \emph{Yogaprakāśikā} (1.53) takes \emph{sauvīra} with \emph{taila}, perhaps to solve the problem of \emph{taila} on its own (see below for more on this). The compound \emph{sauvīrataila} is explained as `oil produced in the place Suvīra' (\emph{suvīradeśodbhavatailam}). According to Ali (1966: 144), Suvīra is known as a country that was also called Suvira (V.79), Sauvira (XVI.21) and Sauvīraka (IV.23). He identifies it with the Rohri/Khairpur region of Sind. \lb
%( S.M. Ali, Geography of the Purānas, Delhi, 1966, p. 144).

The word \emph{taila} could refer to \emph{tilataila}. This is supported by the following rule (\emph{paribhāṣā}) in the \emph{Śārṅgadharasaṃhitā} (48): %
\emph{anuktāvasthāyāṃ paribhāṣāvidhiḥ} [...] \emph{taile ’nukte tilodbhavam}. We thank Dominik Wujastyk for this reference.  \lb

Our translation of \emph{madya} takes into account the following remarks of James McHugh (2021: 8):
% Mchugh, An Unholy Brew)
\begin{quote}
    The most general Sanskrit term to denote drinks that create a drunken state is \emph{madya} “intoxicating [drink].” Translating this word is hard. “Inebriating drink” is clumsy to my ear. “Intoxicating” contains the unfortunate “toxic” element that is not present in the Sanskrit word, though at least in English this is a common word, applicable to various substances and states and lacking any “toxic” associations in everyday usage.
\end{quote}

In the compound \emph{ājādimāṃsa} (1.59c), the adjective \emph{āja} is required for the metre,  so variants beginning with \emph{aja} can be dismissed. Another well-attested reading is \emph{ājāvimāṃsa}. Although this was probably read as `goat and sheep flesh,' \emph{āvi} is not attested as an adjective of sheep, so this reading was probably not original. Moreover, only \emph{ājādimāṃsa} makes good sense. Diwakar Acharya believes that the prohibition of goat flesh and fish in this verse suggests it derives from the North East of India.\lb

The term \emph{kulattha} means a kind of pulse, translated by Dominik Wujastyk (1998: 77) as `poor man's pulse.'\lb
% Dominik Wujastyk, Roots of Ayurveda, Penguin Books, 1998.

The word \emph{kola} is a name for \emph{Zizyphus Jujuba} \textsc{Mill.}~(Nadkarni 1926: 919--920). It is also known as \emph{badara}. This is how Brahmānanda understands it in \emph{Jyotsnā} 1.59 (\emph{kolaṃ kolyāḥ phalaṃ badaram}). According to Nadkarni, the fruit of the wild variety is very acid and astringent. It is believed to purify the blood and assist digestion. The bark is astringent and a simple remedy for diarrhoea. The root is useful as a decoction in fever and delirium. There are references to \emph{kola} being pungent, though this does not seem to indicate sufficiently why \emph{kola} is mentioned separately in the \emph{Haṭhapradīpikā} as an unwholesome food. Diwakar Acharya has informed us that \emph{kola} can refer to a type of banana in some parts of India.\lb

According to Sharma (1982: 69), \emph{piṇyāka} is 'the remnant paste of oily seeds after pressing out the oil content.’ This can then be made into a cake with a strong flavour, which may account for its inclusion in this list of unwholesome foods (we thank Diwakar Acharya for this information).\lb
% P.V. Sharma, Ḍalhaṇa and his commentary on drugs, 1982.

The term \emph{hiṅgu} means Asafoetida (Nadkarni 1926: 360–361). As to why it might be considered unwholesome, the following comments by Nadkarni give some indication:
\begin{quote}
If long continued, even in moderate doses, it gives rise to alliaceous eructations, acrid irritation in the throat, flatulence, diarrhoea and burning in the urine.
\end{quote}
%\emph{laśuna} = garlic (Nadkarni 1926: 45).
\end{philcomm}
%</cm59>

\begin{metre}[hp01_059]
Vasantatilakā 
\end{metre}

%%%%%%%%%%
\subsection*{1.60}
%<*tr60>
\begin{translation}[hp01_060]
One should know food to be unfit if it has been reheated, is dry, too salty or sour, contains an excess of leafy vegetables that are hard to chew, [or] is spoiled.
\end{translation}
%</tr60>

%<*sc60>
%\begin{sources}[hp01_060]
%\end{sources}
%</sc60>

%<*ts60>
\begin{testimonia}[hp01_060]
  \emph{Yogacintāmaṇi} f.\,55v (\attr HP), \emph{Haṭhasaṅketacandrikā} f.\,10v (\attr HP)

  \begin{variants}
    rūkṣaṃ HSC~] tathā YCM\sep
    atilavaṇam amlapṛktaṃ~] atilavaṇaṃ sapalaṃ vā YCM, atilavaṇādikayuktaṃ HSC\sep
    kadaśanaśākotkaṭaṃ duṣṭaṃ HSC~] prasitaṃ śākotkaṭaṃ varjyam YCM
    \end{variants}
%
%\emph{Yogacintāmaṇi} f.\,55v (\attr HP)
%
%\begin{versinnote}
%\tl{bhojanam ahitaṃ vidyāt punar uṣṇīkṛtaṃ tathā/\\+}
%\tl{atilavaṇaṃ sapalaṃ vā prasitaṃ śākotkaṭaṃ varjyam//\\!}
%\end{versinnote}
%
%\emph{Haṭhasaṅketacandrikā} f.\,10v (\attr HP)%
%
%\begin{versinnote}
%\tl{bhojanam ahitaṃ vidyāt punar uṣṇīkṛtaṃ rūkṣaṃ/\\+}
%\tl{atilavaṇādikayuktaṃ kadaśanaśākotkaṭaṃ duṣṭaṃ//\\!}
%\end{versinnote}

\end{testimonia}
%</ts60>

%<*cm60>
\begin{philcomm}[hp01_060]
Brahmānanda glosses his reading\emph{tilapiṇḍa} with \emph{piṇyāka} (on the meaning of which see the notes for the previous verse).\lb

The meaning of the compound \emph{kadaśanaśākotkaṭaṃ} is not clear. Brahmānanda understands it as a list (\emph{dvandva}) consisting of \emph{kadaśana}, \emph{śāka} and \emph{utkaṭa}, which he defines as bad food, prohibited vegetables and pepper, respectively.\lb

There are various possible meanings of \emph{utkaṭa}. According to some Nighaṇṭus, the word \emph{utkaṭā} can mean pepper  (e.g.~\emph{Rājanighaṇṭu} 5.16 \emph{pārvatī śailajā tāmrā lambabījā tathotkaṭā}) and, according to Monier Williams, \emph{utkaṭa} can refer to Saccharum Sara and \emph{utkaṭā} also to Laurus Cassia (cinnamon). \lb

However, \emph{utkaṭa} can be an adjective that means ‘abounding in’ at the end of a compound. Since this verse consists of many adjectives describing food that is unwholesome, it is likely that \emph{kadaśanaśākotkaṭaṃ} was intended as an adjectival \emph{tatpuruṣa}, in which case it means ‘[food] full of vegetables' \emph{śākotkaṭa} that are `bad food' or, perhaps, `bad eating' (\emph{kadaśana}) in the sense of hard to chew. 
\end{philcomm}
%</cm60>

\begin{metre}[hp01_060]
Upagīti 
\end{metre}

%%%%%%%%%%
\subsection*{1.61 heading}
%<*tr61a>
\begin{translation}[hp01_061a]
In the same vein there is a saying by Goraksa:
\end{translation}
%</tr61a>

%<*cm61a>
% \begin{philcomm}[hp01_061a]
% \end{philcomm}
%</cm61a>

%%%%%%%%%%
\subsection*{1.61}
%<*tr61>
\begin{translation}[hp01_061]
One should avoid places near bad people, frequenting fire, women and roads, and observances which harm the body, such as early morning bathing and fasting.
\end{translation}
%</tr61>

%<*sc61>
%\begin{sources}[hp01_061]
%Cf.~Amṛtasiddhi 19.7 % this should be quoted for the verse 3.32

%\begin{versinnote}
%\tl{agnisevābalāsevā pathasevā ca sarvadā/\\+}
%\tl{prathamābhyāsakāle tu saṃtyājyā yoginā sadā// 19.7//\\!}
%\end{versinnote}

%\end{sources}
%</sc61>

%<*ts61>
\begin{testimonia}[hp01_061]
  \emph{Haṭharatnāvalī} 1.73 (\attr Gorakṣa), \emph{Yogacintāmaṇi} f.\,48v, \emph{Yuktabhavadeva} 4.18 (\attr HP)

  \begin{variants}
    prāntaṃ HRĀ\vl~] prāptaṃ HRĀ\vl, prīti HRĀ, prītiṃ YCM YBhD\sep
    tathā HRĀ YCM~] tyajet YBhD
    \end{variants}

%\emph{Haṭharatnāvalī} 1.73 
%
%\begin{versinnote}
%\tl{tathā ca gorakṣavacanam---\\+}
%\tl{varjayed durjanaprītivahnistrīpathasevanam/\\+}
%\tl{prātaḥsnānopavāsādikāyakleśādikaṃ tathā//\\!}
%\end{versinnote}
%\begin{appinnote}
%\tl{\textbf{a} °prīti°] °prāntaṃ P, prāptaṃ T,t1. \ \textbf{d} kāyakleśādikaṃ~] kāyakleśavidhiṃ P,T.\\!}
%\end{appinnote}
%
%\emph{Yogacintāmaṇi} f.\,48v%
%
%\begin{versinnote}
%\tl{haṭhadīpikāyām—\\+}
%\tl{varjayed durjanaprītiṃ vahnistrīpathasevanam/\\+}
%\tl{prātaḥsnānopavāsādikāyakleśādikaṃ tathā//\\!}
%\end{versinnote}
%
%\emph{Yuktabhavadeva} 4.18 (\attr HP)
%\begin{versinnote}
%\tl{varjayed durjanaprītiṃ vahnistrīpathasevanam/\\+}
%\tl{prātaḥsnānopavāsādikāyakleśavidhiṃ tyajet//\\!}
%\end{versinnote}
%
\end{testimonia}
%</ts61>

%<*cm61>
\begin{philcomm}[hp01_061]
Manuscripts \alphaOne\alphaThree\zetaOne\piOne\ have the reading \emph{durjanaprāntaṃ} (1.61a), which is the lectio difficilior in relation to \emph{durjanaprītiṃ} (`the friendship of wicked people'). We have understood \emph{durjanaprānta} in line with Brahmānanda's gloss in \emph{Jyotsnā} 1.64, `dwelling near bad people' (\emph{durjanasamīpavāsam}).  
\end{philcomm}
%</cm61>

%%%%%%%%%%
\subsection*{1.62}
%<*tr62>
\begin{translation}[hp01_062]
The pure grains that are wheat, rice, \emph{śāli} rice, barley, sixty-day \emph{śāli} rice; milk, ghee, cream, fresh butter, ground sugar and honey; dried ginger, fruit of the pointed gourd and so forth; the five vegetables; mung beans and so on; and rain water. [These] are wholesome for great ascetics.%
\end{translation}
%</tr62>

%<*sc62>
%\begin{sources}[hp01_062]
%\end{sources}
%</sc62>

%<*ts62>
\begin{testimonia}[hp01_062]
\emph{Haṭharatnāvalī} 1.71, \emph{Yogacintāmaṇi} f.\,54v (\attr HP), \emph{Yuktabhavadeva} 4.21

\begin{variants}
  ṣaṣṭika HRĀ~] ṣāṣṭika YCM YBhD\sep
  maṇḍana HRĀ YCM~] khaṇḍana YBhD\sep
  phalādi ca YBhD~] phalapatraja HRĀ, phalādika YCM HRĀ\vl\sep
  pañcaśākaṃ HRĀ YCM~] pañcaśāka YBhD\sep
  mudgādi divyam HRĀ YBhD~] mudgādi cālpam YCM\sep
  yamīndra HRĀ~] yatīndra HRĀ\vl, munīndra YCM YBhD 
\end{variants}
%
% 
%\emph{Haṭharatnāvalī} 1.71
%
%\begin{versinnote}
%\tl{godhūmaśāliyavaṣaṣṭikaśobhanānnaṃ\\+}
%\tl{kṣīrājyamaṇḍanavanītasitāmadhūni/\\+}
%\tl{śuṇṭhīpaṭolaphalapatrajapañcaśākaṃ\\+}
%\tl{mudgādidivyam udakaṃ ca yamīndrapathyam//\\!}
%\end{versinnote}
%\begin{appinnote}
%\tl{\textbf{c} °phalapatraja°~] phalādika N,n1,J. \ \textbf{d} yamīndra°~] yatīndra° N,n1,J \\!}
%\end{appinnote}
%
%\emph{Yogacintāmaṇi} f.\,54v (\attr HP)
%
%\begin{versinnote}
%\tl{godhūmaśāliyavaṣāṣṭikaśobhanānnaṃ\\+}
%\tl{kṣīrājyamaṇḍanavanītasitāmadhūni/\\+}
%\tl{śuṇṭhīpaṭolakaphalādikapañcaśākaṃ\\+}
%\tl{mudgādi cālpam udakaṃ ca munīndrapathyam//\\!}
%\end{versinnote}
%
%\emph{Yuktabhavadeva} 4.21
%
%\begin{versinnote}
%\tl{tathā ca śivayoge-\\+}
%\tl{godhūmaśāliyavaṣāṣṭikaśobhanānnaṃ\\+}
%\tl{kṣīrājyakhaṇḍanavanītasitāmadhūni//\\+}
%\tl{śuṇṭhīpaṭolakaphalādi ca pañcaśāka-\\+}
%\tl{mudgādidivyam udakaṃ ca munīndrapathyam//\\!}
%\end{versinnote}
%
\end{testimonia}
%</ts62>

%<*cm62>
\begin{philcomm}[hp01_062]
In 1.62b, \emph{maṇḍa}, which is supported by \textalpha, \textgamma\ and \textpi, is more likely than \emph{khaṇḍa} (`candied sugar') because the other elements of this compound are dairy products. The term \emph{navanīta} is discussed in \emph{Suśrutasaṃhitā, sūtrasthāna,} 45.92. %as follows:%JM:do we need this quote? navanīta is very common
% \begin{quote}
% Fresh butter (\emph{navanīta}) is light soft, sweet, astringent, slightly sour, cold, intellect-promoting, appetiser, cordial, checking, aphrodisiac, non-burning, pacifies pitta and vāta and alleviates wasting, cough, wound, consumption, piles and facial paralysis [...] (trans. Sharma 2018 vol. 1: 434).    
% \end{quote}

The word \emph{sitā} is one of many words for ground sugar. Meulenbeld (1974: 507) comments that \emph{sitā} is `very white and looks like gravel.'\lb

The term \emph{paṭola} can refer to at least two different gourds. Meulenbeld (1974: 569) compiled a list of six possibilities, including \emph{Trichosanthes dioica} \textsc{Roxb}. (`pointed gourd') and \emph{Trichosanthes cucumerina} \textsc{Linn.} (`snake gourd').\lb

% Nadkarni (1954: 863, 518) has two entries on \emph{paṭola}:
%Nadkarni, A. K. (1954). Dr. K. M. Nadkarni’s Indian Materia Medica, with Ayurvedic, Unani-tibbi, Siddha, Allopathic, Homeopathic, Naturopathic & Home Remedies, Appendices & Notes. 2 vols. Bombay: Popular Prakashan. url: https: //archive.org/details/IndianMateriaMedicaK.M.Nadkarni (on 11 Aug. 2017). URL is 1926 edition.
% \begin{enumerate}
% \item Snake gourd is common in Bengal and cultivated in Northern India and Punjab. The unripe fruit of this climbing plant is generally used as a culinary vegetable and is very wholesome, specially suited for the convalescent.

% \item Smooth luffa is a hairy climbing herb extensively cultivated in several parts of India. The fruit is edible. Medicinally it is described as `cool, costive, demulcent, producive of loss of appetite and excitive of wind, bile and phlegm.
% \end{enumerate}
% Sharma (1982: 156) adds that \emph{paṭola} is a synonym of \emph{kulaka} and is well known as \emph{Trichosanthas dioica} \textsc{Roxb.} Brahmānanda glosses \emph{paṭola} as \emph{kośātakī}, which is \emph{Luffa acutangula} \textsc{Roxb} (Meulenbeld 1974: 586), suggesting that he thought it was some sort of luffa. He also mentions the vernacular term \emph{paravara} for \emph{paṭola}.\lb
%JM: under the name parval it is very common in N.India and a popular vegetable in ascetic communities. This seems like much the most likely referent https://en.wikipedia.org/wiki/Trichosanthes_dioica
%JB: okay, shall we change the translation to 'pointed gourd'? and the note to:
Singh and Chunekar (1999: 232) identify \emph{paṭola} as \emph{Trichosanthas dioica} \textsc{Roxb.}, more commonly known as the pointed gourd, and they give its Hindi names as \emph{paravala} and \emph{parorāparorā}. It is very common in North India and is a popular vegetable in ascetic communities. Brahmānanda, however, glosses \emph{paṭola} as \emph{kośātakī}, which Meulenbeld (1974: 586) identifies as \emph{Luffa acutangula} \textsc{Roxb}, suggesting that Brahmānanda thought \emph{paṭola} was some sort of luffa.\lb 
%

Groups of five vegetables (\emph{pañcaśāka} or \emph{śākapañcaka}) have been defined in various yoga texts, but such grouping of vegetables does not seem to occur outside of literature on yoga. The earliest reference to a group of five vegetables known to us is in the sixteenth-century \emph{Yuktabhavadeva} 4.22, which attributes the verse to the \emph{Śivayoga}. The same verse is also quoted in \emph{Jyotsnā} 1.65 with attribution to a medical source (\emph{vaidyake}):

\begin{versinnote}
\tl{sarvaśākam acākṣuṣyaṃ cākṣuṣyaṃ śākapañcakam/\\+}
\tl{jīvantī vāstumatsyākṣī meghanādaḥ punarnavāḥ// iti//\\!}
\end{versinnote}

Another verse on a similar fivefold group of vegetables is also cited in the \emph{Haṭhatattvakaumudī} (4.26)

\begin{versinnote}
\tl{pañcaśākas tu–\\+}
\tl{kṣīraparṇī ca jīvantī matsyākṣī ca punarnavā\\+}
\tl{meghanādaś ceti budhaiḥ pañcaśākaḥ prakīrtitaḥ// iti//\\!}
\end{versinnote}

And a group with more significant differences is mentioned in the \emph{Gheraṇḍasaṃhitā} (5.20; cf.~the additional verses in some witnesses as reported in the apparatus).

\begin{versinnote}
\tl{bālaśākaṃ kālaśākaṃ tathā paṭolapatrakam/\\+}
\tl{pañcaśākaṃ praśaṃsīyād vāstūkaṃ hilamocikāṃ//\\!}
\end{versinnote}

It is not entirely clear how one should understand \emph{divyam} (1.62d). Brahmānanda glosses it with \emph{nirdoṣam} (`defectless') and takes it with \emph{udakam}. Ayurvedic sources indicate more clearly that \emph{divyodaka} was understood as rainwater. In a section on types of water (\emph{jalavarga}) in the \emph{Sūtrasthāna} of the \emph{Carakasaṃhitā} (1.27.196–224), rainwater is referred to as `\emph{divyaṃ udakam}' (1.27.198) in a discussion of the properties of water that has fallen from the sky. The compound \emph{divyodaka} is used in other Āyurvedic works to refer to the use of rainwater in recipes and treatments  (e.g.~\emph{Aṣṭāṅgahṛdaya} 8.42–43). Also, the \emph{Rājanighaṇṭu} (14.4) glosses \emph{divyodaka} as rainwater:

\begin{versinnote}
\tl{divyodakaṃ kharāri syād ākāśasalilaṃ tathā/\\+}
\tl{vyomodakaṃ cāntarikṣajalaṃ ceṣvabhidhāhvayam//\\!}
\end{versinnote}


%62*1 is quoted in the Jyotsna as from a medical work (“vaidyake”).
\end{philcomm}
%</cm62>

\begin{metre}[hp01_062]
Vasantatilakā 
\end{metre}

%%%%%%%%%%
\subsection*{1.63}
%<*tr63>
\begin{translation}[hp01_063]
The yogi should eat food that is sweet, delicious, unctuous, contains cow products, nourishes the bodily constituents, is desired by the mind and is appropriate.
\end{translation}
%</tr63>

%<*sc63>
%\begin{sources}[hp01_063]
%\end{sources}
%</sc63>

%<*ts63>
\begin{testimonia}[hp01_063]
\emph{Haṭharatnāvalī} 1.75, \emph{Yogacintāmaṇi} f.\,54v (\attr HP), \emph{Yuktabhavadeva} 4.23 (\attr \emph{Śivayoga})

\begin{variants}
  mṛṣṭaṃ~] śreṣṭhaṃ HRĀ YBhD, piṣṭaṃ YCM\sep
  sumadhuraṃ YCM YBhD~] samadhuraṃ HRĀ\sep
  yogī bhojanam ācaret YCM YBhD~] caturthāṃśavivarjitam | śivārpitaṃ ca naivedyaṃ yogī bhojanam ācaret HRĀ 
  \end{variants}


%\emph{Haṭharatnāvalī} 1.75%
%
%\begin{versinnote}
%\tl{śreṣṭhaṃ samadhuraṃ snigdhaṃ gavyaṃ dhātuprapoṣaṇam/\\+}
%\tl{manobhilaṣitaṃ yogyaṃ caturthāṃśavivarjitam/\\+}
%\tl{śivārpitaṃ ca naivedyaṃ yogī bhojanam ācaret//\\!}
%\end{versinnote}
%
%\emph{Yogacintāmaṇi} f.\,54v (\attr HP)
%
%\begin{versinnote}
%\tl{piṣṭaṃ sumadhuraṃ snigdhaṃ gavyaṃ dhātuprapoṣaṇam/\\+}
%\tl{mano'bhilaṣitaṃ yogyaṃ yogī bhojanam ācaret// iti//\\!}
%\end{versinnote}
%
%\emph{Yuktabhavadeva} 4.23 (attr. \emph{Śivayoga})
%
%\begin{versinnote}
%\tl{śreṣṭhaṃ sumadhuraṃ snigdhaṃ gavyaṃ dhātuprapoṣaṇam/\\+}
%\tl{mano'bhilaṣitaṃ yogyaṃ yogī bhojanam ācaret//\\!}
%\end{versinnote}
%
\end{testimonia}
%</ts63>

%<*cm63>
\begin{philcomm}[hp01_063]
The variants of 1.63a are all plausible: \emph{mṛṣṭaṃ}, \emph{miṣṭaṃ} and \emph{iṣṭaṃ}, although the last is made tautologous by \emph{mano'bhilaṣitaṃ} in 16.3c. Both \emph{mṛṣṭaṃ} and \emph{miṣṭaṃ} are well attested by manuscripts of important groups and there is hardly any difference in their meaning in this context. We have adopted \emph{mṛṣṭaṃ} as it is supported by \alphaOne\ and \alphaThree.
\end{philcomm}
%</cm63>

%%%%%%%%%%
\subsection*{1.64}
%<*tr64>
\begin{translation}[hp01_064]
Whether young, old, very old, sick or even weak, the diligent [yogi] succeeds in all yogas through practice.
\end{translation}
%</tr64>

%<*sc64>
\begin{sources}[hp01_064]
  \emph{Dattātreyayogaśāstra} 40

  \begin{variants}
    yuvā vṛddho~] yuvāvastho DYŚ\sep
    'ti~] 'pi DYŚ\sep
    durlabho 'pi vā~] vā śanaiḥ śanaiḥ DYŚ    
      \end {variants}
%
%\begin{versinnote}
%\tl{yuvāvastho 'pi vṛddho vā vyādhito vā śanaiḥ śanaiḥ/\\+}
%\tl{abhyāsāt siddhim āpnoti sarvayogeṣv atandritaḥ//\\!}
%\end{versinnote}
%
\end{sources}
%</sc64>

%<*ts64>
\begin{testimonia}[hp01_064]
\emph{Haṭharatnāvalī} 1.23, \emph{Yogacintāmaṇi} 15r (\attr HP)

\begin{variants}
  vṛddho 'tivṛddho vā~] bhavati vṛddho 'pi HRĀ, bālo 'tivṛddho vā YCM
  \end{variants}
  
%\emph{Haṭharatnāvalī} 1.23
%
%\begin{versinnote}
%\tl{yuvā bhavati vṛddho 'pi vyādhito durbalo 'pi vā/\\+}
%\tl{abhyāsāt siddhim āpnoti sarvayogeṣv atandritaḥ//\\!}
%\end{versinnote}
%
%\emph{Yogacintāmaṇi} 15r
%
%\begin{versinnote}
%\tl{haṭhapradīpikāyām—\\+}
%\tl{yuvā bālo 'tivṛddho vā vyādhito durbalo 'pi vā/\\+}
%\tl{abhyāsāt siddhim āpnoti sarvayogeṣv atandritaḥ//\\!}
%\end{versinnote}
%
\end{testimonia}
%</ts64>

%<*cm64>
%\begin{philcomm}[hp01_064]
%\etaOne, the oldest dated manuscript, has a different reading for the last verse quarter (164d): \emph{sarvaṃ yogī yatendriyaḥ} (`the yogi whose senses are restrained wholly succeeds [...]'). Here, \emph{sarvaṃ} is not easy to construe, and the readings of the \textalpha\ manuscripts and other important groups of \emph{Haṭhapradīpikā} manuscripts indicate that \emph{sarvayogeṣv atandritaḥ} was the reading adopted by Svātmārāma, which is more similar to the \emph{Dattātreyayogaśāstra}'s (i.e.~\emph{yoge sarvo 'py atandritaḥ}). 
% MD: This commentary may be not necessary.
%\end{philcomm}
%</cm64>

%%%%%%%%%%
\subsection*{1.65}
%<*tr65>
\begin{translation}[hp01_065]
The postures, various breath retentions, and heavenly techniques: the whole practice of Haṭha [is to be done] until Rājayoga results. 
\end{translation}
%</tr65>

%<*sc65>
%\begin{sources}[hp01_065]
%\end{sources}
%</sc65>

%<*ts65>
\begin{testimonia}[hp01_065]
  \emph{Haṭharatnāvalī} 1.17

  \begin{variants}
    sarvo 'pi~] sāṅgo 'pi HRĀ\sep
    phalāvadhi~] phalārthadaḥ HRĀ
    \end{variants}
  
%\emph{Haṭharatnāvalī} 1.17%
%
%\begin{versinnote}
%\tl{pīṭhāni kumbhakāś citrā divyāni karaṇāni ca/\\+}
%\tl{sāṅgo 'pi ca haṭhābhyāso rājayogaphalārthadaḥ//\\!}
%\end{versinnote}
%
\end{testimonia}
%</ts65>

%<*cm65>
%\begin{philcomm}[hp01_065]
%\end{philcomm}
%</cm65>

%=======
\subsection*{colophon}
%<*trcol>
\begin{translation}[hp01_col]
Thus ends the first chapter in the \emph{Haṭhapradīpikā} composed by the glorious lord among yogis Svātmārāma.    
\end{translation}
%</trcol>

\end{ekdosis}
\end{document}

