\documentclass[10pt]{memoir}
\setstocksize{220mm}{155mm} 	        
\settrimmedsize{220mm}{155mm}{*}	
\settypeblocksize{170mm}{116mm}{*}	
\setlrmargins{18mm}{*}{*}
\setulmargins{*}{*}{1.2}
% \setlength{\headheight}{5pt}
\checkandfixthelayout[lines]
\linespread{1}
\setlength{\parskip}{0.3em}
\setlength\parindent{0pt}

\makepagestyle{HPed}
\makeoddhead{HPed}{\small{HP Transl. \& Comm.}}{}{\small{\today}}
\makeevenhead{HPed}{\small{HP Transl. \& Comm.}}{}{\small{\today}}
\makeoddfoot{HPed}{}{\small{\thepage}}{}
\makeevenfoot{HPed}{}{\small{\thepage}}{}

\usepackage[teiexport=tidy,poetry=verse]{ekdosis}
\usepackage{sanskrit-poetry,libertine,xcolor}
\usepackage[english]{babel}
\setlength{\vindent}{0pt}
\setvnum{}




%%%%%%%%%%%%%%%%%%%% THE  MSS         %%%%%%%%%%%%%%%%%%%%%%%%%%%

%%% Versions
\DeclareWitness{Vu}{\selectlanguage{english}Vulg}{Vulgate, i.e. Brahmānanda's version}[]           
\DeclareWitness{X}{\selectlanguage{english}X}{TenChapter Version, Jodhpur 02228 and 02225 (ed. Lonavla)}[]
\DeclareWitness{Six}{\selectlanguage{english}Ṣ}{SixChapterVersion, ``6ChapterHPms'', fragment of enlarged text, Jodhpur}[]
% Mss. in Geographical Groups
%%%% Varanasi mss (Sampūrṇānanda mss). V1 is Important
\DeclareWitness{V1}{\selectlanguage{english}V\textsubscript{1}}{Sampurnananda Library Sarasvati Bhavan 30109}[]
        \DeclareHand{V1ac}{V1}{\selectlanguage{english}V\rlap{\textsubscript{1}}\textsuperscript{ac}}[] % added by MD
        \DeclareHand{V1pc}{V1}{\selectlanguage{english}V\rlap{\textsubscript{1}}\textsuperscript{pc}}[] % added by MD
\DeclareWitness{V2}{\selectlanguage{english}V\textsubscript{2}}{Sampurnananda Library Sarasvati Bhavan 29869}[]
\DeclareWitness{V3}{\selectlanguage{english}V\textsubscript{3}}{Sampurnananda Library Sarasvati Bhavan 29899}[]
\DeclareWitness{V4}{\selectlanguage{english}V\textsubscript{4}}{Sampurnananda Library Sarasvati Bhavan 29937}[]
\DeclareWitness{V5}{\selectlanguage{english}V\textsubscript{5}}{Sampurnananda Library Sarasvati Bhavan 29938}[]
\DeclareWitness{V6}{\selectlanguage{english}V\textsubscript{6}}{Sampurnananda Library Sarasvati Bhavan 29991}[]
\DeclareWitness{V8}{\selectlanguage{english}V\textsubscript{8}}{Sampurnananda Library Sarasvati Bhavan 30014}[]
\DeclareWitness{V11}{\selectlanguage{english}V\textsubscript{11}}{Sampurnananda Library Sarasvati Bhavan 30029}[]
\DeclareWitness{V12}{\selectlanguage{english}V\textsubscript{12}}{Sampurnananda Library Sarasvati Bhavan 30030}[]
\DeclareWitness{V13}{\selectlanguage{english}V\textsubscript{13}}{Sampurnananda Library Sarasvati Bhavan 30031}[]
\DeclareWitness{V14}{\selectlanguage{english}V\textsubscript{14}}{Sampurnananda Library Sarasvati Bhavan 30050}[]
\DeclareWitness{V15}{\selectlanguage{english}V\textsubscript{15}}{Sampurnananda Library Sarasvati Bhavan 30051}[]
\DeclareWitness{V15pc}{\selectlanguage{english}V\rlap{\textsubscript{15}}\textsuperscript{pc}\space}{}[]
\DeclareWitness{V16}{\selectlanguage{english}V\textsubscript{16}}{Sampurnananda Library Sarasvati Bhavan 30052}[]
\DeclareWitness{V17}{\selectlanguage{english}V\textsubscript{17}}{Sampurnananda Library Sarasvati Bhavan 30053}[] % added by MD
\DeclareWitness{V16pc}{\selectlanguage{english}V\rlap{\textsubscript{16}}\textsuperscript{pc}\space}{}[]
\DeclareWitness{V18}{\selectlanguage{english}V\textsubscript{18}}{Sampurnananda Library Sarasvati Bhavan 30064}[]
\DeclareWitness{V19}{\selectlanguage{english}V\textsubscript{19}}{Sampurnananda Library Sarasvati Bhavan 30069}[]
\DeclareWitness{V21}{\selectlanguage{english}V\textsubscript{21}}{Sampurnananda Library Sarasvati Bhavan 30104}[]
\DeclareWitness{V22}{\selectlanguage{english}V\textsubscript{22}}{Sampurnananda Library Sarasvati Bhavan 30110}[]
\DeclareWitness{V25}{\selectlanguage{english}V\textsubscript{25}}{Sampurnananda Library Sarasvati Bhavan 30122}[]
\DeclareWitness{V26}{\selectlanguage{english}V\textsubscript{26}}{Sampurnananda Library Sarasvati Bhavan 30123}[]
\DeclareWitness{V28}{\selectlanguage{english}V\textsubscript{28}}{Sampurnananda Library Sarasvati Bhavan 30136}[]
\DeclareWitness{W2}{\selectlanguage{english}W\textsubscript{2}}{Wai ??}[]
\DeclareWitness{W4}{\selectlanguage{english}W\textsubscript{4}}{Wai 399-6171}[]

%%%%%%%%%%%%%%%%%%%%%%%%%%%%%%%%%
%%% Jammu & Kaschmir
\DeclareWitness{K1}{\selectlanguage{english}K\textsubscript{1}}{Raghunātha Temple Library 4383}[settlement=Jammu]
        \DeclareWitness{K1ac}{\selectlanguage{english}K\rlap{\textsubscript{1}}\textsuperscript{ac}\space}{}[]
        \DeclareWitness{K1pc}{\selectlanguage{english}K\rlap{\textsubscript{1}}\textsuperscript{pc}\space}{}[]
\DeclareWitness{K3}{\selectlanguage{english}K\textsubscript{3}}{Privat collection}
\DeclareWitness{L1}{\selectlanguage{english}L\textsubscript{1}}{SOAS RE 43454}[settlement=Jammu]
% More details? Catalogue number? L1 And C1 very close (and come from same region)
%%%%%%%%%%%%%%%%%%%%%%%%%%%%%%%%
% Jodhpur
% J10 is important
\DeclareWitness{J10}{\selectlanguage{english}J\textsubscript{10}}{MSPP Jodhpur 2230}[]
        \DeclareHand{J10ac}{J10}{\selectlanguage{english}J\rlap{\textsubscript{10}}\textsuperscript{ac}}[] % modified by MD
        \DeclareHand{J10pc}{J10}{\selectlanguage{english}J\rlap{\textsubscript{10}}\textsuperscript{pc}}[] % modified by MD
\DeclareWitness{J1}{\selectlanguage{english}J\textsubscript{1}}{Jodhpur 02231}[]
\DeclareWitness{J2}{\selectlanguage{english}J\textsubscript{2}}{Jodhpur 02232}[]   
\DeclareWitness{J3}{\selectlanguage{english}J\textsubscript{3}}{Jodhpur 02233}[]
\DeclareWitness{J4}{\selectlanguage{english}J\textsubscript{4}}{Jodhpur 02234}[]
        \DeclareWitness{J4ac}{\selectlanguage{english}J\rlap{\textsubscript{4}}\textsuperscript{ac}\space}{MSPP Jodhpur 02234}[]
        \DeclareWitness{J4pc}{\selectlanguage{english}J\rlap{\textsubscript{4}}\textsuperscript{pc}\space}{MSPP Jodhpur 02234}[]
\DeclareWitness{J5}{\selectlanguage{english}J\textsubscript{5}}{Jodhpur 02235}[]  % 4 chapters, 34 jpgs,   long colophon, missing lines in the beginning.
\DeclareWitness{J6}{\selectlanguage{english}J\textsubscript{6}}{Jodhpur 02237}[]  % 4 chapters, 41 jpgs
%\DeclareWitness{J6ac}{\selectlanguage{english}J\rlap{\textsubscript{6}}\textsubscript{ac}}{Jodhpur 02237}[]  % 4 chapters, 49 jpgs,   1st folio: idaṃ gulābarāyasya
% tulasīrāmaśarmmaṇaḥ putrasya pustakaṃ ...        End: iti śrīsahajānandasantānacintāmaṇisvātmārāmaviracitāyāṃ ..
% saṃvat 1802   (more consistent text)
%\DeclareWitness{J6pc}{\selectlanguage{english}J\rlap{\textsubscript{6}}\textsubscript{pc}}{Jodhpur 02237}[] 
\DeclareWitness{J7}{\selectlanguage{english}J\textsubscript{7}}{Jodhpur 02241}[]  % 4 chapters, 41 jpgs
\DeclareWitness{J8}{\selectlanguage{english}J\textsubscript{8}}{Jodhpur 23709}[]  % 4 chapters,  87 jpgs.   saṃvat 1724
\DeclareHand{J8ac}{J8}{\selectlanguage{english}J\rlap{\textsubscript{8}}\textsuperscript{ac}}[]  % changed by MD
\DeclareHand{J8pc}{J8}{\selectlanguage{english}J\rlap{\textsubscript{8}}\textsuperscript{pc}}[]  % changed by MD
\DeclareWitness{J9}{\selectlanguage{english}J\textsubscript{9}}{Jodhpur 02224}[]  %  fragment, 20 jpgs.
\DeclareWitness{J11}{\selectlanguage{english}J\textsubscript{11}}{Jodhpur 23532}[]
        \DeclareHand{J11ac}{J11}{\selectlanguage{english}J\rlap{\textsubscript{11}}\textsuperscript{ac}}[] % added by MD
        \DeclareHand{J11pc}{J11}{\selectlanguage{english}J\rlap{\textsubscript{11}}\textsuperscript{pc}}[] % added by MD
\DeclareWitness{J12}{\selectlanguage{english}J\textsubscript{12}}{Jodhpur 18552}[] 
\DeclareWitness{J13}{\selectlanguage{english}J\textsubscript{13}}{Jodhpur 02229}[]  %  5 chapters, 93 jpgs.
\DeclareWitness{J14}{\selectlanguage{english}J\textsubscript{14}}{Jodhpur 02239}[]  %  4 chapters
\DeclareWitness{J15}{\selectlanguage{english}J\textsubscript{15}}{Jodhpur 9732A}[]
\DeclareWitness{J16}{\selectlanguage{english}J\textsubscript{16}}{Jodhpur 9732B}[]
\DeclareWitness{J17}{\selectlanguage{english}J\textsubscript{17}}{Jodhpur 3013}[]
% Haṭhapradīpikā with (non-Sanskrit) Bhāṣya RORI Jodhpur ACC.NO.18552
%  Haṭhapradīpikā with (non-Sanskrit) commentary, RORI Alwar 952, 4 chapters,  colophon of the comm:
% iti śrīlāhorīmiśravrajabhūṣanaviracitāyāṃ bhāvārthadīpikāyāṃ caturthodhyāya ..    
%  Haṭhapradīpikā (5 chapter) MSPP Jodhpur ACC.NO.02229/

%%%%%%%%%%        Bodleian, Oxford
\DeclareWitness{B1}{\selectlanguage{english}B\textsubscript{1}}{Bodleian Library No. d.457(8)}[settlement=Oxford]
\DeclareWitness{B2}{\selectlanguage{english}B\textsubscript{2}}{Bodleian Library No. d.458(1)}[settlement=Oxford]
\DeclareWitness{B3}{\selectlanguage{english}B\textsubscript{3}}{Bodleian Library No. d.458(9)}[settlement=Oxford]

%%%%%%%%%%%   Chandigarh
\DeclareWitness{C1}{\selectlanguage{english}C\textsubscript{1}}{Lalchand M-2080}[]%L1 And C1 very close (and come from same region)
\DeclareWitness{C2}{\selectlanguage{english}C\textsubscript{2}}{Lalchand M-6065}[]
\DeclareWitness{C3}{\selectlanguage{english}C\textsubscript{3}}{Lalchand M-1293}[]
\DeclareWitness{C4}{\selectlanguage{english}C\textsubscript{4}}{Lalchand M-2081}[]
\DeclareWitness{C4ac}{\selectlanguage{english}C\rlap{\textsubscript{4}}\textsuperscript{ac}\space}{}[]
\DeclareWitness{C4pc}{\selectlanguage{english}C\rlap{\textsubscript{4}}\textsuperscript{pc}\space}{}[]
\DeclareWitness{C5}{\selectlanguage{english}C\textsubscript{5}}{Lalchand M-2082}[]%doesn't have chapter 1
\DeclareWitness{C6}{\selectlanguage{english}C\textsubscript{6}}{Lalchand M-2089}[]
\DeclareWitness{C7}{\selectlanguage{english}C\textsubscript{7}}{Lalchand M-6494}[]
\DeclareWitness{C8}{\selectlanguage{english}C\textsubscript{8}}{Lalchand M-2091}[]
        \DeclareHand{C8ac}{C8}{\selectlanguage{english}C\rlap{\textsubscript{8}}\textsuperscript{ac}}[]
        \DeclareHand{C8pc}{C8}{\selectlanguage{english}C\rlap{\textsubscript{8}}\textsuperscript{pc}}[]
\DeclareWitness{C9}{\selectlanguage{english}C\textsubscript{9}}{Lalchand M-4530}[]


% %%%%%%%%%%        Nepalese
\DeclareWitness{N1}{\selectlanguage{english}N\textsubscript{1}}{NGMPP A1400-2}[]
\DeclareWitness{N2}{\selectlanguage{english}N\textsubscript{2}}{NGMPP B 39-19}[]
\DeclareWitness{N3}{\selectlanguage{english}N\textsubscript{3}}{NGMPP B 62-20}[]
\DeclareWitness{N5}{\selectlanguage{english}N\textsubscript{5}}{NGMPP A60-15 + A61-1}[]
\DeclareWitness{N4}{\selectlanguage{english}N\textsubscript{4}}{NGMPP A61-2}[]
\DeclareWitness{N6}{\selectlanguage{english}N\textsubscript{6}}{NGMPP A61-6}[]
\DeclareWitness{N9}{\selectlanguage{english}N\textsubscript{9}}{NGMPP A62-33}[]
\DeclareWitness{N10}{\selectlanguage{english}N\textsubscript{10}}{NGMPP A62-37}[]
\DeclareWitness{N11}{\selectlanguage{english}N\textsubscript{11}}{NGMPP A63-15}[]
\DeclareWitness{N12}{\selectlanguage{english}N\textsubscript{12}}{NGMPP A939-19}[]
\DeclareWitness{N13}{\selectlanguage{english}N\textsubscript{13}}{NGMPP A1378-18}[]
\DeclareWitness{N16}{\selectlanguage{english}N\textsubscript{16}}{NGMPP B39-20}[]
\DeclareWitness{N17}{\selectlanguage{english}N\textsubscript{17}}{NGMPP B 111-10}[]
\DeclareWitness{N18}{\selectlanguage{english}N\textsubscript{18}}{NGMPP E 929-3}[]
\DeclareWitness{N19}{\selectlanguage{english}N\textsubscript{19}}{NGMPP E-1528-1 / E-1527-7(4)}[]
\DeclareWitness{N20}{\selectlanguage{english}N\textsubscript{20}}{NGMPP E 2037-13 }[]
\DeclareWitness{N21}{\selectlanguage{english}N\textsubscript{21}}{NGMPP E 2097-31}[]
\DeclareWitness{N22}{\selectlanguage{english}N\textsubscript{22}}{NGMPP G 4-4}[]
\DeclareWitness{N23}{\selectlanguage{english}N\textsubscript{23}}{NGMPP G 25-2}[]
        \DeclareHand{N23ac}{N23}{\selectlanguage{english}N\rlap{\textsubscript{23}}\textsuperscript{ac}}[] % added by MD
        \DeclareHand{N23pc}{N23}{\selectlanguage{english}N\rlap{\textsubscript{23}}\textsuperscript{pc}}[] % added by MD
\DeclareWitness{N24}{\selectlanguage{english}N\textsubscript{24}}{NGMPP G 190-16}[]
\DeclareWitness{N24ac}{\selectlanguage{english}N\rlap{\textsubscript{24}}\textsuperscript{ac}\space}{}[]
\DeclareWitness{N24pc}{\selectlanguage{english}N\rlap{\textsubscript{24}}\textsuperscript{pc}\space}{}[]
\DeclareWitness{N26}{\selectlanguage{english}N\textsubscript{26}}{NGMPP T 24-3}[]

% %%%%%%%%%%        Pune

\DeclareWitness{P1}{\selectlanguage{english}P\textsubscript{1}}{Ānandāśrama S16-3-21}[]
\DeclareWitness{P2}{\selectlanguage{english}P\textsubscript{2}}{Ānandāśrama S16-2-20}[]
\DeclareWitness{P3}{\selectlanguage{english}P\textsubscript{3}}{BISM (79) 314}[]
\DeclareWitness{P4}{\selectlanguage{english}P\textsubscript{4}}{BISM (91) 191}[]
\DeclareWitness{P5}{\selectlanguage{english}P\textsubscript{5}}{BISM (29) 5790}[]
\DeclareWitness{P6}{\selectlanguage{english}P\textsubscript{6}}{BORI 263/1879-80}[]
\DeclareWitness{P7}{\selectlanguage{english}P\textsubscript{7}}{BORI 665/1883-84}[]
\DeclareWitness{P8}{\selectlanguage{english}P\textsubscript{8}}{BORI 316/1895-98}[]
\DeclareWitness{P9}{\selectlanguage{english}P\textsubscript{9}}{BORI 733-1891-95}[]
\DeclareWitness{P10}{\selectlanguage{english}P\textsubscript{10}}{BORI 222-1884-86}[]
\DeclareWitness{P11}{\selectlanguage{english}P\textsubscript{11}}{BORI 221-1882–83}[]
\DeclareWitness{P12}{\selectlanguage{english}P\textsubscript{12}}{Ānandāśrama S16-3-24}[]
\DeclareWitness{P13}{\selectlanguage{english}P\textsubscript{13}}{Ānandāśrama S16-2-22}[]
\DeclareWitness{P14}{\selectlanguage{english}P\textsubscript{14}}{Ānandāśrama S16-3-23}[]
\DeclareWitness{P15}{\selectlanguage{english}P\textsubscript{15}}{BISM (64) 919}[]
\DeclareWitness{P16}{\selectlanguage{english}P\textsubscript{16}}{BISM (64) 1115}[]
\DeclareWitness{P17}{\selectlanguage{english}P\textsubscript{17}}{BISM 620/1886-92}[]
\DeclareWitness{P18}{\selectlanguage{english}P\textsubscript{18}}{BORI 615/1887-91}[]
\DeclareWitness{P19}{\selectlanguage{english}P\textsubscript{19}}{BISM 46-39}[]
\DeclareWitness{P20}{\selectlanguage{english}P\textsubscript{20}}{BISM 39-273}[]
\DeclareWitness{P21}{\selectlanguage{english}P\textsubscript{21}}{BISM 37-743}[]
\DeclareWitness{P22}{\selectlanguage{english}P\textsubscript{22}}{BISM 37-729}[]
\DeclareWitness{P23}{\selectlanguage{english}P\textsubscript{23}}{BISM 33-60}[]
\DeclareWitness{P24}{\selectlanguage{english}P\textsubscript{24}}{BISM 29-5790}[]% =P5!
\DeclareWitness{P25}{\selectlanguage{english}P\textsubscript{25}}{BISM 29-3657}[]
\DeclareWitness{P26}{\selectlanguage{english}P\textsubscript{26}}{BISM 25-281}[]
\DeclareWitness{P27}{\selectlanguage{english}P\textsubscript{27}}{BISM 7-489}[]
\DeclareWitness{P28}{\selectlanguage{english}P\textsubscript{28}}{BORI 399-1895-1902}[]

%%%%%   Mysore
\DeclareWitness{M1}{\selectlanguage{english}M\textsubscript{1}}{P-5682/4}[]
%%%%%   Tübingen
\DeclareWitness{Tue}{\selectlanguage{english}Tü}{Ma I 339}[]
%%%%%%%%%%
\DeclareWitness{YC}{\selectlanguage{english}YC}{Yogacintāmaṇi}[]
\DeclareWitness{ceteri}{\selectlanguage{english}cett.}{ceteri}[]

%%%%%%%%%% Mss with Commentary
\DeclareWitness{A1}{\selectlanguage{english}A\textsubscript{1}}{Alwar 952}[]

\DeclareWitness{Jyo}{\selectlanguage{english}J\textsubscript{yo}}{Brahmānanda's version}[]

%%%%%%%%%%%%%%%%%%%%%%%%%%%%%%%%%%%%%%%%%%%
%List of all Sigla:
%A1,B1,B2,B3,C1,C2,C3,C4,C6,C7,C8,C9,J1,J2,J3,J4,J10,J13,J14,J15,J17,L1,M1,N3,N5,N6,N9,N10,N11,N12,N13,N16,N17,N19,N20,N21,N22,N23,N24,Tü,V1,V2,V3,V4,V5,V6,V8,V11,V19,V22,V26,Vu
%%%%%%%%%%%%%%%%%%%%%%%%%%%%%%%%%%%%%%%%%%%

\DeclareWitness{G4}{\selectlanguage{english}G\textsubscript{4}}{GOML D18885 (Bundle SD5051)}[]
\DeclareWitness{G5}{\selectlanguage{english}G\textsubscript{5}}{GOML R3841/ SR2190}[]
\DeclareWitness{G7}{\selectlanguage{english}G\textsubscript{7}}{GOML D4394}[]

\DeclareWitness{Ko}{\selectlanguage{english}K\textsubscript{o}}{Koba, Gujarat 55626}[]

%
%%%%%                   Abbreviation for the printed apparatus,        xml interface needed
%%%%%                   (synonyms in same line)

% Macro for Editing Abbrevs.
%\def\om{\textrm{\footnotesize \textit{omitted in}\ }} %prints om. for omitted in apparatus
%\def\korr{\textrm{\footnotesize \textit{em.}\ }} %prints em. for emended in apparatus
%\def\conj{\textrm{\footnotesize \textit{conj.}\ }} %prints conj. for conjectured in apparatus


\def\eyeskip{\textrm{{ab.\,oc. }}}   
\def\aberratio{\textrm{{ab.\,oc. }}}
\def\ad{\textrm{{ad}}}   
\def\add{\textrm{{add.\ }}}
\def\ann{\textrm{{ann.\ }}}
\def\ante{\textrm{{ante }}}
\def\post{\textrm{{post }}}
%\def\ceteri{cett.\,}             % for simplifying the apparatus in print                  
\def\codd{\textrm{{codd.\ }}}   %  the same
\def\conj{\textrm{{coni.\ }}}  
\def\coni{\textrm{{coni.\ }}}
\def\contin{\textrm{{contin.\ }}}
\def\corr{\textrm{{corr.\ }}}
\def\del{\textrm{{del.\ }}}
\def\dub{\textrm{{ dub.\ }}}
\def\emend{\textrm{{emend.\ }}}
\def\expl{\textrm{{explic.\ }}}   
\def\explicat{\textrm{{explic.\ }}}
\def\fol{\textrm{{fol.\ }}}         
\def\foll{\textrm{{foll.\ }}}
\def\gloss{\textrm{{glossa ad }}}
\def\ins{\textrm{{ins.\ }}}          \def\inseruit{\textrm{{ins.\ }}}
\def\im{{\kern-.7pt\lower-1ex\hbox{\textrm{\tiny{\emph{i.m.}}}\kern0pt}}}
\def\inmargine{{\kern-.7pt\lower-.7ex\hbox{\textrm{\tiny{\emph{i.m.}}}\kern0pt}}}
\def\intextu{{\kern-.7pt\lower-.95ex\hbox{\textrm{\tiny{\emph{i.t.}}}\kern0pt}}}%\textrm{\scriptsize{i.t.\ }}}               
\def\indist{\textrm{{indis.\ }}}          \def\indis{\textrm{{indis.\ }}}
\def\iteravit{\textrm{{iter.\ }}}          \def\iter{\textrm{{iter.\ }}}  
\def\lectio{\textrm{{lect.\ }}}             \def\lec{\textrm{{lect.\ }}}
\def\leginequit{\textrm{{l.n. }}}         \def\legn{\textrm{{l.n. }}}         \def\illeg{\textrm{{l.n. }}}
\def\om{\textrm{{om. }}}
\def\primman{\textrm{{pr.m.}}}
\def\prob{\textrm{{prob.}}}
\def\rep{\textrm{{repetitio }}}
% \def\secundamanu{\textrm{\scriptsize{s.m.}}}
% \def\secm{{\kern-.6pt\lower-.91ex\hbox{\textrm{\tiny{\emph{s.m.}}}\kern0pt}}}%   \textrm{\scriptsize{s.m.}}}
\def\sequentia{\textrm{{seq.\,inv.\ }}}         \def\seqinv{\textrm{{seq.\,inv.\ }}} \def\order{\textrm{{seq.\,inv.\ }}}
\def\supralineam{{\kern-.7pt\lower-.91ex\hbox{\textrm{\tiny{\emph{s.l.}}}\kern0pt}}} %\textrm{\scriptsize{s.l.}}}
\def\interlineam{{\kern-.7pt\lower-.91ex\hbox{\textrm{\tiny{\emph{s.l.}}}\kern0pt}}}   %\textrm{\scriptsize{s.l.}}}
\def\vl{\textrm{v.l.}}   \def\varlec{\textrm{v.l.}} \def\varialectio{\textrm{v.l.}}
\def\vide{\textrm{{cf.\ }}}           \def\cf{\textrm{{cf.\ }}}
\def\videtur{\textrm{{vid.\,ut}}}
\def\crux{\textup{[\ldots]} }
\def\cruxx{\textup{[\ldots]}}
\def\unm{\textit{unm.}}        % unmetrical
%%%%%%%%%%%%%%%%%%%%%%%%%%%%%%%%%%%%



%%% Local Variables:
%%% mode: latex
%%% TeX-master: t
%%% End:

% addition 2023-12-11 MD:
\TeXtoTEIPat{\begin {metre}[#1]}{<note type="metre" target="##1">}
\TeXtoTEIPat{\end {metre}}{</note>}
\TeXtoTEIPat{\texttheta}{θ}

% change 2023-12-05 mm
\TeXtoTEI{teimute}{} 

% changes/additions 2023-11-27 MM:
\TeXtoTEIPat{\medialink {#1}{#2}}{<ref target="resources/#2">#1</ref>}

% changes/additions 2023-10-25 MM:
% new Sigla
\TeXtoTEIPat{\textAlpha}{Α}
\TeXtoTEIPat{\textalpha}{α}
\TeXtoTEIPat{\textBeta}{Β}
\TeXtoTEIPat{\textbeta}{β}
\TeXtoTEIPat{\textGamma}{Γ}
\TeXtoTEIPat{\textgamma}{γ}
\TeXtoTEIPat{\textDelta}{Δ}
\TeXtoTEIPat{\textdelta}{δ}
\TeXtoTEIPat{\textEpsilon}{Ε}
\TeXtoTEIPat{\textepsilon}{ε}
\TeXtoTEIPat{\textEta}{Η}
\TeXtoTEIPat{\texteta}{η}
\TeXtoTEIPat{\textChi}{Χ}
\TeXtoTEIPat{\textchi}{χ}
\TeXtoTEIPat{\textOmega}{Ω}
\TeXtoTEIPat{\textomega}{ω}

%new environments
\TeXtoTEIPat{\begin {postmula}[#1]}{<note type="postmula" target="##1">}
  \TeXtoTEIPat{\end {postmula}}{</note>}
\TeXtoTEIPat{\begin {altava}[#1]}{<div type="altrec"><note type="avataranika" target="##1">} %%% changed 2023-12-05 mm
  \TeXtoTEIPat{\end {altava}}{</note></div>} %%% changed 2023-12-05 mm
\TeXtoTEIPat{\sgwit {#1}}{<note type="inlineref"><ref>#1</ref></note>}

% changes/additions 2023-10-12 MM:
\TeXtoTEIPat{\\.}{}

% changes/additions 2023-08-15 MD:
\TeXtoTEIPat{\lineom {#1}{#2}}{<note type="omission">#1 omitted in <ref>#2</ref></note>}
\TeXtoTEI{graus}{hi}[rend="grey"]
\TeXtoTEIPat{\startgray}{} %%% changed 2023-12-05 mm
\TeXtoTEIPat{\endgray}{} %%% changed 2023-12-05 mm



% additions/changes 2023-06-05 mm:
%\TeXtoTEIPat{\lineom {#1}}{<note type="omission">Line omitted in <ref>#1</ref></note>}
\TeXtoTEIPat{\NotIn {#1}}{<note type="omission">Stanza omitted in <ref>#1</ref></note>}

% additions 2023-04-16 MD:
\TeXtoTEIPat{\,}{ }

% additions 2023-04-13 mm:
\TeXtoTEIPat{\begin {versinnote}}{<lg>}
  \TeXtoTEIPat{\end {versinnote}}{</lg>}

% additions 2023-04-05 MD:
\TeXtoTEIPat{\begin {testimonia}[#1]}{<note type="testimonia" target="##1">}
  \TeXtoTEIPat{\end {testimonia}}{</note>}
\TeXtoTEI{devnote}{s}[xml:lang="sa-deva"]

% app in philcomm und testimonia %%% added MM 2023-12-02
\TeXtoTEI{var}{note}[type="appinnote"]


\TeXtoTEI{anm}{note}[type="memo"] %% change 2023-04-16 MD
\TeXtoTEI{Anm}{note}[type="memo"] %% change 2023-12-05 MM
\TeXtoTEIPat{\startverse}{} %%% marked for change 2023-04-13 mm
\TeXtoTEIPat{\endverse}{} %%% marked for change 2023-04-13 mm
\TeXtoTEIPat{\newpage}{}
\TeXtoTEIPat{\marma}{}
\TeXtoTEIPat{\marmas}{}
\TeXtoTEIPat{\vin}{} % added by MD 2023-11-14

%%% modify environments and commands
%%% TEI mapping
% additions/changes 2022-06-07 mm:
\TeXtoTEI{grau}{hi}[rend="grey"]
\TeXtoTEIPat{ \& }{ &amp; }

% additions/changes 2022-06-01 mm:
\TeXtoTEI{skp}{seg}[type="deva-ignore"]
\TeXtoTEI{skm}{seg}[type="ltn-ignore"]

\TeXtoTEIPat{\rlap {#1}}{#1}

% additions/changes 2022-04-06 mm:
%\TeXtoTEI{sgwit}{ref}
\TeXtoTEI{textdev}{s}[xml:lang="sa-deva"]
\TeXtoTEIPat{\begin {col}[#1]}{<div type="colophon" xml:id="#1"><p>}
  \TeXtoTEIPat{\end {col}}{</p></div>}
\TeXtoTEIPat{\begin {ava}[#1]}{<note type="avataranika" target="##1">}
  \TeXtoTEIPat{\end {ava}}{</note>}
												   
\TeXtoTEIPat{\outdent}{}
\TeXtoTEIPat{\startaltrecension}{} %%% changed 2023-12-05 mm
\TeXtoTEIPat{\endaltrecension}{} %%% changed 2023-12-05 mm
\TeXtoTEIPat{\startaltnormal}{} % added by MD 2023-11-14 %%% changed 2023-12-05 mm
\TeXtoTEIPat{\endaltnormal}{} % added by MD 2023-11-14 %%% changed 2023-12-05 mm
\TeXtoTEIPat{\begin {alttlg}[#1]}{<div type="altrec"><lg xml:id="#1">}
  \TeXtoTEIPat{\end {alttlg}}{</lg></div>}



% additions/changes 2022-03-12 mm:
\TeXtoTEIPat{\begin {tlg}[#1]}{<lg xml:id="#1">}
  \TeXtoTEIPat{\end {tlg}}{</lg>}

\TeXtoTEIPat{\begin {translation}[#1]}{<note type="translation" target="##1">}
  \TeXtoTEIPat{\end {translation}}{</note>}
\TeXtoTEIPat{\begin {philcomm}[#1]}{<note type="philcomm" target="##1">}
  \TeXtoTEIPat{\end {philcomm}}{</note>}
\TeXtoTEIPat{\begin {sources}[#1]}{<note type="sources" target="##1">}
  \TeXtoTEIPat{\end {sources}}{</note>}


\TeXtoTEIPat{\begin {marma}[#1]}{<note type="marma" target="##1">}
  \TeXtoTEIPat{\end {marma}}{</note>}

\TeXtoTEIPat{\begin {jyotsna}[#1]}{<note type="jyotsna" target="##1">}
  \TeXtoTEIPat{\end {jyotsna}}{</note>}

\EnvtoTEI{description}{list}
\EnvtoTEI{itemize}{list}
\TeXtoTEIPat{\item [#1]}{<label>#1</label>\item}

\TeXtoTEI{tl}{l}
\TeXtoTEI{myfn}{note}[type="myfn"]
\TeXtoTEIPat{\getsiglum {#1}}{<ref target="##1"/>}

\TeXtoTEI{SetLineation}{}
\TeXtoTEI{noindent}{}
\TeXtoTEI{subsection*}{}

\TeXtoTEI{rlap}{}

% end additions/changes
% \TeXtoTEIPat{\skp {#1}}{#1}
% \TeXtoTEIPat{\skm {#1}}{}

\TeXtoTEIPat{\begin {prose}}{<p>}
  \TeXtoTEIPat{\end {prose}}{</p>}

\TeXtoTEIPat{\begin {tlate}}{<p>}
  \TeXtoTEIPat{\end {tlate}}{</p>}

\TeXtoTEI{emph}{hi}
\TeXtoTEI{bigskip}{}
% \TeXtoTEI{/}{|}
\TeXtoTEI{tl}{l}
\TeXtoTEIPat{english}{}
%\TeXtoTEIPat{-}{ } %% change 2023-04-16 MD
%\TeXtoTEIPat{°}{} %% change 2023-04-16 MD
\TeXtoTEIPat{\textcolor {#1}{#2}}{<hi rend="#1">#2</hi>}

% \TeXtoTEIPat{\eyeskip}{}
% \TeXtoTEIPat{\aberratio}{}
% \TeXtoTEIPat{\ad}{}
\TeXtoTEIPat{\add}{<hi rend="italic">add.</hi>} %% change 2023-04-16 MD
% \TeXtoTEIPat{\ann}{}
\TeXtoTEIPat{\ante}{<hi rend="italic">ante</hi> } %% change 2023-04-16 MD
\TeXtoTEIPat{\post}{<hi rend="italic">post</hi> } %% change 2023-04-16 MD
% \TeXtoTEIPat{\codd}{}
% \TeXtoTEIPat{\conj }{}
% \TeXtoTEIPat{\contin}{}
% \TeXtoTEIPat{\corr}{}
% \TeXtoTEIPat{\del}{}
% \TeXtoTEIPat{\dub}{}
% \TeXtoTEIPat{\emend }{}
% \TeXtoTEIPat{\expl}{}
% \TeXtoTEIPat{\ȩxplicat}{}
% \TeXtoTEIPat{\fol}{}
% \TeXtoTEIPat{\gloss}{}
% \TeXtoTEIPat{\ins}{}
% \TeXtoTEIPat{\im}{}
% \TeXtoTEIPat{\inmargine}{}
% \TeXtoTEIPat{\intextu}{}
% \TeXtoTEIPat{\indist}{}
% \TeXtoTEIPat{\iteravit}{}
% \TeXtoTEIPat{\lectio}{}
% \TeXtoTEIPat{\leginequit}{}
% \TeXtoTEIPat{\legn}{}
% \TeXtoTEIPat{\illeg}{<hi rend="italic">illeg.</hi>}
\TeXtoTEIPat{\illeg}{<gap reason="illeg."/>} %%% change 2023-04-11 mm
% \TeXtoTEIPat{\om}{<hi rend="italic">om.</hi>}
\TeXtoTEIPat{\om}{<gap reason="om."/>} %%% change 2023-04-11 mm
% \TeXtoTEIPat{\primman}{}
% \TeXtoTEIPat{\prob}{}
% \TeXtoTEIPat{\rep}{}
% \TeXtoTEIPat{\sequentia}{}
% \TeXtoTEIPat{\supralineam}{}
% \TeXtoTEIPat{\interlineam}{}
\TeXtoTEIPat{\vl}{<hi rend="italic">v.l.</hi>}
% \TeXtoTEIPat{\vide}{}
% \TeXtoTEIPat{\videtur}{}
% \TeXtoTEIPat{\crux}{}
% \TeXtoTEIPat{\cruxxx}{}
\TeXtoTEIPat{\unm}{<hi rend="italic">unm.</hi>}


% List of Scholars
\DeclareScholar{nos}{nos}[
forename=HPP,
surname=Team]


% Nullify \selectlanguage in TEI as it has been used in
% \DeclareWitness but should be ignored in TEI.
\TeXtoTEI{selectlanguage}{}


\SetTEIxmlExport{autopar=false}

%%%%%%%%%%%

\SetTEIxmlExport{autopar=false}
\NewDocumentEnvironment{translation}{O{}}{\textcolor{blue}{\textbf{Translation:}}}{}
\NewDocumentEnvironment{philcomm}{O{}}{
	\textcolor{blue}{\textbf{Commentary:}}}{}
\NewDocumentEnvironment{metre}{O{}}{
	\textcolor{blue}{\textbf{Metre:}}}{} % added MD 2023-12-11
\NewDocumentEnvironment{sources}{O{}}{
	\textcolor{blue}{\textbf{Sources:}}\linebreak}{}
\NewDocumentEnvironment{testimonia}{O{}}{
	\textcolor{blue}{\textbf{Testimonia:}}\linebreak}{}
\NewDocumentEnvironment{versinnote}{O{}}{\begin{ekdverse}}{\end{ekdverse}}
%\newcommand{\var}[1]{\footnotesize\textup{#1}}
\newcommand{\medialink}[2]{\textcolor{green}{\underline{#1}}}
%\TeXtoTEIPat{\medialink {#1}{#2}}{<ref target="/images/#2">#1</ref>}

\NewDocumentCommand{\tl}{m}{#1}

\def\vl{\textit{v.l.}}
\def\var#1{{\footnotesize #1}}
\def\sl#1{\emph{#1}}

%%%%%%%%%%%%

\usepackage{textgreek}

\newcommand{\alphaOne}{\textalpha\textsubscript{1}}% N3
\newcommand{\alphaTwo}{\textalpha\textsubscript{2}}% J5
\newcommand{\alphaThree}{\textalpha\textsubscript{3}}% G4
\newcommand{\betaOne}{\textbeta\textsubscript{1}}% P11
\newcommand{\betaTwo}{\textbeta\textsubscript{2}}% C6
\newcommand{\betaOmega}{\textbeta\textsubscript{\textomega}}% V3
\newcommand{\gammaOne}{\textgamma\textsubscript{1}}% N23
\newcommand{\gammaTwo}{\textgamma\textsubscript{2}}% J7
\newcommand{\deltaOne}{\textdelta\textsubscript{1}}% V19
\newcommand{\deltaTwo}{\textdelta\textsubscript{2}}% K3
\newcommand{\deltaThree}{\textdelta\textsubscript{3}}% C7
\newcommand{\deltaOmega}{\textdelta\textsubscript{\textomega}}% J6
\newcommand{\epsilonOne}{\textepsilon\textsubscript{1}}% P15
\newcommand{\epsilonTwo}{\textepsilon\textsubscript{2}}% N19
\newcommand{\epsilonThree}{\textepsilon\textsubscript{3}}% V15
\newcommand{\epsilonFour}{\textepsilon\textsubscript{4}}% J11
\newcommand{\epsilonOmega}{\textepsilon\textsubscript{\textomega}}% N26
\newcommand{\etaOne}{\texteta\textsubscript{1}}% V1
\newcommand{\etaTwo}{\texteta\textsubscript{2}}% J10
\newcommand{\etaOmega}{\texteta\textsubscript{\textomega}}% N9

%%%%%%%%%%%%%%

\babelhyphenation{%
	Dattā-treya-yoga-śāstra
	Gorakṣa-śataka
	Haṭha-pra-dī-pikā
	Hātha-ratnā-valī
	Svātmā-rāma
	Śiva-saṃhitā
	Vasiṣṭha-saṃhitā
	Viveka-mārtaṇḍa
	Yukta-bhava-deva
	Yoga-cintā-maṇi
	Yoga-yājña-valkya}

\begin{document}
\pagestyle{HPed}
\begin{ekdosis}
\SetLineation{lineation = none,}

%\chapter*{Translation \& philological commentary}

\subsection*{Kj.1 Kālajñāna}
%<*tr1a>
\begin{translation}[Kj01a] 
%for atha kālajñānam
\end{translation}
%</tr1a>

%<*tr1>
\begin{translation}[Kj01] 
Listen, I will teach special omens, from seeing which the adept of yoga knows [the time of] his death.
\end{translation}
%</tr1>
% I think there are a few ways to translate this verse...
% Listen, I will teach the best omens, the seeing of which enables the adept of yoga to know his own death.
% or without the relative clause...
% Listen, I will teach the best omens. By seeing them, the adept of yoga knows his own death. 
%JM: maybe add [time of] before death. Reads oddly in English otherwise.

%<*sc1>
\begin{sources}[Kj01]
\emph{Mārkaṇḍeyapurāṇa} 40.1, cf.~\emph{Śārṅgadharapaddhati} 4566
\begin{variants}
viśiṣṭāni tāni vakṣyāmi tvaṃ śṛṇu~] mahārāja śṛṇu vakṣyāmi tāni te MP, viśiṣṭāni śṛṇu vakṣyāmi tāni te ŚDP    
\end{variants}
% \begin{versinnote}
% \tl{dattātreya uvāca/\\+}
% \tl{ariṣṭāni mahārāja śṛṇu vakṣyāmi tāni te/\\+}
% \tl{yeṣām ālokanān mṛtyuṃ nijaṃ jānāti yogavit//\\!}  
% \end{versinnote}

% Cf.~\emph{Śārṅgadharapaddhati} 4566
% \begin{versinnote}
% \tl{ariṣṭāni viśiṣṭāni śṛṇu vakṣyāmi tāni te/\\+}
% \tl{yeṣām ālokanān mṛtyuṃ nijaṃ jānāti yogavit//\\!}
% \end{versinnote}
\end{sources}
%</sc1>

%<*ts1>
%\begin{testimonia}[Kj01]

%\end{testimonia}
%</ts1>

%<*cm1>
%\begin{philcomm}[Kj01]
%The word order of the first hemistich is different in the HP transmission to that of the MāPur and ŚāDhaPa, which seems clearer to me (Listen to these special omens. I will teach you those which ...)
%\end{philcomm}
%</cm1>
%%%%%

\subsection*{Kj.2}%?? no anusvāra at end of verse [MD: done]
%<*tr2>
\begin{translation}[Kj02]
The man who cannot see the Milky Way, the pole star, Venus, the light of the moon, and Arundhatī (i.e.~the star, Alcor) will not live more than a year.
\end{translation}%?? just Alcor (Cf.~Venus)? JB: is the note below adequate?
%</tr2>
% JM: I think chāyā here means "light" or "beauty"; "will not live"
%NJL: "should/would not live?"

%<*sc2>
\begin{sources}[Kj02]
\emph{Mārkaṇḍeyapurāṇa} 40.2, cf.~\emph{Śārṅgadharapaddhati} 4567
\mylb
% \begin{versinnote}
% \tl{devamārgaṃ dhruvaṃ śukraṃ somacchāyām arundhatīm/\\+}
% \tl{yo na paśyen na jīvet sa naraḥ saṃvatsarāt param//\\!}
% \end{versinnote}

Cf.~\emph{Dharmaputrikā} 198
\begin{versinnote}
\tl{arundhatīṃ dhruvañ caiva somacchāyāṃ mahāpathaṃ/\\+}
\tl{yo na paśyen na jīveta naraḥ saṃvatsarāt paraṃ//\\!}
\end{versinnote}

% Cf.~\emph{Śārṅgadharapaddhati} 4567
% \begin{versinnote}
% \tl{devamārgaṃ dhruvaṃ śukraṃ somacchāyām arundhatīm/\\+}
% \tl{yo na paśyen na jīvet sa naraḥ saṃvatsarāt param//\\!}
% \end{versinnote}
\end{sources}
%</sc2>

%<*ts2>
%\begin{testimonia}[Kj02]

%\end{testimonia}
%</ts2>

%<*cm2>
\begin{philcomm}[Kj02]
We are not certain of the meaning of \emph{devamārga} here. The Monier-Williams dictionary (s.v.: 1899) says it means the sky but one would expect it to have a more specific astrological meaning in a list of stars and planets. It could be a synonym for \emph{devapatha}, which can mean the Milky Way according to one lexicographical source (\emph{Trikāṇḍaśeṣa} 1.1.97).\lb  
%?? JM leave out the M-W sentence? It's based on only Gal. i.e. Galanos' dictionary. JB: It might be accurate, as Goldman and Goldman (1996, vol. V: 531) discuss a commentary of the Rāmāyaṇa that takes devamārga as the sky (another commentary on the same verse takes devamārga as a kenning for the anus: the path of the [lower wind] god).
% ?? JB I've just seen this reference to an ariṣta in the viṣṇudharmottarapurāṇa: na paśyati tu yo vyabhre prabhām arkaniśākṛtoḥ / chāyāṃ candrārkayoś cāpi tayoś ca gamanaṃ tathā // devamārgaprabhāṃ vahner dhruvaṃ tārām arundhatīm / paśyaty adṛṣṭam anyair vā mṛtyuḥ syāt tasya vatsarāt // Its a close parallel. Is it worth citing as a cf. in sources? Perhaps, prabhā suggests its a celestial thing rather than the sky?
    % devamārga - does it mean sky (s.v. MW) here or, given the stars and planets mentioned in this list, could it mean the Milky Way (Cf.~devapatha)?
    %Jürgen: In Böhtlingk dict. most words denoting milky way are prefixed by ``probably, perhaps'' etc.  so he was not sure. devapatha in this sense is only attested in lexicography (Trikāṇḍaśeṣa 1.1.97), I could not look up the passage, here commentaries could be enlightening, but I am not sure whether we can really determine that it is one or the other.
    %JM: I don't think it can be just the sky here, it must be something more specific. Milky Way seems feasible. Perhaps translate "who cannot see..."

Arundhatī is mentioned in the \emph{Bṛhatsaṃhitā} (13.6) and is said to be close to Vasiṣṭha (\emph{tatra vasiṣṭhaṃ munivaram upāśritārundhatī sādhvī}). Vasiṣṭha and Arundhatī are the double star, Mizar and Alcor in the \emph{saptarṣi nakṣātra} (Ursa Major). See Rao 2019: 53.

\end{philcomm}
%</cm2>
%See Rao 2019, Astronomy in Hindu Religion and Culture, International Journal of Jyotish Research, p.53.
%%%%%

\subsection*{Kj.3}
%<*tr3>
\begin{translation}[Kj03]
When he sees a rayless disc of the sun and a fire with a halo, a man lives eleven months and not longer.%?? JM no longer than eleven months; or, better, no emendation and read °māsāṃs (which matches the other verses). JB: see the comments below (One would expect the ablative with ūrdhvaṃ)
\end{translation}
%</tr3>
% literally: a fire with a garland of rays of light
% JM: When he sees... ; disc rather than orb

%<*sc3>
\begin{sources}[Kj03]
\emph{Mārkaṇḍeyapurāṇa} 40.3, cf.~\emph{Śārṅgadharapaddhati} 4568
\begin{variants}
 māsāt tu~] māsebhyo MP, māsāṃś ca ŚDP\sep
 sa ŚDP~] tu MP
\end{variants}

% \begin{versinnote}
% \tl{araśmi bimbaṃ sūryasya vahniṃ caivāṃśumālinam/\\+}
% \tl{dṛṣṭvaikādaśamāsebhyo naro nordhvaṃ tu jīvati//\\!}
% \end{versinnote}

% Cf.~\emph{Śārṅgadharapaddhati} 4568
% \begin{versinnote}
% \tl{araśmi bimbaṃ sūryasya vahniṃ caivāṃśumālinam/\\+}
% \tl{dṛṣṭvaikādaśamāsāṃś ca naro nordhvaṃ sa jīvati//\\!}
% \end{versinnote}

Cf.~\emph{Śivasvarodaya} 343
\begin{versinnote}
\tl{araśmi bimbaṃ sūryasya vahneḥ śītāṃśumālinaḥ/\\+}
\tl{dṛṣṭvaikādaśamāsāyur naraś cordhvaṃ na jīvati//\\!}
\end{versinnote}
\end{sources}
%</sc3>

%<*ts3>
%\begin{testimonia}[Kj03]

%\end{testimonia}
%</ts3>

%<*cm3>
%\begin{philcomm}[Kj03]
%This verse occurs in quite a few texts (with variations), e.g.~ \emph{Rasamañjari} 10.25 (\emph{araśmi bimbaṃ sūryasya vahneś caivāṃśuvarjitam}/ \emph{dṛṣṭvaikādaśamāsāstu naraścordhvaṃ na jīvati}), \emph{Mahāsubhāṣitasaṅgraha} 2844 (\emph{araśmi bimbaṃ sūryasya vahniṃ caivāṃśumālinam}/ \emph{dṛṣṭvaikādaśamāsāttu naro norddhvaṃ tu jīvati}), etc.
%The original idea may be that expressed in the ŚiSvaUd (araśmi bimbaṃ sūryasya vahneḥ śītāṃśumālinaḥ) but the verse we have has changed.

%One would expect the ablative with ūrdhvaṃ (i.e.~°māsāt tu or °māsebhyo)
    % MD: °māsāt seems to be better.
    % Agree: I've emended to °māsāt 
    % MD: Delete the last sentence of the commentary?
%\end{philcomm}
%</cm3>


\subsection*{Kj.4}
%<*tr4>
\begin{translation}[Kj04]
[The man] who emits urine and faeces as gold and silver, either in reality or in a dream, his life [will last] ten months.
\end{translation}
%</tr4>
    %JM ?? will live for ten months (otherwise syntax doesn't work): JB: I have assumed that one has to supply tasya and the verb-to-be with jīvitaṃ daśamāsikam. Otherwise, we would have to emend to the MP's reading to justify your suggested translation. 
    
%<*sc4>
\begin{sources}[Kj04]
\emph{Mārkaṇḍeyapurāṇa} 40.4, cf.~\emph{Śārṅgadharapaddhati} 4569
\begin{variants}
 vamen~] vānte MP, vāntyā ŚDP\sep   
 purīṣaṃ ŚDP~] purīṣe MP\sep
 svarṇaṃ rajataṃ tathā MP~] suvarṇarajataṃ vamet ŚDP\sep
 athavā ŚDP~] kurute MP\sep
 jīvitaṃ ŚDP~] jīvet sa MP
\end{variants}

% \begin{versinnote}
% \tl{vānte mūtrapurīṣe ca yaḥ svarṇaṃ rajataṃ tathā/\\+}
% \tl{pratyakṣaṃ kurute svapne jīvet sa daśamāsikam//\\!}
% \end{versinnote}

% Cf.~\emph{Śārṅgadharapaddhati} 4569
% \begin{versinnote}
% \tl{vāntyā mūtrapurīṣaṃ yaḥ suvarṇarajataṃ vamet//\\+}
% \tl{pratyakṣam athavā svapne jīvitaṃ daśamāsikam//\\!}
% \end{versinnote}
 
Cf.~\emph{Dharmaputrikā} 200
\begin{versinnote}
\tl{chardimūtrapuriṣāṇi yasya hemarajo bhavet/\\+}
\tl{pratyakṣam athavā svapne tv aṣṭau māsān sa jīvati//\\!}
\end{versinnote}

\end{sources}
%</sc4>

%<*ts4>
\begin{testimonia}[Kj04]
\emph{Yogacintāmaṇi} f.\,142v (\attr Mārkaṇḍeya)
\begin{variants}
  mūtra~] mūtraṃ YCM\sep
  yaḥ su~] ca yaḥ YCM
\end{variants}

% \begin{versinnote}
% \tl{vamen mūtraṃ puriṣaṃ ca yaḥ svarṇaṃ rajataṃ tathā/\\+}
% \tl{pratyakṣam athavā svapne jīvitaṃ daśamāsikam//\\!}
% \end{versinnote}
\end{testimonia}
%</ts4>

%<*cm4>
\begin{philcomm}[Kj04]
In the version of this verse transmitted in the \emph{Haṭhapradīpikā}, the relative pronoun \emph{yaḥ} requires a verb in the first hemistich, as seen in the \emph{Yogacintāmaṇi}, whose verse is a close parallel to the one in question, hence our emendation of \emph{vātyāṃ} to \emph{vamen} (cf.~\emph{Yogacintāmaṇi}).\lb
    %The word vātyāṃ appears to be a corruption of vāntyāṃ or vāntyā (Cf.~\emph{Śārṅgadharapaddhati}, which has vamet in pāda 2, and MāPur, which has kurute in pāda 3).

The idea behind this verse is clearly expressed in the \emph{Dharmaputrikā}, where the yogi whose urine and faeces become gold and silver, in reality or in sleep, has eight months left to live.
\end{philcomm}
%</cm4>
    % MD: difficult. Can √vam be used with mūtrapurīṣa? I suspect that yaḥ might be wrong.
    % Yes, mūtrapurīṣaṃ is the object of vamet in the Śārṅgadharapaddhati's version of this verse  
    % NL: perhaps em. to Mārkaṇḍeyapurāṇa's "vānte mūtrapurīṣe". 
    % then we don't have a verb in the relative clause (note that the MāPur has kurute in pāda c, which is different to the HP's version).
    % MD: Both Dharmaputrikā and Mṛtyuvañcanopadeśa have chardi, which corresponds to vānti. And the combination of mūtrapurīṣa and √vam "to vomit" is strange. But I have no idea how to reconstruct the passage.
    % Vāgīśvarakīrtis Mṛtyuvañcanopadeśa
    % chardiṃ mūtraṃ purīṣaṃ ca suvarṇarajataprabham/
    % paśyed ekaikaśas tasya mṛtyur māsāvadher bhavet// VMv_1.68//

\subsection*{Kj.5}
%<*tr5>
\begin{translation}[Kj05]
After seeing Pretas, Piśācas and so on, Gandharva cities and gold-coloured trees, he lives nine months.% Stick to [the yogi] or [a man] rather than change to 'one'? JB:okay
\end{translation}
%</tr5>

%<*sc5>
\begin{sources}[Kj05]
\emph{Mārkaṇḍeyapurāṇa} 40.5, cf.~\emph{Śārṅgadharapaddhati} 4570
\begin{variants}
    varṇān MP~] varṇa ŚDP
\end{variants}
% \begin{versinnote}
% \tl{dṛṣṭvā pretapiśācādīn gandharvanagarāṇi ca/\\+}
% \tl{suvarṇavarṇān vṛkṣāṃś ca nava māsān sa jīvati//\\!}
% \end{versinnote}

% Cf.~\emph{Śārṅgadharapaddhati} 4570
% \begin{versinnote}
% \tl{dṛṣṭvā pretapiśācādīn gandharvanagarāṇi ca/\\+}
% \tl{suvarṇavarṇavṛkṣāṃś ca nava māsān sa jīvati//\\!}
% \end{versinnote}
\end{sources}
%</sc5>

%<*ts5>
\begin{testimonia}[Kj05]
\emph{Yogacintāmaṇi} f.\,143r (\attr Mārkaṇḍeya)
\begin{variants}
    varṇān~] varṇa YCM
\end{variants}
% \begin{versinnote}
% \tl{dṛṣṭvā pretapiśācādīn gandharvanagarāṇi ca/\\+}
% \tl{suvarṇavarṇavṛkṣāṃś ca nava māsān sa jīvati//\\!}
% \end{versinnote}
\end{testimonia}
%</ts5>

%<*cm5>
%\begin{philcomm}[Kj05]
%\end{philcomm}
%</cm5>

\subsection*{Kj.6}
%<*tr6>
\begin{translation}[Kj06]
He who is fat and suddenly becomes thin or who is thin and suddenly becomes fat and who deviates from his constitution has a life-span of eight months.% remove prakṛti? Or if keeping it, prakṛtyāḥ?
\end{translation}
%</tr6>
%JM: maybe just one person is being described here. Suddenly goes with both possible changes.

%<*sc6>
\begin{sources}[Kj06]
\emph{Mārkaṇḍeyapurāṇa} 40.6, cf.~\emph{Śārṅgadharapaddhati} 4571
\begin{variants}
    prakṛtyāś~] prakṛteś MP, ŚDP
\end{variants}
% \begin{versinnote}
% \tl{sthūlaḥ kṛśaḥ kṛśaḥ sthūlo yo 'kasmād eva jāyate/\\+}
% \tl{prakṛteś ca nivarteta tasyāyuś cāṣṭamāsikam//\\!}
% \end{versinnote}

% Cf.~\emph{Śārṅgadharapaddhati} 4571
% \begin{versinnote}
% \tl{sthūlaḥ kṛśaḥ kṛśaḥ sthūlo yo 'kasmād eva jāyate/\\+}
% \tl{prakṛteś ca nivarteta tasyāyuś cāṣṭamāsikam//\\!}
% \end{versinnote}
\end{sources}
%</sc6>

%<*ts6>
% \begin{testimonia}[Kj06]

% \end{testimonia}
%</ts6>

%<*cm6>
%\begin{philcomm}[Kj06]
%prakṛtyāś ca nivartaṃ ca strikes me as odd (ca x2 and nivartam doesnt seem to be well attested in the dictionary). Should we emend to prakṛtyāś ca nivarteta ? 
    % MD: I am in favour of the emendation.
    % Jürgen suggests we emend
    % done.
%\end{philcomm}
%</cm6>

\subsection*{Kj.7}
%<*tr7>
\begin{translation}[Kj07]
He whose footprint in dirt and mud is missing at the heel and toes lives for seven months. 
\end{translation}
%</tr7>


%kardame pāṃśudeśe vā purataḥ pṛṣṭhato 'pi vā/
%khaṇḍaṃ pādodaye nyūnaṃ mṛtyur māsacatuṣṭayāt// VMv_1.85//

%<*sc7>
\begin{sources}[Kj07]
\emph{Mārkaṇḍeyapurāṇa} 40.7, cf.~\emph{Śārṅgadharapaddhati} 4572
\begin{variants}
    pārṣṇau~] pārṣṇyāṃ MP, pārṣṇyoḥ ŚDP\sep
    tathā~] ca vā MP, 'thavā ŚDP\sep
    kardamayor madhye MP~] kardamamadhye vā ŚDP
\end{variants}

% \begin{versinnote}
% \tl{khaṇḍaṃ yasya padaṃ pārṣṇyāṃ pādasyāgre ca vā bhavet/\\+}
% \tl{pāṃśukardamayor madhye sapta māsān sa jīvati//\\!}
% \end{versinnote}
% Cf.~\emph{Śārṅgadharapaddhati} 4572
% \begin{versinnote}
% \tl{khaṇḍaṃ yasya padaṃ pārṣṇyoḥ pādasyāgre 'thavā bhavet/\\+}
% \tl{pāṃśukardamamadhye vā sapta māsān sa jīvati//\\!}
% \end{versinnote}

Cf.~\emph{Liṅgapurāṇa} 1.91.7
\begin{versinnote}
\tl{agrataḥ pṛṣṭhato vāpi khaṇḍaṃ yasya padaṃ bhavet/\\+}
\tl{pāṃsuke kardame vāpi saptamāsān sa jīvati//\\!}
\end{versinnote}

\end{sources}
%</sc7>

%<*ts7>
\begin{testimonia}[Kj07]
\emph{Yogacintāmaṇi} f.\,143r (\attr Mārkaṇḍeya)
\begin{variants}
    pārṣṇau~] pārṣṇyāṃ YCM\sep
    tathā~] 'thavā YCM
\end{variants}

% \begin{versinnote}
% \tl{khaṇḍaṃ yasya padaṃ pārṣṇyāṃ pādasyāgre 'thavā bhavet/\\+}
% \tl{pāṃśukardamayor madhye saptamāsān sa jīvati//\\!}
% \end{versinnote}
\end{testimonia}
%</ts7>

%<*cm7>
%\begin{philcomm}[Kj07]
%\emph{Mārkaṇḍeyapurāṇa} and \emph{Yogacintāmaṇi} have pārṣṇyāṃ. Is this how pārṣṇeḥ is to be understood ('broken at the heel') or should it be read with agre (as I have translated it above)? 
    % MD: emend to pārṣṇau?
    % NJ: I agree with Mitsuyo's suggestion. It explains our variants well and aligns in meaning with the Mārkaṇḍeyapurāṇa variant.
    % okay, I've emended to pārṣṇau 
%It is likely that the original idea of this omen is found in the \emph{Liṅgapurāṇa} and \emph{Śārṅgadharapaddhati}, which say that one's foot is broken either at the front or back or it is in dirt and mud.

%The Śārṅgadharapaddhati has pāṃśukardamamadhye vā ('or is dirt and mud'), which is much clearer. The HP version has lost the vā.
%\end{philcomm}
%</cm7>

\subsection*{Kj.8}
%<*tr8>
\begin{translation}[Kj08]
A pigeon, vulture, raven, crow or [any other] carrion-eating bird perched on his head indicates a lifespan of six months. 
\end{translation}
%</tr8>

%<*sc8>
\begin{sources}[Kj08]
\emph{Mārkaṇḍeyapurāṇa} 40.8, cf.~\emph{Śārṅgadharapaddhati} 4573
\begin{variants}
    kapotagṛdhrau~] gṛdhraḥ kapotaḥ MP, kapotagṛdhra ŚDP\sep
    kākolo MP~] kākolā ŚDP\sep
    vā khago MP~] vāparo ŚDP
\end{variants}

% \begin{versinnote}
% \tl{gṛdhraḥ kapotaḥ kākolo vāyaso vāpi mūrdhani/\\+}
% \tl{kravyādo vā khago līnaḥ ṣaṇmāsāyuḥpradarśakaḥ//\\!}
% \end{versinnote}

% Cf.~\emph{Śārṅgadharapaddhati} 4568
% \begin{versinnote}
% \tl{kapotagṛdhrakākolā vāyaso vāpi mūrdhani/\\+}
% \tl{kravyādo vāparo līnaḥ ṣaṇmāsāyuḥpradarśakaḥ//\\!}
% \end{versinnote}
\end{sources}
%</sc8>

%<*ts8>
\begin{testimonia}[Kj08]
\emph{Yogacintāmaṇi} f.\,143r (\attr Mārkaṇḍeya)
\begin{variants}
    kapotagṛdhrau kākolo~] kapoto gṛdhrakākolau YCM\sep
    līnaḥ~] tiṣṭhet YCM
\end{variants}
% \begin{versinnote}
% \tl{kapoto gṛdhrakākolau vāyaso vāpi mūrdhani/\\+}
% \tl{kravyādo vā khagas tiṣṭhet saṇmāsāyuḥpradarśakaḥ//\\!}
% \end{versinnote}
\end{testimonia}
%</ts8>

%<*cm8>
%\begin{philcomm}[Kj08]
%\end{philcomm}
%</cm8>

\subsection*{Kj.9}
%<*tr9>
\begin{translation}[Kj09]
[If] a man is struck by flocks of crows or a dust storm, or sees his shadow deformed, he lives for four months. 
\end{translation}
%</tr9>

%<*sc9>
\begin{sources}[Kj09]
\emph{Mārkaṇḍeyapurāṇa} 40.9, cf.~\emph{Śārṅgadharapaddhati} 4574
\begin{variants}
    śreṇībhiḥ~] paṅktībhiḥ MP, ŚDP\sep
    sva ŚDP~] svāṃ MP\sep
    vānyathā~] anyathā MP, cānyathā ŚDP\sep
    caturmāsān ŚDP~] catuḥpañca MP
\end{variants}

% \begin{versinnote}
% \tl{hanyate kākapaṅktībhiḥ pāṃśuvarṣeṇa vā naraḥ/\\+}
% \tl{svāṃ chāyām anyathā dṛṣṭvā catuḥpañca sa jīvati//\\!}
% \end{versinnote}

% Cf.~\emph{Śārṅgadharapaddhati} 4574
% \begin{versinnote}
% \tl{hanyate kākapaṅktibhiḥ pāṃsuvarṣeṇa vā naraḥ/\\+}
% \tl{svacchāyāṃ cānyathā dṛṣṭvā caturmāsān sa jīvati//\\!}
% \end{versinnote}
\end{sources}
%</sc9>

%<*ts9>
\begin{testimonia}[Kj09]
\emph{Yogacintāmaṇi} f.\,143r (\attr Mārkaṇḍeya)
\begin{variants}
    śreṇībhiḥ~] paṅktībhiḥ YCM\sep
    svacchāyāṃ vānyathā dṛṣṭvā caturmāsān sa jīvati~] śuṣyec ca vai yasya marma snānād vāmād adhas\-tanam YCM
\end{variants}
% \begin{versinnote}
% \tl{hanyate kākapaṅktībhiḥ pāṃsuvarṣeṇa vā punaḥ/\\+}
% \tl{śuṣyec ca vai yasya marmasnānādvāmādadhastanam//\\!}
% \end{versinnote}
\end{testimonia}
%</ts9>

%<*cm9>
%\begin{philcomm}[Kj09]
%Not sure how to take anyathā. Does it mean that he sees his shadow in a distorted form (i.e.~other than what it normally is)? Do we need to translate the vā in pāda 3?
% Jürgen: Other possibility would be to assume three conditions and make sense of vā in Pāda c. Pāda c would then mean: ``or seeing his shadows deformed''.
%JM: after seeing his shadow deformed ?
%\end{philcomm}
%</cm9>

\subsection*{Kj.10}
%<*tr10>
\begin{translation}[Kj10]
When he sees lightning in a southern direction in a cloudless [sky], or a rainbow in water, he has two or three months of life [left].
\end{translation}
%</tr10>
% MD: Is payas here milk or water? YCM has udaka.
% JM: water makes better sense; original probably MP's rātrau
%<*sc10>
\begin{sources}[Kj10]
\emph{Mārkaṇḍeyapurāṇa} 40.10, cf.~\emph{Śārṅgadharapaddhati} 4575
\begin{variants}
    payasīndradhanur vāpi ŚDP~] rātrāv indradhanuś cāpi MP\sep
    dvitrimāsikam~] hi trimāsikam MP, tridvimāsikam ŚDP
\end{variants}
% \begin{versinnote}
% \tl{anabhre vidyutaṃ dṛṣṭvā dakṣiṇāṃ diśam āśritām/\\+}
% \tl{rātrāv indradhanuś cāpi jīvitaṃ hi trimāsikam//\\!}
% \end{versinnote}

% Cf.~\emph{Śārṅgadharapaddhati} 4575
% \begin{versinnote}
% \tl{anabhre vidyutaṃ dṛṣṭvā dakṣiṇāṃ diśam āśritām/\\+}
% \tl{payasīndredhanur vāpi jīvitaṃ tridvimāsikam//\\!}
% \end{versinnote}
\end{sources}
%</sc10>

%<*ts10>
\begin{testimonia}[Kj10]
\emph{Yogacintāmaṇi} f.\,143r (\attr Mārkaṇḍeya)
\begin{variants}
    payasīndra~] udakendra YCM
\end{variants}
% \begin{versinnote}
% \tl{anabhre vidyutaṃ dṛṣṭvā dakṣiṇāṃ diśam āsthitām/\\+}
% \tl{udakendradhanur vāpi jīvitaṃ dvitrimāsikam//\\!}
% \end{versinnote}
\end{testimonia}
%</ts10>

%<*cm10>
\begin{philcomm}[Kj10]
%The \emph{Mārkaṇḍeyapurāṇa} preserves the more plausible idea of seeing a rainbow at night rather than in water.%?? JM: why more plausible? water seems more plausible to me. JB: I mean it is more plausible as the original reading because these signs are supposed to be unlikely to occur in reality (i.e., lightning in a cloudless sky). I've rewritten the note as follows: 
The \emph{Mārkaṇḍeyapurāṇa} appears to preserve the original idea of seeing a rainbow at night, which is as extraordinary as the first sign (i.e., seeing lightning in a cloudless sky). %

% Should we adopt anabhre or anabhrāṃ?
% Jü: I can see not good arguments, sorry.
% MD: I would keep the locative. 
% Mbh 02,071.025c anabhre vidyutaś cāsan bhūmiś ca samakampata
% Mrtyuvancanopadesa 1.64: ameghe vidyutaṃ paśyet sphurantīṃ dakṣiṇāśritām

\end{philcomm}
%</cm10>

\subsection*{Kj.11}
%<*tr11>
\begin{translation}[Kj11]
He who sees another's headless body [reflected] in ghee, oil, a mirror or water does not live beyond a month. 
\end{translation}
%</tr11>
% MD: in Pāda c, better to emend ti/ca to vā? ca is a conjecture by the editor of J6(δω).

%<*sc11>
\begin{sources}[Kj11]
\emph{Mārkaṇḍeyapurāṇa} 40.11, cf.~\emph{Śārṅgadharapaddhati} 4576
\begin{variants}
    vānātmanas MP~] vāpy ātmanas ŚDP\sep
    aśiraskāṃ ca~] aśiraskāṃ vā MP, aśiraḥskandhāṃ ŚDP
\end{variants}
% \begin{versinnote}
% \tl{ghṛte taile tathādarśe toye vānātmanas tanum/\\+}
% \tl{yaḥ paśyed aśiraskāṃ vā māsād ūrdhvaṃ na jīvati//\\!}
% \end{versinnote}

% Cf.~\emph{Śārṅgadharapaddhati} 4576
% \begin{versinnote}
% \tl{ghṛte taile tathādarśe toye vāpy ātmanas tanum/\\+}
% \tl{yaḥ paśyed aśiraḥskandhāṃ māsād ūrdhvaṃ na jīvati//\\!}
% \end{versinnote}
\end{sources}
%</sc11>

%<*ts11>
\begin{testimonia}[Kj11]
\emph{Yogacintāmaṇi} f.\,143r (\attr Mārkaṇḍeya)
\begin{variants}
    vānātmanas~] vāpy ātmanas YCM\sep
    ca~] sa YCM
\end{variants}
% \begin{versinnote}
% \tl{ghṛte taile tathādarśe toye vāpy ātmanas tanum/\\+}
% \tl{yaḥ paśyed aśiraskāṃ sa māsād ūrdhvaṃ na jīvati//\\!}
% \end{versinnote}
\end{testimonia}
%</ts11>

%<*cm11>
%\begin{philcomm}[Kj11]
%\end{philcomm}
%</cm11>

\subsection*{Kj.12}
%<*tr12>
\begin{translation}[Kj12]
He should know, O king, that a yogi whose body odour is similar to that of bones or a corpse has half a month to live. 
\end{translation}
%</tr12>
    
%<*sc12>
\begin{sources}[Kj12]
\emph{Mārkaṇḍeyapurāṇa} 40.12, cf.~\emph{Śārṅgadharapaddhati} 4577
\begin{variants}
    yasyāsthisadṛśo~] yasya bastasamo MP ŚDP\sep
    tasyārdhamāsikaṃ MP~] tasya māsārdhakaṃ  ŚDP\sep
    yogino nṛpa MP~] yoginaḥ kila ŚDP
\end{variants}

% \begin{versinnote}
% \tl{yasya bastasamo gandhas gātre śavasamo 'pi vā/\\+}
% \tl{tasyārdhamāsikaṃ jñeyaṃ yogino nṛpa jīvitam//\\!}
% \end{versinnote}

% Cf.~\emph{Śārṅgadharapaddhati} 4577
% \begin{versinnote}
% \tl{yasya bastasamo gandho gātre śavasamopi vā/\\+}
% \tl{tasya māsārdhakaṃ jñeyaṃ yoginaḥ kila jīvitam//\\!}
% \end{versinnote}
\end{sources}
%</sc12>

%<*ts12>
% \begin{testimonia}[Kj12]

% \end{testimonia}
%</ts12>

%<*cm12>
%\begin{philcomm}[Kj12]
    % Trans: 'body odour' for gandho gātre (smell in the body) seems more idiomatic in English
    % MD: jīvati in pāda d should be emended to jīvitam.
    % Yes of course. done. thank you!
%\end{philcomm}
%</cm12>

\subsection*{Kj.13}
%<*tr13>
\begin{translation}[Kj13]
He whose chest and legs are dry straight after bathing and who is dehydrated when drinking water lives ten days. 
\end{translation}
%</tr13>

%<*sc13>
\begin{sources}[Kj13]
\emph{Mārkaṇḍeyapurāṇa} 40.13, cf.~\emph{Śārṅgadharapaddhati} 4578
\begin{variants}
    hṛtpādam MP~] hṛttoyam ŚDP\sep
    avaśuṣyati ŚDP~] avaśuṣyate MP
\end{variants}

% \begin{versinnote}
% \tl{yasya vai snātamātrasya hṛtpādam avaśuṣyate/\\+}
% \tl{pibataś ca jalaṃ śoṣo daśāhaṃ so 'pi jīvati//\\!}
% \end{versinnote}

Cf.~\emph{Dharmaputrikā} 40.13 
\begin{versinnote}
\tl{sambhidya māruto yasya marmasthānāc ca bhraśyate/\\+}
\tl{jyotiś caiva na paśyed yo dinam ekaṃ sa jīvati//\\!}
\end{versinnote}

% Cf.~\emph{Śārṅgadharapaddhati} 4578
% \begin{versinnote}
% \tl{yasya vai snātamātrasya hṛttoyam avaśuṣyati/\\+}
% \tl{pibataś ca jalaṃ śoṣo daśāhaṃ sopi jīvati//\\!}
% \end{versinnote}

\end{sources}
%</sc13>

%<*ts13>
% \begin{testimonia}[Kj13]

% \end{testimonia}
%</ts13>

%<*cm13>
%\begin{philcomm}[Kj13]
%Jü: "is dry" in the sense of "dehydrated"? JM: yes, change I think.
%\end{philcomm}
%</cm13>

\subsection*{Kj.14}
%<*tr14>
\begin{translation}[Kj14]
For he whose breath is agitated and cuts the vital points, [and] who does not like the touch of water, death is near.
%Jü: is not excited, i.e. does not like?
%NJL: What shall cutting (kṛntati) the vital points exactly mean? In the sense of destroying? sambhidya māruto yasya marmasthānāc ca bhraśyate ("For the one whose breath shatters and escapes from the vital points ...") of Dharmaputrikā seems so much better to me. In the light of Dharmaputrikā perhaps we can translate our transmission of the verse as "For one whose breath is disturbed [and] tears asunder at the vital points." and take the acc pl as an "accusative object of the target".
% JM: "short" for sambhinnaḥ? "agitated" doesn't seem right.
% ? Apte: sambhinna  (2) Shattered, shaken, agitated.
\end{translation}
%</tr14>

%<*sc14>
\begin{sources}[Kj14]
\emph{Mārkaṇḍeyapurāṇa} 40.14, cf. \emph{Liṅgapurāṇa} 1.91.14
\begin{variants}
    na hṛṣyaty ambusparśāt~] hṛṣyate nāmbusparśāt MP, adbhiḥ spṛṣṭo na hṛṣyeta LP
\end{variants}

% \begin{versinnote}
% \tl{sambhinno māruto yasya marmasthānāni kṛntati/\\+}
% \tl{hṛṣyate nāmbusparśāt tasya mṛtyur upasthitaḥ//\\!}
% \end{versinnote}

% Cf.~\emph{Liṅgapurāṇa} 1.91.14
% \begin{versinnote}
% \tl{saṃbhinno māruto yasya marmasthānāni kṛntati/\\+}
% \tl{adbhiḥ spṛṣṭo na hṛṣyeta tasya mṛtyurupasthitaḥ//\\!} 
% \end{versinnote}

Cf.~\emph{Dharmaputrikā} 213
\begin{versinnote}
\tl{sambhidya māruto yasya marmasthānāc ca bhraśyate/\\+}
\tl{jyotiś caiva na paśyed yo dinam ekaṃ sa jīvati//\\!}
\end{versinnote}
\end{sources}
%</sc14>

%<*ts14>
%\begin{testimonia}[Kj14]
% \begin{versinnote}
% \end{versinnote}
%\end{testimonia}
%</ts14>

%<*cm14>
\begin{philcomm}[Kj14]
It seems more likely that \emph{mārutaḥ} here refers to the breath rather than external wind, since it is difficult to conceive how wind could cut the vital points, which are located inside the body. In his commentary on \emph{Chāndogyopaniṣat} 6.8.6, Śaṅkara describes an internal process in which the breath cuts vital points as a person dies, with the breath merging into heat, which in turn merges into the highest deity (\emph{prāṇaś ca tadordhvocchvāsī svātmany upasaṃhṛta\-bāhya\-karaṇaḥ saṃvarga\-vidyāyāṃ darśanād dhastapādādīn vikṣipan marma\-sthānāni nikṛntan nana ivotsṛjan krameṇopa\-saṃhṛtas tejasi sampadyate} [...]). Similarly, the first half of a parallel verse in the \emph{Dharmaputrikā} (213) appears to be describing an internal process in which the breath is disturbed and then escapes from the vital points. %Nonetheless, whether one interprets \emph{mārutaḥ} as breath or wind, it remains difficult to understand how this relates to the person not delighting in the touch of water. 
% check note

%I have emended to °saṃsparśāt (MāPur) from °saṃsparśe (J6) and °saṃspaśa (V3, E4ac) and saṃsparśa (E4 pc). I think the locative unlikely, judging from the style of these verses?
%MD: māruta is "wind" here.
    % I think it is a synonym for prāṇa (i.e.~an internal process is being described here)
% MD Pāda c: better to adopt hṛṣyaty (πω,ηω), or emend to hṛṣyanty, if marmasthānāni is the subject.
    % I think the subject is the yogi (i.e.~supply yaḥ) otherwise ambusparśāt doesnt make sense. But I've adopted hṛṣyaty (now I see its a class 4 verb) as it's better attested and closer to the MāPur reading. But the 6-ch HP has harṣaty. So, it is still possible.
% Jü: I do not see a necessity for the emendation saMsparze    
% NJL: The locative is often used in the third pāda  
% JB: okay, I've restored the locative.
% MD: The other ηω-mss have the ablative.
% JB :change to śparśāt
\end{philcomm}
%</cm14>

\subsection*{Kj.15}
%<*tr15>
\begin{translation}[Kj15]
The time of death is near for him also who in a dream travels south while singing in a chariot [drawn by] a bear and monkey.%JM: translate api here and in next verse?
\end{translation}
%</tr15>
% Jü: Or he is, despite Vāyupurāṇa 19.13, sitting on an animal? Lesser likely alternative I guess. 

%<*sc15>
\begin{sources}[Kj15]
\emph{Mārkaṇḍeyapurāṇa} 40.15
\begin{variants}
    yugyastho~] yānastho MP\sep
    mṛtyukāla upasthitaḥ~] na mṛtyuḥ kālam icchati MP
\end{variants}

% \begin{versinnote}
% \tl{ṛkṣavānarayānastho gāyan yo dakṣiṇāṃ diśam/\\+}
% \tl{svapne prayāti tasyāpi na mṛtyuḥ kālam icchati//\\!}
% \end{versinnote}

Cf.~\emph{Skandapurāṇa} 1.2.55.76
\begin{versinnote}
\tl{ṛkṣavānarayugyastho gāyan yo dakṣiṇāṃ diśam/\\+}  % Jü: gāyan yo
\tl{yāti majjed [d]adhau paṅke gomaye vā na jīvati//\\!}
\end{versinnote}

Cf.~\emph{Vāyupurāṇa} 19.13
\begin{versinnote}
\tl{ṛkṣavānarayuktena rathenāśāṃ tu dakṣiṇām/\\+}  
\tl{gāyan atha vrajet svapne vidyān mṛtyur upasthitaḥ//\\!}
\end{versinnote}
\end{sources}
%</sc15>

%<*ts15>
%\begin{testimonia}[Kj15]
%\begin{versinnote}
%\end{versinnote}
%\end{testimonia}
%</ts15>

%<*cm15>
\begin{philcomm}[Kj15]
%MD: better to emend yugma to yugya (= yāna).
% I like it! Although °yugma° is well attested. It's in all three witnesses and both the 6 and 10 chapter HPs. And it can be construed (but see below)

The idea behind this verse is more clearly expressed in the \emph{Vāyupurāṇa} (19.13), where a chariot is drawn by a bear and monkey.%?? is this necessary? If so rephrase ("also drawn"?). JB: yugya is an emendation based on the idea in the VP. I've read api with tasya.
\end{philcomm}
%</cm15>

\subsection*{Kj.16}
%<*tr16>
\begin{translation}[Kj16]
[If] in a dream a woman wearing red and black clothes, and singing and laughing, leads him to a southern region, he too will not live.  % Jü: svapne 'pi not translated
\end{translation}
%</tr16>

%<*sc16>
\begin{sources}[Kj16]
\emph{Mārkaṇḍeyapurāṇa} 40.16, cf.~\emph{Śārṅgadharapaddhati} 4581
\begin{variants}
    gāyantī ca hasanty api~] gāyantī hasatī ca yam MP, gītahāsyaparā ca yam ŚDP
\end{variants}

% \begin{versinnote}
% \tl{raktakṛṣṇāmbaradharā gāyantī hasatī ca yam/\\+}
% \tl{dakṣiṇāśāṃ nayen nārī svapne so 'pi na jīvati//\\!}
% \end{versinnote}

% Cf.~\emph{Śārṅgadharapaddhati} 4581
% \begin{versinnote}
% \tl{raktakṛṣṇāmbaradharā gītahāsyaparā ca yam/\\+}
% \tl{dakṣiṇāśāṃ nayen nārī svapne sopi na jīvati//\\!}
% \end{versinnote}
\end{sources}
%</sc16>

%<*ts16>
% \begin{testimonia}[Kj16]

% \end{testimonia}
%</ts16>

%<*cm16>
%\begin{philcomm}[Kj16]
%V3, J6 and E4 have hasanti instead of hasantī, which is unmetrical but grammatically correct. I have adopted the metrical and grammatically correct reading of N26 (i.e.~hasanty api). The testimonia have similar problems in pāda b. For example, MāPur has hasatī for the metre, YCM (ms. N) has an unmetrical reading (i.e.~hasaṃtī ca yam). SkaPur (gāyantīha satī ca yam) and ŚāPa (gītahāsyaparā ca yam) have different readings. 
%\end{philcomm}
%</cm16>
%MD: More precisely, Pi-Omega omits 16d-17a due to an eye-skip caused by svapne, and then omits 17cd. Its reading of 16d given in the apparatus is actually that of 17b.
% JB I think we need to discuss this, as it likely affects a footnote I wrote on 17. V3 has nṛtya*tatparā* deleted, which suggests that 17 was in its exemplar. However, J6 omits v17, suggesting that its exemplar descends from the V3 
%branch. 
%NJL: the south (dakṣiṇa) is the direction of Yama, but why does the women wear red and black? Does anyone know?
\subsection*{Kj.17}
%<*tr17>
\begin{translation}[Kj17]
If [a man] sees in a dream a lone naked Jain ascetic laughing, dancing, and leaping about, he knows death is near.  
% Jü: In German I would translate "tatpara" in such a context just as "dancing", for "engrossed in dancing" and the like lays too much emphasis on a simple thing. 
%JM: delete "thus"?
%  does evaṃ make sense here? We could emend to the Śārṅgadharapaddhati reading ekaṃ. In fact, I've emended to ekaṃ
\end{translation}
%</tr17>


%<*sc17>
\begin{sources}[Kj17]
\emph{Mārkaṇḍeyapurāṇa}  40.17, cf.~\emph{Śārṅgadharapaddhati} 4582
\begin{variants}
    hasantaṃ nṛtyatatparam ŚDP~] hasamānaṃ mahābalam MP\sep
    ekaṃ ŚDP~] evaṃ MP\sep 
    saṃvīkṣya valgantaṃ MP~] vilakṣaṃ vibhrāntaṃ ŚDP
\end{variants}
% \begin{versinnote}
% \tl{nagnaṃ kṣapaṇakaṃ svapne hasamānaṃ mahābalam/\\+}
% \tl{evaṃ saṃvīkṣya valgantaṃ vidyān mṛtyum upasthitam//\\!}
% \end{versinnote}

% Cf.~\emph{Śārṅgadharapaddhati} 4582
% \begin{versinnote}
% \tl{nagnaṃ kṣapaṇakaṃ svapne hasantaṃ nṛtyatatparam/\\+}
% \tl{ekaṃ vilakṣaṃ vibhrāntaṃ vidyān mṛtyum upasthitam//\\!}
% \end{versinnote}

Cf.~\emph{Skandapurāṇa} 1.2.55.75cd–76ab
\begin{versinnote}
\tl{nagnaṃ kṣapaṇakaṃ svapne hasamānaṃ pradṛśya ca//\\+}
\tl{enaṃ ca vīkṣya valgantaṃ taṃ vidyān mṛtyum āgatam/\\!}
\end{versinnote}
\end{sources}
%</sc17>

%<*ts17>
% \begin{testimonia}[Kj17]

% \end{testimonia}
%</ts17>

%<*cm17>
\begin{philcomm}[Kj17]
Other printed versions of the \emph{Mārkaṇḍeyapurāṇa} read \emph{ekaṃ saṃvīkṣya}, e.g.,~\emph{Mār\-kaṇḍe\-ya\-purāṇa} 43.17 (ed.~Vihārilāl Sarkar, Kalikātā-rājadhānyām, 1890)%??ref looks odd.
%MD: better to emend valāṃtaṃ to valgantaṃ.
% Yes, absolutely. I translated as such but forgot to change the text!
\end{philcomm}
%</cm17>

\subsection*{Kj.18}
%<*tr18>
\begin{translation}[Kj18]
Then he who sees oneself in a dream immersed in an ocean of mud from the soles [of the feet] up to the head dies immediately. %?? MD: ā mastalatalād "up to the top of the head"?
\end{translation}
%</tr18>

%<*sc18>
\begin{sources}[Kj18]
\emph{Mārkaṇḍeyapurāṇa} 40.18
\begin{variants}
    yaḥ sadyo mriyate ca saḥ~] sa sadyar mriyate naraḥ MP
\end{variants}
% \begin{versinnote}
% \tl{āmastakatalād yas tu nimagnaṃ paṅkasāgare/\\+}
% \tl{svapne paśyaty athātmānaṃ sa sadyar mriyate naraḥ//\\!} % sadyo 
% \end{versinnote}
\end{sources}
%</sc18>

%<*ts18>
%\begin{testimonia}[Kj18]
% Not in \emph{Śārṅgadharapaddhati} 
%\begin{versinnote}
%\end{versinnote}
%\end{testimonia}
%</ts18>

%<*cm18>
%\begin{philcomm}[Kj18]
%The \emph{yaḥ} in the fourth \emph{pāda} is redundant but perhaps inserted for metrical reasons. 

%Is it possible to make sense of \emph{yathātmānaṃ}? I have emended to \emph{athātmānaṃ}, in light of the \emph{Mārkaṇḍeyapurāṇa}'s reading.
% Jü: the ya in the transmitted reading could be a remnant from the Markand reading pazyatyathAtmAnam 
% MD: I prefer the reading paśyaty athātmānam too. There is no good reason to keep the present participle. In case of paśyan, we should read paśyann because of the Sandhi.
%JM: paśyanyathā° is easy mistake for paśyatyathā° in nāgarī. I've changed tr.
% MD: adopt nimagnaṃ?
% yes, better. done!
%\end{philcomm}
%</cm18>

\subsection*{Kj.19}
%<*tr19>
\begin{translation}[Kj19]
If for ten days he dreams of hair, charcoals, ash, snakes and a river without water, death [occurs] on the eleventh day.
\end{translation}
%</tr19>

%<*sc19>
\begin{sources}[Kj19]
\emph{Mārkaṇḍeyapurāṇa} 40.19
\begin{variants}
    daśāhaṃ~] daśāhāt MP
\end{variants}
% \begin{versinnote}
% \tl{keśāṅgārān tathā bhasmabhujaṅgān nirjalāṃ nadīm/\\+}
% \tl{dṛṣṭvā svapne daśāhāt tu mṛtyur ekādaśe dine//\\!}
% \end{versinnote}

Cf.~\emph{Skandapurāṇa}  1.2.55.77cd–78ab
\begin{versinnote}
\tl{keśāṅgārais tathā bhasmabhujaṅgair nirjalāṃ nadīm//\\+}
\tl{eṣām anyatamaiḥ pūrṇāṃ dṛṣṭvā svapne na jīvati/\\!}
\end{versinnote}

Cf.~\emph{Liṅgapurāṇa} 1.91.19.
\begin{versinnote}
\tl{bhasmāṅgārāṃś ca keśāṃś ca nadīṃ śuṣkāṃ bhujaṅgamān/\\+}
\tl{paśyed yo daśarātraṃ tu na sa jīvati tādṛśaḥ//\\!}
\end{versinnote}
\end{sources}
%</sc19>

%<*ts19>
% \begin{testimonia}[Kj19]
% % not in \emph{Śārṅgadharapaddhati} 

% \end{testimonia}
%</ts19>

%<*cm19>
\begin{philcomm}[Kj19]
The original version of the first line was probably that of the \emph{Skandapurāṇa}, where the verse conveys the idea of a waterless river filled with hair, charcoal, ash or snakes. However, it seems that at some point this idea was lost, and each of these elements came to be treated separately, as in \emph{Liṅgapurāṇa} 1.91.19. 
% check note and translation

%Although there are remnants of this version in the \emph{Haṭhapradīpikā}'s transmission (e.g.~V3 has \emph{keśāṅgārās}, and the \emph{nnirjalāṃ} of J6 might point to \emph{rnnirjalāṃ}), the \emph{Haṭhapradīpikā} manuscripts have something closer to what we see in the \emph{Mārkaṇḍeyapurāṇa}.  
% Jü: Or "hair made of charcoal" (not really better)
% MD: not better to take bhasma and bhujaṅga separately?
% But the compound is plural, suggesting it's not a dvandva (but this is not definite... it could be taken as 'ashes and snakes', but what would be the special significance of seeing ashes and snakes?
% MD: Perhaps keśāṅgāra is a dvandva. Then, we have "hairs, charcoals, ashes, snakes and a waterless river". It may be implied that the first four things are in the river in place of water.
% Fire torch made of human hair ->  kulāṅgāra 1) m. eine Brandfackel des Geschlechts, so v.a. ein die Familie zu Grunde richtendes Glied derselben PRASANNAR. 77,14. 2) f. ī dass. von einem Frauenzimmer gesagt ḥariv. 9940.― pw Vol. 2, S. 82 (Scan anzeigen ) 
%JM: hot coals for charcoals?
% MD: according to the parallel in Liṅgapurāṇa, they are all separate things:
% bhasmāṅgārāṃś ca keśāṃś ca nadīṃ śuṣkāṃ bhujaṅgamān/
% paśyedyo daśarātraṃ tu na sa jīvati tādṛśaḥ// LiP_1,91.19//
\end{philcomm}
%</cm19>


\subsection*{Kj.20}
%<*tr20>
\begin{translation}[Kj20]%?? space after sadyo in ed? JB: I read it as a compound but translated it loosely (i.e, lit: a man becomes one who has a sudden death).
If in a dream a man is struck by stones [thrown] by terrifying, monstrous and malevolent men with raised weapons, he dies suddenly. 
\end{translation}
%</tr20>

%<*sc20>
\begin{sources}[Kj20]
\emph{Mārkaṇḍeyapurāṇa} 40.20, cf.~\emph{Śārṅgadharapaddhati} 4585
\begin{variants}
    vikaṭai rūkṣaiḥ puruṣair~] vikaṭaiḥ kṛṣṇaiḥ puruṣair MP, puruṣaiḥ kṛṣṇair vikaṭair ŚDP\sep
    mṛtyur bhaven naraḥ ŚDP~] mṛtyuṃ labhen naraḥ MP
\end{variants}

% \begin{versinnote}
% \tl{karālair vikaṭaiḥ kṛṣṇaiḥ puruṣair udyatāyudhaiḥ/\\+}
% \tl{pāṣāṇais tāḍitaḥ svapne sadyomṛtyuṃ labhen naraḥ//\\!}
% \end{versinnote}

% Cf.~\emph{Śārṅgadharapaddhati} 4585
% \begin{versinnote}
% \tl{karālaiḥ puruṣaiḥ kṛṣṇair vikaṭair udyatāyudhaiḥ/\\+}
% \tl{pāṣāṇais tāḍitaḥ svapne sadyomṛtyur bhaven naraḥ//\\!}
% \end{versinnote}
\end{sources}
%</sc20>

%<*ts20>
% \begin{testimonia}[Kj20]

% \end{testimonia}
%</ts20>

%<*cm20>
\begin{philcomm}[Kj20]
The syntax of the verse transmitted by the \emph{Haṭhapradīpikā} manuscripts is faulty, since \emph{mṛtyuḥ} appears as the subject, whereas the subject should be a man (\emph{naraḥ}), as found in the \emph{Mārkaṇḍeyapurāṇa} (40.20) and \emph{Śārṅgadharapaddhati} (4585). It makes little sense for death to be struck by stones etc., so the reading of the \emph{Śārṅgadharapaddhati}'s final \emph{pāda} has been adopted.

% N9 reveals an attempt to read a locative absolute (taḍite) but it doesn't seem to work. Should we do anything about this? I can't see an easy fix. % Jü Assume a (wrong) nṛṇaḥ, which is syntactically "correct". According to the motto: better wild conjectures than no fun at all:)
% So maybe keep the verse and translation as they are and write a note?
% before present day yogis believe in slightly wrong omens, I would lean towards restoring the verse according to Mārkaṇḍeyapurāṇa
% I have emended to the Śārṅgadharapaddhati's reading (which is the closest to the transmitted reading) and written a note.
\end{philcomm}
%</cm20>

\subsection*{Kj.21}
%<*tr21>
\begin{translation}[Kj21]
If at sunrise a howling jackal goes in front of, past or around someone, his sudden death is near. 
\end{translation}
%</tr21>

%<*sc21>
\begin{sources}[Kj21]
\emph{Mārkaṇḍeyapurāṇa} 40.21, cf.~\emph{Skandapurāṇa} 1.2.55.79cd–80ab
\begin{variants}
    sadyomṛtyur upasthitaḥ~] sa sadyomṛtyum ṛcchati MP SP
\end{variants}
% \begin{versinnote}
% \tl{sūryodaye yasya śivā krośantī yāti saṃmukham/\\+}
% \tl{viparītaṃ parītaṃ vā sa sadyomṛtyum ṛcchati//\\!}
% \end{versinnote}

% Cf.~\emph{Skandapurāṇa} 1.2.55.79cd–80ab
% \begin{versinnote}
% \tl{sūryodaye yasya śivā krośantī yāti sammukham//\\+}
% \tl{viparītaṃ parītaṃ vā sa sadyomṛtyum ṛcchati/\\!}
% \end{versinnote}
\end{sources}
%</sc21>

%<*ts21>
% \begin{testimonia}[Kj21]
% % not in \emph{Śārṅgadharapaddhati} 4568

% \end{testimonia}
%</ts21>

%<*cm21>
%\begin{philcomm}[Kj21]
% JM: viparītam — "behind"?
% discuss viparītam? I'm not familar with its meaning of 'behind' Write note.
%\end{philcomm}
%</cm21>

\subsection*{Kj.22}
%<*tr22>
\begin{translation}[Kj22]
If [a man's] stomach is afflicted by hunger just after eating and he grinds his teeth, his life is undoubtedly approaching the end.
% Jü: hrdaya is attested as stomach, perhaps mention the possibilty.
% JM: stomach seems better; grinds one's teeth?
% I don't see the reference to hṛdaya as stomach in MW or Apte. How common is this meaning?
%%% 
% NJL: NWS has <Śā, Ges>Magen. Meyer 1926, S. 340, Z. 12.<Meyer 1926: 979> Magen=Stomach in German, Meyer's book is a translation and investigation of the the Arthaśāstra. The passage where he thinks heart must mean stomach goes like this: 
% 4.7.82
% viṣahatasya bhojanaśeṣaṃ vayobhiḥ parikṣet/ 
% hṛdayādd hṛtyāgnau prakṣiptaṃ ciṭaciṭāyād indradhanurvarṇaṃ vā viṣayuktaṃ vidhyāt// 
% dagdhasya hṛdayam adagdhaṃ dṛṣṭvā vā tasya paricārakajanaṃ vā daṇḍhapāruṣyād atimārgeta// 
% 
% He translates: 
% Bei dem durch Gift Getöteten soll man einen Rest dessen, was er gegessen hat mit Hilfe von Vögeln untersuchen. Wenn man etwas aus seinem Magen herausnimmt und es ins Feuer wirft und es dann knistert und % knittert oder regenbogenfarbig wird, dann wisse man, daß es Gift enthält, oder wenn man sieht, dass bei der Verbrennung des Toten sein Magen nicht mitverbrennt. 
% =
% In the case of one who has been killed by poison, one should examine the remainder of what he has eaten with the help of birds. If one takes something from his stomach and throws it into the fire, and it crackles and rustles, or becomes rainbow-colored, then one should know that it contains poison — or if one sees that, during the cremation of the dead, his stomach does not burn along with the rest.

\end{translation}
%</tr22>

%<*sc22>
\begin{sources}[Kj22]
\emph{Mārkaṇḍeyapurāṇa} 40.22
\begin{variants}
    pīḍyate~] bādhyate MP\sep
    asaṃśayaḥ~] na saṃśayaḥ MP
\end{variants}
% \begin{versinnote}
% \tl{yasya vai bhuktamātrasya hṛdayaṃ bādhyate kṣudhā/\\+}
% \tl{jāyate dantagharṣaś ca sa gatāyur na saṃśayaḥ//\\!}
% \end{versinnote}
\end{sources}
%</sc22>

%<*ts22>
\begin{testimonia}[Kj22]
% Not in the \emph{Śārṅgadharapaddhati} 
\emph{Haṭhatattvakaumudī} 56.2
\begin{variants}
    pīḍyate~] bādhate HTK\sep
    asaṃśayaḥ~] asaṃśayam HTK
\end{variants}
% \begin{versinnote}
% \tl{yasya vai bhuktamātrasya hṛdayaṃ bādhate kṣudhā/\\+}
% \tl{jāyate dantagharṣaś ca sa gatāyur asaṃśayam//\\!}
% \end{versinnote}
\end{testimonia}
%</ts22>

%yasya vā snātamātrasya hṛdayaṃ pīḍyate bhṛśam/
%jāyate dantaharṣaś ca taṃ gatāyuṣam ādiśet// LiP_1,91.22//

%<*cm22>
\begin{philcomm}[Kj22]
The meaning of \emph{hṛdaya} as stomach, which makes good sense here, is rare in this type of literature (where it usually means `heart' or `chest') but is attested, e.g.~at \emph{Arthaśāstra} 4.7.12–13. 
\end{philcomm}
%</cm22>

\subsection*{Kj.23}
%<*tr23>
\begin{translation}[Kj23]
He who in a dream cannot smell lamps and the like, by day or by night, and does not see himself [reflected] in someone else’s eyes, does not live. 
\end{translation}
%</tr23>

%<*sc23>
\begin{sources}[Kj23]
\emph{Mārkaṇḍeyapurāṇa} 40.23, cf.~\emph{Śārṅgadharapaddhati} 4586
\begin{variants}
    dīpādigandhaṃ no ŚDP~] dīpagandhaṃ na yo MP\sep
    svapne 'py ahni~] trasyaty ahni MP, paśyaty agniṃ ŚDP\sep
    na sa jīvati MP~] yaḥ mṛtyumān ŚDP
\end{variants}

% \begin{versinnote}
% \tl{dīpagandhaṃ na yo vetti trasyaty ahni tathā niśi/\\+}
% \tl{nātmānaṃ paranetrasthaṃ vīkṣate na sa jīvati//\\!}
% \end{versinnote}

% Cf.~\emph{Śārṅgadharapaddhati} 4586
% \begin{versinnote}
% \tl{dīpādigandhaṃ no vetti paśyaty agniṃ tathā niśi/\\+}
% \tl{nātmānaṃ paranetrasthaṃ vīkṣate yaḥ mṛtyumān//\\!}
% \end{versinnote}
\end{sources}
%</sc23>

%<*ts23>
% \begin{testimonia}[Kj23]

% \end{testimonia}
%</ts23>

%<*cm23>
%\begin{philcomm}[Kj23]
%JM: I removed "even" from "even in a dream", I think api is pretty colourless here.
%\end{philcomm}
%</cm23>

\subsection*{Kj.24}
%<*tr24>
\begin{translation}[Kj24]
On seeing a rainbow at midnight and a cluster of planets during the day, a prudent man should consider his life to be finished.
%NJL: I think we should better translate "planetary group" or "group of planets". I think the term is unusual or even incorrect in astronomy. The term cluster is used almost exclusively in English for: star clusters and galaxy clusters. 
% A search in google books indicates that "cluster of planets" is commonly used.
%??JM: prudent person > yogi? ātmavān is synonym for yogī for Nārāyaṇakaṇṭha and prob elsewhere. Prudent seems a bit weak. JB: I don't find ātmavān as a synonym for yogi in second-millenium yoga texts but it occurs as an adjective (e.g. HTK ātmavān naraḥ). It also occurs in the BhG (2.45), where Śaṅkara glosses it ātmavān apramattaḥ ... careful, prudent. Is self-possessed?
\end{translation}
%</tr24>

%<*sc24>
\begin{sources}[Kj24]
\emph{Mārkaṇḍeyapurāṇa} 1.2.55.40.24, cf.~\emph{Skandapurāṇa} 81cd–82ab
\begin{variants}
    grahagaṇān MP~] vā grahaṇaṃ SP\sep
    saṃkṣīṇam ātmajīvitam MP~] sa kṣīṇam ātmajīvitam SP\sep
    ātmavān~] ātmavit MP, āptavān SP
\end{variants}
% \begin{versinnote}
% \tl{śakrāyudhaṃ cārdharātre divā grahagaṇān tathā/\\+}
% \tl{dṛṣṭvā manyeta saṃkṣīṇam ātmajīvitam ātmavit//\\!}
% \end{versinnote}

% Cf.~\emph{Skandapurāṇa} 81cd–82ab
% \begin{versinnote}
% \tl{śakrāyudhaṃ cārdharātre divā vā grahaṇaṃ tathā// 81//\\+}
% \tl{dṛṣṭvā manyeta sa kṣīṇamātmajīvitamāptavān//\\!}
% \end{versinnote}
\end{sources}
%</sc24>

%<*ts24>
% \begin{testimonia}[Kj24]
% % not in \emph{Śārṅgadharapaddhati}

% \end{testimonia}
%</ts24>

%<*cm24>
%\begin{philcomm}[Kj24]
%\end{philcomm}
%</cm24>

\subsection*{Kj.25}
%<*tr25>
\begin{translation}[Kj25]
Life is over for him whose nose has become crooked, ears are drooping or lifting, and left eye runs.
\end{translation}
%</tr25>

%<*sc25>
\begin{sources}[Kj25]
\emph{Mārkaṇḍeyapurāṇa} 40.25, cf.~\emph{Śārṅgadharapaddhati} 4589
% \begin{versinnote}
% \tl{nāsikā vakratām eti karṇayor namanonnatī/\\+}
% \tl{netraṃ ca vāmaṃ sravati yasya tasyāyur udgatam//\\!}
% \end{versinnote}

% Cf.~\emph{Śārṅgadharapaddhati} 4589
% \begin{versinnote}
% \tl{nāsikā vakratām eti karṇayor namanonnatī/\\+}
% \tl{netraṃ ca vāmaṃ sravati yasya tasyāyur udgatam//\\!}
% \end{versinnote}
\end{sources}
%</sc25>

%<*ts25>
% \begin{testimonia}[Kj25]

% \end{testimonia}
%</ts25>

%<*cm25>
%\begin{philcomm}[Kj25]
%I can't see a way to understand namanonnatā so have emended to namanonnatī (dual f. = Śārṅgadharapaddhati). Thoughts? 
% Jü: not really, since I fail to understand the compound, could also be "raised through bending". A nominal phrase with the transmitted reading would still be possible: "for the ears there is namanonnatA, ..."
% discuss: unnatā is not attested in the dictionary.
%\end{philcomm}
%</cm25>

\subsection*{Kj.26}
%<*tr26>
\begin{translation}[Kj26]
When the face becomes reddish and the tongue is black, the wise man knows that his death is at hand.
\end{translation}
%</tr26>

%<*sc26>
\begin{sources}[Kj26]
\emph{Mārkaṇḍeyapurāṇa} 40.26
\begin{variants}
    vāpy asitā ŚDP~] vā śyāmatāṃ MP\sep
    yadā MP~] bhavet ŚDP\sep
    ātmānam āgatam~] āsannam ātmanaḥ MP, āsannam āgatam ŚDP
\end{variants}
% \begin{versinnote}
% \tl{āraktatām eti mukhaṃ jihvā vā śyāmatāṃ yadā/\\+}
% \tl{tadā prājño vijānīyān mṛtyum āsannam ātmanaḥ//\\!}
% \end{versinnote}

% Cf.~\emph{Śārṅgadharapaddhati} 4588
% \begin{versinnote}
% \tl{āraktatāmeti mukhaṃ jihvā vāpy asitā bhavet/\\+}
% \tl{tadā prājño vijānīyān mṛtyum āsannam āgatam//\\!}
% \end{versinnote}
\end{sources}
%</sc26>

%<*ts26>
% \begin{testimonia}[Kj26]

% \end{testimonia}
%</ts26>

%<*cm26>
%\begin{philcomm}[Kj26]
%Should we consider cāsyāsitā in line with the meaning of the testimonia? Otherwise, we have the opposite idea (i.e.~the tongue becoming white instead of black).
% MD: or cāpy asitā?
%Jü: Asannam ātmanaḥ "near to oneself" seems idiomatic, but I cannot help thinking that "ātmānam āgatam" is strange, for formulated in this way one wants to answer that the self does not die. But just a note, not bad enough for emendation.
% According to my research, a person with a red face and a black tongue is definitely closer to death than a person with sunburn and tongue coating. That said, I think Mitsuyo's suggestion is great. I would go for it and ban the white variant into the critical apparatus. 
%\end{philcomm}
%</cm26>

\subsection*{Kj.27}
%<*tr27>
\begin{translation}[Kj27]
He whose tongue is black and rough, and whose face is lotus-shaped, or whose fleshy region of the cheek is red, is then at the end of his life.%tat = then?
\end{translation}
%</tr27>
% He whose tongue is black and rough, and whose face is lotus-shaped, or whose cheek or boil is red—that is the end of his life.

%<*sc27>
\begin{sources}[Kj27]
% not in the \emph{Mārkaṇḍeyapurāṇa} 
%\begin{versinnote}
%\end{versinnote}
Cf.~\emph{Dharmaputrikā} 212
\begin{versinnote}
\tl{yasya kṛṣṇā kharā jihvā padmavarṇaṃ mukhaṃ bhavet/\\+}
\tl{gaṇḍau tu pītakau raktau dīpagandhaṃ na jighrati//\\!}
\end{versinnote}

Cf.~\emph{Liṅgapurāṇa} 1.91.26
\begin{versinnote}
\tl{yasya kṛṣṇā kharā jihvā padmābhāsaṃ ca vai mukham/\\+}
\tl{gaṇḍe vā piṇḍikārakte tasya mṛtyur upasthitaḥ//\\!}
\end{versinnote}

Cf.~\emph{Kubjikāmatatantra} 23.41
\begin{versinnote}
\tl{yasya kṛṣṇā bhavej jihvā padmavarṇaṃ mukhaṃ bhavet/\\+}
\tl{gaṇḍapṛṣṭhau suraktābhau trirātraṃ ca sa jīvati//\\!}
\end{versinnote}
\end{sources}
%</sc27>

%<*ts27>
\begin{testimonia}[Kj27]
% not in the \emph{Śārṅgadharapaddhati} 
\emph{Haṭhapradīpikā} (10 chapter) 9.35
\begin{variants}
    kṛṣṇā kharā~] kṛṣṇaparā HP10\sep
    ca~] tu HP10\sep
    gaṇḍe~] gaṇḍaṃ HP10
\end{variants}
% \begin{versinnote}
% \tl{yasya kṛṣṇaparā jihvā padmākāraṃ tu vai mukham/\\+}
% \tl{gaṇḍaṃ vā piṇḍikā raktā tad antaṃ tasya jīvitam//\\!}
% \end{versinnote}
\end{testimonia}
%</ts27>

%<*cm27>
\begin{philcomm}[Kj27]
We have understood \emph{gaṇḍe vā piṇḍikā} as the fleshy region on the cheek in line with Mitākṣarā's gloss on \emph{Yājñavalkyasmṛti} 3.97cd (\emph{piṇḍikā māṃsalapradeśaḥ}). The original idea appears to be expressed in the \emph{Dharmaputrikā}, where the cheeks turn yellow and red (\emph{gaṇḍau tu pītakau raktau}), and the introduction of the word \emph{piṇḍikā} has caused confusion. 
%In the Aṣṭāṅgahṛdayasaṅgraha (Nidāna, 2.26), piṇḍikā means calf: ...piṇḍikāpārśvamūrdhaparvāsthirug...// (... pain in the calves, sides, head, joints and bones...). 

%I think the original idea is seen in the Dharmaputrikā: gaṇḍau vā pītakau raktau (`whose cheeks are yellow or red'). I'm almost tempted to emend to this (piṇḍikā is weird!) but perhaps the DhP's meaning has been lost by the time this verse was compiled in the HP. 

% MD: Perhaps corrupted from this?
% yasya kṛṣṇā kharā jihvā padmābhāsaṃ ca vai mukham/
% gaṇḍe vā piṇḍikārakte tasya mṛtyurupasthitaḥ// LiP_1,91.26//
% gaṇḍe seems to be a wrong dual form, but it makes sense: "whose cheeks are red from swelling".
% MD: read tad antaṃ as a compound as in v. 32 and 34?
% Jü: kṛṣṇaparā (that is, one who starts repeating Krsna's name) could be a conscious change by a redactor. 
% NJL: kṛṣṇaparā jihvā = pitch-black tongue, previous vers (with emendation of MD = black = death very near and subsequently tongue turs pitch-black -> death kicks in)?
%I don't think piṇḍikā means swelling in the sense of inflammation. Monier Williams definition of "globular fleshy swelling (in the shoulders, arms, legs, \&c.; esp. the calf of the leg)" is referring to the muscular protuberance of calves and thighs. He cites the Yājñavalkyasmṛti and must be referring to 3.97cd (upajihvā sphijau bāhū jaṅghoruṣu ca piṇḍikā) which Mitākṣarā glosses as "jaṅghoruṣu ca piṇḍikā gaṅghyor ūrvoś ca piṇḍikā māṃsalapradeśaḥ." I think the idea here is that red cheeks are a sign of death. I propose we read what J6 has gaṇḍe piṇḍikā raktā and understand it as the fleshy region on the cheek is red (= fleshy region of the cheeks is red).
\end{philcomm}
%</cm27>


\subsection*{Kj.28}
%<*tr28>
\begin{translation}[Kj28]
[If] the tongue becomes thick at its root when the hairs bristle and he sees the wrist become thick, he dies within a year and a half. 
\end{translation}
%</tr28>
% MD: pāda d: sārdha -> so'rdha°?

%<*sc28>
\begin{sources}[Kj28]
% b not in \emph{Mārkaṇḍeyapurāṇa}
Cf.~\emph{Tantrasadbhava} 24.327cd–328ab
\begin{versinnote}
\tl{yasya jihvā bhavet sthūlā dantāḥ klidyanti bhāmini//\\+}
\tl{mriyate so naro devi varṣānte ca na saṃśayaḥ/\\!}
\end{versinnote}
\end{sources}
%</sc28>

%<*ts28>
\begin{testimonia}[Kj28]
% Not in \emph{Śārṅgadharapaddhati} 4568
\emph{Haṭhapradīpikā} (10 chapter) 9.17, \emph{Haṭhapradīpikā} (6 chapter) 6.284
\begin{variants}
    mūle~] mūlo HP10, mūlaṃ HP6\sep
    sthūlā HP6~] sthūlo HP10\sep
    romoddhṛti HP6~] romaharṣa HP10\sep
    vīkṣya HP10~] vīkṣa HP6\sep
    varṣataḥ HP6~] māsataḥ HP10
\end{variants}

% \begin{versinnote}
% \tl{jihvāmūlo bhavet sthūlo romaharṣasamudgame/\\+}
% \tl{maṇibandhaṃ vīkṣya sthūlaṃ mriyate sārdhamāsataḥ//\\!}
% \end{versinnote}

% \emph{Haṭhapradīpikā} (6 chapter) 6.284
% \begin{versinnote}
% \tl{jihvā mūlaṃ bhavet sthūlā romoddhṛtisamudgame/\\+}
% \tl{maṇibandhaṃ vīkṣa sthūlaṃ mṛyate sārdhavarṣataḥ//\\!}
% \end{versinnote}
\end{testimonia}
%</ts28>

%<*cm28>
\begin{philcomm}[Kj28]
This verse does not appear outside the \emph{Haṭhapradīpikā}'s transmission, yet the notion of the tongue becoming thick seems to be an old omen (see e.g.~the \emph{Tantrasadbhava} parallel). Also, the timeframe is not consistent with the verses that precede and follow it.

%The syntax is a bit odd (finite verb, sati saptamī, gerund, then finite verb). romaharṣa is a emendation suggested by V3 (romahati). E4 has romodvṛtti. ud-vṛtti is not attested in the dictionary and romodvṛtti doesn't occur in my etexts.
% Jü: And the metre in Pada c is wrong.
% JB: yes, you're right! And I can't see a way of fixing it.
% MD: read maṇi as one long syllable and adopt the reading of ηω vīkṣyate?
% NJL: emend to maṇibandhaṃ vṛdhasthūlaṃ = wrist becomes thick and swollen?
% JM: udvṛtti seems ok to me, at least no need to emend
\end{philcomm}
%</cm28>

\subsection*{Kj.29}
%<*tr29>
\begin{translation}[Kj29]
He who experiences a loss of hearing and smell for seven days, [and] has blackness on the teeth and tongue, surely dies in fifteen days.  
\end{translation}
%</tr29>

%<*sc29>
%\begin{sources}[Kj29]
% Not in \emph{Mārkaṇḍeyapurāṇa} 
%\end{sources}
%</sc29>

%<*ts29>
\begin{testimonia}[Kj29]
% not in \emph{Śārṅgadharapaddhati} 

\emph{Haṭhapradīpikā} (6 chapter) 6.285
\begin{variants}
    dhvaṃsaṃ~] pathaṃ HP6
\end{variants}
% \begin{versinnote}
% \tl{śrutipathaṃ vahed yas tu saptāhair gandhanāśanaṃ/\\+}
% \tl{kṛṣṇatvaṃ dantajihvāyāṃ tripaṃcāhne dhruvaṃ mriyet//\\!}
% \end{versinnote}

Cf.~\emph{Haṭhapradīpikā} (10 chapter) 9.13
\begin{versinnote}
\tl{śrutipathaṃ yadā śabdo nādhirohati sarvathā/\\+}
\tl{kṛṣṇatvaṃ dantajihvāyāṃ tripakṣe mriyate dhruvam//\\!}
\end{versinnote}
\end{testimonia}
%</ts29>

%<*cm29>
%\begin{philcomm}[Kj29]
%Does śrutipathaṃ vahet make any sense here? It seems to make some sense in the 10-ch HP version (śrutipathaṃ yadā śabdo nādhirohati sarvathā, 'when the surround sound does not ascend the auditory passage'). I've opted for śrutidhvaṃsaṃ (N9) with vahed. 
% Jü: Best option, but still strange wording.
% NJL: I agree that this is probably the best option. Translate vahed in the sense of enduring or bearing the śrutidhvaṃsa and the gaṃdhanāśana?
%I've assumed one has to understand a yasya with kṛṣṇatvaṃ and saḥ with mriyet. 
%\end{philcomm}
%</cm29>

\subsection*{Kj.30}%??JM space after sadyo?
%<*tr30>
\begin{translation}[Kj30]
One should know that [a man] who in a dream travels south on a vehicle [drawn by] a camel and donkey dies immediately, O Lord.
\end{translation}
%</tr30>

%<*sc30>
\begin{sources}[Kj30]
\emph{Mārkaṇḍeyapurāṇa} 40.27 
\begin{variants}
    vi~] ca MP
\end{variants}
% \begin{versinnote}
% \tl{uṣṭrarāsabhayānena yaḥ svapne dakṣiṇāṃ diśam/\\+}
% \tl{prayāti taṃ ca jānīyāt sadyomṛtyuṃ nareśvara//\\!}
% \end{versinnote}
\end{sources}
%</sc30>

%<*ts30>
%\begin{testimonia}[Kj30]
%Not in \emph{Śārṅgadharapaddhati} 
%\begin{versinnote}
%\end{versinnote}
%\end{testimonia}
%</ts30>

%<*cm30>
\begin{philcomm}[Kj30]
The syntax of the transmitted reading for the fourth \textit{pāda} (\emph{°mṛtyur bhaven nṛṇām}) does not make sense, so the reading of the \emph{Mārkaṇḍeyapurāṇa} has been adopted. A similar idea is expressed in verse 15. 
%I think the transmitted reading for the fourth \textit{pāda} doesn't make sense (and seems grammatically wrong), so I've emended to the Mārkaṇḍeyapurāṇa reading (°mṛtyuṃ na saṃśayaḥ). taṃ is the object of vijānīyāt; yaḥ is the subject of prayāti; so we don't want °mṛtyur bhaven nṛṇām in the final pāda. Too heavy handed? 
% MD: I agree to the emendation. It is certainly no coincidence that this phrase is the same as 20d, which also has a syntactical problem.
% NJL: Probably an eyeskip. I agree, too. 
\end{philcomm}
%</cm30>

\subsection*{Kj.31}
%<*tr31>
\begin{translation}[Kj31]
He who blocks the ears and does not hear the sound arising in oneself, and who does not see a light in his eyes, does not live.% a light in their eyes (jyotir could also be taken with ātmasambhavam)
\end{translation}
%</tr31>

%<*sc31>
\begin{sources}[Kj31]
\emph{Mārkaṇḍeyapurāṇa} 40.28, cf.~\emph{Śārṅgadharapaddhati} 4580
\begin{variants}
    na paśyec~] naśyate MP ŚDP\sep
    yaś ca~] yasya MP ŚDP
\end{variants}

% \begin{versinnote}
% \tl{pidhāya karṇau nirghoṣaṃ na śṛṇoty ātmasambhavam/\\+}
% \tl{naśyate cakṣuṣor jyotir yasya so 'pi na jīvati//\\!}
% \end{versinnote}

% Cf.~\emph{Śārṅgadharapaddhati} 4580
% \begin{versinnote}
% \tl{pidhāya karṇau ca nijau na śṛṇoty ātmasambhavam/\\+}
% \tl{naśyate cakṣuṣor jyotir yasya so 'pi na jīvati//\\!}
% \end{versinnote}
\end{sources}
%</sc31>

%<*ts31>
% \begin{testimonia}[Kj31]

% \end{testimonia}
%</ts31>

%<*cm31>
\begin{philcomm}[Kj31]
A different idea is expressed in the third quarter of the parallel verses of the \emph{Mārkaṇḍeyapurāṇa} and \emph{Śārṅgadharapaddhati} (i.e.~`and the light in his eyes disappears'). %This appears to explain the odd genitive of \emph{cakṣus} in the \emph{Haṭhapradīpikā}'s verse.%JM makes sense as locative
\end{philcomm}
%</cm31>

\subsection*{Kj.32}
%<*tr32>
\begin{translation}[Kj32]
For him who falls into a pit in a dream and its opening is closed, and who cannot get out of the hole, that is the end of his life.
\end{translation}
%</tr32>

%<*sc32>
\begin{sources}[Kj32]
\emph{Mārkaṇḍeyapurāṇa} 40.29, cf.~\emph{Śārṅgadharapaddhati} 4583

% \begin{versinnote}
% \tl{patato yasya vai garte svapne dvāraṃ pidhīyate/\\+}
% \tl{na cottiṣṭhati yaḥ śvabhrāt tad antaṃ tasya jīvitam//\\!}
% \end{versinnote}

% Cf.~\emph{Śārṅgadharapaddhati} 4583
% \begin{versinnote}
% \tl{patato yasya vai garte svapne dvāraṃ pidhīyate/\\+}
% \tl{na cottiṣṭhati yaḥ śvabhrāt tad antaṃ tasya jīvitam//\\!}
% \end{versinnote}
\end{sources}
%</sc32>
% JM: that is the end of [his] life
%<*ts32>
% \begin{testimonia}[Kj32]

% \end{testimonia}
%</ts32>

%<*cm32>
%\begin{philcomm}[Kj32]
%The reading \emph{śvabhrāt} in the \emph{Mārkaṇḍeyapurāṇa} and \emph{Śārṅgadharapaddhati} makes better sense but is not attested in the HP witnesses. Nor is \emph{śvabhrāt} in the 6- and 10-chapter versions of the \emph{Haṭhapradīpikā}, so it seems unlikely to have been in this transmission.  
%\end{philcomm}
%</cm32>
% MD: Yes, but this mistake can be explained by the graphic similarity.
% So do you think we should emend?
% MD: Yes.

\subsection*{Kj.33}
%<*tr33>
\begin{translation}[Kj33]
[If] the eyes [turn] upwards, are unstable and red, and then roll around; [if] the mouth is hot and the navel is cold: [these signs] portend that men will [soon] take another body.
\end{translation}
%</tr33>
% lit: 'gaze' but 'eyes' seems more appropriate with 'red' (raktā) 
% lit: if the mouth has heat...
% people will have another body


%<*sc33>
\begin{sources}[Kj33]
\emph{Mārkaṇḍeyapurāṇa} 40.30, cf.~\emph{Liṅgapurāṇa} 1.91.32
\begin{variants}
    coṣmā MP~] śoṣaḥ LP\sep
    śaṃsanti puṃsām aparaṃ śarīram MP~] atyuṣṇamūtro viṣamastha eva LP
\end{variants}

% \begin{versinnote}
% \tl{ūrdhvā ca dṛṣṭir na ca sampratiṣṭhā raktā punaḥ samparivartamānā/\\+}
% \tl{mukhasya coṣmā śiśirā ca nābhis śaṃsanti puṃsām aparaṃ śarīram//\\!}
% \end{versinnote}
% Cf.~\emph{Liṅgapurāṇa} 1.91.32
% \begin{versinnote}
% \tl{ūrdhvā ca dṛṣṭir na ca sampratiṣṭhā raktā punaḥ samparivartamānā/\\+}
% \tl{mukhasya śoṣaḥ suṣirā ca nābhir atyuṣṇamūtro viṣamastha eva//\\!}
% \end{versinnote}
\end{sources}
%</sc33>

%<*ts33>
\begin{testimonia}[Kj33]
\emph{Haṭhatattvakaumudī} 56.3
% \begin{versinnote}
% \tl{ūrdhvā ca dṛṣṭir na ca saṃpratiṣṭhā  raktā punaḥ samparivarttamānā/\\+} % tā is unmetr.
% \tl{mukhasya coṣmā śiśirā ca nābhiḥ śaṃsanti puṃsāmaparaṃ śarīram//\\!}
% \end{versinnote}
\end{testimonia}
%</ts33>

%<*cm33>
\begin{philcomm}[Kj33]
Metre: upajāti (indravajrā + upendravajrā)   
%According to Apte, praviṣṭa can mean `sunk' in the context of an eye. But the negative makes this meaning less likely, so I have emended to sampratiṣṭhā.
%JM: saṃpratiṣṭhā is odd, saṃpratiṣṭhitā would be usual but unmetrical
%śiśirā (cold) makes better sense than suṣirā (hollow), and seems obviously intended as a contrast with uṣmā, so I've emended. 
\end{philcomm}
%</cm33>

\subsection*{Kj.34}
%<*tr34>
\begin{translation}[Kj34]
He who enters fire in a dream and then does not emerge, or [does not emerge] from entering water, that is the end of his life.
\end{translation}
%</tr34>
%JM: that is the end of [their] life
%<*sc34>
\begin{sources}[Kj34]
\emph{Mārkaṇḍeyapurāṇa} 40.31, cf.~\emph{Śārṅgadharapaddhati} 4584

% \begin{versinnote}
% \tl{svapne 'gniṃ praviśed yas tu na ca niṣkramate punaḥ/\\+}
% \tl{jalapraveśād api vā tad antaṃ tasya jīvitam//\\!}
% \end{versinnote}

% Cf.~\emph{Śārṅgadharapaddhati} 4584
% \begin{versinnote}
% \tl{svapne 'gniṃ praviśed yas tu na ca niṣkramate punaḥ/\\+}
% \tl{jalapraveśād api vā tad antaṃ tasya jīvitam//\\!}
% \end{versinnote}
\end{sources}
%</sc34>

%<*ts34>
% \begin{testimonia}[Kj34]

% \end{testimonia}
%</ts34>

%<*cm34>
%\begin{philcomm}[Kj34]
%\end{philcomm}
%</cm34>

\subsection*{Kj.35}
%<*tr35>
\begin{translation}[Kj35]
A man whose sight is afflicted by spirits at night and then during the day undoubtedly meets his death at the end of a week.%JM: a week?
\end{translation}
%</tr35>

%<*sc35>
\begin{sources}[Kj35]
\emph{Mārkaṇḍeyapurāṇa} 40.32, cf.~\emph{Śārṅgadharapaddhati} 4579
\begin{variants}
    yasyāpi ŚDP~] yaś cābhi MP\sep
    dṛṣṭair ŚDP~] duṣṭair MP\sep
    pumān ŚDP~] naraḥ MP
\end{variants}

% \begin{versinnote}
% \tl{yaś cābhihanyate duṣṭair bhūtair rātrāv atho divā/\\+}
% \tl{sa mṛtyuṃ saptarātrānte naraḥ prāpnoty asaṃśayam//\\!}
% \end{versinnote}

% Cf.~\emph{Śārṅgadharapaddhati} 4579
% \begin{versinnote}
% \tl{yasyāpi hanyate dṛṣṭair bhūtai rātrāv atho divā/\\+}
% \tl{sa mṛtyuṃ saptarātrānte pumān prāproty asaṃśayam//\\!}
% \end{versinnote}
\end{sources}
%</sc35>

%<*ts35>
% \begin{testimonia}[Kj35]
% \emph{Haṭhapradīpikā} (10 chapter) 9.21
% \begin{versinnote}
% \tl{yasyāpi hanyate dṛṣṭir bhūte rātrau divāthavā/\\+}
% \tl{sa mṛtyuṃ saptarātrānte pumān prāpnoty asaṃśayam//\\!}
% \end{versinnote}
% \end{testimonia}
%</ts35>

%<*cm35>
%\begin{philcomm}[Kj35]
%The original reading could have been duṣṭair bhūtair but, even in the \emph{Śārṅgadharapaddhati}, dṛṣṭ° has crept in. 
%In the second hemistich, the manuscript witnesses have \emph{saptame rātrānteṣu} or \emph{saptame rātrau teṣu}. The phrase \emph{saptame rātrau} can be retained but the \emph{teṣu} does not make sense. I have therefore emended it to \emph{pumān} in light of the same verse in the ten-chapter \emph{Haṭhapradīpikā} and \emph{\emph{Śārṅgadharapaddhati}}.
% MD: Shouldn't it be saptamyāṃ, if rātrau? I would prefer to adopt the source text saptarātrānte. In any case, the first mistake should have been the insertion of me in saptame. ṣu of teṣu was probably a misreading of pu of pumān.
% Yes, you're right. saptame doesnt work so we need saptarātrānte.
%\end{philcomm}
%</cm35>

\subsection*{Kj.36}
%<*tr36>
\begin{translation}[Kj36]
If a man sees his spotless, white clothes as red, then black, one should declare that his death is near.
\end{translation}
%</tr36>
% MD: or "sees his spotless white clothes as red and black"? (or: "red, then black" with atha?)
% Jü: makes better sense! 
% Yes, agree!

% MD: Better to read paśyaty athā°? A present participle is unusual. If we read paśyan, then paśyann.
% okay. paśyaty athā°, done.

%<*sc36>
\begin{sources}[Kj36]
\emph{Mārkaṇḍeyapurāṇa} 40.33
\begin{variants}
    athāsitam~] atho 'sitam MP
\end{variants}
% \begin{versinnote}
% \tl{svavastram amalaṃ śuklaṃ raktaṃ paśyaty atho 'sitam/\\+}
% \tl{yaḥ pumān mṛtyum āsannaṃ tasyāpi hi vinirdiśet//\\!}
% \end{versinnote}
\end{sources}
%</sc36>

%<*ts36>
\begin{testimonia}[Kj36]
%\emph{Śārṅgadharapaddhati} 4568
\emph{Yogacintāmaṇi} f.\,144r (\attr \emph{Mārkaṇḍeyapurāṇa})
\begin{variants}
    āsannaṃ~] āpannaṃ YCM
\end{variants}
% \begin{versinnote}
% \tl{svavastram amalaṃ śuklaṃ raktaṃ paśyaty athāsitam//\\+}
% \tl{yaḥ pumān mṛtyum āpannaṃ tasyāpi hi vinirdiśet/\\!}
% \end{versinnote}
\end{testimonia}
%</ts36>

%<*cm36>
%\begin{philcomm}[Kj36]
%\end{philcomm}
%</cm36>

\subsection*{Kj.37}
%<*tr37>
\begin{translation}[Kj37]
They say Yama and Antaka are near to men if there is a reversal of their true nature and an alteration to their constitution. %??MD: The subject may be viparītatva and viparyaya.
\end{translation}
%</tr37>

%<*sc37>
\begin{sources}[Kj37]
\emph{Mārkaṇḍeyapurāṇa} 40.34, cf.~\emph{Śārṅgadharapaddhati} 4587
\begin{variants}
    viparītatvaṃ~] vaiparītyaṃ tu MP, vaiparītyena ŚDP
\end{variants}

% \begin{versinnote}
% \tl{svabhāvavaiparītyaṃ tu prakṛteś ca viparyayaḥ/\\+}
% \tl{kathayanti manuṣyāṇāṃ samāsannau yamāntakau//\\!}
% \end{versinnote}

% Cf.~\emph{Śārṅgadharapaddhati} 4587
% \begin{versinnote}
% \tl{svabhāvavaiparītyena śarīrasya viparyaye/\\+}
% \tl{kathayanti manuṣyāṇāṃ samāsannau yamāntakau//\\!}
% \end{versinnote}
\end{sources}
%</sc37>

%<*ts37>
% \begin{testimonia}[Kj37]

% \end{testimonia}
%</ts37>

%<*cm37>
\begin{philcomm}[Kj37]
We have adopted the readings \emph{°viparītatvaṃ} and \emph{viparyayaḥ} in keeping with the parallel verse in the \emph{Mārkaṇḍeyapurāṇa}. Both \emph{°viparītam} and \emph{viparyayam} in the \emph{Haṭhapradīpikā} witnesses are adjectives without an implied noun. It also seems more probable that \emph{°viparītaṃ ca} is a corruption of \emph{°viparītatvaṃ} than \emph{°vaiparītyaṃ tu}.  
% MD: °viparītatvaṃ + viparyayaḥ? Jü: surely better
% Jü: Or take kathayanti with two acc., but still awkward.
% JB: I've emended to the MāPur readings. °viparītatvaṃ is not attested in any witnesses.
% MD: The corruption from viparītatvaṃ to viparītaṃ ca is more plausible than from vaiparītyaṃ ca. The first ca is dispensable.
% Okay done.
\end{philcomm}
%</cm37>

\subsection*{Kj.38}
%<*tr38>
\begin{translation}[Kj38]
For him who sees in a dream a dwarf holding an iron staff and dressed in black clothes, death occurs after three nights.
\end{translation}
%</tr38>

%<*sc38>
\begin{sources}[Kj38]
% Not in \emph{Mārkaṇḍeyapurāṇa} 
\emph{Vasiṣṭhasaṃhitā} 8.25cd–26ab
\begin{variants}
    hrasvaṃ~] kṛṣṇaṃ VS\sep
    trirātrān~] trimāsān VS
\end{variants}
% \begin{versinnote}
% \tl{lohadaṇḍadharaṃ kṛṣṇaṃ kṛṣṇavastraparicchadam//\\+}
% \tl{svapne prapaśyatas tasya trimāsān maraṇaṃ bhavet/\\!}
% \end{versinnote}
Cf.~\emph{Yogaśāstra} 5.155
\begin{versinnote}
\tl{kṛṣṇaṃ kṛṣṇaparīvāraṃ lohadaṇḍadharaṃ naram/\\+}
\tl{yadā svapne nirīkṣeta mṛtyur māsais tribhis tadā//\\!}
\end{versinnote}
Cf.~\emph{Vivekamārtaṇḍa} (6 chapter) 4.187
\begin{versinnote}
\tl{lohadaṇḍadharaṃ bhīmaṃ puruṣaṃ kṛṣṇapiṅgalam/\\+}
\tl{yaḥ svapne paśyati kruddhaṃ tribhir māsaiḥ sa gacchati//\\!}
\end{versinnote}
\end{sources}
%</sc38>

%<*ts38>
% \begin{testimonia}[Kj38]

% \end{testimonia}
%</ts38>

%<*cm38>
%\begin{philcomm}[Kj38]
%\end{philcomm}
%</cm38>

\subsection*{Kj.39}
%<*tr39>
\begin{translation}[Kj39]
If [a man's] senses do not perceive their respective objects, he will undoubtedly die at the end of a month.  
\end{translation}
%</tr39>

%<*sc39>
\begin{sources}[Kj39]
% Not in \emph{Mārkaṇḍeyapurāṇa} 
\emph{Vasiṣṭhasaṃhitā} 8.26cd–8.27ab
\begin{variants}
    gṛhṇīyuḥ~] gṛhnanti VS\sep
    viṣayān~] viṣayaṃ VS
\end{variants}
% \begin{versinnote}
% \tl{indriyāṇi na gṛhnanti svakīyaṃ viṣayaṃ yadi//\\+}
% \tl{māsānte maraṇaṃ tasya bhaviṣyati na saṃśayaḥ/\\!}
% \end{versinnote}
\end{sources}
%</sc39>

%<*ts39>
%\begin{testimonia}[Kj39]
%\emph{Śārṅgadharapaddhati} 4568
%\begin{versinnote}
%\end{versinnote}
%\end{testimonia}
%</ts39>

%<*cm39>
%\begin{philcomm}[Kj39]
%\end{philcomm}
%</cm39>

\subsection*{Kj.40}
%<*tr40>
\begin{translation}[Kj40]
For him who does not see his own reflection or face in a mirror or in water, death will undoubtedly occur at the end of a month.
\end{translation}
%</tr40>

%<*sc40>
\begin{sources}[Kj40]
% Not in \emph{Mārkaṇḍeyapurāṇa} 
\emph{Vasiṣṭhasaṃhitā} 8.29
\begin{variants}
    māsānte maraṇaṃ tasya~] tasyāpi māsato mṛtyur VS
\end{variants}
% \begin{versinnote}
% \tl{darpaṇe cātmanaś chāyām apsu vā yo na paśyati/\\+}
% \tl{tasyāpi māsato mṛtyur bhaviṣyati na saṃśayaḥ//\\!}
% \end{versinnote}
\end{sources}
%</sc40>
% MD: N9 has chāyā instead of kāyaṃ. Better to read chāyām as in VS?
% The VS has chāyām apsu vā which makes good sense. We could adopt that, but chāyām āsyaṃ vā seems a bit odd (...does not see their reflection or face in a mirror...)

%<*ts40>
\begin{testimonia}[Kj40]
\emph{Haṭhapradīpikā} (10 chapter) 9.16
\begin{variants}
    chāyām apsu~] kāyam āsyaṃ HP10
\end{variants}
% \begin{versinnote}
% \tl{darpaṇe svātmanaḥ kāyam āsyaṃ vā yo na paśyati/\\+}
% \tl{māsante maraṇaṃ tasya bhaviṣyati na saṃśayaḥ//\\!}
% \end{versinnote}
\end{testimonia}
%</ts40>

%<*cm40>
%\begin{philcomm}[Kj40]
%\end{philcomm}
%</cm40>

\subsection*{Kj.41}
%<*tr41>
\begin{translation}[Kj41]
If half of his body is hot and the [other] half cold or if he has lost the hearing in his ears, he will die in a week. 
\end{translation}
%</tr41>

%<*sc41>
\begin{sources}[Kj41]
\emph{Vasiṣṭhasaṃhitā} 8.38
\begin{variants}
    cāpi ca~] vāpy ati VS\sep
    śruti~] smṛti VS\sep
    saptarātre~] saptāhāt sa VS
\end{variants}
% \begin{versinnote}
% \tl{uṣṇaṃ yasya śarīrārdham ardhaṃ vāpy atiśītalam//\\+}
% \tl{karmasmṛtivināśo vā saptāhāt sa mariṣyati/\\!}
% \end{versinnote}
% not in \emph{Mārkaṇḍeyapurāṇa} 
%\begin{versinnote}
%\end{versinnote}
\end{sources}
%</sc41>

%<*ts41>
\begin{testimonia}[Kj41]
\emph{Haṭhapradīpikā} (10 chapter) 9.22
% \begin{versinnote}
% \tl{uṣṇaṃ yasya śarīrārdham arddhaṃ cāpi ca śītalam/\\+}
% \tl{karṇaśrutivināśo vā saptarātre mariṣyati//\\!}
% \end{versinnote}
\end{testimonia}
%</ts41>

%<*cm41>
%\begin{philcomm}[Kj41]
%\end{philcomm}
%</cm41>

\subsection*{Kj.42}
%<*tr42>
\begin{translation}[Kj42]
When the time of death has come for yogis, gnostics or other great sages, [the special omen] should be known by wise people. 
\end{translation}
%</tr42>

%<*sc42>
\begin{sources}[Kj42]
\emph{Mārkaṇḍeyapurāṇa} 40.37
\begin{variants}
    ca MP~] vā ŚDP\sep
    'ntakāle~] tu kāle MP, ca kāle ŚDP\sep
    puruṣais ŚDP~] puruṣas MP\sep
    vijñeyaṃ MP~] vicāryaṃ ŚDP
\end{variants}

% \begin{versinnote}
% \tl{yogināṃ jñānaviduṣām anyeṣāṃ ca mahātmanām/\\+}
% \tl{prāpte tu kāle puruṣas tad vijñeyaṃ vicakṣaṇaiḥ//\\!}
% \end{versinnote}

% Cf.~\emph{Śārṅgadharapaddhati} 4590
% \begin{versinnote}
% \tl{yogināṃ jñānaviduṣām anyeṣāṃ vā mahātmanām/\\+}
% \tl{prāpte ca kāle puruṣais tad vicāryaṃ vicakṣaṇaiḥ//\\!}
% \end{versinnote}
\end{sources}
%</sc42>

%<*ts42>
% \begin{testimonia}[Kj42]

% \end{testimonia}
%</ts42>

%<*cm42>
%\begin{philcomm}[Kj42]
%The manuscripts have \emph{puruṣaṃ} in the third verse quarter but I can't see how that can work, so I've emended to the reading of the Śārṅgadharapaddhati.    % Jü: missing from the apparatus
%I'm not sure how we should understand tad here. Since this section starts with a verse explaining that this discussion is on special ariṣṭas, I've taken tad as referring to that.
%\end{philcomm}
%</cm42>

%<*tr42p>
\begin{translation}[Kj42p]
%for iti kālajñānam
\end{translation}
%</tr42p>


%%%%%%%%%%%%%%%%%%%%%%%%%%%%%%%%%%%%
% videhamukti

\subsection*{Kj.43 Videhamukti}

%<*tr43a>
\begin{translation}[Kj43a]
Now, the explanation of liberation without a body -- 
\end{translation}%?? the teaching on bodiless liberation
%</tr43a>

%<*sc43a>
\begin{sources}[Kj43a]
\emph{Śārṅgadharapaddhati} 4591
\begin{versinnote}
\tl{atha videhamuktikathanam/\\!}
\end{versinnote}
\end{sources}
%</sc43a>

%<*tr43>
\begin{translation}[Kj43]
Whether in the morning, in the afternoon, at midday, at any time of day, or sometime at night, [the yogi] should examine the omen.%?? observe? pay attention to?
\end{translation}
%</tr43>

%<*sc43>
\begin{sources}[Kj43]
\emph{Śārṅgadharapaddhati} 4591, cf.~\emph{Mārkaṇḍeyapurāṇa} 40.42cd–43ab
\begin{variants}
    vā ŚDP~] ca MP\sep
    vā dine kvacit~] vā pare kvacit ŚDP, cāpi taddine MP
\end{variants}


% \begin{versinnote}
% \tl{pūrvāhne vā parāhne vā madhyāhne vā pare kvacit/\\+}
% \tl{yatra vā rajanībhāge tadariṣṭaṃ nirīkṣitam//\\!}
% \end{versinnote}

% Cf.~\emph{Mārkaṇḍeyapurāṇa} 40.42cd–43ab
% \begin{versinnote}
% \tl{pūrvāhne cāparāhne ca madhyāhne cāpi taddine//\\+} 
% \tl{yatra vā rajanībhāge tadariṣṭaṃ nirīkṣitam/\\!}
% \end{versinnote}
\end{sources}
%</sc43>

%<*ts43>
\begin{testimonia}[Kj43]
\emph{Yogacintāmaṇi} f.\,144v (\attr \emph{Mārkaṇḍeyapurāṇa})
\begin{variants}
    vāparāhne vā~] cāparāhṇe ca YCM\sep
    kvacit~] dine YCM
\end{variants}
% \begin{versinnote}
% \tl{pūrvāhṇe cāparāhṇe ca madhyāhne vā dine dine//\\+}
% \tl{yatra vā rajanībhāge tad āriṣṭaṃ nirīkṣitam/\\!}
% \end{versinnote}
\end{testimonia}
%</ts43>

%<*cm43>
%\begin{philcomm}[Kj43]
%\emph{rajanībhavaḥ} is supported by all HP mss., but the original reading was \emph{rajanībhāge}, which makes sense to me (so I've emended to this). But is \emph{rajanībhavaḥ} possible? Perhaps it is because \emph{rātraubhava} is attested as an adjective (happening or occurring at night) in the dictionary. But I don't see \emph{rajanībhavaḥ} in this sense in my etexts.
%\end{philcomm}
%</cm43>

\subsection*{Kj.44}
%<*tr44>
\begin{translation}[Kj44]
Having determined the time of his own [death] according to the external and internal signs, [the yogi] who, through renunciation (\emph{nyāsa}), is serene, free of extremes (i.e.~hot, cold, pain, pleasure, etc.) and has his senses under control,[\dots]  
\end{translation}
%</tr44>

%<*sc44>
\begin{sources}[Kj44]
\emph{Vasiṣṭhasaṃhitā} 6.3
\begin{variants}
    nyāsataḥ~ sa] nirbhayas tu VS
\end{variants}
% \begin{versinnote}
% \tl{vasiṣṭha uvāca\\+}
% \tl{viniścityātmanaḥ kālaṃ bāhyābhyantaralakṣaṇaiḥ/\\+}
% \tl{nirbhayas tu prasannātmā nirdvandvo vijitendriyaḥ//\\!}
% \end{versinnote}
\end{sources}
%</sc44>

%<*ts44>
%\begin{testimonia}[Kj44]
%\emph{Śārṅgadharapaddhati} 4568
%\begin{versinnote}
%\end{versinnote}
%\end{testimonia}
%</ts44>

%<*cm44>
%\begin{philcomm}[Kj44]
%What does nyāsataḥ mean here? %JM: = saṃnyāsa as you translate is best guess
%\end{philcomm}
%</cm44>

\subsection*{Kj.45}
%<*tr45>
\begin{translation}[Kj45]
[\dots] performs the appropriate rites as well as the daily and occasional ones. He should obtain the supreme self in the heart (\emph{guhā}) with the mind, through yoga,[\dots]
\end{translation}
%</tr45>

%<*sc45>
\begin{sources}[Kj45]
\emph{Vasiṣṭhasaṃhitā} 6.4
\begin{variants}
    kurute yukta~] kurvan vidhyukta VS
\end{variants}
% \begin{versinnote}
% \tl{kurvan vidhyuktakarmāṇi nityanaimittikāni ca/\\+}
% \tl{yogena paramātmānaṃ guhāyāṃ prāpya cetasā//\\!}
% \end{versinnote}
\end{sources}
%</sc45>

%<*ts45>
\begin{testimonia}[Kj45]
\emph{Haṭhapradīpikā} (10 chapter) 3cd–4ab
% \begin{versinnote}
% \tl{kurute yuktakarmāṇi nityanaimittikāni ca/\\+}
% \tl{yogena paramātmānaṃ guhāyāṃ prāpya cetasā/\\!}
% \end{versinnote}
\end{testimonia}
%</ts45>

%<*cm45>
%\begin{philcomm}[Kj45]
%The ca at the end of the 1st hemistich suggests that yuktakarma and nityanaimittika are two things, but the latter could qualify the former (i.e.~appropriate rites, both daily and occasional,...). Is the ca important or just a verse filler? JM: two things
%\end{philcomm}
%</cm45>

\subsection*{Kj.46}
%<*tr46>
\begin{translation}[Kj46]
[\dots and] with the breath mastered and free from desire, he should always sacrifice with \textsc{om}. And he who is desireless and devoted to Viṣṇu should repeat the salvific mantra.%om or salvific mantra? Why "a Brahmin"?
\end{translation}
%</tr46>

%<*sc46>
\begin{sources}[Kj46]
\emph{Vasiṣṭhasaṃhitā} 6.5
\begin{variants}
    yajen~] japen VS\sep
    japec ca~] japed vā  VS\sep
    cācyutapriyaḥ~] cācyutaḥ pathāt VS
\end{variants}
% \begin{versinnote}
% \tl{tārakeṇa japen nityaṃ jitāsuḥ kāmavarjitaḥ/\\+}
% \tl{japed vā tārakaṃ brahma niṣkāmaś cācyutaḥ pathāt//\\!}
% \end{versinnote}
\end{sources}
%</sc46>

%<*ts46>
%\begin{testimonia}[Kj46]
%\emph{Śārṅgadharapaddhati} 4568
%\begin{versinnote}
%\end{versinnote}
%\end{testimonia}
%</ts46>

%<*cm46>
\begin{philcomm}[Kj46]
The term \emph{tārakam brahma} (``salvific mantra'') is found in various Purāṇas and more recent Upa\-niṣads. On its meaning in Śaiva sources and its reinterpretion as the six-syllable Rāma mantra by Vaiṣṇava scholars, see Bakker 2019: 467–468.
% Bakker, Hans. 2019. Holy Ground : Where Art and Text Meet : Studies in the Cultural History of India. Leiden: Brill. https://doi.org/10.1163/9789004412071.
%Is japed vā worth adopting? It seems a more meaningful reading (N9 has cā).
% MD: N9's reading is an outlier.
\end{philcomm}
%</cm46>

\subsection*{Kj.47}
%<*tr47>
\begin{translation}[Kj47]
Likewise, for part of that day, the adept of yoga should practise yoga for [attaining] bodiless liberation at death, knowing [the time of death] and being free from the fear of being born [again].
\end{translation}
%</tr47>
%JM: tasya perhaps with ahnaḥ, i.e. on the same day as the practices in the previous verse
% done
%<*sc47>
\begin{sources}[Kj47]
\emph{Śārṅgadharapaddhati} 4592
\begin{variants}
    jananajaṃ~] maraṇajaṃ ŚDP
\end{variants}
% \begin{versinnote}
% \tl{tasya bhāge tathaivāhno yogaṃ yuñjīta yogavit/\\+}
% \tl{videhamuktaye jñānī tyaktvā maraṇajaṃ bhayam//\\!}
% \end{versinnote}

Cf.~\emph{Mārkaṇḍeyapurāṇa} 40.41, 40.42ab
\begin{versinnote}
\tl{dṛṣṭvāriṣṭaṃ tathā yogī tyaktvā maraṇajaṃ bhayam/\\+}
\tl{tatsvabhāvaṃ tadālokya kālo yāvad vipākadaḥ//\\+} 
\tl{tasya bhāge tathaivāhno yogaṃ yuñjīta yogavit/\\!}
\end{versinnote}
\end{sources}
%</sc47>

%<*ts47>
% \begin{testimonia}[Kj47]

% \end{testimonia}
%</ts47>

%<*cm47>
%\begin{philcomm}[Kj47]
%\end{philcomm}
%</cm47>

\subsection*{Kj.48}
%<*tr48>
\begin{translation}[Kj48]
The wise [yogi], seated in lotus posture with his neck in a straight position, should restrain his \emph{prāṇa} and \emph{apāna}, not touch his teeth together,[\dots]
\end{translation}
%</tr48>
%JM: no need for ellipsis, previous verse can be understood as a sentence
% done
%<*sc48>
%\begin{sources}[Kj48]
%\emph{Mārkaṇḍeyapurāṇa} 
%\begin{versinnote}
%\end{versinnote}
%\end{sources}
%</sc48>

%<*ts48>
\begin{testimonia}[Kj48]
\emph{Śārṅgadharapaddhati} 4593
\begin{variants}
    nirudhya prāṇāpānau ca~] niruddhaprāṇapavano ŚDP\sep
    dantān asaṃpṛśan~] dantān na saṃspṛśet ŚDP
\end{variants}
% \begin{versinnote}
% \tl{baddhapadmāsano dhīmān samasaṃsthānakandharaḥ/\\+}
% \tl{niruddhaprāṇapavano dantair dantātra saṃspṛśet//\\!}
% \end{versinnote}
\end{testimonia}
%</ts48>

%<*cm48>
% \begin{philcomm}[Kj48]
% \end{philcomm}
%</cm48>

\subsection*{Kj.49}
%<*tr49>
\begin{translation}[Kj49]
[\dots] mindfully close off the nine apertures [of his body], and shut his eyes. He makes the sound \textsc{oṃ} his bow, fixes \emph{sattva} as the string, [\dots] 
\end{translation}
%</tr49>

%<*sc49>
%\begin{sources}[Kj49]
%\emph{Mārkaṇḍeyapurāṇa} 
%\begin{versinnote}
%\end{versinnote}
%\end{sources}
%</sc49>

%<*ts49>
\begin{testimonia}[Kj49]
\emph{Śārṅgadharapaddhati} 4594
\begin{variants}
    niyojayet~] niyojya ca ŚDP
\end{variants}
% \begin{versinnote}
% \tl{buddhyā nirudhya dvārāṇi nava mīlitalocanaḥ/\\+}
% \tl{oṃkāraṃ tu dhanuḥ kṛtvā guṇaṃ sattvaṃ niyojya ca//\\!}
% \end{versinnote}
\end{testimonia}
%</ts49>

%<*cm49>
% \begin{philcomm}[Kj49]
% \end{philcomm}
%</cm49>

\subsection*{Kj.50}
%<*tr50>
\begin{translation}[Kj50]
[\dots and] the self as the arrow on [the string]. [The arrow] is equipped with the elements, senses and so forth. Situated in the heart lotus, it is shot by letting it fly with the breath and mind. 
\end{translation}
%</tr50>

%<*sc50>
\begin{sources}[Kj50]
\emph{Śārṅgadharapaddhati} 4595
\mylb
% \begin{versinnote}
% \tl{tatrātmānaṃ śaraṃ so 'pi vṛto bhūtendriyādibhiḥ/\\+}
% \tl{prāṇavāyumanaḥkṣepaiḥ kṣipto hṛtkamalasthitaḥ//\\!}
% \end{versinnote}

Cf.~\emph{Muṇḍakopaniṣat} 2.2.4
\begin{versinnote}
\tl{praṇavo dhanuḥ śāro hy ātmā brahma tallakṣyam ucyate/\\+} 
\tl{apramattena veddhavyaṃ śaravat tanmayo bhavet//\\!}
\end{versinnote}
\end{sources}
%</sc50>

%<*ts50>
\begin{testimonia}[Kj50]
\emph{Haṭhapradīpikā} (10 chapter) 10.15cd (ab only)
\begin{variants}
    tatrātmānaṃ śaraṃ so 'pi~] ātmānaṃ prāṇam āsādya HP10
\end{variants}
% \begin{versinnote}
% \tl{ātmānaṃ prāṇam āsādya vṛto bhūtendriyādibhiḥ//\\!}
% \end{versinnote}
\end{testimonia}
%</ts50>

%<*cm50>
%\begin{philcomm}[Kj50]

%Not sure of the meaning of kṣepa. Does it have a technical meaning in archery? śarakṣepa means the range of the arrow-shot. or perhaps kṣepa describes the action of releasing an arrow: kṣepaṇa in archery means `letting fly or go (a bow-string).

%JM letting fly sounds good; maybe vṛtaḥ means something particular for an arrow, but I can't find it in dhanurveda
%\end{philcomm}
%</cm50>
% MD: The 4th Pāda is unmetrical. The reading of ηω (E4) is closer to the Śārṅgadharapaddhati.
% JB: well spotted. I've adopted the ŚDP reading for pāda 4, and changed the translation.
% JM: arrow/ātman is subject in 50 and 51.
%  If vṛta refers to the arrow, perhaps it refers to the action of drawing back the arrow. or maybe just "held back by..."?  
\subsection*{Kj.51}
%<*tr51>
\begin{translation}[Kj51]
Having reached the target by the path to the tenth door, it then dissolves into the supreme self along with the thirty-six ontic principles.
\end{translation}
%</tr51>

%<*sc51>
\begin{sources}[Kj51]
\emph{Śārṅgadharapaddhati} 4596
% \begin{versinnote} 
% \tl{daśamadvāramārgeṇa lakṣyaṃ prāpya tataḥ param/\\+}
% \tl{ṣaṭtriṃśattattvasaṃyuktaḥ paramātmani līyate//\\!}
% \end{versinnote}
\end{sources}
%</sc51>

%<*ts51>
% \begin{testimonia}[Kj51]

% \end{testimonia}
%</ts51>

%<*cm51>
%\begin{philcomm}[Kj51]
%\end{philcomm}
%</cm51>

\subsection*{Kj.52}
%<*tr52>
\begin{translation}[Kj52]
Then, there is supreme space, which is beyond the senses and inaccessible. That which the higher faculty is not able to name does not truly exist.% not an object > does not truly exist?
\end{translation}
%</tr52>
 
%<*sc52>
\begin{sources}[Kj52]
\emph{Śārṅgadharapaddhati} 4597, cf.~\emph{Mārkaṇḍeyapurāṇa} 40.46
\begin{variants}
    paramam ākāśam ŚDP~] paramanirvāṇam MP\sep
    buddhyā ŚDP~] buddher MP\sep
    naiva ŚDP~] yan na MP\sep
    na ca vastu tat~] 'nantam aśnute ŚDP, tat samaśnute MP
\end{variants}

% \begin{versinnote}
% \tl{tataḥ paramam ākāśam atīndriyam agocaram/\\+}
% \tl{yad buddhyā naiva cākhyātuṃ śakyate 'nantam aśnute//\\!}
% \end{versinnote}

% Cf.~\emph{Mārkaṇḍeyapurāṇa} 40.46
% \begin{versinnote}
% \tl{tataḥ paramanirvāṇam atīndriyam agocaram/\\+}
% \tl{yad buddher yan na cākhyātuṃ śakyate tat samaśnute//\\!}
% \end{versinnote}
\end{sources}
%</sc52>

%<*ts52>
\begin{testimonia}[Kj52]
\emph{Haṭhapradīpikā} (10 chapter) 10.17
\begin{variants}
    tataḥ paramam~] etad dhi parama HP10\sep
    yad buddhyā naiva cākhyātuṃ~] yat tu dhyānenākhyātu[ṃ] HP10\sep 
    vastu tat~] vastutaḥ HP10
\end{variants}
% \begin{versinnote}
% \tl{etad dhi parama ākāśam atīndriyam agocaram/\\+}
% \tl{yat tu dhyānenākhyātu[ṃ] śakyate na ca vastutaḥ//\\!}
% \end{versinnote}
\end{testimonia}
%</ts52>

%<*cm52>
\begin{philcomm}[Kj52]
We have adopted the \emph{Śārṅgadharapaddhati}'s reading (\emph{yad buddhyā naiva cākhyātuṃ}) to make sense of third \emph{pāda}. We have retained the unique ending of the fourth \emph{pāda} (\emph{na ca vastu tat}) but the readings of the \emph{Mārkaṇḍeyapurāṇa} (\emph{tat samaśnute}) and \emph{Śārṅgadharapaddhati} (\emph{'nantam aśnute}) are much better.
% check note
\end{philcomm}
%</cm52>

\subsection*{Kj.53 Kālavañcana}
%<*tr53a>
\begin{translation}[Kj53a]
Now, cheating death --
\end{translation}
%</tr53a>
%<*tr53>
\begin{translation}[Kj53]
Hear from me about when a yogi wants to roam the three worlds liberated-in-life, with a body.
\end{translation}
%</tr53>
%JM: Hear from me about when a yogi wants to roam the three worlds liberated-in-life, with a body.
%<*sc53>
\begin{sources}[Kj53]
\emph{Śārṅgadharapaddhati} 4598
\begin{variants}
    sā~] cej ŚDP
\end{variants}
% \begin{versinnote}
% \tl{jīvanmuktaḥ sadeho 'haṃ vicarāmi jagattrayam/\\+}
% \tl{iti cej jāyate vāñchā yoginas tan nibodha me//\\!}
% \end{versinnote}
\end{sources}
%</sc53>

%<*ts53>
\begin{testimonia}[Kj53]
\emph{Yogacintāmaṇi} f.\,108v (\attr Dattātreya), \emph{Haṭhasaṅketacandrikā} f.\,119r (\attr \emph{Yogatattvaprakāśa})
\begin{variants}
    sā~] cej YCM HSC\sep
    vāñchā YCM~] vāṃchāṃ HSC
\end{variants}
%\begin{versinnote}
%\tl{dattātreyaḥ//\\+}
% \tl{jīvanmuktas sadeho 'haṃ vicarāmi jagattrayam/\\+}
% \tl{iti cej jāyate vāñchā yoginas taṃ nibodha me//\\!}
% \end{versinnote}
\end{testimonia}
%</ts53>

%<*cm53>
%\begin{philcomm}[Kj53]
%The word \emph{cet}, which makes the syntax clear, has dropped out of the \emph{Haṭhapradīpikā}'s transmission. 
% MD: better to adopt sā (= vāñchā) of ηω than saṃ° of δω.
% done
%I think the broader context of this verse is that the yogi may choose between videha- (explained in the previous section) and jīvan- muktis... if he chooses jīvanmukti, this is Śiva's (?) explanation of it. 
%\end{philcomm}
%</cm53>

\subsection*{Kj.54}
%<*tr54>
\begin{translation}[Kj54]
Death never spares the body of anyone, anywhere. Therefore, the yogi should make an effort to protect the body. 
\end{translation}
%</tr54>
% MD: better to read tyajaty eva (closer to tyajatyeca E4; some other ηω mss read tyajaty eva). Sa-vipulā with tyajate ca is unlikely.
% JM: Death does not spare (or "never spares" to take eva into account)
% done
%<*sc54>
\begin{sources}[Kj54]
\emph{Śārṅgadharapaddhati} 4599
\begin{variants}
    tyajaty~] nayaty ŚDP
\end{variants}
% \begin{versinnote}
% \tl{śarīraṃ na nayaty eva kālaḥ kasyāpi kutracit/\\+}
% \tl{ataḥ śarīrarakṣārthaṃ yatnaḥ kāryastu yoginā//\\!}
% \end{versinnote}
\end{sources}
%</sc54>

%<*ts54>
\begin{testimonia}[Kj54]
\emph{Yogacintāmaṇi} f.\,108v (\attr Dattātreya), \emph{Haṭhasaṅketacandrikā} f.\,119r (\attr \emph{Yogatattvaprakāśa}) 
\begin{variants}
    tyajaty eva~] tyajed eṣa YCM, tyajed eva HSC
\end{variants}
% \begin{versinnote}
% \tl{śarīraṃ na tyajed eṣa kālaḥ kasyāpi kutracit/\\+}
% \tl{ataḥ śarīrarakṣārthaṃ yatnaḥ kāryas tu yoginā//\\!}
% \end{versinnote}
\end{testimonia}
%</ts54>

%<*cm54>
%\begin{philcomm}[Kj54]
%\end{philcomm}
%</cm54>

\subsection*{Kj.55}
%<*tr55>
\begin{translation}[Kj55]
The yogi should always carefully consider the omens so that, when [the time of death] is known, death does not kill him through deception.
\end{translation}
%</tr55>
% alternatively, ... consider the omens so that, when death is known, it does not kill him through deception. JM: this is better
%??JM: surely it's the yogi's deception? If so this needs rephrasing: through [the yogi’s] deception, death does not... JB: I understood it as the yogi not being deceived by death (i.e., death unexpectedly taking him)... and so translated as above. 

%<*sc55>
\begin{sources}[Kj55]
\emph{Śārṅgadharapaddhati} 4600
% \begin{versinnote}
% \tl{yoginā satataṃ yatnād ariṣṭānāṃ vicāraṇā/\\+}
% \tl{kartavyā yena kālo 'sau jñāto hanti cchalān na tam//\\!}
% \end{versinnote}
\mylb
\end{sources}
%</sc55>

%<*ts55>
\begin{testimonia}[Kj55]
\emph{Yogacintāmaṇi} f.\,108v (\attr Dattātreya), \emph{Haṭhasaṅketacandrikā} f.\,119r (\attr \emph{Yogatattvaprakāśa}) 
\begin{variants}
    'sau jñāto hanti cchalān na~] sāvajñāto na nihanti YCM, sāvajñāto na haṃti HSC
\end{variants}
% \begin{versinnote}
% \tl{yoginā satataṃ yatnād ariṣṭānāṃ vicāraṇā/\\+}
% \tl{kartavyā yena kālo sāvajñāto na nihanti tam//\\!}
% \end{versinnote}
\end{testimonia}
%</ts55>

%<*cm55>
%\begin{philcomm}[Kj55]
%I can't see how the mss. reading ... vicaraṇāt | kartavyo ... can work. So I have emended to the ŚāDhaPa reading.
%\end{philcomm}
%</cm55>

\subsection*{Kj.56}
%<*tr56>
\begin{translation}[Kj56]
Having accurately predicted [the time of] death, he should resort to the place of dissolution and engage in yoga so that the [predicted time of] death comes to nothing.
\end{translation}
%</tr56>
% JM:... he should resort to the place of dissolution and engage in yoga so that the [predicted time of] death comes to nothing. (fruitless doesn't quite work for me)
% JM "his death is fruitless / comes to nothing".
%<*sc56>
\begin{sources}[Kj56]
\emph{Śārṅgadharapaddhati} 4601, cf.~\emph{Mārkaṇḍeyapurāṇa} 40.40
\begin{variants}
    ca kālaṃ~] kālaṃ ca ŚDP MP\sep
    layasthānaṃ sam ŚDP~] abhayasthānam MP\sep
    yogaṃ ŚDP~] yogī MP\sep
    'sya yathāsau jāyate 'phalaḥ ŚDP~] 'sau yathā nāsyāphalo bhavet MP
\end{variants}
% \begin{versinnote}
% \tl{jñātvā kālaṃ ca taṃ samyag layasthānaṃ samāśritaḥ/\\+}
% \tl{yuñjīta yogaṃ kālo 'sya yathāsau jāyate 'phalaḥ//\\!}
% \end{versinnote}

% Cf.~\emph{Mārkaṇḍeyapurāṇa} 40.40
% \begin{versinnote}
% \tl{jñātvā kālaṃ ca taṃ samyag abhayasthānam āśritaḥ/\\+}
% \tl{yuñjīta yogī kālo 'sau yathā nāsyāphalo bhavet/\\!}
% \end{versinnote}
\end{sources}
%</sc56>

%<*ts56>
\begin{testimonia}[Kj56]

\emph{Yogacintāmaṇi} f.\,109r (\attr Dattātreya), \emph{Haṭhasaṅketacandrikā} f.\,119r (\attr \emph{Yogatattvaprakāśa}) 
\begin{variants}
    ca kālaṃ taṃ samyag~] kālaṃ ca taṃ samyag YCM, kālaṃ nijaṃ yogī HSC\sep
    'phalaḥ YCM HSC\vl~] kalaḥ HSC\sep
\end{variants}
% \begin{versinnote}
% \tl{jñātvā kālaṃ ca taṃ samyak layasthānaṃ samāśritaḥ/\\+}
% \tl{yuñjīta yogaṃ kālo 'sya yathāsau jāyate [']phalaḥ//\\!}
% \end{versinnote}
\end{testimonia}
%</ts56>

%<*cm56>
\begin{philcomm}[Kj56]
In \emph{Haṭhapradīpikā} 10.21, Bālakṛṣṇa glosses \emph{layasthānaṃ} (``the place of dissolution'') with \emph{brahmarandhraṃ}. This makes good sense here because in verse 51 the self goes to the tenth door (i.e.~the \emph{brahmarandhra}) to dissolve into the supreme self and, in verse 60, the yogi meditates on dissolving into Śiva, who is on the thousand-petalled lotus, which is usually located at the \emph{brahmarandhra}. 
% 
% JM: I think the commentary is right, it's the place described earlier (51) where the ātman having gone via the tenth door dissolves in paramātman, which the yogi now needs to do so death doesn't take him (but now he doesn't take the final step to videhamukti).  So the pāda means "resorting to the place of dissolution" and goes with yoga yuñjīta
% check note... 
\end{philcomm}
%</cm56>

\subsection*{Kj.57}
%<*tr57>
\begin{translation}[Kj57]
Having adopted the adept's pose (\emph{siddhāsana}), [the yogi] should fill the body with an inhalation. Carefully keeping his spine steady, he should close the ten apertures [of his body].% Sitting in the adept's Pose, he should fill himself up with air. Carefully keeping his spine steady, he should close the ten apertures [of the body].
\end{translation}
%</tr57>
% attentively seems strange in English

%<*sc57>
\begin{sources}[Kj57]
\emph{Śārṅgadharapaddhati} 4602
% \begin{versinnote}
% \tl{baddhasiddhāsano dehaṃ pūrayet prāṇavāyunā/\\+}
% \tl{kṛtvā daṇḍaṃ sthiraṃ buddhyā daśa dvārāṇi rundhayet//\\!}
% \end{versinnote}
\mylb
\end{sources}
%</sc57>

%<*ts57>
\begin{testimonia}[Kj57]
\emph{Yogacintāmaṇi} f.\,109r (\attr Dattātreya), \emph{Haṭhasaṅketacandrikā} f.\,119r (\attr \emph{Yogatattvaprakāśa}) 
\begin{variants}
    rundhayet~] dhārayet YCM, rodhayet HSC
\end{variants}
% \begin{versinnote}
% \tl{baddhasiddhāsano dehaṃ pūrayet prāṇavāyunā/\\+}
% \tl{kṛtvā daṇḍaṃ sthiraṃ buddhyā daśa dvārāṇi dhārayet//\\!}
% \end{versinnote}
\end{testimonia}
%</ts57>

%<*cm57>
%\begin{philcomm}[Kj57]
%\end{philcomm}
%</cm57>

\subsection*{Kj.58}
%<*tr58>
\begin{translation}[Kj58]
He should apply \emph{khecarīmudrā}, and the \emph{jālandhara} [lock] on the neck; and the root lock on [the region] of \emph{apāna}, and the \emph{uḍḍīyāṇa} [lock] on the abdomen.  
\end{translation}
%</tr58>

%<*sc58>
\begin{sources}[Kj58]
\emph{Śārṅgadharapaddhati} 4603
% \begin{versinnote}
% \tl{bandhayet khecarīṃ mudrāṃ grīvāyāṃ ca jalaṃdharam/\\+}
% \tl{apāne mūlabandhaṃ ca uḍḍīyāṇaṃ tathodare//\\!}
% \end{versinnote}
\mylb
\end{sources}
%</sc58>

%<*ts58>
\begin{testimonia}[Kj58]
\emph{Yogacintāmaṇi} f.\,109r (\attr Dattātreya), \emph{Haṭhasaṅketacandrikā} f.\,119v (\attr \emph{Yogatattvaprakāśa}) 
\begin{variants}
    bandhayet YCM~] baddhvā ca HSC
\end{variants}
% \begin{versinnote}
% \tl{bandhayet khecarīṃ mudrāṃ grīvāyāṃ ca jalandharam/\\+}
% \tl{apāne mūlabandhaṃ ca uḍḍiyānaṃ tathodare//\\!}
% \end{versinnote}
\end{testimonia}
%</ts58>

%<*cm58>
%\begin{philcomm}[Kj58]
%\end{philcomm}
%</cm58>

\subsection*{Kj.59}
%<*tr59>
\begin{translation}[Kj59]
Having used pneumatic blows to raise from the base the serpentine power (i.e.~Kuṇḍalinī) situated below, Śiva’s consort who enters Suṣumṇā and pierces the five cakras,[\dots] 
\end{translation}
% JB is this translation clear? JM: how about "He should use blows [from the breath] to raise from the base the serpentine power[, Kuṇḍalinī,] situated below, Śiva's consort who is in the Suṣumṇā and is the piercer of the five cakras... 
%</tr59>
% MD: bhujaṅgīṃ -> bhujagīṃ metri causa.

%<*sc59>
\begin{sources}[Kj59]
\emph{Śārṅgadharapaddhati} 4604
\begin{variants}
    mūlād~] mūla ŚDP
\end{variants}
% \begin{versinnote}
% \tl{utthāpya bhujagīṃ śaktiṃ mūlavātair adhaḥsthitām/\\+}
% \tl{suṣumṇāntargatāṃ pañcacakrāṇāṃ bhedinīṃ śivām//\\!}
% \end{versinnote}
\end{sources}
%</sc59>

%<*ts59>
\begin{testimonia}[Kj59]
\emph{Yogacintāmaṇi} f.\,109r (\attr Dattātreya), \emph{Haṭhasaṅketacandrikā} f.\,119v (\attr \emph{Yogatattvaprakāśa}) 
\begin{variants}
    mūlād ghātair adhaḥsthitām~] mūlādhārāmbujasthitām YCM HSC
\end{variants}
% \begin{versinnote}
% \tl{utthāpya bhujagīṃ śaktiṃ mūlādhārāmbujasthitām/\\+}
% \tl{suṣumṇāntargatāṃ pañca cakrāṇāṃ bhedanīṃ śivām/\\!}
% \end{versinnote}
\end{testimonia}
%</ts59>

%<*cm59>
%\begin{philcomm}[Kj59]
%Should we emend \emph{mūlād ghātair} to \emph{mūlād vātair} (Cf.~ŚāDhaPa)? I suspect that \textit{ghāta} was understood as pneumatic blows, perhaps like \emph{udghāta}. I think \emph{mūlād vātair} is better than \emph{mūlavātair}.
%\end{philcomm}
%</cm59>
%JM ghātair is good, she's not normally said to be woken by the breath

\subsection*{Kj.60}
%<*tr60>
\begin{translation}[Kj60]
[\dots] [the yogi] should lead the \emph{jīva} to the seat of the heart and visualise [Kuṇḍalinī] moving [upwards] together with the higher faculty and mind and dissolving into Śiva in a thousand-petalled lotus.
\end{translation}%JM: lead the \emph{jīva} to the seat in the heart and visualise [Kuṇḍalinī] moving [upwards] together with the higher faculty and mind and dissolving into Śiva in the thousand-petalled lotus.
%</tr60>

%<*sc60>
\begin{sources}[Kj60]
\emph{Śārṅgadharapaddhati} 4605
\begin{variants}
    hṛdyā~] hṛdā ŚDP\sep
    buddhi~] buddhiṃ ŚDP\sep
    sthe~] stha ŚDP\sep
    vicintayet~] sudhāmaye ŚDP
\end{variants}
% \begin{versinnote}
% \tl{jīvaṃ hṛdāśrayaṃ nītvā yāntīṃ buddhiṃ manoyutām/\\+}
% \tl{sahasradalapadmasthaśive līnāṃ sudhāmaye//\\!}
% \end{versinnote}
\end{sources}
%</sc60>

%<*ts60>
\begin{testimonia}[Kj60]
\emph{Yogacintāmaṇi} f.\,109r (\attr Dattātreya), \emph{Haṭhasaṅketacandrikā} f.\,119v (\attr \emph{Yogatattvaprakāśa}) 
\begin{variants}
        hṛdyā~] hṛdā YCM HSC\sep
        sthe YCM~] stha HSC\sep
        vicintayet~] sudhāmaye YCM HSC
\end{variants}
% \begin{versinnote}
% \tl{jīvaṃ hṛdāśrayaṃ nītvā yāntīṃ buddhimanoyutām/\\+}
% \tl{sahasradalamadhyasthe śive līnāṃ sudhāmaye//\\!}
% \end{versinnote}
\end{testimonia}
%</ts60>

%<*cm60>
%\begin{philcomm}[Kj60]
%The testimonia has jīvaṃ instead of bandhaṃ. jīvaṃ makes sense to me (whereas I can't make sense of bandhaṃ), so I've emended. Can we make sense of bandhaṃ?
% JM not me
%The ms. reading of yānti buddhi(ṃ)manojitam does not make sense as there's no plural subject. I've emended to the YCM reading.
%
%The reading °sthe śive līnāṃ in the YCM seems closer to the mss. °sthā śive līnāṃ and °sthāṃ śive līnāṃ, but the ŚāDhaPa's °sthaśive līnāṃ is also possible.
%\end{philcomm}
%</cm60>

\subsection*{Kj.61}
%<*tr61>
\begin{translation}[Kj61]
Then, he should visualise [Kuṇḍalinī] sprinkling and flooding the whole body from the base [upwards] with the nectar of immortality produced by the moon.
\end{translation}%JM: then he should visualise [Kuṇḍalinī] sprinkling and flooding the whole body from the base [upwards] with the nectar of immortality produced by the moon.
%</tr61>

%<*sc61>
\begin{sources}[Kj61]
\emph{Śārṅgadharapaddhati} 4606
\begin{variants}
    tataḥ~] pītvā ŚDP
\end{variants}
% \begin{versinnote}
% \tl{pītvā sudhākarodbhūtam amṛtaṃ tena mūlataḥ/\\+}
% \tl{siñcantīṃ sakalaṃ dehaṃ plāvayantīṃ vicintayet//\\!}
% \end{versinnote}
\end{sources}
%</sc61>

%<*ts61>
\begin{testimonia}[Kj61]
\emph{Yogacintāmaṇi} f.\,109r (\attr Dattātreya), \emph{Haṭhasaṅketacandrikā} f.\,119v (\attr \emph{Yogatattvaprakāśa}) 
% \begin{versinnote}
% \tl{tataḥ sudhākarodbhūtam amṛtaṃ tena mūlataḥ/\\+}
% \tl{siñcantīṃ sakalaṃ dehaṃ plāvayantīṃ vicintayet//\\!}
% \end{versinnote}
% \emph{Haṭhasaṅketacandrikā}
% \begin{versinnote}
% \tl{tataḥ sudhākarodbhūtam amṛtaṃ tena mūlataḥ/\\+}
% \tl{siṃcantī sakalaṃ dehaṃ plāvayantī vicintayet//\\!}
% \end{versinnote}
\end{testimonia}
%</ts61>

%<*cm61>
%\begin{philcomm}[Kj61]
%The common term for moon sudhākara appears to have become jumbled in this transmission: aśrudhārākarodbhūta, aśudhārākarodbhūta. A second hand in J6 has tried to restore sudhā but the compound sudhādhārākarodbhūta is also strange and unattested. I have reverted to the YCM and HSC reading, but the ŚDP reading is the best (but seems more remote from the HP transmission).

%I've emended the mss mūlitaḥ to the testimonia's mūlataḥ.

%Syntax is a bit odd: I've read tataḥ sudhākarodbhūtam amṛtaṃ as one statement. Then, tena mūlataḥ siñcantīṃ sakalaṃ dehaṃ plāvayantīṃ vicintayet... yogī vicintayet kuṇḍalinīṃ siñcantīṃ ... tena amṛtena... mūlataḥ mūlādhārataḥ.
%\end{philcomm}
%</cm61>

\subsection*{Kj.62}
%<*tr62>
\begin{translation}[Kj62]
Then, together with her the yogi attains oneness with Śiva. He becomes full of supreme bliss and gives up even mental activity.% JM: also leaves > gives up even
\end{translation}
%</tr62>

%<*sc62>
\begin{sources}[Kj62]
\emph{Śārṅgadharapaddhati} 4607
% \begin{versinnote}
% \tl{tayā sārdhaṃ tato yogī śivenaikātmatāṃ vrajet/\\+}
% \tl{parānandamayo bhūtvā cidvṛttim api saṃtyajet//\\!}
% \end{versinnote}
\mylb
\end{sources}
%</sc62>

%<*ts62>
\begin{testimonia}[Kj62]

\emph{Yogacintāmaṇi} f.\,109r (\attr Dattātreya), \emph{Haṭhasaṅketacandrikā} f.\,119v (\attr \emph{Yogatattvaprakāśa}) 
\begin{variants}
    bhūtvā YCM HSC\vl~] sūtva HSC
\end{variants}
% \begin{versinnote}
% \tl{tayā sārddhaṃ tato yogī śivenaikātmatāṃ vrajet/\\+}
% \tl{parānandamayo bhūtvā cidvṛttim api saṃtyajet//\\!}
% \end{versinnote}
\end{testimonia}
%</ts62>

% MD: J15 has the missing second line: parānaṃdamayo bhūtvā cidvṛtim api saṃtyajet.
% JB I have reinstated this hemistich because its in all the testimonia and the 10-chapter HP.
%<*cm62>
%\begin{philcomm}[Kj62]
%\end{philcomm}
%</cm62>

\subsection*{Kj.63}
%<*tr63>
\begin{translation}[Kj63]
After that, how can death kill he who is invisible, unmanifest, free from ego, and completely without a conception of the body? 
\end{translation}
%</tr63>

%<*sc63>
\begin{sources}[Kj63]
\emph{Śārṅgadharapaddhati} 4608
% \begin{versinnote}
% \tl{tato lakṣyam anābhāsam ahaṃbhāvavivarjitam/\\+}
% \tl{sarvāṅgakalpanāhīnaṃ kathaṃ kālo nihanti tam//\\!}
% \end{versinnote}
\mylb
\end{sources}
%</sc63>

%<*ts63>
\begin{testimonia}[Kj63]

\emph{Yogacintāmaṇi} f.\,109r (\attr Dattātreya), \emph{Haṭhasaṅketacandrikā} f.\,119v (\attr \emph{Yogatattvaprakāśa}) 
\begin{variants}
    sarvāṅga YCM HSC\vl~] sarvāṅgai HSC
\end{variants}
% \begin{versinnote}
% \tl{tato lakṣyam anābhāsam ahaṃbhāvavivarjitam/\\+}
% \tl{sarvāṅgakalpanāhīnaṃ kathaṃ kālo nihanti tam//\\!}
% \end{versinnote}
\end{testimonia}
%</ts63>

%<*cm63>
%\begin{philcomm}[Kj63]
%The original reading was likely 'lakṣyam anābhāsam, but this has become lakṣamanābhyāsam and lakṣyamano'bhyāsam in the HP transmission. We need bahuvrīhis, and I can't see how the HP readings can make sense. So I've emended to 'lakṣyam anābhāsam.  
%\end{philcomm}
%</cm63>

\subsection*{Kj.64}
%<*tr64>
\begin{translation}[Kj64]
He alone is death, he is Śiva, he is everything and nothing. Who is killed by whom? In that [state], no one dies. 
\end{translation}
%</tr64>
% JM not mriyate in ed?
% I've changed it without reporting it on the assumption that we are standardising the spelling here.

%<*sc64>
\begin{sources}[Kj64]
\emph{Śārṅgadharapaddhati} 4609
\begin{variants}
    eva~] eka ŚDP
\end{variants}

% \begin{versinnote}
% \tl{sa eka kālaḥ sa śivaḥ sa sarvaṃ nāpi kiṃ cana/\\+}
% \tl{kaḥ kena hanyate tatra mriyate nāpi kaś cana//\\!}
% \end{versinnote}
\end{sources}
%</sc64>

%<*ts64>
\begin{testimonia}[Kj64]

\emph{Yogacintāmaṇi} f.\,109r (\attr Dattātreya), \emph{Haṭhasaṅketacandrikā} f.\,119v (\attr \emph{Yogatattvaprakāśa}) 
% \begin{versinnote}
% \tl{sa eva kālas sa śivas sarvaṃ nāpi kiṃ cana/\\+}
% \tl{kaḥ kena hanyate tatra mriyate nāpi kaś cana//\\!}
% \end{versinnote}
\end{testimonia}
%</ts64>

%<*cm64>
%\begin{philcomm}[Kj64]
%\end{philcomm}
%</cm64>

\subsection*{Kj.65}
%<*tr65>
\begin{translation}[Kj65]
Then when the time has passed for the puzzled Death, the yogi is awakened by knowledge, like one who has arisen from sleep.
\end{translation}
%</tr65>
% JM Then when the time has passed for Death [I would capitalise], who is puzzled...

%<*sc65>
\begin{sources}[Kj65]
\emph{Śārṅgadharapaddhati} 4610
\begin{variants}
    bodhaṃ yāti prabodhataḥ~] prabodhaṃ yāti bodhitaḥ ŚDP
\end{variants}
% \begin{versinnote}
% \tl{tato vyatīte samaye kālasya bhrāntirūpiṇaḥ/\\+}
% \tl{yogī suptotthita iva prabodhaṃ yāti bodhitaḥ//\\!}
% \end{versinnote}
\end{sources}
%</sc65>

%<*ts65>
\begin{testimonia}[Kj65]

\emph{Yogacintāmaṇi} f.\,109r (\attr Dattātreya), \emph{Haṭhasaṅketacandrikā} f.\,119v (\attr \emph{Yogatattvaprakāśa}) 
\begin{variants}
    bodhaṃ yāti prabodhataḥ~] prabodhe pratibodhitaḥ YCM, pratibodhe prabodhitaḥ HSC
\end{variants}
% \begin{versinnote}
% \tl{tato vyatīte samaye kālasya bhrāntirūpiṇaḥ/\\+}
% \tl{yogī suptotthita iva prabodhe pratibodhitaḥ//\\!}
% \end{versinnote}
\end{testimonia}
%</ts65>

%<*cm65>
\begin{philcomm}[Kj65]
Death is puzzled (\emph{bhrāntirūpiṇaḥ}) because the yogi has cheated death. 
% JM: add a note to explain
\end{philcomm}
%</cm65>


\subsection*{Kj.66}
%<*tr66>
\begin{translation}[Kj66]
Thus, the yogi becomes perfected, having duly and with extraordinary valour cheated death, the creator of transmigration.%?? samsara for transmigration?
\end{translation}
%</tr66>
% JM: cheated death, the creator of  transmigration
%<*sc66>
\begin{sources}[Kj66]
\emph{Śārṅgadharapaddhati} 4611
% \begin{versinnote}
% \tl{evaṃ siddho bhaved yogī vañcayitvā vidhānataḥ/\\+}
% \tl{kālaṃ kalitasaṃsāraṃ pauruṣeṇādbhutena hi//\\!}
% \end{versinnote}
\mylb
\end{sources}
%</sc66>

%<*ts66>
\begin{testimonia}[Kj66]
\emph{Yogacintāmaṇi} f.\,109r (\attr Dattātreya), \emph{Haṭhasaṅketacandrikā} f.\,119v (\attr \emph{Yogatattvaprakāśa}) 
% \begin{versinnote}
% \tl{evaṃ siddho bhaved yogī vañcayitvā vidhānataḥ/\\+}
% \tl{kālaṃ kalitasaṃsāraṃ pauruṣeṇādbhutena hi //\\!}
% \end{versinnote}
\end{testimonia}
%</ts66>

%<*cm66>
%\begin{philcomm}[Kj66]
%\end{philcomm}
%</cm66>


\subsection*{Kj.67}
%<*tr67>
\begin{translation}[Kj67]
The singular yogi wanders there in the three worlds, seeing the wonder of worldly life, as he pleases, without ego. 
\end{translation}%?? JM: maybe eka eva = unique? Hard to square with eka in the next verse otherwise. JB: singular?
%</tr67>

%<*sc67>
\begin{sources}[Kj67]
\emph{Śārṅgadharapaddhati} 4612
\begin{variants}
    tatra~] tatas ŚDP\sep
    viharaty~] vicaraty ŚDP
\end{variants}
% \begin{versinnote}
% \tl{tatas tribhuvane yogī vicaraty eka eva saḥ/\\+}
% \tl{paśyan saṃsāravaicitryaṃ svecchayā nirahaṃkṛtiḥ//\\!}
% \end{versinnote}
\end{sources}
%</sc67>

%<*ts67>
\begin{testimonia}[Kj67]

\emph{Yogacintāmaṇi} f.\,109r–109v (\attr Dattātreya), \emph{Haṭhasaṅketacandrikā} f.\,119v (\attr \emph{Yogatattvaprakāśa}) 
\begin{variants}
    tatra~] tatas YCM HSC\sep
    viharaty~] vicaraty YCM HSC
\end{variants}

% \begin{versinnote}
% \tl{tatas tribhuvane yogī vicaraty eka eva saḥ/\\+}
% \tl{paśyan saṃsāravaicitryaṃ svecchayānir ahaṃkṛtiḥ//\\!}
% \end{versinnote}
\end{testimonia}
%</ts67>

%<*cm67>
%\begin{philcomm}[Kj67]
%\end{philcomm}
%</cm67>


\subsection*{Kj.68}
%<*tr68>
\begin{translation}[Kj68]
A sun-stone manifests fire through contact with the rays of the sun, not when it is alone. That is an example of a yogi. 
\end{translation}
%</tr68>
% I think the verse means "A sun-stone manifests fire through contact with the rays of the sun, not when it is alone. That is an example of a yogi." But I don't know why it is an example: what does the yogi need to join with? Śiva? Kuṇḍalinī?

%<*sc68>
\begin{sources}[Kj68]
\emph{Śārṅgadharapaddhati} 4613, cf.~\emph{Mārkaṇḍeyapurāṇa} 43.49
\begin{variants}
    dṛṣṭāntas tu sa~] dṛṣṭāntaḥ sa tu ŚDP, upamā sāpi MP
\end{variants}
% \begin{versinnote}
% \tl{yathārkaraśmisaṃyogād arkakānto hutāśanam/\\+}
% \tl{āviṣkaroti naikaḥ san dṛṣṭāntaḥ sa tu yoginaḥ//\\!}
% \end{versinnote}

% Cf.~\emph{Mārkaṇḍeyapurāṇa} 43.49
% \begin{versinnote}
% \tl{yathārkaraśmisaṃyogād arkakānto hutāśanam/\\+}
% \tl{āviṣkaroti naikaḥ san upamā sāpi yoginaḥ//\\!}
% \end{versinnote}
\end{sources}
%</sc68>

%<*ts68>
%\begin{testimonia}[Kj68]

%\end{testimonia}
%</ts68>

%<*cm68>
\begin{philcomm}[Kj68]
The reason the sun-stone seems to be an example of a yogi here is that this stone was used to create fire like a magnifying glass, when it was in contact with the sun, much like the yogi is liberated-in-life when he becomes one with Śiva (62b). The fire-producing quality of the sun-stone (more commonly known as \emph{sūryakānta} or \emph{agnimaṇi}, etc.) is described in \emph{Rājanighaṇṭu} 13.205cd: ``That is a sun-stone from which real fire is emitted upon contact with the sun rays'' (\emph{yaḥ sūryāṃśusparśaniṣṭhyūtavahnir jātyaḥ so 'yaṃ jāyate sūryakāntaḥ}).
\end{philcomm}
%</cm68>

%<*tr68p>
\begin{translation}[Kj68p]
%for iti kālavañcanam
\end{translation}
%</tr68p>

\end{ekdosis}
\end{document}

