\documentclass[10pt]{memoir}
\setstocksize{220mm}{155mm} 	        
\settrimmedsize{220mm}{155mm}{*}	
\settypeblocksize{170mm}{116mm}{*}	
\setlrmargins{18mm}{*}{*}
\setulmargins{*}{*}{1.2}
% \setlength{\headheight}{5pt}
\checkandfixthelayout[lines]
\linespread{1.16}

\setlength{\footmarkwidth}{1.3em}
\setlength{\footmarksep}{0em}
\setlength{\footparindent}{1.3em}
\footmarkstyle{\textsuperscript{#1} }
\usepackage{fnpos}
\makeFNbottom

\usepackage[teiexport=tidy,poetry=verse]{ekdosis}
\usepackage{sanskrit-poetry}

\usepackage[english]{babel}
\usepackage{babel-iast,xparse,xcolor}
\babelfont[iast]{rm}[Renderer=Harfbuzz, Scale=1.5]{AdishilaSan}
\babelfont[english]{rm}[Scale=0.9]{Adobe Text Pro}
\babeltags{dev = iast}
\babeltags{eng = english}

\SetHooks{
	lemmastyle=\bfseries,
	refnumstyle=\selectlanguage{english}\color{blue}\bfseries, 
	}
\newif\ifinapparatus
\DeclareApparatus{default}[
	lang=english,
	sep = {] },
	delim=\hskip 0.75em,
	rule=none,
	]
\DeclareApparatus{notes}[
	lang=english,
	sep = {},
	delim=\hskip 0.75em,
	rule=\rule{0.7in}{0.4pt},
	]

\DeclareShorthand{conj}{\texteng{\emph{conj.}}}{ego}
\DeclareShorthand{emend}{\texteng{\emph{em.}}}{ego}

\setlength{\vrightskip}{-10pt}
\setlength{\vgap}{3mm}
\verselinenumfont{\footnotesize\selectlanguage{english}\normalfont}




%%%%%%%%%%%%%%%%%%%% THE  MSS         %%%%%%%%%%%%%%%%%%%%%%%%%%%

%%% Versions
\DeclareWitness{Vu}{\selectlanguage{english}Vulg}{Vulgate, i.e. Brahmānanda's version}[]           
\DeclareWitness{X}{\selectlanguage{english}X}{TenChapter Version, Jodhpur 02228 and 02225 (ed. Lonavla)}[]
\DeclareWitness{Six}{\selectlanguage{english}Ṣ}{SixChapterVersion, ``6ChapterHPms'', fragment of enlarged text, Jodhpur}[]
% Mss. in Geographical Groups
%%%% Varanasi mss (Sampūrṇānanda mss). V1 is Important
\DeclareWitness{V1}{\selectlanguage{english}V\textsubscript{1}}{Sampurnananda Library Sarasvati Bhavan 30109}[]
        \DeclareHand{V1ac}{V1}{\selectlanguage{english}V\rlap{\textsubscript{1}}\textsuperscript{ac}}[] % added by MD
        \DeclareHand{V1pc}{V1}{\selectlanguage{english}V\rlap{\textsubscript{1}}\textsuperscript{pc}}[] % added by MD
\DeclareWitness{V2}{\selectlanguage{english}V\textsubscript{2}}{Sampurnananda Library Sarasvati Bhavan 29869}[]
\DeclareWitness{V3}{\selectlanguage{english}V\textsubscript{3}}{Sampurnananda Library Sarasvati Bhavan 29899}[]
\DeclareWitness{V4}{\selectlanguage{english}V\textsubscript{4}}{Sampurnananda Library Sarasvati Bhavan 29937}[]
\DeclareWitness{V5}{\selectlanguage{english}V\textsubscript{5}}{Sampurnananda Library Sarasvati Bhavan 29938}[]
\DeclareWitness{V6}{\selectlanguage{english}V\textsubscript{6}}{Sampurnananda Library Sarasvati Bhavan 29991}[]
\DeclareWitness{V8}{\selectlanguage{english}V\textsubscript{8}}{Sampurnananda Library Sarasvati Bhavan 30014}[]
\DeclareWitness{V11}{\selectlanguage{english}V\textsubscript{11}}{Sampurnananda Library Sarasvati Bhavan 30029}[]
\DeclareWitness{V12}{\selectlanguage{english}V\textsubscript{12}}{Sampurnananda Library Sarasvati Bhavan 30030}[]
\DeclareWitness{V13}{\selectlanguage{english}V\textsubscript{13}}{Sampurnananda Library Sarasvati Bhavan 30031}[]
\DeclareWitness{V14}{\selectlanguage{english}V\textsubscript{14}}{Sampurnananda Library Sarasvati Bhavan 30050}[]
\DeclareWitness{V15}{\selectlanguage{english}V\textsubscript{15}}{Sampurnananda Library Sarasvati Bhavan 30051}[]
\DeclareWitness{V15pc}{\selectlanguage{english}V\rlap{\textsubscript{15}}\textsuperscript{pc}\space}{}[]
\DeclareWitness{V16}{\selectlanguage{english}V\textsubscript{16}}{Sampurnananda Library Sarasvati Bhavan 30052}[]
\DeclareWitness{V17}{\selectlanguage{english}V\textsubscript{17}}{Sampurnananda Library Sarasvati Bhavan 30053}[] % added by MD
\DeclareWitness{V16pc}{\selectlanguage{english}V\rlap{\textsubscript{16}}\textsuperscript{pc}\space}{}[]
\DeclareWitness{V18}{\selectlanguage{english}V\textsubscript{18}}{Sampurnananda Library Sarasvati Bhavan 30064}[]
\DeclareWitness{V19}{\selectlanguage{english}V\textsubscript{19}}{Sampurnananda Library Sarasvati Bhavan 30069}[]
\DeclareWitness{V21}{\selectlanguage{english}V\textsubscript{21}}{Sampurnananda Library Sarasvati Bhavan 30104}[]
\DeclareWitness{V22}{\selectlanguage{english}V\textsubscript{22}}{Sampurnananda Library Sarasvati Bhavan 30110}[]
\DeclareWitness{V25}{\selectlanguage{english}V\textsubscript{25}}{Sampurnananda Library Sarasvati Bhavan 30122}[]
\DeclareWitness{V26}{\selectlanguage{english}V\textsubscript{26}}{Sampurnananda Library Sarasvati Bhavan 30123}[]
\DeclareWitness{V28}{\selectlanguage{english}V\textsubscript{28}}{Sampurnananda Library Sarasvati Bhavan 30136}[]
\DeclareWitness{W4}{\selectlanguage{english}W\textsubscript{4}}{Wai 399-6171}[]

%%%%%%%%%%%%%%%%%%%%%%%%%%%%%%%%%
%%% Jammu & Kaschmir
\DeclareWitness{K1}{\selectlanguage{english}K\textsubscript{1}}{Raghunātha Temple Library 4383}[settlement=Jammu]
        \DeclareWitness{K1ac}{\selectlanguage{english}K\rlap{\textsubscript{1}}\textsuperscript{ac}\space}{}[]
        \DeclareWitness{K1pc}{\selectlanguage{english}K\rlap{\textsubscript{1}}\textsuperscript{pc}\space}{}[]
\DeclareWitness{L1}{\selectlanguage{english}L\textsubscript{1}}{SOAS RE 43454}[settlement=Jammu]
% More details? Catalogue number? L1 And C1 very close (and come from same region)
%%%%%%%%%%%%%%%%%%%%%%%%%%%%%%%%
% Jodhpur
% J10 is important
\DeclareWitness{J10}{\selectlanguage{english}J\textsubscript{10}}{MSPP Jodhpur 2230}[]
        \DeclareHand{J10ac}{J10}{\selectlanguage{english}J\rlap{\textsubscript{10}}\textsuperscript{ac}}[] % modified by MD
        \DeclareHand{J10pc}{J10}{\selectlanguage{english}J\rlap{\textsubscript{10}}\textsuperscript{pc}}[] % modified by MD
\DeclareWitness{J1}{\selectlanguage{english}J\textsubscript{1}}{Jodhpur 02231}[]
\DeclareWitness{J2}{\selectlanguage{english}J\textsubscript{2}}{Jodhpur 02232}[]   
\DeclareWitness{J3}{\selectlanguage{english}J\textsubscript{3}}{Jodhpur 02233}[]
\DeclareWitness{J4}{\selectlanguage{english}J\textsubscript{4}}{Jodhpur 02234}[]
        \DeclareWitness{J4ac}{\selectlanguage{english}J\rlap{\textsubscript{4}}\textsuperscript{ac}\space}{MSPP Jodhpur 02234}[]
        \DeclareWitness{J4pc}{\selectlanguage{english}J\rlap{\textsubscript{4}}\textsuperscript{pc}\space}{MSPP Jodhpur 02234}[]
\DeclareWitness{J5}{\selectlanguage{english}J\textsubscript{5}}{Jodhpur 02235}[]  % 4 chapters, 34 jpgs,   long colophon, missing lines in the beginning.
\DeclareWitness{J6ac}{\selectlanguage{english}J\rlap{\textsubscript{6}}\textsubscript{ac}}{Jodhpur 02237}[]  % 4 chapters, 49 jpgs,   1st folio: idaṃ gulābarāyasya
% tulasīrāmaśarmmaṇaḥ putrasya pustakaṃ ...        End: iti śrīsahajānandasantānacintāmaṇisvātmārāmaviracitāyāṃ ..
% saṃvat 1802   (more consistent text)
\DeclareWitness{J6pc}{\selectlanguage{english}J\rlap{\textsubscript{6}}\textsubscript{pc}}{Jodhpur 02237}[] 
\DeclareWitness{J7}{\selectlanguage{english}J\textsubscript{7}}{Jodhpur 02241}[]  % 4 chapters, 41 jpgs
\DeclareWitness{J8}{\selectlanguage{english}J\textsubscript{8}}{Jodhpur 23709}[]  % 4 chapters,  87 jpgs.   saṃvat 1724
\DeclareHand{J8ac}{J8}{\selectlanguage{english}J\rlap{\textsubscript{8}}\textsuperscript{ac}}[]  % changed by MD
\DeclareHand{J8pc}{J8}{\selectlanguage{english}J\rlap{\textsubscript{8}}\textsuperscript{pc}}[]  % changed by MD
\DeclareWitness{J9}{\selectlanguage{english}J\textsubscript{9}}{Jodhpur 02224}[]  %  fragment, 20 jpgs.
\DeclareWitness{J11}{\selectlanguage{english}J\textsubscript{11}}{Jodhpur 23532}[]
\DeclareWitness{J12}{\selectlanguage{english}J\textsubscript{12}}{Jodhpur 18552}[] 
\DeclareWitness{J13}{\selectlanguage{english}J\textsubscript{13}}{Jodhpur 02229}[]  %  5 chapters, 93 jpgs.
\DeclareWitness{J14}{\selectlanguage{english}J\textsubscript{14}}{Jodhpur 02239}[]  %  4 chapters
\DeclareWitness{J15}{\selectlanguage{english}J\textsubscript{15}}{Jodhpur 9732A}[]
\DeclareWitness{J17}{\selectlanguage{english}J\textsubscript{17}}{Jodhpur 3013}[]
% Haṭhapradīpikā with (non-Sanskrit) Bhāṣya RORI Jodhpur ACC.NO.18552
%  Haṭhapradīpikā with (non-Sanskrit) commentary, RORI Alwar 952, 4 chapters,  colophon of the comm:
% iti śrīlāhorīmiśravrajabhūṣanaviracitāyāṃ bhāvārthadīpikāyāṃ caturthodhyāya ..    
%  Haṭhapradīpikā (5 chapter) MSPP Jodhpur ACC.NO.02229/

%%%%%%%%%%        Bodleian, Oxford
\DeclareWitness{B1}{\selectlanguage{english}B\textsubscript{1}}{Bodleian Library No. d.457(8)}[settlement=Oxford]
\DeclareWitness{B2}{\selectlanguage{english}B\textsubscript{2}}{Bodleian Library No. d.458(1)}[settlement=Oxford]
\DeclareWitness{B3}{\selectlanguage{english}B\textsubscript{3}}{Bodleian Library No. d.458(9)}[settlement=Oxford]

%%%%%%%%%%%   Chandigarh
\DeclareWitness{C1}{\selectlanguage{english}C\textsubscript{1}}{Lalchand M-2080}[]%L1 And C1 very close (and come from same region)
\DeclareWitness{C2}{\selectlanguage{english}C\textsubscript{2}}{Lalchand M-6065}[]
\DeclareWitness{C3}{\selectlanguage{english}C\textsubscript{3}}{Lalchand M-1293}[]
\DeclareWitness{C4}{\selectlanguage{english}C\textsubscript{4}}{Lalchand M-2081}[]
\DeclareWitness{C4ac}{\selectlanguage{english}C\rlap{\textsubscript{4}}\textsuperscript{ac}\space}{}[]
\DeclareWitness{C4pc}{\selectlanguage{english}C\rlap{\textsubscript{4}}\textsuperscript{pc}\space}{}[]
\DeclareWitness{C5}{\selectlanguage{english}C\textsubscript{5}}{Lalchand M-2082}[]%doesn't have chapter 1
\DeclareWitness{C6}{\selectlanguage{english}C\textsubscript{6}}{Lalchand M-2089}[]
\DeclareWitness{C7}{\selectlanguage{english}C\textsubscript{7}}{Lalchand M-6494}[]
\DeclareWitness{C8}{\selectlanguage{english}C\textsubscript{8}}{Lalchand M-2091}[]
\DeclareWitness{C8pc}{\selectlanguage{english}C\rlap{\textsubscript{8}}\textsuperscript{pc}\space}{}[]
\DeclareWitness{C9}{\selectlanguage{english}C\textsubscript{9}}{Lalchand M-4530}[]

% %%%%%%%%%%        Nepalese
\DeclareWitness{N1}{\selectlanguage{english}N\textsubscript{1}}{NGMPP A1400-2}[]
\DeclareWitness{N2}{\selectlanguage{english}N\textsubscript{2}}{NGMPP B 39-19}[]
\DeclareWitness{N3}{\selectlanguage{english}N\textsubscript{3}}{NGMPP B 62-20}[]
\DeclareWitness{N5}{\selectlanguage{english}N\textsubscript{5}}{NGMPP A60-15 + A61-1}[]
\DeclareWitness{N6}{\selectlanguage{english}N\textsubscript{6}}{NGMPP A61-6}[]
\DeclareWitness{N9}{\selectlanguage{english}N\textsubscript{9}}{NGMPP A62-33}[]
\DeclareWitness{N10}{\selectlanguage{english}N\textsubscript{10}}{NGMPP A62-37}[]
\DeclareWitness{N11}{\selectlanguage{english}N\textsubscript{11}}{NGMPP A63-15}[]
\DeclareWitness{N12}{\selectlanguage{english}N\textsubscript{12}}{NGMPP A939-19}[]
\DeclareWitness{N13}{\selectlanguage{english}N\textsubscript{13}}{NGMPP A1378-18}[]
\DeclareWitness{N16}{\selectlanguage{english}N\textsubscript{16}}{NGMPP B39-20}[]
\DeclareWitness{N17}{\selectlanguage{english}N\textsubscript{17}}{NGMPP B 111-10}[]
\DeclareWitness{N18}{\selectlanguage{english}N\textsubscript{18}}{NGMPP E 929-3}[]
\DeclareWitness{N19}{\selectlanguage{english}N\textsubscript{19}}{NGMPP E-1528-1 / E-1527-7(4)}[]
\DeclareWitness{N20}{\selectlanguage{english}N\textsubscript{20}}{NGMPP E 2037-13 }[]
\DeclareWitness{N21}{\selectlanguage{english}N\textsubscript{21}}{NGMPP E 2097-31}[]
\DeclareWitness{N22}{\selectlanguage{english}N\textsubscript{22}}{NGMPP G 4-4}[]
\DeclareWitness{N23}{\selectlanguage{english}N\textsubscript{23}}{NGMPP G 25-2}[]
\DeclareWitness{N24}{\selectlanguage{english}N\textsubscript{24}}{NGMPP G 190-16}[]
\DeclareWitness{N24ac}{\selectlanguage{english}N\rlap{\textsubscript{24}}\textsuperscript{ac}\space}{}[]
\DeclareWitness{N24pc}{\selectlanguage{english}N\rlap{\textsubscript{24}}\textsuperscript{pc}\space}{}[]

\DeclareWitness{P28}{\selectlanguage{english}P\textsubscript{28}}{BORI 399-1895-1902}[]

%%%%%   Mysore
\DeclareWitness{M1}{\selectlanguage{english}M\textsubscript{1}}{P-5682/4}[]
%%%%%   Tübingen
\DeclareWitness{Tü}{\selectlanguage{english}Tü}{Ma I 339}[]
%%%%%%%%%%
\DeclareWitness{YC}{\selectlanguage{english}YC}{Yogacintāmaṇi}[]
\DeclareWitness{ceteri}{\selectlanguage{english}cett.}{ceteri}[]

%%%%%%%%%% Mss with Commentary
\DeclareWitness{A1}{\selectlanguage{english}A\textsubscript{1}}{Alwar 952}[]


%%%%%%%%%%%%%%%%%%%%%%%%%%%%%%%%%%%%%%%%%%%
%List of all Sigla:
%A1,B1,B2,B3,C1,C2,C3,C4,C6,C7,C8,C9,J1,J2,J3,J4,J10,J13,J14,J15,J17,L1,M1,N3,N5,N6,N9,N10,N11,N12,N13,N16,N17,N19,N20,N21,N22,N23,N24,Tü,V1,V2,V3,V4,V5,V6,V8,V11,V19,V22,V26,Vu
%%%%%%%%%%%%%%%%%%%%%%%%%%%%%%%%%%%%%%%%%%%

\DeclareShorthand{x}{\selectlanguage{english}δ}{J10,J17,N17,P28,W4}


%%% Local Variables:
%%% mode: latex
%%% TeX-master: t
%%% End:


%%%%%                   Abbreviation for the printed apparatus,        xml interface needed
%%%%%                   (synonyms in same line)

% Macro for Editing Abbrevs.
%\def\om{\textrm{\footnotesize \textit{omitted in}\ }} %prints om. for omitted in apparatus
%\def\korr{\textrm{\footnotesize \textit{em.}\ }} %prints em. for emended in apparatus
%\def\conj{\textrm{\footnotesize \textit{conj.}\ }} %prints conj. for conjectured in apparatus


\def\eyeskip{\textrm{{ab.\,oc. }}}   
\def\aberratio{\textrm{{ab.\,oc. }}}
\def\ad{\textrm{{ad}}}   
\def\add{\textrm{{add.\ }}}
\def\ann{\textrm{{ann.\ }}}
\def\ante{\textrm{{ante }}}
\def\post{\textrm{{post }}}
%\def\ceteri{cett.\,}             % for simplifying the apparatus in print                  
\def\codd{\textrm{{codd.\ }}}   %  the same
\def\conj{\textrm{{coni.\ }}}  
\def\coni{\textrm{{coni.\ }}}
\def\contin{\textrm{{contin.\ }}}
\def\corr{\textrm{{corr.\ }}}
\def\del{\textrm{{del.\ }}}
\def\dub{\textrm{{ dub.\ }}}
\def\emend{\textrm{{emend.\ }}}
\def\expl{\textrm{{explic.\ }}}   
\def\explicat{\textrm{{explic.\ }}}
\def\fol{\textrm{{fol.\ }}}         
\def\foll{\textrm{{foll.\ }}}
\def\gloss{\textrm{{glossa ad }}}
\def\ins{\textrm{{ins.\ }}}          \def\inseruit{\textrm{{ins.\ }}}
\def\im{{\kern-.7pt\lower-1ex\hbox{\textrm{\tiny{\emph{i.m.}}}\kern0pt}}}
\def\inmargine{{\kern-.7pt\lower-.7ex\hbox{\textrm{\tiny{\emph{i.m.}}}\kern0pt}}}
\def\intextu{{\kern-.7pt\lower-.95ex\hbox{\textrm{\tiny{\emph{i.t.}}}\kern0pt}}}%\textrm{\scriptsize{i.t.\ }}}               
\def\indist{\textrm{{indis.\ }}}          \def\indis{\textrm{{indis.\ }}}
\def\iteravit{\textrm{{iter.\ }}}          \def\iter{\textrm{{iter.\ }}}  
\def\lectio{\textrm{{lect.\ }}}             \def\lec{\textrm{{lect.\ }}}
\def\leginequit{\textrm{{l.n. }}}         \def\legn{\textrm{{l.n. }}}         \def\illeg{\textrm{{l.n. }}}
\def\om{\textrm{{om. }}}
\def\primman{\textrm{{pr.m.}}}
\def\prob{\textrm{{prob.}}}
\def\rep{\textrm{{repetitio }}}
% \def\secundamanu{\textrm{\scriptsize{s.m.}}}
% \def\secm{{\kern-.6pt\lower-.91ex\hbox{\textrm{\tiny{\emph{s.m.}}}\kern0pt}}}%   \textrm{\scriptsize{s.m.}}}
\def\sequentia{\textrm{{seq.\,inv.\ }}}         \def\seqinv{\textrm{{seq.\,inv.\ }}} \def\order{\textrm{{seq.\,inv.\ }}}
\def\supralineam{{\kern-.7pt\lower-.91ex\hbox{\textrm{\tiny{\emph{s.l.}}}\kern0pt}}} %\textrm{\scriptsize{s.l.}}}
\def\interlineam{{\kern-.7pt\lower-.91ex\hbox{\textrm{\tiny{\emph{s.l.}}}\kern0pt}}}   %\textrm{\scriptsize{s.l.}}}
\def\vl{\textrm{v.l.}}   \def\varlec{\textrm{v.l.}} \def\varialectio{\textrm{v.l.}}
\def\vide{\textrm{{cf.\ }}}           \def\cf{\textrm{{cf.\ }}}
\def\videtur{\textrm{{vid.\,ut}}}
\def\crux{\textup{[\ldots]} }
\def\cruxx{\textup{[\ldots]}}
\def\unm{\textit{unm.}}        % unmetrical
%%%%%%%%%%%%%%%%%%%%%%%%%%%%%%%%%%%%



%%% Local Variables:
%%% mode: latex
%%% TeX-master: t
%%% End:

% additions/changes 2024-07-04 mm:
\TeXtoTEIPat{\lb}{<lb/>}
\TeXtoTEIPat{\begin {quote}}{<q>}
  \TeXtoTEIPat{\end {quote}}{</q>}
\TeXtoTEIPat{\begin {enumerate}}{<list rend="numbered">}
  \TeXtoTEIPat{\end {enumerate}}{</list>}
\TeXtoTEI{item}{item}

% additions/changes 2024-07-01 mm:
\TeXtoTEIPat{\unavbl {#1}}{<note type="foliolost">Folio lost in <ref>#1</ref></note>}
\TeXtoTEIPat{\NotIn {#1}}{<note type="omission">Omitted in <ref>#1</ref></note>}
\TeXtoTEI{graus}{span}[type="altrec"]
\TeXtoTEI{grau}{span}[type="altrec"]

% addition 2024-03-15 MD
\TeXtoTEI{manuref}{}

\TeXtoTEIPat{\alphaOne}{α<hi rend="sub">1</hi>}% N3
\TeXtoTEIPat{\alphaTwo}{α<hi rend="sub">2</hi>}% J5
\TeXtoTEIPat{\alphaThree}{α<hi rend="sub">3</hi>}% G4
\TeXtoTEIPat{\betaOne}{β<hi rend="sub">1</hi>}% P11
\TeXtoTEIPat{\betaTwo}{β<hi rend="sub">2</hi>}% C6
\TeXtoTEIPat{\betaOmega}{β<hi rend="sub">ω</hi>}% V3
\TeXtoTEIPat{\gammaOne}{γ<hi rend="sub">1</hi>}% N23
\TeXtoTEIPat{\gammaTwo}{γ<hi rend="sub">2</hi>}% J7
\TeXtoTEIPat{\deltaOne}{δ<hi rend="sub">1</hi>}% V19
\TeXtoTEIPat{\deltaTwo}{δ<hi rend="sub">2</hi>}% K3
\TeXtoTEIPat{\deltaThree}{δ<hi rend="sub">3</hi>}% C7
\TeXtoTEIPat{\deltaOmega}{δ<hi rend="sub">ω</hi>}% J6
\TeXtoTEIPat{\epsilonOne}{ε<hi rend="sub">1</hi>}% P15
\TeXtoTEIPat{\epsilonTwo}{ε<hi rend="sub">2</hi>}% N19
\TeXtoTEIPat{\epsilonThree}{ε<hi rend="sub">3</hi>}% V15
\TeXtoTEIPat{\epsilonFour}{ε<hi rend="sub">4</hi>}% J11
\TeXtoTEIPat{\epsilonOmega}{ε<hi rend="sub">ω</hi>}% N26
\TeXtoTEIPat{\etaOne}{η<hi rend="sub">1</hi>}% V1
\TeXtoTEIPat{\etaTwo}{η<hi rend="sub">2</hi>}% J10
\TeXtoTEIPat{\etaOmega}{η<hi rend="sub">ω</hi>}% N9

% addition 2023-12-11 MD:
\TeXtoTEIPat{\begin {metre}[#1]}{<note type="metre" target="##1">}
\TeXtoTEIPat{\end {metre}}{</note>}
\TeXtoTEIPat{\texttheta}{θ}

% change 2023-12-05 mm
\TeXtoTEI{teimute}{} 

% changes/additions 2023-11-27 MM:
\TeXtoTEIPat{\medialink {#1}{#2}}{<ref target="resources/#2">#1</ref>}

% changes/additions 2023-10-25 MM:
% new Sigla
\TeXtoTEIPat{\textAlpha}{Α}
\TeXtoTEIPat{\textalpha}{α}
\TeXtoTEIPat{\textBeta}{Β}
\TeXtoTEIPat{\textbeta}{β}
\TeXtoTEIPat{\textGamma}{Γ}
\TeXtoTEIPat{\textgamma}{γ}
\TeXtoTEIPat{\textDelta}{Δ}
\TeXtoTEIPat{\textdelta}{δ}
\TeXtoTEIPat{\textEpsilon}{Ε}
\TeXtoTEIPat{\textepsilon}{ε}
\TeXtoTEIPat{\textEta}{Η}
\TeXtoTEIPat{\texteta}{η}
\TeXtoTEIPat{\textChi}{Χ}
\TeXtoTEIPat{\textchi}{χ}
\TeXtoTEIPat{\textOmega}{Ω}
\TeXtoTEIPat{\textomega}{ω}

%new environments
\TeXtoTEIPat{\begin {postmula}[#1]}{<div type="postmula" xml:id="#1">} %%% changed 2024-07-01 mm
  \TeXtoTEIPat{\end {postmula}}{</div>}  %%% changed 2024-07-01 mm
  
\TeXtoTEIPat{\begin {altpostmula}[#1]}{<div type="altrec"><div type="postmula" xml:id="#1">} %%% added 2024-07-03 md
  \TeXtoTEIPat{\end {altpostmula}}{</div></div>} %%% added 2024-07-03 md

\TeXtoTEIPat{\begin {altava}[#1]}{<div type="altrec"><div type="avataranika" xml:id="#1">} %%% changed 2024-07-01 mm
  \TeXtoTEIPat{\end {altava}}{</div></div>} %%% changed 2024-07-01 mm

\TeXtoTEIPat{\sgwit {#1}}{<note type="inlineref"><ref>#1</ref></note>}

% changes/additions 2023-10-12 MM:
\TeXtoTEIPat{\\.}{}

% changes/additions 2023-08-15 MD:
\TeXtoTEIPat{\lineom {#1}{#2}}{<note type="omission">#1 omitted in <ref>#2</ref></note>}
%\TeXtoTEIPat{\startgray}{} %%% changed 2023-12-05 mm; not used 2024-03-26 MD
%\TeXtoTEIPat{\endgray}{} %%% changed 2023-12-05 mm; not used 2024-03-26 MD

% additions/changes 2023-06-05 mm:
%\TeXtoTEIPat{\lineom {#1}}{<note type="omission">Line omitted in <ref>#1</ref></note>}

% additions 2023-04-16 MD:
\TeXtoTEIPat{\,}{ }

% additions 2023-04-13 mm:
\TeXtoTEIPat{\begin {versinnote}}{<lg>}
  \TeXtoTEIPat{\end {versinnote}}{</lg>}

% additions 2023-04-05 MD:
\TeXtoTEIPat{\begin {testimonia}[#1]}{<note type="testimonia" target="##1">}
  \TeXtoTEIPat{\end {testimonia}}{</note>}
\TeXtoTEI{devnote}{s}[xml:lang="sa-deva"]

% app in philcomm und testimonia %%% added MM 2023-12-02
\TeXtoTEI{var}{note}[type="appinnote"]


\TeXtoTEI{anm}{note}[type="memo"] %% change 2023-04-16 MD
\TeXtoTEI{Anm}{note}[type="memo"] %% change 2023-12-05 MM
\TeXtoTEIPat{\startverse}{} %%% marked for change 2023-04-13 mm
\TeXtoTEIPat{\endverse}{} %%% marked for change 2023-04-13 mm
\TeXtoTEIPat{\newpage}{}
\TeXtoTEIPat{\marmas}{ } % changed 2024-03-17 MD
\TeXtoTEIPat{\marma}{}
\TeXtoTEIPat{\vin}{} % added by MD 2023-11-14

%%% modify environments and commands
%%% TEI mapping
% additions/changes 2022-06-07 mm:
\TeXtoTEIPat{ \& }{ &amp; }

% additions/changes 2022-06-01 mm:
\TeXtoTEI{skp}{seg}[type="deva-ignore"]
\TeXtoTEI{skm}{seg}[type="ltn-ignore"]

\TeXtoTEIPat{\rlap {#1}}{#1}

% additions/changes 2022-04-06 mm:
%\TeXtoTEI{sgwit}{ref}
\TeXtoTEI{textdev}{s}[xml:lang="sa-deva"]
\TeXtoTEIPat{\begin {col}[#1]}{<div type="colophon" xml:id="#1">}
  \TeXtoTEIPat{\end {col}}{</div>}
\TeXtoTEIPat{\begin {ava}[#1]}{<div type="avataranika" xml:id="#1">} %%% changed 2024-07-01 mm
  \TeXtoTEIPat{\end {ava}}{</div>} %%% changed 2024-07-01 mm
												   
\TeXtoTEIPat{\outdent}{}
\TeXtoTEIPat{\startaltrecension}{} %%% changed 2023-12-05 mm
\TeXtoTEIPat{\endaltrecension}{} %%% changed 2023-12-05 mm
\TeXtoTEIPat{\startaltnormal}{} % added by MD 2023-11-14 %%% changed 2023-12-05 mm
\TeXtoTEIPat{\endaltnormal}{} % added by MD 2023-11-14 %%% changed 2023-12-05 mm
\TeXtoTEIPat{\begin {alttlg}[#1]}{<div type="altrec"><lg xml:id="#1">}
  \TeXtoTEIPat{\end {alttlg}}{</lg></div>}



% additions/changes 2022-03-12 mm:
\TeXtoTEIPat{\begin {tlg}[#1]}{<lg xml:id="#1">}
  \TeXtoTEIPat{\end {tlg}}{</lg>}

\TeXtoTEIPat{\begin {translation}[#1]}{<note type="translation" target="##1">}
  \TeXtoTEIPat{\end {translation}}{</note>}
\TeXtoTEIPat{\begin {philcomm}[#1]}{<note type="philcomm" target="##1">}
  \TeXtoTEIPat{\end {philcomm}}{</note>}
\TeXtoTEIPat{\begin {sources}[#1]}{<note type="sources" target="##1">}
  \TeXtoTEIPat{\end {sources}}{</note>}


\TeXtoTEIPat{\begin {marma}[#1]}{<note type="marma" target="##1">}
  \TeXtoTEIPat{\end {marma}}{</note>}

\TeXtoTEIPat{\begin {jyotsna}[#1]}{<note type="jyotsna" target="##1">}
  \TeXtoTEIPat{\end {jyotsna}}{</note>}

\EnvtoTEI{description}{list}
\EnvtoTEI{itemize}{list}
\TeXtoTEIPat{\item [#1]}{<label>#1</label>\item}

\TeXtoTEI{tl}{l}
\TeXtoTEI{myfn}{note}[type="myfn"]
\TeXtoTEIPat{\getsiglum {#1}}{<ref target="##1"/>}

\TeXtoTEI{SetLineation}{}
\TeXtoTEI{noindent}{}
\TeXtoTEI{subsection*}{}

\TeXtoTEI{rlap}{}

% end additions/changes
% \TeXtoTEIPat{\skp {#1}}{#1}
% \TeXtoTEIPat{\skm {#1}}{}

\TeXtoTEIPat{\begin {prose}}{<p>}
  \TeXtoTEIPat{\end {prose}}{</p>}

\TeXtoTEIPat{\begin {tlate}}{<p>}
  \TeXtoTEIPat{\end {tlate}}{</p>}

\TeXtoTEI{emph}{hi}
\TeXtoTEI{bigskip}{}
% \TeXtoTEI{/}{|}
\TeXtoTEI{tl}{l}
\TeXtoTEIPat{english}{}
%\TeXtoTEIPat{-}{ } %% change 2023-04-16 MD
%\TeXtoTEIPat{°}{} %% change 2023-04-16 MD
\TeXtoTEIPat{\textcolor {#1}{#2}}{<hi rend="#1">#2</hi>}

% \TeXtoTEIPat{\eyeskip}{}
% \TeXtoTEIPat{\aberratio}{}
% \TeXtoTEIPat{\ad}{}
\TeXtoTEIPat{\add}{<hi rend="italic">add.</hi>} %% change 2023-04-16 MD
% \TeXtoTEIPat{\ann}{}
\TeXtoTEIPat{\ante}{<hi rend="italic">ante</hi> } %% change 2023-04-16 MD
\TeXtoTEIPat{\post}{<hi rend="italic">post</hi> } %% change 2023-04-16 MD
% \TeXtoTEIPat{\codd}{}
% \TeXtoTEIPat{\conj }{}
% \TeXtoTEIPat{\contin}{}
% \TeXtoTEIPat{\corr}{}
% \TeXtoTEIPat{\del}{}
% \TeXtoTEIPat{\dub}{}
% \TeXtoTEIPat{\emend }{}
% \TeXtoTEIPat{\expl}{}
% \TeXtoTEIPat{\ȩxplicat}{}
% \TeXtoTEIPat{\fol}{}
% \TeXtoTEIPat{\gloss}{}
% \TeXtoTEIPat{\ins}{}
% \TeXtoTEIPat{\im}{}
% \TeXtoTEIPat{\inmargine}{}
% \TeXtoTEIPat{\intextu}{}
% \TeXtoTEIPat{\indist}{}
% \TeXtoTEIPat{\iteravit}{}
% \TeXtoTEIPat{\lectio}{}
% \TeXtoTEIPat{\leginequit}{}
% \TeXtoTEIPat{\legn}{}
% \TeXtoTEIPat{\illeg}{<hi rend="italic">illeg.</hi>}
\TeXtoTEIPat{\illeg}{<gap reason="illeg."/>} %%% change 2023-04-11 mm
% \TeXtoTEIPat{\om}{<hi rend="italic">om.</hi>}
\TeXtoTEIPat{\om}{<gap reason="om."/>} %%% change 2023-04-11 mm
% \TeXtoTEIPat{\primman}{}
% \TeXtoTEIPat{\prob}{}
% \TeXtoTEIPat{\rep}{}
% \TeXtoTEIPat{\sequentia}{}
% \TeXtoTEIPat{\supralineam}{}
% \TeXtoTEIPat{\interlineam}{}
\TeXtoTEIPat{\vl}{<hi rend="italic">v.l.</hi>}
% \TeXtoTEIPat{\vide}{}
% \TeXtoTEIPat{\videtur}{}
% \TeXtoTEIPat{\crux}{}
% \TeXtoTEIPat{\cruxxx}{}
\TeXtoTEIPat{\unm}{<hi rend="italic">unm.</hi>}
\TeXtoTEIPat{\lacuna}{<gap reason="lac."/>} % addition 2024-03-24 MD
\TeXtoTEIPat{\lost}{<gap reason="lost"/>} % addition 2024-06-24 MD

% List of Scholars
\DeclareScholar{nos}{nos}[
forename=HPP,
surname=Team]

% Nullify \selectlanguage in TEI as it has been used in
% \DeclareWitness but should be ignored in TEI.
\TeXtoTEI{selectlanguage}{}



\NewDocumentCommand{\skp}{m}{}
\NewDocumentCommand{\skm}{m}{\unless\ifinapparatus#1-\fi}

\SetTEIxmlExport{autopar=false}
\NewDocumentEnvironment{tlg}{O{}}{
	\begin{ekdverse}
	\indentpattern{0000}}{
	\end{ekdverse}
	\vskip 0.75\baselineskip}
\NewDocumentEnvironment{alttlg}{O{}}{}{}
\NewDocumentCommand{\tl}{m}{#1}

%%%%%%

\def\startaltrecension#1{
  \stopvline
  \begin{ekdverse}[type=altrecension]
    \indentpattern{0000} 
    \begin{patverse*}
      \color{gray}
      \setvnum{#1}}
\def\endaltrecension{
  \end{patverse*}
  \end{ekdverse}
  \vskip 0.75\baselineskip
  \startvline}

%%%%%%

\newcommand{\myfn}[1]{\footnote{\texteng{#1}}}
\renewcommand{\thefootnote}{\texteng{\arabic{footnote}}}
\newcommand{\devnote}[1]{\selectlanguage{iast}{\scriptsize #1}\selectlanguage{english}}
\newcommand{\outdent}{\hspace{-\vgap}}
\newcommand{\sgwit}[1]{{\small (\getsiglum{#1})}\selectlanguage{iast}}
\newcommand{\NotIn}[1]{\texteng{\footnotesize (om. \getsiglum{#1})}\selectlanguage{iast}}

\def\om{\emph{om.}} % \!}
\def\illeg{\emph{illeg.}} %\!}
\def\unm{\emph{unm.\:}}
\def\recte{\texteng{r.\:}}
\def\for{\texteng{for }}
\def\sic{\emph{sic}}

\makepagestyle{HPed}
\makeoddhead{HPed}{\small\texteng{HP1 OldMss}}{}{\small\texteng{\today}}
\makeevenhead{HPed}{\small\texteng{HP1 OldMss}}{}{\small\texteng{\today}}
\makeoddfoot{HPed}{}{\small\texteng{\thepage}}{}
\makeevenfoot{HPed}{}{\small\texteng{\thepage}}{}

\newcommand{\myfn}[1]{\footnote{\texteng{#1}}}
\setlength{\footmarkwidth}{1.3em}
\setlength{\footmarksep}{0em}
\setlength{\footparindent}{1.3em}
\footmarkstyle{\textsuperscript{#1} }
\usepackage{fnpos}
\makeFNbottom

% additions/changes 2022-03-12 mm:
\SetTEIxmlExport{autopar=false}
\NewDocumentEnvironment{marma}{O{}}{}{}
\def\sthana{}

\begin{document}
\pagestyle{HPed} %
\begin{ekdosis}
 \SetLineation{lineation = none,}

 \chapter{Variants and Versions}

\sthana{24a}

 \begin{marma}[hp01_024]
        \begin{description}
%        \item[(ku(r)k(k)uṭāsana-)]{}
        \item[-vat kṛtvā]{V1, N22}
        \item[-vat kratvā]{P6}
        \item[-bandhastho]{A1, J8, L1, M1, V6, B1, B2 (in marg.), B3, Bo2, C1, C2, C3, C4
           (\emph{madhyastho} is also given above), C6, C8, J1, J3, J6, J13, J14, K1, Ko, LD1, LD2, N1, V2, N2,
           V3 (about \emph{stho}), N4a, V8, N5, N8, N9, N11, V21, N13, N16, N18, N19, N20, N21, N24,
           N25, N26, P1, P2, P3, P4 (\emph{... sanaṃ}), P7, Tue,V17, V14, R3, R5, V13, V10}
        \item[-bandhasthe]{J7, V19}
        \item[-badhascho]{Ba, Bo3, V28, V22}
        \item[-baṃdhascho]{J5, J11, J15, V15}
        \item[-baddhastho]{N12}
        \item[-bandhastu]{Bo1}
        \item[-madhyastho]{C9, J10, J12, J16 (\emph{baṃ} is given in margin), J17, N17, V4, N6, N7,
           N10, V20, V18, P5, P8, V12, LD1}
        \item[-madhyasthā]{P9}
        \item[-madhyasthau]{V16}
        \item[-baṃdhasthau]{N23}
        \item[-badhestu]{J2}
        \item[(unavailable/illegible)]{Ba5, C7, J4 (\emph{gudaṃ nibadhya gulphābhyāṃ}), C5, R2}
        \end{description}
\end{marma}


\begin{marma}[hp01_027]

 \sthana{27d}

 \begin{description}
\item[jaṭharapradīptaṃ]        V6 (jaraṭha), B1, B2, B3, Bo1, C1, J13("pradīpta"), N1, V22, N13, N21 (No anusvara over "ṭha" of "pīṭha"), P2, R3(...pīṭha)
\item[jaṭharapradīpti]        Ba5, V10
\item[jaṭharapradiptaṃ]        P4(...piṭhaṃ)
\item[jaṭharaprakṛṣtaṃ]        V14
\item[jaṭharapradīpaṃ]        P6("piṭhaṃ" changed to "pīṭhaṃ", "dipaṃ" changed to "dīpaṃ")
\item[jaṭharapradīptiṃ]        Bo3, Ko, N4a, Tue, R5
\item[jaṭharapracaṇḍaṃ]        K1, V1, N22, V15
\item[jaṭharaṃ pracaṇḍaṃ]        J11("jvalanaṃ pradīptaṃ" is also given above),
\item[jaṭharapracaṃḍa]        J5
\item[matsyendra-vīra-jaṭhara-pracaṇḍaṃ]              J4, N19 (V1 (above) also has "jaṭhara-pracaṇḍaṃ") [vīra-ja is unmetrical?]
\item[budhaṃ pracaṇḍa]        V8
\item[jaṭharapravṛddha]        J10, J17, V2, N8, N11
\item[jaṭharapravṛddhau]        J15
\item[jaṭharapravṛddhaṃ]        N24
\item[jaṭharapravṛddhiṃ]        C1
\item[jaṭharaprabuddhaṃ]        C8, J14, N2, N12, V21, N23, V12
\item[jaṭharaprabuddha]        Ba,
\item[jaṭharaprabuddhau]        C4 ("jvalanapradīptam" is also given above), L1, V19, V28, N5, N25, P3
\item[jaṭharaprabodhaṃ]        A1, C6, P7,
\item[...bodhaṃ]           R2(only "bodhaṃ" available)
\item[jaṭharāprabodhaṃ]        V20
\item[jaṭharaṃ prabadhaṃ]        J2
\item[jvalanapradīptaṃ]        LD1, Bo2, C3, C9, J1, J3, J6, J8, N17, V3, V4, N6 ("na" is written as a later addition), N9, N18 (No anusvāra visible on "ṭha"), V18, N26, P1, P5,V17, P8('pradī' is correction at bottom of folio), V13, V16, P9
\item[jvalanaṃ pradīptaṃ]        J7, J12, LD2, N10, N16, N20(It is "di", short "i". Also, "ṭhaṃ" is not there.)
\item[t(?)valanaṃ pradiptaṃ]        J16,
\item[jvalatpradīptaṃ]        N7
\item[(unavailable/illegible)]        C5, C7, M1
        \end{description}
\end{marma}


\begin{marma}[hp01_027]

  \sthana{27d}
  
        \begin{description}
        \item[candrasthiratvaṃ ca dadāti] Ba5, J7, J10ac, J17, LD1, LD2, V6, B2, Bo1, Bo3, N4a, V4, V22, N10,
         N13, P1, Tue, P8, R5, V13, V10
         \item[candraṃ sthiratvaṃ ca dadāti] P9("ca" in margin)
        \item[candre sthiratvaṃ ca dadāti] C9,
        \item[candrasthiratvaṃ hi dadāti] B1, C2, N1, R3
        \item[cādrasthiratvaṃ hi dadāti] B3
        \item[candrasthiratvaṃ pradadāti] C1, N24
        \item[caṇḍasthiratvaṃ ca(?) dadāti] N7
        \item[daṇḍasthiratvaṃ ca dadāti] A1, Ba, Bo2, C3, C6, C8, J1, J2, J8, J10pc, J11, J12, J14,
         J15, J16("ḍaṃ" is given in margin), K1, Ko, M1, N17, V1, V2, V3, N6, N9, N12, V20, N16, N18,
         N19, N20, N22, N26, P6, V17, V12, R2
        \item[daṇḍāsthiratvaṃ ca dadāti] P5
        \item[daṇḍāsthitatvaṃ ca dadāmi] V14
        \item[daṇḍaṃ sthiratvaṃ ca dadāti] P7
        \item[daṇḍasthiratvaṃ hi dadāti] J13, P2 (anusvāra over "da" is missing)
        \item[daṇḍasthiratvaṃ hi dadātti] P4
        \item[daṇḍasthiratāṃ ca dadāti] V21
        \item[daṃḍaschiratvaṃ dadāti] J5,
        \item[daṇḍasthiratāṃ sa dadāti] J3, J6,
        \item[daṇḍasthiratvaṃ vidadhāti] C4, L1, V28, V18, N25, V16
        \item[daṇḍasthiratvāṃ vidadhāti] P3
        \item[daṇḍasthiratvaṃ pradadāti] N8, N11, V19, N23
        \item[daṃḍe sthiratvaṃ pradadāti] N5
        \item[daṇḍasthiratvaṃ jvalanaṃ] V8
        \item[(daṇḍa?)sthirasvaṃ ce dadāti] N2
        \item[dehasthiratvaṃ ca dadāti] N21
        \item[kāyasthiratvaṃ ca dadāti] V15
        \item[kaḍasthiratvaṃ ca dadāti(?)] J4
        \item[(unavailable/illegible)] C5, C7
        \end{description}
\end{marma}



\begin{marma}[hp01_028]
\sthana{1.28b}
  \begin{description}
  \item[dorbhyāṃ padāgradvitayaṃ] A1, Bo1, C2, C6, C8, J4, V1, M1, B2, B3, N1, N4a, V21, V20, V22,
        V28, N16, N19, N21, N22, N23, N24, P2, P6, R5, V13, R2
  \item[dorbhyāṃ padāgredvitayaṃ] V14
  \item[dorbhyāṃ padāgradvītayaṃ] P7
  \item[dorbhyāṃ padāgnadvittayaṃ] P4 (could be "padāgra")
  \item[dorbhyāṃ padāgraṃ dvitīyaṃ] Ko, V15
  \item[dorbhyāṃ padāgradvitīyaṃ] R3
\item[dorbhyāṃ padagradvitayaṃ] Bo3, Ba5, K1, N13, Tue ("pa" uncertain), V10
\item[dorbhyāṃ padāgryau dvitayaṃ] J5
\item[dorbhyāṃ pādāgrādvitayaṃ] B1
\item[dorbhyāṃ ca pādadvitayaṃ] C4, L1, V19, N5, N7, N8, N11, N12, P3
\item[dorbhyāṃ ca pādaṃ dvitayaṃ] C1
\item[dorbhyāṃ ca pādau dvitayaṃ] N25
\item[dorbhyāṃ ca pādāgradvitayaṃ] J13
\item[dorbhyāṃ sa pādāgradvayaṃ] J2
\item[dvābhyāṃ karābhyāṃ dvitayaṃ] Bo2, C3, C9, J1, J3, J6, J7, J8, J10, J15, J16, J17, LD1, LD2, N17, V6, V4, V3, N6, N9, N10, N18, N20, V18, N26, P1, P5, V17, V16, LD1, P9("karābhyāṃ" in margin)
\item[dvābhyāṃ karābhyāṃ hitayaṃ] J12, V12
\item[dvābhyāṃ padāgraṃ dvitayaṃ] J11
\item[dvābhyāṃ pādāgrādvitīyaṃ] P8
\item[(unavailable/illegible)] Ba, C7
\item[dorbhyāṃ padagraṃthidvayaṃ] J14, N2
\item[.. rbhyāṃ pādaṃbhyāṃ dvitayaṃ] V8
\item[(unavailable/illegible)] C5,
        \end{description}
\end{marma}



\begin{marma}[hp01_035]
\sthana{35b}
  \begin{description}
\item[asyahṛdaye dhṛtvā]        J7("eva niyataṃ" in below margin), J12, A1, V4, V18, V16, P8
\item[asya hṛdayaṃ dhṛtvā]   P9
\item[āsyahṛdaye dhṛtvā]        N10, V1, P1, P6
\item[āsyahridayaṃ dhṛtvā]        J16
\item[eva hṛdaye dhṛtvā] C4 ('vidadhe' is also given above), C7, C8, J5, J10, J11, J15, J17, Ko,
 V19, N2, N5, V21, N25, P3, V15, V14, R8,V12
\item[eva hṛdaye kṛtvā]        Bo3, K1, N4a, V22, N13, N19, N24 (seems later addition), Tue, V10
\item[eva niyataṃ dhṛtvā]        C1, C3, J14, V8, V28
\item[eva niyataṃ kṛtvā] B1, B3, Bo2, C2, C6, J1, J3, J6("sudṛdha" is also given above), J8, J13,
 L1, N17, LD1, N1, V3, N12, V20, N16 (It seems "meḍhra" here), N18, N20, N21, N26, P2, P4, P7, R3,
 R5('hṛdaye' canceled and replaced by 'niyataṃ')
\item[ekahṛdaye dhṛtvā]        J4, N3, N23
\item[ekahṛdayo dhṛtvā]        J2, M1
\item[eva sudṛḍhaṃ kṛtvā]        N8, N11
\item[vāmaniyataṃ dhṛtvā]        V6
\item[samyak(ū?) hṛdaye dhṛtvā]        N9    
\item[āsyahṛdaye kṛtvā]        N22, V17
\item[asya hṛdayeḥ dhṛtvā]        C9 ("āsyaṃ hṛdaye yasya saḥ" is given above)
\item[meḍhre pādam atho hṛdi svacibukaṃ]        N7
\item[meṃḍhraṃ pādam athaikam]                Bo1, LD1, LD2
\item[meḍhre pādam athaikm eva hṛdaye]        Ba5, N7
\item[meṃḍhreṃ pādam athaikam]                LD2
\item[meḍhraṃ pādam athaika samyak meva hṛdaye dhṛtvā]               P5
\item[(unavailable/illegible)]        Ba, C5, V13, R2
        \end{description}
\end{marma}

   

\begin{marma}[hp01_038]
\sthana{38a}
  \begin{description}
  \item[yameṣv eva mitāhāra(ḥ)] C2, J10, J17, M1, V1, V6, P8
  \item[yameṣv eva sitāhāra] V16
  \item[yameṣv eva mitāhāro] N7
  \item[yameṣv eva mitāhāram] C4, B2, J14, N2, N5
  \item[yameṣv eva ātmacetana] V28
  \item[yameṣv iva mitāhāra(ḥ)] B1, B3, Bo1, Bo2, C3, C9, J3, J6, J8, J13, J15, K1, Ko, LD1, N17, N1,
        V3, V4, N9, N10, N14, N16, N18, N21, N26, P1 (seems to be "mitāhārāḥ"), P2, P4, R3(after
        correction), V12, LD1, P9
  \item[jameṣṭiva mitāhāram] J5
  \item[yameṣv iva mitāhāro] V18
  \item[yameṣv iva mitāhāre] P5
  \item[yameṣv iva mītāhāra] J16,
  \item[yameṣv iva mitāhāram] A1, Bo3, Ba5, C1, C7, C8, J4, J11, L1, LD2, V19, N4a, V22, V21, N13, N23,
        N24('va' of 'iva' not clear), N25, P3, P7("mitah.." earlier), Tue, V15, R5, V13, V10
  \item[yamiṣviva mitāhāram] V14
  \item[yameṣv iva mitāhā(ḥ)] V8
  \item[niyameṣu mitāhāro] N8, N11, N12
  \item[niyameṣu mitāhārā] J12,
  \item[(unavailable/illegible)] Ba, V20, R2
  \item[yameśiva mitāhāram] N19
  \item[yameṣv eva mitāhāra] N20, V17
  \item[yameṣu ca nirāhārī] N22, P6
  \item[yameṣu ca mitāhārm] C6,
  \item[yamaṣviva mitāharm] J1
  \item[niyameṣu mitāhari] J2
  \item[yameṣv api mitahāraḥ] J7("yamesṣv iva" is deleted)
   \item[(unavailable/illegible)] C5,
        \end{description}
\end{marma}

\sthana{1.38b}
  \begin{description}
\item[ahiṃ niyameṣv iva]        B1
\item[ahiṃsā niyameṣu ca]        J4, M1, V1, N22, P6
\item[ahiṃsā niyameṣv api]        B2, J14, N2, N24, V14
\item[ahiṃsā niyameṣv iva] A1, B3, Ba5, Bo1, Bo2, Bo3, C2, C3, C4, C7, C8, C9, J1, J3, J5, J6, J7, J8,
 J10, J11, J13, J15, J17, Ko, LD1, LD2, N17, V3, V4, N5 (there is an extra mātrā after "me" and makes it
 look like "mo"), N9, V21, N13, N14, N16, N19, N20, N21, V18, N25, N26, P1 (was "ivā", the long
 mātrā seems canceled), P2, P4, P7, V16, P8, R3, R5, V13, V12, LD1, P9
\item[ahiṃsā niyāmaṣv iva]        P5
\item[ahiṃsā niyameṣu eva]        V17
\item[āhiṃsā niyameṣv iva]        Ko,
\item[ahiṃsā nīyame sthitā]        K1,
\item[ahisā nīyameṣv iva]        J16,
\item[aṃhiṃsā niyamaṣv iva]        J12,
\item[svāhiṃsā niyameṣv iva]        V6
\item[ahiṃsāṃ niyameṣv iva] L1, V19, N4a, V22, P3, Tue, V15, V10
\item[ahiṃsāṃ niyameṣu ca]        C6,
\item[ahiṃsā niyaṃmai(?)ṣv iva]         N1
\item[(a)hiṃsā ca niyameṣv iva]        N7("a" could be supplied due to its potential lopa due to sandhi)
\item[yathāhiṃsā yameṣv iva]        N8, N11
\item[ahiṃsā niyameṣv iha]        N10
\item[asiṃsā niyame sthata(ḥ)]               V8
\item[yameṣv evātma-cintanaṃ]        N12
\item[na hiṃsāṃ niyameṣv iva]        N23
\item[kṣamasye ātmacintane]        J2
\item[(unavailable/illegible)]        Ba, C5, V28, V20, N18 (only "ahisā" is visible), R2
        \end{description}
\end{marma}

\begin{marma}[hp01_039]
\sthana{39d}           
          \begin{description}
(ṇ/n and ś/s irrelevant)
\item[suṣumṇā iva nāḍikā]        V1, N22 (anusvāra over "nā"), P6
\item[suṣumṇā iva nāḍiṣu]         B2, J4, J14, K1, Ko, N2, V8, N19 (seems "suṣumṇa", short "a"), V12
\item[suṣumṇā ca nāḍiṣu]        J5,
\item[suṣumṇā eva nāḍiṣu ]        V20
\item[suṣumṇā iva nāḍiṣuḥ]        N8
\item[suṣumiva nāḍiṣu]         V28
\item[suṣumṇām eva nāḍiṣu]        J10ac, J17, N24, V14
\item[suṣumṇām iva nāḍiṣu ]        A1, C1, C4 ("nāḍīnāṃ malaśodhane" is given above), C6, C7, C8, C9 (in marg.), J1, J2, J3, J6, J11, J15 (suṣūmṇām), L1, M1, V19, N5, N7, N12, N16, N23 (sukhumṇām), N25, V15, V13
\item[suṣuṇmām iva nāḍiṣu]        P7
\item[suṣumṇām iva ḍinoṣu]        P3
\item[nāḍīnāṃ malaśodhanam]        B1, B3, Ba5, Bo1 (it seems 'मळ'), Bo3, C2, C3, C9, J8, J12, J13, LD2, N1, N4a, N10, N13, N18 (It is "ḍi" here. Short "i"), N20, N21 ("nāṃ" seems written differently), N26, V3, V22, P1 (was "śodhane", mātrā of "e" was canceled and anusvāra was added), P2, P4, Tue("naḍī.." earlier. Mark over "na" probably intends to indicate it as "nā"), V17, R3, R5, V10
\item[nāḍināṃ malasodhanaṃ]        J16,
\item[nāḍīnāṃ malaśodhanā]   P9
\item[nāḍīnāṃ malaśodhane]        J7, J10pc, LD1, N9, N17, V6, V21, N14(seems "dī"), V18, P5, V16, P8, V12, LD1
\item[suṣumṇādiṣu śodhanam ]         V4
\item(unavailable/illegible)]        Ba, Bo2, C5, R2
        \end{description}
\end{marma}

\begin{marma}[hp01_040]
\sthana{1.40e}
  \begin{description}
\item[śramādau bahubhiḥ pīṭhaiḥ] V1
\item[śramadair bahubhiḥ pīṭhaiḥ] B1, B3, Bo1, Bo2, C1, C2, C6, C8, J3, J6, J13, J14, K1, M1,
 N1(not sure if it is 'śra'), N12, N16, N21("daiḥ" instead of "dair"), N22, N23, N24, P2, P4,
 P7(bbahu), V14, R3, V13
\item[śramadair bahubhiḥ pīṭhai]        P6("piṭhai" changed to "pīṭhai")
\item[śramadai bahubhiḥ pīṭhaiḥ]        Ba, C9mg, J5,
\item[śramadair bahubhi pīṭhai]        N8
\item[śramādaiḥ bahubhi pīṭhaiḥ]        V21
\item[śramadairghyādibhiḥ pīṭhaiḥ]        C4("kim ādyair bahubhiḥ pīṭhai" is given above), C7, L1, V19, N5, N25, P3
\item[śramadairdhyādibhiḥ pīṭhaiḥ]        V28
\item[śamardda bahubhiḥ pīṭhaiḥ]        J1
\item[sanastai bahubhyaḥ pīṭai]        V8
\item[(samastair?) bahubhiḥ pīṭhaiḥ]        N2
\item[samastair bahubhiḥ pīṭhaiḥ]        J2, J4, J11, Ko,
\item[samasta bahubhiḥ pīṭhai]        V15
\item[kim ādyair bahubhiḥ pīṭhaiḥ] A1, C9("śramādai bahubhiḥ pīthaiḥ" is also given above), J7, J8,
 J10, J17, LD2, N17, V6, V3, V4, N7, N14(visarga at the end not visible due to bad picture), V20,
 N18(No visarga at the end), N20, V18, N26, P5, V17, P8, V12, LD1, R2, P9("rba" in margin)
\item[kim ādyair bahubhiḥ pīṭheḥ]        N9, V16
\item[kim adyair bahubhiḥ pīṭhaiḥ]        N10(N10 has this pāda as the 1st pāda, not the 5th)
\item[kim ādyair bahubhiḥ piṭhaiḥ]        J15, LD1, P1
\item[kim anyair bahubhiḥ pīṭhaiḥ]        Ba5, C3, N4a, N13, Tue, R5, V10
\item[kim anyair bahuniḥ pīṭhaḥ]        Bo3, V22
\item[kim arthai bahubhiḥ pīṭhai]        J12, J16("adyair" is also given),
\item[kiṃ śramair bahubhiḥ pīṭhaiḥ]        B2
\item[pramadair bahubhiḥ pīṭheḥ]        N19
\item[(unavailable/illegible)] C5,
        \end{description}

\sthana{1.40f}
  \begin{description}
\item[sadā siddhāsanasthiteḥ]        A1, V20, V12, R2
\item[sadā siddhāsane sati] B1, B3, Bo1, C2, C9("kiṃ sadā sevanena tu" is also given above), J7,
 J8, J10, J11("siddhaḥ" is given above), J12, J15, J17, LD1, LD2, N17, V1, V6, N1('kiṃ' and 'śa' seem faded
 and thus, I guess, deleted.), V4, N7, N9, N10(N10 has this pāda as the 2nd pāda, not the 6th),
 N20("sa" of "sati" is not clearly visible), N22, V18, N26, P1, P2, P4, P5, P6, V17, V16, P8, LD1, P9
\item[sidhe siddhāsane sati]        V10
\item[sadā siddhāsane satī]        J16,
\item[sadā siddhāsane satya]                  V3
\item[sadā siddhāsanaṃ(?)]        V8
\item[sadā siddhāsanā nyāsāt]        V22
\item[sadā siddhāsanābhyām sādyogī]        V14
\item[yadā siddhāsane sati]        J2, J4, N2
\item[yadā siṃhāsane sati]        Ko,
\item[siddhe siddhāsane sati]        Ba5, Bo3, C3, N4a, N13, Tue
\item[siddhe siddhāsane miti]        R5('mi' seems tampered with)
\item[kiṃ sadā siddhasādhanaiḥ]        M1
\item[kiṃ syāt siddhāsane sati] Bo2, C1, C4("sadā siddhāsane sati" is also given above), C7, J1,
 J3, J5, J6, J13, L1, V19, B2, V28, N5, N16, N21("sīddh..." long"i"), N25, P3, V15, R3
\item[kiṃ syāt siddhāsane mati(ṃ?)]        N8
\item[kiṃ syāt siddhāsane sthite]        C6, C8, V21
\item[kiṃ syā siddhāsane sthite ]        P7
\item[kiṃ syāt siddhāsana sthite ]        N23
\item[kiṃ syāt siddhāsane schite]        Ba
\item[kiṃ vā siddhāsane sati]        J14, N12, N24
\item[kiṃ syāt siddhāsanai(?)ti]        N18
\item[kiṃ sadā sevanena tu]         K1, N19(1.40h is "baddhe siddhāsane sadā")
\item[(unavailable/illegible/unable to read)]        C5, N14, V13
        \end{description}
\end{marma}

\begin{marma}[hp01_046]
\sthana{1.46cd}
\item[(utta(m)bhya cibukaṃ ~)]
\item[vakṣa sthāpayet ]        J8, J10, J17, LD2, N18, V6, N7, P8, N18, V12, V3, P9("thā" unclear)
\item[vakṣaḥ sthāpayet]         Bo1, B2, C3, J1, J12, J15, J16(sthāpaye),  K1, LD1, V18, P1, P5, V16, N10, N22(vakṣyaḥ), LD1
\item[vandā sthāpayet]        V17
\item[vakṣa sthāpayat]         N17
\item[vakṣe saṃsthāpayat(?)]        Ko,
\item[vakṣe uthāpayet]        V4
\item[vakṣa sthāpya?]        J7(A word is deleted there), C9, N5(sthāpya sevanaṃ)
\item[vatsa sthāpayet ]        N20, N26
\item[vakṣaṃ sthāpayet]         V1
\item[vakṣasy āsthāpya]         C4, L1, M1, V19, P3
\item[vakṣasy utthāya]        Bo2, C6, N16, N8, N11, N16
\item[vakṣasy utthāpya] B5, Bo3, C1, J13, J14, N13, N21, N23, N24(unclear if it is "syu"), Tue,
 N1(unclear), N4a, R5
\item[vakṣasy utṭhāpyo]        J4, J11,
\item[vakṣasy utṭhāgre]        J6
\item[vakṣasy oschāpyo]        J5
\item[vakṣyasy utthāpyot patane...]        N19
\item[vakṣyasy asthāpya]        N25
\item[vakṣyapyutchāpya]        Ba
\item[vakṣe sthāpayet]        N14 (seems "vacche"), P2, P4, N9
\item[vakṣe syu sthāpya pavanaṃ]         V10
\item[vakṣye sthāpayet]        C2
\item[vakṣye pyutthāya]        C8
\item[vakṣyesu uttānya]        V14
\item[vikṣe sthāpayet]        R3
\item[cakṣasya utthāpya]        V15
\item[svakṣasy utthāpya]        P7
\item[urghātopaticotanau karau kṛtvā tato dṛśaṃ?]        J2
\item[dakṣasy utthāgret(?)]        J3
\item[utkuṃcya cibukaṃ vakṣa sthāpayet ]           P6
\item[sannāya cibukaṃ vakṣyasy utthāpya]          N2
\item[yuktaṃ vakṣastha-cibukaṃ mākramet]        N12
\item[(unavailable/illegible/unable to read)]        C5, V13, R2
        \end{description}
\end{marma}


\begin{marma}[hp01_048]
\sthana{1.48b}
  \begin{description}
\item[(gāḍhaṃ vakṣasi sannidhāya cibukaṃ ~)]
\item[dhyānaṃ tataś cetasi]        J2, V1
\item[dhyānaṃ tataś cepsitaṃ]        J15, C9
\item[dhyānaṃ tataś cośitaṃ]        V6
\item[dhyānaṃ tataś conmitaṃ]        P5
\item[dhyānaṃ tataḥ conmī]        N1(some syllables seem deleted, 'nmī' is added later)
\item[dhyānaṃ ca tac cetasi] B2, B3, Ba, Bo1, Bo2, C1, C2, C4, C6, C7, C8, J4, J5(caitasi), J8,
 J11, J12, J13, K1, Ko, L1, V19, N25, N26, P2, P3 ("cubukaṃ"), P4, P6, P7(cetasī), V4(ṃ omitted),
 V17, V16, N3, N5, N8, N11, V15, V14, P8, R3, V13, N14, N16, N17,N19, V12, V3, N21, N23, N24(saṃniyamya), R2, P9
\item[dhyānaṃ ca tecesi]        V8
\item[dhyānaṃ ca tac cepsitaṃ] A1, J1, J3, J6, J10, J14, J16, J17, LD1, P1, N7(later attempt to form 'dhyā'), N9, N10(saṃvidhāya), N12, N18(dhānaṃ), LD1, N20, N22
\item[dhyānaṃ ca yac cepsitaṃ]        C3,
\item[dhyānaṃ ca tac cepsite]        J7,
\item[dhyānaṃ ca tatrepsitaṃ]        LD2, N2
\item[dhyāyaṃś ca tac cetasi]        Ba5, Bo3, M1, Tue, N4a, N13, V10, V22
\item[dhyāyaṃś ca taṃ cepsitaṃ]        R5(something is canceled and 'psitaṃ' is added)
\item[(unavailable/illegible)]   C5,
        \end{description}

\sthana{1.48c}
 \begin{description}
\item[(vāraṃ vāram apānam ūrdhvam anilaṃ ~)]
\item[protsālayan pūrayan]        V1
\item[protsārayan pūrayan]        J7, V6, N7, N12
\item[protsārayaḥ pūrayem]        V8
\item[protsārayan pūritaṃ]        A1, Ba5, Bo3, C1, C6, C8, J10, Tue, N4a, V14, R5, V13, N13, V10, V22, R2("n pū" not legible due to damage)
\item[protsārayat pūritaṃ]        P7
\item[protsyā?rayat pūritaṃ]        Ba
\item[protsārayan pūrayet]        C4("proccālayan pūritaṃ" is also given above),L1, N25, P3
\item[protsārayet pūrayan]        V19
\item[protsārayet pūrayet]        C7
\item[protsāraṇaṃ pūrayet]        N5
\item[protthāya yat pūrayan]        B3
\item[protthāpanṃ pūrayet]        Bo1,
\item[protthāpayan pūrayan]        C2, P2, P4, N1, R3, P9(not sure of "tthā")
\item[procārayet pūritaṃ]        J8
\item[prottāsayan pūritaṃ]        V12
\item[prollāsayan pūritaṃ]        J10, J17, N17, P1, P5
\item[prollāśayet pūritaṃ]        J15, N9
\item[prolāsayet pūrayan]        J12,
\item[prollāsayet pūrayan]        N10
\item[prollosayana pūrayan]   LD2
\item[prollāsayet pūrayet]        J16,
\item[prollāsayan pūrayan]        C9, V16, P8, V4
\item[pro(llāpa?)yan pūrayan] LD1
\item[prollāsat pūritaṃ]        C3,
\item[prollāghayan pūrayan]        LD1
\item[procāraranu pūritaṃ]        J5
\item[proccārayan pūritaṃ]        M1, V17, N3(vālaṃ vāram)
\item[proccārayan pūrayan]        J4,
\item[proccārayat puraya yataṃ(?)]        J1,
\item[proccārayat pūritaṃ]        N2
\item[proccārayet pūritaṃ]        V15, V3, N23
\item[proccārayet pūrayet]        N19
\item[proccālayan]        K1
\item[proccālayan pūrayen]        B2
\item[proccālayan pūrayet]        J13,
\item[proccālayaṃn pūrayan]        P6, N22(vvāraṃ cāram apānamūlapranilaṃ), \item[proccālayan pūrayan]        N18, N24(ayāmūrdh...)
\item[proccālayan pūritaṃ]        J2, J14, Ko, N26, N20(proccālayaṃn)
\item[proccālayet pūritaṃ]        Bo2, J3, J6("proccāṭayan pūrayet" is also given below), N16
\item[proccālitaṃ pūritaṃ]        N14(vālaṃ vālam)
\item[proccānalaṃ pūrayen]        N21
\item[prodbhā(ṭa?)yan pūrayet]        N8
\item[prod(ghā?)ṭayan pūrayet]        N11


\sthana{1.48d}
 \begin{description}
\item[(muñcan prāṇam upaiti bodham atulaṃ ~)]
\item[śaktiprabhāvān naraḥ] B3, Ba5, Bo2, Bo3, C1, C2, C6, C8, J1, J3, J6, J8, J11, J13, J14, K1, Ko,
 M1, V1, P2, P4, Tue, N2, N3, V15, V14, R3, R5, V13, N13, N16('vā' at bottom of page), N19, V10, V22, N23
\item[śaktiprabhāvān nara]        V3
\item[śaktiprabhāvā naraḥ]        V8, N22(mucan...)
\item[śaktiḥ prabhāvān naraḥ]        Ba,
\item[śaktiṃ prabhāvān naraḥ]        Bo1,
\item[śaktiṃ svabhāvān naraḥ]   N21
\item[saktiprabhāvān]        J5
\item[śaktiprabhāvon naraḥ]        N1, N7
\item[śaktiprabhāvīn naraḥ]        P6
\item[śaktiprabhāvaṃ naraḥ]        J12, N4a
\item[śaktiprabhān naraḥ]        B2
\item[śaktiprabhāvāṃtaraḥ]        N18
\item[śaktiprabhāvodayāt] C4, C7, L1, V19, N25 (1st part of pāda: prāṇaṃ muñcyati yāti bo...), V16,
 N5, N14(sakti)('da' seems later correction)
\item[śaktiḥ prasādān naraḥ]        J2,
\item[śaktimabhāvodayāt]        P3 ("ma" could well be "pra)
\item[śaktiḥ prabhāvodayāt]        A1, R2("vo" illegible due to damage)
\item[śaktiprabodhodayāt]        N20
\item[śaktiprabhāvodayaḥ]        J7, V6
\item[śaktiprabhāvād ataḥ]        C3, C9, J10, J16, P5, P7, N9, N10, N11, N12, P8, V12, LD1, N24, P9
\item[śaktīprabhāvad ataḥ]        LD2
\item[śaktiprabhāvad ata]        J15, N8, V4
\item[śaktiprabhāvataḥ]        J17
\item[śaktiprabhāvadhataḥ]        V17
\item[śaktiprabodhodayāt]        N17, N26
\item[śaktiprabhāvodhayaḥ]        P1
\item[bhakti prabhāvodayaḥ]        J4,
\item[(unavailable/illegible)]   C5,

        \end{description}
\end{marma}

\begin{marma}[hp01_049]
\sthana{1.49c}
 \begin{description}
\item[mārutaṃ nayate yas tu]        C6, C8,  
\item[mārutaṃ niyato yas tu]        Ba,
\item[mārutaṃ niyatam yaskta]        V14
\item[mārutaṃ niyatam yas tu]        V8, N23(mār(ū?)taṃ)
\item[mārutaṃ nīyate yas tu]        C1,
\item[mārutaṃ pīyate yas tu]        V1
\item[mārutaṃ pīvyate yas tu]        J8ac, V3
\item[mārutaṃ yaḥ pīvet nitya]        V13
\item[mārutaṃ pibati yas tu]        J8pc
\item[mārutaṃ pibate yas tu]        J11,
\item[mārutaṃ yas tu pibati] A1, C9, J7, J10, J12, J15, J17, LD2, N17, V6, N26, P1, P5, P6, P7, V17, V16,
 N7, N9, N10, P8, N14, V4, LD1, N20(piba?ti), R2(earlier reading cancelled and overwritten), P9
\item[māruto yas tu pibati]        N22
\item[mārutaṃ yaska pibati]        V12
\item[māruta yas tu pibati]         R1
\item[mārutaṃ yas tu pibaṃti]        C3,
\item[mārutaṃ yas tu pībaṃti]        J16,
\item[mārutaṃ dhārayet yas tu] Ba5, Bo2, Bo3, C2, C7, J1, J3, J6, J13, L1, M1, V19, N25 ("dhārayed"), P2
 ("dhārayed"), P3("), P4("), Tue("), N4a("), N11("), N12("), R5("), N13("), N16("), V10, N24("dhārayed")
\item[māruto dhārayed yas tu]        N1, R3, V22
\item[māruta dhāraye?as tu]        N21
\item[mārutaṃ dhāraye yas tu]        N8, N18
\item[marutaṃ dhārayed yas tu]        C4, N5, V15
\item[mārutaṃ māryatd yas tu]        J5
\item[mārutaṃ mriyate yas tu]        N3
\item[mārutaṃ mriyate yasmāt]        Ko,
\item[māruto mriyate yas tu]        J4
\item[māruto bhidyate yasya]        B2, Bo1, J14, K1, N2, N19
\item[mahato dhārayed yastu]        B3
\item[pavanṃ prāsayeta saṃstu]        J2
\item[(unavailable/illegible)]    C5, 
        \end{description}
\end{marma}

\begin{marma}[hp01_051]
\sthana{1.51d}
 \begin{description}
\item[nāsāgre nyastalocanaḥ]        B3, C2, C9, J7, J13, J16, N17, V1, V6, P2, P4, N1, N7, R3, N22
\item[nāsāgre nyastalocanā]        P6
\item[nāsāgra-nyastalocanaḥ]        A1, Bo1, C3, J10, J17, N26, P1, P5, V16, N10, V12, V4, LD1, LD2, N20, R2
\item[nāsāgraṃ nyastalocanaḥ]  P9
\item[nāsāgranyastalocanaṃ]        J8
\item[nāsāgraṃ nyastalocanaṃ]        P7, V17, V3
\item[nāsāgra susamāhitaḥ]        C8,
\item[nāsāgraṃ susamāhitaḥ] Bo2, Bo3, C1, C4("nāsāgra nyastalocanaḥ" is also given), J1, J2, J3, J6,
 J11, J14, J15, K1, Ko, L1, M1, V19, N25, P3, Tue, N2, N4a, N5, N12, V15, V14, V13, N13, N16, V10,
 V8
\item[nāsāgrastu samāhitaḥ]   N23
\item[nāśāgre susamāhitaḥ]        N19,
\item[nāsyagraṃ tu samāhitaḥ]        J5,
\item[nāsāgraṃ susamāhitaṃ]        Ba5, C6, V22
\item[nāsāgryaṃ susamāhitaṃ]        N18
\item[nāsāgraṃ susamāhita]        N8
\item[nāsāṃgraṃ susamāhitaḥ]        N11
\item[nāśāgraṃ tu samāhitaḥ]        N3,
\item[nāsāgraṃ tu samāhitaḥ]   N24
\item[nāsāgre susamāhitaḥ]        B2, Ba, K1, N21
\item[nāsāyāṃ susamāhitaḥ]        J4,
\item[nāsāṃ tu samāhitaḥ]        R5
\item[nāsāgre neṣu nocana]        N14
\item[(unavailable)]        C5, C7,
        \end{description}
\end{marma}


         
\begin{marma}[hp01_052]
\sthana{1.52d}
 \begin{description}
 \item[kurute cāsanottamam] B3, Ba, Ba5, Bo1, Bo2, Bo3, C1, C2, C4, C6, C8, J1, J3, J5, J6, J7, J8,
        J10pc, J13, J15, L1, N17, V6, N26, P2, P3, P4, P5, P7, Tue, N1, V17, N4a, N5, N8, N11(anusvāra
        over 'cā' seems canceled), N12, V15, R3(visarga is also there), R5, V13, N13, N14, N18, N19,
        V8, V3, N20, N21, N23, R2
\item[kuruta cāsanottamam]        V22
\item[kurute cāsanomatam]        V10
\item[kurute cāsanottame]        N16
\item[kurute cāsanottamaḥ]        J2,
\item[kurute cāsanotraye]        V14
\item[kurute cāsanottama]        A1,
\item[kurute cāsanontayam]   N24
\item[kurute vāsanottamam]        V1, V19, N3
\item[kurute sādhanottamam]        C3, J12, J10ac, J16, J17, P1, V16, N9, N10, V12, V4, P9
\item[kurute sādhanottaramaṃ]        P8(perhaps 'ra' is attempted to be canceled)
\item[kurute sādhakottamam]        C9, N7,
\item[kurute sādhakottamaḥ]        LD1, LD2
\item[kurute cāsanam uttamaṃ]         N25
\item[sevate cāsanottamaiḥ]        J14,
\item[sevate cāsanottamaṃ]        J11, Ko(casano°), N2
\item[sevyate cāsanottamaṃ]        J4,
\item[(unavailable)]        B2, C5, C7, K1, M1, P6, N22
        \end{description}
\end{marma}


         
\begin{marma}[hp01_054]
\sthana{1.54b}
 \begin{description}
 \item[sarvavyādhi-viṣāpaham] C3, C9, J5, J7, J8, J10, J12, J15, J16, J17, LD1, V1pc, V6ac, P1,
        P5(sarvva), P6, P7, V17, N3, V16, N9, N10, P8, R1, V12, V3, V4, N22, P9
 \item[sarvavyādhī-viṣāpaham] LD2
 \item[sarvavyādhi-vināśanam] A1, B2, B3, Ba, Ba5, Bo1, Bo2, Bo3, C1, C2, C4("viṣāpahaṃ" is also given above), C6, C7, C8, J1, J2, J3, J4, J6, J11, J13, K1, Ko, L1, N17, V1ac, V6pc, V19, N25, N26(Seems ...sanam), P2, P3, P4, Tue, N1, N4a, N5, N7, N11, N12, V15, V14, R3(seems an extra mātrā after 'vyā'), R5, N13, N14(sarvva...sanaṃ), N16, N18(sanaṃ), N19(anusvāra seems erased due to damage), V10, V8, V22, N23, N24(sanaṃ), R2
\item[sarvavyāvināśanam]   N21
\item[sarvadhyādhi-vighātakaṃ]   N20
\item[sarvavyādhi-praṇāśanaṃ]        J14, N2
\item[kaphavātāmayāpahaṃ]        M1
\item[(unclear/unable to read)]        C5, N8(last two syllables seem erased), V13
        \end{description}
\end{marma}


         
\begin{marma}[hp01_055]
\sthana{1.55 additional verses}

kriyāyuktasya siddhiḥ syād akriyasya kathaṃ bhavet |
na śāstrapāṭhamātreṇa yogasiddhiḥ prajāyate ||
na veṣadhāraṇaṃ siddheḥ kāraṇam na ca tatkathā |
kriyaiva kāraṇaṃ siddheḥ satyam etan na saṃśayaḥ |
śiśnodararatāyeha na deyo veṣadhāriṇaḥ ||

\begin{description}
\item[Yes] B1, B2, B3, Ba, Bo1, Bo2, C1, C3, C4, C6, C8, C9, J1, J3, J6, J7(58 num.), J8, J10, J11,
 J12, J13, J14, J15, J16, J17, K1, L1, LD1, LD2, N5, N17, N25(the first 3 lines only?), V3, V17, V25, P8,
 R1('śiśno....' not there), R3, N14, N16, N18, V12, LD1, N7, N9, N10, N13(appears way later, after the verse "yuvā vṛddho...yogeṣvatandritaḥ", and last line not there), N20, N21, N26, P1, P3, P4, P5, P7, R2, P9
\item[Yes], but the 4th line appears as the 2nd line: C2, J13, N1, P2
\item[No] A1, Ba5, Bo3, C1, C5, C7, J2, J4, J5, Ko, M1, V1, V6, V8, V19, R5, N19(mudrādipavanakriyām also
 not there), V10, V22, V4 (unclear/unable to read) V13, N3, N4a, N8, N11, N12, N22, N23, N24, Tue, P6
        \end{description}

\sthana{1.55 third additional verse}

mayi bodhāmbudhau svacche tuccho 'yaṃ viśvabudbudaḥ |
pralīna udito veti vikalpapaṭalaḥ kutaḥ ||

\begin{description}
\item[Yes] B1, B3, Ba, Bo2, C1, C2, C4, C7, C8, J1, J3, J6, J7, J13, L1, N5, N17, N25, V6, V19,
  R3, N14(bodhibudhau, malīna), N16, N18, N1 (word "tathā" before "pralīna", also some letters before "svacche", perhaps "ttu", seem faded and not taken into consideration), N8, N11, P2, P3, P4, N13(at the end of 1st chapter)
\item[No] A1, B2, Ba5, Bo3, C3, C5, C6, C9, J2, J4, J8, J5, J10, J11, J12, J14, J15, J16, J17, K1, Ko, LD1, LD2, M1,
  V1, V3, V8, V17, V25, P8, R1, R5, N19, V12, V10, V22, V4, LD1 (unclear/unable to read) V13, N2, N3, N4a, N7, N9, N10, N12, N20, N21, N22, N23, N24, N26, Tue, P1, P5, P6, P7, R2, P9
        \end{description}

\sthana{1.55 fourth additional verse}
% P: What is the difference between this and the previous extra verse?

mayi bodhāmbudhau svacche tuccho 'yaṃ viśvabudbudaḥ |
pralīna udito veti vikalpapaṭalaḥ kutaḥ ||

\begin{description}
\item[Yes] B1, B3, Ba, Bo2, C1, C2, C4, C7, C8, J1, J3, J6, J7, J13, L1, N5, N17, N25, V6, V19, R3,
 N14(bodhibudhau, malīna), N16, N18, N8, N11, P2, P3, P4
\item[No] A1, B2, Bo3, C3, C5, C6, C9, J2, J4, J8, J5, J10, J11, J12, J14, J15, J16, J17, K1, Ko, M1,
 V1, V3, V8, V17, V25, P8, R1, R5, N13, N19, V12, V10, V22, V4, LD1 (unclear/unable to read) V13, N2, N3, N4a, N7, N9, N10, N12, N20, N21, N22, N23, N24, N26, Tue, P1, P5, P6, P7, R2, P9
        \end{description}


\begin{marma}[hp01_056]
\sthana{1.56a}
 \begin{description}
 \item[āsanaṃ kumbhakaś citro] J8pc, N24
 \item[āsanaṃ kumbhakaś citraṃ] B2, C3, J7, J8ac, J10, J12, J15, N17, P8, N14(ā(?)naṃ), V12, LD1, LD2, N4a, N10, N12, N20, P1, P5
 \item[āsanaṃ kumbhavaś citra] J2,
\item[āsanaṃ kumbhavaś ci(nt?)aṃ] P9
 \item[āsana kumbhakaś citraṃ] V3
\item[āsanaṃ kumbhakāś citrāḥ]  N8, N11
\item[āsanaṃ kumbhakāś citrā]   P7
\item[āsanaṃ kumbhakācitraṃ]   Tue
 \item[āsanaṃ kumbhakaṃś citraṃ] C9, J17
 \item[āsanaṃ kumbhakaṃ citraṃ] A1, B1, B3, Ba, Ba5, Bo1, Bo2, Bo3, C1, C2, C4, C6, C7, C8, J1, J3, J4, J5, J6, J11, J13, J14(55), K1, Ko, L1, M1, V19, R3, R5, N13, N16, N19, V10, V22, V4, N1, N5, N23, N25, N26, P2, P3, P4, P6, R2
\item[āsanaṃ kumbhakaṃ śvitraṃ]   N21   
 \item[āsanaṃ kumbhaka citrāṃ] J16, N7
 \item[āsanaṃ kumbhaka citaṃ] N18
\item[āsanaṃ kubhakaṃ citraṃ] N2 (appears before kriyāyuktasya....)
\item[asanaṃ kumbhakaṃ citraṃ] N3, N9
 \item[(unclear/unable to read)] V13
\item[(unavailable)]        C5, N22, R1
        \end{description}

\sthana{1.56b}
 \begin{description}
 \item[mudrādikaraṇāni ca] A1, B2, C3, C9, J7, J8pc, J10, J12, J15, J16, J17, LD1, N17, P8, N14, V12, N7, N9, N10, P1("kā" changed to "ka" later), P5, P7, R2, P9
 \item[mudrānikaraṇāni ca] LD2
\item[mudrādikāraṇāni ca]        V4
\item[vicitrīkaraṇāni ca]        J2,
\item[mudrādi pavanakriyā]        B1, B3, Bo1, C2, J5, J8ac, J13, R3, V3, N1, N20, P2, P4
\item[mudrādi pavanakriyāḥ]   N26
\item[mudrākhyaṃ karaṇaṃ tathā] Ba, Ba5, Bo2, Bo3, C1, C3 ("mudrādikāraṇāni ca" is also given), C6, C7,
 C8, J1, J3, J11, L1, M1, V19, R5, N13, N16, N18, N19, V10, V22, N5, N11, N12, N21, N23, N25, Tue, P3, P6
\item[mudrākhaṃ karaṇaṃ tathā]   N24
\item[mudrākhyaṃ karaṇaṃ bhavet]   N3
\item[mudrākhyakaraṇaṃ tathā]        Ko, N4a, N8
\item[mudrādi karaṇaṃ tathā]        K1,
\item[mudrākṣaṃ karaṇaṃ tathā]        J4, J6,
\item[mudrakhecaraṇīṃ tathā]        J14(given at 55)
\item[mudrakhecariṇīṃ tathā]        N2 (appears before kriyāyuktasya....)
\item[(unclear/unable to read)]        V13
\item[(unavailable)]         C5, N22, R1
        \end{description}
\end{marma}


         
\begin{marma}[hp01_060]
\sthana{1.60cd}
 \begin{description}
\item[(atilavaṇaṃ tilapinḍaṃ kadaśanaśākotkaṭaṃ varjyam |)]
\item[atilavaṇaṃ tilapinḍaṃ]        J17, LD2, N10, R1
\item[atilavaṇaṃ tilapinḍa]        C3, J8pc, J10, J15, J16, LD1, N17, V4, N9, N20, P1, P5, P8, P9
\item[atilavaṇaṃ tilapiṇḍā?]   N14
\item[atilavaṇaṃ tilaṃ pinḍa]        J8ac
\item[atilavaṇaṃ tilaṃ pinḍaṃ]         V3
\item[atilavaṇaṃ tilapiṇyākadarśanam]        N26
\item[atilavaṇam amlayuktaṃ]        A1, B3, Ba, Ba5, Bo3, V22, V10, N1, N4a, N11 (original letters "...sacapalaṃ" striked off and "...amlayuktaṃ"), N13 mentioned in bottom margin), Tue, P2("amla" in margin), R2, R5(some doubt in "amla")
\item[atilavaṇama malayuktaṃ]        C2,
\item[atilavaṇamalayuktaṃ]   P4, R3
\item[atilavaṇāmlayuktaṃ]   P7
\item[atilavaṇa amlayuktaṃ]        C6, C8, V13
\item[atilavaṇa amlayutaṃ]        J11,
\item[atilavatilapi]        V6
\item[atilavaṇatiktapiṇḍa]        B2
\item[atilavaṇatilapiṇḍakaṃ]        C9
\item[atilavaṇaṃ dyuṣṇaṃ]        V1
\item[atilavaṇam apramuṣṇaṃ]        M1
\item[atilavaṇam ca apramitaṃ]        V8
\item[atilavaṇam cāpramitaṃ]   P6
\item[atilavaṇaṃ ca tilapiḍaṃ]   N7
\item[atilavaṇam pramitaṃ]        J12, K1,
\item[atilavaṇādi prasaktaṃ]        J14,
\item[atilavaṇādi proktaṃ]   N22
\item[atilavaṇacāmlapiktaṃ]        N19
\item[atilavaṇakavalṃ cāpamitaṃ]        J1,
\item[atilavaṇasavapalala]        V19
\item[atilavaṇasavapalalaṃ]   N25
\item[atilavaṇaṃ sala(pa?)laṃ]   N8
\item[atilavaṇaṃ palilaṃ]         R4
\item[atilavaṇāsavapalalaṃ]        C1, C7, L1, N5
\item[atilavaṇātiproktaṃ]        J2,
\item[atilavaṇādiyutaṃ]        J4,
\item[atilavaṇādyaptataptaṃ]        J5,
\item[atilavaṇādyapran(t?)aṃ]  N3
\item[atilavaṇādyapratapta]   N12
\item[atilavanā?āna(va?)lala]   N24
\item[atilavaṇamatprayuktaṃ]   N21
\item[atilavaṇāmavapalalaṃ]        C4 ("atilavaṇaṃ tilapinḍaṃ" is also given),
\item[atilavaṇāya priyatavṃ]        Ko
\item[atilavaṇaṃ sapalaṃ ca]        Bo2, J3, J6,
\item[atilavaṇasalilaṃ kadaṣana]        B1, Bo1, J13,
\item[atilavaṇetilapiṃḍakvadaśanaṃ]        J7
\item[atilavaṇaprasūtaṃ]         N2
\item[atilavaṇaṃ kavalaṃ cāpramitaṃ piṇḍa]   N18
\item[atīlavaṇāmlapṛktaṃ]   N23
\item[(illegible/unavailable)]  C5, N16, P3

* * *

\item[śākotkaṭaṃ] A1, B2, B3, Ba5, Bo1, Bo3, C1, C2, C3, C4, C6, C8, C9, J2, J3, J4, J6, J10, J11, J12, J13, J14, J15, J17, K1, Ko, L1, LD1, LD2, V6, V19, V22, V4, V10, V13, N2, N3, N4a, N7, N11, N13, N18, N19, N21, N24, N25, Tue, P2, P5, P6, P7, P8, R1, R2, R3, R4, R5, P9
\item[śākotkaṭhaṃ]        V8
\item[śākotkavaṃ]        J5,
\item[śākotkaṭa]        Bo2, J8pc, J16, N1, N9, N26(??kotkaṭa)
\item[śākotkarṭeṃ]  P4
\item[śākātkaṭa]        V3
\item[śākotchadaṃ]        Ba,
\item[śākolkaṭaṃ]   P1
\item[śākokṣadaṃ]   N23
\item[śāṃkotkaṭaṃ]   N10
\item[śāketkaṭaṃ]        C7
\item[śokātkaṭa]        J8ac, V1
\item[śokotkaṭaṃ]        B1, J7,
\item[śāśotka]        M1
\item[śātkṭaṃ]        J1,
\item[śa?tāsanaḥ]   N14
\item[sākotkaṭa]   N20
\item[sākotkaṭaṃ]  N22
\item[syacotkaṭaṃ] N12
\item[(illegible/unavailable)]        C5, N17, N8, N16, P3
\item[kotkaṭaṃ]  N5

* * *
\item[varjyaṃ] A1, B1, B2, B3, Ba5, Bo1, Bo3, C1, C2, C4, C7, C9, J3, J6, J13, J15, K1, L1, LD2, V3, V6, V19,
 V22, N1, N4a, N5, N11, N12(va?jyaṃ), N13, N22, N25, Tue, P2, P4, P6, P7, R3, R4, R5
\item[varjya] V8, V10
\item[varjyāṃ] J16
\item[varj(j)aṃ] J7, J10, J12, J17, M1, N9, R1
\item[varjjyaṃ]  P5
\item[varj(j)itaṃ] J1, Bo2, J8, LD1, V3, N18, N20, N26
\item[varyyaṃ]   N7, R2, P9
\item[varjjāṃ]   N19
\item[varjet]  N10
\item[varjayet] C3, P1("varjaṃ" earlier)
\item[vayīṃ]   P8
\item[duṣṭaṃ] C6, C8, J4, J5, J11, J14, Ko, V13, N2, N21, N23, N24
\item[du(ṣl?)aṃ]  N3
\item[dṛṣṭaḥ]
\item[durjvaṃ(?)] Ba,
\item[(illegible/unavailable)] C5, N17, V1, V4, N8, N14, N16, P3
        \end{description}
\end{marma}


         
\begin{marma}[hp01_062]
\sthana{1.62d}
 \begin{description}
 \item[mudgādi divyam udakaṃ ca] A1, B2(dīvyam), Ba, Ba5, Bo3, C6("dgā" is missing), C8, J5, J7, J10, J11, J12, J16, J17, K1, L1pc, LD1, LD2, N17, V1, V6, V8, V22, V4, V13, N1 ("mudgā" correction in folio margin), N10, N12, N13, N14(not sure for "dgā" and "ṃca"), N20, N22, N23(something after "mudgādi" but perhaps it is cancelled), N24(is it "mudrā"?), N26, Tue, P1, P6, P8, R1, R2, P9
\item[mudgādi divyam udakaṃ]   N18
\item[mudgādi divyam udakaṃ ka]   N4a
\item[mudgādi divyam udake ca]   R5
\item[mudgādi divyam udakaṃ yamīndra]        V3, N2
\item[mudgādi divyam udakaṃ hi]        C3, J14, J15, N9, P5
\item[mudgādi pivyam udakaṃ ca]        J2,
\item[mudgājya divyam udakaṃ ca]        C9
\item[mudgādivyam udakaṃ ca]        C4, J8, M1, V10, N3
\item[mudgādi divyagaditaṃ ca]   N7
\item[mudgādi divyamuktaṃ]        Ko,
\item[mudgādi cālpam udakaṃ ca]        B1, B3, Bo1, Bo2, C1, C2, C7, V19, N5, N11 ("ca" in margin), N21, P2(not sure of "cā"), P4, R3, R4
\item[mudgādi cālyam udakaṃ ca]        L1ac, N8(no "ca")
\item[mudgādi cājyam udakaṃ ca]        N16
\item[mudgādi cālpa sudakaṃ ca]         J1
\item[mudgādi dālpam udakaṃ ca]        J3, J6("cālpa" is also given)] J13,
\item[mudgādi dālpamudgaṃ ca]        N25
\item[mudrādi divyam udakaṃ ca]  P7
\item[mudrādi divyam udakaṃ ca mudakaṃ]        N19
\item[(illegible/unavailable)]       C5, J4, P3
        \end{description}

\sthana{1.62d}
 \begin{description}
 \item[munīndrapathyaṃ (or: munindra-)] B1, B2, B3, Bo1, C1, C2, C4, C7, J3, J5, J6, J13, J16, J17, K1, L1, LD2, V1, V19, N1, N5, N11, N19, N21, N22, P2, P4, R4
\item[munīdrapathyaṃ]   N8, R3
\item[munīndrapathaṃ]   N25
\item[mahendrapathyaṃ]  N9, P5
 \item[yamīndrapathyaṃ (or: yamindra-)] A1, Ba, Ba5, Bo3, C6, C8, C9, J7, J8, J10, J11, J14, LD1, M1, N17, V6, V22, V3, V4 (pathyaṃ missing), V10, V13, N2, N3, N4a, N12, N13, N14, N20, N26, Tue, P1, P6, P7, P8, R2, R5, P9
\item[yamīṃdrapraparyya]   N24
\item[yatīṃdrapathyaṃ]        J2, N10, N23
\item[yamendrapathyaṃ]        C3,
\item[yogīdrapathya]        J1,         
\item[mūnidra pathayam]        V8
\item[munidra pathayam]        J12,
\item[munīdramukhyaiḥ]        N7
\item[muṃhanīdrapathya]        J15
\item[suyogipathyaṃ]         N16
\item[varayogipathyaṃ]        N18
\item[jamodūpathyaṃ]        R1
\item[paramaṃ hitaṃ hi]        Ko,
\item[(illegible/unavailable)]   C5, J4, P3
        \end{description}
\end{marma}


         
\begin{marma}[hp01_063]
\sthana{1.63a}
 \begin{description}
\item[iṣṭaṃ sumadhuraṃ]        A1, J3, J6, J10, N18(appears after one verse), P7, R2, P9
\item[iṣṭaṃ samadhuraṃ]        C9, J7, J17, LD2, N17, V19, V4, LD1, P1, P8, R1
\item[iṣṭaṃ saṃmadhuraṃ]   N7
\item[daṣṭaṃ samudhūraṃ]        J16,
\item[dṛṣṭaṃ samūdhuraṃ]        K1
\item[dṛṣṭaṃ sumūdhuraṃ]        N10
\item[piṣṭaṃ sumadhuraṃ]        C8, N8, N11, N14("śu")
\item[pathyaṃ sumadhuraṃ]        J12,
\item[puṣṭaṃ sumadhuraṃ]        Ba5, Bo3, N4a, N13, Tue
\item[puṣṭaṃ samadhuraṃ]   R5
\item[yeṣṭaṃ sumadhura]        V13
\item[yuṣṭaṃ sumadhuraṃ]        V10(mu°), V22
\item[mṛṣṭaṃ sumadhuraṃ]        M1, V1
\item[mṛṣṭaṃ sumadhurāṃ]        Ko
\item[mṛṣṭaṃ sumadhusnigdhaṃ]   N3
\item[mṛṣṭaṃ samadhuraṃ]        C3,
\item[miṣṭaṃ samadhuraṃ]        J8, Ba, C4, N20(appears after one verse), N26(appears after one verse), P5
\item[miṣṭaṃ sumadhuraṃ]        Bo2, C6, J1, J11, J14, N2, N16(appears after one verse), N19(miṣtaṃ??), N21(appears after one verse), N22, N23(appears after one verse), R4(appears after one verse)
\item[miṣṭaṃ samudhuraṃ]        J5, J15,[%P: How is this reading different from the reading above the reading above this?]
\item[miṣṭaṃ sūmadhuraṃ]  P6(appears after one verse)
\item[miṣṭaṃ sumadhura]   N25
\item[miṣṭaṃ madhuraṃ]        V3
\item[miṣṭaṃ ca madhuraṃ]        B2
\item[maṃ(ṣṭ?)aṃ samudhuraṃ]  N9
\item[sumiṣṭaṃ madhuraṃ]        N12(appears after one verse)
\item[uṣṇaṃ sumadhuraṃ]        B1, Bo1, C2, C7, L1, J13, N1 (seems "uṣṇaṃ"), P2
\item[uṣṇaṃ samadhuraṃ]        C1,
\item[uṣma(?) sumadhuraṃ]        B3
\item[uṣmaṃ sumadhuraṃ]        P4, R3(appears after one verse)
\item[uśmā?mumadhuraṃ]  N5
\item[suvṛkṣaṃ madhuraṃ]        V6
\item[ghṛtam ca madhuraṃ]        V8
\item[vṛṣyaṃ sumadhuraṃ]        J2,
\item[tiṣṭaṃ tu madhuraṃ]  N24
\item[(illegible/unavailable)]   C5, J4, P3
        \end{description}
\end{marma}


\begin{marma}[hp02_003]
\sthana{2.3b}
\item[jīvanam] Ba, Ba5, Bo2, Bo3, C1, C6, J1, J2, J3, J6, J12, J13, J14, J16, N17, V1, N4a, N7, N8, N9, N10, N11, N12, N13, N16, V16, V17, V10, N18, N21, V21, Tue, N23, V6, V7, V20, V28, V18
\item[jīvaṃnam] N14
\item[jīvitam] A1, B1, B2, B3, Bo1, C2, C3, C4, C8, J4, J5, J7, J8, J10, J11, LD1, LD2, N1, N2, N5, V8, N20, N22, V14, V3, N24, N25, V15, V5, V22, V2, V12, V13, V9
\item[jivitam] J15,
\item[jīvita] N26
\item[jīvitum] J17, N19
\item[jītavim] N3
\item[jīvo na] Ko
\item[(unavailable/illegible)] C5, C9, 

 \begin{description}

        \end{description}
\end{marma}

\begin{marma}[hp02_003]
\sthana{2.3d}
\item[nibandhayet] Ba, Bo2, C6, V1, J2, J4, J6, J8, J11, J14, N12, N16, N19, N22, V14, V3, V13
\item[nirundhayet] A1, B1, B2, B3, C1, C3, C9, J5, J10, J13, J15, J17, LD1, LD2, N17, V19, W4, N2, N7, N9, N10, N14("ye" unclear),V16, V17, N20, V21, N23, N26, V20, V5, V2, V18, V12, V9
\item[nirodhayet] Ba5, Bo1, Bo3, C2, C4, C8, J1, J3, J6pc, J7, V19, N1, N4a, N5, N8(seems unclear), V8, N11, N13, V10, N18(“ye” is doubtful), N21, Tue, N24, N25, V6, V7, V15, V28, V22
\item[nisaṃdhayet] J16,
\item[nirodhanaṃ] J12,
\item[nibandhanāt] V19, N3
\item[nibadhnayāt] Ko,
\item[(unavailable/illegible)] C5,
 \begin{description}

        \end{description}
\end{marma}



\begin{marma}[hp02_004]
\sthana{2.4d}
\item[kāyasiddhiḥ] A1, B1, B2, C2, C6, J8, J14, V1, N1, N8, N11, V17, N24, V20, V15, V9
\item[kāyāsiddhiḥ] J2, N3,
\item[kāyāsuddhi] V6
\item[kayasiddhiḥ] N20
\item[kāyasiddhiḥ] J5,
\item[kāyaśuddhiḥ] N23
\item[kāryasiddhiḥ] B3, Ba, Ba5, Bo1, Bo2, Bo3, C1, C3, C4, C8, C9, J3, J4, J6, J10, J11, J13, J17, Ko, LD1, N17, V1, W4, N2, N4a, N5, N9, N12, N13, V16, V10, N19, N21, Tue, N25, N26, V70, V5, V28, V22, V2, V18, V12, V13
\item[kāryasiddhiṃ] J15, 
\item[kāryāsiddhiḥ] J1
\item[kāryasiddhi] J12, LD2, N7, V8, N10, N14("rya" unclear), N16, N18, N22, V21, V3 
\item[kāryasiddhī] J16,
\item[kāryasuddhiḥ] V14
\item[kāyāśuddhiḥ] J7,
\item[(unavailable/illegible)] C5,

 \begin{description}

        \end{description}
\end{marma}


\begin{marma}[hp02_006]
\sthana{2.6c}
\item[suṣumnāsu snighdhā] J4, V1, V15
\item[suṣumnānāḍisthā] Ba5, N13, Tue, 
\item[suṣumṇānāḍīsthā] N4a, N21(“ṇā” doubtful), 
\item[suṣumṇānāmadhysthā] V2
\item[suṣumṇānāḍīstho] N24
\item[suṣumnānāḍīstho] J14,
\item[suṣumṇānāḍīschā] Bo3, V28, V22
\item[suṣumṇāṃtasthhā] J15,
\item[suṣumnāṃtaraschā] J11
\item[sukhumnāṃtarasthā] V19
\item[sukhumnāṃtarālasthā] B1,
\item[suṣumnāṃtarālasthā] B3, Bo1, Bo2, C1, C2, J3, J6, J13, N1, N8, N11
\item[suṣumnāṃtarālasyā] J1
\item[suṣumnāṃtarājasthā] N16
\item[suṣumṇāṃtarasthā]  N5, C4
\item[suṣumṇāṃtarāla] V5
\item[suṣumnā pāśvaṃsthā] A1, V20
\item[yathā suṣumṇā pārśvaḥsthā] Ba
\item[yathā suṣumṇā pārśvasthā] C6, C8, J7, N23
\item[suṣumṇāpañcasthā] N19
\item[yathā suṣumnā mukhasthā] V17
\item[yathā suṣumnā nāḍisthamala] V10
\item[yathā suṣumnņayā śvaḥsthā] V8, V21
\item[suṣumņā sustabdhāḥ] N12
\item[yathā suṣumnā susvasthā][a] J5, N22, V14, V3, N25, N3
\item[suṣumņā suṣasthā] N20
\item[suṣumnņayā ścasyā][b] V6, V13
\item[suṣumnāvasthāyaṃ] B1, B2
\item[suṣumnānāḍibhyāṃ] V7
\item[suṣumnāsvasthāyan] N2
\item[suṣama susvasthā] J8
\item[suṣumnāsui svasthā] J8ac %H: I didn’t find this reading in ms. I don’t know who added this.
\item[suṣumnāsaṃsvasthā] J2,
\item[sukham avasthāya] C3, C9, J10, J12, J17, LD2, N17, W4, N7, N9, V16, V18, V12, V9
\item[sukham avasthāyaḥ] J16,
\item[sukham avasthāyā] N14
\item[sukham avasthāpyā] LD1
\item[sukham avaschāṃya] N10
\item[suṣam avasthāya] J8pc, N18,
\item[suṣumṇām avasthāpya] J7pc
\item[suṣumnāṃtvarayāpalaḥ] N25
\item[(unavailable/illegible)] C5, Ko, 

 \begin{description}

        \end{description}
\end{marma}


\begin{marma}[hp02_009]
\sthana{2.09 additional verses}
dvādaśamātraḥ kanīyānmadhyamo mātracaturvviśaḥ |
uttamaḥ ṣaṭtriṃśanmātro mātrābhedāḥ smṛtā tajñaiḥ |
vaśiṣṭasaṃhitādau tu pūrakaṃ ṣoḍaśamātrabhiḥ |
kuṃbhakaṃ ca catuṣṣaṣṭimātrābhiḥ recakaṃ tu |
dvātriṃśanmātrābhiḥ vidhivatkuṃbhakaṃ kṛtvā punaścaṃdreṇa recayet ||
\begin{description}
\item[Yes] J17
\item[No] A1, B1, B2, B3, Ba, Ba5, Bo1 Bo2, C1, C2, C3, C4, C5, C6, C8, C9, J1, J2, J3, J5, J10, J11, J12, J14, J15, J16, LD2, N17, V1, J4, J6, K7. J8, J13, Ko, LD1, W4, V19, N1, N2, N3, N4a, N5, N7, N8, N9, V28, V8, N10, N11, N12, N13, N14, N16, V16, V17, V10, N18, N19, N20, N21, N22, V14, V21, V3, Tue, N23, N24, N25, N26, V6, V7, V20, V15, V5, V22, V2, V18, V12, V13, V9
        \end{description}

\begin{marma}[hp02_011]
\sthana{2.11 additional verses}
iḍayā piva ṣoḍaśabhiḥ pavanaṃ |
kuru ṣaṣṭicatuṣṭayamaṃtagataṃ |
tyaja piṃgalayā sanakaiḥ sanakaiḥ |
daśabhirdaśabhirdaśabhirdvyadhikaiḥ ||
\begin{description}
\item[Yes] B1, B3, C9 (2.13), V19, N1 (2.13), N5, N10 (some change), N12, N21(some changes), N25, V28
\item[No] A1, B2, Ba, Ba5, Bo1, Bo2, C1, C2, C3, C4, C5, C6, C8, J1, J2, J3, J10, J13, J17, J4, J5, J6, J7, J11, J12, J14, J15, J16, Ko, LD1, LD2, N17, V1, J8, W4, N2, N3, N4a, N7, N8, N9, V8, N11, N13, N14, N16, V16, V17, V10, N18, N19, N20, N22, V14, V21, V3, Tue, N23, N24, N26, V6, V7, V20, V15, V5, V22, V2, V18, V12, V13, V9
        \end{description}


\begin{marma}[hp02_012]
\sthana{2.12cd}
\item[uttānaṃ cottame prāṇarodhe padmāsane muhuḥ] Ba, C1, C6, V14
\item[uttānaṃ cottame prāṇe stebdhe? padmāsane muhuḥ] C8,
\item[uttame sthānam āpnoti tato vāyuṃ nibaṃdhayet] N13, V10, Tue, V7
\item[utiṣṭhe tūtame prāṇā yāvat padmāsane sūmaṃ] J2,
\item[uttiṣṭhaty uttame prāṇaṃ rodheḥ padmāsanaṃ .uhuḥ] V1
\item[uttiṣṭhāṃty uttame prāṃṇā baddhe padmāsane uttama] J16
\item[uttiṣṭhaty uttame prāṇē baddhe padmāsanae dṛḍhe] C2
\item[uttiṣṭhaty uttame prāṇaḥ baddhe padmāsanae dṛḍhe] J13,
\item[uttiṣṭhaty uttame prāṇaṃ baddhā padmāsane dṛḍhe] B1
\item[uttiṣṭhaty uttame prāṇo baddhe padmāsane dṛḍhe] B2, B3,
\item[uttiṣṭhaṃty uttame prāṇā baddhe padmāsane muhuḥ] LD2, J10
\item[uttiṣṭhaṃty ūttame prāṇā baddhe padmāsane muhuḥ] J15,
\item[uttiṣṭhaṃty uttame prāṇā baddha padmāsane muhuḥ] C3, N10
\item[uttisthaṃ cottame prāṇā baddha padmāsane muhu] N14
\item[uttiṣṭhaty uttame prāṇa baddhe padmāsanaṃ muhuḥ] J8ac, J14, V3
\item[uttiṣṭhaty uttame prāṇa baddho padmāsane muhuḥ] V8
\item[uttiṣṭhatm uttame prāṇa baddhe padmāsane muhuḥ] V16, V17, V21, V20, V18, V13
\item[uttiṣṭhaty uttame prāṇo baddhe padmāsane muhuḥ] C9, 
\item[uttiṣṭhaty uttame prāṇo baddha padmāsane muhuḥ] A1
\item[uttiṣṭhaty uttame prāṇo badhye padmāsane muhuḥ] N9
\item[uttiṣṭhaty uttame prāṇo baddhapadmāsane muhuḥ] N7
\item[uttiṣṭhaty uttame prāṇē baddhe padmāsanaṃ muhuḥ] J8pc
\item[uttiṣṭhaty uttame prāṇē baddha padmāsane haye] Bo1
\item[uttiṣṭhaṃty uttare prāṇā baddhe padmāsane muhuḥ] J17
\item[uttiṣṭhaṃty uttame prāṇā baddhe padmāsane muhuḥ] W4
\item[uttiṣṭhaty uttame prāṇā baddhe padmāsane muhuḥ] N20
\item[uttiṣṭhaty uttame prāṇo baddhe padmāsane mukuḥ] N26
\item[uttiṣṭhaty uttame prāṇa rodhe padmāsana sthitaḥ] Bo2, J1, J3, N8, N11, N16, V2
\item[uttiṣṭhaty uttame prāṇa rodhe padmāsana sthite] C4, N25
\item[uttiṣṭhaty uttame prāṇāṃ rodhe padmāsane sthite] V19
\item[uttiṣṭhaty uttame prāṇa rodhe padmāsane sthite] V5, V28
\item[uttiṣṭhaty uttame prāṇa rodhe padmāsane sthitaḥ] J6, N2
\item[uttiṣṭaty uttame prāṇa rodhe padmāsanasthite]  N5
\item[uttiṣṭhaty uttame prāṇa rodhe padmāsanaṃ muhuḥ] N3, V15
\item[uttiṣṭhaty uttāme prāṇarodhe padmāsanaṃ muhuḥ] J5, J11,
\item[uttiṣṭhaty uttāme prāṇarodhe padmāsano muhuḥ] J7
\item[tattiṣṭhaty uttāmaprāṇarodhya padmāsanaṃ muhuḥ] J4
\item[uttiṣ(?)yaty uttame prāṇa rodho padmāsanaṃ muhuḥ] N24
\item[uttiṣṭhaty uttame kārye rodhe padmāsanae muhu] Ko,
\item[uttiṣṭhaṃty uttame prāṇa rodho padmāsane sthitaḥ] N18
\item[uṭṭiṣṭhaṃ cottare prāṇā baddhe padmāsane muhuḥ] N17
\item[uttiṣṭaty attame prāṇe baddhe padmāsane dṛḍhe] N1
\item[uttiṣṭhaty uttame prāṇe yāvat padmāsane samaḥ] N12
\item[uttiṣṭhaty uttare prāṇā baddhe padmāsane dṛḍhe] J12,
\item[uttiṣṭhaty unname prāṇe baddhe padmāsane dṛḍhe] N19
\item[uttiṣṭhaty uttame prāṇa baddha padmāsanaṃ dṛḍhaṃ] N21
\item[uttiṣṭety uname prāṇaḥ rodhe pavāsane sthite] N22
\item[uttiṣṭhaty uttame prāṇa rāvai padmāsano muhuḥ] N23
\item[uttiṣṭhaty uttame prāṇa rodhe padmāsano muhuḥ] V6
\item[uṭṭiṣṭhnty uttame prāṇā baddhe padmāsane muhuḥ] LD1,
\item[(illegible/unavailable)]   Ba5, Bo3, C5, N4a, V22, V12, V9
 \begin{description}

        \end{description}
\end{marma}

\begin{marma}[hp02_019]
\sthana{2.19c}
\item[kṛśatā] A1, B3, Ba, Ba5, Bo1, Bo2, Bo3, C1, C2, C6, C8, J1, J4, J5, J6, J7, J11, J13, J14, V19, N1, N2, N4a, N5, N8, N11, N12, N13, N16, N18, N19, N22, C4, V17, V10, V21, Tue, N23, N26, V7, V20, V5, V28, V22, V2, V18
\item[kṛsatā] Ko, J3,  
\item[kṛśate] J10pc
\item[kriyate] C3, J10, J12, J16, J17, LD1, N7, N9, N10, V12, C9, J2, N17, LD2, V19
\item[kṛyate] J15, N14
\item[kṛśatāṃ] J8, V8, N21, V3, N25, V6
\item[kṛṣitāṃ] N24
\item[kṛtā] N3, N20, V14
\item[krūratā] J2
\item[(illegible/unavailable)]   B1, B2, C5, V16, V15, V13, V9
 \begin{description}

        \end{description}
\end{marma}

\begin{marma}[hp02_021]
\sthana{2.22a}
\item[medaḥśoṣādikaṃ pūrvaṃ] V1
\item[medaḥśleṣmādikaṃ surve] V6
\item[medaśokādikaṃ pūrvaṃ] N22
\item[medaḥśleṣmādi nāśārthe] A1, C3,
\item[medaḥśleṣmādi nāśārtha] LD2,
\item[medaḥśleṣmādi ṣatkarmāṅi] V28
\item[medaḥśleṣmādi nāśārthaṃ] B1, B3, Bo1, C2, J10, J12, J13, J15, J16, J17, LD1, N7, V21, V20, V12
\item[medaḥśleṣmādināśīrṣaṃ] N9
\item[medaḥ tuṣmād adhikei purva śaṭ karmaṃ] V8
\item[medaśleṣmādi nāśārthaṃ] C9, N1, N10, N14, N21, V18
\item[medaśleṣādi nāśārthe] V17
\item[medaśleṣmādi nāsārthe] N20, N26
\item[medaśleṣma nivṛtyarthaṃ] Bo2, C1, J1, J3, J6, N2, N8, N11, N16, N18, V15, V5, V2
\item[medaḥśleṣma nivṛtyarthaṃ] N5, C4
\item[??śleṣma nivṛtyarthaṃ] N25
\item[medaḥśleṣmādhikaḥ pūrvaṃ] Ba, C8, J5, V10, V14, V13
\item[medaḥśleṣmādhikaḥ pūrva] J7,
\item[medaḥśleṣmādhikaḥ arthaṃ] V16
\item[medai śleṣmāṃdhika purṇa] J2
\item[medaśleṣmāṃdhikaḥ pūrvaṃ] Ba5, Bo3, V3
\item[medaḥ ślṣāṃdhikaḥ pūrvaṃ] C6, J14, 
\item[medaśleṣmādhikaḥ pūrvaṃ]  J11, N3, N4a, N12, N13, Tue, V22
\item[medāśleṣmāṣikaḥ pūrvva] N23, V7
\item[medaḥśleṣmaṇisapurṇa] N24 
\item[mitaḥ sthūlādikaiḥ kāryyaṃ] B2
\item[medodhikastu havabhiḥ] Ko,
\item[(unavailable/illegible)] C5, J4, N19, V9
 \begin{description}

        \end{description}
\end{marma}


\begin{marma}[hp02_021d]
\sthana{2.22d}
\item[apy abhāvataḥ] V1
\item[api bhāvataḥ] C3
\item[samabhāvataḥ] A1, B1, Ba5, Bo1, Bo3, C9, J1, J3, J6, J10, J11, J12, J13, J14, J15, J17, LD1, LD2, N3, N4a, N7, N8, N10, N11, N12, N13, V10, N20, N21, V14,V3, Tue, N26(perhaps left margin says something), V7, V20, V15, V28, V22, V18, V12, V13
\item[samabhāvat] J16,
\item[samupāgataḥ] J7,
\item[samabhyāvataḥ] N24
\item[api samācaret] V16, V17, V21
\item[samabhāvata]  N9
\item[samabhāvanā] N22
\item[samabhāvanāt] N22
\item[smabhāvata]  N14
\item[sabhāgataḥ] Ba
\item[samabhāgataḥ] B2, C6, C8, V8, N18, N23, 
\item[samabhāgatā] Ko,
\item[samabhāgata] V6
\item[samabhāgikaḥ] B3, C2, N1
\item[samabhārat] J2
\item[samatā yataḥ]  Bo2, C1, C4, N2, N16, N25, V2
\item[samatā yathā]   N5,
\item[samatāpataḥ] V5
\item[(unavailable/illegible)] C5, J4, N19
 \begin{description}

        \end{description}
\end{marma}


\begin{marma}[hp02_022a-c]
\sthana{}
 \begin{description}
        \end{description}
\end{marma}


\begin{marma}[hp02_023c]
\sthana{2.22c}
\item[vicitraguṇasandhāyī(i)] A1, B1, B3, Ba, Ba5, Bo1, Bo2, Bo3, C1, C2, C3, C8, C9, J1, J2, J3, J4, J5, J6, J7, J11, J13, J14, J15, LD1, LD2, V1, W4, J10, N17, J17, J8ac, Ko, N1, N3, N4a, N7, N9, V8, N10, N16, V16, V17, V10, N18(...saṃdhāī, no “y”), N19(...saṃdhāi, no “y”), N21, N22(...saṃdhāī, no “y”), V14, V3, Tue, N23, V7, V15, V5, V28, V22, V2, V18, V12, V13
\item[vicītraguṇasandhāyi] J16,
\item[vicitraguṇasandhāya]  C6, N12
\item[vicitraguṇasandhāyā] N20, V20
\item[vicitragunasandhāyai]  N14, V21, V6
\item[vicitraguṇasandhāryā]  N13
\item[vicitraguṇasandhāryaṃ]  J12,
\item[vicitraguṇasandhānaṃ] N26
\item[vicitraguṇasadyāpi] N2
\item[vicitraguptasaṃ?jñopi] N5
\item[vicitraguṇasaṃjñopi] C4, N25
\item[vicitraguṇasaṃdhāyoga]  N8("vivi?"), N11 ("yogasiddhidāyi" in brackets, then "yi" of "sandhāyi" )
\item[trividhaṃ guṇasandhāyi] B2
\item[(unavailable/illegible)] C5, N24
 \begin{description}

        \end{description}
\end{marma}


\begin{marma}[hp02_24c]
\sthana{2.24c}
\item[tataḥ pratyaharec caitad] J12, J16, J17, N23
\item[punaḥ pratyahare caitad] J13, J14, 
\item[punaḥ pratyaharec caitaḥ] J15,
\item[(unavailable/illegible)]
 \begin{description}

        \end{description}
\end{marma}


\begin{marma}[hp02_24d]
\sthana{2.24d}
\item[prakṣālyaṃ dhautikarma tat] Ba, C6, V21
\item[prakṣālanaṃ dhautikarma tat] V13
\item[prakṣālye? dhautikarma tat] C8
\item[ādāyaṃ dhautikarma tat] V14
\item[udgāraṃ dhautikarma tat] V1,
\item[sodgāraṃ dhautikarmavit] N12
\item[uditaṃ dhautikarma tat] A1, Ba5, Bo1, Bo3, C9, J10pc, J12, J15, LD1, LD2, N4a, N7, N9, N10, N13, N14("ti" added later, anusvara on "ka" seems cancelled), V16, V17, V10, N20, N21(“u” of uditaṃ doubtful), Tue, N26, V7, V20, V15, V18, V12
\item[aditaṃ dhautikarma tat] J17,
\item[uditaṃ dhautikarma kṛt] J10, V22
\item[up(?)itaṃ dhautikarma tat] C3
\item[abhyāsād dhautikarmavit] B1, B2, C4, J1, J3, J6, J13, V5, V28, V2
\item[abhyāsād dhautikarmasu] N25
\item[abhyāsā dhautikarmavit] N18
\item[abhyāso dhautikarmavit] N19
\item[abhyāse dhautakarmavit] J14,
\item[abhyāsād dhotikarmavit] C1
\item[abhyāsād dhautikarmavat] N5
\item[abhyāsaud dhautikarmavit] Bo2, C2,
\item[abhyāsāsau dhautikarmavit] V6
\item[abhyāsau dhautikarmatat] V8
\item[abhyāsād dhautikarmavita] N2
\item[abhyāsād dhautikarma tat] N1 (after correction),
\item[dukhalātā dhautikarmataḥ] V3
\item[duḥkhalṃ dhautikarma tat] Ko,
\item[duṣālaṃ dhautikarma tat] J4, J11, 
\item[uṣālaṃ dhautikarma tat] J5,
\item[abhyāsa dhautakarka tat] N22
\item[abhyāsād dhauti bhavati karma tat] N8("dhīta" instead of "dhauti"), N11 ("bhavati" in brackets)
\item[uṃtthānaṃ dhautikarma tat] N23
\item[dākṣālaṃ dhautikarma tat] N24
\item[utthānaṃ dhautikarma tat] J7,
\item[dugdhāraṃ dhautikarma tat] J2
\item[syā sā dhautikarmakṛt?] B3
\item[?s?vāla dhautikarmakṛt] N3
\item[(unavailable/illegible)] C5, N16
 \begin{description}

        \end{description}
\end{marma}

\begin{marma}[hp02_25d]
\sthana{2.25d}
\item[chadhyate ca na] J16,
\item[dhāvaty eva] V1, V8, V3, V2
\item[dhāvate ca na] B2
\item[dhāvaṃta ca na] N19
\item[dhāvaṃte ca na] J5,
\item[dhavaṃte ca na] N22
\item[dhruvntye neva] Ko,
\item[gachaty eva na] N21
\item[gachaṃty eva na] Bo1
\item[prayāntyeva na] Ba, N4a (2.26a), Bo2, Bo3, C1, C4, C6, C8, J1, J3, J6, J7, N5, N8, N11, N13, N16, V10, Tue, N25, V28, V22
\item[prayānteva na] N23, V7
\item[śudhyatty eva] A1,
\item[śudhyaty eva] LD2,
\item[śudhyate nātra] J12,
\item[śudhyate ca na] N10
\item[śudhyanty eva] B1, C3, C9, J10, J13, J17, N1 (not sure for "eva")
\item[śudhyant eva] B3,
\item[śudhyaṃteva na] J15, N9, N14, V16, V17
\item[śudhyaṃte ca na] LD1, 
\item[suddhyaṃtevaṃ na] N18
\item[śudhyaṃte ca na] N7
\item[śudhyaṃ sevana] C2,
\item[kṣīyaṃte saklā malā] J14,
\item[?īyatsyeva na] N2
\item[bhavaṃty eva na] N3, V14
\item[naśyaṃteva] N12
\item[naśyaṃte] J11
\item[nāśaṃ yāṃti] J4, 
\item[naśyaṃty eva na] J2,
\item[jayaty eva na] N24
\item[śuṣyaṃte sakalāmalāḥ] N20, N26
\item[(illegible/unavailable)] Ba5,C5, V21, V6, V20, V15, V5, V18, V12, V13
 \begin{description}

        \end{description}
\end{marma}          

\begin{marma}[hp02_26a]
\sthana{2.26a}
\item[udaragatapadārtham udvahatī] J12,
\item[udaragatapadārtham udvahantī] Ba, C3, J1, J7, Ko, V1, V16, N18, N22, N23, V6, V28, V2
\item[udaragatapadārthamūrdhāhatī] J16,
\item[uragatapadārtham udvahantī] B2,
\item[udaragatapadārthān udvahaṃti] C6
\item[udaragatapadārtham uşmahantī] V17
\item[udaragatapadārtham udvamaṃti(ī)] B1, B3, Ba5, Bo2, C2, C8, C9, J3, J6, J11, J14, LD1, LD2, N20, N21, V3, N26, V20, V22, V18, V12, V13
\item[udaragatapadārtham udvavaṃtī] J17,
\item[udaragatapadārtham udvavaṃtī] J10,
\item[udaragatapadārtham udvamaṃta] J4,
\item[udaragatapadārtham udvamaṃte] V15
\item[udaragatapadārthām udyamaṃte] N19
\item[udaragatapadārthaṃ śuddhapetaṃ] N25
\item[udaragatapadyarthān udvamaṃtī] A1
\item[udaragatapadārtham uḍhasaṃtī] N24
\item[udaragatapadārthaśodhanaṃ syāt] Bo1
\item[udaragatapadārtha śuddhayeta] C4
\item[udaragatapadārtha śuddhaye vai] J15, 
\item[udaragatapadārtha suddhayeta] J13,
\item[udagatapadārtha uddhanaṃtī] J5,
\item[(illegible/unavailable)] Bo3, C1, J2, V19, V10, V14, Tue, C5, V21, V7, V5, V22
 \begin{description}
        \end{description}
\end{marma}

\begin{marma}[hp02_26b]
\sthana{2.26b}
\item[pavanam apānam udīrya kaṇṭhanāle] V1, A1, B2, Ba, Ba5, C6, C8, C9, J3, J4, J6, J7, J10, J11, J14, J15, J17, Ko, LD1, LD2, N17, W4, J8, V16, N18, N20, N22, V3, N23, V20, V15, V28, V18, V12, V13 
\item[pavanam apānam udārya kaṇṭhanālo] J16,
\item[pavanām apānam udīrya kaṇṭhanālṃ] J1
\item[pavana apānam udīrya kaṇṭhanāle] V6
\item[pavanam upānam udīrya kaṇṭhanāle] C3
\item[pavanam upānam urīrya kaṇṭhanāle] C4
\item[pavanam apānam udīrya kaṇṭhanālo] N19
\item[pavanam apānam udīrya kaṇṭhanālāt] B3, C2, J13, N26, V2
\item[pavanam apām udirya kaṃnāle] N25
\item[pavanam apāsenam udāryaṃ kaṇṭhanāle] J5,
\item[pavanam mudārya kaṇṭhanālo] J12
\item[pavanam udirya kaṇṭhanāle] N21
\item[mapānam udīrya kaṇṭhanāle] B1,
\item[pavanamapānam udīrya kaṇṭhanāle] Bo1, Bo2,
\item[pavanam athānam utiya kaṇṭhanāle] N24
\item[pavanam apānam udīrya karāḍanāle] V17
\item[(illegible/unavailable)] Bo3, C5, J2, V19, V10, V14, V21, Tue, V7, V5, V22        
\end{description}
\end{marma}

\begin{marma}[hp02_26c]
\sthana{2.26c}
\item[karibhir iva jalasya vāyuvegāt] A1, B1, B3, C9, J1, J3, J6, J8, J10, J17, N17, W4, V16, V17, N18, N20, N26, V2, V18, V12 
\item[karibhir iva gajasya vāyuvegāt] V20
\item[karibhir iva jalasya vāyuvegā] B2, C3, J12, J16, 
\item[karibhir iva jalaṃ ca vāyuvegā] Ba
\item[karibhiriva jalaṃsya vāyuvegāt] Bo2, 
\item[karibhiriva jalasya vāyuvegāt] C2, LD1, 
\item[karibhiriva jalasya vāyuvego] J13,
\item[karibhiriva jalasya vāyuyogāt] V28
\item[kramamarivayavasya vāyur yo] V1
\item[kramaparicayavaśya nāḍicakrā] Ba5
\item[kramaparicayavaśyā nāḍīmārgā] C6, J7,
\item[kramaparicayavaśya vāyumārga] C8
\item[kramaparicayavasya vāyumārgre] J14,
\item[kramaparicayavaśya vāyumārgāt] Bo1
\item[kramaparicayavaśya vāyumārgaṃ] N21
\item[kramac pacya paricaya vaśyā vāyumārgaṃ] J5
\item[kramaparicayavaśyaṃ vāthagargāṃ] N24
\item[kramaparicayavaśya mārgte(??)] N23
\item[kramaparicayavasya vāyumārge] N22
\item[kramapārīcita vāyumārge] J15,
\item[kramaparicitavasya vāyuyogo] Ko,
\item[kramapārīcitavaśya vāyumārg(oṃ)?] N19
\item[kramapārīcitavaśya vāyumārgo] J11
\item[kramapārīcitavaśya vāyumārge] J4, 
\item[kramapārecitavaktu? vāyumārge] C4
\item[karamaparicayavaśya nāḍīmārgā] LD2
\item[kramaparicittakakrū vāyumārge] N25
\item[(illegible/unavailable)] Bo3, C1, J2, V19, V10, V14, Tue, C5, V21, V6, V7, V15, V5, V22, V13
 \begin{description}
        \end{description}
\end{marma}

\begin{marma}[hp02_26d]
\sthana{2.26d}
\item[gajakaraṇānti nigadyate haṭhajṇau] Ko,
\item[gajakaraṇīti nigadyate haṭhajñaiḥ] B2, Ba, Ba5, Bo1, C3, C4, C6, J4, J6, J7, J11, J15, V1, N19, N23, N25(“dya” unclear), V20, V15, V13
\item[gajakaraṇīti nigadate haṭhajñaiḥ] J14,
\item[gajakaraṇīti nigadyāta haṭhaj~naiḥ] J5,
\item[gajakaraṇīti nigadyaṃte haṭhajñaiḥ] LD2,
\item[gajakaraṇīti nigadyate] N24, V2
\item[gajakarṇīti nigadyate haṭhaj~naiḥ] A1, C8
\item[gajakaraṇīva nigadyate haṭhaj~naiḥ] B1
\item[gajakariṇīva nigadyate haṭhajñaiḥ] C2
\item[gajakaraṇīti nigachate haṭhaj~naiḥ] B3
\item[gajakāraṇī nigadyate haṭhaj~naiḥ] Bo2,V3, V6
\item[gajakaraṇīti nigadyate haṭhakṣaiḥ] N22
\item[gajakaraṇī ca nigadyate dṛḍhajñaiḥ] J13,
\item[jalagajakaraṇī nigadyate haṭhaj~naiḥ] J1, J3, 
\item[jalakaraṇīti nigadyate haṭhajñaiḥ] C9, J12, J16, LD1, N17, W4, N20, N26, 
\item[jalakaraṇīti nigadyate mahāhaṭhaj~naiḥ] V21
\item[jalakariṇīti nigadyate haṭhajñaiḥ] J10, J17, V16, V28, V18, V12
\item[jalakaraṇānī nigadyate haṭhaj~naiḥ] V17
\item[jalagajakaraṇīti nigadyate haṭhaj~naiḥ] J8, N18
\item[gajakaraṇīti nirādyate haṭhaj~naiḥ] N21(unusual way of writing “dya”)
\item[(illegible/unavailable)] Bo3, C1, J2, C5, V19, V10, V14, Tue, V7, V5, V22
 \begin{description}
        \end{description}
\end{marma}

\begin{marma}[hp02_29c]
\sthana{2.29c}
\item[aseṣadoṣān api yan nihanyād] Ko, 
\item[aśeṣadoṣopacayaṃ nihanyad] J8, N17, V1, W4, V16, V17, V10, V6, V7, V15, 
\item[aśeṣadoṣoyathā nihanyānād] V14
\item[aśeṣadoṣopacayaṃ nihanyād] B1, B2, B3, Ba, Ba5, Bo1, Bo2, Bo3, C1, C3, C4, C6, C8, C9, J2, J3, J5, J6, J7, J10, J11, J13, J14, J15, J17, LD1, LD2, N18(“aseṣa”), N19, N20, Tue, N25, N26, V20, V5, V28, V22, V2. V18, V12
\item[aśeṣadoṣopacaya no hanyād] J16,
\item[aśeṣadoṣopacayā nihanyād] V13
\item[aśeṣadoṣopacayaṃ nihanyā] N23(space for “da” left blank with an overhead line, verse comes before the earlier verse in this list)
\item[aśeṣadoṣopacaya nihanyād] N24(verse comes before the earlier verse in this list)
\item[aśeṣa?oṣovacayaṃ nihanyād] N21(verse comes before the earlier verse in this list)
\item[aśeṣadoṣepacayaṃti hanyād] J12,
\item[aśeṣadoṣepacayaṃ nihanyad] C2,
\item[aśeṣadoṣāpacayaṃ nihanyad] V21, V3
\item[aśeṣadoṣopacitiṃ hathajñād] J4,
\item[aśeṣadoṣopacayaṃ ca hanyad] A1
\item[aśeṣadoṣaprabhavaṃ nihanyad] V19
\item[aśeṣadoṣasya cayana hatyā] N22
\item[(illegible/unavailable)] C5,
 \begin{description}

        \end{description}
\end{marma}

\begin{marma}[hp02_29d]
\sthana{2.29d}
\item[abhyasyamānaṃ jalavastikarma] Ba5, Bo1, Bo2, Bo3, C1, C3, C6, C8, J1, J2, J4, J5, J6, J7, J8, J11, J12, J13, J14, J15, J16, LD1, N17, W4, V16, V17, V10, N18, N19, N21, V21, V3, Tue, N26, V6, V7, V20, V15, V5, V28, V22, V2, V18, V13
\item[anyasyamānaṃ jalavastikarma] J3,
\item[abhyasyamānaṃ nijavastikarma] V12
\item[abhyasyamānaṃ jalavastikarmaḥ] Ko,
\item[abhyasyamānaṃ khalu vastikarma] C9,
\item[abhyasyamāna ja?vastikarma] N20(something seems added above)
\item[abhyasyamānaṃ jalavastitakramāt] N24
\item[praamāńam jalavastikarma] V14
\item[abhyasyamāno jalavastikarma] B3, C2, 
\item[abhyasyamā jalavastikarma] B1,
\item[abhyāsamānaṃ jalavastikarma] C4, V1, N25
\item[abhyāsyamānaṃ jalavastikarma] A1, Ba
\item[abhyāsamānaṃ jalavasti] V19
\item[abhyasyamānaṃ javastikarma] J10, J17,
\item[abhyasyamānaṃ jvalabastikarmaḥ] LD2,
\item[bhyasyabhānaṃ jalavestikarma] N23(space for “da” left blank with an overhead line)
\item[abhyasya śūnyaṃ jalavastikarma] B2,
\item[dattasya mānaṃ yanabasyakarmaḥ] N22
\item[(illegible/unavailable)] C5,
 \begin{description}

        \end{description}
\end{marma}

\begin{marma}[hp02_30]
\sthana{2.30}
\item[atha natīḥ] J1
\item[atha netī(ḥ)] Ba, Ba5, Bo1, Bo2, C1, C2, C4, J3, J6, J8, V1, V10, N18, V21, V3, N23, N25, N26, V7, V5, V2
\item[atha neti] J2, Ko, N20, N24, V6, V28
\item[atha netiḥ] J7, Tue, V22, V12
\item[atha netikarma] C3, C9, J10, J17, LD1, W4, V16, V17, N21, V14, V15, V18, V13
\item[atha netikarmaḥ] C8, J12, J15, N22
\item[atha netīkarma] A1, B1, B2, C6, J13, J14, J16, N17, V20
\item[atha netīkarmaḥ] LD2
\item[atha nītī] J5,
\item[atha nītikarma] J4,
\item[atha nītīkarma] N19
\item[prathama netīkarma] B3
\item[(illegible/unavailable)] Bo3, C5, J11, 
 \begin{description}

        \end{description}
\end{marma}

\begin{marma}[hp02_30c]
\sthana{2.30c}
\item[mukhanirgamanād eva] A1, B2, C3, J10, J17, LD1, N17, V1, W4, V16, N22, V20, V18, V12, V13
\item[mukhanirgamanād evaḥ] J15,
\item[mukhanirgamanād deva] C9
\item[mukhanirgatanād evaṃ] J12,
\item[mukhanirgatamanād eva] V14
\item[mukhanirgamayedeti] J2
\item[mukhan nirgamanād eva] B3,
\item[mukhān nirgamanād eva] C2,
\item[mukha nīrgamanod eva] J16, 
\item[mukhān nirgamayec caiṣā] Ba, Ba5, Bo1, C8, J8, J11, LD2, V17, N19, N20, V21, Tue, N26, 
\item[mukhān nirgamyate caiṣā] J4,
\item[mukhān virgamayed eṣā] N21
\item[mukhā nirgamayec caiṣā] J7, V10, V3, V6, V7, V15, V22
\item[mukhe nirgamayec caiṣā] C6
\item[mu nirgamayai caiṣā] J5,
\item[mukhān nirgamaye doṣā] N24
\item[mukhān nirgamayet sāpi] V19
\item[mukhān nirgāmayet sā hi] B1, Bo2, J13, J14, N18, N25  
\item[mukhān nirgamayet sā hi] J6,
\item[mukhā nirgāmaye sā hi] V2
\item[mukhā nirgāmayet so hi] Ko,
\item[mukhā nirgamanā sā hi] V5
\item[mukhān nigarmayet sā hi] C4, J3,
\item[mukhā nigarmayet sā hi] J1, V28  
\item[mukhān nirgatya nāsāyam] C1,
\item[mukhaśca nirgamec caiṣā] N23
\item[(illegible/unavailable)] Bo3, C5
 \begin{description}

        \end{description}
\end{marma}

\begin{marma}[hp02_33a]
\sthana{2.33a}
\item[ghaṭanaṃ] J2, 
\item[muñcakaṃ] Ko,
\item[mucyate] J4,
\item[mocanaṃ] B2, Ba5, C8, J7, J12, N18, N21, N22, Tue, N23, V7
\item[mocakaṃ] Bo2, C1, J1, J3, J5, J6, J11, J14, N19, V28
\item[moṭakaṃ] V1, Bo1, N25 
\item[toṭakaṃ] V19, V16
\item[tāṭakaṃ] V17
\item[trāṭakaṃ] A1, C4, C6, V10, V14, V21, V3, V6, V20, V15, V5, V28, V22, V2, V18, V12, V13
\item[troṭakaṃ] N24
\item[rocanaṃ?] Ba,
\item[sphoṭanaṃ] B1, B3, C2, C3, C9, J8, J10, J13, J15, J16, J17, LD2, N17, W4, N20, N26,  
\item[sphoṭakaṃ] LD1,
\item[(illegible/unavailable)] Bo3, C5
 \begin{description}

        \end{description}
\end{marma}

\begin{marma}[hp02_33b]
\sthana{2.33b}
\item[jadrānāṃ ca kapāṭakaṃ] J16,
\item[tandrādīnāṃ gapāṭa[ka]m] V1,
\item[tandrādīnāṃca pāṭanaṃ] V19,
\item[tandrādīnāṃ ca kapāṭakaṃ] B1,
\item[tandrādīnāṃ kapāṭakam] B2, B3, Ba, Ba5, Bo1, C1, C2, C4, C6, C8, J3, J5, J6, J7, J8, J11, J12, Ko, LD1, LD2, W4, V17, V10, N21, N22, V3, Tue, V7, V5, V28, V22, V2, V13
\item[tandrādīṇāṃ kapāṭakam] J13, 
\item[tandrāṇāṃ ca kapāṭakam] J15, 
\item[tandrādīnāṃ kapālakam] J1
item[tandrāegaṃ kathapāṭakam] V21
\item[tandrādīnāṃ kapāṭakai] Bo2
\item[tandrādiṇāṃ kapāṭakam] J8, N18, N23,
\item[tandrādiṇāṃ ca pāṭakam] J14, N19
\item[tandrādīṇāṃ ca pāṭakam] C9
\item[tandrādīṇāṃ ca pāṭhakam] V15
\item[tandrādīṇāṃ ca pāṭavam] J4,
\item[tandrāṇāṃ ca kapāṭakam] C3, J10, J17, V16, N20, V14, N26, V20, V18
\item[tadvāṇāṃ ca kapāṭakam] A1,
\item[tandrāṇāṃ tu kapāṭakam] N17
\item[sandrādīnāṃ kapātakāṃ] N24
\item[nidrārttīnāṃ kapāṭakam] N25
\item[hetādīnāṃ? kapāṭhakam] J2,
\item[(illegible/unavailable)] Bo3, C5, V6, V12
 \begin{description}

        \end{description}
\end{marma}

\begin{marma}[hp02_34b]
\sthana{2.34b}
\item[gudaṃ] V1
\item[gopyaṃ] V16, V17, V10, N18, V14, V21, V3,V6, V7, V20, V15, V28, V22, V2, V18, V12, V13
\item[tundaṃ] A1, B3, Ba5, Bo2, C2, C8, C9, J3, J6, J7, J8, J13, J14, J16, LD1, LD2, N19, N20, Tue, N26, 
\item[tunduṃ] N23
\item[tunda] N17
\item[tunde] C1, C4,  
\item[tuṃḍe(?)] N25, V5
\item[tundat] B2, J10, J17, 
\item[tundet] J15, 
\item[truṃdaṃ] N21
\item[tuda] W4
\item[tudaṃ] B1, C3, C6, J1, J5, J12,  
\item[tundan] J4,
\item[tudan] Ba, Bo1, J11, 
\item[tulyaṃ] Ko,
\item[tujyaṃ] V19
\item[t(?)uddhaṃ] N22
\item[puṃda] N24
\item[saṃdada] J2
\item[(illegible/unavailable)] Bo3, C5,
 \begin{description}

        \end{description}
\end{marma}

\begin{marma}[hp02_35a]
\sthana{2.35a}
\item[pāvanādi] V1
\item[pāvanāni] N18,
\item[pāvanāgniḥ] N25
\item[pāvanādi] Ko,
\item[pāvakādi] C3, J10, J16,
\item[pāvakādī] J12,
\item[pāvakāgni] N26
\item[pācakādi] C2, C9, J17, LD1, LD2, N17, W4, V16, V17, N20, V14, V21, V18, V13
\item[pācatādi] Bo2, 
\item[pācanādi] A1, B1, B2, B3, Ba, Ba5, Bo1, C6, C8, J4, J5, J7, J8, J11, J14, V19, V10, N19, N21, V3, Tue, N23, V6, V7, V20, V15, V28, V22
\item[pācanāgni] J3, J6, J13,  
\item[pācakāgniḥ] J15
\item[pācakāgni] V2
\item[pācakāri?] J2,
\item[pācanāṃnī] N24
\item[kārkādeśaṃ] N22
\item[(illegible/unavailable)] Bo3, C5, V5, V12
 \begin{description}

        \end{description}
\end{marma}

\begin{marma}[hp02_35c]
\sthana{2.35c}
\item[apa śoṣaṇī ca] Ba5,
\item[mala śoṣiṇī va] V1,
\item[mala śoṣaṇī ca] B2, C3, J8, LD1, N17, W4, V16, V14, V3, V7, V18, V12
\item[maya śoṣaṇī ca] V21, V15, V28, V13
\item[maya śodhaṇī yā] N23
\item[maya śodhaṇī cā] V6, 
\item[mala śoṣaṇī va] J10, J17,
\item[mala śoṣaṇī vaḥ] J16,
\item[doṣāmaya śoṣaṇī ca] C1, C2, C4, C6, C8, C9, J3, J4, J11, J13, J14,  Ko, N18, N21, Tue, N25, V22, V2
\item[doṣāmaya śauṣaṇī caḥ] J15, 
\item[doṣān mala śoṣaṇī ca] J12,
\item[doṣāmaya śoṣaṇīya] J1
\item[doṣāmaya śoṣaṇīyaṃ] J2
\item[doṣāmaya śoṣiṇī ca] A1, B3, Ba, Bo1, J6, V19, N19
\item[doṣāmala śoṣaṇī ca] N20
\item[doṣāmaya śoṣiṇāṃ ca] Bo2
\item[doṣāmaya śodhinī ca] J7, 
\item[doṣābhayahā soṣaṇī ca] J5,
\item[dauṣāmaya śoṣaṇī ca] LD2,
\item[doṣāmaya goṣaṇī ca] N24
\item[mala śoṣiṇī ca] V17, N26,
\item[doṣāṇapi śoṣaṇā ca] B1,
\item[doṣāmaya śoṣaṇī ca] V10
\item[doṣapra śoṣiṇī] V5
\item[doṣānala śoṣaṇī bahava] N22
\item[(illegible/unavailable)] Bo3, C5, V20, 
 \begin{description}

        \end{description}
\end{marma}

\begin{marma}[hp02_35d]
\sthana{2.35d}
\item[mūlam iyaṃ hi] J14, V1
\item[mūlam idaṃ hi] Ko, J4, J11, N19
\item[mūlam ayaṃ ca mauliḥ] N22
\item[maulir iyaṃ hi] A1, C1, C4, J3, J5, V19, V21, N25+, V15, V2
\item[maulir iyaṃ] J1
\item[maulir iyaṃ ca] Ba, Ba5, Bo1,Bo2, C6, C8, J6, J7, N21, Tue, N23(“yaṃ” looks like “paṃ”), V6, V7, V20, V28, V22
\item[saulir iya ca] N24
\item[maulir ayaṃ hi] J2
\item[sau jayatīha] J10, J17, N17, W4, N18, N20, N26
\item[jayatīha] B1, B3, C2, C3, C9, J8, J12, J13, J15, J16, LD1, LD2, V3
\item[sauryakarīha?] B2
\item[(illegible/unavailable)] Bo3, C5, V16, V17, V10, V14, V5, V18, V12, V13
 \begin{description}

        \end{description}
\end{marma}

\begin{marma}[hp02_38a]
\sthana{2.38a}
\item[prāṇāyāmena vai] Ko, J4, J11, J14, V1, N19, V15
\item[prāṇāyāmair eva] B1, B2, B3, Ba, Ba5, Bo1, Bo2, C1, C2, C4, C6, C8, J1, J3, J5, J6, J7, J12, J13, J15, LD2, N1 (not sure of "eva"), N3, N4a, N5, N8, N11, N12, N13, N16, N18, N22, V3, Tue, N23, N24, N25, V6, V7, V5, V28, V22, V2, V12, V13
\item[prāṇāyāmaiḥ sukhāt] A1, C3, C9, J10, J17, LD1, N2 ("śu"), N9, N10,
\item[prāṇāyāme sukhāt] J16,
\item[prāṇāyāmaiḥ sukhāḥ] N26
\item[prāṇāyāmair mu?khāṃ] N20
\item[prāṇāyāmena sarvepi] N21
\item[praṇāyāme kṛte samyak] J2
\item[prāṇāyāme śukhā] N14, V18
\item[prāṇāyāme tataḥ] V16, V17, V10, V14, V21, V20
\item[praṇāyāmāt sukhāt] N7
\item[(illegible/unavailable)] Bo3, C5,
 \begin{description}

        \end{description}
\end{marma}

\begin{marma}[hp02_38b]
\sthana{2.38b}
\item[malā api] J4, J11, V19, N21, V15, V28
\item[malād api] Ko,
\item[malā iti] Ba, Ba5, Bo1, Bo2, C1, C6, C8, J1, J2, J3, J5, J6, J7, LD2, N3, N4a, N5, N8, N11, N13, N16, V10, N18, N19, Tue, N23, N24, N25, V6, V7, V22
\item[malā itiḥ] J15,
\item[malā ime] J14, 
\item[malāśayaḥ] J10,
\item[malāśayāḥ] A1, B1, B3, C2, C9, J13, LD1, N1, N2, N10, V17, V14, V20, V18, V13
\item[malākulaṃ] B2, N22
\item[malākulāḥ] N20, N26
\item[malāśayā(ḥ)] C3, C4, J16, J17, N9, V16
\item[marāśayā] N14
\item[malāsayā] J12, V21, V3, V2
\item[malāśrayāḥ] N7
\item[malāyinaḥ] N12
\item[(illegible/unavailable)] Bo3, C5, V5, V12
 \begin{description}

        \end{description}
\end{marma}

\begin{marma}[hp02_39]
\sthana{2.39}
\item[]
 \begin{description}

        \end{description}
\end{marma}

\begin{marma}[hp02_40b]
\sthana{2.40b}
\item[cirāśrayam] LD2,
\item[nirāśrayam] A1, Ba, C1, C3, C6, C8, C9, J2, J4, J5, J7, J8, J10, J11, J15, J17, LD1, V1, W4, N3, N9, N10, V17, N20, N22, V21, V3, N23, N26, V6, V15, V28, V12, V13
\item[nirāśrayaḥ] J12,
\item[nirodhā vādham] V8
\item[nirāśayam] Bo1,
\item[nirāsnayām] J16,
\item[nirāmayam] B2, Bo2, C4, J1, J3, J6, J14, Ko, N17, V19, N2, N5, N7 ("ma" seems tampered with later), N8, N11, N12, N14, N16, V16, N18, N19, N21, N24, N25, V20, V5, V2, V18
\item[nirākulam] B1, B3, Ba5, C2, J13, N1 (2.45), N4a, N13, V10, Tue, V7, V22
\item[(illegible/unavailable)] Bo3, C5, V14
 \begin{description}

        \end{description}
\end{marma}

\begin{marma}[hp02_40c]
\sthana{2.40c}
\item[akṣir?] C8,
\item[īkṣed] Ba, J5, 
\item[īkṣet] C6, V13
\item[īkṣe] C1, J7, 
\item[paśyed] A1, Bo1, C9, J4, J8, J10, J11, J14, J15, J16, J17, Ko, LD1, N17, V1, W4, N9, V8, N10, V16, V17, N20, N22(pasyed), V21, V3, N26, V15, V12
\item[paśyad] J12,
\item[paśyet] V20
\item[paśyeta] LD2,
\item[paśye] N19, N24
\item[paśe] C3
\item[paśyan] N7
\item[pasya?] N14  
\item[dṛṣṭir] B3, Ba5, C2, C4, J6, V19, N1, N2, N4a, N11, N16, V10, N21, N25, V7, V5
\item[dṛṣṭi] B1, B2, Bo2, J1, J2, J3, J13, N5, N8, N12, N13, N18, Tue, V28, V22, V2 
\item[vīkṣed] N3
\item[iched] N23
\item[(illegible/unavailable)] Bo3, C5, V14, V6, V18
 \begin{description}

        \end{description}
\end{marma}

\begin{marma}[hp02_43a]
\sthana{2.43a}
\item[vidhānajñā] C9, J5, J10, J11, Ko, V1, N2 ("vi" seems faded), N4a, N7, N10, N11, N12, N13, N14, N16, V16, V17, V10, N19, N20, N21, V14, V3, Tue, N23, N26, V6, V7, V15, V22, V18, V13
\item[vidhānaḥjñā] J2
\item[vidhānajñe] V2
\item[vidhanajñā] N18
\item[vidhaustajñā] V20
\item[vidhānujñaś] J1
\item[vidhānajñāś] Bo2, C1, C3, C8, J4, J6, J7, J12, 
\item[vidhānajñaś] B1, B3, Ba, Ba5, Bo1, Bo2, C1, C2, C3, C8, J3, J16, N3, N24,
\item[vidhānajña] N22, V12
\item[vidhānejñaś] N8
\item[vidhijñās tu] C6,
\item[vidhānajño] C4, J15, J17, LD1, LD2, N5, N25, V28
\item[vidhānajñoś] J13, 
\item[vividhānajño] B2, 
\item[vidhāna?ī] N9, V8
\item[vidhānajñaiś] N1 (unsure about "ai" matra)
\item[vidhiṃstajñaś] A1
\item[(illegible/unavailable)] Bo3, C5, J14, V21, V5
 \begin{description}

        \end{description}
\end{marma}

\begin{marma}[hp02_44cd]
\sthana{2.44cd}
\item[bhastrakā bhrāmarī vedhā kevalaṃ cāṣṭakumbhakāḥ] J2
\item[bhastrikā bhrāmarī mūrcchā kevalī cāṣṭakumbhakāḥ] V1, V16, V17, 
\item[bhastrikā bhrāmarī mūrcchā kevala cāṣṭakumbhakāḥ] V20, V18, V12, V13
\item[bhastrikā bhrāmarī mūrcchā vinītyaṣṭakumbhakāḥ] V10
\item[bhastrikā bhrāmarī mūrcchā vinītyaṣṭakumbhakān] V28
\item[bhastrikā bhrāmarī mūr??ā kevalaś cāṣṭakumbhakā] N1,
\item[bhastrīkā bhrāmarī mūrchā kevalaś cāṣṭakumbhakāḥ] C9, J13, J15, LD1, LD2, N21
\item[bhaṃstrikā bhramarī mūrchā kāṃbalā atha? kumbhakāḥ] J5
\item[bhaṃstrīkā bhramarī mūrchā kevala cāṣṭakumbhakāḥ] N22, V21, V3
\item[bhastṛ(?)kā bhrāmarī mūrcchā kevalaś cāṣṭakumbhakāḥ] N26
\item[bhastrikā bhrāmarī mūrchā kaivalaś cā sa kumbhakaḥ] B3,
\item[bhastrikā bhramarī mūrcchā cāṣṭakumbhakāḥ sahita?] N2 (after exchange of words' positions)
\item[bhastrikā bhramarī mūrchā kevalaś cāṣṭakumbhakāḥ] A1, B1, C3, J7pc, J10, J12, J16, J17, N3, N7, N10("ri" instead of "rī"),
\item[bhastrikā bhrāmarī mūrchā kevalaś cāṣṭakumbhakāḥ] N20
\item[bhastrikā marī mūrchā kevalaṃ cāṣṭakumbhakāḥ] B2,
\item[bhastrikā bhrāmalī mūrchā kevalaś cāṣṭakumbhakā] N14
\item[gachati tiṣṭto kāryā mujjayākhyāṃ cākumbhakāḥ] V8
\item[bhastrikā bhramarī mūrchā kevalaś cāṣṭakumbhakā] N9
\item[bhastrikā bhrāmarī mūrchā plāvanī saṣṭhakumbhakāḥ] N4a
\item[bhastrikā bhramarī mūrcchā plāvanī tvaṣṭakumbhakāḥ] N12, V6, V7
\item[bhastrikā bhramarī mūrcchā pāvanīty eṣakumbhakāḥ] Ko,
\item[bhastrikā bhramarī mūrcchā pāvanīty aṣṭakumbhakān] J4,
\item[bhastrikā bhrāmarī mūrchā plāvinītyaṣṭakumbhakāḥ] Ba5, Bo1, C1, N13, V14, Tue(no visarga at the end), V15, V22
\item[bhastrikā bhrāmarī mūrchā prāvatītyaṣṭakumbhakaḥ] N23 
\item[bhastrikā bhramarī mūrchā plāvanītyaṣṭakumbhakāḥ] Ba, C8, J7, 
\item[bhastrikā bhrāmarī mūrcchā pālāvanīty aṣṭakumbhakāḥ] J11,
\item[bhastrikā bhramarī mūrchā pratvanītyaṣṭakumbhakāḥ] C6
\item[bhastrikā bhramarī mūrchā sahitāś cāṣṭakumbhakāḥ] C4, N5
\item[bhastikā bhrāmarī mūrchā sahitāś cāṣṭakumbhakam] N25, V5
\item[bhastrikabhrama ma mūrchā saṃhataṃ cāṣṭakumbhakā] N8
\item[bhastrikā bhramarī mūrchā saṃhataṃ cāṣṭakumbhakāḥ] N11(visarga added later), N16 ("saṃhitaṃ"),
\item[bhastrikā bhramarī mūrchā saṃhitaṃ cāṣṭakumbhakāḥ] Bo2, J3, J6, V2
\item[bhastrikā bhramarī mūrchā saṃprataṃ cāṣṭakumbhakān] N18
\item[bhadrikā? bhramarī mūrchā saṃprataṃ rāṣṭakumbhakāṃ] J1
\item[bhastrikā śītalī bhramarī mūrchā plāvanītyaṣtakumbhakāḥ] N19 
\item[sastrikā bhramarī mūrchā kavalaś cāṣṭakumbhakā] N24
\item[(illegible/unavailable)] Bo3, C5, J14, 
 \begin{description}

        \end{description}
\end{marma}

\begin{marma}[hp02_46]
\sthana{2.46}
\item[om.] V19,
\item[adhas tāt kuñcanenāśu kaṇṭhasaṃkocane kṛte | madhye paścimatānena syāt prāṇo brahmanāḍīkaḥ ||] Ba5
\item[adhas tāt kuñcane vāyuḥ kaṃṭhasaṃkocane kṛte | madhye paścimatāṇena syāt prāṇo brahmanāmīgaḥ ||] Ko,
\item[adhas tāt kuñcanenāśu kaṇṭhasaṃkocane kṛte | madhye paścimatānena syāt prāṇo brahmanāḍīkagḥ ||] V22
\item[adhas tāt kuñcanenāsuṃ kaṇṭhasaṃkocane kṛte | madhye paścimatānena syāt prāṇo brahmanāḍīkāṃ ||] C6
\item[adhas tā kuñcakīnālau kaṇṭhasaṃkocane kṣate | madhyaṃ paścimatāṇena syāt prāṇo brahmanāḍīkaḥ ||] J2
\item[adhas tāt kuñcanenāśu kaṇṭhasaṃkocane kṛte | madhye paścimatānena syāt prāṇo brahmanāḍigaḥ ||] A1, Bo1, C8, J10, J11, V1, N3, N13, Tue, V7, V20, V15, V28, V12, 
\item[adhastvākuñcanenāśu kaṇṭhasaṃkocane kṛte | madhye paścimatānena syāt prāṇo brahmanāḍigaḥ ||] C9
\item[adhas tāt kuñcanenāśu kaṇṭhasaṃkocane kṣate | madhye paścimatānena syāt prāṇo brahmanāḍigaḥ ||] C1,
\item[adhastāt kuñcanenāśu kaṇṭhasaṃkocane kṛte | madhyapaścimatānena syāt prāṇo brahmanāḍigaḥ ||] C3, LD1, N12, N26
\item[adhastā kuñcanenāśu kaṇṭhasaṃkocane kṛte | madhyapaścimatānena syāt prāṇo brahmanāḍigaḥ ||] B2, J15, V10, V21
\item[adhasthāt kuñcanenāśu kaṇṭhasaṃkocane kṛte | madhye paścimatānena syāt prāṇo brahmanāḍigaḥ ||] J7,
\item[adhastāt kuñcanenāśu kaṇṭhasaṃkocane kṛte | madhyapaścimatānena syāt prāṇo madhadhyanāḍigaḥ ||] J6, J17, 
\item[adhastāt kuñcanenāśu kaṇṭhasaṃkocane kṛte | madhye paścimatānena syāt prāṇo madhadhyanāḍigaḥ ||] Bo2, N10, V16, V17, V14, V3, V6
\item[adhastā kuñcanenāśu kaṇṭhasaṃkocane kṛte | madhye paścimatānena syāt prāṇo madhadhyanāḍigaḥ ||] J3,
\item[adhastā kuñcanenāśu kaṇṭhaśaṃkocane kṛte | madhye paścimatānena syāt prāṇo madhadhyanāḍiga ||] J12,
\item[adhas tāt kuñcanenāśu kaṇṭhasaṃkocane kṛte | madhye paścimatāna syāt prāṇo brahmāṇḍanāḍigaḥ ||]  Ba, N4a
\item[adhastākuñcanenāśu kaṇṭhasaṃkocane kṛte | madhyapaścimatānena syā? prāṇo brahmanāḍikaḥ ||]  N5
\item[adhastāt kuñcanenāśu kaṇṭhasaṃkocane kṛte | madhyapaścimatānena syāt prāṇo brahmanāḍikaḥ ||] C4
\item[adhastākuñcaṇeṇāśu kaṇṭhaśaṇkocane kṛ(?)te | madhye paścamatānena syāt prāṇo madhye nāḍikā ||] N14
\item[adhastāt kuñcanenaiva kaṇṭhasaṃkocane kṛte | madhye paścimatānena syāt prāṇo brahmanāḍikaḥ ||] N1
\item[adhastāt kuñcanenaiva kaṇṭhaṃ saṃkocane kṛte | madhye paścimatānena syāt prāṇo brahmanāḍikaḥ ||] B3
\item[adhas tāt kuñcanenāśu kaṇṭhasaṃkocane kṛte | madhye paścimatāṇena syāt prāṇo madhyanāgiḍaḥ ||] J4,
\item[adhas tāt kuñcanenaiva kaṇṭhasaṃkocane kṛte | madhye paścimatānena syāt prāṇo madhyanāgiḍaḥ ||] J4,
\item[adhamān kuñcanenāśu kaṇṭhasaṃkocane kṛte | madhye paścimatāṇena syāt prāṇo brahmanāḍigaḥ ||] J5,
\item[adhastāt kuñcanenaiva kaṇṭhaṃ saṃkocane kṛte | madhye paścimatānena syāt prāṇo brahmanāḍigaḥ ||] C2
\item[adhastāt kuñcanenaiva kaṇṭhasaṃkocane kṛte | madhyapaścimatānena syāt prāṇo brahmanāḍigaḥ ||] B1
\item[adhasthāt kuñcanenaiva kaṇṭhasokocane kṛte | madhyapaścimatānena syāt prāṇo brahmanāḍigaḥ ||] N21
\item[adhastāt kuñcanenāsu kaṇṭhasaṃkocane kṛte | madhyapaścimatānena syāt prāṇo madhyanādigaḥ ||] J16, 
\item[adhastā kuñcanenāsu kaṇṭhasaṃkocane kṛte | madhye paścimatāṇena syāt prāṇo madhyanāḍigaḥ ||] LD2,
\item[adhastvā kuñcanenāsu kaṇṭhasaṃkocane kṛtaṃ | madhyapaścimatānena syāt prāṇo madhyanāḍigaḥ ||] N7, V13
\item[adhastvā kuñcanenāsu kaṇṭhasaṃkocane kṛte | madhyapaścimatānena syāt prāṇe brahmanāḍigaḥ ||] N25
\item[adhastāt kuñcanenāśu kaṇṭhasaṃkocane kṛte | madhyapaścimatānena syāt prāṇo madhyanāḍigaḥ ||] N16
\item[adhastāt kuñcanenāśu kaṇṭhasaṃkocane kṛte | madhye paścimatānena syāt prāṇo madhyanāḍigaḥ ||] J1, N18, N19
\item[adhastāt kuñcanenāsu kekatvaṃ saṃkocane krate madhye paścimatānena syāt prāṇo vasta(?)nāḍ?gaḥ ||] N9
\item[adhastāt kuñcatenāśu kaṇṭhasaṃkocane kṛte | madhyapaścimatānena syāt prāṇo brahmanāḍitāḥ ||] N20
\item[adhas tā kucanenāmaḥ kaṃcasaṃkocane kṛte | madhye paścimatānena prāṇasyā brahmanāḍīkaḥ ||] N22
\item[adhas ta kuñcanenāśu kaṇṭhasaṃkocane kṛte | madhyapaścimatānena syāt prāṇo brahmanābhirāt ||] N23
\item[adhas tāt kuñcanaunāśu kaṇṭhakocane kṛte | madhye paścimatāne tasyāt pāno madhyanāḍigaḥ ||] N24
\item[adhas tāt kuñcanaunāśu kaṇṭhakocane kṛte | madhye paścimatāne tasyāt prāno madhyanāḍigaḥ ||] V18
\item[(illegible/unavailable)] Bo3, C5, J14, N2, N8, N11, V5, V2
 \begin{description}

        \end{description}
\end{marma}

\begin{marma}[hp02_48b]
\sthana{2.48b}
\item[āsthāpya] V1, V19
\item[sansthāpya] V8
\item[utthāpya] A1, B1, B2, B3, Ba, Ba5, Bo1, Bo2, C2, C3, C4, C6, C8, C9, J1, J3, J6, J7, J8, J10, J11, J13, J17, LD1, N17, W4, N2, N3, N4a, N5, N8 (not sure of "pya"), N9, N10(seems "chā" instead of "thā"), N11, N12, N13, N16, V16, V17, V10, N18, N21, V14, V21, V3, Tue, N23, N24, N26, V6, V7, V15, V28, V22, V18, V12, V13,
\item[ūtthāpya] J16,
\item[uthāpya] Ko,
\item[usthāpya] J2, J12, 
\item[udhyāpya] N19
\item[ur(?)thāpya] N14
\item[utthāya] C1, N7, N20,
\item[utprāpya] N22
\item[uschāpya] N1 (not sure)
\item[usthāpya] J4,
\item[uchāpya] J5, J15, 
\item[utchāpya?] LD2,
\item[(illegible/unavailable)] Bo3, C5, J14, N25, V20, V5, V2

 \begin{description}

        \end{description}
\end{marma}

\begin{marma}[hp02_49b]
\sthana{2.49b}
\item[virōdhāvadhi kumbhayet]
\item[nīrodhāvadhi kumbhayet] J15, J16, J17, V21
\item[nīrodhānadhi kumbhayet] V17
\item[nirodhaṃ kumbhayet tataḥ] J13, 
\item[(illegible/unavailable)] J14,

 \begin{description}

        \end{description}
\end{marma}



\begin{marma}[hp02_54a]
\sthana{2.54a}
\item[śītkaraḥ] B1, Ba, C8, 
\item[śītkaram] C6,
\item[śīkāra] J7,
\item[śītkāraḥ] C1, B3, J13, J15, V17, V21 
\item[śītakārāī] LD1, 
\item[sītkara(ḥ)] B1, Ba, Bo2, C2, C3, C4, J3, J4, 
\item[sītkāraḥ] J6
\item[śītkārī] J16, 
\item[sītkāṃ] N18, N22, Tue, N24, N25, V15, V28, V22, V10
\item[sītkād] V14
\item[sītkā] Ko,
\item[śītkāṃ] A1
\item[sītkarā] J17,
\item[sītkārī] Ba5, Bo1, J5, J12, 
\item[sītakārī] C9
\item[sītakarākarmma] J10, 
\item[śītikārī] LD2,
\item[sītalī] J2
\item[sikta?] J1
\item[sītkīṃ] V19,
\item[sitkāṃ] J11, N19
\item[sinktā?] B2
\item[sātkāṃ] N21
\item[dhikt(?)āṃ] N23
\item[kumbhaṃ] V3,J8, N20,V14, V21, N26, V16  
\item[kumbhakaṃ] V7, V20, V18, V12, V13, 
\item[kumbhakaḥ] V3
\item[]
\item[(illegible/unavailable)] Bo3, C5, J14, V1, V6, V5, V2, 
 \begin{description}

        \end{description}
\end{marma}


\begin{marma}[hp02_55a]
\sthana{2.55a}
\item[cakram āsādya(ḥ)] A1, B3, Ba, Bo1, C2, V1, J8pc, J10, J17, P28, V12, N21(unusual way of writing “dya”), V14, V20, V18, V16
\item[cakrasāmānya(ḥ)] B1, J8ac, J11, J13, N19, N24, V3, V22
\item[cakrasammānyaḥ] Ba5, Tue, V10
\item[cakrastamānyaṃ] J2
\item[cakrasāmānyaṃ] LD2
\item[cakramādyāya] J12,
\item[cakrasāmānya] J4, J5,
\item[cakraṃ samāsādya] N17, W4
\item[cakrasamāsādya] LD1,
\item[cakranāśāya] N20, N26, V17
\item[cakrabhogyaś ca] N22
\item[cakrasevyas tu] Bo2, C1, C4, C6, C8, J1, J3, J6, J7, J15, V19, N18(“ca” doubtful), V21, N25, V15, V28, V2, V13, V21
\item[cakrasevyas ta] N23
\item[cakrabhogyas tu] N22
\item[cakrābhimānyaḥ] Ko,
\item[vaktram āsādya] C3, C9
\item[cakramadyāya] J16,
\item[(illegible/unavailable)] Bo3, C5, J14, V6, V5
 \begin{description}

        \end{description}
\end{marma}


\begin{marma}[hp02_56*1]
\sthana{2.55*1[c][d]}
\item[satyāmūlena randhreṇa] C4, J6, J8pc, J10, J17, V21, N25
\item[satyāmūlena randhrena] J15,
\item[sadāmūlena randhreṇa] J3, J6pc
\item[satyāmūlena randhrenayaḥ] V18, V16, V21
\item[spṛśyamūlena randhreṇa] B1, B3, C2, J13,  
\item[rasanātālumūlena] N17, 
\item[rasanāmūla randhreṇa] C9, 
\item[rasavatyāmūla randhre yaḥ] N26
\item[]
\item[]
\item[(illegible/unavailable)] A1, B2, Ba, Ba5, Bo1, Bo2, Bo3, C1, C3, C4, C5, C6, C8, J1, J2, J4, J5, J7, J11, J12, J14, J16, Ko, LD1, LD2, P28, W4, V3, V6, V20, V15, V5, V28, V22, V2, V12, V13, V10, V14, V17
 \begin{description}

        \end{description}
\end{marma}


\begin{marma}[hp02_57a]
\sthana{2.57a}
\item[rasanātālumūlena] A1, B1, B2, B3, Bo1, C2, C4, C9, J4, J10, J12, J13, J15, J16, J17, Ko, LD1, N21, N22(tālū), N25, N26, V1, V20, V15, V28
\item[rasanātālumūle yaḥ] Ba,
\item[rasanātālumūlenaya] V14,
\item[rasanātālumūlenayaḥ] V2, V18, V12, V3, V16, V17
\item[rasanāttālumūlena] N20
\item[rasanātālumūlaṃ ca] J5, 
\item[rasanātāluyogena] Bo2, C6, C8, J3, J6, J7, N18, V21, N23, V13
\item[rasanātālayogena] J1,
\item[rasanāṃ tālamūlena] J11
\item[rasanānāṃ tālamūlena] N19
\item[jihvayā vāyum ākṛṣya] Ba5,
\item[satyāṃ mūlena randhreṇa] C3,
\item[]
\item[(illegible/unavailable)] Bo3, C5, J2, J14, Tue, LD2, N24, V6, V5, V22, V10
 \begin{description}

        \end{description}
\end{marma}

\begin{marma}[hp02_58b]
\sthana{2.58b}
\item[pūrvavat kumbhakād anu] C4, C6, C8, J1, J4, J6pc, J7, J11, Ko, N21, V14, V21,  N24, N25, V6, V19, V15, V28, V2
\item[pūrvavat kūmbhakād anu] Bo2,
\item[pūrvavat kumbhakād anmuṃ] Bo1,
\item[pūrvavat kumbhayed anu] A1, V20
\item[pūrvavat kumbhakād anū] N19
\item[pūrve ce kumbhanād agu] J5,
\item[pūrvavat kumbhakādane] N23, V7
\item[pūrvavat kumbhakeṣv anu] B2
\item[pūrvavat kumbhasādhanaṃ] B1, B3, Ba, Ba5, Bo3, C1, C2, C3, C9, J2, J3, J6, J10, J12, J13, J15, J16, J17, LD1, LD2, N18(“sā” unclear), N22, V3, Tue, V22, V18, V12, V13, V3, V10, V16
\item[kākacaṃcusamannayāt] V17
\item[kākacaṃcusamunnayāt] N26
\item[kākacaṃcusamunnarayāt] N20(mark over “kā”, perhaps “ra” is cancelled), 
\item[(illegible/unavailable)] C5, J14, V5
 \begin{description}

        \end{description}
\end{marma}


\begin{marma}[hp02_62b]
\sthana{2.62b}
\item[kapāle ca samantataḥ] V1 
\item[kapāle niḥsvanaṃ tathā] Ba, 
\item[kapole niḥsvanaṃ tadā] C1, 
\item[kapole nisvanaṃ tadā] J2, 
\item[kapole svāsanaṃ tataḥ] N26
\item[kapole svasanaṃ tataḥ] V18
\item[kapāla śvasamaṃ tataḥ] J4, 
\item[kapāla śasanaṃ tataḥ] J15, 
\item[kapāle śvasanaṃ tataḥ] J11, V17 
\item[kapāle svaśanaṃ tataḥ] LD1, 
\item[kapāla śparśanaṃ tataḥ] J16, 
\item[kapole svaśanaṃ tataḥ] LD2, V16
\item[kapole śvavaṇe tathā] B2,
\item[kapole śvavaṇe tataḥ] V13, V3
\item[kapola saśvanaṃ tataḥ] J5,
\item[kapole sca sanantataḥ] V3
\item[kayolesvaśanaṃtataḥ] N24
\item[kapāla sasvanaṃ tataḥ] N21, V12
\item[kapālasya samaṃ tataḥ] C3
\item[kapālasyarśanaṃ tataḥ] J10, 
\item[kapāle sasvanaṃ tataḥ] Bo1, V14
\item[kapāle mukhano marut] Ko,
\item[kapālādhi sasvanaṃ tataḥ] V6
\item[kapālasparśanaṃ tataḥ] C9, J17,
\item[kapālasya śanaṃ tataḥ] J12,
\item[kapālādhi niḥsvanaṃ] C6,
\item[kapālāvadhi sasvanaṃ] A1, Ba5, C8, J7, Tue, V7, V20, V28, V22, V10
\item[kapālāvadhi sasanaṃ] V21
\item[kapālāvadhi pūrayet] B1, B3, Bo2, C2, C4, J1, J3, J6, J13, N25, V15, V2
\item[kepālavadhi sasvanaṃ] N23, 
 
\item[(illegible/unavailable)] Bo3, C5, J14, V5
 \begin{description}

        \end{description}
\end{marma}

\begin{marma}[hp02_65cd]
\sthana{2.65cd}
\item[dhārayen nāsikāmadhye aṅgulībhyāṃ vinā dṛḍham] V1, V21
\item[dhārayen nāsikāmadhye aṅgulībhyāṃ tathā dṛḍham] LD1,
\item[dhārayen nāsikāmadhyaṃ aṅgulībhyāṃ tathā dṛḍham] A1, C9, 
\item[dhāraye nāsikāmadhyaṃ maṅgulānāṃ tathā dṛḍham] J12,
\item[dhārayen nāsikāmadhaṃ aṅgulībhyāṃ tathā dṛḍham] V20
\item[dhārayen nāsikāmadhya aṅgulībhyāṃ tathā dṛḍham] J10,
\item[dhārayen nāsikāmadhye aṅgulībhyāṃ tathā dṛḍham] B2, C3, J17, V18, V12, V16, V17
\item[dhārayen nāsikāmadhyai aṅgulībhyāṃ tathā dṛḍham] J15,
\item[dhārayen nāsīkāmadhye maṃgulībhā tathā dṛḍham] J16,
\item[dhārayen nāsīkāmadhye aṅgulībhyāṃ dṛḍham tathā] LD2,
\item[dhārayen nāsikāmadhye aṅguṣṭhābhyāṃ tathā dṛḍham] N26,
\item[dhārayen nāsikāmadhyṃ tarjanībhyāṃ vinā dṛḍham] Ko,
\item[dhārayen nāsikāmadhyaṃ tarjanībhyāṃ vinā dṛḍham] J11,
\item[dhārayen nāsikāmadhye tarjanībhyāṃ vinā dṛḍham] C6, J5, V3, N24, V6, V7, V28, V22, V2, V3, V14
\item[dhāraye nāsikāmadhye tarjanībhyāṃ vinā dṛḍham] Ba,
\item[dhārayen nāsikāṃ madhya tarjjanībhyāṃ vinā dṛḍham] Ba5, Bo2, J3,
\item[dhāraye nāsikāṃ madhyā tarjjanībhyāṃ vinā dṛḍham] Ba5, V15
\item[dhārayen nāsikāṃ madhyā tarjjanībhyāṃ vinā dṛḍham] J7,
\item[dhāraye nāsikāṃ madhye tarjjanībhyāṃ vinā dṛḍham] Bo1,
\item[dhārayen nāsikāmadhye tarjanībhyāṃ tathā dṛḍham] B1, B3, J13, V13
\item[dhārayan nāsikāmadhye tarjanībhyāṃ tathā dṛḍham] C2, 
\item[dhārayen nāsikāmadhya tarjjanībhyāṃ vinā dṛḍhām] J1
\item[dhārayen nāsikā madhye tarjanībhyāṃ vinā madam] J4,
\item[dhārayen nāsikāṃ madhya tarjjanībhyāṃ yathā dṛḍham] C8
\item[dhārayen nāsikāṃ madhya tarjjanībhyāṃ vinā dṛḍham] J6pc,
\item[kārayen nāsikāṃ madhya tarjjanībhyāṃ vinā dṛḍham] J6,
\item[dhārayen nāsikāṃ madhyāt tarjanībhyāṃ vinā dṛḍham] Tue
\item[dhārayen nāsikāmadhyāt tarjanībhyāṃ vinā dṛḍham] C4, N25, V10
\item[dhārayaṃ nāśikāmadhye tarjanabhyāṃ vinā dṛḍham] N23
\item[dhāraae vāsikāmadhye tarjanībhyāṃ vinā dṛḍham] J2,
\item[(illegible/unavailable)] Bo3, C1, C5, J14, V5
 \begin{description}

        \end{description}
\end{marma}


\begin{marma}[hp02_67ab]
\sthana{2.67ab}
\item[kaṇdalībodhakaṃ viprabhavaghnaṃ sukhadaṃ śubhaṃ] V1
\item[kuṇdalībodhanaṃ sarva doṣaghnaṃ sukhadaṃ śivaṃ] Bo1
\item[kuṇdalībodhanaṃ sarva doṣaghnaṃ sukhadaṃ śuvaṃ] V15
\item[kaṇdalībodhakaṃ ksipraṃpavanaṃ sukhadaṃ hitaṃ] V7
\item[kuṇḍalo bodhaka kṣipraṃ pavanaṃ sukhadaṃ hitaṃ] J2
\item[kuṇḍalībodhakaṃ kṣipraṃ pavanaṃ sukhadaṃ hitaṃ] Ba5, V22
\item[kuṇḍalabodhakaṃ pāpāghnaṃ suṣadaṃ śubhaṃ] Ko,
\item[kaṇḍalībodhanaṃ kuryāt pāpaghnaṃ sukhadaṃ śubhaṃ] V6, V2
\item[kaṇḍalībodhanaṃ cakaṃ pāpaghnaṃ sukhada śubhaṃ] V14
\item[kuṇḍalībodhanaṃ kuryāt pāpaghnaṃ sukhadaṃ śubhaṃ] B2, Bo2, C4, J3, J6, J7, J11,  
\item[kaṇḍalībodhanaṃ kuryāt pāpaghnaṃ subhaṃ tad śukhaṃ] V28
\item[kuṇdalībādhakaṃ ksipraṃ pāpghnaṃ sukhadaṃ hitaṃ] Ba, 
\item[kuṇḍalībodhakaḥ karthūḥ roghnaṃḥ sukhadaḥ śubhaḥ] A1, V20
\item[kuṇḍalībodhanaṃ karttur bhavaghnam sukhadaṃ śubhaṃ] C6,
\item[kuṇḍalībodhanaṃ karttur bhavaghnam sukhaṃ śubhaṃ] V21
\item[kuṇḍalībodhakaṃ miśrabhavaghnam sukhadaṃ śubhaṃ] J4,
\item[kuṇḍalībodhanaṃ kartuṃ bhavaghnam sukhadaṃ śubhadaṃ] C8,
\item[kuṇḍalībodhakaḥ kumbho roghnaḥ sukhadaḥ śubhaḥ] C3, J10, J17, 
\item[kuṇḍalībodhakaṃ kumbhoḥ roghnaḥ sukhadaḥ śubhaḥ] J15, J16,
\item[kuṇḍalībodhakaḥ kumbho roghnaḥ suṣadaḥ śubhaḥ] C9, 
\item[kuṇḍalībodhakaḥ kumbho rogaghnaḥ sukhadaḥ śubhaḥ] J12, LD1, LD2, V16 
\item[kuṇḍalībodhakaḥ kumbho rogaghnaḥ sukhadaḥ śubhaḥ] B1, B3, C2, J13, V18, V12, V13, V3, V17 
\item[kuṇḍalīno bodhacakraṃ? bhāvadaṃ supada śubhāṃ] J5
\item[(illegible/unavailable)] Bo3, C1, C5, J14, V5, V10
 \begin{description}

        \end{description}
\end{marma}


\begin{marma}[hp02_68a]
\sthana{2.68a}
\item[guṇatrayasamudbhūtaṃ] Bo1,
\item[samagātrasamudbhūtaṃ] J4,
\item[samyagātrasamudbhūtaṃ] J7,
\item[samyagātrasamudbhutaṃ] J15,
\item[samyaggātrasamudbhūtaṃ] B2, Bo2, C4, C8, J2, J11, V1, V7, V2, V3, V17
\item[samya gātre samudbhūtaṃ] J5,
\item[samyaggātre samudbhūtaṃ] J6, V28
\item[samyaggātre samudbhūta] J3,
\item[samyag gātraṃ samudbhūtaṃ] Ko,
\item[samyaggātasamudbhūtam] V20, V15
\item[samyagataḥ samudbhūtam] V14
\item[samyak gātrasamudbhūta] Ba, V22
\item[samyag antaḥ samudbhū] A1
\item[samyag antaḥ samudbhūta] C6
\item[samyak bhastrā samudbhūtā] B1, B3, C2, J12, J13, J16,  
\item[samyak bhastrā samudbhūtaḥ] V21  
\item[samyak bhastrā samabhyāsaḥ] V13
\item[samyag bhastrā samudbhūtāṃ] LD1, LD2, 
\item[samyak bhastrā samudbhūto] C3, J10, J17, 
\item[samyag bhastri samudbhūtaṃ] C9
\item[sasyagātrasamudbhūtaṃ] J1
\item[samagātrasamudbhūtaṃ] V6, V16
\item[samagātrasamudbhūtaghnaṃ] V18
\item[(illegible/unavailable)] Ba5, Bo3, C1, C5, J14, V5, V12, V10
 \begin{description}

        \end{description}
\end{marma}


\begin{marma}[hp02_69a]
\sthana{2.69a}
\item[vegair ghoṣaṃ] V1
\item[vegākṛṣṭaṃ] Ba, J2, V15
\item[vego tad ghoṣa] J1
\item[vego ghoṣam] C3, C9, J16,  
\item[vego ghoṣaḥ] J15,
\item[vegāt ghoṣaṃ] A1, B2, B3, Ba5, C9, LD1, V6, V7, V22, V18, V12, V13, V14, V16, V17
\item[vegā ghoṣaṃ] V3
\item[vegot ghoṣaṃ] J12, V2
\item[vegot ghoṣe] V2,
\item[vegodveṣaḥ] J4,
\item[vaiṣodvaiṣaṃ] J5,
\item[veget vege] V28
\item[vegād ghoṣaṃ] J10, J17, V20
\item[vegod ghoṣaṃ] Bo2, C6, J10pc, J11,
\item[vedod ghoṣaṃ] V21
\item[vegodyoṣaṃ] J7,
\item[vemod ghoṣaṃ] J3, J6, 
\item[vegodvegaṃ] C4, J13, 
\item[vegād aghoṣaṃ] B1,
\item[vegād doṣaṃ] Bo1,
\item[veṇu doṣaṃ] C2,
\item[(illegible/unavailable)] Bo3, C1, C5, J14, Ko, V5, V10
 \begin{description}

        \end{description}
\end{marma}


\begin{marma}[hp02_73a]
\sthana{2.73a}
\item[ārecyāpūrya yat kuryāt] 
\item[ārecyāpūrya vat kuryāt] J13,
\item[recya vā pūrya sanakaiḥ] J11,
\item[recake pūrakaṃ kuryāt] J12, J16, 
\item[recakaḥ pūrakaṣ kāryyaḥ] J17,
\item[recakaḥ pūrakaḥ kāryaḥ] J15, 
\item[recayitvā pūrakaḥ kāryaḥ] J14, 
\item[recake pūrakaṃ uktwā] V14
\item[recaka pūraka kumbhai] V3, V16
\item[recaka pūraka kumbhakaiḥ] V17
\item[recaka pūraka varjite] V10
\item[(illegible/unavailable)]
 \begin{description}

        \end{description}
\end{marma}


\begin{marma}[hp02_73b]
\sthana{2.73b}
\item[sa vai sahitakumbhakaḥ] J12, J13, J15, 
\item[sarve sahitakumbhakaḥ] J17,
\item[sa vai sahitakumbhak] J16, 
\item[sa taiḥ sahitakumbhakaḥ] J14,
\item[sa vai kevalakumbhakaḥ] V10
\item[kāryaḥ sahitakumbhakaḥ] J11,
\item[kāryaḥ śanai sahitakumbhakā] V3, V14
\item[kāryaḥ śarveiḥ sahitakumbhakeiḥ] V16
\item[kāryaḥ śarveiḥ sahitakumbhaka] V17
\item[(illegible/unavailable)]
 \begin{description}

        \end{description}
\end{marma}


\begin{marma}[hp02_78a]
\sthana{kumbhakaḥ prāṇarecānte} C3, J2, J3, J13, V1, V6, V28, V18, V16
\sthana{kumbhakāt prāṇarecānte} V17
\item[kumbhaka prāṇarecānte] A1, B1, B2, Bo2, C1, J1, J5, LD1, V2, V12
\item[kumbhake siddha reca puraka] V21
\item[kumbhaka prāṇa cātte?] J10, J17, 
\item[kumbhaka prāṇa recāttaḥ] J16,
\item[kumbhakaṃ prāṇarecānte] B3, C4, C9, J6, 
\item[kumbhakaṃ prāṇarecātteḥ] J15,
\item[kumbhakaṃ prāṇareṃcānte] J12,
\item[kumbhataḥ prāṇarecānte] J7,
\item[kumbhata prāṇarecānte] V3
\item[kumbhakaṃ ghrāṇarecānte] LD2, 
\item[kumbhitaḥ prāṇarecānte] Bo1, V15
\item[kumbhītaḥ prāṇarecānte] Ko, J11, J14, 
\item[kumbhīte prāṇarecānte] V14
\item[kumbhītaprāṇareṣānte] J4,
\item[tad etat prāṇarrecānte] C6, C8, 
\item[kumbhakaḥ prāṇarodhānte] Ba, Ba5, V7, V20, V22, V10
\item[(illegible/unavailable)] Bo3, C1, C5, V5, V13
 \begin{description}

        \end{description}
\end{marma}


\end{ekdosis}
\end{document}

%%% Local Variables:
%%% mode: latex
%%% TeX-master: t
%%% End:





	

	

[a]Perhaps "yathā" can be removed from these readings
[b]Perhaps "yathā" can be removed from these readings
[c]I don't know what part of a verse this number means
[d]I think it is 56*1 @nilsliersch@web.de isn't it? 2nd thing..do we need full verse or just a line or a word here?