Notes
    • P16 has chapter 1 till 1.45
    • P22 has only 3 folios of chapter 1
\documentclass[10pt]{memoir}
\setstocksize{220mm}{155mm} 	        
\settrimmedsize{220mm}{155mm}{*}	
\settypeblocksize{170mm}{116mm}{*}	
\setlrmargins{18mm}{*}{*}
\setulmargins{*}{*}{1.2}
% \setlength{\headheight}{5pt}
\checkandfixthelayout[lines]
\linespread{1.16}

\setlength{\footmarkwidth}{1.3em}
\setlength{\footmarksep}{0em}
\setlength{\footparindent}{1.3em}
\footmarkstyle{\textsuperscript{#1} }
\usepackage{fnpos}
\makeFNbottom

\usepackage[teiexport=tidy,poetry=verse]{ekdosis}
\usepackage{sanskrit-poetry}

\usepackage[english]{babel}
\usepackage{babel-iast,xparse,xcolor}
\babelfont[iast]{rm}[Renderer=Harfbuzz, Scale=1.5]{AdishilaSan}
\babelfont[english]{rm}[Scale=0.9]{Adobe Text Pro}
\babeltags{dev = iast}
\babeltags{eng = english}

\SetHooks{
	lemmastyle=\bfseries,
	refnumstyle=\selectlanguage{english}\color{blue}\bfseries, 
	}
\newif\ifinapparatus
\DeclareApparatus{default}[
	lang=english,
	sep = {] },
	delim=\hskip 0.75em,
	rule=none,
	]
\DeclareApparatus{notes}[
	lang=english,
	sep = {},
	delim=\hskip 0.75em,
	rule=\rule{0.7in}{0.4pt},
	]

\DeclareShorthand{conj}{\texteng{\emph{conj.}}}{ego}
\DeclareShorthand{emend}{\texteng{\emph{em.}}}{ego}

\setlength{\vrightskip}{-10pt}
\setlength{\vgap}{3mm}
\verselinenumfont{\footnotesize\selectlanguage{english}\normalfont}




%%%%%%%%%%%%%%%%%%%% THE  MSS         %%%%%%%%%%%%%%%%%%%%%%%%%%%

%%% Versions
\DeclareWitness{Vu}{\selectlanguage{english}Vulg}{Vulgate, i.e. Brahmānanda's version}[]           
\DeclareWitness{X}{\selectlanguage{english}X}{TenChapter Version, Jodhpur 02228 and 02225 (ed. Lonavla)}[]
\DeclareWitness{Six}{\selectlanguage{english}Ṣ}{SixChapterVersion, ``6ChapterHPms'', fragment of enlarged text, Jodhpur}[]
% Mss. in Geographical Groups
%%%% Varanasi mss (Sampūrṇānanda mss). V1 is Important
\DeclareWitness{V1}{\selectlanguage{english}V\textsubscript{1}}{Sampurnananda Library Sarasvati Bhavan 30109}[]
        \DeclareHand{V1ac}{V1}{\selectlanguage{english}V\rlap{\textsubscript{1}}\textsuperscript{ac}}[] % added by MD
        \DeclareHand{V1pc}{V1}{\selectlanguage{english}V\rlap{\textsubscript{1}}\textsuperscript{pc}}[] % added by MD
\DeclareWitness{V2}{\selectlanguage{english}V\textsubscript{2}}{Sampurnananda Library Sarasvati Bhavan 29869}[]
\DeclareWitness{V3}{\selectlanguage{english}V\textsubscript{3}}{Sampurnananda Library Sarasvati Bhavan 29899}[]
\DeclareWitness{V4}{\selectlanguage{english}V\textsubscript{4}}{Sampurnananda Library Sarasvati Bhavan 29937}[]
\DeclareWitness{V5}{\selectlanguage{english}V\textsubscript{5}}{Sampurnananda Library Sarasvati Bhavan 29938}[]
\DeclareWitness{V6}{\selectlanguage{english}V\textsubscript{6}}{Sampurnananda Library Sarasvati Bhavan 29991}[]
\DeclareWitness{V8}{\selectlanguage{english}V\textsubscript{8}}{Sampurnananda Library Sarasvati Bhavan 30014}[]
\DeclareWitness{V11}{\selectlanguage{english}V\textsubscript{11}}{Sampurnananda Library Sarasvati Bhavan 30029}[]
\DeclareWitness{V12}{\selectlanguage{english}V\textsubscript{12}}{Sampurnananda Library Sarasvati Bhavan 30030}[]
\DeclareWitness{V13}{\selectlanguage{english}V\textsubscript{13}}{Sampurnananda Library Sarasvati Bhavan 30031}[]
\DeclareWitness{V14}{\selectlanguage{english}V\textsubscript{14}}{Sampurnananda Library Sarasvati Bhavan 30050}[]
\DeclareWitness{V15}{\selectlanguage{english}V\textsubscript{15}}{Sampurnananda Library Sarasvati Bhavan 30051}[]
\DeclareWitness{V15pc}{\selectlanguage{english}V\rlap{\textsubscript{15}}\textsuperscript{pc}\space}{}[]
\DeclareWitness{V16}{\selectlanguage{english}V\textsubscript{16}}{Sampurnananda Library Sarasvati Bhavan 30052}[]
\DeclareWitness{V17}{\selectlanguage{english}V\textsubscript{17}}{Sampurnananda Library Sarasvati Bhavan 30053}[] % added by MD
\DeclareWitness{V16pc}{\selectlanguage{english}V\rlap{\textsubscript{16}}\textsuperscript{pc}\space}{}[]
\DeclareWitness{V18}{\selectlanguage{english}V\textsubscript{18}}{Sampurnananda Library Sarasvati Bhavan 30064}[]
\DeclareWitness{V19}{\selectlanguage{english}V\textsubscript{19}}{Sampurnananda Library Sarasvati Bhavan 30069}[]
\DeclareWitness{V21}{\selectlanguage{english}V\textsubscript{21}}{Sampurnananda Library Sarasvati Bhavan 30104}[]
\DeclareWitness{V22}{\selectlanguage{english}V\textsubscript{22}}{Sampurnananda Library Sarasvati Bhavan 30110}[]
\DeclareWitness{V25}{\selectlanguage{english}V\textsubscript{25}}{Sampurnananda Library Sarasvati Bhavan 30122}[]
\DeclareWitness{V26}{\selectlanguage{english}V\textsubscript{26}}{Sampurnananda Library Sarasvati Bhavan 30123}[]
\DeclareWitness{V28}{\selectlanguage{english}V\textsubscript{28}}{Sampurnananda Library Sarasvati Bhavan 30136}[]
\DeclareWitness{W4}{\selectlanguage{english}W\textsubscript{4}}{Wai 399-6171}[]

%%%%%%%%%%%%%%%%%%%%%%%%%%%%%%%%%
%%% Jammu & Kaschmir
\DeclareWitness{K1}{\selectlanguage{english}K\textsubscript{1}}{Raghunātha Temple Library 4383}[settlement=Jammu]
        \DeclareWitness{K1ac}{\selectlanguage{english}K\rlap{\textsubscript{1}}\textsuperscript{ac}\space}{}[]
        \DeclareWitness{K1pc}{\selectlanguage{english}K\rlap{\textsubscript{1}}\textsuperscript{pc}\space}{}[]
\DeclareWitness{L1}{\selectlanguage{english}L\textsubscript{1}}{SOAS RE 43454}[settlement=Jammu]
% More details? Catalogue number? L1 And C1 very close (and come from same region)
%%%%%%%%%%%%%%%%%%%%%%%%%%%%%%%%
% Jodhpur
% J10 is important
\DeclareWitness{J10}{\selectlanguage{english}J\textsubscript{10}}{MSPP Jodhpur 2230}[]
        \DeclareHand{J10ac}{J10}{\selectlanguage{english}J\rlap{\textsubscript{10}}\textsuperscript{ac}}[] % modified by MD
        \DeclareHand{J10pc}{J10}{\selectlanguage{english}J\rlap{\textsubscript{10}}\textsuperscript{pc}}[] % modified by MD
\DeclareWitness{J1}{\selectlanguage{english}J\textsubscript{1}}{Jodhpur 02231}[]
\DeclareWitness{J2}{\selectlanguage{english}J\textsubscript{2}}{Jodhpur 02232}[]   
\DeclareWitness{J3}{\selectlanguage{english}J\textsubscript{3}}{Jodhpur 02233}[]
\DeclareWitness{J4}{\selectlanguage{english}J\textsubscript{4}}{Jodhpur 02234}[]
        \DeclareWitness{J4ac}{\selectlanguage{english}J\rlap{\textsubscript{4}}\textsuperscript{ac}\space}{MSPP Jodhpur 02234}[]
        \DeclareWitness{J4pc}{\selectlanguage{english}J\rlap{\textsubscript{4}}\textsuperscript{pc}\space}{MSPP Jodhpur 02234}[]
\DeclareWitness{J5}{\selectlanguage{english}J\textsubscript{5}}{Jodhpur 02235}[]  % 4 chapters, 34 jpgs,   long colophon, missing lines in the beginning.
\DeclareWitness{J6ac}{\selectlanguage{english}J\rlap{\textsubscript{6}}\textsubscript{ac}}{Jodhpur 02237}[]  % 4 chapters, 49 jpgs,   1st folio: idaṃ gulābarāyasya
% tulasīrāmaśarmmaṇaḥ putrasya pustakaṃ ...        End: iti śrīsahajānandasantānacintāmaṇisvātmārāmaviracitāyāṃ ..
% saṃvat 1802   (more consistent text)
\DeclareWitness{J6pc}{\selectlanguage{english}J\rlap{\textsubscript{6}}\textsubscript{pc}}{Jodhpur 02237}[] 
\DeclareWitness{J7}{\selectlanguage{english}J\textsubscript{7}}{Jodhpur 02241}[]  % 4 chapters, 41 jpgs
\DeclareWitness{J8}{\selectlanguage{english}J\textsubscript{8}}{Jodhpur 23709}[]  % 4 chapters,  87 jpgs.   saṃvat 1724
\DeclareHand{J8ac}{J8}{\selectlanguage{english}J\rlap{\textsubscript{8}}\textsuperscript{ac}}[]  % changed by MD
\DeclareHand{J8pc}{J8}{\selectlanguage{english}J\rlap{\textsubscript{8}}\textsuperscript{pc}}[]  % changed by MD
\DeclareWitness{J9}{\selectlanguage{english}J\textsubscript{9}}{Jodhpur 02224}[]  %  fragment, 20 jpgs.
\DeclareWitness{J11}{\selectlanguage{english}J\textsubscript{11}}{Jodhpur 23532}[]
\DeclareWitness{J12}{\selectlanguage{english}J\textsubscript{12}}{Jodhpur 18552}[] 
\DeclareWitness{J13}{\selectlanguage{english}J\textsubscript{13}}{Jodhpur 02229}[]  %  5 chapters, 93 jpgs.
\DeclareWitness{J14}{\selectlanguage{english}J\textsubscript{14}}{Jodhpur 02239}[]  %  4 chapters
\DeclareWitness{J15}{\selectlanguage{english}J\textsubscript{15}}{Jodhpur 9732A}[]
\DeclareWitness{J17}{\selectlanguage{english}J\textsubscript{17}}{Jodhpur 3013}[]
% Haṭhapradīpikā with (non-Sanskrit) Bhāṣya RORI Jodhpur ACC.NO.18552
%  Haṭhapradīpikā with (non-Sanskrit) commentary, RORI Alwar 952, 4 chapters,  colophon of the comm:
% iti śrīlāhorīmiśravrajabhūṣanaviracitāyāṃ bhāvārthadīpikāyāṃ caturthodhyāya ..    
%  Haṭhapradīpikā (5 chapter) MSPP Jodhpur ACC.NO.02229/

%%%%%%%%%%        Bodleian, Oxford
\DeclareWitness{B1}{\selectlanguage{english}B\textsubscript{1}}{Bodleian Library No. d.457(8)}[settlement=Oxford]
\DeclareWitness{B2}{\selectlanguage{english}B\textsubscript{2}}{Bodleian Library No. d.458(1)}[settlement=Oxford]
\DeclareWitness{B3}{\selectlanguage{english}B\textsubscript{3}}{Bodleian Library No. d.458(9)}[settlement=Oxford]

%%%%%%%%%%%   Chandigarh
\DeclareWitness{C1}{\selectlanguage{english}C\textsubscript{1}}{Lalchand M-2080}[]%L1 And C1 very close (and come from same region)
\DeclareWitness{C2}{\selectlanguage{english}C\textsubscript{2}}{Lalchand M-6065}[]
\DeclareWitness{C3}{\selectlanguage{english}C\textsubscript{3}}{Lalchand M-1293}[]
\DeclareWitness{C4}{\selectlanguage{english}C\textsubscript{4}}{Lalchand M-2081}[]
\DeclareWitness{C4ac}{\selectlanguage{english}C\rlap{\textsubscript{4}}\textsuperscript{ac}\space}{}[]
\DeclareWitness{C4pc}{\selectlanguage{english}C\rlap{\textsubscript{4}}\textsuperscript{pc}\space}{}[]
\DeclareWitness{C5}{\selectlanguage{english}C\textsubscript{5}}{Lalchand M-2082}[]%doesn't have chapter 1
\DeclareWitness{C6}{\selectlanguage{english}C\textsubscript{6}}{Lalchand M-2089}[]
\DeclareWitness{C7}{\selectlanguage{english}C\textsubscript{7}}{Lalchand M-6494}[]
\DeclareWitness{C8}{\selectlanguage{english}C\textsubscript{8}}{Lalchand M-2091}[]
\DeclareWitness{C8pc}{\selectlanguage{english}C\rlap{\textsubscript{8}}\textsuperscript{pc}\space}{}[]
\DeclareWitness{C9}{\selectlanguage{english}C\textsubscript{9}}{Lalchand M-4530}[]

% %%%%%%%%%%        Nepalese
\DeclareWitness{N1}{\selectlanguage{english}N\textsubscript{1}}{NGMPP A1400-2}[]
\DeclareWitness{N2}{\selectlanguage{english}N\textsubscript{2}}{NGMPP B 39-19}[]
\DeclareWitness{N3}{\selectlanguage{english}N\textsubscript{3}}{NGMPP B 62-20}[]
\DeclareWitness{N5}{\selectlanguage{english}N\textsubscript{5}}{NGMPP A60-15 + A61-1}[]
\DeclareWitness{N6}{\selectlanguage{english}N\textsubscript{6}}{NGMPP A61-6}[]
\DeclareWitness{N9}{\selectlanguage{english}N\textsubscript{9}}{NGMPP A62-33}[]
\DeclareWitness{N10}{\selectlanguage{english}N\textsubscript{10}}{NGMPP A62-37}[]
\DeclareWitness{N11}{\selectlanguage{english}N\textsubscript{11}}{NGMPP A63-15}[]
\DeclareWitness{N12}{\selectlanguage{english}N\textsubscript{12}}{NGMPP A939-19}[]
\DeclareWitness{N13}{\selectlanguage{english}N\textsubscript{13}}{NGMPP A1378-18}[]
\DeclareWitness{N16}{\selectlanguage{english}N\textsubscript{16}}{NGMPP B39-20}[]
\DeclareWitness{N17}{\selectlanguage{english}N\textsubscript{17}}{NGMPP B 111-10}[]
\DeclareWitness{N18}{\selectlanguage{english}N\textsubscript{18}}{NGMPP E 929-3}[]
\DeclareWitness{N19}{\selectlanguage{english}N\textsubscript{19}}{NGMPP E-1528-1 / E-1527-7(4)}[]
\DeclareWitness{N20}{\selectlanguage{english}N\textsubscript{20}}{NGMPP E 2037-13 }[]
\DeclareWitness{N21}{\selectlanguage{english}N\textsubscript{21}}{NGMPP E 2097-31}[]
\DeclareWitness{N22}{\selectlanguage{english}N\textsubscript{22}}{NGMPP G 4-4}[]
\DeclareWitness{N23}{\selectlanguage{english}N\textsubscript{23}}{NGMPP G 25-2}[]
\DeclareWitness{N24}{\selectlanguage{english}N\textsubscript{24}}{NGMPP G 190-16}[]
\DeclareWitness{N24ac}{\selectlanguage{english}N\rlap{\textsubscript{24}}\textsuperscript{ac}\space}{}[]
\DeclareWitness{N24pc}{\selectlanguage{english}N\rlap{\textsubscript{24}}\textsuperscript{pc}\space}{}[]

\DeclareWitness{P28}{\selectlanguage{english}P\textsubscript{28}}{BORI 399-1895-1902}[]

%%%%%   Mysore
\DeclareWitness{M1}{\selectlanguage{english}M\textsubscript{1}}{P-5682/4}[]
%%%%%   Tübingen
\DeclareWitness{Tü}{\selectlanguage{english}Tü}{Ma I 339}[]
%%%%%%%%%%
\DeclareWitness{YC}{\selectlanguage{english}YC}{Yogacintāmaṇi}[]
\DeclareWitness{ceteri}{\selectlanguage{english}cett.}{ceteri}[]

%%%%%%%%%% Mss with Commentary
\DeclareWitness{A1}{\selectlanguage{english}A\textsubscript{1}}{Alwar 952}[]


%%%%%%%%%%%%%%%%%%%%%%%%%%%%%%%%%%%%%%%%%%%
%List of all Sigla:
%A1,B1,B2,B3,C1,C2,C3,C4,C6,C7,C8,C9,J1,J2,J3,J4,J10,J13,J14,J15,J17,L1,M1,N3,N5,N6,N9,N10,N11,N12,N13,N16,N17,N19,N20,N21,N22,N23,N24,Tü,V1,V2,V3,V4,V5,V6,V8,V11,V19,V22,V26,Vu
%%%%%%%%%%%%%%%%%%%%%%%%%%%%%%%%%%%%%%%%%%%

\DeclareShorthand{x}{\selectlanguage{english}δ}{J10,J17,N17,P28,W4}


%%% Local Variables:
%%% mode: latex
%%% TeX-master: t
%%% End:


%%%%%                   Abbreviation for the printed apparatus,        xml interface needed
%%%%%                   (synonyms in same line)

% Macro for Editing Abbrevs.
%\def\om{\textrm{\footnotesize \textit{omitted in}\ }} %prints om. for omitted in apparatus
%\def\korr{\textrm{\footnotesize \textit{em.}\ }} %prints em. for emended in apparatus
%\def\conj{\textrm{\footnotesize \textit{conj.}\ }} %prints conj. for conjectured in apparatus


\def\eyeskip{\textrm{{ab.\,oc. }}}   
\def\aberratio{\textrm{{ab.\,oc. }}}
\def\ad{\textrm{{ad}}}   
\def\add{\textrm{{add.\ }}}
\def\ann{\textrm{{ann.\ }}}
\def\ante{\textrm{{ante }}}
\def\post{\textrm{{post }}}
%\def\ceteri{cett.\,}             % for simplifying the apparatus in print                  
\def\codd{\textrm{{codd.\ }}}   %  the same
\def\conj{\textrm{{coni.\ }}}  
\def\coni{\textrm{{coni.\ }}}
\def\contin{\textrm{{contin.\ }}}
\def\corr{\textrm{{corr.\ }}}
\def\del{\textrm{{del.\ }}}
\def\dub{\textrm{{ dub.\ }}}
\def\emend{\textrm{{emend.\ }}}
\def\expl{\textrm{{explic.\ }}}   
\def\explicat{\textrm{{explic.\ }}}
\def\fol{\textrm{{fol.\ }}}         
\def\foll{\textrm{{foll.\ }}}
\def\gloss{\textrm{{glossa ad }}}
\def\ins{\textrm{{ins.\ }}}          \def\inseruit{\textrm{{ins.\ }}}
\def\im{{\kern-.7pt\lower-1ex\hbox{\textrm{\tiny{\emph{i.m.}}}\kern0pt}}}
\def\inmargine{{\kern-.7pt\lower-.7ex\hbox{\textrm{\tiny{\emph{i.m.}}}\kern0pt}}}
\def\intextu{{\kern-.7pt\lower-.95ex\hbox{\textrm{\tiny{\emph{i.t.}}}\kern0pt}}}%\textrm{\scriptsize{i.t.\ }}}               
\def\indist{\textrm{{indis.\ }}}          \def\indis{\textrm{{indis.\ }}}
\def\iteravit{\textrm{{iter.\ }}}          \def\iter{\textrm{{iter.\ }}}  
\def\lectio{\textrm{{lect.\ }}}             \def\lec{\textrm{{lect.\ }}}
\def\leginequit{\textrm{{l.n. }}}         \def\legn{\textrm{{l.n. }}}         \def\illeg{\textrm{{l.n. }}}
\def\om{\textrm{{om. }}}
\def\primman{\textrm{{pr.m.}}}
\def\prob{\textrm{{prob.}}}
\def\rep{\textrm{{repetitio }}}
% \def\secundamanu{\textrm{\scriptsize{s.m.}}}
% \def\secm{{\kern-.6pt\lower-.91ex\hbox{\textrm{\tiny{\emph{s.m.}}}\kern0pt}}}%   \textrm{\scriptsize{s.m.}}}
\def\sequentia{\textrm{{seq.\,inv.\ }}}         \def\seqinv{\textrm{{seq.\,inv.\ }}} \def\order{\textrm{{seq.\,inv.\ }}}
\def\supralineam{{\kern-.7pt\lower-.91ex\hbox{\textrm{\tiny{\emph{s.l.}}}\kern0pt}}} %\textrm{\scriptsize{s.l.}}}
\def\interlineam{{\kern-.7pt\lower-.91ex\hbox{\textrm{\tiny{\emph{s.l.}}}\kern0pt}}}   %\textrm{\scriptsize{s.l.}}}
\def\vl{\textrm{v.l.}}   \def\varlec{\textrm{v.l.}} \def\varialectio{\textrm{v.l.}}
\def\vide{\textrm{{cf.\ }}}           \def\cf{\textrm{{cf.\ }}}
\def\videtur{\textrm{{vid.\,ut}}}
\def\crux{\textup{[\ldots]} }
\def\cruxx{\textup{[\ldots]}}
\def\unm{\textit{unm.}}        % unmetrical
%%%%%%%%%%%%%%%%%%%%%%%%%%%%%%%%%%%%



%%% Local Variables:
%%% mode: latex
%%% TeX-master: t
%%% End:

% additions/changes 2024-07-04 mm:
\TeXtoTEIPat{\lb}{<lb/>}
\TeXtoTEIPat{\begin {quote}}{<q>}
  \TeXtoTEIPat{\end {quote}}{</q>}
\TeXtoTEIPat{\begin {enumerate}}{<list rend="numbered">}
  \TeXtoTEIPat{\end {enumerate}}{</list>}
\TeXtoTEI{item}{item}

% additions/changes 2024-07-01 mm:
\TeXtoTEIPat{\unavbl {#1}}{<note type="foliolost">Folio lost in <ref>#1</ref></note>}
\TeXtoTEIPat{\NotIn {#1}}{<note type="omission">Omitted in <ref>#1</ref></note>}
\TeXtoTEI{graus}{span}[type="altrec"]
\TeXtoTEI{grau}{span}[type="altrec"]

% addition 2024-03-15 MD
\TeXtoTEI{manuref}{}

\TeXtoTEIPat{\alphaOne}{α<hi rend="sub">1</hi>}% N3
\TeXtoTEIPat{\alphaTwo}{α<hi rend="sub">2</hi>}% J5
\TeXtoTEIPat{\alphaThree}{α<hi rend="sub">3</hi>}% G4
\TeXtoTEIPat{\betaOne}{β<hi rend="sub">1</hi>}% P11
\TeXtoTEIPat{\betaTwo}{β<hi rend="sub">2</hi>}% C6
\TeXtoTEIPat{\betaOmega}{β<hi rend="sub">ω</hi>}% V3
\TeXtoTEIPat{\gammaOne}{γ<hi rend="sub">1</hi>}% N23
\TeXtoTEIPat{\gammaTwo}{γ<hi rend="sub">2</hi>}% J7
\TeXtoTEIPat{\deltaOne}{δ<hi rend="sub">1</hi>}% V19
\TeXtoTEIPat{\deltaTwo}{δ<hi rend="sub">2</hi>}% K3
\TeXtoTEIPat{\deltaThree}{δ<hi rend="sub">3</hi>}% C7
\TeXtoTEIPat{\deltaOmega}{δ<hi rend="sub">ω</hi>}% J6
\TeXtoTEIPat{\epsilonOne}{ε<hi rend="sub">1</hi>}% P15
\TeXtoTEIPat{\epsilonTwo}{ε<hi rend="sub">2</hi>}% N19
\TeXtoTEIPat{\epsilonThree}{ε<hi rend="sub">3</hi>}% V15
\TeXtoTEIPat{\epsilonFour}{ε<hi rend="sub">4</hi>}% J11
\TeXtoTEIPat{\epsilonOmega}{ε<hi rend="sub">ω</hi>}% N26
\TeXtoTEIPat{\etaOne}{η<hi rend="sub">1</hi>}% V1
\TeXtoTEIPat{\etaTwo}{η<hi rend="sub">2</hi>}% J10
\TeXtoTEIPat{\etaOmega}{η<hi rend="sub">ω</hi>}% N9

% addition 2023-12-11 MD:
\TeXtoTEIPat{\begin {metre}[#1]}{<note type="metre" target="##1">}
\TeXtoTEIPat{\end {metre}}{</note>}
\TeXtoTEIPat{\texttheta}{θ}

% change 2023-12-05 mm
\TeXtoTEI{teimute}{} 

% changes/additions 2023-11-27 MM:
\TeXtoTEIPat{\medialink {#1}{#2}}{<ref target="resources/#2">#1</ref>}

% changes/additions 2023-10-25 MM:
% new Sigla
\TeXtoTEIPat{\textAlpha}{Α}
\TeXtoTEIPat{\textalpha}{α}
\TeXtoTEIPat{\textBeta}{Β}
\TeXtoTEIPat{\textbeta}{β}
\TeXtoTEIPat{\textGamma}{Γ}
\TeXtoTEIPat{\textgamma}{γ}
\TeXtoTEIPat{\textDelta}{Δ}
\TeXtoTEIPat{\textdelta}{δ}
\TeXtoTEIPat{\textEpsilon}{Ε}
\TeXtoTEIPat{\textepsilon}{ε}
\TeXtoTEIPat{\textEta}{Η}
\TeXtoTEIPat{\texteta}{η}
\TeXtoTEIPat{\textChi}{Χ}
\TeXtoTEIPat{\textchi}{χ}
\TeXtoTEIPat{\textOmega}{Ω}
\TeXtoTEIPat{\textomega}{ω}

%new environments
\TeXtoTEIPat{\begin {postmula}[#1]}{<div type="postmula" xml:id="#1">} %%% changed 2024-07-01 mm
  \TeXtoTEIPat{\end {postmula}}{</div>}  %%% changed 2024-07-01 mm
  
\TeXtoTEIPat{\begin {altpostmula}[#1]}{<div type="altrec"><div type="postmula" xml:id="#1">} %%% added 2024-07-03 md
  \TeXtoTEIPat{\end {altpostmula}}{</div></div>} %%% added 2024-07-03 md

\TeXtoTEIPat{\begin {altava}[#1]}{<div type="altrec"><div type="avataranika" xml:id="#1">} %%% changed 2024-07-01 mm
  \TeXtoTEIPat{\end {altava}}{</div></div>} %%% changed 2024-07-01 mm

\TeXtoTEIPat{\sgwit {#1}}{<note type="inlineref"><ref>#1</ref></note>}

% changes/additions 2023-10-12 MM:
\TeXtoTEIPat{\\.}{}

% changes/additions 2023-08-15 MD:
\TeXtoTEIPat{\lineom {#1}{#2}}{<note type="omission">#1 omitted in <ref>#2</ref></note>}
%\TeXtoTEIPat{\startgray}{} %%% changed 2023-12-05 mm; not used 2024-03-26 MD
%\TeXtoTEIPat{\endgray}{} %%% changed 2023-12-05 mm; not used 2024-03-26 MD

% additions/changes 2023-06-05 mm:
%\TeXtoTEIPat{\lineom {#1}}{<note type="omission">Line omitted in <ref>#1</ref></note>}

% additions 2023-04-16 MD:
\TeXtoTEIPat{\,}{ }

% additions 2023-04-13 mm:
\TeXtoTEIPat{\begin {versinnote}}{<lg>}
  \TeXtoTEIPat{\end {versinnote}}{</lg>}

% additions 2023-04-05 MD:
\TeXtoTEIPat{\begin {testimonia}[#1]}{<note type="testimonia" target="##1">}
  \TeXtoTEIPat{\end {testimonia}}{</note>}
\TeXtoTEI{devnote}{s}[xml:lang="sa-deva"]

% app in philcomm und testimonia %%% added MM 2023-12-02
\TeXtoTEI{var}{note}[type="appinnote"]


\TeXtoTEI{anm}{note}[type="memo"] %% change 2023-04-16 MD
\TeXtoTEI{Anm}{note}[type="memo"] %% change 2023-12-05 MM
\TeXtoTEIPat{\startverse}{} %%% marked for change 2023-04-13 mm
\TeXtoTEIPat{\endverse}{} %%% marked for change 2023-04-13 mm
\TeXtoTEIPat{\newpage}{}
\TeXtoTEIPat{\marmas}{ } % changed 2024-03-17 MD
\TeXtoTEIPat{\marma}{}
\TeXtoTEIPat{\vin}{} % added by MD 2023-11-14

%%% modify environments and commands
%%% TEI mapping
% additions/changes 2022-06-07 mm:
\TeXtoTEIPat{ \& }{ &amp; }

% additions/changes 2022-06-01 mm:
\TeXtoTEI{skp}{seg}[type="deva-ignore"]
\TeXtoTEI{skm}{seg}[type="ltn-ignore"]

\TeXtoTEIPat{\rlap {#1}}{#1}

% additions/changes 2022-04-06 mm:
%\TeXtoTEI{sgwit}{ref}
\TeXtoTEI{textdev}{s}[xml:lang="sa-deva"]
\TeXtoTEIPat{\begin {col}[#1]}{<div type="colophon" xml:id="#1">}
  \TeXtoTEIPat{\end {col}}{</div>}
\TeXtoTEIPat{\begin {ava}[#1]}{<div type="avataranika" xml:id="#1">} %%% changed 2024-07-01 mm
  \TeXtoTEIPat{\end {ava}}{</div>} %%% changed 2024-07-01 mm
												   
\TeXtoTEIPat{\outdent}{}
\TeXtoTEIPat{\startaltrecension}{} %%% changed 2023-12-05 mm
\TeXtoTEIPat{\endaltrecension}{} %%% changed 2023-12-05 mm
\TeXtoTEIPat{\startaltnormal}{} % added by MD 2023-11-14 %%% changed 2023-12-05 mm
\TeXtoTEIPat{\endaltnormal}{} % added by MD 2023-11-14 %%% changed 2023-12-05 mm
\TeXtoTEIPat{\begin {alttlg}[#1]}{<div type="altrec"><lg xml:id="#1">}
  \TeXtoTEIPat{\end {alttlg}}{</lg></div>}



% additions/changes 2022-03-12 mm:
\TeXtoTEIPat{\begin {tlg}[#1]}{<lg xml:id="#1">}
  \TeXtoTEIPat{\end {tlg}}{</lg>}

\TeXtoTEIPat{\begin {translation}[#1]}{<note type="translation" target="##1">}
  \TeXtoTEIPat{\end {translation}}{</note>}
\TeXtoTEIPat{\begin {philcomm}[#1]}{<note type="philcomm" target="##1">}
  \TeXtoTEIPat{\end {philcomm}}{</note>}
\TeXtoTEIPat{\begin {sources}[#1]}{<note type="sources" target="##1">}
  \TeXtoTEIPat{\end {sources}}{</note>}


\TeXtoTEIPat{\begin {marma}[#1]}{<note type="marma" target="##1">}
  \TeXtoTEIPat{\end {marma}}{</note>}

\TeXtoTEIPat{\begin {jyotsna}[#1]}{<note type="jyotsna" target="##1">}
  \TeXtoTEIPat{\end {jyotsna}}{</note>}

\EnvtoTEI{description}{list}
\EnvtoTEI{itemize}{list}
\TeXtoTEIPat{\item [#1]}{<label>#1</label>\item}

\TeXtoTEI{tl}{l}
\TeXtoTEI{myfn}{note}[type="myfn"]
\TeXtoTEIPat{\getsiglum {#1}}{<ref target="##1"/>}

\TeXtoTEI{SetLineation}{}
\TeXtoTEI{noindent}{}
\TeXtoTEI{subsection*}{}

\TeXtoTEI{rlap}{}

% end additions/changes
% \TeXtoTEIPat{\skp {#1}}{#1}
% \TeXtoTEIPat{\skm {#1}}{}

\TeXtoTEIPat{\begin {prose}}{<p>}
  \TeXtoTEIPat{\end {prose}}{</p>}

\TeXtoTEIPat{\begin {tlate}}{<p>}
  \TeXtoTEIPat{\end {tlate}}{</p>}

\TeXtoTEI{emph}{hi}
\TeXtoTEI{bigskip}{}
% \TeXtoTEI{/}{|}
\TeXtoTEI{tl}{l}
\TeXtoTEIPat{english}{}
%\TeXtoTEIPat{-}{ } %% change 2023-04-16 MD
%\TeXtoTEIPat{°}{} %% change 2023-04-16 MD
\TeXtoTEIPat{\textcolor {#1}{#2}}{<hi rend="#1">#2</hi>}

% \TeXtoTEIPat{\eyeskip}{}
% \TeXtoTEIPat{\aberratio}{}
% \TeXtoTEIPat{\ad}{}
\TeXtoTEIPat{\add}{<hi rend="italic">add.</hi>} %% change 2023-04-16 MD
% \TeXtoTEIPat{\ann}{}
\TeXtoTEIPat{\ante}{<hi rend="italic">ante</hi> } %% change 2023-04-16 MD
\TeXtoTEIPat{\post}{<hi rend="italic">post</hi> } %% change 2023-04-16 MD
% \TeXtoTEIPat{\codd}{}
% \TeXtoTEIPat{\conj }{}
% \TeXtoTEIPat{\contin}{}
% \TeXtoTEIPat{\corr}{}
% \TeXtoTEIPat{\del}{}
% \TeXtoTEIPat{\dub}{}
% \TeXtoTEIPat{\emend }{}
% \TeXtoTEIPat{\expl}{}
% \TeXtoTEIPat{\ȩxplicat}{}
% \TeXtoTEIPat{\fol}{}
% \TeXtoTEIPat{\gloss}{}
% \TeXtoTEIPat{\ins}{}
% \TeXtoTEIPat{\im}{}
% \TeXtoTEIPat{\inmargine}{}
% \TeXtoTEIPat{\intextu}{}
% \TeXtoTEIPat{\indist}{}
% \TeXtoTEIPat{\iteravit}{}
% \TeXtoTEIPat{\lectio}{}
% \TeXtoTEIPat{\leginequit}{}
% \TeXtoTEIPat{\legn}{}
% \TeXtoTEIPat{\illeg}{<hi rend="italic">illeg.</hi>}
\TeXtoTEIPat{\illeg}{<gap reason="illeg."/>} %%% change 2023-04-11 mm
% \TeXtoTEIPat{\om}{<hi rend="italic">om.</hi>}
\TeXtoTEIPat{\om}{<gap reason="om."/>} %%% change 2023-04-11 mm
% \TeXtoTEIPat{\primman}{}
% \TeXtoTEIPat{\prob}{}
% \TeXtoTEIPat{\rep}{}
% \TeXtoTEIPat{\sequentia}{}
% \TeXtoTEIPat{\supralineam}{}
% \TeXtoTEIPat{\interlineam}{}
\TeXtoTEIPat{\vl}{<hi rend="italic">v.l.</hi>}
% \TeXtoTEIPat{\vide}{}
% \TeXtoTEIPat{\videtur}{}
% \TeXtoTEIPat{\crux}{}
% \TeXtoTEIPat{\cruxxx}{}
\TeXtoTEIPat{\unm}{<hi rend="italic">unm.</hi>}
\TeXtoTEIPat{\lacuna}{<gap reason="lac."/>} % addition 2024-03-24 MD
\TeXtoTEIPat{\lost}{<gap reason="lost"/>} % addition 2024-06-24 MD

% List of Scholars
\DeclareScholar{nos}{nos}[
forename=HPP,
surname=Team]

% Nullify \selectlanguage in TEI as it has been used in
% \DeclareWitness but should be ignored in TEI.
\TeXtoTEI{selectlanguage}{}



\NewDocumentCommand{\skp}{m}{}
\NewDocumentCommand{\skm}{m}{\unless\ifinapparatus#1-\fi}

\SetTEIxmlExport{autopar=false}
\NewDocumentEnvironment{tlg}{O{}}{
	\begin{ekdverse}
	\indentpattern{0000}}{
	\end{ekdverse}
	\vskip 0.75\baselineskip}
\NewDocumentEnvironment{alttlg}{O{}}{}{}
\NewDocumentCommand{\tl}{m}{#1}

%%%%%%

\def\startaltrecension#1{
  \stopvline
  \begin{ekdverse}[type=altrecension]
    \indentpattern{0000} 
    \begin{patverse*}
      \color{gray}
      \setvnum{#1}}
\def\endaltrecension{
  \end{patverse*}
  \end{ekdverse}
  \vskip 0.75\baselineskip
  \startvline}

%%%%%%

\newcommand{\myfn}[1]{\footnote{\texteng{#1}}}
\renewcommand{\thefootnote}{\texteng{\arabic{footnote}}}
\newcommand{\devnote}[1]{\selectlanguage{iast}{\scriptsize #1}\selectlanguage{english}}
\newcommand{\outdent}{\hspace{-\vgap}}
\newcommand{\sgwit}[1]{{\small (\getsiglum{#1})}\selectlanguage{iast}}
\newcommand{\NotIn}[1]{\texteng{\footnotesize (om. \getsiglum{#1})}\selectlanguage{iast}}

\def\om{\emph{om.}} % \!}
\def\illeg{\emph{illeg.}} %\!}
\def\unm{\emph{unm.\:}}
\def\recte{\texteng{r.\:}}
\def\for{\texteng{for }}
\def\sic{\emph{sic}}

\makepagestyle{HPed}
\makeoddhead{HPed}{\small\texteng{HP1 OldMss}}{}{\small\texteng{\today}}
\makeevenhead{HPed}{\small\texteng{HP1 OldMss}}{}{\small\texteng{\today}}
\makeoddfoot{HPed}{}{\small\texteng{\thepage}}{}
\makeevenfoot{HPed}{}{\small\texteng{\thepage}}{}

\newcommand{\myfn}[1]{\footnote{\texteng{#1}}}
\setlength{\footmarkwidth}{1.3em}
\setlength{\footmarksep}{0em}
\setlength{\footparindent}{1.3em}
\footmarkstyle{\textsuperscript{#1} }
\usepackage{fnpos}
\makeFNbottom

% additions/changes 2022-03-12 mm:
\SetTEIxmlExport{autopar=false}
\NewDocumentEnvironment{marma}{O{}}{}{}
\def\sthana{}

\begin{document}
\pagestyle{HPed} %
\begin{ekdosis}
  \SetLineation{lineation = none,}
 
  \chapter{Variants and Versions}

\sthana{24a}
 
  \begin{marma}[hp01_024]
    \begin{description}
%    \item[(ku(r)k(k)uṭāsana-)]{}
    \item[-vat kṛtvā]{V1, N22}
    \item[-vat kratvā]{P6}
    \item[-bandhastho]{A1, J8, L1, M1, V6, B1, B2, B3, Bo2, C1, C2, C3, C4, N2, C6, C8, J1, J3, J6, J13, J14, K1, Ko, LD1, LD2, N1, V2, V3 (about \emph{stho}), N4a, V8, N5, N8, N9, N11, V21, N13, N16, N18, N19, N20, N21, N24, N25, N26, P1, P2, P3, P4, P7, P10, P11, P14, P15, P17, P20, P21, P22, P23, R4, Tue, V17, V14, R3, R5, V13, V10, G1
    \item[-bandhasthe]{J7, V19}
    \item[-badhascho]{Ba, Bo3, V28, V22}
    \item[-baṃdhascho] J5, J11, J15, P12, P13, P18, P27, V15
    \item[-baddhastho] N12, P25, 
    \item[-bandhastu]{Bo1}
    \item[-madhyastho]{C9, J10, J12, J16 (\emph{baṃ} is given in margin), J17, N17, V4, N6, N7, N10, V20, V18, P5, P8, P24, P26, V12, LD1}
     \item[-madhyasthā]{P9}
    \item[-madhyasthau]{V16}
    \item[-baṃdhasthau]{N23}
    \item[-badhestu]{J2}
    \item[-vadhnastho] P16, 
    \item[(unavailable/illegible)] Ba5, C7, J4, N14, C5, G11, R2, T, Ba3
    \end{description}
 \end{marma}


\begin{marma}[hp01_025]

  \sthana{25c}

  \begin{description}
\item[dhanurākarṣaṇaṃ kṛtvā]
\item[dhanusaṃkarṣaṇaṃ kuryād] Bo2,
\item[dhanuḥsaṃkarṣaṇaṃ kuryād] N9,
\item[dhanurākarkhaṇa kṛtvā] V19,
\item[dhanurākarṣakākuṣṭaṃ] M3,
\item[dhanurākarṣaṇā[kṛ]ṣṇaṃ] P7,
\item[dhanurākarṣaṇaḥ kṛṣṭaṃ] V1,
\item[dhanurākarṣaṇākarṣaṃ] Ko,
\item[dhanurākarṣaṇākṛṣṭaṃ] Ba, C6, C8, G5, G11, J5, J11, M1, N12, N20, P15, V15, V2, V25, 
\item[dhanurākarṣeṇākṛṣṭaṃ] J2,
\item[dhanurākarṣaṇaṃ kṛtvā] B2, C1, C4, J1, J3, J6, J10, J14, J15, K1, L1, MYO20, N5, N11, N16, N18, N19, N22, N24, N25, P3, P6, P13, P23, Pen,
\item[dhanurākarṣana kṛtvā] N21,
\item[dhanurākarṣaṇaṃ kṛṣṭa] J4,
\item[dhanurākarṣaṇaṃ kṛṣṭaṃ] C6,
\item[dhanurākarṣaṇaṃ kṛuṣṭaṃ] G4,
\item[dhanurākarṣaṇaṃ kuryād] A1, B1, B3, Ba5, Bo1, Bo3, C2, C3, C9, J7, J10, J12, J13, J16, J17, Jyo, LD1, LD2, N13, N17, P1, P2, P4, P5, P7, P9, P12, P17, V16, V18, 
\item[dhanurākarṣaṇāt kaṣṭaṃ] J8, P11, V3, 
\item[dhanurākarṣaṇāt kṛṣṭaṃ] L2, N2, N23, P11, V14, V17, 
\item[jānūrākarṣaṇa kṛuṣiḥ] Ba4, 
\item[(unavailable/illegible)] A1, C5, C7, K2, K3, N3, N14, V9, P10, 
\end{description}
 \end{marma}


\begin{marma}[hp01_027]

  \sthana{27d}

  \begin{description}
\item[jaṭharapradīptaṃ]    V6 (jaraṭha), B1, B2, B3, Bo1, J13, N1, V22, N13, N21, P2, P12, P13, P14, P17, R3, R4, 
\item[jaṭharapradīpti]    Ba5, V10
\item[jaṭharapradiptaṃ]    P4,
\item[jaṭharaprakṛṣtaṃ]    V14
\item[jaṭharapradīpaṃ]    P6,
\item[jaṭharapradīptiṃ]    Bo3, Ko, N4a, P18, P22, Tue, R5
\item[jaṭharāgnidīptiṃ]    P22a,
\item[jaṭharapracaṇḍaṃ]    K1, V1, N22, V15, G1
\item[jaṭharaṃ pracaṇḍaṃ]    J11, P16, 
\item[jaṭharapracaṃḍa]    J5, P11, P27, 
\item[matsyendra-vīra-jaṭhara-pracaṇḍaṃ]          J4, N19, V1 [vīra-ja is unmetrical?]
\item[budhaṃ pracaṇḍa]    V8
\item[jaṭharapravṛdhaḥ]    P21, P23, 
\item[jaṭharapravṛddha]    J10, J17, V2, N8, N11
\item[jaṭharapravṛddhau]    J15, P20, 
\item[jaṭharapravṛddhaṃ]    N24
\item[jaṭharapravṛddhiṃ]    C1
\item[jaṭharaprabuddhaṃ]    C8, J14, N2, N12, V21, N23, V12,
\item[jaṭharaprabuddha]    Ba, P25, 
\item[jaṭharaprabuddhau]    C4 ("jvalanapradīptam" is also given above), L1, V19, V28, N5, N25, P3
\item[jaṭharaprabodhaṃ]    A1, C6, P7, P26,
\item[...bodhaṃ]       R2,
\item[jaṭharāprabodhaṃ]    V20
\item[jaṭharaṃ prabadhaṃ]    J2
\item[jaṭhara pravaddhaṃ] N2, 
\item[jvalanapradīptaṃ]    LD1, Bo2, C3, C9, J1, J3, J6, J8, N17, V3, V4, N6,, N9, N18, V18, N26, P1, P5, V17, P8, P24, V13, V16, P9
\item[jvalanaṃ pradīptaṃ]    J7, J12, LD2, N10, N16, N20, P10, 
\item[t(?)valanaṃ pradiptaṃ]    J16,
\item[jvalatpradīptaṃ]    N7,
\item[vīrapracaṃḍaṃ] P15,
\item[(unavailable/illegible)]    C5, C7, M1, N14, G11, T, Ba3
    \end{description}
 \end{marma}


 \begin{marma}[hp01_027]

   \sthana{27d}
   
    \begin{description}
    \item[candrasthiratvaṃ ca dadāti] Ba5, J7, J10ac, J17, LD1, LD2, V6, B2, Bo1, Bo3, N4a, V4, V22, N10, N13, P1, Tue, P8, P22, R5, V13, V10
      \item[candraṃ sthiratvaṃ ca dadāti] P9("ca" in margin)
    \item[candre sthiratvaṃ ca dadāti] C9,
    \item[candrasthiratvaṃ hi dadāti] B1, C2, N1, R3
    \item[cādrasthiratvaṃ hi dadāti] B3
    \item[candrasthiratvaṃ pradadāti] C1, N24
    \item[caṇḍasthiratvaṃ ca(?) dadāti] N7,
    \item[daṇḍasthiratvaṃ ca dadāti] A1, Ba, Bo2, C3, C6, C8, J1, J2, J8, J10pc, J11, J12, J14, J15, J16, K1, Ko, N2, M1, N17, V1, V2, V3, N6, N9, N12, P10, V20, N16, N18, N19, N20, N22, N26, P6, P18, P25, V17, V12, R2
    \item[daṇḍāsthiratvaṃ ca dadāti] P5
    \item[daṇḍasthira.ntya karoti] G1
    \item[daṇḍāsthitatvaṃ ca dadāmi] V14
    \item[daṇḍaṃ sthiratvaṃ ca dadāti] P7, P11, 
    \item[daṇḍasthiratvaṃ hi dadāti] J13, P2
    \item[daṇḍasthiratvaṃ hi dadātti] P4
    \item[daṃḍaschiratvaṃ hi dadāti] P12, 
    \item[daṇḍasthiratāṃ ca dadāti] V21
    \item[daṃḍaschiratvaṃ dadāti] J5,
    \item[daṇḍasthiratāṃ sa dadāti] J3, J6,
    \item[daṇḍasthiratvaṃ vidadhāti] C4, L1, V28, V18, N25, V16
    \item[daṇḍasthiratvāṃ vidadhāti] P3
    \item[daṇḍasthiratvaṃ pradadāti] N8, N11, V19, N23
    \item[daṇḍaṃ sthiratvaṃ pradadāti] P23, 
    \item[daṃḍe sthiratvaṃ pradadāti] N5
    \item[daṇḍasthiratvaṃ jvalanaṃ] V8
    \item[(daṇḍa?)sthirasvaṃ ce dadāti] N2
    \item[dehasthiratvaṃ ca dadāti] N21
    \item[kāyasthiratvaṃ ca dadāti] V15
    \item[kaḍasthiratvaṃ ca dadāti(?)] J4
    \item[vīryasthiratvaṃ ca dadāti] P22a,
    \item[pracaṃḍahrunmaṃḍalakhaḍanāstraṃ] P21, 
    \item[pracaṃḍarugmaṃḍalakhaṃḍanāstraṃ] P13, P14, P17, P20, P24, P26, R4, 
    \item[rugmaṃḍalākhaṃḍanakhaṃḍanāstraṃ] P27,
    \item[picaṃḍarukmāṃḍanakhaṃḍanāstraṃ] P15,
    \item[vicaṃḍayugmaṃḍalakhaṃḍanāstraṃ] P16,
    \item[(unavailable/illegible)] C5, C7, N14, T, Ba3
    \end{description}
 \end{marma}

 
 
\begin{marma}[hp01_028]
\sthana{1.28b}
   \begin{description}
   \item[dorbhyāṃ padāgradvitayaṃ] A1, Bo1, C2, C6, C8, J4, V1, M1, B2, B3, N1, N4a, P11, P14, P18, P20, V21, V20, V22, G1, V28, N16, N19, N21, N22, N23, N24, P2, P6, P12, P13, P17, P26, R4, R5, V13, R2
   \item[dorbhyāṃ padāgredvitayaṃ] V14
   \item[dorbhyāṃ padāgradvītayaṃ] P7
   \item[dorbhyāṃ padāgnadvittayaṃ] P4 (could be "padāgra")
   \item[dorbhyāṃ padāgraṃ dvitīyaṃ] Ko, V15
   \item[dorbhyāṃ padāgradvitīyaṃ] R3, P25, P27, 
   \item[dorbhyāṃ padāgraṃthidvayaṃ] N2,
\item[dorbhyāṃ padagradvitayaṃ] Bo3, Ba5, K1, N13, P14, P22, Tue, V10
\item[dorbhyāṃ padāgryau dvitayaṃ] J5
\item[dorbhyāṃ pādāgrādvitayaṃ] B1
\item[dorbhyāṃ ca pādadvitayaṃ] C4, L1, V19, N5, N7, N8, N11, N12, P3, P16, P23, 
\item[dorbhyāṃ ca pādaṃ dvitayaṃ] C1
\item[dorbhyāṃ ca pādau dvitayaṃ] N25
\item[dorbhyāṃ ca pādāgradvitayaṃ] J13
\item[dorbhyāṃ sa pādāgradvayaṃ] J2
\item[dvābhyāṃ karābhyāṃ dvitayaṃ] Bo2, C3, C9, J1, J3, J6, J7, J8, J10, J15, J16, J17, LD1, LD2, N17, V6, V4, V3, N9, N10, N18, N20, V18, N26, P1, P5, P10, P24, V17, V16, LD1, P9("karābhyāṃ" in margin)
\item[dorbhyāṃ ca pādadvayakaṃ] P21, 
\item[dvābhyāṃ karābhyāṃ dvitīyaṃ] N6, 
\item[dvābhyāṃ karābhyāṃ hitayaṃ] J12, V12
\item[dvābhyāṃ padāgraṃ dvitayaṃ] J11
\item[dvābhyāṃ pādāgrādvitīyaṃ] P8
\item[(unavailable/illegible)] Ba, C7, G11
\item[dorbhyāṃ padagraṃthidvayaṃ] J14, N2
\item[.. rbhyāṃ pādaṃbhyāṃ dvitayaṃ] V8
\item[(unavailable/illegible)] C5, N14, T, Ba3, 
    \end{description}
 \end{marma}



 \begin{marma}[hp01_035]
 \sthana{35b}
   \begin{description}
\item[asyahṛdaye dhṛtvā]    J7, J12, A1, V4, V18, V16, P8
\item[asya hṛdayaṃ dhṛtvā]   P9
\item[āsyahṛdaye dhṛtvā]    N10, V1, P1, P6
\item[āsyahridayaṃ dhṛtvā]    J16
\item[hṛdayaṃ kṛtvā] G1
\item[eva hṛdaye dhṛtvā] C4, C7, C8, J5, J10, J11, J15, J17, Ko, V19, N2, N5, V21, N25, P3, P20, P23, P24, P25, V15, V14, R8, V12
\item[ava hṛdaye dhṛtvā] P11, 
\item[eva hṛdaye dhṛtvā]    P21,
\item[eva hṛdaye kṛtvā]    Bo3, K1, N4a, V22, N13, N19, N24, P14, P27, Tue, V10
\item[eva niyataṃ dhṛtvā]    C1, C3, J14, R4, V8, V28
\item[eva niyataṃ kṛtvā] B1, B3, Bo2, C2, C6, J1, J3, J6("sudṛdha" is also given above), J8, J13, L1, N17, LD1, N1, V3, N12, V20, N16, N18, N20, N21, N26, P2, P4, P7, P12, P13, P17, P18, P26, R3,
  R5('hṛdaye' canceled and replaced by 'niyataṃ')
\item[ekahṛdaye dhṛtvā]    J4, N3, N23, P15, G11
\item[ekahṛdayo dhṛtvā]    J2, M1
\item[eksak ahṛdaye dhṛtvā]    P10, 
\item[eva sudṛḍhaṃ kṛtvā]    N8, N11
\item[eva dṛḍhayan dhṛtvā]    P16,
\item[vāmaniyataṃ dhṛtvā]    V6
\item[samyak(ū?) hṛdaye dhṛtvā]    N9    
\item[āsyahṛdaye kṛtvā]    N22, V17
\item[asya hṛdayeḥ dhṛtvā]    C9,
\item[meḍhre pādam atho hṛdi svacibukaṃ]    N7
\item[meṃḍhraṃ pādam athaikam]            Bo1, LD1, LD2
\item[meḍhre pādam athaikm eva hṛdaye]    B2, Ba5,
\item[meṃḍhreṃ pādam athaikam]            LD2
\item[meḍhraṃ pādam athaika samyak meva hṛdaye dhṛtvā]           P5
\item[(unavailable/illegible)]    Ba, C5, V13, N6, N14, P22, R2, T, Ba3
    \end{description}
 \end{marma}

\begin{marma}[hp01_037]
 \sthana{Before 1.37}
   \begin{description}
\item[pūrvoktam eva matsyāsanaṃ] N12,
\item[pūrvoktam eva saṃmataṃ] P11,
\item[pūrvoktam eva matsaṃmataṃ] J14, Ko,
\item[pūrvoktam eva matsāsanaṃ] N24,
\item[pūrvoktam evāsanaṃ mataṃ] J2,
\item[pūrvoktam eva matsyendramataṃ] B2, Ba5, C6, J5, J11, N23,
\item[dvāvektimeva matseṃdramataṃ] C8, 
\item[(unavailable/illegible)]  A1, B1, B3, Ba, Bo1, Bo2, Bo2, C2, C3, C4, C5, C9, J1, J3, J4, J6, J7, J8, J12, J13, J15, J16, J17, K1, K2, LD1, LD2, N13, N14, N16, N17, N18, N20, N21, N22, N25, P1, P2, P3, P4, P5, P6, P8, P9, P10, P11, P12, P13, 
\end{description}
 \end{marma}


    

 \begin{marma}[hp01_038]
 \sthana{38a}
   \begin{description}
   \item[yameṣv eva mitāhāra(ḥ)] C2, J10, J17, M1, V1, V6, P8
   \item[yameṣv eva sitāhāra] V16
   \item[yameṣv eva mitāhāro] N7,
   \item[yameṣv eva mitāhāram] C4, B2, J14, N2, N5
   \item[yameṣv eva ātmacetana] V28
   \item[yameṣv iva mitāhāra(ḥ)] B1, B3, Bo1, Bo2, C3, C9, J3, J6, J8, J13, J15, K1, Ko, LD1, N17, N1, V3, V4, N9, N10, N14, N16, N18, N21, N26, P1, P2, P4, P12, P17, R3, V12, LD1, P9
   \item[jameṣṭiva mitāhāram] J5
   \item[yameṣv iva mitāhāro] P13, R4, V18
   \item[yameṣv iva mitāhāre] P5, P24, 
   \item[yameṣv iva mītāhāra] J16,
   \item[yameṣv iva mitāhāram] A1, Bo3, Ba5, C1, C7, C8, J4, J11, L1, LD2, V19, N4a, V22, V21, N13, N23, G11, N24, N25, P3, P7, P11, P14, P15, P16, P18, P20, P21, P23, P25, P26, P27, Tue, V15, R5, V13, V10
   \item[yamiṣviva mitāhāram] V14
   \item[yameṣv iva mitāhā(ḥ)] V8
   \item[yameṣv iva mityāhāraḥ] P10, 
   \item[niyameṣu mitāhāro] N8, N11, N12
   \item[niyameṣu mitāhārā] J12,
   \item[(unavailable/illegible)] Ba, V20, R2
   \item[yameśiva mitāhāram] N19
   \item[yameṣv eva mitāhāra] N20, V17
   \item[yameṣu ca nirāhārī] N22, P6
   \item[yameṣu ca mitāhārm] C6,
   \item[yamaṣviva mitāharm] J1,
   \item[niyameṣu mitāhari] J2
   \item[yameṣv api mitahāraḥ] J7("yamesṣv iva" is deleted)
   \item[(unavailable/illegible)] C5, N6, T, Ba3, G1, P22,
    \end{description}
 \end{marma}

\sthana{1.38b}
   \begin{description}
\item[ahiṃ niyameṣv iva]    B1
\item[ahiṃsā niyameṣu ca]    J4, M1, V1, N22, P6
\item[ahiṃsā niyameṣv api]    B2, J14, N2, N24, V14
\item[ahiṃsā niyameṣv iva] A1, B3, Ba5, Bo1, Bo2, Bo3, C2, C3, C4, C7, C8, C9, J1, J3, J5, J6, J7, J8, G11, J10, J11, J13, J15, J17, Ko, LD1, LD2, N17, V3, V4, N5, N9, V21, N13, N14, N16, N19, N20, N21, V18, N25, N26, P1, P2, P4, P7, P10, P11, P12, P13, P15, P16, P17, P18, P23, P24, P25, P26, V16, P8, R3, R5, V13, V12, LD1, P9, P14, 
\item[ahiṃsā niyameṣv ivaṃ] P21,
\item[ahiṃsā niyāmaṣv iva]    P5
\item[ahiṃsā niyameṣu eva]    V17
\item[āhiṃsā niyameṣv iva]    Ko,
\item[ahiṃsā nīyame sthitā]    K1,
\item[ahiṃsāṃ nīyame sthitā]    P27, 
\item[ahisā nīyameṣv iva]    J16,
\item[aṃhiṃsā niyamaṣv iva]    J12,
\item[svāhiṃsā niyameṣv iva]    V6
\item[ahiṃsāṃ niyameṣv iva] L1, V19, N4a, V22, P3, P20, Tue, V15, V10
\item[ahiṃsāṃ niyameṣu ca]    C6,
\item[ahiṃsā niyaṃmai(?)ṣv iva]     N1
\item[(a)hiṃsā ca niyameṣv iva]    N7,
\item[yathāhiṃsā yameṣv iva]    N8, N11
\item[ahiṃsā niyameṣv iha]    N10
\item[asiṃsā niyame sthata(ḥ)]           V8
\item[yameṣv evātma-cintanaṃ]    N12
\item[hiṃsā ca niyameṣv iva] R4, 
\item[na hiṃsāṃ niyameṣv iva]    N23
\item[kṣamasye ātmacintane]    J2
\item[(unavailable/illegible)]    Ba, C5, N6, V28, V20, N18, R2, T, G1, Ba3, P22, P25, 
    \end{description}
 \end{marma}

\begin{marma}[hp01_039]
\sthana{39d}       
       \begin{description}
 (ṇ/n and ś/s irrelevant)
\item[suṣumṇā iva nāḍikā]    V1, N22, P6
\item[suṣumṇā iva nāḍiṣu]     B2, J4, J14, K1, Ko, N2, V8, N19, V12, G11
\item[suṣumṇā caiva nāḍiṣu]     P26, 
\item[suṣumṇā ca nāḍiṣu]    J5,
\item[suṣumṇā eva nāḍiṣu ]    V20
\item[suṣumṇā iva nāḍiṣuḥ] N8, P15, 
\item[suṣumiva nāḍiṣu]     V28
\item[suṣumṇām eva nāḍiṣu]    J10ac, J17, N24, V14
\item[suṣumṇām iva nāḍiṣu ]    A1, C1, C4, C6, C7, C8, J1, J2, J3, J6, J11, J15, L1, M1, V19, N5, N7, N12, N16, N23, N25, P11, P16, P20, P21, P23, P25, P27, V15, V13
\item[suṣuṇmām iva nāḍiṣu]    P7
\item[suṣumṇāśva nāḍiṣu]    N11,
\item[suṣumṇām iva ḍinoṣu]    P3
\item[nāḍīnāṃ malaśodhanam]    B1, B3, Ba5, Bo1, Bo3, C2, C3, C9, J8, J12, J13, LD2, N1, N4a, N10, N13, N18, N20, N21, N26, V3, V22, P1, P2, P4, P12, P13, P14, P17, P18, Tue, V17, R3, R5, V10
\item[nāḍiṣu malaśodhanam]  R4,
\item[nāḍināṃ malasodhanaṃ]    J16,
\item[nāḍīnāṃ malaśodhanā]   P9
\item[nāḍīnāṃ malaśodhane]    J7, J10pc, LD1, N9, N17, V6, V21, N14(seems "dī"), V18, P5, P10, V16, P8, P24, V12, LD1
\item[suṣumṇādiṣu śodhanam ]     V4
\item(unavailable/illegible)]    Ba, Bo2, C5, N6, R2, T, G1, Ba3, P22, 
    \end{description}
 \end{marma}

 \begin{marma}[hp01_040]
 \sthana{1.40e}
   \begin{description}
\item[śramādau bahubhiḥ pīṭhaiḥ] V1
\item[śramadair bahubhiḥ pīṭhaiḥ] B1, B3, Bo1, Bo2, C1, C2, C6, C8, J3, J6, J13, J14, K1, M1, G11, N1, N11, N12, N16, N21, N22, N23, N24, P2, P4, P7, P11, P12, P13, P16, P17, P21, P23, P25, P27, V14, R3, V13
\item[śramadair bahubhiḥ pīṭhai]    P6,
\item[śramadai bahubhiḥ pīṭhaiḥ]    Ba, C9mg, J5,
\item[śramadair bahubhi pīṭhai]    N8
\item[śramādaiḥ bahubhi pīṭhaiḥ]    V21
\item[śramadairghyādibhiḥ pīṭhaiḥ]    C4, C7, L1, V19, N5, N25, P3, P20, 
\item[śramadairdhyādibhiḥ pīṭhaiḥ]    V28
\item[śamardda bahubhiḥ pīṭhaiḥ]    J1,
\item[sanastai bahubhyaḥ pīṭai]    V8
\item[(samastair?) bahubhiḥ pīṭhaiḥ]    N2
\item[samastair bahubhiḥ pīṭhaiḥ]    J2, J4, J11, Ko, N2, P15, 
\item[samasta bahubhiḥ pīṭhai]    V15
\item[kim ādyair bahubhiḥ pīṭhaiḥ] A1, C9, J7, J8, J10, J17, LD2, N17, P26, V6, V3, V4, N7, N14, V20,
  N18(No visarga at the end), N20, V18, N26, P5, P10, V17, P8, V12, LD1, R2, P9
\item[kim ādyair bahubhiḥ pīṭheḥ]    N9, V16
\item[kim adyair bahubhiḥ pīṭhaiḥ]    N10,
\item[kim ādyair bahubhiḥ piṭhaiḥ]    J15, LD1, P1, P24, 
\item[kim anyair bahubhiḥ pīṭhaiḥ]    Ba5, C3, N4a, N13, P14, P18, Tue, R5, V10
\item[kim anyair bahuniḥ pīṭhaḥ]    Bo3, V22
\item[kim arthai bahubhiḥ pīṭhai]    J12, J16("adyair" is also given),
\item[kiṃ śramair bahubhiḥ pīṭhaiḥ]    B2
\item[pramadair bahubhiḥ pīṭheḥ]    N19
\item[tathaikasmin eva dṛḍham]    R4, 
\item[(unavailable/illegible)] C5, N6, T, G1, Ba3, P22,
    \end{description}

 \sthana{1.40f}
   \begin{description}
\item[sadā siddhāsanasthiteḥ]    A1, V20, V12, R2
\item[sadā siddhāsanasthiteḥ]    P26, 
\item[sadā siddhāsane sati] B1, B3, Bo1, C2, C9, J7, J8, J10, J11, J12, J15, J17, LD1, LD2, N17, V1, V6, N1, V4, N7, N9, N10, G11, N20, N22, V18, N26, P1, P2, P4, P5, P6, P8, P9, P10, P12, P24, V17, V16, LD1,
\item[sidhe siddhāsane sati]    V10
\item[siddhe siddhāsane sati] P18,
\item[sadā siddhāsane satī]    J16,
\item[sadā siddhāsane satya]              V3
\item[sadā siddhāsanaṃ(?)]    V8
\item[sadā siddhāsanā nyāsāt]    V22
\item[sadā siddhāsanābhyām sādyogī]    V14
\item[yadā siddhāsane sati]    J2, J4, N2, P15,  
\item[yadā siṃhāsane sati]    Ko,
\item[siddhe siddhāsane sati]    Ba5, Bo3, C3, N4a, N13, P14, Tue
\item[siddhe siddhāsane miti]    R5,
\item[kiṃ sadā siddhasādhanaiḥ]    M1, P11, 
\item[kiṃ syāt siddhāsane sati] Bo2, C1, C4, C7, J1, J3, J5, J6, J13, L1, P13, P16, P20, V19, B2, V28, N5, N16, N21, N25, P3, P17, P21, P23, V15, R3
\item[kiṃ syāt siddhāsane mati(ṃ?)]    N8
\item[kiṃ syāt siddhāsane sthite]    C6, C8, P25, V21
\item[kiṃ syā siddhāsane sthite ]    P7
\item[kiṃ syāt siddhāsana sthite ]    N23
\item[kiṃ syāt siddhāsane schite]    Ba,
\item[kiṃ vā siddhāsane sati]    J14, N12, N24
\item[kiṃ syāt siddhāsanai(?)ti]    N18
\item[kiṃ sadā sevanena tu]     K1, N19(1.40h is "baddhe siddhāsane sadā")
\item[baṃdhe siddhāsane sadā]  P27,
\item[baddhe siddhāsane sadā]  R4,
\item[va??t siddhāsane sadā]  N11,
\item[(unavailable/illegible/unable to read)]    C5, N6, N14, V13, T, G1, Ba3, P22,
    \end{description}
 \end{marma}

 \begin{marma}[hp01_046]
 \sthana{1.46cd}
\item[(utta(m)bhya cibukaṃ ~)]
\item[vakṣa sthāpayet ]    J8, J10, J17, LD2, V6, N7, P8, N18, V12, V3, P9("thā" unclear)
\item[vakṣaḥ sthāpayet]     Bo1, B1, B2, C3, J1, J12, J15, J16(sthāpaye),  K1, LD1, V18, P1, P5, P10, P24, R4, V16, N10, N22, LD1
\item[vakṣaḥ sthāpaye]     P26, 
\item[vandā sthāpayet]    V17
\item[vakṣa sthāpayat]     N17
\item[vakṣe saṃsthāpayat(?)]    Ko,
\item[vakṣe uthāpayet]    V4
\item[vakṣa sthāpya?]    J7(A word is deleted there), C9, N5(sthāpya sevanaṃ)
\item[vatsa sthāpayet ]    N20, N26
\item[vakṣaṃ sthāpayet]     V1
\item[vakṣasy āsthāpya]     C4, L1, M1, V19, P3
\item[vakṣasy utthāya]    Bo2, C6, N16, N8, N11,
item[vakṣasy stthāpya] G11
\item[vakṣasy utthāpya] Ba5, Bo3, C1, J13, J14, N2, N13, N21, N23, N24, P13, Tue, N1, N4a, P14, P18, P23,  R5
\item[vakṣasy usthāpya] P21,
\item[vakṣasy utthāyo] P15,
\item[vakṣasy utṭhāpyo]    J4, J11,
\item[vakṣasy utṭhāgre]    J6
\item[vakṣasy oschāpyo]    J5
\item[vakṣasy osthāpyot]    P11, 
\item[vakṣyasy utthāpyot patane...]    N19
\item[vakṣyasy asthāpya]    N25
\item[vakṣyasy āsthāpya]    P20,
\item[vakṣyapyutchāpya]    Ba, 
\item[vakṣyapyutthāpya]    P25, 
\item[vakṣe sthāpayet]    N14, P2, P4, N9
\item[vakṣe syu sthāpya pavanaṃ]     V10
\item[vakṣye sthāpayet]    B3, C2, P17, 
\item[vakṣye schāpayet]    P12, 
\item[vakṣye pyutthāya]    C8
\item[vakṣyesu uttānya]    V14
\item[vikṣe sthāpayet]    R3
\item[cakṣasya utthāpya]    V15
\item[svakṣasy utthāpya]    P7
\item[urghātopaticotanau karau kṛtvā tato dṛśaṃ?]    J2
\item[dakṣasy utthāgret(?)]    J3
\item[utkuṃcya cibukaṃ vakṣa sthāpayet ]       P6
\item[sannāya cibukaṃ vakṣyasy utthāpya]      N2
\item[yuktaṃ vakṣastha-cibukaṃ mākramet]    N12
\item[(unavailable/illegible/unable to read)]    A1, C5, N6, P16, V13, R2, T, G1, Ba3, P22, P27, 
    \end{description}
 \end{marma}

 
 \begin{marma}[hp01_048]
 \sthana{1.48b}
   \begin{description}
\item[(gāḍhaṃ vakṣasi sannidhāya cibukaṃ ~)]
\item[dhyānaṃ tataś cetasi]    J2, V1
\item[dhyānaṃ tataś cepsitaṃ]    J15, C9
\item[dhyānaṃ tataś cośitaṃ]    V6
\item[dhyānaṃ tataś conmitaṃ]    P5, P24, 
\item[dhyānaṃ tataḥ conmī]    N1(some syllables seem deleted, 'nmī' is added later)
\item[dhyānaṃ ca tac cetasi] B1, B2, B3, Ba, Bo1, Bo2, C1, C2, C4, C6, C7, C8, J4, J5, J8, G11, J11, J12, J13, K1, Ko, L1, V19, N25, N26, P2, P3, P4, P6, P7, P12, P13, P15, P17, P26, V4(ṃ omitted), V17, V16, N3, N5, N8, N11, V15, V14, P8, P20, P25, R3, V13, N14, N16, N17,N19, V12, V3, N21, N23,  N24(saṃniyamya), R2, P9
\item[dhyāyaṃś ca tat cetasi] P14,
\item[dhyānaṃ ca tecesi]    V8
\item[dhyānena tac cetasi] P21, P23, 
\item[dhyānaṃ ca tac cepsitaṃ] A1, J1, J3, J6, J10, J14, J16, J17, LD1, P1, P10, N7, N9, N10, N12, N18, LD1, N20, N22, P27, 
\item[dhyānaṃ ca yac cepsitaṃ]    C3,
\item[dhyānaṃ ca tac cepsite]    J7,
\item[dhyānaṃ ca tatrepsitaṃ]    LD2, N2,
\item[dhyāyaṃś ca tac cetasi]    Ba5, Bo3, M1, Tue, N4a, N13, P18, V10, V22
\item[dhyāyaṃś ca taṃ cepsitaṃ]    R5
\item[(unavailable/illegible)]   C5, N6, T, G1, Ba3, P11, P16, P22, R4, 
    \end{description}

\sthana{1.48c}
  \begin{description}
\item[(vāraṃ vāram apānam ūrdhvam anilaṃ ~)] G11
\item[protsālayan pūrayan]    V1
\item[protsārayan pūrayan]    J7, V6, N7, N12
\item[protsārayaḥ pūrayem]    V8
\item[protsārayan pūritaṃ]    A1, Ba5, Bo3, C1, C6, C8, J10, P14, P15, P26, Tue, N4a, V14, R5, V13, N13, P18, V10, V22, R2, G11
\item[protsārayat pūritaṃ]    P7, P25, 
\item[protsyā?rayat pūritaṃ]    Ba,
\item[protsārayan pūrayet]    C4("proccālayan pūritaṃ" is also given above),L1, N25, P3
\item[protsārayet pūrayan]    P20, V19
\item[protsārayet pūrayet]    C7
\item[protsāraṇaṃ pūrayet]    N5, P21, P23, 
\item[protthāya yat pūrayan]    B3
\item[protthāpanṃ pūrayet]    Bo1,
\item[prodhāyana pūrayet]    B1,
\item[protthāpayan pūrayan]    C2, P2, P4, N1, R3, P9, P12, P17, 
\item[procārayet pūritaṃ]    J8
\item[prottāsayan pūritaṃ]    V12
\item[prollāsayan pūritaṃ]    J10, J17, N17, P1, P5, P24, 
\item[prollāśayet pūritaṃ]    J15, N9
\item[prolāsayet pūrayan]    J12,
\item[prollāsayet pūrayan]    N10
\item[prollosayana pūrayan]   LD2
\item[prollāsayet pūrayet]    J16,
\item[prollāsayan pūrayan]    C9, V16, P8, V4
\item[pro(llāpa?)yan pūrayan] LD1
\item[prollāsat pūritaṃ]    C3,
\item[prollāghayan pūrayan]    LD1
\item[procāraranu pūritaṃ]    J5
\item[proccārayan pūritaṃ]    M1, V17, N3(vālaṃ vāram)
\item[proccārayan pūrayan]    J4,
\item[proccārayat puraya yataṃ(?)]    J1,
\item[proccārayat pūritaṃ]    N2, P27, 
\item[proccārayet pūritaṃ]    V15, V3, N23
\item[proccārayet pūrayet]    N19
\item[proccālayan]    K1
\item[proccālayan pūrayen]    B2
\item[proccālayan pūrayet]    J13,
\item[proccālayaṃn pūrayan]    P6, N22(vvāraṃ cāram apānamūlapranilaṃ), \item[proccālayan pūrayan]    N18, N24(ayāmūrdh...)
\item[proccālayan pūritaṃ]    J2, J14, Ko, N26, N20(proccālayaṃn)
\item[proccālayet pūritaṃ]    Bo2, J3, J6("proccāṭayan pūrayet" is also given below), N16
\item[proccālitaṃ pūritaṃ]    N14(vālaṃ vālam)
\item[….... śayat pūritaṃ]    P10, 
\item[proccānalaṃ pūrayen]   P13, N21
\item[prodbhā(ṭa?)yan pūrayet]    N8
\item[prod(ghā?)ṭayan pūrayet]    N6, N11
\item[(unavailable/illegible)] T, G1, Ba3, P11, P16, P22, R4, 

\sthana{1.48d}
  \begin{description}
\item[(muñcan prāṇam upaiti bodham atulaṃ ~)]
\item[śaktiprabhāvān naraḥ] B1, B3, Ba5, Bo2, Bo3, C1, C2, C6, C8, J1, J3, J6, J8, J11, J13, J14, K1, Ko, M1, V1, P2, P4, Tue, N2, N3, V15, V14, R3, R5, V13, N13, N16, N19, P12, P13, P14, P15, P17, P18, P25, V10, V22, N23, P27, G11
\item[śaktiprabhāvān nara]    V3
\item[śaktiprabhāvā naraḥ]    V8, N22(mucan...)
\item[śaktiḥ prabhāvān naraḥ]    Ba,
\item[śaktiṃ prabhāvān naraḥ]    Bo1,
\item[śaktiṃ svabhāvān naraḥ]   N21
\item[saktiprabhāvān]    J5
\item[śaktiprabhāvon naraḥ]    N1, N7
\item[śaktiprabhāvīn naraḥ]    P6
\item[śaktiprabhāvaṃ naraḥ]    J12, N4a
\item[śaktiprabhān naraḥ]    B2
\item[śaktiprabhāvāṃtaraḥ]    N18
\item[śaktiprabhāvodayāt] C4, C7, L1, V19, N25, P20, V16, N5, N14, P21, P23, P26, 
\item[śaktiḥ prasādān naraḥ]    J2,
\item[śaktimabhāvodayāt]    P3 ("ma" could well be "pra)
\item[śaktiḥ prabhāvodayāt]    A1, R2("vo" illegible due to damage)
\item[śaktiprabodhodayāt]    N20
\item[śaktiprabhāvodayaḥ]    J7, P10, V6
\item[śaktiprabhāvād ataḥ]    C3, C9, J10, J16, P5, P7, N9, N10, N11, N12, P8, V12, LD1, N24, P9
\item[śaktīprabhāvad ataḥ]    LD2, P24, 
\item[śaktiprabhāvad ata]    J15, N8, V4
\item[śaktiprabhāvataḥ]    J17
\item[śaktiprabhāvadhataḥ]    V17
\item[śaktiprabodhodayāt]    N17, N26
\item[śaktiprabhāvodhayaḥ]    P1
\item[bhakti prabhāvodayaḥ]    J4,
\item[(unavailable/illegible)]   C5, N6, T, G1, Ba3, P11, P16, P22, R4, 

    \end{description}
 \end{marma}

\begin{marma}[hp01_049]
\sthana{1.49c}
  \begin{description}
\item[mārutaṃ nayate yas tu]    C6, C8, P11, 
\item[mārutaṃ niyato yas tu]    Ba, P25, 
\item[mārutaṃ niyatam yaskta]    V14
\item[mārutaṃ niyatam yas tu]    V8, N23(mār(ū?)taṃ)
\item[mārutaṃ mriyate yas tu] G11
\item[mārutaṃ nīyate yas tu]    C1,
\item[mārutaṃ pīyate yas tu]    V1
\item[mārutaṃ pīvyate yas tu]    J8ac, V3
\item[mārutaṃ yaḥ pīvet nitya]    V13
\item[mārutaṃ pibati yas tu]    J8pc
\item[mārutaṃ pibate yas tu]    J11,
\item[mārutaṃ yas tu pibati] A1, C9, J7, J10, J12, J15, J17, LD2, N17, V6, N26, P1, P5, P6, P7, V17, V16, N7, N9, N10, P8, N14, V4, LD1, N20, P24, R2, P9
\item[mārutaṃ yas tu pibatī] P21, 
\item[māruto bhidyate yasya]    N2, 
\item[māruto yas tu pibati]    N22
\item[mārutaṃ yaska pibati]    V12
\item[māruta yas tu pibati]     R1
\item[mārutaṃ yas tu pibaṃti]    C3,
\item[mārutaṃ yas tu pībaṃti]    J16,
\item[mārutaṃ yas tu vidhivat]    P26,
\item[mārutaṃ dhārayet yas tu] Ba5, Bo2, Bo3, C2, C7, J1, J3, J6, J13, L1, M1, V19, N25, P2, P3, P4, P14, P18, P20, P23, Tue, N4a, N11, N12, P12, P13, R5("), N13, N16, V10, N24,
\item[māruto dhārayed yas tu]    N1, P17, R3, V22
\item[māruta dhāraye?as tu]    N21
\item[mārutaṃ dhāraye yas tu]    N8, N18
\item[marutaṃ dhārayed yas tu]    C4, N5, V15
\item[mārutaṃ mārayed yas tu]    J5
\item[mārutaṃ mriyate yas tu]    N3
\item[mārutaṃ mriyate yasmāt]    Ko,
\item[māruto mriyate yas tu]    J4, P15, 
\item[māruto kṣiyate yasya] P27,
\item[māruto bhidyate yasya]    B1, B2, Bo1, J14, K1, N2, N19
\item[mahato dhārayed yastu]    B3
\item[pavanṃ prāsayeta saṃstu]    J2
\item[(unavailable/illegible)]    C5, N6, T, G1, Ba3, P10, P16, P22, R4, 
    \end{description}
 \end{marma}

\begin{marma}[hp01_051]
\sthana{1.51d}
  \begin{description}
\item[nāsāgre nyastalocanaḥ]    B1, B3, C2, C9, J7, J13, J16, N17, V1, V6, P2, P4, P17, N1, N7, P12, R3, N22
\item[nāsāgre nyastalocanā]    P6, R4, 
\item[nāsāgra-nyastalocanaḥ]    A1, Bo1, C3, J10, J17, N26, P1, P5, V16, N10, V12, V4, LD1, LD2, N20, P24, R2
\item[nāsāgraṃ nyastalocanaḥ]  P9
\item[nāsāgranyastalocanaṃ]    J8
\item[nāsāgra vyasalocanaḥ]    P26, 
\item[nāsāgranyastaloca....]    P10,
\item[nāsāgraṃ nyastalocanaṃ]    P7, V17, V3
\item[nāsāgra susamāhitaḥ]    C8,
\item[nāsāgraṃ susamāhitaḥ] Bo2, Bo3, C1, C4("nāsāgra nyastalocanaḥ" is also given), J1, J2, J3, J6, J11, J14, J15, K1, Ko, L1, M1, V19, N25, P3, Tue, N2, N4a, N5, N12, V15, V14, V13, N13, N16, P13, P14, P15, P18, P23, V10, V8, G11
\item[nāsāgrastu samāhitaḥ]   N23
\item[nāśāgre susamāhitaḥ]    N19,
\item[nāsāgre susamāhitaḥ]    P27,
\item[nāsyagraṃ tu samāhitaḥ]    J5,
\item[nāsāgre susamāhitaṃ]    P21, 
\item[nāsāgraṃ susamāhitaṃ]    Ba5, C6, V22
\item[nāsāgryaṃ susamāhitaṃ]    N18
\item[nāsāgraṃ susamāhita]    N8
\item[nāsāṃgraṃ susamāhitaḥ]    N11
\item[nāśāgraṃ tu samāhitaḥ]    N3,
\item[nāsāgraṃ tu samāhitaḥ]   N24, P20, 
\item[nāsāgre susamāhitaḥ]    B2, Ba, K1, N21, P25, 
\item[nāsāyaṃ susamāhitaḥ]    J4, P11,
\item[nāsāyāṃ susamāhitaḥ]    J4,
\item[nāsāṃ tu samāhitaḥ]    R5
\item[nāsāgre neṣu nocana]    N14
\item[(unavailable/illegible)]    C5, C7, N6, T, G1, Ba3, P16, P22,
    \end{description}
 \end{marma}


      
\begin{marma}[hp01_052]
\sthana{1.52d}
  \begin{description}
  \item[kurute cāsanottamam] B3, Ba, Ba5, Bo1, Bo2, Bo3, C1, C2, C4, C6, C8, J1, J3, J5, J6, J7, J8, J10pc, J13, J15, L1, N17, V6, N26, P2, P3, P4, P5, P7, Tue, N1, V17, N4a, N5, N8, N11, N12, P12,  P13, P14, P17, P18, P20, P21, P23, P24, V15, R3, R5, V13, N13, N14, N18, N19, V8, V3, N20, N21, N23, P25, R2, R4, G11
\item[kuruta cāsanottamam]    V22
\item[kurute cāsanomatam]    V10
\item[kurute cāsanottame]    B1, N16
\item[kurute cāsanottamaḥ]    J2,
\item[kurute cāsanotraye]    V14
\item[kurute cāsanottama]    A1, P26, 
\item[kurute cāsanontayam]   N24
\item[kurute vāsanottamam]    V1, V19, N3, P11, 
\item[kurute sādhanottamam]    C3, J12, J10ac, J16, J17, P1, V16, N9, N10, P9, P10, V12, V4,
\item[kurute sādhanottaramaṃ]    P8(perhaps 'ra' is attempted to be canceled)
\item[kurute sādhakottamam]    C9, N7,
\item[kurute sādhakottamaḥ]    LD1, LD2
\item[kurute cāsanam uttamaṃ]     N25
\item[sevate cāsanottamaiḥ]    J14,
\item[sevate cāsanottamaṃ]    J11, Ko, N2, P15,  
\item[sevyate cāsanottamaṃ]    J4,
\item[(unavailable/illegible)]    B2, C5, C7, K1, M1, N6, N22, P6, T, G1, Ba3, P16, P22, P27, 
    \end{description}
 \end{marma}


      
\begin{marma}[hp01_054]
\sthana{1.54b}
  \begin{description}
  \item[sarvavyādhi-viṣāpaham] C3, C9, J5, J7, J8, J10, J12, J15, J16, J17, LD1, V1pc, V6ac, P1, P5, P6, P7, P24, V17, N3, V16, N9, N10, P8, R1, V12, V3, V4, N22, P9, P11, G11
  \item[sarvavyādhī-viṣāpaham] LD2
  \item[sarvavyādhi-vināśanam] A1, B1, B2, B3, Ba, Ba5, Bo1, Bo2, Bo3, C1, C2, C4, C6, C7, C8, J1, J2, J3, J4, J6, J11, J13, K1, Ko, L1, N17, V1ac, V6pc, V19, N25, N26, P2, P3, P4, P17, P25, Tue, N1, N4a, N5, N7, N11, N12, P12, P13, P14, P15, P18, P20, V15, V14, R3, R5, N13, N14, N16, N18, N19, V10, V8, V22, N23, N24, P21, P23, P26, P27, R2, R4, 
\item[sarvavyāvināśanam]   N21
\item[sarvadhyādhi-vighātakaṃ]   N20
\item[sarvavyādhi-praṇāśanaṃ]    J14, N2, 
  \item[sarvavyādhivikṣāpaham] P10, 
\item[kaphavātāmayāpahaṃ]    M1
\item[(unavailable/illegible)]    C5, N6, N8, V13, T, G1, Ba3, P16, P22,
    \end{description}
 \end{marma}


      
\begin{marma}[hp01_055]
\sthana{1.55 additional verses}
 
kriyāyuktasya siddhiḥ syād akriyasya kathaṃ bhavet |
na śāstrapāṭhamātreṇa yogasiddhiḥ prajāyate ||
na veṣadhāraṇaṃ siddheḥ kāraṇam na ca tatkathā |
kriyaiva kāraṇaṃ siddheḥ satyam etan na saṃśayaḥ |
śiśnodararatāyeha na deyo veṣadhāriṇaḥ ||

\begin{description}
\item[Yes] B1, B2, B3, Ba, Bo1, Bo2, C1, C3, C4, C6, C8, C9, J1, J3, J6, J7, J8, J10, J11, J12, J13, J14, J15, J16, J17, K1, L1, LD1, LD2, N5, N17, N25, P10, V3, V17, V25, P8, R1, R3, N14, N16, N18, V12, LD1, N7, N9, N10, N13, N20, N21, N26, P1, P3, P4, P5, P7, P14, P17, P21, R2, P9, P12, P24, P25, P26, P27, 
\item[Yes], but the 4th line appears as the 2nd line: C2, J13, N1, P2
\item[No] A1, Ba5, Bo3, C1, C5, C7, J2, J4, J5, Ko, M1, V1, V6, V8, V19, R5, N19, P18, P20, V10, V22, V4 (unclear/unable to read) V13, N2, N3, N4a, N6, N8, N11, N12, N22, N23, N24, Tue, P6, P11, P13, P15, P16, P22, P23, R4, G11, T, G1, Ba3
    \end{description}

\sthana{1.55 third additional verse}
 
mayi bodhāmbudhau svacche tuccho 'yaṃ viśvabudbudaḥ |
pralīna udito veti vikalpapaṭalaḥ kutaḥ ||

 \begin{description}
 \item[Yes] B1, B3, Ba, Bo1, Bo2, C1, C2, C4, C7, C8, J1, J3, J6, J7, J13, L1, N5, N17, N25, P12, P17, P20, P21, P23, P26, V6, V19, R3, N14, N16, N18, N1, 

 \item[No] A1, B2, Ba5, Bo3, C3, C5, C6, C9, J2, J4, J8, J5, J10, J11, J12, J14, J15, J16, J17, K1, Ko, LD1, LD2, M1, V1, V3, V8, V17, V25, P8, R1, R5, N19, V12, V10, V22, V4, LD1, V13, N2, N3, N4a, N6, N7, N9, N10, N12, N20, N21, N22, N23, N24, N26, P11, P18, Tue, P1, P5, P6, P7, P9, P10, P13, P14, P15, P16, P24, P27, R2, R4, G11, T, G1, Ba3, P22, P25, 
    \end{description}

\sthana{1.55 fourth additional verse}
% P: What is the difference between this and the previous extra verse?
 
mayi bodhāmbudhau svacche tuccho 'yaṃ viśvabudbudaḥ |
pralīna udito veti vikalpapaṭalaḥ kutaḥ ||

\begin{description}
\item[Yes] B1, B3, Ba, Bo2, C1, C2, C4, C7, C8, J1, J3, J6, J7, J13, L1, N5, N17, N25, V6, V19, R3, N14, N16, N18, N8, N11, P2, P3, P4, P12, P21, P23, 
\item[No] A1, B2, Bo3, C3, C5, C6, C9, J2, J4, J8, J5, J10, J11, J12, J14, J15, J16, J17, K1, Ko, M1, V1, V3, V8, V17, V25, P8, R1, R5, N13, N19, V12, V10, V22, V4, LD1, V13, N2, N3, N4a, N6, N7, N9, N10, N12, N20, N21, N22, N23, N24, N26, Tue, P1, P5, P6, P7, R2, P9, P10, P11, P13, P14, P15, P16, P17, P18, P20, G11, T, G1, Ba3
    \end{description}


\begin{marma}[hp01_056]
\sthana{1.56a}
  \begin{description}
  \item[āsanaṃ kumbhakaś citro] J8pc, N24
\item[pīṭhāni kuṃbhakāś citro] G11
\item[pīṭhādi kuṃbhakāś citram] R4, 
  \item[āsanaṃ kumbhakaś citraṃ] B2, C3, J7, J8ac, J10, J12, J15, N17, P8, N14(ā(?)naṃ), V12, LD1, LD2, N4a, N10, N12, N20, P1, P5, P24, P27, 
  \item[āsanaṃ kumbhavaś citra] J2,
\item[āsanaṃ kumbhavaś ci(nt?)aṃ] P9
  \item[āsana kumbhakaś citraṃ] V3
\item[āsanaṃ kumbhakāś citrāḥ]  N8, N11
  \item[āsanaṃ kumbhavaś cittaṃ] P10, 
\item[āsanaṃ kumbhakāś citrā]   P7
\item[āsanaṃ kumbhakācitraṃ]   Tue
  \item[āsanaṃ kumbhakaṃś citraṃ] C9, J17
  \item[āsanaṃ kumbhakaṃ citraṃ] A1, B1, B3, Ba, Ba5, Bo1, Bo2, Bo3, C1, C2, C4, C6, C7, C8, J1, J3, J4, J5, J6, J11, J13, J14, K1, Ko, L1, M1, P11, P12, P13, P18, V19, R3, R5, N13, N16, N19, P14, P17, P20, P21, P23, P25, V10, V22, V4, N1, N5, N23, N25, N26, P2, P3, P4, P6, P26, R2
  \item[āsanaṃ kumbhakaṃ citra] P15,
\item[āsanaṃ kumbhakaṃ śvitraṃ]   N21   
  \item[āsanaṃ kumbhaka citrāṃ] J16, N7,
  \item[āsanaṃ kumbhaka citaṃ] N18
\item[āsanaṃ kubhakaṃ citraṃ] N2, 
\item[asanaṃ kumbhakaṃ citraṃ] N3, N9
  \item[(unclear/unable to read)] V13
\item[(unavailable/illegible)]    C5, N6, N22, R1, T, G1, Ba3, P16, P22,
    \end{description}

\sthana{1.56b}
  \begin{description}
  \item[mudrādikaraṇāni ca] A1, B2, C3, C9, J7, J8pc, J10, J12, J15, J16, J17, LD1, N17, P8, N14, V12, N7, N9, N10, P1, P5, P7, P9, P10, P24, R2, R4, 
  \item[mudrānikaraṇāni ca] LD2
\item[mudrādikāraṇāni ca]    V4
\item[vicitrīkaraṇāni ca]    J2,
  \item[mudrādikaraṇakriyā] P27,
\item[mudrādi pavanakriyā]    B1, B3, Bo1, C2, J5, J8ac, J13, R3, V3, N1, N20, P2, P4, P12, P17, 
\item[mudrādi pavanakriyāḥ]   N26
\item[mudrākhyaṃ karaṇaṃ tathā] Ba, Ba5, Bo2, Bo3, C1, C3, C6, C7,  C8, J1, J3, J11, L1, M1, P11, P14, P15, P18, P20, P23, P25, P26, V19, R5, N13, N16, N18, N19, V10, V22, N5, N11, N12, N21, N23, N25, Tue, P3, P6, P21, 
\item[mudrākhaṃ karaṇaṃ tathā]   N24
\item[mudrākhyaṃ karaṇaṃ bhavet]   N3
\item[mudrākhyakaraṇaṃ tathā]    Ko, N4a, N8, P13, 
\item[mudrādi karaṇaṃ tathā]    K1,
\item[mudrākṣaṃ karaṇaṃ tathā]    J4, J6,
\item[mudrakhecaraṇīṃ tathā]    J14(given at 55)
\item[mudrakhecariṇīṃ tathā]    N2, 
\item[(unclear/unable to read)]    V13
\item[(unavailable)]     C5, N6, N22, R1, G11, T, G1, Ba3, P16, P22,
    \end{description}
 \end{marma}


      
\begin{marma}[hp01_060]
\sthana{1.60cd}
  \begin{description}
\item[(atilavaṇaṃ tilapinḍaṃ kadaśanaśākotkaṭaṃ varjyam |)]
\item[atilavaṇaṃ tilapinḍaṃ]    J17, LD2, N10, R1
\item[atilavaṇaṃ tilapinḍa]    C3, J8pc, J10, J15, J16, LD1, N17, V4, N9, N20, P1, P5, P8, P9
\item[atilavaṇaṃ tilapiṇḍā?]   N14
\item[atilavaṇatilapiṇḍakadaśanaṃ]  P10, P24, 
\item[atilavaṇaṃ tilaṃ pinḍa]    J8ac
\item[atilavaṇaṃ tilaṃ pinḍaṃ]     V3
\item[atilavaṇaṃ tilapiṇyākadarśanam]    N26
\item[atilavaṇam amlayuktaṃ]    A1, B3, Ba, Ba5, Bo3, V22, V10, N1, N4a, N11, N13, Tue, P2, P12, P14, P17, P18, P25, P26, R2, R5
\item[atilavaṇama malayuktaṃ]    C2,
\item[atilavaṇāmalpayuktaṃ]    P13, 
\item[atilavaṇamalayuktaṃ]   P4, R3
\item[atilavaṇāmlayuktaṃ]   P7
\item[atilavaṇa amlayuktaṃ]    C6, C8, V13
\item[atilavaṇa amlayutaṃ]    J11,
\item[atilavatilapi]    V6
\item[atilavaṇatiktapiṇḍa]    B2
\item[atilavaṇatilapiṇḍakaṃ]    C9
\item[atilavaṇaṃ dyuṣṇaṃ]    V1
\item[atilavaṇam apramuṣṇaṃ]    M1
\item[atilavaṇam ca apramitaṃ]    V8
\item[atilavaṇam cāpramitaṃ]   P6, P27, 
\item[atilavaṇaṃ ca tilapiḍaṃ]   N7,
\item[atilavaṇam pramitaṃ]    J12, K1,
\item[atilavaṇādi prasaktaṃ]    J14,
\item[atilavaṇādi proktaṃ]   N22
\item[atilavaṇacāmlapiktaṃ]    N19
\item[atilavaṇakavalṃ cāpamitaṃ]    J1,
\item[atilavaṇasavapalala]    V19
\item[atilavaṇasavapalalaṃ]   N25
\item[atilavaṇāsavasalilaṃ]   P21, 
\item[atilavaṇaṃ sapalalaṃ]   P20,
\item[atilavaṇaṃ sala(pa?)laṃ]   N8
\item[atilavaṇaṃ palilaṃ]     R4
\item[atilavaṇāsavapalalaṃ]  C1, C7, L1, N5
\item[atilavaṇāsavapalaṃla]  P23, 
\item[atilavaṇātiproktaṃ]    J2,
\item[atilavaṇādiyutaṃ]    J4,
\item[atilavaṇādi prayuktaṃ] P15,
\item[atilavaṇādi puttaṃkdaśanaṃ]    P11,
\item[atilavaṇādyaptataptaṃ]    J5,
\item[atilavaṇādyapran(t?)aṃ]  N3
\item[atilavaṇādyapratapta]   N12
\item[atilavanā?āna(va?)lala]   N24
\item[atilavaṇamatprayuktaṃ]   N21
\item[atilavaṇāmavapalalaṃ]    C4 ("atilavaṇaṃ tilapinḍaṃ" is also given),
\item[atilavaṇāya priyatavṃ]    Ko
\item[atilavaṇaṃ sapalaṃ ca]    Bo2, J3, J6,
\item[atilavaṇasalilaṃ kadaṣana]    B1, Bo1, J13,
\item[atilavaṇetilapiṃḍakvadaśanaṃ]    J7
\item[atilavaṇaprasūtaṃ]     N2, 
\item[atilavaṇaṃ kavalaṃ cāpramitaṃ piṇḍa]   N18
\item[atīlavaṇāmlapṛktaṃ]   N23
\item[(illegible/unavailable)]  C5, N6, N16, P3, G11, T, G1, Ba3, P16, P22,

* * *

\item[śākotkaṭaṃ] A1, B2, B3, Ba5, Bo1, Bo3, C1, C2, C3, C4, C6, C8, C9, J2, J3, J4, J6, J10, J11, J12, J13, J14, J15, J17, K1, Ko, L1, LD1, LD2, V6, V19, V22, V4, V10, V13, N2, N3, N4a, N7, N11, N13, N18, N19, N21, N24, N25, Tue, P2, P5, P6, P7, P8, P9, P10, P11, P12, P13, P14, P15, P17, P18, P20, P21, P23, P24, P26, R1, R2, R3, R4, R5,
\item[śākotkaṭhaṃ]    V8
\item[śākotkavaṃ]    J5,
\item[śākotkaṭa]    Bo2, J8pc, J16, N1, N9, N26, P27,
\item[śākoktaṃ]    P25, 
\item[śākotkarṭeṃ]  P4
\item[śākātkaṭa]    V3
\item[śākotchadaṃ]    Ba,
\item[śākolkaṭaṃ]   P1
\item[śākokṣadaṃ]   N23
\item[śāṃkotkaṭaṃ]   N10
\item[śāketkaṭaṃ]    C7
\item[śokātkaṭa]    J8ac, V1
\item[śokotkaṭaṃ]    B1, J7,
\item[śāśotka]    M1
\item[śātkṭaṃ]    J1,
\item[śa?tāsanaḥ]   N14
\item[sākotkaṭa]   N20
\item[sākotkaṭaṃ]  N22
\item[syacotkaṭaṃ] N12
\item[(illegible/unavailable)]    C5, N6, N17, N8, N16, P3, G11, T, G1, Ba3, P16, P22,
\item[kotkaṭaṃ]  N5

* * *
\item[varjyaṃ] A1, B1, B2, B3, Ba5, Bo1, Bo3, C1, C2, C4, C7, C9, J3, J6, J13, J15, K1, L1, LD2, V3, V6, V19, V22, N1, N4a, N5, N11, N12(va?jyaṃ), N13, N22, N25, Tue, P2, P4, P6, P7, P12, P14, P17, P18, P20, P21, P23, P27, R3, R4, R5
\item[varjya] V8, V10
\item[varjyāṃ] J16
\item[varj(j)aṃ] J7, J10, J12, J17, M1, N9, P10, P26, R1
\item[varjjyaṃ]  P5, P24, 
\item[varj(j)itaṃ] J1, Bo2, J8, LD1, V3, N18, N20, N26
\item[varyyaṃ]   N7, R2, P9
\item[varjjāṃ]   N19
\item[varjet]  N10
\item[varjayet] C3, P1("varjaṃ" earlier)
\item[vayīṃ]   P8
\item[duṣṭaṃ] C6, C8, J4, J5, J11, J14, Ko, V13, N2, N21, N23, N24, P13, P15, P25, 
\item[du(ṣl?)aṃ]  N3
\item[dṛṣṭaḥ]
\item[durjvaṃ(?)] Ba,
\item[(illegible/unavailable)] C5, N6, N17, V1, V4, N8, N14, N16, P3, G11, T, G1, Ba3, P11, P16, P22,
    \end{description}
 \end{marma}


      
\begin{marma}[hp01_062]
\sthana{1.62d} 
  \begin{description}
  \item[mudgādi divyam udakaṃ ca] A1, B2(dīvyam), Ba, Ba5, Bo3, C6("dgā" is missing), C8, J5, J7, J10, J11, J12, J16, J17, K1, L1pc, LD1, LD2, N17, V1, V6, V8, V22, V4, V13, N1, N10, N12, N13, N14, N20, N22, N23, N24, N26, Tue, P1, P6, P8, P9, P14, P17, P18, P23, P25, P26, P27, R1, R2,
\item[mudgādi divyam udakaṃ]   N18, P11,
\item[mudgādi divyam udakaṃ ka]   N4a
\item[mudgādi divyam udake ca]   R5
\item[mudgādi divyam udakaṃ yamīndra]    V3, N2
\item[mudgādi divyam udakaṃ hi]    C3, J14, J15, N9, P5, P10, P15, P24, 
\item[mudgādi pivyam udakaṃ ca]    J2,
\item[mudgājya divyam udakaṃ ca]    C9
\item[mudgādivyam udakaṃ ca]    C4, J8, M1, V10, N3
\item[mudgādi divyagaditaṃ ca]   N7,
\item[mudgādi divyamuktaṃ]    Ko,
\item[mudgādi cālpam udakaṃ ca]    B1, B3, Bo1, Bo2, C1, C2, C7, V19, N5, N11, N21, P2, P4, P12, P13, P20, P21, R3, R4
\item[mudgādi cālyam udakaṃ ca]    L1ac, N8(no "ca")
\item[mudgādi cājyam udakaṃ ca]    N16
\item[mudgādi cālpa sudakaṃ ca]     J1,
\item[mudgādi dālpam udakaṃ ca]    J3, J6("cālpa" is also given)] J13,
\item[mudgādi dālpamudgaṃ ca]    N25
\item[mudrādi divyam udakaṃ ca]  P7
\item[mudrādi divyam udakaṃ ca mudakaṃ]    N19
\item[(illegible/unavailable)]       C5, J4, N6, P3, G11, T, G1, Ba3, P16, P22,
    \end{description}

\sthana{1.62d}
  \begin{description}
  \item[munīndrapathyaṃ (or: munindra-)] B1, B2, B3, Bo1, C1, C2, C4, C7, J3, J5, J6, J13, J16, J17, K1, L1, LD2, V1, V19, N1, N5, N11, N19, N21, N22, P2, P4, P12, P13, P17, P20, P21, P23, P26, R4
\item[munīdrapathyaṃ]   N8, R3
\item[munīndrapathaṃ]   N25
\item[mahendrapathyaṃ]  N9, P5, P10, P24, 
  \item[yamīndrapathyaṃ (or: yamindra-)] A1, Ba, Ba5, Bo3, C6, C8, C9, J7, J8, J10, J11, J14, LD1, M1, N17, V6, V22, V3, V4 (pathyaṃ missing), V10, V13, N2, N3, N4a, N12, N13, N14, N20, N26, Tue, P1, P6, P7, P8, P14, P18, P25, P27, R2, R5, P9
\item[yamīṃdrapraparyya]   N24
\item[yamīḍyapathyaṃ] P15,
\item[yatīṃdrapathyaṃ]    J2, N10, N23
\item[yavatīṃdrapathyaṃ]    P11,
\item[yamendrapathyaṃ]    C3,
\item[yogīdrapathya]    J1,     
\item[mūnidra pathayam]    V8
\item[munidra pathayam]    J12,
\item[munīdramukhyaiḥ]    N7,
\item[muṃhanīdrapathya]    J15
\item[suyogipathyaṃ]     N16
\item[varayogipathyaṃ]    N18
\item[jamodūpathyaṃ]    R1
\item[paramaṃ hitaṃ hi]    Ko,
\item[(illegible/unavailable)]   C5, J4, N6, P3, G11, T, G1, Ba3, P16, P22,
    \end{description}
 \end{marma}


      
\begin{marma}[hp01_063]
\sthana{1.63a}
  \begin{description}
\item[iṣṭaṃ sumadhuraṃ]    A1, J3, J6, J10, N18(appears after one verse), P7, R2, P9
\item[iṣṭaṃ sumadhura]    P11,
\item[iṣṭaṃ samadhuraṃ]    C9, J7, J17, LD2, N17, V19, V4, LD1, P1, P8, R1
\item[iṣṭaṃ saṃmadhuraṃ]   N7,
\item[iṣṭaṃ samakaraṃ]   P10, 
\item[daṣṭaṃ samudhūraṃ]    J16,
\item[dṛṣṭaṃ samūdhuraṃ]    K1
\item[dṛṣṭaṃ sumūdhuraṃ]    N10
\item[piṣṭaṃ sumadhuraṃ]    C8, N8, N11, N14("śu")
\item[pathyaṃ sumadhuraṃ]    J12,
\item[puṣṭaṃ sumadhuraṃ]    Ba5, Bo3, N4a, N13, P14, P18, Tue
\item[puṣṭaṃ samadhuraṃ]   R5
\item[yeṣṭaṃ sumadhura]    V13
\item[yuṣṭaṃ sumadhuraṃ]    V10(mu°), V22
\item[mṛṣṭaṃ sumadhuraṃ]    M1, V1
\item[mṛṣṭaṃ sumadhurāṃ]    Ko
\item[mṛṣṭaṃ sumadhusnigdhaṃ]   N3
\item[mṛṣṭaṃ samadhuraṃ]    C3,
\item[miṣṭaṃ samadhuraṃ]    J8, Ba, C4, N20, N26, P5, P24, 
\item[miṣṭaṃ sumadhuraṃ]    Bo2, C6, J1, J11, J14, N2, N16, N19, N21, N22, N23, P13, P15, P21, P25, P27, R4
\item[miṣṭaṃ samudhuraṃ]    J5, J15,
\item[miṣṭaṃ sūmadhuraṃ]  P6
\item[miṣṭaṃ sumadhura]   N25
\item[miṣṭaṃ madhuraṃ]    V3
\item[miṣṭaṃ ca madhuraṃ]    B2
\item[maṃ(ṣṭ?)aṃ samudhuraṃ]  N9
\item[sumiṣṭaṃ madhuraṃ]    N12
\item[uṣṇaṃ sumadhuraṃ]    B1, Bo1, C2, C7, L1, J13, N1, P2, P17, P23, 
\item[uṣṇaṃ sumudhuraṃ]    P20,
\item[uṣṇaṃ sumadhura]    P12, 
\item[uṣṇaṃ samadhuraṃ]    C1,
\item[uṣma(?) sumadhuraṃ]    B3
\item[uṣmaṃ sumadhuraṃ]    P4, R3(appears after one verse)
\item[uśmā?mumadhuraṃ]  N5
\item[suvṛkṣaṃ madhuraṃ]    V6
\item[ghṛtam ca madhuraṃ]    V8
\item[vṛṣyaṃ sumadhuraṃ]    J2,
\item[tiṣṭaṃ tu madhuraṃ]  N24
\item[(illegible/unavailable)]   C5, J4, N6, P3, G11, T, G1, Ba3, P16, P22, P26,
    \end{description}
 \end{marma}

Notes:
    • P3, R1, has not 2nd chapter
    • Folios of P4 are scattered 
    • P16 has only 3 folios of chapter 2 at the end
    • P22 has only 3 folios of chapter 1
    • R3 has only 1 folio of chapter 2
    • 1 & 2 Chapters are mixed in R4
    • R5 has only one folio of chapter 2
    • C5 has not 1st and 2nd chapters
    • U1 has only chapter 2
\begin{marma}[hp02_003]
\sthana{2.3b}
\item[jīvanam] Ba, Ba5, Bo2, Bo3, C1, C6, J1, J2, J3, J6, J12, J13, J14, J16, N17, P1, V1, N4a, N7, N8, N9, N10, N11, N12, N13, N16, P10, P13, P14, V16, V17, V10, N18, N21, P18, P20, P25, P26, P27, R5, V21, Tue, N23, V6, V7, V20, V28, V18
\item[jīvaṃnam] N14
\item[jīvitam] A1, B1, B2, B3, Bo1, C2, C3, C4, C8, J4, J5, J7, J8, J10, J11, LD1, LD2, N1, N2, N5, V8, N20, N22, P2, P6, P9, P17, P21, P23, V14, V3, N24, N25, P12, P15, R1, R3, V15, V5, V22, V2, V12, V13, V9, G11, Ba3, G1, U1, 
\item[jīvatam] P5, P24,
\item[jivitam] J15,
\item[jīvita] N26
\item[jīviti] P11, 
\item[jīvitum] J17, N19
\item[jītavim] N3
\item[jīvo na] Ko, P7, P8, 
\item[(unavailable/illegible)] C5, C9, N6, T, P16, P22,

  \begin{description}

    \end{description}
 \end{marma}

 \begin{marma}[hp02_003]
\sthana{2.3d}
\item[nibandhayet] Ba, Bo2, C6, V1, J2, J4, J6, J8, J11, J14, N12, N16, N19, N22, P11, P15, P25, V14, V3, V13
\item[nibaddhayet] G11, P6, P27,
\item[nirundhayet] A1, B1, B2, B3, C1, C3, C9, J5, J10, J13, J15, J17, LD1, LD2, N17, P1, V19, W4, N2, N7, N9, N10, N14, P5, P8, P9, P10, P23, V16, V17, N20, V21, N23, N26, P2, P7, P26, V20, V5, V2, V18, V12, V9, Ba3
\item[nirodhayet] Ba5, Bo1, Bo3, C2, C4, C8, J1, J3, J6pc, J7, V19, N1, N4a, N5, N8, R3, V8, N11, N13, V10, N18, N21, P13, P14, P17, P18, P20, Tue, N24, N25, P12, P21, R5, V6, V7, V15, V28, V22, U2,
\item[nirudhayet] P24,
\item[niruddhayet] R1, 
\item[nisaṃdhayet] J16,
\item[nirodhanaṃ] J12,
\item[nibandhanāt] V19, N3, G1
\item[nibadhnayāt] Ko,
\item[(unavailable/illegible)] C5, N6, T, P16, P22,
  \begin{description}

    \end{description}
 \end{marma}



 \begin{marma}[hp02_004]
\sthana{2.4d}
\item[kāyasiddhiḥ] A1, B1, B2, C2, C6, J5, J8, J14, V1, N1, N8, N11, P2, P12, P17, P26, R3, V17, N24, V20, V15, V9, G11, G1
\item[kāyāsiddhiḥ] J2, N3,
\item[kāyāsuddhi] V6
\item[kāyāsiddhi] P6, P27, 
\item[kayasiddhiḥ] N20
\item[kāyaśuddhiḥ] N23
\item[kāryasiddhiḥ] B3, Ba, Ba5, Bo1, Bo2, Bo3, C1, C3, C4, C8, C9, J3, J4, J6, J10, J11, J13, J17, Ko, LD1, N17, P10, V1, W4, N2, N4a, N5, N9, N12, N13, V16, V10, N19, N21, P1, P5, P6, P7, P8, P9, P11, P13, P14, P15, P18, P20, P21, P23, P24, P25, R1, Tue, N25, N26, V70, V5, V28, V22, V2, V18, V12, V13, Ba3
\item[kāryasiddhaḥ] R5,
\item[kāryasiddhiṃ] J15, 
\item[kāryāsiddhiḥ] J1,
\item[kāryasiddhi] J12, LD2, N7, V8, N10, N14("rya" unclear), N16, N18, N22, V21, V3 
\item[kāryasiddhī] J16,
\item[kāryasuddhiḥ] V14
\item[kāyāśuddhiḥ] J7,
\item[(unavailable/illegible)] C5, N6, T, P16, P22,

  \begin{description}

    \end{description}
 \end{marma}


 \begin{marma}[hp02_006]
\sthana{2.6c}
\item[suṣumnāsu snighdhā] J4, V1, V15
\item[suṣumnāḥ snighdhā] P15, 
\item[suṣumnānāḍisthā] Ba5, N13, P13, P14, P20, Tue, 
\item[suṣumṇāṃ nāḍisthāḥ] P21, 
\item[suṣumṇānāḍīsthā] N4a, N21, P18,  
\item[suṣumṇānāmadhysthā] V2
\item[suṣumnā nāḍīsthaṃ] G1
\item[suṣumṇānāḍīstho] N24
\item[suṣumnānāḍīstho] J14,
\item[suṣumṇānāḍīschā] Bo3, V28, V22
\item[suṣumṇāṃtasthhā] J15,
\item[suṣumnāṃtaraschā] J11
\item[sukhumnāṃtarasthā] V19
\item[suṣumnāṃtarālasthā] B3, Bo1, Bo2, C1, C2, J3, J6, J13, N1, N8, N11, P2, P17, P23, 
\item[suṣumnāṃtarālaschā] P12, 
\item[suṣumnāṃtarālasyā] J1,
\item[suṣumnāṃtarājasthā] N16
\item[suṣumṇāṃtarasthā]  N5, C4
\item[suṣumṇāṃtarāla] V5
\item[suṣumnā pāśvaṃsthā] A1, V20
\item[yathā suṣumṇā pārśvaḥsthā] Ba,
\item[yathā suṣumṇā pārśvasthā] C6, C8, J7, N23, P7, P25, P26, 
\item[suṣumṇāpañcasthā] N19
\item[yathā suṣumnā mukhasthā] V17
\item[yathā suṣumnā sumalāḥ] P27,
\item[yathā suṣumnā nāḍisthamala] V10
\item[yathā suṣumnņayā śvaḥsthā] V8, V21
\item[yathā suṣumņā sustakcya?] P11, 
\item[suṣumņā sustabdhāḥ] N12
\item[yathā suṣumnā susvasthā] J5, N22, N26, V14, V3, N3
\item[suṣumņā suṣasthā] N20
\item[suṣumnņayā ścasyā] V6, V13
\item[suṣumnāvasthāyaṃ] B1, B2
\item[suṣumnānāḍibhyāṃ] V7
\item[suśumnāsusvasthā] G11
\item[suṣumnāsvasthāyan] N2
\item[suṣama susvasthā] J8
\item[suṣumnāsui svasthā] J8ac,
\item[suṣumnāsaṃsvasthā] J2,
\item[suṣumṇā sūmalā] P6,
\item[sukham avasthāya] C3, C9, J10, J12, J17, LD2, N17, P1, P5, P24, W4, N7, N9, V16, V18, V12, V9, Ba3
\item[sukhamalāsyāya] P9, 
\item[sukham avasthāyaḥ] J16,
\item[sukham avasthāyā] N14, P8, 
\item[sukham avasthāpyā] LD1
\item[sukham avaschāya] R1,
\item[sukham avaschāṃya] N10
\item[sukha........] P10,
\item[suṣam avasthāya] J8pc, N18,
\item[suṣumṇām avasthāpya] J7pc
\item[suṣumnāṃtvarayāpalaḥ] N25
\item[(unavailable/illegible)] C5, Ko, N6, T, P16, P22, R5, 

  \begin{description}

    \end{description}
 \end{marma}


\begin{marma}[hp02_009]
\sthana{2.09 additional verses}
dvādaśamātraḥ kanīyānmadhyamo mātracaturvviśaḥ |
uttamaḥ ṣaṭtriṃśanmātro mātrābhedāḥ smṛtā tajñaiḥ |
vaśiṣṭasaṃhitādau tu pūrakaṃ ṣoḍaśamātrabhiḥ |
kuṃbhakaṃ ca catuṣṣaṣṭimātrābhiḥ recakaṃ tu |
dvātriṃśanmātrābhiḥ vidhivatkuṃbhakaṃ kṛtvā punaścaṃdreṇa recayet ||
\begin{description}
\item[Yes] J17, R1, 
\item[No] A1, B1, B2, B3, Ba, Ba5, Bo1 Bo2, C1, C2, C3, C4, C5, C6, C8, C9, J1, J2, J3, J5, J10, J11, J12, J14, J15, J16, LD2, N17, V1, J4, J6, K7. J8, J13, Ko, LD1, W4, V19, N1, N2, N3, N4a, N5, N6, N7, N8, N9, V28, V8, N10, N11, N12, N13, N14, N16, P5, P10, V16, V17, V10, N18, N19, N20, N21, N22, P1, P6, P7, P8, P9, P11, P12, P13, P14, P15, P16, P17, P18, P20, P21, P22, P23, P24, P25, P26, P27, V14, V21, V3, Tue, N23, N24, N25, N26, P2, R5, V6, V7, V20, V15, V5, V22, V2, V18, V12, V13, V9, G11, T, Ba3, G1
    \end{description}

\begin{marma}[hp02_011]
\sthana{2.11 additional verses}
iḍayā piva ṣoḍaśabhiḥ pavanaṃ |
kuru ṣaṣṭicatuṣṭayamaṃtagataṃ |
tyaja piṃgalayā sanakaiḥ sanakaiḥ |
daśabhirdaśabhirdaśabhirdvyadhikaiḥ ||
\begin{description}
\item[Yes] B1, B3, C, V19, N1, N5, N10 , N12, N21, N25, P12, P13, P26, P17, P20, P21, V28
\item[No] A1, B2, Ba, Ba5, Bo1, Bo2, C1, C2, C3, C4, C5, C6, C8, C9, J1, J2, J3, J10, J13, J17, J4, J5, J6, J7, J11, J12, J14, J15, J16, Ko, LD1, LD2, N17, P1, V1, J8, W4, N2, N3, N4a, N6, N7, N8, N9, V8, N11, N13, N14, N16, P5, P6, P8, P15, P18, V16, V17, V10, N18, N19, N20, N22, V14, V21, V3, Tue, N23, N24, N26, P2, P7, P9, P10, P11, P14, P16, P22, P23, P24, P25, P27, R1, R5, V6, V7, V20, V15, V5, V22, V2, V18, V12, V13, V9, G11, T, Ba3, G1
    \end{description}


 \begin{marma}[hp02_012]
\sthana{2.12cd}
\item[uttānaṃ cottame prāṇarodhe padmāsane muhuḥ] Ba, C1, C6, P25, V14
\item[uttānaṃ cottame prāṇe rodhe padmāsane muhuḥ] P7, 
\item[uttānaṃ cottame prāṇe stebdhe? padmāsane muhuḥ] C8,
\item[uttame sthānam āpnoti tato vāyuṃ nibaṃdhayet] N13, V10, Tue, V7
\item[utiṣṭhe tūtame prāṇā yāvat padmāsane sūmaṃ] J2,
\item[uttiṣṭhaty uttame prāṇaṃ rodheḥ padmāsanaṃ .uhuḥ] V1
\item[uttiṣṭhāṃty uttame prāṃṇā baddhe padmāsane uttama] J16
\item[uttiṣṭhaty uttame prāṇē baddhe padmāsanae dṛḍhe] C2, P17, 
\item[uttiṣṭhaty uttame prāṇaḥ baddhe padmāsanae dṛḍhe] J13,
\item[uttiṣṭhaty uttame prāṇaṃ baddhā padmāsane dṛḍhe] B1
\item[uttiṣṭhaty uttame prāṇo baddhe padmāsane dṛḍhe] B2, B3, P2,
\item[uttiṣṭhaty uttame prāṇa baddhe padmāsanaṃ dṛḍhaṃ] Ba3
\item[uttiṣṭhaty uttame prāṇo baddhe padmāsane dṛḍhe] P13, 
\item[uttiṣṭhaty uttame prāṇaḥ badhe padmāsane dṛḍhe] P27,
\item[uttiṣṭhaty uttame prāṇo baddhe padmāsane sadṛḍhe] P12,
\item[uttiṣṭhaṃty uttame prāṇā baddhe padmāsane muhuḥ] LD2, J10, P1, P8, P9, P10, P24, 
\item[uttiṣṭhaṃty ūttame prāṇā baddhe padmāsane muhuḥ] J15, P5, R1, 
\item[uttiṣṭhaṃty uttame prāṇā baddha padmāsane muhuḥ] C3, N10
\item[uttisthaṃ cottame prāṇā baddha padmāsane muhu] N14
\item[uttiṣṭhaty uttame prāṇa baddhe padmāsanaṃ muhuḥ] J8ac, J14, V3
\item[uttiṣṭhaty uttame prāṇa baddho padmāsane muhuḥ] V8
\item[uttiṣṭhatm uttame prāṇa baddhe padmāsane muhuḥ] V16, V17, V21, V20, V18, V13
\item[uttiṣṭhaty uttame prāṇo baddhe padmāsane muhuḥ] C9, 
\item[uttiṣṭhaty uttame prāṇo baddha padmāsane muhuḥ] A1
\item[uttiṣṭhaty uttame prāṇo badhye padmāsane muhuḥ] N9
\item[uttiṣṭhaty uttame prāṇo baddhapadmāsane muhuḥ] N7
\item[uttiṣṭhaty uttame prāṇē baddhe padmāsanaṃ muhuḥ] J8pc
\item[uttiṣṭhaty uttame prāṇē baddhapadmāsane muhuḥ] P26, 
\item[uttiṣṭhaty uttame prāṇē baddha padmāsane haye] Bo1
\item[uttiṣṭhaṃty uttare prāṇā baddhe padmāsane muhuḥ] J17
\item[uttiṣṭhaṃty uttame prāṇā baddhe padmāsane muhuḥ] W4
\item[uttiṣṭhaty uttame prāṇā baddhe padmāsane muhuḥ] N20
\item[uttiṣṭhaty uttame prāṇo baddhe padmāsane mukuḥ] N26
\item[uttiṣṭhaty uttame prāṇa rodhe padmāsana sthitaḥ] Bo2, J1, J3, N8, N11, N16, V2
\item[uttiṣṭhaty uttame prāṇa rodhe padmāsana sthite] C4, N25
\item[uttiṣṭhaty uttame prāṇa rodhe padmāsane sthite] P21, P23, 
\item[uttiṣṭhaty uttame prāṇāṃ rodhe padmāsane sthite] V19
\item[uttiṣṭhaty uttame prāṇo rodhe padmāsane sthite] P20, V5, V28
\item[uttiṣṭhaty uttame prāṇa rodhe padmāsane sthitaḥ] J6, N2
\item[uttiṣṭaty uttame prāṇa rodhe padmāsanasthite]  N5, P6, 
\item[uttiṣṭhaty uttame prāṇa rodhe padmāsanaṃ muhuḥ] N3, , P15, V15
\item[uttiṣṭhaty uttāme prāṇarodhe padmāsanaṃ muhuḥ] J5, J11, G1
\item[uttiṣṭhaty uttāme prāṇarodhe padmāsano muhuḥ] J7
\item[tattiṣṭhaty uttāmaprāṇarodhya padmāsanaṃ muhuḥ] J4
\item[uttiṣ(?)yaty uttame prāṇa rodho padmāsanaṃ muhuḥ] N24
\item[uttiṣṭhaty uttame prāṇaṃ rodhe padmāsanaṃ muhuḥ] P11, 
\item[uttiṣṭhaty uttame kārye rodhe padmāsanae muhu] Ko,
\item[uttiṣṭhaṃty uttame prāṇa rodho padmāsane sthitaḥ] N18
\item[uṭṭiṣṭhaṃ cottare prāṇā baddhe padmāsane muhuḥ] N17
\item[uttiṣṭhatyuttame prāṇarodhe padmāsane muhuḥ] G11
\item[uttiṣṭaty attame prāṇe baddhe padmāsane dṛḍhe] N1
\item[uttiṣṭhaty uttame prāṇe yāvat padmāsane samaḥ] N12
\item[uttiṣṭhaty uttare prāṇā baddhe padmāsane dṛḍhe] J12,
\item[uttiṣṭhaty unname prāṇe baddhe padmāsane dṛḍhe] N19
\item[uttiṣṭhaty uttame prāṇa baddha padmāsanaṃ dṛḍhaṃ] N21
\item[uttiṣṭety uname prāṇaḥ rodhe pavāsane sthite] N22
\item[uttiṣṭhaty uttame prāṇa rāvai padmāsano muhuḥ] N23
\item[uttiṣṭhaty uttame prāṇa rodhe padmāsano muhuḥ] V6
\item[uṭṭiṣṭhnty uttame prāṇā baddhe padmāsane muhuḥ] LD1,
\item[uttame sthānaṃ āpnoti tato vāyuṃ nibandhayet] P14, P18, 
\item[(illegible/unavailable)]   Ba5, Bo3, C5, N4a, N6, P22, V22, V12, V9, T, P16, R5, 
  \begin{description}

    \end{description}
 \end{marma}

\begin{marma}[hp02_019]
\sthana{2.19c}
\item[kṛśatā] A1, B3, Ba, Ba5, Bo1, Bo2, Bo3, C1, C2, C6, C8, J1, J4, J5, J6, J7, J11, J13, J14, V19, N1, N2, N4a, N5, N8, N11, N12, N13, N16, N18, N19, N22, P2, P5, P6, P7, P11, P12, P13, P14, P15, P17, P18, P20, P23, P25, P26, P27, C4, V17, V10, V21, Tue, N23, N26, V7, V20, V5, V28, V22, V2, V18, G11, G1
\item[kṛsatā] Ko, J3,  
\item[kṛśate] J10pc
\item[kriyate] C3, J10, J12, J16, J17, LD1, N7, N9, N10, P1, P8, P9, P10, R1, V12, C9, J2, N17, LD2, V19, Ba3
\item[kṛyate] J15, N14
\item[kṛśatāṃ] J8, V8, N21, V3, N25, P21, V6
\item[kṛṣitāṃ] N24
\item[kṛtā] N3, N20, V14
\item[krūratā] J2
\item[(illegible/unavailable)]   B1, B2, C5, N6, V16, V15, V13, V9, T, P16, P22, P24, R5,
  
\begin{description}

    \end{description}
 \end{marma}

 \begin{marma}[hp02_021]
\sthana{2.22a}
\item[medaḥśoṣādikaṃ pūrvaṃ] P6, V1
\item[medaḥśleṣmādikaṃ surve] V6
\item[medaśokādikaṃ pūrvaṃ] N22
\item[medaḥśleṣmādi nāśārthe] A1, C3, P26, 
\item[medaḥśleṣmādi nāśārtha] LD2,
\item[medaḥśleṣmādi ṣatkarmāṅi] V28
\item[medaḥśleṣmādi nāśārthaṃ] B1, B3, Bo1, C2, J10, J12, J13, J15, J16, J17, LD1, N7, N17, P1, P2, P5, P8, P9, P12, P13, P17, R1, V21, V20, V12
\item[medaḥśleṣmādhi nāśārthaṃ] P27, 
\item[madaṃ śleṣmādi nāśārthe] P10, 
\item[nādaśleṣmādinā śāstraṃ] Ba3
\item[medaḥśleṣmādināśīrṣaṃ] N9
\item[medaḥ tuṣmād adhikei purva śaṭ karmaṃ] V8
\item[medaśleṣmādi nāśārthaṃ] C9, N1, N10, N14, N21, V18
\item[medaśleṣādi nāśārthe] V17
\item[medaśleṣmādi nāsārthe] N20, N26
\item[medaśleṣma nivṛtyarthaṃ] Bo2, C1, J1, J3, J6, N2, N8, N11, N16, N18, P20, P21, P23, V15, V5, V2
\item[medaḥśleṣma nivṛtyarthaṃ] N5, C4
\item[??śleṣma nivṛtyarthaṃ] N25
\item[medaḥśleṣmādhikaḥ pūrvaṃ] Ba, C8, J5, P7, P11, P14, P18, V10, V14, V13, G11
\item[medaḥśleṣmādhikaḥ pūrva] J7,
\item[medaḥśleṣmādhikaḥ arthaṃ] V16
\item[medai śleṣmāṃdhika purṇa] J2
\item[medaśleṣmāṃdhikaḥ pūrvaṃ] Ba5, Bo3, V3
\item[medaḥ ślṣāṃdhikaḥ pūrvaṃ] C6, J14, 
\item[medaśleṣmādhikaḥ pūrvaṃ]  J11, N3, N4a, N12, N13, P25, Tue, V22
\item[medāśleṣmāṣikaḥ pūrvva] N23, V7
\item[medaḥśleṣmaṇisapurṇa] N24 
\item[mitaḥ sthūlādikaiḥ kāryyaṃ] B2
\item[medodhikastu havabhiḥ] Ko,
\item[(unavailable/illegible)] C5, J4, N6, N19, P15, V9, T, G1, P16, P22, P24, R5,
  \begin{description}

    \end{description}
 \end{marma}


 \begin{marma}[hp02_021d]
\sthana{2.22d}
\item[apy abhāvataḥ] V1
\item[api bhāvataḥ] C3
\item[samabhā] R1,
\item[samabhāvataḥ] A1, B1, Ba5, Bo1, Bo3, C9, J1, J3, J6, J10, J11, J12, J13, J14, J15, J17, LD1, LD2, N3, N4a, N7, N8, N10, N11, N12, N13, P1, P5, P10, V10, N20, N21, P6, P8, P9, P11, P13, P14, P18, V14, V3, Tue, Ba3, N26, V7, V20, V15, V28, V22, V18, V12, V13, G11
\item[samabhāvat] J16, N17, 
\item[samupāgataḥ] J7,
\item[samabhyāvataḥ] N24
\item[api samācaret] V16, V17, V21
\item[samabhāvata]  N9	
\item[samabhāvanā] N22
\item[smabhāvata]  N14
\item[sabhāgataḥ] Ba,
\item[samabhāgataḥ] B2, C6, C8, P7, V8, N18, N23, P25, P27, 
\item[samabhāgatā] Ko,
\item[samabhāgata] V6
\item[samabhāgikaḥ] B3, C2, N1, P2, P12, P17, 
\item[samabhārat] J2
\item[samahāvata]  P26, 
\item[samatā yataḥ]  Bo2, C1, C4, N2, N16, N25, P20, P23, V2
\item[samatā yatāṃ]  P21, 
\item[samatā yathā]   N5,
\item[samatāpataḥ] V5
\item[(unavailable/illegible)] C5, J4, N6, N19, P15, T, G1, P16, P22, P24, R5,
  \begin{description}

    \end{description}
 \end{marma}


 \begin{marma}[hp02_022a-c]
\sthana{}
  \begin{description}
    \end{description}
 \end{marma}


 \begin{marma}[hp02_023c]
\sthana{2.22c}
\item[vicitraguṇasandhāyī(i)] A1, B1, B3, Ba, Ba5, Bo1, Bo2, Bo3, C1, C2, C3, C8, C9, J1, J2, J3, J4, J5, J6, J7, J11, J13, J14, J15, J16, LD1, LD2, P7, V1, W4, J10, N17, J17, J8ac, Ko, N1, N3, N4a, N7, N9, N10, N16, P1, P2, P6, P9, P10, P11, P12, P13, P15, P17, P18, P21, P24, P25, P27, R1, V8, V16, V17, V10, N18, N19, N21, N22, P14, V14, V3, Tue, N23, V7, V15, V5, V28, V22, V2, V18, V12, V13, G11, Ba3
\item[vicitraguṇasadhāyī] P26,
\item[vicitraguṇaśandhāyī]  P8, 
\item[vicītraguṇaṃsandhāyi] P5, 
\item[vicitraguṇasaṃyāyi] P23,
\item[vicitraguṇasandhāya]  C6, N12
\item[vicitraguṇasandhāyā] N20, V20
\item[vicitragunasandhāyai]  N14, V21, V6
\item[vicitraguṇasandhāryā]  N13
\item[vicitraguṇasandhāryaṃ]  J12,
\item[vicitraguṇasandhānaṃ] N26
\item[vicitraguṇasadyāpi] N2
\item[vicitraguptasaṃ?jñopi] N5
\item[vicitraguṇasaṃjñopi] C4, P20, N25
\item[vicitraguṇasaṃdhāyoga]  N8, N11,
\item[trividhaṃ guṇasandhāyi] B2
\item[(unavailable/illegible)] C5, N6, N24, T, G1, P16, P22, R5,
  \begin{description}

    \end{description}
 \end{marma}


 \begin{marma}[hp02_24c]
\sthana{2.24c}
\item[graset pratyāharec caitad] P25, 
\item[tataḥ pratyāharec caitad] A1, Ba, Ba3, Bo1, B3, C8, G11, J12, J16, J17, N9, N12, N14, N17, N20, N21, N23, P1, P5, P8, P9, P10, P11, P12, P13, P15, P24, P26, P27, R1,
\item[tatao pratyāharec caitad] C9,
\item[tataḥ pratyāhare caitad] C6,
\item[tataḥ pratyaharec cetad] N10
\item[tataḥ pratyaharec caitad] N13, P7, 
\item[tata pratyaharec caiva] N24,
\item[punaḥ pratyaharec caitad] B1, B2, P2, P14, P17, P18, Tue,
\item[punaḥ pratyahare caitad] J13, J14, 
\item[punaḥ pratyāhare caitad] N19, 
\item[punaḥ pratyāhare caitā] N22,
\item[pūnaḥ pratyāharec caityāt] P6,
\item[punaḥ pratyahared etad] C1, C2, C3, C4, J1, N8, N11, N16, N18, N25, N26, P20, P21, P23, 
\item[punaḥ pratyaharec caitaḥ] J15,
\item[(unavailable/illegible)] C5, J1, N6, N7, T, G1, P16, P22, R5,
  \begin{description}

    \end{description}
 \end{marma}

 \begin{marma}[hp02_24d]
\sthana{2.24d}
\item[prakṣālyaṃ dhautikarma tat] Ba, C6, P7, P25, V21
\item[prakṣālanaṃ dhautikarma tat] V13
\item[prakṣālye? dhautikarma tat] C8
\item[ādāyaṃ dhautikarma tat] V14
\item[udgāraṃ dhautikarma tat] V1,
\item[sodgāraṃ dhautikarmavit] N12
\item[uditaṃ dhautikarma tat] A1, Ba5, Bo1, Bo3, C9, J10pc, J12, J15, LD1, LD2, N4a, N7, N9, N10, N13, N14, N17, P5, P8, P9, P10, P14, P18, P24, V16, V17, V10, N20, N21, Tue, N26, V7, V20, V15, V18, V12
\item[uditaṃ dhātikarma tat] P13, 
\item[udita dhaitikarma tat] G11
\item[aditaṃ dhautikarma tat] J17,
\item[uditaṃ dhautikarma kṛt] J10, P1, R1, V22
\item[up(?)itaṃ dhautikarma tat] C3
\item[abhyāsād dhautikarmavit] B1, B2, C2, C4, J1, J3, J6, J13, P2, P12, P17, P20, P26, V5, V28, V2
\item[abhyāsād dhautikarmasu] N25
\item[abhyāsā dhautikarmavit] N18
\item[abhyāsā dhautikarma tat] P6,
\item[abhyāsā dhautikarmavit] P21,
\item[abhyāso dhautikarmavit] N19
\item[abhyase dhautikarmavit] P27,
\item[abhyāse dhautakarmavit] J14,
\item[abhyāsād dhotikarmavit] C1
\item[abhyāsād dhautikarmavat] N5
\item[abhyāsaud dhautikarmavit] Bo2,
\item[abhyāsāsau dhautikarmavit] V6
\item[abhyāsau dhautikarmatat] V8
\item[abhyāsād dhautikarmavita] N2
\item[abhyāsād dhautikarma tat] N1 (after correction),
\item[dukhalātā dhautikarmataḥ] V3
\item[duḥkhalṃ dhautikarma tat] Ko,
\item[duṣālaṃ dhautikarma tat] J4, J11, 
\item[duḥkhahaṃ dhautikarma tat] P11, 
\item[uṣālaṃ dhautikarma tat] J5, P15, 
\item[abhyāsa dhautakarka tat] N22
\item[abhyāsād dhauti bhavati karma tat] N8, N11 ("bhavati" in brackets)
\item[uṃtthānaṃ dhautikarma tat] N23
\item[dākṣālaṃ dhautikarma tat] N24
\item[utthānaṃ dhautikarma tat] J7,
\item[dugdhāraṃ dhautikarma tat] J2
\item[syā sā dhautikarmakṛt?] B3
\item[?s?vāla dhautikarmakṛt] N3
\item[(unavailable/illegible)] C5, N6, N16, T, G1, P16, P22, P23, R5,
  \begin{description}

    \end{description}
 \end{marma}

 \begin{marma}[hp02_25d]
\sthana{2.25d}
\item[bādhate yaṃ na] P27,
\item[chadhyate ca na] J16,
\item[dhāvaty eva] V1, V8, V3, V2
\item[dhāvaṃtyeva na] P6, P15, 
\item[dhāvate ca na] B2
\item[dhāvaṃta ca na] N19
\item[dhāvaṃte ca na] J5,
\item[dhavaṃte ca na] N22
\item[dhruvntye neva] Ko,
\item[gachaty eva na] N21, P13, 
\item[gachaṃty eva na] Bo1
\item[prayāntyeva na] Ba, N4a, Bo2, Bo3, C1, C4, C6, C8, J1, J3, J6, J7, N5, N8, N11, N13, N16, P7, P11, P14, P18, P20, P21, P25, V10, Tue, N25, V28, V22
\item[dhāvatyeva na] G11
\item[prayānteva na] N23, V7
\item[śudhyatty eva] A1,
\item[śudhyaty eva] LD2, P2, P9, 
\item[śudhyate nātra] J12,
\item[śudhyate ca na] N10, P26, 
\item[śudhyanty eva] B1, C3, C9, J10, J13, J17, N1, N17, P5, P8, P12, P17, Ba3
\item[śudhyanteva] B3, P10, 
\item[śudhyaṃteva na] J15, N9, N14, P24, V16, V17
\item[śvadhvatyeva na] R1,
\item[śudhyaṃte ca na] LD1, N7, P1, 
\item[suddhyaṃtevaṃ na] N18
\item[śudhyaṃ sevana] C2,
\item[kṣīyaṃte saklā malā] J14,
\item[?īyatsyeva na] N2
\item[bhavaṃty eva na] N3, V14
\item[naśyaṃteva] N12
\item[naśyaṃte] J11
\item[nāśaṃ yāṃti] J4, 
\item[naśyaṃty eva na] J2,
\item[jayaty eva na] N24
\item[śuṣyaṃte sakalāmalāḥ] N20, N26
\item[(illegible/unavailable)] Ba5,C5, N6, V21, V6, V20, V15, V5, V18, V12, V13, T, G1, P16, P22, P23, R5,
  \begin{description}

    \end{description}
 \end{marma}      
 
 \begin{marma}[hp02_26a]
\sthana{2.26a}
\item[udaragatapadārtham udvahatī] J12,
\item[udaragatapadārthaṃ  utbhavanti] G1
\item[udaragatapadārtham udvahati] R1,
\item[udaragatapadārtham udvahantī] Ba, C3, J1, J7, Ko, N10, P6, V1, V16, N12, N18, N22, N23, P20, P27, V6, V28, V2
\item[udaragatapadārthamūrdhāhatī] J16,
\item[uragatapadārtham udvahantī] B2,
\item[udaragatapadārthān udvahaṃti] C6, P7, 
\item[udaragatapadārtham uşmahantī] V17
\item[udaragatapadārtham udvamaṃti(ī)] B1, B3, Ba5, Bo2, C2, C8, C9, J3, J6, J11, J14, LD1, LD2, N7, N13, N16, N16, N20, N21, N26, P2, P8, P9, P11, P12, P13, P14, P17, P18, P21, P25, P26, V3, V20, V22, V18, V12, V13, Ba3
\item[udaragatapadārtham udvayant] P5,
\item[udaragatapadārtham uddhayaṃt] P24, 
\item[udaragatpadārtham udvamaṃtī] N9,  
\item[udaragatapadārtham udvavaṃtī] J10, J17, P1,
\item[udaragatapadārtham udvamaṃta] J4, P10, 
\item[udaragatapadārtham udvamaṃte] P15, V15
\item[udaragatapadārthām udyamaṃte] N19
\item[udaragatapadārthaṃ śuddhapetaṃ] N25
\item[udaragatapadyarthān udvamaṃtī] A1
\item[udaragatapadārtham uḍhasaṃtī] N24
\item[udaragatapadārtham udvapanti] N14, 
\item[udaragatapadārthaśodhanaṃ syāt] Bo1
\item[udaragatapadārtha śuddhayeta] C4
\item[udaragatapadārtha śuddhaye vai] J15, 
\item[udaragatapadārtha suddhayeta] J13,
\item[udagatapadārtha uddhanaṃtī] J5,
\item[(illegible/unavailable)] Bo3, C1, J2, N6, N8, N11, R5, V19, V10, V14, Tue, C5, V21, V7, V5, V22, G11, T, P16, P22, P23, 
  \begin{description}
    \end{description}
 \end{marma}

 \begin{marma}[hp02_26b]
\sthana{2.26b}
\item[pavanam apānam udīrya kaṇṭhanāle] V1, A1, B2, Ba, Ba5, C6, C8, C9, J3, J4, J6, J7, J10, J11, J14, J15, J17, Ko, LD1, LD2, N7, N9, N10, N13, N16, N17, P6, P7, P8, P9, P10, P13, P18, W4, J8, V16, N18, N20, N22, P14, P15, P20, P25, P27, V3, N23, V20, V15, V28, V18, V12, V13, G1
\item[pevanam apānam udīrya kaṇṭhanāle] P5, P24, 
\item[pavanam apānam udīrya kaṇṭhānālī] Ba3
\item[pavanam apānam udīrya kaṇṭhanālaiḥ] N12, 
\item[pavanam apānam udārya kaṇṭhanālo] J16,
\item[pavanām apānam udīrya kaṇṭhanālṃ] J1,
\item[pavana apānam udīrya kaṇṭhanāle] V6
\item[pavanam upānam udīrya kaṇṭhanāle] C3
\item[pavanam upānam urīrya kaṇṭhanāle] C4
\item[pavanam apānam udīrya kaṇṭhanālo] N19, R1,
\item[pavanam apānam udīrya kaṇṭhanālāt] B3, C2, J13, N26, P1, P2, P12, P17, P21, P26, V2
\item[pavanam apām udirya kaṃnāle] N25
\item[pavanam apāsenam udāryaṃ kaṇṭhanāle] J5,
\item[pavanam apānam udīryakaṃ jāle] P11, 
\item[pavanam mudārya kaṇṭhanālo] J12
\item[pavanam udirya kaṇṭhanāle] N21
\item[mapānam udīrya kaṇṭhanāle] B1,
\item[pavanamapānam udīrya kaṇṭhanāle] Bo1, Bo2,
\item[pavanam athānam utiya kaṇṭhanāle] N24
\item[pavanam apānam udīrya karāḍanāle] V17
\item[pavanam apānam udīrya kaṇṭhaḍālai] N14, 
\item[(illegible/unavailable)] Bo3, C1, C5, J2, N6, N8, N11, R5, V19, V10, V14, V21, Tue, V7, V5, V22, G11, T, P16, P22, P23, 
\end{description}
 \end{marma}

 \begin{marma}[hp02_26c]
\sthana{2.26c}
\item[karibhir iva jalasya vāyuvegāt] A1, B1, B3, C2, C9, J1, J3, J6, J8, J10, J17, LD1, N7, N10, N16,  N17, P1, P2, P9, P12, P17, W4, V16, V17, N18, N20, N26, V2, V18, V12, Ba3
\item[karibhir iva jalasya vāyuvegaj] R1,
\item[karibhir iva gajasya vāyuvegāt] V20
\item[karibhir iva jalasya vāyuvegā] B2, C3, J12, J16, N14, P8, 
\item[karibhir iva jalasya vāyubagā] P26,
\item[karibhir iva jalasya vāyuvege] N9,
\item[karibhir iva jalaṃ ca vāyuvegā] Ba,
\item[karibhir iva jalaṃ ca vāyuvegāt] P25, 
\item[karibhiriva jalaṃsya vāyuvegāt] Bo2, 
\item[karibhiriva jalasya vāyuvego] J13,
\item[karibhiriva jalasya vāyuyogāt] V28
\item[karibhirivadāyikā jalabhya vāyuvegāt] P21, 
\item[kramamarivayavasya vāyur yo] V1
\item[kramaparicayavaśya nāḍicakrā] Ba5, N13, P14, P18,  
\item[kramaparicayavaśyā nāḍīmārgā] C6, J7, LD2, 
\item[kramaparicayavaśyā nāḍīmārgāt] N12,
\item[kramaparicayavaśya vāyumārga] C8
\item[kramaparicayavaśya vāyumārgā] P7, P20, 
\item[kramaparicayavasya vāyuvegā] P27,
\item[kramaparicayavaśa nāḍimārgā] P11,  
\item[kramaḥ paricayavasya vāyuḥ mārgre] P6,
\item[kramaparicayavasya vāyumārgre] J14,
\item[kramaparicayavaśya vāyumārgāt] Bo1
\item[kramaparicayavaśya vāyumārgaṃ] N21, P13, 
\item[kramac pacya paricaya vaśyā vāyumārgaṃ] J5
\item[kramaparicayavaśyaṃ vāthagargāṃ] N24
\item[kramaparicayavaśya mārgte(??)] N23
\item[kramaparicayavasya vāyumārge] N22
\item[kramapārīcita vāyumārge] J15,
\item[kramaparījita vāyuḥ] P15, 
\item[kramapārīcitavacana vāyumārgā] P5, P24, 
\item[kramaparicitavasya vāyuyogo] Ko,
\item[kramapārīcitavaśya vāyumārg(oṃ)?] N19
\item[kramapārīcitavaśya vāyumārgo] J11
\item[kramapārīcitavaśya vāyumārge] J4, 
\item[kramapārecitavaktu? vāyumārge] C4
\item[kramaparicittakakrū vāyumārge] N25
\item[kirimir iva jalasya vāyuvegāt] P10, 
\item[(illegible/unavailable)] Bo3, C1, J2, N6, N8, N11, R5, V19, V10, V14, Tue, C5, V21, V6, V7, V15, V5, V22, V13, G11, T, G1, P16, P22, P23, 
  \begin{description}
    \end{description}
 \end{marma}

 \begin{marma}[hp02_26d]
\sthana{2.26d}
\item[gajakaraṇānti nigadyate haṭhajṇau] Ko,
\item[gajakarīṇīti kathyate haṭhavidyā] G1
\item[gajakaraṇīti nigadyate haṭhajñaiḥ] B2, Ba, Ba5, Bo1, C3, C4, C6, J4, J6, J7, J11, J15, N12, N13, P5, P6, P7, P9, P10, P14, P18, P24, P26, P27, V1, N19, N23, N25, P13, V20, V15, V13
\item[gajjakaraṇīti nigadyate haṭhajñaiḥ] P8, 
\item[gajakaraṇī nigadyate haṭhajñaiḥ] N16, 
\item[gajakaraṇīti nigadate haṭhajñaiḥ] J14,
\item[gajakaraṇīti nigadyāta haṭhaj~naiḥ] J5,
\item[gajakaraṇīti nigadyaṃte haṭhajñaiḥ] LD2,
\item[gajakaraṇīti nigadyate] N24, V2
\item[gajakarṇīti nigadyate haṭhaj~naiḥ] A1, C8
\item[gajakaraṇīva nigadyate haṭhaj~naiḥ] B1
\item[gajakariṇīva nigadyate haṭhajñaiḥ] C2
\item[gajakaraṇīti nigachate haṭhaj~naiḥ] B3
\item[gajakāraṇī nigadyate haṭhaj~naiḥ] Bo2,V3, V6
\item[gajakaraṇīti nigadyate haṭhakṣaiḥ] N22
\item[gajakaraṇī ca nigadyate dṛḍhajñaiḥ] J13,
\item[gajakaranīti nigadye haṭhajña] N14, 
\item[gogajakaṇīti nigadyate haṭhajñaiḥ] P15, 
\item[jalagajakaraṇī nigadyate haṭhaj~naiḥ] J1, J3, 
\item[jalakaraṇīti nigadyate haṭhajñaiḥ] C9, J12, J16, LD1, N7, N17, P1, W4, N20, N26, P21, P25, 
\item[jalakaraṇīti nigadyato haṭhajñaiḥ] N10,
\item[jalakaraṇīti nigadyate mahāhaṭhaj~naiḥ] V21
\item[jalakariṇīti nigadyate haṭhajñaiḥ] J10, J17, N9, P20, R1, V16, V28, V18, V12, Ba3
\item[jalakaraṇānī nigadyate haṭhaj~naiḥ] V17
\item[jalagajakaraṇīti nigadyate haṭhaj~naiḥ] J8, N18, P2,
\item[jalakaraṇīva nigadyate haṭhajñaiḥ] P12,
\item[jalagajakaraṇīva nigadyate haṭhajñaiḥ] P17, 
\item[gajakaraṇīti nirādyate haṭhaj~naiḥ] N21,
\item[najakariṇīti nigadyate haṭhajñaiḥ] P11, 
\item[(illegible/unavailable)] Bo3, C1, J2, C5, N6, N8, N11, V19, V10, V14, Tue, V7, V5, V22, G11, T, P16, P22, P23, R5,
  \begin{description}
    \end{description}
 \end{marma}

 \begin{marma}[hp02_29c]
\sthana{2.29c}
\item[aseṣadoṣān api yan nihanyād] Ko, 
\item[aśeṣadoṣopacayaṃ nihanyad] J8, N17, V1, W4, V16, V17, V10, V6, V7, V15, 
\item[aśeṣadoṣoyathā nihanyānād] V14
\item[aśeṣadoṣāmaya śoṣaṇāya] G1
\item[aśeṣadoṣopacayaṃ nihanyād] B1, B2, B3, Ba, Ba5, Bo1, Bo2, Bo3, C1, C3, C4, C6, C8, C9, J2, J3, J5, J6, J7, J10, J11, J13, J14, J15, J17, LD1, LD2, N7, N8, N11, N13, N16, N18, N19, N20, P1, P2, P5, P8, P9, P12, P13, P14, P15, P17, P18, P20, P21, P24, P25, P27, Tue, R1, N25, N26, V20, V5, V28, V22, V2. V18, V12
\item[aśeṣadoṣopacayaṃti hanyād] J1,
\item[aśeṣadopacayaṃ nihanyād] P23,
\item[aśeṣa doṣā pacayan niśamyaha nyāda] G11
\item[aśeṣadoṣo ca nayaṃti hanyād] N10,
\item[aśeṣadoṣopacaya no hanyād] J16,
\item[aśeṣadoṣopacayā nihanyād] V13
\item[aśeṣadoṣopacayān nihanyād] N12,
\item[aśeṣadoṣapacayaṃ nihanyād] N9,
\item[aśeṣadoṣopacayaṃ nihanti] P7, 
\item[aśeṣadoṣopacayaṃ nihanyā] N23,
\item[aśeṣadoṣopacaya nihanyād] N24, P10, P26, 
\item[aśeṣa?oṣovacayaṃ nihanyād] N21,
\item[aśeṣadoṣasya cayaṃti hanyād] P6,
\item[aśeṣadoṣepacayaṃti hanyād] J12, N14, 
\item[aśeṣa doṣo vacayan nihanyād] Ba3
\item[aśeṣadoṣepacayaṃ nihanyad] C2,
\item[aśeṣadoṣāpacayaṃ nihanyad] V21, V3
\item[aśeṣadoṣopacayaṃ nihatyad] P11, 
\item[aśeṣadoṣopacitiṃ hathajñād] J4,
\item[aśeṣadoṣopacayaṃ ca hanyad] A1
\item[aśeṣadoṣaprabhavaṃ nihanyad] V19
\item[aśeṣadoṣasya cayana hatyā] N22
\item[(illegible/unavailable)] C5, N6, T, P16, P22, R5,
  \begin{description}

    \end{description}
 \end{marma}

 \begin{marma}[hp02_29d]
\sthana{2.29d}
\item[abhyasyamānaṃ jalavastikarma] Ba5, Bo1, Bo2, Bo3, C1, C3, C6, C8, J1, J2, J4, J5, J6, J7, J8, J11, J12, J13, J14, J15, J16, LD1, N8, N9, N10, N11, N12, N13, N14, N16, N17, P1, P2, P8, P9, P11,  P12, P14, P15, P18, P20, P23, W4, V16, V17, V10, N18, N19, N21, P13, P25, P27, V21, V3, Tue, N26, V6, V7, V20, V15, V5, V28, V22, V2, V18, V13, G11, G1
\item[abhyasyamānaṃ janabastikarma] R1,
\item[abhyasyamānaṃ jalavāstikarma] P5, P24, 
\item[anyasyamānaṃ jalavastikarma] J3,
\item[abhyasyamānaṃ nijavastikarma] V12
\item[abhyasyamānaṃ jalavastikarmaḥ] Ko, P6, 
\item[abhyamānaṃ jalabastikarma] P21, 
\item[abhyasyamānaṃ khalu vastikarma] C9,
\item[abhyasyamāna ja?vastikarma] N20(something seems added above)
\item[abhyasyamānaṃ jalavastitakramāt] N24
\item[praamāńam jalavastikarma] V14
\item[abhyasyamāno jalavastikarma] B3, C2, P17, 
\item[abhyasyamā jalavastikarma] B1,
\item[abhyāsamānaṃ jalavastikarma] C4, V1, N7, N25, P10, P26, 
\item[abhyāsyamānaṃ jalavastikarma] A1, Ba,
\item[abhyāsamānaṃ jalavasti] V19
\item[abhyasya mānam jalapastikarma] Ba3
\item[abhyasyamānaṃ javastikarma] J10, J17,
\item[abhyasyamānaṃ jvalabastikarmaḥ] LD2,
\item[bhyasyabhānaṃ jalavestikarma] N23(space for “da” left blank with an overhead line)
\item[abhyasya śūnyaṃ jalavastikarma] B2,
\item[cābhyasyamānaṃ jalavastikarma] P7, 
\item[dattasya mānaṃ yanabasyakarmaḥ] N22
\item[(illegible/unavailable)] C5, N6, T, P16, P22, R5,
  \begin{description}

    \end{description}
 \end{marma}

 \begin{marma}[hp02_30]
\sthana{2.30}
\item[atha natīḥ] J1,
\item[atha ti] P26, 
\item[atha netī(ḥ)] Ba, Ba5, Bo1, Bo2, C1, C2, C4, J3, J6, J8, N8, N11, N16, V1, V10, N18, P20, P21, P23, V21, V3, N23, N25, N26, V7, V5, V2
\item[atha neti] J2, Ko, N7, N9, N10, N13, N14, N20, N24, P7, P9, V6, V28
\item[atha netiḥ] J7, N12, P14, P18, P25, Tue, V22, V12, Ba3
\item[atha netikarma] C3, C9, J10, J17, LD1, P1, P2, P5, P8, P10, P24, W4, V16, V17, N21, P13, V14, V15, V18, V13
\item[atha netikarmaḥ] C8, J12, J15, N22, P6, 
\item[atha netīkarma] A1, B1, B2, C6, J13, J14, J16, N17, P12, P17, P27, R1, V20
\item[atha netīkarmaḥ] LD2
\item[atha nītī] J5,
\item[atha nīti] P11, 
\item[atha nītikarma] J4,
\item[atha nītīkarma] N19
\item[prathama netīkarma] B3
\item[(illegible/unavailable)] Bo3, C5, N6, J11, G11, P15, T, G1, P16, P22, R5,
  \begin{description}

    \end{description}
 \end{marma}

 \begin{marma}[hp02_30c]
\sthana{2.30c}
\item[mukhanirgamanād eva] A1, B2, C3, J10, J17, LD1, N7, N9, N17, P1, P5, P6, P8, P9, P10, P24, P26, V1, W4, V16, N22, V20, V18, V12, V13, Ba3
\item[mukhanirgamanād evaḥ] J15,
\item[mukhanirgamanād deva] C9
\item[mukhanirgatanād evaṃ] J12,
\item[mukhanirgatamanād eva] V14
\item[mukhanirgamayedeti] J2,
\item[mukhan nirgamanād eva] B3,
\item[mukhān nirgamanād eva] C2, P12, P17, 
\item[muvi nirgamanād eva] R1,
\item[mukhā nirgamanād eva] P2, 
\item[mukhṃ nirgamanād eva] N14, 
\item[mukha nīrgamanod eva] J16, 
\item[mukhān nirgamayec caiṣā] Ba, Ba5, Bo1, C8, J8, J11, LD2, N12, N13, V17, N19, N20, P14, P18, P25, V21, Tue, N26, 
\item[mukhān nirgamyate caiṣā] J4,
\item[mukhān nirgamayed eṣā] P13, 
\item[sukhān nirgamayeccaiṣā] G11
\item[mukhān virgamayed eṣā] N21
\item[mukhā nirgamayec caiṣā] J7, P15, V10, V3, V6, V7, V15, V22
\item[mukhe nirgamayec caiṣā] C6, P7, 
\item[mukhe nirgamaye caiṣā] P11, 
\item[mu nirgamayai caiṣā] J5,
\item[mukhān nirgamaye doṣā] N24
\item[mukhān nirgamayet sāpi] V19
\item[mukhān nirgāmayet sā hi] B1, Bo2, J13, J14, N11, N18, N25  
\item[mukhān nirgamanāt sā hi] P21, P23, 
\item[mukhān nirgamayet sā hi] J6, N8, N20, P20, P27
\item[mukhā nirgāmaye sā hi] V2
\item[mukhā nirgāmayet so hi] Ko,
\item[mukhā nirgamanā sā hi] V5
\item[mukhān nigarmayet sā hi] C4, J3, N16, 
\item[mukhā nigarmayet sā hi] J1, V28  
\item[mukhān nirgatya nāsāyam] C1,
\item[mukhaśca nirgamec caiṣā] N23
\item[sukhanirgamanād eva] N10,
\item[mukhaṃ nirgamayedeṣu nīti] G1
\item[(illegible/unavailable)] Bo3, C5, N6, T, P16, P22, R5,
  \begin{description}

    \end{description}
 \end{marma}

 \begin{marma}[hp02_31cd]
\sthana{2.30cd}
\item[] 
\item[(illegible/unavailable)] 
  \begin{description}

    \end{description}
 \end{marma}

 \begin{marma}[hp02_33a]
\sthana{2.33a}
\item[ghaṭanaṃ] J2, 
\item[muñcakaṃ] Ko,
\item[mucyate] J4,
\item[mocanaṃ] B2, Ba5, C8, J7, J12, N12, N13, N18, N21, N22, P13, P14, P16, P18, P26, Tue, N23, V7
\item[mocakaṃ] Bo2, C1, J1, J3, J5, J6, J11, J14, N16, N19, P15, P20, P21, P23, V28
\item[moṭakaṃ] V1, Bo1, N25 
\item[meṭakaṃ] P11, 
\item[mokṣakaṃ] N8, N11, 
\item[toṭakaṃ] V19, V16
\item[tāṭakaṃ] V17
\item[trāṭakaṃ] A1, C4, C6, P6, P7, P27, V10, V14, V21, V3, V6, V20, V15, V5, V28, V22, V2, V18, V12, V13, G11, Ba3, G1
\item[troṭakaṃ] N24
\item[rocanaṃ] Ba, P25, 
\item[sphoṭanaṃ] B1, B3, C2, C3, C9, J8, J10, J13, J15, J16, J17, LD2, N7, N14, N17, P1, P8, W4, N20, N26, P2, P5, P9, P10, P12, P17, P24, Ba3  
\item[sphoṭana] R1,
\item[spheṭanaṃ] N10,
\item[sphoṭakaṃ] LD1,
\item[sphoṭa] N9, 
\item[(illegible/unavailable)] Bo3, C5, N6, T, P22, R5,
  \begin{description}

    \end{description}
 \end{marma}

 \begin{marma}[hp02_33b]
\sthana{2.33b}
\item[jadrānāṃ ca kapāṭakaṃ] J16,
\item[tandrādīnāṃ gapāṭa[ka]m] V1,
\item[tandrādīnāṃca pāṭanaṃ] P15, V19,
\item[ta.ntryādīnāṃ kavāṭakam] G1
\item[tandrādīnāṃ ca kapāṭakaṃ] B1,
\item[tandrādīnāṃ kapāṭakam] B2, B3, Ba, Ba5, Bo1, C1, C2, C4, C6, C8, J3, J5, J6, J7, J8, J11, J12, Ko, LD1, LD2, N7, N8, N11, N13, N16, P2, P6, P7, P12, P13, P14, P16, P17, P18, P23, P25, P26, P27, W4, V17, V10, N21, N22, V3, Tue, V7, V5, V28, V22, V2, V13, Ba3
\item[tadrādīnāṃ kapāṭakam] P21,
\item[tandrādīnāṃ kavāṭakam] G11
\item[tandrānāṃ tu kapāṭakam] N14,
\item[tandrādīnāṃ kṛpāṭakam] N12, 
\item[tandrādīṇāṃ kapāṭakam] J13, 
\item[tandrāṇāṃ ca kapāṭakam] J15, N9, N10, P1, P8, P10, R1, 
\item[tadbhānaṃ ca kapāṭakam] P5, P24, 
\item[tandrādīnāṃ kapālakam] J1,
item[tandrāegaṃ kathapāṭakam] V21
\item[tandrādīnāṃ kapāṭakai] Bo2
\item[tandrādiṇāṃ kapāṭakam] J8, N18, N23,
\item[tandrādīnāṃ kaṃpāṭa] P9, 
\item[tandrādīn kapāṭakam] P11,
\item[tandrādiṇāṃ ca pāṭakam] J14, N19
\item[tandrādīṇāṃ ca pāṭakam] C9
\item[tandrādīṇāṃ ca pāṭhakam] V15
\item[tandrādīṇāṃ ca pāṭavam] J4,
\item[tandrāṇāṃ ca kapāṭakam] C3, J10, J17, V16, N20, V14, N26, V20, V18
\item[tadvāṇāṃ ca kapāṭakam] A1,
\item[tandrāṇāṃ tu kapāṭakam] N17
\item[sandrādīnāṃ kapātakāṃ] N24
\item[nidrādīnāṃ kapāṭakam] P20,
\item[nidrārttīnāṃ kapāṭakam] N25
\item[hetādīnāṃ? kapāṭhakam] J2,
\item[(illegible/unavailable)] Bo3, C5, N6, P22, R5, V6, V12, T
  \begin{description}

    \end{description}
 \end{marma}

 \begin{marma}[hp02_34b]
\sthana{2.34b}
\item[gudaṃ] V1
\item[gopyaṃ] P1, V16, V17, V10, N14, N18, V14, V21, V3,V6, V7, V20, V15, V28, V22, V2, V18, V12, V13, G11, Ba3, G1
\item[tuda] N10,,
\item[tundaṃ] A1, B3, Ba5, Bo2, C2, C8, C9, J3, J6, J7, J8, J13, J14, J16, LD1, LD2, N8, N9, N11, N13, N16, N19, N20, P2, P17, P18, P20, P26, Tue, N26, G11, P11, 
\item[tunduṃ] N23
\item[tunda] N12, N17 P16, 
\item[tunde] C1, C4, P21, P23,  
\item[tuṃḍe(?)] N25, V5
\item[tundat] B2, J10, J17, P8, P10, R1, 
\item[tundet] J15, 
\item[truṃdaṃ] N21, P15, 
\item[tuda] W4
\item[tudaṃ] B1, C3, C6, J1, J5, J12, P6, P7, P12, P13, P14, P27, 	
\item[tundan] J4,
\item[tudan] Ba, Bo1, J11, P9, P25,  
\item[tudat] N7, P5, P24, 
\item[tulyaṃ] Ko,
\item[tujyaṃ] V19
\item[t(?)uddhaṃ] N22
\item[puṃda] N24
\item[saṃdada] J2
\item[(illegible/unavailable)] Bo3, C5, N6, P22, R5, T
  \begin{description}

    \end{description}
 \end{marma}

 \begin{marma}[hp02_35a]
\sthana{2.35a}
\item[pāvanādi] V1
\item[pāvanāni] N18,
\item[pāvanāgniḥ] N25
\item[pāvanādi] Ko,
\item[pāvakādi] C3, J10, J16, N10, Ba3, R1, 
\item[pāvakādī] J12,
\item[pāvakāgni] N26
\item[pācakādi] C2, C9, J17, LD1, LD2, N7, N9, N14, N17, P1, P5, P8, P9, P10, P24, W4, V16, V17, N20, V14, V21, V18, V13
\item[pācatādi] Bo2, 
\item[pācanādi] A1, B1, B2, B3, Ba, Ba5, Bo1, C6, C8, J4, J5, J7, J8, J11, J14, N12, N13, P6, P7, P14, P17, P25, P27, V19, V10, N19, N21, P2, P12, P13, P18, P20, V3, Tue, N23, V6, V7, V20, V15, V28, V22, G11, G1
\item[pācanādīn P26, 
\item[pācanāgni] C1, C4, J1, J3, J6, J13, N8, N11, N16, P16, P23, 
\item[pācanāgnī] P21,
\item[pācakāgniḥ] J15
\item[pācakāgni] V2
\item[pācakāri?] J2,
\item[pācanānī] P11, P15, 
\item[pācanāṃnī] N24
\item[kārkādeśaṃ] N22
\item[(illegible/unavailable)] Bo3, C5, N6, P22, R5, V5, V12, T
  \begin{description}

    \end{description}
 \end{marma}

 \begin{marma}[hp02_35c]
\sthana{2.35c}
\item[apa śoṣaṇī ca] Ba5,
\item[mala śoṣiṇī va] V1,
\item[mala śoṣaṇī ca] B2, C3, J8, LD1, N9, N17, N20, P1, P5, P10, P16, P24, W4, V16, V14, V3, V7, V18, V12
\item[maya śoṣaṇī ca] V21, V15, V28, V13
\item[maya śodhaṇī yā] N23
\item[maya śodhaṇī cā] V6, 
\item[mala śoṣaṇīva] J10, J17, P9,
\item[mala śoṣaṇī vaḥ] J16,
\item[doṣāmaya śoṣaṇī ca] C1, C2, C4, C6, C8, C9, J3, J4, J11, J13, J14, Ko, N12, N13, N18, N21, P2, P6, P7, P11, P12, P13, P14, P15, P17, P18, P20, P21, P23, P27, Tue, N25, V22, V2, G11, Ba3
\item[doṣāmaya śauṣaṇī caḥ] J15, 
\item[doṣamala śoṣaṇī ca] P8,  
\item[doṣāmala soṣaṇi ca] N14, R1,  
\item[doṣān mala śoṣaṇī ca] J12, N10, 
\item[doṣāmaya śoṣaṇīya] J1,
\item[doṣāmaya śoṣaṇīyaṃ] J2
\item[doṣāmaya śoṣiṇī ca] A1, B3, Ba, Bo1, J6, N8, N11, N16, V19, N19, P25, 
\item[doṣāmaya śoṣiṇāṃ ca] Bo2
\item[doṣāmaya śodhinī ca] J7, 
\item[doṣāmaya śodhanī ca] P26,
\item[doṣābhayahā soṣaṇī ca] J5,
\item[dauṣāmaya śoṣaṇī ca] LD2,
\item[doṣāmaya goṣaṇī ca] N24
\item[mala śoṣiṇī ca] V17, N7, N26,
\item[doṣāṇapi śoṣaṇā ca] B1,
\item[doṣāmaya śoṣaṇī ca] V10
\item[doṣapra śoṣiṇī] V5
\item[doṣānala śoṣaṇī bahava] N22
\item[(illegible/unavailable)] Bo3, C5, N6, P22, V20, R5, T, G1
  \begin{description}

    \end{description}
 \end{marma}

 \begin{marma}[hp02_35d]
\sthana{2.35d}
\item[mūlam iyaṃ hi] J14, P15, V1
\item[mūlam idaṃ hi] Ko, J4, J11, N19
\item[mau.riyaṃ hi] G1
\item[mūlam ayaṃ ca mauliḥ] N22
\item[maulir iyaṃ hi] A1, C1, C4, J3, J5, N8, N11, N16, P6, P20, P23, P27, V19, V21, N25, V15, V2
\item[maulir iyaṃ] J1,
\item[maulir iyaṃ ca] Ba, Ba5, Bo1,Bo2, C6, C8, J6, J7, N12, N13, N21, Tue, N23, P7, P11, P13, P14, P16, P18, P21, P25, V6, V7, V20, V28, V22, G11
\item[saulir iya ca] N24
\item[maulir ayaṃ hi] J2
\item[sau jayatīha] J10, J17, N17, W4, N18, N20, N26
\item[jayatīha] B1, B3, C2, C3, C9, J8, J12, J13, J15, J16, LD1, LD2, N7, N9, N10, N14, P1, P2, P5, P8, P9, P10, P12, P17, P24, P26, R1, V3
\item[sauryakarīha?] B2
\item[(illegible/unavailable)] Bo3, C5, N6, P22, V16, V17, V10, V14, V5, V18, V12, V13, T, Ba3, R5,
  \begin{description}

    \end{description}
 \end{marma}

 \begin{marma}[hp02_38a]
\sthana{2.38a}
\item[prāṇāyāmena vai] Ko, J4, J11, J14, V1, N19, P15, V15
\item[prāṇāyāmair eva] B1, B2, B3, Ba, Ba5, Bo1, Bo2, C1, C2, C4, C6, C8, J1, J3, J5, J6, J7, J12, J13, J15, LD2, N1, N3, N4a, N5, N8, N11, N12, N13, N16, N18, N22, P2, P7, P11, P12, P14, P17, P18, P20, P21, P23, P25, V3, Tue, N23, N24, N25, V6, V7, V5, V28, V22, V2, V12, V13, G11, G1
\item[prāṇāyāmaiḥ sukhāt] A1, C3, C9, J10, J17, LD1, N2, N9, N10, N17, P1, P5, P8, P9, P10, P24, P26, R1, Ba3
\item[prāṇāyāme sukhāt] J16,
\item[prāṇāyāmais tu sā sarvaiḥ] P16,
\item[prāṇāyāmaiḥ sukhāḥ] N26
\item[prāṇāyāmair mu?khāṃ] N20
\item[prāṇāyāmena sarvepi] N21, P6, P13, P27, 
\item[praṇāyāme kṛte samyak] J2
\item[prāṇāyāme śukhā] N14, V18
\item[prāṇāyāme tataḥ] V16, V17, V10, V14, V21, V20
\item[praṇāyāmāt sukhāt] N7
\item[(illegible/unavailable)] Bo3, C5, N6, P22, R5, T
  \begin{description}

    \end{description}
 \end{marma}

 \begin{marma}[hp02_38b]
\sthana{2.38b}
\item[malā api] J4, J11, V19, N21, P6, P13, P15, P27, V15, V28
\item[malād api] Ko,
\item[malā iti] Ba, Ba5, Bo1, Bo2, C1, C6, C8, J1, J2, J3, J5, J6, J7, LD2, N3, N4a, N5, N8, N11, N13, N16, P7, P11, P14, P18, P21, P23, P25, V10, N18, N19, P20, Tue, N23, N24, N25, V6, V7, V22, G11, G1
\item[malā itiḥ] J15,
\item[malā ime] J14, 
\item[malāśayaḥ] J10,
\item[malāśayāḥ] A1, B1, B3, C2, C9, J13, LD1, N1, N2, N10, N17, P5, P10, P12, P17, P24, P26, R1, V17, V14, V20, V18, V13
\item[malākulaṃ] B2, N22
\item[malākulāḥ] N20, N26
\item[malāśayā] C3, C4, J16, J17, N9, P1, P2, P8, P9, P16, V16
\item[marāśayā] N14
\item[malāsayā] J12, V21, V3, V2
\item[malāśrayāḥ] N7
\item[malāyinaḥ] N12
\item[(illegible/unavailable)] Bo3, C5, N6, P22, P24, V5, V12, T, Ba3, R5,
  \begin{description}

    \end{description}
 \end{marma}

 \begin{marma}[hp02_39ab]
\sthana{2.39ab}
\item[brahmādayo 'pi tridaśāḥ pavanābhyāsatatparāḥ] A1, B2, B3, Ba, Bo1, C1, C4, C6, C8, C9, G1, J1, P11, P12, P13, P14, P15, P16, P18, P20, P27, P21, P23, P25, P26,
\item[brahmādayo 'pi tṛdaśāḥ pavanābhyāsatatparāḥ] P17,
\item[ṣaṭkarmayogam āpnoti pavanābhyāsatatparāḥ] B1, C2, 
\item[ṣaṭkarmayogam āpnoti pavanābhyāsatatparaḥ] C3, R1,
\item[(illegible/unavailable)] C5, P22, P24, R5, T
  \begin{description}

    \end{description}
 \end{marma}

\begin{marma}[hp02_39cd]
\sthana{2.39cd}
\item[abhūvannantakabhayāt tasmat pavanam abhyaset] G1,
\item[abhūvannantakabhayāt tasmāt pavanam abhyaset] A1, Ba, Bo1,C1, C6, C8, P11, P13, P14, P15, P18, P20a, P27,
\item[abhūvannātankabhayāt tasmāt pavanam abhyaset] B1,
\item[abhūvannatakabhayāṃ tasmat pavanam abhyaset] P26,
\item[abhavannantakabhayāt tasmat pavanam abhyaset] P25,
\item[abhūvan nanekabhayāt tasmat pavanam abhyaset] P16,
\item[te ca siddhiṃ gatāḥ sarve tasmat pavanam abhyaset] P12,
\item[tena siddhi gatās te ca tasmat pavanam abhyaset] P21, P23, 
\item[tena siddhiṃ gatā yasmāt tasmāt pavanam abhyaset] C4,
\item[tena siddhiṃ gatās te ca tasmat pavanam abhyaset] B2, C9,  J1, 
\item[tena siddhi gatāḥ satyaḥ tasmat pavanam abhyaset] P20,
\item[naiva siddhiṃ gatāḥ sarve tasmat pavanam abhyaset] B3, P17,
\item[sumanskāṃtako bhavyaḥ tasmat pavanam abhyaset] R1,
\item[samanskāṃtako bhavyaḥ tasmāt pavanam abhyaset] C2,
\item[samanskoraho bhavyaḥ tasmāt pavanam abhyaset] C3,
\item[(illegible/unavailable)] C5, P22, P24, R5, T
  \begin{description}

    \end{description}
 \end{marma}


 \begin{marma}[hp02_40b]
\sthana{2.40b}
\item[cirāśrayam] LD2,
\item[nirāśrayam] A1, Ba, C1, C3, C6, C8, C9, J2, J4, J5, J7, J8, J10, J11, J15, J17, LD1, V1, W4, N3, N9, N10, V17, N20, N22, P1, P5, P6, P8, P9, P10, P11, P15, P20a, P24, P25, P27, V21, V3, N23, N26, R1, V6, V15, V28, V12, V13, G1
\item[nirāśrayaḥ] J12,
\item[niśaṃśrayaṃ] G11
\item[nirodhā vādham] V8
\item[nirāśayam] Bo1,
\item[nirāsnayām] J16,
\item[nirāmayam] B2, Bo2, C4, J1, J3, J6, J14, Ko, N17, V19, N2, N5, N7, N8, N11, N12, N14, N16, V16, N18, N19, N21, N24, N25, P7, P13, P16, P20, P21, P23, P26, V20, V5, V2, V18, Ba3,
\item[nirākulam] B1, B3, Ba5, C2, J13, N1, N4a, N13, P2, P12, P14, P17, P18, V10, Tue, V7, V22
\item[(illegible/unavailable)] Bo3, C5, N6, P22, P24, R5, V14, T
  \begin{description}

    \end{description}
 \end{marma}

 \begin{marma}[hp02_40c]
\sthana{2.40c}
\item[akṣir?] C8,
\item[īkṣed] Ba, J5, P25,
\item[īkṣet] C6, V13, G1
\item[īkṣe] C1, J7, P7, 
\item[īkṣye] P11, 
\item[paśyed] A1, Bo1, C9, J4, J8, J10, J11, J14, J15, J16, J17, Ko, LD1, N17, P15, V1, W4, N9, V8, N10, P5, P8, P10, P24, P26, R1, V16, V17, N20, N22, V21, V3, N26, V15, V12, Ba3
\item[paśyer] P1,
\item[paśyad] J12,
\item[paśyet] V20
\item[paśyeta] LD2,
\item[paśye] N19, N24
\item[paśe] C3,
\item[paśed] P9, 
\item[paśyan] N7
\item[paśca] P6, 
\item[pasya?] N14  
\item[dṛṣṭir] B3, Ba5, C2, C4, J6, V19, N1, N2, N4a, N11, N16, P2, P12, P14, P17, P18, P20, P23, V10, N21, N25, V7, V5
\item[dṛṣṭi] B1, B2, Bo2, J1, J2, J3, J13, N5, N8, N12, N13, N18, P13, P21, Tue, V28, V22, V2 
\item[vīkṣed] N3, G11
\item[iched] N23
\item[(illegible/unavailable)] Bo3, C5, N6, V14, V6, V18, T, P16, P22, P24, P27, R5,
  \begin{description}

    \end{description}
 \end{marma}

 \begin{marma}[hp02_43a]
\sthana{2.43a}
\item[vidhānajñā] C9, J5, J10, J11, Ko, V1, N2 ("vi" seems faded), N4a, N7, N10, N11, N12, N13, N14, N16, P6, P8, P9, V16, V17, V10, N19, N20, N21, V14, V3, Tue, N23, N26, V6, V7, V15, V22, V18, V13, G11, G1
\item[vidhānajñāḥ] Ba3
\item[vidhānaḥjñā] J2
\item[vidhānajñe] V2
\item[vidhanajñā] N18
\item[vidhaustajñā] V20
\item[vidhānujñaś] J1,
\item[vidhānataś] P23, 
\item[vidhānajñāś] Bo2, C3, C8, J4, J6, J7, J12, P5, P13, P14, P15, P24, P25, P27, 
\item[vidhānajñaś] B1, B3, Ba, Ba5, Bo1, Bo2, C1, C2, J3, J16, N3, N24, P2, P11, P12, P17, P18, P21, 
\item[vidhānajña] N22, V12
\item[vidhānejñaś] N8
\item[vidhijñās tu] C6,
\item[vidhānajño] C4, J15, J17, LD1, LD2, N5, N17, N25, P1, P10, P20, V28
\item[vidhānajñoś] J13, 
\item[viddhānajño?] R1,
\item[vividhānajño] B2, 
\item[vidhāna?ī] N9, V8
\item[vidhānajñaiś] N1 (unsure about "ai" matra)
\item[vidhistajñāś] P26,
\item[vidhiṃstajñaś] A1
\item[vidhiṃstajñaiś] P7,
\item[(illegible/unavailable)] Bo3, C5, N6, J14, V21, V5, T, P16, P22, P24, R5,
  \begin{description}

    \end{description}
 \end{marma}

 \begin{marma}[hp02_44cd]
\sthana{2.44cd}
\item[bhastrakā bhrāmarī vedhā kevalaṃ cāṣṭakumbhakāḥ] J2
\item[bhastrikā bhramarikā mūrcchā kevaleṣṭā ca kumbhakā] G1
\item[bhastrikā bhrāmarī mūrcchā kevalī cāṣṭakumbhakāḥ] V1, V16, V17, 
\item[bhastrikā bhrāmarī mūrcchā kevala cāṣṭakumbhakāḥ] V20, V18, V12, V13, G11, Ba3
\item[bhastrikā bhrāmarī mūrcchā vinītyaṣṭakumbhakāḥ] V10
\item[bhastrikā bhrāmarī mūrcchā vinītyaṣṭakumbhakān] V28
\item[bhastrikā bhrāmarī mūr??ā kevalaś cāṣṭakumbhakā] N1,
\item[bhastrīkā bhrāmarī mūrchā kevalaś cāṣṭakumbhakāḥ] C9, J13, J15, LD1, LD2, N21
\item[bhastrīkā bhrāmarī mūrchā kevalaś cāṣṭakumbhakaḥ] P12, 
\item[bhastikā bhrāmarī mūrchā kevalaś cāṣṭakumbhakaḥ] P26,
\item[bhaṃstrikā bhramarī mūrchā kāṃbalā atha? kumbhakāḥ] J5
\item[bhaṃstrīkā bhramarī mūrchā kevala cāṣṭakumbhakāḥ] N22, V21, V3
\item[bhastṛ(?)kā bhrāmarī mūrcchā kevalaś cāṣṭakumbhakāḥ] N26
\item[bhastrikā bhrāmarī mūrchā kaivalaś cā sa kumbhakaḥ] B3,
\item[bhastrikā bhramarī mūrcchā cāṣṭakumbhakāḥ sahita?] N2 (after exchange of words' positions)
\item[bhastrikā bhramarī mūrdhā kevalaś cāṣṭakumbhakāḥ] C2,
\item[bhastrikā bhramarī mūrchā kevalaś cāṣṭakumbhakāḥ] A1, B1, C3, J7pc, J10, J12, J16, J17, N3, N7, N9, N10, N17, P1, P2, P6, P9, P10, P17, P20, P27, R1, 
\item[bhastrikā bhramarī mūrchā kevalaś caṣṭakumbhakāḥ] P5, P24, 
\item[bhastrikā bhrāmarī mūrchā kevalaś cāṣṭakumbhakāḥ] N20, P8, 
\item[bhastrikā marī mūrchā kevalaṃ cāṣṭakumbhakāḥ] B2,
\item[bhastrikā bhrāmalī mūrchā kevalaś cāṣṭakumbhakā] N14
\item[gachati tiṣṭto kāryā mujjayākhyāṃ cākumbhakāḥ] V8
\item[bhastrikā bhramarī mūrchā kevalaś cāṣṭakumbhakā] N9
\item[bhastrakā bhramarī mūrchā kevalāś cāṣṭakumbhakā] P13,
\item[bhastrikā bhrāmarī mūrchā plāvanī saṣṭhakumbhakāḥ] N4a
\item[bhastrikā bhramarī mūrcchā plāvanī tvaṣṭakumbhakāḥ] N12, V6, V7
\item[bhastrikā bhramarī mūrcchā plāvanī tyaṣṭakumbhikā] P11, 
\item[bhastrikā bhramarī mūrcchā pāvanīty eṣakumbhakāḥ] Ko,
\item[bhastrikā bhramarī mūrcchā pāvanīty aṣṭakumbhakān] J4,
\item[bhastrikā bhrāmarī mūrchā plāvinītyaṣṭakumbhakāḥ] Ba5, Bo1, C1, N13, P14, P18, V14, Tue, V15, V22
\item[bhastrīkā bhrāmarī mūrchā plāvinītyaṣṭakumbhakāḥ] P25,
\item[bhastrikā bhrāmarī mūrchā prāvatītyaṣṭakumbhakaḥ] N23
\item[bhastrikā bhrāmarī mūrchā pratyahītyaṣṭakumbhakāḥ] P7,  
\item[bhastrikā bhramarī mūrchā plāvanītyaṣṭakumbhakāḥ] Ba, C8, J7, P15,  
\item[bhastrikā bhrāmarī mūrcchā pālāvanīty aṣṭakumbhakāḥ] J11,
\item[bhastrikā bhramarī mūrchā pratvanītyaṣṭakumbhakāḥ] C6
\item[bhastrikā bhrāmarī mūrchā sahita cāṣṭakumbhakā] P23,
\item[bhastrikā bhramarī mūrchā sahitāś cāṣṭakumbhakāḥ] C4, N5
\item[bhastikā bhrāmarī mūrchā sahitāś cāṣṭakumbhakam] N25, V5
\item[bhastikā bhrāmarī mūrchā sahitāś cāṣṭakumbhakān] P21, 
\item[bhastrikabhrama ma mūrchā saṃhataṃ cāṣṭakumbhakā] N8
\item[bhastrikā bhramarī mūrchā saṃhataṃ cāṣṭakumbhakāḥ] N11(visarga added later), N16 ("saṃhitaṃ"),
\item[bhastrikā bhramarī mūrchā saṃhitaṃ cāṣṭakumbhakāḥ] Bo2, J3, J6, V2
\item[bhastrikā bhramarī mūrchā saṃprataṃ cāṣṭakumbhakān] N18
\item[bhadrikā? bhramarī mūrchā saṃprataṃ rāṣṭakumbhakāṃ] J1,
\item[bhastrikā śītalī bhramarī mūrchā plāvanītyaṣtakumbhakāḥ] N19 
\item[sastrikā bhramarī mūrchā kavalaś cāṣṭakumbhakā] N24
\item[(illegible/unavailable)] Bo3, C5, N6, J14, T, P16, P22, R5,
  \begin{description}

    \end{description}
 \end{marma}

 \begin{marma}[hp02_46]
\sthana{2.46}
\item[om.] V19,
\item[adhas tāt kuñcanenāśu kaṇṭhasaṃkocane kṛte | madhye paścimatānena syāt prāṇo brahmanāḍīkaḥ ||] Ba5
\item[adhastāt kuñcanādāśu kaṇṭhasaṅkocane kṛte | madhye paścimatāṇena syāt prāṇo brahmanāḍigaḥ ] G1, P11, P18, 
\item[adhastāt kuñcanādāśu kaṇṭhasaṃcane kṛte | madhye paścimatāṇena syāt prāṇo brahmanāḍigaḥ ] P25,
\item[adhas tāt kuñcanenāśu kaṇṭhasaṃkocane kṛte | madhye paścimatānena syāt prāṇo brahmanāḍīkā ||] P7, 
\item[athastātkuñcanenāśukarṇasaṅkocane kṛte | madhye paścimatānena syāt prāṇo madhyanāḍigaḥ] Ba3
\item[adhas tāt kuñcane vāyuḥ kaṃṭhasaṃkocane kṛte | madhye paścimatāṇena syāt prāṇo brahmanāmīgaḥ ||] Ko,
\item[adhas tāt kuñcanenāśu kaṇṭhasaṃkocane kṛte | madhye paścimatānena syāt prāṇo brahmanāḍīkagḥ ||] V22
\item[adhas tāt kuñcanenāsuṃ kaṇṭhasaṃkocane kṛte | madhye paścimatānena syāt prāṇo brahmanāḍīkāṃ ||] C6
\item[adhas tā kuñcakīnālau kaṇṭhasaṃkocane kṣate | madhyaṃ paścimatāṇena syāt prāṇo brahmanāḍīkaḥ ||] J2
\item[adhas tāt kuñcanenāśu kaṇṭhasaṃkocane kṛte | madhye paścimatānena syāt prāṇo brahmanāḍigaḥ ||] A1, Bo1, C8, J10, J11, V1, N3, N13, P14, P20, Tue, V7, V20, V15, V28, V12, G11
\item[adhastā kuñcanenāśu kaṇṭhasaṃkocane kṛte | madhye paścamatānena syāt prāṇo brahmanāḍigaḥ ||] P21, 
\item[adhastvākuñcanenāśu kaṇṭhasaṃkocane kṛte | madhye paścimatānena syāt prāṇo brahmanāḍigaḥ ||] C9
\item[adhas tāt kuñcanenāśu kaṇṭhasaṃkocane kṣate | madhye paścimatānena syāt prāṇo brahmanāḍigaḥ ||] C1,
\item[adhastāt kuñcanenāśu kaṇṭhasaṃkocane kṛte | madhyapaścimatānena syāt prāṇo brahmanāḍigaḥ ||] C3, LD1, N12, N26
\item[adhastā kuñcanenāśu kaṇṭhasaṃkocane kṛte | madhyapaścimatānena syāt prāṇo brahmanāḍigaḥ ||] B2, J15, V10, V21
\item[adhasthāt kuñcanenāśu kaṇṭhasaṃkocane kṛte | madhye paścimatānena syāt prāṇo brahmanāḍigaḥ ||] J7,
\item[adhasthāt kuñcanenāśu kaṇṭhasaṃkocane kṛte | madhye paścimatānenya syāt prāṇo brahmanāḍigaḥ ||] P10,
\item[adhas thāt kuñjakenāśu kaṇṭhasaṃkocane kṛte | madhye paścimatāṇāsthā syāt prāṇo brahmanāḍīgataḥ ||] P15, 
\item[adhastvākuñcanenāśu kaṇṭhasaṃkocane kṛte | madhya paścimatānena syāt prāṇo madhyanāḍigaḥ ||] P1, P9, 
\item[adhastāt kuñcanenāśu kaṇṭhasaṃkocane kṛte | madhyapaścimatānena syāt prāṇo madhadhyanāḍigaḥ ||] J6, J17, P2,
\item[adhastāt kuñcanenāśu kaṇṭhasaṃkocane kṛte | madhye paścimatānena syāt prāṇo madhadhyanāḍigaḥ ||] Bo2, N10, V16, V17, V14, V3, V6
\item[addhastāt kuñcanenāsu kaṃṭhasaṃkocane krate | madhye paścimatānena syāt prāṇo madhadhyanāḍigaḥ ||] R1, 
\item[adhastā kuñcanenāśu kaṇṭhasaṃkocane kṛte | madhye paścimatānena syāt prāṇo madhadhyanāḍigaḥ ||] J3,
\item[adhastāt kuñcanenāśu kaṇṭhasaṃkocane kṛte | madhye paścimatānena syāt prāṇo madhyanāḍikaḥ ||] P17, 
\item[adhastā kuñcanenāśu kaṇṭhasaṃkocane kṛte | madhya paścimatānena syāt prāṇo madhadhyanāḍigaḥ ||] N17,
\item[adhastā kuñcanenāśu kaṇṭhaśaṃkocane kṛte | madhye paścimatānena syāt prāṇo madhadhyanāḍiga ||] J12,
\item[adhas tāt kuñcanenāśu kaṇṭhasaṃkocane kṛte | madhye paścimatāna syāt prāṇo brahmāṇḍanāḍigaḥ ||]  Ba, N4a
\item[adhastākuñcanenāśu kaṇṭhasaṃkocane kṛte | madhyapaścimatānena syā? prāṇo brahmanāḍikaḥ ||]  N5
\item[adhastāt kuñcanenāśu kaṇṭhasaṃkocane kṛte | madhyapaścimatānena syāt prāṇo brahmanāḍikaḥ ||] C4
\item[adhastākuñcaṇeṇāśu kaṇṭhaśaṇkocane kṛ(?)te | madhye paścamatānena syāt prāṇo madhye nāḍikā ||] N14
\item[adhastāt kuñcanenaiva kaṇṭhasaṃkocane kṛte | madhye paścimatānena syāt prāṇo brahmanāḍikaḥ ||] N1,
\item[adhastāt kuñcanenaiva kaṇṭhasaṃkocane kṛte | madhye paścimatānena syāt prāṇo brahmanāḍigaḥ ||] P13, P27, 
\item[adhastāt kuñcane caiva kaṇṭhasaṃkocane krate | madhye paścimatānena syāt prāṇo brahmanāḍigaḥ ||] P6,
\item[adhastāt kuñcanenaiva kaṇṭhaṃ saṃkocane kṛte | madhye paścimatānena syāt prāṇo brahmanāḍikaḥ ||] B3
\item[adhas tāt kuñcanenāśu kaṇṭhasaṃkocane kṛte | madhye paścimatāṇena syāt prāṇo madhyanāgiḍaḥ ||] J4,
\item[adhas tāt kuñcanenaiva kaṇṭhasaṃkocane kṛte | madhye paścimatānena syāt prāṇo madhyanāgiḍaḥ ||] J4,
\item[adhas tāt kuñcanenaiva kaṇṭhasaṃkocane kṛte | madhya paścimatānena syāt prāṇo madhyanāgiḍaḥ ||] P12,
\item[adhamān kuñcanenāśu kaṇṭhasaṃkocane kṛte | madhye paścimatāṇena syāt prāṇo brahmanāḍigaḥ ||] J5,
\item[adhastāt kuñcanenaiva kaṇṭhaṃ saṃkocane kṛte | madhye paścimatānena syāt prāṇo brahmanāḍigaḥ ||] C2
\item[adhastāt kuñcanenaiva kaṇṭhasaṃkocane kṛte | madhyapaścimatānena syāt prāṇo brahmanāḍigaḥ ||] B1
\item[adhastāt kuñcanenaiva kaṇṭhasaṃkocane kṛte | madhyapaścimatānena syāt prāṇo brahmanāḍigaḥ ||] N1
\item[adhasthāt kuñcanenaiva kaṇṭhasokocane kṛte | madhyapaścimatānena syāt prāṇo brahmanāḍigaḥ ||] N21
\item[adhastāt kuñcanenāsu kaṇṭhasaṃkocane kṛte | madhyapaścimatānena syāt prāṇo madhyanādigaḥ ||] J16, 
\item[adhastā kuñcanenāsu kaṇṭhasaṃkocane kṛte | madhye paścimatāṇena syāt prāṇo madhyanāḍigaḥ ||] LD2,
\item[adhastāt kuñcānenāśu kaṇṭhasaṃkocane kṛte | madhye paścimatānena syāt prāṇo madhyanāḍigaḥ ||] P26,
\item[adhaścā kuñcanenāsu kaṇṭhasaṃkocane kṛte | madhye paścimatānyena syāt prāṇo madhyanāḍigaḥ ||] P5, P24, 
\item[aṃtastvā kuñcanenāsu kaṇṭhasaṃkocane kṛte | madhyapaścimatānena syāt prāṇo madhyanāḍigaḥ ||] P8, 
\item[adhastvā kuñcanenāsu kaṇṭhasaṃkocane kṛtaṃ | madhyapaścimatānena syāt prāṇo madhyanāḍigaḥ ||] N7, V13
\item[adhastvā kuñcanenāsu kaṇṭhasaṃkocane kṛte | madhyapaścimatānena syāt prāṇe brahmanāḍigaḥ ||] N25
\item[adhastāt kuñcanenāśu kaṇṭhasaṃkocane kṛte | madhyapaścimatānena syāt prāṇo madhyanāḍigaḥ ||] N16
\item[adhastāt kuñcanenāśu kaṇṭhasaṃkocane kṛte | madhye paścimatānena syāt prāṇo madhyanāḍigaḥ ||] J1, N18, N19
\item[adhastāt kuñcanenāsu kekatvaṃ saṃkocane krate madhye paścimatānena syāt prāṇo vasta(?)nāḍ?gaḥ ||] N9
\item[adhastāt kuñcatenāśu kaṇṭhasaṃkocane kṛte | madhyapaścimatānena syāt prāṇo brahmanāḍitāḥ ||] N20
\item[adhas tā kucanenāmaḥ kaṃcasaṃkocane kṛte | madhye paścimatānena prāṇasyā brahmanāḍīkaḥ ||] N22
\item[adhas ta kuñcanenāśu kaṇṭhasaṃkocane kṛte | madhyapaścimatānena syāt prāṇo brahmanābhirāt ||] N23
\item[adhas tāt kuñcanaunāśu kaṇṭhakocane kṛte | madhye paścimatāne tasyāt pāno madhyanāḍigaḥ ||] N24
\item[adhas tāt kuñcanaunāśu kaṇṭhakocane kṛte | madhye paścimatāne tasyāt prāno madhyanāḍigaḥ ||] V18
\item[(illegible/unavailable)] Bo3, C5, J14, N6, N2, N8, N11, V5, V2, T, P16, P22, P23, R5,
  \begin{description}

    \end{description}
 \end{marma}

 \begin{marma}[hp02_48b]
\sthana{2.48b}
\item[āsthāpya] V1, V19
\item[sansthāpya] P26, P27, V8
\item[utthāpya] A1, B1, B2, B3, Ba, Ba5, Bo1, Bo2, C2, C3, C4, C6, C8, C9, J1, J3, J6, J7, J8, J10, J11, J13, J17, LD1, N17, W4, N2, N3, N4a, N5, N8, N9, N11, N12, N13, N16, P1, P2, P5, P6, P7, P8, P11, P12, P13, P14, P15, P17, P18, P20, P24, P25, V16, V17, V10, N18, N21, V14, V21, V3, Tue, N23, N24, N26, V6, V7, V15, V28, V22, V18, V12, V13, G11
\item[utthāya] P23,
\item[uddhāpya] Ba3
\item[ūtthāpya] J16,
\item[uthāpya] Ko, R1, 
\item[usthāpya] J2, J12, 
\item[udhyāpya] N19
\item[ur(?)thāpya] N14
\item[utthāya] C1, N7, N20, P21, 
\item[utprāpya] N22
\item[uschāpya] N1, 
\item[usthāpya] J4,
\item[uchāpya] J5, J15, N10, P10, 
\item[utchāpya?] LD2, P9, 
\item[(illegible/unavailable)] Bo3, C5, J14, N6, N25, V20, V5, V2, T, G1, P16, P22, R5,

  \begin{description}

    \end{description}
 \end{marma}

 \begin{marma}[hp02_49b]
\sthana{2.49b}
\item[virōdhāvadhi kumbhayet]
\item[nīrādhāvadhi kumbhayet] N23,
\item[nīrodhāvadhi kambhayet] Ba,
\item[nirodhābadhi kumbhayet] C2,
\item[nirodhāvadhi kumbhayet] A1, B1, B3, C3, C8, C9, J15, J16, J17, N4a, N10, N13, P1, P2, P5, P8, P9, P10, P11,  P12, P14, P17, P18, P24, P25, P27, Tue, V21, G11, Ba3
\item[nirodhāccādhi kumbhayet] C6,
\item[nirodhāvadhi kumbhakaḥ] Bo1, 
\item[nīrodhāvadhi kumbhata] R1,
\item[nīrodhavadhi kumbhayet] P13, 
\item[nīrovadhi kumbhayet] P15,
\item[nīrodhān adhikumbhayet] N20, 
\item[nirodho parikumbhake] G1
\item[nīrodhāvadhi kumbhayat] N3, N7, N12, P20a,   
\item[nīrodhāvadhi kumbhavat] N21, 
\item[nīrodhāvadhi kumbhakaṃ] J1, N2, N11,  
\item[nīrodhāvadhi kumbhake] N18, 
\item[nīrodhānadhi kumbhayet] N1, N26, V17
\item[nīrodhāvahi kuṃbhayet] N9, N24,
\item[nīrodhīvadhi kumbhayet] N19, 
\item[nirodhaṃ kumbhayet tataḥ] J13, 
\item[nirodhāścādhi kumbhayet] P7, 
\item[nīrodhāvadhikam bhavet] P26,
\item[śirodhāvadhi kumbhakaṃ] C1, C4, N5, N8, N16, P21, P23,  
\item[śirodhāvati kumbhakān] P6,
\item[śirodhyavadhi kumbhakaṃ] N25, P20, 
\item[śiro ‘pya?vadhi kumbhayet] N17,
\item[śiroyāvadti kumbhakāt] B2,
\item[sirodhāvadhi kumbhakāt] N22,
\item[(illegible/unavailable)] C5, J14, N6, N14, T, P16, P22, R5,

  \begin{description}

    \end{description}
 \end{marma}

\begin{marma}[hp02_53a]
\sthana{2.53a}
\item[jālodare] C3, C9,  
\item[jālodarerā] P24, 
\item[jalodar] A1,
\item[jalodara] C6,
\item[jalodarā] C1, C8, P13, P14, P15, P18, P20, P21, P23, P27, 
\item[jalodaro] B2,
\item[jalodāraṃ] P17,
\item[jalodaraṃ] B1, B3, Ba, Bo1, C2, C4, J1, P25, P26, 
\item[jālodared] R1,
\item[(illegible/unavailable)] C5, P16, P22, R5,

  \begin{description}

    \end{description}
 \end{marma}


\begin{marma}[hp02_55ab]
\sthana{2.55ab}
\item[śītkāṃ dadyāt sadā cakre ghraṇe caiva vijṛmbhikām]
\item[śītkāṃ dadyāt sadā vaktre ghraṇe naiva vijṛmbhikām] P14, P18, 
\item[śītkāṃ dadyāt sadā vaktre ghraṇe caiva vijṛmbhikām] C6, 
\item[chikkāṃ dadyāt sadā valke ghraṇe naiva vijṛmbhikām] Bo1, 
\item[sītkāṃ dadyāt sadā vaktre ghraṇe caiva vijṛmbhikām] P27,
\item[sītkāṃ dadyāt sadā vaktre ghraṇe naiva visarjjayet] B2,
\item[sātkāṃ dadyāt sadā vaktre prāṇe caiva vijṛmbhite] P13,
\item[satkāṃ dadyāt sadā vaktre prāṇe cātha vijambhakāṃ] P15,
\item[śītkāṃ kuryāt sadā vaktrāt ghraṇo caiva vijṛ-kām] A1,
\item[śītkāṃ kuryāt sadā vaktrāt ghraṇe naiva vijṛmbhakān] P26,
\item[sītkāṃ kuryāt tathā vaktre ghraṇe naiva vijṛmbhikām] C1,
\item[sītkāṃ kuryā tathā vaktre ghraṇe naiva vivarjayet] C4,
\item[sītkāṃ kuryāt tathā vaktre ghraṇe caiva vijṛmbhikām] P25,  
\item[śītkāṃ kuryāt tathā vaktre ghraṇe caiva vijṛmbhikām] Ba,  
\item[śītkāṃ kuryāt tathā vaktre ghraṇe caiva vijṛmbhikā] C8,  
\item[śītkāṃ kuryāt yathā vaktre prāṇaṃ caiva visarjjayet] P20, 
\item[śitkāṃ kuryāt sadā vaktre prāṇenaiva visarjayet] P21,
\item[śitkāṃ kuryāt tathā vaktre ghrāṇenaiva visarjayet] P23, 
\item[sitkāṃ kuryāt tayā vaktre ….nenaiva visarjayet] J1,
\item[kuṃbhaṃ kuryāt sadā vaktre grāṇenaiva visarjayet] B1, B3, C2,  
\item[kuṃbhaṃ kuryāt sadā vaktrāt grāṇe caiva vijṛmbhikām] C3,
\item[kuṃbhaṃ kuryāt sadā vaktrat ghrāṇenaiva vijaṃbhitaṃ] R1, C9,  
\item[(illegible/unavailable)] C5, P16, P22, P24, R5,

\item[]
\item[(illegible/unavailable)] 

  \begin{description}

    \end{description}
 \end{marma}

 \begin{marma}[hp02_54a]
\sthana{2.54a}
\item[śītkaraḥ] B1, C8, P17, 
\item[śītkaram] C6,
\item[śīkāra] J7,
\item[śītkāraḥ] C1, B3, J13, J15, P7, P20, P25, V17, V21 
\item[śītakāraḥ] N9, 
\item[śītakāra] P1, R1, 
\item[śitakāra] P6, 
\item[śītakārāī] LD1, 
\item[śīghrā] P11,
\item[śītalī] P15,  
\item[sītkara(ḥ)] Ba, Bo2, C2, C3, C4, J3, J4, N6, 
\item[sītkāraḥ] J6, N8, N11, N16, P2, P12, P13, 
\item[sītkāra] N17, P5, P24, 
\item[śītkārī] J16, 
\item[śītakārī] P8, P26, 
\item[śītakarī] P9,
\item[sītkāṃ] N5, N18, N22, Tue, N24, N25, V15, V28, V22, V10, G11
\item[sītkād] V14
\item[sitkā] N3, N14, 
\item[sītkā] Ko, N12, P27, 
\item[śītkāṃ] A1
\item[sītātkārī] N13,
\item[sītkaraḥ] N1,
\item[sītkāra] P23,
\item[sītkārā] P21,
\item[sītkarā] J17, 
\item[sītkārī] Ba5, Bo1, J5, J12, N2, N4a, N7, N10, P14, P18,    
\item[sītakārī] C9
\item[sītakarākarmma] J10, 
\item[śītikārī] LD2,
\item[sītalī] J2
\item[sikta?] J1,
\item[sītkīṃ] V19,
\item[sitkāṃ] J11, N19
\item[sinktā?] B2
\item[sātkāṃ] N21
\item[dhikt(?)āṃ] N23
\item[kumbhaṃ] V3,J8, N20,V14, V21, N26, V16, Ba3 
\item[kumbhakaṃ] V7, V20, V18, V12, V13, 
\item[kumbhakaḥ] V3
\item[(illegible/unavailable)] Bo3, C5, J14, P10, V1, V6, V5, V2, T, G1, P16, P22, R5,
  \begin{description}

    \end{description}
 \end{marma}


 \begin{marma}[hp02_55a]
\sthana{2.55a}
\item[cakram āsādya(ḥ)] A1, B3, Ba, Bo1, C2, V1, J8pc, J10, J17, N1, N7, N10, N28, V12,
 N21, P1, P2, P5, P8, P9, P12, P13, P17, P24, P25, P26, P27, V14, V20, V18, V16, Ba3
\item[cakramāsādyamān] 
\item[cakrasāmānya(ḥ)] B1, J8ac, J11, J13, N19, N24, P11, P15, P18, V3, V22
\item[cakrasammānyaḥ] Ba5, N4a, N13, P14, Tue, V10, G1
\item[cakrastamānyaṃ] J2
\item[cakramāsādyamān] G11
\item[cakram āsāya] N9,
\item[cakrasāmānyaṃ] LD2
\item[cakramādyāya] J12,
\item[cakrasāmānya] J4, J5,
\item[cakrasāmānyaḥ] N3, 
\item[cakraṃ samāsādya] N6, N14, N17, W4
\item[cakrasamāsādya] LD1,
\item[cakranāśāya] N20, N26, P10, V17
\item[cakranāsānyaḥ] N12, 
\item[cakrabhogyaś ca] B2,
\item[cakrasevyas tu] Bo2, C1, C4, C6, C8, J1, J3, J6, J7, J15, N2, N5, N8, N11, N16, P7, V19, N18, V P20, P21, P23, 21, N25, V15, V28, V2, V13, V21
\item[cakrasevyas ta] N23
\item[cakrabhogyas tu] N22, P6, 
\item[cakrābhimānyaḥ] Ko,
\item[vaktram āsādya] C3, C9
\item[cakramadyāya] J16,
\item[(illegible/unavailable)] Bo3, C5, J14, V6, V5, T, P16, P22, R1, R5,
  \begin{description}

    \end{description}
 \end{marma}


 \begin{marma}[hp02_56*1]
\sthana{2.55*1}
\item[satyāmūlena randhreṇa] C4, J6, J8pc, J10, J17, N5, N9, P1, P10, V21, N25
\item[satyāmūlena randhrena] J15,
\item[satyaṃ mūlena randhreṇa] P5, P24, 
\item[sadāmūlena randhreṇa] J3, J6pc
\item[satyāmūlena randhrenayaḥ] V18, V16, V21
\item[spṛśyamūlena randhreṇa] B1, B3, C2, J13, P2, P12, P17, 
\item[rasanātālumūlena] N6, N14, N17, 
\item[rasanāmūla randhreṇa] C9, 
\item[rasavatyāmūla randhre yaḥ] N20, N26
\item[]
\item[]
\item[(illegible/unavailable)] A1, B2, Ba, Ba5, Bo1, Bo2, Bo3, C1, C3, C5, C6, C8, J1, J2, J4, J5, J7, J11, J12, J14, J16, Ko, LD1, LD2, N1, N2, N3, N4a, N7, N8, N10, N11, N12, N13, N16, N18, N19, N21, N22, N23, N24, P6, P7, P8, P9, P11, P13, P14, P15, P18, P20, P21, P22, P23, P25, P26, P27, P28, R1, R5, W4, V3, V6, V20, V15, V5, V28, V22, V2, V12, V13, V10, V14, V17, G11, Tue, T, Ba3, G1, P16,
  \begin{description}

    \end{description}
 \end{marma}


 \begin{marma}[hp02_57a]
\sthana{2.57a}
\item[rasanātālumūlena] A1, B1, B2, B3, Bo1, C2, C4, C9, J4, J10, J12, J13, J15, J16, J17, Ko, LD1, N1, N2, N3, N6, N7, N10, N11, N12, N14, N17, N21, N22, N25, N26, P1, P2, P5, P6, P8, P9, P10, P11, P12, P13, P15, P20, P23, P24, P26, P27, R1, V1, V20, V15, V28, Ba3
\item[rasanātālumūle yaḥ] Ba,
\item[rasanātālumūlenaya] V14,
\item[rasanātālumūlena yaḥ] V2, V18, V12, V3, V16, V17, G11
\item[rasanātāgru mūlena yaḥ] G1
\item[rasanāttālumūlena] N20
\item[rasanātālumūlaṃ ca] J5, 
\item[rasanātāluyogena] Bo2, C6, C8, J3, J6, J7, N8, N16, N18, P7, V21, N23, V13
\item[rasanātālayogena] J1,
\item[rasanā tālamūlena] N5, N9, 
\item[rasanāṃ tālamūlena] J11
\item[rasanāṃ tālumūlena] P25,
\item[rasanāṃ tāluyogena] P21, 
\item[rasanānāṃ tālamūlena] N19
\item[jihvayā vāyum ākṛṣya] Ba5,
\item[satyāṃ mūlena randhreṇa] C3,
\item[]
\item[(illegible/unavailable)] Bo3, C1, C5, J2, J14, N4a, Tue, LD2, N13, N24, P14, P17, P18, P22, V6, V5, V22, V10, T, P16, R5,
  \begin{description}

    \end{description}
 \end{marma}

 \begin{marma}[hp02_58b]
\sthana{2.58b}
\item[pūrvavat kumbhakād anu] C4, C6, C8, J1, J4, J6pc, J7, J11, Ko, N2, N5, N8, N11, N21, P13, P20, P23, V14, V21, N3, N24, N25, P27, V6, V19, V15, V28, V2, Ba3, G1
\item[pūrvavat kuṃbhanādan] G11
\item[pūrvavat kūmbhakād anu] Bo2,
\item[pūrvavat kumbhayed anu] P26,
\item[pūrvavat kumbhanāda tu] P11,
\item[pūrvavat kumbhakād anmuṃ] Bo1,
\item[pūrvavat kumbhayed anu] A1, V20
\item[pūrvavat kumbhakād anū] N19
\item[pūrve ce kumbhanād agu] J5,
\item[pūrvavat kumbhakādane] N23, V7
\item[pūrvavat kumbhakeṣv anu] B2
\item[pūrvavat kumbhasādhanaṃ] B1, B3, Ba, Ba5, Bo3, C1, C2, C3, C9, J2, J3, J6, J10, J12, J13, J15, J16, J17, N4a, LD1, LD2, N6, N9, N10, N12, N13, N14, N17, N18, N22, P2, P5, P6, P10, P12, P24, P25, R1, V3, Tue, V22, V18, V12, V13, V3, V10, V16
\item[pūrvaṃ ca  kumbhasādhanaṃ] P1, P8, P9, 
\item[pūrvava kumbhasādhanaṃ] N7, 
\item[kākacaṃcusamannayāt] V17
\item[pūrvavat kumbhasādhayet] N1, P17, 
\item[pūrvakumbhakavad bhavet] P7,
\item[kākacaṃcusamunnayāt] N16, N26
\item[kākacaṃcusamunnarayāt] N20, 
\item[kākacaṃcusamanvayāt] P21, 
\item[(illegible/unavailable)] C5, J14, P14, P15, V5, T, P16, P18, P22, R5,
  \begin{description}

    \end{description}
 \end{marma}

\begin{marma}[hp02_62b]
\sthana{2.62b}
\item[kapāle ca samantataḥ] V1 
\item[kapāla pavanaṃ tataḥ] N6, 
\item[kapāle niḥsvanaṃ tathā] Ba,
\item[kapole niḥsvanaṃ tathā] P25, 
\item[kapole niḥsvanaṃ tadā] C1, 
\item[kapole nisvanaṃ tadā] J2, 
\item[kapole svāsanaṃ tataḥ] N26
\item[kapole svasanaṃ tataḥ] V18
\item[kapāla śvasamaṃ tataḥ] J4, P10, 
\item[kapāla śasanaṃ tataḥ] J15, 
\item[kapāle śasanaṃ tataḥ] Ba3
\item[kapāle śvasanaṃ tataḥ] J11, N3, P20a, V17 
\item[kapāle svaśanaṃ tataḥ] LD1, 
\item[kapāla śparśanaṃ tataḥ] J16, N9, N10, P5, P24, 
\item[kapole svaśanaṃ tataḥ] LD2, V16
\item[kapole śvāsanaṃ tataḥ] N20, 
\item[kapole svasanaṃ tataḥ] N7, P8, 
\item[kapole śvavaṇe tathā] B2,
\item[kapole śvavaṇe tataḥ] V13, V3
\item[kapāle sasvanaṃ tataḥ] P13, P15, P27, 
\item[kapole sasvanaṃ tataḥ] P9,
\item[kapolaṃ sasvaramarut] G11
\item[kapolaṃ sasvanaṃ tataḥ] P11, 
\item[kapola saśvanaṃ tataḥ] J5,
\item[kapole sca sanantataḥ] V3
\item[kayolesvaśanaṃtataḥ] N24
\item[kapāla pavanaṃ tata] N14, 
\item[kapāla sasvanaṃ tataḥ] N21, V12
\item[kapāla svasanaṃ tataḥ] N19, 
\item[kapālasya samaṃ tataḥ] C3
\item[kapālasyarśanaṃ tataḥ] J10, 
\item[kapāle sasvanaṃ tataḥ] Bo1, V14
\item[kapāle mukhano marut] Ko,
\item[kapālādhi sasvanaṃ tataḥ] V6
\item[kapāla pavanaṃ tataḥ] N17,
\item[kapāle susvanam] G1
\item[kapālasparśanaṃ tataḥ] C9, J17, P1,
\item[kapālasya śanaṃ tataḥ] J12,
\item[kapālādhi niḥsvanaṃ] C6,
\item[kapālavidhi sasvanaṃ] P26,
\item[kapālāvadhi sasvanaṃ] A1, Ba5, C8, J7, N4a, N12, N13, P7, P14, P18, Tue, V7, V20, V28, V22, V10
\item[kapālāvadhi sasanaṃ] V21
\item[kapālāvadhi pūrayet] B1, B3, Bo2, C2, C4, J1, J3, J6, J13, N1, N2, N5, N8, N11, N16, N18, N25, P2, P12, P17, P20, P21, P23, V15, V2
\item[kepālavadhi sasvanaṃ] N23, 
\item[kepālena samaṃ tataḥ] P6, 
\item[kepālenastha sarīrasthaṃ] N22,
 \item[(illegible/unavailable)] Bo3, C5, J14, V5, T, P16, P22, R1, R5,
  \begin{description}

    \end{description}
 \end{marma}

\begin{marma}[hp02_65cd]
\sthana{2.65cd}
\item[dhārayan nāsikāmadhyṃ aṃgulībhyāṃ tathā dṛḍham] N10,
\item[dhārayennasikā madhye aṅgulīnāṃ tathā dṛḍham] Ba3
\item[dhārayen nāsikāmadhye aṅgulībhyāṃ vinā dṛḍham] V1, V21
\item[dhārayen nāsikāmadhye aṅgulībhyāṃ tathā dṛḍham] LD1,
\item[dhārayen nāsikāmadhyaṃ aṅgulībhyāṃ tathā dṛḍham] A1, C9, P26, 
\item[dhāraye nāsikāmadhyaṃ maṅgulānāṃ tathā dṛḍham] J12,
\item[dhārayen nāsikāmadhaṃ aṅgulībhyāṃ tathā dṛḍham] V20
\item[dhārayen nāsikāmadhya aṅgulībhyāṃ tathā dṛḍham] J10,
\item[dhārayen nāsikāmadhye aṅgulībhyāṃ tathā dṛḍham] B2, C3, J17, N6, N14, N17, P5, P8, P10, P24, V18, V12, V16, V17, G1
\item[dhāraye nāsikāmadhye aṅgulībhyāṃ tathā dṛḍham] N7, 
\item[dhārayen nāsikāmadhyai aṅgulībhyāṃ tathā dṛḍham] J15,
\item[dhārayen nāsikāmadhye aṅgulībhyāṃ tadā dṛḍham] N9, P1, 
\item[dhārayen nāsīkāmadhye maṃgulībhā tathā dṛḍham] J16,
\item[dhārayen nāsīkāmadhye aṅgulībhyāṃ dṛḍham tathā] LD2,
\item[dhārayen nāsikāmadhye aṅguṣṭhābhyāṃ tathā dṛḍham] N20, N26,
\item[dhārayen nāsikāṃ madhye aṅgulībhyāṃ tathā dṛḍham] P9, 
\item[dhārayen nāsikābhyāṃatarjabhyāṃ tathā dṛḍham] R1,
\item[dhārayen nāsikāmadhyṃ tarjanībhyāṃ vinā dṛḍham] Ko,
\item[dhārayen nāsikāmadhyaṃ tarjanībhyāṃ vinā dṛḍham] J11, N3, 
\item[dhārayen nāsikāmadhye tarjanībhyāṃ vinā dṛḍham] C6, J5, N12, N18, N19, P6, V3, N21, N24, P13, P17, P25, P27, V6, V7, V28, V22, V2, V3, V14, G11
\item[dhārayen nāsikāmadhyaṃ tarjanībhyāṃ vinā dṛḍham] P7,
\item[dhāraye nāsikāmadhye tarjanībhyāṃ vinā dṛḍham] Ba,
\item[dhārayen nāsikāṃ madhya tarjjanībhyāṃ vinā dṛḍham] Ba5, Bo2, J3, N11, N16, P11, P14, P20, 
\item[dhārayen nāsikāṃ madhyā tarjjanībhyāṃ vinā dṛḍham] P23,
\item[dhāraye nāsikāṃ madhyā tarjjanībhyāṃ vinā dṛḍham] P18,
\item[dhāraye nāsikāṃ madhyā tarjjanībhyāṃ vinā dṛḍham] Ba5, N4a, N13, V15
\item[dhāraye nāsikāṃ madhya tarjjanībhyāṃ vinā dṛḍham] N8, 
\item[dhārayen nāsikāṃ madhyā tarjjanībhyāṃ vinā dṛḍham] J7, P21, 
\item[dhāvayen nāsikāmadhyāṃ tarjanībhyāṃ vinā dṛḍham] P15,
\item[dhāraye nāsikāṃ madhye tarjjanībhyāṃ vinā dṛḍham] Bo1,
\item[dhārayen nāsikāmadhye tarjanībhyāṃ tathā dṛḍham] B1, B3, J13, N1, N2, P2, P12, V13
\item[dhārayan nāsikāmadhye tarjanībhyāṃ tathā dṛḍham] C2, 
\item[dhārayen nāsikāmadhya tarjjanībhyāṃ vinā dṛḍhām] J1,
\item[dhārayen nāsikā madhye tarjanībhyāṃ vinā madam] J4,
\item[dhārayen nāsikāṃ madhya tarjjanībhyāṃ yathā dṛḍham] C8
\item[dhārayen nāśikāmadhye sarjanībhyāṃ yathā dṛḍham] N22,
\item[dhārayen nāsikāṃ madhya tarjjanībhyāṃ vinā dṛḍham] J6pc,
\item[kārayen nāsikāṃ madhya tarjjanībhyāṃ vinā dṛḍham] J6,
\item[dhārayen nāsikāṃ madhyāt tarjanībhyāṃ vinā dṛḍham] Tue
\item[dhārayen nāsikāmadhyāt tarjanībhyāṃ vinā dṛḍham] C4, N5, N25, V10
\item[dhārayaṃ nāśikāmadhye tarjanabhyāṃ vinā dṛḍham] N23
\item[dhāraae vāsikāmadhye tarjanībhyāṃ vinā dṛḍham] J2,
\item[(illegible/unavailable)] Bo3, C1, C5, J14, V5, T, P16, P22, R5,
  \begin{description}

    \end{description}
 \end{marma}

\begin{marma}[hp02_67ab]
\sthana{2.67ab}
\item[kaṇdalībodhakaṃ viprabhavaghnaṃ sukhadaṃ śubhaṃ] V1,
\item[kuṇḍalī bodhakaḥ kubho rogaghnaḥ sukhada śubhaḥ] Ba3
\item[kuṇdalībodhanaṃ sarva doṣaghnaṃ sukhadaṃ śivaṃ] Bo1
\item[kuṇdalībodhanaṃ sarva doṣaghnaṃ sukhadaṃ śuvaṃ] V15
\item[kuṇdalībodhanaṃ sarva doṣa?? sūkhadaṃ sūkhaṃ] N24,
\item[kaṇdalībodhakaṃ ksipraṃ pavanaṃ sukhadaṃ hitaṃ] Tue, V7
\item[kuṇḍalo bodhaka kṣipraṃ pavanaṃ sukhadaṃ hitaṃ] J2,
\item[kuṇḍalībodhakaṃ kṣipraṃ pavanaṃ sukhadaṃ hitaṃ] Ba5, N13, P14, P18, V22
\item[kuṇḍalabodhakaṃ pāpāghnaṃ suṣadaṃ śubhaṃ] Ko,
\item[kaṇḍalībodhanaṃ kuryāt pāpaghnaṃ sukhadaṃ śubhaṃ] V6, V2
\item[kaṇḍalībodhanaṃ cakaṃ pāpaghnaṃ sukhada śubhaṃ] V14
\item[kuṇḍalībodhanaṃ kuryāt pāpaghnaṃ sukhadaṃ śubhaṃ] B2, Bo2, C4, J1, J3, J6, J7, J11, N16, N19, N23, N25, P6, P21, P23, P27, 
\item[kuṇḍalībodhanaṃ kuryāt pāpaghnaṃ śubhadaṃ sukhaṃ] P20,
\item[kuḍalībodhanaṃ kuryāt pāpaghna suṣadaṃ subhaṃ] N18, 
\item[kuṇḍaṃlībodhanaṃ kuryāt pāpaghnaṃ sukhadaṃ śubhaṃ] N8, N11, 
\item[kuṇḍalabodhanaṃ kuryāt pāpaghnaṃ sukhadaṃ śubhaṃ] N5, 
\item[kaṇḍalībodhanaṃ kuryāt pāpaghnaṃ subhaṃ tad śukhaṃ] V28
\item[kuṇdalībādhakaṃ ksipraṃ pāpghnaṃ sukhadaṃ hitaṃ] Ba,
\item[kuṇdalībodhakaṃ ksipraṃ pāpaghnaṃ sukhadaṃ hitaṃ] P25, 
\item[kumḍalīśhodhanaṃ kuyāt pāpaghnaṃ ca sukhaṃ śhubhaṃ] N22, 
\item[kuṇdalībādhakaṃ kṣipraṃ pavanaṃ sukhadaṃ hitaṃ] N4a,  
\item[kaṇḍalībodhanaṃ kuryāt tāpaghnaṃ sukhadaṃ śubhaṃ] N2, 
\item[kuṇḍalībodhakaḥ kuṃbhaḥ rogaghnaḥ sukhadaḥ śubhaḥ] N6,
\item[kuḍalībodhaka kuṃbhao rogaghnaḥ sukhadaḥ śubhaḥ] P5, P8, P9, P24, 
\item[kuḍalībodhanaḥ kuṃbhao rogaghnaḥ sukhadaḥ śubhaḥ] P7,
\item[kuṇḍalībodhakaḥ kuṃbho rogaghnaḥ sukhadaṃ śubhaḥ] N9, N10, N17, N20,  N26, P1, P10, 
\item[kuḍalībodhakaḥ kuṃbhao rogaghnaṃ sukhadaṃ śubhaṃ] P17,
\item[kuṇḍalībodhakaḥ karthūḥ roghnaṃḥ sukhadaḥ śubhaḥ] A1, V20
\item[kuṇḍalībodhaka kumbhao rogaghnaṃ sukhadaḥ śubham] N14, P2,
\item[kuṇḍalībodhakaṃ cakraṃ bhāvaghnam sukhadaṃ śubhaṃ] N3,
\item[kuṇḍalībodhanaṃ cakra bhāvaghnam sukhadaṃ śubhaṃ] P11,  
\item[kuṇḍalībodhakaṃ kṣipram bhavaghnam sukhadaṃ śubhaṃ] N12,
\item[kuṇḍalībodhakaṃ vipram bhavaghnam sukhadaṃ śubhaṃ] P15,
\item[kuṇḍalībodhanaṃ karttur bhavaghnam sukhadaṃ śubhaṃ] C6,
\item[kuṇḍalībodhanaṃ karttur bhavaghnam sukhaṃ śubhaṃ] V21
\item[kuṇḍalībodhakaṃ miśrabhavaghnam sukhadaṃ śubhaṃ] J4,
\item[kuṇḍalībodhanaṃ kartuṃ bhavaghnam sukhadaṃ śubhadaṃ] C8,
\item[kuṇḍalībodhakaḥ kartuḥ rogaghnasukhadaḥ śubhaṃ] P26, 
\item[kuṇḍalībodhakaḥ kumbhoḥ ragaghnaḥ sukhadaḥ śubhaḥ] N7, 
\item[kuṇḍalībodhakaḥ kumbho roghnaḥ sukhadaḥ śubhaḥ] C3, J10, J17, 
\item[kuṇḍalībodhakaṃ kumbhoḥ roghnaḥ sukhadaḥ śubhaḥ] J15, J16,
\item[kuṇḍalībodhakaḥ kumbho roghnaḥ suṣadaḥ śubhaḥ] C9, 
\item[kuṇḍalībodhakaḥ kumbho rogaghnaḥ sukhadaḥ śubhaḥ] J12, LD1, LD2, V16 
\item[kuṇḍalībodhakaḥ kumbho rogaghnaḥ sukhadaḥ śubhaḥ] B1, B3, C2, J13, V18, V12, V13, V3, V17 
\item[kuṇḍalībodhakaḥ kumbho rogaghnaṃ sukhadaṃ śubhaṃ] P12,
\item[kuṇḍali bodhakaṃ vakrabhāvaghnaṃ sukhadaṃ śubhaṃ] G11
\item[kuṇḍalībodhakaṃ sarva doṣaghnaṃ sukhadaṃ hitaṃ] N21, P13, 
\item[kuṇḍalībodhakaḥ kumbho rogaghnaṃ sukhadaṃ śubhaṃ] N1, 
\item[kuṇḍalīno bodhacakraṃ? bhāvadaṃ supada śubhāṃ] J5,
\item[(illegible/unavailable)] Bo3, C1, C5, J14, V5, V10, T, G1, P16, P22, R1, R5,
  \begin{description}

    \end{description}
 \end{marma}

\begin{marma}[hp02_68a]
\sthana{2.68a}
\item[guṇatrayasamudbhūtaṃ] Bo1,
\item[guṇatrayasamudbhūta] N21,
\item[samagātrasamudbhūtaṃ] J4, N7, 
\item[samagātrasamā....] P10, 
\item[samyagātrasamudbhūtaṃ] J7,
\item[samyagātrasamudbhutaṃ] J15,
\item[samyagātrasamudbhuta] P14,
\item[samyaggātrasamudbhūtaṃ] B2, Bo2, C4, C8, J2, J11, N3, N9, N16, N19, P13, V1, V7, V2, V3, V17
\item[samyagatrasamudbhutaṃ] N12,
\item[samyaggātrasamudbhūta] N2, N4a, N11, N13, N20, N23, N26, P15, P18, P25, P27, Tue, 
\item[samyagātrā sumuvbhutaḥ?] N24,
\item[samya gātre samudbhūtaṃ] J5,
\item[samyaggātre samudbhūtaṃ] J6, N5, N8, N25, P20, P23, V28
\item[samyaggātre samudbhūta] J3,
\item[samyaggātre samudbhūtat] P6, 
\item[samyag gātraṃ samudbhūtaṃ] Ko,
\item[samyaggātasamudbhūtam] V20, V15
\item[samyagataḥ samudbhūtam] V14
\item[samyak gātrasamudbhūta] Ba, V22, G11, P11, 
\item[samyag antaḥ samudbhū] A1
\item[samyag antaḥ samudbhūta] C6
\item[samyak bhastrā samudbhūtā] B1, B3, C2, J12, J13, J16, N1, N10, P1, P2, P8, P9, P12, P17, Ba3
\item[samyak bhastrā samudbhūtaḥ] P26, V21  
\item[samyak bhastrā samabhyāsaḥ] V13
\item[samyag bhastrā samudbhūtāṃ] LD1, LD2, N18, P7, P21, 
\item[samyak bhastrā samudbhūto] C3, J10, J17, N14, N17, P5, P24, 
\item[samyag bhastri samudbhūtaṃ] C9
\item[samyag bhastrī samudbhūto] N6, 
\item[sasyagātrasamudbhūtaṃ] J1,
\item[samagātrasamudbhūtaṃ] V6, V16
\item[samagātre samudbhūta] N22,
\item[samagātrasamudbhūtaghnaṃ] V18
\item[(illegible/unavailable)] Ba5, Bo3, C1, C5, J14, V5, V12, V10, T, G1, P16, P22, R1, R5,
  \begin{description}

    \end{description}
 \end{marma}

\begin{marma}[hp02_69a]
\sthana{2.69a}
\item[vegair ghoṣaṃ] V1
\item[vegākṛṣṭaṃ] Ba, J2, N21, P25, V15
\item[vegākṛṣṇaṃ] P13,
\item[vegā gho] N23,
\item[vego tad ghoṣa] J1,
\item[vego ghoṣam] C3, C8, J16, N3, N9, N16, N20, P1, P5, P10, P24, P27, 
\item[vego ghoṣa] N24,
\item[vego ghoṣaḥ] J15,
\item[vegāt ghoṣaṃ] A1, B2, B3, Ba5, C9, N4a, LD1, N13, N26, P14, P17, P18, P26, R1, Tue, V6, V7, V22, V18, V12, V13, V14, V16, V17, Ba3
\item[vegad ghoṣa] N10,
\item[vegad ghoṣaṃ] N7, 
\item[vegā ghoṣaṃ] P8, V3
\item[vegot ghoṣaṃ] J12, N14, N17, P11, V2, G11
\item[vegot ghoṣe] V2,
\item[vegodveṣaḥ] J4,
\item[vegodvegaṃ] P20a,
\item[vegodveṣaṃ] N5, P15, P20, 
\item[vaiṣodvaiṣaṃ] J5,
\item[veget vege] V28
\item[vegād ghoṣaṃ] J10, J17, N1, V20
\item[vegod ghoṣaṃ] Bo2, C6, J10pc, J11, N18, P6, P9, P21, 
\item[vegod ghoraṃ] P23, 
\item[vedod ghoṣaṃ] V21
\item[vedyo ghoṣaṃ] N22,
\item[vegodyoṣaṃ] J7,
\item[vegopyoyaṃ] N19, 
\item[vemod ghoṣaṃ] J3, J6, 
\item[vegodvegaṃ] C4, J13, N25, 
\item[vegod deyaṃ] P7, 
\item[vegād aghoṣaṃ] B1,
\item[vegād doṣaṃ] Bo1, N2, 
\item[vegod doṣaṃ] N6, 
\item[veṇu doṣaṃ] C2,
\item[veṇu ghoṣaṃ] P2,
\item[veṇud ghoṣaṃ] P12, 
\item[(illegible/unavailable)] Bo3, C1, C5, J14, Ko, N8, N11, N12, V5, V10, T, G1, P16, P22, R5,
  \begin{description}

    \end{description}
 \end{marma}

\begin{marma}[hp02_73a]
\sthana{2.73a}
\item[ārecyāpūrya yat kuryāt] N2, N5, 
\item[ārecya pūryaṃ yat kuryāt] C4,
\item[ārecya āpūrya yat kuryāt] P23,
\item[ārecyāpūrya ?????] N11, 
\item[ārecyāpūrya vat kuryāt] J13,
\item[ārecyāpūrvavat kuryāt] N25,
\item[caraikaṃ? pūrakaṃ kuryāt] C2,
\item[recakaṃ pūrakaṃ kuryāt] N1, N10, P2, 
\item[recya vā pūrya sanakaiḥ] J11,
\item[recake pūrakaṃ kuryāt] J12, J16, 
\item[recayet pūrakaṃ kuryāt] J1, N16, N19, N23, P6, 
\item[recakaṃ pūrakaṃ kuryāt] P12, P17, 
\item[ārecyāpūrya kuṃ?yāṃt] N8,
\item[recakaḥ pūrakaṣ kāryyaḥ] J17,
\item[recakaḥ pūrakaḥ kāryaḥ] A1, C3, C9, N9, J15, N17, N26, P1, P5, P9, P24, R1, 
\item[recakaḥ pūrakaḥ kuryāt] P21, 
\item[recanaṃ pūraṇam kuryāt] P13,
\item[recakaḥ pūrakaḥ kā....] P10,
\item[recakaḥ pūrakaḥ kāryāḥ] Bo1, N7, 
\item[recakaḥ pūrakaḥ kāryaṃ] N20
\item[recakaḥ pūrakaḥ kāryo] P8, 
\item[recakaḥ pūrvavat kāryaḥ] N3, 
\item[recakaṃ pūrakaṃ kuryāt] B1, B3, 
\item[ārecaṃ pūrvavat kuryāt] P20,
\item[recayitvā pūrakaḥ kāryaḥ] J14, 
\item[reco vā pūrakaḥ kāryaḥ] P11, 
\item[rocyatvā pūrakaḥ kuṃbhakaḥ] N24,
\item[recakaḥ pūrakaṃ tyatvā] N18, 
\item[recakaṃ pūrakaṃ muktā] N22,
\item[recake pūrakaṃ uktwā] V14, G11, Ba3
\item[recake pūrṇaṃ kāryaṃ] Ba,
\item[recake pūraṇaṃ kāryaṃ] P25,
\item[recanaṃ pūranaṃ kuryāt] N21,
\item[recaka pūraka kumbhai] V3, V16
\item[recaka pūraka kumbhakaiḥ] V17
\item[recakaṃ pūrakaṃ muktyā] G1
\item[recaka pūraka varjite] V10
\item[virecya pūrakaḥ kāryaḥ] C6, C8, P7,
\item[prāṇāyāmoyam ity uktaḥ] N4a, N12, P14, P18, Tue,
\item[(illegible/unavailable)] B2, C1, C5, N6, N13, N14, P15, T, P16, P22, P26, P27, R5,
  \begin{description}

    \end{description}
 \end{marma}

\begin{marma}[hp02_73b]
\sthana{2.73b}
\item[śanaiḥ saṃhitakumbhakaḥ] N16, 
\item[sarve kevalakumbhakam] G1
\item[śanaiḥ sahitakumbhakaḥ] Bo1, C6, C8, J1, N23, P7, 
\item[śanaiḥ sahitakumbhakaiḥ] P11,
\item[śanaiḥ sahitakumbhakaṃ] P6, 
\item[sa vai kevalakumbhakaḥ] N4a, N18, N22, P18, P26, Tue, 
\item[sa vai sahitakumbhakaḥ] A1, B1, B2, B3, C2, C4, J12, J13, J15, N1, N3, N5, N10, N17, N21, P2, P8, P12, P13, P17, P20, P21, P23, P27, 
\item[sarve sahitakumbhakaḥ] J17, N20, 
\item[savyai sahitakumbhakaḥ] N19, 
\item[sarvai sahitakumbhakaḥ] N9, N12, N26, P1, P9, P10, R1, 
\item[sarvaiḥ sahitakumbhakaḥ] C3, C9, 
\item[sarvaiḥ sahitakumbhakam] Ba, P25,
\item[sarvai sā?ti kumbhakaṃ] N25,
\item[sarvai kevalakumbhaka] N14,
\item[sarvair ahitakumbhakaḥ] N7, 
\item[sa vai sahitakumbhak] J16, 
\item[sa taiḥ sahitakumbhakaḥ] J14, N24, 
\item[sa vai kevalakumbhakaḥ] Ba3, G11, P14, V10
\item[sarvaiḥ sahitapūrakaḥ] P5, P24, 
\item[kāryaḥ sahitakumbhakaḥ] J11,
\item[kuryat sahitakumbhakaḥ] N2, N8, 
\item[kuryat saṃhitakumbhakaḥ] N11, 
\item[kāryaḥ śanai sahitakumbhakā] V3, V14
\item[kāryaḥ śarveiḥ sahitakumbhakeiḥ] V16
\item[kāryaḥ śarveiḥ sahitakumbhaka] V17
\item[(illegible/unavailable)] C1, C5, N6, N13, P15, T, P16, P22, R5,
  \begin{description}

    \end{description}
 \end{marma}

\begin{marma}[hp02_78a]
\sthana{2.78}
\item[kumhakaḥ prāṇarecānte] C3, J2, J3, J13, N8, N11, N16, N21, P5, P24,V1, V6, V28, V18, V16, G11,
\item[kumkāt prāṇarecānte] N20, N26, V17
\item[kumbhaka prāṇarecānte] A1, B1, B2, Bo2, C2, J1, J5, LD1, N2, N7, N9, N10, N12, N14, N17, N22, N25, P1, P2, P9, P10, P12, P13, P17, P20, P21, P23, V2, V12, Ba3
\item[kumbhaka prāṇarecyaṃte] N1,
\item[kumbhake siddha reca puraka] V21
\item[kumbhaka prāṇa cātte?] J10, J17, 
\item[kumbhaka prāṇa recāttaḥ] J16,
\item[kumbhaka prāṇarecātte] P8, P26,
\item[kumbhakaṃ prāṇarecānte] B3, C4, C9, J6, N5, N19, P6, P27, G1, R1, 
\item[kumbhakaṃ prāṇarecātteḥ] J15,
\item[kumbhakaṃ prāṇareṃcānte] J12,
\item[kumbhataḥ prāṇarecānte] J7,
\item[kumbhata prāṇarecānte] V3
\item[kumbhakaṃ ghrāṇarecānte] LD2, 
\item[kumbhitaḥ prāṇarecānte] Bo1, N23, N24, P11, P15, V15
\item[kumbhītaḥ prāṇarecānte] Ko, J11, J14, N3,  
\item[kumbhīte prāṇarecānte] V14
\item[kumbhītaprāṇareṣānte] J4,
\item[tad etat prāṇarrecānte] C6, C8, P7, 
\item[kumbhakaḥ prāṇarodhānte] Ba, Ba5, P14, P18, P25, Tue, V7, V20, V22, V10
\item[kumbhaka prāṇarodhānte] N4a, N13,
\item[(illegible/unavailable)] Bo3, C1, C5, N6, N18, V5, V13, T, P16, P22, R5,
  \begin{description}

    \end{description}
 \end{marma}

\end{ekdosis}
\end{document}

%%% Local Variables:
%%% mode: latex
%%% TeX-master: t
%%% End:









    • Verses are found till 42ab in Bo1 ms
    • Chapter 3 is not available in Bo2; Bo3; C5; P2;
    • Many verses of chapter 1 & 4 are given in chapter 3 of J4 i.e. khecarī etc...
    • K2 has only till sthāna 65d of chapter 3
    • N4a has only till 76b of chapter 3
    • N4b starts from the sthana 56ab
    • N20 has only till sthāna 3.29
    • O1, O2, O3 have only few folios of chapter 1
\begin{marma}[hp03_010]

\sthana{3.10cd}

\item[pādaṃ prasāritaṃ dhṛtvā karābhyāṃ pūrayen mukhaṃ] C4, C8, J1, J3, J6, J7, J14, K1, N5, N8, N11, N25, P6, 
\item[padaṃ prasāritaṃ dhṛtvā karābhyaṃ pūrayen mukhe] N12,
\item[pādaṃ prasāritaṃ kṛtvā karābhyāṃ pūrayen mukham] C6, C9, J10a, J11,
\item[pādaṃ prasāritaṃ dhṛtvā karābhyāṃ dhārayan mukhaṃ] N17,
\item[pādaṃ prasāritaṃ dhṛtvā karābhyāṃ dhārayen mukham] C3, J10, N20, P9, 
\item[pādaprasaritaṃ dhṛtvā karābhyāṃ dhārayen mukham] P10,
\item[pādaprasāritaṃ dhṛtvā karābhyāṃ dhārayen mukham] J17, P1, P8, 
\item[pādaṃ prasāritaṃ dhṛtvāḥ karābhyāṃ dhārayen mukham] J15, N9, 
\item[pādaṃ prasāritaṃ kṛtvāḥ karābhyāṃ dhārayen mukham] P5,
\item[pādaṃ prasāritaṃ dhṛtvā karābhyāṃ dhārayed dṛḍhaṃ] A1, C2, LD1, P2, P4, P12, P13, 
\item[pādaṃ prasāritaṃ kṛtvā hastābhyāṃ dhāraye dṛḍhaṃ] N24,
\item[pādaṃ prasāritaṃ dhṛtvā karābhyāṃ dhāraye dṛḍhaṃ] J13,
\item[pādaṃ prasāritaṃ kṛtvā karābhyāṃ dhāraye dṛḍhaṃ] J12, Ko, 
\item[pādaṃ prasāritaṃ kṛtvā karābhyāṃ dhārayet dṛḍhaṃ] LD2, N4a, N10,
\item[pādaṃ prasārita kṛtvā kārābhyāṃ dhāraye dṛḍhe] J16,
\item[pāda prasāritaṃ dhṛtvā karābhyāṃ dhāraye dṛḍhaṃ] B3,
\item[pādaṃ prasāritaṃ dhṛtvā karābhyāṃ dhārayet sukhaṃ] N6, N7, 
\item[pādaṃ praśāritaṃ dhṛtvā kalābhyāṃ dhārayet sukhaṃ] N14,
\item[pādaṃ prasāritaṃ dhṛtvā karābhyāṃ purayed dṛḍhaṃ] B1, 
\item[pādaḥ prasaritaṃ dhṛtvā karābhyāṃ dhārayen mukhaṃ] N15, N16,
\item[pādaṃ prasaritaṃ dhṛtvā karābhyāṃ dhārayen mukhe] B2,
\item[pādaṃ prasāritaṃ kṛtvā karābhyāṃ dhārayed dhiyā] Bo1,  
\item[padaṃ prasāritaṃ kṛtvā karābhyaṃ pūrayen mukhe] J4, J5, J8, N3, P11, 
\item[pādaṃ prasāritaṃ kṛtvā karābhyaṃ pūrayen mukhaṃ] N22,
\item[pādaṃ prasāritaṃ dhṛtvā karābhyaṃ pūrayan mukhaṃ] N18,
\item[pada prasāritaṃ dhṛtvā karābhyaṃ pūrayen mukhaṃ] N23, P7, 
\item[padaṃ prasāritaṃ dhṛtvā karābhyaṃ pūrayen sukhaṃ] K2,
\item[prasāritaṃ padaṃ kṛtvā karābhyaṃ pūrayen mukhe] Ba,
\item[prasāritaṃ paraṃ kṛtvā karābhyaṃ pūrayen mukhe] J2,
\item[prasāritaṃ padaṃ kṛtvā karābhyaṃ dhāraye dṛḍhaṃ] N21,
\item[prasāritaṃ padaṃ kṛtvā karābhyaṃ dhārayed dṛḍhaṃ] Ba5, N13, P3,

\item[(illegible/unavailable)] Bo2, Bo3, C5, N4b, 
  \begin{description}
    \end{description}

  \end{marma}


\begin{marma}[hp03_012]

\sthana{3.12cd}

\item[tadāsau maraṇāvasthā jayate dvipuṭāśritā] P11,
\item[tadāsau maraṇāvasthā jāyate dvipuṭāśritā] A1, J5, N3, 
\item[tadāsau maraṇāvasthā jāyate dviputāśritā] N24,
\item[tadāsau maraṇāvasthā jāyate dviputāśritā] N21,
\item[tadāsau maraṇāvasthā jāyate dviputāśritāḥ] P13,
\item[tadāsau maraṇāvasthā jāyate dvipuṭāśrayā] C8,
\item[tadāsau maraṇāvasthā jāyate dvipuṭā hi sā] C6, P7,
\item[tadāsau maraṇāvaschā jāyate dvipuṭāśritā] Bo1, 
\item[tadāsau maraṇāvasthā jāyate tripurāśritā] J4, 
\item[tadāsau maraṇāvasthā harate dvipuṭamāśritā] B1, 
\item[tadāsau maraṇāvasthā harate dvipūtāśrayī] N23,
\item[tadāsau maraṇāvasthā harate dvipuṭaśritā] C2,
\item[tadāsau maraṇāvasthā harate dvipuṭāśritā] B3, J15, 
\item[tadāsau maraṇāvasthāṃ harate dvipuṭāśritā] J13, 
\item[tadāsau maraṇāvasthā jāyate tripuṭāśritaḥ] Ko,
\item[tadāsau maraṇāvasthā jāyate tripuṭāśritāḥ] N12,
\item[tadāsau maraṇāvasthā jāyate trimuṭāśratāḥ] J2,
\item[tadāsau ramaṇāvasthā jāyate dvipuṭāśritā] J8,
\item[tadeyaṃ maraṇāvasthā jāyate dvipuṭāśritā] J14,
\item[yathāsau maraṇāvatthāṃ harate dvipaṭāśrayaṃ] J1,
\item[yathāso maraṇāvasthā jāyate dvipūṭāśrītāḥ] J16,
\item[yathāsau maraṇāvasthāṃ harate dvipaṭāśrayaṃ] J6,
\item[yathāsau maraṇāvasthāṃ harate dvipuṭāśritāṃ] P2, P4, 
\item[yathāsau maraṇāvasthāṃ harate dvipadāśrayaṃ] N16,
\item[tathāsau maraṇāvasthāṃ harate dvipadāśrayaṃ] J6a,
\item[tathāsau maraṇāvasthāṃ harate dvipuṭāśrayā] K1, K2, N11, N22, 
\item[tathāsau maraṇāvasthāṃ harate vapurāśrayā] C4,
\item[tathāsau maraṇāvasthāṃ harate vapurāśrayāṃ] J7,
\item[tathāsau maraṇāvasthāṃ harate dvipuṭāśrayāṃ] J3,
\item[tathāsau maraṇāvasthāṃ haraṃte dvipadāśrayā] N5,
\item[tathāsau maraṇāvasthāṃ harate dvipaṭāśrayā] N8,
\item[tathā sā maraṇāvasthā harate dvipuṭāśrayā] N25,
\item[tadāsau maraṇāvasthāṃ harate dvipuṭāśritā] P12,
\item[tadāsau maraṇāvaschā harate vapurāśrayi] Ba, 
\item[tada sā maraṇāvasthā jāyate dvipaṭaśritā] N17,
\item[tadā sā maraṇāvasthā jāyate dvipaṭaśritā] N7,
\item[tathā sā maraṇāvasthā haste dvipuṭamāsrayāt] B2,
\item[tadā sā maraṇāvasthā jāyate dvipuṭāśayā] Ba5,
\item[tadā sā maraṇāvasthā jāyate dvipuṭāśrayā] N4a, N13, P1, 
\item[tadā sā maraṇāvasthā jāyate dvipuṭāśritā] C9, J10, J17, LD1, N6, N15, N20, P5, P8, P9, P10, 
\item[tathā sa maraṇāvasthā jāyate dviputāśṛtā] N14,
\item[tadā sā maraṇāvasthā jāyate dvipaṭāśritā] N9,
\item[tadā sā maraṇāvasthāṃ harate dvipuṭāśrayā] C9a,
\item[yadā sā maraṇā ’vasthā jāyate dvipuṭāsthitā] C3,
\item[yadā sā maraṇāvasthā jāyate dvipuṭāśritā] J12, N10,
\item[tadā sā maraṇāvasthā jāyate tripuṭāsanā] J11,
\item[tadā sā maraṇāvasthā jāyate tripuṭīśritā] LD2,
\item[yathāsau maraṇāvasthāṃ harate dvipuṭaśritā] N18,
\item[yathāsau maraṇāvasthā jāyate dvipūṭāśrayā] P6,
  \item[(illegible/unavailable)] Bo2, Bo3, C5, N4b, P3,

  \begin{description}
    \end{description}

\end{marma}  

\begin{marma}[hp02_014]

\sthana{2.14a}

\item[mahākleśādayo doṣā] A1, Ba, C2, C6, C8, J2, J8, J11, J15, J17, LD2, N3, N14, N16, N20, N22, P6, P7, P8, P10,
\item[mahān kleśādayo doṣā] J10, P1, 
\item[mahākleśā.........doṣā] K2,
\item[mahāklesādayo doṣā] C9,
\item[mahākleśayato doṣā] N23,
\item[mahākleśā mahādoṣā] C4, J13, N5, N25, 
\item[mahākleśādayo doṣāḥ] B1, B2, B3, Ba5, C3, J1, J3, J6, J14, LD1, N4a, N6, N7, N9, N10, N13, N15, N17, N21, P2, P4, P5, P12, P13, 
\item[mahākleśādaye doṣāḥ] P9,
\item[mahākleśāyate doṣāḥ] N24,
\item[māhākleśādayo doṣāḥ] N18,
\item[mahākleśāyato doṣa] J7, 
\item[mahākleśāpanā doṣā] Bo1,
\item[mahākleśā yathā doṣā] Ko,
\item[mahākleśā yathā dehe] J4, 
\item[mahākleśāyata doṣā] P11,
\item[mahatkāśādayo doṣāḥ] N12,
\item[mahatakuśodayo] J5,
\item[mahīkleśādayo doṣā] J12, J16, 
  \item[(illegible/unavailable)] Bo2, Bo3, C5, K1, N4b, N8, N11, P3,

  \begin{description}
    \end{description}

  \end{marma}

\begin{marma}[hp02_018]

\sthana{2.18b}

\item[mahāsiddhikarī nṛṇām] B1, B3, Ba, Bo1, Ba5, C2, C3, J1, J2, J4, J6, J8, J10, J11, J12, J13, J14, J17, Ko, LD1, LD2, N4a, N6, N7, N9, N10, N12, N13, N14, N15, N16, N17, N21, N24, P1, P2, P4, P5, P6, P8, P10, P12, P13, 
\item[mahāsiddhiḥkarī nṛṇām] N18,
\item[mahāsiddhikari nṛṇām] J16,
\item[mahāsidhikarī nṛṇām] C9,
\item[mahāsiddhikarī nṛṇām] J15,
\item[mahāsiddhikaraṃ nṛṇām] N22,
\item[mahāsiddhipradāyakaḥ] J3,
\item[mahāmudrākarī nṛṇām] P9,
\item[sarvasiddhikarī nṛṇām] N20,
\item[jarāmṛtyuvināśinī] A1, C4, C6, C8, J7, K1, K2, N3, N5, N8, N11, N23, P7, 
\item[jarāmṛtyuvivarjitaṃ] N25,
\item[nṛṇāṃ mṛtyuvināśinī] J5,
\item[na deyā yasya kasya cit] B2,
  \item[(illegible/unavailable)] Bo2, Bo3, C5, N4b, P3, P11, 

  \begin{description}
    \end{description}

\end{marma}
  
\begin{marma}[hp02_022]

\sthana{2.22cd}
\item[rājadantadvayaṃ tatra jihvayottaṃbhaved iti] J7, N24, 
\item[rājadadatadvayaṃ tatra jihvayottaṃbhaved iti] N23, 
\item[rājadaṃtadvaya tatra jihvayotta bhaved iti] J5,
\item[rājadantadvayaṃ tatra jihvayotaṃbhayed iti] C8, J14, 
\item[rājadantavilaṃ tatra jihvayottaṃbhayed iti] A1, C6, J6a,
\item[rājadantavilaṃ tatra jihvāsstaṃbhayed iti] P7,
\item[rājadantavilaṃ tatra jihvayottaṃbhayod iti] N16,
\item[rājadaṃtavilaṃ yatra jihvayottaṃbhayed iti] N20,
\item[rājadantavilaṃ jatra jihvayottaṃbhayed iti] J8,
\item[rājadantavilaṃ tatra jihvayottaṃtrayed iti] J6,
\item[rājadantavilaṃ (yaṃtra) jihvayottaṃ bhaved iti] J1,
\item[rājadantavalaṃ tadvaj jihvayā staṃbhayed iti] J11,
\item[rājadantavalaṃ tadvaj jihvā staṃbhayed iti] N25,
\item[rājadantavilaṃ haṃti jihvāyām staṃbhayoditā] Ko,
\item[rājadantasyabilaṃ tadvaj juhvayāstaṃbhayed iti] P5,
\item[rājadantadvayaṃ tatra jihvayā (ruraṃ)bhayed iti] Bo1, 
\item[rājadaṃtasya jihvāyaṃ baṃdhaṃ ca staṃbhayed iti] Ba, 
\item[rājadaṃtasthajihvāyaṃ baṃdhaṃ saṃbhedayed iti] N12,
\item[dājadaṃtasthi jihvāyaṃ baṃdhaṃ ca sthaṃbhayed iti] J2,
\item[rājadaṃtasya jihvāyaṃ baṃdha śasto bhave dvijaḥ] C2,
\item[rājadaṃta sajīhvāyā baṃdhaś ca staṃbhayed iti] P6,
\item[rājadaṃtasya jihvāyaṃ baṃdhaḥ śasto bhaveddhitaḥ] B1, B3, J10, J13, J15, J17, LD1, LD2, N6, N7, N9, N15, N17, N18, P1, P2, P4, P8, P9, P12, 
\item[rājadaṃtasya jihvāyaṃ baṃdhaḥ śasto bhaved iha] J12,
\item[rājadaṃtasya jihvāyaṃ baṃdhaḥ śasto bhaved ehi] N10,
\item[rājadaṃtasya jihvāyoṃ baṃdhaḥ śasto bhaved ihaḥ] J16,
\item[rājadaṃtasya jihvāyaṃ baṃdhaḥ sasto bhaveddhitaḥ] C9,
\item[rājadaṃtasya jihvāyaṃ baṃdhaḥ sasto bhaved iti] C9a, N21, P13, 
\item[rājadaṃtasya jihvāyaṃ baṃdhaḥ śasto bhaved iti] N4a, N13, 
\item[rājadaṃtasya jihvāyaṃ baṃdhaḥ śasnā bhaveddhitaḥ] C3,
\item[rājadaṃtasya jihvāyaṃ baṃdhaḥ śasto bhaved iti] Ba5,
\item[rājadaṃtasya jihvāyaṃ baṃdhaṃ śasto bhaveddhitaḥ] B2,
\item[rātadaṃtasajihvāyaṃ baṃdha śasto bhaveddhitaḥ] K2,
\item[śatadaṃtasya jihvāyaṃ baṃdhaḥ śasto bhaveddhitaḥ] C4,
\item[śatadaṃtasya jihvāyaṃ baṃdhaḥ śasto bhavedihaṃ] N5,
\item[(illegible/unavailable)] Bo2, Bo3, C5, J3, J4, K1, Ko, N3, N4b, N8, N11, N14, N22, P3, P10, P11, 

  \begin{description}
    • It is given at image no. 19 in J5
    \end{description}

  \end{marma}

\begin{marma}[hp02_024]

\sthana{2.24d}

\item[kedāraṃ prāpayen manaḥ] A1, B1, B2, B3, Ba5, Bo1, C2, C3, C4, C6, C9, J1, J6, J7, J8, J10, J12, J13, J15, J16, J17, K1, K2, LD1, LD2, N3, N4a, N5, N6, N7, N8, N9, N12, N13, N15, N16, N17, N20, N21, N22, N23, N24, N25, P1, P2, P5, P6, P8, P9, P10, P12, P13, 
\item[kedāraṃ prāpayan manaḥ] N5, N18, P11, 
\item[kedāraṃ prāppayen manaḥ] Ba, C8, 
\item[kedāraṃ prāpayen mama] N10,
\item[kedāraṃ prāpayen mamaḥ] P4,
\item[kedāraṃ prāpayen manī] P7,
\item[kedāraṃ prāpayen naraṃ] J11,
\item[keharaṃ prāpayen naraṃ] Ko,
\item[kehāraṃ prāpayen naraṃ] J4, 
\item[kedāraṃ prāpyate naraḥ] J5,
\item[kedāraṃ prāpnuyān naraḥ] J14,
\item[kedāraṃ prāsayen raḥ] J2,
  \item[(illegible/unavailable)] Bo2, Bo3, C5, J3, N4b, N11, N14, P3,

  \begin{description}
    \end{description}

  \end{marma}

  \begin{marma}[hp02_026]

\sthana{2.26cd}

\item[vāyunā gatim āvṛtya nibhṛtaṃ kaṇṭhamudrayā] B2, C6, C8, J8, J12, J17, K2, LD1, N3, N4a, N5, N9, N11, N20, N21, P9, P10, P12, P13, 
\item[vāyurnā gatim āvṛtya nimṛtaṃ kaṇṭhamudrayā] N24,
\item[vāyunā gatim ākṛkhya nibhṛtakaṃṭhamudrayā] N18,
\item[vāyunā gātim āvṛtya nibhūtaṃ kaṇthamuṃdrayā] J15,
\item[vāyunā gātim āvṛtya nibhṛtaṃ kaṇthamudrayaṃ] C3,
\item[vāyunā gātim āvṛtya nibhṛtaṃ kaṇthamudrāyā] J10,
\item[vāyunā gātim āvṛtya nibhṛtaṃ kathamudrayā] Ko,
\item[vāyunāṃ gātim āvṛtya vibhṛtaṃ kaṃthamudrayā] C9, J14, 
\item[vāyunā gātim āvṛtya niścitaṃ kaṃḍamudrayā] Ba, J2, 
\item[vāyunā gatim āvṛtya nivṛtaṃ kaṇṭhamudrayā] P8,
\item[vāyunāṃ gatim āvṛtya nibhṛtaṃ kaṃṭhamudrayā] J3, J6a, J7, K1, N7, N10, N13, P4, P7, P11, 
\item[vāyunāṃ gātim āvṛtya nibhūtaṃ kaṃṭhamudrayā] J16,
\item[vāyunāṃ gātim āvṛtti nibhṛtaṃ kaṃṭhamudrayā] J4, J6,
\item[vāyunā gatim ākṛṣya nibhṛtaṃ kaṇṭhamudrayā] A1, B1, J13, LD2, P1, P2, 
\item[vāyunā gatim ākṛṣya nibhṛta kaṇṭhamudrayā] J23, N23,
\item[dhāvāyunā gatim ākṛṣya nibhṛtaṃ kaṇṭhamudrayā] J5,
\item[vāyunāṃ gatim ākṛṣya nibhṛtaṃ kaṃṭhamudrayā] B3, P6, 
\item[vāyūnāṃ gatim ākṛṣya nibhṛtaṃ kaṃṭhamudrayā] C2, N16,
\item[vāyūnāṃ gatim ākṛṣya nibhṛtaṃ haṭhamudrayā] J1,
\item[vāyūnām atimāvṛttya nibhṛtaṃ kaṃṭhamudrayā] N6,
\item[vāyūnāṃ gatim āvṛttya nibhṛtaṃ kaṃṭhamudrayā] C4, J11, N12, N25, 
\item[vāyunā gamitikṛṣya nibhṛtaṃ kaṇṭhamudrayā] N22,
\item[vāyunāṃ gatim āvṛtya nibhaṃ taṃ kaṃṭhamudrayā] Bo1, 
\item[vāyunāmatim āvṛtya nibhṛtaṃ kaṇthamudrayā] N17,
\item[vāyunā gatim āvṛtya nirbhṛtaṃ kaṇṭhamudrayā] P5,
\item[(illegible/unavailable)] Bo2, Bo3, C5, N4b, N8, N14, N15, P3,

    \begin{description}
      The meaning of \emph{nibhṛtam} is not clear. 
    \end{description}

  \end{marma}

\begin{marma}[hp02_028]

\sthana{3.28ab}

\item[somasūryāgnisambandhāj jāyate cāmṛtāyate]
\item[somasūryāgnisambandhāt jāyate cāmṛtāya vai] B2, J14, K1, Ko, N3, N21, N22, P6, P11, P13, 
\item[somasūryāgnisambandho jāyate cāmṛtāya vai] Ba, Ba5, N4a, N12, N13, 
\item[somasūryāgnisambandho jāyate cāmṛtāya va] N24,
\item[somasūryāgnisambandhāj jāyate ca mṛtāya vai] J8, J11a, N20,
\item[somasūryāgnisambandho jāyate cāmṛtāya vau] J2,
\item[somasūryāgnisambandho jāyate cāmṛtāyate] J12,
\item[somasūryāgnisambandhāj jāyate cāmṛtāya vau] J5,
\item[somasūryāgnisambandha jāyate cāmṛtāyate] N16,
\item[somasūryāgnisambandhāj jāyate cāmṛtāyate] A1, C6, J7, N23, P7, 
\item[somasūryāgnisambandhāj jāyate cāmṛtāya tu] C8,
\item[somasūryāgnisambandhāj jāyate cāmṛtāya ca] Bo1, N7,
\item[somasūryāgnisaṃbaṃdhāj jāyate camṛtiṃ jayaḥ] B1, B3, C2, J13, LD2, P2, P4, P12, 
\item[somasūryāgnisaṃbaṃdhā jāyate mṛtur jjayaḥ] N15,
\item[somasūryāgnisaṃbaṃdhāj jāyate ca mṛtur jjayaḥ] J17, LD1, P8, 
\item[somasūryāgnisaṃbaṃdhāj jāyate ca mṛtejaṃ yaḥ] C3,
\item[somasūryāgnisaṃbaṃdhā jāyate ca mṛtyuṃjayaḥ] C9,
\item[somasūryāgnisaṃbaṃdhāj jāyate ca mṛturjayaḥ] J10,
\item[somasūryāgnisaṃbaṃdhāj jāyate ca mṛtyor jjayaḥ] N17,
\item[somasūryāgnisaṃbaṃdhāj jāte ca mṛtyor jayaḥ] N9,
\item[somasūryāgnisaṃbaṃdhāj jāyate ca mṛter jayaḥ] J10a, J15, N10, P1, P5,
\item[somasūryāgnisaṃbaṃdhāj jāyate ca mṛtter jayaḥ] P9,
\item[somasūryāgnisaṃbaṃdhā jāyate ca mṛter jayā] J16,
\item[somasūryāgnisaṃbaṃdhāj jāyate ca mṛtyu jayaḥ] P10,
\item[somasūryāgnisaṃdhānā jāyate cāmṛtāyate] N25,
\item[somasūryāgnisaṃdhānāj jāyate cāmṛtāyate] C4,
\item[somasūryāgnisaṃdhānā jāyate vāmṛtāyate] N5,
\item[somasūryāgnisaṃdhānāj jāyate cāmṛtāya ca] J11,
\item[somasūryāgnisaṃdhānāṃ  jāyate cāmṛtāya ca] J4, 
\item[somasūryāgnisaṃdhānaṃ jāyate cāmṛtāyate] J6, N11, N18,
\item[somasūryāgnisaṃdhānā jāyate bāmṛtāyate] K2,
\item[somasūryāgnisaṃdhānaṃ jāyate vāmṛtāyate] J3,
\item[somasūryāgnisaṃdhānaṃj jāyate cāmṛtāyate] J1,
\item[somasūryāgnisambandhāj jāyate jāyate tripuṭāśrayaḥ] N6,
\item[somasūryāgnisambaddha mṛtyur jjayati yogavit] N14, 
  \item[(illegible/unavailable)] Bo2, Bo3, C5, N4b, N8, P3,

  \begin{description}
  \end{description}

\sthana{3.28cd}

\item[mṛtāvasthā samutpannā tato mṛtyubhayaṃ kutaḥ] A1, B1, B2, C2, C4, C6, C8, J1, J4, J5, J6, J7, J8, J11, J13, J14, K1, K2, N3, N5, N11, N12, N16, N20, N22, P2, P4, P6, P11, P12, 
\item[mṛtāvasthā samutpannā tato mṛtyubhayaṃ jayet] J12, J16, N10,
\item[mṛtāvasthā samutpannā tato mṛtyubhayaḥ kutaḥ] B3,
\item[mṛtāvaschā samutpaṃnnā tato mṛtyubhayaḥ kutaḥ] Ba,
\item[mṛtāvasthā samotpannā tato mṛtyubhayaṃ kutaḥ] N25,
\item[mṛtavasthā samutpannā tato mṛtyubhayaṃ kutaḥ] P7,
\item[mṛtāvasthā samutpannā tato vāyuṃ virecayet] Ba5, C9, J10, J17, LD1, N4a, N6, N7, N9, N13, N14, N17, N18, N21, P1, P5, P9, 
\item[mṛtāvasthā samutpannā tato vāyu virecayet] P8, P10, 
\item[mṛtāvasthāṃ samutpannā tato vāyuṃ viruṃcyatet] LD2,
\item[mṛtāvasthā samutpannā tato vāyu virsarjjayet] Ko,
\item[mṛtā ‘vasthā samutpannā tato vāyuṃ virecayet] C3, 
\item[mṛtāvaschā samutpannā tato vāyuṃ viracayet] Bo1,
\item[mṛtāvastu samutpanna tato vāyuṃ virecayet] J2,
\item[samutpannā mṛtāvasthā tato vāyuṃ virecayet] N24,
\item[mṛtāvasthā samutpannā recayec ca tato vāyu] N15,
\item[mṛtavasthā tato vāyuṃ niruṃdhayet kuṃbhakena] N23,
\item[(illegible/unavailable)] Bo2, Bo3, C5, N4b, N8, P3, P13, 

  \begin{description}
    \end{description}  

\end{marma}
  
\begin{marma}[hp03_029]

\sthana{3.29c}

\item[valīpalitavedhagnaḥ]
\item[valīpalitavedhaghnaḥ] P13,
\item[valipalitavedhaghnau] J2,
\item[valīpalitavandhaḥ] C3, 
\item[valīpalitavepaghnaḥ] Ba5, J1, J3, J6, J14, N4a, N11, N12, N13, N16, N18, N23, 
\item[valipalitaveghaghnaḥ] N22,
\item[valīpalitaveghaghnaḥ] K1, Ko, N20, N24, P11, 
\item[valīpalitavelaghnaḥ] N21,
\item[valīpalitavegāghnaḥ] N25,
\item[valīpalitabaṃdhaghnaḥ] J7, J10, J12, J16, J17, LD1, N6, N9, N17, P1, P5, P8, P9, 
\item[valipalitabaṃdhaghnaḥ] N7, N10, N15, 
\item[valīpalītabaṃdhagghnaḥ] LD2,
\item[valīpalitavegaghnaḥ] C4, K2, N5,  
\item[valipalitavegaghnaḥ] J15,
\item[valipalitavedhaghna] P6, P7,
\item[valītpalitavedhaghna] P10,
\item[valīpalitavedhaghnaṃ] J8,
\item[valīpalitaveghaghnaṃ] N3,
\item[valīpalīvedhaghnaḥ] Ba,
\item[valiḥ palitavedhaghnaḥ] J5,
\item[valīpalitavedhaghnaḥ] C6, J4, J11, 
\item[vallīpalitavedhaghnaḥ] A1, B2, Bo1, 
\item[valīpalitarogaghnaḥ] C8,
\item[valīpalitavaṃdhaghnaḥ] C9,
\item[vallīpalitanirmuktaḥ] B1, B3, C2, J13, P2, P4, P12, 
  \item[(illegible/unavailable)] Bo2, Bo3, C5, N4b, N8, N14, P3,

    \begin{description}
 The occurance of \emph{vedha} in third \emph{pāda} is rather strange as it does not seem related to the other two symptoms of old age (i.e., wrinkles and grey hair) that this \emph{mudrā} can cure. The reading \emph{vepa} (`trembling') of the \emph{Yogacintāmaṇi} and \emph{Jyotsnā} is much better in this regard. Also, \emph{vega} (`agitation' or `shock') renders a somewhat plausible meaning, but is not strongly attested.      
    \end{description}

  \end{marma}

\begin{marma}[hp03_039]

\sthana{2.39c}

\item[vrajaty ūrdhvaṃ haṭhāc] C3, 
\item[vrajaty ūrdhvaṃ haṭhāchaktyā] J3, J10, 
\item[vrajaty ūrdhvaṃ haṭhācchaktyā] J11, J14, N12, 
\item[vrajaty ūrdhvaṃ haṭha śaktyā] N22,
\item[vrajaty ūrdhvahaṭhāchaktyā] J7, J12, 
\item[vrajaty ūrdhvaṃ haṭhā chaktā] J4,
\item[vrajaṃty ūrhaṃ haṭhā chaktā] J16,
\item[vrajaty ūrdhvaṃ haṭhā ch] A1, C6, C9,
\item[vrajaty ūrdha haṭhā chāktā] J1,
\item[vrajaty ūrdhvaṃ haṭhā śaktyā] C8, P6, 
\item[vrajaty ūrdhvaṃ haṭhāt śaktyā] P7,
\item[vrajaty ūrdhvaṃ haṭhaḥ śaktyā] P2, P12, 
\item[vrajaty ūrdhvaṃ haṭhā chaktyā] J3, J8, N7, N9, N10, N16, N18, P1, P5, P8, P10, 
\item[vrajaty ūrdhvaṃ haṭhāc chaktyā] J6, J15, J17, N6,
\item[vrajaty ūrdhvaṃ haṭhāt....tkṣā] P9,
\item[vrajaty ūrdhvaṃ hi tac chaktyā] J6a, N11, N25, 
\item[vrajaty urdhaṃ hi chaktyā] N5,
\item[vrajaty ūrdhvaṃ hitachaktyā] C4, N8,
\item[vrajaty ūrdhvaṃ tataḥ śaktyā] K1,
\item[vrajaty ūrdhaṃ tataḥ ṣaktya] B2,
\item[vrajaty ūrdhaṃ hataḥ śaktyā] Ba, C2,
\item[vrajaty ūrdhaṃ haṭaḥ śaktyā] P4,
\item[vrajaty ūrdhaṃ hṛtaḥ śaktyā] Ba5, N13, 
\item[vrajaty ūrddhvaṃ hṛtaḥ śaktyā] N4a,
\item[vrajety ūrdhvaṃ haṭhāt chaktyā] J5, 
\item[vrajety ūrdhvaṃ haṭhāt śhaktyā] J13,
\item[nayed ūrdhvaṃ haṭhātchaktyā] Bo1,
\item[vrajaty ūrdhvaṃ hi tachaktyā] K2,
\item[vṛajaty ūrdhva hatāśatsā] J2,
\item[vṛjaṃty ū......] N3,
\item[vrajya urddhe haṭhā śaktyā] LD2,
\item[nayed urddhvahaṭhāchaktyā] N21, P13, 
\item[vrajaty ūrdhvamukhe chāktyā] N15,
\item[jajatyūrdhvaṃ hatāchatkā] N23,
  \item[(illegible/unavailable)] B1, B3, Bo2, Bo3, C5, Ko, LD1, N4b, N14, N17, N20, N24, P3, P11, 

    \begin{description}
    • It is at 45c in A1; C6;
    • It is at 62c in C2
    • It is at 63c in J1; J3;
    • It is at 64c in J13;
    • It is at 43c in C8
    • It is at Image no. 38 in J4
    • It is at 48c in J12
      Most of the collated manuscripts have the reading \emph{haṭhāt śaktyā}, which does not make good sense. The \emph{Vivekamārtaṇḍa}'s reading \emph{haṭaḥ śaktyā} (`struck by kuṇḍalinī') is not found among the \emph{Haṭhapradīpikā}'s manuscripts, but is close to N23. The reading of 3a2 (and the YCM), \emph{hi tacchaktyā} appears to be the most plausible reading of the collated manuscripts. It could be understood as `by the power of  khecarī', which would mean that \emph{yonimudrā} blocks \emph{bindu}'s downward course and \emph{khecarī} causes it to go upwards.
    \end{description}

\end{marma}

\begin{marma}[hp03_046]

\sthana{3.46f}

\item[vimalaṃ dhārāmṛtaṃ yaḥ pibet] C2, C3, C6, C8, C9, J3, J4, J5, J8, J13, J14, J15, J17, Ko, LD1, N7, N8, N9, N11, N15, N16, N25, P1, P6, P8, P9, P13, 
\item[vimalaṃ dhālāmṛtaṃ yaḥ pibet] N14,
\item[sumūla dhārāmṛtaṃ yaḥ pibet] J12,
\item[vimalaṃ dhārāmṛtaṃ yaḥ pībe] N22,
\item[vimala dhārāmṛtaṃ yaḥ pivet] J1, N3, 
\item[vimalaṃ dhārāmṛtyaṃ yaḥ pibet] P10,
\item[vimalaṃ dhāraṃ mṛtaṃ yaḥ pibet] B3, 
\item[vimalaṃ dhārāmṛtaṃ yaḥ piven] A1, B2, C4, J6, J7, J10, J11, K2, N5, N13, N17, N18, N21, N23, P2, P4, P5, P7, P11, P12, 
\item[vimalaṃ dhārāmṛta yaḥ piven] N6,
\item[vimalaṃ dhārāmṛtaṃ yaḥ piveṃn] LD2,
\item[vimalaṃ dhārāmṛtaṃ yaḥ pīvet] K1,
\item[vimalaṃ dhārāmṛtaṃ yaḥ pīven] J16,
\item[vimalaṃ dhārāmayaṃ yaḥ pibet] Ba5, N4a, N12, 
\item[vivare dhārāmayaṃ yaḥ pibet] Ba, J2, 
\item[vimalaṃ dhārāmṛtaṃ yaḥ pibaṃn] N10,
\item[vimalaṃ dharamṛtaṃ yaḥ piven] N24,
\item[(illegible/unavailable)] B1, Bo1, Bo2, Bo3, C5, N4b, N20, P3, 

  \begin{description}
    • It is given at 75 in B3
    • It is given at 72f in C2
    • It is given at 47f in C8
    \end{description}

\end{marma}

\begin{marma}[hp03_049]

\sthana{3.49}

\item[pātāle yadviśati suṣiraṃ merumūle yadasti | 
tadvaccaitat pravadati sudhīs tanmukhaṃ nimnagānām | 
candrātsāraḥ sravati vapuṣastena mṛtyur narāṇāṃ 
taṃ badhnīyāt sukaraṇam atho nānyathā kāyasiddhiḥ ||] J14, 

\item[pātāle yadvaśati sukhiraṃ merumūle yadasti
tadvac caitat pravadati sudhīs tanmukhaṃ nimnagānāṃ
caṃdrātsāraḥ śravati vapuṣos tena mṛtyur narāṇāṃ
taṃ badhnīyāt sukaraṇamatho nānyathā kāryasiddhi] P13,

\item[pātāle ye dvitayasūṣire merumūle tadasmins
tatvaṃ yattat pravadati sudhīs tanmukhaṃ nimnagānāṃ ||
caṃdraṃ sāraḥ śravaṃtyau vapuṣas tanema mṛtyuṃ nnarāṇāṃ
bandhiyātakaraṇamamṛtaṃ nānyathā kāryasiddhi ||] P6,

\item[yatpātāle viśati suṣiraṃ merumūle yadasti
tasmins tatvaṃ pravadati sudhīs taṃnmukhaṃ nimnagānāṃ ||
caṃdrātsāraḥ sravati vapuṣaḥ stena mṛtyur narāṇāṃ
taṃ badhnīyāt sukaraṇamatho nānyathā kāyasiddhiḥ ||] P2,

\item[yatpātāle viśati sukhiraṃ merumūle yadasti
tasmiṃs tatvaṃ pravadati sudhīs tanmukhaṃ nimnagānā ||
caṃdrātsāraḥ sravati vapuṣas tena mṛtyur narāṇāṃ
taṃ badhnīyāt sukaraṇamatho nānyathā kāyasiddhiḥ ||] P4,

\item[yatpātāle viśati sukhiraṃ merumūle yadasti
tasmiṃs tatvaṃ pravadati sudhīs tanmukhaṃ nimnagānāṃ ||
caṃdrātsāraḥ sravati vapuṣas tena mṛtyur narāṇāṃ
taṃ badhnīyāt sukaraṇamatho nānyathā kāyasiddhi ||] P12,

\item[yahapāledviśati sukhiramahamūle yadāsti
taviśve tat pravadati sudhā tanmukhaṃ nimnagānāṃ
caṃdrāsāra sravati vapuṣaṣto na matyu narāṇāṃ
taṃ vadhnīyāt sukaraṇamatho nānyathā kāryyasiddhi] N24, 

\item[yatprāleyaṃ pihītasuṣiraṃ merūmurddhātathāṃtaṃ
tasmiṃstatvaṃ pravadati sudhīs tanmukhaṃ nimnagānāṃ
caṃdrātsāraḥ śravati vapuṣas tena mṛtyur nnarāṇāṃ 
taṃ badhnīyāt sukaraṇamatho nānyathā kāryasiddhi] J15,

\item[yatprāleyaṃ cāpihitamukhisorūmūddhnaṃ tathyaṃ
tasmiṃs tatvaṃ pravadaṃti sudhīs tanmukhaṃ nimnagānāṃ ||
caṃdrāt sāraḥ śravati vapuṣes tena mṛtyuṃ nnarāṇāṃ
taṃ vadhnīyāt sukaraṇamatho nānyathā kāyasiddhiḥ ||] P10,

\item[pātāle yadvitayuśubhiraṃ merumūlai tadasmin
tatvaṃ caitat pravadati sudhīs tanmukhaṃ nimnagānāṃ
caṃ taṃ prasāraṃ grasati vapuṣā doṣamṛtyunarāṇāṃ
taṃ badhnīyāt sukaraṇamṛdā nānyathā kāryasiddhi] P11,

\item[yatprāleyaṃ pihitasukhiraṃ merumūrdhdhnaṃ tatathyaṃ
tasmiṃs tatvaṃ pravadati sudhīs tanmukhaṃ nimnagānāṃ ||
caṃdrātsāraḥ śravati vapuṣas tena mṛtyur narāṇāṃ
taṃ vadhnīyāt sukaraṇamatho nānyathā kāyasiddhiḥ ||] P9,

\item[yatproleyaṃ pihitasukhiraṃ merumūrdhnyātathyaṃ
tasmistatvaṃ pravadati sudhīstanmukhaṃ nimnagānāṃ
candrochāraḥ śravati vapuṣestenamṛtyunnarāṇāṃ
taṃ pātāle yadviśatisuṣiraṃ merumūle yasti
tadvaccaitat pravadati sudhīs tanmukhaṃ nimnagānāṃ
vadhnāyāstu karaṇamatho caṃdrātsāraḥ
sravati vapuṣestena mṛtyurnnarāṇāṃ
taṃ badhnīyāt sukaraṇamatho nānyathā kāyasiddhiḥ] P8,

\item[pātāle yadvitaya sukhiraserumūles tadasmin
statvaṃ yatat pravadaṃti sudhās tanmukhaṃ nimnagānāṃ
cāṃdraṃ sāraṃ śravati vapuṣas tena mṛtyur narāṇāṃ
badhnayātat kāraṇamamataṃ nānyathā kāryasiddhiḥ] N22,

\item[pātālād yadviśatiśukhiraṃ merumūle tadasti
tatvaṃ caitat pravadati sudhīḥ tatsukhaṃ nimnagānāṃ ||42||
caṃdrāt sāraṃ sravati vapuṣā tena mṛtyur narāṇāṃ
taṃ badhnīyāt sukaramṛdā nānyathā kāryasiddhiḥ ||] P7,

\item[yatprāleyaṃ cāpihitaṃ suṣiraṃ merumūrddhataḥ 
tasmiṃs tatvaṃ pravadaṃti sudhīs tanmukhaṃ nimnagānām
candrāt sravatipīyūṣas tena mṛtyur nnarāṇāṃ
taṃ vadhnīyāt sukaraṇamatho nānyathā kāryyasiddhi] N16,

\item[yatprāleyaṃ pihitasukhiraṃ merumūrddhnitithyaṃ 
tasmiṃs tatvaṃ pravadati sudhīs tanmukhaṃ nimnagānāṃ ||
caṃdrātsāraṃ śravati vapuṣas tena mṛtyuṃ narāṇāṃ
taṃ badhnīyāt sukaraṇamatho nānyathā kā prasiddhiḥ ||] P5, 

\item[yatprāleyaṃ pihitaśiṣire merūmūrddhaṃtathāṃtaṃ
tasmiṃstatvaṃ pravadati sudhīs tanmukhaṃ nimnagānāṃ
caṃdrātsāraḥ śravati vapus tena mṛtyur nnarāṇāṃ
taṃ vadhnīyād varūṇamamṛtaṃ nānyathā kāryasiddhiḥ] N25,

\item[yatprāleyaṃ pihitasukhiraṃ me tu samūrddhye tathyaṃ
tasmistatvaṃ pravadati sudhīs tanmukhaṃ nimnagānāṃ ||
caṃdrātsāraḥ śravati vapuṣas tena mṛtyurnnarāṇāṃ 
taṃ badhnīyāt sukaraṇamatho nānyathā kāryasiddhiḥ ||] P1,

\item[yatprāleyaṃ pihitasukhiraṃ merumuddhnastitathyaṃ
tasmin tatvaṃ pravadati sudhīs tanmukhaṃ nimnagānāṃ 
caṃdrātsaraṃ sravati vapuṣas tena mṛtyur narāṇāṃ
taṃ vadhnīyāt sukaraṇamatho nānyathā kāryyasiddhi] N18,

\item[yatprāleyaṃ pihitasukhiraṃ merumūrddhnāṃ tathyaṃ
tasmiṃs tatvaṃ pravadati sudhī yas tanmukhaṃ nimnagānāṃ ||
candrātsāraḥ sravati vapuṣas tena mṛtyur nnarāṇāṃ
taṃ vadhnīyād varuṇamamṛtaṃ nānyathā kāryyasiddhiḥ ||] N17,

\item[yatprāle pihitasukhiraṃ merumūrddhyaṃ || 
tathyaṃ tasmīs tatvaṃ pravadati sudhiḥ tanmukhaṃ nimnagānāṃ
cadrātsāraḥ śravati vapuṣas tena mṛtyur narāṇāṃ ||
ta vadhnīyāt sukaraṇamano nānyathā kāryasiddhi ||] N15,

\item[yatprāleya prahitasukhiraṃ merūmūrddhāṃtarasthaṃs
tasmiṃs tatvaṃ pravadati sudhīs tanmukhaṃ nimnagānāṃ
caṃdrātsāraḥ sravati vapuṣaḥ tena mṛtyur narāṇāṃ
tadvadhnīyāt sukaraṇamatho nānyathā kāyasiddhiḥ] N13,

\item[pātāle yadvitayasuṣiraṃ merumūle tadasmiṃs
tatvaṃ caitat pravadati sudhī tanmukhaṃ nimnagānāṃ ||
candrāsāraṃ sravati vapuṣo doṣamṛtyur narāṇaāṃ
tad badhnīyāt sukaraṇamaho nānyathā kāryasiddhiḥ ||] N10,

\item[pātāle yad vasati sukhiraṃ merumūle yadasti
tadvac caitat pravadati sudhīs tanmukhaṃ nimnagānāṃ
caṃdrātsāraḥ śravati vapuṣas tena mṛtyur narāṇā
taṃ vaddhniyāt sukaraṇamatho nānyathā kāryasiddhiḥ] N21,

\item[yatprāleyaṃ cāpihittasukhiraṃ meru mūddhā tathyaṃ
tasmis tatvaṃ pravati sudhīs tanmukhaṃ nimnāgānā
caṃdrāt sāraḥ śravati vapuṣes tena mṛtyuṃ narāṇāṃ
na vadhnīyāt sukaraṇāmadho nānyathā kāyasiddhi] N9,

\item[pātāle yadvicayaṃ suṣiraṃ meruūle tadasmin
tatvaṃ caitat pravadati sudhīs tanmukhaṃ nimnagāyāḥ |
candrāsāraṃ grasati vapuṣā doṣamṛtyur narāṇāṃ
taṃ vadhnīyāt sukaraṇamahā nānyathā kāryasiddhiḥ ||] N12,

\item[yatprāleyaṃ pihitasuṣiraṃ merumūrddhnyāṃ tathyaṃ
tasmis tatvaṃ pravadati sudhīs tanmukhaṃ nimnagānāṃ |
caṃdrātsāraḥ sravati vapuṣas tena mṛtyur nnarāṇāṃ
taṃ taṃ vadhnīyād varuṇamamṛtaṃ nānyathā kāryyasiddhi ||] N6,

\item[yatprāyaṃ pihitaḥ sukhiraṃ merumūrddāṃtarasthaṃ
tasmin tatvaṃ pravadati sudhīs tanmukhaṃ nimnagānāṃ |
caṃdrātsāraḥ sravati vapuṣe tena mṛtyur nnarāṇāṃ
taṃ taṃ vadhnīyād sukhakaramathā nānyathā kāryasiddhi ||] N7,

\item[yatprāleyaṃ pihitaśikharaṃ merumūrdhnisthitaṃ yat
tasmin tatvaṃ pravadati sudhīs tanmukhaṃ nimnagānāṃ
caṃdrātsāraḥ śravati vapuṣas tena mṛtyur narāṇāṃ
tad vadhnīyāt karaṇamamṛtaṃ nānyathā kāryasiddhiḥ] Ko,

\item[pātāle yadvitayasukhiraṃ merūmūle tadasmiṃs
taṃtvaṃ cetat pravadatī sudhāṃ tanmūkhaṃ nimnagānāṃ 
caṃdrāsāraṃ sravati vapuṣo doṣamṛtyu narāṇāṃ
tadbadhnīyāt sukarora hī nānyathā kāryaṃ siddhī] J16, 

\item[pātāle ya dvitiya sukhiraṃ merumūle tadasmin
statvaṃ caitat pravadati sudhā tanmukhaṃ nimnagānāṃ
caṃdrāsāraṃ sravati vapuṣo doṣamṛtyur narāṇāṃ
tad badhnīyāt sukarāgama hi nānyathā kāryasiddhiḥ] J12,

\item[yatprāleya prahitasuṣiraṃ merumūrdhnāstitathyaṃ
tasmiṃstatvaṃ pravadati sudhīs tanmukhaṃ nimnagānāṃ
caṃdrātsāraṃ sravati vapuṣo doṣamṛtyur narāṇāṃ
tadvadhnīyāt sakaraṇamayo nānyathā kāryasiddhiḥ] K1,

\item[pātāle ya dvitayasuṣiraṃ merumūle tadasmin
tadvac caitā pravadati sudhī tanmukhaṃ nimnagānāṃ
caṃdrāt sāraḥ sravati vapuṣas tenamṛtyur narāṇāṃ
taṃ vadhnīyāt sukaraṇahā nānyathā kāryasiddhiḥ] J11, 

\item[pātāle yad viśati sukhiraṃ merumūle yadasti 
tadva caitat pravadati sudhīs tanmukhaṃ nimnagānāṃ ||
caṃdrāt sāraḥ sravati vapuṣas tena mṛtyur narāṇāṃ
tad vadhnīyāt sukhakaramatho nānyathā kāyasiddhiḥ ||] J7,

\item[yatprāleyapihitasukhiraṃ merumūrddhni sthitaṃ
tasmiṃs tatvaṃ pravadati sudhīs tanmukhaṃ nimnagānāṃ |
caṃdrāt sāraḥ śravati vapuṣas tena mṛtyur nnarāṇāṃ
taṃ vadhnīyāt sukaraṇamatho nānyathā kāyasiddhiḥ ||] J10,

\item[yatprāleyaṃ pihita suṣiraṃ merumūdhnāṃ tathāṃntyaṃ
tasmiṃstatvaṃ prabadati sudhīs tanmukhaṃ nimnagānāṃ ||
caṃdrātsāraḥ sravati bapuṣas tena mṛtyur jarāṇāṃ |
taṃ badhnīyadguṇamamṛtaṃ nānyathā kāryasiddhiḥ] K2, 

\item[yatprāleyaṃ pihitasukhiraṃ merumūrddhni sthitaṃ
tasmiṃs tatvaṃ pravadaṃti sudhīs tanmukhaṃ nimnagānāṃ
candāt sāraḥ śravati vapuṣes tena mṛtyun narāṇāṃ
taṃ badhnīyāt sukaraṇamatho nānyathā kāyasiddhiḥ] J17,

\item[pātāle yadvitayasukhiraṃ merumūle tadasti
tatvaṃ caitat pravadati sudhīs tanmukhaṃ nimnagānāṃ
caṃdrāsāraṃ sravati vapuṣo doṣamtyu narāṇāṃ
na badhnīyāt sukaraṇamaho nānyathā kāryasiddhiḥ] LD1,

\item[pātāle yadviśati sukhira merūmūle pakṣasti
tadvac caitat pravadati sudhās tanmukhaṃ niṣagmanāṃ ||
caṃdrātsāraḥ rapativapuṣas tena mṛtyur narāṇāṃ 
tacha-yāt sukhakaraṇamartho nāmarthā kāyasiddhiḥ ||] N23,

\item[yatpātāle vrajaddhitapasukhiraṃ merumūle ca
tasmin tattvaṃ pravadati sudhīs tanmukhaṃ korthaḥ nāḍīnā kāryaḥ |
caṃdrāt sāraḥ śravati vapuṣas tena mṛtyur nnarāṇāṃ
taṃ vadhnīyāt sukaraṇamatho nānyathā kāyasiddhiḥ ||] J10a, 

\item[yatprālepaṃ pihitasukhire merumūle yadastī
tasmi tvaṃ pravadaṃti sudhīs tanmukhaṃ nimnagānāṃ
caṃdrātsāraṃ śravati vapuṣaṃ stenamṛtyuṃr nnarāṇāṃ |
taṃ badhnīyāt sukaraṇamṛdā nānyathā kāryasiddhiḥ ||] J5,

\item[yatprāleyaṃ cāpihitaḥ sukhiraṃ merumūrddhātaḥ 79 ||
tasmin tatvaṃ pravadati sudhīs tanmukhaṃ nimnagānāṃ ||
caṃdrotsāraḥ śravati vapuṣas tena mṛtyur narāṇāṃ ||
taṃ badhnīyāt sukaraṇamatho nānyathā kāryasiddhiḥ ||] J13, 

\item[yatprāleyaṃ cāpi hitasukhiraṃ meha mūrddhāṃtaḥ
tasmin tattvaṃ pravadati sudhīs tanmukhaṃ bhinnagānāṃ
caṃdrātsāraḥ sravati ca pīyūṣas tenamṛtyur narāṇāṃ
taṃ badhnīyāt sukaraṇamatho nānyathā kāyasiddhiḥ] J6,

\item[yat prāleyaṃ cāpi hitasukhiraṃ merumūrddhvataḥ ||
tasmiṃs tatvaṃ pravadati sudhyaṃs tanmukhaṃ nimnagānāṃ ||74||
candrāt sāraḥ śravati vapayuṣes tena mṛtyun narāṇāṃ |
taṃ vadhnīyātasukaraṇamatho nānyathā kāyasiddhiḥ ||] J8, 

\item[yatprāleyaṃ vahati suṣiraṃ meha mūrddhāṃtaḥ
tasmin tattvaṃ pravadati sudhīs tanmukhaṃ bhinnagānāṃ
caṃdrātsāraḥ sravati satataṃ tenamṛtyur narāṇāṃ
taṃ badhnīyāt sukaraṇamatho nānyathā kāyasiddhiḥ] J6a,

\item[pātāle yannitayasuṣiraṃ merumūle tad asti
tadvac caitat pravadati dhīstanmuratanimnagānāṃ ||
caṃdrāsāraṃ sravati vapuṣes tenaćtyur narāṇāṃ
tad vadhnīyātyakaraṇaćdā nānyathā kāryasiddhi ||] J4,

\item[pātālād yad viśati suśiraṃ merumūle tadasti
tatvaṃ caitat pravadati sudhīs tatsukhaṃ nimnagānām |
caṃdrātsāraṃ sravati vapuṣā tenamṛtyur narāṇāṃ
taṃ badhnīyāt sukaraṇamṛdā nānyathā kāyasiddhiḥ ||] C6,

\item[pātālād yadi viśati sukhiraṃ merumūle tadasthi
tatvaṃ caitat pravadati sudhīs tatsukhaṃ nimnamānām |
caṃdrātsāraṃ sravati vapuṣā tena mṛtyur narāṇāṃ
taṃ badhnīyāt sukaraṇamṛdunā nānyathā kāryasiddhiḥ ||] C8,

\item[yat pātāle viśati suṣiraṃ merumūle yadasti | 
tasmiṃstatvaṃ pravadati sudhīs tan mukhaṃ nimnagānāṃ
candrātsāraḥ sravati vapuṣastena mṛtyur narāṇāṃ 
taṃ badhnīyāt sukaraṇam atho nānyathā kāyasiddhiḥ ||] B3, 

\item[yatpātālapihītasuśīraṃ merumūlena yadasti
tasmiṃstatvaṃ pravadati sudhīs ta(nsu)ṣaṃ nimnagānāṃ ||55||
caṃdrāsāra śravati vapuṣas tena mṛtyuṃ narāṇāṃ
taṃ vadhniyāt sukaraṇaṃ yatho nānyathā kāryasiddhiḥ ||56||] LD2,

\item[yat pātāle vikhati suṣiraṃ merumūle yadasti | 
tasmiṃstatvaṃ pravadati sudhīs tan mukhaṃ nimnagānāṃ
candrātsāraḥ sravati vapuṣastena mṛtyur narāṇāṃ 
taṃ badhnīyāt sukaraṇam atho nānyathā kāyasiddhiḥ ||] C2,

\item[prāleya prahita suṣiraṃ merumūrdhāṃtarasthaṃ
tasmiṃs tatvaṃ pravadati sudhīs tan mukhaṃ nimnagānām |
caṃdrāt sāraḥ sravati vapuṣas tena mṛtyur narāṇāṃ
tad badhnīyāt sukaraṇamato nānyathā kāyasiddhiḥ ||] Ba5,

\item[yatprāleya prahita suṣiraṃ merumūrdhāntarasthaṃ
tasmin tatvaṃ pravadati sudhīs tan mukhaṃ nimnagānām |
candrāt sāraṃ sravati vapuṣas tena mṛtyur narāṇāṃ
tad vadhnīyāt sukaraṇamato nānyathā kāyasiddhiḥ ||] N4a,

\item[yatprāleyaṃ pihitasukhiraṃ merumūrddhnāsmitathyaṃ 
tasmiṃs tatvaṃ pravadati sudhis tanmukhaṃ nimnagāṇāṃ 
caṃdrāt sāra śravati vapuṣi tenaćtyu narāṇaāṃ
taṃ vadhniyāt sukaraṇapatho nānyathā kāryasiddhiḥ] C9,

\item[yatprāleyaṃ vahati suṣiraṃ cordhvamūrdhvāṃtasaṃsyaṃ ||
tasmin tatvaṃ pravadati sudhīs tanmukhaṃ bhinnagānāṃ ||76|
caṃdrātsāraḥ sravati satataṃ tena mṛtyur narāṇāṃ ||
taṃ badhnīyāt sukhakaramatho nānyathā kāyasiddhiḥ ||] J3,

\item[yatprāleya prahita sukhiraṃ mesamūrdhāṃ ca tathyaṃ
tasmiṃs tatvaṃ pravadati sudhīs tan mukhaṃ nimnagānaṃ ||
caṃdrāt sāraḥ śravati vapuṣas tena mṛtyur narāṇāṃ
tad badhnīyāt sukaraṇamato nānyathā kāyasiddhiḥ ||] C3,

\item[yatprāleyaṃ pihita suṣiraṃ merumūrdhnā tathāṃ taṃ
tasmiṃs tatvaṃ pravadati sudhīs tan mukhaṃ nimagnānāṃ ||
caṃdrāt sāraḥ sravati vapuṣas tena mṛtyur narāṇāṃ
taṃ badhnīyād guruṇmamṛtaṃ nānyathā kāryasiddhiḥ ||] C4,

\item[yatprāleyaṃ pihita suṣiraṃ merumūrdhā? tathāṃtaṃ
tasmiṃs tatvaṃ pravadati sudhīs tan mukhaṃ nimagnānāṃ ||
caṃdrāt sāraḥ śravati vapuṣas tena mṛtyur narāṇāṃ
taṃ badhnīyād guruṇamamṛtaṃ nānyathā kāryasiddhiḥ ||] N5,

\item[pātāle yadvitay suṣi-
stitatvaṃ caitat pravadati sudhis tanmūkhaṃ ni(śū)gānāṃ
candrātsāraḥ sravati vapuṣo doṣamṛtyur narāṇāṃ 
taṃ badhnīyāt suka(ṭha)............nānyathā kāyasiddhi ||] Ba, 

\item[yatprāreyaṃ pihitaṃ sukhiraḥ merumurddhyastitathyaṃ
tasmistatvaṃ pravadaṃti sudhīs tanmukhaṃ nimnagānāṃ ||76||
candrātsāraḥ śravati ca vapīyūṣastenamṛtyurnarāṇāṃ ||
taṃ badhnīyāt sukaraṇatho nānyathā kāryasiddhiḥ ||] J1, 

\item[(illegible/unavailable)] A1, B1, B2, Bo1, Bo2, Bo3, C5, J2, N3, N4b, N8, N11, N14, N20, P3,

  \begin{description}
    • It is given at 80 in B3
    • It is given at 75 in C2
    • It is given at 41 in C8
    \end{description}

  \end{marma}

\begin{marma}[hp03_056]

\sthana{3.56ab}

\item[tato jātau vahnyapānau prāṇam uṣmasvarūpakau] J3, J7, J11, 
\item[tato jātau vahnipānau prāṇamūlasvarūpakau] N18,
\item[tato jātau vahnyapānau prāṇamūktasvarūpakaṃ] K2,
\item[tato jātau vahnyapānau prāṇamūktāsvarūpakaṃ] J15,
\item[tato jātau vahnipānau prāṇemuṣmasvarūpakāṃ] N24,
\item[tato jātau vahijānau prāṇamūlesvarūpakau] J1,
\item[tato jātau vahnipānau prāṇam uṣmasvarūpakaṃ] C8,
\item[tato jātau vahnyapānau prāṇamuktasvarūpakaṃ] C4, N5, 
\item[tato jātau vahnyapānau prāṇamuktasvarūpakau] J6, J14, N16, 
\item[tato jātau vaha?pānau prāṇamuktasvarūpakau] N11,
\item[tato jātau vahnyapānau prāṇamūlasvarūpakaṃ] Ko,
\item[tato jātau vardhapāne prāṇamuṣmasvarūpakaṃ] J8,
\item[tato jāto vahipānau prāṇamūlasvarūpaka] N7,
\item[tato vāyu vahniyogau prāṇamūlasvarūmukaḥ] N17,
\item[tato yoga vahniyonau prāṇamūlasvarūmukaḥ] P8,
\item[vato yāto vanhyapānau prāṇam uṣmasvarūpakaṃ] P12,
\item[vato yātau vahnyapānau prāṇam uṣmasvarūpakaṃ] J4, LD2, P7, P13, 
\item[vato yātau vahnyapānau prāṇamukhasvarūpakaṃ] N21,
\item[vato vātau vanhipānau prāṇam uṣmasvarūpakaṃ] K1, 
\item[vato vāyū vanhipānau prāṇarvddhrcarāvamū] N6,
\item[tato yātau vahnyapānau prāṇam uṣmasvarūpakaṃ] C2,
\item[tato yātau brahmapānau prāṇam uṣmasvarūpakaṃ] Ba5, 
\item[tato yātau vrahmapānau prāṇam uktasvarūpakaṃ] N25,
\item[tato yātau bāhyapānau prāṇam uṣmasvarūpakaṃ] C6,
\item[tato jātā vaṃdhapānau prāṇamurasvarūpaṃke] N23,
\item[tato yātau vāhyāpānau praṇam uṣṇasvarūpakaṃ] P11,
\item[tato yātau vāhyāpānau prāṇam uṣṇasvarūpakaṃ] J13, P2, P4,  
\item[tato yāto vahir yonau prāṇamūlasvarūpakaḥ] C9,
\item[tato yāttau vahnipātau prāṇmūlasvarūpakaṃ] J5,
\item[tato yātau vahnipānau prāṇmūlasvarūpakau] N8,
\item[tato yāmau vahnipānau prāṇamūlasvarūpakaṃ] N3,
\item[tato vahnir apānau ca prāṇmūlasvarūpakaṃ] B2,
\item[tato cāptau brahmapānau prāṇam uspasvarūpakaṃ] N22,
\item[tato yo vahni yo vahni yonau prāṇaṃ mūlasvarūpakaḥ] N15,
\item[tato yāte vahnir yonau prāṇe mūlasvarūpakaḥ] C3,
\item[tato yāte vahniyonau prāṇa mūlasvarūpakaḥ] J17, P1, P5, P10, 
\item[tato yāte vahnyar yono prāṇe mūlasvarūpake] J16,  
\item[tato yāte bahir yonau prāṇe mūlasvarūpake] J12,
\item[tato yāte vanhipānau prāṇe mūlasvarūpake] N10,
\item[tato yāne vahnipānau prāṇamūlasvarūpakaḥ] N9,
\item[tato yāto vahni yonau prāṇamūlasvarūpakaḥ] J10,
\item[tato yāto vanhyapānau prāṇamūṣmasvarūpakaṃ] N4a, N13,
\item[tato yāto vanhyapānau prāṇamūṣṇasvarūpakaṃ] N4b,
\item[tato pātau vahnyāpānau prāṇam uṣṇasvarūpakaṃ] A1, B3,
\item[tato yāto vahnipānau prāṇamuṣṇasvarūpakaṃ] LD1,
\item[tenātyaṃta pradīpastu jvalano dehajas tathā] Ba, 
\item[tato vāyuvahnir yātau prāṇmūrañ caturmukha] N14,
\item[tato yātau vahnyapārṣṇau prāṇamuhmyasvarūpakaṃ] P6,
\item[tato sthito vanhiyonau prāṇamūlasvarūpakaḥ] P9,
\item[(illegible/unavailable)] B1, Bo1, Bo2, Bo3, C5, J2, N12, N20, P3,

  \begin{description}
    • It is given at 74ab at Ba
    • It is given at 85ab in C2
    \end{description}

  \end{marma}

  \begin{marma}[hp03_056]

\sthana{3.57d}

\item[niḥśvasya] A1, B2, Ba, Ba5, C6, J14, LD1, N6, N10, N17, P4, P12,
\item[niḥśvayād] J13,
\item[niścitam] J1, J3, J7, Ko, N8, N11, N16, N18, N23, V19, N19, 
\item[niśvasyād] C4, N25, P5, 
\item[niścayād] N5,
\item[niśvasya] B3, C2, C8, C9, J2, J4, J12, J15, J16, J17, Jyo, N3, N4a, N4b, N7, N9, N12, N13, N15, N21, N22, N24, P1, P2, P7, P8, P11, P13, 
\item[nīśvasya] LD2,
\item[niśrayād] K2,
\item[niśvāsya] J5,
\item[nisvan] K1,
\item[nisvasya] C3, P9, P10, 
\item[nisvatya] N14,
\item[viśvasya] V3, J8, J11, V1, J10, P6, 
\item[(illegible/unavailable)] B1, Bo1, Bo2, Bo3, C5, J6, N20, P3,

  \begin{description}
    • It is given at 86d in C2
    • Folio no. 25 is missing in J6
    \end{description}

\end{marma}

\begin{marma}[hp03_059]

\sthana{3.59}

\item[uḍḍiyānākhyo] B2, C9, J4, J7,J10, J15, J16, LD1, N6, P8, P9,
\item[uḍḍiyānāṣo] N15,
\item[uḍiyaṇākhyo] J5,  
\item[uḍiyanākhyo] J17,
\item[uḍḍiyaṇākhyo] B3,
\item[uḍḍiyanākhyo] N7,
\item[uḍḍiyanākhaṃ] P7,
\item[uḍiyaṇākhyoyaṃ] P11,
\item[uḍḍiyaṇākhyoyaṃ] C2, J3, J12, J13, 
\item[uḍḍiyāṇākhyoyaṃ] K1, P12, 
\item[uḍḍiyāṇākhyoyāṃ] P13,
\item[uḍḍiyāṇākhyo] N21,
\item[uḍḍiyāṇakākhyo] Ko,
\item[uḍḍiyanākheyaṃ] J1, J11, N3, 
\item[uḍiyānākhyo] P10,
\item[uḍiyāṇākhyo] P4,
\item[uḍḍiyānākhyoyaṃ] N13, N16, N25, 
\item[uḍiyanākheyaṃ] J8,
\item[uddiyanākhyoyaṃ] N23,
\item[uḍḍiyanākhyoyaṃ] C4, K2, N18, 
\item[uḍḍīyanākhyoyaṃ] N4a, N5, N8, N11, N12, 
\item[uḍiyanākhyoyaṃ] N10,
\item[uḍiyakākhyoyaṃ] N24,
\item[uḍīyāṇākhyo] B1, P2, 
\item[uḍīyānākhyo] C3, N9, P1, P5, 
\item[uḍḍīyānākhyo] LD2,
\item[uḍyāṇākhyo] J2,
\item[uḍiyānākhyaṃ] C6, C8, 
\item[uḍḍiyānākhyaṃ] A1, 
\item[uḍḍiyanākhoyaṃ] Ba, 
\item[uḍḍiyabaṃdhoyaṃ] J14,
\item[udiyānākhya] N14,
\item[udiyānākhyo] N17,
\item[uḍāyamāṇākhyo] N22,
\item[uddīyeṇākhyoyaṃ] P6,
  \item[(illegible/unavailable)] Ba5, Bo1, Bo2, Bo3, C5, J6, N4b, N20, P3,

  \begin{description}
    • It is given at 93cd in B1
    \end{description}

  \end{marma}

\begin{marma}[hp03_060]

\sthana{3.60d}

\item[tatra] A1, B2, Ba5, C6, C8, C9, J5, J7, J8, J10a, J11, J12, J14, K1, Ko, LD1, LD2, N4a, N7, N10, N11, N13, N22, N23, N24, P7, P10, V3, N19, P6, P11, V1, Jyo
\item[atra] B1, B3, C2, J4, J13, N21, P2, P4, P12, P13, 
\item[kṣatra] N15,
\item[kṣetra] C3, J10, J15, J16, J17, N6, N9, N14, N17, P1, P8, P9, 
\item[kṣetraṃ] P5,
\item[mūla] C4, J1, J3, K2, N3, N5, N8, N12, N16, N18, N25, V19  
  \item[(illegible/unavailable)] Ba, Bo1, Bo2, Bo3, C5, J2, J6, N4b, N20, P3,

  \begin{description}
    • It is given at 94cd in B1
    \end{description}

\end{marma}

\begin{marma}[hp03_061]

\sthana{3.61ab}

\item[udare paścimaṃ tānaṃ nābher ūrdhvaṃ ca kārayet] B3, Ba5, C2, C3, C4, C6, J3, J4, J5, J7, J15, K2, N4a, N5, N13, N25, P2, P4, P7, P11, P12, 
\item[udare paścimetānaṃ nābher ūrdhvaṃ tu kārayet] N6, N14, 
\item[udare paścimaṃ tānaṃ tānaṃ nābher ūrdhvaṃ tu kārayet] N15,
\item[udare paścime tāle nābher ūrdhvaṃ ca kārayet] N8,
\item[udare paścimetānaṃ nābhe nābher ūrdhvaṃ tu kārayet] N7,
\item[udare paścimatāo nābher ūrdhvaṃ ca kārayet] J16,
\item[udare paścimaṃ tānaṃ nābher ūrdhve ca kārayet] J13, 
\item[udare paścime tānaṃ nābher ūrdhvaṃ ca kārayet] N16,
\item[udare paścime tālaṃ nābher ūrdhve ca kārayet] N11,
\item[udare paścime tānā nābher ūrdhvaṃ ca kārayet] N18,
\item[udare paścimaṃ tānāṃ nābher ūrdhvaṃ ca kārayet] J12,
\item[udare paścime tānaṃ nābher ūrdhvaṃ ca kārayet] J11, N12, 
\item[udarāt paścime bhāge nābher ūrdhvaṃ ca kārayet] J8,
\item[udarāt paścime bhāge adho nābher nigadyate] LD2,
\item[udare paścimatānaṃ nābher ūrdhvaṃ ca kārayet] B1, N3, 
\item[udare paścime yātuṃ nābhiruddhe ca kārayet] J1,
\item[udare paścimatānena nābher ūrdhvaṃ ca kārayet] C8,
\item[udare paścime tānaṃ nābher ūrdhvaṃ tu kārayet] N17,
\item[udare paścimaṃ tānaṃ nābher ūrdhvaṃ tu kārayet] A1, B2, C9, K1, LD1, N9, N21, N22, N24, P5, P6, P8, P9, P13, 
\item[udare paścimatānaṃ nābher ūrdhve tu kārayet] J17,
\item[udare paścimatānaṃ nābher ūrdhvaṃ ca kārayet] N10,
\item[udare paścimottānaṃ nābher ūrddhe tu kārayet] P1,
\item[udare paścimaṃ tānaṃ nābher ūrddhe tu kārayet] J10,
\item[udara paścimantānaṃ nābhebher ūrdhaṃ akārayet] N23,
\item[udara paścimantānaṃ nābhebher ūrdhaṃ tu kārayet] B1,
\item[udaraṃ paścimaṃ tāṇaṃ nābher ūrdhve tu kārayet] P10,
\item[udaraṃ paścimaṃ tāṇaṃ nābher ūrdhvaṃ ca kārayet] J14, Ko, 
\item[paścimaṃ tānam ure (tā)raye dṛḍhatāṃ gate] Ba, 
  \item[(illegible/unavailable)] Bo1, Bo2, Bo3, C5, J2, J6, N4b, N20, P3,

    \begin{description}
      Tāna|tāṇa: the latter is because of influence from vernacular pronunciation.
    • It is given at 95ab in B1
    • It is given at 95ab in B3
    \end{description}

\end{marma}


\begin{marma}[hp03_062]

  \sthana{3.62b}

\item[guruṇā kathitaṃ yathā] C4, J1, J3, J7, J14, K2, N5, N8, N11, N16, N18, N23, N25, 
\item[guruṇā kathitaṃ tathā] B2, K1, N22, P6, 
\item[guruṇā kathitaṃ tadā] Ba5, 
\item[guruṇā kathitaṃ sadā] A1, B1, B3, C2, C3, C6, C8, C9, J4, J5, J10, J11, J12, J13, J16, J17, Ko, LD1, N3, N4a, N6, N7, N9, N10, N12, N13, N14, N15, N17, N21, P1, P2, P4, P5, P7, P8, P9, P10, P11, P12, P13, 
\item[guruṇā kathita sadā] J15,
\item[guruṇāṃ kathitaṃ sadā] J8,
\item[guruṇāṃ kathitaṃ sadāḥ] LD2,
\item[kathitaṃ guruṇā sadā] N24,
\item[(illegible/unavailable)] Ba, Bo1, Bo2, Bo3, C5, J2, J6, N4b, N20, P3,

  \begin{description}
    The word \emph{sadā} (instead of \emph{yathā}) is also well attested. However, the first hemistich seems more difficult to understand with  \emph{sadā}. Perhaps, the easiest way to make sense of \emph{sadā} is to read it with the verb \emph{abhyaset}; `Even an old man who regularly and tirlessly practises \emph{uḍḍiyāṇa}, which is easy when taught by a guru, becomes young.'
    • It is given at 95ab in B1
    • It is given at 96b in B3

  \end{description}

\end{marma}

  \sthana{3.62cd}

\item[abhyaset tad anaṃdas tu vṛddho’pi taruṇāyate ||] N16,
\item[abhyaset tad atandras tu vṛddho’pi taruṇāyate ||] C4, J1, J3, J7, K2, N5, N11, N25, 
\item[abhyase tad atandras tu vṛddho’pi taruṇāyate ||] N8,
\item[abhyāsed astataṃdras tu vṛddho’pi taruṇo bhavet] J5, P6, 
\item[abhyāsed astatadras tu vṛddho’pi taruṇo bhavet] N3,
\item[abhyāset satataṃ yastu vṛddho’pi taruṇo bhavet] P1,
\item[abhyāset dastataṃdras tu vṛddho’pi taruṇo bhavet] N22,
\item[abhyāsen na taṃdras tu vṛddhāpi taruṇāyate ||] N23,
\item[abhyāsataḥ svataṃtras tu vṛddho’pi taruṇāyate] A1, C6, C8, P7, 
\item[abhyased asya taṃtraṃ tu vṛddho’pi taruṇāyate] J11,
\item[abhyaset satataṃ yas tu vṛddho’pi tarṇāyate] Ba5, N4a, N13, N18,
\item[abhyased aniśaṃ yogī vṛddho’pi taruṇāyate ||] J14,
\item[abhyased aniśaṃ yogī vṛddho’pi taruṇo bhavet ||] N21, P13, 
\item[abhyaset satataṃ yas tu vṛddas tu taruṇo bhavet] LD1,
\item[abhyaset satataṃ yas tu vṛddho’pi taruṇo bhavet] B1, C2, C3, J10, J13, J17, LD2, N6, N7, N10, N12, N15, N17, P2, P4, P5, P8, P9, P10, 
\item[abhyaset satataṃ yas tu vṛddho’pi sa taruṇo bhavet] P12,
\item[abhiset satataṃ yas tu vṛddho’pi taruṇo bhavet] N9,
\item[abhyased astadataṃdras tu vṛddho’pi taruṇo bhavet] Ko,
\item[abhyaset satataṃ yas tu vṛddhopi tarūṃṇau bhavet] J15, J16, 
\item[abhyaset satataṃ yas tu vṛdho’pi taruṇo bhavet] C9,
\item[abhyasyat satataṃ yas tu vṛddho’pi taruṇaṃ bhavet] B3,
\item[abhyāse satataṃ yas tu vṛddho’pi taruṇo bhavet] J12, N14,
\item[abhyāsāt satataṃ yas tu vṛddho’pi taruṇo bhavet] J8, 
\item[abhyased asvataṃtran tu vṛddho’pi taruṇo bhavet] B2, 
\item[abhyased asvataṃtras tu vṛddho’pi taruṇo bhavet] J4, K1, P11, 
\item[abhyased ataṃ saṃyogād vaddhopi taruṇo bhavet] N24, 
\item[(illegible/unavailable)] Ba, Bo1, Bo2, Bo3, C5, J2, J6, N4b, N20, P3,

  \begin{description}
The reading \emph{astatandraḥ} is probably the original one in the \emph{Dattātreyayogaśāstra} and is attested in some manuscripts of the \emph{Haṭhapradīpikā}. However, it is not strongly supported by the witnesses of the important groups and so may have changed to something simpler, such as \emph{tad atandras tu}.
    • It is given at 96cd in B1
    • It is given at 96cd in B3

  \end{description}

\end{marma}

\begin{marma}[hp03_064]

  \sthana{3.64ab}

\item[sati vajrāsane pādau karābhyāṃ kāraye dṛḍham] J4,
\item[sati vajrāsanaṃ pādau karābhyāṃ kāraye dṛḍham] N3,
\item[sati vajrāsane pādau karābhyāṃ kārayed dṛḍham] N5,
\item[sati vajrāsane pādau karābhyāṃ dhārayed dṛḍham] A1, Ba5, C2, C4, C6, J3, J14, J17, LD1, N6, N12, N13, N14, N25, P5, P6, 
\item[sati vajrāsanaṃ pādau karābhyāṃ dhārayed dṛḍham] P11,
\item[sata bajrāsane pādau karābhyāṃ dhāraye dṛḍham] K2,
\item[sati vajrāsane pādau karābhyāṃ dhāraye dṛḍham] C3, C8, J1, J7, J11, J13, J15, K1, N7, N9, N15, N17, N18, N21, N22, N24, P1, P2, P4, P8, P9, P10, P12,
\item[sati vajrāsane pādau karābhyāṃ dhāraye dṛḍhau] N10,
\item[satī vajrāsane pādau karābhyāṃ dhāraye dṛḍham] LD2,
\item[sati vajrāsane pādau karābhyāṃ dhāraye dṛḍhai] J12,
\item[sati vajrāsane pādo karābhyāṃ dhāraye dṛḍham] J8,
\item[sati vajrāsane pādau karā- sandhāraye dṛḍhe] N23,
\item[sati vajrāsane pādau karābhyāṃ dhāraye draḍham] C9,
\item[sati vajrāsane pādau karābhyāṃ kāraṃye draḍham] Ko,
\item[sati vajrāsane pādau karābhyāṃ dhārayad draḍham] J10,
\item[śanai vajrāsane pādau karābhyāṃ dhāraye dṛḍham] N8, N11,
\item[sati vajrāsane pādau karābhyāṃ dhāraye dhruvam] B3, 
\item[saṃti vajrāsane pādau kadābhyāṃ dhāraye dṛḍhāṃ] P13,
\item[saṃti vajñāsane pādau kadābhyāṃ dhāraye dṛḍham] J2, P7, 
\item[saṃti jjāsane pādau karābhyāṃ kāraye dṛḍham] J5,
\item[saṃsthi vajrāsane pādau karābhyāṃ dhāraye dṛḍham] N16,
\item[saṃsthita vajrāsane pādau karābhyāṃ dhārayed dṛḍham] B1,
\item[satīva āsane pad....rābhyaṃ dhāraye dṛḍhaṃ] B2,
\item[śatri vajrāsane pādau karābhyāṃ dhāraye dṛḍho] J16,
\item[(illegible/unavailable)] Bo1, Bo2, Bo3, C5, J6, N4a, N4b, N20, P3,

  \begin{description}
    • It is given at 98ab in B1
    • It is given at 116ab in Ba5
  \end{description}

\end{marma}

\begin{marma}[hp03_064cd]

  \sthana{3.64cd}

\item[ulphadeśasamīpe ca udaraṃ tat prapīḍayet] J1, J3, N11,
\item[ulphadeśe samīpe ca udaraṃ tat prapīḍayet] N16,
\item[gulphadeśasamīpe ca kandaṃ tatra prapīḍayet] B1, J12, K1, Ko, N10, N13, P7, 
\item[gulphadeśasamīpe ca kaṃda tatra prapīḍayet] P13,
\item[gulphadesasamīpe ca kandaṃ tatra prapīdayet] N21, 
\item[gulphadeśasamīpa ca kaṃdaṃ tatra prapīdayat] N23,
\item[gulphadeśasamīpe ca kaṃpaṃ tatra prapīḍayet] B3, J7, J16, 
\item[gulphe deśe samīpe ca kaṃdaṃ tatra prapīḍyate] C3,
\item[gulphe deśe samīpe ca kaṃpaṃ tatra prapīḍayet] C2, P12, 
\item[gulphadeśe samīpe ca kaṃpaṃ tatra prapīḍayet] J13, P2, P4, 
\item[gulphadeśe samīpaś ca kandaṃ tatra prapīḍayet] B2,
\item[gulphadeśe samīpaṃ ca kaṃdaṃ tatra prapīḍayet] J4,
\item[gulphadeśasamīpaṃ ca kaṃdaṃ tatra prapīḍayet] N5, N25, 
\item[gulphadeśasamīpe ca gudaṃ tatra prapīḍayet] J14,
\item[gulphadeśasamīpaṃ ca skaṃdhaṃ tatra prapīḍayet] K2,
\item[gulphadeśe samīpaṃ ca skaṃdhaṃ tatra prapīḍayet] C4, 
\item[gulphadeśasamīpe ca kanda tatra prapīḍayet] A1, 
\item[gulphadeśasamīpe ca kadaṃ tatra prapīḍayet] Ba5, C6, 
\item[gulphadeśasamīpe ca kaṇṭhaṃ tatra prapīḍayet] N12, P11, 
\item[gulphadesasamīpe ca kaṃṭhaṃ tacca prapīḍayet] LD2,
\item[gulmadeśasamīpe ca kaṃṭha tacca prapīḍyate] N7,
\item[gulphadesasamīpe ca kaṃdaṃ tacca prapīḍyate] C9, N17, P8, P9, 
\item[gulphadeśasamīpe ca kaṃdaṃ ta prapīḍyate] P10,
\item[gulphadeśasamīpe ca kaṃdaṃ tacca prapīḍyate] P5,
\item[gulmadeśasamīpe ca kaṃdaṃ tatra prapiḍayet] N22,
\item[gulmadeśasamīpe ca kaṃdaṃ tacca prapīḍyate] J17, N6, N9, P1, 
\item[gulphadeśasamīpe ca kaṃdaṃ tatra prapīḍyate] N15,
\item[gulphadesasamīpe ca kaṃdaṃ tatra prapīḍyate] J15,
\item[gulphadeśasamīpe ca kaṃdaṃ tacca prapīḍyate] J8, J10, 
\item[gulphadeśasamīpe ca kaṃdaṃ tatra prapīḍayet] J11,
\item[gulphadeśaṃ samīpe ca kaṃdaṃ tava prapīḍyate] N3,
\item[gulphadeśasamīpe ca kaṃdhan tasya prapīḍyate] N14,
\item[gulphadeśasamīpe ca kaṃde tacca prapīḍayet] LD1,
\item[gulphadeśasamīpe ca kaṃ(dha) tatra prapīḍayet] C8, J6, 
\item[gulphadeśe samāvaṃtu kaṃṭha tatrāpi pīḍayet] J2,
\item[gulphadeśe samitye vā kaṃṭhaṃ tatra pradāyate] J5,
\item[gulphadeśasamīpe ca udaraṃ tat prapīḍayet] N8,
\item[gudadeśasamīpe ca kaṃdaṃ tacca prapīḍayet] N18,
\item[gulphadeśasamīpe tu kandaṃ tatra prapīḍayet] N24,
\item[gulpadeśasamipe ca kaṃdaṃ tatra sapīḍayet] P6,
\item[(illegible/unavailable)] Bo1, Bo2, Bo3, C5, N4a, N4b, N20, P3,

  \begin{description}
In the \emph{Gorakṣaśataka} this verse comes before any instruction to perform \emph{uḍḍiyāna}, so it appears to refer to pressing the \emph{kanda} with the feet, which is plausible if the \emph{kanda} is located in the gential region. However, Sundaradeva and Bhavadevamiśra think the \emph{kanda} is located near the navel. In their view, the yogi presses the \emph{kanda} by holding the legs near the ankles with the hands and performing \emph{uḍḍiyānabandha} (at the end of \emph{kumbhaka} and the beginning of exhalation, according to Sundaradeva). In this view, it is the backward stretch in the abdomen (\emph{udare paścimatāna}) that presses the \emph{kanda} in the navel.
  \end{description}

\end{marma}

\begin{marma}[hp03_065]

  \sthana{3.65b}

\item[upīpa? hṛdaye gale] N15,
\item[kāraye hṛdayaṃ gataḥ] J4,
\item[kārayed dhṛdaye gale] A1,  
\item[kārayed hṛdaye gale] B2, C6, J5, J7, J11, LD1, LD2, N3, N6, N12, N17, N21, N22, N24, P7, P11, 
\item[kāraye hṛdaye gale] C8, J8, K1, P13, 
\item[kārayed hṛdaye gataiḥ] N23,
\item[kārayedradaye gale] P6,
\item[cibukaṃ hṛdi] J1,
\item[kārayec cibukaṃ hṛdi] B1, C2, C4, J3, J13, J14, K2, N5, N8, N11, N16, N18, N25, P2, P12, 
\item[kāraye cibukaṃ hṛdi] B3, Ko, 
\item[kārayec cubukaṃ hṛdi] P4,
\item[kāraye dṛḍhaye nare] N14,
\item[pīḍayetadaye gate] J12,
\item[pīḍayeddhadaye gate] J16,
\item[pīḍayed hṛdaye gate] N10,
\item[pīḍayed radaye gale] P8,
\item[pīḍayed hṛdaye gale] C3, J10, J17, N7, N9, P1, P5, P9, P10, 
\item[pīḍayed hṛdaye galaṃ] J15,
\item[pīḍaye hṛdaye gale] C9,
\item[(illegible/unavailable)] Ba5, Bo1, Bo2, Bo3, C5, J2, J6, N4a, N4b, N13, N20, P3,

  \begin{description}
  \end{description}

\end{marma}

\begin{marma}[hp03_065]

  \sthana{3.65d}

\item[ūrddhvaasaṃdhiṃ na gachati] J11,
\item[tundasaṃdhiṃ na gacchati] J4, P11, 
\item[tundasaṃsi na gachati] N22,
\item[tundasaṃdhiṃ na gachati] J7, N3, 
\item[(khedaṃ) saṃdhiṃ na gacchati] J1,
\item[daṃtaṃ gaṃdhi nigachati] N15,
\item[datta siddhir nigacchati] P7,
\item[daṃtaṃ gaṃdhiṃ nigacchati] P8,
\item[daṃtasaṃdhiṃ na gacchati] C9,
\item[daṃtaṃgaṃdhi nigacchati] P9,
\item[tanosiṃdhiṃ na gachati] Ko,
\item[tudasadhi na gachati] N11,
\item[tudan sādhiṃ na gachati] B2,
\item[taṃda saṃdhiṃ na ga] N23,
\item[dasaṃdhiṃ nā gachati] J8,
\item[tuṇḍasaṃdhiṃ na gacchati] N12,
\item[tundasiddhiṃ nigacchati] A1, C6, 
\item[tudasadhi na gachati] N8,
\item[tudasaṃdhi na gachati] P6,
\item[(ruṃṭa)siddhaṃ ca gacchati] C8,
\item[tathā saṃdhiṃ na gachati] N24,
\item[tadā saṃdhiṃ ca gacchati] C4, K2, N5, N25, 
\item[nāḍīsaṃdhiṃ na gacchati] C3, J11a, N21, P5, 
\item[nāḍīsaṃdhi na gachati] J13, N9, P10, P13, 
\item[nāḍīsiddhiṃ ca gachati] N7,
\item[nāḍīsiddhiṃ nigachati] P1,
\item[nāḍīsiddhiṃ na gachati] LD1,
\item[nāḍīsaṃdhi nigacchati] J10, J17, 
\item[nāḍīsiddhi niyacchati] J12,
\item[nāḍīsidhiṃ niyachati] J
\item[nāḍisaṃdhi nīyacchatī] J16,
\item[nāḍīśuddhiṃ nigachati] LD2,
\item[rudasaṃdhiṃ na gachati] K1,
\item[skaṃdasaṃdhi na gachati] J5,
\item[skandhasaṃdhin na gacchati] N16,
\item[skandasaṃdhiṃ na gacchati] J3,
\item[skandasaṃdhiṃ nirgachati] J14,
\item[skandasaṃghena gacchati] B1,  
\item[skandasaṃgena gachati] B3, C2, 
\item[tadā gaṃ?dhiṃ pragacchati] N6,
\item[tadā gaṃdhiṃ pragachati] N17,
\item[uttamodyadyānavandhanāt] N14,
\item[ṣedaṃ saṃdhi na gakṣati] N18,
\item[kaṃdasagena gachati] P2,
\item[kaṃdasaṃgena gachati] P4, P12, 
\item[(illegible/unavailable)] Ba5, Bo1, Bo2, Bo3, C5, J2, J6, J15, N4a, N4b, N13, N20, P3,

  \begin{description}
  \end{description}

\end{marma}

\begin{marma}[hp03_076]

  \sthana{3.76b}

\item[mukhabandhanam] A1, B1, B2, B3, C2, C4, C6, C8, C9, J1, J2, J3, J6, J7, J8, J10, J11, J12, J13, J14, J15, J16, K1, LD1, LD2, N4a, N5, N7, N8, N10, N11, N12, N16, N18, N21, N22, N23, N25, P2, P4, P6, P7, P10, P11, 
\item[mukhavañcanam] Jyo, N4b, N13, 
\item[mukha(caṃ)canam] Ba5,
\item[mukhamaṃdhavaṃ] J5, 
\item[(illegible/unavailable)] Bo1, Bo2, Bo3, C3, C5, J4, J17, K2, Ko, N3, N6, N9, N14, N15, N17, N20, N24, P1, P3, P5, P8, P9, P12, P13, 

  \begin{description}
    • It is given at 111c in B1 
    • It is given at 106b in C2
    • It is striked in N4b
No source? Only Brahmānanda reads mukhavañcanam, which makes better sense.
  \end{description}

\end{marma}

\begin{marma}[hp03_081]

  \sthana{3.81b}

\item[ṣaṇmāsārdhena naśyati] J1, J3, J4, J6a, J7, K1, N5, N8, N16, N18, N23, N25, V19
\item[ṣaṇmāsārdhaṃ praṇasyate] N22,
\item[ṣaṇmāsārdhena paśyati] J6,
\item[ṣanmāsānna ca  dṛśyate] C9,
\item[mūrdham ākṛṣya rakṣayet] B2,
\item[ṣaṇmāsordhvaṃ na dṛśyate] Ba5, N4b,
\item[ṣaṇmāsārddhe na dṛśyate] J5,
\item[ṣaṇmāsārddhaṃ na dṛśyate] J8, N11, N24, 
\item[ṣaṇmāsārdhān na dṛśyate] A1, C6, C8, J11, J14, N3, N9, P7, P11, 
\item[ṣaṇmāsārdhaṃ na dṛśyati] Ko,
\item[ṣaṇyāsorddhaṃ na drasyate] J2,
\item[ṣaṇmāsorddhvaṃ pranasyate] P6,
\item[ṣaṇmāsa tu na dṛśyate] N14,
\item[ṣaṇmāsā tu na dṛśyate] C3, J16, 
\item[ṣaṇmāsāt tu dṛśyate] J15,
\item[ṣaṇmāsāṃ tu na dṛśyate] N7,
\item[ṣaṇmāsān naiva dṛśyate] N12,
\item[ṣaṇmāsāt tu na dṛśyate] B1, B3, C2, J10, J12, J13, J17, LD1, LD2, N6, N10, N15, N17, N21, P1, P2, P4, P5, P9, P10, P12, P13, 
\item[parāmāsāt tu na dṛśyate] P8,
\item[(illegible/unavailable)] Bo1, Bo2, Bo3, C4, C5, K2, N4a, N13, N20, P3,

  \begin{description}
    • It is given at 116b in B1
    • It is given at 111b in C2
  \end{description}

\end{marma}

\begin{marma}[hp03_082]

  \sthana{3.82d}

\item[siddhabhājanaṃ] LD2,
\item[siddhibhājanaṃ] B1, B2, B3, Ba5, C2, C6, C8, J1, J3, J4, J6, J7, J8, J15, N4b, N8, N9, N23, V19, C7, K3, Ko,  N19, V15, J11, J13, K1, N3, N11, N12, N16, N18, N21, N22, N24, N25, P2, P4, P6, P10, P11, P13, P23, V1, V3, Jyo
\item[siddhibhājana] J5,
\item[siddhibhājanaḥ] J14, P7, 
\item[vidhibhājanaṃ] J2,
\item[siddhibhāg bhavet] J12,
\item[siddhimān bhavet] A1, C3, C9, J10, J17, LD1, N6, N7, N10, N14, N15, N17, P1, P5, P8, P9, 
\item[sīddhimān bhavet] J16,
\item[(illegible/unavailable)] Bo1, Bo2, Bo3, C4, C5, K2, N4a, N5, N13, N20, P3, P12, 

  \begin{description}
    • It is noted in C4 that “vajrolī graṃthāṃtare likhitāsīt”
  \end{description}

\end{marma}

\begin{marma}[hp03_090]

  \sthana{3.90b}

\item[bījaṃ] V19, N19, V15, J11, V1, J10, J17, P1, P8, 
\item[bijaṃ] J16, LD2,
\item[bīnduṃ] B2, C6, K1, P7,
\item[vindaṃ] J2,
\item[bīndu] C8,
\item[dhanāṃ] A1, 
\item[jīvaṃ] C7, P23
\item[nijaṃ] Ba5, N4b, N13, Jyo
\item[bījabiṃdu] P9,
\item[bījabinduṃ] C3,
\item[bījaṃ binduṃ ca] J1, J3, J6, J12, J14, LD1, N5, N8, N6, N9, N11, N14, N16, N17, P5, 
\item[bījaṃ bindu ca] N10,
\item[svīyaṃ bindu ca] P11,
\item[rajaṃ bīnduṃ ca] P10,
\item[stribaṃdu caiva] N22,
\item[striṃ biṃduṃ ca] P6,
\item[striyaṃ biṃduṃ ca] N12,
\item[striyo biṃduś ca] N7,
\item[striyā biṃduṃ ca] N23,
\item[jayaṃ bindu ca] J8,
\item[bīrjaṃ binduṃ ca] C9,
\item[jīvaṃ bindu ca] N18,
\item[vīryaṃ binduṃ ca] J5,
\item[nijabinduṃ] B1, B3, C2, Jyo, N21, P2, P4, P12, P13, 
\item[nījaṃ binduṃ ca] J13,
\item[nijaṃ binduṃ ca] J15,
\item[nijaṃ bindu] J4,
\item[jayaṃ] V3
\item[striyā] J7,N23
\item[(illegible/unavailable)] Bo1, Bo2, Bo3, C4, C5, K2, Ko, N3, N4a, N15, N20, N24, N25, P3,

  \begin{description}
    • It is given at 126b in B1 
  \end{description}

\end{marma}

\begin{marma}[hp03_092]

  \sthana{3.92c}

\item[jaleṣu bhasma] B2, Ba5, C3, C9, J1, J3, J6, J7, J10, J11, J12, J14, J15, J16, J17, K1, Ko, LD1, LD2, N5, N6, N8, N7, N9, N10, N11, N13, N14, N16, N17, N18, P1, P5, P6, P7, P8, P9, P10, 
\item[jalesu bhasma] N4b,
\item[jaleṣu bhaśma] J2,
\item[jale bhasma] B1, B3, C2, J4, J13, N21, P4, P12, 
\item[jale tu bhasma] A1,  
\item[taleṣu tasma] N22,
\item[(illegible/unavailable)] Bo1, Bo2, Bo3, C4, C5, C6, C8, J5, J8, K2, N3, N4a, N12, N15, N20, N23, N24, N25, P2, P3, P11, P13, 

  \begin{description}
  \end{description}

\end{marma}

\begin{marma}[hp03_096]

  \sthana{3.96ab}

\item[amarīṃ yaḥ piben nityaṃ nasyaṃ kurvan dine dine] Ba5, C6, C8, J11, N13, P7,
\item[amarīṃ yaḥ piven nityaṃ nasyaṃ kurvan dine dine] N4b,
\item[amarīṃ yaḥ piben nityaṃ naśyaṃ kurvan dine dine] J14,
\item[amarīṃ yo pive nityaṃ nasyaṃ kuryād dine dine] N17,
\item[amarī yaḥ pibe nityaṃ nasya kuryā dine dine] J5,
\item[amarāvyapiva nityaṃ nasyaṃ kurvan dine dine] N22,
\item[amarī ya piben nityaṃ naśyaṃ kurvaṃ dine dine] J1,
\item[amarī yaḥ piven nityaṃ nasyaṃ kurvana dine dine] J8,
\item[amarī yaḥ pīven nītyaṃ nasyaṃ kuryā dīne dīne] LD2,
\item[amarī yaḥ piben nityaṃ naśya kurvvaṃ dine dine] J2,
\item[amarī yaḥ ṣpiven nityaṃ naśyaṃ kurvva dine dine] N18,
\item[amarī yo piben nityaṃ nasyaṃ kuryād dine dine] P9,
\item[amarī yo piven nityaṃ sammyak kuryād dine dine] N14,
\item[amarī yaḥ piven nityaṃ taśya kurvan dine dine] P11,
\item[amarī yaṃ piven nityaṃ paśyan kurvan dine dine] N12,
\item[amarīṃ yaḥ piben nityaṃ tasya kurvan dine dine] A1, J3, J6, 
\item[amaṃrī yaḥ piven nityaṃ tasya kuryā dine dine] C3, J7,
\item[amaṃrī yā piben nityaṃ tapaḥ kuryā dine dinai] J15,
\item[amarīṃ yo piven nityaṃ tasyaṃ kuryā dine dine] P10,
\item[amarīṃ yo piven nityaṃ tasyaṃ kuryyād dine dine] J10,
\item[amarīṃ yaṣ piven nityaṃ tasya kuryyād dine dine] P5,
\item[amarīṃ yo piben nityaṃ tasya kuryyād dine dine] LD1, N9,
\item[amarīṃ yo piben nityaṃ tasyāṃ kuryyād dine dine] P1,
\item[amarī yo pive nityaṃ tasya kuryā dine dine] P8,
\item[amarīṃ yaḥ piben nityaṃ vasyaṃ kuryyād dine dine] B1, 
\item[yamarī yo piben nityaṃ tasyāṃ kuryyād dine dine] B2,
\item[aamarīṃ yo piven nityaṃ nasyaṃ kuryād dine dine] N6,
\item[amarīṃ yo piven nityaṃ tasyaṃ kuryā dine dine] J17,
\item[ammuṃ yogī pibe nityaṃ naśyaṃ kuryā dine dine] B3, C2, 
\item[amarīṃ yaḥ pive nityaṃ nasyaṃ kuryā dine dine] C9,
\item[amarīṃ yaḥ piven nityaṃ naśyaṃ kuryād dine dine] Ko,
\item[amarī yaḥ piven nityaṃ nasyaṃ kuryād dine dine] N7,
\item[amarīṃ jaḥ piven nityaṃ nāśāraṃdhrād dine dine] J10a,
\item[amarīṃ yaḥ piben nityaṃ kriyāṃ kurvan dine dine] J6a,
\item[amarīṃ yaḥ piben nityaṃ vaśyaṃ kuryā dine dine] J4, J13, N21, 
\item[amarī yaḥ piben nityaṃ vaśyaṃ kuryā dine dine] P13,
\item[amarī yeḥ piben nityaṃ sa bhavet makaradhvajaḥ] J12,
\item[amariṃ yaḥ pibin nityaṃ sa bhavet makaṃradhvaja] J16,
\item[amariṃ yaḥ pibin nityaṃ saṃ bhavet makaṃradhvajaḥ] N10,
\item[aṃvarī ca piven nityaṃ naśyaṃ kurvan dine dine] N16,
\item[amuṃ yogī piben nityaṃ nasyaṃ kuryād dine dine] P12,
\item[(illegible/unavailable)] Bo1, Bo2, Bo3, C4, C5, K1, K2, N3, N4a, N5, N8, N11, N15,
 N20, N23, N24, N25, P2, P3, P4, P6, 


  \begin{description}
  \end{description}

\end{marma}

\begin{marma}[hp03_099*1]

  \sthana{3.99*1 additional line}

\item[meḍhram ākuṃcanād ūrdhvaṃ rajasāpi hi yoginī] J17, N17,
\item[meḍhram akuṃcanād ūrdhvaṃ rajasyāpi hi yoginī] N9, P1,
\item[meḍhrām ākuṃcanād ūrdhvaṃ rajasāpi hi yoginī] J6, N6,
\item[meḍhrām ākuṃcanād ūrdhvarajasāpi hi yoginī] P10,
\item[meḍhrād ākuṃcanād ūrdhvaṃ rajasāpi hi yoginī] J3, J6a,
\item[meḍhram ākuṃcayed ūrdhvaṃ rajasāpi hi yoginī] C3, J12, 
\item[meḍhram ākuṃcayed ūrdhvaṃ rajaḥsāpi hi yoginī] N10,
\item[meḍhram ākuṃcanād ūrdhvaṃ rajasāpi hi yogīnī] J15,
\item[meḍhrām ākuṃcanād ūrdhvaṃ rajasāpi hi yoginī] J10, 
\item[meḍhrasyākuṃcanād ūrdhvaṃ rajasyāpi hi yoginī] P5,
\item[meḍyam ākuṃcanād ūrdhvaṃ rajasopi hi yoginī] J11,
\item[meḍhreṇākuñcanād ūrdhvaṃ rajasāpi hi yoginī] A1,
\item[meṃḍham ākuṃcayed ūrdhvaṃ rarajasārpi hi yogīnīḥ] J16,
\item[meḍhrāeṇākarṣayed ūrdhvaṃ samaggamaṅi pā(ṭa)vāt] C8,
\item[rakṣed ākuṃcanād ūrdhvaṃ yā rajaḥ sa hi yoginī] N13,
\item[mehanākuṃcatād ūrdhvaṃ rajasāpi ca yoginī] N16,
\item[mehanākuñcanād ūrdhvaṃ rajasāpi ca yoginī] B1, J4, J13, J14, 
\item[mehanākuñcanād ūrdhvaṃ rajasāpī ca yoginī] N12,
\item[mehanākuñcanād ūrdhvā rajasāpī hi yoginī] P8,
\item[mehanākuñcanād ūrdhvaṃ rajasāpi hi yoginī] K1, N7, N21, P7, P9, P13, 
\item[mehanākuṃcanād ūrdhvaṃ na yasyāpi hi yoginī] C2,
\item[mehenākuñcanād ūrddhaṃ rajasāpi hi yoginī] B2,
\item[mehenākucanād ūrddhva rajasāpi hi yogīnā] N22,
\item[meḍhrmākuñcanād ūrddhvaḥ rajasyāpi hi yoginī] N14,
\item[mehenākuñcanād ūrddhv rajasāpi hi yoginī] J7,
\item[meruākuñcanād ūrdhvaṃ nayaty apī hi yogīnī] B3, 
\item[mehanākuñcanād ūrdhvaṃ na yasyāpi hi yoginī] P2, P4, P12, 
\item[rakṣedākuñcanād ūrdhvaṃ yā rajaḥ sā hi yoginī] N4b, P6, 
\item[(illegible/unavailable)] Ba5, Bo1, Bo2, Bo3, C4, C5, C6, C9, J1, J2, J5, J8, K2, LD1, LD2, N3, N4a, N5, N8, N11, N15, N18, N20, N23, N24, N25, P3, P11, 
  \begin{description}
  \item[Yes] A1; B1; B2; J1; J6; J7; J11; J10; Jyo; P2;
    \item[No] Ba5, C9; J2; J5; K2; Ko; LD1; LD2; N3; N4a; N5; N8; N10; N11; N15; N20; N23; N24; N25; P3;
  \end{description}

\end{marma}

\begin{marma}[hp03_099]

  \sthana{3.99b}

\item[svadehajau] B1, C9, J1, J2, J4, J7, J6, J8, K1, Ko, LD1, V1, V3, N10, V15, J11, J13, J14, N7, N12, N16, N18, N21, P1, P11, 
\item[svadehaje] P6,
\item[svadehajo] LD2,
\item[svadehagau] Jyo, N4b, N13, 
\item[svadehajaḥ] B3, C2, C8, P2, P4, P12, 
\item[svadehajaiḥ] A1, C3, C6, J10, J12, J15, J16, J17, N6, N9, N17, P7, P8, P9, P10, 
\item[svadehijaḥ] P5,
\item[svadehaya] N14,
\item[svadehajaṃ] C7, J5, 
\item[svadehijau] P13,
\item[svadehi(nau)] Ba5,
\item[sadehajaṃ] V19
\item[sadehabhājāṃ] P23
\item[śca dehajaiḥ] B2,
\item[(illegible/unavailable)] Bo1, Bo2, Bo3, C4, C5, J3, K2, N3, N4a, N5, N8, N10, N11, N15, N20, N22, N23, N24, N25, P3,

  \begin{description}
  \end{description}

\end{marma}

\begin{marma}[hp03_101]

  \sthana{3.101b}

\item[bhogabhuktivimuktidaḥ] B1, C2, J4, J13, LD1, P4, 
\item[bhogamuktinabhuktida] N22,
\item[bhogībhuktivimuktibhiḥ] K1,
\item[bhoge mukti vimuktiha] J16,
\item[bhogamuktivimuktidaḥ] N17, P2, P5, P8, P9, 
\item[bhogamuktivimuktidaṃ] N14,
\item[bhogamuktavimuktidaḥ] P1,
\item[bhogabhuktivimuktadaḥ] C9,
\item[bhogayuktavimuktidaḥ] LD2,
\item[bhogair yuktetha muktidaḥ] N12,
\item[bhoge muktepi muktitaḥ] P13,
\item[bhoge bhukte ca muktidaḥ] J1, N18,
\item[bhoge bhuktepi muktidaḥ] J7, J8, N4b, N13, N16, N21, P6, P11, 
\item[bhoge bhuktopi muktida] J15, N9,
\item[bhojayuktepi muktidaḥ] J11,
\item[bhogabhuktepi muktidaḥ] P10,
\item[bhogabhuktepi muktidā] Ko,
\item[bhogabhukte vimuktidaḥ] J10,
\item[bhoge mukte vimuktidaḥ] P12,
\item[bhoge mukte ca muktidaḥ] J3, J6, 
\item[bhogayogepi muktidaḥ] C6, 
\item[bhoge yogepi muktidaḥ] C8,
\item[bhogamuktivimuktidaḥ] B2, C3, B3, J2, J12, J17, N6, N7,
\item[bhogo muktepi muktidaḥ] Ba5,
\item[bhogayukte’pi muktidaḥ] A1, 
\item[bhogayukto’pi muktidaḥ] J14, P7,
\item[(illegible/unavailable)] Bo1, Bo2, Bo3, C4, C5, J5, K2, N3, N4a, N5, N8, N10, N11, N15, N20, N23, N24, N25, P3,


  \begin{description}
  \end{description}

\end{marma}

\begin{marma}[hp03_110]

  \sthana{3.110}

\item[paristhitā caiva phaṇāvatī sā, 
prātaś ca sāyaṃ praharārdhamātraṃ | 
prasārya sā cārpy avidhānayuktyā 
pragṛhya nityaṃ paricālanīyā ||] C6,

\item[paristhitā caiva phaṇāvatī sā, 
prātaś ca sāyaṃ praharārdhamātraṃ | 
prasārayet tat paridhānayuktyā
pragṛhyate tat paripālanīyā ||] LD1,

\item[paristhitāṃ caiva phaṇāvatī sā 
prātaś ca sāyaṃ praharārdhamātraṃ ||
prātar ya sūryāyavidhānayuktā
pragṛhya paripālanīyā ||] P7,

\item[avasthitasyāṃ phaṇārpayatī tāṃ
prātaś ca sāyaṃ prarārddhamātraṃ 
prayārya yāttat paridhānamuktā
pragṛhyate tat paricālanīyā] P10,

\item[paristhitā caiva phaṇāvatī sā
prātaś ca sāyaṃ praharārddhamātraṃ ||
prapurya suryāt paridhānayuktā
pragṛhyato yat paricālanīyā ||] P6,

\item[avasthitasthaṃdaphaṇārpayaṃtī
prātaś ca sāyaṃ praharārdhamātraṃ ||
prasāryatte tat paridhānamuktā
pragṛhyate tat paripālanīyā ||] P9,

\item[paristhitā caiva phaṇāvatī sā
prātaś ca sāyaṃ praharārddhamātraṃ
prapūrya sūryāparidhānamuktā
pragṛhya niryāty avicāni sā] N25,

\item[paristhitā caiva phaṇāvatī sā
prātaś ca sāyaṃ praharārddha mātraṃ
prapūrya sūryā paridhārna muktā
pragṛhyato yatparicālanīyā] N22,

\item[paristhitā caiva phaṇāvatī sā
prātaś ca sāyaṃ praharārddhamātraṃ 
prasāryya yat tat paridhāya yuktyā
pragṛhyate tat paricālanīyā] N18,

\item[paristhitā caiva phaṇāvatī sā 
prātaś ca sāyaṃ praharārddhamātraṃ
prapūryya sauryyāt paridhāyamuktā
pragṛhya niryāti vicālitā sā] N16, 

\item[avasthitasyāṃdaphaṇārpayaṃtī tāṃ
prātaś ca sāyaṃ praharārddhamātraṃ
prasāhya yaṃ tatparidhānamuktā
yad gṛhyate tat paripālanīyā] P8,

\item[praviśya tasyaiva phaṇāvatī sā
prātaś ca sāyaṃ praharārddhamātraṃ |
prapūrya sūryātmavidhānayuktā
pragṛhya tiryak pravicālitā sā ||] N12,

\item[śayyāsanasthāsya phaṇāvatī sā 
prātaś ca sāyaṃ praharārdhramātraṃ ||
prapūrya śauryāt paridhānayuktyā
pragṛhya nityaṃ paricālanīyā ||] P2,

\item[śayyāsanasthāsvaphaṇāvatī sā
prātaś ca sāyaṃ praharārddhamātraṃ ||
prapūrya śauryāt paridhānayuktyā
pragṛhya nityaṃ paricālanīyā ||] P12,

\item[avasthita syā phaṇārpayaṃtī tāṃ
prātaś ca sāyaṃ praharārddhamātraṃ
prasārya yastāt paridhānamuktā
pragṛhyate tatparicālanīyā ||] N9,

\item[paristhitā caiva phaṇāvati sā 
prātaś ca sāyaṃ praharārddhamātraṃ
prasārya yat tat paridhānayuktyā
pragṛhyate tat paripālanīyā||] LD2,

\item[avasthitā saiva phaṇāvatī sā 
prātaśca sāyaṃ praharārddhamātraṃ
prasārya yāvat paridhānamuktā
pragṛhya niryāti bicālitī sā] J15, 

\item[avaschitasyāṃ phaṇārpayaṃtī tāṃ
prātaś ca sāyaṃ praharārddhamātraṃ ||
prasārya yāvat paridhānamukta
pragṛhyate tat paridhānanīyā ||] P1,

\item[avasthitā caiva phaṇāvatī sā
prātaś ca sāyaṃ praharārddhamātraṃ
prapūya sūryāt paridhānayuktyā
pragṛhya nityaṃ paricālanīyā] N13,

\item[avasthitā caiva phaṇāvati tāṃ
prātaś ca sāyaṃ praharārddhamātraṃ 
prasāryya sūryāt paridhānamuktā 
pragṛhyate te tatparicālanīyā] N17,

\item[avasthitā caiva phanāvatin tāṃ
prātaś ca sāyaṃ praharārddhamātraṃ
prasāryya sūryyā paridhānamuktā
pragṛhyate tat paricālanīyā] N14, 

\item[avasthitasya phaṇārpyayantīyaṃ ||
prātaś ca sāyaṃ praharārddhamātraṃ
prasārya sūryāt paridhānayuktyā |
pragṛhya tyakta paricālanīyāṃ] J17,

\item[avasthitā caiva phaṇāvatīṃ tāṃ
prātaś ca sāyaṃ praharārddhamātraṃ |
prasāryya sūryāt paridhānamuktā
pragṛhyate tat paricālanīyā ||] N6,

\item[avasthitā caiva phaṇārpyaṃatīṃ 
prātaś ca sāyaṃ praharārddhamātraṃ |
āpūrya sūryāt paridhānamuktā
prasahyate tat paricālanīyā ||] N10,

\item[paristhitā caiva phaṇāvatī sā, 
prātaś ca sāyaṃ praharārdhamātraṃ | 
prapūrya sūryāt paridhānayuktā
pragṛhya nityaṃ paricālanīyā ||] J11,

\item[avasthitasyā phaṇārppayaṃtī tāṃ
prātaś ca sāyaṃ praharārddhamātraṃ ||
prasāryya yāttatparidhānamuktā
pragṛhyate tat paricālanīyāḥ ||] P5,

\item[saṃprasthitā caiva phaṇāvatī sā
prātaś ca sāyaṃ praharārddhamātraṃ ||
prapūrya sūryāt paridhānayuktyā
prasāhaṇī yā paricālanīyā ||] N7,

\item[paristhitā caiva phaṇāvatī sā, 
prātaś ca sāyaṃ praharārdhamātraṃ | 
prapūrya sūryāt paridhānayuktā
pragṛhyatīr yat paricālanīyā ||] ko,

\item[paristhi caiva phaṇāvatī sā
prātaś ca sāyaṃ praharārddhamātraṃ
prapūrya sūryāt paridhānayuktyā
pragṛhya niryāt paricālanīyā] K1,

\item[paristhitā caiva phaṇāvatī sā, 
prātaś ca sāyaṃ praharārdhamātraṃ | 
prapūrya sūryāt paridhānamuktā
pragṛhya niryāti vicālitā sā ||] N11,

\item[avasthitā syāt phaṇāvatī sā
prātaś ca sāyaṃ praharārddhamātraṃ
prasahyate tat paricālanīyā
pragṛhya niryāti bicālati sā] J16,

\item[avasthitasyaṃda phaṇāvatī tāṃ
prātaḥ sāyaṃ praharārddhamātraṃ ||
prasārya yat tat pradhānamuktā
pragṛhya tatparicālanīyā ||] N15,

\item[paristhitā caiva phaṇāvatī sā 
prātaś ca sāyaṃ praharārdhamātraṃ | 
prapūrya sūryāt paridhānayuktā 
pragṛhya nipītyavicālinī sā ||] J7, 

\item[avasthitasyaṃdaphaṇārpayāṃtī
prātaś ca sāyaṃ praharārddhamātraṃ ||
prasāryarpāṣṇāt paridhānayuktyā
pragṛhyateta parivālanīyā ||] J10,

\item[avasthitasyaṃdaphaṇārpayāṃtī
prātaś ca sāyaṃ praharārddhamātraṃ ||
prasāryasūrjāt paridhānayuktyā
pragṛhyateta parivālanīyā ||] J10a,

\item[pṛṣṭhisthitasyaiva phaṇāvatī sā
prātaś ca sāyaṃ praharārdharātram |
prapūryyasūryyāt paridhāyayuktā
pragṛhya niryyāt paricālanīyā ||] J8, 

\item[pravistṛtasyavaphaṇāvatī sā
prātaś ca sāyaṃ praharārddharātraṃ
prasāryya sūryātmavidhānayuktyā
pragṛhya niryāt paricālanīyā] P11,

\item[paristhitā caiva phaṇāvatī sā
prātaś ca sāyaṃ praharārdharātram |
prapūrya sūryāt paridhīyayuktā
pragṛhya niryāti vicālitā sā ||] J14,

\item[paristhitā caika phaṇāvatī sā, 
prātaś ca sāhaṃ praharārdhamātraṃ | 
prapūrja śauryāt paridhāyamuktā 
pragṛhya niryāti vicālitā sā ||] J3,

\item[paristhitā caika phaṇāvatī sā, 
prātaś ca sāhaṃ praharārdhamātraṃ | 
prasūrya sauryāt paridhāyamuktā 
pragṛhya niryāti vicālitā sā ||] J6, 

\item[paristhitā caika phaṇāvatī sā, 
prātaś ca sāyaṃ praharārdhamātraṃ | 
prapūrya śauryāt paridhānamuktā 
pragṛhya niryāti vicālitā sā ||] J6a,

\item[paristhitā caiva phaṇāvatī sā 
prātaś ca sāyaṃ harārddhamātraṃ ||
prapuryāt paridhāyayuktāṃ 
pragṛhya nirghātivicāritāsāṃ ||] J1,

\item[paristhitā caiva phaṇāvatī sā, 
paristhitāyaṃ praharārdhamātraṃ | 
prasārya sūryā(rpya) avidhānayuktyā 
yat gṛhya nityaṃ paricālanīyā ||] C8,

\item[paristhitā caiva phaṇāvatī sā, 
prātaś ca sāyaṃ praharārdhamātraṃ | 
prapūrya saudā paridhāna muktā, 
pragṛhya niryāty avicālinī sā ||] C4,

\item[paristhītā caiva phaṇāvatī sā, 
prātaś ca sāyaṃ praharārdhamātraṃ | 
prasūrya sauryā paridhāna muktā, 
pragṛhya niryāty avicālinī sā ||] N5,

\item[paristhītā caiva phaṇāvatī sā, 
prātaś ca sāyaṃ praharārdhamātraṃ | 
prapūrya suryā paridhānamuktā, 
pragṛhya niryāti vicālitā sā ||] N8,

\item[praviśyataviśya tasyaiva phaṇāvatī sā
prātastvasāyaṃ praharārdhamātraṃ ||
māpūryasūryāt paridhānivuktā
pragṛhyatīrthāt paripālanāya] J2,

\item[samyāsanasthasya phaṇāvatī sā 
prātaś ca sāyaṃ prahārārddhamātraṃ ||
prapūrya sauryāt paridhānayuktyā
pragṛhya niryātyavicālitā sā ||] J4, J13, 

\item[avasthitā styāt phaṇam arpayantī
prātaś ca sāyaṃ praharārddhamātram 
prasāryya yā tat paridhānamuktā
pragṛhyate tat paricālanīyā ] C3, 

\item[avasthitāṃ spaṃda phaṇāvatīṃ tāṃ
prātaś ca sāyaṃ praharārdhamātraṃ 
prasārya yatnāt paridhāna(yuktyā)
pragrahvate tatparicālanīyā] C9, 

\item[avasthitā syāt phaṇamṛpiyaṃtī 
prātaś ca sāyaṃ praharārddhamātraṃ
āpūrya sūryāt paridhānayuktā
pravakṣyate tatparimelanīyā] J12, 

\item[paristhitā caiva phaṇāvatī sā 
prātaś ca sāyaṃ praharārdhamātram | 
prasārya sūryāya vidhānayuktā 
pragṛhya niryaṃ paricālanīyā ||] A1,

\item[paristhitā caiva phaṇāvatī sā 
prātaś ca sāyaṃ praharārdhamātram | 
prapūrya sūryāt paridhānayuktyā 
pragṛhya niryāt paricālanīyā ||] B2, 

\item[śaivāsanasthāsya phaṇavatī sā 
prātaś ca sāyaṃ praharārddhamātraṃ |
prapūrya śauryāt paridhānayuktyā
pragṛhya nityaṃ paricālanīyā ||] B3,

\item[śayyāsanasthasya phaṇāvatī sā
prātaś ca sāyaṃ praharārtdhamātraṃ
prapūrya sūryāt paridhānayuktyā
pragṛhya nityaṃ paricālanīyā] P13,

\item[śayyāsanasthāsya phaṇavatī sā 
prātaś ca sāyaṃ praharārddhamātraṃ |
prapūrya śauryāt paridhānayuktā
pragṛhya nityaṃ paricālanīyā ||] C2,

\item[śayyāsanasthāsya phaṇāvatī sā
prātaś ca sāyaṃ praharārddhamatraṃ ||
prapūrya śauryāt paridhānayuktyā
pragṛhya nityaṃ paricālanīyā ||] P4,

\item[avasthitā caiva phaṇāvatī sā
prātaś ca sāyaṃ praharārddhamātram ||
prapūrya sūryāt paridhānayuktyā
pragṛhya nityaṃ paricālanīyā ||] Ba5, N4b,  

\item[śayyāśanasthasya phaṇāvatī sā
prātaś ca sāyaṃ prahārddhamātram
prasūryasūryāt paridhānayuktyā
pragṛhya nityaṃ paricālanīyā] N21,

\item[(illegible/unavailable)] B1, Bo1, Bo2, Bo3, C5, J5, K2, N3, N4a, N20, N23, N24, P3, 

  \begin{description}
    • It is given at 114 in C2
    • It is given at 137 in J1
We need this verse from every mss. 
    
    No known source. Sundaradeva makes sense of this verse by equating the tongue with kuṇḍalinī. This enables him to understand the reference to the cloth (\emph{paridhāna}) as the technique of wrapping the tongue in a cloth and milking it, which is a practice associated with khecarī mudrā. This interpretation also makes sense of the next verse in the \emph{Haṭhapradīpikā}, which describes the cloth.
  \end{description}

\end{marma}


\begin{marma}[hp03_114]

  \sthana{3.114}

\item[nāsādakṣiṇavāhimārgapavano ghrāṇena dīrghī kṛtaḥ
candrāmbhaḥ paripūritāmṛtatanuḥ prā ghaṇṭikāyās tathā/
(chaṃdaḥ) kālaviśālavahnipavanān bhrūrandhranāḍīgaṇān
tatkāyaṃ kurute punar navataraṃ jīrṇadrumaskaṃdhavat//] C6,

\item[nāsādakṣiṇamārgavāhipavanāt prāṇotidīrghīkṛtaś 
caṃdrāṃtaḥ paripūritāmatatanuḥ prākghaṃṭikāyās tathā ||
bhiṃdatkālaviśālavanhivaśagān bhrūraṃdhranāḍīgaṇān
tatkāryaṃ kurute punar navataraṃ jīrṇadrumaskaṃdhavat ||] P2,

\item[nāsādakṣiṇamārgavāhipavanāt prāṇotidīrghīkṛtaś
caṃdrāṃtaḥ paripuritāmṛtatanuḥ prāgghaṃṭikāyās tathā
bhiṃdatkālaviśālavanhivaśagān bhrūraṃdhranāḍīgaṇān
tatkāryaṃ kurute punarnavataraṃ jīrṇadrumaskaṃdhavat ||] P4, 

\item[nāsādakṣiṇamārgavāhipavanaḥ prāṇotidīrghikṛtaḥ
ścaṃdrābhaḥ paripūritāmṛtatanuḥ prākghaṃṭikāyāstathā ||
chiṃdan kālaviśālavahnivaśagāt bhrūraṃdhranāḍīgaṇāt
taṃ kāyaṃ kurute pūnar navatanuṃ jīrṇakramaḥ skaṃdhavat ||] P6,

\item[nāsādakṣiṇamārgavāhipavanāṃ prāṇebhidīrghaśkṛteś
caṃdrāṃbhaḥ paripūratorimṛtatanuḥ prāgghaṃṭikāyās tathā ||
chinnaṃ kālaviśālavanhivaśagaṃ prāgraṃdhanāḍīgaṇā
tatkāyaṃ kurute punar navataraṃ chinnadrumaskaṃdhavat ||] P9,

\item[nāsādakṣiṇamārgavāhipavanaḥ prāṇotidīrghākṛtaś
caṃdrābhaḥ paripūritāmṛtanuḥ prāgghaṃṭikāyās tathā ||
chidat kālaviśālavahnivaśagaṃ bhrūraṃdhranāḍīgaṇās
tatkayaṃ kurute punarnavataraṃ chinnadramaskaṃdhavat ||] N7,

\item[nāsādakṣiṇavārgāvāhipavanāṃtaprāṇebhidīghīkṛteś
caṃdrāṃbha paripūritāmṛtatanuḥ prāḍaghaṃṭikāyās tathā
chinnatkālaviśālavahnivaśaṃgaṃ prāgurdhanāḍīgaṇāt
tatkāryaṃ kurute punar nnavattaraṃ chinnadrumaskaṃdhavat] P8,

\item[nāsādakṣiṇamārgavāhipavanāt prāṇobhidīrghākṛteś
caṃdrābhaḥ paripūritāṃ mṛtatanuḥ prāgbaṃdhakāyāyethā ||
chinnatkālaviśālavahnivaśagaghyāghragranāḍīgaṇāt
tatkāryaṃ kurute punarnnavataraṃ chinnadrumaskaṃdhavat ||] P1,

\item[nāsādakṣiṇamārggavāhipavanāt prāṇobhidīrghākṛteś
caṃdrābhaḥ paripūritāṃmṛtatanuḥ prāgghaṃṭikāyas tathā ||
chinnatkālaviśālavahnivaśagaṃ prāgadhonāḍīgaṇān
na tatkāryyāt kurute punarnnavataraṃ sthichinnaṅgulamaskaṃdhavat ||] P5,

\item[nāsādakṣiṇamārgavāhipavanāt prāṇotidīrghīkṛtaḥ
caṃdrātaḥ paripūritāmṛtatanu prāgghaṃṭikāyās tathā ||
bhiṃdat kālaviśālavanhivaśagān bhrūraṃdhranāḍīgaṇān
tatkāryaṃ kurute punar navataraṃ jīrṇadrumaskaṃdhavat ||] P12,

\item[nāsādakṣiṇamārgavāhipavanāt prāṇopi dīrghākṛtaś
candrāntaḥ paripūritāmṛtatanuḥ prāghaṇṭikāyā yathā |
chindan kālavikālavahniparamān bhrūrandhranāḍīgaṇān
taṃ kāyaḥ kurute punarnavataraṃ jīrṇādayaḥ skaṃdhavat ||] N12,

\item[nāśādakṣiṇavartavāhipavano prāṇeti dīrghākṛtaṃ
caṃdrāṃtaḥ paripūrya pūritatanuḥ prāgghaṃṭikāyās tathā
bhiṃdat kālaviśālavahnivaśagān bhrūraṃdhranāḍigaṇās
taṃ kāyaṃ kurute punar navataraṃ jīrṇadrumaskaṃdhavat] N25,

\item[nāśādakṣiṇamārgravāhipavanāt prāṇopi dīrghīkṛtaḥ
caṃdrāṃśāt paripūritāmṛtatanuḥ prāgghaṃtikāyāḥ pathā
chiṃdat kālaviśālavadriparamān bhrūraṃdhranāḍīguṇān
taṃ kāyaṃ kurute punar navataraṃ jīrṇadrumaskaṃdhavat] P11,

\item[nāsādakṣiṇamārgavāhipavanāt prāṇebhidīrghākṛteś
caṃdrāṃtaḥ paripūratāmṛtatanuḥ prāgghaṃṭikāyās tathā
cchinakālaviśālavahnivaśagaṃ prāgaṃdhvanāḍīgaṇāt
tatkāryaṃ kurute punarnnavattaraṃ cchinnadramuskaṃdhavat] P10,

\item[nāsādakṣiṇamārgavāhipavanoghrāṇenadīrghākṛtaṃ
caṃdrāṃgaḥ paripūritāmṛtatanuḥ prāgghaṭhikāyās tathā ||
chaṃdaḥ kālaviśālavanhipavanād bhrūraṃdhranāḍīgaṇā
tatkāryaṃ kurute punarnavataraṃ jīrṇadrumaskaṃdhavat ||] P7,

\item[nāsādakṣiṇamārgavāhipavanāt prāṇopi dīrghīkṛtaś
candrāmbhaḥ paripūritāmṛtatanuḥ prākghaṃṭikāyas tathā
bhindan kālaviśālavanhivaśagān bhūrandhranāḍīgaṇān
taṃ kāyaṃ kurute punar navataran jīrṇadrumaskandhavat] N16,

\item[nāsādakṣiṇamārggavāhipavanāt prāṇobhidīrghākṛtauś
candrāṃbhaḥ paripūritāmṛtatanuḥ prāgghaṭikāyās tathā
chinnatkālaviśālavahnivaśagā bhrūraṃdhranāḍīgaṇāt
tatkāryyaṃ kurute punarnnavataraṃ chinnadrumaskaṃdhavat] N17,

\item[nāsādakṣiṇamārgāvāhipavanāt prāṇotidīrghīkṛtaḥ
caṃdrāṃbhaḥ paripūritāmṛtatanuḥ prāgghaṃṭikāyās tathā 
bhiṃdan kālaviśālavahnivasagān bhūraṃdhranāḍīgaṇān
tatkāryyaṃ kurute punar nnavataraṃ chinnabhramaskaṃdhavat] N18,

\item[nasādakṣiṇamārgavāhipavanāt prāṇotidīrghokṛtaś
caṃdrābhaḥ paripūritāmṛtatanuḥ prāgghaṭikāyāḥ sadā
bhinatkālaviśālavahnivaśagān bhrūraṃdhranāḍīgaṇān
tatkālaṃ kurute punar navataraṃ chinnadrumaskaṃdhavat] N21,

\item[nāsādakṣīṇamārggavāhipavanāt prāṇobhidīrghākṛtoś
candrāmbhaḥ paripūritāmatatanu prāghaṇṭikāyās tathā |
cihnatkālavisālavahnivasagād bhrūrandhranāḍiganāt
tatkāryyaṃ kulute puṇarṇṇavataraṃ cihnadrumaskandhavat ||] N14,

\item[nāsādakṣiṇamārgavāhipavanāt prāṇotidīrghākṛtaḥ
caṃdrāṃbhaḥ paripūritāmṛtanuḥ prāgghaṃṭikāyāḥ sadā
bhiṃdan kālaviśālavahniviśagān bhrūraṃdhranāḍīgaṇān
tatkālaṃ kurute punarvanataraṃ chinnadamaskaṃdhavat] P13,

\item[nāsādakṣiṇamārgavāhipavanāt prāṇopidīghīkrataś
caṃdrābhaḥ pariparitānanatanuḥ prāgghaṃṭikāyās tathā ||
chinnaṃ kālaviśālavahnivaśagaṃ bhrūraṃdhranāḍīgaṇān
tatkayaṃ kurute punarnavataraṃ chinnadramaskaṃdhavat ||] N10,

\item[nāsādakṣiṇavāhimārgapavanāt prāṇotidīrghīkṛtaḥ
caṃdrābhaḥ paripūrṇatāmṛtatanuḥ prāgghaṃṭikāyāḥ sadā
chiṃdan kālaviśālavahnivaśagān bhrūraṃdhranāḍīgaṇān
tatkālaṃ kurute punarnavataraṃ chiṃdan drumaskaṃdhavat] J11,

\item[nāsādakṣiṇamārthethavāhipavanaprāṇotidisīkṛta
caṃdrābhā paripūritāmrāṃmṛtatanuḥ prāke praghaṃṭikāyās tathā
vidatkālaviśālavahnivaśagād bhūraṃdhranāḍīgaṇā
taṃ kāyaṃ kurute punar navatanuṃ jīrṇadrumaskaṃdhavat] N22,

\item[nāsādakṣiṇamārgavāhipavanāṃ prāṇotīdīrghīkṛteś
caṃdrāṃbhaḥ paripūritāmṛtatanūḥ prāgghaṭikās tathā ||
chinan kālaviśālavahnivaśagaṃ prāgraṃdhanāḍīgaṇā
tatkāryaṃ kurute purnavataraṃ chidrumaskaṃdhavat ||] N15,

\item[nāsādakṣiṇamārgavāhapavanaghrāṇotidīrghākṛtaś
caṃdrābhāparipūritāmṛtatanuḥ prāk ghaṃṭikāyās tathā 
bhiṃdan kālaviśālavahnivaśagān bhrūraṃdhranāḍīgaṇān
taṃ kāyaṃ kurute punarnavataraṃ jīrṇadrumaskaṃdhavat ||] N11,

\item[nāsaṃdakṣiṇamārgavāhipavanāt prāṇobhidīrghākṛteś
caṃdrāṃbhaḥ paripuritātanuḥ prāgghaṭikāyās tathā ||
chinnakālavisālāvahnivaśagaṃ prāgradhanāḍīgaṇā
tatkāryaṃ kurute punar navattaraṃ chinnadrumaskaṃdhavat ||] N9,

\item[nāsādakṣiṇavāhimārgapavanāt prāṇotidīrghīkṛtaś
caṃdrābhaḥ paripūrṇatāmṛtatanuḥ prāghaṃṭikāyās tathā
chiṃdan kālaviśālavahnivaśagān bhrūraṃdhranāḍīgaṇān
taṃ kāyaṃ kurute punarnavataraṃ jīrṇaṃ drumaskaṃtvat] K1,

\item[nāsādakṣiṇavāhimārgapavanāt prāpṇotidīrghīkṛtiṃ
caṃdrābhaḥ paripūrṇatāmṛtatanuḥ prodghāṭikāyās sadā
chīdan kālavisālavahnivasagān bhruraṃdhranāḍīgaṇān
tat kāyaṃ kurute punarnavataraṃ chīrdaṃ drumaskaṃdavat] Ko,

\item[nāsādakṣiṇavāhimārgayavanāt prāṇotidīrghīkṛtaś
caṃdrāṃbhaḥ paripūrṇatāmṛtatanuḥ prāgghaṃṭikāyās tathā
chiṃdan kālaviśālavahnivaśagān bhrūraṃdhranāḍīgaṇān
tatkālaṃ kurute punarnavataraṃ chiṃdan drumaskaṃdavat] J14,

item[nāsādakṣiṇavāhimārgayavano prāṇotidīrghīkṛtaś
caṃdrāṃbhaḥ paripūritāmṛtatanuḥ prāgghaṃṭikāyās tathā
chinnat kālaviśālavahnivaśagaṃ bhrūraṃdhranāḍīgaṇāt
svaṃ kāyaṃ kurute punarnavataraṃ chinna drumaskaṃdhavat] J15,

\item[nāsādakṣiṇamārgavāhipavanāt prāṇeti dīrghākṛti |
ścaṃdrāṃbhaḥ paripūritāmṛtatanu prāgghaṭikāyās tathā |
chinnatkālaviśālavahnivaśagāṃ prāgraṃdhranāḍīgaṇāt |
tatkāryaṃ kurute punannavataraṃ chinnadrūmaskaṃdhavat ||] J17,

\item[nāsādakṣiṇamārggavāhipavanāt prāṇebhidīrghākṛtauś
caṃdrāṃbhaḥ paripūritāmṛtatanuḥ rprāgghaṃṭikāyās tathā |
cchinnat kālaviśālavahnivaśagāṃ bhrūraṃdhranāḍīgaṇāt
tatkāryyaṃ kurute punarnnavataraṃ cchinnadrumaskaṃdhavat ||] N6,

\item[nāsādakṣiṇamārgavāhipavano prāṇena dīrghīkṛtoś
caṃdrāṃbhaḥ paripūritāmṛtatanuṃ prāgghaṃṭikāyās tadā
chinnat kālaviśālavahnivaśagaṃ bhrūraṃdhranāḍīgaṇās
tatkāryaṃ kurute punarnavataraṃ chinnadrumaskaṃdhavat] LD1, 

\item[nāsādakṣaṇamārgavāhīpavanaprāṇo na dīrghā kṛteś
caṃdrāṃbhaḥ paripūrītāmṛtatanu prāgghaṃṭikāyās tathā
chinnat kālaviśālavahnīvaśagaṃ bhrūraṃdhranāḍīgaṇās
ta kāryaṃ kurute punarnavataraṃ chīnnadrumaskaṃdhavat ||] LD2, 

Item\[nāsādakṣiṇamārgavāhipavanāt prāṇotīdīrghākṛtaiś
caṃdrābha paripūritāmṛtatanu prāg ghaṭikāyās tathā
chinnaṃ kālaviśālavahnivaśagaṃ bhrūraṃdhranāḍīgaṇāt
svaṃ kāryaṃ kute punarnavataraṃ chinnadrumaskaṃdhavat] J16,

\item[nāsādakṣiṇamargavāhipavanāt prāṇoditi dīrghate
caṃdrābhaḥ paripūritāmṛtatanuḥ prāg ghaṭikāyā tathā
chinnaṃ kālaviśālavahnivaśagaṃ bhrūraṃdhranāḍīgaṇāt
kāryaṃ tat kurute vapur navataraṃ chinnaṃ drumaskaṃdhavat] J12,

\item[nāsādakṣiṇamārgavāhipavanodghrāṇetidīrghī kṛteś
candrābhaḥ paripūritāmṛtatanuḥ prāgghaṇṭikāyās tadā |
bhiṃdankālaviśālavahnivaśagān bhrūrandhranāḍīgaṇān
taṃ kāyaṃ kurute punar navataraṃ jīrṇadrumaskaṃdhavat//] J7,

\item[nāsādakṣiṇamārgavāhipavanāt prāṇo dīrghīkṛtaḥ |
candrāṃgāt paripūritāmṛtyutanuḥ prāgghaṃṭikāyā yathā ||140||
(bhiṃnat)kālaviśālavahnivaśagān mṛṃraṃdhranāḍīgaṇān ||
taṃ kāyaṃ kurute punar nnavatarajīrṇaṃ drumaskaṃdhavat ||] J8, 

\item[nāsādakṣiṇamārgavāhipavanāt prāṇotidīrghākṛtiś
caṃdrāṃbhaḥ paripūritāṃmatatanuḥ prāgvvaṃḍhikāyās tathā||13||
chinnat kālaviśālavahnivaśagāt prāgraṃdhranāḍīgaṇā ||
ttatkāryaṃ kurute punar nnavataraṃ chinnadrumaskaṃdhavat ||] J10,

\item[nāsādakṣiṇamārgavāhipavanaḥ prāṇotidīrghākṛtiś
caṃdrāṃbhaḥ paripūritāṃmatatanuḥ prāgvvaṃḍhikājāḥ yathā||13||
chinnat kālaviśālavahnivaśagāt prāgraṃdhranāḍīgaṇā ||
ttatkāryaṃ kurute punar nnavataraṃ chinnadrumaskaṃdhavat ||] J10a,

\item[nāsādakṣiṇamārgavāhipavanāt prāṇotidīrghī kṛtaḥ ||
caṃdrāṃtaḥ paritāmṛtanuḥ māgdhaṃṭikāyās tathā ||57||
bhidat kālaviśālavanhivaśagān bhrūṃraṃdhranāḍīgaṇāt ||
tatkāryaṃ kurute punar navataraṃ jīrṇadrumaskaṃdhavat ||] J4,

\item[nāsādakṣiṇamārgavāhipavanāt prāṇotidīrghī kṛtaś
caṃdrāṃbaḥ paripūrṇatāmṛtanu prāpnoti kāye sadā
biṃdūn kālaviśālavahnivaśagān bhrūraṃdhranāḍīgaṇān ||
tatkāryaṃ kurute punar navataraṃ jīrṇadrumaskaṃdhavat ||] J13, 

\item[nāsādakṣiṇavāhimārgapavano prāṇatidīrghākṛtā
caṃdrābhaḥ paripūrṇatāmṛtatanu prāpnoti kāye sadā |
viṃdūnakālaviśālapāśavtṛ bhūradhanāḍīgaṇān
tatkālaṃ kurute punarnavatanu chinne dumaskaṃḍavat ||] J2, 

\item[nāsāt dakṣiṇamārgaṃ vāhi pavanān prāṇodīrghītikṛta
caṃdrābhaḥ paripūritāmṛtatanuḥ prākghaṇṭiyās tathā ||
bhidan kālaviśālavahi vaśan bhūraṃdhranāḍīgaṇān ||
tat kāryaṃ kurute punar navataraṃ jīrṇadrumaskaṃdhavat ||] J1,

\item[nāsādakṣiṇamārgavāhipavanāt prāṇotidīrghīkṛtś
caṃdrābhaḥ paripūritāmṛtatanuḥ prāgghaṃṭikāyās tathā
chiṃdat kālavisālavanhivasagāt bhrūraṃdhranāḍīgaṇāt
tatkāryaṃ kurute punar navataṭaṃ chinnadrumaskaṃdhavat] C9, 

\item[nāsādakṣiṇamārgavāhipavanāt prāṇotra dīrghīkṛtḥ
aṃtas tat paripūritāmṛtatanuḥ prāk ghaṃṭikāyās tathā
bhiṃdan kālavisālavahnivasagān bhrūraṃdhranāḍīgaṇān
taṃ kāyaṃ kurute punar navataṭaṃ jīrṇaṃdrumaskaṃdhavat] J3,

\item[nāsādakṣiṇamārgavāhipavanāt prāṇotra dīrghīkṛtaḥ
aṃtastāt paripūritāćtatanuḥ prāk ghaṃṭikāyās tathā 
bhiṃdan kālaviśālavahnivaśagān bhrūraṃdhranāḍīgaṇān
taṃ kāyaṃ kurute punarnavataraṃ jīrṇaṃ drumaṃ skaṃdhavat] J6, 

\item[nāsādakṣiṇavāhimārgapavano ghrāṇo na dīrghī kṛtaḥ
ścandrābhaḥ paripūritāṃmṛtatanuḥ prā ghaṭṭikāyāstathā/
cheda kālaviśālavahnipavanād bhrūrandhranāḍīgaṇāt
tatkāyaṃ kurute punar navataraṃ jīrṇaḥ drumaḥ skaṃdhavat//] C8,

\item[nāsādakṣiṇavartmavāhipavano ghrāṇe ca dīrghīkṛtaś
candrāmtaḥparipūryapūritatanuḥ prāg ghaṃṭikāyās tathā/
bhiṃdan kālaviśālavahnivaśagāṃ bhrūraṃdhranāḍīgaṇāṃs
taṃ kāryaṃ kurute punar navataraṃ jīrṇadrumaskaṃdhavat//] C4,

\item[nāsādakṣiṇavartmavāhipavano ghrāṇoti dīrghīkṛta
Candrāmtaḥ paripūryapūritatanuḥ prāg ghaṃṭikās tathā/
bhidat kālaviśālavahnivaśagāṃ bhrūraṃdhranāḍīgaṇāṃs
taṃ kāryaṃ kurute punar navataraṃ jīrṇadrumaskaṃdhavat//] N5, 
\item[nāsādakṣiṇamārgavāhipavano ghrāṇeti dīrghrākṛtaś
candrābhā paripūritāmṛtanuḥ prāk ghaṃṭikāyās tathā/
bhiṃdaṃn kālaviśālavahnivaśagān bhrūraṃdhranāḍīgaṇāt
taṃ kāyaṃ kurute punar navataraṃ jīrṇadrumaskaṃdhavat//] N8,

\item[nāsādakṣiṇamārgavāhipavanāt prāṇo ’tidīrghīkṛtaś
candrāmbhaḥparipūritāmṛtatanuḥ prāg ghaṇṭikāyās tataḥ/
chitvā kālaviśālavahnivaśagaṃ bhrūrandhranāḍīgataṃ
tat kā(yaṃ) kurute punar navataraṃ chinnaṃ drumaṃ skandhavat//] Ba5, N4b,

\item[nāsādakṣiṇamārgavāhipavano ghrāṇena dīrghaṃkṛtaś
candrāmbhaḥ paripūritāmṛtatanuḥ prāg ghaṇṭṭikāyās pathā/
chinnaṃ kālaviśālavahnivaśagān prāgruddhanāḍīgaṇāt
tat kāyaṃ kurute punar navataraṃ chinnadrumaskandhavat//] A1, 

\item[nāsādakṣiṇamārgavāhipavanāt prāṇotidīrghākṛtaḥ
caṃdrātaḥ paripūritāmṛtatanuḥ prāg ghaṇṭakāyās tathā/
bhindet kālaviśālavanhivaśagān bhrūraṃdhranāḍīgaṇān
tat kāryaṃ kurute punar navataraṃ jīrṇadrumaskandavat//] B1,

\item[nāsādakṣiṇamārgavāhipavanāt prāṇotidīrghākṛtaś
caṃdrātaḥ paripūritāmṛtatanū prāg ghaṃṭikāyaṃs tathā/
bhiṃdaṃt kālaviśālavanhivaśagān bhrūraṃdhranāḍīgaṇān
tat kāryaṃ kurute punar navataraṃ jīrṇadrumaskaṃdavat//] B3,

\item[nāsādakṣiṇamārggavāhipavanaḥ prāṇo(dgidīrvvī) kṛtaś
candrāmbhaḥparipūritāmṛtatanuḥ prāg ghaṇṭikāyās tathā/
chindan kālaviśālavahnivasagān bhrūraṃdhranāḍīgatān
taṃ kāyaṃ kurute punar navataraṃ chinnadrumaskandhavat//] B2, 

\item[nāsādakṣiṇamārgavāhipavanāt prāṇotidīrghākṛtaś
caṃdrāṃtaḥ paripūritāmṛtatanū prāg ghaṃṭikāyās tathā/
bhiṃdat kālaviśālavanhivaśagān bhrūraṃdhranāḍīgaṇān
tat kāryaṃ kurute punar navataraṃ jīrṇadrumaskaṃdavat//] C2,

\item[nāsādakṣiṇamārgavāhipavanāt prā?bhidīrghā kṛte
ścaṃdrāṃbhaḥ paripūritāmṛtatanuḥ prāg ghaṭikāyās tathā/
chiṃdat kālaviśālavanhivaśagān prāgraṃdhranāḍīglāt
tat kāyaṃ kurute punar navataraṃ chinnadrumaskaṃdhavat//] C3,


\item[(illegible/unavailable)] Bo1, Bo2, Bo3, C5, J5, K2, N3, N4a, N13, N20, N23, N24, P3,

  \begin{description}
 Please make notes on the position of the verse.
    • It is given at 158 in B1; B3
    • It is given at 148 in C2
    • It is given at 114 in C8; J10; J15; Ko; P6; J8; N17;
    • It is given at 115 in P8; P9;
    • It is given at 110 in C9
    • It is given at 141 in J1; N18;
    • It is given at 142 in J3
    • It is given at 139 in J6
    • It is given at 117 in J7; P5; P7;
    • It is given at image no.54 in J4
    • It is given at image no.00016 in J11;
    • It is given at 128 in J12; N10;
    • It is given at 125 in N4b
    • It is given at 135 in N9
    • It is given at 137 in P10
    • It is given at 157 in J13
    • It is given at 119 in J14
    • It is given at PDF pg no. 95 in J16
    • It is given at 113 in J17; N6; N14;
    • It is given at 110 in K1
    • It is given at 120 in LD1
    • It is given at 121 in N7
    • It is given at 122 in LD2
    • It is given at PDF no. 13 in N11
    • It is given at JPG no. 1341 lower in N12
    • It is given at PDF no. 21 in N5
    • It is given at PDF no. 28 in N8
    • It is given at PDF no. 09 in N15
    • It is given at 138 in N16
    • It is given at 148 in N21; P2; P12; P13;
    • It is given at 147 in P4;
    • It is given at 134 in P1
    • It is given at PDF no. 23 in N22
    • It is given at PDF no. 20 in N25

  \end{description}

\end{marma}

\begin{marma}[hp03_119b]

  \sthana{3.119b}

\item[śambhugarbhām arundhatīm] J7, J11, J14, 
\item[śambhugarbhām arundhatī] J2, N21, 
\item[śambhugabham irundhatī] P13,
\item[śabdagarbhām aruṃdhatī] J5,
\item[śabdagarbhbhām aruṃddhatī] J8,
\item[śabdagaṃdhām aruṃddhatī] N3,
\item[śabdagarbhām sarasvatīm] A1, Ko, P11, 
\item[śaṃbhugarbhām sarasvatīm] N12, P7,
\item[śabdagabhām sarasvatīm] C6,
\item[sraṣṭugarbhā sarasvatī] C8,
\item[sukhasuptām aruṃdhatī] B1, J15, J17, N6, N9, P9, 
\item[saṣumnāyām aruṃdhatī] C3, J1, J12, K1, 
\item[saṣumṇāyām aruṃdhatīṃ] N10,
\item[saṣumṇāyārūṃdhatī] J16,
\item[sukhasuptām aruṃdhati] N15,
\item[sukhasuptām aruṃdhatīḥ] B3,
\item[sukhasuptām aruṃdhatīm] Ba5, C9, J3, J6, LD1, N5, N13, P4, 
\item[sukhasuptām aruṃdhatīm] P10,
\item[susuptām aruṃdhatīm] P5,
\item[suṣasuptām aruṃdhatī] N7,
\item[sukhaṃ suptāṃ maruṃdhatīm] J4, J13, 
\item[sukhaṃ suptāṃ maruṃdhatī] LD2,
\item[susuptā māmaruṃdhati] N14,
\item[suptā prāṇaruṃdhati] P1,
\item[suṣasuptām aruṃdhatī] P8,
\item[suṣasuptām aruṃdhatīṃ] C2, J10, N17,
\item[susuptāyām aruṃdhatīṃ] P12,
\item[suṣumnāyām aruṃdhatīṃ] N18,
\item[sukhasuptām aruṃdhatām] B2,
\item[(illegible/unavailable)] Bo1, Bo2, Bo3, C4, C5, K2, N4a, N4b, N8, N11, N16, N20, N22, N23, N24, N25, P2, P3, P6,

  \begin{description}
Could \emph{śaṃbhugarbhām} here mean that kuṇḍalinī is Śiva's offspring? The literal sense of a bahuvrīhi (in whose womb is Śiva) does not seem to make sense in this context.
  \end{description}

\end{marma}

\begin{marma}[hp03_127]

  \sthana{3.127ab}

\item[iyaṃ tu madhyamā nāḍī dṛḍhābhyāsena yoginām] N4b,
\item[īḍāvimadhyamā nāḍī iḍhābhyāsena yogini] J2,
\item[kuṭilā madhyamā nāḍī dṛḍhābhyāsena yoginām] J3, J6a, 
\item[pittāpi madhyamā nāḍī dṛḍhābhyāsena yoginām] J1, J6, 
\item[khilāpi madhyamā nāḍī dṛḍhābhyāsena yoginaṃ] N3, N22, 
\item[khilāpi madhyamā nāḍī dṛḍhābhyāsena yoginām] C2, J8, K1, 
\item[khilāpi madhyamā nāḍī pri?ḍābhyāsena yogināṃ] N10,
\item[khilāpi madhyamā nāḍī dṛḍhābhyasepi yogminā] J5,
\item[khilāpi madhyamā nāḍī dṛḍhābhyāsena yoginā] C6, P6, P7,
\item[khilāpi madhya nāḍī ca dṛḍhābhyāsena yoginā] P11,
\item[khilāpi madhyamā nāḍī iḍābhyāsena yoginā] C8,
\item[khilāpi madhyamā nāḍī dṛḍhābhyāsena yoginaḥ] B2, 
\item[...lāpi madhyamā nāḍī dṛḍhābhyāsena yoginaḥ] N7,
\item[kṣīṇāpi madhyamā nāḍī dṛḍhābhyāsena yoginām] N12,
\item[calāpi madhyamā nāḍī dṛḍhābhyāsena yoginām] J11,
\item[cālitā madhyamā nāḍī dṛḍhābhyāsena yoginā] Ko,
\item[pittāpi madhyamā nāḍī dṛḍhābhyāsena yoginām] N16,
\item[vināpi madhyamā nāḍī dṛḍhābhyāsena yoginaḥ] C9, N17, 
\item[vināpi madhyamā nāḍīm iḍābhyāsena yoginaḥ] J12, J15, P9, 
\item[vināpi madhyamā nāḍī dṛḍhābhyāsena yoginā] A1, B3, N14,
\item[iyaṃ tu madhyamā nāḍī dṛḍhābhyāsena yoginā] Ba5,
\item[iyaṃ tu madhyamā nāḍī dṛḍhābhyāsena yogināṃ] N13, 
\item[vināpi madhyamā nāḍī dṛḍhābhyāsena yogināṃ] B1, J4, J7, J13, P2, P4, P12, 
\item[vināpi madhyamā nāḍī dṛḍhābhyāsena yonināṃ] C4, N5, N18,
\item[vināpī madhyamā nāḍī driḍhābhyāsena yogīnaḥ] LD2, N6, 
\item[vināpi madhyamāṃ nāḍī driḍhābhyāsena yoginaḥ] J10, P8, 
\item[vināpi madhyamāṃ nāḍī drāḍhābhyāsena yogina] J17,
\item[vināpi madhyamāṃ nāḍīm iḍābhyāsena yoginaḥ] C3, LD1, P5, P10,  
\item[vināpi madhyamā nāḍiṃ iḍābhyāsena ?????] N15,
\item[vināpi madhyamāṃ nāḍīm iḍābhyāsena yoginaḥ] P1,
\item[mināpi madhyamāṃ nāḍīm iḍābhyāsena yoginaḥ] N9,
\item[vināpī madhyamāṃ nāḍīm iḍābhyāsena yoginīḥ] J16,
\item[calāpi madhyamā nāḍī dṛḍhābhyāsena yonināṃ] J14,
\item[ṣilāṃpi madhyamā nāḍī dṛḍhābhyāsena yogināṃ] N8,
\item[ṣilāpi madhyamā nāḍī dṛḍhābhyāsena yogināṃ] N11,
\item[kṣīṇāpi madhyamā nāḍī dṛḍhābhyāsena yogināṃ] N21, P13, 
\item[(illegible/unavailable)] Bo1, Bo2, Bo3, C5, K2, N4a, N20, N23, N24, N25, P3,

  \begin{description}
    khila seems odd (could it mean defective or unperfected here?), but the alternative readings are not helpful (except iyaṃ tu, which seems like a patch).
  \end{description}

\end{marma}

\begin{marma}[hp03_128]

  \sthana{3.128ab}

\item[upāsanavinidrāṇāṃ rājayogaḥ samudrakaḥ]
\item[upāsanaṃ vinidrāṇāṃ rājayogaḥ samudrakaḥ] J7,
\item[upāsanavinidrāṇāṃ rājayogaḥ sasadgatiḥ] J6, 
\item[upāsanavinidrāṇāṃ rājayogaḥ sasadgatiḥ] J3,
\item[upāsanepinidrāṇāṃ rājayogasamudbhavān] J5,
\item[upāsane viniprā rājayogasamudravat] N3,
\item[abhyāseṣu vinidrāṇāṃ manodhrata samādhinā] J2,
\item[upāsano vinidrāṇāṃ rājayogaḥ samaṃ gati] J1,
\item[upadeśaṃ hi mudrāṇāṃ yo dhatte sāṃpraḍyikaḥ] C9,
\item[upadeśaṃ hi mudrāṇāṃ yo datte sāṃpraḍāyakaḥ] P12,
\item[upadeśaṃ hi mudrāṇāṃ yo dhatte sāṃpraḍāyikaḥ] P13,
\item[upadeśaṃ hi mudrāṇāṃ yo dhatte sāṃpraḍāyikaṃ] N22, P5, 
\item[upadeśaṃ hi mudrāṇāṃ yo dhatte sāṃpraḍāyikāṃ] P8,
\item[upadeśo hi mudrāṇā yo datte ṣāṃpraḍayakaḥ] N21,
\item[upadeśaṃ hi mudrāṇāṃ yo datte śāṃpradīpikaṃ] N18, 
\item[upāṃsanavidhinṛṇāṃ yo datte saṃprādāyikaṃ] N8, N11,
\item[upāsanavinidrāṇāṃ rājayogaḥ samudragaḥ] K1,
\item[upāsanavinidrāṇāṃ rājayogaḥ samāhitaḥ] C4,
\item[upāsanavinidrāṇāṃ rājayogaḥ samāhakaḥ] N5,
\item[upāsanavinidrāṇāṃ rājayogaḥ sasadgatiḥ] N16,
\item[abhyāse tu vinidrāṇāṃ mano dhṛtvā samādhinā] Ba5, N4b,
\item[abhyāsena hi mudrāṇāṃ tad udeti samādhinā] A1, B2, C3, J10, J12, LD1, N6, N7, N9, N10, N15, N17, P9, P10, 
\item[abhyāsena hi mudrāṇāṃ na tūdeti samādhinā] P2, P4, 
\item[abhyāsena hi mudrāṇāṃ tad udeti samādhinaḥ] P1,
\item[abhyāsena hi mudrāṇāṃ tad udetī samādhīnā] LD2,
\item[abhyāsena hi mudrānāṃ tad udetu samādhīnāṃ] N14,
\item[abhyāsena hī muṃdrāṇāṃ tad udeti samādhīnā] J15,
\item[abhyāseṣu vinidrāṇāṃ anu(dgata) samādhinām] C6,
\item[abhyāseṣu vinidrāṇāṃ anudruta samādhiṣu] P11,
\item[abhyāseṣu vinidrāṇāṃ manodhṛta samādhināṃ] J14,
\item[abhyāse tu vinidrāṇāṃ mano dhṛtvā samādhināṃ] N13,
\item[abhyāsena vinidrāṇāṃ manodhṛti samādhinā] N12, 
\item[abhyāseṣu vinidrāṇāṃ anudbhūta samādhinām] J11,
\item[abhyāseṣu vinidrāṇāṃ rājayogaṃ samudragaḥ] P6,
\item[abhyāseṣu vinidrāṇāṃ rājayoga samudrakaṃ] J10a,
\item[abhyāseṣu vinidrāṇāṃ manovṛti samādhināṃ] Ko,
\item[abhyāseṣu vinidrāṇāṃ manudbhūta samādhiṣu] C8,
\item[abhyāseṣu vinidrāṇāṃ manudṛta samādhiṣu] J8,
\item[abhyāsena hi mudrāṇāṃ tad udeti samādhināṃ] J17,
\item[abhyāsena hī muṃdrāṇāṃ tad udeti samādhinā] J16,
\item[abhyāsena hi mudrāṇāṃ na tūdeti samādhinā] C2, J4, J13, 
\item[abhyāsena hi mudrāṇāṃ na tudeti samādhinā] B1, B3, 
\item[abhyāse vinimudrāṇāṃ manudbhutasamādhiṣu] P7,
\item[(illegible/unavailable)] Bo1, Bo2, Bo3, C5, K2, N4a, N20, N23, N24, N25, P3,

  \begin{description}
    No idea. We need more readings. 
  \end{description}

\end{marma}



%%% Local Variables:
%%% mode: latex
%%% TeX-master: t
%%% End: