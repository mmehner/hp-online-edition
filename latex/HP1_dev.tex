\documentclass[10pt]{memoir}
\setstocksize{220mm}{155mm} 	        
\settrimmedsize{220mm}{155mm}{*}	
\settypeblocksize{170mm}{116mm}{*}	
\setlrmargins{18mm}{*}{*}
\setulmargins{*}{*}{1.2}
% \setlength{\headheight}{5pt}
\checkandfixthelayout[lines]
\linespread{1.2}

\setlength{\footmarkwidth}{1.3em}
\setlength{\footmarksep}{0em}
\setlength{\footparindent}{1.3em}
\footmarkstyle{\textsuperscript{#1} }
\usepackage{fnpos}
\makeFNbottom

\usepackage[teiexport=tidy,poetry=verse]{ekdosis}
\usepackage{libertine}
\usepackage{sanskrit-poetry}
\usepackage{xcolor}

\usepackage[english]{babel}
\usepackage{babel-iast,xparse,xcolor}
\babelfont[iast]{rm}[Renderer=Harfbuzz, Scale=1.5]{AdishilaSan}
%\babelfont[english]{rm}[Scale=0.9]{Adobe Text Pro}
\babeltags{dev = iast}
\babeltags{eng = english}

\SetHooks{
	lemmastyle=\bfseries,
	refnumstyle=\selectlanguage{english}\color{blue}\bfseries, 
	}
\newif\ifinapparatus
\DeclareApparatus{default}[
	lang=english,
	sep = {] },
	delim=\hskip 0.75em,
	rule=none,
	]
\DeclareApparatus{notes}[
	lang=english,
	sep = {},
	delim=\hskip 0.75em,
	rule=\rule{0.7in}{0.4pt},
	]

%\DeclareShorthand{conj}{\texteng{\emph{conj.}}}{ego}
\DeclareShorthand{emend}{\texteng{\emph{em.}}}{ego}

\setlength{\vrightskip}{-10pt}
%\setlength{\vgap}{3mm} % default 1.5em
\verselinenumfont{\footnotesize\selectlanguage{english}\normalfont}
\setlength{\stanzaskip}{0.6\baselineskip}

%Define two commands: \skp ("sanskrit plus"), to be ignored by TeX in
%the edition text, but processed in the TEI output. Conversely, \skm
%("sanskrit minus") is to be processed in the edition text, but
%ignored if found in the apparatus criticus and in the TEI output:

\NewDocumentCommand{\skp}{m}{}
%\NewDocumentCommand{\skm}{m}{\unless\ifinapparatus#1-\fi}
\NewDocumentCommand{\skm}{m}{\unless\ifinapparatus#1\fi} % modified by MD 2022-05-31

%


%%%%%%%%%%%%%%%%%%%% THE  MSS         %%%%%%%%%%%%%%%%%%%%%%%%%%%

%%% Versions
\DeclareWitness{Vu}{\selectlanguage{english}Vulg}{Vulgate, i.e. Brahmānanda's version}[]           
\DeclareWitness{X}{\selectlanguage{english}X}{TenChapter Version, Jodhpur 02228 and 02225 (ed. Lonavla)}[]
\DeclareWitness{Six}{\selectlanguage{english}Ṣ}{SixChapterVersion, ``6ChapterHPms'', fragment of enlarged text, Jodhpur}[]
% Mss. in Geographical Groups
%%%% Varanasi mss (Sampūrṇānanda mss). V1 is Important
\DeclareWitness{V1}{\selectlanguage{english}V\textsubscript{1}}{Sampurnananda Library Sarasvati Bhavan 30109}[]
        \DeclareHand{V1ac}{V1}{\selectlanguage{english}V\rlap{\textsubscript{1}}\textsuperscript{ac}}[] % added by MD
        \DeclareHand{V1pc}{V1}{\selectlanguage{english}V\rlap{\textsubscript{1}}\textsuperscript{pc}}[] % added by MD
\DeclareWitness{V2}{\selectlanguage{english}V\textsubscript{2}}{Sampurnananda Library Sarasvati Bhavan 29869}[]
\DeclareWitness{V3}{\selectlanguage{english}V\textsubscript{3}}{Sampurnananda Library Sarasvati Bhavan 29899}[]
\DeclareWitness{V4}{\selectlanguage{english}V\textsubscript{4}}{Sampurnananda Library Sarasvati Bhavan 29937}[]
\DeclareWitness{V5}{\selectlanguage{english}V\textsubscript{5}}{Sampurnananda Library Sarasvati Bhavan 29938}[]
\DeclareWitness{V6}{\selectlanguage{english}V\textsubscript{6}}{Sampurnananda Library Sarasvati Bhavan 29991}[]
\DeclareWitness{V8}{\selectlanguage{english}V\textsubscript{8}}{Sampurnananda Library Sarasvati Bhavan 30014}[]
\DeclareWitness{V11}{\selectlanguage{english}V\textsubscript{11}}{Sampurnananda Library Sarasvati Bhavan 30029}[]
\DeclareWitness{V12}{\selectlanguage{english}V\textsubscript{12}}{Sampurnananda Library Sarasvati Bhavan 30030}[]
\DeclareWitness{V13}{\selectlanguage{english}V\textsubscript{13}}{Sampurnananda Library Sarasvati Bhavan 30031}[]
\DeclareWitness{V14}{\selectlanguage{english}V\textsubscript{14}}{Sampurnananda Library Sarasvati Bhavan 30050}[]
\DeclareWitness{V15}{\selectlanguage{english}V\textsubscript{15}}{Sampurnananda Library Sarasvati Bhavan 30051}[]
\DeclareWitness{V15pc}{\selectlanguage{english}V\rlap{\textsubscript{15}}\textsuperscript{pc}\space}{}[]
\DeclareWitness{V16}{\selectlanguage{english}V\textsubscript{16}}{Sampurnananda Library Sarasvati Bhavan 30052}[]
\DeclareWitness{V17}{\selectlanguage{english}V\textsubscript{17}}{Sampurnananda Library Sarasvati Bhavan 30053}[] % added by MD
\DeclareWitness{V16pc}{\selectlanguage{english}V\rlap{\textsubscript{16}}\textsuperscript{pc}\space}{}[]
\DeclareWitness{V18}{\selectlanguage{english}V\textsubscript{18}}{Sampurnananda Library Sarasvati Bhavan 30064}[]
\DeclareWitness{V19}{\selectlanguage{english}V\textsubscript{19}}{Sampurnananda Library Sarasvati Bhavan 30069}[]
\DeclareWitness{V21}{\selectlanguage{english}V\textsubscript{21}}{Sampurnananda Library Sarasvati Bhavan 30104}[]
\DeclareWitness{V22}{\selectlanguage{english}V\textsubscript{22}}{Sampurnananda Library Sarasvati Bhavan 30110}[]
\DeclareWitness{V25}{\selectlanguage{english}V\textsubscript{25}}{Sampurnananda Library Sarasvati Bhavan 30122}[]
\DeclareWitness{V26}{\selectlanguage{english}V\textsubscript{26}}{Sampurnananda Library Sarasvati Bhavan 30123}[]
\DeclareWitness{V28}{\selectlanguage{english}V\textsubscript{28}}{Sampurnananda Library Sarasvati Bhavan 30136}[]
\DeclareWitness{W4}{\selectlanguage{english}W\textsubscript{4}}{Wai 399-6171}[]

%%%%%%%%%%%%%%%%%%%%%%%%%%%%%%%%%
%%% Jammu & Kaschmir
\DeclareWitness{K1}{\selectlanguage{english}K\textsubscript{1}}{Raghunātha Temple Library 4383}[settlement=Jammu]
        \DeclareWitness{K1ac}{\selectlanguage{english}K\rlap{\textsubscript{1}}\textsuperscript{ac}\space}{}[]
        \DeclareWitness{K1pc}{\selectlanguage{english}K\rlap{\textsubscript{1}}\textsuperscript{pc}\space}{}[]
\DeclareWitness{L1}{\selectlanguage{english}L\textsubscript{1}}{SOAS RE 43454}[settlement=Jammu]
% More details? Catalogue number? L1 And C1 very close (and come from same region)
%%%%%%%%%%%%%%%%%%%%%%%%%%%%%%%%
% Jodhpur
% J10 is important
\DeclareWitness{J10}{\selectlanguage{english}J\textsubscript{10}}{MSPP Jodhpur 2230}[]
        \DeclareHand{J10ac}{J10}{\selectlanguage{english}J\rlap{\textsubscript{10}}\textsuperscript{ac}}[] % modified by MD
        \DeclareHand{J10pc}{J10}{\selectlanguage{english}J\rlap{\textsubscript{10}}\textsuperscript{pc}}[] % modified by MD
\DeclareWitness{J1}{\selectlanguage{english}J\textsubscript{1}}{Jodhpur 02231}[]
\DeclareWitness{J2}{\selectlanguage{english}J\textsubscript{2}}{Jodhpur 02232}[]   
\DeclareWitness{J3}{\selectlanguage{english}J\textsubscript{3}}{Jodhpur 02233}[]
\DeclareWitness{J4}{\selectlanguage{english}J\textsubscript{4}}{Jodhpur 02234}[]
        \DeclareWitness{J4ac}{\selectlanguage{english}J\rlap{\textsubscript{4}}\textsuperscript{ac}\space}{MSPP Jodhpur 02234}[]
        \DeclareWitness{J4pc}{\selectlanguage{english}J\rlap{\textsubscript{4}}\textsuperscript{pc}\space}{MSPP Jodhpur 02234}[]
\DeclareWitness{J5}{\selectlanguage{english}J\textsubscript{5}}{Jodhpur 02235}[]  % 4 chapters, 34 jpgs,   long colophon, missing lines in the beginning.
\DeclareWitness{J6ac}{\selectlanguage{english}J\rlap{\textsubscript{6}}\textsubscript{ac}}{Jodhpur 02237}[]  % 4 chapters, 49 jpgs,   1st folio: idaṃ gulābarāyasya
% tulasīrāmaśarmmaṇaḥ putrasya pustakaṃ ...        End: iti śrīsahajānandasantānacintāmaṇisvātmārāmaviracitāyāṃ ..
% saṃvat 1802   (more consistent text)
\DeclareWitness{J6pc}{\selectlanguage{english}J\rlap{\textsubscript{6}}\textsubscript{pc}}{Jodhpur 02237}[] 
\DeclareWitness{J7}{\selectlanguage{english}J\textsubscript{7}}{Jodhpur 02241}[]  % 4 chapters, 41 jpgs
\DeclareWitness{J8}{\selectlanguage{english}J\textsubscript{8}}{Jodhpur 23709}[]  % 4 chapters,  87 jpgs.   saṃvat 1724
\DeclareHand{J8ac}{J8}{\selectlanguage{english}J\rlap{\textsubscript{8}}\textsuperscript{ac}}[]  % changed by MD
\DeclareHand{J8pc}{J8}{\selectlanguage{english}J\rlap{\textsubscript{8}}\textsuperscript{pc}}[]  % changed by MD
\DeclareWitness{J9}{\selectlanguage{english}J\textsubscript{9}}{Jodhpur 02224}[]  %  fragment, 20 jpgs.
\DeclareWitness{J11}{\selectlanguage{english}J\textsubscript{11}}{Jodhpur 23532}[]
\DeclareWitness{J12}{\selectlanguage{english}J\textsubscript{12}}{Jodhpur 18552}[] 
\DeclareWitness{J13}{\selectlanguage{english}J\textsubscript{13}}{Jodhpur 02229}[]  %  5 chapters, 93 jpgs.
\DeclareWitness{J14}{\selectlanguage{english}J\textsubscript{14}}{Jodhpur 02239}[]  %  4 chapters
\DeclareWitness{J15}{\selectlanguage{english}J\textsubscript{15}}{Jodhpur 9732A}[]
\DeclareWitness{J17}{\selectlanguage{english}J\textsubscript{17}}{Jodhpur 3013}[]
% Haṭhapradīpikā with (non-Sanskrit) Bhāṣya RORI Jodhpur ACC.NO.18552
%  Haṭhapradīpikā with (non-Sanskrit) commentary, RORI Alwar 952, 4 chapters,  colophon of the comm:
% iti śrīlāhorīmiśravrajabhūṣanaviracitāyāṃ bhāvārthadīpikāyāṃ caturthodhyāya ..    
%  Haṭhapradīpikā (5 chapter) MSPP Jodhpur ACC.NO.02229/

%%%%%%%%%%        Bodleian, Oxford
\DeclareWitness{B1}{\selectlanguage{english}B\textsubscript{1}}{Bodleian Library No. d.457(8)}[settlement=Oxford]
\DeclareWitness{B2}{\selectlanguage{english}B\textsubscript{2}}{Bodleian Library No. d.458(1)}[settlement=Oxford]
\DeclareWitness{B3}{\selectlanguage{english}B\textsubscript{3}}{Bodleian Library No. d.458(9)}[settlement=Oxford]

%%%%%%%%%%%   Chandigarh
\DeclareWitness{C1}{\selectlanguage{english}C\textsubscript{1}}{Lalchand M-2080}[]%L1 And C1 very close (and come from same region)
\DeclareWitness{C2}{\selectlanguage{english}C\textsubscript{2}}{Lalchand M-6065}[]
\DeclareWitness{C3}{\selectlanguage{english}C\textsubscript{3}}{Lalchand M-1293}[]
\DeclareWitness{C4}{\selectlanguage{english}C\textsubscript{4}}{Lalchand M-2081}[]
\DeclareWitness{C4ac}{\selectlanguage{english}C\rlap{\textsubscript{4}}\textsuperscript{ac}\space}{}[]
\DeclareWitness{C4pc}{\selectlanguage{english}C\rlap{\textsubscript{4}}\textsuperscript{pc}\space}{}[]
\DeclareWitness{C5}{\selectlanguage{english}C\textsubscript{5}}{Lalchand M-2082}[]%doesn't have chapter 1
\DeclareWitness{C6}{\selectlanguage{english}C\textsubscript{6}}{Lalchand M-2089}[]
\DeclareWitness{C7}{\selectlanguage{english}C\textsubscript{7}}{Lalchand M-6494}[]
\DeclareWitness{C8}{\selectlanguage{english}C\textsubscript{8}}{Lalchand M-2091}[]
\DeclareWitness{C8pc}{\selectlanguage{english}C\rlap{\textsubscript{8}}\textsuperscript{pc}\space}{}[]
\DeclareWitness{C9}{\selectlanguage{english}C\textsubscript{9}}{Lalchand M-4530}[]

% %%%%%%%%%%        Nepalese
\DeclareWitness{N1}{\selectlanguage{english}N\textsubscript{1}}{NGMPP A1400-2}[]
\DeclareWitness{N2}{\selectlanguage{english}N\textsubscript{2}}{NGMPP B 39-19}[]
\DeclareWitness{N3}{\selectlanguage{english}N\textsubscript{3}}{NGMPP B 62-20}[]
\DeclareWitness{N5}{\selectlanguage{english}N\textsubscript{5}}{NGMPP A60-15 + A61-1}[]
\DeclareWitness{N6}{\selectlanguage{english}N\textsubscript{6}}{NGMPP A61-6}[]
\DeclareWitness{N9}{\selectlanguage{english}N\textsubscript{9}}{NGMPP A62-33}[]
\DeclareWitness{N10}{\selectlanguage{english}N\textsubscript{10}}{NGMPP A62-37}[]
\DeclareWitness{N11}{\selectlanguage{english}N\textsubscript{11}}{NGMPP A63-15}[]
\DeclareWitness{N12}{\selectlanguage{english}N\textsubscript{12}}{NGMPP A939-19}[]
\DeclareWitness{N13}{\selectlanguage{english}N\textsubscript{13}}{NGMPP A1378-18}[]
\DeclareWitness{N16}{\selectlanguage{english}N\textsubscript{16}}{NGMPP B39-20}[]
\DeclareWitness{N17}{\selectlanguage{english}N\textsubscript{17}}{NGMPP B 111-10}[]
\DeclareWitness{N18}{\selectlanguage{english}N\textsubscript{18}}{NGMPP E 929-3}[]
\DeclareWitness{N19}{\selectlanguage{english}N\textsubscript{19}}{NGMPP E-1528-1 / E-1527-7(4)}[]
\DeclareWitness{N20}{\selectlanguage{english}N\textsubscript{20}}{NGMPP E 2037-13 }[]
\DeclareWitness{N21}{\selectlanguage{english}N\textsubscript{21}}{NGMPP E 2097-31}[]
\DeclareWitness{N22}{\selectlanguage{english}N\textsubscript{22}}{NGMPP G 4-4}[]
\DeclareWitness{N23}{\selectlanguage{english}N\textsubscript{23}}{NGMPP G 25-2}[]
\DeclareWitness{N24}{\selectlanguage{english}N\textsubscript{24}}{NGMPP G 190-16}[]
\DeclareWitness{N24ac}{\selectlanguage{english}N\rlap{\textsubscript{24}}\textsuperscript{ac}\space}{}[]
\DeclareWitness{N24pc}{\selectlanguage{english}N\rlap{\textsubscript{24}}\textsuperscript{pc}\space}{}[]

\DeclareWitness{P28}{\selectlanguage{english}P\textsubscript{28}}{BORI 399-1895-1902}[]

%%%%%   Mysore
\DeclareWitness{M1}{\selectlanguage{english}M\textsubscript{1}}{P-5682/4}[]
%%%%%   Tübingen
\DeclareWitness{Tü}{\selectlanguage{english}Tü}{Ma I 339}[]
%%%%%%%%%%
\DeclareWitness{YC}{\selectlanguage{english}YC}{Yogacintāmaṇi}[]
\DeclareWitness{ceteri}{\selectlanguage{english}cett.}{ceteri}[]

%%%%%%%%%% Mss with Commentary
\DeclareWitness{A1}{\selectlanguage{english}A\textsubscript{1}}{Alwar 952}[]


%%%%%%%%%%%%%%%%%%%%%%%%%%%%%%%%%%%%%%%%%%%
%List of all Sigla:
%A1,B1,B2,B3,C1,C2,C3,C4,C6,C7,C8,C9,J1,J2,J3,J4,J10,J13,J14,J15,J17,L1,M1,N3,N5,N6,N9,N10,N11,N12,N13,N16,N17,N19,N20,N21,N22,N23,N24,Tü,V1,V2,V3,V4,V5,V6,V8,V11,V19,V22,V26,Vu
%%%%%%%%%%%%%%%%%%%%%%%%%%%%%%%%%%%%%%%%%%%

\DeclareShorthand{x}{\selectlanguage{english}δ}{J10,J17,N17,P28,W4}


%%% Local Variables:
%%% mode: latex
%%% TeX-master: t
%%% End:

%
%%%%%                   Abbreviation for the printed apparatus,        xml interface needed
%%%%%                   (synonyms in same line)

% Macro for Editing Abbrevs.
%\def\om{\textrm{\footnotesize \textit{omitted in}\ }} %prints om. for omitted in apparatus
%\def\korr{\textrm{\footnotesize \textit{em.}\ }} %prints em. for emended in apparatus
%\def\conj{\textrm{\footnotesize \textit{conj.}\ }} %prints conj. for conjectured in apparatus


\def\eyeskip{\textrm{{ab.\,oc. }}}   
\def\aberratio{\textrm{{ab.\,oc. }}}
\def\ad{\textrm{{ad}}}   
\def\add{\textrm{{add.\ }}}
\def\ann{\textrm{{ann.\ }}}
\def\ante{\textrm{{ante }}}
\def\post{\textrm{{post }}}
%\def\ceteri{cett.\,}             % for simplifying the apparatus in print                  
\def\codd{\textrm{{codd.\ }}}   %  the same
\def\conj{\textrm{{coni.\ }}}  
\def\coni{\textrm{{coni.\ }}}
\def\contin{\textrm{{contin.\ }}}
\def\corr{\textrm{{corr.\ }}}
\def\del{\textrm{{del.\ }}}
\def\dub{\textrm{{ dub.\ }}}
\def\emend{\textrm{{emend.\ }}}
\def\expl{\textrm{{explic.\ }}}   
\def\explicat{\textrm{{explic.\ }}}
\def\fol{\textrm{{fol.\ }}}         
\def\foll{\textrm{{foll.\ }}}
\def\gloss{\textrm{{glossa ad }}}
\def\ins{\textrm{{ins.\ }}}          \def\inseruit{\textrm{{ins.\ }}}
\def\im{{\kern-.7pt\lower-1ex\hbox{\textrm{\tiny{\emph{i.m.}}}\kern0pt}}}
\def\inmargine{{\kern-.7pt\lower-.7ex\hbox{\textrm{\tiny{\emph{i.m.}}}\kern0pt}}}
\def\intextu{{\kern-.7pt\lower-.95ex\hbox{\textrm{\tiny{\emph{i.t.}}}\kern0pt}}}%\textrm{\scriptsize{i.t.\ }}}               
\def\indist{\textrm{{indis.\ }}}          \def\indis{\textrm{{indis.\ }}}
\def\iteravit{\textrm{{iter.\ }}}          \def\iter{\textrm{{iter.\ }}}  
\def\lectio{\textrm{{lect.\ }}}             \def\lec{\textrm{{lect.\ }}}
\def\leginequit{\textrm{{l.n. }}}         \def\legn{\textrm{{l.n. }}}         \def\illeg{\textrm{{l.n. }}}
\def\om{\textrm{{om. }}}
\def\primman{\textrm{{pr.m.}}}
\def\prob{\textrm{{prob.}}}
\def\rep{\textrm{{repetitio }}}
% \def\secundamanu{\textrm{\scriptsize{s.m.}}}
% \def\secm{{\kern-.6pt\lower-.91ex\hbox{\textrm{\tiny{\emph{s.m.}}}\kern0pt}}}%   \textrm{\scriptsize{s.m.}}}
\def\sequentia{\textrm{{seq.\,inv.\ }}}         \def\seqinv{\textrm{{seq.\,inv.\ }}} \def\order{\textrm{{seq.\,inv.\ }}}
\def\supralineam{{\kern-.7pt\lower-.91ex\hbox{\textrm{\tiny{\emph{s.l.}}}\kern0pt}}} %\textrm{\scriptsize{s.l.}}}
\def\interlineam{{\kern-.7pt\lower-.91ex\hbox{\textrm{\tiny{\emph{s.l.}}}\kern0pt}}}   %\textrm{\scriptsize{s.l.}}}
\def\vl{\textrm{v.l.}}   \def\varlec{\textrm{v.l.}} \def\varialectio{\textrm{v.l.}}
\def\vide{\textrm{{cf.\ }}}           \def\cf{\textrm{{cf.\ }}}
\def\videtur{\textrm{{vid.\,ut}}}
\def\crux{\textup{[\ldots]} }
\def\cruxx{\textup{[\ldots]}}
\def\unm{\textit{unm.}}        % unmetrical
%%%%%%%%%%%%%%%%%%%%%%%%%%%%%%%%%%%%



%%% Local Variables:
%%% mode: latex
%%% TeX-master: t
%%% End:

% additions/changes 2024-07-04 mm:
\TeXtoTEIPat{\lb}{<lb/>}
\TeXtoTEIPat{\begin {quote}}{<q>}
  \TeXtoTEIPat{\end {quote}}{</q>}
\TeXtoTEIPat{\begin {enumerate}}{<list rend="numbered">}
  \TeXtoTEIPat{\end {enumerate}}{</list>}
\TeXtoTEI{item}{item}

% additions/changes 2024-07-01 mm:
\TeXtoTEIPat{\unavbl {#1}}{<note type="foliolost">Folio lost in <ref>#1</ref></note>}
\TeXtoTEIPat{\NotIn {#1}}{<note type="omission">Omitted in <ref>#1</ref></note>}
\TeXtoTEI{graus}{span}[type="altrec"]
\TeXtoTEI{grau}{span}[type="altrec"]

% addition 2024-03-15 MD
\TeXtoTEI{manuref}{}

\TeXtoTEIPat{\alphaOne}{α<hi rend="sub">1</hi>}% N3
\TeXtoTEIPat{\alphaTwo}{α<hi rend="sub">2</hi>}% J5
\TeXtoTEIPat{\alphaThree}{α<hi rend="sub">3</hi>}% G4
\TeXtoTEIPat{\betaOne}{β<hi rend="sub">1</hi>}% P11
\TeXtoTEIPat{\betaTwo}{β<hi rend="sub">2</hi>}% C6
\TeXtoTEIPat{\betaOmega}{β<hi rend="sub">ω</hi>}% V3
\TeXtoTEIPat{\gammaOne}{γ<hi rend="sub">1</hi>}% N23
\TeXtoTEIPat{\gammaTwo}{γ<hi rend="sub">2</hi>}% J7
\TeXtoTEIPat{\deltaOne}{δ<hi rend="sub">1</hi>}% V19
\TeXtoTEIPat{\deltaTwo}{δ<hi rend="sub">2</hi>}% K3
\TeXtoTEIPat{\deltaThree}{δ<hi rend="sub">3</hi>}% C7
\TeXtoTEIPat{\deltaOmega}{δ<hi rend="sub">ω</hi>}% J6
\TeXtoTEIPat{\epsilonOne}{ε<hi rend="sub">1</hi>}% P15
\TeXtoTEIPat{\epsilonTwo}{ε<hi rend="sub">2</hi>}% N19
\TeXtoTEIPat{\epsilonThree}{ε<hi rend="sub">3</hi>}% V15
\TeXtoTEIPat{\epsilonFour}{ε<hi rend="sub">4</hi>}% J11
\TeXtoTEIPat{\epsilonOmega}{ε<hi rend="sub">ω</hi>}% N26
\TeXtoTEIPat{\etaOne}{η<hi rend="sub">1</hi>}% V1
\TeXtoTEIPat{\etaTwo}{η<hi rend="sub">2</hi>}% J10
\TeXtoTEIPat{\etaOmega}{η<hi rend="sub">ω</hi>}% N9

% addition 2023-12-11 MD:
\TeXtoTEIPat{\begin {metre}[#1]}{<note type="metre" target="##1">}
\TeXtoTEIPat{\end {metre}}{</note>}
\TeXtoTEIPat{\texttheta}{θ}

% change 2023-12-05 mm
\TeXtoTEI{teimute}{} 

% changes/additions 2023-11-27 MM:
\TeXtoTEIPat{\medialink {#1}{#2}}{<ref target="resources/#2">#1</ref>}

% changes/additions 2023-10-25 MM:
% new Sigla
\TeXtoTEIPat{\textAlpha}{Α}
\TeXtoTEIPat{\textalpha}{α}
\TeXtoTEIPat{\textBeta}{Β}
\TeXtoTEIPat{\textbeta}{β}
\TeXtoTEIPat{\textGamma}{Γ}
\TeXtoTEIPat{\textgamma}{γ}
\TeXtoTEIPat{\textDelta}{Δ}
\TeXtoTEIPat{\textdelta}{δ}
\TeXtoTEIPat{\textEpsilon}{Ε}
\TeXtoTEIPat{\textepsilon}{ε}
\TeXtoTEIPat{\textEta}{Η}
\TeXtoTEIPat{\texteta}{η}
\TeXtoTEIPat{\textChi}{Χ}
\TeXtoTEIPat{\textchi}{χ}
\TeXtoTEIPat{\textOmega}{Ω}
\TeXtoTEIPat{\textomega}{ω}

%new environments
\TeXtoTEIPat{\begin {postmula}[#1]}{<div type="postmula" xml:id="#1">} %%% changed 2024-07-01 mm
  \TeXtoTEIPat{\end {postmula}}{</div>}  %%% changed 2024-07-01 mm
  
\TeXtoTEIPat{\begin {altpostmula}[#1]}{<div type="altrec"><div type="postmula" xml:id="#1">} %%% added 2024-07-03 md
  \TeXtoTEIPat{\end {altpostmula}}{</div></div>} %%% added 2024-07-03 md

\TeXtoTEIPat{\begin {altava}[#1]}{<div type="altrec"><div type="avataranika" xml:id="#1">} %%% changed 2024-07-01 mm
  \TeXtoTEIPat{\end {altava}}{</div></div>} %%% changed 2024-07-01 mm

\TeXtoTEIPat{\sgwit {#1}}{<note type="inlineref"><ref>#1</ref></note>}

% changes/additions 2023-10-12 MM:
\TeXtoTEIPat{\\.}{}

% changes/additions 2023-08-15 MD:
\TeXtoTEIPat{\lineom {#1}{#2}}{<note type="omission">#1 omitted in <ref>#2</ref></note>}
%\TeXtoTEIPat{\startgray}{} %%% changed 2023-12-05 mm; not used 2024-03-26 MD
%\TeXtoTEIPat{\endgray}{} %%% changed 2023-12-05 mm; not used 2024-03-26 MD

% additions/changes 2023-06-05 mm:
%\TeXtoTEIPat{\lineom {#1}}{<note type="omission">Line omitted in <ref>#1</ref></note>}

% additions 2023-04-16 MD:
\TeXtoTEIPat{\,}{ }

% additions 2023-04-13 mm:
\TeXtoTEIPat{\begin {versinnote}}{<lg>}
  \TeXtoTEIPat{\end {versinnote}}{</lg>}

% additions 2023-04-05 MD:
\TeXtoTEIPat{\begin {testimonia}[#1]}{<note type="testimonia" target="##1">}
  \TeXtoTEIPat{\end {testimonia}}{</note>}
\TeXtoTEI{devnote}{s}[xml:lang="sa-deva"]

% app in philcomm und testimonia %%% added MM 2023-12-02
\TeXtoTEI{var}{note}[type="appinnote"]


\TeXtoTEI{anm}{note}[type="memo"] %% change 2023-04-16 MD
\TeXtoTEI{Anm}{note}[type="memo"] %% change 2023-12-05 MM
\TeXtoTEIPat{\startverse}{} %%% marked for change 2023-04-13 mm
\TeXtoTEIPat{\endverse}{} %%% marked for change 2023-04-13 mm
\TeXtoTEIPat{\newpage}{}
\TeXtoTEIPat{\marmas}{ } % changed 2024-03-17 MD
\TeXtoTEIPat{\marma}{}
\TeXtoTEIPat{\vin}{} % added by MD 2023-11-14

%%% modify environments and commands
%%% TEI mapping
% additions/changes 2022-06-07 mm:
\TeXtoTEIPat{ \& }{ &amp; }

% additions/changes 2022-06-01 mm:
\TeXtoTEI{skp}{seg}[type="deva-ignore"]
\TeXtoTEI{skm}{seg}[type="ltn-ignore"]

\TeXtoTEIPat{\rlap {#1}}{#1}

% additions/changes 2022-04-06 mm:
%\TeXtoTEI{sgwit}{ref}
\TeXtoTEI{textdev}{s}[xml:lang="sa-deva"]
\TeXtoTEIPat{\begin {col}[#1]}{<div type="colophon" xml:id="#1">}
  \TeXtoTEIPat{\end {col}}{</div>}
\TeXtoTEIPat{\begin {ava}[#1]}{<div type="avataranika" xml:id="#1">} %%% changed 2024-07-01 mm
  \TeXtoTEIPat{\end {ava}}{</div>} %%% changed 2024-07-01 mm
												   
\TeXtoTEIPat{\outdent}{}
\TeXtoTEIPat{\startaltrecension}{} %%% changed 2023-12-05 mm
\TeXtoTEIPat{\endaltrecension}{} %%% changed 2023-12-05 mm
\TeXtoTEIPat{\startaltnormal}{} % added by MD 2023-11-14 %%% changed 2023-12-05 mm
\TeXtoTEIPat{\endaltnormal}{} % added by MD 2023-11-14 %%% changed 2023-12-05 mm
\TeXtoTEIPat{\begin {alttlg}[#1]}{<div type="altrec"><lg xml:id="#1">}
  \TeXtoTEIPat{\end {alttlg}}{</lg></div>}



% additions/changes 2022-03-12 mm:
\TeXtoTEIPat{\begin {tlg}[#1]}{<lg xml:id="#1">}
  \TeXtoTEIPat{\end {tlg}}{</lg>}

\TeXtoTEIPat{\begin {translation}[#1]}{<note type="translation" target="##1">}
  \TeXtoTEIPat{\end {translation}}{</note>}
\TeXtoTEIPat{\begin {philcomm}[#1]}{<note type="philcomm" target="##1">}
  \TeXtoTEIPat{\end {philcomm}}{</note>}
\TeXtoTEIPat{\begin {sources}[#1]}{<note type="sources" target="##1">}
  \TeXtoTEIPat{\end {sources}}{</note>}


\TeXtoTEIPat{\begin {marma}[#1]}{<note type="marma" target="##1">}
  \TeXtoTEIPat{\end {marma}}{</note>}

\TeXtoTEIPat{\begin {jyotsna}[#1]}{<note type="jyotsna" target="##1">}
  \TeXtoTEIPat{\end {jyotsna}}{</note>}

\EnvtoTEI{description}{list}
\EnvtoTEI{itemize}{list}
\TeXtoTEIPat{\item [#1]}{<label>#1</label>\item}

\TeXtoTEI{tl}{l}
\TeXtoTEI{myfn}{note}[type="myfn"]
\TeXtoTEIPat{\getsiglum {#1}}{<ref target="##1"/>}

\TeXtoTEI{SetLineation}{}
\TeXtoTEI{noindent}{}
\TeXtoTEI{subsection*}{}

\TeXtoTEI{rlap}{}

% end additions/changes
% \TeXtoTEIPat{\skp {#1}}{#1}
% \TeXtoTEIPat{\skm {#1}}{}

\TeXtoTEIPat{\begin {prose}}{<p>}
  \TeXtoTEIPat{\end {prose}}{</p>}

\TeXtoTEIPat{\begin {tlate}}{<p>}
  \TeXtoTEIPat{\end {tlate}}{</p>}

\TeXtoTEI{emph}{hi}
\TeXtoTEI{bigskip}{}
% \TeXtoTEI{/}{|}
\TeXtoTEI{tl}{l}
\TeXtoTEIPat{english}{}
%\TeXtoTEIPat{-}{ } %% change 2023-04-16 MD
%\TeXtoTEIPat{°}{} %% change 2023-04-16 MD
\TeXtoTEIPat{\textcolor {#1}{#2}}{<hi rend="#1">#2</hi>}

% \TeXtoTEIPat{\eyeskip}{}
% \TeXtoTEIPat{\aberratio}{}
% \TeXtoTEIPat{\ad}{}
\TeXtoTEIPat{\add}{<hi rend="italic">add.</hi>} %% change 2023-04-16 MD
% \TeXtoTEIPat{\ann}{}
\TeXtoTEIPat{\ante}{<hi rend="italic">ante</hi> } %% change 2023-04-16 MD
\TeXtoTEIPat{\post}{<hi rend="italic">post</hi> } %% change 2023-04-16 MD
% \TeXtoTEIPat{\codd}{}
% \TeXtoTEIPat{\conj }{}
% \TeXtoTEIPat{\contin}{}
% \TeXtoTEIPat{\corr}{}
% \TeXtoTEIPat{\del}{}
% \TeXtoTEIPat{\dub}{}
% \TeXtoTEIPat{\emend }{}
% \TeXtoTEIPat{\expl}{}
% \TeXtoTEIPat{\ȩxplicat}{}
% \TeXtoTEIPat{\fol}{}
% \TeXtoTEIPat{\gloss}{}
% \TeXtoTEIPat{\ins}{}
% \TeXtoTEIPat{\im}{}
% \TeXtoTEIPat{\inmargine}{}
% \TeXtoTEIPat{\intextu}{}
% \TeXtoTEIPat{\indist}{}
% \TeXtoTEIPat{\iteravit}{}
% \TeXtoTEIPat{\lectio}{}
% \TeXtoTEIPat{\leginequit}{}
% \TeXtoTEIPat{\legn}{}
% \TeXtoTEIPat{\illeg}{<hi rend="italic">illeg.</hi>}
\TeXtoTEIPat{\illeg}{<gap reason="illeg."/>} %%% change 2023-04-11 mm
% \TeXtoTEIPat{\om}{<hi rend="italic">om.</hi>}
\TeXtoTEIPat{\om}{<gap reason="om."/>} %%% change 2023-04-11 mm
% \TeXtoTEIPat{\primman}{}
% \TeXtoTEIPat{\prob}{}
% \TeXtoTEIPat{\rep}{}
% \TeXtoTEIPat{\sequentia}{}
% \TeXtoTEIPat{\supralineam}{}
% \TeXtoTEIPat{\interlineam}{}
\TeXtoTEIPat{\vl}{<hi rend="italic">v.l.</hi>}
% \TeXtoTEIPat{\vide}{}
% \TeXtoTEIPat{\videtur}{}
% \TeXtoTEIPat{\crux}{}
% \TeXtoTEIPat{\cruxxx}{}
\TeXtoTEIPat{\unm}{<hi rend="italic">unm.</hi>}
\TeXtoTEIPat{\lacuna}{<gap reason="lac."/>} % addition 2024-03-24 MD
\TeXtoTEIPat{\lost}{<gap reason="lost"/>} % addition 2024-06-24 MD

% List of Scholars
\DeclareScholar{nos}{nos}[
forename=HPP,
surname=Team]

% Nullify \selectlanguage in TEI as it has been used in
% \DeclareWitness but should be ignored in TEI.
\TeXtoTEI{selectlanguage}{}



% additions/changes 2022-04-06 mm:
%\NewDocumentEnvironment{ava}{O{}}{\begin{ekdpar}\SetLineation{lineation=none}}{\end{ekdpar}}
%\NewDocumentEnvironment{col}{O{}}{\begin{ekdpar}\SetLineation{lineation=none}}{\end{ekdpar}}

% end additions
% added by MM 2022-10-25:
\NewDocumentEnvironment{postmula}{O{}}{
  \begin{ekdverse}
    \hspace{-\vgap}}{
  \end{ekdverse}
  \vskip 0.6\baselineskip
}
% modified by MD 2022-05-8:
\NewDocumentEnvironment{ava}{O{}}{
  \begin{ekdverse}
    \hspace{-\vgap}}{
  \end{ekdverse}
  \vskip 0.6\baselineskip
}
\NewDocumentEnvironment{col}{O{}}{
  \medskip
  \setvnum{col}
%  \selectlanguage{iast}
  \begin{ekdverse}
    \hspace{-\vgap}}{
  \end{ekdverse}
}

        
% modifications/additions by MM 2022-06-07:
\NewDocumentEnvironment{altava}{O{}}{
  \begin{ekdverse}\color{gray}
    \hspace{-\vgap}}{
  \end{ekdverse}
  \vskip 0.6\baselineskip
}   

% end additions

\SetTEIxmlExport{autopar=false}

\NewDocumentEnvironment{tlg}{O{}}{
  \begin{ekdverse}}{
  \end{ekdverse}
  \vskip 0.6\baselineskip}

% additions/changes 2022-08-22 mm:
\NewDocumentEnvironment{alttlg}{O{}}{
%  \stopvline
%  \addtocounter{saved@poemline}{-1}
%  \setvnum{\hindsection.\arabic{saved@poemline}*\arabic{poemline}}
%  \selectlanguage{iast}
  \begin{ekdverse}[type=altrecension]
    \color{gray}
  }{
  \end{ekdverse}
  \vskip 0.6\baselineskip
%  \addtocounter{saved@poemline}{1}
%  \startvline
%  \setvnum{\hindsection.\arabic{poemline}}
%  \selectlanguage{iast}
}

% additions/changes 2022-08-22 mm:
\def\startaltrecension{
  \stopvline
  \addtocounter{saved@poemline}{-1}
  \setvnum{\hindsection.\arabic{saved@poemline}*\arabic{poemline}}
	%\selectlanguage{iast}
	%\begin{ekdverse}[type=altrecension]
	%\color{gray}
	\small  % added 2023-10-12 MD
	}
\def\endaltrecension{
	%\end{ekdverse}
	%\vskip 0.75\baselineskip
  \addtocounter{saved@poemline}{1}
  \startvline
  \setvnum{\hindsection.\arabic{poemline}}
%  \selectlanguage{iast}
	\normalsize  % added 2023-10-12 MD
	}

\def\startaltnormal{
	\stopvline
	\addtocounter{saved@poemline}{-1}
	\setvnum{\hindsection.\arabic{saved@poemline}*\arabic{poemline}}}
\def\endaltnormal{\endaltrecension}



\NewDocumentCommand{\tl}{m}{#1}

%%%%%%

\def\startverse{\begin{ekdverse}} % übergangsweise
\def\endverse{\end{ekdverse}\vskip 0.6\baselineskip} % übergangsweise
\def\startgray{\color{gray}} % NEW! 2023-06-16
\def\endgray{\color{black}} % NEW! 2023-06-16


%%%%%%

\newcommand{\myfn}[1]{\footnote{\texteng{#1}}}
\renewcommand{\thefootnote}{\texteng{\arabic{footnote}}}
\newcommand{\devnote}[1]{\textdev{\scriptsize #1}}
%\newcommand{\outdent}{\hspace{-\vgap}}
\newcommand{\sgwit}[1]{{\footnotesize (\getsiglum{#1})}}
\newcommand{\NotIn}[1]{\texteng{\footnotesize (om. \getsiglum{#1})}}
\newcommand{\lineom}[2]{\texteng{\footnotesize (#1 om. \getsiglum{#2})}}
\newcommand{\grau}[1]{\textcolor{gray}{#1}} % partial altrecension
\newcommand{\graus}[1]{\small\textcolor{gray}{#1}\normalsize} % partial altrecension
\newcommand{\Anm}[1]{\begin{ekdverse}
	\texteng{\footnotesize (#1)}
	\end{ekdverse}
	\vskip 0.6\baselineskip}
\newcommand{\anm}[1]{\texteng{\footnotesize [#1]}}

\def\om{\texteng{\emph{om.\kern-0.8ex}}}
\def\illeg{\texteng{\emph{illeg.\kern-0.8ex}}} 
\def\damaged{\texteng{\emph{damaged}}} 
\def\unm{\texteng{\emph{unm.\ }}}
\def\gap{\texteng{\emph{gap}}}
%\def\recte{\texteng{r.\:}}
%\def\for{\texteng{for\ }}
%\def\sic{\texteng{\emph{sic}}}
%\def\oder{\texteng{\emph{or\ }}}
\def\ante{\texteng{\normalfont\emph{ante\ }}}
\def\add{\texteng{\normalfont\emph{add.}}}
\def\post{\texteng{\normalfont\emph{post\ }}}
\def\antecorr{\texteng{\textsubscript{ac}}}
\def\postcorr{\texteng{\textsubscript{pc}}}
\def\marma{\texteng{\textsuperscript{\#}}}
\def\marmas{\texteng{\textsuperscript{\#}} }
\def\crux{\texteng{\textsuperscript{\textdagger}}}


%\usepackage{textgreek}

%%% Gr1,4b,6
\DeclareWitness{N3}{\texteng{\textalpha\textsubscript{1}}}{NGMPP B 62-20}[]
        \DeclareHand{N3ac}{N3}{\texteng{\textalpha\rlap{\textsubscript{1}}\textsuperscript{ac}}}[]
        \DeclareHand{N3pc}{N3}{\texteng{\textalpha\rlap{\textsubscript{1}}\textsuperscript{pc}}}[]
\DeclareWitness{J5}{\texteng{\textalpha\textsubscript{2}}}{Jodhpur 02235}[]
\DeclareWitness{G4}{\texteng{\textalpha\textsubscript{3}}}{GOML 18885}[]% Telugu script
\DeclareWitness{N24}{\texteng{\textalpha\textsubscript{4}}}{NGMPP G 190-16}[]
\DeclareWitness{Gr1r}{\texteng{\textAlpha *}}{Gr1 reconstructed}[]

\DeclareWitness{P11}{\texteng{\textbeta\textsubscript{1}}}{}[]
\DeclareWitness{C6}{\texteng{\textbeta\textsubscript{2}}}{Lalchand M-2089}[]

\DeclareWitness{V3}{\texteng{\textbeta\textsubscript{\textomega}}}{Sampurnananda Library Sarasvati Bhavan 29899}[]

%%% Gr2

\DeclareWitness{N23}{\texteng{\textgamma\textsubscript{1}}}{NGMPP G 25-2}[]
        \DeclareHand{N23ac}{N23}{\texteng{\textgamma\rlap{\textsubscript{1}}\textsuperscript{ac}}}[]
        \DeclareHand{N23pc}{N23}{\texteng{\textgamma\rlap{\textsubscript{1}}\textsuperscript{pc}}}[]
\DeclareWitness{J7}{\texteng{\textgamma\textsubscript{2}}}{Jodhpur 02241}[]
%\DeclareWitness{V6}{\texteng{V\textsubscript{6}}}{Sampurnananda Library Sarasvati Bhavan 29991}[]
\DeclareWitness{K1}{\texteng{K\textsubscript{1}}}{Raghunātha Temple Library 4383}[settlement=Jammu]
        \DeclareWitness{K1ac}{\texteng{K\rlap{\textsubscript{1}}\textsuperscript{ac}\space}}{}[]
        \DeclareWitness{K1pc}{\texteng{K\rlap{\textsubscript{1}}\textsuperscript{pc}\space}}{}[]


%%% Gr3

\DeclareWitness{V19}{\texteng{\textdelta\textsubscript{1}}}{Sampurnananda Library Sarasvati Bhavan 30069}[]
\DeclareWitness{K3}{\texteng{\textdelta\textsubscript{2}}}{Privat collection}
\DeclareWitness{C7}{\texteng{\textdelta\textsubscript{3}}}{Lalchand M-6494}[]
%\DeclareWitness{C1}{\texteng{C\textsubscript{1}}}{Lalchand M-2080}[]%L1 And C1 very close (and come from same region)
%\DeclareWitness{P23}{\texteng{P\textsubscript{23}}}{}[]
%\DeclareWitness{L1}{\texteng{L\textsubscript{1}}}{SOAS RE 43454}[settlement=Jammu]

\DeclareWitness{J6}{\texteng{\textdelta\textsubscript{\textomega}}}{Jodhpur 02237}[]
        \DeclareHand{J6ac}{J6}{\texteng{\textdelta\rlap{\textomega}\textsuperscript{ac}}}[]
        \DeclareHand{J6pc}{J6}{\texteng{\textdelta\rlap{\textomega}\textsuperscript{pc}}}[]

%%% Gr4c

\DeclareWitness{P15}{\texteng{\textepsilon\textsubscript{1}}}{}[]
\DeclareWitness{N19}{\texteng{\textepsilon\textsubscript{2}}}{NGMPP E-1528-1 / E-1527-7(4)}[]
\DeclareWitness{V15}{\texteng{\textepsilon\textsubscript{3}}}{Sampurnananda Library Sarasvati Bhavan 30051}[]
        \DeclareHand{V15ac}{V15}{\texteng{\textepsilon\rlap{\textsubscript{3}}\textsuperscript{ac}}}[]
        \DeclareHand{V15pc}{V15}{\texteng{\textepsilon\rlap{\textsubscript{3}}\textsuperscript{pc}}}[]
\DeclareWitness{J11}{\texteng{\textepsilon\textsubscript{4}}}{Jodhpur 23532}[]
        \DeclareHand{J11ac}{J11}{\texteng{\textepsilon\rlap{\textsubscript{4}}\textsuperscript{i.t.}}}[]
        \DeclareHand{J11pc}{J11}{\texteng{\textepsilon\rlap{\textsubscript{4}}\textsuperscript{mg.}}}[alternative reading written by the first hand in margin or interlinearly (J11)]
%\DeclareWitness{J14}{\texteng{\textepsilon\textsubscript{5}}}{Jodhpur 02239}[]

%\DeclareWitness{L2}{\texteng{L\textsubscript{2}}}{Wellcome Collection O.36]}
\DeclareWitness{M1}{\texteng{M\textsubscript{1}}}{P-5682/4}[]

\DeclareWitness{N26}{\texteng{\textepsilon\textsubscript{\textomega}}}{NGMPP}[]
%\DeclareWitness{V17}{\texteng{\textepsilon\textsubscript{\textomega 3}}}{Sampurnananda Library Sarasvati Bhavan 30053}[]

\DeclareWitness{V1}{\texteng{\texteta\textsubscript{1}}}{Sampurnananda Library Sarasvati Bhavan 30109}[]
        \DeclareHand{V1ac}{V1}{\texteng{\texteta\rlap{\textsubscript{1}}\textsuperscript{ac}}}[]
        \DeclareHand{V1pc}{V1}{\texteng{\texteta\rlap{\textsubscript{1}}\textsuperscript{pc}}}[]

%%% Gr4d

\DeclareWitness{J10}{\texteng{\texteta\textsubscript{2}}}{MSPP Jodhpur 2230}[]
        \DeclareHand{J10ac}{J10}{\texteng{\texteta\rlap{\textsubscript{2}}\textsuperscript{ac}}}[]
        \DeclareHand{J10pc}{J10}{\texteng{\texteta\rlap{\textsubscript{2}}\textsuperscript{pc}}}[]

\DeclareWitness{N9}{\texteng{\texteta\textsubscript{\textomega}}}{NGMPP A62-33}[]
%\DeclareWitness{J15}{\texteng{\textepsilon\textsubscript{\textomega 4}}}{Jodhpur 9732A}[]

%%%

\DeclareWitness{Jyo}{\texteng{\textchi}}{Brahmānanda's version}[]
%\DeclareWitness{Tue}{\texteng{Tü}}{Ma I 339}[]

\DeclareWitness{ceteri}{\texteng{cett.}}{ceteri}[]

%%% Group Sigla

\DeclareWitness{Gr1}{\texteng{\textAlpha}}{N3,J5,G4}

\DeclareWitness{Gr2}{\texteng{\textGamma}}{N23,J7}
%\DeclareWitness{Gr2}{\texteng{%
%	\textbeta\textsubscript{1}%
%	\textbeta\textsubscript{2}%
%	}}{N23,J7}
\DeclareWitness{Gr3a}{\texteng{\textDelta}}{V19,K3,C7}
\DeclareWitness{Gr4b}{\texteng{%
	\textbeta\textsubscript{1}%
	\textbeta\textsubscript{2}%
	}}{C6,P11}
\DeclareWitness{GrB}{\texteng{%
	\textbeta\textsubscript{1}%
	\textbeta\textsubscript{2}%
	\textbeta\textsubscript{\textomega}%
	}}{C6,P11,V3}
\DeclareWitness{Gr4c}{\texteng{\textEpsilon}}{P15,N19,V15}

% \DeclareWitness{Gr4d}{\texteng{%
	% \texteta\textsubscript{1}%
	% \texteta\textsubscript{2}%
	% }}{V1,J10}
\DeclareWitness{Gr6}{\texteng{\textOmega}}{V3,J6,N9,N26}

\usepackage{textgreek}

%%% Gr1,4b,6
\DeclareWitness{N3}{\texteng{\textalpha\textsubscript{1}}}{NGMPP B 62-20}[]
        \DeclareHand{N3ac}{N3}{\texteng{\textalpha\rlap{\textsubscript{1}}\textsuperscript{ac}}}[]
        \DeclareHand{N3pc}{N3}{\texteng{\textalpha\rlap{\textsubscript{1}}\textsuperscript{pc}}}[]
\DeclareWitness{J5}{\texteng{\textalpha\textsubscript{2}}}{Jodhpur 02235}[]
\DeclareWitness{G4}{\texteng{\textalpha\textsubscript{3}}}{GOML 18885}[]% Telugu script
\DeclareWitness{N24}{\texteng{\textalpha\textsubscript{4}}}{NGMPP G 190-16}[]
\DeclareWitness{Gr1r}{\texteng{\textalpha{*}}}{Gr1 reconstructed}[]

\DeclareWitness{C6}{\texteng{\textbeta\textsubscript{1}}}{Lalchand M-2089}[]
\DeclareWitness{P11}{\texteng{\textbeta\textsubscript{2}}}{}[]

\DeclareWitness{V3}{\texteng{\textbeta\textsubscript{\textomega}}}{Sampurnananda Library Sarasvati Bhavan 29899}[]

%%% Gr2

\DeclareWitness{N23}{\texteng{\textgamma\textsubscript{1}}}{NGMPP G 25-2}[]
        \DeclareHand{N23ac}{N23}{\texteng{\textgamma\rlap{\textsubscript{1}}\textsuperscript{ac}}}[]
        \DeclareHand{N23pc}{N23}{\texteng{\textgamma\rlap{\textsubscript{1}}\textsuperscript{pc}}}[]
\DeclareWitness{J7}{\texteng{\textgamma\textsubscript{2}}}{Jodhpur 02241}[]
%\DeclareWitness{V6}{\texteng{V\textsubscript{6}}}{Sampurnananda Library Sarasvati Bhavan 29991}[]
\DeclareWitness{K1}{\texteng{K\textsubscript{1}}}{Raghunātha Temple Library 4383}[settlement=Jammu]
        \DeclareWitness{K1ac}{\texteng{K\rlap{\textsubscript{1}}\textsuperscript{ac}\space}}{}[]
        \DeclareWitness{K1pc}{\texteng{K\rlap{\textsubscript{1}}\textsuperscript{pc}\space}}{}[]


%%% Gr3

\DeclareWitness{V19}{\texteng{\textdelta\textsubscript{1}}}{Sampurnananda Library Sarasvati Bhavan 30069}[]
\DeclareWitness{K3}{\texteng{\textdelta\textsubscript{2}}}{Privat collection}
\DeclareWitness{C7}{\texteng{\textdelta\textsubscript{3}}}{Lalchand M-6494}[]
%\DeclareWitness{C1}{\texteng{C\textsubscript{1}}}{Lalchand M-2080}[]%L1 And C1 very close (and come from same region)
%\DeclareWitness{P23}{\texteng{P\textsubscript{23}}}{}[]
%\DeclareWitness{L1}{\texteng{L\textsubscript{1}}}{SOAS RE 43454}[settlement=Jammu]

\DeclareWitness{J6}{\texteng{\textdelta\textsubscript{\textomega}}}{Jodhpur 02237}[]
        \DeclareHand{J6ac}{J6}{\texteng{\textdelta\rlap{\textomega}\textsuperscript{ac}}}[]
        \DeclareHand{J6pc}{J6}{\texteng{\textdelta\rlap{\textomega}\textsuperscript{pc}}}[]

%%% Gr4c

\DeclareWitness{P15}{\texteng{\textepsilon\textsubscript{1}}}{}[]
\DeclareWitness{N19}{\texteng{\textepsilon\textsubscript{2}}}{NGMPP E-1528-1 / E-1527-7(4)}[]
\DeclareWitness{V15}{\texteng{\textepsilon\textsubscript{3}}}{Sampurnananda Library Sarasvati Bhavan 30051}[]
        \DeclareHand{V15ac}{V15}{\texteng{\textepsilon\rlap{\textsubscript{3}}\textsuperscript{ac}}}[]
        \DeclareHand{V15pc}{V15}{\texteng{\textepsilon\rlap{\textsubscript{3}}\textsuperscript{pc}}}[]
\DeclareWitness{J11}{\texteng{\textepsilon\textsubscript{4}}}{Jodhpur 23532}[]
        \DeclareHand{J11ac}{J11}{\texteng{\textepsilon\rlap{\textsubscript{4}}\textsuperscript{i.t.}}}[]
        \DeclareHand{J11pc}{J11}{\texteng{\textepsilon\rlap{\textsubscript{4}}\textsuperscript{mg.}}}[alternative reading written by the first hand in margin or interlinearly (J11)]
\DeclareWitness{J14}{\texteng{\textepsilon\textsubscript{5}}}{Jodhpur 02239}[]

%\DeclareWitness{L2}{\texteng{L\textsubscript{2}}}{Wellcome Collection O.36]}
\DeclareWitness{M1}{\texteng{M\textsubscript{1}}}{P-5682/4}[]

\DeclareWitness{N26}{\texteng{\textepsilon\textsubscript{\textomega}}}{NGMPP}[]
%\DeclareWitness{V17}{\texteng{\textepsilon\textsubscript{\textomega 3}}}{Sampurnananda Library Sarasvati Bhavan 30053}[]

\DeclareWitness{V1}{\texteng{\texteta\textsubscript{1}}}{Sampurnananda Library Sarasvati Bhavan 30109}[]
        \DeclareHand{V1ac}{V1}{\texteng{\texteta\rlap{\textsubscript{1}}\textsuperscript{ac}}}[]
        \DeclareHand{V1pc}{V1}{\texteng{\texteta\rlap{\textsubscript{1}}\textsuperscript{pc}}}[]

%%% Gr4d

\DeclareWitness{J10}{\texteng{\texteta\textsubscript{2}}}{MSPP Jodhpur 2230}[]
        \DeclareHand{J10ac}{J10}{\texteng{\texteta\rlap{\textsubscript{2}}\textsuperscript{ac}}}[]
        \DeclareHand{J10pc}{J10}{\texteng{\texteta\rlap{\textsubscript{2}}\textsuperscript{pc}}}[]

\DeclareWitness{N9}{\texteng{\texteta\textsubscript{\textomega}}}{NGMPP A62-33}[]
%\DeclareWitness{J15}{\texteng{\textepsilon\textsubscript{\textomega 4}}}{Jodhpur 9732A}[]

%%%

\DeclareWitness{Jyo}{\texteng{\textchi}}{Brahmānanda's version}[]
%\DeclareWitness{Tue}{\texteng{Tü}}{Ma I 339}[]

\DeclareWitness{ceteri}{\texteng{cett.}}{ceteri}[]

%%% Group Sigla

% \DeclareWitness{Gr1}{\selectlanguage{english}Gr\textsubscript{1}}{N3,J5,G4}
%\DeclareWitness{GrA1}{\texteng{A}}{N3,C6,V3}
%\DeclareWitness{GrA2}{\texteng{A}}{N3,P11,C6,V3}

\DeclareWitness{Gr2}{\texteng{\textGamma}}{N23,J7}
%\DeclareWitness{Gr2}{\texteng{%
%	\textbeta\textsubscript{1}%
%	\textbeta\textsubscript{2}%
%	}}{N23,J7}
\DeclareWitness{Gr3a}{\texteng{\textDelta}}{V19,K3,C7}
\DeclareWitness{Gr4b}{\texteng{%
	\textbeta\textsubscript{1}%
	\textbeta\textsubscript{2}%
	}}{C6,P11}
\DeclareWitness{GrB}{\texteng{\textBeta}}{C6,P11,V3}
\DeclareWitness{Gr4c}{\texteng{\textEpsilon}}{P15,N19,V15}

\DeclareWitness{Gr4d}{\texteng{%
	\texteta\textsubscript{1}%
	\texteta\textsubscript{2}%
	}}{V1,J10}
\DeclareWitness{Gr6}{\texteng{\textOmega}}{V3,J6,N9,N26}

\makepagestyle{HPed}
\makeoddhead{HPed}{\small\texteng{HP1}}{}{\small\texteng{\today}}
\makeevenhead{HPed}{\small\texteng{HP1}}{}{\small\texteng{\today}}
\makeoddfoot{HPed}{}{\small\texteng{\thepage}}{}
\makeevenfoot{HPed}{}{\small\texteng{\thepage}}{}
\def\hindsection{1}

% Ms list: N3,C6,V3,N23,V19,K3(34-),C7(2 fols. missing),P15,V15,V1,J10,Jyo
% deleted: C1,C8,V17,J10pc,N17,P28,W4,N19
% + J6?


\begin{document}
\pagestyle{HPed} %
\begin{otherlanguage}{iast}
\begin{ekdosis}

\begin{tlg}[hp01_001]
\tl{
\pada{\app{\lem[wit={ceteri}]{śrīādināthāya}
	\rdg[wit={P15}]{anādināthāya}
	\rdg[wit={V1}]{ādīśanāthāya}} namo'stu tasmai}\\+}
\tl{
\pada{yenopadiṣṭā haṭhayogavidyā/}\\+}
\tl{
\pada{\app{\lem[wit={ceteri}]{virājate}% +P11
	\rdg[wit={C6,Jyo}]{vibhrājate}} pronnata%
\app{\lem[wit={ceteri},alt={rājasaudham}]{rājasaudha\skp{m}} % °saudhasa N23, <<rāja>>saudham N3
	\rdg[wit={V3,J10,Jyo}]{rājayogam}}}-\\+}%
\tl{
\pada{\app{\lem[wit={ceteri},alt={āroḍhum}]{\skm{m }āroḍhu\skp{m}}
	\rdg[wit={P15}]{ārūḍham}}m icchor adhi% syllable "cchora" gap P15
\app{\lem[wit={ceteri}]{rohiṇīva}
	\rdg[wit={N23}]{rohiṇī ca}
	\rdg[wit={V3}]{roha eva}}//}
	\anm{1.1--33 lost \getsiglum{K3}}\\!}
\end{tlg}

%1.2
\begin{tlg}[hp01_002]
\tl{
\pada{praṇamya śrīguruṃ nāthaṃ}
\pada{svātmārāmeṇa
\app{\lem[wit={ceteri}]{yoginā}
	\rdg[wit={Gr3a,V15}]{dhīmatā}}/}\\+}
\tl{
\pada{kevalaṃ rājayogāya}
\pada{\app{\lem[wit={ceteri}]{haṭhavidyo} % V3 di omitted and inserted in margin
	\rdg[wit={J10}]{haṭhayogo}}padiśyate//}\\!}
\end{tlg}

%1.3
\begin{tlg}[hp01_003]
\tl{
\pada{\app{\lem[wit={J5,C6,N23,V15,V1,J10,Jyo}]{bhrāntyā}% +G4
	\rdg[wit={V3,Gr3a}]{bhrāntvā}
	\rdg[wit={N3,P15}]{bhrāntā}}
	bahu\app{\lem[wit={ceteri}]{matadhvānte} % thāṃte N3
	\rdg[wit={C6,N23}]{matadhvāntai}% ##
	\rdg[wit={V1}]{mataṃ bhrāntaṃ}}}
\pada{rāja\app{\lem[wit={N3,N23},alt={mārgam}]{mārga\skp{m}}
	\rdg[wit={ceteri}]{yogam}}% 
\app{\lem[wit={ceteri},alt={ajānatām}]{\skm{m }ajānatām}
	\rdg[wit={Gr3a,V15}]{ajānataḥ}}/}\\+}
\tl{
\pada{haṭhapradīpikāṃ  % °dipikāṃ N23
	\app{\lem[wit={ceteri}]{dhatte}
	\rdg[wit={C6}]{datte}}}
\pada{\app{\lem[wit={ceteri}]{svātmārāmaḥ}
	\rdg[wit={V3}]{svātmārāma}
	\rdg[wit={N23}]{svātmārame}}
\app{\lem[wit={N3,C6,N23,P15,V15,V1,Jyo}]{kṛpākaraḥ}
	\rdg[wit={Gr3a}]{kṛpāparaḥ}
	\rdg[wit={V3,J10}]{prakāśyate}}//}\\!}
\end{tlg}

%1.4
\begin{tlg}[hp01_004]
\tl{
\pada{haṭha\app{\lem[wit={ceteri}]{vidyāṃ hi}
	\rdg[wit={N23,V1}]{vidyā hi}
	\rdg[wit={N3}]{vidyo hi}}
\app{\lem[wit={ceteri}]{matsyendra}
	\rdg[wit={N23}]{tachaṃdra}}}%
\pada{gorakṣādyā % °dya C6
	\app{\lem[wit={ceteri}]{vijānate}% +G4
	\rdg[wit={N3}]{virājate}}/}\\+} % gorakṣādiṣu rājate J5
\tl{
\pada{\app{\lem[wit={ceteri}]{svātmārāmo}
	\rdg[wit={V19,P15}]{ātmārāmo}}}%
\pada{\app{\lem[wit={ceteri}]{'thavā}
	\rdg[wit={C7}]{yathā}
	\rdg[wit={P15}]{mahā}} yogī
\app{\lem[wit={ceteri}]{jānīte}
	\rdg[wit={V3}]{jānaṃte}}
	tatprasādataḥ//}\\!} % V3 ta omitted and inserted in margin
\end{tlg}


%1.5
\begin{tlg}[hp01_005]
\tl{
\pada{\app{\lem[wit={ceteri}]{śrīādinātha}% nāthā N23
	\rdg[wit={V1,J10}]{ādināthādi}}%
	matsyendra}%
\pada{\app{\lem[wit={N3,C6,N23,J10,Jyo}]{śābarā}
	\rdg[wit={V1}]{śabarā}
	\rdg[wit={Gr3a}]{śāradā}
	\rdg[wit={V3}]{śāgarā}
	\rdg[wit={P15,V15}]{sāgarā}}nanda%
	bhairavāḥ/}\\+} % °vaḥ N3, °va V3
\tl{
\pada{\app{\lem[wit={ceteri}]{cauraṅgī}
	\rdg[wit={P15}]{coraṃgī}}%
\app{\lem[wit={ceteri}]{mīna}
	\rdg[wit={P15}]{mena}
	\rdg[wit={V19}]{ṣīna}}gorakṣa}%
\pada{\app{\lem[wit={ceteri}]{virūpākṣa}
	\rdg[wit={N3,V19}]{virūpākṣaḥ}
	\rdg[wit={C7}]{virūpākhya}}%
\app{\lem[wit={ceteri}]{bileśayāḥ} % V3 om. ḥ; +G4
	\rdg[wit={N3,J5,V19},post=\texteng{(savālmikaḥ \getsiglum{N3ac})}]{savālikaḥ}}//}\\!}% +J5
\end{tlg}

%1.6
\begin{tlg}[hp01_006]
\tl{
\pada{\app{\lem[wit={ceteri}]{manthāna}
	\rdg[wit={Jyo}]{manthāno}}%
\app{\lem[wit={ceteri}]{bhairavo}
	\rdg[wit={N23}]{bhairavā}} yogī}
\pada{\app{\lem[wit={ceteri}]{siddha}% +P11
	\rdg[wit={Jyo}]{siddhir}
	\rdg[wit={C6,J10}]{śuddha}% + Jyo-mss; śraddha C6
	\rdg[wit={V1}]{suddha}}%
\app{\lem[wit={ceteri}]{buddhaś ca}% +P11,G4
	\rdg[wit={N3,C6,V3}]{buddhiś ca}% +J5; budhi V3
	\rdg[wit={C7}]{pādaś ca}}
\app{\lem[wit={Gr3a,Jyo}]{kanthaḍiḥ}% +L2
%	\rdg[wit={C7}]{kathaḍiḥ}
	\rdg[wit={J5,C6,V3,V15}]{kanthaḍī}
	\rdg[wit={N23}]{kanthaḍīḥ}
	\rdg[wit={P15}]{kanthaliḥ}
	\rdg[wit={N3,V1}]{kanthalī}
	\rdg[wit={J10}]{kandalī}}/}\\+}
\tl{
\pada{\app{\lem[wit={N3,J5},prewit=\texteng{\textless\ goraṃṭaka}]{goraṇṭakaḥ}
	\rdg[wit={C6,N23,P15,Jyo}]{koraṇṭakaḥ}
	\rdg[wit={V15}]{koraṃḍakaḥ}
	\rdg[wit={V3,C7,V1,J10}]{pauraṇṭakaḥ}
	\rdg[wit={V19},alt={\om}]{\skp{\om}}} % +C1
\app{\lem[wit={ceteri}]{surānandaḥ}
	\rdg[wit={V3}]{surānanda}
	\rdg[wit={N3}]{kṣurānaṃda}
	\rdg[wit={V19},alt={\om}]{\skp{\om}}}}
\pada{\app{\lem[wit={ceteri}]{siddhapādaś ca}
	\rdg[wit={V15}]{śrīpādaś caiva}
	\rdg[wit={V19},alt={\om}]{\skp{\om}}}
\app{\lem[wit={N23,V15,J10,Jyo}]{carpaṭiḥ}
	\rdg[wit={N3,V3}]{carpaṭi}% paryaṭī J5
	\rdg[wit={C6,C7,V1}]{carpaṭī}
	\rdg[wit={P15}]{karpaṭiḥ}
	\rdg[wit={V19},alt={\om}]{\skp{\om}}}//} \lineom{cd}{V19}\myfn{%
	Usually omitted in \getsiglum{Gr3a}. \getsiglum{C7} seems to have supplied this hemistich from a ms of another branch.}\\!}
\end{tlg}

\newpage
%1.7
\begin{tlg}[hp01_007]
\tl{
\pada{\app{\lem[resp=emend]{kāṇerī}
	\rdg[wit={J5,C6,N23,V1}]{kaṇerī}
	\rdg[wit={N3}]{kaṇeri}
	\rdg[wit={P15}]{kaṇeriḥ} % kanerī L2,N19
	\rdg[wit={V15}]{kāṇeriḥ}
	\rdg[wit={V3,J10,Jyo}]{kānerī}
	\rdg[wit={V19}]{kariṇī}
	\rdg[wit={C7}]{karaṇī}}
\app{\lem[wit={ceteri}]{pūjya}
	\rdg[wit={V1}]{pūrya}
	\rdg[wit={P15,J10}]{pūrva}}pādaś ca}
\pada{\app{\lem[wit={ceteri}]{nityanātho}
	\rdg[wit={J10}]{dhvaninātho}
	\rdg[wit={Gr3a}]{bilvanātho}}
	nirañjanaḥ/}\\+}  % °naṃ V3
\tl{
\pada{\app{\lem[wit={ceteri}]{kapālī}
	\rdg[wit={C7}]{kapāsī}} bindunāthaś ca}
\pada{kāka\app{\lem[wit={C6,N23,P15,V15,V1,Jyo}]{caṇḍīśvarāhvayaḥ}% svarā C6
	\rdg[wit={V3,Gr3a,J10}]{caṇḍīśvarādayaḥ} % mayaḥ M1,P11,etc.
	\rdg[wit={N3,J5,G4},post=\texteng{caṃḍe° \getsiglum{N3}, gajaḥ \getsiglum{J5}}]{caṃḍīśvaro gayaḥ}}//}\\!}
\end{tlg}

%\newpage

%1.8
\begin{tlg}[hp01_008]
\tl{
\pada{\app{\lem[wit={G4}]{allama}% +G3,G4,M1?
	\rdg[wit={N3}]{alama}
	\rdg[wit={V3,P15,V1,J10,Jyo}]{allamaḥ}% +N4,N13; allāmaḥ Tue,ed.
	\rdg[wit={J5}]{allumaḥ}
	\rdg[wit={N23}]{allasaḥ}
	\rdg[wit={C6}]{alasaḥ}
	\rdg[wit={V15}]{amelleḥ}
	\rdg[wit={Gr3a}]{sukṣamaḥ}}prabhudevaś ca}
\pada{\app{\lem[wit={J5,V3,V15,Jyo}]{ghoḍācolī}
	\rdg[wit={G4,N23}]{ghoḍācūlī}% cūli? G4
	\rdg[wit={N3}]{ghoḍāculī}
	\rdg[wit={C6}]{goḍācūlī}
	\rdg[wit={Gr3a}]{ghoṭācolī}
	\rdg[wit={V1,J10}]{ghorācolī}
	\rdg[wit={P15}]{\_\,gacolī}}
\app{\lem[wit={ceteri}]{ca}
	\rdg[wit={V19}]{sa}
	\rdg[wit={C7}]{gha}}
\app{\lem[wit={V1,J10}]{ṭiṇṭiṇī}
	\rdg[wit={N3,Jyo}]{ṭiṃṭiṇiḥ}% ##
	\rdg[wit={V19}]{ṭiṃṭiniḥ}
	\rdg[wit={N23}]{tīṭiṇiḥ}
	\rdg[wit={C6}]{ṭiṃṭaṇī}% +ḥ C6pc
	\rdg[wit={C7}]{ṭimbhiṇiḥ}
	\rdg[wit={J5}]{ṭīṃcaṇī}
	\rdg[wit={V3}]{ciṃciṇī}
	\rdg[wit={V15}]{ciṃciṇīḥ}
	\rdg[wit={P15}]{caṃcaṇiḥ}}/}\\+}
\tl{
\pada{\app{\lem[wit={C6,V1,J10}]{bhālukī}% Jyo-mss,N19; bhālakī L2, bhālukin MYO
	\rdg[wit={N3}]{bhāluki}
	\rdg[wit={N23}]{bhānukin}
	\rdg[wit={Jyo}]{bhānukī}
	\rdg[wit={J5}]{tālukī}
	\rdg[wit={Gr3a}]{vālukir}
	\rdg[wit={V3,P15}]{vāsukir}
	\rdg[wit={V15}]{vāsukīr}
	\rdg[wit={G4}]{vāsuki}}
\app{\lem[wit={G4,V3,N23,C7,V1,J10}]{nāgabodhaś ca}% °bīdhaś ca? G4
	\rdg[wit={P11,P15,V15}]{nāgabodhiś ca}
	\rdg[wit={V19}]{nāgarodhaś ca}
	\rdg[wit={C6}]{nāgadevaś ca}
	\rdg[wit={N3}]{namioḍḍīśa}
	\rdg[wit={J5}]{nāma tuṃḍīśa}
	\rdg[wit={Jyo}]{nāradevaś ca}}} % nāgadevaś Jyo-mss
\pada{\app{\lem[wit={G4,C6,V3,J10}]{khaṇḍa}% Jyo-mss; khaṃḍikaḥ pā° C6
	\rdg[wit={V15,Jyo}]{khaṇḍaḥ}
	\rdg[wit={N23,V1}]{khaṃḍaṃ}
	\rdg[wit={P15}]{kaṃṭhaḥ}
	\rdg[wit={C7}]{caṇḍa} %+C1
	\rdg[wit={V19}]{caṃḍī}
	\rdg[wit={N3,J5}]{siddhaḥ}}kāpālikas tathā//}\\!}
\end{tlg}

%1.9
\begin{tlg}[hp01_009]
\tl{
\pada{ityādayo mahāsiddhā}
\pada{haṭhayoga\app{\lem[wit={ceteri}]{prabhāvataḥ}
	\rdg[wit={C6,V3,N23}]{prasādataḥ}}/}\\+}
\tl{
\pada{\app{\lem[wit={ceteri}]{khaṇḍayitvā}
	\rdg[wit={V19}]{khaṃḍaïtvā}} kāladaṇḍaṃ} % kāra N3
\pada{\app{\lem[wit={ceteri}]{brahmāṇḍe}
	\rdg[wit={P15}]{brahmāṇḍaṃ}}
\app{\lem[wit={G4,C6,N23,C7,P15,V1,J10,Jyo}]{vicaranti}
	\rdg[wit={V3}]{tu caranti}
	\rdg[wit={N3,J5,V19,V15}]{°ṣu caranti}} te//}\\!}
\end{tlg}

%1.10
\begin{tlg}[hp01_010]
\tl{
\pada{\app{\lem[wit={ceteri}]{saṃsāratāpa}
	\rdg[wit={V3}]{saṃsārāśrama}
	\rdg[wit={J10}]{saṃsāraśrama}
	\rdg[wit={C6,Jyo}]{aśeṣatāpa}}taptānāṃ}
\pada{\app{\lem[wit={C6,N23,Gr3a,P15,V15,Jyo}]{samāśraya}
	\rdg[wit={V1}]{samāśrayo}
	\rdg[wit={V3,J10}]{āśrayo'yaṃ}
	\rdg[wit={N3}]{samagrapra°}}%
\app{\lem[wit={C6,N23,C7,P15,Jyo}]{maṭho haṭhaḥ}
	\rdg[wit={V19}]{mato haṭhaḥ}% +L2
	\rdg[wit={V15}]{mahāmaṭhaḥ}
	\rdg[wit={V1}]{haṭho maṭhaḥ}
	\rdg[wit={V3,J10}]{haṭho mataḥ}
	\rdg[wit={N3}]{°thamo haṭhaḥ}}/}\\+}
\tl{
\pada{\app{\lem[wit={ceteri}]{aśeṣa}
	\rdg[wit={P15}]{samasta}
	\rdg[wit={N3},alt={\om}]{\skp{\om}}}yoga%
\app{\lem[wit={N23,Gr3a,P15,V15,J10},alt={jagatām}]{jagatā\skp{m}}
	\rdg[wit={V1}]{jagatīm}
	\rdg[wit={C6,V3,Jyo}]{yuktānām}
	\rdg[wit={N3},alt={\om}]{\skp{\om}}}}%
\pada{\app{\lem[wit={C6,N23,P15,V15,V1,Jyo},alt={ādhāra}]{\skm{m }ādhāra}
	\rdg[wit={V3,Gr3a,J10}]{ādhāraḥ}
	\rdg[wit={N3},alt={\om}]{\skp{\om}}}%
\app{\lem[wit={ceteri}]{kamaṭho haṭhaḥ}
	\rdg[wit={V1}]{kahaṭho maṭhaḥ}
	\rdg[wit={N3},alt={\om}]{\skp{\om}}}//} \lineom{cd}{N3}\\!}
\end{tlg}

%1.11
\begin{tlg}[hp01_011]
\tl{
\pada{haṭhavidyā paraṃ
\app{\lem[wit={ceteri}]{gopyā}
	\rdg[wit={V3}]{gopyaṃ}}
\app{\lem[wit={N3,C6,N23,V15,V1,J10}]{yogināṃ}
	\rdg[wit={V3}]{yogīnāṃ}
	\rdg[wit={Gr3a,P15,Jyo}]{yoginā}}}
\pada{siddhi\app{\lem[wit={N3,V3,N23,V15,V1,J10},alt={icchatām}]{\skm{m }icchatām}
	\rdg[wit={C6,C7,P15,Jyo}]{icchatā}
	\rdg[wit={V19}]{icchitā}}/}\\+}
\tl{
\pada{bhaved vīryavatī guptā}
\pada{\app{\lem[wit={ceteri}]{nirvīryā} % viryā N3, nivīryā C6
	\rdg[wit={V19}]{nirvījā}
	\rdg[wit={J10}]{nirvāryā}}
\app{\lem[wit={ceteri}]{tu}% +J5,G4
	\rdg[wit={N3,V19,V15}]{ca}
	\rdg[wit={P15}]{sā}} prakāśitā//}\\!}
\end{tlg}

%1.12
\begin{tlg}[hp01_012]
\tl{
\pada{surājye dhārmike deśe}
\pada{subhikṣe % surbhikṣe C6
\app{\lem[wit={ceteri}]{nirupadrave}
	\rdg[wit={P15}]{kṣemabhadrade}}/}%
\myfn{\getsiglum{Jyo} has \devnote{dhanuḥpramāṇaparyantaṃ śilāgnijalavarjite} between the hemistiches.}\\+}
\tl{
\pada{\app{\lem[wit={N3,C6,N23,Gr3a,P15,V15}]{ekānta}% +G4
	\rdg[wit={V3,V1,J10,Jyo}]{ekānte}}% +J5
	maṭhikāmadhye} % sidhyai C6
\pada{sthātavyaṃ haṭhayoginā//}\\!} % yogināṃ N3
\end{tlg}

\newpage
%1.13
\begin{tlg}[hp01_013]
\tl{
\pada{\app{\lem[wit={ceteri}]{alpadvāram arandhra}
	\rdg[wit={N23}]{ākalpadvā<<ra>>raṃdhra}
	\rdg[wit={P15}]{alpadvāram aruṃ<<dha>>}}garta%
\app{\lem[wit={V19}]{piṭharaṃ}
	\rdg[wit={C7}]{piṭhiraṃ}
	\rdg[wit={N23}]{piṭhikaṃ}
	\rdg[wit={N3,P15,V15}]{piṭ(h)akaṃ}% ## peṭanaṃ G4
	\rdg[wit={C6}]{paṭikaṃ}
	\rdg[wit={J5}]{viṭakaṃ}
	\rdg[wit={V3,J10}]{viṭapaṃ}
	\rdg[wit={Jyo}]{vivaraṃ}
	\rdg[wit={V1}]{sahitaṃ}}
\app{nātyucca\lem[wit={ceteri}]{nīcā}
	\rdg[wit={N3pc}]{nītā}
	\rdg[wit={V3}]{noccā}
	\rdg[wit={N23}]{naṃcā}}%
\app{\lem[wit={ceteri}]{yataṃ}
	\rdg[wit={J10}]{yitaṃ}
	\rdg[wit={V1}]{yutaṃ}
	\rdg[wit={V15}]{vṛtaṃ}}}\\+}
\tl{
\pada{samyaggomaya% saṃgomaya N3
\app{\lem[wit={C6,N23,Gr3a,V15,V1,Jyo}]{sāndra}
	\rdg[wit={V3,J10}]{sārdra}
	\rdg[wit={N3}]{sāpra}
	\rdg[wit={P15}]{lipta}}%
\app{\lem[wit={G4,C6,N23,V19,J10,Jyo}]{liptam amalaṃ}
	\rdg[wit={N3,J5,P11,V3,C7,V15}]{liptavimalaṃ}
	\rdg[wit={V1}]{liptam abilaṃ}
	\rdg[wit={P15}]{sāndravimalaṃ}}
\app{\lem[wit={ceteri}]{niḥśeṣa}% +J5,G4
	\rdg[wit={N3,J10}]{nirdoṣa}}%
\app{\lem[wit={V3,P15,V1,J10}]{bādhojjhitam} % bādhodgataṃ N19 ##
	\rdg[wit={C6}]{vātojjhitaṃ}
	\rdg[wit={Gr3a,V15,Jyo}]{jantūjjhitam}
	\rdg[wit={N23}]{jaṃbhūdgitaṃ}
	\rdg[wit={N3}]{jyaṃtyūpsitaṃ}}/}\\+}
\tl{
\pada{bāhye maṇḍapa%
\app{\lem[wit={ceteri}]{vedikūpa}
	\rdg[wit={V15}]{kūpavedi}}%
\app{\lem[wit={N3,C6,V3,N23,V15,Jyo}]{ruciraṃ}
	\rdg[wit={Gr3a,P15,V1,J10}]{racitaṃ}}
	prākārasaṃveṣṭitaṃ}\\+}
\tl{
\pada{proktaṃ yogamaṭhasya lakṣaṇam idaṃ siddhair haṭhābhyāsibhiḥ//}\\!}
% mahasya, siddhai, °bhyāsabhiḥ N3
\end{tlg}

%\newpage

%1.14
\begin{tlg}[hp01_014]
\tl{
\pada{evaṃvidhe maṭhe sthitvā} % viddhe C6
\pada{sarvacintā\app{\lem[wit={ceteri}]{vivarjitaḥ}
	\rdg[wit={P15}]{vivarjite}}/}\\+}
\tl{
\pada{gurūpadiṣṭamārgeṇa} % gurupa V3; padeśa C6
\pada{\app{\lem[wit={N3,C6,N23,Gr3a,V15,Jyo}]{yogam eva}
	\rdg[wit={P15,V1,J10}]{yogam evaṃ}
	\rdg[wit={V3}]{yogamārgaṃ}}
\app{\lem[wit={N3pc,J5,V3,N23,P15,V1,J10,Jyo}]{samabhyaset}
	\rdg[wit={N3ac,G4,C6,Gr3a,V15}]{sadābhyaset}}//}\\!} % ##
\end{tlg}

%1.15
\begin{tlg}[hp01_015]
\tl{
\pada{\app{\lem[wit={C6,N23,C7,V1,J10,Jyo}]{atyāhāraḥ}
	\rdg[wit={V15}]{atyāhāra}
	\rdg[wit={V19}]{alpāhāra}
	\rdg[wit={N3}]{alpāhāro}
	\rdg[wit={V3}]{ātmāhāraḥ}
	\rdg[wit={P15}]{natyāhāsaḥ}}
\app{\lem[wit={ceteri}]{prayāsaś ca}
	\rdg[wit={V19}]{prayāsaś cā}
	\rdg[wit={N3,N23}]{pravāsaś ca}}}
\pada{prajalpo%
\app{\lem[wit={ceteri}]{'niyama}% ,postwit={(\getsiglum{N17} with an explicit Avagraha)}
	\rdg[wit={V15}]{niyamā}% J10 corrects to niyamo, then deletes the inserted "o"
	\rdg[wit={P15}]{nama}}grahaḥ/}\\+}
\tl{
\pada{\app{\lem[wit={ceteri}]{janasaṅgaś ca}
	\rdg[wit={C6}]{janasaṃgaṃ ca}
	\rdg[wit={P15}]{janasaṃkara}} laulyaṃ ca}
\pada{ṣaḍbhi\app{\lem[wit={ceteri},alt={yogo vinaśyati}]{\skm{r }yogo vinaśyati}
	\rdg[wit={N23}]{yogā vinaśyati}
	\rdg[wit={J10}]{yogaś ca naśyati}
	\rdg[wit={V1}]{yogaḥ praṇaśyati}}//}\\!}
\end{tlg}


%1.16
\begin{tlg}[hp01_016]
\tl{
\pada{\app{\lem[wit={ceteri},alt={utsāhāt/n}]{utsāhā\skp{t}} % °h<<ā>><n> V15
	\rdg[wit={V3}]{utsāha}}%
\app{\lem[wit={N23,Gr3a,Jyo},alt={sāhasād}]{\skm{t }sāhasā\skp{d}}
	\rdg[wit={N3,V3,P15,V15,V1,J10}]{niścayād}
	\rdg[wit={C6}]{niyamād}}%
\app{\lem[wit={C6,N23,Gr3a,P15,V15,Jyo},alt={dhairyāt}]{\skm{d }dhairyā\skp{t}}
	\rdg[wit={N3,V3}]{dhairyā}
	\rdg[wit={V1,J10}]{vairyāt}}t}
\pada{tatva\app{\lem[wit={N23,Jyo}]{jñānāc ca niścayāt}
	\rdg[wit={C6,Gr3a}]{jñānaviniścayāt}
	\rdg[wit={N3,P11,V3,P15,V15,V1,J10}]{jñānāc ca darśanāt}}/}\\+} % V3 °jñānā ca; +M1,M4
\tl{
\pada{\app{\lem[wit={ceteri}]{jana}
	\rdg[wit={N23}]{nija}}saṅgaparityāgāt}%
\pada{ṣaḍbhi\app{\lem[wit={ceteri}, alt={yogaḥ prasidhyati}]{\skm{r }yogaḥ prasidhyati}
	\rdg[wit={V1,J10}]{yogas tu sidhyati} % V3 om. ḥ
	\rdg[wit={V19}]{yogo prasidhyati}}//}\\!}
\end{tlg}
%\myfn{In \getsiglum{J8} the following verses are inserted after 1.16:\\
%\textdev{ahiṃsā satyamasteyaṃ brahmacaryaṃ kṣamā dhṛtiḥ/
%dayārjavamitāhāraḥ śaucaṃ caiva yamā daśa//}\\
%\textdev{tapaḥ saṃtoṣamāstikyaṃ dānamīśvarapūjanam/
%siddhāntaṃ śravaṇaṃ caiva hrī matiśca japo hutam//}}\\!}


\startaltrecension
\begin{alttlg}[hp01_016_1]
\tl{
\pada{ahiṃsā satyam asteyaṃ}
\pada{brahmacaryaṃ kṣamā dhṛtiḥ/}\\+
}\tl{
\pada{\app{\lem[wit={C7}]{dayārjavaṃ}
	\rdg[wit={V3}]{dayārjava}} mitāhāraḥ}
\pada{śaucaṃ caiva yamā daśa//} \sgwit{V3,C7}\\!
}\end{alttlg}

\begin{alttlg}[hp01_016_2]
\tl{
\pada{tapaḥ saṃtoṣam āstikyaṃ}
\pada{dānam īśvarapūjanam/}\\+
}\tl{
\pada{\app{\lem[resp=emend]{siddhānta}
	\rdg[wit={V3}]{siddhāntaṃ}}śravaṇaṃ caiva}
\pada{hrī matiś ca japo hutam//} \sgwit{V3}\\!
}\end{alttlg}
\endaltrecension

% \begin{alttlg}[hp01_016_3]
% \tl{
% \pada{niyamā daśa saṃproktā}
% \pada{yogaśāstraviśāradaiḥ//} \sgwit{V17}\\!
% }\end{alttlg}

\newpage
\begin{altava}[hp01_017]
\grau{\app{\lem[wit={P15}]{athāsanāni}
	\rdg[wit={V15}]{<<atha>> āsanāni}}/
	\sgwit{P15,V15}}
\end{altava}

%1.17
\begin{tlg}[hp01_017]
\tl{
\pada{haṭhasya prathamāṅgatvād}
\pada{āsanaṃ pūrvam ucyate/}\\+}
\tl{
\pada{\app{\lem[wit={ceteri},alt={tat kuryād}]{tat kuryā\skp{d}}
	\rdg[wit={N3}]{na kuryād}
	\rdg[wit={J10,Jyo}]{kuryāt tad}}%
\app{\lem[wit={J5,P11,N23,C7,V1,Jyo},alt={āsanaṃ sthairyam}]{\skm{d }āsanaṃ sthairya\skp{m}}% +J5,G4
	\rdg[wit={V15}]{āsane sthairyaṃ} % °yaṃ a°(!) V15
	\rdg[wit={N3,C6,V3,V19,P15}]{āsanasthairyam}
	\rdg[wit={J10}]{āsanaṃ tasmād}}}%
\pada{m ārogyaṃ cāṅga\app{\lem[wit={N3,P11,V3,P15,V1,J10}]{pāṭavam}% +G4, pāṭave J5
	\rdg[wit={C6,N23,Gr3a,V15,Jyo}]{lāghavaṃ}}//}%
\myfn{1.17--18 are transposed in \getsiglum{V19,V15}. \getsiglum{C7} has it twice -- in the regular place and after 1.18. Readings of the first occurence are closer to \getsiglum{J10} (\devnote{kuryāt tad āsanaṃ tasmād} and \devnote{aṅgapāṭavam}) and the second to \getsiglum{V19} which belongs to the same ms group as \getsiglum{C7}. In the apparatus, readings of the second occurence only are included.}\label{I17}\\!}
\end{tlg}

%1.18
\begin{tlg}[hp01_018]
\tl{
\pada{vasiṣṭhādyaiś ca munibhi}%
\pada{r matsyendrādyaiś ca yogibhiḥ/}\\+}
\tl{
\pada{aṅgīkṛtāny āsanāni}
\pada{\app{\lem[wit={C6,V3,N23,P15,V15,J10,Jyo}]{kathyante}
	\rdg[wit={V1}]{likhyante}
	\rdg[wit={Gr3a}]{vakṣyante}}
kānicin mayā//}\linelabel{v18d}
	\anm{1.18--28 lost \getsiglum{N3}}\\!}
\end{tlg}



%1.19
\begin{tlg}[hp01_019]
\tl{
\pada{\app{\lem[wit={ceteri},alt={jānūrvor}]{jānūrvo\skp{r}} % jānu° V3
	\rdg[wit={V19}]{jaṃtūrvo}}%
\app{\lem[wit={ceteri},alt={antare}]{\skm{r }antare}
	\rdg[wit={N23,P15}]{antaraṃ}}
samyak}\pada{kṛtvā pādatale
\app{\lem[wit={ceteri}]{ubhe}
	\rdg[wit={N23}]{śubhe}}/}\\+}
\tl{
\pada{\app{\lem[wit={ceteri}]{ṛjukāyaḥ}
	\rdg[wit={N23ac,V1}]{ṛjukāya}
	\rdg[wit={V3}]{ṛjuḥ kāya}}
\app{\lem[wit={ceteri}]{samāsīnaḥ}
	\rdg[wit={V3}]{samāsīnaṃ}}}
\pada{svastikaṃ \app{\lem[wit={ceteri},alt={tat}]{ta\skp{t}}
	\rdg[wit={N23}]{ca}}t pracakṣate//}
	\anm{found before \ref{I17} in \getsiglum{P15}}\\!}
\end{tlg}

%1.20
\begin{tlg}[hp01_020]
\tl{
\pada{savye
\app{\lem[wit={ceteri}]{dakṣiṇa}
	\rdg[wit={V3}]{dakṣaṇa}}%
\app{\lem[wit={ceteri}]{gulphaṃ}
	\rdg[wit={C7}]{gulphe}} tu}
\pada{pṛṣṭhapārśve niyojayet/}\\+} % pṛṣṭi C6
\tl{
\pada{dakṣiṇe\app{\lem[wit={ceteri}]{'pi}% +J5,G4
	\rdg[wit={N23}]{ca}
	\rdg[wit={Gr3a}]{tu}} tathā
\app{\lem[wit={ceteri}]{savyaṃ} % BKhP
	\rdg[wit={V1}]{savye}}}
\pada{\app{\lem[wit={ceteri}]{gomukhaṃ}
	\rdg[wit={C7}]{gomukhe}}  %+C1
\app{\lem[wit={G4,Gr3a,V15,V1,J10}]{gomukhaṃ yathā}
	\rdg[wit={V3}]{gomukhaṃ tathā}
	\rdg[wit={J5}]{gomukhaṃ bhavet}
	\rdg[wit={Jyo}]{gomukhākṛti}
	\rdg[wit={C6,P11,N23}]{gomukhākṛtiḥ}}//} \NotIn{P15}\\!}
\end{tlg}


%1.21
\begin{tlg}[hp01_021]
\tl{
\pada{\app{\lem[wit={ceteri}]{ekaṃ}
	\rdg[wit={Gr3a,V15}]{eka}}
	pāda\app{\lem[wit={G4,V1},alt=athaikasmin]{\skm{m }athaikasmi\skp{n}}
	\rdg[wit={J5,C6,V3,N23,P15,J10,Jyo}]{tathaikasmin}%A1 ##
	\rdg[wit={Gr3a}]{yathaikasmi}}}%
\pada{\app{\lem[wit={V3,V1},alt={vinyasyoruṇi}]{\skm{n }vinyasyoruṇi}
	\rdg[wit={J10}]{vinyasyoruṃni}
	\rdg[wit={G4}]{+\,veśyorūṇi}
	\rdg[wit={C6,P11,N23,Gr3a,V15,Jyo}]{vinyased ūru}%  ##
	\rdg[wit={J5}]{vinyaserupa°}
	\rdg[wit={P15}]{vinasyorasi}}
\app{\lem[wit={ceteri}]{saṃsthitam}
	\rdg[wit={P15}]{saṃsthitaḥ}
	\rdg[wit={P11,N23,Jyo}]{°ṇi sthitaṃ}
	\rdg[wit={C6}]{°ṇi sthitaḥ}
	\rdg[wit={J5}]{°ri sthitaṃ}}/}\\+}
\tl{
\pada{\app{\lem[wit={ceteri}]{itarasmiṃ}
	\rdg[wit={V19}]{itarāsmiṃ}}s tathā
\app{\lem[wit={ceteri}]{coruṃ}% +G4, coru J7
	\rdg[wit={C6}]{corau}
	\rdg[wit={V19}]{ce\,..}}}
\pada{vīrāsanam 
\app{\lem[wit={ceteri}]{itīritam} % īḍitaṃ C6
	\rdg[wit={J5}]{iti smṛtam}
	\rdg[wit={G4}]{udāhṛtaṃ}}//}\\!}
\end{tlg}


\begin{tlg}[hp01_022]
\tl{
\pada{gudaṃ
\app{\lem[wit={C6,N23,Jyo}]{nirudhya}
	\rdg[wit={V15,V1}]{nibadhya}
	\rdg[wit={P15}]{nabadhya}
	\rdg[wit={V3}]{nibaddhi}
	\rdg[wit={J5,P11,Gr3a}]{niyamya}
	\rdg[wit={G4,J10}]{niṣpīḍya}} gulphābhyāṃ}
\pada{vyutkrameṇa
\app{\lem[wit={ceteri}]{samāhitaḥ}
	\rdg[wit={V3}]{samāhitaṃ}}/}\\+}
\tl{
\pada{\app{\lem[wit={ceteri}]{kūrmāsanaṃ}
	\rdg[wit={C6,N23}]{yogāsanaṃ}} bhaved eta}%
\pada{\app{\lem[wit={ceteri},alt={iti}]{\skm{d }iti}
	\rdg[wit={C6}]{sarve}} yogavido viduḥ//}\\!}
\end{tlg}


\newpage
\begin{tlg}[hp01_023]
\tl{
\pada{padmāsanaṃ
\app{\lem[wit={G4,C6,V3,N23,V19}]{su}% +P11
	\rdg[wit={C7,P15,V15,V1,J10,Jyo}]{tu}
	\rdg[wit={J5}]{stu}}%
\app{\lem[wit={ceteri}]{saṃsthāpya}
	\rdg[wit={Gr3a}]{saṃyojya}}}
\pada{jānūrvor antare karau/}\\+} % antaraḥ? P15
\tl{
\pada{niveśya bhūmau saṃsthāpya}
\pada{\app{\lem[wit={ceteri}]{vyomasthaḥ}
	\rdg[wit={C6,N23,Jyo}]{vyomasthaṃ}} % +P11
\app{\lem[wit={N23,V19,P15,V15,Jyo}]{kukkuṭā}
	\rdg[wit={V3,V1,J10}]{kurk(k)uṭā}
	\rdg[wit={C6}]{kukuṭā}}sanam//}
	\anm{1.23c--29d lost \getsiglum{C7}}\\!}
\end{tlg}

%\newpage
%1.24
\begin{tlg}[hp01_024]
\tl{
\pada{\app{\lem[wit={N23,V19,P15,V15,Jyo}]{kukkuṭā}
	\rdg[wit={V3,V1,J10}]{kurk(k)uṭā}
	\rdg[wit={C6}]{kukuṭā}}sana% °sanaṃ V3 
\app{\lem[wit={C6,V3,N23,P15,V15,Jyo}]{bandhastho} % sthau N23? scho V15
	\rdg[wit={V19}]{bandhasthe}% stho C1
	\rdg[wit={J10}]{madhyastho}
	\rdg[wit={V1}]{vat kṛtvā}}}\marmas
\pada{dorbhyāṃ % ddorbhyāṃ J10ac
\app{\lem[wit={ceteri}]{saṃbadhya}
	\rdg[wit={N23}]{saṃveṣṭa}
	\rdg[wit={C6}]{saṃhṛtya}}
\app{\lem[wit={ceteri}]{kandharām}
	\rdg[wit={V3}]{kandharam}
	\rdg[wit={P15}]{kandaraṃ}}/}\\+}
\tl{
\pada{\app{\lem[wit={J5,P15,V15,V1,J10}]{śete}
	\rdg[wit={V3}]{śene}
	\rdg[wit={P11,N23}]{sthitaḥ}% ##
	\rdg[wit={G4}]{sthite}
	\rdg[wit={C6}]{sthitvā}
	\rdg[wit={V19,Jyo}]{bhavet}}
\app{\lem[wit={ceteri},alt={kūrmavad}]{kūrmava\skp{d}}
	\rdg[wit={V3}]{kūrmarad}}%
\app{\lem[wit={J5,V3,N23,J10,Jyo},alt={uttāna}]{\skm{d }uttāna}
	\rdg[wit={G4,C6,V19,P15,V15,V1}]{uttānam}}}
\pada{etad uttāna\app{\lem[wit={ceteri}]{kūrmakam}
	\rdg[wit={P15}]{pūrvakaṃ}}//}\\!}
\end{tlg}


%1.25
\begin{tlg}[hp01_025]
\tl{
\pada{pādāṅguṣṭhau % ṣṭho N23
\app{\lem[wit={ceteri}]{tu}
	\rdg[wit={V19}]{ca}}
\app{\lem[wit={ceteri}]{pāṇibhyāṃ}
	\rdg[wit={V1,J10}]{bāhubhyāṃ}}}
\pada{gṛhītvā % gṛhitvā V3
\app{\lem[wit={ceteri}]{śravaṇāvadhi}
	\rdg[wit={C6}]{śravaṇāvadhiḥ}
	\rdg[wit={V1}]{śravaṇāvadhiṃ}
	\rdg[wit={V3}]{śravaṇavidhi}}/}\\+}
\tl{
\pada{dhanurā\app{\lem[resp=emend,postwit=\texteng{(=\,\getsiglum{N19})}]{karṣaṇaṃ kṛtvā} % = L1,N19
	\rdg[wit={V19}]{karkhaṇa kṛtvā}
	\rdg[wit={G4}]{karṣaṇaṃ kṛṣṭaṃ}
	\rdg[wit={V1}]{karṣaṇaḥ kṛṣṭaṃ}
	\rdg[wit={J5,C6,P15,V15}]{karṣaṇākṛṣṭaṃ}% +G11
	\rdg[wit={P11,V3,N23}]{karṣaṇāt kṛṣṭaṃ} % kaṣṭaṃ V3; originally ākarṣavat?
	\rdg[wit={J10,Jyo}]{karṣaṇaṃ kuryād}
	}}
\pada{dhanurāsana%
\app{\lem[wit={C6,V3,N23,V1,J10,Jyo},alt={ucyate}]{\skm{m }ucyate}
	\rdg[wit={V19,P15,V15}]{īritam}}//}\\!}
\end{tlg}

%\newpage
%1.26
\begin{tlg}[hp01_026]
\tl{
\pada{vāmorumūlārpita\app{\lem[wit={ceteri}]{dakṣapādaṃ} % pāda C6
	\rdg[wit={J5,G4},post=\texteng{(°pādau \getsiglum{J5})}]{dakṣapādo}}}\\+}
\tl{
\pada{\app{\lem[wit={N23,V1,J10,Jyo},alt={jānor}]{jāno\skp{r}} % +C1
	\rdg[wit={V3}]{jānaur}
	\rdg[wit={C6,V19,P15,V15}]{jānvor}}%
\app{\lem[wit={ceteri},alt={bahirveṣṭita}]{\skm{r }bahirveṣṭita}
	\rdg[wit={N23ac}]{bahiṣṭhita}
	\rdg[wit={N23pc}]{bahisaṃṣṭhita}}%
\app{\lem[wit={J5,C6,V19}]{dakṣadoṣṇā}
	\rdg[wit={N23}]{dakṣadorbhyāṃ}
	\rdg[wit={V3,V1,J10,Jyo}]{vāmapādam}
	\rdg[wit={G4,V15}]{vāmadoṣṇā}% °doṣā? G4
	\rdg[wit={P15}]{vāmadoṣṇi}}/}\\+}
\tl{
\pada{pragṛhya tiṣṭhet
	pari\app{\lem[wit={ceteri}]{vartitāṅgaḥ}% °āṃmaḥ P15
	\rdg[wit={V19}]{marditāṅgaḥ}}}\\+}
\tl{
\pada{śrī\app{\lem[wit={ceteri}]{matsyanātho}
	\rdg[wit={V3}]{matsyadinātho}}ditam āsanaṃ syāt//}\\!}
\end{tlg}


%1.27
\begin{tlg}[hp01_027]
\tl{
\pada{matsyendra\app{\lem[wit={ceteri}]{pīṭhaṃ}% pīṭho C6
	\rdg[wit={V1}]{vīraṃ}
	\rdg[wit={P15}]{vīra}}
\app{\lem[wit={ceteri}]{jaṭhara}%
	\rdg[wit={V3}]{jvalana}
	\rdg[wit={P15}]{vīra}}% again!
\app{\lem[wit={J10}]{pravṛddha}
	\rdg[wit={N23}]{pravuddhaṃ} % prabuddhaṃ L2
	\rdg[wit={V19}]{pravuddhau}
	\rdg[wit={C6}]{prabodhaṃ}
	\rdg[wit={P15,V15,V1}]{pracaṇḍaṃ}
	\rdg[wit={J5,G4,P11}]{pracaṇḍa}
	\rdg[wit={V3}]{pradiptaṃ}
	\rdg[wit={Jyo}]{pradīptiṃ}}}\marma-\\+}
\tl{
\pada{\app{\lem[wit={C6,V3,N23,V19,J10,Jyo}]{pracaṇḍa}
	\rdg[wit={J5,P15}]{picaṇḍa}
	\rdg[wit={P11}]{vicaṃḍa}
	\rdg[wit={V1}]{viccaṇḍa}
	\rdg[wit={V15},postwit=\texteng{(inserts {khaṃḍaḷa} between {ruṅmaṇḍala} and {khaṇḍanāstram} instead)},alt={\om}]{\skp{\om}}}%
\app{\lem[wit={C6,V3,V19,V15,J10,Jyo},alt={ruṅ-/rugmaṇḍala}]{ruṅmaṇḍala}
	\rdg[wit={V1}]{rūrmaṇḍala}
	\rdg[wit={N23}]{rugmaṃḍana}
	\rdg[wit={P15}]{ruk(!)māṃḍana}}%
\app{\lem[wit={ceteri}]{khaṇḍanāstram}% +C1
	\rdg[wit={N23}]{khaṇḍanāmaṃ}
	\rdg[wit={V19}]{khaṇḍalāsyam}}/}\\+}
\tl{
\pada{abhyāsataḥ kuṇḍalinīprabodhaṃ}\\+} % ābhy° N23, °bodha N23
\tl{
\pada{\app{\lem[wit={ceteri}]{daṇḍa}
	\rdg[wit={J10ac,Jyo}]{candra}% +C1
	\rdg[wit={V15}]{kāya}}\marma%
\app{\lem[wit={ceteri}]{sthiratvaṃ}
	\rdg[wit={V19}]{sthitatvaṃ}}
\app{\lem[wit={ceteri}]{ca dadāti}
	\rdg[wit={N23ac}]{dadāti}% +J5; ca karoti G4
	\rdg[wit={N23pc,V19}]{pradadāti}} puṃsām//}\\!}
\end{tlg}


\newpage
%1.28
\begin{tlg}[hp01_028]
\tl{
\pada{prasārya pādau bhuvi daṇḍarūpau}\\+}
\tl{
\pada{\app{\lem[wit={ceteri}]{dorbhyāṃ}
	\rdg[wit={V3,J10}]{dvābhyāṃ}}
\app{\lem[wit={C6,N23pc,P15,V15,V1,Jyo}]{padāgra} % °graṃ V15
	\rdg[wit={N23ac}]{dāgra}
	\rdg[wit={V3,J10}]{karābhyāṃ}
	\rdg[wit={V19}]{ca pādaṃ}}% pāda V19 
	dvitayaṃ\marmas % dvitiyaṃ V3, dvitīyaṃ C1
	gṛhītvā/}\\+} % gṛhitvā V3
\tl{
\pada{jānūparinyastalalāṭa% nyasya C6
\app{\lem[wit={ceteri}]{deśo} % +P11; deso N23
	\rdg[wit={C6,V3}]{deśe}}}\\+} % dese V3; deśaḥ C6pc
\tl{
\pada{\app{\lem[wit={C6,V3,N23,P15,V1,Jyo},alt={vased}]{vase\skp{d}}
	\rdg[wit={V19}]{'bhyased}
	\rdg[wit={V15}]{bhaved}
	\rdg[wit={J10},alt={d (two syllables omitted)}]{d}}d idaṃ
	paścimatāna% tāṇa P15,V1
\app{\lem[wit={ceteri},alt={āhuḥ}]{\skm{m }āhuḥ} % āhu V3
	\rdg[wit={V15,V1}]{baṃdhaḥ}
	\rdg[wit={P15}]{baṃdhaṃ}}//}\\!}
\end{tlg}

%\newpage
%1.29
\begin{tlg}[hp01_029]
\tl{
\pada{iti paścimatāna% tāṇa P15,V1; tāyām N3
\app{\lem[wit={N3,C6,V3,V15,V1,Jyo},alt={āsanāgryaṃ}]{\skm{m }āsanāgryaṃ}% +C1
	\rdg[wit={J10}]{āsanāśāgryaṃ}
	\rdg[wit={V19,P15}]{āsanākhyaṃ}
	\rdg[wit={N23}]{āyanaṃ}}}\\+}
\tl{
\pada{pavanaṃ
\app{\lem[wit={ceteri}]{paścima}
	\rdg[wit={V19}]{paścimā}}%
\app{\lem[wit={ceteri}]{vāhinaṃ}
	\rdg[wit={N3,V3,P15}]{vāhanaṃ}} % +C1
	karoti/}\\+}
\tl{
\pada{udayaṃ
\app{\lem[wit={ceteri}]{jaṭharānalasya}
	\rdg[wit={V19}]{jaṭharānilasya}} kuryā-}\\+}
\tl{
\pada{d udare
\app{\lem[wit={ceteri},alt={kārśyam}]{kārśya\skp{m}}
	\rdg[wit={N23}]{kāryam}
	\rdg[wit={V3}]{kṛśyam}}%
\app{\lem[wit={N3,C6,P15,V15,V1pc,Jyo},alt={arogatāṃ}]{\skm{m }arogatāṃ}% °tā C6
	\rdg[wit={N23}]{alogatāṃ}
	\rdg[wit={V19,V1ac,J10}]{arogitāṃ}
	\rdg[wit={V3}]{arogyatāṃ}}
ca puṃsām//}\\!}
\end{tlg}


%1.30

\begin{tlg}[hp01_030]
\tl{
\pada{dharām avaṣṭabhya
\app{\lem[wit={C6,V3,N23,V1,J10,Jyo}]{karadvayena} % dvayaina N23
	\rdg[wit={N3,J5,V15}]{karadvayābhyāṃ}
	\rdg[wit={G4,P15}]{karasthalābhyāṃ}
	\rdg[wit={C7}]{puraḥ karābhyāṃ} %+C1
	\rdg[wit={V19}]{punaḥ karābhyāṃ}}}\\+} % mix of two recensions?
\tl{
\pada{ta\app{\lem[wit={C6,P15,V15,V1,J10,Jyo},alt={kūrpara}]{\skm{t }kūrpara}
	\rdg[wit={N3,V3,N23}]{kurpara}
	\rdg[wit={Gr3a}]{kurpare}}%
	sthāpitanābhi%
\app{\lem[wit={ceteri}]{pārśvaḥ}
	\rdg[wit={V3,Gr3a}]{pārśve}}/}\\+}
\tl{
\pada{\app{\lem[wit={ceteri}]{uccāsano}
	\rdg[wit={N3}]{uccāsanā}
	\rdg[wit={C7}]{uccāsane} %+C1
	\rdg[wit={V19}]{taccāsanaṃ}}
daṇḍava\app{\lem[wit={ceteri},alt={utthitaḥ khe}]{\skm{d }utthitaḥ khe} % ḥ om. N23ac; +P11,J5(ṣe)
	\rdg[wit={N3}]{utthitaś cet}
	\rdg[wit={C6}]{ucchritaś ca}
	\rdg[wit={P15}]{utthitaḥ khaṃ}
	\rdg[wit={V3}]{utthitasya}}}\\+} % uthitasya V3
\tl{
\pada{\app{\lem[wit={C6,N23,Gr3a,P15,V15,J10},alt={māyūram}]{māyūra\skp{m}}
	\rdg[wit={N3,V3,V1,Jyo}]{mayūram}}m etat pravadanti
\app{\lem[wit={ceteri}]{pīṭham}
	\rdg[wit={Gr3a}]{santaḥ}}//}\\!}
\end{tlg}

%\newpage
%1.31
\begin{tlg}[hp01_031]
\tl{
\pada{harati sakala%
\app{\lem[wit={ceteri}]{rogān āśu}
	\rdg[wit={P15}]{rogān śvāsa}
	\rdg[wit={J10}]{doṣān āśu}
	\rdg[wit={V3}]{doṣān asu}}
\app{\lem[wit={ceteri}]{gulmo}
	\rdg[wit={N23}]{gulpho}
	\rdg[wit={N3}]{gulphau}}da%
\app{\lem[wit={ceteri},alt={ādīn}]{\skm{r}ādī\skp{n}}
	\rdg[wit={V3}]{ādin}}}-\\+}
\tl{
\pada{\app{\lem[wit={Gr3a,P15,V1,J10,Jyo},alt={abhibhavati ca}]{\skm{n }abhibhavati ca} % na bhavati bhava P11; N12 richtig
	\rdg[wit={N3,V15}]{abhibhavati}
	\rdg[wit={N23}]{abhavati ca}
	\rdg[wit={V3}]{abhavati}
	\rdg[wit={C6}]{na hi bhavati ca}}
	doṣān āsanaṃ śrīmayūram/}\\+} % doṣām P11, doṣam C6
\tl{
\pada{bahukadaśanabhuktaṃ
\app{\lem[wit={ceteri}]{bhasma}
	\rdg[wit={V19}]{tac ca}}
	kuryā\app{\lem[wit={ceteri}, alt={aśeṣaṃ}]{\skm{d }aśeṣaṃ}
	\rdg[wit={V1}]{aśeṣo}
	\rdg[wit={P15}]{iśutraṃ}
	\rdg[wit={C6},alt={\om}]{\skp{\om}}}}\\+}
\tl{
\pada{janayati
\app{\lem[wit={ceteri}]{jaṭharāgniṃ}
	\rdg[wit={P15}]{jaṭharāgraṃ}
	\rdg[wit={V3}]{vaḍavāgniṃ}}
\app{\lem[wit={ceteri},alt={jārayet}]{jāraye\skp{t}}
	\rdg[wit={J10}]{jvālayet}}t kālakūṭam//}\\!} % kāra N23ac
\end{tlg}

\newpage
%1.32
\begin{tlg}[hp01_032]
\tl{
\pada{\app{\lem[wit={ceteri}]{uttānaṃ}
	\rdg[wit={N3,C6,N23,J10}]{uttāna}}
	śavavad bhūmau}
\pada{\app{\lem[wit={N3,C6,V3,N23,P15,V1}]{śayanaṃ tu śavāsanam} % savā° N23
	\rdg[wit={V15,J10}]{śayanaṃ ca śavāsanam}
	\rdg[wit={Jyo}]{śayanaṃ tat śavāsanam}
	\rdg[wit={Gr3a}]{śavāsanam idaṃ smṛtam}}/}\\+}
\tl{
\pada{\app{\lem[wit={C6,N23,P15,V1,J10}]{sarvāsana}
	\rdg[wit={V3}]{savāsana}
	\rdg[wit={N3,Gr3a,V15,Jyo}]{śavāsanaṃ}}%
\app{\lem[wit={ceteri}]{śrānti}
	\rdg[wit={V15}]{śrama}}haraṃ}
\pada{cittaviśrānti%
\app{\lem[wit={N3,C6,V3,N23,V19,P15}]{sādhanam}
	\rdg[wit={C7,V15,V1,J10,Jyo}]{kārakam}}//}\\!}
\end{tlg}

%\newpage
%1.33
\begin{tlg}[hp01_033]
\tl{
\pada{caturāśītyāsanāni}
\pada{\app{\lem[wit={ceteri}]{śivena}
	\rdg[wit={P15}]{sarvāṇi}
	\rdg[wit={C6}]{kathitāni}}
\app{\lem[wit={ceteri}]{kathitāni}
	\rdg[wit={V3}]{kathitāna}
	\rdg[wit={C6}]{śivena}}
\app{\lem[wit={N3,G4,C6,V3,P15,V15,V1,J10}]{tu}
	\rdg[wit={N23,V19,C7}]{vai}
	\rdg[wit={J5,Jyo}]{ca}}/}\\+}
\tl{
\pada{tebhyaś catuṣkam ādāya} % ca(tuṣka)m P15 (two akṣaras as a gap)
\pada{sārabhūtaṃ bravīmy aham//}\\!}
\end{tlg}

%\newpage
%1.34
\begin{tlg}[hp01_034]
\tl{
\pada{siddhaṃ
\app{\lem[wit={ceteri}]{padmaṃ}
	\rdg[wit={V19}]{bhadraṃ}}
\app{\lem[wit={ceteri}]{tathā}
	\rdg[wit={V3}]{yathā}}
\app{\lem[wit={C6,V3,N23,V1,J10,Jyo}]{siṃhaṃ} % sīhaṃ N3
	\rdg[wit={P15,V15}]{saiṃhaṃ}
	\rdg[wit={V19}]{padmaṃ}
	\rdg[wit={C7}]{bhadraṃ}}} %+C1
\pada{\app{\lem[wit={ceteri}]{bhadraṃ}
	\rdg[wit={V19,C7}]{siṃhaṃ}}
\app{\lem[wit={C6,N23,V19,C7,J10,Jyo}]{ceti}
	\rdg[wit={N3,V1}]{caiva}
	\rdg[wit={V3,P15}]{caitac}
	\rdg[wit={V15}]{cātha}}
\app{\lem[wit={ceteri}]{catuṣṭayam}
	\rdg[wit={V15}]{catuṣkakaṃ}}/}\\+}
\tl{
\pada{śreṣṭhaṃ
\app{\lem[wit={N3,C6,V3,N23}]{tatrāpi ca sakhe}
	\rdg[wit={V1,J10,Jyo}]{tatrāpi ca sukhe}
	\rdg[wit={V15}]{tatrāpi ca sukhaṃ}
	\rdg[wit={K3}]{tathāpi ca sakhe}
	\rdg[wit={V19,C7}]{tathāpi bhadraṃ [ca]}
	\rdg[wit={P15}]{tatra viśeṣeṇa}}}
\pada{\app{\lem[wit={N3,C6,V3,N23}]{tiṣṭha}
	\rdg[wit={V15}]{tiṣṭhat}
	\rdg[wit={Gr3a,V1,J10,Jyo}]{tiṣṭhet}
	\rdg[wit={P15}]{śreṣṭhaṃ}}
\app{\lem[wit={ceteri}]{siddhāsane}
	\rdg[wit={V19}]{siṃhāsane}
	\rdg[wit={P15}]{padmāsanaṃ}} sadā//} 
%	\anm{\getsiglum{K3} begins}
	\\!}
\end{tlg}


%\begin{alttlg}[hp01_035ava]
\begin{ava}[hp01_035]
\app{\lem[wit={N23}]{atha}
	\rdg[wit={N3,C6,J10}]{tatra}}
\app{\lem[wit={N3,C6,N23,J10}]{siddhāsanam}}/
	\sgwit{N3,N23,J10} %(not Gr3a,V15,V1,Jyo)
\end{ava}
%\end{alttlg}


%1.35
\begin{tlg}[hp01_035]
\tl{
\pada{yoni\app{\lem[wit={ceteri}]{sthānaka}% +G4
	\rdg[wit={N3,C6,N23,Gr3a}]{dvāraka}}% +J5 ##
\app{\lem[wit={ceteri},alt={°m aṅghrimūla}]{\skm{°}m aṅghrimūla}
	\rdg[wit={V19}]{°m aṅghrimūlā}
	\rdg[wit={J10}]{mūlāṅghri}}%
\app{\lem[wit={ceteri}]{ghaṭitaṃ}
	\rdg[wit={P15}]{puṭakaṃ}} kṛtvā
\app{\lem[wit={ceteri}]{dṛḍhaṃ}
	\rdg[wit={P15,V15}]{dhruvaṃ}} vinyase}- \\+}
\tl{
\pada{\app{\lem[wit={ceteri},alt={meḍhre}]{\skm{n }meḍhre}
	\rdg[wit={N23}]{medhre}
	\rdg[wit={V19}]{madhye}}
	pādam athaika%
\app{\lem[wit={V1},alt={āsyahṛdaye}]{\skm{m }āsyahṛdaye}
	\rdg[wit={N3,N23,P15}]{ekahṛdaye}
	\rdg[wit={J5,P11,Gr3a,V15,J10,Jyo}]{eva hṛdaye}% ##
	\rdg[wit={C6,V3}]{eva niyataṃ}}\marmas
\app{\lem[wit={ceteri}]{dhṛtvā}
	\rdg[wit={C6,V3,Jyo}]{kṛtvā}}
\app{\lem[wit={ceteri}]{samaṃ}
	\rdg[wit={Jyo}]{hanuṃ}}
\app{\lem[wit={ceteri}]{vigraham}
	\rdg[wit={Jyo}]{susthiraṃ}}/}\\+}
\tl{
\pada{\app{\lem[wit={ceteri}]{sthāṇuḥ}
	\rdg[wit={V19}]{sthāṇu}}
	saṃyamitendriyo'caladṛśā % J10 writes avagraha!
\app{\lem[wit={N3,C6,V3,C7,V1,J10}]{paśyan}
	\rdg[wit={N23}]{paśyad}
	\rdg[wit={G4,V19,K3,P15,V15,Jyo}]{paśyed}} % ##
	bhruvor antaraṃ} \\+}
\tl{
\pada{\app{\lem[wit={Gr3a,P15,V15,V1},alt={etan}]{eta\skp{n}}
	\rdg[wit={N3,C6,V3,N23,J10}]{caitan}
	\rdg[wit={Jyo}]{hy etan}}n%
	mokṣakapāṭabheda%
\app{\lem[wit={ceteri}]{janakaṃ}
	\rdg[wit={K3,P15}]{jananaṃ}} siddhāsanaṃ procyate//}\\!}
\end{tlg}


\begin{ava}[hp01_036]
\app{\lem[wit={N3,C6,N23,J10,Jyo}]{matāntare tu}
	\rdg[wit={V3,P15,V15,V1}]{matāntare}
	\rdg[wit={V19}]{matsyendraḥ| matāntaraṃ tu}
	\rdg[wit={K3,C7}]{etan matsyendramataṃ matāntare tu}}/
\end{ava}

%1.36
\begin{tlg}[hp01_036]
\tl{
\pada{meḍhrād upari % meṃḍhrād P15, meḍhrādhaḥpari N23
\app{\lem[wit={ceteri}]{vinyasya}% °syaṃ V19
	\rdg[wit={N3,V3}]{nikṣipya}
	\rdg[wit={J10}]{niḥkṣipya}}}
\pada{\app{\lem[wit={N3,C6,V3,N23,P15,J10}]{savya}
	\rdg[wit={V15,Jyo}]{savyaṃ}
	\rdg[wit={V1},post={\unm}]{savyaṃ tu}
	\rdg[wit={Gr3a}]{vāma}}gulphaṃ
	tathopari/}\\+}  % tato° N3; °riḥ J10
\tl{
\pada{gulphāntaraṃ
\app{\lem[wit={ceteri}]{ca}
	\rdg[wit={V1}]{tu}}
\app{\lem[wit={ceteri}]{nikṣipya}
	\rdg[wit={J10}]{niḥkṣipya}
	\rdg[wit={V19}]{vinyasya}}}
\pada{siddhāsanam idaṃ bhavet//}\\!}
\end{tlg}

\newpage
\begin{postmula}[hp01_036]
pūrvoktam eva
\app{\lem[wit={N3},post=\texteng{\textless\  matsamaṃtaṃ}]{matsaṃmatam} % cf. N19,P6
	\rdg[wit={P15,V15}]{matsyamataṃ}
	\rdg[wit={C6,N23}]{matsyendramatam}
	\rdg[wit={P11}]{saṃmataṃ}}/
	\sgwit{N3,C6,P11,N23,P15,V15}
\end{postmula}

%\newpage
%1.37
\begin{tlg}[hp01_037]
\tl{
\pada{\app{\lem[wit={ceteri},alt={etat}]{eta\skp{t}}
	\rdg[wit={K3,C7}]{iti}
	\rdg[wit={V19}]{kecit}}t siddhāsanaṃ prāhu}%
\pada{\app{\lem[wit={ceteri}]{\skm{r }anye}
	\rdg[wit={N3}]{anyathā}} vajrāsanaṃ viduḥ/}\\+}
\tl{
\pada{\app{\lem[wit={ceteri}]{muktāsanaṃ}
	\rdg[wit={V19}]{muktvāsanaṃ}
	\rdg[wit={N3}]{guptāsanaṃ}}
\app{\lem[wit={ceteri}]{vadanty eke}% vadaṃteke N3
	\rdg[wit={V19,K3}]{vadanty anye}
	\rdg[wit={C7}]{vidur anye}}}
\pada{prāhu\app{\lem[wit={ceteri},alt={guptāsanaṃ}]{\skm{r }guptāsanaṃ}
	\rdg[wit={N3}]{muktāsanaṃ}} pare//}\\!}% from pare on lost N3
\end{tlg}

%\newpage
%1.38
\begin{tlg}[hp01_038]
\tl{
\pada{\app{\lem[wit={ceteri}]{yameṣv iva}
	\rdg[wit={V1,J10}]{yameṣv eva}
	\rdg[wit={C6}]{yameṣu ca}}
\app{\lem[wit={ceteri},alt={mitāhāram}]{mitāhāra\skp{m}}
	\rdg[wit={V3,V1}]{mitāhāra}
	\rdg[wit={J10}]{mitāhāraḥ}}}%
\pada{\app{\lem[wit={C6,V19,K3,V15,Jyo},alt={ahiṃsāṃ}]{\skm{m }ahiṃsāṃ}
	\rdg[wit={N23}]{nahiṃsāṃ}
	\rdg[wit={V3,C7,P15,V1,J10}]{ahiṃsā}}
\app{\lem[wit={ceteri}]{niyameṣv iva}
	\rdg[wit={C6,V1}]{niyameṣu ca}}/}\marma\\+}
\tl{
\pada{mukhyaṃ sarvāsane%
\app{\lem[wit={ceteri},alt={ekaṃ}]{ṣv ekaṃ}
	\rdg[wit={V19}]{evaṃ}
	\rdg[wit={V15}]{eke}}}
\pada{\app{\lem[wit={ceteri}]{siddhāḥ siddhāsanaṃ}% siddhā V1ac,V3
	\rdg[wit={C6}]{etat siddhāsanaṃ}
	\rdg[wit={Gr3a}]{siddhāsanam idaṃ}}
	viduḥ//} \anm{1.38--46 lost \getsiglum{N3}}\\!}
\end{tlg}

%1.39
\begin{tlg}[hp01_039]
\tl{
\pada{caturāśītipīṭheṣu}
\pada{\app{\lem[wit={ceteri}]{siddham eva sadābhyaset}
	\rdg[wit={V19,C7}]{siddhāsanaṃ sadābhyaset}
	\rdg[wit={K3}]{abhyaset siddhāsanaṃ sadā}}/}\\+}
\tl{
\pada{\app{\lem[wit={ceteri}]{dvāsaptati}
	\rdg[wit={V15}]{dvisaptati}}% +C8
\app{\lem[wit={ceteri}]{sahasreṣu}
	\rdg[wit={V15}]{sahasrāsu}
	\rdg[wit={Jyo}]{sahasrāṇāṃ}}} % V3 sahaśreṣu
\pada{\app{\lem[wit={C6,Gr3a,V15,J10},alt={suṣumṇām}]{suṣumṇā\skp{m}}% suṣunmām J10
	\rdg[wit={N23}]{sukhumṇām}
	\rdg[wit={P15,V1}]{suṣumṇā}
	\rdg[wit={V3,Jyo}]{nāḍīnāṃ}}%
\app{\lem[wit={C6,N23,Gr3a,P15,V15},alt={iva nāḍiṣu}]{\skm{m }iva nāḍiṣu}
	\rdg[wit={V1}]{iva nāḍikā}
	\rdg[wit={J10}]{eva nāḍiṣu}
	\rdg[wit={V3,Jyo}]{malaśodhanam}% sodhanaṃ V3
	}\marma//}\\!}
\end{tlg}


%1.40
\begin{tlg}[hp01_040]
\tl{
\pada{\app{\lem[wit={ceteri}]{ātmadhyāyī}
	\rdg[wit={V15}]{ātmādhyāyī}}
\app{\lem[wit={ceteri}]{mitāhārī}
	\rdg[wit={Gr3a,P15}]{mitāhāro}}}
\pada{yāvad dvādaśavatsaram/}\\+} % yāvadvā° V15, yatavādadvā° C6
\tl{
\pada{sadā siddhāsanā\app{\lem[wit={ceteri},alt={°bhyāsād}]{bhyāsā\skp{d}}
	\rdg[wit={P15}]{°bhyāsā}
	\rdg[wit={V19}]{°bhyānād}}}%
\pada{\app{\lem[wit={ceteri},alt={yogī}]{\skm{d }yogī}
	\rdg[wit={V15}]{yoga}}
\app{\lem[wit={ceteri}]{niṣpattim āpnuyāt}% +G4
	\rdg[wit={J10},post={\unm}]{niṣpattim avāpnuyāt}
	\rdg[wit={V3}]{siddhim avāpnuyāt}}/}\\+} % +J5
\tl{
\pada{\app{\lem[wit={C6,N23}]{śramadair bahubhiḥ}
	\rdg [wit={V1}]{śramādau bahubhiḥ}
	\rdg[wit={P15}]{samastair bahubhiḥ}
	\rdg[wit={V15}]{samastabahubhiḥ}
	\rdg[wit={Gr3a}]{śramadairghyādibhiḥ}
	\rdg[wit={V3,J10}]{kim ādyair bahubhiḥ}
	\rdg[wit={Jyo}]{kim anyair bahubhiḥ}}
	pīṭhaiḥ}
\pada{\app{\lem[wit={C6,N23,Gr3a,V15},alt={kiṃ syāt}]{kiṃ syā\skp{t}}
	\rdg[wit={V3,V1,J10}]{sadā}
	\rdg[wit={P15}]{yadā}
	\rdg[wit={Jyo}]{siddhe}}t siddhāsane
\app{\lem[wit={ceteri}]{sati}
	\rdg[wit={V3}]{satya}
	\rdg[wit={C6,N23}]{sthite}}\marma//}\\!}
\end{tlg}

%\newpage

%1.41
\begin{tlg}[hp01_041]
\tl{
\pada{\app{\lem[wit={ceteri}]{prāṇānile}
	\rdg[wit={V3}]{prāṇānale}}
\app{\lem[wit={N23,K3,P15,V1,J10}]{sāvadhānaṃ}
	\rdg[wit={C7},alt={°na}]{sāvadhāna}
	\rdg[wit={C6,V3,V19,Jyo},alt={°ne}]{sāvadhāne} % +C1
	\rdg[wit={V15},alt={°no}]{sāvadhāno}}}
\pada{\app{\lem[wit={ceteri}]{baddhe}
	\rdg[wit={V3}]{baṃdhe}
	\rdg[wit={P15}]{baṃdhaḥ}
	\rdg[wit={N23}]{baddhvai}}
\app{\lem[wit={Gr3a,V15,V1,J10,Jyo}]{kevalakumbhake}
	\rdg[wit={J5,G4,P11,V3,P15}]{kevalakumbhakaḥ}
	\rdg[wit={C6,N23}]{kevalakumbhataḥ}}/}\\+}
\tl{
\pada{\app{\lem[wit={ceteri}]{utpadyate}
	\rdg[wit={C6,V3,V1}]{utpadyaṃte}}
	nirāyāsāt} % °yāsā K3, nirāyāt C6
\pada{\app{\lem[wit={ceteri}]{svayam evonmanī}
	\rdg[wit={C6},alt={\om},post=\texteng{(eye-skip)}]{\skp{\om}}} 
	\app{\lem[wit={ceteri}]{yathā} % evvonmanīṃ N23
	\rdg[wit={V1}]{tathā}
	\rdg[wit={Jyo}]{kalā}
	\rdg[wit={C6},alt={\om}]{\skp{\om}}}//} % by eye-skip
	\anm{1.41c--47d lost \getsiglum{C7}}\\!}
\end{tlg}

\newpage
%1.42
\begin{tlg}[hp01_042]
\tl{
\pada{\app{\lem[wit={J5,G4,P11,V3,V1,J10,Jyo}]{tathaika}
	\rdg[wit={N23,Gr3a,V15}]{athaika}}sminn eva % ##
\app{\lem[wit={J5,G4,P11,V3,Gr3a}]{dṛḍhaṃ}
	\rdg[wit={V15,V1,J10,Jyo}]{dṛḍhe}
	\rdg[wit={N23}]{dṛdhe}}}
\pada{\app{\lem[wit={ceteri}]{baddhe}
	\rdg[wit={N23}]{baddha}}
\app{\lem[wit={ceteri}]{siddhāsane}
	\rdg[wit={V19}]{siṃhāsane}}
\app{\lem[wit={ceteri}]{sadā}
	\rdg[wit={N23}]{tadā}
	\rdg[wit={Jyo}]{sati}}/}
	\lineom{ab}{C6,P15}\\+}         %  V1 3r
\tl{
\pada{bandhatrayam anāyāsāt}
\pada{svayam evopajāyate//} \lineom{c}{C6}\\!}
\end{tlg}

%\newpage
%1.43
\begin{tlg}[hp01_043]
\tl{
\pada{\app{\lem[wit={C6,N23,Gr3a,V15,Jyo}]{nāsanaṃ siddhasadṛśaṃ}
	\rdg[wit={P15}]{nāsanaṃ siddhasadanaṃ} % siṃdhu° P15ac
	\rdg[wit={V1,J10}]{na cāsanaṃ siddhasamaṃ}}}
\pada{na
\app{\lem[wit={J5,C6,N23,Gr3a,P15,V15,Jyo}]{kumbhaḥ kevalopamaḥ}
	\rdg[wit={V1,J10}]{kumbhasadṛśo'nilaḥ}}/}\\+}
% J10 has vs no. 43 here, but counts atha padmāsanam as 45.
\tl{
\pada{na khecarīsamā mudrā}
\pada{na
\app{\lem[wit={ceteri}]{nāda}
	\rdg[wit={V1}]{nādaḥ}}sadṛśo layaḥ//} \NotIn{V3}%
\myfn{In \getsiglum{V3} this verse is omitted here and found at the beginning of the Khecarī section in Chapter 3:\\
\devnote{nāsanaṃ siddhasadṛśaṃ na kuṃbha kevalokanaṃ/ na khecarīsamā mudrā na nādasadṛśo layaḥ//}}\\!}
\end{tlg}


%\newpage
\begin{ava}[hp01_044]
\app{\lem[wit={ceteri}]{atha}
	\rdg[wit={V1}]{tathā}
	\rdg[wit={V19,P15},alt={\om}]{\skp{\om}}}
\app{\lem[wit={ceteri}]{padmāsanam}
	\rdg[wit={V19,P15},alt={\om}]{\skp{\om}}}/
\end{ava}


%1.44
\begin{tlg}[hp01_044]
\tl{
\pada{vāmorūpari
\app{\lem[wit={ceteri}]{dakṣiṇaṃ}
	\rdg[wit={V3}]{dakṣaṇaṃ}
	\rdg[wit={V19}]{vidakṣiṇaṃ}}
\app{\lem[wit={ceteri}]{ca}
	\rdg[wit={C6,N23,Gr3a}]{hi}}
	caraṇaṃ saṃsthāpya vāmaṃ % saṃsthāya C6
\app{\lem[wit={ceteri}]{tathā}
	\rdg[wit={V3}]{tato}}}\\+}
\tl{
\pada{\app{\lem[wit={J5,G4,P11,V3,P15,V1}]{yāmyo}
	\rdg[wit={C6,N23,Gr3a,V15,Jyo}]{dakṣo}
	\rdg[wit={J10}]{jānvo}}rūpari
\app{\lem[wit={ceteri}]{paścimena vidhinā}
	\rdg[wit={V3}]{tasya bandhanavidhau}}
\app{\lem[wit={ceteri}]{dhṛtvā}
	\rdg[wit={V3}]{pṛṣṭe}} karābhyāṃ dṛḍham/}\\+}
\tl{
\pada{aṅguṣṭhau hṛdaye nidhāya
	cibukaṃ nāsāgram ālokaye-}\\+} % cabukaṃ V15
\tl{
\pada{d etad vyādhi\app{\lem[wit={P15,V15,V1,J10,Jyo}]{vināśakāri}% +G4
	\rdg[wit={C6}]{vināśakāya}
	\rdg[wit={V3},post={\unm}]{vināsanaṃ}
	\rdg[wit={N23}]{vināśam āśu}
	\rdg[wit={Gr3a}]{vikāranāśa°}} % vighātakāra° J5; vivātakāri P11
\app{\lem[wit={ceteri}]{yamināṃ}
	\rdg[wit={N23}]{janakaṃ}
	\rdg[wit={Gr3a}]{°nakaraṃ}}
	padmāsanaṃ procyate//}\\!}
\end{tlg}

\begin{ava}[hp01_045]
\app{\lem[wit={J5,G4,V3,V1,Jyo}]{matāntare}
	\rdg[wit={N23}]{matāntaraṃ}
	\rdg[wit={V15}]{matāntara}
	\rdg[wit={C6,K3,P15,J10}]{matāntare tu}
	\rdg[wit={V19}]{matabhede}}/
\end{ava}

%1.45
\begin{tlg}[hp01_045]
\tl{
\pada{uttānau caraṇau kṛtvā}
\pada{\app{\lem[wit={ceteri}]{ūru}% ūrū V15,K3
	\rdg[wit={V3}]{kuru}}saṃsthau
\app{\lem[wit={ceteri}]{prayatnataḥ}
	\rdg[wit={V19}]{vidhānataḥ}}/}\\+}
\tl{
\pada{ūrumadhye  % ūrū K3,V15; madhyai N23
\app{\lem[wit={ceteri}]{tathottānau}% tathātānau C6
	\rdg[wit={V19,J10ac}]{tathauttānau}}}
\pada{pāṇī kṛtvā % pāṇiṃ C6
\app{\lem[wit={ceteri}]{tato dṛśau}
	\rdg[wit={V3}]{tato dṛśai}
	\rdg[wit={Gr3a}]{tu tādṛśau}}//}\\!}
\end{tlg}

%\newpage

%1.46
\begin{tlg}[hp01_046]
\tl{
\pada{\app{\lem[wit={ceteri}]{nāsāgre}
	\rdg[wit={V19}]{nāsagre}}
\app{\lem[wit={ceteri},alt={vinyased}]{vinyase\skp{d}}
	\rdg[wit={V3}]{vinyasya}}%
\app{\lem[wit={P11,N23,Gr3a,J10,Jyo},alt={rāja}]{\skm{d }rāja}
	\rdg[wit={P15}]{rājā}
	\rdg[wit={V15,V1}]{dṛṣṭiṃ}
	\rdg[wit={V3}]{dṛṣṭī}
	\rdg[wit={C6},alt={\om}]{\skp{\om}}}}%
\pada{danta\app{\lem[wit={P11,V3,N23,P15,V15,V1,J10}]{mūlaṃ}
	\rdg[wit={C6,Gr3a,Jyo}]{mūle}}
\app{\lem[wit={J5,G4,V3,Gr3a,V15,V1,J10}]{ca}
	\rdg[wit={C6,P11,N23,P15,Jyo}]{tu}} jihvayā/}\\+}
\tl{
\pada{\app{\lem[wit={N23,P15,J10}]{uttabhya}
	\rdg[wit={C6,P11,V3,Gr3a,V15,V1,Jyo}]{uttaṃbhya}}
	cibukaṃ % cubukaṃ V15
\app{\lem[wit={C6,P11,N23,Gr3a,P15,Jyo},alt={vakṣasy}]{vakṣa\skp{sy}} % vakṣyasy C6
	\rdg[wit={V15}]{cakṣasy}
	\rdg[wit={V1}]{vakṣaṃ}
	\rdg[wit={V3,J10}]{vakṣa}}}%
\pada{\app{\lem[wit={Gr3a},alt={āsthāpya},postwit=\texteng{(cf. DYŚ)}]{\skm{sy }āsthāpya}% +G11
	\rdg[wit={N23,Jyo}]{utthāpya}
	\rdg[wit={C6}]{utthāya}
	\rdg[wit={P15}]{utthāyot}
	\rdg[wit={V15}]{otthāpya}
	\rdg[wit={J5,P11}]{osthāpyot}
	\rdg[wit={V3,V1,J10}]{sthāpayet}}\marmas
	pavanaṃ śanaiḥ//}%
\myfn{Incomplete description. See the philological commentary.}\\!}
\end{tlg}

\newpage
%1.47
\begin{tlg}[hp01_047]
\tl{
\pada{idaṃ padmāsanaṃ
\app{\lem[wit={ceteri}]{proktaṃ}
	\rdg[wit={V19}]{praktaṃ}}}
	\pada{sarvavyādhivināśanam/}\\+}
\tl{
\pada{durlabhaṃ	yena kenāpi} % durlabha V19
\pada{\app{\lem[wit={ceteri}]{dhīmatā labhyate}
	\rdg[wit={C6}]{dhīmatāṃ labhyate}
	\rdg[wit={J7}]{dhīmatāṃ labhate} % +J17
	\rdg[wit={P15,V15}]{labhyate dhīmatā}}
	bhuvi//}\\!}
\end{tlg}


\begin{postmula}[hp01_047]
\app{\lem[wit={N3,C6,N23}]{paścād uktam eva} % evā N3
	\rdg[wit={V3,V19,C7,V1,J10}]{paścād uktaṃ}% Jyo-Ms?
	\rdg[wit={K3}]{etat paścād uktaṃ}
	\rdg[wit={P15,V15}]{idam api}}
\app{\lem[wit={N3}]{matsaṃmatam}
	\rdg[wit={V3,P15,V15,V1,J10}]{matsyamatam}
	\rdg[wit={C6,N23,Gr3a}]{matsyendramatam}}/ % machendra N23
	\NotIn{P11,Jyo} % Jyo-Mss have this?	 
\end{postmula}


%\newpage

%1.48
\begin{tlg}[hp01_048]
\tl{
\pada{\app{\lem[wit={ceteri}]{kṛtvā}
	\rdg[wit={Gr3a}]{dhṛtvā}}
	saṃpuṭitau % °to N23
\app{\lem[wit={ceteri}]{karau}
	\rdg[wit={N23},alt={\om}]{\skp{\om}}}
	dṛḍhataraṃ baddhvā % dṛḍhatarau C6
\app{\lem[wit={ceteri}]{tu}
	\rdg[wit={Gr3a}]{ca}} padmāsanaṃ}\\+}
\tl{
\pada{gāḍhaṃ vakṣasi % gāḍhāṃ N3
\app{\lem[wit={ceteri}]{saṃnidhāya}
	\rdg[wit={V1,J10}]{saṃvidhāya}
	\rdg[wit={N3}]{nidhāya}} % Stronger evidence
	cibukaṃ % cubukaṃ V15
\app{\lem[wit={ceteri}]{dhyānaṃ}
	\rdg[wit={Jyo}]{dhyāyaṃś}}
\app{\lem[wit={ceteri},alt={ca tac}]{ca ta\skp{c}}
	\rdg[wit={V1}]{tataś}}%
\app{\lem[wit={ceteri},alt={cetasi}]{\skm{c }cetasi}
	\rdg[wit={J10}]{cepsitaṃ}}\marma/}\\+}
\tl{
\pada{vāraṃ vāram apānam ūrdhvam anilaṃ
\app{\lem[resp=emend]{proccālayan}
	\rdg[wit={N3}]{proccārayaṃn}
	\rdg[wit={N23,V15}]{proccārayet}
	\rdg[wit={V3}]{procārayet}
	\rdg[wit={V1}]{protsālayan}
	\rdg[wit={C6,P15,Jyo}]{protsārayan}
	\rdg[wit={V19,C7}]{protsārayet}
	\rdg[wit={K3}]{protsāray.\,n/t}
	\rdg[wit={J10}]{prollāsayan}}
\app{\lem[wit={ceteri}]{pūritaṃ}
	\rdg[wit={V19,V1}]{pūrayan}
	\rdg[wit={K3,C7}]{pūrayet}}}\marma\\+}
\tl{
\pada{\app{\lem[wit={ceteri}]{muñcan prāṇam upaiti} % upeti N3
	\rdg[wit={V3,J10}]{muñcat prāṇam upaiti}
	\rdg[wit={Gr3a}]{prāṇaṃ muñcati yāti}}
	bodham atulaṃ śakti%
\app{\lem[wit={ceteri}]{prabhāvān naraḥ}
	\rdg[wit={J10}]{prabhāvād ataḥ}
	\rdg[wit={Gr3a}]{prabhāvodayāt}}\marma//} \NotIn{P11}\\!}
\end{tlg}


%1.49
\begin{tlg}[hp01_049]
\tl{
\pada{\app{\lem[wit={ceteri}]{padmāsana}% °sane J8
	\rdg[wit={J10,Jyo}]{padmāsane}}sthito yogī}
\pada{nāḍī\app{\lem[wit={ceteri}]{dvāreṣu} % nāḍi N23
	\rdg[wit={V3,Jyo}]{dvāreṇa}}
\app{\lem[wit={G4,V15,J10}]{pūrayan}
	\rdg[wit={N3,J5,C6,V3,N23,Gr3a,P15,V1}]{pūrayet}
	\rdg[wit={Jyo}]{pūritaṃ}}/}\\+}
\tl{
\pada{\app{\lem[wit={ceteri}]{mārutaṃ}
	\rdg[wit={P15}]{māruto}}
\app{\lem[wit={Gr3a,V15,Jyo}]{dhārayed yas tu}
	\rdg[wit={N23}]{niyataṃ yas tu} % nīyate yas tu C1
	\rdg[wit={C6,P11}]{nayate yas tu}% +P11
	\rdg[wit={V1}]{pīyate yas tu}
	\rdg[wit={V3}]{pīvyate yas tu}
	\rdg[wit={J10}]{yas tu pibati}
	\rdg[wit={N3,G4,P15}]{mriyate yas tu}
	\rdg[wit={J5}]{mārayed yas tu}}}\marmas% dhriyate yas tu?
\pada{sa mukto nātra saṃśayaḥ//}\\!}
\end{tlg}



%\newpage
\begin{ava}[hp01_050]
\app{\lem[wit={ceteri}]{atha siṃhāsanam}
	\rdg[wit={N3}]{atha sīṃhāna}
	\rdg[wit={V19}]{siṃhāsana yathā}}/
\end{ava}

%1.50
\begin{tlg}[hp01_050]
\tl{
\pada{gulphau
\app{\lem[wit={ceteri}]{ca}
	\rdg[wit={V15}]{tu}} vṛṣaṇasyādhaḥ} % vṛṣaṇ<asy>ādhaḥ P15, vṛṣa<ṇa>syādhaḥ N3
\pada{\app{\lem[wit={ceteri}]{sīvanyāḥ} % siva° N23, sīvā° N3
	\rdg[wit={P15}]{sīvinyāḥ}
	\rdg[wit={Gr3a}]{sīmanyāḥ}}
	pārśvayoḥ kṣipet/}\label{VuI50}\\+} % prā° N23, °yo N3
\tl{
\pada{\app{\lem[wit={ceteri}]{dakṣiṇe}
	\rdg[wit={V3}]{dakṣaṇe}}
	savyagulphaṃ % 'savya! K3
\app{\lem[wit={ceteri}]{tu}
	\rdg[wit={V19,K3}]{ca}}}
\pada{\app{\lem[wit={ceteri}]{dakṣagulphaṃ}
	\rdg[wit={K3}]{vāme caivā°}}
\app{\lem[wit={ceteri}]{tu}
	\rdg[wit={V19,V1}]{ca}
	\rdg[wit={K3}]{°pi}}
\app{\lem[wit={ceteri}]{savyake}
	\rdg[wit={K3}]{savyakam}
	\rdg[wit={V19}]{guhyake}}//}
	\lineom{cd}{C7} \anm{eye-skip?}\\!}
\end{tlg}

\newpage
%1.51
\begin{tlg}[hp01_051]
\tl{
\pada{hastau
\app{\lem[wit={N3,C6,V3,N23}]{ca jānvoḥ}% jāhno N3
	\rdg[wit={P15,V15,V1,Jyo}]{tu jānvoḥ}
	\rdg[wit={J10}]{tu jānunoḥ}
	\rdg[wit={V19,K3}]{jānvoś ca}}
\app{\lem[wit={ceteri}]{saṃsthāpya}
	\rdg[wit={J10}]{sthāpya}}}
\pada{\app{\lem[wit={N3,V3,V15,V1,J10,Jyo}]{svāṅgulīḥ}% °liṃḥ N3
	\rdg[wit={C6,N23,P15}]{svāṅgulī}
	\rdg[wit={V19}]{aṅgulīḥ}
	\rdg[wit={K3}]{hy aṅgulīḥ}}
\app{\lem[wit={ceteri}]{saṃprasārya}
	\rdg[wit={N23}]{yaṃ prasārmya}} ca/}\\+}
\tl{
\pada{\app{\lem[wit={ceteri}]{vyātta}
	\rdg[wit={V3}]{vyāta}
	\rdg[wit={V19}]{vyālā}}%
\app{\lem[wit={ceteri}]{vaktro} % vaktrā? N23
	\rdg[wit={P15}]{vaktrau}
	\rdg[wit={V3}]{vakro}}
\app{\lem[wit={ceteri}]{nirīkṣeta}
	\rdg[wit={V3}]{nirīkṣet}
	\rdg[wit={J10}]{nirīkṣyeta}
	\rdg[wit={N23}]{nirīkṣeya}}}
\pada{\app{\lem[wit={N3,C6,V19,K3,P15,V15,Jyo}]{nāsāgraṃ} % nāśā N3
	\rdg[wit={V3,N23,J10}]{nāsāgra}
	\rdg[wit={V1}]{nāsāgre}}
\app{\lem[wit={V19,K3,P15,V15,Jyo}]{susamāhitaḥ}
	\rdg[wit={N23}]{stusamāhitaḥ}
	\rdg[wit={N3,J5}]{tu samāhitaḥ}
	\rdg[wit={C6}]{susamāhitaṃ}
	\rdg[wit={V1,J10}]{nyastalocanaḥ}
	\rdg[wit={V3}]{nyastalocanaṃ}}\marma//} \NotIn{C7}\\!}
\end{tlg}

%\newpage
%1.52
\begin{tlg}[hp01_052]
\tl{
\pada{siṃhāsanaṃ
	bhave\app{\lem[wit={ceteri},alt={etat}]{\skm{d }eta\skp{t}}
	\rdg[wit={N23}]{evaṃ}}}%
\pada{\app{\lem[wit={ceteri},alt={pūjitaṃ}]{\skm{t }pūjitaṃ}
	\rdg[wit={V15}]{pūjītaṃ}
	\rdg[wit={J10}]{pūjita}}
\app{\lem[wit={ceteri}]{yogibhiḥ sadā}% yogabhiḥ
	\rdg[wit={V19,K3}]{munipuṅgavaiḥ}
	\rdg[wit={Jyo}]{yogipuṅgavaiḥ}}/}\\+}
\tl{
\pada{bandha\app{\lem[wit={N3,C6,V3,P15,J10,Jyo}]{tritaya} % <<ba>>ndha N23
	\rdg[wit={V1}]{tritīya}
	\rdg[wit={N23,V19,K3,V15}]{trayasya}}%
\app{\lem[wit={ceteri}]{sandhānaṃ}
	\rdg[wit={P15}]{saṃdhāyi}}}
\pada{\app{\lem[wit={ceteri}]{kurute}
	\rdg[wit={P15}]{sevate}}
\app{\lem[wit={C6,V3,N23,K3,P15,V15,Jyo}]{cāsanottamam}
	\rdg[wit={N3,V19,V1}]{vāsanottamam}
	\rdg[wit={J10}]{sādhanottamam}}\marma//} \NotIn{C7}\\!}
\end{tlg}

%\newpage
\begin{ava}[hp01_053]
\app{\lem[wit={ceteri}]{atha bhadrāsanam} % V3 om. ṃ
	\rdg[wit={V19}]{atha bhadraṃ}
	\rdg[wit={N3,C7,P15,V15},alt={\om}]{\skp{\om}}}/ \NotIn{N3,C7,P15,V15}
\end{ava}

%1.53
\begin{tlg}[hp01_053]
\tl{
\pada{gulphau ca vṛṣaṇasyādhaḥ}
\pada{\app{\lem[wit={N3,C6,N23,V15,V1,Jyo}]{sīvanyāḥ}
	\rdg[wit={J10}]{sīvanyā}
	\rdg[wit={V19,K3}]{sīmanyāḥ}
	\rdg[wit={V3,P15},alt={\om}]{\skp{\om}}} pārśvayoḥ kṣipet/}
	\anm{=\,\ref{VuI50}ab} \lineom{ab}{C7,P15,V3}\\+}
\tl{
\grau{\pada{savyagulphaṃ tathā savye} % savyaṃ C6
\pada{\app{\lem[wit={C6,N23}]{dakṣa} % +C1
	\rdg[wit={Jyo}]{dakṣe}}gulphaṃ 
	\app{\lem[wit={N23,Jyo}]{tu}
	\rdg[wit={C6}]{ca}} dakṣiṇe/}
	\sgwit{C6,N23,Jyo} \anm{cd not in \getsiglum{P11}}}\\+} % not in V19,P15,V15,P11!
\tl{
\pada{\app{\lem[wit={N3,C6,V3,N23,V1,J10,Jyo}]{pārśva}
	\rdg[wit={Gr3a,V15}]{pārśve}
	\rdg[wit={P15}]{pārśvau}}pādau ca
	pāṇibhyāṃ}
\pada{dṛḍhaṃ
\app{\lem[wit={ceteri}]{baddhvā}
	\rdg[wit={V19}]{baddhaṃ}}
\app{\lem[wit={C6,V3,V15,V1,J10,Jyo}]{suniścalam}
	\rdg[wit={N3}]{suniścalaḥ}
	\rdg[wit={K3,C7}]{suniścayam}
	\rdg[wit={N23}]{stu niścalaṃ}
	\rdg[wit={P15}]{tu niścalaṃ}
	\rdg[wit={V19}]{suniścitaṃ}}//}\\!}
\end{tlg}


%1.54
\begin{tlg}[hp01_054]
\tl{
\pada{bhadrāsanaṃ bhaved etat}
\pada{sarvavyādhi\app{\lem[wit={N3,P11,V3,V1pc,J10},postwit=\texteng{(\getsiglum{V1pc} by the first hand)}]{viṣāpaham}
	\rdg[wit={C6,N23,Gr3a,P15,V15,V1ac,Jyo}]{vināśanam}}\marma/}
	\\+}    % V1(3v) correction by first hand!! vināśanaṃ -> viṣāpahaṃ
\tl{
\pada{gorakṣāsanam ity āhu}% gorikṣā° P15
\pada{r i\app{\lem[wit={ceteri},alt={idaṃ}]{\skp{i}daṃ}
	\rdg[wit={N23}]{evaṃ}}
	\app{\lem[wit={ceteri}]{vai siddhayoginaḥ}
	\rdg[wit={C6}]{siddhāś ca yoginaḥ}}//}\\!}
\end{tlg}

\begin{altava}[hp01_054]
\grau{ity āsanāni/ \sgwit{K3}}
\end{altava}

%\newpage
%1.55
\begin{tlg}[hp01_055]
\tl{
\pada{\app{\lem[wit={ceteri}]{evam āsana}
	\rdg[wit={P15}]{pavanāsana}}bandheṣu}
\pada{\app{\lem[wit={ceteri}]{yogīndro}
	\rdg[wit={P15}]{yogeṃdro}}
\app{\lem[wit={N3,P11,K3,C7,P15,V15}]{vijitaśramaḥ}
	\rdg[wit={N23}]{vijitaḥ śramaḥ}
	\rdg[wit={V19}]{vijiteṃśramaḥ}
	\rdg[wit={V3}]{vijitaśramāṃ}
	\rdg[wit={C6,V1,J10,Jyo}]{vigataśramaḥ}}/}\\+}
\tl{
\pada{\app{\lem[wit={C6,N23,Gr3a,J10},alt={athābhyasen}]{athābhyase\skp{n}}% °bhyasyen P11
	\rdg[wit={P15}]{athābhyase}
	\rdg[wit={V15}]{athābhyāse}
	\rdg[wit={V3}]{athābhyāsaṃ}
	\rdg[wit={V1}]{athabhyā[g]e}
	\rdg[wit={Jyo}]{abhyasen}
	\rdg[wit={N3}]{abhyāse}}% athābhyāsen nāśuddhiḥ syāt J5!
\app{\lem[wit={N3,C6,V3,N23,Gr3a,V15,V1},alt={nāḍi}]{\skm{n }nāḍi} % ra vipulā (with caesura after the 4th)
	\rdg[wit={J10},post={\unm}]{nāḍī} % ma vipulā (but no caesura after 5th)
	\rdg[wit={P15},post=\texteng{(with both vowel signs)}]{nāḍi/ḍī}
	\rdg[wit={Jyo}]{nāḍikā}}%
\app{\lem[wit={P11,V3,Gr3a,V15,J10,Jyo}]{śuddhiṃ}
	\rdg[wit={C6,N23}]{śuddhi}
	\rdg[wit={V1}]{śvaddhiṃ}
	\rdg[wit={N3}]{śuddhiḥ syān}
	\rdg[wit={P15},post=\texteng{(one syllbale too short)}]{°ṣu}}}
\pada{\app{\lem[wit={ceteri}]{mudrādi}
	\rdg[wit={K3,C7}]{mudrayā}
	\rdg[wit={V19}]{subaddhvā}}%
pavana\app{\lem[wit={ceteri}]{kriyām} % pa<<va>>na N23
	\rdg[wit={N3,P15,V15}]{kriyāḥ}
	\rdg[wit={N23}]{kriyā}}//}\\!}
\end{tlg}

% These verses are in the J10 branch, with V3 also. (but not in V1,V19,C1)

\newpage
\startaltrecension
\begin{alttlg}[hp01_055_1]
\tl{
\pada{kriyāyuktasya siddhiḥ syād}% siddhi V3
\pada{akriyasya kathaṃ bhavet/} \\+
}\tl{
\pada{na śāstrapāṭhamātreṇa}
\pada{yogasiddhiḥ prajāyate//} % siddhi V3
\sgwit{C6,V3,J10,Jyo}\myfn{\getsiglum{Jyo} has this verse and the next one (without the 3rd line) after 1.64.}\\! % not in P11!
}\end{alttlg}

%\newpage
\begin{alttlg}[hp01_055_2]
\tl{
\pada{na
\app{\lem[wit={C6,J10,Jyo}]{veṣadhāraṇaṃ}
	\rdg[wit={V3}]{veṣṭadhāriṇyo}}
\app{\lem[wit={C6,J10,Jyo}]{siddheḥ}
	\rdg[wit={V3}]{siddhi}}}
\pada{kāraṇaṃ
\app{\lem[wit={C6,J10,Jyo}]{na ca}
	\rdg[wit={V3}]{ca}}
\app{\lem[wit={C6,V3,Jyo}]{tatkathā}
	\rdg[wit={J10}]{tatkathāḥ}}/}\\+
}\tl{
\pada{kriyaiva kāraṇaṃ
\app{\lem[wit={J10,Jyo}]{siddheḥ}
	\rdg[wit={V3}]{siddhi}
	\rdg[wit={C6}]{siddhaṃ}}}
\pada{satya\app{\lem[wit={J10,Jyo},alt={etan}]{\skm{m }eta\skp{n}}
	\rdg[wit={C6}]{eva}
	\rdg[wit={V3}]{eva tat}}n na saṃśayaḥ/}\\+
}\tl{
\pada{\app{\lem[wit={C6,V3,J10}]{śiśnodara}
	\rdg[wit={Jyo},alt={\om}]{\skp{\om}}}%
\app{\lem[wit={J10}]{ratāyeha}
	\rdg[wit={V3}]{ratāyena}
	\rdg[wit={C6}]{ratāyasya}
	\rdg[wit={Jyo},alt={\om}]{\skp{\om}}}}
\pada{\app{\lem[wit={C6,V3,J10}]{na}
	\rdg[wit={Jyo},alt={\om}]{\skp{\om}}}
\app{\lem[resp=emend]{deyā} % de(2)na(1)yā J7
	\rdg[wit={C6,V3}]{deyo}
	\rdg[wit={J10}]{dayo}
	\rdg[wit={Jyo},alt={\om}]{\skp{\om}}}
\app{\lem[wit={C6,J10}]{veṣa}
	\rdg[wit={V3}]{viṣa}
	\rdg[wit={Jyo},alt={\om}]{\skp{\om}}}%
\app{\lem[wit={V3,J10}]{dhāriṇaḥ}
	\rdg[wit={C6}]{dhāriṇe}
	\rdg[wit={Jyo},alt={\om}]{\skp{\om}}}//} \sgwit{C6,V3,J10,Jyo}\\!
}\end{alttlg}


%\newpage
\begin{alttlg}[hp01_055_3]
\tl{
\pada{\app{\lem[wit={K3,C7}]{mayi}% C8,J7
	\rdg[wit={V19}]{miyi}}
\app{\lem[wit={K3,C7}]{bodhāmbudhau}
	\rdg[wit={V19}]{bodhoṃbudhau}}
	svacche} % L1
\pada{\app{\lem[wit={V19,K3}]{tuccho'yaṃ}
	\rdg[wit={C7}]{tuccho yo}} viśvabudbudaḥ/}\\+
}\tl{
\pada{pralīna udito veti}
\pada{vikalpapaṭalaḥ kutaḥ//} \sgwit{Gr3a}\\!
}\end{alttlg}


\begin{alttlg}[hp01_055_4]
\tl{
\pada{śruti\app{\lem[wit={K3,C7}]{pratītiḥ}
		\rdg[wit={V19}]{prītaḥ}}
	svagurupratītiḥ}\\+
}\tl{
\pada{svātmapratītir manaso%
\app{\lem[resp=emend]{'pi rodhaḥ}% J7/N11/J13/C8
	\rdg[wit={Gr3a}]{'pi bodhaḥ}}/}\\+ % L1/N5/V19
}\tl{
\pada{etāni sarvāṇi
\app{\lem[wit={V19}]{samuccitāni} % C1/N11/J13/V19
	\rdg[wit={K3,C7}]{samuddhṛtāni}}}\\+ % L1/N5
}\tl{
\pada{matāni dhīrair iha sādhanāni//} \sgwit{Gr3a}\\!
}\end{alttlg}
\endaltrecension

%\newpage
%1.56
\begin{tlg}[hp01_056]
\tl{
\pada{āsanaṃ % āsana V3
\app{\lem[wit={ceteri}]{kumbhakaṃ}
	\rdg[wit={V3,J10}]{kumbhakaś}}
\app{\lem[wit={ceteri}]{citraṃ}
	\rdg[wit={N23,P15}]{citra}}}\marmas
\pada{\app{\lem[wit={C6,Gr3a,P15,V15,Jyo}]{mudrākhyaṃ}
	\rdg[wit={N3}]{mudrākhya}
	\rdg[wit={N23}]{mudrāśyaṃ}
	\rdg[wit={V3,J10}]{mudrādi}}
\app{\lem[wit={ceteri}]{karaṇaṃ tathā}
	\rdg[wit={J10}]{karaṇāni ca}
	\rdg[wit={V3}]{pavanakriyā}}\marma/}\\+}
\tl{
\pada{atha nādānu% nādāna° P15
\app{\lem[wit={ceteri},alt={sandhānam}]{sandhāna\skp{m}}
	\rdg[wit={V3}]{sandhāna}}}%
\pada{\app{\lem[wit={ceteri},alt={abhyāsā°}]{\skm{m }abhyāsā}
	\rdg[wit={C6,V19}]{abhyāsyā}
	\rdg[wit={V3}]{syābhyāsā}}%
\app{\lem[wit={C6,V3,N23,V15,J10,Jyo}]{nukramo haṭhe}
	\rdg[wit={N3}]{nukramo haṭhaḥ}
	\rdg[wit={P15}]{nukramo haṭho}
	\rdg[wit={Gr3a}]{nukrameṇa tu}}//} \NotIn{V1}\\!}
\end{tlg}


%\newpage
%1.57
\begin{tlg}[hp01_057]
\tl{
\pada{brahmacārī
\app{\lem[wit={ceteri}]{mitāhārī}
	\rdg[wit={V19}]{mitāhāro}}}
\pada{\app{\lem[wit={N3,P11,Gr3a,P15,V1}]{yogī}} % +J5,G4,N19
	\rdg[wit={C6,V3,N23,V15,J10,Jyo}]{tyāgī}
	yoga\app{\lem[wit={ceteri}]{parāyaṇaḥ}
	\rdg[wit={K3}]{parākramaḥ}}/}\\+}
\tl{
\pada{abdād ūrdhvaṃ % V3,J10 om. anusvāra
	bhave\app{\lem[wit={ceteri},alt={siddho}]{\skm{t }siddho}
	\rdg[wit={N3,C7}]{siddhir}
	\rdg[wit={J10}]{siddhīn}}}
\pada{nātra
\app{\lem[wit={ceteri}]{kāryā}
	\rdg[wit={V3,N23ac}]{kārya}
	\rdg[wit={C6}]{kāryo}}
\app{\lem[wit={ceteri}]{vicāraṇā}
	\rdg[wit={J10}]{vicāraṇāt}
	\rdg[wit={C6}]{vicāraṇe}}//}\\!}
\end{tlg}

\newpage
%1.58
\begin{tlg}[hp01_058]
\tl{
\pada{susnigdhamadhu\app{\lem[wit={ceteri},alt={āhāraś}]{\skm{r}āhāra\skp{ś}}
	\rdg[wit={N3,C6}]{āhāra}
	\rdg[wit={V3}]{āhāraṃ}}}%
\pada{\app{\lem[wit={ceteri},alt={caturthāṃśa}]{\skm{ś }caturthāṃśa}%
	\rdg[wit={N23}]{caturthāśā}}%
\app{\lem[wit={ceteri}]{vivarjitaḥ}
	\rdg[wit={V3}]{vivarjitam}}/}\\+}
\tl{
\pada{\app{\lem[wit={ceteri}]{bhujyate}% bhū° N3
	\rdg[wit={V3}]{bhuṃjyate}}
	śiva\app{\lem[wit={ceteri}]{saṃprītyai} % °tyaiḥ C6
	\rdg[wit={V1}]{saṃpritya}}}
\pada{\app{\lem[wit={ceteri}]{mitāhāraḥ}
	\rdg[wit={N23}]{mitāhārī}}
\app{\lem[wit={ceteri}]{sa ucyate}
	\rdg[wit={V3,V1}]{samucyate}}//}\\!}
\end{tlg}


%\newpage
%1.59
\begin{tlg}[hp01_059]
\tl{
\pada{\app{\lem[wit={ceteri}]{kaṭvamla}%
	\rdg[wit={V3,V1}]{kaṭvāmla}}%
\app{\lem[wit={ceteri}]{tīkṣṇalavaṇoṣṇa}
	\rdg[wit={V19}]{tiktalavaṇoṣṇa}}%
\app{\lem[wit={N3,Gr3a,P15,V1,J10,Jyo}]{harīta}
	\rdg[wit={V15}]{hārīta}
	\rdg[wit={C6,V3,N23}]{harita}}%
\app{\lem[wit={N3,C6,N23,Gr3a,P15,V15,Jyo}]{śāka}
	\rdg[wit={V3,V1}]{śākaṃ}
	\rdg[wit={J10}]{sāka}}}-\\+}
\tl{
\pada{sauvīrataila%
%\myfn{\emph{sauvīra} is glossed as \emph{kāṃjī} in \getsiglum{J8}. Cf. Brahmānanda's comm.: \emph{sauvīraṃ kāñjikam}.}%
	\app{\lem[wit={ceteri}]{tila}
	\rdg[wit={V1},alt={\illeg}]{\skp{\illeg}}
	\rdg[wit={C6,N23},alt={\om}]{\skp{\om}}}sarṣapa%
\app{\lem[wit={ceteri}]{matsyamadyam}
	\rdg[wit={P15}]{matsyamadyāḥ}
	\rdg[wit={N3}]{machyamadyā}
	\rdg[wit={V1}]{tsyamaghaṃ}
	\rdg[wit={N23ac}]{madyama\,..\,n}
	\rdg[wit={C6,N23pc,Jyo}]{madyamatsyān}}/}\\+}
\tl{
\pada{\app{\lem[wit={P15,Jyo}]{ājādi}
	\rdg[wit={P11,N23,V15}]{ajādi}
	\rdg[wit={J10}]{ājāvi}
	\rdg[wit={V3,V1,Gr3a}]{ajāvi}
	\rdg[wit={N3}]{ājīvi}
	\rdg[wit={C6}]{ajavya}}%
\app{\lem[wit={ceteri}]{māṃsadadhi}
	\rdg[wit={V15}]{māṃsaṃ dadhi}}takra%
\app{\lem[wit={ceteri}]{kulattha}% Jyo: kulattha or kulittha
	\rdg[wit={V3,V19}]{kulatha}
	\rdg[wit={V1}]{kulatthya}
	\rdg[wit={C7}]{kuluttha}}%
\app{\lem[wit={N3,C6,V3,V19,C7,V15,Jyo}]{kola} % lokola C6
	\rdg[wit={K3}]{kaula}
	\rdg[wit={P15}]{kela}
	\rdg[wit={J10}]{kodra}
	\rdg[wit={V1}]{koṣṇā}
	\rdg[wit={N23}]{kāla}}-}\\+}% kolyā?
%\myfn{\emph{kola} is glossed as \emph{bhaṭavāsa} in \getsiglum{J8}.}
\tl{
\pada{\app{\lem[wit={ceteri}]{piṇyāka}
	\rdg[wit={J10}]{piṃṇyāka}
	\rdg[wit={V3}]{pinnāka}
	\rdg[wit={N23},alt={\om}]{\skp{\om}}}%
	hiṅgu\app{\lem[wit={ceteri},alt={laśunā-/lasunādyam}]{laśunādya\skp{m}}% laśunā C7,V15; lasunā N23,V1,V3,J10,V19
	\rdg[wit={P15}]{laśanādyam}
	\rdg[wit={C7}]{laśunādim}}%
\app{\lem[wit={ceteri},alt={apathyam}]{\skm{m }apathya\skp{m}}
	\rdg[wit={C7}]{asūram}}m āhuḥ//}\\!} % āhu N3
\end{tlg}

%\newpage
%1.60
\begin{tlg}[hp01_060]
\tl{
\pada{bhojanam ahitaṃ
\app{\lem[wit={N23,V19,K3,P15,Jyo},alt={vidyāt}]{vidyā\skp{t}}
	\rdg[wit={C7,V1}]{vidyā}
	\rdg[wit={N3,C6,V3,V15,J10}]{viṃdyāt}}% =A1
\app{\lem[wit={ceteri},alt={punar apy}]{\skm{t }punar a\skp{py}}
	\rdg[wit={Gr3a,P15}]{punar}}%
\app{\lem[wit={ceteri},alt={uṣṇīkṛtaṃ}]{\skm{py }uṣṇīkṛtaṃ}
	\rdg[wit={V3}]{uśnakrataṃ}
	\rdg[wit={Gr3a}]{uṣṇībhūtam}
	\rdg[wit={N3}]{uṣṇi}
	\rdg[wit={C6}]{asvīkṛtaṃ}}
\app{\lem[wit={N3,C6,V3,N23,P15,V15,J10,Jyo}]{rūkṣam}
	\rdg[wit={V1}]{rūkṣa}
	\rdg[wit={V19,K3}]{apramitaṃ}
	\rdg[wit={C7}]{apratihataṃ}}/}\\+}
\tl{
\pada{\app{\lem[resp=emend]{atilavaṇam amlapṛktaṃ}
	\rdg[wit={N23}]{atīlavaṇāmlapṛktaṃ}
	\rdg[wit={C6}]{atīlavaṇāmlayuktaṃ}
	\rdg[wit={Jyo}]{atilavaṇam amlayuktaṃ}
	\rdg[wit={N3}]{atilavaṇādyaprantaṃ}
	\rdg[wit={V1}]{atilavaṇādyuṣṇataṃ}
	\rdg[wit={P15}]{atilavaṇādiprayuktaṃ}
	\rdg[wit={V15}]{atilavaṇādiyuktaṃ}
	\rdg[wit={J10}]{atilavaṇaṃ tilapiṇḍa}
	\rdg[wit={V3}]{atilavaṇaṃ tilaṃ piṇḍa}
	\rdg[wit={V19}]{atilavaṇasavapalala}
	\rdg[wit={K3}]{atilavaṇasavapalalaṃ}
	\rdg[wit={C7}]{atilavaṇāsavapalalaṃ}
	}\marmas
\app{\lem[wit={ceteri}]{kadaśana}
	\rdg[wit={V3,J10}]{kadaśanaṃ}}%
\app{\lem[wit={ceteri}]{śākotkaṭaṃ}
	\rdg[wit={V3}]{śākātkaṭa}
	\rdg[wit={V1}]{śokātkaṭa}
	\rdg[wit={N23}]{śākokṣadaṃ}}
\app{\lem[wit={C6,N23,P15,V15}]{duṣṭam}
	\rdg[wit={N3}]{duṣṇaṃ}
	\rdg[wit={Gr3a,Jyo}]{varjyam}
	\rdg[wit={J10}]{varjjaṃ}
	\rdg[wit={V3}]{varjitaṃ}
	\rdg[wit={V1},alt={\illeg}]{\skp{\illeg}}}\marma//}
	\anm{Upagīti}\\!}
\end{tlg}



%\newpage
\begin{ava}[hp01_061]
\app{\lem[wit={C6,V3,N23,Jyo}]{tathā hi}
	\rdg[wit={N3,V1,J10}]{tathā}
	\rdg[wit={V19,C7,P15,V15,Jyo},alt={\om}]{\skp{\om}}}
\app{\lem[wit={N3,C6,V3,N23,V1,J10}]{gorakṣavacanam}
	\rdg[wit={V19,C7,P15,V15,Jyo},alt={\om}]{\skp{\om}}}/
	\sgwit{N3,C6,V3,N23,V1,J10}
\end{ava}

%1.61
\begin{tlg}[hp01_061]
\tl{
\myfn{\getsiglum{Jyo} adds here:
\devnote{vahnistrīpathisevānām ādau varjanam ācaret} (cf. HP 3.32cd; Amaraugha (short recension) 35)}% REF
\pada{\app{\lem[alt={\ante varjayed \add},nosep]{\skp{\ante varjayed \add}}
	\rdg[wit={V1}]{tailāmlāloṇītīnikālikābhāi\,(?)}}%
\app{\lem[wit={C6,Gr3a,P15,V15,J10,Jyo},alt={varjayed}]{varjaye\skp{d}}
	\rdg[wit={N3,V3,N23,V1}]{varjaye}
}\app{\lem[wit={ceteri},alt={durjana}]{\skm{d }durjana}
	\rdg[wit={C6}]{durjanaṃ}
	\rdg[wit={V1}]{tarjana}}%
\app{\lem[wit={N3,P11,P15,Jyo}]{prāntaṃ}
	\rdg[wit={V1}]{prātaṃ}
	\rdg[wit={C6}]{prāpte}
	\rdg[wit={N23,Gr3a,V15,J10}]{prītiṃ}
	\rdg[wit={V3}]{prīti}}}
\pada{\app{\lem[wit={ceteri}]{vahnistrī} % stri N3
	\rdg[wit={V19}]{vastrī}}%
\app{\lem[wit={ceteri}]{patha}
	\rdg[wit={V3}]{pathya}
	\rdg[wit={Jyo}]{pathi}
	\rdg[wit={V19}]{madhu}}sevanam/}\\+} % sevana N3
\tl{
\pada{\app{\lem[wit={ceteri},alt={prāta(ḥ)snāno}]{prātaḥsnāno} % V1,J10,V19,C1 prātasnāno
	\rdg[wit={V3}]{prātaśnāno}}pavāsādi}
\pada{kāya\app{\lem[wit={N3,C6,V3,N23,P15,V15,Jyo}]{kleśavidhiṃ} % kāyā N3; kleśaṃ P15
	\rdg[wit={Gr3a,V1,J10}]{kleśādikaṃ}}
\app{\lem[wit={ceteri}]{tathā}
	\rdg[wit={V19}]{yathā}}//}\\!}
\end{tlg}


\newpage
%1.62
\begin{tlg}[hp01_062]
\tl{
\pada{\app{\lem[wit={ceteri}]{godhūma}
	\rdg[wit={V19}]{godhūmā}}śāli% śālī N23
\app{\lem[wit={ceteri}]{yava}
	\rdg[wit={C6,V1}]{java}}%
\app{\lem[wit={ceteri}]{ṣaṣṭika}
	\rdg[wit={N3,C6,Jyo}]{ṣāṣṭika}
	\rdg[wit={N23}]{māṣikaṃ}
	\rdg[wit={P15}]{piṣṭaka}}%
\app{\lem[wit={ceteri}]{śobhanānnaṃ}
	\rdg[wit={V3,N23}]{śobhanānna}
	\rdg[wit={V1}]{śobhanānnānī}}}\\+}
\tl{
\pada{kṣīrājya\app{\lem[wit={N3,C6,N23,Gr3a}]{maṇḍa}% +J5,P11
	\rdg[wit={P15,V15,J10,Jyo}]{khaṇḍa}
	\rdg[wit={V3,V1}]{ṣaṃḍa}}%
	nava\app{\lem[wit={ceteri}]{nīta} % va<<na>>nīta N23
	\rdg[wit={V1}]{nīti}}%
\app{\lem[wit={ceteri}]{sitā}
	\rdg[wit={V1}]{śītā}
	\rdg[wit={P15}]{sudhā}}madhūni/}\\+}
\tl{
\pada{śuṇṭhī % suṃṭhī V1,V3, suṭhī V19
\app{\lem[wit={ceteri}]{paṭolaka} % vaṭo° C7
	\rdg[wit={P15,V1}]{paṭolika}
	\rdg[wit={J10}]{paṭola}}%
%\myfn{\emph{paṭolaka} is glossed as \emph{palavala} in \getsiglum{J8}. Cf. Brahmānanda's comm.: \emph{paṭolakaphalaṃ paravara iti bhāṣāyāṃ prasiddhaṃ}.}%
\app{\lem[wit={C6,P11,V3}]{phalādi ca}
	\rdg[wit={V15}]{phalādiṣu}
	\rdg[wit={G4,N23,V19,K3,P15,Jyo}]{phalādika} % ##
	\rdg[wit={N3}]{phalādi<<ka>>}
	\rdg[wit={J5}]{phalādi}
	\rdg[wit={C7}]{phalādi\,..}
	\rdg[wit={J10}]{phalakādi ca}
	\rdg[wit={V1}]{phipalādika}}
\app{\lem[wit={ceteri}]{pañcaśākaṃ} % ṃ oṃ? V15
	\rdg[wit={N3}]{pacyaśākaṃ}
	\rdg[wit={J10}]{śākabhuktaṃ}}} \\+}
\tl{
\pada{\app{\lem[wit={ceteri}]{mudgādi}
	\rdg[wit={N3,V3,V15}]{mudgā}
	\rdg[wit={C6}]{mu\,\_\,di}}
\app{\lem[wit={ceteri},alt={divyam}]{divya\skp{m}}
	\rdg[wit={V19,C7}]{cālpam}}m udakaṃ\marmas
\app{\lem[wit={ceteri}]{ca}
	\rdg[wit={P15}]{hri\,(?)}
	\rdg[wit={V3},alt={\om}]{\skp{\om}}}
\app{\lem[wit={N3,C6,V3,P15,J10,Jyo}]{yamīndra}% yamīdrya P15, yamidra N3
	\rdg[wit={N23}]{yatīndra}
	\rdg[wit={Gr3a,V15,V1}]{munīndra}}pathyam\marma//}% ##
%\myfn{\getsiglum{N23} inserts the following verse here, which is cited in the Jyotsnā: \devnote{tad uktaṃ vaidyake}---
%	\devnote{sarvaśākamacākṣuṣyaṃ cakṣapyaṃ (\recte cākṣuṣyaṃ) śākapañcakaṃ/ jīvantī vāstu matsyākṣī meghanādaḥ punarnavā//}\\
%	\getsiglum{V3} has instead:
%	\devnote{kṣīravarṇī (\recte °parṇī) ca jaivantī (\recte jīvantī) matsāṣī (\recte matsyākṣī) ca punarnavā/
%	meghanādīti pañcaite śākanāma(?) prakīrtitā(?)//}\\
%	\getsiglum{V15,V3} have instead:
%	\devnote{kṣīraparṇī ca jīvantī matsyākṣī ca punarnavā/
%	meghanādaś ca pañcaite pañcaśākāḥ (śākanāma \getsiglum{V3}) prakīrtitāḥ//}}
	\\!}
\end{tlg}

%\newpage
 \startaltrecension
 \begin{alttlg}[hp01_062_1]
 \tl{\pada{sarvaśākamacākṣuṣyaṃ}
 \pada{\app{\lem[resp=emend]{cākṣuṣyaṃ}\rdg[wit={N23}]{cakṣapyaṃ}}
 śākapañcakaṃ}/\\+}
 \tl{
 \pada{jīvantī vāstu matsyākṣī}
 \pada{meghanādaḥ punarnavā}// \sgwit{N23}\\!}
 %\myfn{Cited in Jyotsnā: \devnote{taduktaṃ vaidyake} ...}
 \end{alttlg}


 \begin{alttlg}[hp01_062_2]
   \tl{
     \pada{\app{\lem[wit={V15}]{kṣīraparṇī}
	 \rdg[wit={V3}]{kṣīravarṇī}} ca
       \app{\lem[wit={V15}]{jīvantī}
	   \rdg[wit={V3}]{jaivantī}}}
     \pada{\app{\lem[wit={V15}]{matsyākṣī}
	 \rdg[wit={V3}]{matsāṣī}} ca punarnavā}/\\+}
   \tl{
     \pada{\app{\lem[wit={V15}]{meghanādaś ca}
	 \rdg[wit={V3}]{meghanādīti}}} pañcaite
     \pada{\app{\lem[wit={V3}]{śākanāma prakīrtitā} 
     % [wit={W4}]{śākasaṃjñāḥ prakīrtitāḥ}
	 \rdg[wit={V15},post=\texteng{(°kāḥ °tāḥ \getsiglum{V15pc})}]{pañcaśākaḥ prakīrtitaḥ}}\,\texteng{\small (sic)}//} \sgwit{V3,V15}\\!}% Not in N9
 \end{alttlg}
 \endaltrecension


%\newpage
%1.63
\begin{tlg}[hp01_063]
\tl{
\pada{\app{\lem[wit={N3,V15,V1}]{mṛṣṭaṃ}% +M1,M3
	\rdg[wit={J5,C6,P11,V3,N23,P15}]{miṣṭaṃ}% +N19; damaged G4
	\rdg[wit={V19,J10}]{iṣṭaṃ}
	\rdg[wit={Jyo}]{puṣṭaṃ}
	\rdg[wit={K3,C7}]{uṣṇaṃ}}\marmas
\app{\lem[wit={ceteri}]{sumadhuraṃ} % raṃ om. N3
	\rdg[wit={V3,V19,V15}]{samadhuraṃ}} % +C1
	snigdhaṃ gavyaṃ
dhātu\app{\lem[wit={ceteri}]{prapoṣaṇam}
	\rdg[wit={C7,V15}]{prapoṣakaṃ}}/}\\+}
\tl{
\pada{mano'bhilaṣitaṃ % °lakhitaṃ N23, °lāṣitaṃ V1, °liṣitaṃ N3
\app{\lem[wit={N3,P11,N23,Gr3a,V15,J10,Jyo}]{yogyaṃ}
	\rdg[wit={V3}]{yonyaṃ}
	\rdg[wit={P15,V1}]{bhojyaṃ}
	\rdg[wit={C6}]{divyaṃ}} yogī % yogi N3
\app{\lem[wit={ceteri},alt={bhojanam}]{bhojana\skp{m}}
	\rdg[wit={J10}]{bhojanasam}}m ācaret//}\\!}
\end{tlg}


%\newpage
%1.64
\begin{tlg}[hp01_064]
\tl{
\pada{yuvā vṛddho%
\app{\lem[wit={ceteri}]{'tivṛddho}
	\rdg[wit={K3,C7}]{pi vṛddho}} vā} % vṛddhā? N23
\pada{vyādhito
\app{\lem[wit={ceteri}]{durbalo'pi vā} % durbalā? N23
	\rdg[wit={C6,J10}]{durbalas tathā}}/}\\+}
\tl{
\pada{abhyāsāt siddhim āpnoti}
\pada{\app{\lem[wit={ceteri}]{sarvayogeṣv atandritaḥ}
	\rdg[wit={K3}]{sarvo yogeṣv atandritaḥ}
	\rdg[wit={V3}]{sarvayogeṣu taṃdritaḥ}
	\rdg[wit={V1}]{sarvaṃ yogī yateṃdriyaḥ}}//}\\!}
\end{tlg}


%\newpage
%1.65
\begin{tlg}[hp01_065]
\tl{
\pada{\app{\lem[wit={ceteri}]{pīṭhāni}
	\rdg[wit={Gr3a}]{pīṭhādi}}
\app{\lem[wit={ceteri},alt={kumbhakāś}]{kumbhakā\skp{ś}} % °kāḥ C7
	\rdg[wit={V3,V15ac,J10}]{kumbhakaś}}%
\app{\lem[wit={ceteri},alt={citrā}]{\skm{ś }citrā}
	\rdg[wit={J10}]{citraṃ}}}
\pada{\app{\lem[wit={ceteri}]{divyāni}
	\rdg[wit={V1,J10}]{mudrādi}}
	karaṇāni ca/}\\+}
\tl{
\pada{\app{\lem[wit={N3,C6,P15,V15,V1}]{sarvo'pi ca}% sarvā J5, sarve P11, illeg. G4
	\rdg[wit={N23}]{sarve py ayaṃ}
	\rdg[wit={Gr3a}]{sarvo pi hi}
	\rdg[wit={V3,J10,Jyo}]{sarvāṇy api}}
haṭhā\app{\lem[wit={N3,C6,V3,N23,V15}]{bhyāso}% +J5
	\rdg[wit={V1,J10,Jyo}]{bhyāse}% +G4
	\rdg[wit={Gr3a}]{bhyāsād}% +P11
	\rdg[wit={P15}]{bhyā}}}
\pada{rājayoga% rājya P15
\app{\lem[wit={N3,P15,V15,V1pc,J10,Jyo}]{phalāvadhi}% +J5,P11
	\rdg[wit={C6,V3,N23}]{phalāvadhiḥ}% +G4
	\rdg[wit={V1ac}]{yugāvadhi}
	\rdg[wit={Gr3a}]{prasiddhaye}}//}\\!}
% \myfn{%
	% \getsiglum{V17} inserts here:
	% \devnote{svastheṣu cittasamitāsanabandhayukte
	% prāṇaṃ prapūratritayaṃ ghaṭavat prayukte/
	% savyāpasavyakramavṛddhi yathoktam eva
	% mātrā ca dvādaśa punar daśadvādaśābde//}}
\end{tlg}

\newpage
\begin{col}[hp01_col]
iti \app{\lem[wit={N3,N23,P15,V15,V1}]{svātmārāma}
	\rdg[wit={J10}]{ātmārāma}
	\rdg[wit={C6,V3}]{śrīsvātmārāma}
	\rdg[wit={Jyo}]{śrīsahajānadasaṃtānaciṃtāmaṇisvātmārāma}
	\rdg[wit={Gr3a},alt={\om}]{\skp{\om}}}%
\app{\lem[wit={C6,N23,P15,V15,V1,J10,Jyo}]{yogīndra}
	\rdg[wit={V3}]{yogendra}
	\rdg[wit={N3}]{mahāyogeṃdra}
	\rdg[wit={Gr3a},alt={\om}]{\skp{\om}}}%
\app{\lem[wit={ceteri}]{viracitāyāṃ}
	\rdg[wit={Gr3a},alt={\om}]{\skp{\om}}}
\app{\lem[wit={ceteri}]{haṭha}
	\rdg[wit={C7}]{śrīhaṭha}}pradīpikāyāṃ
\app{\lem[alt={\ante prathamo° \add},nosep]{\skp{\ante prathamo° \add}}
	\rdg[wit={V15}]{āsanayogo nāma}
	\rdg[wit={Jyo}]{āsanavidhikathanaṃ nāma}}
\app{\lem[wit={ceteri}]{prathamopadeśaḥ}% V3 om. ḥ
	\rdg[wit={C6,V15,J10}]{prathama upadeśaḥ}
	\rdg[wit={V1}]{prathamo'dhyāyaḥ}
	\rdg[wit={C7}]{prathamo\,..\,+}}//
      
\end{col}

\end{ekdosis}
\end{otherlanguage}

%\newpage
%\hrule
\vfill
%\bigskip
\small
%\section*{List of sigla}
% N3,C6,V3,N23,V19,K3,C7,P15,V15,V1,J10,Jyo
%\newpage
\begin{tabular}{llll}
%\hspace{4em}\=\hspace{10cm}\=\kill
%\\
%\hline
\multicolumn{4}{l}{\textbf{List of Sigla}} \\
\\
\getsiglum{N3} & N3 & Gr1 & 2 folios missing in Ch. 1 (1.18--28, 38--46)\\
\getsiglum{J5} & J5 & Gr1 & consulted sporadically \\
\getsiglum{G4} & G4 & Gr1 & consulted sporadically\\
\getsiglum{C6} & C6 & Gr4b\\
\getsiglum{P11} & P11 & Gr4b & consulted sporadically\\
\getsiglum{V3} & V3 & Gr6\\
\getsiglum{N23} & N23 & Gr2\\
%\getsiglum{J7} & J7 & Gr2\\
\getsiglum{V19} & V19 & Gr3\\
\getsiglum{K3} & K3 & Gr3 & the first folios missing (1.1--33)\\
\getsiglum{C7} & C7 & Gr3 & 2 folios missing in Ch. 1 (1.23c--29d, 41c--47b)\\
\getsiglum{P15} & P15 & Gr4c\\
\getsiglum{N19} & N19 & Gr4c & consulted sporadically\\
\getsiglum{V15} & V15 & Gr4c & influenced by Gr3\\
\getsiglum{V1} & V1 & Gr4c/4d\\
\getsiglum{J10} & J10 & Gr4d\\
\getsiglum{Jyo} & Jyo & Gr4a &  Brahmānanda's version, based on the edition 1972 \\
\end{tabular}
\vfill
\end{document}
