\documentclass[10pt]{memoir}
\setstocksize{220mm}{155mm} 	        
\settrimmedsize{220mm}{155mm}{*}	
\settypeblocksize{170mm}{116mm}{*}	
\setlrmargins{18mm}{*}{*}
\setulmargins{*}{*}{1.2}
% \setlength{\headheight}{5pt}
\checkandfixthelayout[lines]
\linespread{1}
\setlength{\parskip}{0.3em}
\setlength\parindent{0pt}

\makepagestyle{HPed}
\makeoddhead{HPed}{\small{HP Transl. \& Comm.}}{}{\small{\today}}
\makeevenhead{HPed}{\small{HP Transl. \& Comm.}}{}{\small{\today}}
\makeoddfoot{HPed}{}{\small{\thepage}}{}
\makeevenfoot{HPed}{}{\small{\thepage}}{}

\usepackage[teiexport=tidy,poetry=verse]{ekdosis}
\usepackage{sanskrit-poetry,libertine,xcolor}
\usepackage[english]{babel}
\setlength{\vindent}{0pt}
\setvnum{}




%%%%%%%%%%%%%%%%%%%% THE  MSS         %%%%%%%%%%%%%%%%%%%%%%%%%%%

%%% Versions
\DeclareWitness{Vu}{\selectlanguage{english}Vulg}{Vulgate, i.e. Brahmānanda's version}[]           
\DeclareWitness{X}{\selectlanguage{english}X}{TenChapter Version, Jodhpur 02228 and 02225 (ed. Lonavla)}[]
\DeclareWitness{Six}{\selectlanguage{english}Ṣ}{SixChapterVersion, ``6ChapterHPms'', fragment of enlarged text, Jodhpur}[]
% Mss. in Geographical Groups
%%%% Varanasi mss (Sampūrṇānanda mss). V1 is Important
\DeclareWitness{V1}{\selectlanguage{english}V\textsubscript{1}}{Sampurnananda Library Sarasvati Bhavan 30109}[]
        \DeclareHand{V1ac}{V1}{\selectlanguage{english}V\rlap{\textsubscript{1}}\textsuperscript{ac}}[] % added by MD
        \DeclareHand{V1pc}{V1}{\selectlanguage{english}V\rlap{\textsubscript{1}}\textsuperscript{pc}}[] % added by MD
\DeclareWitness{V2}{\selectlanguage{english}V\textsubscript{2}}{Sampurnananda Library Sarasvati Bhavan 29869}[]
\DeclareWitness{V3}{\selectlanguage{english}V\textsubscript{3}}{Sampurnananda Library Sarasvati Bhavan 29899}[]
\DeclareWitness{V4}{\selectlanguage{english}V\textsubscript{4}}{Sampurnananda Library Sarasvati Bhavan 29937}[]
\DeclareWitness{V5}{\selectlanguage{english}V\textsubscript{5}}{Sampurnananda Library Sarasvati Bhavan 29938}[]
\DeclareWitness{V6}{\selectlanguage{english}V\textsubscript{6}}{Sampurnananda Library Sarasvati Bhavan 29991}[]
\DeclareWitness{V8}{\selectlanguage{english}V\textsubscript{8}}{Sampurnananda Library Sarasvati Bhavan 30014}[]
\DeclareWitness{V11}{\selectlanguage{english}V\textsubscript{11}}{Sampurnananda Library Sarasvati Bhavan 30029}[]
\DeclareWitness{V12}{\selectlanguage{english}V\textsubscript{12}}{Sampurnananda Library Sarasvati Bhavan 30030}[]
\DeclareWitness{V13}{\selectlanguage{english}V\textsubscript{13}}{Sampurnananda Library Sarasvati Bhavan 30031}[]
\DeclareWitness{V14}{\selectlanguage{english}V\textsubscript{14}}{Sampurnananda Library Sarasvati Bhavan 30050}[]
\DeclareWitness{V15}{\selectlanguage{english}V\textsubscript{15}}{Sampurnananda Library Sarasvati Bhavan 30051}[]
\DeclareWitness{V15pc}{\selectlanguage{english}V\rlap{\textsubscript{15}}\textsuperscript{pc}\space}{}[]
\DeclareWitness{V16}{\selectlanguage{english}V\textsubscript{16}}{Sampurnananda Library Sarasvati Bhavan 30052}[]
\DeclareWitness{V17}{\selectlanguage{english}V\textsubscript{17}}{Sampurnananda Library Sarasvati Bhavan 30053}[] % added by MD
\DeclareWitness{V16pc}{\selectlanguage{english}V\rlap{\textsubscript{16}}\textsuperscript{pc}\space}{}[]
\DeclareWitness{V18}{\selectlanguage{english}V\textsubscript{18}}{Sampurnananda Library Sarasvati Bhavan 30064}[]
\DeclareWitness{V19}{\selectlanguage{english}V\textsubscript{19}}{Sampurnananda Library Sarasvati Bhavan 30069}[]
\DeclareWitness{V21}{\selectlanguage{english}V\textsubscript{21}}{Sampurnananda Library Sarasvati Bhavan 30104}[]
\DeclareWitness{V22}{\selectlanguage{english}V\textsubscript{22}}{Sampurnananda Library Sarasvati Bhavan 30110}[]
\DeclareWitness{V25}{\selectlanguage{english}V\textsubscript{25}}{Sampurnananda Library Sarasvati Bhavan 30122}[]
\DeclareWitness{V26}{\selectlanguage{english}V\textsubscript{26}}{Sampurnananda Library Sarasvati Bhavan 30123}[]
\DeclareWitness{V28}{\selectlanguage{english}V\textsubscript{28}}{Sampurnananda Library Sarasvati Bhavan 30136}[]
\DeclareWitness{W4}{\selectlanguage{english}W\textsubscript{4}}{Wai 399-6171}[]

%%%%%%%%%%%%%%%%%%%%%%%%%%%%%%%%%
%%% Jammu & Kaschmir
\DeclareWitness{K1}{\selectlanguage{english}K\textsubscript{1}}{Raghunātha Temple Library 4383}[settlement=Jammu]
        \DeclareWitness{K1ac}{\selectlanguage{english}K\rlap{\textsubscript{1}}\textsuperscript{ac}\space}{}[]
        \DeclareWitness{K1pc}{\selectlanguage{english}K\rlap{\textsubscript{1}}\textsuperscript{pc}\space}{}[]
\DeclareWitness{L1}{\selectlanguage{english}L\textsubscript{1}}{SOAS RE 43454}[settlement=Jammu]
% More details? Catalogue number? L1 And C1 very close (and come from same region)
%%%%%%%%%%%%%%%%%%%%%%%%%%%%%%%%
% Jodhpur
% J10 is important
\DeclareWitness{J10}{\selectlanguage{english}J\textsubscript{10}}{MSPP Jodhpur 2230}[]
        \DeclareHand{J10ac}{J10}{\selectlanguage{english}J\rlap{\textsubscript{10}}\textsuperscript{ac}}[] % modified by MD
        \DeclareHand{J10pc}{J10}{\selectlanguage{english}J\rlap{\textsubscript{10}}\textsuperscript{pc}}[] % modified by MD
\DeclareWitness{J1}{\selectlanguage{english}J\textsubscript{1}}{Jodhpur 02231}[]
\DeclareWitness{J2}{\selectlanguage{english}J\textsubscript{2}}{Jodhpur 02232}[]   
\DeclareWitness{J3}{\selectlanguage{english}J\textsubscript{3}}{Jodhpur 02233}[]
\DeclareWitness{J4}{\selectlanguage{english}J\textsubscript{4}}{Jodhpur 02234}[]
        \DeclareWitness{J4ac}{\selectlanguage{english}J\rlap{\textsubscript{4}}\textsuperscript{ac}\space}{MSPP Jodhpur 02234}[]
        \DeclareWitness{J4pc}{\selectlanguage{english}J\rlap{\textsubscript{4}}\textsuperscript{pc}\space}{MSPP Jodhpur 02234}[]
\DeclareWitness{J5}{\selectlanguage{english}J\textsubscript{5}}{Jodhpur 02235}[]  % 4 chapters, 34 jpgs,   long colophon, missing lines in the beginning.
\DeclareWitness{J6ac}{\selectlanguage{english}J\rlap{\textsubscript{6}}\textsubscript{ac}}{Jodhpur 02237}[]  % 4 chapters, 49 jpgs,   1st folio: idaṃ gulābarāyasya
% tulasīrāmaśarmmaṇaḥ putrasya pustakaṃ ...        End: iti śrīsahajānandasantānacintāmaṇisvātmārāmaviracitāyāṃ ..
% saṃvat 1802   (more consistent text)
\DeclareWitness{J6pc}{\selectlanguage{english}J\rlap{\textsubscript{6}}\textsubscript{pc}}{Jodhpur 02237}[] 
\DeclareWitness{J7}{\selectlanguage{english}J\textsubscript{7}}{Jodhpur 02241}[]  % 4 chapters, 41 jpgs
\DeclareWitness{J8}{\selectlanguage{english}J\textsubscript{8}}{Jodhpur 23709}[]  % 4 chapters,  87 jpgs.   saṃvat 1724
\DeclareHand{J8ac}{J8}{\selectlanguage{english}J\rlap{\textsubscript{8}}\textsuperscript{ac}}[]  % changed by MD
\DeclareHand{J8pc}{J8}{\selectlanguage{english}J\rlap{\textsubscript{8}}\textsuperscript{pc}}[]  % changed by MD
\DeclareWitness{J9}{\selectlanguage{english}J\textsubscript{9}}{Jodhpur 02224}[]  %  fragment, 20 jpgs.
\DeclareWitness{J11}{\selectlanguage{english}J\textsubscript{11}}{Jodhpur 23532}[]
\DeclareWitness{J12}{\selectlanguage{english}J\textsubscript{12}}{Jodhpur 18552}[] 
\DeclareWitness{J13}{\selectlanguage{english}J\textsubscript{13}}{Jodhpur 02229}[]  %  5 chapters, 93 jpgs.
\DeclareWitness{J14}{\selectlanguage{english}J\textsubscript{14}}{Jodhpur 02239}[]  %  4 chapters
\DeclareWitness{J15}{\selectlanguage{english}J\textsubscript{15}}{Jodhpur 9732A}[]
\DeclareWitness{J17}{\selectlanguage{english}J\textsubscript{17}}{Jodhpur 3013}[]
% Haṭhapradīpikā with (non-Sanskrit) Bhāṣya RORI Jodhpur ACC.NO.18552
%  Haṭhapradīpikā with (non-Sanskrit) commentary, RORI Alwar 952, 4 chapters,  colophon of the comm:
% iti śrīlāhorīmiśravrajabhūṣanaviracitāyāṃ bhāvārthadīpikāyāṃ caturthodhyāya ..    
%  Haṭhapradīpikā (5 chapter) MSPP Jodhpur ACC.NO.02229/

%%%%%%%%%%        Bodleian, Oxford
\DeclareWitness{B1}{\selectlanguage{english}B\textsubscript{1}}{Bodleian Library No. d.457(8)}[settlement=Oxford]
\DeclareWitness{B2}{\selectlanguage{english}B\textsubscript{2}}{Bodleian Library No. d.458(1)}[settlement=Oxford]
\DeclareWitness{B3}{\selectlanguage{english}B\textsubscript{3}}{Bodleian Library No. d.458(9)}[settlement=Oxford]

%%%%%%%%%%%   Chandigarh
\DeclareWitness{C1}{\selectlanguage{english}C\textsubscript{1}}{Lalchand M-2080}[]%L1 And C1 very close (and come from same region)
\DeclareWitness{C2}{\selectlanguage{english}C\textsubscript{2}}{Lalchand M-6065}[]
\DeclareWitness{C3}{\selectlanguage{english}C\textsubscript{3}}{Lalchand M-1293}[]
\DeclareWitness{C4}{\selectlanguage{english}C\textsubscript{4}}{Lalchand M-2081}[]
\DeclareWitness{C4ac}{\selectlanguage{english}C\rlap{\textsubscript{4}}\textsuperscript{ac}\space}{}[]
\DeclareWitness{C4pc}{\selectlanguage{english}C\rlap{\textsubscript{4}}\textsuperscript{pc}\space}{}[]
\DeclareWitness{C5}{\selectlanguage{english}C\textsubscript{5}}{Lalchand M-2082}[]%doesn't have chapter 1
\DeclareWitness{C6}{\selectlanguage{english}C\textsubscript{6}}{Lalchand M-2089}[]
\DeclareWitness{C7}{\selectlanguage{english}C\textsubscript{7}}{Lalchand M-6494}[]
\DeclareWitness{C8}{\selectlanguage{english}C\textsubscript{8}}{Lalchand M-2091}[]
\DeclareWitness{C8pc}{\selectlanguage{english}C\rlap{\textsubscript{8}}\textsuperscript{pc}\space}{}[]
\DeclareWitness{C9}{\selectlanguage{english}C\textsubscript{9}}{Lalchand M-4530}[]

% %%%%%%%%%%        Nepalese
\DeclareWitness{N1}{\selectlanguage{english}N\textsubscript{1}}{NGMPP A1400-2}[]
\DeclareWitness{N2}{\selectlanguage{english}N\textsubscript{2}}{NGMPP B 39-19}[]
\DeclareWitness{N3}{\selectlanguage{english}N\textsubscript{3}}{NGMPP B 62-20}[]
\DeclareWitness{N5}{\selectlanguage{english}N\textsubscript{5}}{NGMPP A60-15 + A61-1}[]
\DeclareWitness{N6}{\selectlanguage{english}N\textsubscript{6}}{NGMPP A61-6}[]
\DeclareWitness{N9}{\selectlanguage{english}N\textsubscript{9}}{NGMPP A62-33}[]
\DeclareWitness{N10}{\selectlanguage{english}N\textsubscript{10}}{NGMPP A62-37}[]
\DeclareWitness{N11}{\selectlanguage{english}N\textsubscript{11}}{NGMPP A63-15}[]
\DeclareWitness{N12}{\selectlanguage{english}N\textsubscript{12}}{NGMPP A939-19}[]
\DeclareWitness{N13}{\selectlanguage{english}N\textsubscript{13}}{NGMPP A1378-18}[]
\DeclareWitness{N16}{\selectlanguage{english}N\textsubscript{16}}{NGMPP B39-20}[]
\DeclareWitness{N17}{\selectlanguage{english}N\textsubscript{17}}{NGMPP B 111-10}[]
\DeclareWitness{N18}{\selectlanguage{english}N\textsubscript{18}}{NGMPP E 929-3}[]
\DeclareWitness{N19}{\selectlanguage{english}N\textsubscript{19}}{NGMPP E-1528-1 / E-1527-7(4)}[]
\DeclareWitness{N20}{\selectlanguage{english}N\textsubscript{20}}{NGMPP E 2037-13 }[]
\DeclareWitness{N21}{\selectlanguage{english}N\textsubscript{21}}{NGMPP E 2097-31}[]
\DeclareWitness{N22}{\selectlanguage{english}N\textsubscript{22}}{NGMPP G 4-4}[]
\DeclareWitness{N23}{\selectlanguage{english}N\textsubscript{23}}{NGMPP G 25-2}[]
\DeclareWitness{N24}{\selectlanguage{english}N\textsubscript{24}}{NGMPP G 190-16}[]
\DeclareWitness{N24ac}{\selectlanguage{english}N\rlap{\textsubscript{24}}\textsuperscript{ac}\space}{}[]
\DeclareWitness{N24pc}{\selectlanguage{english}N\rlap{\textsubscript{24}}\textsuperscript{pc}\space}{}[]

\DeclareWitness{P28}{\selectlanguage{english}P\textsubscript{28}}{BORI 399-1895-1902}[]

%%%%%   Mysore
\DeclareWitness{M1}{\selectlanguage{english}M\textsubscript{1}}{P-5682/4}[]
%%%%%   Tübingen
\DeclareWitness{Tü}{\selectlanguage{english}Tü}{Ma I 339}[]
%%%%%%%%%%
\DeclareWitness{YC}{\selectlanguage{english}YC}{Yogacintāmaṇi}[]
\DeclareWitness{ceteri}{\selectlanguage{english}cett.}{ceteri}[]

%%%%%%%%%% Mss with Commentary
\DeclareWitness{A1}{\selectlanguage{english}A\textsubscript{1}}{Alwar 952}[]


%%%%%%%%%%%%%%%%%%%%%%%%%%%%%%%%%%%%%%%%%%%
%List of all Sigla:
%A1,B1,B2,B3,C1,C2,C3,C4,C6,C7,C8,C9,J1,J2,J3,J4,J10,J13,J14,J15,J17,L1,M1,N3,N5,N6,N9,N10,N11,N12,N13,N16,N17,N19,N20,N21,N22,N23,N24,Tü,V1,V2,V3,V4,V5,V6,V8,V11,V19,V22,V26,Vu
%%%%%%%%%%%%%%%%%%%%%%%%%%%%%%%%%%%%%%%%%%%

\DeclareShorthand{x}{\selectlanguage{english}δ}{J10,J17,N17,P28,W4}


%%% Local Variables:
%%% mode: latex
%%% TeX-master: t
%%% End:

%
%%%%%                   Abbreviation for the printed apparatus,        xml interface needed
%%%%%                   (synonyms in same line)

% Macro for Editing Abbrevs.
%\def\om{\textrm{\footnotesize \textit{omitted in}\ }} %prints om. for omitted in apparatus
%\def\korr{\textrm{\footnotesize \textit{em.}\ }} %prints em. for emended in apparatus
%\def\conj{\textrm{\footnotesize \textit{conj.}\ }} %prints conj. for conjectured in apparatus


\def\eyeskip{\textrm{{ab.\,oc. }}}   
\def\aberratio{\textrm{{ab.\,oc. }}}
\def\ad{\textrm{{ad}}}   
\def\add{\textrm{{add.\ }}}
\def\ann{\textrm{{ann.\ }}}
\def\ante{\textrm{{ante }}}
\def\post{\textrm{{post }}}
%\def\ceteri{cett.\,}             % for simplifying the apparatus in print                  
\def\codd{\textrm{{codd.\ }}}   %  the same
\def\conj{\textrm{{coni.\ }}}  
\def\coni{\textrm{{coni.\ }}}
\def\contin{\textrm{{contin.\ }}}
\def\corr{\textrm{{corr.\ }}}
\def\del{\textrm{{del.\ }}}
\def\dub{\textrm{{ dub.\ }}}
\def\emend{\textrm{{emend.\ }}}
\def\expl{\textrm{{explic.\ }}}   
\def\explicat{\textrm{{explic.\ }}}
\def\fol{\textrm{{fol.\ }}}         
\def\foll{\textrm{{foll.\ }}}
\def\gloss{\textrm{{glossa ad }}}
\def\ins{\textrm{{ins.\ }}}          \def\inseruit{\textrm{{ins.\ }}}
\def\im{{\kern-.7pt\lower-1ex\hbox{\textrm{\tiny{\emph{i.m.}}}\kern0pt}}}
\def\inmargine{{\kern-.7pt\lower-.7ex\hbox{\textrm{\tiny{\emph{i.m.}}}\kern0pt}}}
\def\intextu{{\kern-.7pt\lower-.95ex\hbox{\textrm{\tiny{\emph{i.t.}}}\kern0pt}}}%\textrm{\scriptsize{i.t.\ }}}               
\def\indist{\textrm{{indis.\ }}}          \def\indis{\textrm{{indis.\ }}}
\def\iteravit{\textrm{{iter.\ }}}          \def\iter{\textrm{{iter.\ }}}  
\def\lectio{\textrm{{lect.\ }}}             \def\lec{\textrm{{lect.\ }}}
\def\leginequit{\textrm{{l.n. }}}         \def\legn{\textrm{{l.n. }}}         \def\illeg{\textrm{{l.n. }}}
\def\om{\textrm{{om. }}}
\def\primman{\textrm{{pr.m.}}}
\def\prob{\textrm{{prob.}}}
\def\rep{\textrm{{repetitio }}}
% \def\secundamanu{\textrm{\scriptsize{s.m.}}}
% \def\secm{{\kern-.6pt\lower-.91ex\hbox{\textrm{\tiny{\emph{s.m.}}}\kern0pt}}}%   \textrm{\scriptsize{s.m.}}}
\def\sequentia{\textrm{{seq.\,inv.\ }}}         \def\seqinv{\textrm{{seq.\,inv.\ }}} \def\order{\textrm{{seq.\,inv.\ }}}
\def\supralineam{{\kern-.7pt\lower-.91ex\hbox{\textrm{\tiny{\emph{s.l.}}}\kern0pt}}} %\textrm{\scriptsize{s.l.}}}
\def\interlineam{{\kern-.7pt\lower-.91ex\hbox{\textrm{\tiny{\emph{s.l.}}}\kern0pt}}}   %\textrm{\scriptsize{s.l.}}}
\def\vl{\textrm{v.l.}}   \def\varlec{\textrm{v.l.}} \def\varialectio{\textrm{v.l.}}
\def\vide{\textrm{{cf.\ }}}           \def\cf{\textrm{{cf.\ }}}
\def\videtur{\textrm{{vid.\,ut}}}
\def\crux{\textup{[\ldots]} }
\def\cruxx{\textup{[\ldots]}}
\def\unm{\textit{unm.}}        % unmetrical
%%%%%%%%%%%%%%%%%%%%%%%%%%%%%%%%%%%%



%%% Local Variables:
%%% mode: latex
%%% TeX-master: t
%%% End:

% additions/changes 2024-07-04 mm:
\TeXtoTEIPat{\lb}{<lb/>}
\TeXtoTEIPat{\begin {quote}}{<q>}
  \TeXtoTEIPat{\end {quote}}{</q>}
\TeXtoTEIPat{\begin {enumerate}}{<list rend="numbered">}
  \TeXtoTEIPat{\end {enumerate}}{</list>}
\TeXtoTEI{item}{item}

% additions/changes 2024-07-01 mm:
\TeXtoTEIPat{\unavbl {#1}}{<note type="foliolost">Folio lost in <ref>#1</ref></note>}
\TeXtoTEIPat{\NotIn {#1}}{<note type="omission">Omitted in <ref>#1</ref></note>}
\TeXtoTEI{graus}{span}[type="altrec"]
\TeXtoTEI{grau}{span}[type="altrec"]

% addition 2024-03-15 MD
\TeXtoTEI{manuref}{}

\TeXtoTEIPat{\alphaOne}{α<hi rend="sub">1</hi>}% N3
\TeXtoTEIPat{\alphaTwo}{α<hi rend="sub">2</hi>}% J5
\TeXtoTEIPat{\alphaThree}{α<hi rend="sub">3</hi>}% G4
\TeXtoTEIPat{\betaOne}{β<hi rend="sub">1</hi>}% P11
\TeXtoTEIPat{\betaTwo}{β<hi rend="sub">2</hi>}% C6
\TeXtoTEIPat{\betaOmega}{β<hi rend="sub">ω</hi>}% V3
\TeXtoTEIPat{\gammaOne}{γ<hi rend="sub">1</hi>}% N23
\TeXtoTEIPat{\gammaTwo}{γ<hi rend="sub">2</hi>}% J7
\TeXtoTEIPat{\deltaOne}{δ<hi rend="sub">1</hi>}% V19
\TeXtoTEIPat{\deltaTwo}{δ<hi rend="sub">2</hi>}% K3
\TeXtoTEIPat{\deltaThree}{δ<hi rend="sub">3</hi>}% C7
\TeXtoTEIPat{\deltaOmega}{δ<hi rend="sub">ω</hi>}% J6
\TeXtoTEIPat{\epsilonOne}{ε<hi rend="sub">1</hi>}% P15
\TeXtoTEIPat{\epsilonTwo}{ε<hi rend="sub">2</hi>}% N19
\TeXtoTEIPat{\epsilonThree}{ε<hi rend="sub">3</hi>}% V15
\TeXtoTEIPat{\epsilonFour}{ε<hi rend="sub">4</hi>}% J11
\TeXtoTEIPat{\epsilonOmega}{ε<hi rend="sub">ω</hi>}% N26
\TeXtoTEIPat{\etaOne}{η<hi rend="sub">1</hi>}% V1
\TeXtoTEIPat{\etaTwo}{η<hi rend="sub">2</hi>}% J10
\TeXtoTEIPat{\etaOmega}{η<hi rend="sub">ω</hi>}% N9

% addition 2023-12-11 MD:
\TeXtoTEIPat{\begin {metre}[#1]}{<note type="metre" target="##1">}
\TeXtoTEIPat{\end {metre}}{</note>}
\TeXtoTEIPat{\texttheta}{θ}

% change 2023-12-05 mm
\TeXtoTEI{teimute}{} 

% changes/additions 2023-11-27 MM:
\TeXtoTEIPat{\medialink {#1}{#2}}{<ref target="resources/#2">#1</ref>}

% changes/additions 2023-10-25 MM:
% new Sigla
\TeXtoTEIPat{\textAlpha}{Α}
\TeXtoTEIPat{\textalpha}{α}
\TeXtoTEIPat{\textBeta}{Β}
\TeXtoTEIPat{\textbeta}{β}
\TeXtoTEIPat{\textGamma}{Γ}
\TeXtoTEIPat{\textgamma}{γ}
\TeXtoTEIPat{\textDelta}{Δ}
\TeXtoTEIPat{\textdelta}{δ}
\TeXtoTEIPat{\textEpsilon}{Ε}
\TeXtoTEIPat{\textepsilon}{ε}
\TeXtoTEIPat{\textEta}{Η}
\TeXtoTEIPat{\texteta}{η}
\TeXtoTEIPat{\textChi}{Χ}
\TeXtoTEIPat{\textchi}{χ}
\TeXtoTEIPat{\textOmega}{Ω}
\TeXtoTEIPat{\textomega}{ω}

%new environments
\TeXtoTEIPat{\begin {postmula}[#1]}{<div type="postmula" xml:id="#1">} %%% changed 2024-07-01 mm
  \TeXtoTEIPat{\end {postmula}}{</div>}  %%% changed 2024-07-01 mm
  
\TeXtoTEIPat{\begin {altpostmula}[#1]}{<div type="altrec"><div type="postmula" xml:id="#1">} %%% added 2024-07-03 md
  \TeXtoTEIPat{\end {altpostmula}}{</div></div>} %%% added 2024-07-03 md

\TeXtoTEIPat{\begin {altava}[#1]}{<div type="altrec"><div type="avataranika" xml:id="#1">} %%% changed 2024-07-01 mm
  \TeXtoTEIPat{\end {altava}}{</div></div>} %%% changed 2024-07-01 mm

\TeXtoTEIPat{\sgwit {#1}}{<note type="inlineref"><ref>#1</ref></note>}

% changes/additions 2023-10-12 MM:
\TeXtoTEIPat{\\.}{}

% changes/additions 2023-08-15 MD:
\TeXtoTEIPat{\lineom {#1}{#2}}{<note type="omission">#1 omitted in <ref>#2</ref></note>}
%\TeXtoTEIPat{\startgray}{} %%% changed 2023-12-05 mm; not used 2024-03-26 MD
%\TeXtoTEIPat{\endgray}{} %%% changed 2023-12-05 mm; not used 2024-03-26 MD

% additions/changes 2023-06-05 mm:
%\TeXtoTEIPat{\lineom {#1}}{<note type="omission">Line omitted in <ref>#1</ref></note>}

% additions 2023-04-16 MD:
\TeXtoTEIPat{\,}{ }

% additions 2023-04-13 mm:
\TeXtoTEIPat{\begin {versinnote}}{<lg>}
  \TeXtoTEIPat{\end {versinnote}}{</lg>}

% additions 2023-04-05 MD:
\TeXtoTEIPat{\begin {testimonia}[#1]}{<note type="testimonia" target="##1">}
  \TeXtoTEIPat{\end {testimonia}}{</note>}
\TeXtoTEI{devnote}{s}[xml:lang="sa-deva"]

% app in philcomm und testimonia %%% added MM 2023-12-02
\TeXtoTEI{var}{note}[type="appinnote"]


\TeXtoTEI{anm}{note}[type="memo"] %% change 2023-04-16 MD
\TeXtoTEI{Anm}{note}[type="memo"] %% change 2023-12-05 MM
\TeXtoTEIPat{\startverse}{} %%% marked for change 2023-04-13 mm
\TeXtoTEIPat{\endverse}{} %%% marked for change 2023-04-13 mm
\TeXtoTEIPat{\newpage}{}
\TeXtoTEIPat{\marmas}{ } % changed 2024-03-17 MD
\TeXtoTEIPat{\marma}{}
\TeXtoTEIPat{\vin}{} % added by MD 2023-11-14

%%% modify environments and commands
%%% TEI mapping
% additions/changes 2022-06-07 mm:
\TeXtoTEIPat{ \& }{ &amp; }

% additions/changes 2022-06-01 mm:
\TeXtoTEI{skp}{seg}[type="deva-ignore"]
\TeXtoTEI{skm}{seg}[type="ltn-ignore"]

\TeXtoTEIPat{\rlap {#1}}{#1}

% additions/changes 2022-04-06 mm:
%\TeXtoTEI{sgwit}{ref}
\TeXtoTEI{textdev}{s}[xml:lang="sa-deva"]
\TeXtoTEIPat{\begin {col}[#1]}{<div type="colophon" xml:id="#1">}
  \TeXtoTEIPat{\end {col}}{</div>}
\TeXtoTEIPat{\begin {ava}[#1]}{<div type="avataranika" xml:id="#1">} %%% changed 2024-07-01 mm
  \TeXtoTEIPat{\end {ava}}{</div>} %%% changed 2024-07-01 mm
												   
\TeXtoTEIPat{\outdent}{}
\TeXtoTEIPat{\startaltrecension}{} %%% changed 2023-12-05 mm
\TeXtoTEIPat{\endaltrecension}{} %%% changed 2023-12-05 mm
\TeXtoTEIPat{\startaltnormal}{} % added by MD 2023-11-14 %%% changed 2023-12-05 mm
\TeXtoTEIPat{\endaltnormal}{} % added by MD 2023-11-14 %%% changed 2023-12-05 mm
\TeXtoTEIPat{\begin {alttlg}[#1]}{<div type="altrec"><lg xml:id="#1">}
  \TeXtoTEIPat{\end {alttlg}}{</lg></div>}



% additions/changes 2022-03-12 mm:
\TeXtoTEIPat{\begin {tlg}[#1]}{<lg xml:id="#1">}
  \TeXtoTEIPat{\end {tlg}}{</lg>}

\TeXtoTEIPat{\begin {translation}[#1]}{<note type="translation" target="##1">}
  \TeXtoTEIPat{\end {translation}}{</note>}
\TeXtoTEIPat{\begin {philcomm}[#1]}{<note type="philcomm" target="##1">}
  \TeXtoTEIPat{\end {philcomm}}{</note>}
\TeXtoTEIPat{\begin {sources}[#1]}{<note type="sources" target="##1">}
  \TeXtoTEIPat{\end {sources}}{</note>}


\TeXtoTEIPat{\begin {marma}[#1]}{<note type="marma" target="##1">}
  \TeXtoTEIPat{\end {marma}}{</note>}

\TeXtoTEIPat{\begin {jyotsna}[#1]}{<note type="jyotsna" target="##1">}
  \TeXtoTEIPat{\end {jyotsna}}{</note>}

\EnvtoTEI{description}{list}
\EnvtoTEI{itemize}{list}
\TeXtoTEIPat{\item [#1]}{<label>#1</label>\item}

\TeXtoTEI{tl}{l}
\TeXtoTEI{myfn}{note}[type="myfn"]
\TeXtoTEIPat{\getsiglum {#1}}{<ref target="##1"/>}

\TeXtoTEI{SetLineation}{}
\TeXtoTEI{noindent}{}
\TeXtoTEI{subsection*}{}

\TeXtoTEI{rlap}{}

% end additions/changes
% \TeXtoTEIPat{\skp {#1}}{#1}
% \TeXtoTEIPat{\skm {#1}}{}

\TeXtoTEIPat{\begin {prose}}{<p>}
  \TeXtoTEIPat{\end {prose}}{</p>}

\TeXtoTEIPat{\begin {tlate}}{<p>}
  \TeXtoTEIPat{\end {tlate}}{</p>}

\TeXtoTEI{emph}{hi}
\TeXtoTEI{bigskip}{}
% \TeXtoTEI{/}{|}
\TeXtoTEI{tl}{l}
\TeXtoTEIPat{english}{}
%\TeXtoTEIPat{-}{ } %% change 2023-04-16 MD
%\TeXtoTEIPat{°}{} %% change 2023-04-16 MD
\TeXtoTEIPat{\textcolor {#1}{#2}}{<hi rend="#1">#2</hi>}

% \TeXtoTEIPat{\eyeskip}{}
% \TeXtoTEIPat{\aberratio}{}
% \TeXtoTEIPat{\ad}{}
\TeXtoTEIPat{\add}{<hi rend="italic">add.</hi>} %% change 2023-04-16 MD
% \TeXtoTEIPat{\ann}{}
\TeXtoTEIPat{\ante}{<hi rend="italic">ante</hi> } %% change 2023-04-16 MD
\TeXtoTEIPat{\post}{<hi rend="italic">post</hi> } %% change 2023-04-16 MD
% \TeXtoTEIPat{\codd}{}
% \TeXtoTEIPat{\conj }{}
% \TeXtoTEIPat{\contin}{}
% \TeXtoTEIPat{\corr}{}
% \TeXtoTEIPat{\del}{}
% \TeXtoTEIPat{\dub}{}
% \TeXtoTEIPat{\emend }{}
% \TeXtoTEIPat{\expl}{}
% \TeXtoTEIPat{\ȩxplicat}{}
% \TeXtoTEIPat{\fol}{}
% \TeXtoTEIPat{\gloss}{}
% \TeXtoTEIPat{\ins}{}
% \TeXtoTEIPat{\im}{}
% \TeXtoTEIPat{\inmargine}{}
% \TeXtoTEIPat{\intextu}{}
% \TeXtoTEIPat{\indist}{}
% \TeXtoTEIPat{\iteravit}{}
% \TeXtoTEIPat{\lectio}{}
% \TeXtoTEIPat{\leginequit}{}
% \TeXtoTEIPat{\legn}{}
% \TeXtoTEIPat{\illeg}{<hi rend="italic">illeg.</hi>}
\TeXtoTEIPat{\illeg}{<gap reason="illeg."/>} %%% change 2023-04-11 mm
% \TeXtoTEIPat{\om}{<hi rend="italic">om.</hi>}
\TeXtoTEIPat{\om}{<gap reason="om."/>} %%% change 2023-04-11 mm
% \TeXtoTEIPat{\primman}{}
% \TeXtoTEIPat{\prob}{}
% \TeXtoTEIPat{\rep}{}
% \TeXtoTEIPat{\sequentia}{}
% \TeXtoTEIPat{\supralineam}{}
% \TeXtoTEIPat{\interlineam}{}
\TeXtoTEIPat{\vl}{<hi rend="italic">v.l.</hi>}
% \TeXtoTEIPat{\vide}{}
% \TeXtoTEIPat{\videtur}{}
% \TeXtoTEIPat{\crux}{}
% \TeXtoTEIPat{\cruxxx}{}
\TeXtoTEIPat{\unm}{<hi rend="italic">unm.</hi>}
\TeXtoTEIPat{\lacuna}{<gap reason="lac."/>} % addition 2024-03-24 MD
\TeXtoTEIPat{\lost}{<gap reason="lost"/>} % addition 2024-06-24 MD

% List of Scholars
\DeclareScholar{nos}{nos}[
forename=HPP,
surname=Team]

% Nullify \selectlanguage in TEI as it has been used in
% \DeclareWitness but should be ignored in TEI.
\TeXtoTEI{selectlanguage}{}


\SetTEIxmlExport{autopar=false}

%%%%%%%%%%%

\SetTEIxmlExport{autopar=false}
\NewDocumentEnvironment{translation}{O{}}{\textcolor{blue}{\textbf{Translation:}}}{}
\NewDocumentEnvironment{philcomm}{O{}}{
	\textcolor{blue}{\textbf{Commentary:}}}{}
\NewDocumentEnvironment{metre}{O{}}{
	\textcolor{blue}{\textbf{Metre:}}}{} % added MD 2023-12-11
\NewDocumentEnvironment{sources}{O{}}{
	\textcolor{blue}{\textbf{Sources:}}\linebreak}{}
\NewDocumentEnvironment{testimonia}{O{}}{
	\textcolor{blue}{\textbf{Testimonia:}}\linebreak}{}
\NewDocumentEnvironment{versinnote}{O{}}{\begin{ekdverse}}{\end{ekdverse}}
%\newcommand{\var}[1]{\footnotesize\textup{#1}}
\newcommand{\medialink}[2]{\textcolor{green}{\underline{#1}}}
%\TeXtoTEIPat{\medialink {#1}{#2}}{<ref target="/images/#2">#1</ref>}

\NewDocumentCommand{\tl}{m}{#1}

\def\vl{\textit{v.l.}}
\def\var#1{{\footnotesize #1}}
\def\sl#1{\emph{#1}}

%%%%%%%%%%%%

\usepackage{textgreek}

\newcommand{\alphaOne}{\textalpha\textsubscript{1}}% N3
\newcommand{\alphaTwo}{\textalpha\textsubscript{2}}% J5
\newcommand{\alphaThree}{\textalpha\textsubscript{3}}% G4
\newcommand{\betaOne}{\textbeta\textsubscript{1}}% P11
\newcommand{\betaTwo}{\textbeta\textsubscript{2}}% C6
\newcommand{\betaOmega}{\textbeta\textsubscript{\textomega}}% V3
\newcommand{\gammaOne}{\textgamma\textsubscript{1}}% N23
\newcommand{\gammaTwo}{\textgamma\textsubscript{2}}% J7
\newcommand{\deltaOne}{\textdelta\textsubscript{1}}% V19
\newcommand{\deltaTwo}{\textdelta\textsubscript{2}}% K3
\newcommand{\deltaThree}{\textdelta\textsubscript{3}}% C7
\newcommand{\deltaOmega}{\textdelta\textsubscript{\textomega}}% J6
\newcommand{\epsilonOne}{\textepsilon\textsubscript{1}}% P15
\newcommand{\epsilonTwo}{\textepsilon\textsubscript{2}}% N19
\newcommand{\epsilonThree}{\textepsilon\textsubscript{3}}% V15
\newcommand{\epsilonFour}{\textepsilon\textsubscript{4}}% J11
\newcommand{\epsilonOmega}{\textepsilon\textsubscript{\textomega}}% N26
\newcommand{\etaOne}{\texteta\textsubscript{1}}% V1
\newcommand{\etaTwo}{\texteta\textsubscript{2}}% J10
\newcommand{\etaOmega}{\texteta\textsubscript{\textomega}}% N9

%%%%%%%%%%%%%%

\babelhyphenation{%
	Dattā-treya-yoga-śāstra
	Gorakṣa-śataka
	Haṭha-pra-dī-pikā
	Hātha-ratnā-valī
	Svātmā-rāma
	Śiva-saṃhitā
	Vasiṣṭha-saṃhitā
	Viveka-mārtaṇḍa
	Yukta-bhava-deva
	Yoga-cintā-maṇi
	Yoga-yājña-valkya}

\begin{document}
\pagestyle{HPed}
\begin{ekdosis}
\SetLineation{lineation = none,}

%\chapter*{Translation \& philological commentary}

%%%%%%%%%%
\subsection*{4.1 heading}
\begin{translation}[hp04_001a]
Now \emph{samādhi}:
\end{translation}

% \begin{philcomm}[hp04_001a]
% \end{philcomm}

%%%%%%%%%%
\subsection*{4.0*1}
\begin{translation}[hp04_000_1] 
Homage to the guru, Śiva, who consists of \emph{nāda}, \emph{bindu} and \emph{kalā}. [The yogi] who is constantly devoted to him attains the untainted state (\emph{nirañjanapada}).
\end{translation}

%\begin{sources}[hp04_000_1]
%\end{sources}

\begin{testimonia}[hp04_000_1]
\emph{Haṃsavilāsa} 14 (p. 47)
\begin{versinnote}
\tl{namaḥ śivāya gurave nādabindukalātmane |\\+}
\tl{nirañjanapadaṃ yāti yatra yogī parāyaṇaḥ iti dhyānam || 14 ||\\!}
\end{versinnote}
\end{testimonia}

\begin{philcomm}[hp04_000_1]
%The compound \emph{yatnaparāyaṇa} is rather unusual and generally not attested elsewhere (the only exception known  to us is the \emph{Amarasiddhi}). The reading  \emph{yogaparāyaṇa} in some secondary manuscripts makes much better sense (i.e., 'one devoted to yoga') but is likely a substitution for a more difficult reading.
%adopt yatraparāyaṇaḥ [MD: done], and assume tasmai in 4.1ab



%Look into referent of nirañjana in these texts. Judit: best to take it generally (like nirāmaya, etc.).
Verses 4.0*1–4.0*16 are omitted by the \textalpha\ group and are likely not to be original. The first additional verse resembles a \emph{maṅgala} verse that one might expect to see at the beginning of a text. The second is a verse from the \emph{Gorakṣaśataka} that introduces the topic of \emph{samādhi}. Some manuscripts of the \textbeta, \textepsilon, \textzeta\ and \texteta\ groups have the two verses on the synonyms of \emph{rājayoga} here (on their position in the \textalpha\ group and other manuscripts, see the note to 4.32). The rest of the additional verses (4.0*5–4.0*16) are a motley collection on \emph{samādhi}, \emph{rājayoga}, the importance of the guru, dissolving the breath, \emph{suṣumṇā}, etc. In contrast to this, the \textalpha\ group begins with a cohesive discusson on absorption (4.1–3) that transitions to the gaze (4.4) and a brief discussion of \emph{śāmbhavī} and \emph{khecarī mudrā}s (4.5–4.8). The main topic of the chapter, which is meditating on the internal sound (\emph{nādānusandhāna}), begins at 4.12 in the \textalpha\ group (whereas in other groups it begins after fifty or so verses). The emphasis on \emph{nādānusandhāna} in the fourth chapter of the \textalpha\ group is consistent with the statement in verse 1.56 that \emph{nādānusandhāna} is the fourth component of Haṭhayoga.\lb

Its likely that the term \emph{nirañjanapada} was understood here as \emph{samādhi} because \emph{nirañjana} is included in a list of synonyms of \emph{samādhi} later in this chapter (4.32).\lb

The triad \emph{nāda}, \emph{bindu} and \emph{kalā} occurs in earlier works, in particular Śaiva Tantras, where it appears in contexts of enunciating mantras (\emph{mantroccāra}, e.g., \emph{Kubjikāmatatantra} 7.65, \emph{Jñānārṇavatantra} 2.4, \emph{Īśānaśivagurudevapaddhati} 18-110, etc.) and sometimes qualifies dieties (e.g., \emph{Parākhyatantra} 5.156ab) and gurus (e.g., \emph{Gurugītā} 64). The context can change the meaning of these terms, so we have chosen not to translate them. For a discussion of their various meanings, see \emph{Tāntrikābhidhānakośa} 2004 vol. 2: 68–73, 2013 vol. 3, 277–279.  

%nāda-bindu-kalā, referring to inner sound, generative fluid, and digits of the moon (or unmanīkalā) in the context of HY. Judit: kalā is like śakti (initial source)?
%Refer to TĀK

\end{philcomm}

\begin{metre}[hp04_000_1]
Anuṣṭubh (a: na-vipulā)
\end{metre}

%%%%%%%%%%
\subsection*{4.0*2}
\begin{translation}[hp04_000_2]
So now I will teach the best way to \emph{samādhi}. It destroys death, has an easy method and brings about the bliss of Brahman.
\end{translation}

\begin{sources}[hp04_000_2]
\emph{Gorakṣaśataka} 64
\begin{versinnote}
\tl{athedānīṃ pravakṣyāmi samādhikramam uttamam | \\+}
\tl{mṛtyughnaṃ tu sukhopāyair brahmānandakaraṃ sadā ||\\+}
\tl{\var{64c tu sukhopāyair ] T; sukhadopāyaṃ GU}\\!}
\end{versinnote}
\end{sources}

%\begin{testimonia}[hp04_000_2]
%\end{testimonia}

%\begin{philcomm}[hp04_000_2]
%It appears that the \emph{Gorakṣaśataka}'s verse was changed by whoever added it to reflect the discursive nature of the discussion on \emph{samādhi} that follows. By changing \emph{samādhikramam uttamam} to \emph{samādhikramalakṣaṇam} (and perhaps \emph{sadā} to \emph{param}) the verse foreshadows a broader discussion of the topic samādhi. But it doesn't really work, so adopt uttamam. [JB: We have uttamam now, so this note is not needed]
%\end{philcomm}

%%%%%%%%%%
\subsection*{4.0*3}
\begin{translation}[hp04_000_3]
The sovereign yoga (\emph{rājayoga}), meditative absorption (\emph{samādhi}), the beyond mind state (\emph{unmanī}), the transmental state (\textit{manonmanī}), [the sovereign yoga of] the lineage of immortals (\emph{amaraugha}), dissolution [of mind] (\emph{laya}), the [ultimate] reality (\emph{tattva}), void and not void (\textit{śūnyāśūnya}), the highest state (\emph{para pada}), [\dots]
\end{translation}

\subsection*{4.0*4}
\begin{translation}[hp04_000_4]
[\dots] no-mind (\emph{amanaska}), non-duality (\emph{advaita}), without support (\emph{nirālamba}), pure (\emph{nirañjana}), liberation in life (\emph{jīvanmukti}), innate (\emph{sahaja}) and the fourth [state] (\emph{turya}) are synonyms.
\end{translation}

%%%%%%%%%%
\subsection*{4.0*5}
\begin{translation}[hp04_000_5]
The unity of the self and mind arises in the same way that salt becomes identical with water through contact [with it]. That is called \emph{samādhi}.
\end{translation}

\begin{sources}[hp04_000_5]
\emph{Vivekamārtaṇḍa} 161
\begin{versinnote}
\tl{ambusaindhavayoḥ sāmyaṃ yathā bhavati yogataḥ |\\+}
\tl{tathātmamanasor aikyaṃ samādhiḥ so’bhidhīyate ||\\!}
\end{versinnote}
\end{sources}

\begin{testimonia}[hp04_000_5]
\emph{Haṭharatnāvalī} 4.1
\begin{versinnote}
\tl{salile saindhavaṃ yadvat sāmyaṃ bhavati yogavit |\\+}
\tl{tathātmamanasor aikyaṃ samādhiḥ so 'bhidhīyate ||\\!}
\end{versinnote}

\emph{Yuktabhavadeva} 11.29 (attr.~to Gorakṣanātha)
\begin{versinnote}
\tl{ambusaindhavayor aikyaṃ yathā bhavati yogataḥ |\\+}
\tl{tathātmamanasor aikyaṃ samādhir abhidhīyate ||\\!}
\end{versinnote}

\emph{Haṭhatattvakaumudī} 51.72
\begin{versinnote}
\tl{tad uktaṃ yogacandrikāyām –\\+}
\tl{ambusaindhavayor aikyaṃ yathā bhavati yogataḥ |\\+}
\tl{tathātmanasor aikyaṃ samādhiḥ sa vidhīyate ||\\+}
\tl{aikyaṃ abhinnatvam | yogataḥ yogābhyāsāt ||\\!}
\end{versinnote}
\end{testimonia}

%\begin{philcomm}[hp04_000_5]
%The change \emph{salile saindhavaṃ} appears to have occured because of the verb \emph{bhajati}, which is different to the \emph{Vivekamartaṇḍa}'s reading of \emph{ambusaindhavayor aikyaṃ [...] bhavati [...]}. However, since groups 2 and 3 support \emph{salile}, it is possible that Svātmārāma accepted it.
%SS sees this verse as a nirukti of samādhi using sāmyam
%\end{philcomm}

%%%%%%%%%%
\subsection*{4.0*6}
\begin{translation}[hp04_000_6]
The identity of the individual and universal selves is called \emph{samādhi}, in which all thoughts disappear.
\end{translation}

\begin{sources}[hp04_000_6]
\emph{Vivekamārtaṇḍa} 163
\begin{versinnote}
\tl{yat samatvaṃ dvayor atra jīvātmaparamātmanoḥ |\\+}
\tl{samastanaṣṭasaṃkalpaḥ samādhiḥ so’bhidhīyate ||\\!}
\end{versinnote}
\end{sources}

\begin{testimonia}[hp04_000_6]
\emph{Haṭharatnāvalī} 4.2
\begin{versinnote}
\tl{tat samatvaṃ bhaved atra jīvātmaparamātmanoḥ |\\+}
\tl{samastanaṣṭasaṃkalpaḥ samādhiḥ so 'bhidhīyate ||\\!}
\end{versinnote}

\emph{Yuktabhavadeva} 11.28 (attr.~to Gorakṣanātha)
\begin{versinnote}
\tl{yat sarvadvandvayor aikyaṃ jīvātmaparamātmanoḥ |\\+}
\tl{samastanaṣṭasaṃkalpaḥ samādhiḥ so 'bhidhīyate ||\\!}
\end{versinnote}

\end{testimonia}

%\begin{philcomm}[hp04_000_6]
%\end{philcomm}


%%%%%%%%%%
\subsection*{4.0*7}
\begin{translation}[hp04_000_7]
Who indeed truly knows the majesty of Rājayoga? From knowledge, liberation becomes steady [and] power (\emph{siddhi}) is obtained by means of the guru's teaching.
\end{translation}

\begin{sources}[hp04_000_7]
\emph{Amanaska} 2.5
\begin{versinnote}
\tl{rājayogasya māhātmyaṃ ko vā jānāti tattvataḥ |\\+}
\tl{jñānāt siddhir muktir iti guror jñānaṃ ca labhyate ||\\+}
\tl{\var{5c jñānāt siddhir muktir iti ] Bl Jb K Ma Va:
jñānāt siddhimuktir iti Pa Pc Tr Ua Ea:
jñānasiddhir muktir iti Vb Vd:
jñānān mukteḥ siddhir iti Ja:
jñānāt sidhyati muktir hi AllN (except Na Nm Ve Ea):
jñānāt sidhyate muktir hi Ve:
jñānāt sidhyanti muktiṃ hi Na:
jñānāt sidhyati muktiṃ hi Nm:
nānāsiddhir muktir iti Tha:
jñānasiddhir bhavaty eva Mb:
jñānasiddhimuktisiddhi Pb:
jñānāt siddhimuktor iti Je}\\!}
\end{versinnote}
\end{sources}

\begin{testimonia}[hp04_000_7]
\emph{Yogacintāmaṇi} f.~37v (citing the \emph{Rājayoga} [aka \emph{Amanaska}])
\begin{versinnote}
\tl{rājayogasya māhātmyaṃ ko hi jānāti tattvataḥ |\\+}
\tl{tajjñānī vasate yatra sadeśaḥ puṇyabhājanam ||\\!}
\end{versinnote}

%\emph{Yogamārgaprakāśikā} 4.11
%\begin{versinnote}
%\tl{rājayogasya māhātmyaṃ ko vā śaknoti varṇitum | \\+}
%\tl{yogasyāsya ca kartāro vijñeyās te maheśvarāḥ ||\\!}
%\end{versinnote}
\end{testimonia}

\begin{philcomm}[hp04_000_7]
The third verse quarter of this verse has been subject to much revision in the \emph{Haṭhapradīpikā} and the source text, the \emph{Amanaska}. Unlike the manuscripts of the \emph{Amanaska}, those of the \emph{Haṭhapradīpikā} transmit \emph{sthirā}, \emph{sthitiḥ} or \emph{sthite} after \emph{muktiḥ} or \emph{mukti}. We have adopted \emph{muktiḥ sthirā}, the reading of \textepsilon, an important group for the grey-scaled verses. 
%  attested by \etaTwo, 
% adopt muktisthitiḥ and rewrite note. 
% MD: The actual reading is muktiḥ sthirā.
\end{philcomm}

%%%%%%%%%%
\subsection*{4.0*8}
\begin{translation}[hp04_000_8]
Letting go of sense objects, seeing the truth, [and] realising the innate state are difficult without the compassion of a good guru.
\end{translation}

%\begin{sources}[hp04_000_8]
%\end{sources}

%\begin{testimonia}[hp04_000_8]
%\end{testimonia}

%\begin{philcomm}[hp04_000_8]
%No source?
%\end{philcomm}

%%%%%%%%%%
\subsection*{4.0*9}
\begin{translation}[hp04_000_9]
When the primal \emph{śakti} (i.e.~\emph{kuṇḍalinī}) has been woken up by means of the various postures, retentions and wonderful techniques [i.e.~\emph{mudrā}s], the breath dissolves into the void.
\end{translation}

%\begin{sources}[hp04_000_9]
%\end{sources}

\begin{testimonia}[hp04_000_9]
\emph{Yogacintāmaṇi}
\begin{versinnote}
\tl{haṭhapradīpikāyām—\\+}
\tl{vividhair āsanaiḥ kumbhair vicitrakaraṇair api |\\+}
\tl{prabuddhāyām ādiśaktau prāṇaḥ śūnye vilīyate ||\\!}
\end{versinnote}
\end{testimonia}

\begin{philcomm}[hp04_000_9]
In \emph{Jyotsnā} 4.10, Brahmānanda understands `the void' (\emph{śūnya}) as the central channel. In \emph{Haṭhapradīpikā} 3.4, \emph{śūnyapādavī} is a synonym of Suṣumnā.
\end{philcomm}

\begin{metre}[hp04_000_9]
Anuṣṭubh (c: ra-vipulā)
\end{metre}

%%%%%%%%%%
\subsection*{4.0*10}
\begin{translation}[hp04_000_10]
For the yogi whose Kuṇḍalinī has awakened and who has given up all activity, the innate state automatically shines forth.
\end{translation}

%\begin{sources}[hp04_000_10]
%\end{sources}

\begin{testimonia}[hp04_000_10]
\emph{Yogacintāmaṇi} f.~9r (attr.~to the \emph{Haṭhapradīpikā})
\begin{versinnote}
\tl{utpannaśaktibodhasya tyaktaniḥśeṣakarmaṇaḥ |\\+}
\tl{yoginaḥ sahajāvasthā svayam eva prajāyate ||\\!}
\end{versinnote}
\end{testimonia}

%\begin{philcomm}[hp04_000_10]
%\end{philcomm}

%

%%%%%%%%%%
\subsection*{4.0*11}
\begin{translation}[hp04_000_11]
When the breath is flowing in the central channel, and the mind enters the void, the expert destroys all actions.
\end{translation}
%?

% \begin{sources}[hp04_000_11]
% \end{sources}

\begin{testimonia}[hp04_000_11]
\emph{Yogacintāmaṇi} f.~9r (attr.~to the \emph{Haṭhapradīpikā})
\begin{versinnote}
\tl{suṣumṇāvāhini prāṇe śūnye viśati mārute |\\+}
\tl{tathā samastakarmāṇi nirmūlayati yogavit ||\\!}
\end{versinnote}

\emph{Upāsanāsārasaṅgraha} p.~66 (attr.~to the \emph{Haṭhapradīpikā})
\begin{versinnote}
\tl{suṣumṇāvāhini prāṇe śūnye viśati mānase |\\+}
\tl{tadā samastakarmāṇi nirmūlayati marmavit ||\\!}
\end{versinnote}
\end{testimonia}

%\begin{philcomm}[hp04_000_11]
%\emph{tathā} may connect the two locative absolutes.
%\end{philcomm}

%%%%%%%%%%
\subsection*{4.0*12}
\begin{translation}[hp04_000_12]
O Lord of the Immortals, homage to you. You have slain even death, into whose mouth this world, with everything that is moving and unmoving, has fallen.
\end{translation}
%?? MD: Should we perhaps read 'amaraugha' instead of 'amareśa'? Considering 4.11*30, it is possible for inanimate things to become objects of worship. And it may not be a coincidence that the next verse is a quotation from the Amaraugha.

\begin{sources}[hp04_000_12]
\end{sources}

\begin{testimonia}[hp04_000_12]
\end{testimonia}

%\begin{philcomm}[hp04_000_12]
%It appears that this verse, which is in manuscripts of the \textbeta\ and \textzeta\ groups, has become corrupt in manuscripts of \textgamma\, perhaps because of eye-skip (i.e., \emph{amano nirmanāḥ śūnyaṃ jagad etac carācaram}).
%

%JT: alt 3 is original, makes best sense, others lost pādas b and c and tried to correct. Adopt. [MD: done]
%Eye skip might explain how amarāya namas became amano nirmanaḥ śūnyaṃ jagad ... and the tadāmaraugha was dropped because it was difficult to understand or was corrupt.
%\end{philcomm}

%%%%%%%%%%
\subsection*{4.0*13}
\begin{translation}[hp04_000_13]
When equanimity has been obtained, and the breath is moving into the central channel, then the \emph{vajrolī} of the lineage of immortals arises †even for the person whose hope is for that†.
\end{translation}
%?? MD: tadâśā jīvite'pi ca? "then there is also the prospect of staying alive"?

\begin{sources}[hp04_000_13]
\emph{Amaraugha} 7
\begin{versinnote}
\tl{citte tu sattvam āpanne vāyau vrajati madhyame |\\+}
\tl{tadāmaraughavajrolis tadāśājīvitasya ca ||\\+}
\tl{\var{citte tu sattvam ] Ga, Ae (=\emph{Amaraugha}) : citte samatvam Ba Ad Ed : cittau samatvam Gb Ta.
tadāmaraughavajrolis ] \textit{conj.} :
tadāmaraughavajroli Ga :
tadāmaraudhavajrolī Ae :
eṣāmarauli vajroli Gb :
eṣāmarāli vajroli Ba :
eṣāmarauli vajrauli Ta :
yeṣāmarāli vajrauli Ad :
eṣāmarolī vajrolī Ed.
tadāśājīvitasya ca ] Ga, Ae (=\emph{Amaraugha}) :
tadā meni mameti ca Ba :
tadā mati mateti ca Ed :
mameti ca mameti ca Ad :
tadā meti mameti mateti ca Gb (\textit{unmetr.}) :
mateti mame++ Ta.
}\\!}
\end{versinnote}

\end{sources}

%\begin{testimonia}[hp04_000_13]
%\end{testimonia}

\begin{philcomm}[hp04_000_13]
The second line of this verse is problematic in both the source text (the \emph{Amaraugha}) and the manuscripts of the \emph{Haṭhapradīpikā}. For the last \emph{pāda}, we have adopted the closest reading to that of the source text (the \emph{Amaraugha}) but cannot make sense of it so have cruxed it. The line was rewritten in \etaTwo\ and other manuscripts, as well as the \emph{Jyotsnā} (4.14cd), to include the two variations of \emph{vajrolī}, namely \emph{amarolī} and \emph{sahajolī}.%
%C2 (Grp 7) tadāmarolī ... sahajolī prajāyate (=Jyotsnā)
\end{philcomm}

%%%%%%%%%%
\subsection*{4.0*14}
\begin{translation}[hp04_000_14]
How can knowledge exist, when the mind is alive, O goddess? So long as the breath lives, the mind does not die. Breath and mind: [when] this pair goes to dissolution, a man attains liberation; in no way [does any] other [man].
\end{translation}

\begin{sources}[hp04_000_14]
\emph{Candrāvalokana} 6
\begin{versinnote}
\tl{jñānaṃ kuto manasi jīvati devi yāvat \\+}
\tl{prāṇo pi jīvati mano mriyate na yāvat |\\+}
\tl{prāṇo mano dvayam idaṃ vilayaṃ prayāti\\+}
\tl{mokṣaṃ sa gacchati naro na kathaṃcid anyaḥ ||\\+}
\tl{\var{devi ] 7970 4340 T00788, kūdṛṣṭi D4345 (unmetr.)}\\!}
\end{versinnote}
\end{sources}

\begin{testimonia}[hp04_000_14]
\emph{Yogacintāmaṇi} f.~17r
\begin{versinnote}
\tl{atra śivavākyam |\\+}
\tl{jñānaṃ kuto manasi jīvati devi tāvat\\+}
\tl{prāṇo 'pi jīvati mano mriyate na yāvat |\\+}
\tl{prāṇo mano dvayam idaṃ vilayaṃ nayed yo \\+}
\tl{mokṣaṃ sa gacchati naro na kathañcid anyaḥ ||\\!}
\end{versinnote}

\emph{Haṭhasaṅketacandrikā} f.~2r
\begin{versinnote}
\tl{jñānaṃ kuto manasi jīvati durvikalpe \\+}
\tl{prāṇe 'pi jīvati mano mṛyate na yāvat |\\+}
\tl{prāṇo mano dvayam idaṃ na vilīyate 'tra \\+}
\tl{mokṣa na gacchati naro 'tra kathaṃ cid eva ||\\!}
\end{versinnote}
\end{testimonia}

%\begin{philcomm}[hp04_000_14]
%\end{philcomm}

\begin{metre}[hp04_000_14]
Vasantatilakā 
\end{metre}

%%%%%%%%%%
\subsection*{4.0*15}
\begin{translation}[hp04_000_15]
Both mercury and the mind are flighty by nature. When mercury is stabilised [or] the mind is stabilised, nothing in the world is impossible.
\end{translation}

%\begin{sources}[hp04_000_15]
%\end{sources}

\begin{testimonia}[hp04_000_15]
\emph{Yogacintāmaṇi} f.~103v
\begin{versinnote}
\tl{śāstrāntare śivavākyam—\\+}
\tl{rasasya manasaś caiva cañcalatvaṃ samāsataḥ |\\+}
\tl{raso baddho mano baddhaṃ kiṃ na sidhyati bhūtale ||\\!}
\end{versinnote}
\end{testimonia}

%\begin{philcomm}[hp04_000_15]
%LO: rasa implies bindu (and the next verse is about bindu when in the AS). But the next verse is more likely from the Rasārṇava (look at the context there).
%\end{philcomm}

%%%%%%%%%%
\subsection*{4.0*16}
\begin{translation}[hp04_000_16]
Stabilised, mercury and breath cure disease; stilled, they automatically bring back life; and bound they bestow the ability to fly, O Bhairavī.
\end{translation}

\begin{sources}[hp04_000_16]
\emph{Rasārṇava} 1.19
\begin{versinnote}
\tl{mūrchito harati vyādhiṃ mṛto jīvayati svayam |\\+}
\tl{baddhaḥ khecaratāṃ kuryāt raso vāyuś ca bhairavi ||\\!}
\end{versinnote}

Cf.~\emph{Amṛtasiddhi} 7.7
\begin{versinnote}
\tl{mūrcchito harate vyādhiṃ baddhaḥ khecaratāṃ nayet |\\+}
\tl{sarvasiddhikaro līno niścalo muktidāyakaḥ ||\\!}
\end{versinnote}
\end{sources}

\begin{testimonia}[hp04_000_16]
\emph{Yogacintāmaṇi} f.~103v (attr.~to \emph{śāstrāntare śivavākyam})
\begin{versinnote}
\tl{mūrchito harate vyādhiṃ mṛto jīvayati svayam |\\+}
\tl{baddhaḥ khecaratāṃ dhatte mano vāyuś ca bhairavi ||\\!}
\end{versinnote}
\end{testimonia}

\begin{philcomm}[hp04_000_16]
The \emph{Rasārṇava} is the likely source of this verse given the vocative, \emph{bhairavi}. Similar verses are found in other yoga texts (see Mallinson and Szántó 2021: 57).\lb

Disregarding the technical meaning of the terms, the verse plays with apparent contradictions (\emph{virodhābhāsa}) in saying literally that, when mercury and the breath are unconscious, they cure disease; when dead, they restore life and, when bound, they enable one to fly up.
\end{philcomm}

%%%%%%%%%%
\subsection*{4.1}
\begin{translation}[hp04_001]
%Now, \emph{samādhi}–
Mind is the master of the senses; breath is the master of the mind [and] dissolution [of the mind] is the master of the breath. [The yogi] should take refuge in that master, dissolution [of the mind].
\end{translation}
%

%\begin{sources}[hp04_001]
%\end{sources}

\begin{testimonia}[hp04_001]
\emph{Haṭharatnāvalī} 4.4
\begin{versinnote}
\tl{indriyāṇāṃ mano nātho manonāthas tu mārutaḥ |\\+}
\tl{mārutasya layo nāthaḥ sa layo nādam āśritaḥ ||\\!}
\end{versinnote}

\emph{Yogacintāmaṇi} f.~23r
\begin{versinnote}
\tl{haṭhapradīpikāyām—\\+}
\tl{indriyāṇāṃ mano nātho manonāthaś ca mārutaḥ |\\+}
\tl{mārutasya layo nāthaḥ sa layo nādam āśritaḥ ||\\!}
\end{versinnote}

\end{testimonia}

%\begin{philcomm}[hp04_001]
%
%\end{philcomm}

%%%%%%%%%%
\subsection*{4.1*1}
\begin{translation}[hp04_001_1]
Whether or not this [dissolution] is called liberation in another school, an extraordinary bliss from the dissolution of mind and breath arises in me.
\end{translation}
%\begin{sources}[hp04_001_1]
%\end{sources}

\begin{testimonia}[hp04_001_1]
\emph{Yogacintāmaṇi} f.~23r (attr.~to the \emph{Haṭhapradīpikā})
\begin{versinnote}
\tl{ayam eva tu mokṣākhyo astu vāpi matāntare |\\+}
\tl{manaḥprāṇalayo nādo na ca kaś cit vibhidyate ||\\!}
\end{versinnote}
\end{testimonia}

\begin{philcomm}[hp04_001_1]
The manuscripts attest two versions of the second line. It appears that the reading of \textbeta, \textepsilon\ and \textzeta\ groups, which we have adopted, was changed to remove the first person pronoun. The modified reading is prevalent in the \textgamma\ and \textdelta\ manuscripts (\emph{manaḥprāṇalayānando nāpi kaś cit vibhidyate}).  
\end{philcomm}
%  

%%%%%%%%%%
\subsection*{4.2}
\begin{translation}[hp04_002]
The yogis' motionless and unchanging dissolution [of mind], in which inhalation and exhalation have disappeared [and] perception of sense objects has ceased, is supreme.
\end{translation}

\begin{sources}[hp04_002]
\emph{Amanaska} 2.21
\begin{versinnote}
\tl{pranaṣṭocchvāsaniśvāsaḥ pradhvastaviṣayagrahaḥ |\\+}
\tl{niśceṣṭo nirgatārambho hy ānandaṃ yāti yogavit ||\\+}
\tl{\var{pranaṣṭocchvāsa° ] Jb Pa Va Bl Ja Ad AllSI  AllN, 
pranaṣṭasvāsa°  Mb,
pranaṣṭāsvāsa° Ua,
praṇaśocchvāsa° Ma,
pranaṣṭaḥ svāsa° Je Vb,
pranaṣṭo śvāsa°ḥ Pc,
prāṇastho śvāsa° Pb,
pranaṣṭe svāsa° K}\\!} % retained because we've read against alpha.
%\tl{\var{hy ānandaṃ yāti yogavit ]  AllNI (except  Ma Pa Pb Pc Tr Va),
%hy ānandaṃ yāti tattvavit Pa Tr Va Ea,
%svānandaṃ yāti yogavit Vd,
%hy ānandayati yoginaḥ Ma Tha,
%hy ānanday[a]ti yogī kaḥ Pc,
%hy ānanda iva yoginaḥ AllN (except  Na Nm Ea Ve),
%hy ānanda iva yogi Na Nm,
%hy ānandaś ca yoginaḥ Ve,
%hy ānandasukhayogavit Pb,
%layo jayati yoginaḥ Ad,
%layo jayati yoginām Cc}\\!} %retained to show that jayati was in the South Indian transmissin of the Amanaska.
\end{versinnote}
\end{sources}

\begin{testimonia}[hp04_002]
\emph{Yogacintāmaṇi} f.~27v (attr.~to the \textit{Rājayoga})
\begin{versinnote}
\tl{praṇaṣṭocchvāsaniḥśvāsa[ḥ] vidhvastaviṣayagrahaḥ |\\+}
\tl{niśceṣṭo nirgatārambho hy ānandayati yogikaḥ ||\\!}
\end{versinnote}

\emph{Haṭhasaṅketacandrikā} f.~117 (attr.~to the \emph{Haṭhapradīpikā})
\begin{versinnote}
\tl{pranaṣṭaśvāsaniḥśvāsaḥ pradhvastaviṣayajvaraḥ |\\+}
\tl{niśceṣṭo nirvikāraś ca layo jayati yoginā ||\\!}
\end{versinnote}
\end{testimonia}

\begin{philcomm}[hp04_002]
The fourth verse quarter of this verse differs from that of the North Indian recension of the \emph{Amanaska} (quoted as the source) but is found in two South Indian manuscripts of the \emph{Amanaska} (Ad and Cc).
\end{philcomm}

%%%%%%%%%%
\subsection*{4.3}
\begin{translation}[hp04_003]
May an extraordinary absorption reign supreme, in which all volition has been cut off and all activity ceased, intelligible [only] by means of itself and ineffable.%
%% My translation in the Amanaska (reading jāyate): [This] extraordinary absorption by which all volition has been cut off and in which all movement has ceased, is intelligible [only] by means of itself and is beyond the scope of words.
% According to the Amanaska's transmission, jayatāṃ (and its similar readings) arose at a later time.  
\end{translation}

\begin{sources}[hp04_003]
\emph{Amanaska} 2.22
\begin{versinnote}
\tl{ucchinnasarvasaṅkalpo niḥśeṣāśeṣaceṣṭitaḥ |\\+}
\tl{svāvagamyo layaḥ ko 'pi jāyate vāgagocaraḥ ||\\+}
\tl{\var{jāyate ] AllNI (except  Ma Pc), AllSI (except Ad Tha) Ea, jayatāṃ Ma Ad Nb Eb, jayatā Pc, jñāyatāṃ Tha, jagatāṃ AllN (except Nb Nu Ea Eb), japatāṃ Ca, layatāṃ Nu}\\!}
\end{versinnote}
\end{sources}

\begin{testimonia}[hp04_003]
\emph{Yogacintāmaṇi} f.~27v (attr.~to the \textit{Rājayoga})
\begin{versinnote}
\tl{ucchinnasarvasaṃkalpo nirgatāśeṣaceṣṭakaḥ |\\+}
\tl{svāvagamyo layaḥ ko 'pi jayatāṃ vāgagocaraḥ ||\\!}
\end{versinnote}
\end{testimonia}

%\begin{philcomm}[hp04_003]
%\end{philcomm}

%%%%%%%%%%
\subsection*{4.4}
\begin{translation}[hp04_004]
Dissolution, which is eternal in the elements and senses, occurs where the gaze is. The gaze that has dissolved into the focus becomes the power of living beings.%%
%
\end{translation}


\begin{sources}[hp04_004]
\emph{Candrāvalokana} 8cd–9ab
\begin{versinnote}
\tl{yatra dṛṣṭir manas tatra bhūtendriyasanātanaṃ |\\+}
\tl{sā śaktis sarvabhūtānāṃ dṛṣṭir lakṣye layaṃ gatā ||\\+}
\tl{\var{sā śaktis ] 4340 4345 T00788, sa śāntis  7970}\\!}
\end{versinnote}

\emph{Kaulajñānanirṇaya} 3.2
\begin{versinnote}
\tl{yatra dṛṣṭir manas tatra bhūtendriya sapudgalaḥ \\+} 
\tl{svaśaktijīvabhūtā hi dṛṣṭilakṣair layaṃ gatāḥ \\+}
\tl{\var{°bhūtā hi ] conj. Hatley, °bhūtāni ABCed}\\!} 
% 
\end{versinnote}
\end{sources}

\begin{testimonia}[hp04_004]
\emph{Yogacintāmaṇi} f.~24r
\begin{versinnote}
\tl{rājayoge—\\+}
\tl{yatra dṛṣṭir layas tatra bhūtendriyasanātanī |\\+}
\tl{syāc chaktiḥ sarvabhūtānāṃ dṛṣṭir lakṣeṇa saṃgatā ||\\!}
\end{versinnote}

\emph{Haṭhasaṅketacandrikā} f.~128r
\begin{versinnote}
\tl{tad uktaṃ haṭhapradīpikāyām– \\+}
\tl{yatra dṛṣṭir layas tatra bhūtendriyasanātanī |\\+}
\tl{syāc chaktiḥ sarvabhūtānāṃ dṛṣṭir lakṣeṇa saṃgatā ||\\+}
\tl{\var{syāc chaktiḥ sarva°] B220, syārā śakti serva° 2244}\\+}
\tl{\var{lakṣeṇa saṃgatā ] 2244, lakṣyakṣayaṃ gatā B220}\\!}
\end{versinnote}

\emph{Haṭhatattvakaumudī} 49.25
\begin{versinnote}
\tl{tathā coktaṃ haṭhapradīpikāyām –\\+}
\tl{yatra dṛṣṭir layas tatra bhūtendriyasanātanaḥ |\\+}
\tl{na sā śaktir bhāvabhūtānāṃ dṛṣṭe lakṣe kṣayaṃ gatā ||\\!}
\end{versinnote}
\end{testimonia}

%\begin{philcomm}[hp04_004]
%Adopt dṛṣṭir lakṣye layaṃ gatā. [MD: Done.]
%PYŚ: 2.18 on bhūtendriya and drś
%\end{philcomm}

%%%%%%%%%%
\subsection*{4.5}
\begin{translation}[hp04_005]
The Vedas, Shastras and Puranas are like common prostitutes. \emph{Śāmbhavī mudrā} is unique and guarded like a woman of a good family.
\end{translation}

\begin{sources}[hp04_005]
\emph{Amanaska} 2.9
\begin{versinnote}
\tl{vedaśāstrapurāṇāni sāmānyagaṇikā iva |\\+}
\tl{ekaiva śāmbhavī mudrā guptā kulavadhūr iva ||\\!}
%\tl{\var{vedaśāstrapurāṇāni ] AllNI (except Jb Mb) AllSI AllN (except Nb Nu), vedaśāstrapurāṇādyāḥ Mb, vedaśāstrāṇi sarvāṇi Jb: vedaśāstrapurāṇādi Nb Nu}\\!}
\end{versinnote}
\end{sources}

\begin{testimonia}[hp04_005]
\emph{Haṭharatnāvalī} 4.27
\begin{versinnote}
\tl{vedaśāstrapurāṇāni sāmānyagaṇikā iva |\\+}
\tl{ekaiva śāmbhavī mudrā guptā kulavadhūr iva ||\\!}
\end{versinnote}

\emph{Yogacintāmaṇi} f.~24v (attr.~to the \textit{Rājayoga})
\begin{versinnote}
\tl{vedaśāstrapurāṇaughāḥ sāmānyagaṇikā iva |\\+}
\tl{ekaiva śāmbhavī mudrā guptā kulavadhūr iva ||\\!}
\end{versinnote}

\emph{Haṭhatattvakaumudī} 49.26 (attr.~to the \emph{Haṭhapradīpikā})
\begin{versinnote}
\tl{vedaśāstrapurāṇāni sāmānyagaṇikā iva |\\+}
\tl{ekaiva śāmbhavī mudrā sarvatantreṣu gopitā || 49.26\\!}
\end{versinnote}

%\emph{Vimalāgama} 73-005
%\begin{versinnote}
%\tl{skānde-\\+}
%\tl{vedaśāstrapurāṇādyāḥ aṣṭā veśy[ā]ṅganā iva |\\+}
%\tl{yā punaḥ śāṅkarī vidyā guptā kulavadhūr iva ||\\!}
%\end{versinnote}
\end{testimonia}

\begin{philcomm}[hp04_005]
See Birch 2013: 286 for more parallel verses.
\end{philcomm}

%%%%%%%%%%
\subsection*{4.6}
\begin{translation}[hp04_006]
The focus is internal, the gaze external, unblinking: this is the \emph{śāmbhavī mudrā} concealed in all the Tantras.%

\end{translation}

\begin{sources}[hp04_006]
\emph{Amanaska} 2.10
\begin{versinnote}
\tl{antar lakṣyaṃ bahir dṛṣṭir nimeṣonmeṣavarjitā |\\+}
\tl{eṣā hi śāmbhavī mudrā sarvatantreṣu gopitā ||\\+}
\tl{\var{antar lakṣyaṃ ] AllNI (except Ja K Pb Tr Ua Vb) Tha Vd, antarlakṣya Ad Cc K: antarlakṣaṃ Tr Nw Ve: antarlakṣā Ja Ua Ea Eb, antarlakṣo Na Nq Nu, antarlakṣyo Nb Nm: antaryogaṃ Vb: antarlīno Pb}\\!}%JM: delete variants?
% JB: the lakṣyaṃ/lakṣa spelling is an issue in both the Amanaska and HP mss (inlcuding alpha).
\end{versinnote}

\emph{Candrāvalokana} 1
\begin{versinnote}
\tl{antar lakṣyaṃ bahir dṛṣṭir nimeṣonmeṣavarjitaḥ |\\+}
\tl{iyaṃ sā śāṃbhavī mudrā sarvatantreṣu gopitā |\\!}
\end{versinnote}
\end{sources}

\begin{testimonia}[hp04_006]
\emph{Yogacintāmaṇi} f.~24v (attr.~to the \textit{Rājayoga})
\begin{versinnote}
\tl{antar lakṣyaṃ bahir dṛṣṭir nimeṣonmeṣavarjitā |\\+}
\tl{eṣā tu śāmbhavī mudrā sarvaśāstreṣu gopitā ||\\!}
\end{versinnote}
\end{testimonia}

\begin{philcomm}[hp04_006]
As seen in the witnesses of the \emph{Amanaska}, the source text, the spellings \emph{lakṣya} and \emph{lakṣa} occur randomly in the manuscripts. See Birch 2013: 287 for more parallel verses.
\end{philcomm}

%%%%%%%%%%
\subsection*{4.7}
\begin{translation}[hp04_007]
When the yogi's mind and breath have dissolved in the internal focus while he is looking outwards and down with a gaze in which the pupils are unmoving, even though he's not looking [at anything], this indeed is \emph{khecarīmudrā}. O guru, it manifests because of your favour and is that reality which is Śambhu's state, free from what is void and not void.
\end{translation}


\begin{sources}[hp04_007]
\emph{Candrāvalokana} 2
\begin{versinnote}
\tl{antarlakṣyavilīnacittapavano yogī yadā vartate \\+}
\tl{dṛṣṭyā niścalatārayā bahir adhaḥ paśyan na paśyet sadā |\\+}
\tl{mudreyaṃ khalu khecarī bhavati sā yuktaprasādāt guroḥ\\+}
\tl{śūnyāśūnyavivarjitaṃ sphurati yat tattvaṃ padaṃ śāṃbhavaṃ ||\\+}
\tl{\var{°pavano ] 4340 4345 T00788, °pavane 7970}\\+}
\tl{\var{sā yukta°] 4340 4345, sā yuktā  T00788, satyuṣṭa° 7970}\\+}
\tl{\var{prasādād guroḥ ] 4340 4345 T00788, prasādāt guro 7970}\\+}
\tl{\var{yat tattvaṃ ] 7970 4340 4345, cet tatvaṃ T00788}\\!}
\end{versinnote}
\end{sources}

\begin{testimonia}[hp04_007]
\emph{Yogacintāmaṇi} f.~24v
\begin{versinnote}
\tl{haṭhapradīpikāyām—\\+}
\tl{antarlakṣyavilīnacittapavano yogī yadā vartate\\+}
\tl{dṛṣṭyā niścalatārayā bahir asau paśyann apaśyann api |\\+}
\tl{mudreyaṃ khalu śāṃbhavī bhavati sā yuṣmatprasādād guro\\+}
\tl{śūnyāśūnyavivarjitaṃ sphurati yat tatvaṃ padaṃ śāṃbhavam ||\\!}
\end{versinnote}

\emph{Anubhavanivedana} 1
\begin{versinnote}
\tl{antarlakṣyavilīnacittapavano yogī yadā vartate\\+}
\tl{dṛṣṭyā niścalatārayā bahir asau paśyann apaśyann api |\\+}
\tl{mudreyaṃ khalu śāmbhavī bhavati sā yuṣmatprasādād guro\\+}
\tl{śūnyāśūnyavivarjitaṃ bhavati yat tattvaṃ padaṃ śāmbhavam ||1 ||\\!}
\end{versinnote}

\emph{Haṭhatattvakaumudī} 49.27 (attr.~to the \emph{Haṭhapradīpikā})
\begin{versinnote}
\tl{antarlakṣavilīnacittapavano yogī yadā varttate\\+}
\tl{dṛṣṭyā niścalatārayā bahir adhaḥ paśyann apaśyann api |\\+}
\tl{mudreyaṃ khalu śāmbhavī bhavati sā yuṣmatprasādāt guroḥ\\+}
\tl{śūnyāśūnyavivarjitaṃ sphurati yat tatvaṃ padaṃ śāmbhavam ||\\!}
\end{versinnote}
\end{testimonia}

\begin{philcomm}[hp04_007]
On the similarity of \emph{śāmbhavī} and \emph{khecarī mudrās} in 4.5–7, see the note on 4.8. 
\end{philcomm}

\begin{metre}[hp04_007]
Śārdūlavikrīḍita 
\end{metre}

%%%%%%%%%%
\subsection*{4.8}
\begin{translation}[hp04_008]
There is no difference between the states of \emph{śāmbhavī} and \emph{khecarī}.
\end{translation}
% Adopt avasthāyām abhedatā (N10, N12, T11) (Cf.~avasthāyām abhedataḥ C6)
%
% śrīśāmbhavyāś ca khecaryā avasthāyām abhedataḥ |M1,J10
% bhavec cittalayākārā unmanī citsvarūpiṇī | J10
% The beyond-mind [state], whose form is the dissolution of mind and own nature is consciousness, is identical in the state of Śāmbhavī and Khecarī.

%\begin{sources}[hp04_008]
%\end{sources}

\begin{testimonia}[hp04_008]
\emph{Yogacintāmaṇi} f.~25r (attr.~to the \emph{Haṭhapradīpikā})
\begin{versinnote}
\tl{śrīśāṃbhavyā khecaryāś ca avasthāṃ ca labhed ataḥ |\\+}
\tl{tāre jyotiṣi saṃyojya kiñ cid uccālayed bhruvau ||\\!}
\end{versinnote}

\emph{Haṭhasaṅketacandrikā} f.~128r–128v (attr.~to the \emph{Haṭhapradīpikā})
\begin{versinnote}
\tl{tāre jyotiṣi saṃyojya kiṃ cid unnamayed bhuvau [|]\\+}
\tl{śrīśāṃbhavyāś ca khecaryā avasthā[ṃ] tu labhed ataḥ ||\\+}
\tl{\var{unnamayed ] 2244, saṃcālayed B220}\\+}
\tl{\var{unmanīkaraṇaṃ kṣaṇāt ] 2244, unmanīkāralakṣaṇam B220}\\+}
\tl{\var{tu labhed ataḥ ] 2244, labhate daśa B220}\\!}
\end{versinnote}

\end{testimonia}

\begin{philcomm}[hp04_008]
This line may be authorial because it is explaining the fact that 4.6 and 4.7 teach similar techniques called \emph{śāmbhavī mudrā} and \emph{khecarīmudrā} respectively. One would expect a verse on \emph{khecarī mudrā} to be explaining the insertion of the tongue in the nasopharyngeal cavity (cf.~3.33–48), but the practice of meditating by fixing the gaze is called \emph{khecarī mudrā} in 4.7 (as attested by \textalpha, \textbeta\ and \texteta\ manuscripts) and manuscripts of the \emph{Candrāvalokana}, the text from which Svātmārāma borrowed this verse. It is, therefore, likely that he added 4.8 to explain that \emph{śāmbhavī} and \emph{khecarī mudrās} are the same with respect to the gaze. A similar conflation occurs in the \textit{Śivayogapradīpikā} (5.3).
% NJ: Maybe mention Śivayogapradīpikā 5.3 and 5.4 
%% 5.3ab lakṣyalīnamanasānilena yo vartate 'calitatārako bhavet |
%% 5.3cd khecarīyam atha saiva śāṃbhavīmudrayā sakṛtayā jagadguruḥ 
%% (5.3) For one whose eyes remain unmoving [while open], with the mind and breath dissolved in the [inner] lakṣya, this khecarī indeed is śāmbhavī. Through the proper practice of this mudrā, he shall be the guru of the world.
%% 5.4ab loke 'antaḥkhecarī mudrām ajñātvā bāhyakhecarīm
%% 5.4cd avalambya tvarāḥ kecij jihvāchedanakarmaṇā
%% (5.4) Not knowing the inner khecarīmudrā in this world, some have hastily taken to the external khecarī through the practice of cutting the tongue.
%%%%
%Judit would greyscale 4.8, thinking that it is a marginal note. And greyscale the pātāle verses.
%avasthā: Brahmānanda explains this as referring to the place (dhāman) of khecarī (the middle of the brow) and śāmbhavī (the heart).
\end{philcomm}

%%%%%%%%%%
\subsection*{4.9}
\begin{translation}[hp04_009]
That which enters the aperture into the underworld, which exists at the base of Meru, the wise yogin say that is the truth, the source of [all] rivers. The essence of the body flows from the moon. Because of that, death arises for people. One should dam that [essence] with the earth that is the good \textit{mudrā}. Otherwise, there is no bodily perfection.
\end{translation}
% See 3.49*2

% The commentary has been moved from the third chapter. Should be modified.

\begin{testimonia}[hp04_009]
\emph{Haṭharatnāvalī} 4.30
\begin{versinnote}
\tl{pātāle yad viśati suṣiraṃ merūmūlaṃ tad asti\\+}
\tl{tattvaṃ caitad vadati sudhā tanmukhaṃ nimnagānām |\\+}
\tl{candrāt sāraṃ sravati vapuṣas tena mṛtyur narāṇāṃ \\+}
\tl{tad badhnīyāt sukharatimṛdur nānyathā kāryasiddhiḥ ||\\!}
\end{versinnote}
% after (ekaṃ sṛṣṭimayaṃ + mano yatra vilīyeta) and before (śarīraṃ tāvad eva)
%the context of this verse in the HR is rājayoga.

\emph{Yogacintāmaṇi} f.~75r (attr.~to the \emph{Haṭhapradīpikā})
\begin{versinnote}
\tl{tat pātālād viyati śikhare merumūle tad asti \\+}
\tl{tattvaṃ caitat pravadati sudhīḥ saṃmukhe nimnagānām |\\+}
\tl{candrāt sāraḥ sravati vapuṣas tena mṛtyur narāṇām \\+}
\tl{tad badhnīyāt svakaraṇamṛdā nānyathā kāyasiddhiḥ ||\\!}
\end{versinnote}
% after the verse (cittāyattaṃ nṛṇāṃ śukraṃ) and before (indhanāni yathā vahnis)
% context is khecarī (3rd chapter material)


\emph{Yuktabhavadeva} 7.220 (attr.~to Gorakṣanātha)
\begin{versinnote}
\tl{pātāle yad viśati suṣiraṃ merumūle yad asti \\+}
\tl{tadvac caitat pravadati sudhīs tanmukhaṃ nimnagānām |\\+}
\tl{candrāt sāraḥ sravati vapuṣas tena mṛtyur narāṇām \\+}
\tl{badhnīyāt tat sukaraṇam atho nānyathā kāyasiddhiḥ ||\\!}
\end{versinnote}
% context is khecarī, (3rd chapter material)
% after (ekaṃ sṛṣṭimayaṃ bījaṃ) and before (atha uḍḍiyānabandhaḥ)

\emph{Haṭhatattvakaumudī} 14.26 (attr.~to the \emph{Haṭhapradīpikā})
\begin{versinnote}
\tl{pātāle yad vitatasuṣiraṃ merumūle tad asmin \\+}
\tl{tadvac caitat pravadati sudhīs tanmukhaṃ nimnagānām |\\+}
\tl{candrāt sāraḥ sravati vapuṣas tena mṛtyur narāṇāṃ \\+}
\tl{taṃ badhnīyāt sukaraṇamṛdā nānyathā kāyasiddhiḥ ||\\!}
\end{versinnote}
% context is khecarī 3rd chapter material
% after (cumbantī yadi lambikā) and before (cittaṃ vicarati gagane)

\end{testimonia}

\begin{philcomm}[hp04_009]
%Workshop notes
% Sven: the first line could refer to Kuṇḍalinī
% J5 has this verse here.
% mṛdā = using clay to make a dam?
% Also see Netratantra 7.32
% Could this verse be about mūlabandha? (reading pātāle and mūle)
%J5: yat prāleyam pihitasukhire merumūle yad astī — this seems helpful.
% PN has a nice-ish reading: yat prāleyaṃ pihita sukhiraṃ merumūrdhnaṃ. mūrdhni etc. is much better.
% J10 has merumūrdhni sthitaṃ yat, which is good.
% If going with prāleyam, adopt yat prāleyam pihitasuṣiraṃ merumūrdhny asti tathyaṃ (~K1 Grp2)
% ... tasmin tattvaṃ pravadati
%Haru: Perhaps, praleyam was corrupted to pātāle and then the rest of the line changed.


% α1,δ1 - in Ch. 4 only
% β1,β2 - after 3.39
% all the other mss - at 3.49*2
% among them α2,α3,gamma have it in both Ch. 3 and 4
On the position of these verses in the various recensions of the text, see the introduction (ref??).\lb

%This verse occurs in various places in the different recensions of the text. It is here in chapter four in the \textalpha\ and \textgamma\ groups (\alphaTwo\ and \alphaThree\ also have it in the third chapter at 3.49*2). It occurs after 3.39 in \betaOne\ and \betaTwo, and in both the third and fourth chapters in the \textgamma\ manuscripts. It must have entered the transmission of the \emph{Haṭhapradīpikā} at an early stage, and has moved around, perhaps because the name of the technique alluded to is not stated and the meaning of the first half of the verse is somewhat vague without its original context (we are yet to identify a source).

The first half of the verse has many variants among the witnesses. \alphaTwo\ has:
\begin{versinnote}
\tl{yat prāleyaṃ pihitasukhire merumūle yad astī \\+}
\tl{tasmiṃs tattvaṃ pravadati sudhīs tan mukhaṃ nimnagānām\\!}
\end{versinnote}
This version is close to that of other important manuscripts in the \textbeta, \textgamma\ and \texteta\ groups. With the help of these manuscripts (in particular \getsiglum{K1} of the \textgamma\ group), the above can be emended and understood as follows:
\begin{versinnote}
\tl{yat prāleyaṃ pihitasuṣiraṃ merumūrdhny asti tathyaṃ \\+}
\tl{tasmiṃs tattvaṃ pravadati sudhīs tan mukhaṃ nimnagānām\\+}
\tl{\var{°suṣiraṃ ] \getsiglum{K1}, °sukhire \alphaTwo. °mūrdhny ] \getsiglum{P8}. °mūle \alphaTwo. asti tathyaṃ ] \getsiglum{K1}, yad astī \alphaTwo.}\\!}
“That cool liquid by which the aperture is filled at the top of Meru and exists as the truth, the wise [yogi] says that is the source of [all] rivers.” 
\end{versinnote}

% Other manuscripts have something more similar to:
% \begin{versinnote}
% \tl{pātāle yad viśati suṣiraṃ merumūlaṃ tad asti \\+}
% \tl{tattvaṃ caitat pravadati sudhīs tan mukhaṃ nimnagānām\\!}
% \end{versinnote}

\end{philcomm}

% MD: It is most likely that these and two other stanzas were here first and were moved to the 3rd chapter by the person who reorganised the HP. I would therefore suggest reactivating these stanzas here. Agreed.

\begin{metre}[hp04_009]
Mandākrāntā 
\end{metre}

%%%%%%%%%%
\subsection*{4.10}
\begin{translation}[hp04_010]
The sun devours whatever nectar flows from the divine moon. As a result, the body is afflicted by old age.
\end{translation}

%\begin{sources}[hp04_010]
%\end{sources}

\begin{testimonia}[hp04_010]
\emph{Haṭharatnāvalī} 2.72 
\begin{versinnote}
\tl{atha viparītakaraṇī–\\+}
\tl{yat kiñ cit sravate candrād amṛtaṃ divyarūpi ca |\\+}
\tl{tatsarvaṃ grasate sūryas tena piṇḍaṃ vināśi ca ||\\!}
\end{versinnote}
 
\emph{Yogacintāmaṇi} f.~77v
\begin{versinnote}
\tl{haṭhapradīpikāyām—\\+}
\tl{yat kiṃ cin sravate candrād amṛtaṃ divyarūpi ca |\\+}
\tl{tat sarvaṃ grasate sūryas tena piṇḍaṃ vināśi ca ||\\!}
\end{versinnote}

Cf.~\emph{Haṭhayogasaṃhitā} 38 (p. 26) % Gheraṇḍasaṃhitā 3.33
\begin{versinnote}
\tl{nābhimūle vaset sūryas tālumūle ca candramāḥ |\\+}
\tl{amṛtaṃ grasate sūryas tato mṛtyuvaśo naraḥ ||\\!}
\end{versinnote}
\end{testimonia}

\begin{philcomm}[hp04_010]
%No source? %?? MD: VM 113-114 (Jim's working edition)
% nābhideśe vasaty eva bhāskaro dahanātmakaḥ |
% amṛtātmā sthito nityaṃ tālumadhye ca candramāḥ || 113 ||
% varṣaty adhomukhaś candro gṛhṇāty ūrdhvamukho raviḥ |
% jñātavyaṃ kāraṇaṃ tatra yena pīyūṣam āpyate || 114 ||
%(cited also in Jyotsnā 3.77)

%The term \emph{piṇḍa} is rarely neuter, hence the change to the masculine in the \emph{Jyotsnā}.

%are found in alpha in ch.4, but should they be there? They are very physical and not suited to ch.4. But here they are unusual because they are preamble and the practices already described do not have preamble. Perhaps drop them altogether. But then there is no explanation of how VK works.

%Supposedly these verses were originally here as in the alpha mss, first moved to the end of the Viparītakaraṇī section (as in the epsilon), and finally moved to the beginning of the same section (as in many other groups).
\end{philcomm}

%%%%%%%%%%
\subsection*{4.11}
\begin{translation}[hp04_011]
There is a divine bodily position for this, which blocks the mouth of the sun. It is to be known from the teaching of a guru and not through the countless interpretations of scriptures.
\end{translation}
% greyscaled

%\begin{sources}[hp04_011]
%\end{sources}

\begin{testimonia}[hp04_011]
Haṭharatnāvali 2.73 (on \emph{viparītakaraṇī})
\begin{versinnote}
\tl{tatrāsti divyaṃ karaṇaṃ sūryasya mukhabandhanam |\\+}
\tl{gurūpadeśato jñeyaṃ na tu śāstrārthakoṭibhiḥ || \\!}
\end{versinnote}

\emph{Yogacintāmaṇi} f.~77v (attr.~to the \emph{Haṭhapradīpikā})
\begin{versinnote}
\tl{tatrāsti karaṇaṃ divyaṃ sūryasya mukhabandhanam |\\+}
\tl{gurūpadeśato jñeyaṃ na tu śāstrārthakoṭibhiḥ ||\\!}
\end{versinnote}

%Yogakarṇikā 147c–148b
%\begin{versinnote}
%\tl{tat sarvaṃ grasate sūryas tena piṇḍo jarāyutaḥ || 147 ||\\+}
%\tl{tatrāsti karaṇaṃ divyaṃ sūryasya mukhavañcanam |\\!}
%\end{versinnote}
\end{testimonia}

\begin{philcomm}[hp04_011]
%See the note on the previous verse.
%No source? Only Brahmānanda reads mukhavañcanam, which makes better sense. Judit; bandhanaṃ makes sense with grasate sūryas
\end{philcomm}

%%%%%%%%%%
\subsection*{4.11*1}
\begin{translation}[hp04_011_1]
[The yogi] should fix the pupils on a light and slightly raise the eyebrows. This is the way of the preliminary yoga, which immediately brings about the beyond-mind state.
\end{translation}
% unmanīkārakaḥ ?

\begin{sources}[hp04_011_1]
\emph{Amanaska} 1.8 (South Indian Recension)
\begin{versinnote}
\tl{netre jyotiṣi saṃyojya kiṃ cid unnamayed bhruvau |\\+}
\tl{pūrvayogasya mārgo 'yam unmanīkārakaḥ kṣaṇāt ||\\+}
\tl{\var{unnamayed ] Cb Vd: unnamayan Bb Tha: unmīlaye W: unmilayet Uc: unmanaya Pe: unmīlya Cc}\\!}
%\tl{\var{unmanīkārakaḥ kṣaṇāt ] Bb Tha W Vd Pe: unmunikārakakṣaṇāt Uc: unmanīkāraṇaṃ kṣaṇāt Cc: unmanākāśakaḥ kṣaṇāt Ad Ca: unmanākāśakakarṣaṇāt Cb}\\!} 
\end{versinnote}
\end{sources}

\begin{testimonia}[hp04_011_1]
\emph{Yogacintāmaṇi} f.~25r (attr.~to the \emph{Haṭhapradīpikā})
\begin{versinnote}
\tl{tāre jyotiṣi saṃyojya kiṃ cid uccālayed bhruvau |\\+}
\tl{pūrvayogasya mārgo 'yam unmanīkaraṇaḥ kṣaṇāt ||\\!}
\end{versinnote}

\emph{Haṭhasaṅketacandrikā} ff. 128r-128v (attr.~to the \emph{Haṭhapradīpikā})
\begin{versinnote}
\tl{tāre jyotiṣi saṃyojya kiṃ cid unnamayed bhuvau |\\+}
\tl{pūrvayogasya mārgo [']yam unmanīkaraṇaṃ kṣaṇāt ||\\!}
\end{versinnote}

%Rājayogāmṛta 3.7c-d $-$ 3.8a-b
%\begin{versinnote}
%\tl{tārāj jyotiṣi saṃyojya kiṃcid unmīlayed bhruvau |\\+}
%\tl{pūrvayogasya mārgo 'yaṃ unmanitārakaṃ kṣaṇāt ||\\!}
%\end{versinnote}
\end{testimonia}

\begin{philcomm}[hp04_011_1]
This and the next verse (4.11*1–2) are from the South Indian recension of the \emph{Amanaska}, a later rewriting of that work which incorporates these verses in an additional passage on Tārakayoga, added here to elaborate on the meditation of fixing the gaze that is taught in 4.4–7. 
\end{philcomm}

%%%%%%%%%%
\subsection*{4.11*2}
\begin{translation}[hp04_011_2]
Some are confused by a multitude of tantric texts, some by a mass of vedic texts and some by reasoning. They do not know what causes one to cross over (\emph{tārakam}).%
\end{translation}


\begin{sources}[hp04_011_2]
\emph{Amanaska} 1.11 (South Indian Recension)
\begin{versinnote}
\tl{ke cid āgamajālena ke cin nigamasaṅkulaiḥ |\\+}
\tl{ke cit tarkeṇa muhyanti naiva jānanti tārakam ||\\!}
\end{versinnote}
\end{sources}

\begin{testimonia}[hp04_011_2]
\emph{Yogacintāmaṇi} f.~25r (attr.~to the \emph{Haṭhapradīpikā})
\begin{versinnote}
\tl{ke cid āgamajālena ke cin niyamasaṃkulāḥ |\\+}
\tl{ke cit tarkeṇa muhyanti naiva jānanti tārakam ||\\!}
\end{versinnote}

\emph{Haṭhatattvakaumudī} 49.29 (attr.~to the \emph{Haṭhapradīpikā})
\begin{versinnote}
\tl{ke cid āgamajālena ke cin nigamasaṃkule |\\+}
\tl{ke cit tarkeṇa muhyanti naiva jānanti tārakam ||\\!}
\end{versinnote}
\end{testimonia}

\begin{philcomm}[hp04_011_2]
In the source text \emph{tārakam} refers to Tāraka yoga, one of two yogas taught in the South Indian recension of the \emph{Amanaska}.\lb

On why this verse is in greyscale, see the note to 4.11*1.    
\end{philcomm}

%%%%%%%%%%
\subsection*{4.11*3}
\begin{translation}[hp04_011_3]
By leading the moon and sun to dissolution in a motionless state, the [yogi], his eyes half open, mind steady, and gaze placed at the tip of the nose, attains the supreme reality (\textit{vastu}), the state that is the highest principle (\textit{tattva}), 
% attains the state of the highest reality, the supreme essence (\textit{vastu}), 
whose form is light and which is devoid of anything external and shining intensely. What more is to be said here?
\end{translation}

\begin{sources}[hp04_011_3]
\emph{Candrāvalokana} 3
\begin{versinnote}
\tl{ardhodghāṭitalocana[ḥ] sthiramanā nāsāgradattekṣaṇaś\\+}
\tl{candrārkāv api līnatām upagatau niṣpandarūpaṃ vapuḥ |\\+}
\tl{jyotīrūpam aśeṣabāhyarahitaṃ dedīpyamānaṃ paraṃ\\+}
\tl{tatvaṃ tatpadam eti yat tu paramaṃ vācyaṃ kim atrādhikaṃ ||\\+}
\tl{\var{sthiramanā ] 4340 4345 T00788, sphuramanā 7970}\\+}
\tl{\var{nāsāgradattekṣaṇaś ] 7970 4345 T00788, nāsāgradaṭhattakṣaṇaś 4340}\\+}
\tl{\var{tatvaṃ tatpadam eti yat tu] 4340 4345 T00788, tatvaṃ-n-tatpadam eti yat kṛ 7970}\\!}
\end{versinnote}
\end{sources}

\begin{testimonia}[hp04_011_3]
\emph{Yogacintāmaṇi} ff. 24v–25r (attr.~to the \emph{Haṭhapradīpikā})
\begin{versinnote}
\tl{ardhodghāṭitalocanaḥ sthiramanā nāsāgradattekṣaṇaḥ\\+}
\tl{candrārkāv api līnatām upanayen niḥspandabhāvottare |\\+}
\tl{jyotīrūpam aśeṣabāhyarahitaṃ dedīpyamānaṃ paraṃ  \\+}
\tl{tatvaṃ tat padam eti vastu paramaṃ vācyaṃ kim atrādhikam ||\\!}
\end{versinnote}

\emph{Anubhavanivedana} 2
\begin{versinnote}
\tl{ardhodghāṭitalocanaḥ sthiramanā nāsāgradattekṣaṇaś\\+}
\tl{candrārkāv api līnatām upagatau trispandabhāvāntare |\\+}
\tl{jyotīrūpam aśeṣabāhyarahitaṃ caikaṃ pumāṃsaṃ param\\+}
\tl{tattvaṃ tatpadam eti vastu paramaṃ vācyaṃ kim atrādhikam ||2 ||\\!}
\end{versinnote}

\emph{Haṭhatattvakaumudī} 49.30 (citing the \emph{Haṭhapradīpikā})
\begin{versinnote}
\tl{ardhodghāṭitalocanaḥ sthiramanā nāsāgradattekṣaṇaḥ\\+}
\tl{candrārkāvapi līnatāmupanayennispandavācyaṃ tataḥ |\\+}
\tl{jyotīrūpaviśeṣabāhyarahitaṃ dedīpyamānaṃ paraṃ\\+}
\tl{tatvaṃ tatparamasti vastu paramaṃ vācyaṃ kimatrādhikam || iti ||\\!}
\end{versinnote}

\end{testimonia}

\begin{philcomm}[hp04_011_3]
This verse is not in the \textalpha\ manuscripts and was probably added because it elaborates on the gaze. In the \emph{Candrāvalokana} (i.e., the source text) and the \emph{Anubhavanivedana}, it follows 4.7, which may explain its position in the \textgamma\ and \textdelta\ groups. The attentive reader will notice that we have decided against manuscripts that agree with the source text. The reason is based on our stemmatic considerations. Cumulative evidence supports the origin of the additional verses at the beginning of the fourth chapter in the hyparchetype of the \textepsilon\ manuscripts. We suspect these verses contaminated the \textbeta\ group's copies very early. To complicate matters, the scribe was familiar with the source text and corrected the readings accordingly.
\end{philcomm}


\begin{metre}[hp04_011_3]
Śārdūlavikrīḍita 
\end{metre}


%%%%%%%%%%
\subsection*{4.11*4}
\begin{translation}[hp04_011_4]
[The yogi] should not worship the \emph{liṅga} by day, nor should he worship it by night. He should worship the \emph{liṅga} constantly, by suppressing day and night.
% he should worship it day and night. Beta 2
\end{translation}

\begin{sources}[hp04_011_4]
\emph{Khecarīvidyā} 3.19
\begin{versinnote}
\tl{na divā pūjayel liṅgaṃ na rātrau ca maheśvari |\\+}
\tl{sarvadā pūjayel liṅgaṃ divārātrinirodhataḥ ||\\!}
\end{versinnote}
\end{sources}

\begin{testimonia}[hp04_011_4]
\emph{Haṭhasaṅketacandrikā} f.~128v (attr.~Haṭhapradīpikā)
\begin{versinnote}
\tl{divā na pūjayel liṃga[ṃ] rātrau naiva prapūjayet [|]\\+}
\tl{satataṃ pūjayel liṃga[ṃ] divārātrau ca pūjayet [||]\\!}
\end{versinnote}
\end{testimonia}

\begin{philcomm}[hp04_011_4]
The manuscripts transmit readings for the last verse quarter that either contradict or repeat the statements in the first line. This problem likely occurred through some kind of dittographical error. We have therefore adopted the reading of \emph{Jyotsnā} 4.42d (°nirodhataḥ), which is the same as the source text. 
\end{philcomm}

%%%%%%%%%%
\subsection*{4.11*5 heading}
\begin{translation}[hp04_011_5a]
Now \emph{khecarī}:
\end{translation}

% \begin{philcomm}[hp04_011_5a]
% \end{philcomm}

%%%%%%%%%%
\subsection*{4.11*5}
\begin{translation}[hp04_011_5]
There is a hollow that generates knowledge and has five streams. \emph{Khecarīmudrā} is situated in that pure void.
\end{translation}
%
%\begin{sources}[hp04_011_5]
%\end{sources}

\begin{testimonia}[hp04_011_5]
\emph{Haṭhasaṅketacandrikā} f.~128v
\begin{versinnote}
\tl{suṣiraṃ jñānajanakaṃ paṃcastr[o]taḥsamanvitaṃ |\\+}
\tl{tiṣṭhate khecarī mudrā tasmin [ś]ūnye niraṃjane  ||\\!}
\end{versinnote}
\end{testimonia}

\begin{philcomm}[hp04_011_5]
This verse may not be referring to the cavity in which the tongue is placed but perhaps to a more esoteric sense based on the meaning of \textit{pañcasrotas} as the five streams of tantric Śaiva teachings, which in this case generate knowledge. On \textit{pañcasrotas}, see \emph{Tāntrikābhidhānakośa} 2013, vol. 3: 361.
\end{philcomm}

\begin{metre}[hp04_011_5]
Anuṣṭubh (a: na-vipulā)
\end{metre}

%%%%%%%%%%
\subsection*{4.11*6}
\begin{translation}[hp04_011_6]
The breath in the left and right channels moves into the middle. Without doubt, \emph{khecarīmudrā} abides in that place.
\end{translation}

%\begin{sources}[hp04_011_6]
%\end{sources}

%\begin{testimonia}[hp04_011_6]
%\emph{Haṭhasaṅketacandrikā} f. 246v (Mysore ORI Manuscript B220) 
%\begin{versinnote}
%\tl{savyadakṣiṇanāḍīstho madhye carati mārutaḥ |\\+}
%\tl{tiṣṭhati khecarī mudrā tasmin sthāne na saṃśayaḥ ||\\!}
%\end{versinnote}
%\end{testimonia}

%\begin{philcomm}[hp04_011_6]
%The term \emph{madhye} appears to be referring to the central channel here. 
%\end{philcomm}

%%%%%%%%%%
\subsection*{4.11*7}
%\begin{translation}[hp04_011_7]
%The mind moves in the void because the tongue moves in the void, which is why that \emph{mudrā} is worshipped by the adepts with the name “she who moves in the void” (\emph{khecarī}).
%\end{translation}

% \begin{sources}[hp04_011_7]
% \emph{Vivekamārtaṇḍa} 50
% \begin{versinnote}
% \tl{cittaṃ carati khe yasmāj jihvā carati khe gatā |\\+}
% \tl{[tenai]ṣā khecarī nāma mudrā siddhair namaskṛtā ||\\!}
% \end{versinnote}

% %\emph{Skandapurāṇa} 4.41.45
% %\begin{versinnote}
% %\tl{cittaṃ carati khe yasmā jihvā carati khe gatā |\\+}
% %\tl{tenaiṣā khecarī nāma mudrā siddhair niṣevitā। ||\\!}
% %\end{versinnote}
% \end{sources}

% \begin{testimonia}[hp04_011_7]
% \emph{Yogacintāmaṇi} [attrib. Skandapurāṇa]
% %\begin{versinnote}
% %\tl{cittaṃ carati khe yasmāj jihvā carati khe gatā |\\+}
% %\tl{tenaiṣā khecarī nāma mudrā siddhair niṣevitā ||\\!}
% %\end{versinnote}

% \emph{Yuktabhavadeva} 212

% \end{testimonia}

\begin{philcomm}[hp04_011_7]
See 3.37, where this verse is also found.
\end{philcomm}

%%%%%%%%%%
\subsection*{4.11*8}
\begin{translation}[hp04_011_8]
At the juncture of Iḍā and Piṅgalā, the void devours the breath. \emph{Khecarīmudrā} abides there. This is undoubtedly true.
\end{translation}

%\begin{sources}[hp04_011_8]
%\end{sources}

\begin{testimonia}[hp04_011_8]
\emph{Upāsanāsārasaṅgraha} (IFP Transcript T1095) p.~42.
\begin{versinnote}
\tl{iḍāpiṅgalayor yoge śūnye caivānilaṃ graset |\\+}
\tl{tiṣṭhate khecarī mudrā tatra satyaṃ punaḥ punaḥ ||\\!}
\end{versinnote}
\end{testimonia}

\begin{philcomm}[hp04_011_8]
Cf.~Rāghavabhaṭṭa \emph{ad Śāradātilaka} 25.43: \emph{suṣumṇāyām eteṣu parvasu iḍāpiṅgalayor yogo bhavatīti jñeyam}.
\end{philcomm}

%%%%%%%%%%
\subsection*{4.11*9}
\begin{translation}[hp04_011_9]
The \emph{mudrā} situated in the cakra of the void (\emph{vyomacakre}) in the middle of the moon and sun on an unsupported surface is [the \emph{mudrā}] called \emph{khecarī}.
\end{translation}

%\begin{sources}[hp04_011_9]
%\end{sources}

\begin{testimonia}[hp04_011_9]
\emph{Upāsanāsārasaṅgraha} (IFP Transcript T1095) p.~41.
\begin{versinnote}
\tl{somasūryadvayor madhye nirālambe tale punaḥ |\\+}
\tl{saṃsthitā vyomacakre sā mudrā nāma ca khecarī ||\\!}
\end{versinnote}

\emph{Gorakṣasiddhāntasaṅgraha} p.~37
\begin{versinnote}
\tl{sūryācandramasormadhye nirālambe'nile punaḥ |\\+}
\tl{saṃsthitā vyomacakre yā sā mudrā nāma khecarī ||\\!}
\end{versinnote}
\end{testimonia}

\begin{philcomm}[hp04_011_9]
The reading \emph{tale} is suspect, as is \emph{nirālambāntare}, the conjecture of Brahmānanda.

The \emph{vyomacakra} (`the cakra of space') is also mentioned in a half-verse that was added to some later recensions of the \emph{Haṭhapradīpikā} (see 3.35) and states that \emph{vyomacakra} is another name for \emph{khecarīmudrā}. In \emph{Jyotsnā} 4.45, Brahmānanda states that the \emph{vyomacakra} is associated with all the voids in the middle of the brow (\emph{bhrūmadhye sarvakhānāṃ samanvayāt}) and, in the \emph{Haṭhasaṅketacandrikā} (f. 129v), Sundaradeva says that it is called the Brahmarandhra, which is between Iḍā and Piṅgalā (\emph{iḍāpiṅgalāntargataṃ brahmarandhrākhyaṃ vyomacakraṃ tat khecarīmudrāṃ śaktiṃ kuryād ...}). 
%  bhruvormadhye dvyasraṃ vyomacakraṃ tad dhi cinmayasya jyotiṣa upalabdhisthānam  Commentary on LYV 9.82.
% Haṭhasaṅketacandrika (4192): śūnyaṃ vyomacakravyavasthitam | iḍāpiṃgalāṃtargataṃ brahmaraṃdhrākhyaṃ vyomacakraṃ [...]. 
\end{philcomm}

%%%%%%%%%%
\subsection*{4.11*10}
\begin{translation}[hp04_011_10]
Pierced by me, lovely, and clearly beloved of Śiva, the Suṣumṇā should fill herself with the divine air through her rear mouth.
\end{translation}

%\begin{sources}[hp04_011_10]
%\end{sources}

\begin{testimonia}[hp04_011_10]
\emph{Yogasārasaṅgraha} p. 61 (attrib. Praṇavacintāmaṇi)
\begin{versinnote}
\tl{sā mayā viditā yā māyā sākṣācchivavallabhā |\\+}% unmetrical
\tl{pūrayen mārutaṃ divyaṃ suṣumnā paścime mukhe ||\\!}
\end{versinnote}
\end{testimonia}

\begin{philcomm}[hp04_011_10]
The referent of \emph{mayā} is unspecified, and no source text has been identified. It is most probably the goddess. The \textepsilon\ manuscripts have \emph{māyodbheditā} for \emph{mayodbheditā}, but it is hard to make sense of. If correct, \emph{udbheditā} would more likely mean “produced” or “made manifest”. Alternatively, the first line of this verse could be referring to \emph{khecarīmudrā}, in which case its being produced by \emph{māyā} would make more sense. In the second line, Brahmānanda’s reading \emph{suṣumnāṃ} is tempting. The line would then mean, “[The yogi] should fill Suṣumnā with divine breath through the rear opening”.
\end{philcomm}


%%%%%%%%%%
\subsection*{4.11*11}
\begin{translation}[hp04_011_11]
And if [she] fills herself from the front, \emph{khecarī} definitely arises. [The yogi] should practise \emph{khecarīmudrā}. The state beyond mind arises.
\end{translation}

%\begin{sources}[hp04_011_11]
%\end{sources}

\begin{testimonia}[hp04_011_11]
\emph{Upāsanāsārasaṅgraha} p.~135
\begin{versinnote}
\tl{purastāc caiva pūryeta niścitā khecarī bhavet |\\+}
\tl{abhyaset khecarīmudrāmunmanī saṃprajāyate ||\\!}
\end{versinnote}
\end{testimonia}

\begin{philcomm}[hp04_011_11]
We have understood \emph{niścitā} as an adverb. No witnesses have \emph{niścitam}, but we see no other way of taking \emph{niścitā}.
\end{philcomm}

%%%%%%%%%%
\subsection*{4.11*12}
\begin{translation}[hp04_011_12]
[The yogi] should practise \emph{khecarīmudrā} until he falls into a yogic sleep. For one who has attained yogic sleep, death never arises.
\end{translation}

%\begin{sources}[hp04_011_12]
%\end{sources}

\begin{testimonia}[hp04_011_12]
\emph{Upāsanāsārasaṅgraha} p.~135
\begin{versinnote}
\tl{abhyaset khecarīṃ tāvat yāvat syād yoganidrataḥ |\\+}
\tl{saṃprāptayoganidrasya kālo nāsti kadācana ||\\!}
\end{versinnote}
\end{testimonia}

\begin{philcomm}[hp04_011_12]
\end{philcomm}

%%%%%%%%%%


\subsection*{4.11*13}
\begin{translation}[hp04_011_13]
Between the eyebrows is the place of Śiva. The mind dissolves there. That level should be known as the fourth state. Death does not exist there.
\end{translation}
%

%\begin{sources}[hp04_011_13]
%\end{sources}

\begin{testimonia}[hp04_011_13]
\emph{Upāsanāsārasaṅgraha} p.~135
\begin{versinnote}
\tl{bhruvor madhye śivasthānaṃ manas tatra vilīyate |\\+}
\tl{jñātavyaṃ tat paraṃ turyaṃ tatra kālo na vidyate ||\\!}
\end{versinnote}
\end{testimonia}

%\begin{philcomm}[hp04_011_13]
%\end{philcomm}

%%%%%%%%%%
\subsection*{4.11*14}
\begin{translation}[hp04_011_14]
Between the moon and the sun, [the yogi] should apply \emph{khecarīmudrā}, which is situated in the supportless, great void, the \emph{vyomacakra}.
\end{translation}

\begin{sources}[hp04_011_14]
\emph{Jñānasāra} 3.3cd
\begin{versinnote}
\tl{candrasūryadvayor madhye muḍādadyā? tu khecarīm |\\!}
\end{versinnote}

\end{sources}

%\begin{testimonia}[hp04_011_14]
%\end{testimonia}

% \begin{philcomm}
% \end{philcomm}

%%%%%%%%%%
\subsection*{4.11*15}
\begin{translation}[hp04_011_15]
[The yogi] should make the mind supportless and think of nothing at all. He assuredly remains like a pot in the ether, [empty] inside and outside.
\end{translation}

\begin{sources}[hp04_011_15]
\emph{Jñānasāra} 3.4
\begin{versinnote}
\tl{nirālambaṃ manaḥ kṛtvā na kiñcid api cintayet || 3 ||\\+}
\tl{sa bāhyābhyantare yogī ghaṭavat tiṣṭhate priye |\\!}
\end{versinnote}

% \emph{Śivasaṃhitā} 5.210
% \begin{versinnote}
% \tl{nirālambo bhavej jīvo jñātvā vedāntayuktitaḥ |\\+}
% \tl{nirālambaṃ manaḥ kṛtvā na kiṃ cic cintayet sudhīḥ ||\\!}
% \end{versinnote}
\end{sources}

%\begin{testimonia}[hp04_011_15]
%\end{testimonia}

%\begin{philcomm}[hp04_011_15]
%\end{philcomm}

%%%%%%%%%%
\subsection*{4.11*16}
\begin{translation}[hp04_011_16]
Just as the external air has dissolved into the void, the breath is sure to go to its place †with the mind on the side of the sun†.
%When the external air has dissolved into the void, the breath surely goes to its own place, † and the sun together with the mind to the fire.†
\end{translation}

%\begin{sources}[hp04_011_16]
%\end{sources}

\begin{testimonia}[hp04_011_16]
\emph{Haṭhapradīpikā} (with ten-chapters) 7.52 
\begin{versinnote}
\tl{bāhyavāyur yathā līnaḥ svasya madhye na saṃśayaḥ |\\+}
\tl{svasthānaṃ gacchati prāṇaḥ sūryo 'gnau pavane tathā ||\\!}
\end{versinnote}
\end{testimonia}

\begin{philcomm}[hp04_011_16]
% sūryāgnir manasā tathā M1
It is hard to make sense of \emph{sūryāṅge} here (`on the side of the sun' ?). The terms \emph{sūryāṅge} and \emph{candrāṅge} occur in the third chapter (3.15) in the sense of the right and left sides of the body, respectively. However, this meaning does not seem relevant here. Since we are not sure of the meaning of \emph{sūryāṅge} and since the variant readings with \emph{pavana} are not clear either, we have cruxed the fourth verse quarter.      
\end{philcomm}

%

%%%%%%%%%%
\subsection*{4.11*17}
\begin{translation}[hp04_011_17]
For [the yogi] practising in this way day and night on the path of the breath, as a result of the practice the breath is consumed, [and] the mind dissolves into it.
\end{translation}

%\begin{sources}[hp04_011_17]
%\end{sources}

\begin{testimonia}[hp04_011_17]
\emph{Haṭhasaṅketacandrikā} f. 129v
\begin{versinnote}
\tl{abhyāsāl līyate vāyuḥ manas tatra vilīyate ||\\!}
\end{versinnote}
\end{testimonia}

\begin{philcomm}[hp04_011_17]
Brahmānanda identifies the path of the breath (\emph{vāyumārga}) with Suṣumṇā.
\end{philcomm}

%%%%%%%%%%
\subsection*{4.11*18}
\begin{translation}[hp04_011_18]
[The yogi] should flood the body with nectar from the soles of the feet to the head. [His] body is perfected forever, and he has great strength and valour.\\Thus ends \emph{khecarī}.
\end{translation}

%\begin{sources}[hp04_011_18]
%\end{sources}

\begin{testimonia}[hp04_011_18]
\emph{Haṭhasaṅketacandrikā} f. 129v
\begin{versinnote}
\tl{siddhaty evaṃ tadā kāyo mahābalaparākramaḥ ||\\!}
\end{versinnote}
\end{testimonia}

%\begin{philcomm}[hp04_011_18]
%\end{philcomm}

%%%%%%%%%%
% \subsection*{4.11*18 ending}
% iti khecarī/
% \begin{translation}[hp04_011_18p]
% \end{translation}

% \begin{philcomm}[hp04_011_18p]
% \end{philcomm}

%%%%%%%%%%
\subsection*{4.11*19 heading}
\begin{translation}[hp04_011_19a]
Now \emph{śāmbhavī}:
\end{translation}

% \begin{philcomm}[hp04_011_19a]
% \end{philcomm}

%%%%%%%%%%
\subsection*{4.11*19}
\begin{translation}[hp04_011_19]
{[}The yogi] should [put] the mind in Śakti and Śakti in the mind, observe the mind with the mind, and meditate on it as the supreme state.
\end{translation}
%V3 is closer to the CA reading


\begin{sources}[hp04_011_19]
\emph{Candrāvalokana} 27
\begin{versinnote}
\tl{śaktimadhye manaḥ kṛtvā manaś śaktes tu madhyamam |\\+}
\tl{manasā mana ālokya taṃ dhyāyet paramaṃ padaṃ ||\\+}
\tl{\var{madhyamam ] madhyagaṃ 4340}\\+}
\tl{\var{taṃ dhyāyet ] tad dhyāyet 4340}\\!}
\end{versinnote}
\end{sources}

\begin{testimonia}[hp04_011_19]
\emph{Haṭhasaṅketacandrikā} f.~129v–130r
\begin{versinnote}
\tl{śaktimadhyo manaḥ kṛtvā śaktiṃ ca svāntamadhyagāṃ |\\+}
\tl{manasā mana ālokya tad dhyāyet paramaṃ padam ||\\!}
\end{versinnote}
\end{testimonia}

\begin{philcomm}[hp04_011_19]
In the second verse quarter, \emph{mana} for \emph{mano} is for the metre.
\end{philcomm}

%%%%%%%%%%
\subsection*{4.11*20}
\begin{translation}[hp04_011_20]
Put the self in space and put space in the self. [The yogi] should make the self consist of space and think of nothing at all.
\end{translation}

\begin{sources}[hp04_011_20]
\emph{Uttaragītā} 1.9
\begin{versinnote}
\tl{khamadhye kuru cātmānam ātmamadhye ca khaṃ kuru |\\+}
\tl{ātmānaṃ khamayaṃ kṛtvā na kiṃ cid api cintayet  ||\\!}
\end{versinnote}
\end{sources}

\begin{testimonia}[hp04_011_20]
\emph{Haṭharatnāvalī} 4.45
\begin{versinnote}
\tl{khamadhye kuru cātmānam ātmamadhye ca khaṃ kuru |\\+}
\tl{sarvaṃ ca khamayaṃ kṛtvā na kiṃ cid api cintayet ||\\!}
\end{versinnote}
\end{testimonia}

%\begin{philcomm}[hp04_011_20]
% Critical edition of the Mbh: 
% 06,040.078d@003A_0073 ātmānaṃ khamayaṃ kṛtvā na kiṃ cid api cintayet || (36) Add to testimonia khamadhye kuru cātmānam ātmamadhyaṃ ca khaṃ kuru || (35)
%06,040.078d@003A_0072 khamadhye ca praveṣṭavyaṃ khaṃ ca brahma sanātanam |
%06,040.078d@003A_0073 ātmānaṃ khamayaṃ kṛtvā na kiṃ cid api cintayet || (36)
%\end{philcomm}

%%%%%%%%%%
\subsection*{4.11*21}
\begin{translation}[hp04_011_21]
Like an empty pot in air, [the yogī] is empty on the inside and empty on the outside. Like a full pot in the ocean, [the yogi] is full on the inside and full on the outside.%
\end{translation}

\begin{sources}[hp04_011_21]
\emph{Laghuyogavāsiṣṭha} 6.15.79 (\emph{Mokṣopāya} 6.155.25)
\begin{versinnote}
\tl{antaḥśūnyo bahiḥśūnyaḥ śūnyakumbha ivāmbare |\\+}
\tl{antaḥpūrṇo bahiḥpūrṇaḥ pūrṇakumbha ivārṇave ||\\!}
\end{versinnote}
\end{sources}

\begin{testimonia}[hp04_011_21]
\emph{Haṭharatnāvalī} 4.46
\begin{versinnote}
\tl{antaḥpūrṇo bahiḥpūrṇaḥ pūrṇakumbha ivāmbhasi |\\+}
\tl{antaḥśūnyaṃ bahiḥśūnyaṃ śūnyakumbha ivāmbare ||\\!}
\end{versinnote}

%\emph{Yogacintāmaṇi} f. 47v (attrib.~Vasiṣṭha, = \emph{Laghuyogavāsiṣṭha} 15.79)
%\begin{versinnote}
%\tl{antaḥśūnyo bahiḥśūnyaḥ śūnyakumbha ivāmbare |\\+}
%\tl{antaḥpūrṇo bahiḥpūrṇaḥ pūrnakumbha ivārṇave ||\\!}
%\end{versinnote}

\end{testimonia}

%\begin{philcomm}[hp04_011_21]
%\end{philcomm}

%%%%%%%%%%
\subsection*{4.11*22}
\begin{translation}[hp04_011_22]
Do not think about the external or internal. [The yogi] should abandon all thought and think of nothing at all.
\end{translation}

%\begin{sources}[hp04_011_22]
%\end{sources}

\begin{testimonia}[hp04_011_22]
\emph{Haṃsavilāsa} p.~48
\begin{versinnote}
\tl{bāhyacintā na kartavyā tathaivāntaracintanam |\\+}
\tl{sarvacintāṃ parityajya na kiñcid api cintayet ||\\!}
\end{versinnote}
\end{testimonia}

%\begin{philcomm}[hp04_011_22]
%°cintanā is not attested elsewhere.
%\end{philcomm}

%%%%%%%%%%
\subsection*{4.11*23}
\begin{translation}[hp04_011_23]
The whole world is but a construct of mere ideation. A construct of mere ideation is an affectation of the mind. So jettison this ideation. Take refuge in a resolve that is free of ideation, and obtain peace, O Rāma.
\end{translation}

\begin{sources}[hp04_011_23]
\emph{Laghuyogavāsiṣṭha} 7.27
\begin{versinnote}
\tl{saṃkalpajālakalanaiva jagat samagraṃ\\+}
\tl{saṃkalpajālakalanāt tu manovilāsaḥ |\\+}
\tl{saṃkalpamātram alam utsṛja nirvikalpam\\+}
\tl{āśritya niścayam avāpnuhi rāma śāntim ||\\!}
\end{versinnote}
\end{sources}

%\begin{testimonia}[hp04_011_23]
%\end{testimonia}

\begin{philcomm}[hp04_011_23]

\end{philcomm}

\begin{metre}[hp04_011_23]
Vasantatilakā
\end{metre}

%%%%%%%%%%
\subsection*{4.11*24}
\begin{translation}[hp04_011_24]
Just as camphor in fire and salt in water, so the mind, on being brought into contact with the highest reality, dissolves into it.
\end{translation}

%\begin{sources}[hp04_011_24]
%\end{sources}

\begin{testimonia}[hp04_011_24]
\emph{Haṭharatnāvalī} 4.43
\begin{versinnote}
\tl{karpūram anale yadvat saindhavaṃ salile yathā |\\+}
\tl{tathā sandhīyamānaṃ hi manas tatraiva līyate ||\\!}
\end{versinnote}
\end{testimonia}

%\begin{philcomm}[hp04_011_24]
%\end{philcomm}

%%%%%%%%%%
\subsection*{4.11*25}
\begin{translation}[hp04_011_25]
Mind is said to be all that is to be known, [all] that has been perceived, and [all] knowledge of that. Knowledge and what is to be known are destroyed together. There is no other path.
\end{translation}

%\begin{sources}[hp04_011_25]
%\end{sources}

\begin{testimonia}[hp04_011_25]
\emph{Haṭhatattvakaumudī} 51.35
\begin{versinnote}
\tl{jñeyaṃ sarvapratītaṃ ca tajjñānaṃ mana ucyate |\\+}
\tl{jñānaṃ jñeyaṃ samaṃ naṣṭaṃ nānyaḥ panthā dvitīyakaḥ ||\\!}
\end{versinnote}
\end{testimonia}

%\begin{philcomm}[hp04_011_25]
%Function of ca?
%\end{philcomm}

%%%%%%%%%%
\subsection*{4.11*26}
\begin{translation}[hp04_011_26]
All this, everything moving and unmoving, is [just] a vision of the mind. For when the mind has become free of the mind, it is called the absence of duality.
\end{translation}

\begin{sources}[hp04_011_26]
\emph{Amanaska} 2.79
\begin{versinnote}
\tl{manodṛśyam idaṃ sarvaṃ yat kim cit sacarācaram |\\+}
\tl{manaso hy unmanībhāve 'dvaitabhāvaṃ pracakṣate ||\\!}
\end{versinnote}

Cf. Gauḍapāda's \emph{Māṇḍūkyopaniṣatkārikā} 3.31
\begin{versinnote}
\tl{manodṛśyam idaṃ dvaitaṃ yat kiṃ cit sacarācaram |\\+}
\tl{manaso hy amanībhāve dvaitaṃ naivopalabhyate ||\\!}
\end{versinnote}

\end{sources}

\begin{testimonia}[hp04_011_26]
\emph{Yogacintāmaṇi} f. 27r (attr.~to the \emph{Rājayoga}, aka. \emph{Amanaska})
\begin{versinnote}
\tl{manodṛśyam idaṃ sarvaṃ yat kiṃ cit sacarācaram |\\+}
\tl{manasas tūnmanībhāvo 'dvaitābhāvaṃ pracakṣate ||\\!}
\end{versinnote}
\end{testimonia}

%\begin{philcomm}[hp04_011_26]
%\end{philcomm}

%%%%%%%%%%
\subsection*{4.11*27}
\begin{translation}[hp04_011_27]
As a result of abandoning the things that are to be known, the mind attains dissolution. When the mind has attained dissolution, liberation (\textit{kaivalyam}) remains.
\end{translation} %

%\begin{sources}[hp04_011_27]
%\end{sources}

\begin{testimonia}[hp04_011_27]
\emph{Haṭharatnāvalī} 4.44
\begin{versinnote}
\tl{jñeyavastuparityāgād vilayaṃ yāti mānasaḥ |\\+}
\tl{mānase vilayaṃ yāte kaivalyam upajāyate ||\\!}
\end{versinnote}
\end{testimonia}

%\begin{philcomm}[hp04_011_27]
%\end{philcomm}

%%%%%%%%%%
\subsection*{4.11*28}
\begin{translation}[hp04_011_28]
“Dissolution, dissolution”, they say. What kind of characteristics does dissolution have? Because subliminal impressions do not arise again, dissolution is the forgetting of the objects of the senses.
\end{translation}

%\begin{sources}[hp04_011_28]
%\end{sources}

\begin{testimonia}[hp04_011_28]
\emph{Haṭharatnāvalī} 1.13
\begin{versinnote}
\tl{layo laya iti prāhuḥ kīdṛśaṃ layalakṣaṇam |\\+}
\tl{apunarbhavasaṃsthānaṃ layo viṣayavismṛtiḥ ||\\!}
\end{versinnote}

\emph{Yogasārasaṅgraha} p.~52 (attr. to Śrīdatta)
\begin{versinnote}
\tl{layo laya iti prāhur īdṛśaṃ lakṣaṇaṃ sphuṭam |\\+}
\tl{tatra sarvasamādhāne layo viṣayavismṛtiḥ ||\\!}
\end{versinnote}
\end{testimonia}

\begin{philcomm}[hp04_011_28]
This verse may have been inspired by the \emph{Mokṣopāya} (e.g.~1.2.2).
\end{philcomm}

%%%%%%%%%%
\subsection*{4.11*29}
\begin{translation}[hp04_011_29]
Various methods like these, which are understood properly by one's own experience, are taught as paths to \textit{samādhi} by the great-souled teachers of former times.
\end{translation}

%\begin{sources}[hp04_011_29]
%\end{sources}

%\begin{testimonia}[hp04_011_29]
%\end{testimonia}

%\begin{philcomm}[hp04_011_29]
%\end{philcomm}

\begin{metre}[hp04_011_29]
Anuṣṭubh (c: bha-vipulā)
\end{metre}

%%%%%%%%%%
% \subsection*{4.11*30 heading}
% atha viśrāntiḥ/
% \begin{translation}[hp04_011_30a]
% \end{translation}

% \begin{philcomm}[hp04_011_30a]
% \end{philcomm}

%%%%%%%%%%
\subsection*{4.11*30}
\begin{translation}[hp04_011_30]
Homage to Suṣumṇā, to Kuṇḍalinī, to the nectar in the orb of the moon, to the mind beyond mind state, to you whose nature is consciousness, the great Śakti.%
\end{translation}

%\begin{sources}[hp04_011_30]
%\end{sources}

%\begin{testimonia}[hp04_011_30]
%\end{testimonia}

%\begin{philcomm}[hp04_011_30]
%\end{philcomm}

\begin{metre}[hp04_011_30]
Anuṣṭubh (a: ra-vipulā)
\end{metre}

%%%%%%%%%%
\subsection*{4.11*31}
\begin{translation}[hp04_011_31]
The cultivation of the inner sound taught by Gorakṣanātha [and] approved even for deluded people for whom realisation of the highest reality is impossible is [now] taught.
\end{translation} %

%\begin{sources}[hp04_011_31]
%\end{sources}

%\begin{testimonia}[hp04_011_31]
%\end{testimonia}

%\begin{philcomm}[hp04_011_31]
%\end{philcomm}


%%%%%%%%%%
\subsection*{4.12}
\begin{translation}[hp04_012]
The twelve and a half million methods of dissolution taught by glorious Śiva are the best. We consider one of the dissolutions in particular to be especially worthy of honour, concentration on the inner sound.
\end{translation}
% MD: From a stemmatic point of view, jayante (not -ti) is slightly preferable.

\begin{sources}[hp04_012]
\emph{Yogatārāvalī} 2
\begin{versinnote}
\tl{sadā śivoktāni sapādalakṣa-\\+}
\tl{layāvadhānāni lasantu loke |\\+}
\tl{nādānusandhānasamādhim ekaṃ\\+}
\tl{manyāmahe mānyatamaṃ layānām ||\\!}
%\tl{\var{lasantu ] Sa : lasanti Pa : vasanti Ad W Eb Ec Ed : ca santi  Ea : hi sarva Ma}\\+}
%\tl{\var{loke ]  Ma  W Eall : bhūmau  Ad Pa Sa}\\!}
\end{versinnote}
\end{sources}

\begin{testimonia}[hp04_012]
\emph{Haṭharatnāvalī}~1.12
\begin{versinnote}
\tl{śrīādināthena sapādakoṭi-\\+}
\tl{layaprakārāḥ kathitā jayantu  |\\+}
\tl{nādānusandhānakam eva kāryaṃ\\+}
\tl{manyāmahe mānyatamaṃ layānām ||\\!}
\end{versinnote}

\emph{Yogacintāmaṇi} f.~23v (attr. to the \emph{Haṭhapradīpikā})
\begin{versinnote}
\tl{śrīādināthena sapādakoṭi-\\+}
\tl{layaprakārāḥ kathitā jayanti |\\+}
\tl{nādānusandhānakam eva kāryam\\+}
\tl{manyāmahe nānyatamaṃ layānām ||\\!}
\end{versinnote}

\end{testimonia}

%\begin{philcomm}[hp04_012]
%\end{philcomm}

\begin{metre}[hp04_012]
Upajāti
\end{metre}

%%%%%%%%%%
\subsection*{4.13}
\begin{translation}[hp04_013]
Seated in the pose of the liberated ones (\emph{muktāsanasthitaḥ}), the yogi should adopt \emph{śāṃbhavī mudrā} and, with his mind one-pointed, listen to the inner sound in his right ear.
\end{translation}

%\begin{sources}[hp04_013]
%\end{sources}

\begin{testimonia}[hp04_013]
\emph{Yogacintāmaṇi} f.~23v (attr. to the \emph{Haṭhapradīpikā})
\begin{versinnote}
\tl{muktāsanasthito yogī mudrāṃ sandhāya śāṃbhavīm |\\+}
\tl{śṛṇuyād dakṣiṇe karṇe nādam antargataṃ sadā ||\\!}
\end{versinnote}

\emph{Haṭhasaṅketacandrikā} f.~124r (attr. to the \emph{Haṭhapradīpikā})
\begin{versinnote}
\tl{muktāsanasthito yogī mudrāṃ saṃdhāya śāṃbhavīṃ [|]\\+}
\tl{śṛṇuyād dakṣiṇe karṇe nādam ekāntike sudhīḥ  [||]\\+}
\tl{\var{dakṣiṇe karṇe ] B220,  dakṣirṇe 2244}\\+}
\tl{\var{ekāṃtike ] 2244, ekāṃtate B220}\\!}
\end{versinnote}
\end{testimonia}

\begin{philcomm}[hp04_013]
In verse 1.37, \emph{muktāsana} is said to be the same as \emph{siddhāsana}.
We read this verse here (as attested by \textalpha, etc.) rather than at \manuref{4.36*1} because it follows on from the Śāmbhavī section and is unnecessary and repetitive at \manuref{4.36*1} because \manuref{4.37–38} explain the practice.
\end{philcomm}

%%%%%%%%%%
\subsection*{4.14}
\begin{translation}[hp04_011]
A fire that has been set on wood disappears together with the wood; the mind set on the inner sound dissolves together with the inner sound.
\end{translation}

%\begin{sources}[hp04_011]
%\end{sources}

\begin{testimonia}[hp04_011]
\emph{Haṭharatnāvalī} 4.15
\begin{versinnote}
\tl{kāṣṭhe pravartito vahniḥ kāṣṭhena saha līyate |\\+}
\tl{nāde pravartitaṃ cittaṃ nādena saha śāmyati ||\\!}
\end{versinnote}

\emph{Yogacintāmaṇi} f.~23v (attr. to the \emph{Haṭhapradīpikā})
\begin{versinnote}
\tl{kāṣṭhe pravartito vahniḥ kāṣṭhena saha śāmyati |\\+}
\tl{nāde pravartitaṃ cittaṃ nādena saha līyate ||\\!}
\end{versinnote}

\emph{Haṭhasaṅketacandrikā} f.~124r (attr. to the \emph{Haṭhapradīpikā})
\begin{versinnote}
\tl{kāṣṭhaiḥ pravartito vahniḥ kāṣṭhena saha śāmyati |\\+}
\tl{nāde pravartitaṃ cittaṃ nādena saha līyate ||\\!}
\end{versinnote}
\end{testimonia}

%\begin{philcomm}[hp04_011]
%\end{philcomm}

%%%%%%%%%%
\subsection*{4.15}
\begin{translation}[hp04_012]
Having forgotten everything external, the mind becomes one with the inner sound like milk and water, then quickly dissolves into the space of consciousness.
\end{translation}

%\begin{sources}[hp04_012]
%\end{sources}

\begin{testimonia}[hp04_012]
\emph{Yogacintāmaṇi} f.~23v (attr. to the \emph{Haṭhapradīpikā})
\begin{versinnote}
\tl{vismṛtya sakalaṃ bāhyaṃ nāde dugdhāmbuvan naraḥ |\\+}
\tl{ekībhūyātha sahasā cidākāśe vilīyate ||\\!}
\end{versinnote}

\emph{Haṭhasaṅketacandrikā} f.~124r (attr. to the \emph{Haṭhapradīpikā})
\begin{versinnote}
\tl{vismṛtya sakalaṃ bāhyaṃ nāde dugdhāṃbuvan manaḥ |\\+}
\tl{ekībhūyātha sahasā cidākāśe vilīyate ||\\!}
\end{versinnote}

\emph{Upāsanāsārasaṅgraha} p.~106 (attr. to the \emph{Haṭhapradīpikā})
\begin{versinnote}
\tl{vismṛtya sakalaṃ bāhyaṃ nāde dagdhāṃbuvan manaḥ |\\+}
\tl{ekībhūtaṃ tanyā cittaṃ rājayogābhidānakaṃ ||\\!}
\end{versinnote}

\emph{Nādabindūpaniṣat} 39
\begin{versinnote}
\tl{vismṛtya sakalaṃ bāhyaṃ nāde dugdhāmbuvan manaḥ |\\+}
\tl{ekībhūyātha sahasā cidākāśe vilīyate ||\\!}
\end{versinnote}
\end{testimonia}

%\begin{philcomm}[hp04_012]
%\end{philcomm}

\begin{metre}[hp04_012]
Anuṣṭubh (c: na-vipulā)
\end{metre}

%%%%%%%%%%
\subsection*{4.16}
\begin{translation}[hp04_016]
Having become intent on indifference through regular practice, the ascetic should concentrate on nothing but the inner sound, which immediately brings about the [state] beyond mind.%.
\end{translation}

%\begin{sources}[hp04_016]
%\end{sources}

\begin{testimonia}[hp04_016]
\emph{Yogacintāmaṇi} f.~23v (attr. to the \emph{Haṭhapradīpikā})
\begin{versinnote}
\tl{audāsīnyaparo bhūtvā sadābhyāsena saṃyamī |\\+}
\tl{unmanīkaraṇaṃ sadyo nādam evāvadhārayet ||\\!}
\end{versinnote}

\emph{Haṭhasaṅketacandrikā} f.~124r (attr. to the \emph{Haṭhapradīpikā})
\begin{versinnote}
\tl{audāsīnyaparo bhūtvā sadābhyāsena saṃyamī |\\+}
\tl{unmanīkārakaṃ sadyo nādam evāvadhārayet ||\\+}
\tl{\var{evāvadhārayet ] emend., evaṃ vadhārayet ms. 2244}\\!}
\end{versinnote}

\emph{Upāsanāsārasaṅgraha} p.~106 (attr. to the \emph{Haṭhapradīpikā})
\begin{versinnote}
\tl{audāsīnyaparo bhūtvā sadābhyāsena saṃyamī |\\+}
\tl{unmanīkaraṇam sadyo nādam evānu[...]yet ||\\!}
\end{versinnote}

\emph{Nādabindūpaniṣat} 40
\begin{versinnote}
\tl{udāsīnas tato bhūtvā sadābhyāsena saṃyamī |\\+}
\tl{unmanīkārakaṃ sadyo nādam evāvadhārayet ||\\!}
\end{versinnote}

\end{testimonia}

%\begin{philcomm}[hp04_016]
%\end{philcomm}

%%%%%%%%%%
\subsection*{4.17 heading}
\begin{translation}[hp04_017a]
What kind of indifference?
\end{translation}

% \begin{philcomm}[hp04_017a]
% \end{philcomm}

%%%%%%%%%%
\subsection*{4.17}
\begin{translation}[hp04_017]
In the cold season, [indifference towards] whether [one has] an open tent or a quilt, with regard to good nourishment, whether it is cow's milk or water, with regard to food, whether it is lots of alms [or] forest roots, and with regard to the vessel for food, whether it is the hand or some kind of bowl.
\end{translation}

% MD: pāda c: better to read "bhojye bhikṣā kandam āraṇyakaṃ vā"?
% JB: kāpi droṇī: a fine wooden vessel

%\begin{sources}[hp04_017]
%\end{sources}

\begin{testimonia}[hp04_017]
\emph{Haṭharatnāvalī} 4.7
\begin{versinnote}
\tl{śīte kāle dvau paṭī vā paṭī vā pathyāhāre gopayo vā payo vā |\\+}
\tl{bhakṣye bhojye vṛttim āraṇyakaṃ vā pāṇī droṇī ko 'pi vā bhakṣyapātre ||\\!}
\end{versinnote}

\emph{Yogacintāmaṇi} f.~23v (attr. to the \emph{Haṭhapradīpikā})
\begin{versinnote}
\tl{kīdṛśaṃ caudāsīnyam—\\+}
\tl{śīte kāle kāpaṭī vā paṭī vā
pathyāhāre gopayo vā payo vā |\\+}
\tl{bhakṣye bhikṣāvṛndam āraṇyakandam
pāṇau droṇī kā parā bhojapātram ||\\!}
\end{versinnote}

\emph{Haṭhatattvakaumudī}  54.39
\begin{versinnote}
\tl{audāsīnyaṃ śītakāle paṭī vā
pathyāhāro gopayo vā payo vā |\\+}
\tl{bhojyaṃ bhikṣāvṛndam āraṇyakandaṃ
pāṇī droṇī kāpi vā bhojyapātram ||\\!}
\end{versinnote}

\end{testimonia}

\begin{philcomm}[hp04_017]
On \emph{caupaṭī} in the first \emph{pāda}, McGregor (1994: s.v.) and Callewaert (2009: s.v.) in their dictionaries of modern and old Hindi both give the meaning ``open all around" for \emph{caupaṭ}. McGregor derives it from Sanskrit \emph{catuṣpaṭṭa}. Molesworth (1857: s.v.) in his Marathi dictionary gives ``A quadrangular expanse or space, esp. as open and extended: also a broad and level tract" for \emph{caupaṭṭā}.
\end{philcomm}

\begin{metre}[hp04_017]
Śālinī 
\end{metre}

%%%%%%%%%%
\subsection*{4.18}
\begin{translation}[hp04_018]
Having forever abandoned all worry and all activity, as a result of meditating on nothing but the inner sound the mind dissolves into the inner sound.
\end{translation}

%\begin{sources}[hp04_018]
%\end{sources}

\begin{testimonia}[hp04_018]
\emph{Haṭharatnāvalī} 4.13
\begin{versinnote}
\tl{sarvacintāṃ samutsṛjya sarvaceṣṭāṃ ca sarvadā |\\+}
\tl{nādam evānusandhānān nāde cittaṃ vilīyate ||\\!}
\end{versinnote}

\emph{Haṭhasaṅketacandrikā} f.~124r (attr. to the \emph{Haṭhapradīpikā})
\begin{versinnote}
\tl{sarvaciṃtā[ṃ] parityajya sarvakāle ca sarvadā |\\+}
\tl{nādam evānusandhatte nāde cittaṃ vilīyate ||\\!}
\end{versinnote}

\emph{Nādabindūpaniṣat} 41
\begin{versinnote}
\tl{sarvacintāṃ samutsṛjya sarvaceṣṭāvivarjitaḥ |\\+}
\tl{nādam evānusaṃdadhyān nāde cittaṃ vilīyate  ||\\!}
\end{versinnote}
\end{testimonia}

\begin{philcomm}[hp04_018]
%The occurrence of two finite verbs in the second hemistich of some witnesses (groups 2 and 3) is infelicitous and perhaps points to an error that arose from the \textit{nādam evānusandhānān} reading in the group one mss. It seems the original reading was probably \textit{nāda evānusandhānāt}, and the -m- crept in as a hiatus break. 

The third verse quarter varies considerably among the witnesses and testimonia. We have accepted the reading of \alphaOne\ (\emph{nādam evānusandhānān}), which is also attested by the \textbeta\ group and manuscripts of the \emph{Haṭharatnāvalī}, on the assumption that the -\emph{m}- at the end of \emph{nādam} is a hiatus break, the intended reading being \textit{nāda evānusandhānāt}. A hiatus break is also at 3.90.
\end{philcomm}

%%%%%%%%%%
\subsection*{4.19}
\begin{translation}[hp04_019]
Beginning (\emph{ārambhaḥ}), union (\emph{ghaṭaḥ}), accumulation (\emph{paricayaḥ}) and completion (\emph{niṣpattiḥ}): those are the stages of yoga in all yogas.
\end{translation}

\begin{sources}[hp04_019]
\emph{Amaraugha} 34
\begin{versinnote}
\tl{ārambhaś ca ghaṭaś caiva paricayas tṛtīyakaḥ |\\+}
\tl{niṣpattiḥ sarvayogeṣu yogāvasthā caturvidhā ||\\!}
\end{versinnote}

\emph{Amṛtasiddhi} 19.2
\begin{versinnote}
\tl{ārambhaś ca ghaṭaś caiva paricayas tṛtīyakaḥ |\\+}
\tl{niṣpannaḥ sarvaśeṣeṣu yogāvasthāḥ prakīrtitāḥ ||\\!}
\end{versinnote}

\emph{Śivasaṃhitā} 3.31
\begin{versinnote}
\tl{ārambhaś ca ghaṭaś caiva tathā paricayas tathā |\\+}
\tl{niṣpattiḥ sarvayogeṣu yogāvasthā bhavanti tāḥ ||\\+}
\tl{\var{tathā ] tataḥ, tadā, smṛtāḥ, tv athā}\\!}
\end{versinnote}
\end{sources}

\begin{testimonia}[hp04_019]
\emph{Haṭharatnāvalī} 4.17
\begin{versinnote}
\tl{ārambhaś ca ghaṭaś caiva tathā paricayas tathā |\\+}
\tl{niṣpattiḥ sarvayogeṣu yogāvasthā bhavanti tāḥ ||\\!}
\end{versinnote}

\emph{Yogacintāmaṇi} f. 111v
\begin{versinnote}
\tl{īśvaraprokte—\\+}
\tl{ārambhaś ca ghaṭaś caiva tathā paricayo 'pi ca |\\+}
\tl{niṣpattiḥ sarvayogeṣu yogāvasthā bhavanti tāḥ ||\\!}
\end{versinnote}

\emph{Yuktabhavadeva} 7.135
\begin{versinnote}
\tl{tad uktaṃ śivayoge—\\+}
\tl{ārambhaś ca ghaṭaś caiva tathā paricayaḥ punaḥ |\\+}
\tl{niṣpattiś ceti yogasya syād avasthācatuṣṭayam ||\\!}
\end{versinnote}

\emph{Haṭhatattvakaumudī} 54.14 (attr. to the \emph{Haṭhapradīpikā})
\begin{versinnote}
\tl{ārambhaś ca ghaṭaś caiva tathā paricayas tathā|\\+}
\tl{niṣpattiḥ sarvayogeṣu syād avasthācatuṣṭayam  ||\\!}
\end{versinnote}

\end{testimonia}

\begin{philcomm}[hp04_019]
On these four stages, see Birch 2019: 968–969; Mallinson and Szanto 2021:19–20.
\end{philcomm}

%%%%%%%%%%
\subsection*{4.20 heading}
\begin{translation}[hp04_020a]
Among these, the beginning stage is [as follows]:
\end{translation}

% \begin{philcomm}[hp04_020a]
% \end{philcomm}

%%%%%%%%%%
\subsection*{4.20}
\begin{translation}[hp04_020]
As a result of the piercing of the knot of Brahmā, bliss arises in the void [and] an unstruck sound which has various tones is heard in the body.
\end{translation}


\begin{sources}[hp04_020]
\emph{Amaraugha} 35
\begin{versinnote}
\tl{brahmagranthes tathā bhedād ānandaḥ śūnyasambhavaḥ |\\+}
\tl{vicitrakvaṇako dehe 'nāhataḥ śrūyate dhvaniḥ ||\\!}
\end{versinnote}
\end{sources}

\begin{testimonia}[hp04_020]
\emph{Haṭharatnāvalī} 4.18
\begin{versinnote}
\tl{brahmarandhre bhaved bhedo yo nādaḥ sūryasaṃbhavaḥ |\\+}
\tl{vicitrakvaṇado dehe 'nāhataḥ śrūyate dhvaniḥ ||\\+}
\tl{\var{yo nādaḥ ] ānandaḥ \vl}\\+}
\tl{\var{vicitrakvaṇado ] vicitrakvaṇako \vl}\\!}
\end{versinnote}

\emph{Yogacintāmaṇi} f. 25r
\begin{versinnote}
\tl{brahmagranthir bhaved bhinna ānandaḥ śūnyasaṃbhavaḥ |\\+}
\tl{vicitrakṣaṇiko deho 'nāhataḥ śrūyate dhvaniḥ ||\\!}
\end{versinnote}

\emph{Haṭhatattvakaumudī} 54.15 (attr. to the \emph{Haṭhapradīpikā})
\begin{versinnote}
\tl{tatra ārambhaḥ –\\+}
\tl{brahmagranthir bhaved bhinnād ānandaḥ śūnyasambhavaḥ |\\+}
\tl{vicitrakvaṇiko dehe 'nāhataḥ śrūyate dhvaniḥ ||\\!}
\end{versinnote}
\end{testimonia}

\begin{philcomm}[hp04_020]
In \emph{Jyotsnā} 4.70, Brahmānanda understands the voids associated with each stage to be places in the body. However, in the \emph{Amṛtasiddhi} and \emph{Amaraugha}, the source text of this verse, the series of voids (along with their respective blisses and sounds) derives from a tetrad of meditative voids in Vajrayāna traditions (Mallinson and Szanto 2021: 18; Birch 2019: 968). 
\end{philcomm}

%%%%%%%%%%
\subsection*{4.21}
\begin{translation}[hp04_021]
With a divine body, radiant, smelling heavenly, free from disease [and] his heart full [of bliss], in the void in the beginning [stage] the practitioner becomes a yogi.

\end{translation}

\begin{sources}[hp04_021]
\emph{Amaraugha} verse 36
\begin{versinnote}
\tl{divyadehaś ca tejasvī divyagandho hy arogavān |\\+}
\tl{saṃpūrṇahṛdaye śūnye tv ārambhe yogavān bhavet ||\\+}
\tl{\var{°hṛdaye ] \emph{Amaraugha}, °hṛdayaḥ \emph{Amaraughaprabodha}}\\!}
\end{versinnote}
\end{sources}

\begin{testimonia}[hp04_021]
\emph{Haṭharatnāvalī} 4.19
\begin{versinnote}
\tl{divyadehaḥ sutejasvī divyagandhas tv arogavān |\\+}
\tl{saṃpūrṇahṛdaye śūnye tv ārambhe yogavān bhavet ||\\!}
\end{versinnote}

\emph{Haṭhatattvakaumudī} 54.18 (attr. to the \emph{Haṭhapradīpikā})
\begin{versinnote}
\tl{tejasvī divyagandhaś ca divyadeho 'py arogavān |\\+}
\tl{sampūrṇahṛdaye śūnye tv ārambhe yogavān bhavet ||\\!}
\end{versinnote}
\end{testimonia}

%\begin{philcomm}[hp04_021]
%\end{philcomm}

%%%%%%%%%%
\subsection*{4.22 heading}
\begin{translation}[hp04_022a]
Now the unified stage:
\end{translation}

% \begin{philcomm}[hp04_022a]
% \end{philcomm}

%%%%%%%%%%
\subsection*{4.23}
\begin{translation}[hp04_022]
In the second stage the breath, after bringing about union, goes into the middle [channel]. Then the yogi has a firm posture [and] he becomes a gnostic and equal to a god.
\end{translation}


\begin{sources}[hp04_022]
\emph{Amaraugha} 37
\begin{versinnote}
\tl{dvitīye saṅghaṭīkṛtya vāyur bhavati madhyagaḥ |\\+}
\tl{dṛḍhāsano bhaved yogī jñānī devasamas tadā ||\\+}
\tl{\var{dvitīye saṅ- ] \emph{Amaraugha} : dvitīyāyāṃ \emph{Amaraughaprabodha}}\\!}
\end{versinnote}
\end{sources}

\begin{testimonia}[hp04_022]
\emph{Haṭharatnāvalī} 4.20
\begin{versinnote}
\tl{dvitīyāyāṃ ghaṭīkṛtya vāyur bhavati madhyagaḥ |\\+}
\tl{dṛḍhāsano bhaved yogī kāmadevasamas tadā ||\\!}
\end{versinnote}

\emph{Haṭhatattvakaumudī} 54.15 (attr. to the \emph{Haṭhapradīpikā})
\begin{versinnote}
\tl{atha ghaṭāvasthā –\\+}
\tl{dvitīyāyāṃ ghaṭīṃ kṛtvā vāyur bhavati madhyagaḥ |\\+}
\tl{dṛḍhāsano bhaved yogī jñānī devasamas tathā ||\\!}
\end{versinnote}
\end{testimonia}

\begin{philcomm}[hp04_022]
The name of the second stage, \emph{ghaṭa} (and the related form \emph{ghaṭīkṛtya} found in the first \emph{pāda} of this verse) can be understood in at least three ways, as union, activation or pot (the latter with an alchemical connotation, on which see 3.12 and Mallinson and Szanto 2021: 20–23). In the \emph{Dattātreyayogaśāstra} (verse 90) its primary meaning is union and the united pairs are \emph{prāṇa} and \emph{apāna}, \emph{jīvātman} and \emph{paramātman}, and \emph{nāda} and \emph{bindu}.
\end{philcomm}

%%%%%%%%%%
\subsection*{4.23}
\begin{translation}[hp04_021]
Then, as a result of the piercing of the knot of Viṣṇu there is a hint of supreme bliss in total emptiness (\emph{atiśūnye}), and then the pounding sound of a kettle drum occurs.%
\end{translation}

\begin{sources}[hp04_021]
\emph{Amaraugha} 38
\begin{versinnote}
\tl{viṣṇugranthes tato bhedāt paramānandasūcakaḥ |\\+}
\tl{atiśūnye vimardaś ca bherīśabdas tato bhavet ||\\+}
\tl{\var{atiśūnye ] Ad Gb Ta : atiśūnyo \emph{Amaraugha}, Ba}\\+}
\tl{\var{tato ] \emph{Amaraugha} : tathā  Ba : tadā Ad Gb Ta}\\!}
\end{versinnote}
\end{sources}

\begin{testimonia}[hp04_021]
\emph{Haṭharatnāvalī} 4.21
\begin{versinnote}
\tl{viṣṇugranthes tathā bhedaḥ paramānandasūcakaḥ |\\+}
\tl{atiśūnye vimardaś ca bherīśabdas tathā bhavet ||\\!}
\end{versinnote}

\emph{Yogacintāmaṇi} f. 25r
\begin{versinnote}
\tl{viṣṇugranthir yadā bhinnaḥ paramānandasūcakaḥ |\\+}
\tl{atiśūnyavibhedaś ca bherīśabdas tadā bhavet ||\\!}
\end{versinnote}

\emph{Haṭhatattvakaumudī} 54.21 (attr. to the \emph{Haṭhapradīpikā})
\begin{versinnote}
\tl{viṣṇugranthir yadā bhinnā paramānandasūcikā |\\+}
\tl{atiśūnyavibhedaś ca bherīśabdas tathā bhavet ||\\!}
\end{versinnote}
\end{testimonia}

\begin{philcomm}[hp04_021]
%V3 and HR have the reading of the \textit{Amaraugha} (\textit{atiśūnye vimardaś ca}), which was adapted from the Buddhist concept of a series of four void-like meditative states and moments. In the \emph{Amaraugha},  two of the moments were left out and two were reinterpreted (as in the case of \emph{vimarda}, which seems to have be retained with the more general sense of a pounding sound). %It is possible that Svātmārāma adopted this reading but he may have also changed it to \textit{atiśūnyavibheda} (as attested by groups 2 and 3)  in an attempt to make sense of the pāda (i.e., the variety, atiśūnya, arises).
\end{philcomm}

%%%%%%%%%%
\subsection*{4.24 heading}
\begin{translation}[hp04_024a]
Now the accumulation stage:
\end{translation}

% \begin{philcomm}[hp04_024a]
% \end{philcomm}

%%%%%%%%%%
\subsection*{4.24}
\begin{translation}[hp04_024]
In the third [stage], having pierced [the knot of Viṣṇu], the sound of a bass drum arises in space. Then [the yogi] reaches the great void, the abode of all supernatural powers.%
\end{translation}
%

\begin{sources}[hp04_024]
\emph{Amaraugha} 39
\begin{versinnote}
\tl{tṛtīyāyāṃ tato bhittvā ninādo mardaladhvaniḥ |\\+}
\tl{mahāśūnyaṃ tato jātaṃ sarvasiddhisamāśrayam ||\\+}
\tl{\var{bhittvā ninādo ] Ga : bhītvādinādau Ae : bhittvā vipāko Ba Ad Gb Ta }\\!}
\end{versinnote}
\end{sources}

\begin{testimonia}[hp04_024]
\emph{Haṭharatnāvalī} 4.22
\begin{versinnote}
\tl{tṛtīyāyāṃ tato nityaṃ āviṣkāro marddladhvaniḥ |\\+}
\tl{mahāśūnyaṃ tato yāti sarvasiddhisamāśrayaḥ ||\\!}
\end{versinnote}

\emph{Yogacintāmaṇi} f. 25r
\begin{versinnote}
\tl{tṛtīyāyāṃ tato bhitvā vimāyo mardaladhvaniḥ |\\+}
\tl{mahāśūnyaṃ tathā yāti sarvasiddhisamāśrayam ||\\!}
\end{versinnote}

\emph{Haṭhatattvakaumudī} 54.22 (attr. to the \emph{Haṭhapradīpikā})
\begin{versinnote}
\tl{atha paricayāvasthā ||\\+}
\tl{tṛtīyāyāṃ tato jitvā sahajānandasambhavaḥ |\\+}
\tl{doṣaduḥkhajarāmṛtyuḥ kṣudhānidrāvivarjitaḥ ||\\!}
\end{versinnote}
\end{testimonia}

\begin{philcomm}[hp04_024]
Here the object of piercing is unspecified, but it is likely to refer to the knot of Viṣṇu mentioned in the previous verse.\lb

We have understood \emph{vihāyo}° (`space') as referring to the state of total emptiness (\emph{atiśunya}) that was mentioned in the previous verse. 

\end{philcomm}

%%%%%%%%%%
\subsection*{4.25}
\begin{translation}[hp04_025]
Having overcome the [supreme] bliss of the mind, there arises innate bliss. [The yogi] becomes free of disease, suffering, old age, death, hunger and sleep.
\end{translation}
% MD: cittānanda or cidānanda?

\begin{sources}[hp04_025]
\emph{Amaraugha} 40
\begin{versinnote}
\tl{paramānandarocitvāt sahajānandasambhavaḥ |\\+}
\tl{doṣaduḥkhajarāmṛtyukṣudhānidrāvivarjitaḥ ||\\+}
\tl{\var{paramānanda° ] \emph{Amaraugha}; cittānandaṃ \emph{Amaraughaprabodha}.}\\+}
\tl{\var{°rocitvāt ] \emph{Amaraugha}; tato jitvā \emph{Amaraughaprabodha}}\\!}
\end{versinnote}
\end{sources}

\begin{testimonia}[hp04_025]
\emph{Haṭharatnāvalī} 4.23
\begin{versinnote}
\tl{cidānandaṃ tato jitvā paramānandasaṃbhavaḥ |\\+}
\tl{doṣaduḥkhajarāmṛtyukṣudhānidrāvivarjitaḥ ||\\!}
\end{versinnote}

\emph{Yogacintāmaṇi} f. 25r
\begin{versinnote}
\tl{cittānandaṃ tato jitvā sahajānandasaṃbhavaḥ |\\+}
\tl{doṣaduḥkhakṣudhānidrājarāmṛtyuvivarjitaḥ ||\\!}
\end{versinnote}

\emph{Haṭhatattvakaumudī} 54.22 (attr. to the \emph{Haṭhapradīpikā})
\begin{versinnote}
\tl{tṛtīyāyāṃ tato jitvā sahajānandasambhavaḥ |\\+}
\tl{doṣaduḥkhajarāmṛtyuḥ kṣudhānidrāvivarjitaḥ ||\\!}
\end{versinnote}
\end{testimonia}

\begin{philcomm}[hp04_025]
The reading of the first verse quarter, \emph{cittānāndaṃ tato jitvā}, is likely a patch for the somewhat obscure \emph{paramānandarocitvāt} in the \emph{Amaraugha} (40a), which is the source text.
\end{philcomm}

%%%%%%%%%%
\subsection*{4.26 heading}
\begin{translation}[hp04_026a]
Now the perfection stage:
\end{translation}

% MD: niṣpattyavasthā -> niṣpattiḥ?

% \begin{philcomm}[hp04_026a]
% \end{philcomm}

%%%%%%%%%%
\subsection*{4.26}
\begin{translation}[hp04_026]
Then, having pierced Rudra's knot, the breath goes to all the seats [of the deities in the body]. In the perfected [stage] the sound of a flute becomes the sound of a resonating lute.
\end{translation}


\begin{sources}[hp04_026]
\emph{Amaraugha} 41
\begin{versinnote}
\tl{rudragranthiṃ tato bhittvā sarvapīṭhagato 'nilaḥ |\\+}
\tl{niṣpanno vaiṇavaḥ śabdaḥ kvaṇadvīṇākvaṇo bhavet ||\\+}
\tl{\var{sarva° ] Ae : sarvaṃ Ga : sattva° \emph{Amaraughaprabodha}}\\+}
\tl{\var{niṣpanno ] Ga : niṣpannā Ae : niṣpattau Ad Gb T : ++ttau Ba }\\+}
\tl{\var{kvaṇadvīṇākvaṇo ] conj. : kvaṇañ cailakvaṇo S1 : kvaṇañ caiva kvaṇo S2 : kvaṇadvitakvaṇo L3 : kvaṇanvitakvaṇo L2 : kvaṇanvītakva+ L4 : kvaṇatbhakvaṇo L1}\\!}
\end{versinnote}
\end{sources}

\begin{testimonia}[hp04_026]
\emph{Haṭharatnāvalī} 4.24
\begin{versinnote}
\tl{rudragranthiṃ tato bhitvā śarvapīṭhagato 'nilaḥ |\\+}
\tl{niṣpattau vaiṇavaḥ śabdaḥ kvanadvīṇākvaṇo bhavet ||\\!}
\end{versinnote} 

\emph{Yogacintāmaṇi} f. 25r
\begin{versinnote}
\tl{rudragranthiṃ tato bhitvā sarvapīṭhagato 'nilaḥ |\\+}
\tl{niṣṭhāto vaiṇavaḥ śabdaḥ kvaṇadvīṇākvaṇo bhavet ||\\!}
\end{versinnote}

\emph{Haṭhatattvakaumudī} 54.24 (attr. to the \emph{Haṭhapradīpikā})
\begin{versinnote}
\tl{atha niṣpattiḥ –\\+}
\tl{rudragranthiṃ tato bhitvā śarvapīṭhagato'nalaḥ |\\+}
\tl{niṣpanno vaiṇavaḥ śabdo kvaṇadvīṇākvaṇo bhavet ||\\!}
\end{versinnote}
\end{testimonia}

\begin{philcomm}[hp04_026]
The reading we have adopted for the fourth verse quarter, \emph{kvaṇadvīṇākvaṇo}, Birch’s conjecture in his edition of the \emph{Amaraugha} is supported by several testimonia and \emph{Amṛtasiddhi} 31.2, where it is said that the sound of a \emph{vīṇā} arises in the fourth stage.\lb

The reading \emph{śarva}, i.e.~Śiva, found in some testimonia and the \emph{Jyotsnā} makes good sense, but in its description of this stage, the \emph{Amṛtasiddhi}, which is the ultimate source of this passage has \emph{sarva} (30.1).
\end{philcomm}

%%%%%%%%%%
\subsection*{4.27}
\begin{translation}[hp04_027]
Then the mind becomes one [with the sound]. This is called Rājayoga. He becomes a creator and destroyer, an equal to a lord among yogis.%
\end{translation}

\begin{sources}[hp04_027]
\emph{Amaraugha} 42
\begin{versinnote}
\tl{ekībhūtaṃ tadā cittaṃ rājayogābhidhānakam |\\+}
\tl{sṛṣṭisaṃhārakartāsau yogīśvarasamo bhavet ||\\+}
\tl{\var{rājayogābhidhānakam ] rājayogo 'bhidhīyate Ae}\\!}
\end{versinnote}
\end{sources}

\begin{testimonia}[hp04_027]
\emph{Yogacintāmaṇi} f. 25
\begin{versinnote}
\tl{ekībhūtaṃ tathā cittaṃ rājayogābhidhāyakam |\\!}
\end{versinnote}

\emph{Upāsanāsārasaṅgraha} p.~106
\begin{versinnote}
\tl{vismṛtya sakalaṃ bāhyaṃ nāde dagdhāmbuvan manaḥ |\\+}
\tl{ekībhūtaṃ tathā cittaṃ rājayogābhidhānakam \\!}
\end{versinnote}

\end{testimonia}

\begin{philcomm}[hp04_027]
%The bahuvrīhi rājayogābhidhānakaṃ means that the mind is called rājayoga, which is very unusual. Perhaps °bhidhāyakaṃ is better, meaning "expresses rājayoga", but it is only in J7 and V3.

According to Brahmānanda (\emph{Jyotsnā} 4.77), the idea that the yogi becomes a creator and destroyer means that he becomes equal to God. Consequently he construes the line as \emph{asau yogī īśvarasamo bhavet}.
\end{philcomm}

%%%%%%%%%%
\subsection*{4.27*1}
\begin{translation}[hp04_027_1]
The dissolution which arises from the inner sound  instantly gives proof of [its efficacy] [and] is an easy method for attaining the state of Rājayoga [even] for foolish people.
\end{translation}

%\begin{sources}[hp04_027_1]
%\end{sources}

%\begin{testimonia}[hp04_027_1]
%\end{testimonia}

%\begin{philcomm}[hp04_027_1]
%\end{philcomm}

%%%%%%%%%%
\subsection*{4.28}
\begin{translation}[hp04_028]
Whether or not this is liberation, in this very state a great, unbroken pleasure, which is rich in the nectar of absorption, is attained from Rājayoga.
\end{translation}

%\begin{sources}[hp04_028]
%\end{sources}

\begin{testimonia}[hp04_028]
\emph{Haṭharatnāvalī} 4.16
\begin{versinnote}
\tl{astu vā māstu vā muktir atraivākhaṇḍitaṃ mahat |\\+}
\tl{layāmṛtaṃ laye saukhyaṃ rājayogād avāpyate ||\\!}
\end{versinnote}

\emph{Yogacintāmaṇi} f. 113v (attr. to īśvara)
\begin{versinnote}
\tl{astu vā māstu vā siddhir atraivākhaṇḍitaṃ sukham |\\!}
\end{versinnote}

\emph{Haṭhatattvakaumudī} 54.35 (attr. to the \emph{Haṭhapradīpikā})
\begin{versinnote}
\tl{astu vā māstu vā muktir atraivākhaṇḍitaṃ mahat |\\+}
\tl{layāmṛtamayaṃ saukhyaṃ rājayogād avāpyate ||\\!}
\end{versinnote}
\end{testimonia}

%\begin{philcomm}[hp04_028]
%\end{philcomm}

%%%%%%%%%%
\subsection*{4.29}
\begin{translation}[hp04_029]
Rājayoga without Haṭha and Haṭha without Rājayoga do not succeed so [the yogi] should practise both until the perfection stage.
\end{translation}

\begin{sources}[hp04_029]
\emph{Śivasaṃhitā} 5.222
\begin{versinnote}
\tl{haṭhaṃ vinā rājayogo rājayogaṃ vinā haṭhaḥ |\\+}
\tl{na sidhyati tato yugmam āniṣpatteḥ samabhyaset |\\+}
\tl{tasmāt pravartate yogī haṭhe sadgurumārgataḥ ||\\+}
\tl{\var{na ... samabhyaset ] \emph{om.} I, III, IV, VII, IX, X, XII, XIV–XVI}\\!}
\end{versinnote}
\end{sources}

\begin{testimonia}[hp04_029]
\emph{Haṭharatnāvalī} 1.19
\begin{versinnote}
\tl{haṭhaṃ vinā rājayogo rājayogaṃ vinā haṭhaḥ |\\+}
\tl{vyāptiḥ syād avinābhūtā śrīrājahaṭhayogayoḥ ||\\!}
\end{versinnote}

\emph{Yogacintāmaṇi} f. 21r (attr. to the \emph{Haṭhapradīpikā})
\begin{versinnote}
\tl{haṭhaṃ vinā rājayogaṃ rājayogaṃ vinā haṭham |\\+}
\tl{na siddhyati tato yugmaṃ manīṣyetau samabhyaset ||\\!}
\end{versinnote}

\emph{Yuktabhavadeva} 7.127 (attr. to the \emph{Haṭhapradīpikā})
\begin{versinnote}
\tl{haṭhaṃ vinā rājayogo rājayogaṃ vinā haṭhaḥ |\\+}
\tl{na sidhyati tato yugmamāniṣpatteḥ samācaret ||\\!}
\end{versinnote}

\emph{Haṭhatattvakaumudī} 55.1
\begin{versinnote}
\tl{haṭhaṃ vinā rājayogo rājayogaṃ vinā haṭhaḥ  |\\+}
\tl{na sidhyati tato yugmam āniṣpatteḥ samabhyaset  ||\\!} % = HP 2.77
\end{versinnote}
\end{testimonia}

%\begin{philcomm}[hp04_029]
%\end{philcomm}

\begin{metre}[hp04_029]
Anuṣṭubh (a: ra-vipulā)
\end{metre}

%%%%%%%%%%
\subsection*{4.30}
\begin{translation}[hp04_030]
I consider those who are ignorant of Rājayoga and only work at Haṭha to be like farmhands who get no reward for their efforts.%
\end{translation}

%\begin{sources}[hp04_030]
%\end{sources}

\begin{testimonia}[hp04_030]
\emph{Haṃsavilasa} p.~49
\begin{versinnote}
\tl{rājayogam ajānantaḥ kevalaṃ haṭhakarmagāḥ|\\+}
\tl{ye tān karmavaśān manye prayāsaphalavarjitāḥ ||\\!}
\end{versinnote}
\end{testimonia}

%\begin{philcomm}[hp04_030]
%\end{philcomm}

%%%%%%%%%%
\subsection*{4.31}
\begin{translation}[hp04_031]
The supreme reality is the seed, Haṭha the ground and indifference water. With these three the wish-fulfilling vine that is the beyond-mind state immediately grows.
\end{translation}
%  JB: wish-Fulfilling vine for kalpalatikā?
% PhD thesis (2011). Lin, Nancy Grace. "Adapting the Buddha's Biographies: A Cultural History of the Wish-Fulfilling Vine in Tibet, Seventeenth to Eighteenth Centuries." A cultural history of the Kṣemendra's Bodhisattvāvadānakalpalatā in Tibet.


%\begin{sources}[hp04_031]
%\end{sources}

\begin{testimonia}[hp04_031]
\emph{Yogacintāmaṇi} f. 24r (attr. to the \emph{Haṭhapradīpikā})
\begin{versinnote}
\tl{nādo bījaṃ haṭhaḥ kṣetram audāsīnyaṃ jalaṃ smṛtam |\\+}
\tl{unmanīkalpalatikā sadya evodbhaviṣyati ||\\!}
\end{versinnote}
\end{testimonia}

\begin{philcomm}[hp04_031]
The meaning of \emph{tattva} here is unclear. It is a synonym for \emph{samādhi} and \emph{unmanī} in the next verses. However, this meaning seems unlikely here as \emph{tattva} is one of three factors that is supposed to lead to \emph{unmanī}. In \emph{Jyotsnā} 4.104, Brahmānanda says that \emph{tattva} means \emph{citta}, which makes sense in so far as the mind grows to the beyond-mind state. However, this interpretation seems somewhat contrived. We have understood \emph{tattva} in the sense of the `highest reality' (\emph{paramatattva}) on the assumption that, as the seed, it is the latent cause of the state beyond mind.

\end{philcomm}

\begin{metre}[hp04_031]
Anuṣṭubh (c: na-vipulā)
\end{metre}

%%%%%%%%%%
\subsection*{4.32}
\begin{translation}[hp04_032]
The sovereign yoga (\emph{rājayoga}), meditative absorption (\emph{samādhi}), the beyond mind state (\emph{unmanī}), transmental state (\textit{manonmanī}), [the sovereign yoga of] the lineage of immortals (\emph{amaraugha}), non-duality (\emph{advaita}), without support (\emph{nirālamba}), pure (\emph{nirañjana}), [\dots]
\end{translation}
% 

%\begin{sources}[hp04_032]
%\end{sources}

\begin{testimonia}[hp04_032]
\emph{Yogacintāmaṇi} f.~6r
\begin{versinnote}
\tl{haṭhapradīpikāyām—\\+}
\tl{rājayogaḥ samādhiś ca unmanī ca manonmanī |\\+}
\tl{amaraughaughacāndrīva nirālambaṃ nirañjanam ||\\!}
\end{versinnote}

\emph{Upāsanāsārasaṅgraha} p.~106
\begin{versinnote}
\tl{haṭhapradīpikāyāṃ |\\+}
\tl{rājayogas samādhiś cāpy unmanī ca manonmanī |\\+}
\tl{amaraugho pi cādvaitaṃ nirālambaṃ niraṃjanaṃ  ||\\!}
\end{versinnote}

\emph{Yogasārasaṅgraha} p.~60 (attr. to the \emph{Nandikeśvaratārāvalī})
\begin{versinnote}
\tl{rājayogaḥ samādhiś conmanī ca manonmanī |\\+}
\tl{śivayogo layas tatvaṃ śūnyāśūnyaṃ nirañjanam |\\!}
\end{versinnote}

\emph{Haṃsavilāsa} p.~47
\begin{versinnote}
\tl{rājayogaḥ samādhiśca unmanī ca manonmanī |\\+}
\tl{amaraughālayas tatra śūnyāc chūnya paraṃ padam ||\\!}
\end{versinnote}
\end{testimonia}

%\begin{philcomm}[hp04_032]
%\emph{amaraughaughacāndrī} appears to be a mistake perhaps because the word \emph{amaraugha} became \emph{amarolī}, hence prompting the change to \emph{candrī}.
%\end{philcomm}

%%%%%%%%%%
\subsection*{4.33}
\begin{translation}[hp04_033]
[\dots]  no-mind (\emph{amanaska}), dissolution [of mind] (\emph{laya}), the [ultimate] reality (\emph{tattva}), void and not void (\textit{śūnyāśūnya}), the highest state (\emph{para pada}), liberation in life (\emph{jīvanmukti}), innate (\emph{sahaja}) and the fourth [state] (\emph{turya}) are synonyms.
\end{translation}
% 

\begin{sources}[hp04_033]
\emph{Yogacintāmaṇi} f.~6r (attr. to the \emph{Haṭhapradīpikā})
\begin{versinnote}
\tl{amanasko layaś caiva śūnyāśūnyaṃ parāparam |\\+}
\tl{jīvanmuktiś ca sahajaṃ turyaṃ cety ekavācakam iti ||\\!}
\end{versinnote}

\emph{Upāsanāsārasaṅgraha} p.~106 (attr. to the \emph{Haṭhapradīpikā})
\begin{versinnote}
\tl{ama[na]sko layas tatvaṃ śūnyāśūnyaparaṃ  padaṃ |\\+}
\tl{jīvanmuktiś ca sahajaṃ turyaṃ cety ekavācakaṃ  ||\\!}
\end{versinnote}

\emph{Yogasārasaṅgraha} p.~60 (attr. to the \emph{Nandikeśvaratārāvalī})
\begin{versinnote}
\tl{amanaskaṃ yathā caitan nirālambaṃ nirañjanam |\\+}
\tl{jīvanmuktiś ca sahajam ity adir hy ekavācakam ||\\!}
\end{versinnote}
\end{sources}

%\begin{testimonia}[hp04_033]
%\end{testimonia}

%\begin{philcomm}[hp04_033]
%tattva may have dropped out of Gr2 and 3 because of the awkwardness of including it after tattvaṃ bījam in the previous verse (if tattva was considered something else).
%\end{philcomm}

\begin{metre}[hp04_033]
Anuṣṭubh (c: na-vipulā)
\end{metre}

%%%%%%%%%%
\subsection*{4.34}
\begin{translation}[hp04_034]
Two paths for the quick attainment of the beyond-mind state are approved by me: [cultivating] the ultimate reality (\emph{tattva}) or supreme pleasure. And focusing on the inner sound...
\end{translation}

% MD: pāda d: comment on ca vs vā?
% JB: the ultimate reality [of Rājayoga], which is supreme bliss, or the cultivation of inner resonance, (either .... or... if we read vā in c and d)

%\begin{sources}[hp04_034]
%\end{sources}

\begin{testimonia}[hp04_034]
\emph{Yogacintāmaṇi} f.~23v (attr. to the \emph{Haṭhapradīpikā})
\begin{versinnote}
\tl{unmanyavāptaye śīghraṃ dvau mārgau mama saṃmatau |\\+}
\tl{tatvaṃ paramasaukhyaṃ vā nādopāsanam eva vā ||\\!}
\end{versinnote}

\emph{Upāsanāsārasaṅgraha} p.~106 (attr. to the \emph{Haṭhapradīpikā})
\begin{versinnote}
\tl{unmanyavāptaye śīghraṃ mārgau dvau mama sammatau |\\+}
\tl{tattvaṃ paramasākhyaṃ vā nādopāsanam eva vā  ||\\!}
\end{versinnote}
\end{testimonia}

%\begin{philcomm}[hp04_034]
%\end{philcomm}


%%%%%%%%%%
\subsection*{4.35}
\begin{translation}[hp04_035]
 is approved even for foolish people whose minds are intent upon pleasure. The dissolution which arises from the inner sound instantly bestows bliss.
\end{translation}
% JB [worldly] pleasures -> [worldly] bliss ... translate saukhya the same way as previous verse?

%\begin{sources}[hp04_035]
%\end{sources}


\begin{testimonia}[hp04_035]
\emph{Yogacintāmaṇi} f.~23v (attr. to the \emph{Haṭhapradīpikā})
\begin{versinnote}
\tl{saukhyapraviṣṭacittānāṃ mūḍhānām api saṃmatam |\\+}
\tl{sadya ānandasandhāyī jāyate nādajo layaḥ ||\\!}
\end{versinnote}

\emph{Upāsanāsārasaṅgraha} p.~106 (attr. to the \emph{Haṭhapradīpikā})
\begin{versinnote}
\tl{sāṅkhye praviṣṭacittānāṃ mūḍhānām api saṃmateḥ |\\+}
\tl{tasya svānaṃdasa [...] ryo jāyate nādajo layaḥ ||\\!}
\end{versinnote}
\end{testimonia}

%\begin{philcomm}[hp04_035]
%J5 has saṃmataḥ, agreeing with layaḥ
%\end{philcomm}

%%%%%%%%%%
\subsection*{4.35*1}
\begin{translation}[hp04_035_1]
There is one seed [syllable] consisting of creation and one \emph{mudrā}, \emph{khecarī}, one god, the unsupported, [and] one state, mind beyond the mind.
\end{translation}

\begin{sources}[hp04_035_1]
Cf. \emph{Tantrāloka} 32.64
\begin{versinnote}
\tl{ekaṃ sṛṣṭimayaṃ bījaṃ yadvīryaṃ sarvamantragam |\\+}
\tl{ekā mudrā khecarī ca mudraughaḥ prāṇito yayā ||\\!}
\end{versinnote}

\emph{Tantrālokaviveka} 32.63
Cf \begin{versinnote}
\tl{yad āgamaḥ –\\+}
\tl{ekaṃ sṛṣṭimayaṃ bījam ekā mudrā ca khecarī	|\\+}
\tl{dvāvekaṃ yo vijānāti sa vai pūjyaḥ kulāgame	||\\!}
\end{versinnote}

\emph{Śivasūtravimarśinī} 5
\begin{versinnote}
\tl{ekaṃ sṛṣṭimayaṃ [sṛṣṭimayaṃ bījam iti mantravīryarūpam aham iti bījam | 
mudrā parabhairavīyātmā |] bījam ekā mudrā ca khecarī |\\+}
\tl{dvāv etau yasya jāyete so'tiśāntapade sthitaḥ ||\\!}
\end{versinnote}
\end{sources}

\begin{testimonia}[hp04_035_1]
\emph{Haṭharatnāvalī} 4.28
\begin{versinnote}
\tl{ekaṃ sṛṣṭimayaṃ bījam ekā mudrā ca khecarī |\\+}
\tl{eko devo nirālambaḥ ekāvasthā manonmanī ||\\!}
\end{versinnote}

\emph{Yogacintāmaṇi} f.~75r (attr. to the \emph{Haṭhapradīpikā})
\begin{versinnote}
\tl{ekaṃ sṛṣṭimayaṃ bījam ekā mudrā ca khecarī |\\+}
\tl{eko deśo nirālamba ekāvasthā manonmanī ||\\!}
\end{versinnote}

\emph{Yuktabhavadeva} 7.219 (attr. to Gorakṣanātha)
\begin{versinnote}
\tl{ekaṃ sṛṣṭimayaṃ bījam ekā mudrā ca khecarī |\\+}
\tl{eko devo nirālamba ekāvasthā manonmanī ||\\!}
\end{versinnote}

%\emph{Nādabindūpaniṣad} 52cd-53ab
%\begin{versinnote}
%\tl{śaṅkhadundubhinādaṃ ca na śruṇoti kadācana || 52 ||\\+}
%\tl{kāṣṭhavaj jñāyate deha unmanyāvasthayā dhruvam |\\!}
%\end{versinnote}
\end{testimonia}

\begin{philcomm}[hp04_035_1]
See 3.49.
\end{philcomm}

%%%%%%%%%%
\subsection*{4.35*2}
\begin{translation}[hp04_035_2]
[The yogi] never hears the sounds of [even] conch shells and large drums. As a result of the state of no mind, the body assuredly becomes as [insentient as a piece of] wood.
\end{translation}
%

\begin{sources}[hp04_035_2]
\emph{Jñānasāra} 3.7
\begin{versinnote}
\tl{śaṅkhadundubhinādena na śṛṇoti kadācana |\\+}
\tl{kāṣṭava[j] jñāyate yogī notpattyā vai prajāyate ||\\!}
\end{versinnote}
\end{sources}

\begin{testimonia}[hp04_035_2]
\emph{Haṭhasaṅketacandrika} f.~120v (attr. to the \emph{Haṭhapradīpikā})
\begin{versinnote}
\tl{haṭhapradīpikāyāṃ \\+}
\tl{śaṃkhaduṃdubhinādaṃ ca n[a] śṛṇoti kadācana |\\+}
\tl{kāṣṭavaj jāyate dehe unmanyā'vasthayāś dhruvaṃ ||\\!}
\end{versinnote}

\end{testimonia}

%\begin{philcomm}[hp04_035_2]
%\end{philcomm}

%%%%%%%%%%
\subsection*{4.35*3}
\begin{translation}[hp04_035_3]
Free from all states [of mind] and all thought, the yogi is as if dead. He is liberated. In this there is no doubt.
\end{translation}

%\begin{sources}[hp04_035_3]
%\end{sources}

\begin{testimonia}[hp04_035_3]
\emph{Nādabindūpaniṣad} 51cd-52ab
\begin{versinnote}
\tl{sarvāvasthāvinirmuktaḥ sarvacintāvivarjitaḥ || 51 ||\\+}
\tl{mṛtavat tiṣṭhate yogī sa mukto nātra saṃśayaḥ |\\!}
\end{versinnote}

\emph{Haṭhatattvakaumudī} 51.75 (attr. to the \emph{Haṭhapradīpikā})
\begin{versinnote}
\tl{sarvāvasthāvinirmuktaḥ sarvacintāvivarjitaḥ |\\+}
\tl{mṛtavat tiṣṭhate yogī sa mukto nātra saṃśayaḥ ||\\!}
\end{versinnote}
\end{testimonia}

%\begin{philcomm}[hp04_035_3]
%Maybe kāṣṭhavat crept in as a repetition of the previous verse.
%\end{philcomm}

%%%%%%%%%%
\subsection*{4.35*4}
\begin{translation}[hp04_035_4]
The yogi in \textit{samādhi} experiences neither cold nor heat, neither suffering nor pleasure, neither praise nor scorn.
\end{translation}

\begin{sources}[hp04_035_4]
\emph{Vivekamārtaṇḍa} 166
\begin{versinnote}
\tl{nābhijānāti śītoṣṇaṃ na duḥkhaṃ na sukhaṃ tathā |\\+}
\tl{na mānaṃ nāpamānaṃ ca yogī yuktaḥ samādhinā ||\\!}
\end{versinnote}
\end{sources}

%\begin{testimonia}[hp04_035_4]
%\end{testimonia}

%\begin{philcomm}[hp04_035_4]
%\end{philcomm}

%%%%%%%%%%
\subsection*{4.35*5}
\begin{translation}[hp04_035_5]
The yogi in \textit{samādhi} cannot be wounded by any weapon, killed by any person or overpowered by mantras and magic.
\end{translation}


\begin{sources}[hp04_035_5]
\emph{Vivekamārtaṇḍa} 168
\begin{versinnote}
\tl{abhedyaḥ sarvaśastrāṇām avadhyaḥ sarvadehinām |\\+}
\tl{agrāhyo mantratantrāṇāṃ yogī yuktaḥ samādhinā ||\\!}
\end{versinnote}
\end{sources}

%\begin{testimonia}[hp04_035_5]
%\end{testimonia}

\begin{philcomm}[hp04_035_5]
The collated manuscripts have \emph{avadhyaḥ} in the first and second verse quarters. Although \emph{avadhya} can be construed in both quarters, the repetition appears to be a dittographical error that changed \emph{avedyaḥ}, which is close to the reading of the source text, into \emph{avadhyaḥ}. We have therefore emended accordingly. 
\end{philcomm}

%%%%%%%%%%
\subsection*{4.35*6}
\begin{translation}[hp04_035_6]
The yogi in \textit{samādhi} does not experience smell, taste, form, touch, sound, himself nor anyone else.
\end{translation}

\begin{sources}[hp04_035_6]
\emph{Vivekamārtaṇḍa} 165
\begin{versinnote}
\tl{na gandhaṃ na rasaṃ rūpaṃ na ca sparśaṃ na nisvanam |\\+}
\tl{nātmānaṃ na paraṃ vetti yogī yuktaḥ samādhinā ||\\!}
\end{versinnote}
\end{sources}

\begin{testimonia}[hp04_035_6]
\emph{Yuktabhavadeva} 11.31 (attr. to Gorakṣanātha)
\begin{versinnote}
\tl{na gandhaṃ na rasaṃ sparśaṃ na rūpaṃ na ca niḥsvanam |\\+}
\tl{nātmānaṃ ca paraṃ vetti yogī yuktaḥ samādhinā ||\\!}
\end{versinnote}
\end{testimonia}

%\begin{philcomm}[hp04_035_6]
%\end{philcomm}

%%%%%%%%%%
\subsection*{4.35*7}
\begin{translation}[hp04_035_7]
He is indeed truly liberated whose mind is neither asleep nor awake, has no memory nor otherwise, and neither stops nor starts.
\end{translation}

\begin{sources}[hp04_035_7]
\emph{Gorakṣaśataka} 7
\begin{versinnote}
\tl{cittaṃ prasuptaṃ yogena jāgratsuptaṃ na cānyathā |\\+}
\tl{nāstam eti na codeti yasyāsau mukta eva hi ||\\+}
\tl{\var{7a cittaṃ prasuptaṃ yogena ] T; cittaṃ na suptaṃ no jāgrac G.}\\+}
\tl{\var{7b jāgratsuptaṃ na cānyathā ] em.; jāgratsūtir na *nyathā T, chrutimadvacanasya ca G.}\\!}
\end{versinnote}
%
\end{sources}

\begin{testimonia}[hp04_035_7]
\emph{Haṭhasaṅketacandrika} f.~120v (attr. to the \emph{Haṭhapradīpikā})
\begin{versinnote}
\tl{cittaṃ na suptaṃ no jāgrat smṛtivarṇaṃ na cānyathā |\\+}
\tl{nāstam eti na codeti yasyāsau mukta eva saḥ ||\\!}
\end{versinnote}
\end{testimonia}

\begin{philcomm}[hp04_035_7]
%The first line is corrupt in the \emph{Gorakṣaśataka} witnesses; its text has been constituted from the \emph{Haṭhapradīpikā}. 
The first line of this verse is different to the version in the source text, the \emph{Gorakṣaśataka}, as the words \emph{jāgrat} and \emph{suptaṃ} are found in the first quarter. The second quarter is corrupt in many of the \emph{Haṭhapradīpikā} manuscripts but \emph{smṛti} and \emph{nānyathā} are well attested. % We have therefore adopted the reading of G11 (group 4c), which in this case is close to the readings of the \textBeta\ group.
% MD: to delete the last sentence? G11 = ε1 is now fully collated.
\end{philcomm}

\begin{metre}[hp04_035_7]
Anuṣṭubh (a: ma-vipulā)
\end{metre}

%%%%%%%%%%
\subsection*{4.35*8}
\begin{translation}[hp04_035_8]
[The yogi] who remains at ease as though asleep in the waking state, without breathing in and out, is definitely liberated.
\end{translation}

\begin{sources}[hp04_035_8]
\emph{Amanaska} 2.59
\begin{versinnote}
\tl{sadā jāgradavasthāyāṃ suptavad yo 'vatiṣṭhate |\\+}
\tl{niśvāsocchvāsahīnas ca niścitaṃ mukta eva saḥ ||\\+}
\tl{\var{sadā jāgradavasthāyāṃ ] \vl
sadā jāgṛvadasthāyāṃ,
sadā jāgrat apasthāyāṃ,
sadā jāgradavasther ya,
sa jāgras tadavasthāyāṃ,
yadā jāgṛdavasthāyāṃ,
yo jāgrad yad avasthāyāṃ,
svapnajāgradavasthāyāṃ,
suptajāgradavasthāyāṃ}\\!}
\end{versinnote}
\end{sources}

\begin{testimonia}[hp04_035_8]
\emph{Kulārṇavatantra} 9.11
\begin{versinnote}
\tl{svapnajāgradavasthāyāṃ suptavat yo 'avatiṣṭhate | \\+}
\tl{niśvāsocchvāsahīnaś ca niścitaṃ mukta eva saḥ || 11 ||\\!}
\end{versinnote}

\emph{Yogacintāmaṇi} f.~27v (attr. to the \emph{Rājayoga} [aka. \emph{Amanaska}])
\begin{versinnote}
\tl{sadā jāgradavasthāyāṃ suptavada yo'vatiṣṭhate ||\\+}
\tl{niḥśvāsocchāsahīnaś ca niścitaṃ mukta eva saḥ|\\!}
\end{versinnote}

\emph{Haṭhatattvakaumudī} 55.24 (attr. to the \emph{Rājayoga} [aka. \emph{Amanaska}])
\begin{versinnote}
\tl{sadā jāgradavasthāyāṃ suptavad yo ‘vatiṣṭhate |\\+}
\tl{niśvāsocchvāsavihīnaś ca niścitaṃ mukta eva saḥ || 55.24 = AY 2.59\\!}
\end{versinnote}


%\emph{Yuktabhavadeva} 1.64 (attr. to the \emph{Kulārṇava})
%\begin{versinnote}
%\tl{svapnajāgradavasthāyāṃ suptavat yo 'vatiṣṭhate |\\+}
%\tl{niśvāsocchvāsahīnaś ca niścitaṃ mukta eva saḥ ||64||\\!}
%\end{versinnote}
\end{testimonia}

%\begin{philcomm}[hp04_035_8]
%\end{philcomm}

%%%%%%%%%%
\subsection*{4.36}
\begin{translation}[hp04_036]
Only the glorious guru lord knows the unique ineffable bliss that has arisen in the hearts of lords among yogis who experience \textit{samādhi} by concentrating on the inner sound.
\end{translation}
% JB: Perhaps a note is needed on yogīśvara (here or 4.27). Brahmānanda takes yogīśvarāḥ as yogiṣu yogayukteṣu īśvarāḥ samarthāḥ. i.e., 'lords among yogis' or perhaps 'adepts among yogis'?
%% Jayaratha (TĀ 2.43) yogīśvaraḥ = yoginām apīśvaraḥ 'a lord of even yogis'


\begin{sources}[hp04_036]
\emph{Yogatārāvalī} 3
\begin{versinnote}
\tl{nādānusandhānasamādhibhājāṃ yogīśvarāṇāṃ hṛdaye pragūḍham |\\+}
\tl{ānandamātraṃ vacasām avācyaṃ jānāti taṃ śrīgurunātha ekaḥ ||\\+}
\tl{\var{ekaḥ ] Pa : eva Ad}\\!}
\end{versinnote}
\end{sources}

\begin{testimonia}[hp04_036]
\emph{Haṭharatnāvalī} 4.5
\begin{versinnote}
\tl{nādānusandhānasamādhibhājāṃ yogīśvarāṇāṃ hṛdaye prarūḍham |\\+}
\tl{ānandam ekaṃ vacaso 'py agamyaṃ jānāti taṃ śrīgurunātha eva ||\\!}
\end{versinnote}

\emph{Yogacintāmaṇi} f.~24r (attr. to the \emph{Haṭhapradīpikā})
\begin{versinnote}
\tl{nādānusandhānasamādhibhājāṃ
yogīśvarāṇāṃ hṛdaye prarūḍhaṃ |\\+}
\tl{ānandam ekaṃ vacasām avācyaṃ
jānāti tatvaṃ gurunātha eva ||\\!}
\end{versinnote}

\end{testimonia}

%\begin{philcomm}[hp04_036]
%Check consistency on tr. of anusandhāna.
%\end{philcomm}

\begin{metre}[hp04_036]
Upajāti
\end{metre}

%%%%%%%%%%
\subsection*{4.36*1}
\begin{translation}[hp04_036_1]
Seated in the pose of the liberated, the yogi should adopt \emph{śāṃbhavī mudrā} and listen continuously to the inner sound in his right ear.
\end{translation}
%%JB: I've changed our translation of muktāsana to the pose of the liberated (instead of 'liberated ones') in keeping with the way we translated it in ch. 1

%\begin{sources}[hp04_036_1]
%\end{sources}

\begin{testimonia}[hp04_036_1]
\emph{Yogacintāmaṇi} f.~23v (attr. to the \emph{Haṭhapradīpikā})
\begin{versinnote}
\tl{muktāsanasthito yogī mudrāṃ sandhāya śāṃbhavīm |\\+}
\tl{śṛṇuyād dakṣiṇe karṇe nādam antargataṃ sadā ||\\!}
\end{versinnote}

\emph{Haṭhasaṅketacandrikā} f.~124r (attr. to the \emph{Haṭhapradīpikā})
\begin{versinnote}
\tl{muktāsanasthito yogī mudrāṃ saṃdhāya śāṃbhavīṃ [|]\\+}
\tl{śṛṇuyād dakṣiṇe karṇe nādam ekāṃtike sudhīḥ  [||]\\+}
\tl{\var{dakṣiṇe karṇe ] B220,  dakṣirṇe 2244}\\!}
\end{versinnote}

\emph{Nādabindūpaniṣat} 31
\begin{versinnote}
\tl{siddhāsane sthito yogī mudrāṃ sandhāya vaiṣṇavīm |\\+}
\tl{śṛṇuyād dakṣiṇe karṇe nādam antargataṃ sadā ||\\!}
\end{versinnote}
\end{testimonia}

\begin{philcomm}[hp04_036_1]
See 4.13.
\end{philcomm}

%%%%%%%%%%
\subsection*{4.37}
\begin{translation}[hp04_037]
[The yogi] who desires yogic sovereignty should abandon all thought and concentrate with an attentive mind on nothing but the internal sound.
\end{translation}

%\begin{sources}[hp04_037]
%\end{sources}

\begin{testimonia}[hp04_037]
\emph{Haṭharatnāvalī} 4.14
\begin{versinnote}
\tl{sarvacintāṃ parityajya sāvadhānena cetasā |\\+}
\tl{nāda evānusandheyaḥ yogasāmrājyasiddhaye ||\\!}
\end{versinnote}

\emph{Yogacintāmaṇi} f.~23v (attr. to the \emph{Haṭhapradīpikā})
\begin{versinnote}
\tl{sarvacintāṃ parityajya sāvadhānena cetasā |\\+}
\tl{nādam evānusandhatte yogasāmrājyam icchatā ||\\+}
\tl{\var{°sāmrājyam icchatā ] U, °sāmrājyadhiṣṭhitaḥ N}\\!}
\end{versinnote}

\emph{Haṭhasaṅketacandrikā} f. 124r (attr. to the \emph{Haṭhapradīpikā})
\begin{versinnote}
\tl{sarvacittaṃ parityajya sāvadhānena cetasā |\\+}
\tl{nāda evānusaṃdheyo yogasāmrājyam icchatā ||\\!}
\end{versinnote}
\end{testimonia}

%\begin{philcomm}[hp04_037]
%N3 icchatā
%yogasāmrājya = rājayogasiddhi
%numbering is array in the apparatus
%\end{philcomm}

%%%%%%%%%%
\subsection*{4.38}
\begin{translation}[hp04_038]
The sage should block his ears with cotton and fix the mind on the sound which he hears, until he attains a state of stillness.
\end{translation}

%\begin{sources}[hp04_038]
%\end{sources}

\begin{testimonia}[hp04_038]
\emph{Haṭharatnāvalī} 4.8
\begin{versinnote}
\tl{karṇau pidhāya tūlena yaḥ śṛṇoti dhvaniṃ yamī |\\+}
\tl{tatra cittaṃ sthiraṃ kuryād yāvat sthirapadaṃ vrajet ||\\!}
\end{versinnote}

\emph{Yogacintāmaṇi} f.~24r (attr. to the \emph{Haṭhapradīpikā})
\begin{versinnote}
\tl{karṇau pidhāya hastena yaḥ śṛṇoti dhvaniṃ muniḥ |\\+}
\tl{tāvac cittaṃ sthiraṃ kuryād yāvat sthirapadaṃ vrajet ||\\!}
\end{versinnote}

\emph{Haṭhasaṅketacandrikā} f.~124r
\begin{versinnote}
\tl{karṇau pidhāya hastābhyāṃ yaś śṛṇoti dhvaniṃ muniḥ |\\+}
\tl{tatra cittaṃ sthiraṃ kuryād yāvat sthirapadaṃ vrajet ||\\!}
\end{versinnote}
\end{testimonia}

\begin{philcomm}[hp04_038]
%
%mūlena (alpha) probably came from tūlena. The hasta variations seem impractible because the yogi would have to hold the arms up, but may have been adopted by yogis who did not have access to cotton (?).JM: I think these comments are unnecessary (and surely it is possible/practicable to use the fingers?). 

%JB: I've tweaked the note below to inform the reader why we've adopted tūlena.
The reading \emph{tūlena} (`with cotton'), which is attested by \alphaThree, makes good sense and is close to the reading of \emph{mūlena} in \alphaOne\ and \alphaTwo. Manuscripts of several other groups instead have \emph{hastābhyām} (`with the hands') or \emph{hastena} (`with the hands'). This reading was inspired by the technique of blocking the ears and other orifices with the fingers in order to listen to the inner sounds. This practice is attested as early as the \textit{Svacchandatantra} in which it is called \textit{ṣaṇmukhīkaraṇa} (Vasudeva 2004: 272 n.66). In this \emph{karaṇa}, the other openings of the head are also blocked with the fingers. \emph{Śivasaṃhitā} 5.36–46 teaches a similar practice. In the \textit{Haṭhayogasaṃhitā} (p. 68), the practice of blocking the ears with the hands is stipulated for \textit{bhrāmarī kumbhaka}.
\end{philcomm}

%%%%%%%%%%
\subsection*{4.39}
\begin{translation}[hp04_039]
When this inner sound is being cultivated, it drowns out external sound. After a fortnight the yogi overcomes all distraction and becomes happy.
\end{translation}

%\begin{sources}[hp04_039]
%\end{sources}

\begin{testimonia}[hp04_039]
\emph{Yogacintāmaṇi} f.~24r (attr. to the \emph{Haṭhapradīpikā})
\begin{versinnote}
\tl{abhyasyamāno nādo 'yaṃ bāhyam āvartayed dhvanim |\\+}
\tl{paścād vikṣepam akhilaṃ jitvā yogī sukhī bhavet ||\\!}
\end{versinnote}

\emph{Haṭhasaṅketacandrikā} f.~124r
\begin{versinnote}
\tl{abhyasyamāno nādo 'yaṃ bāhyam āvarttayet dhvaniṃ |\\+}
\tl{pakṣād vikṣepam akhilaṃ jitvā yogī sukhībhavet ||\\!}
\end{versinnote}

\emph{Nādabindūpaniṣat} 32
\begin{versinnote}
\tl{abhyasyamāno nādo 'yaṃ bāhyam āvṛṇute dhvanim  |\\+}
\tl{pakṣād vipakṣam akhilaṃ jitvā turyapadaṃ vrajet ||\\!}
\end{versinnote}
\end{testimonia}

%\begin{philcomm}[hp04_039]
%\end{philcomm}

\begin{metre}[hp04_039]
Anuṣṭubh (a: ma-vipulā; c: na-vipulā)
\end{metre}

%%%%%%%%%%
\subsection*{4.40}
\begin{translation}[hp04_040]
In the first stage of practice, a loud sound of various kinds is heard. Then, as the practice progresses, a quieter and quieter sound is heard.
\end{translation}

%\begin{sources}[hp04_040]
%\end{sources}

\begin{testimonia}[hp04_040]
\emph{Haṭharatnāvalī} 4.9
\begin{versinnote}
\tl{śrūyate prathamābhyāse nādo nānāvidho bahuḥ |\\+}
\tl{vardhamāne tato 'bhyāse śrūyate sūkṣmasūkṣmataḥ ||\\!}
\end{versinnote}

\emph{Yogacintāmaṇi} f.~24r (attr. to the \emph{Haṭhapradīpikā})
\begin{versinnote}
\tl{śrūyate prathamābhyāse nādo nānāvidho mahān |\\+}
\tl{vartamāne tato 'bhyāse śrūyate sūkṣmasūkṣmataḥ ||\\!}
\end{versinnote}

\emph{Haṭhatattvakaumudī} 54.31
\begin{versinnote}
\tl{śrūyate prathamābhyāse nādo nānāvidho bahuḥ |\\+}
\tl{vardhamāne tato 'bhyāse śrūyate sūkṣmasūkṣmataḥ ||\\!}
\end{versinnote}
\end{testimonia}

%\begin{philcomm}[hp04_040]
%Adopt sūkṣmasūkṣmataḥ (alpha)[MD: Done.]
%sūkṣma seems to mean more than just quiet.
%\end{philcomm}

%%%%%%%%%%
\subsection*{4.41}
\begin{translation}[hp04_041]
In the first stage, [the sounds] are those that are produced by the ocean, a [storm] cloud, a kettle drum and a waterfall. In the intermediate stage, they are [the sounds] produced by a bass drum and conch and a bell and trumpet.
\end{translation}

%\begin{sources}[hp04_041]
%\end{sources}

\begin{testimonia}[hp04_041]
\emph{Haṭharatnāvalī} 4.10
\begin{versinnote}
\tl{ādau jaladhijīmūtabherīnirjharasaṃbhavāḥ |\\+}
\tl{madhye marddalaśaṃkhotthā ghaṇṭākāhalakās tathā ||\\!}
\end{versinnote}

\emph{Yogacintāmaṇi} f.~24r (attr. to the \emph{Haṭhapradīpikā})
\begin{versinnote}
\tl{ādau jaladhijīmūtabherījharjharasaṃbhavaḥ |\\+}
\tl{madhye mardalaśaṅkhottha ghaṇṭākāhalakās tathā ||\\!}
\end{versinnote}    

\emph{Haṭhatattvakaumudī} 54.32
\begin{versinnote}
\tl{ādau jaladhijīmūtabherīnirjharasambhavaḥ |\\+}
\tl{madhye marddalaśaṃkhotthā ghaṃṭākāhalakās tathā ||\\!}
\end{versinnote}
\end{testimonia}

\begin{philcomm}[hp04_041]
We have translated \emph{kāhala} as `trumpet' on the basis of e.g. \emph{Viśvalocanakośa}, \emph{lāntavarga} 161 (\emph{dhvaninālā tu vīṇāyāṃ veṇukāhalayor api}), but it can also mean a type of drum (see e.g. \emph{Śabdakalpadruma} s.v. \emph{kāhala} where it is said to be a \emph{bṛhaḍḍhakkā}, a big drum).%whose mouth looks like a datura flower. NWS: "a musical instrument; a trumpet. EI 24 (Index) (Sircar 1966: 138)".
%Schmidt, Nachträge: "eine Art Trommel (\emph{kāhalāsu dhattūrapuṣpākāramukhabherīṣu}), \emph{Yaśastilakam}  (nach dem Ko.) = asphuṭaśabda" [\ldots]. 
\end{philcomm}

%%%%%%%%%%
\subsection*{4.42}
\begin{translation}[hp04_042]
In the final stage, there are the sounds of little bells, a bamboo flute, a veena and a bee. These various sound are heard in the body.
\end{translation}

%\begin{sources}[hp04_042]
%\end{sources}

\begin{testimonia}[hp04_042]
\emph{Haṭharatnāvalī} 4.11 
\begin{versinnote}
\tl{ante tu kiṃkiṇīvṛndavīṇābhramaraniḥsvanāḥ |\\+}
\tl{iti nānāvidhā nādāḥ śrūyante dehamadhyataḥ ||\\!}
\end{versinnote}

\emph{Yogacintāmaṇi} f.~24r (attr. to the \emph{Haṭhapradīpikā})
\begin{versinnote}
\tl{anye tu kiṅkiṇīvṛndavīṇābhramaraniḥsvanāḥ |\\+}
\tl{iti nānāvidho nādaḥ śrūyate dehamadhyagaḥ ||\\!}
\end{versinnote}

\emph{Haṭhatattvakaumudī} 54.33
\begin{versinnote}
\tl{ante tu kiṃkiṇī vaṃśanādā bhramaraniḥsvanāḥ |\\+}
\tl{iti nānāvidhā nādāḥ śrūyante yatra madhyataḥ ||\\!}
\end{versinnote}
\end{testimonia}

%\begin{philcomm}[hp04_042]
%\end{philcomm}

%%%%%%%%%%
\subsection*{4.43}
\begin{translation}[hp04_043]
Even if a loud noise such as that of a [storm] cloud or kettle drum is being heard, the [yogi] should concentrate on only the very quietest sound in it.
\end{translation}

%\begin{sources}[hp04_043]
%\end{sources}

\begin{testimonia}[hp04_043]
\emph{Yogacintāmaṇi} f.~24r (attr. to the \emph{Haṭhapradīpikā})
\begin{versinnote}
\tl{mahati śrūyamāṇe 'pi meghabheryādike dhvanau |\\+}
\tl{tataḥ sūkṣmāt sūkṣmataraṃ nādam eva parāmṛśet ||\\!}
\end{versinnote}

\emph{Haṭhatattvakaumudī} 54.34
\begin{versinnote}
\tl{mahati śrūyamāṇe 'pi meghabheryādike svane |\\+}
\tl{tatra sūkṣmāt sūkṣmataraṃ nādam eva parāmṛśet ||\\!}
\end{versinnote}
\end{testimonia}

%\begin{philcomm}[hp04_043]
%\end{philcomm}

\begin{metre}[hp04_043]
Anuṣṭubh (c: bha-vipulā)
\end{metre}

%%%%%%%%%%
\subsection*{4.44}
\begin{translation}[hp04_044]
Or, the [yogi] should filter out the gross sound for the subtle, or the subtle for the gross, or, abandoning both, be [focused] on [a sound] in the middle [and] not move the mind elsewhere.%
\end{translation} %
% MD: how did we understand syādvā?
%\begin{sources}[hp04_044]
%\end{sources}

\begin{testimonia}[hp04_044]
\emph{Yogacintāmaṇi} f.~24r (attr. to the \emph{Haṭhapradīpikā})
\begin{versinnote}
\tl{ghanam utsṛjya vā sūkṣmaṃ sūkṣmaṃ pramṛjya vā ghanam |\\+}
\tl{paraṃ tatraiva niḥkṣipya mano nānyatra cālayet ||\\!}
\end{versinnote}

\emph{Haṭhatattvakaumudī} 54.35
\begin{versinnote}
\tl{ghanam utsṛjya vā sūkṣme sūkṣmam utsṛjya vā ghane |\\+}
\tl{ramamāṇam api kṣipraṃ mano nātra pracālayet ||\\!}
\end{versinnote}

\emph{Nādabindūpaniṣat} 37
\begin{versinnote}
\tl{ghanam utsṛjya vā sūkṣme sūkṣmam utsṛjya vā ghane |\\+}
\tl{ramamāṇam api kṣiptaṃ mano nānyatra cālayet ||\\!}
\end{versinnote}
\end{testimonia}

%\begin{philcomm}[hp04_044]
%\end{philcomm}

%%%%%%%%%%
\subsection*{4.45}
\begin{translation}[hp04_045]
Alternatively, the mind fixes upon whatever sound it first attaches to and dissolves together with it.
\end{translation}

%\begin{sources}[hp04_045]
%\end{sources}

\begin{testimonia}[hp04_045]
\emph{Yogacintāmaṇi} f.~24r (attr. to the \emph{Haṭhapradīpikā})
\begin{versinnote}
\tl{yatra kutrāpi vā nāde prathamaṃ viśate manaḥ |\\+}
\tl{tatraiva susthiraṃ kuryāt tena sārdhaṃ vilīyate ||\\!}
\end{versinnote}

\emph{Haṭhatattvakaumudī} 54.36
\begin{versinnote}
\tl{yatra kutrāpi vā nāde lagati prathamaṃ manaḥ |\\+}
\tl{tatraiva susthirībhūtvā tena sārdhaṃ vilīyate ||\\!}
\end{versinnote}

\emph{Nādabindūpaniṣat} 37
\begin{versinnote}
\tl{yatra kutrāpi vā nāde lagati prathamaṃ manaḥ  |\\+}
\tl{tatra tatra sthirībhūtvā tena sārdhaṃ vilīyate ||\\!}
\end{versinnote}
\end{testimonia}

%\begin{philcomm}[hp04_045]
%\end{philcomm}

%%%%%%%%%%
\subsection*{4.46}
\begin{translation}[hp04_046]
Just as a bee drinking nectar has no regard for fragrances, so the mind attached to the inner sound does not desire the objects of the senses.
\end{translation}

%\begin{sources}[hp04_046]
%\end{sources}

\begin{testimonia}[hp04_046]
\emph{Haṭharatnāvalī} 4.12
\begin{versinnote}
\tl{makarandaṃ pibed bhṛṅgo gandho na prekṣyate yathā |\\+}
\tl{nādāsaktaṃ tathā cittaṃ viṣayān na hi kāṃkṣate ||\\!}
\end{versinnote}

\emph{Yogacintāmaṇi} f.~24r (attr. to the \emph{Haṭhapradīpikā})
\begin{versinnote}
\tl{makarandaṃ piban bhṛṅgo gandhān nāpekṣate yathā |\\+}
\tl{nādāsaktaṃ tathā cittaṃ viṣayān na hi kāṅkṣati ||\\!}
\end{versinnote}

\emph{Haṭhatattvakaumudī} 54.41
\begin{versinnote}
\tl{makarandaṃ piban bhṛṃgo gandhaṃ nāpekṣate yathā |\\+}
\tl{nādāsaktaṃ tathā cittaṃ viśayān naiva kāṃkṣati ||\\!}
\end{versinnote}

\emph{Nādabindūpaniṣat} 42
\begin{versinnote}
\tl{makarandaṃ piban bhṛṅgo gandhān nāpekṣate tathā |\\+}
\tl{nādāsaktaṃ sadā cittaṃ viṣayaṃ na hi kāṅkṣati ||\\!}
\end{versinnote}
\end{testimonia}

%\begin{philcomm}[hp04_046]
%gandhān is the alpha reading and is possible. [MD: adopted]
%\end{philcomm}

%%%%%%%%%%
\subsection*{4.47}
\begin{translation}[hp04_047]
When the mercury of the mind is bound and has cast off its fickle nature because it has been assimilated with the sulphur of the internal resonance, it attains the immobility called the unsupported (i.e., \textit{samādhi}).
\end{translation}

%\begin{sources}[hp04_047]
%\end{sources}

\begin{testimonia}[hp04_047]
\emph{Yogacintāmaṇi} f.~26v (attr. to the \emph{Haṭhapradīpikā})
\begin{versinnote}
\tl{purā matsyendrabodhāya ādināthoditavacaḥ |\\+}
\tl{manaḥpākam avāpnoti nirālambākhyaghoṭanam ||\\!}
\end{versinnote}

\emph{Haṭhatattvakaumudī} 54.42
\begin{versinnote}
\tl{baddhaṃ viyuktaṃ cāpalyaṃ nādagandhakajāraṇāt |\\+}
\tl{manaḥpāradam āpnoti nirālambākhyakheṭakam ||\\!}
\end{versinnote}
\end{testimonia}

\begin{philcomm}[hp04_047]
See Hellwig 2009: 204–206 on \emph{khoṭa}, “lame”, which in alchemy is a technical term use to describe mercury that has been processed many times using the \emph{māraṇa} technique and no longer moves.
\end{philcomm}

%%%%%%%%%%
\subsection*{4.47*1}
\begin{translation}[hp04_047_1]
Bound by the sulphur of the inner sound, the lord that is the mercury of the mind immediately casts off its fickle nature and attains fame as “[the bird] with clipped wings”.
\end{translation} %

%\begin{sources}[hp04_047_1]
%\end{sources}

\begin{testimonia}[hp04_047_1]
\emph{Haṭhatattvakaumudī} 54.43
\begin{versinnote}
\tl{baddhaḥ sugandhanādena sadyaḥ santyaktacāpalaḥ |\\+}
\tl{prayāti sūtacittendraḥ pakṣacchinna ivāprabhaḥ ||\\!}
\end{versinnote}
\end{testimonia}

\begin{philcomm}[hp04_047_1]
On \emph{pakṣaccheda} in alchemical processes of immobilizing mercury and for references in Rasaśāstra, see Hellwig 2009: 276–278.

% Hellwig, O. (2009). Wörterbuch der mittelalterlichen indischen Alchemie. Barkhuis ; University of Groningen.


%Cf.~\emph{Rasendracūḍāmaṇi} 16.52--55
%\begin{versinnote}
%\tl{evaṃ ca pañcamo grāsaḥ pradātavyo 'ṣṭamāṃśataḥ |\\+} 
%\tl{sa pātrastho 'gnisaṃtapto na gacchati kathañ cana ||16.52||\\+}
%\tl{sa \textit{pakṣacchinna} ity uktaḥ sa mukto 'khiladurguṇaiḥ |\\+}
%\tl{so 'yaṃ niṣevitaḥ sūtas trimāsaṃ rājikāmitaḥ ||16.53||\\+}
%\tl{viḍaṅgatriphalākṣaudraiḥ khe devaiḥ saha saṅgamam |\\+}
%\tl{ghrāṇamātreṇa \textit{sūtendraḥ} sarvaroganikṛntanaḥ ||16.54||\\+}
%\tl{guṇā ete vinirdiṣṭā rasasya rasavādibhiḥ |\\+}
%\tl{sakalāste guṇāḥ satyā bhairaveṇa prakīrtitāḥ ||16.55||\\!}
%\end{versinnote}

%Cf.~also NWS (ref) s.v.~\textit{pakṣaccheda}.

\end{philcomm}

\begin{metre}[hp04_047_1]
Anuṣṭubh (c: ma-vipulā)
\end{metre}

%%%%%%%%%%
\subsection*{4.48}
\begin{translation}[hp04_048]
As a result of listening to the inner sound, the snake that is the mind forgets everything and, one-pointed, does not dart off anywhere.
\end{translation}

% MD: Why do some good mss have saṃsmṛtya for vismṛtya

%\begin{sources}[hp04_048]
%\end{sources}

\begin{testimonia}[hp04_048]
\emph{Yogacintāmaṇi} f.~26v (attr. to the \emph{Haṭhapradīpikā})
\begin{versinnote}
\tl{nādaśravaṇataś cittam antaraṅgaturaṅgamaḥ |\\+}
\tl{viśūnyaṃ sarvam ekāgryaṃ kutra cin na hi dhāvati ||\\!}
\end{versinnote}

\emph{Haṭhatattvakaumudī} 54.44
\begin{versinnote}
\tl{nādaśravaṇataś cittam antaraṃgakuraṃgakaḥ |\\+}
\tl{vismṛtya viśvam ekāgraḥ kutra cin na hi dhāvati ||\\!}
\end{versinnote}
\end{testimonia}

\begin{philcomm}[hp04_048]
The metaphor is that of snake charming: beguiled by the inner sound, the snake that is the mind becomes transfixed. Witnesses of the \textgamma \ and \textdelta \ groups have \emph{turaṅgamaḥ} instead of \emph{bhujaṅgamaḥ}, perhaps because forms from \emph{dhāv} are unusual with the latter, but it is found at e.g.~\emph{Garuḍapurāṇa} (1.113.33ab).\lb

In the third \emph{pāda} we have read against \alphaOne\ and \alphaTwo\ (as well as \betaTwo\ and \epsilonOne), which have \emph{saṃsmṛtya sarvam} instead of \emph{vismṛtya sarvam}. The former could be understood to mean “with complete concentration” but we have adopted \emph{vismṛtya} on semantic grounds.
% JB: but we have adopted \emph{vismṛtya} because its meaning here is more consistent with the description of \textit{nādānusandhāna} in 4.15ab. 
% Padmapurāṇa
\end{philcomm}

%%%%%%%%%%
\subsection*{4.49}
\begin{translation}[hp04_049]
This inner sound is a sharpened goad with the power to restrain the bull elephant in must that is the mind as it wanders about in the garden of the sense objects.
\end{translation} %

%\begin{sources}[hp04_049]
%\end{sources}

\begin{testimonia}[hp04_049]
\emph{Yogacintāmaṇi} f.~23r (attr. to the \emph{Haṭhapradīpikā})
\begin{versinnote}
\tl{manomattagajendrasya viṣayodyānacāriṇaḥ |\\+}
\tl{niyāmanasamartho 'yaṃ ninādo niśitāṅkuśaḥ ||\\!}
\end{versinnote}

\emph{Nādabindūpaniṣat} 44cd–45ab
\begin{versinnote}
\tl{manomattagajendrasya viṣayodyānacāriṇaḥ ||\\+}
\tl{niyāmanasamartho 'yaṃ ninādo niśitāṅkuśaḥ |\\!}
\end{versinnote}
\end{testimonia}

\begin{philcomm}[hp04_049]
The unusual form \emph{niyāmana} is also found in Rasaśāstra works where it occurs in the context of restraining mercury and is a topic of discussion (e.g.~\emph{Rasaprakāśasudhākara} 1.23, \emph{Ānandakanda} 1.4.58–59).
\end{philcomm}


%%%%%%%%%%
\subsection*{4.50}
\begin{translation}[hp04_050]
Cultivation of the inner sound is a bolt for [the stable door of] the swift horse of the mind so the yogi should regularly focus on it.
\end{translation} 

%\begin{sources}[hp04_050]
%\end{sources}

\begin{testimonia}[hp04_050]
\emph{Haṭhatattvakaumudī} 54.46
\begin{versinnote}
\tl{antaraṃgaturaṃgasya vājinaḥ paridhāvataḥ |\\+}
\tl{nādopāstikhalīnaṃ hi niyāmanakaraṃ dṛḍham ||\\!}
\end{versinnote}
\end{testimonia}

\begin{philcomm}[hp04_050]
%We understand \textit{ninādaḥ}, which is the subject of the previous line, as the subject in the first line of this verse. Several witness have \emph{paridhāyate} instead of \emph{parighāyate}. The latter is unattested but could mean “is a halter" based on \emph{abhidhānī}’s meaning of “halter”.
\end{philcomm}

\begin{metre}[hp04_050]
Anuṣṭubh (a: na-vipulā)
\end{metre}

%%%%%%%%%%
\subsection*{4.50*1}
\begin{translation}[hp04_050_1]
The inner sound is a net for trapping the deer of the mind and a hunter for corraling the antelope of the mind.
\end{translation}
%  JB: we have not translated antaraṅga° in pāda a. Change to 'trapping the inner deer that is the mind'? JM: antaraṅga = mind
% JB: acts as a net (vāgurāyate) or becomes a net
%?? MD: Pāda c turaṅga is better attested than kuraṅga.

%\begin{sources}[hp04_050_1]
%\end{sources}

\begin{testimonia}[hp04_050_1]
\emph{Yogacintāmaṇi} f.~26v (attr. to the \emph{Haṭhapradīpikā})
\begin{versinnote}
\tl{nādo 'ntaraṅgasāraṅgabandhane vāgurāyate ||\\+}
\tl{antaraṅgaturaṅgasya bandhane līyate 'pi ca ||\\!}
\end{versinnote}

\emph{Haṭhatattvakaumudī} 47
\begin{versinnote}
\tl{nādo'ntaraṃgasāraṃgabandhane vāgurāyate |\\+}
\tl{antaraṃgakuraṃgasya nādo vyādhāyate 'pi ca  ||\\!}
\end{versinnote}
\end{testimonia}

\begin{philcomm}[hp04_050_1]
%Adopt V15 67cd
%
\end{philcomm}



%%%%%%%%%%
\subsection*{4.51}
\begin{translation}[hp04_051]
Striking the deer of the mind when, focused upon inner sounds such as that of a bell, it is transfixed, is very easy if the archer is skilful.
\end{translation}
% ghaṇṭādinādasaktastabdhāntaḥkaraṇahariṇasya |
% praharaṇam atisukaraṃ syāc charasaṃdhātā pravīṇaś cet ||
%MD: I would prefer nādasakta to °sakti. The latter is not well attested and sounds unnatural to me. If you take ghaṇṭā°...°sakta and stabdhāntaḥkaraṇa as a dvandva compound, there is no grammatical problem.

%\begin{sources}[hp04_051]
%\end{sources}

%\begin{testimonia}[hp04_051]
%\end{testimonia}

%\begin{philcomm}[hp04_051]
%N3 has a problem with the metre.
%\end{philcomm}

\begin{metre}[hp04_051]
Upagīti 
\end{metre}


%%%%%%%%%%
\subsection*{4.52}
\begin{translation}[hp04_052]
The tone of that sound is that of the unstruck sound. A light is inside the tone [and] the mind is inside the light. That mind dissolves. That is the supreme state of Viṣṇu.
\end{translation}

\begin{sources}[hp04_052]
\emph{Uttaragītā} 41cd–42 %(source) JB: I think we can say that the UG is the source of 4.52*1 (but not 4.52)
\begin{versinnote}
\tl{anāhatasya śabdasya tasya śabdasya yo dhvaniḥ ||\\+}
\tl{dhvaner antargataṃ jyotir jyotirantargataṃ manaḥ |\\+}
\tl{tan mano vilayaṃ yāti tad viṣṇoḥ paramaṃ padam  ||\\!}
\end{versinnote}
\end{sources}

\begin{testimonia}[hp04_052]
\emph{Yogacintāmaṇi} f.~26v (attr. to the \emph{Haṭhapradīpikā})
\begin{versinnote}
\tl{anāhatasya śabdasya tasya śabdasya yo dhvaniḥ ||\\+}
\tl{dhvaner antargataṃ jyotir jyotirantargataṃ manaḥ |\\+}
\tl{yan mano vilayaṃ yāti tad viṣṇoḥ paramaṃ padam ||\\!}
\end{versinnote}

\emph{Haṭhayogasaṃhitā} p.~68
\begin{versinnote}
\tl{anāhatasya śabdasya tasya śabdasya yo dhvaniḥ |\\+}
\tl{dhvaner antargataṃ jyotir jyotiṣo 'ntargataṃ manaḥ ||\\!}
\end{versinnote}
\end{testimonia}

\begin{philcomm}[hp04_052]
% to discuss. It appears that the following lines were added (in Gr2 and Gr3) as a note or supplement to 4.68. In particular, the hemistich anāhatasya śabdasya tasya... seems to comment on anāhatadhvaner.
%\begin{versinnote}
%\tl{anāhatasya śabdasya tasya śabdasya yo dhvaniḥ ||\\+}
%\tl{dhvaner antargataṃ jyotir jyotirantargataṃ manaḥ |\\+}
%\tl{tanmano vilayaṃ yāti tad viṣṇoḥ paramaṃ padam  || \\!}
%\end{versinnote}
The source of these lines may be the \emph{Uttaragītā} as they occur in a published version. However, in a manuscript of the \emph{Uttaragītā} (NGMPP E 2098-11) these three lines are omitted from Kṛṣṇa's words, which start with \emph{oṃkāra}. The author of the \emph{Upāsanāsārasangraha} (f. 111) quoted these lines and attributed them to the \emph{Gītāsāra}. 
%HI: tan could mean tatra.
%Looks like the last line was rewritten to include tatra.

% MD: I adopted the version transmitted in Gr4c and J10 (anāhatadhvaner antar jñeyaṃ yat sūkṣmasūkṣmakaṃ ...).
% Which is the more likely reading of the second line for the alternative version?
%A - dhvaner antargataṃ jyotir jyotirantargataṃ manaḥ (J5,G4,Gr2,Gr3)
%B - dhvaner antargataṃ jñeyaṃ jñeyasyāntargataṃ manaḥ (N3?,F,Jyo) JM: the second I think the second. J

% & Note about the note for JB
% switch 4.49 with 4.49*1
% Explain why we have adopted Alt2 (probably borrowed from the Uttaragītā) and the likelihood that Alt 1 was an attempt to replace it with something more coherent.
% We are not sure what the difference between dhvani and śabda is... and have translated them as ...
% MD 2024-09: G11 has both versions! It seems that the editor of the extended HP created the 4-pāda version, but kept the 6-pāda version.
\end{philcomm}


%%%%%%%%%%
\subsection*{4.52*1}
\begin{translation}[hp04_052]
When the mind dissolves into that which is the most subtle object of perception in the unstruck sound, that is the supreme state of Viṣṇu.
\end{translation}
% anāhatadhvaner antar jñeyaṃ yat sūkṣmasūkṣmakaṃ/
% manas tatra layaṃ yāti tad viṣṇoḥ paramaṃ padam//

%\begin{sources}[hp04_052]
%\end{sources}

\begin{testimonia}[hp04_052*1]
\emph{Hathatattvakaumudī} 54.48
\begin{versinnote}
\tl{anāhatadhvaner antar jñeyaṃ yat sūkṣmasūkṣmakaṃ ||\\+}
\tl{manas tatra layaṃ yāti tad viṣṇoḥ paramaṃ padam ||\\!}
\end{versinnote}

\end{testimonia}

%\begin{philcomm}[hp04_052*1]
%HI: tan could mean tatra.
%Looks like the last line was rewritten to include tatra
% MD: I adopted the version transmitted in Gr4c and J10 (anāhatadhvaner antar jñeyaṃ yat sūkṣmasūkṣmakaṃ ...).
% JB I have divided 4.49 and 4.49*1, and changed the sources and testimonia accordingly. But it seems that alpha has 4.49*1 and not 4.49, so have we made a mistake in greyscaling 4.49*1? A note is definately needed, but I'm not sure what our thinking is here
%\end{philcomm}

%%%%%%%%%%
\subsection*{4.53}
\begin{translation}[hp04_053]
As long as sound exists, there is a concept of space. That which is soundless is the supreme Brahman and is called the supreme self.
\end{translation}
% adopt samīryate J7, V3

\begin{sources}[hp04_053]
\emph{Vivekamārtaṇḍa} (six-chapters) 5.15 %JB: This was in testimonia but it must be the source as we now know that Svātmārāma used the 6ch VM.
\begin{versinnote}
\tl{tāvad ākāśasaṅkalpo yāvac chabdaḥ pravartate |\\+}
\tl{niḥśabdaṃ tat parabrahma paramātmā sa gīyate || 15 ||\\!}
\end{versinnote}
\end{sources}

\begin{testimonia}[hp04_053]
\emph{Yogacintāmaṇi} f.~27r (attr. to the \emph{Haṭhapradīpikā})
\begin{versinnote}
\tl{tāvad ākāśasaṃkalpo yāvac chabdaḥ pravartate |\\+}
\tl{niḥśabdaṃ tatparaṃ brahma paramātmā samīryate ||\\!}
\end{versinnote}

\emph{Nādabindūpaniṣat} 47cd–48ab
\begin{versinnote}
\tl{tāvadākāśasaṅkalpo yāvacchabdaḥ pravatate |\\+}
\tl{niḥśabdaṃ tatparaṃ brahma paramātmā samīryate ||\\!}
\end{versinnote}
\end{testimonia}

%\begin{philcomm}[hp04_053]
%\end{philcomm}
%%%%%%%%%%
\subsection*{4.54}
\begin{translation}[hp04_054]
Whatever is heard as the inner sound is nothing but Śakti. The formless one which hears it is nothing but the supreme lord.
\end{translation}
% adopt V3 reading nādarupeṇa


%\begin{sources}[hp04_054]
%\end{sources}

\begin{testimonia}[hp04_054]
\emph{Yogacintāmaṇi} f.~27r (attr. to the \emph{Haṭhapradīpikā})
\begin{versinnote}
\tl{yat kiñ cin nāmarūpeṇa śrūyate śaktir eva sā |\\+}
\tl{yas tacchrotā nirākāraḥ sa eva parameśvaraḥ ||\\!}
\end{versinnote}

\emph{Haṭhasaṅketacandrikā} f.~123r (attr. to the \emph{Haṭhapradīpikā})
\begin{versinnote}
\tl{yat kiṃ cin nāmarūpeṇa śrūyate śaktir eva sā |\\+}
\tl{yasya śrottā nirākaraḥ sa eva parameśvaraḥ ||\\!}
\end{versinnote}
\end{testimonia}

%\begin{philcomm}[hp04_054]
%\end{philcomm}

%%%%%%%%%%
\subsection*{4.55}
\begin{translation}[hp04_055]
Blocking of the ears, mouth, eyes, and nose should not be performed [when] a clear inner sound is heard distinctly in the purified Suṣumṇā channel.
\end{translation}

%\begin{sources}[hp04_055]
%\end{sources}

\begin{testimonia}[hp04_055]
\emph{Yogacintāmaṇi} f.~26v
\begin{versinnote}
\tl{haṭhapradīpikāyām—\\+}
\tl{śravaṇapuṭanayananāsāpuṭarodhanaṃ kāryam |\\+}
\tl{śrīśuddhasuṣumṇāsaraṇau sphuṭam amalaḥ śrūyate nādaḥ ||\\!}
\end{versinnote}

\emph{Haṭhasaṅketacandrikā} 123v–124r
\begin{versinnote}
\tl{haṭhapradīpikāyāṃ \\+}
\tl{śravaṇapuṭanayanayugulanāsāmukharodham eva kartavyaṃ |\\+}
\tl{śuddhasaṣumṇāśaraṇe sphuṭam amalaḥ śrūyate nādaḥ ||\\!}
\end{versinnote}

\emph{Saubhāgyalakṣmyupaniṣad} 4
\begin{versinnote}
\tl{śravaṇamukhanayananāsānirodhanenaiva kartavyam |\\+}
\tl{śuddhasuṣumnāsaraṇau sphuṭam amalaṃ śrūyate nādaḥ ||\\!}
\end{versinnote}
\end{testimonia}

\begin{philcomm}[hp04_055]
%N3 is missing an akṣara (nirodhaṃ aiva)
%J5 has nirodhaṃ naiva
% Adopt alpha reading mukha°, naiva. HI: it is strongly attested (groups 4c, C6, etc) and more likely explains the change to °saṃrodhanaṃ kāryam. 
% & Note: blocking of ears, etc., has not been mentioned until now. 
\end{philcomm}

\begin{metre}[hp04_055]
Upagīti 
\end{metre}

%%%%%%%%%%
\subsection*{4.55*1}
\begin{translation}[hp04_055_1]
The inner sound is called Śakti; knowledge of the inner sound is Sadāśiva. But when knowledge and the object of knowledge have disappeared, only the beyond-mind [state] remains.
\end{translation}
%

%\begin{sources}[hp04_055_1]
%\end{sources}

\begin{testimonia}[hp04_055_1]
\emph{Upāsanāsārasaṅgraha} f. 107 (attr. to the \emph{Haṭhapradīpikā})
\begin{versinnote}
\tl{nādaga...r iti jñeyaṃ nādo jñānaṃ sadāśivaḥ |\\+}
\tl{jñeyajñāne vilīne [']ṃtaḥ sonmany evāviśiṣyate ||\\!}
\end{versinnote}

\emph{Haṭhatattvakaumudī} 54.50
\begin{versinnote}
\tl{nādaḥ śaktir iti khyāto nādajñānaṃ sadāśivaḥ |\\+}
\tl{nādajñāne vinaṣṭe ca tad unmany eva śiṣyate ||\\!}
\end{versinnote}
\end{testimonia}

%\begin{philcomm}[hp04_055_1]
%V15 reading for the third pāda is better than N19. JB: If we adopt the zeta 3 (V15) for the third pāda, then we have to adopt tad unmany eva śiṣyate in the fourth (otherwise 4th pāda is unmetrical with evāvaśiṣyate). 
%\end{philcomm}

%%%%%%%%%%
\subsection*{4.55*2}
\begin{translation}[hp04_055_2]
As long as there is the inner sound there is mind. At the end of the inner sound the mind beyond mind state [arises]. The void is said to be sonorous and Brahma is silent.
\end{translation}

%\begin{sources}[hp04_055_2]
%\end{sources}

\begin{testimonia}[hp04_055_2]
\emph{Haṭhatattvakaumudī} 54.51
\begin{versinnote}
\tl{nādo yāvan manas tāvan nādānte ca manonmanī |\\+}
\tl{saśabdaṃ kathitaṃ vāte niḥśabdaṃ brahma kathyate ||\\!}
\end{versinnote}
\end{testimonia}

%\begin{philcomm}[hp04_055_2]
%\end{philcomm}

%%%%%%%%%%
\subsection*{4.55*3}
\begin{translation}[hp04_055_3]
When the store of subliminal impressions has been destroyed as a result of continuously concentrating on the inner sound, the mind and breath are sure to dissolve into the untainted [god] (\emph{nirañjane}).
\end{translation}%. 
%

%\begin{sources}[hp04_055_3]
%\end{sources}

\begin{testimonia}[hp04_055_3]
%\emph{Upāsanāsārasaṅgraha} f. 107 Jb: problematic as most of the first line is missing in ms. no. 12170 and we havent been using the IFP transcript (which does have this verse)
%\begin{versinnote}
%\tl{sadā nādānusandhānāt saṃkṣīṇe vāsanā bhavet |\\+}
%\tl{nirañjane vilīneti niścalitaṃ mārutaṃ manaḥ ||\\!}
%\end{versinnote}

\emph{Haṭhatattvakaumudī} 54.52
\begin{versinnote}
\tl{sadā nādānusandhānāt saṃkṣīṇe vāsanākṣaye |\\+}
\tl{nirañjane ca līyate niścitaṃ cittamārutau ||\\!}
\end{versinnote}

\emph{Nādabindūpaniṣat} 49
\begin{versinnote}
\tl{saśabdaś cākṣare kṣīṇe niḥśabdaṃ paramaṃ padam |\\+}
\tl{sadā nādānusandhānāt saṃkṣīṇā vāsanā bhavet ||\\!}
\end{versinnote}
\end{testimonia}

\begin{philcomm}[hp04_055_3]
The term \emph{nirañjanaḥ} likely refers to \emph{devo nirañjanaḥ} (`untainted god') in the next verse (4.52*4), an expression which refers to the supreme deity.\\
The compound \emph{manamārutau} with the \emph{aiśa} form \emph{mana} is attested in both the \texteta\ and \textepsilon\ groups. Cf.~4.11*22b \emph{manamadhyagām} (in a verse taken from the \emph{Candrāvalokana}).
\end{philcomm}

%%%%%%%%%%
\subsection*{4.55*4}
\begin{translation}[hp04_055_4]
Thousands of crores of inner sounds and hundreds of crores of visual focal points all dissolve into the place of the untainted god (\emph{devo nirañjanaḥ}).
%
\end{translation}
%% JB: does bindu 
%\begin{sources}[hp04_055_4]
%\end{sources}

\begin{testimonia}[hp04_055_4]
\emph{Yogacintāmaṇi} f.~27r (attr. to the \emph{Haṭhapradīpikā})
\begin{versinnote}
\tl{nādakoṭisahasrāṇi bindukoṭiśatāni ca |\\+}
\tl{sarve tatra layaṃ yānti yatra devo nirañjanaḥ ||\\!}
\end{versinnote}

\emph{Haṭhatattvakaumudī} 54.53
\begin{versinnote}
\tl{nādakoṭisahasrāṇi bindukoṭiśatāni ca |\\+}
\tl{sarve tatra layaṃ yānti yatra devo nirañjanaḥ ||\\!}
\end{versinnote}

\emph{Nādabindūpaniṣat} 50cd--51ab
\begin{versinnote}
\tl{nādakoṭisahasrāṇi bindukoṭiśatāni ca ||\\+}
\tl{sarve tatra layaṃ yānti brahmapraṇavanādake |\\!}
\end{versinnote}

Cf. \emph{Śabdakalpadruma} (s.v.~\emph{dharmaghaṭa})
\begin{quote}
\dots ante yāti paraṃ sthānaṃ yatra devo nirañjanaḥ |
iti bhaviṣyapurāṇoktā dharmaghaṭavratakathā samāptā ||
\end{quote}
\end{testimonia}

\begin{philcomm}[hp04_055_4]
On the meaning of \emph{devo nirañjana}, see the note to 4.55*3.

%Compare the quotation in the \emph{Śabdakalpadruma}, s.v.~\emph{dharmaghaṭa}:
%\begin{quote}
%“\dots ante yāti paraṃ sthānaṃ yatra devo nirañjanaḥ||"
%iti bhaviṣyapurāṇoktā dharmaghaṭavratakathā samāptā ||
%\end{quote}
%JB. I'm not sure I see the point of this comparison. Delete? The hemistich seems generic and not necessarily related to the dharmaghaṭavrata. 
\end{philcomm}

%%%%%%%%%%
\subsection*{4.55*4 ending}
% iti nādānusaṃdhānam/
\begin{translation}[hp04_055_4p]
\end{translation}

% \begin{philcomm}[hp04_055_4p]
% \end{philcomm}


%%%%%%%%%%
\subsection*{4.55*5}
\begin{translation}[hp04_055_5]
All the methods of Haṭha and Laya [should be practised] until the attainment of the state of Rājayoga. Having attained the state of Rājayoga, [the yogi] becomes untainted.
\end{translation}

%\begin{sources}[hp04_055_5]
%\end{sources}

%\begin{testimonia}[hp04_055_5]
%\end{testimonia}

% \begin{philcomm}[hp04_055_5]
% \end{philcomm}

%%%%%%%%%%
\subsection*{4.56}
\begin{translation}[hp04_056]
Enough of the verbosity of a learned gathering! O friend, listen to this instruction formerly taught by Śiva for awakening Matsyendra.
\end{translation}
%JB shouldnt it be kāṣṭhāgoṣṭhī - as kāṣṭhagoṣṭhī means a wooden assembly [or now I'm thinking... wooden cowpen). I think we need a note. 

%\begin{sources}[hp04_056]
%\end{sources}

\begin{testimonia}[hp04_056]
\emph{Yogacintāmaṇi} f.~26v (attr. to the \emph{Haṭhapradīpikā})
\begin{versinnote}
\tl{kāṣṭhagoṣṭhīprasaṅgena nādam antargataṃ śṛṇu |\\+}
\tl{purā matsyendrabodhāya ādināthoditaṃ vacaḥ ||\\!}
\end{versinnote}
\end{testimonia}

\begin{philcomm}[hp04_056]
This verse may have been composed by Svātmārāma to introduce the next two verses, which are from the \emph{Candrāvalokana}, a dialogue between Matsyendra and Śiva. Verse 1.34, which may also be authorial like this one, has the vocative \emph{sakhe}.
% JB: In support of this, we could add that other yoga texts which are not dialogues, like the Yogatārāvalī (26), Śivayogapradīpikā (2.59, 3.41, etc.) and Yogabhāskara (2), appear to use the vocative sakhe to address the reader. 
\end{philcomm}

%%%%%%%%%%
\subsection*{4.57}
\begin{translation}[hp04_057]
As long as the moving breath does not enter the middle path, as long as bindu is not stable, restrained by the breath, as long as [realisation of] the ultimate truth (\emph{tattvam}), which is as natural as the sky, does not arise, then all that one says is deceitful, and false prattle.
\end{translation}% tr of 3rd pāda. JM thinks it's "doesn't become as natural as...". Current formulation doesn't work for me (if it's sahaja, it has arisen already)
%% JB: Maybe we should understand tattva here as samādhi (Brahmānanda takes it as citta: yāvat tattvaṃ cittaṃ dhyāne dhyeyacintane sahajasadṛśaṃ svābhāvikadhyeyākāravṛttipravāhavan naiva jāyate). Nonetheless, I think the idea is that all one says is deceitful if the ultimate reality/samādhi, which is naturally similar to the sky, hasn't arisen (in the case of the ultimate reality, hasn't revealed itself to that person).
%?? MD: Do we keep °prabaddhaḥ in the second line? Or adopt °prabandhaḥ as in the source text?

\begin{sources}[hp04_057]
\emph{Candrāvalokana} 14
\begin{versinnote}
\tl{yāvan naiva praviśati caran māruto madhyamārge\\+}
\tl{yāvad bindur na bhavati dṛḍhaḥ prāṇavātaprabandhaḥ |\\+}
\tl{yāvad vyomnā sadṛśa sarasaṃ jāyate nonmanatvaṃ {[}em.~vyomnā sadṛśam arasaṃ?]\\+}
\tl{tāvat sarvaṃ yadi ca vadate dambhamithyāpralāpaḥ ||\\+}
\tl{\var{14b prāṇavātaprabandhaḥ~] 4345, prāṇaghātaprabuddhaḥ 75278, prāṇavāyuḥ prabuddhaḥ 7970, prāṇavātaprabuddhaḥ T00788}\\!}
%14d sarvam] 7970, T00788, 75278 : vighnaṃ 4345
\end{versinnote}
\end{sources}

\begin{testimonia}[hp04_057]
\emph{Yogacintāmaṇi} f.~22a
\begin{versinnote}
\tl{haṭhapradīpikāyāṃ—\\+}
\tl{yāvan naiva praviśati caran māruto madhyamārge\\+}
\tl{yāvad bindur na bhavati dṛḍhaḥ prāṇavātaprabaddhaḥ |\\+}
\tl{yāvad vyomnaḥ sahajasadṛśaṃ jñāyate naiva tatvam \\+}
\tl{tāvat sarvaṃ vadati yad idaṃ dambhamithyāpralāpaḥ\\!}
\end{versinnote}

\emph{Upāsanāsārasaṅgraha} f.~110–111
\begin{versinnote}
\tl{haṭhapradīpikāyāṃ—\\+}
\tl{yāvan naiva praviśati caran māruto madhyamārge\\+}
\tl{yāvad bindur na bhavati dṛḍhaḥ prāṇavātaprabaddhe |\\+}
\tl{yāvad vyomnā sahajasadṛśaṃ jāyate nātmatattvaṃ \\+}
\tl{tāvat sarvaṃ vadati yad idaṃ dambhamithyāpralāpaḥ\\!}
\end{versinnote}

\emph{Haṭhatattvakaumudī} 2.2
\begin{versinnote}
\tl{yāvan naiva praviśati caran māruto madhyamārgaṃ\\+}
\tl{yāvat sūkṣmo na bhavati dṛḍhaḥ prāṇavātaprabandhaḥ |\\+}
\tl{yāvad vyomnā sahajasadṛśaṃ jāyate naiva cittaṃ\\+}
\tl{tāvat sarvaṃ vadati tad idaṃ dambhamithyāpralāpaḥ ||\\!}
\end{versinnote}
\end{testimonia}

%\begin{philcomm}[hp04_057]
%J5 has sthiraḥ (instead of dṛḍhaḥ) supporting the visarga (rather than dṛdhaṃ N3)
%tattva has been understood here (and in the verse with bīja) as the mind.
%\end{philcomm}

\begin{metre}[hp04_057]
Mandākrāntā 
\end{metre}

%%%%%%%%%%
\subsection*{4.58}
\begin{translation}[hp04_058]
Having learnt the correct piercing of Suṣumṇā, [the yogi] should make the breath go into the central channel, put it in the place of the moon and block the nostrils.
\end{translation}
%nītvāsāv aindavasthāne
% 
% MD 2024-8: The IFP-ms has dhṛtvā taṃ caindavasthāne, dhṛtvā is also supported by α3. "dhṛtvā taṃ caindave sthāne" may be the original reading.

\begin{sources}[hp04_058]
\emph{Candrāvalokana} 32
\begin{versinnote}
\tl{jñātvā suṣumnāsadbhedaṃ kṛtvā vāyuṃ ca madhyagam |\\+}
\tl{kṛtvāsāv aindave sthāne ghrāṇarandhre nirodhayet ||\\+}
\tl{\var{satbhedaṃ ] 4345, 4340, T00788 : tatbhedaṃ  7970.}\\+}
\tl{\var{kṛtvāsāv aindave ] 7970, 4340, T00788 : kṛtvā*d*baindave 4345}\\!}
\end{versinnote}
\end{sources}

\begin{testimonia}[hp04_058]
\emph{Upāsanāsārasaṅgraha} p.~31 (attr. to the \emph{Candrāvalokana})
\begin{versinnote}
\tl{jñatvā suṣumnāṃ tadbhedaṃ kṛtvā vāyuṃ ca madhyagam |\\+}
\tl{kṛtvāsau baindavasthāne ghraṇarandhre nirodhayet ||\\!}
\end{versinnote}

\emph{Haṭhasaṅketacandrikā} f.~107v–108r (attr. to the \emph{Haṭhapradīpikā})
\begin{versinnote}
\tl{jñātvā suṣumnāsadbhedaṃ kṛtvā vāyuṃ ca madhyagam |\\+}
\tl{sthitvā sadaiva svasthena prāṇarandhraṃ nirodhayet ||\\!}
\end{versinnote}

\emph{Yogakuṇḍalinyupaniṣat} 7cd–8ab
\begin{versinnote}
\tl{jñātvā suṣumnāṃ tadbhedaṃ kṛtvā vāyuṃ ca madhyagam ||\\+}
\tl{sthitvāsau baindavasthāne ghrāṇarandhre nirodhayet |\\!}
\end{versinnote}
\end{testimonia}

%\begin{philcomm}[hp04_058]
%\end{philcomm}

\begin{metre}[hp04_058]
Anuṣṭubh (a: ma-vipulā)
\end{metre}

%%%%%%%%%%
\subsection*{4.59 heading}
\begin{translation}[hp04_059a]
And so, Vasiṣṭha [said]:
\end{translation}

% \begin{philcomm}[hp04_059a]
% \end{philcomm}

%%%%%%%%%%
\subsection*{4.59}
\begin{translation}[hp04_059]
The moon and sun move in Iḍā and Piṅgalā. The moon is said to be of the nature of \emph{tamas} and the sun of \emph{rajas}.
\end{translation}

\begin{sources}[hp04_059]
\emph{Vasiṣṭhasaṃhitā} 2.28–29ab
\begin{versinnote}
\tl{iḍāyāṃ piṅgalāyāṃ ca carataś candrabhāskarau |\\+}
\tl{iḍāyāṃ candramā jñeyaḥ piṅgalāyāṃ raviḥ smṛtaḥ ||\\+}
\tl{candras tāmasa ity uktaḥ sūryo rājasa ucyate |\\!}
\end{versinnote}

Cf.~\emph{Yogayājñavalkya} 4.32cd–33
\begin{versinnote}
\tl{iḍāyāṃ piṅgalāyāṃ ca carataś candrabhāskarau ||\\+}
\tl{iḍāyāṃ candramā jñeyaḥ piṅglāyāṃ raviḥ smṛtaḥ |\\+}
\tl{candras tāmasa ity uktaḥ sūryo rājasa ucyate ||\\!}
\end{versinnote}

Cf.~\emph{Matsyendrasaṃhitā} 4.41cd
\begin{versinnote}
\tl{iḍāyāṃ piṅgalāyāṃ ca parataś candrabhāskarau ||\\!} 
\end{versinnote}

\end{sources}

\begin{testimonia}[hp04_059]
\emph{Haṭharatnāvalī} 4.36cd–37ab
\begin{versinnote}
\tl{iḍāyāṃ piṃgalāyāṃ ca somasūryau pratiṣṭhitau ||\\+}
\tl{tāmaso rājasaś caiva savyadakṣinasaṃsthitau |\\!}
\end{versinnote}

\emph{Yogacintāmaṇi} f.~59v (attr. to Yājñavalkya)
\begin{versinnote}
\tl{iḍāyāṃ piṃgalāyāṃ ca carataś candrabhāskarau |\\+}
\tl{iḍāyāṃ candramā jñeyaḥ piṃgalāyāṃ raviḥ smṛtaḥ ||\\+}
\tl{candras tāmasa ity uktas sūryo rājasa ucyate |\\!}
\end{versinnote}
\end{testimonia}

%\begin{philcomm}[hp04_059]
%\end{philcomm}

%%%%%%%%%%
\subsection*{4.60}
\begin{translation}[hp04_060]
Those two bring about the entirety of time, which consists of night and day. Suṣumṇā consumes time. This which has been taught is secret.
\end{translation}

\begin{sources}[hp04_060]
\emph{Vasiṣṭhasaṃhitā} 2.29cd–30ab
\begin{versinnote}
\tl{tāv eva sakalaṃ dhattaḥ kālaṃ rātrindivātmakam |\\+}
\tl{bhoktrī suṣumṇā kālasya guhyam etad udāhṛtam ||\\!}
\end{versinnote}

Cf. \emph{Yogayājñavalkya} 4.34cd–35ab
\begin{versinnote}
\tl{tāv eva dhattaḥ sakalaṃ kālaṃ rātridivātmakam |\\+}
\tl{bhoktrī suṣumnā kālasya guhyam etad udāhṛtam ||\\!}
\end{versinnote}
\end{sources}

\begin{testimonia}[hp04_060]
\emph{Yogacintāmaṇi} (attr. to Yājñavalkya)
\begin{versinnote}
\tl{tāv eva dhattaḥ sakalaṃ kālaṃ rātriṃ divātmakam |\\+}
\tl{bhoktrī suṣumṇā kālasya guhyam etad udāhṛtam ||\\!}
\end{versinnote}

\emph{Haṭhasaṅketacandrikā} f. 95v
\begin{versinnote}
\tl{tathā coktaṃ haṭhapradīkāyāṃ–\\+}
\tl{sūryācandramasau dhattaḥ kālaṃ rātridinātmakam ||\\+}
\tl{bhoktrī suṣumnā kālasya guhyate tad udīritaṃ ||\\!}
\end{versinnote}
\end{testimonia}

\begin{philcomm}[hp04_060]
The variant readings of \emph{pāda} a which name the sun and moon are likely to have arisen due to the absence of the preceding verse in 
\textepsilon, \texteta, and \textzeta.
\end{philcomm}

\begin{metre}[hp04_060]
Anuṣṭubh (a: bha-vipulā; c: ma-vipulā)
\end{metre}

%%%%%%%%%%
\subsection*{4.61 heading}
\begin{translation}[hp04_061a]
For as the tetrad of verses called the Saubhadra has it:
\end{translation}

\begin{philcomm}[hp04_061a]
We do not know why this tetrad of verses is called Saubhadra.
\end{philcomm}

%%%%%%%%%%
\subsection*{4.61}
\begin{translation}[hp04_061]
There are six cakras, sixteen supports, three focal points and three \emph{guṇa}s. Everything else is [just] the prolixity of texts. Trikūṭa is the supreme place.
\end{translation}

%\begin{sources}[hp04_061]
%\end{sources}

\begin{testimonia}[hp04_061]
Cf.~6-chapter \emph{Vivekamārtaṇḍa} 6.3
\begin{versinnote}
\tl{ṣaṭcakraṃ ṣoḍaśādhāraṃ trilakṣaṃ vyomapañcakam |\\+}
\tl{svadehe ye na jānanti kathaṃ sidhyanti yoginaḥ ||\\!}
\end{versinnote}
\end{testimonia}

\begin{philcomm}[hp04_061]
The three components of the yogic body listed here are found together in other texts, the earliest being \emph{Netratantra} 7.1ab (\emph{ṛtucakraṃ svarādhāraṃ trilakṣyaṃ vyomapañcakam}). However, we are yet to find a source for this list that includes the three \emph{guṇa}s.
\end{philcomm}

%%%%%%%%%%
\subsection*{4.62}
\begin{translation}[hp04_062]
Kuṇḍalinī is taught to have a curved shape like a snake. She is Śakti. [The yogi] who has made her move is undoubtedly liberated.
\end{translation}

%\begin{sources}[hp04_062]
%\end{sources}

\begin{testimonia}[hp04_062]
\emph{Yogacintāmaṇi} f.~79r (attr. to the \emph{Haṭhayoga})
\begin{versinnote}
\tl{kuṇḍalī kuṭilākārā sarpavat parikīrtitā |\\+}
\tl{sā śaktiścālitā yena sa mukto nātra saṃśayaḥ ||\\!}
\end{versinnote}

\emph{Upāsanāsārasaṅgraha} f. 51 (attr. to a \emph{yogaśāstra})
\begin{versinnote}
\tl{kuṇḍalī kuṭilākārā sarpavat parikīrtitā |\\+}
\tl{sā śaktiścālitā yena sa mukto nātra saṃśayaḥ ||\\!}
\end{versinnote}
\end{testimonia}

%\begin{philcomm}[hp04_062]
%\end{philcomm}

%%%%%%%%%%
\subsection*{4.63}
\begin{translation}[hp04_061]
When the \emph{kūṭa} is situated at Trikūṭa [then] the mind is wonderful and uninterrupted. By means of Kuṇḍalinī, [the yogi] is undoubtedly liberated.
\end{translation}

%\begin{sources}[hp04_061]
%\end{sources}

\begin{testimonia}[hp04_061]
\emph{Upāsanāsārasaṅgraha} f. 51 (attr. to a \emph{yogaśāstra})
\begin{versinnote}
\tl{yadā kūṭaṃ trikūṭasthaṃ cittaṃ cittaṃ niraṃtaram |\\+}
\tl{kuṇḍalyās tu prayogeṇa sa mukto nātra saṃśayaḥ ||\\!}
\end{versinnote}
\end{testimonia}

\begin{philcomm}[hp04_061]
We are unsure of the meaning of \emph{kūṭa} here. As suggested to us by Sven Sellmer, it may mean the tip of the tongue which in, for example \emph{Khecarīvidyā} 1.65–67 and 3.16–17, is to be placed at \emph{trikūṭa} as part of the practice of \emph{khecarīmudrā}.
\end{philcomm}

%%%%%%%%%%
\subsection*{4.64}
\begin{translation}[hp04_064]
There are seventy-two thousand openings  of the channels in the cage [that is the body]. Suṣumṇā is the Śāmbhavī Śakti while the other [channels] are pointless.
\end{translation}% 

%\begin{sources}[hp04_064]
%\end{sources}

\begin{testimonia}[hp04_064]
\emph{Haṭhasaṅketacandrikā} f.~108r (attr. to the \emph{Haṭhapradīpikā})
\begin{versinnote}
\tl{sūryā[c]andramasau kṛtvā viditvā karapaṃjare | \\+}
\tl{suṣumṇā śāṃbhavī śaktiḥ śeṣās tv eva nirarthakāḥ ||\\!}
\end{versinnote}

\emph{Upāsanāsārasaṅgraha} f.~111 (attr. to the \emph{Haṭhapradīpikā})
\begin{versinnote}
\tl{dvāsaptatisahasrāṇi nāḍīdvārāṇi paṃjare |\\+}
\tl{suṣumnā śāṃbhavī śaktiḥ śeṣās tv eva nirarthakāḥ ||\\!}
\end{versinnote}

\emph{Yogaśikhopaniṣat} 6.17cd–18ab
\begin{versinnote}
\tl{dvisaptatisahasrāṇi nāḍīdvārāṇi pañjare ||\\+}
\tl{suṣumnā śāmbhavī śaktiḥ śeṣās tv anye nirarthakāḥ |\\!}
\end{versinnote}
\end{testimonia}

\begin{philcomm}[hp04_064]
The compound \emph{nāḍīdvāra} is not found elsewhere (other than as \emph{nāḍīdvāreṇa}) and its meaning here is unclear. Brahmānanda understands \emph{dvārāṇi} to refer to routes by which breath enters the body (\emph{dvārāṇi vāyupraveśamārgāḥ}) and we have translated \emph{nāḍīdvārāṇi} accordingly.
%Consider °nāḍīnāṃ dehapañjare JM: probably a patch? JB: Agree
\end{philcomm}

%%%%%%%%%%
\subsection*{4.65}
\begin{translation}[hp04_065]
The breath, having been carefully accumulated, together with fire awakens Kuṇḍalinī and enters Suṣumṇā without obstruction.

\end{translation}

\begin{sources}[hp04_065]
\emph{Dattātreyayogaśāstra} 108
\begin{versinnote}
\tl{vāyuḥ paricito yatnād agninā saha kuṇḍalīm |\\+}
\tl{bodhayitvā suṣumnāyāṃ praviśed anirodhataḥ ||\\!}
\end{versinnote}
\end{sources}

\begin{testimonia}[hp04_065]
\emph{Śārṅgadharapaddhati} 4399
\begin{versinnote}
\tl{vāyuḥ paricito yasmād agninā saha kuṇḍalīm  |\\+}
\tl{bodhayitvā suṣumnāyāṃ praviśed anirodhataḥ ||\\!}
\end{versinnote}

\emph{Haṭhasaṅketacandrikā} ff.~197v–180r (attr. to the \emph{Haṭhapradīpikā})
\begin{versinnote}
\tl{vāyuḥ paricito yasmād agninā saha kuṃḍalī |\\+}
\tl{bodhayitvā suṣumnāyāṃ praviśed anirodhata iti\\!}
\end{versinnote}
\end{testimonia}

%\begin{philcomm}[hp04_065]
%\end{philcomm}

%%%%%%%%%%
\subsection*{4.66}
\begin{translation}[hp04_066]
When the breath is flowing in Suṣumṇā, the transmental state is attained. Otherwise [i.e.~if the breath is not flowing in Suṣumṇā], the various practices [of yoga] lead to nothing but exertion for yogis.
\end{translation}%

%\begin{sources}[hp04_066]
%\end{sources}

\begin{testimonia}[hp04_066]
\emph{Upāsanāsārasaṅgraha} p.~108 (attr. to the \emph{Haṭhapradīpikā})
\begin{versinnote}
\tl{haṭhapradīpikāyām–\\+}
\tl{suṣumnāvāhinī prāṇe siddhaty eva manonmani |\\+}
\tl{anyathā vividhābhyāso prayāsāyaiva yoginām ||\\!}
\end{versinnote}

\emph{Haṭhasaṅketacandrikā} f.~113v
\begin{versinnote}
\tl{suṣumṇāvāhini prāṇe sidhyaty eva manonmanī  |\\+}
\tl{anye ye vividhābhyāsāḥ prayāsā eva yogināṃ  ||\\!}
\end{versinnote}
\end{testimonia}

%\begin{philcomm}[hp04_066]
%\end{philcomm}

%%%%%%%%%%
\subsection*{4.67}
\begin{translation}[hp04_067]
The mind is bound by the very same thing that binds the breath and the breath is bound by that which binds the mind.
\end{translation}

%\begin{sources}[hp04_067]
%\end{sources}

\begin{testimonia}[hp04_067]

\emph{Haṭhasaṅketacandrikā} f.~67r
\begin{versinnote}
\tl{tathā coktaṃ haṭhapradīpikāyāṃ–\\+}
\tl{pavano badhyate yena manas tenaiva b[a]dhyate \\+}
\tl{mānaś ca badhyate yena pavanas tena badhyate [|]\\!}
\end{versinnote}
\end{testimonia}

\begin{philcomm}[hp04_067]
Brahmānanda understands \emph{yena} here to refer to the yogi. We have taken it to refer to a practice.
\end{philcomm}

%%%%%%%%%%
\subsection*{4.68}
\begin{translation}[hp04_068]
The mind has two impulses: past impressions (\emph{vāsanā}) and the breath. When one of those two disappears, both soon disappear.%
\end{translation}

\begin{sources}[hp04_068]
\emph{Gorakṣaśataka} 9
\begin{versinnote}
\tl{hetudvayañ ca cittasya vāsanā ca samīraṇaḥ |\\+}
\tl{tayor vinaṣṭa ekasmin drutaṃ dvāv api naśyataḥ |\\+}
\tl{\var{9d drutaṃ dvāv api ] em.; dhṛtaṃ dvāv api T, tasmai dvāv api G1, nasmai dvāv api G2, tad dvāv api vi° U}\\!}
\end{versinnote}

Cf. \emph{Mokṣopāya} V.92.48
\begin{versinnote}
\tl{dve bīje rāma cittasya prāṇaspandanavāsane |\\+}
\tl{ekasmiṃś ca tayoẖ kṣīṇe kṣipraṃ dve api naśyataḥ ||\\!}
\end{versinnote}
\end{sources}

\begin{testimonia}[hp04_068]
\emph{Yogakuṇḍalinyupaniṣat} 1
\begin{versinnote}
\tl{hetudvayaṃ hi cittasya vāsanā ca samīraṇaḥ |\\+}
\tl{tayor vinaṣṭa ekasmiṃs tad dvāv api vinaśyataḥ ||\\!}
\end{versinnote}
\end{testimonia}

\begin{philcomm}[hp04_068]
The emendation of \emph{drutaṃ} in the last verse quarter has been made to restore the faulty readings of the \textalpha\ manuscripts (i.e., \emph{dguttaṃ, dhṛtaṃ, dṛtaṃ}), which are similar to an incorrect reading in an important witness of the source text, the \emph{Gorakṣaśataka} (T, \emph{dhṛtaṃ}). The emendation to \emph{drutaṃ} is based on the parallel verse in the \emph{Mokṣopāya} and its related recensions, which have \emph{kṣipraṃ} instead.\lb

It is difficult to say whether the \emph{Gorakṣaśataka}'s reading \emph{cittasya} was changed to \emph{manaso} by the author when this verse was borrowed in order to make the terminology consistent with the previous verse, or whether this change occurred at a later time.
\end{philcomm}

%%%%%%%%%%
\subsection*{4.69}
\begin{translation}[hp04_069]
The breath dissolves where the mind dissolves; the mind dissolves right where the breath dissolves.
\end{translation}

%\begin{sources}[hp04_069]
%\end{sources}

\begin{testimonia}[hp04_069]
\emph{Haṭharatnāvalī} 4.29
\begin{versinnote}
\tl{mano yatra vilīyeta pavanas tatra līyate ||\\!}
\end{versinnote}
\end{testimonia}

\begin{philcomm}[hp04_069]
%HI: tatra = tāvat.
\end{philcomm}

%%%%%%%%%%
\subsection*{4.70}
\begin{translation}[hp04_070]
 Like water mixed with milk, mind and breath are always joined, behaving in the same way.  As long as there is mind the breath is active, and as long as there is breath the mind is active.
\end{translation}


\begin{sources}[hp04_070]
\emph{Amanaska} 2.27
\begin{versinnote}
\tl{dugdhāṃbuvat sammilitau sadaiva\\+}
\tl{tulyakriyau mānasamārutau ca |\\+}
\tl{yāvan manas tatra marutpravṛttir \\+}
\tl{yāvan maruc cāpi manaḥpravṛttiḥ |\\+}
\tl{\var{sadaiva ] NI, S, N : tathaiva Cc : sad eva Na}\\!}
\end{versinnote}
\end{sources}

\begin{testimonia}[hp04_070]
\emph{Yogacintāmaṇi} f.~19r (attr. to the \emph{Rājayoga} (aka. \emph{Amanaska})
\begin{versinnote}
\tl{rājayoge—\\+}
\tl{dugdhāmbuvat saṃmilitau sadaiva\\+}
\tl{tulyakriyau mānasamārutau ca |\\+}
\tl{yāvan manas tatra marutpravṛttir \\+}
\tl{yāvan marut tatra manaḥpravṛttiḥ ||\\!}
\end{versinnote}

\emph{Haṭhatattvakaumudī} 2.5
\begin{versinnote}
\tl{dugdhāmbuvat saṃmilitau sadaiva \\+}
\tl{tulyakriyau mānasamārutau ca | \\+}
\tl{yāvan manas tatra marutpravṛttis \\+}
\tl{tatraikanāśād aparasya nāśaḥ ||\\!}
\end{versinnote}
\end{testimonia}

\begin{philcomm}[hp04_070]
%These verses (4.66–67) seem to elevate the possible role of the mind in achieving \emph{samādhi} (as opposed to the standard Haṭha method of stopping the breath to achieve samādhi). 
Complementing his understanding of the previous verse, Brahmānanda has \emph{yato\dots tatra} in 4.70cd, taking it to mean \emph{yatra\dots tatra} and to be referring to cakras.
\end{philcomm}

\begin{metre}[hp04_070]
Upajāti
\end{metre}

%%%%%%%%%%
\subsection*{4.71}
\begin{translation}[hp04_071]
As a result of one of those two disappearing the other disappears and as a result of one being active the other is active. And when neither has disappeared there is perception through all the sense faculties. When both have disappeared the state of liberation is attained.
\end{translation}


\begin{sources}[hp04_071]
\emph{Amanaska} 2.28
\begin{versinnote}
\tl{tatraikanāśād aparasya nāśa\\+}
\tl{ekapravṛtter aparapravṛttiḥ |\\+}
\tl{adhvastayoś cendriyavargabuddhir \\+}
\tl{vidhvastayor mokṣapadasya siddhiḥ ||\\+}
\tl{\var{vargabuddhir ] NI and S : vargavṛttir Pa Tr Va Nb Ea Eb : vargavṛddhiḥ VbVd: vargavṛddhir N : vargavidhi Pc : sargabuddhir Cc: sargaviddhir Nu}\\!}
\end{versinnote}
\end{sources}

\begin{testimonia}[hp04_071]
\emph{Yogacintāmaṇi} f.~19r (attr. to the \emph{Rājayoga} (aka. \emph{Amanaska})
\begin{versinnote}
\tl{tatraikanāśād aparasya nāśaḥ\\+}
\tl{ekapravṛtter aparapravṛttiḥ |\\+}
\tl{adhvastayoḥ svendriyavargavṛddhir\\+}
\tl{vidhvastayor mokṣapadasya siddhiḥ ||\\!}
\end{versinnote}

\emph{Haṭhatattvakaumudī} 2.6
\begin{versinnote}
\tl{ekapravṛttāv aparapravṛttir \\+}
\tl{ekasya nāśād aparasya nāśaḥ |\\+}
\tl{adhvastayor indriyavargavṛttir \\+}
\tl{vidhvastayoḥ mokṣapadasya siddhiḥ ||\\!}
\end{versinnote}
\end{testimonia}

%\begin{philcomm}[hp04_071]
%Destroy etc. for disappear?
%J5 śuddhir (for buddhir)
%\end{philcomm}

\begin{metre}[hp04_071]
Upajāti
\end{metre}

%%%%%%%%%%
\subsection*{4.72}
\begin{translation}[hp04_072]
When there is no movement in the path of the wind, [the yogi]
obtains the whole earth and the eight lordly powers. True, true is this, O beautiful one.
\end{translation}

\begin{sources}[hp04_072]
\emph{Jñānasāra} 3.5–6
\begin{versinnote}
\tl{vāyuvegena deveśi sakalāṃ bhramate mahīm |\\+}
\tl{aṣṭadhāguṇam aiśvaryaṃ satyaṃ satyaṃ na cānyathā ||\\!}
\end{versinnote}
\end{sources}

\begin{testimonia}[hp04_072]
\emph{Haṭhasaṅketacandrikā} f.~117r
\begin{versinnote}
\tl{tathā coktaṃ haṭhapradīpikāyāṃ–\\+}
\tl{vāyumārge [']py asaṃcāre sakalāṃ bhramate mahīm | \\+}
\tl{tathā'ṣṭāguṇam aiśvaryam ity āha bhagavān śiva iti || \\!}
\end{versinnote}
\end{testimonia}

\begin{philcomm}[hp04_072]
As it is found in its source text, the \emph{Jñānasāra}, this verse says that the yogi flies around the world with the speed of the wind (\emph{vāyuvegena}). We have understood Svātmārāma to have edited the verse to reflect the subject of the previous two verses in which the breath is to be stopped.
%Unidentified source. Jnānasāra removes vocative.
%vāyumārga is an expression for flying (Rāmāyaṇa 7.33.3, etc., etc.)
%labhate needed for aṣṭaguṇaṃ
\end{philcomm}

%%%%%%%%%%
\subsection*{4.73 heading}
\begin{translation}[hp04_073a]
Thus, Viśvarūpācārya [said]:
\end{translation}

% \begin{philcomm}[hp04_073a]
% \end{philcomm}

%%%%%%%%%%
\subsection*{4.73}
\begin{translation}[hp04_073]
When the breath is destroyed and the mind dissolves, all experience is the same (\emph{samarasatva}). That is called \emph{samādhi}.
\end{translation}

\begin{sources}[hp04_073]
\emph{Vivekamārtaṇḍa} 163
\begin{versinnote}
\tl{yadā saṃkṣīyate prāṇo mānasaṃ ca vilīyate |\\+}
\tl{tadā samarasatvaṃ ca samādhiḥ so 'bhidhīyate ||\\!}
\end{versinnote}
\end{sources}

\begin{testimonia}[hp04_073]
\emph{Yuktabhavadeva} 11.30 (attr. to Gorakṣanātha)
\begin{versinnote}
\tl{yadā saṃkṣīyate prāṇo mānasaṃ ca pralīyate |\\+}
\tl{yadā samarasatvaṃ ca samādhiḥ procyate tadā ||\\+}
\tl{\var{yadā ] tadā}\\!}
\end{versinnote}

\emph{Haṭhasaṅketacandrikā} f.~117v
\begin{versinnote}
\tl{tathā ca viśvarūpāyāryāḥ–\\+}
\tl{yadā saṃkṣīyate prāṇo mānasaṃ ca pralīyate | \\+}
\tl{tadā samarasatvaṃ yat samādhiḥ so 'bhidhīyata iti || \\+}
\tl{\var{mānasaṃ ca ] B220, mānaseva 2244}\\!}
\end{versinnote}
\end{testimonia}

\begin{philcomm}[hp04_073]
The six-chapter \emph{Vivekamārtaṇḍa} is attributed to Viśvarūpa, which perhaps explains the attribution of this verse to him.
\end{philcomm}

%%%%%%%%%%
\subsection*{4.74}
\begin{translation}[hp04_074]
When the mind is still the breath is still, from which semen becomes still.  As a result of semen becoming still, my son, the body becomes still.
\end{translation}

%\begin{sources}[hp04_074]
%\end{sources}

\begin{testimonia}[hp04_074]
\emph{Yogacintāmaṇi}~f.19v
\begin{versinnote}
\tl{haṭhapradīpikāyām—\\+}
\tl{manaḥsthairye sthiro vāyus tato binduḥ sthiro bhavet |\\+}
\tl{bindusthairyād athāpannaṃ piṇḍasthairyaṃ prajāyata iti ||\\!}
\end{versinnote}

Cf. \emph{Haṭhatattvakaumudī} 43.19
\begin{versinnote}
\tl{cittasthairye mārutasusthiraḥ syāt
tasmād bindususthiro yogino 'ṃge~|\\+}
\tl{bindusthairye syād dayā satvam ojaḥ
piṇḍasthairyaṃ kāyasampad balaṃ ca || 19 ||\\!}
\end{versinnote}
\end{testimonia}

%\begin{philcomm}[hp04_074]
The vocative \emph{putra} in \emph{pāda} c suggests that this verse is from a source text that we are yet to identify.
%88c āpanna not well attested and N3 has bindusthairyodayāt putra: perhaps this comes from an unknown source text addressed to a son.
%\end{philcomm}

%%%%%%%%%%
\subsection*{4.75}
\begin{translation}[hp04_075]
Only he whose gaze is steady without a visible object, whose breath is steady without effort [and] whose mind is steady without a support is a yogi, is a guru, is to be served.
\end{translation}


\begin{sources}[hp04_075]
\emph{Amanaska} 2.44
\begin{versinnote}
\tl{dṛṣṭiḥ sthirā yasya vinaiva dṛśyād\\+}
\tl{vāyuḥ sthiro yasya vinā prayatnāt |\\+}
\tl{cittaṃ sthiraṃ yasya vināvalambāt\\+}
\tl{sa eva yogī sa guruḥ sa sevyaḥ ||\\!}
%\tl{\var{vinaiva dṛśyād  AllNI AllSI Eb: vinā sudṛśyād Ua: vinā nimeṣād Pb AllN : vinaiva lakṣyād Cc: vinaiva dṛṣṭyād Jb: vinakadṛśyad Ma: vinānaṅgame Nb}\\!}
\end{versinnote}
\end{sources}

\begin{testimonia}[hp04_075]
\emph{Haṭharatnāvalī} 4.25
\begin{versinnote}
\tl{dṛṣṭiḥ sthirā yasya vinaiva lakṣyāt\\+}
\tl{vāyuḥ sthiro yasya vinā prayatnāt |\\+}
\tl{cittaṃ sthiraṃ yasya vināvalambanāt\\+}
\tl{sa eva yogī sa guruḥ sa sevyaḥ ||\\!}
\end{versinnote}

\emph{Yogacintāmaṇi} f. 48r (attr. to the \emph{Rājayoga})
\begin{versinnote}
\tl{dṛṣṭiḥ sthirā yasya vinaiva dṛśyaṃ\\+}
\tl{vāyuḥ sthiro yasya vinā prayatnaṃ |\\+}
\tl{cittaṃ sthiraṃ yasya vināvalaṃbaṃ\\+}
\tl{sa eva yogī sa guruḥ saṃsevyaḥ ||\\!}
\end{versinnote}

Cf. \emph{Kulārṇavatantra} 13.70
\begin{versinnote}
\tl{dṛśyaṃ vinā sthirā dṛṣṭir manaś cālambanaṃ  vinā |\\+}
\tl{vināyāsaṃ sthiro vāyur yasya syāt sa guruḥ priye ||\\!}
\end{versinnote}

\emph{Haṭhasaṅketacandrikā}~f. 3v (attr. to the \emph{Haṭhapradīpikā})
\begin{versinnote}
\tl{atha gurulakṣaṇam |\\+}
\tl{dṛṣṭi[ḥ] sthirā yasya vinaiva dṛśyād\\+}
\tl{vāyuḥ sthiro yasya vinā prayatnāt |\\+}
\tl{cittaṃ sthiraṃ yasya vināvalambaṃ \\+}
\tl{sa rājayogī sa guruḥ sa sevyaḥ ||\\+}
\tl{\var{rāja°] B220, rāva° 2244}\\!}
\end{versinnote}
\end{testimonia}

%\begin{philcomm}[hp04_075]
%\end{philcomm}

\begin{metre}[hp04_075]
Upajāti
\end{metre}

%%%%%%%%%%
\subsection*{4.76}
\begin{translation}[hp04_076]
[The yogi] whose breath does not move in, out, left, right, up or down is liberated. In this there is no doubt.
\end{translation}

\begin{sources}[hp04_076]
\emph{Gorakṣaśataka} 8
\begin{versinnote}
\tl{praveśe nirgame vāme dakṣiṇe cordhvam apy adhaḥ |\\+}
\tl{na yasya vāyur vrajati sa mukto nātra saṃśayaḥ ||\\!}
\end{versinnote}
\end{sources}

%\begin{testimonia}[hp04_076]
%\end{testimonia}

%\begin{philcomm}[hp04_076]
%\end{philcomm}

\begin{metre}[hp04_076]
Anuṣṭubh (c: bha-vipulā)
\end{metre}

%%%%%%%%%%
\subsection*{4.77}
\begin{translation}[hp04_077]
All the methods of Haṭha and Laya are for mastering Rājayoga. The man who has ascended to Rājayoga cheats death.
\end{translation}

%\begin{sources}[hp04_077]
%\end{sources}

\begin{testimonia}[hp04_077]
\emph{Yogacintāmaṇi} f. 8r (attrib.~to the \emph{Haṭhapradīpikā})
\begin{versinnote}
\tl{sarve haṭhalayābhyāsād rājayogasya siddhaye |\\+}
\tl{rājayogaṃ samārūḍhaḥ puruṣaḥ kālavañcaka iti ||\\!}
\end{versinnote}

\emph{Haṭhatattvakaumudī} 55.34
\begin{versinnote}
\tl{haṭhapradīpikāyām–\\+}
\tl{sarve haṭhalayopāyā rājayogasya siddhaye |\\+}
\tl{rājayogasamārūḍhaḥ puruṣaḥ kālavañcakaḥ ||\\!}
\end{versinnote}

\emph{Haṃsavilāsa} p.~49
\begin{versinnote}
\tl{sarve haṭhalayopāyā rājayogāya kevalam |\\+}
\tl{rājayogaṃ samārūḍhaḥ puruṣaḥ kālavañcakaḥ ||\\!}
\end{versinnote}
\end{testimonia}

%\begin{philcomm}[hp04_077]
%\end{philcomm}

%%%%%%%%%%
\subsection*{4.77*1}
\begin{translation}[hp04_077_1]
Iḍā is the divine Gaṅgā, Piṅgalā is the river Yamunā. Between those two is Suṣumṇā, who is to be recognised as Sarasvatī.
\end{translation}

%\begin{sources}[hp04_077_1]
%\end{sources}

%\begin{testimonia}[hp04_077_1]
%\end{testimonia}

%\begin{philcomm}[hp04_077_1]
%\end{philcomm}

%%%%%%%%%%
\subsection*{4.77*2}
\begin{translation}[hp04_077_2]
The place of the Triveṇī confluence is called the king of sacred sites. One should bathe there [and] be freed from all sins.
\end{translation}

%\begin{sources}[hp04_077_2]
%\end{sources}

%\begin{testimonia}[hp04_077_2]
%\end{testimonia}

%\begin{philcomm}[hp04_077_2]
%\end{philcomm}

%%%%%%%%%%
\subsection*{4.78}
\begin{translation}[hp04_078]
O great yogis, experience the nectar of Haṭha, the essence extracted from the ocean of all yoga scriptures after it has thus been churned, if you wish not to grow old and die.
\end{translation} %JB: if you do not wish for old age and death.

%\begin{sources}[hp04_078]
%\end{sources}

\begin{testimonia}[hp04_078]
\emph{Haṭhasaṅketacandrikā} f.~145v
\begin{versinnote}
\tl{iti tu sakalayogaśāstrasindhoḥ \\+}
\tl{parimathitād avakṛṣya sārabhūtaṃ [|]\\+}
\tl{anubhavata haṭhāmṛtaṃ yamīndrā\\+}
\tl{yadi bhavatām ajarāmaratvavāṃchā [||]\\!}
\end{versinnote}
\end{testimonia}

\begin{philcomm}[hp04_078]
\end{philcomm}

\begin{metre}[hp04_078]
Puṣpitāgrā
\end{metre}

%%%%%%%%%%
\subsection*{4.78*1}
\begin{translation}[hp04_75_1] The wise people in the world wash away sin at the sacred site of knowledge (\emph{vidyātīrthe}), the virtuous at the sacred site of truth (\emph{satyatīrthe}), the impure-minded at the sacred site of the Gaṅgā (\emph{gaṅgātīrthe}), yogis at the sacred site of knowledge (\emph{jñānatīrthe}), kings at the sacred site of the streams (\emph{dhārātīrthe}), the rich at the sacred site of charity (\emph{dānatīrthe}) [and] women of good family at the sacred site of modesty. (\emph{lajjātīrthe})
\end{translation}
%

%\begin{sources}[hp04_75_1]
%\end{sources}

%\begin{testimonia}[hp04_75_1]
%\emph{Yogacintāmaṇi} 
%\end{testimonia}

%\begin{philcomm}[hp04_75_1]
%\end{philcomm}

\begin{metre}[hp04_078_1]
Mandākrāntā 
\end{metre}

%%%%%%%%%%
\subsection*{colophon}
\begin{translation}[hp04_col]
Thus ends the fourth chapter in the \emph{Haṭhapradīpikā} composed by the glorious lord among yogis Svātmārāma.    
\end{translation}



\end{ekdosis}
\end{document}
