\documentclass[11pt,twoside]{article}
\usepackage[papersize={17cm,24cm},
	centering, textwidth=12.5cm, textheight=17.9cm,
	asymmetric]{geometry} % so that the orgvnr always appear on the right side
%\linespread{1.1}
\sloppy

\usepackage{xcolor, xparse, xspace, pifont, datetime2}
\usepackage{enumitem}

\newcommand{\HP}{\textit{Haṭha\-pra\-dī\-pikā}\xspace}

\usepackage{fancyhdr}
	\renewcommand{\headrulewidth}{0pt}
	\fancyhead[EL]{\small\texteng{\thepage}}
	\fancyhead[ER]{\small\texteng{Critical Edition \& Translation}}
	\fancyhead[OL]{\small\texteng{HP\hindsection}}
	\fancyhead[OR]{\small\texteng{\thepage}}
	\fancyhead[C]{}
	\fancyfoot[C]{}
%	\pagestyle{fancy}

\fancypagestyle{firstpage}{%
	\fancyhead[OL]{\small\texteng{(\Today)}}
	\fancyhead[OR]{\small\texteng{\thepage}}
	\fancyhead[C]{}
	}

\fancypagestyle{HPed}{%
	\renewcommand{\headrulewidth}{0pt}
	\fancyhead[EL]{\small\texteng{\thepage}}
	\fancyhead[ER]{\small\texteng{[\rightmark\textendash}}
	\fancyhead[OL]{\small\texteng{\textendash\leftmark]}}
	\fancyhead[OR]{\small\texteng{\thepage}}
	\fancyhead[C]{\small\texteng{Critical Edition}}
	\fancyfoot[C]{}
	}
\pagestyle{HPed}

\usepackage{scrextend}
	\deffootnote[]{0em}{1em}{}
%	\deffootnote[2.1em]{0em}{1.5em}{\color{blue}\texteng{\textbf{Verse \thefootnotemark}}}
\usepackage{fnpos}
	\makeFNbottom
%\footnoterulefalse
\setlength{\footnotesep}{1em}
%\dimen\footins=??
%\renewcommand{\footnotesize}{\small}  % Wirkt auf App und Fn beides.

\usepackage[english]{babel}
\usepackage{babel-iast}
\babelfont[iast]{rm}[Renderer=Harfbuzz, Scale=1.1]{AdishilaSan}
\babeltags{dev = iast}
\babeltags{eng = english}
\usepackage{libertine}

\usepackage[teiexport=tidy,poetry=verse]{ekdosis}
\usepackage{sanskrit-poetry}
\usepackage{textgreek}

%%% Gr1,4b,6
\DeclareWitness{N3}{\texteng{\textalpha\textsubscript{1}}}{NGMPP B 62-20}[]
        \DeclareHand{N3ac}{N3}{\texteng{\textalpha\rlap{\textsubscript{1}}\textsuperscript{ac}}}[]
        \DeclareHand{N3pc}{N3}{\texteng{\textalpha\rlap{\textsubscript{1}}\textsuperscript{pc}}}[]
\DeclareWitness{J5}{\texteng{\textalpha\textsubscript{2}}}{Jodhpur 02235}[]
\DeclareWitness{G4}{\texteng{\textalpha\textsubscript{3}}}{GOML 18885}[]% Telugu script
\DeclareWitness{N24}{\texteng{\textalpha\textsubscript{4}}}{NGMPP G 190-16}[]
\DeclareWitness{Gr1r}{\texteng{\textAlpha *}}{Gr1 reconstructed}[]

\DeclareWitness{P11}{\texteng{\textbeta\textsubscript{1}}}{}[]
\DeclareWitness{C6}{\texteng{\textbeta\textsubscript{2}}}{Lalchand M-2089}[]

\DeclareWitness{V3}{\texteng{\textbeta\textsubscript{\textomega}}}{Sampurnananda Library Sarasvati Bhavan 29899}[]

%%% Gr2

\DeclareWitness{N23}{\texteng{\textgamma\textsubscript{1}}}{NGMPP G 25-2}[]
        \DeclareHand{N23ac}{N23}{\texteng{\textgamma\rlap{\textsubscript{1}}\textsuperscript{ac}}}[]
        \DeclareHand{N23pc}{N23}{\texteng{\textgamma\rlap{\textsubscript{1}}\textsuperscript{pc}}}[]
\DeclareWitness{J7}{\texteng{\textgamma\textsubscript{2}}}{Jodhpur 02241}[]
%\DeclareWitness{V6}{\texteng{V\textsubscript{6}}}{Sampurnananda Library Sarasvati Bhavan 29991}[]
\DeclareWitness{K1}{\texteng{K\textsubscript{1}}}{Raghunātha Temple Library 4383}[settlement=Jammu]
        \DeclareWitness{K1ac}{\texteng{K\rlap{\textsubscript{1}}\textsuperscript{ac}\space}}{}[]
        \DeclareWitness{K1pc}{\texteng{K\rlap{\textsubscript{1}}\textsuperscript{pc}\space}}{}[]


%%% Gr3

\DeclareWitness{V19}{\texteng{\textdelta\textsubscript{1}}}{Sampurnananda Library Sarasvati Bhavan 30069}[]
\DeclareWitness{K3}{\texteng{\textdelta\textsubscript{2}}}{Privat collection}
\DeclareWitness{C7}{\texteng{\textdelta\textsubscript{3}}}{Lalchand M-6494}[]
%\DeclareWitness{C1}{\texteng{C\textsubscript{1}}}{Lalchand M-2080}[]%L1 And C1 very close (and come from same region)
%\DeclareWitness{P23}{\texteng{P\textsubscript{23}}}{}[]
%\DeclareWitness{L1}{\texteng{L\textsubscript{1}}}{SOAS RE 43454}[settlement=Jammu]

\DeclareWitness{J6}{\texteng{\textdelta\textsubscript{\textomega}}}{Jodhpur 02237}[]
        \DeclareHand{J6ac}{J6}{\texteng{\textdelta\rlap{\textomega}\textsuperscript{ac}}}[]
        \DeclareHand{J6pc}{J6}{\texteng{\textdelta\rlap{\textomega}\textsuperscript{pc}}}[]

%%% Gr4c

\DeclareWitness{P15}{\texteng{\textepsilon\textsubscript{1}}}{}[]
\DeclareWitness{N19}{\texteng{\textepsilon\textsubscript{2}}}{NGMPP E-1528-1 / E-1527-7(4)}[]
\DeclareWitness{V15}{\texteng{\textepsilon\textsubscript{3}}}{Sampurnananda Library Sarasvati Bhavan 30051}[]
        \DeclareHand{V15ac}{V15}{\texteng{\textepsilon\rlap{\textsubscript{3}}\textsuperscript{ac}}}[]
        \DeclareHand{V15pc}{V15}{\texteng{\textepsilon\rlap{\textsubscript{3}}\textsuperscript{pc}}}[]
\DeclareWitness{J11}{\texteng{\textepsilon\textsubscript{4}}}{Jodhpur 23532}[]
        \DeclareHand{J11ac}{J11}{\texteng{\textepsilon\rlap{\textsubscript{4}}\textsuperscript{i.t.}}}[]
        \DeclareHand{J11pc}{J11}{\texteng{\textepsilon\rlap{\textsubscript{4}}\textsuperscript{mg.}}}[alternative reading written by the first hand in margin or interlinearly (J11)]
%\DeclareWitness{J14}{\texteng{\textepsilon\textsubscript{5}}}{Jodhpur 02239}[]

%\DeclareWitness{L2}{\texteng{L\textsubscript{2}}}{Wellcome Collection O.36]}
\DeclareWitness{M1}{\texteng{M\textsubscript{1}}}{P-5682/4}[]

\DeclareWitness{N26}{\texteng{\textepsilon\textsubscript{\textomega}}}{NGMPP}[]
%\DeclareWitness{V17}{\texteng{\textepsilon\textsubscript{\textomega 3}}}{Sampurnananda Library Sarasvati Bhavan 30053}[]

\DeclareWitness{V1}{\texteng{\texteta\textsubscript{1}}}{Sampurnananda Library Sarasvati Bhavan 30109}[]
        \DeclareHand{V1ac}{V1}{\texteng{\texteta\rlap{\textsubscript{1}}\textsuperscript{ac}}}[]
        \DeclareHand{V1pc}{V1}{\texteng{\texteta\rlap{\textsubscript{1}}\textsuperscript{pc}}}[]

%%% Gr4d

\DeclareWitness{J10}{\texteng{\texteta\textsubscript{2}}}{MSPP Jodhpur 2230}[]
        \DeclareHand{J10ac}{J10}{\texteng{\texteta\rlap{\textsubscript{2}}\textsuperscript{ac}}}[]
        \DeclareHand{J10pc}{J10}{\texteng{\texteta\rlap{\textsubscript{2}}\textsuperscript{pc}}}[]

\DeclareWitness{N9}{\texteng{\texteta\textsubscript{\textomega}}}{NGMPP A62-33}[]
%\DeclareWitness{J15}{\texteng{\textepsilon\textsubscript{\textomega 4}}}{Jodhpur 9732A}[]

%%%

\DeclareWitness{Jyo}{\texteng{\textchi}}{Brahmānanda's version}[]
%\DeclareWitness{Tue}{\texteng{Tü}}{Ma I 339}[]

\DeclareWitness{ceteri}{\texteng{cett.}}{ceteri}[]

%%% Group Sigla

\DeclareWitness{Gr1}{\texteng{\textAlpha}}{N3,J5,G4}

\DeclareWitness{Gr2}{\texteng{\textGamma}}{N23,J7}
%\DeclareWitness{Gr2}{\texteng{%
%	\textbeta\textsubscript{1}%
%	\textbeta\textsubscript{2}%
%	}}{N23,J7}
\DeclareWitness{Gr3a}{\texteng{\textDelta}}{V19,K3,C7}
\DeclareWitness{Gr4b}{\texteng{%
	\textbeta\textsubscript{1}%
	\textbeta\textsubscript{2}%
	}}{C6,P11}
\DeclareWitness{GrB}{\texteng{%
	\textbeta\textsubscript{1}%
	\textbeta\textsubscript{2}%
	\textbeta\textsubscript{\textomega}%
	}}{C6,P11,V3}
\DeclareWitness{Gr4c}{\texteng{\textEpsilon}}{P15,N19,V15}

% \DeclareWitness{Gr4d}{\texteng{%
	% \texteta\textsubscript{1}%
	% \texteta\textsubscript{2}%
	% }}{V1,J10}
\DeclareWitness{Gr6}{\texteng{\textOmega}}{V3,J6,N9,N26}




%%%%%%%%%%%%%%%%%%%% THE  MSS         %%%%%%%%%%%%%%%%%%%%%%%%%%%

%%% Versions
\DeclareWitness{Vu}{\selectlanguage{english}Vulg}{Vulgate, i.e. Brahmānanda's version}[]           
\DeclareWitness{X}{\selectlanguage{english}X}{TenChapter Version, Jodhpur 02228 and 02225 (ed. Lonavla)}[]
\DeclareWitness{Six}{\selectlanguage{english}Ṣ}{SixChapterVersion, ``6ChapterHPms'', fragment of enlarged text, Jodhpur}[]
% Mss. in Geographical Groups
%%%% Varanasi mss (Sampūrṇānanda mss). V1 is Important
\DeclareWitness{V1}{\selectlanguage{english}V\textsubscript{1}}{Sampurnananda Library Sarasvati Bhavan 30109}[]
        \DeclareHand{V1ac}{V1}{\selectlanguage{english}V\rlap{\textsubscript{1}}\textsuperscript{ac}}[] % added by MD
        \DeclareHand{V1pc}{V1}{\selectlanguage{english}V\rlap{\textsubscript{1}}\textsuperscript{pc}}[] % added by MD
\DeclareWitness{V2}{\selectlanguage{english}V\textsubscript{2}}{Sampurnananda Library Sarasvati Bhavan 29869}[]
\DeclareWitness{V3}{\selectlanguage{english}V\textsubscript{3}}{Sampurnananda Library Sarasvati Bhavan 29899}[]
\DeclareWitness{V4}{\selectlanguage{english}V\textsubscript{4}}{Sampurnananda Library Sarasvati Bhavan 29937}[]
\DeclareWitness{V5}{\selectlanguage{english}V\textsubscript{5}}{Sampurnananda Library Sarasvati Bhavan 29938}[]
\DeclareWitness{V6}{\selectlanguage{english}V\textsubscript{6}}{Sampurnananda Library Sarasvati Bhavan 29991}[]
\DeclareWitness{V8}{\selectlanguage{english}V\textsubscript{8}}{Sampurnananda Library Sarasvati Bhavan 30014}[]
\DeclareWitness{V11}{\selectlanguage{english}V\textsubscript{11}}{Sampurnananda Library Sarasvati Bhavan 30029}[]
\DeclareWitness{V12}{\selectlanguage{english}V\textsubscript{12}}{Sampurnananda Library Sarasvati Bhavan 30030}[]
\DeclareWitness{V13}{\selectlanguage{english}V\textsubscript{13}}{Sampurnananda Library Sarasvati Bhavan 30031}[]
\DeclareWitness{V14}{\selectlanguage{english}V\textsubscript{14}}{Sampurnananda Library Sarasvati Bhavan 30050}[]
\DeclareWitness{V15}{\selectlanguage{english}V\textsubscript{15}}{Sampurnananda Library Sarasvati Bhavan 30051}[]
\DeclareWitness{V15pc}{\selectlanguage{english}V\rlap{\textsubscript{15}}\textsuperscript{pc}\space}{}[]
\DeclareWitness{V16}{\selectlanguage{english}V\textsubscript{16}}{Sampurnananda Library Sarasvati Bhavan 30052}[]
\DeclareWitness{V17}{\selectlanguage{english}V\textsubscript{17}}{Sampurnananda Library Sarasvati Bhavan 30053}[] % added by MD
\DeclareWitness{V16pc}{\selectlanguage{english}V\rlap{\textsubscript{16}}\textsuperscript{pc}\space}{}[]
\DeclareWitness{V18}{\selectlanguage{english}V\textsubscript{18}}{Sampurnananda Library Sarasvati Bhavan 30064}[]
\DeclareWitness{V19}{\selectlanguage{english}V\textsubscript{19}}{Sampurnananda Library Sarasvati Bhavan 30069}[]
\DeclareWitness{V21}{\selectlanguage{english}V\textsubscript{21}}{Sampurnananda Library Sarasvati Bhavan 30104}[]
\DeclareWitness{V22}{\selectlanguage{english}V\textsubscript{22}}{Sampurnananda Library Sarasvati Bhavan 30110}[]
\DeclareWitness{V25}{\selectlanguage{english}V\textsubscript{25}}{Sampurnananda Library Sarasvati Bhavan 30122}[]
\DeclareWitness{V26}{\selectlanguage{english}V\textsubscript{26}}{Sampurnananda Library Sarasvati Bhavan 30123}[]
\DeclareWitness{V28}{\selectlanguage{english}V\textsubscript{28}}{Sampurnananda Library Sarasvati Bhavan 30136}[]
\DeclareWitness{W2}{\selectlanguage{english}W\textsubscript{2}}{Wai ??}[]
\DeclareWitness{W4}{\selectlanguage{english}W\textsubscript{4}}{Wai 399-6171}[]

%%%%%%%%%%%%%%%%%%%%%%%%%%%%%%%%%
%%% Jammu & Kaschmir
\DeclareWitness{K1}{\selectlanguage{english}K\textsubscript{1}}{Raghunātha Temple Library 4383}[settlement=Jammu]
        \DeclareWitness{K1ac}{\selectlanguage{english}K\rlap{\textsubscript{1}}\textsuperscript{ac}\space}{}[]
        \DeclareWitness{K1pc}{\selectlanguage{english}K\rlap{\textsubscript{1}}\textsuperscript{pc}\space}{}[]
\DeclareWitness{K3}{\selectlanguage{english}K\textsubscript{3}}{Privat collection}
\DeclareWitness{L1}{\selectlanguage{english}L\textsubscript{1}}{SOAS RE 43454}[settlement=Jammu]
% More details? Catalogue number? L1 And C1 very close (and come from same region)
%%%%%%%%%%%%%%%%%%%%%%%%%%%%%%%%
% Jodhpur
% J10 is important
\DeclareWitness{J10}{\selectlanguage{english}J\textsubscript{10}}{MSPP Jodhpur 2230}[]
        \DeclareHand{J10ac}{J10}{\selectlanguage{english}J\rlap{\textsubscript{10}}\textsuperscript{ac}}[] % modified by MD
        \DeclareHand{J10pc}{J10}{\selectlanguage{english}J\rlap{\textsubscript{10}}\textsuperscript{pc}}[] % modified by MD
\DeclareWitness{J1}{\selectlanguage{english}J\textsubscript{1}}{Jodhpur 02231}[]
\DeclareWitness{J2}{\selectlanguage{english}J\textsubscript{2}}{Jodhpur 02232}[]   
\DeclareWitness{J3}{\selectlanguage{english}J\textsubscript{3}}{Jodhpur 02233}[]
\DeclareWitness{J4}{\selectlanguage{english}J\textsubscript{4}}{Jodhpur 02234}[]
        \DeclareWitness{J4ac}{\selectlanguage{english}J\rlap{\textsubscript{4}}\textsuperscript{ac}\space}{MSPP Jodhpur 02234}[]
        \DeclareWitness{J4pc}{\selectlanguage{english}J\rlap{\textsubscript{4}}\textsuperscript{pc}\space}{MSPP Jodhpur 02234}[]
\DeclareWitness{J5}{\selectlanguage{english}J\textsubscript{5}}{Jodhpur 02235}[]  % 4 chapters, 34 jpgs,   long colophon, missing lines in the beginning.
\DeclareWitness{J6}{\selectlanguage{english}J\textsubscript{6}}{Jodhpur 02237}[]  % 4 chapters, 41 jpgs
%\DeclareWitness{J6ac}{\selectlanguage{english}J\rlap{\textsubscript{6}}\textsubscript{ac}}{Jodhpur 02237}[]  % 4 chapters, 49 jpgs,   1st folio: idaṃ gulābarāyasya
% tulasīrāmaśarmmaṇaḥ putrasya pustakaṃ ...        End: iti śrīsahajānandasantānacintāmaṇisvātmārāmaviracitāyāṃ ..
% saṃvat 1802   (more consistent text)
%\DeclareWitness{J6pc}{\selectlanguage{english}J\rlap{\textsubscript{6}}\textsubscript{pc}}{Jodhpur 02237}[] 
\DeclareWitness{J7}{\selectlanguage{english}J\textsubscript{7}}{Jodhpur 02241}[]  % 4 chapters, 41 jpgs
\DeclareWitness{J8}{\selectlanguage{english}J\textsubscript{8}}{Jodhpur 23709}[]  % 4 chapters,  87 jpgs.   saṃvat 1724
\DeclareHand{J8ac}{J8}{\selectlanguage{english}J\rlap{\textsubscript{8}}\textsuperscript{ac}}[]  % changed by MD
\DeclareHand{J8pc}{J8}{\selectlanguage{english}J\rlap{\textsubscript{8}}\textsuperscript{pc}}[]  % changed by MD
\DeclareWitness{J9}{\selectlanguage{english}J\textsubscript{9}}{Jodhpur 02224}[]  %  fragment, 20 jpgs.
\DeclareWitness{J11}{\selectlanguage{english}J\textsubscript{11}}{Jodhpur 23532}[]
        \DeclareHand{J11ac}{J11}{\selectlanguage{english}J\rlap{\textsubscript{11}}\textsuperscript{ac}}[] % added by MD
        \DeclareHand{J11pc}{J11}{\selectlanguage{english}J\rlap{\textsubscript{11}}\textsuperscript{pc}}[] % added by MD
\DeclareWitness{J12}{\selectlanguage{english}J\textsubscript{12}}{Jodhpur 18552}[] 
\DeclareWitness{J13}{\selectlanguage{english}J\textsubscript{13}}{Jodhpur 02229}[]  %  5 chapters, 93 jpgs.
\DeclareWitness{J14}{\selectlanguage{english}J\textsubscript{14}}{Jodhpur 02239}[]  %  4 chapters
\DeclareWitness{J15}{\selectlanguage{english}J\textsubscript{15}}{Jodhpur 9732A}[]
\DeclareWitness{J16}{\selectlanguage{english}J\textsubscript{16}}{Jodhpur 9732B}[]
\DeclareWitness{J17}{\selectlanguage{english}J\textsubscript{17}}{Jodhpur 3013}[]
% Haṭhapradīpikā with (non-Sanskrit) Bhāṣya RORI Jodhpur ACC.NO.18552
%  Haṭhapradīpikā with (non-Sanskrit) commentary, RORI Alwar 952, 4 chapters,  colophon of the comm:
% iti śrīlāhorīmiśravrajabhūṣanaviracitāyāṃ bhāvārthadīpikāyāṃ caturthodhyāya ..    
%  Haṭhapradīpikā (5 chapter) MSPP Jodhpur ACC.NO.02229/

%%%%%%%%%%        Bodleian, Oxford
\DeclareWitness{B1}{\selectlanguage{english}B\textsubscript{1}}{Bodleian Library No. d.457(8)}[settlement=Oxford]
\DeclareWitness{B2}{\selectlanguage{english}B\textsubscript{2}}{Bodleian Library No. d.458(1)}[settlement=Oxford]
\DeclareWitness{B3}{\selectlanguage{english}B\textsubscript{3}}{Bodleian Library No. d.458(9)}[settlement=Oxford]

%%%%%%%%%%%   Chandigarh
\DeclareWitness{C1}{\selectlanguage{english}C\textsubscript{1}}{Lalchand M-2080}[]%L1 And C1 very close (and come from same region)
\DeclareWitness{C2}{\selectlanguage{english}C\textsubscript{2}}{Lalchand M-6065}[]
\DeclareWitness{C3}{\selectlanguage{english}C\textsubscript{3}}{Lalchand M-1293}[]
\DeclareWitness{C4}{\selectlanguage{english}C\textsubscript{4}}{Lalchand M-2081}[]
\DeclareWitness{C4ac}{\selectlanguage{english}C\rlap{\textsubscript{4}}\textsuperscript{ac}\space}{}[]
\DeclareWitness{C4pc}{\selectlanguage{english}C\rlap{\textsubscript{4}}\textsuperscript{pc}\space}{}[]
\DeclareWitness{C5}{\selectlanguage{english}C\textsubscript{5}}{Lalchand M-2082}[]%doesn't have chapter 1
\DeclareWitness{C6}{\selectlanguage{english}C\textsubscript{6}}{Lalchand M-2089}[]
\DeclareWitness{C7}{\selectlanguage{english}C\textsubscript{7}}{Lalchand M-6494}[]
\DeclareWitness{C8}{\selectlanguage{english}C\textsubscript{8}}{Lalchand M-2091}[]
        \DeclareHand{C8ac}{C8}{\selectlanguage{english}C\rlap{\textsubscript{8}}\textsuperscript{ac}}[]
        \DeclareHand{C8pc}{C8}{\selectlanguage{english}C\rlap{\textsubscript{8}}\textsuperscript{pc}}[]
\DeclareWitness{C9}{\selectlanguage{english}C\textsubscript{9}}{Lalchand M-4530}[]


% %%%%%%%%%%        Nepalese
\DeclareWitness{N1}{\selectlanguage{english}N\textsubscript{1}}{NGMPP A1400-2}[]
\DeclareWitness{N2}{\selectlanguage{english}N\textsubscript{2}}{NGMPP B 39-19}[]
\DeclareWitness{N3}{\selectlanguage{english}N\textsubscript{3}}{NGMPP B 62-20}[]
\DeclareWitness{N5}{\selectlanguage{english}N\textsubscript{5}}{NGMPP A60-15 + A61-1}[]
\DeclareWitness{N4}{\selectlanguage{english}N\textsubscript{4}}{NGMPP A61-2}[]
\DeclareWitness{N6}{\selectlanguage{english}N\textsubscript{6}}{NGMPP A61-6}[]
\DeclareWitness{N9}{\selectlanguage{english}N\textsubscript{9}}{NGMPP A62-33}[]
\DeclareWitness{N10}{\selectlanguage{english}N\textsubscript{10}}{NGMPP A62-37}[]
\DeclareWitness{N11}{\selectlanguage{english}N\textsubscript{11}}{NGMPP A63-15}[]
\DeclareWitness{N12}{\selectlanguage{english}N\textsubscript{12}}{NGMPP A939-19}[]
\DeclareWitness{N13}{\selectlanguage{english}N\textsubscript{13}}{NGMPP A1378-18}[]
\DeclareWitness{N16}{\selectlanguage{english}N\textsubscript{16}}{NGMPP B39-20}[]
\DeclareWitness{N17}{\selectlanguage{english}N\textsubscript{17}}{NGMPP B 111-10}[]
\DeclareWitness{N18}{\selectlanguage{english}N\textsubscript{18}}{NGMPP E 929-3}[]
\DeclareWitness{N19}{\selectlanguage{english}N\textsubscript{19}}{NGMPP E-1528-1 / E-1527-7(4)}[]
\DeclareWitness{N20}{\selectlanguage{english}N\textsubscript{20}}{NGMPP E 2037-13 }[]
\DeclareWitness{N21}{\selectlanguage{english}N\textsubscript{21}}{NGMPP E 2097-31}[]
\DeclareWitness{N22}{\selectlanguage{english}N\textsubscript{22}}{NGMPP G 4-4}[]
\DeclareWitness{N23}{\selectlanguage{english}N\textsubscript{23}}{NGMPP G 25-2}[]
        \DeclareHand{N23ac}{N23}{\selectlanguage{english}N\rlap{\textsubscript{23}}\textsuperscript{ac}}[] % added by MD
        \DeclareHand{N23pc}{N23}{\selectlanguage{english}N\rlap{\textsubscript{23}}\textsuperscript{pc}}[] % added by MD
\DeclareWitness{N24}{\selectlanguage{english}N\textsubscript{24}}{NGMPP G 190-16}[]
\DeclareWitness{N24ac}{\selectlanguage{english}N\rlap{\textsubscript{24}}\textsuperscript{ac}\space}{}[]
\DeclareWitness{N24pc}{\selectlanguage{english}N\rlap{\textsubscript{24}}\textsuperscript{pc}\space}{}[]
\DeclareWitness{N26}{\selectlanguage{english}N\textsubscript{26}}{NGMPP T 24-3}[]

% %%%%%%%%%%        Pune

\DeclareWitness{P1}{\selectlanguage{english}P\textsubscript{1}}{Ānandāśrama S16-3-21}[]
\DeclareWitness{P2}{\selectlanguage{english}P\textsubscript{2}}{Ānandāśrama S16-2-20}[]
\DeclareWitness{P3}{\selectlanguage{english}P\textsubscript{3}}{BISM (79) 314}[]
\DeclareWitness{P4}{\selectlanguage{english}P\textsubscript{4}}{BISM (91) 191}[]
\DeclareWitness{P5}{\selectlanguage{english}P\textsubscript{5}}{BISM (29) 5790}[]
\DeclareWitness{P6}{\selectlanguage{english}P\textsubscript{6}}{BORI 263/1879-80}[]
\DeclareWitness{P7}{\selectlanguage{english}P\textsubscript{7}}{BORI 665/1883-84}[]
\DeclareWitness{P8}{\selectlanguage{english}P\textsubscript{8}}{BORI 316/1895-98}[]
\DeclareWitness{P9}{\selectlanguage{english}P\textsubscript{9}}{BORI 733-1891-95}[]
\DeclareWitness{P10}{\selectlanguage{english}P\textsubscript{10}}{BORI 222-1884-86}[]
\DeclareWitness{P11}{\selectlanguage{english}P\textsubscript{11}}{BORI 221-1882–83}[]
\DeclareWitness{P12}{\selectlanguage{english}P\textsubscript{12}}{Ānandāśrama S16-3-24}[]
\DeclareWitness{P13}{\selectlanguage{english}P\textsubscript{13}}{Ānandāśrama S16-2-22}[]
\DeclareWitness{P14}{\selectlanguage{english}P\textsubscript{14}}{Ānandāśrama S16-3-23}[]
\DeclareWitness{P15}{\selectlanguage{english}P\textsubscript{15}}{BISM (64) 919}[]
\DeclareWitness{P16}{\selectlanguage{english}P\textsubscript{16}}{BISM (64) 1115}[]
\DeclareWitness{P17}{\selectlanguage{english}P\textsubscript{17}}{BISM 620/1886-92}[]
\DeclareWitness{P18}{\selectlanguage{english}P\textsubscript{18}}{BORI 615/1887-91}[]
\DeclareWitness{P19}{\selectlanguage{english}P\textsubscript{19}}{BISM 46-39}[]
\DeclareWitness{P20}{\selectlanguage{english}P\textsubscript{20}}{BISM 39-273}[]
\DeclareWitness{P21}{\selectlanguage{english}P\textsubscript{21}}{BISM 37-743}[]
\DeclareWitness{P22}{\selectlanguage{english}P\textsubscript{22}}{BISM 37-729}[]
\DeclareWitness{P23}{\selectlanguage{english}P\textsubscript{23}}{BISM 33-60}[]
\DeclareWitness{P24}{\selectlanguage{english}P\textsubscript{24}}{BISM 29-5790}[]% =P5!
\DeclareWitness{P25}{\selectlanguage{english}P\textsubscript{25}}{BISM 29-3657}[]
\DeclareWitness{P26}{\selectlanguage{english}P\textsubscript{26}}{BISM 25-281}[]
\DeclareWitness{P27}{\selectlanguage{english}P\textsubscript{27}}{BISM 7-489}[]
\DeclareWitness{P28}{\selectlanguage{english}P\textsubscript{28}}{BORI 399-1895-1902}[]

%%%%%   Mysore
\DeclareWitness{M1}{\selectlanguage{english}M\textsubscript{1}}{P-5682/4}[]
%%%%%   Tübingen
\DeclareWitness{Tue}{\selectlanguage{english}Tü}{Ma I 339}[]
%%%%%%%%%%
\DeclareWitness{YC}{\selectlanguage{english}YC}{Yogacintāmaṇi}[]
\DeclareWitness{ceteri}{\selectlanguage{english}cett.}{ceteri}[]

%%%%%%%%%% Mss with Commentary
\DeclareWitness{A1}{\selectlanguage{english}A\textsubscript{1}}{Alwar 952}[]

\DeclareWitness{Jyo}{\selectlanguage{english}J\textsubscript{yo}}{Brahmānanda's version}[]

%%%%%%%%%%%%%%%%%%%%%%%%%%%%%%%%%%%%%%%%%%%
%List of all Sigla:
%A1,B1,B2,B3,C1,C2,C3,C4,C6,C7,C8,C9,J1,J2,J3,J4,J10,J13,J14,J15,J17,L1,M1,N3,N5,N6,N9,N10,N11,N12,N13,N16,N17,N19,N20,N21,N22,N23,N24,Tü,V1,V2,V3,V4,V5,V6,V8,V11,V19,V22,V26,Vu
%%%%%%%%%%%%%%%%%%%%%%%%%%%%%%%%%%%%%%%%%%%

\DeclareWitness{G4}{\selectlanguage{english}G\textsubscript{4}}{GOML D18885 (Bundle SD5051)}[]
\DeclareWitness{G5}{\selectlanguage{english}G\textsubscript{5}}{GOML R3841/ SR2190}[]
\DeclareWitness{G7}{\selectlanguage{english}G\textsubscript{7}}{GOML D4394}[]

\DeclareWitness{Ko}{\selectlanguage{english}K\textsubscript{o}}{Koba, Gujarat 55626}[]

% addition 2023-12-11 MD:
\TeXtoTEIPat{\begin {metre}[#1]}{<note type="metre" target="##1">}
\TeXtoTEIPat{\end {metre}}{</note>}
\TeXtoTEIPat{\texttheta}{θ}

% change 2023-12-05 mm
\TeXtoTEI{teimute}{} 

% changes/additions 2023-11-27 MM:
\TeXtoTEIPat{\medialink {#1}{#2}}{<ref target="resources/#2">#1</ref>}

% changes/additions 2023-10-25 MM:
% new Sigla
\TeXtoTEIPat{\textAlpha}{Α}
\TeXtoTEIPat{\textalpha}{α}
\TeXtoTEIPat{\textBeta}{Β}
\TeXtoTEIPat{\textbeta}{β}
\TeXtoTEIPat{\textGamma}{Γ}
\TeXtoTEIPat{\textgamma}{γ}
\TeXtoTEIPat{\textDelta}{Δ}
\TeXtoTEIPat{\textdelta}{δ}
\TeXtoTEIPat{\textEpsilon}{Ε}
\TeXtoTEIPat{\textepsilon}{ε}
\TeXtoTEIPat{\textEta}{Η}
\TeXtoTEIPat{\texteta}{η}
\TeXtoTEIPat{\textChi}{Χ}
\TeXtoTEIPat{\textchi}{χ}
\TeXtoTEIPat{\textOmega}{Ω}
\TeXtoTEIPat{\textomega}{ω}

%new environments
\TeXtoTEIPat{\begin {postmula}[#1]}{<note type="postmula" target="##1">}
  \TeXtoTEIPat{\end {postmula}}{</note>}
\TeXtoTEIPat{\begin {altava}[#1]}{<div type="altrec"><note type="avataranika" target="##1">} %%% changed 2023-12-05 mm
  \TeXtoTEIPat{\end {altava}}{</note></div>} %%% changed 2023-12-05 mm
\TeXtoTEIPat{\sgwit {#1}}{<note type="inlineref"><ref>#1</ref></note>}

% changes/additions 2023-10-12 MM:
\TeXtoTEIPat{\\.}{}

% changes/additions 2023-08-15 MD:
\TeXtoTEIPat{\lineom {#1}{#2}}{<note type="omission">#1 omitted in <ref>#2</ref></note>}
\TeXtoTEI{graus}{hi}[rend="grey"]
\TeXtoTEIPat{\startgray}{} %%% changed 2023-12-05 mm
\TeXtoTEIPat{\endgray}{} %%% changed 2023-12-05 mm



% additions/changes 2023-06-05 mm:
%\TeXtoTEIPat{\lineom {#1}}{<note type="omission">Line omitted in <ref>#1</ref></note>}
\TeXtoTEIPat{\NotIn {#1}}{<note type="omission">Stanza omitted in <ref>#1</ref></note>}

% additions 2023-04-16 MD:
\TeXtoTEIPat{\,}{ }

% additions 2023-04-13 mm:
\TeXtoTEIPat{\begin {versinnote}}{<lg>}
  \TeXtoTEIPat{\end {versinnote}}{</lg>}

% additions 2023-04-05 MD:
\TeXtoTEIPat{\begin {testimonia}[#1]}{<note type="testimonia" target="##1">}
  \TeXtoTEIPat{\end {testimonia}}{</note>}
\TeXtoTEI{devnote}{s}[xml:lang="sa-deva"]

% app in philcomm und testimonia %%% added MM 2023-12-02
\TeXtoTEI{var}{note}[type="appinnote"]


\TeXtoTEI{anm}{note}[type="memo"] %% change 2023-04-16 MD
\TeXtoTEI{Anm}{note}[type="memo"] %% change 2023-12-05 MM
\TeXtoTEIPat{\startverse}{} %%% marked for change 2023-04-13 mm
\TeXtoTEIPat{\endverse}{} %%% marked for change 2023-04-13 mm
\TeXtoTEIPat{\newpage}{}
\TeXtoTEIPat{\marma}{}
\TeXtoTEIPat{\marmas}{}
\TeXtoTEIPat{\vin}{} % added by MD 2023-11-14

%%% modify environments and commands
%%% TEI mapping
% additions/changes 2022-06-07 mm:
\TeXtoTEI{grau}{hi}[rend="grey"]
\TeXtoTEIPat{ \& }{ &amp; }

% additions/changes 2022-06-01 mm:
\TeXtoTEI{skp}{seg}[type="deva-ignore"]
\TeXtoTEI{skm}{seg}[type="ltn-ignore"]

\TeXtoTEIPat{\rlap {#1}}{#1}

% additions/changes 2022-04-06 mm:
%\TeXtoTEI{sgwit}{ref}
\TeXtoTEI{textdev}{s}[xml:lang="sa-deva"]
\TeXtoTEIPat{\begin {col}[#1]}{<div type="colophon" xml:id="#1"><p>}
  \TeXtoTEIPat{\end {col}}{</p></div>}
\TeXtoTEIPat{\begin {ava}[#1]}{<note type="avataranika" target="##1">}
  \TeXtoTEIPat{\end {ava}}{</note>}
												   
\TeXtoTEIPat{\outdent}{}
\TeXtoTEIPat{\startaltrecension}{} %%% changed 2023-12-05 mm
\TeXtoTEIPat{\endaltrecension}{} %%% changed 2023-12-05 mm
\TeXtoTEIPat{\startaltnormal}{} % added by MD 2023-11-14 %%% changed 2023-12-05 mm
\TeXtoTEIPat{\endaltnormal}{} % added by MD 2023-11-14 %%% changed 2023-12-05 mm
\TeXtoTEIPat{\begin {alttlg}[#1]}{<div type="altrec"><lg xml:id="#1">}
  \TeXtoTEIPat{\end {alttlg}}{</lg></div>}



% additions/changes 2022-03-12 mm:
\TeXtoTEIPat{\begin {tlg}[#1]}{<lg xml:id="#1">}
  \TeXtoTEIPat{\end {tlg}}{</lg>}

\TeXtoTEIPat{\begin {translation}[#1]}{<note type="translation" target="##1">}
  \TeXtoTEIPat{\end {translation}}{</note>}
\TeXtoTEIPat{\begin {philcomm}[#1]}{<note type="philcomm" target="##1">}
  \TeXtoTEIPat{\end {philcomm}}{</note>}
\TeXtoTEIPat{\begin {sources}[#1]}{<note type="sources" target="##1">}
  \TeXtoTEIPat{\end {sources}}{</note>}


\TeXtoTEIPat{\begin {marma}[#1]}{<note type="marma" target="##1">}
  \TeXtoTEIPat{\end {marma}}{</note>}

\TeXtoTEIPat{\begin {jyotsna}[#1]}{<note type="jyotsna" target="##1">}
  \TeXtoTEIPat{\end {jyotsna}}{</note>}

\EnvtoTEI{description}{list}
\EnvtoTEI{itemize}{list}
\TeXtoTEIPat{\item [#1]}{<label>#1</label>\item}

\TeXtoTEI{tl}{l}
\TeXtoTEI{myfn}{note}[type="myfn"]
\TeXtoTEIPat{\getsiglum {#1}}{<ref target="##1"/>}

\TeXtoTEI{SetLineation}{}
\TeXtoTEI{noindent}{}
\TeXtoTEI{subsection*}{}

\TeXtoTEI{rlap}{}

% end additions/changes
% \TeXtoTEIPat{\skp {#1}}{#1}
% \TeXtoTEIPat{\skm {#1}}{}

\TeXtoTEIPat{\begin {prose}}{<p>}
  \TeXtoTEIPat{\end {prose}}{</p>}

\TeXtoTEIPat{\begin {tlate}}{<p>}
  \TeXtoTEIPat{\end {tlate}}{</p>}

\TeXtoTEI{emph}{hi}
\TeXtoTEI{bigskip}{}
% \TeXtoTEI{/}{|}
\TeXtoTEI{tl}{l}
\TeXtoTEIPat{english}{}
%\TeXtoTEIPat{-}{ } %% change 2023-04-16 MD
%\TeXtoTEIPat{°}{} %% change 2023-04-16 MD
\TeXtoTEIPat{\textcolor {#1}{#2}}{<hi rend="#1">#2</hi>}

% \TeXtoTEIPat{\eyeskip}{}
% \TeXtoTEIPat{\aberratio}{}
% \TeXtoTEIPat{\ad}{}
\TeXtoTEIPat{\add}{<hi rend="italic">add.</hi>} %% change 2023-04-16 MD
% \TeXtoTEIPat{\ann}{}
\TeXtoTEIPat{\ante}{<hi rend="italic">ante</hi> } %% change 2023-04-16 MD
\TeXtoTEIPat{\post}{<hi rend="italic">post</hi> } %% change 2023-04-16 MD
% \TeXtoTEIPat{\codd}{}
% \TeXtoTEIPat{\conj }{}
% \TeXtoTEIPat{\contin}{}
% \TeXtoTEIPat{\corr}{}
% \TeXtoTEIPat{\del}{}
% \TeXtoTEIPat{\dub}{}
% \TeXtoTEIPat{\emend }{}
% \TeXtoTEIPat{\expl}{}
% \TeXtoTEIPat{\ȩxplicat}{}
% \TeXtoTEIPat{\fol}{}
% \TeXtoTEIPat{\gloss}{}
% \TeXtoTEIPat{\ins}{}
% \TeXtoTEIPat{\im}{}
% \TeXtoTEIPat{\inmargine}{}
% \TeXtoTEIPat{\intextu}{}
% \TeXtoTEIPat{\indist}{}
% \TeXtoTEIPat{\iteravit}{}
% \TeXtoTEIPat{\lectio}{}
% \TeXtoTEIPat{\leginequit}{}
% \TeXtoTEIPat{\legn}{}
% \TeXtoTEIPat{\illeg}{<hi rend="italic">illeg.</hi>}
\TeXtoTEIPat{\illeg}{<gap reason="illeg."/>} %%% change 2023-04-11 mm
% \TeXtoTEIPat{\om}{<hi rend="italic">om.</hi>}
\TeXtoTEIPat{\om}{<gap reason="om."/>} %%% change 2023-04-11 mm
% \TeXtoTEIPat{\primman}{}
% \TeXtoTEIPat{\prob}{}
% \TeXtoTEIPat{\rep}{}
% \TeXtoTEIPat{\sequentia}{}
% \TeXtoTEIPat{\supralineam}{}
% \TeXtoTEIPat{\interlineam}{}
\TeXtoTEIPat{\vl}{<hi rend="italic">v.l.</hi>}
% \TeXtoTEIPat{\vide}{}
% \TeXtoTEIPat{\videtur}{}
% \TeXtoTEIPat{\crux}{}
% \TeXtoTEIPat{\cruxxx}{}
\TeXtoTEIPat{\unm}{<hi rend="italic">unm.</hi>}


% List of Scholars
\DeclareScholar{nos}{nos}[
forename=HPP,
surname=Team]


% Nullify \selectlanguage in TEI as it has been used in
% \DeclareWitness but should be ignored in TEI.
\TeXtoTEI{selectlanguage}{}



\setlength\parindent{1em}
\SetLineation{lineation=none}
\poemlines{0}

\SetHooks{
	lemmastyle=\bfseries,
	refnumstyle=\selectlanguage{english}\color{blue}\bfseries, 
	appfontsize=\footnotesize
	}
\DeclareApparatus{default}[
	lang=english,
	sep = {] },
	delim=\hskip 0.75em,
	%	rule=none,
	]
\DeclareApparatus{anmkg}[
	notelang=english,
	sep = { },
	delim=\texteng{\ \textbullet\ \ },
%	rule=\relax
	rule=\rule{0.15\columnwidth}{0.4pt}
	]

\newcommand{\mydelim}{\xspace\textcolor{violet}{\textbullet}\ \ }
\newcommand{\mylem}[1]{\texteng{\textcolor{violet}{#1}}}
\setlength{\vrightskip}{-15pt}
\setlength{\vgap}{-3em} % default 1.5em
\verselinenumfont{\footnotesize\selectlanguage{english}\normalfont}

\newlength{\myoutdent}\setlength{\myoutdent}{2em}

\DeclareShorthand{emend}{\texteng{\emph{em.}}}{ego}
%\DeclareShorthand{conj}{\texteng{\emph{conj.}}}{ego}

%Define two commands: \skp ("sanskrit plus"), to be ignored by TeX in
%the edition text, but processed in the TEI output. Conversely, \skm
%("sanskrit minus") is to be processed in the edition text, but
%ignored if found in the apparatus criticus and in the TEI output:

\newif\ifinapparatus
\NewDocumentCommand{\skp}{m}{}
\NewDocumentCommand{\skm}{m}{\unless\ifinapparatus#1\fi}

\SetTEIxmlExport{autopar=false}

\newcommand{\versenr}{\ \themyvnum//}

\NewDocumentEnvironment{tlg}{O{}}{
	\def\hpvnum{\texteng{\thepoemline}}
	\markboth{\hpvnum}{\hpvnum}
	\setcounter{myvnum}{\value{poemline}}
	\begin{ekdverse}
	\Large}{\normalsize
	\end{ekdverse}
	%\smallskip
%  \stepcounter{myvnum}
}

\NewDocumentEnvironment{alttlg}{O{}}{
	\setvnum{\hindsection.\arabic{saved@poemline}*\arabic{poemline}}
	\def\hpvnum{\texteng{\hindsection.\arabic{saved@poemline}*\arabic{poemline}}}
	\markboth{\hpvnum}{\hpvnum}
	\setcounter{altvnum}{\value{poemline}}
	\begin{ekdverse}[type=altrecension]
	\color{gray}
	\Large}{\normalsize
	\end{ekdverse}
	%\smallskip
}

\NewDocumentCommand{\tl}{m}{#1}

\NewDocumentEnvironment{ava}{O{}}{
	\setvnum{prescript:}
	\begin{ekdverse}
	\hspace{-\myoutdent}
	\Large}{\normalsize
	\end{ekdverse}
	\smallskip
}

\NewDocumentEnvironment{altava}{O{}}{
	\setvnum{prescript:}
	\begin{ekdverse}[type=altrecension]
	\color{gray}
	\hspace{-\myoutdent}
	\Large}{\normalsize
	\end{ekdverse}
	\smallskip
}   

\NewDocumentEnvironment{postmula}{O{}}{
	\setvnum{postscript:}
	\smallskip
	\begin{ekdverse}
	\hspace{-\myoutdent}
	\Large}{\normalsize
	\end{ekdverse}
}

\NewDocumentEnvironment{altpostmula}{O{}}{
	\setvnum{postscript:}
	\smallskip
	\begin{ekdverse}[type=altrecension]
	\color{gray}
	\hspace{-\myoutdent}
	\Large}{\normalsize
	\end{ekdverse}
}

\NewDocumentEnvironment{col}{O{}}{
	\setvnum{colophon:}
	\medskip
	\begin{ekdverse}%
	\hspace{-2.5em}%
	\Large%
	}{\normalsize
	\end{ekdverse}
	%\smallskip
      }
      
\NewDocumentCommand{\tcommref}{m}{}
\NewDocumentCommand{\ttransref}{m}{}
\NewDocumentCommand{\tnocomm}{}{}


\def\startaltrecension{
	\setcounter{altvnum}{0}
	\stopvline
	\addtocounter{saved@poemline}{-1}
	\renewcommand{\versenr}{\ \themyvnum *{\small \arabic{poemline}}//}
%	\small
	}
	
\def\endaltrecension{
	\addtocounter{saved@poemline}{1}
	\startvline
	\setvnum{\hindsection.\arabic{poemline}}
	\renewcommand{\versenr}{\ \themyvnum//}
%	\normalsize
	}

\def\startaltnormal{
	\startaltrecension
	\setvnum{\hindsection.\arabic{saved@poemline}*\arabic{poemline}}
	}

\def\endaltnormal{\endaltrecension}

%%%%%%

\newcommand{\teionly}[1]{}
\newcommand{\teimute}[1]{#1}
\newcommand{\manuref}[1]{#1}
\newcounter{myvnum}\setcounter{myvnum}{0}
\newcounter{altvnum}\setcounter{altvnum}{0}
\newcounter{mynotenr}\setcounter{mynotenr}{0}
%\newcommand{\myfn}[1]{\footnote{\texteng{#1}}}

\newcommand{\myfn}[1]{%
	\setcounter{ekd@padanum}{0} % um Pāda-Nummer zu unterdrücken
	\stepcounter{mynotenr}%
	\linelabel{note\themynotenr}%
	\note[type=anmkg, labelb={note\themynotenr}]{#1}
	}

% \newcommand{\myfnx}[1]{%
	% \setcounter{ekd@padanum}{0} % um Pāda-Nummer zu unterdrücken
	% \stepcounter{mynotenr}%
	% \linelabel{note\themynotenr}%
	% \note[type=anmkg, labelb={note\themynotenr},num]{#1}
	% }

\renewcommand{\thefootnote}{\texteng{\arabic{footnote}}}
\newcommand{\devnote}[1]{{\small\textdev{#1}}}
\newcommand{\devtext}[1]{{\normalsize\textdev{#1}}}
%\newcommand{\vsn}[1]{{\footnotesize\texteng{#1}}}
\newcommand{\graus}[1]{\small\textcolor{gray}{#1}\normalsize} % partial altrecension
\newcommand{\grau}[1]{\textcolor{gray}{#1}} % partial altrecension
\newcommand{\Anm}[1]{\begin{ekdverse}
	\texteng{\footnotesize (#1)}
	\end{ekdverse}
	}

%\newcommand{\sgwit}[1]{{\footnotesize (\getsiglum{#1})}}
%\newcommand{\NotIn}[1]{\texteng{\footnotesize (om. \getsiglum{#1})}}
%\newcommand{\lineom}[2]{\texteng{\footnotesize (#1 om. \getsiglum{#2})}}
%\newcommand{\anm}[1]{\texteng{\footnotesize [#1]}}
\newcommand{\sgwit}[1]{}% Nur für Online version; Change TEI too!!
%\newcommand{\lineom}[2]{\myfn{#1 om. \getsiglum{#2}}}
\newcommand{\anm}[1]{\myfn{#1}}
%\newcommand{\unavbl}[1]{\marginpar{\scriptsize\texteng{−\,\getsiglum{#1}}}}
%\newcommand{\unavbl}[1]{\myfn{Folio lost in \getsiglum{#1}}}
\newcommand{\textapp}[1]{\texteng{\textsf{#1}}}
\newcommand{\unavbl}{\textapp{folio lost}}
\newcommand{\incl}{\textapp{included in}}
\newcommand{\only}{\textapp{only included in}}
\newcommand{\also}{\textapp{also included in}}
\newcommand{\excl}{\textapp{included in all except}}
\newcommand{\NotIn}{\om}
\newcommand{\expnr}[1]{\textcolor{magenta}{#1}}% X\kern 1pt

\def\om{\texteng{\emph{om.\@}}}% \kern-0.3ex
\def\illeg{\texteng{\emph{illeg.\@}}} 
\def\lost{\texteng{\emph{lost}}} 
\def\lacuna{\texteng{\emph{lac.\@}}}
\def\unm{\texteng{\emph{unm.\ }}}
\def\ante{\texteng{\normalfont\textapp{ante\ }}}
\def\add{\texteng{\normalfont\emph{add.\@}}}
\def\post{\texteng{\normalfont\textapp{post\ }}}
\def\antecorr{\texteng{\textsubscript{ac}}}
\def\postcorr{\texteng{\textsubscript{pc}}}
\def\marmas{\ }%\texteng{\textsuperscript{\#}}\ }
\def\marma{}%\texteng{\textsuperscript{\#}}}
\def\crux{\texteng{\textsuperscript{\textdagger}}}

%%%%%%% Commentary part

\usepackage{catchfilebetweentags}

\NewDocumentEnvironment{translation}{O{}}{%
	\selectlanguage{english}}{%
	\selectlanguage{iast}}
	
\NewDocumentEnvironment{sources}{O{}}{%
	\selectlanguage{english}%
	\begin{description}[leftmargin=1em, 
		topsep=0pt, parsep=0pt, partopsep=0pt,
		listparindent=0pt, labelwidth=1em, labelsep=0pt]
	\item[\ding{118}\ Sources]
	\item %
	}{\end{description}\selectlanguage{iast}}

\NewDocumentEnvironment{testimonia}{O{}}{%
	\selectlanguage{english}%
	\begin{description}[leftmargin=1em,
		topsep=0pt, parsep=0pt, partopsep=0pt,
		listparindent=0pt, labelwidth=1em, labelsep=0pt]
	\item[\ding{118}\ Testimonia]
	\item %
	}{\end{description}\selectlanguage{iast}}
	
\NewDocumentEnvironment{philcomm}{O{}}{%
	\selectlanguage{english}%
	\begin{description}[leftmargin=1em, 
		topsep=0pt, parsep=0pt, partopsep=0pt,
		listparindent=1.5em,
		labelwidth=1em, labelsep=0pt]
	\item[\ding{118}\ Commentary]\ %
	\newline
	}{\end{description}\selectlanguage{iast}}

\newenvironment{variants}{%
	\begin{description}[%
		leftmargin=4em,
		topsep=3.5pt,
		parsep=0pt,
	%	partopsep=0pt,
		listparindent=-1.5em,
		labelwidth=2.5em,
		labelsep=0pt]
	\item\scriptsize}{%
	\end{description}
	}
 
\newenvironment{versinnote}{%
	\setlength{\vindent}{0pt}
%	\poemlines{0}
	\vspace{4pt plus 2pt minus 2pt}
	\begin{ekdverse}
	\linespread{0.9}\normalsize\selectlanguage{iast}}{%
	\linespread{1}\selectlanguage{english}\end{ekdverse}
	\vspace{4pt plus 2pt minus 2pt}
%	\poemlines{1}
	\addtocounter{poemline}{-1}
	}

  \newenvironment{versinnoterm}{%
	\setlength{\vindent}{0pt}
	\vspace{1pt}
	\begin{ekdverse}
		\itshape}{%
		\rmfamily
	\end{ekdverse}
	\vspace{1pt}
	\addtocounter{poemline}{-1}
	}

\newenvironment{appinnote}{% still in use: 1.16, 1.30, 2.50, 2.77, 3.25, 3.34, 3.39*1, 4.9
	\setlength{\vindent}{0pt}
	\begin{ekdverse}
	\scriptsize\selectlanguage{english}}{%
	\selectlanguage{iast}\end{ekdverse}
	\vspace{3pt minus 1pt}
	\addtocounter{poemline}{-1}
}

%\newcommand{\vnumfix}{\addtocounter{poemline}{1}}
%\TeXtoTEIPat{\vnumfix}{}
\newcommand{\labelincomm}{\smallskip\newline\noindent}
%\TeXtoTEIPat{\labelincomm}{<lb/>} % >> PreambleComm.tex
%\newcommand{\tre}{\ }
%\TeXtoTEIPat{\tre}{}
\newcommand{\skx}[2]{#1} % sandhi between pādas
%\TeXtoTEIPat{\skx {#1}{#2}}{#2} % >> PreambleComm.tex
%\TeXtoTEIPat{\commcitecore}{}
%\TeXtoTEIPat{\commcite}{}
%\TeXtoTEIPat{\commciterange}{}
%\TeXtoTEIPat{\altcommcite}{}
%\TeXtoTEIPat{\avacite}{}
%\TeXtoTEIPat{\colcite}{}
%\TeXtoTEIPat{\trcite}{}

%\TeXtoTEIPat{\labelvnum}{}
%\TeXtoTEIPat{\commvnum}{}

\newcommand{\myvspace}{\vspace{-3pt plus 3pt minus 3pt}}
\newcommand{\commlabel}{\hfill\texteng{\raisebox{0pt}{\textbf{[\hindsection.\labelvnum]}}}\hfill}

\newcommand{\comminfn}{%
	\footnotetext{%
	\commlabel
	\ExecuteMetaData[\commfilename]{sc\commvnum}%
	\ExecuteMetaData[\commfilename]{ts\commvnum}%
	\ExecuteMetaData[\commfilename]{cm\commvnum}%
	}}
	
\newcommand{\commcitecore}{%
	\myvspace
	\begin{quote}%
	\ExecuteMetaData[\commfilename]{tr\commvnum}
	\texteng{(\labelvnum)}\comminfn
	\end{quote}}

\def\commfilename{HP\hindsection_comm.tex}
\newcommand{\commcite}{%
	\def\commvnum{\themyvnum}%
	\def\labelvnum{\themyvnum}%
	\commcitecore}

\newcommand{\commciterange}[2]{%
	\def\commvnum{#1}%
	\def\labelvnum{#2}%
	\commcitecore}
	
\newcommand{\altcommcite}{%
	\def\commvnum{\themyvnum-\thealtvnum}%
	\def\labelvnum{\themyvnum*\thealtvnum}%
	\myvspace
	\begin{quote}%
	\textcolor{gray}{%
	\ExecuteMetaData[\commfilename]{tr\commvnum}
	\texteng{(\labelvnum)}}\comminfn
	\end{quote}}

\newcommand{\avacite}[1]{%
	\bigskip%
	\setlength{\parindent}{1em} %\hspace*{0.5em} in HP4X
	\ExecuteMetaData[\commfilename]{tr#1}
	\vspace{-3pt}
	}

\newcommand{\trcite}[1]{
	\myvspace
	\begin{quote}
	\ExecuteMetaData[\commfilename]{tr#1}
	\texteng{(#1)}
	\end{quote}
	}

\newcommand{\alttrcite}{
	\def\commvnum{\themyvnum-\thealtvnum}%
	\def\labelvnum{\themyvnum*\thealtvnum}%
	\myvspace
	\begin{quote}
	\textcolor{gray}{\ExecuteMetaData[\commfilename]{tr\commvnum}
	\texteng{(\labelvnum)}}
	\end{quote}
	}

\newcommand{\colcite}{
	\medskip
	\noindent
	\ExecuteMetaData[\commfilename]{trcol}
	}


\newcommand{\closer}{\vspace{-1ex}}
\newcommand{\lb}{\par}
\newcommand{\mylb}{\smallskip\lb}
\newcommand{\sep}{\par}
% \TeXtoTEIPat{\sep}{<lb/>}% oder besser mit einem Trennzeichen in einer Zeile lassen?
\def\vl{\textit{v.l.}\xspace}
%\newcommand{\var}[1]{\texteng{\scriptsize #1}}
%\newcommand{\varsep}{\xspace\texteng{\textbullet}\xspace}

\def\sl#1{\emph{#1}}
\newcommand{\medialink}[2]{\textcolor{violet}{\underline{#1}}}
%\TeXtoTEIPat{\medialink {#1}{#2}}{<ref target="/images/#2">#1</ref>}
\usepackage{url}

\newcommand{\alphaOne}{\textalpha\textsubscript{1}}% N3
\newcommand{\alphaTwo}{\textalpha\textsubscript{2}}% J5
\newcommand{\alphaThree}{\textalpha\textsubscript{3}}% G4
\newcommand{\gammaOne}{\textgamma\textsubscript{1}}% N23
\newcommand{\gammaTwo}{\textgamma\textsubscript{2}}% J7
\newcommand{\deltaOne}{\textdelta\textsubscript{1}}% V19
\newcommand{\deltaTwo}{\textdelta\textsubscript{2}}% K3
\newcommand{\deltaThree}{\textdelta\textsubscript{3}}% C7
\newcommand{\deltaOmega}{\textdelta\textsubscript{\textomega}}% J6
\newcommand{\epsilonOne}{\textepsilon\textsubscript{1}}% G11
\newcommand{\epsilonTwo}{\textepsilon\textsubscript{2}}% G5
\newcommand{\zetaOne}{\textzeta\textsubscript{1}}% P15
\newcommand{\zetaTwo}{\textzeta\textsubscript{2}}% N19
\newcommand{\zetaThree}{\textzeta\textsubscript{3}}% V15
\newcommand{\zetaFour}{\textzeta\textsubscript{4}}% J11
\newcommand{\zetaOmega}{\textzeta\textsubscript{\textomega}}% N26
\newcommand{\etaOne}{\texteta\textsubscript{1}}% V1
\newcommand{\etaTwo}{\texteta\textsubscript{2}}% J10
\newcommand{\etaOmega}{\texteta\textsubscript{\textomega}}% E4
\newcommand{\piOne}{\textpi\textsubscript{1}}% P11
\newcommand{\piTwo}{\textpi\textsubscript{2}}% C6
\newcommand{\piOmega}{\textpi\textsubscript{\textomega}}% V3

\def\attr{\hbox{attrib.}\xspace}
\babelhyphenation{%
	Dattā-treya-yoga-śāstra
	Gorakṣa-śataka
	Go-rakṣa-nātha
	Haṭha-pra-dī-pikā
	Haṭha-ratnā-valī
	Haṭha-tattva-kaumudī
	Jāran-dhara
	Rāja-yoga
	Śām-bhavī
	Śāṃ-bhavī
	Śārṅga-dhara-pad-dhati
	Svātmā-rāma
	Śiva-saṃhitā
	Vasiṣṭha-saṃhitā
	Viveka-mārtaṇḍa
	Yukta-bhava-deva
	Yoga-cintā-maṇi
	Yoga-tattva-pra-kāśa
	Yoga-yājña-valkya
	}

\def\hindsection{3}


% N3,J5,(G4), P11,C6,V3, N23,J7, V19,E2,C7(Vajrolī only), G11,G5(from Vajrolī on), P15(up to 13a),N19,V15, V1,J10,Jyo
% J6,E4 (for the Khecaryabhyāsakrama only)
% not included: C1(lost),P23
% discarded: C8,N17,J11,J15

% 3.26b apparatus format check

\begin{document}
\thispagestyle{firstpage}
\begin{center}
\section*{Chapter 3}
\end{center}
\bigskip
\begin{otherlanguage}{iast}
\begin{ekdosis}

\begin{tlg}[hp03_001]
\tl{
\pada{\app{\lem[wit={ceteri}]{saśaila}
		\rdg[wit={V3}]{saśaile}
		\rdg[wit={J5}]{śaila}}%
	\app{\lem[wit={ceteri}]{vana}
		\rdg[wit={N23}]{vane}}%
	\app{\lem[wit={ceteri}]{dhātrīṇāṃ}% °<<ṇāṃ>> J7
		\rdg[wit={P11,C6}]{dhātṝṇāṃ}% dhātṛṇāṃ P11
		\rdg[wit={J5}]{dhātrāṇāṃ}}}
\pada{yathādhāro%
	\app{\lem[wit={ceteri}]{'hināyakaḥ}
		\rdg[wit={J5}]{hi nāthakaḥ}
		\rdg[wit={J7}]{himālayaḥ}
		}/} \\+}
\tl{
\pada{sarveṣāṃ
	\app{\lem[wit={ceteri}]{yoga}% yo<<ga>> J7
		\rdg[wit={GrB}]{haṭha}
		}tantrāṇāṃ}
\pada{tathādhāro hi kuṇḍalī//\versenr}\\!} %3.1
\end{tlg}
\commcite\newpage

\begin{tlg}[hp03_002]
\tl{
\pada{suptā % guptā? E2
	\app{\lem[wit={ceteri}]{guruprasādena}
		\rdg[wit={E2,G11}]{gurūpadeśena}}} % prasānena N23; saptāṃgulaprasādena J5
\pada{\app{\lem[wit={N3,J5,G11,V15,J10,P11,V3,Jyo}]{yadā jāgarti kuṇḍalī}
		\rdg[wit={P15,N19,V1,C6}]{yathā jāgarti kuṇḍalī}% +N24
		\rdg[wit={Gr2,Gr3}]{bodhitā sukhadā bhavet}}/}\\+}
\tl{
\pada{\app{\lem[wit={N3,Gr2,G11,V15,J10,GrB,Jyo}]{tadā}% +J17
		\rdg[wit={J5,G4,V19,P15,N19,V1}]{tathā}% śtathā? N19; +K3,C7
		\rdg[wit={E2}]{tayā}}
	\app{\lem[wit={ceteri}]{sarvāṇi padmāni}
		\rdg[wit={J10}]{padmāni sarvāṇi}
		\rdg[wit={V19}]{pi sarvapadmāni}}}
\pada{bhidyante granthayo'pi ca//\versenr}\\!}  %3.2 % grathīyā N23, gaṃthiyo N3
\end{tlg}
\commcite%\newpage

\begin{tlg}[hp03_003]
\tl{\app{\lem[alt={\ante prāṇasya \add},nosep]{\skp{\ante prāṇasya \add}}
	\rdg[wit={V15}]{suṣumnā\,|}}%
\pada{\app{\lem[wit={ceteri}]{prāṇasya}
		\rdg[wit={J5}]{praṇasya}
		\rdg[wit={N19}]{praṇamya}% +K3
		\rdg[wit={P11}]{praṇavasya}
		\rdg[wit={C6}]{prāṇa}}
	śūnya% śūnyā J7, śunya N3, sūnya P11
	\app{\lem[wit={ceteri}]{padavī}
		\rdg[wit={N19,V15}]{padavīṃ}}%
		}
\pada{\app{\lem[nolem]{\skp{pāda b}}
	\rdg[wit={P15,N19,V15},alt={\om},post=\texteng{(jump to \textit{śūnyapadavī} in the next verse)}]{\skp{\om\ (jump to \textit{śūnyapadavī} in the next verse.)}}}%
\app{\lem[wit={N3,J5,Gr2,G11,J10,V3,Jyo}]{tadā}
		\rdg[wit={V19,P11,C6}]{tathā}% +K3,C7
		\rdg[wit={E2}]{tayā}
		\rdg[wit={V1}]{yathā}}
\app{\lem[wit={ceteri}]{rājapathāyate}% raja° C6
		\rdg[wit={V1}]{rājapadāyate}
		\rdg[wit={J5},alt={\om}]{\skp{\om}}}/}\\+}
\tl{
\pada{\app{\lem[nolem]{\skp{pāda c}}
	\rdg[wit={P15,N19,V15},alt={\om}]{\skp{\om}}}%
\app{\lem[wit={N3,J5,C6pc,J7,G11,J10,P11,V3,Jyo}]{tadā}
		\rdg[wit={C6ac,V19,V1}]{tathā} % tathac V19; +C7
		\rdg[wit={E2}]{tayā}
		\rdg[wit={N23}]{yadā}}
	cittaṃ nirālambaṃ} % °<<laṃ>>baṃ C6
\pada{\app{\lem[nolem]{\skp{pāda d}}
	\rdg[wit={P15,N19,V15},alt={\om}]{\skp{\om}}}%
\app{\lem[wit={N3,J5,Gr2,G11,J10,GrB,Jyo}]{tadā}
		\rdg[wit={V19,V1}]{tathā}% +K3,C7
		\rdg[wit={E2}]{tayā}}
		kālasya vañcanam//\versenr}% kāralya V19
%		\myfn{After pāda a, \getsiglum{P15,N19,V15} jump to \textit{śūnyapadavī} in the next verse.}
%		\lineom{bcd}{P15,N19,V15} \anm{eye-skip}
		\\!}  %3.3
\end{tlg}
\commcite\newpage

\begin{ava}[hp03_004a]
\app{\lem[wit={N3,Gr2,G11,P11,C6}]{śūnyapadavīti kim} % +K3,C7
		\rdg[wit={E2}]{atha śūnyapadavīti kam}
		\rdg[wit={J10}]{atha śūnyapadavīm iti kim ucyate}
		\rdg[wit={J5}]{punyapadavīti}
		\rdg[wit={V19,P15,N19,V15,V1,V3,Jyo},alt={\om}]{\skp{\om}}% +N24
		}/ 
%	\NotIn{V19,P15,N19,V15,V1,V3,Jyo}
%	\sgwit{N3,J5,Gr2,E2,K3,C7,G11,J10,P11,C6}
\end{ava}

\begin{tlg}[hp03_004]
\tl{
\pada{\app{\lem[nosep]{\skp{pāda a}}
	\rdg[wit={P15,N19,V15},alt={\om}]{\skp{\om}}}%
	suṣumṇā śūnyapadavī} % padavīva N3; sukhu<<m>>ṇā, śunya N23
\pada{brahma\app{\lem[wit={J5,N23,V1}]{randhra}
		\rdg[wit={ceteri}]{randhraṃ}}%
	mahā\app{\lem[wit={G4,P15,V15,G11,V1,J10,Jyo}]{pathaḥ}
		\rdg[wit={J5,Gr2,Gr3,N19,GrB}]{pathaṃ}
		\rdg[wit={N3}]{pathāḥ}}/}
%		\lineom{a}{P15,N19,V15}
		\\+}
\tl{
\pada{\app{\lem[wit={ceteri}]{śmaśānaṃ}% sma° V3
		\rdg[wit={V19}]{śmaśāne}
		\rdg[wit={J7,E2}]{śmaśānī}% +K3,C7
		\rdg[wit={N23}]{aiśānī}
		} śāmbhavī
	\app{\lem[wit={Gr1,Gr2,Gr3,P15,V15,P11,Jyo}]{madhya}
		\rdg[wit={G11,V1,J10,C6,V3}]{madhyaṃ}
		\rdg[wit={N19}]{madhye}
		}}%
\pada{\app{\lem[wit={ceteri}]{mārgaś cety eka} % māgraśceteka P15, māga° V15; catye°? J5, °gaśutyeka V19
		\rdg[wit={G11}]{margeś cety eka}
		\rdg[wit={E2}]{margeś cety eva}
		\rdg[wit={V1}]{mārge vety eka}
		\rdg[wit={N19}]{mārgapratyeka}
%		\rdg[wit={K3,C7}]{mārgaḥ śūnyeva}
		}%
	\app{\lem[wit={J7,G11,J10,Jyo}]{vācakāḥ}
		\rdg[wit={N23}]{vācakā}
		\rdg[wit={N3,C6}]{vācakaḥ}
		\rdg[wit={V3}]{vācaka}
		\rdg[wit={J5,Gr3,P15,N19,V15,V1,P11}]{vācakaṃ}% +F,N24
		}//\versenr}\\!}  %3.4
\end{tlg}

\avacite{4a}
\commcite\newpage

\begin{tlg}[hp03_005]
\tl{
\pada{tasmāt sarvaprayatnena}
\pada{\app{\lem[wit={Gr2,G11,P15,V15,J10,Jyo},alt={prabodhayitum}]{prabodhayitu\skp{m}}
		\rdg[wit={N3}]{prabodhayītum}
		\rdg[wit={J5,N19,V1,C6,V3}]{prabodhayatum}
		\rdg[wit={P11}]{prabodhayatām}
		\rdg[wit={Gr3}]{tāṃ bodhayituṃ}}%
	\app{\lem[wit={N3,J7,E2,G11,P15,V15,V1,J10,C6,Jyo},alt={īśvarīṃ}]{\skm{m }īśvarīm}
		\rdg[wit={J5,N23,N19,V3}]{īśvarī}% °rīṃ J8
		\rdg[wit={V19,P11}]{īśvaraṃ}}/}\\+}
\tl{
\pada{brahma\app{\lem[wit={ceteri}]{dvāra}
		\rdg[wit={P11}]{dvāraṃ}
		\rdg[wit={P15,N19}]{dvāre}}%
	\app{\lem[wit={ceteri}]{mukhe}
		\rdg[wit={N23}]{mukha}
		\rdg[wit={P15}]{sukhe}}
	\app{\lem[wit={ceteri}]{suptāṃ}
		\rdg[wit={G4,G11}]{suptā}
		\rdg[wit={V3}]{supto}}}
\pada{mudrā\app{\lem[wit={N3,P15,N19,V15,V1,J10,P11,V3,Jyo},alt={°bhyāsaṃ samācaret}]{\skp{°}bhyāsaṃ samācaret}% bhyāsa V3
		\rdg[wit={J5}]{bhyāse samācaret}
		\rdg[wit={G11}]{bhyāsaṃ sadābhyaset}
		\rdg[wit={Gr2,Gr3,C6}]{bhyāsena bodhayet}}//\versenr}\\!}  %3.5
\end{tlg}
\commcite%\newpage


\begin{tlg}[hp03_006]
\tl{
\pada{mahāmudrā mahābandho} % mudrāṃ N3, mudrāmudrā N19
\pada{\app{\lem[nolem]{\skp{pāda b}}
	\rdg[wit={V19},alt={\om}]{\skp{\om}}}%
	mahāvedhaś ca khecarī/}\\+} % °vedhoś N19; ṣecarī J5
\tl{
\pada{\app{\lem[nolem]{\skp{pāda c}}
	\rdg[wit={V19},alt={\om}]{\skp{\om}}}%
	\app{\lem[wit={N3pc,J7,N19}]{uḍḍīyāṇaṃ}
		\rdg[wit={N3ac}]{uḍḍīyānaṃ}
		\rdg[wit={N23,E2,J10}]{uḍḍiyānaṃ}% +K3,C7
		\rdg[wit={C6}]{uḍiyānaṃ}% +J17
		\rdg[wit={J5,V3}]{uḍiyāṇaṃ}% ~N24
		\rdg[wit={V1}]{uḍḍīyāṇo}
		\rdg[wit={P11}]{uddhriyānaṃ}
		\rdg[wit={Jyo}]{uḍyānaṃ}
		\rdg[wit={P15,V15}]{uḍyāṇa}% uḍyāṇaṃ F
		\rdg[wit={G11}]{oḍyāṇaṃ}
%		\rdg[wit={V19},alt={\om}]{\skp{\om}}
		}
  \app{\lem[wit={C6},alt={mūlabandhas}]{mūlabandha\skp{s}}
		\rdg[wit={J10}]{mūlabandhaḥ}
		\rdg[wit={P11,V3}]{mūlabandha}
		\rdg[wit={N3}]{mūlabandhaṃ}% +N24
		\rdg[wit={J5}]{mahābandhaṃ}
		\rdg[wit={Gr2,E2}]{mūlabandho}% muḷa° V15 % ra-vipulā
		\rdg[wit={G11,P15,N19,V15,Jyo}]{mūlabandhaś ca}% +F
		\rdg[wit={V1}]{mūlabandhāś ca}
%		\rdg[wit={V19},alt={\om}]{\skp{\om}}
		}}%
\pada{\app{\lem[wit={N3,J5,P15,V15,J10,GrB},alt={tato}]{\skm{s }tato}
		\rdg[wit={G4,J7,Gr3,G11,N19,V1,Jyo}]{bandho}% +N24 & 3 testimonies
		\rdg[wit={N23}]{bandhā}
		}
	\app{\lem[wit={ceteri}]{jālandharā}% jalaṃ° J10ac
		\rdg[wit={N3,P11}]{jālāṃdharā}
		\rdg[wit={G11}]{jālāntarā}
		\rdg[wit={V1}]{jālaṃjarā}}bhidhaḥ//\versenr}
%		\lineom{bc}{V19}
		\\!}  %3.6
\end{tlg}
\commcite\newpage


\begin{tlg}[hp03_007]
\tl{
\pada{\app{\lem[wit={ceteri}]{karaṇī} % karanī N23
		\rdg[wit={G4,P11}]{karaṇaṃ}}
	\app{\lem[wit={ceteri}]{viparītākhyā}
		\rdg[wit={G4,N19}]{viparītākhyaṃ}
		\rdg[wit={P11}]{viparītā syāt}
		\rdg[wit={P15}]{viparītāni}}}
\pada{\app{\lem[wit={ceteri}]{vajrolī} % varjrolī N23, vajolī N19, vajroli V15
		\rdg[wit={N3}]{vajrālī}% $$
		\rdg[wit={P15}]{varjālī}
		\rdg[wit={V15}]{vajroli}
		\rdg[wit={V19}]{vajro}} śakticālanam/} \\+}
\tl{
\pada{\app{\lem[nolem]{\skp{pāda c}}
	\rdg[wit={P11},alt={\om}]{\skp{\om}}}%
\app{\lem[wit={N3,J5}]{idaṃ mudrādi}
		\rdg[wit={G11,Jyo}]{idaṃ hi mudrā}% +N24,N12,M1; evaṃ F,G7
		\rdg[wit={P15,N19}]{idaṃ tu mudrā}
		\rdg[wit={V15,V1,J10,V3}]{idaṃ ca mudrā}
		\rdg[wit={Gr2,Gr3,C6}]{etad dhi mudrā}% etadvi? N23; +P7
		}daśakaṃ}
\pada{\app{\lem[nolem]{\skp{pāda d}}
	\rdg[wit={P11},alt={\om}]{\skp{\om}}}%
jarā\app{\lem[wit={ceteri}]{maraṇa}
		\rdg[wit={V3}]{marṇavi}
		\rdg[wit={N23}]{maṇa}}%
	\app{\lem[wit={ceteri}]{nāśanam}
		\rdg[wit={Gr3}]{varjitaṃ}}//\versenr}
%	\lineom{cd}{P11}
	\\!}  %3.7
\end{tlg}
\commcite\newpage


\begin{tlg}[hp03_008]
\tl{\app{\lem[alt=\mylem{8ab},nosep,nonum]{\skp{pāda ab}}
	\rdg[wit={V15},alt=\textapp{found after 3.9ab}]{\skp{found after 3.9ab}}}%
\pada{\app{\lem[wit={ceteri}]{ādinātho}
		\rdg[wit={Gr3}]{ādīśvaro}% adī° V19
		}ditaṃ
	\app{\lem[wit={ceteri},alt={divyam}]{divya\skp{m}}% dinyaṃ J5
		\rdg[wit={J10}]{sarvaṃ}}m}
\pada{aṣṭaiśvarya\app{\lem[wit={ceteri}]{pradāyakam}% °drāyakaṃ N19
		\rdg[wit={C6}]{phalapradaṃ}}/}% aṣṭaiḥ N3, aṣṭe° N23
%	\anm{after 3.9ab \getsiglum{V15}}
	\\+}
\tl{
\pada{vallabhaṃ sarva% sarve J17
	\app{\lem[wit={ceteri}]{siddhānāṃ}% +J5; siḥddhā° N19
		\rdg[wit={N3}]{siddhīnāṃ}
		\rdg[wit={G4,V15}]{vidyānāṃ}}}
\pada{durlabhaṃ % dulabhaṃ V15
	\app{\lem[wit={ceteri},alt={marutām}]{marutā\skp{m}}
	\rdg[wit={C6}]{mahatām}}m api//\versenr}\\!}  %3.8
\end{tlg}
\commcite%\newpage


\begin{tlg}[hp03_009]
\tl{
\pada{gopanīyaṃ prayatnena} 
\pada{yathā ratnakaraṇḍakam/} \\+}
\tl{
\pada{kasyacin naiva % kasyacci V3
	\app{\lem[wit={ceteri}]{vaktavyaṃ}
		\rdg[wit={V1}]{vaktavyā}
		\rdg[wit={P15,V3}]{kartavyaṃ}}}
\pada{\app{\lem[wit={ceteri}]{kulastrīsurataṃ}
		\rdg[wit={V1}]{kulastrīṣu rataṃ}
		\rdg[wit={P11}]{kulastrīyasukhaṃ}
		\rdg[wit={V3}]{kulastrīsukharataṃ}}
	\app{\lem[wit={ceteri}]{yathā}
		\rdg[wit={J5,J10}]{tathā}}//\versenr}%
	\myfn{In \getsiglum{G4}, this verse is followed by \emph{Vivekamārtaṇḍa} 58 and \HP\ 3.17(c)d--18ab. 3.17cd is written at the correct place again, but 3.18 is omitted there. The confusion may be due to an eye-skip caused by \textit{gopanīyaṃ/yā prayatnena} in 3.9a and 3.18c.}
	%\devnote{śoṣaṇaṃ malajālasya ghāṭanaṃ sūryacaṃdrayoḥ/
	%rasānāṃ śodhana} + + + + + + + + + + ||
	%+ + + + + + + + + + + + + + \devnote{bhyaset/
	%kathiteyaṃ mahāmudrā mahāsidhikari nṛṇāṃ//}
	%The first part is an unknown verse, and the latter part is the same as 3.17d--18ab. 
	\\!}  %3.9
\end{tlg}
\commcite\newpage


\startaltrecension
\begin{alttlg}[hp03_009_1]
\tl{\app{\lem[nolem]{}
	\rdg[wit={J10,C6,V3}]{\incl}}% +E4
\pada{\app{\lem[wit={J10,V3},alt={vajrolīr},post=\texteng{(\emph{r} as a hiatus bridge)}]{vajrolī}
		\rdg[wit={C6}]{vajrolī tv}}%
	\app{\lem[wit={C6},alt={amarolī}]{\skm{r }amarolī}
		\rdg[wit={V3}]{amarolīś}
		\rdg[wit={J10}]{amaroliś}} ca}
\pada{\app{\lem[wit={C6,V3}]{sahajolī}
		\rdg[wit={J10}]{sahajolis}} tridhā
	\app{\lem[wit={C6}]{matāḥ}
		\rdg[wit={J10}]{mataḥ}
		\rdg[wit={V3}]{magaḥ}}/} \\+}
\tl{
\pada{\app{\lem[wit={J10,V3}]{eteṣāṃ}
		\rdg[wit={C6}]{etāsāṃ}}} lakṣaṇaṃ vakṣye % vakṣe V3
\pada{kartavyaṃ ca viśeṣataḥ//\versenr} 
%\sgwit{J10,C6,V3}
\\!}  % 3.9*1  NOT IN P11!
\end{alttlg}
\altcommcite%\newpage
\endaltrecension


%%%%%%%%%%%%%%

\begin{ava}[hp03_010a]
	\app{\lem[wit={J5,G11,V1,J10,P11}]{tatra mahāmudrā}
		\rdg[wit={C6}]{tatha mahāmudrā}
		\rdg[wit={P15}]{tatra mahāmudrā yathā}
		\rdg[wit={N23,E2,N19,V3,Jyo}]{atha mahāmudrā}
		\rdg[wit={V15}]{atha tatra mahāmudrā}
		\rdg[wit={V19}]{mahāmudrā}% +K3
		\rdg[wit={N3,G4,J7},alt={\om}]{\skp{\om}}}/
%	\NotIn{N3,G4,J7}
\end{ava}

\begin{tlg}[hp03_010]
\tl{
\pada{pādamūlena vāmena} % pāde G11; °mūle<<na>> C8
\pada{\app{\lem[wit={ceteri}]{yoniṃ} %%% CHECK! MD
		\rdg[wit={N3,J5,V3}]{yoni}% yoniṃ J8
		\rdg[wit={N19}]{yoniḥ}}
	\app{\lem[wit={ceteri}]{saṃpīḍya dakṣiṇam} % °ṇāṃ N3
		\rdg[wit={P15,N19}]{pīḍya dakṣiṇam}
		\rdg[wit={J5}]{saṃpīḍaye kṣaṇaṃ}}/}\\+}
\tl{
\pada{\app{\lem[wit={ceteri}]{pādaṃ}
		\rdg[wit={J5,J10}]{pāda}
		\rdg[wit={V3}]{padaṃ}
		\rdg[wit={Jyo}]{prasā°}}
	\app{\lem[wit={ceteri}]{prasāritaṃ}
		\rdg[wit={V3}]{prasaritaṃ}
		\rdg[wit={V1}]{prasāditaṃ}
		\rdg[wit={Jyo}]{°ritaṃ padaṃ}}
	\app{\lem[wit={J7,E2,G11,V15,V1,J10}]{dhṛtvā}% +K3,C7
		\rdg[wit={N3,J5,N23,V19,P15,N19,GrB,Jyo}]{kṛtvā}% +F
		}}
\pada{karābhyāṃ
	\app{\lem[wit={ceteri},alt={pūrayen}]{pūraye\skp{n}}% so in Amaraugha
		\rdg[wit={J10}]{dhārayen}
		\rdg[wit={Jyo}]{dhārayed}}%
	\app{\lem[wit={N3,J5,G11,N19,P11,V3},alt={mukhe}]{\skm{n }mukhe}% so in Amaraugha
		\rdg[wit={Gr2,Gr3,P15,V15,V1,J10,C6}]{mukham}
		\rdg[wit={Jyo}]{dṛḍham}}\marma//\versenr}\\!}  %3.10
\end{tlg}

\avacite{10a}
\commcite\newpage


\begin{tlg}[hp03_011]
\tl{
\pada{\app{\lem[wit={ceteri}]{kaṇṭhe}% +J8
		\rdg[wit={V19,G11,N19,V3}]{kaṇṭha}}
	\app{\lem[wit={J5,G4,J7,V19,P15,V15,J10,P11,Jyo}]{bandhaṃ}
		\rdg[wit={N23,G11,V1,C6,V3}]{bandha}
		\rdg[wit={N19}]{bandhaḥ}
		\rdg[wit={E2}]{bandhe}% +K3
		\rdg[wit={N3}]{budha}
		} 
	\app{\lem[wit={ceteri}]{samāropya}
		\rdg[wit={G11}]{°nam āropya}}}
\pada{\app{\lem[wit={ceteri},alt={dhārayed}]{dhāraye\skp{d}} % dvārayed P11
		\rdg[wit={V19}]{dhānayed}}d 
		vāyum ūrdhvataḥ/}\\+} % °ta N3
\tl{
\pada{\app{\lem[wit={ceteri}]{yathā}
		\rdg[wit={V1}]{pathi}}
	\app{\lem[wit={N3,G4,P15,V15,J10,P11,V3}]{daṇḍāhataḥ} % +J7pc; °hata V3
		\rdg[wit={J5,Gr2,Gr3,G11,N19,V1,C6,Jyo}]{daṇḍahataḥ}
		} sarpo}
\pada{\app{\lem[wit={ceteri}]{daṇḍākāraḥ} % kāra J5,P11
		\rdg[wit={Gr2,G11,N19}]{daṇḍakāraḥ}} % kāraṃ J7ac?
	\app{\lem[wit={ceteri}]{prajāyate}
		\rdg[wit={V1}]{prayujyate}
		}//\versenr}\\!}  %3.11
\end{tlg}
\commcite\newpage


\begin{tlg}[hp03_012]
\tl{
\pada{\app{\lem[wit={J7,Gr3,G11,V15,V1,J10,P11,Jyo}]{ṛjvībhūtā}
		\rdg[wit={C6}]{ṛjvībhūtvā}
		\rdg[wit={N3,V3}]{rujvībhūtvā}% °bhūtatathā śaktiḥ J5; +F
		\rdg[wit={N19}]{rajvībhūtā}
		\rdg[wit={J5}]{rajvībhū}
		\rdg[wit={P15}]{vajrībhūtā}
		\rdg[wit={N23}]{ṛ\,\_\,bhūtrā}}
	\app{\lem[wit={ceteri}]{tathā}
		\rdg[wit={N19}]{yathā}}
	\app{\lem[wit={ceteri}]{śaktiḥ}
		\rdg[wit={V19,N19,V1,P11,V3}]{śakti}}}
\pada{kuṇḍalī sahasā bhavet/}\\+} % kuṃḍali N23; saha bhāvayet G11
\tl{
\pada{\app{\lem[wit={N3,J5,Gr2,P15,N19,V1,GrB}]{tadāsau}
		\rdg[wit={Gr3}]{tathāsau}
		\rdg[wit={G4,V15,J10,Jyo}]{tadā sā}
		\rdg[wit={G11}]{tadā sa}} % tada N17
	\app{\lem[wit={ceteri}]{maraṇā}% +J17
		\rdg[wit={P15}]{maraṇa}
		\rdg[wit={V1}]{maraṇī}
		\rdg[wit={V3}]{ramaṇā}
		\rdg[wit={J10}]{maṇā}
		\rdg[wit={G11}]{marasā}}%
	\app{\lem[wit={ceteri}]{vasthā}
		\rdg[wit={J7,Gr3,V1}]{vasthāṃ}
		\rdg[wit={P15}]{sthā}}}
\pada{\app{\lem[wit={ceteri}]{jāyate}% jayate P11
		\rdg[wit={P15}]{yāyate}
		\rdg[wit={Gr2,Gr3}]{harate}}
	\app{\lem[wit={N3,J5,Gr3,G11,V1,J10,GrB,Jyo}]{dvipuṭā}
		\rdg[wit={N23}]{dvipūtā}
		\rdg[wit={P15,N19}]{nṛpuṭā}
		\rdg[wit={G4,V15}]{tripuṭā}
		\rdg[wit={J7}]{vapurā}}%
	\app{\lem[wit={N3,J5,N19,J10,P11,V3},alt={°śritā}]{\skp{°}śritā}
		\rdg[wit={N23}]{śrayī}
		\rdg[wit={J7}]{śrayāṃ}
		\rdg[wit={V19,G11,Jyo}]{śrayā}% +K3
		\rdg[wit={G4,E2}]{śrayaḥ}
		\rdg[wit={V1}]{[śr]i\,..}
		\rdg[wit={P15}]{smṛtā}
		\rdg[wit={V15}]{sanāṃ}
		\rdg[wit={C6}]{hi sā}}\marma//\versenr}
	\\!}  %3.12
\end{tlg}
\commcite\newpage


\begin{tlg}[hp03_013]
\tl{%
\pada{tataḥ śanaiḥ % stataḥ P11
	\app{\lem[wit={ceteri}]{śanair eva} % śanai(1) P11,V3,V19; śanaiḥr N3
		\rdg[wit={N23}]{śanair yeca}
		\rdg[wit={P15}]{\textapp{breaks off after} śanai}}
		}
\pada{\app{\lem[wit={ceteri},alt={recayen}]{recaye\skp{n}}% recaye J7
		\rdg[wit={N19}]{recaya}}%
	\app{\lem[wit={ceteri},alt={na tu}]{\skm{n }na tu}
		\rdg[wit={E2}]{tanu}% recayet tanu! E2,K3
		\rdg[wit={V3}]{na ca}
		\rdg[wit={Jyo}]{naiva}} vegataḥ/}\\+}
\tl{
\pada{\app{\lem[wit={ceteri}]{iyaṃ}
		\rdg[wit={V3}]{idaṃ}} khalu mahāmudrā}
\pada{mahā\app{\lem[wit={ceteri}]{siddhaiḥ}% siddhai J5,P11,V19
		\rdg[wit={N19,V15}]{siddhiḥ}}
	\app{\lem[wit={J5,Jyo}]{pradarśitā}
		\rdg[wit={N3}]{pradarśanā}
		\rdg[wit={N23,J7,V19,E2,G11,V15,V1,J10,P11,C6,V3}]{praśasyate}% +G11,F,N24; prasaśyate V3
		\rdg[wit={N19}]{prajāyate}}//\versenr}%
\myfn{\getsiglum{P15} breaks off at pāda a.\mydelim %
The \textdelta\ group has a different verse order: 
16 \rightarrow\ 15 \rightarrow\ 13 \rightarrow\ 14.}
%	\unavbl{P15}
%	\anm{\getsiglum{P15} breaks off at 3.13a}
	\\!}  %3.13
\end{tlg}
\commcite\newpage


\begin{tlg}[hp03_014]
\tl{
\pada{\app{\lem[wit={ceteri}]{mahā}% maha P11
		\rdg[wit={J10}]{mahān}
		\rdg[wit={J5}]{mahata}}%
	\app{\lem[wit={N3,G4,V1,J10,C6,V3,Jyo}]{kleśādayo}
		\rdg[wit={J7}]{kleśā yato}
		\rdg[wit={N23}]{kleśa yato}
		\rdg[wit={P11}]{kleśāyatā}
		\rdg[wit={N19}]{kleśā yathā}
		\rdg[wit={V15}]{kleśa yathā}
		\rdg[wit={Gr3}]{kleśā mahā}
		\rdg[wit={G11}]{rogā mahā}
		\rdg[wit={J5}]{kuśodayo}
		}\marmas
	\app{\lem[wit={ceteri}]{doṣā}
		\rdg[wit={J10,Jyo}]{doṣāḥ}
		\rdg[wit={J7}]{doṣa}
		\rdg[wit={G11}]{kleśā}
		\rdg[wit={J5},alt={\om}]{\skp{\om}}}}
\pada{\app{\lem[wit={G4,Gr2,Gr3,G11,V1,P11,C6}]{jīryante}% +F,N24; jīryate N23
		\rdg[wit={V15}]{jīyaṃte}
		\rdg[wit={N19}]{jāyante}
		\rdg[wit={N3}]{hīyante}% cf. Mbh ṣaḍ doṣāḥ puruṣeṇeha hātavyāḥ
		\rdg[wit={J5}]{hrīyaṃte}
		\rdg[wit={J10,V3,Jyo}]{kṣīyante}% +J14
		}
		maraṇādayaḥ/}\\+} % °ādaya P11,V3,N19, °odayaḥ J5
\tl{
\pada{\app{\lem[nolem]{\skp{pāda c}}
	\rdg[wit={Gr3},alt={\om}]{\skp{\om}}}%
mahā\app{\lem[wit={G11,V15,J10,GrB,Jyo}]{mudrāṃ}% mahāṃ P11
		\rdg[wit={Gr1,Gr2,N19,V1}]{mudrā}% mudrāḥ? V1
%		\rdg[wit={Gr3},postwit=\texteng{(Pādas c--d omitted)},alt={\om}]{\skp{\om}}
		}
	\app{\lem[wit={ceteri}]{ca}
		\rdg[wit={N3}]{tu}} % +K3
	\app{\lem[wit={ceteri}]{tenaiva}
		\rdg[wit={N23}]{tenai} % haplo
		\rdg[wit={V15}]{tenetāṃ}
%		\rdg[wit={Gr3},alt={\om}]{\skp{\om}}
		}}
\pada{\app{\lem[nolem]{\skp{pāda d}}
	\rdg[wit={Gr3},alt={\om}]{\skp{\om}}}%
vadanti \app{\lem[wit={ceteri}]{vibudhottamāḥ}
		\rdg[wit={P11,V3}]{vibudhottamā}
		\rdg[wit={N3,J5}]{vibudhottamaḥ}
		\rdg[wit={J7}]{vividhottamāḥ}
%		\rdg[wit={Gr3},alt={\om}]{\skp{\om}}
		}//\versenr} 
%	\lineom{cd}{Gr3}\unavbl{P15}
	\\!}
\end{tlg}
\commcite\newpage


\begin{tlg}[hp03_015]
\tl{
\pada{\app{\lem[wit={ceteri}]{candrāṅge} % °rṃge
		\rdg[wit={V1}]{cāndrāṅge}
		\rdg[wit={N19}]{candrāṃgaṃ}
		\rdg[wit={Gr3}]{candrāṃśaṃ}
		\rdg[wit={P11}]{caṃdrāṃśe}}
	\app{\lem[wit={ceteri}]{tu}
		\rdg[wit={J10}]{ca}} samabhyasya}
\pada{\app{\lem[wit={ceteri}]{sūryāṅge}
		\rdg[wit={V1}]{sūryāṅge°}
		\rdg[wit={N19}]{sūryāṃgaṃ}
		\rdg[wit={Gr3}]{sūryāṃśaṃ}}
	\app{\lem[wit={N3,J5,Gr2,G11,J10,GrB,Jyo}]{punar abhyaset}
		\rdg[wit={Gr3,N19,V15}]{tu samabhyaset}
		\rdg[wit={V1}]{°ṣu samabhyaset}}/}\\+}
\tl{
\pada{\app{\lem[nolem]{\skp{pāda c}}
	\rdg[wit={V15},alt={\om}]{\skp{\om}}}%
\app{\lem[wit={ceteri},alt={yāvat}]{yāva\skp{t}}
		\rdg[wit={P11}]{ye ca}}%
%	\app{\lem[wit={N3,J5,Gr2,G11,N19,V1,GrB,Jyo},alt={tulyā}]{\skm{t }tulyā} % tulyāṃ V1? yāvaścalyā? J5; G11 nicht nach ly aus, aber G5 und M3 have lyā
	\app{\lem[wit={ceteri},alt={tulyā}]{\skm{t }tulyā}
		\rdg[wit={J10}]{saṃkhyā}
		\rdg[wit={Gr3}]{tayor}
%		\rdg[wit={V15},alt={\om}]{\skp{\om}}
		}
	\app{\lem[wit={ceteri},alt={bhavet}]{bhave\skp{t}}
		\rdg[wit={J7,V1}]{bhavat}
%		\rdg[wit={V15},alt={\om}]{\skp{\om}}
		}%
%	\app{\lem[wit={N3,J5,Gr2,G11,V1,GrB,Jyo},alt={saṃkhyā}]{\skm{t }saṃkhyā}% sakhyā J5
	\app{\lem[wit={ceteri},alt={saṃkhyā}]{\skm{t }saṃkhyā}% sakhyā J5
		\rdg[wit={N19}]{saṃkṣā}
		\rdg[wit={Gr3}]{sāmyaṃ}
		\rdg[wit={J10}]{tulyā}
%		\rdg[wit={V15},alt={\om}]{\skp{\om}}
		}}
\pada{\app{\lem[nolem]{\skp{pāda d}}
	\rdg[wit={V15},alt={\om}]{\skp{\om}}}%
tato mudrāṃ % mudrā C8
	\app{\lem[wit={ceteri}]{visarjayet}
		\rdg[wit={V19}]{visaryayet}
		\rdg[wit={P11,V3}]{vivarjayet}% +F
%		\rdg[wit={V15},alt={\om}]{\skp{\om}}
		}//\versenr}
%	\lineom{cd}{V15}
	\\!}  %3.15
\end{tlg}
\commcite\newpage



\begin{tlg}[hp03_016]
\tl{
\pada{\app{\lem[wit={ceteri}]{na hi pathyam apathyaṃ vā}% pathām apathaṃ P11, <<vā>> V1 
		\rdg[wit={J10}]{nāpathyāṃ na hi pathyaṃ ca}
		\rdg[wit={N19}]{na hi madhyaṃ vā}}} % 3 akṣaras missing
\pada{rasāḥ  % rasā J5,P11,C6,V3
		sarve\app{\lem[wit={ceteri}]{'pi}
		\rdg[wit={V19}]{hi}} nīrasāḥ/}\\+}
\tl{
\pada{\app{\lem[wit={ceteri}]{api bhuktaṃ}
		\rdg[wit={N23}]{api muktaṃ}
		\rdg[wit={N19,V15}]{ahimuktaṃ}
		\rdg[wit={P11}]{api viṣaṃ}}
	\app{\lem[wit={ceteri}]{viṣaṃ ghoraṃ}% ghāraṃ? N23, raghāraṃ V1
		\rdg[wit={P11}]{ghora bhuktaṃ}}}
\pada{\app{\lem[wit={ceteri},alt={pīyūṣam}]{pīyūṣa\skp{m}}% pi° G11
		\rdg[wit={J5,V3}]{piyuṣam}}%
	\app{\lem[wit={ceteri},alt={iva}]{\skm{m }iva}
		\rdg[wit={P11}]{api}% +K3
		}
	\app{\lem[wit={ceteri}]{jīryate}% jīrjate V19
		\rdg[wit={Gr2,G11,Jyo}]{jīryati}% +F
		\rdg[wit={J5}]{jāyate}}//\versenr}
	\\!}
\end{tlg}
\commcite%\newpage

\begin{tlg}[hp03_017]
\tl{\app{\lem[nolem]{}
	\rdg[wit={P11},alt={\om}]{\skp{\om}}}%
\pada{kṣaya\app{\lem[wit={ceteri}]{kuṣṭha} % kṣayaṃ J5,N19,F
		\rdg[wit={J5,V1}]{kuṣṭhaṃ}% +F
		}%
	\app{\lem[wit={ceteri}]{gudā}
		\rdg[wit={V19,N19,V15}]{mudā}}varta}%
\pada{\app{\lem[wit={N3,N19,V15,V1,J10,V3,Jyo}]{gulmājīrṇa}
		\rdg[wit={J5,Gr3,G11,C6}]{gulmajīrṇa}% jīṇa J5
		\rdg[wit={Gr2}]{gulmaplīha}}%
	\app{\lem[wit={ceteri}]{purogamāḥ}
		\rdg[wit={J5,V3}]{purogamā}
		\rdg[wit={V19}]{jvarās tathā}% +K3
		\rdg[wit={E2}]{jarādayaḥ}}/}\\+}
\tl{
\pada{\app{\lem[wit={ceteri}]{tasya doṣāḥ} % doṣā J5,V3,V19
		\rdg[wit={V1,J10}]{doṣāḥ sarve}}
		kṣayaṃ yānti} % jāṃti V19, yā[ṃ]ti N3
\pada{mahāmudrāṃ \app{\lem[wit={ceteri}]{tu yo'bhyaset}
		\rdg[wit={V15,C6}]{ca yo bhyaset}
		\rdg[wit={V3}]{yo<<ma>>bhyaset}}//\versenr}
%		\NotIn{P11}
		\\!}  %3.17
\end{tlg}
\commcite\newpage


\begin{tlg}[hp03_018]
\tl{\app{\lem[nolem]{}
	\rdg[wit={P11},alt={\om}]{\skp{\om}}}%
\app{\lem[alt=\mylem{18ab},nosep]{\skp{pāda ab}}
	\rdg[wit={G4},alt=\textapp{found after 3.9}]{\skp{found after 3.9}}}%
\pada{\app{\lem[wit={ceteri}]{kathiteyaṃ} % °ya V15
		\rdg[wit={J5,N19,V3}]{kathitoyaṃ}
		} mahāmudrā}
\pada{\app{\lem[wit={G4,G11,N19,V15,V1,J10,V3,Jyo}]{mahāsiddhikarī nṛṇām}% +N24,M3; keep! ## nṛṇā V15,V1
		\rdg[wit={N3,Gr2,Gr3,C6}]{jarāmṛtyuvināśinī} % +M1,F; °nāsinī V19, °vināśanaṃ G7
		\rdg[wit={J5}]{nṛṇāṃ mṛtyuvināśinī}}\marma/}\\+}
\tl{
\pada{\app{\lem[nolem]{\skp{pāda c}}
	\rdg[wit={G4},alt={\om}]{\skp{\om}}}%
	\app{\lem[wit={ceteri}]{gopanīyā}% +ḥ N23
		\rdg[wit={J5,N19,V3}]{gopanīyaṃ}
		\rdg[wit={J10}]{gopanīyāṃ}
		} prayatnena}
\pada{\app{\lem[nolem]{\skp{pāda d}}
	\rdg[wit={G4},alt={\om}]{\skp{\om}}}%
	na
	\app{\lem[wit={ceteri}]{deyā}
		\rdg[wit={V3}]{deyaṃ}}
	yasya kasyacit//\versenr}   % yasyā J7ac 
%	\NotIn{P11}
	\\!} %3.18
\end{tlg}
\commcite\newpage

%%%%%%%%%%%%%%

\begin{ava}[hp03_019a]
  \app{\lem[nolem]{}
    \rdg[wit={P11},alt={\om}]{\skp{\om}}}%
  \app{\lem[wit={ceteri}]{atha}
		\rdg[wit={Gr2},alt={\om}]{\skp{\om}} % +K3
		}
	\app{\lem[wit={ceteri}]{mahābandhaḥ}
		\rdg[wit={N3,N19,V3}]{mahābandha}
		\rdg[wit={E2}]{mahāvedhaḥ}
		}/
%	\NotIn{P11}
\end{ava}

\begin{tlg}[hp03_019]
\tl{
\pada{\app{\lem[wit={N3,G4,Gr3,G11,V15,V1,Jyo}]{pārṣṇiṃ}
%		\rdg[wit={J8}]{pārṣmiṃ}
		\rdg[wit={J5,J7,N19,J10,GrB}]{pārṣṇi}% pāṣṇi J5
		\rdg[wit={N23}]{yāṣi}}
	\app{\lem[wit={ceteri}]{vāmasya}
		\rdg[wit={J10}]{bhāgena}} pādasya}
\pada{yonisthāne % yonī V15, yoniḥ N19
	\app{\lem[wit={ceteri}]{niyojayet}
		\rdg[wit={N19}]{yojayet}}/}\\+}
\tl{
\pada{\app{\lem[nolem]{\skp{pāda c}}
	\rdg[wit={P11},alt={\om}]{\skp{\om}}}%
vāmorūpari saṃsthāpya}
\pada{\app{\lem[nolem]{\skp{pāda d}}
	\rdg[wit={P11},alt={\om}]{\skp{\om}}}%
\app{\lem[wit={ceteri}]{dakṣiṇaṃ}
		\rdg[wit={J5,V3}]{dakṣaṇaṃ}
		\rdg[wit={E2}]{dakṣiṇe}
		} caraṇaṃ tathā//\versenr}
%		\lineom{cd}{P11}
		\myfn{%
\manuref{3.19c–3.20d} are omitted in \getsiglum{P11} and \getsiglum{N19,V15}, probably due to an eye-skip caused by \textit{niyojayet} in 3.19b and 3.20d. In the latter manuscripts, however, the lines are inserted after \ref{matam}ab. It seems that they were supplied from a manuscript of the \textalpha\ or \textdelta\ group, as they are followed by \ref{III21} in \getsiglum{N19}.}
		\\!}
\end{tlg}

\avacite{19a}
\commcite\newpage
		%3.19


\begin{tlg}[hp03_020]
\tl{\app{\lem[nolem]{}
	\rdg[wit={P11},alt={\om}]{\skp{\om}}}%
\pada{pūrayitvā
	\app{\lem[wit={N3,J5,Gr2,Gr3,C6}]{mukhe}
		\rdg[wit={G11,V1,J10,V3,Jyo}]{tato}
		\rdg[wit={N19,V15}]{tathā}}
	\app{\lem[wit={ceteri}]{vāyuṃ}
		\rdg[wit={J5,Gr2,V3}]{vāyu}}}
\pada{hṛdaye % gudaye J5
	\app{\lem[wit={ceteri}]{cibukaṃ}% cibuke V19ac, cubukaṃ Jyo
		\rdg[wit={V15}]{sasvanaṃ}
		\rdg[wit={N19}]{svasanaṃ}}
	\app{\lem[wit={ceteri}]{dṛḍham}% tathā dṛḍhaṃ V19ac!
		\rdg[wit={C6}]{tathā}}/}\\+}
\tl{
\pada{\app{\lem[wit={N3,V3}]{nibhṛtya} % = Amaraugha
		\rdg[wit={C6}]{nibhṛtaṃ}
		\rdg[wit={N19,V15}]{nivṛtya}
		\rdg[wit={E2,G11}]{nipīḍya}% +K3
		\rdg[wit={Gr2,V19,V1,Jyo},alt={niṣ-/niḥpīḍya}]{niṣpīḍya}% niḥ- Gr2,V19; +G5,F,C7
		\rdg[wit={J10}]{nikṣipya}
		\rdg[wit={J5}]{nitya}} 
	yoni\app{\lem[wit={ceteri},alt={ākuñcya}]{\skm{m }ākuñcya}% °kuṃcca N23
		\rdg[wit={J5}]{samākuñcya}}}
\pada{\app{\lem[wit={ceteri}]{mano}
		\rdg[wit={V15}]{tato}} madhye niyojayet//\versenr}   % nijojayet N23
%	\NotIn{P11}
	\\!}
\end{tlg}
\commcite\newpage


\begin{tlg}[hp03_021]
\tl{\app{\lem[nolem]{}
	 \rdg[wit={Gr1,Gr3,N19}]{\incl}}%
\pada{\app{\lem[wit={N3,J5,Gr3,N19}]{recayec ca śanair eva}% evaṃ J5
		\rdg[wit={G4}]{vased evaṃ mahābaṃdho}}}
\pada{\app{\lem[wit={N3,J5,Gr3,N19}]{mahābandho'yam ucyate}
		\rdg[wit={G4}]{recayec ca śanaiḥś śanaiḥ}
		}//\versenr}\label{III21}
		%\anm{included in \getsiglum{Gr1,Gr3,N19}}
		\\!}
\end{tlg}
\commcite%\newpage

\startaltnormal
\begin{alttlg}[hp03_021_1]
\tl{
\pada{\app{\lem[wit={ceteri}]{dhārayitvā}
		\rdg[wit={N19}]{cālayitvā}}
	\app{\lem[wit={G11,N19,V15,J10,P11,V3}]{yathāśaktyā}% +F
		\rdg[wit={Gr2,V1,C6,Jyo}]{yathāśakti}
		}}
\pada{recaye\app{\lem[wit={ceteri},alt={anilaṃ}]{\skm{d }anilaṃ} 
		\rdg[wit={G11}]{aniśaṃ}}
		śanaiḥ % śanai P11,V3
		/}\\+
\pada{savyāṅge % °gai P11
	\app{\lem[wit={J7,G11,N19,V15,V1,J10}]{ca samabhyasya}% °sye N3
		\rdg[wit={Jyo}]{tu samabhyasya}
		\rdg[wit={GrB}]{pūrvam abhyasya}
		\rdg[wit={N23},alt={\om}]{\skp{\om}}}}
\pada{\app{\lem[wit={G11,N19},alt={dakṣāṅge ca sam°}]{dakṣāṅge ca sama}% +G7,F
		\rdg[wit={V15,V1,C6}]{dakṣiṇāṅge sam°}
		\rdg[wit={N23}]{sam°}
		\rdg[wit={J7,V3,Jyo}]{dakṣāṅge punar}% +M3
		\rdg[wit={P11}]{dakṣiṇāṅge punar}
		\rdg[wit={J10}]{dakṣiṇe punar}
		}bhyaset//\versenr}
		\anm{included in \getsiglum{Gr2,G11,N19,V15,V1,J10,GrB,Jyo} as a substitute for 3.21.\mydelim 3.21*1cd = 3.22cd.} \\!}
\end{alttlg}
\altcommcite\newpage


\begin{alttlg}[hp03_021_2]
\tl{\app{\lem[nolem]{}
	\rdg[wit={G4,Gr2,G11,N19,V15,V1,J10,GrB,Jyo}]{\incl}}%
\pada{\app{\lem[wit={G4,Gr2,G11,N19,V15,P11,C6,Jyo}]{matam atra}
		\rdg[wit={V1}]{matam etat}
		\rdg[wit={V3}]{matāntare}
		\rdg[wit={J10}]{matārettamaṃtra}}
	\app{\lem[wit={ceteri}]{tu}
		\rdg[wit={Gr2}]{ca}} keṣāṃcit} % ci V3
\pada{\app{\lem[wit={ceteri}]{kaṇṭhabandhaṃ}% ṃ om. V3
		\rdg[wit={G4,J10}]{kaṇṭhe bandhaṃ}
		\rdg[wit={N23}]{kaṃḍhayaṃ}}
	\app{\lem[wit={G4,G11,N19,V15,P11,V3}]{visarjayet}
		\rdg[wit={V1,J10,Jyo}]{vivarjayet}
		\rdg[wit={Gr2,C6}]{tu varjayet}}/}
		\label{matam}\\+}
\tl{\app{\lem[alt=\mylem{21*2cd},nosep]{\skp{pāda cd}}
	\rdg[wit={N19,V15},alt=\textapp{found after \ref{somasurya}}]{\skp{found after 3.27}}}%
\pada{\app{\lem[wit={G11,C6}]{rājadantabilaṃ tatra}% +P7
		\rdg[wit={V3}]{rājadantabilaṃ jatra}
		\rdg[wit={P11}]{virājaṃti bilaṃ tatra}
		\rdg[wit={N19,V15}]{rājadantabalaṃ haṃti}% rājaddaṃntabaḷaṃ V15
		\rdg[wit={Gr2}]{rājadantadvayaṃ tatra}
		\rdg[wit={V1,Jyo}]{rājadantasthajihvāyā(ṃ)}% ṃ om. V1
		\rdg[wit={J10}]{rājadantasya jihvāyāṃ}}\marma}
\pada{\app{\lem[wit={N19,V15,C6},alt={jihvayottambhayed}]{jihvayottambhaye\skp{d}}
		\rdg[wit={Gr2,V3}]{jihvayottaṃbhaved}
		\rdg[wit={G11}]{jihvayoktaṃ bhajed}
		\rdg[wit={P11}]{jihvādaṃ staṃbhayed}
		\rdg[wit={V1}]{bandhaś ca staṃbhayed}
		\rdg[wit={J10,Jyo}]{bandhaḥ śasto bhaved}}%
	\app{\lem[wit={Gr2,G11,N19,V15,GrB,Jyo},alt={iti}]{\skm{d }iti}
		\rdg[wit={J10}]{dhitaḥ}
		\rdg[wit={V1}]{dhi tat}}//\versenr} 
	%	\NotIn{N3,J5,Gr3}
	\\!} %3.22; not in K1
\end{alttlg}
\altcommcite\newpage
\endaltnormal


\begin{tlg}[hp03_022]
\tl{\app{\lem[nolem]{}
	\rdg[wit={Gr1}]{\incl}}%
\pada{\app{\lem[wit={N3}]{amuṃ}
		\rdg[wit={J5}]{asaṃ}
		\rdg[wit={G4}]{ayaṃ}
		}
	\app{\lem[wit={J5}]{yogī}
		\rdg[wit={N3}]{yoga}
		\rdg[wit={G4}]{yogo}
		}
	mahā\app{\lem[wit={N3}]{bandhaṃ}
		\rdg[wit={J5}]{bandho}
		\rdg[wit={G4}]{bandh.}
		}}
\pada{mahāsiddhi% 
	\app{\lem[wit={N3}]{pradāyakam}
		\rdg[wit={J5}]{pradāyakaḥ} % +N24; dā om. V1, °yaka P11
		\rdg[wit={G4},alt={\lost}]{\skp{\lost}}
		}/}\\+}
\tl{
\pada{savyāṅge % sarvāṃge J5, °gai P11
	\app{\lem[wit={N3}]{ca samabhyasya}% °sye N3
		\rdg[wit={J5}]{tu samabhyasya}
		\rdg[wit={G4},alt={\lost}]{\skp{\lost}}}}
\pada{\app{\lem[wit={N3},alt={dakṣāṅge ca sam°}]{dakṣāṅge ca sama}% +G7,F
		\rdg[wit={J5}]{dakṣacāṃge sam°}
		\rdg[wit={G4}]{dakṣāṅge punar}
		}bhyaset//\versenr}\label{III22}
%		\sgwit{Gr1} 
		\\!}
\end{tlg}
\commcite%\newpage

\startaltnormal
\begin{alttlg}[hp03_022_1]
\tl{\app{\lem[nolem]{}
	\rdg[wit={Gr2,Gr3,G11,N19,V15,V1,J10,GrB,Jyo}]{\incl}}%
%\texteng{[Alt] }%
\pada{ayaṃ
	\app{\lem[wit={ceteri}]{khalu}
%		\rdg[wit={V3}]{ṣalu}
		\rdg[wit={V1,J10}]{kila}
		}
	mahā\app{\lem[wit={ceteri}]{bandho}
		\rdg[wit={J10}]{bandhaḥ}
		}}
\pada{\app{\lem[wit={ceteri}]{mahā}
		\rdg[wit={N23}]{sahā}
		\rdg[wit={J10}]{sarva}}siddhi% siddhiṃ P11
		pradāyakaḥ/} % dā om. V1, °yaka P11
		\\+}
\tl{
\pada{kāla\app{\lem[wit={ceteri}]{pāśa} % pāsa V3
		\rdg[wit={N23}]{pāśaṃ}}%
		mahā\app{\lem[wit={ceteri}]{bandha}
		\rdg[wit={N23}]{bandho}
		\rdg[wit={N19}]{baddho}}}%
\pada{\app{\lem[wit={ceteri}]{vimocana}
		\rdg[wit={V3}]{mocayec ca}}% 
	\app{\lem[wit={ceteri}]{vicakṣaṇaḥ}% +J5
		\rdg[wit={P11},alt={°ṇa}]{vicakṣaṇa}
		\rdg[wit={V3},alt={°ṇaṃ}]{vicakṣaṇam}
		\rdg[wit={G4}]{kṛtakṣayaḥ}}//\versenr}\label{kAlapAza}
\myfn{This verse is included in all manuscripts except \getsiglum{Gr1} as a substitute for 3.22.
\mydelim 
The second half is also found in \getsiglum{J5,G4}, most likely as a result of contamination. In \getsiglum{J5}, it is found between \ref{III22}ab and cd. In \getsiglum{G4}, it can be supposed from the surviving letters that the text of the original version was followed by that of the expanded version.}
%; it includes \ref{matam},  after which a half verse is lost. Then follow \ref{kAlapAza} and \ref{triveni}.
		\\!}  %3.22*1
\end{alttlg}
\altcommcite\newpage
\endaltnormal

\begin{tlg}[hp03_023]
\tl{
\pada{\app{\lem[nolem]{\skp{pāda a}}
	\rdg[wit={J5},alt={\om}]{\skp{\om}}}%
ayaṃ \app{\lem[wit={ceteri}]{ca}
		\rdg[wit={Gr2,Gr3,Jyo}]{tu}} sarvanāḍīnā}%
\pada{\app{\lem[nolem]{\skp{pāda b}}
	\rdg[wit={J5},alt={\om}]{\skp{\om}}}%
\app{\lem[wit={ceteri},alt={ūrdhvaṃ}]{\skm{m }ūrdhvaṃ}
		\rdg[wit={N3,V1,N23}]{ūrdhva}}% -ṃ ū- V1
	\app{\lem[wit={N3,V15,V1,J10,P11,V3}]{gativibodhakaḥ}% ḥ oṃ P11,V3
		\rdg[wit={N19}]{gatinibodhakāḥ}
		\rdg[wit={Jyo}]{gatinirodhakaḥ}
		\rdg[wit={G11}]{gativiśodhanaḥ}% °naṃ G5; °kaḥ +G7,M3
		\rdg[wit={Gr2,Gr3,C6}]{gamanabodhakaḥ}}/}
%	\lineom{ab}{J5}
	\\+} 
\tl{
\pada{triveṇīsaṃgamaṃ dhatte} % °veṇīṃ, dhartte N23; °veṇi V15; sagamaṃ J7
\pada{kedāraṃ
	\app{\lem[wit={ceteri},alt={prāpayen}]{prāpaye\skp{n}}% °ye<n> V1
		\rdg[wit={J5}]{prāpyate}
		}%
	\app{\lem[wit={ceteri},alt={manaḥ}]{\skm{n }manaḥ} % mana V3
		\rdg[wit={J5,G11,V1}]{naraḥ}
		\rdg[wit={N19}]{naraṃ}
		}\marma//\versenr}\label{triveni}%
	\myfn{\getsiglum{Jyo} has a different verse order: \ref{triveni}ab \rightarrow\ \ref{kAlapAza} \rightarrow\ \ref{triveni}cd.}\\!}  %3.23
\end{tlg}
\commcite\newpage

%%%%%%%%%%%%%%

% \begin{altava}[hp03_025]
% \app{\lem[wit={C7,N19,C6}]{atha} % °vedha N19,V3
		% \rdg[wit={Gr2,K3},alt={\om}]{\skp{\om}} 
		% mahāvedhaḥ}/
		% \sgwit{Gr2,K3,C7,N19,C6} %\NotIn{V19,N3,P11}
% \end{altava}

\begin{tlg}[hp03_024]
\tl{
\pada{rūpa\app{\lem[wit={ceteri}]{lāvaṇya}
		\rdg[wit={E2}]{lāvaṇa}
		\rdg[wit={G11}]{yauvana}}% lāvanya V19
	\app{\lem[wit={ceteri}]{saṃpannā}% sapannā J7
		\rdg[wit={J5}]{saṃpanū}
		\rdg[wit={N23}]{saṃpattī}
		\rdg[wit={V19}]{saṃyuktā}}}
\pada{yathā \app{\lem[wit={ceteri}]{strī puruṣaṃ}
		\rdg[wit={V19}]{nārī patiṃ}} vinā/}\\+}
\tl{
\pada{\app{\lem[nolem]{\skp{pāda c}}
	\rdg[wit={V15},alt={\om}]{\skp{\om}}}%
mahāmudrā%
	\app{\lem[wit={J7,Gr3,G11,C6,Jyo}]{mahābandhau}
		\rdg[wit={N3,J5,N23,N19,J10,P11,V3}]{mahābandho}
		\rdg[wit={V1}]{mahābandha}
		%\rdg[wit={V15},alt={\om}]{\skp{\om}}
		}}
\pada{\app{\lem[nolem]{\skp{pāda d}}
	\rdg[wit={V15},alt={\om}]{\skp{\om}}}%
\app{\lem[wit={J7,Gr3,G11,J10,C6,Jyo}]{niṣphalau}% niḥ° J7,V19,J10
		\rdg[wit={N23,P11}]{niṣphalo}% niḥphalo N23
		\rdg[wit={J5}]{niṣkalaḥ}
		\rdg[wit={N3}]{niṣkalā}
		\rdg[wit={N19}]{mahābaṃdha}
		\rdg[wit={V1,V3}]{mahāvedha}% +J14; °vedhaṃ J8,F
		%\rdg[wit={V15},alt={\om}]{\skp{\om}}
		}
	\app{\lem[wit={Gr2,Gr3,G11,C6,Jyo}]{vedhavarjitau} % vetha N23
		\rdg[wit={J5,P11}]{vedhavarjitaḥ}
		\rdg[wit={N3}]{vedhavarttina}
		\rdg[wit={J10}]{vedhavarttitau}
		\rdg[wit={N19,V1}]{vinā tathā}% +J14,F
		\rdg[wit={V3}]{vinānyathā}
		%\rdg[wit={V15},alt={\om}]{\skp{\om}}
		}//\versenr}\label{VuIII25} 
%		\lineom{cd}{V15}
		\\!}  %3.24
\end{tlg}
\commcite\newpage

% \begin{postmula}[hp03_025]
% iti mahābandhaḥ/ \sgwit{V1}
% \end{postmula}

\begin{ava}[hp03_025a]
\app{\lem[wit={V15,J10,V3,Jyo}]{atha mahāvedhaḥ}
		\rdg[wit={E2,N19,C6}]{\textapp{found before \ref{VuIII25}}}% °vedha N19,V3
		\rdg[wit={G11}]{atha vedhaḥ}
		\rdg[wit={Gr2}]{mahāvedhaḥ \textapp{(before \ref{VuIII25})}}
		\rdg[wit={V1}]{iti mahābandhaḥ}
		\rdg[wit={J5}]{atha mahābaṃdhaḥ \textapp{(after \ref{VuIII26}ab)}}
		\rdg[wit={N3,V19,P11},alt={\om}]{\skp{\om}}
	}/ 
%	\NotIn{N3,V19,P11}
%	\sgwit{V15,J10,V3,Jyo}
\end{ava}


\begin{tlg}[hp03_025]
\tl{
\pada{\app{\lem[wit={Gr1,G11,N19,V15,V1,Jyo}]{mahābandha}
		\rdg[wit={J7},post=\texteng{(followed by a double daṇḍa and corrected to °vedhaḥ)}]{mahābandhaḥ}
		\rdg[wit={N23,P11}]{mahābandho}% māhā° N23
		\rdg[wit={Gr3,C6}]{mahāvedhe}
		\rdg[wit={J10,V3}]{mahāvedha}}%
	\app{\lem[wit={ceteri}]{sthito}
		\rdg[wit={N23}]{sthite}
		\rdg[wit={J10}]{sthitau}} yogī}
\pada{kṛtvā pūraka%
	\app{\lem[wit={J7,G11,V15,V1,J10,C6,Jyo},alt={ekadhīḥ}]{\skm{m }ekadhīḥ}
		\rdg[wit={N3}]{ekadhī}
		\rdg[wit={P11}]{edhakī}
		\rdg[wit={V19,N19}]{ekadhā}% ekayā K3,C7
		\rdg[wit={G4}]{ekadhaḥ}
		\rdg[wit={N23}]{eva dhīḥ}
		\rdg[wit={V3}]{eva dhī}
		\rdg[wit={E2}]{eva vā}
		\rdg[wit={J5}]{eva ca dhā}}/}\\+}
\tl{
\pada{\app{\lem[wit={J7,Gr3,G11,V15,P11,Jyo}]{vāyūnāṃ}
		\rdg[wit={V1}]{vāyunāṃ}
		\rdg[wit={Gr1,N23,N19,J10,C6,V3}]{vāyunā}}
	\app{\lem[wit={ceteri}]{gatim āvṛtya}
		\rdg[wit={J5,N23,V15}]{gatim ākṛṣya}}} % gatin? V15
\pada{nibhṛtaṃ\marmas % ṃ om. N23
	kaṇṭha\app{\lem[wit={ceteri}]{mudrayā}
		\rdg[wit={J10}]{mudrāyā}
		}//\versenr}\label{VuIII26}\\!}  %3.26
\end{tlg}

\avacite{25a}
\commcite\newpage


\begin{tlg}[hp03_026]
\tl{
\pada{\app{\lem[wit={ceteri}]{samahasta}
		\rdg[wit={N3}]{samahāsta}
		\rdg[wit={N23}]{samahaste}
		\rdg[wit={J10}]{samahastā}
		\rdg[wit={C6}]{samau hasta}
		\rdg[wit={J5}]{nyastahasta}}%
	\app{\lem[wit={N23,V19,G11,J10,P11,V3,Jyo}]{yugo}
		\rdg[wit={J7,E2,V15,V1,C6}]{yugau}% +K3,C7
		\rdg[wit={N3,N19}]{yuge}
		\rdg[wit={J5}]{yuga}} 
		bhūmau% 
	\app{\lem[alt={\post bhūmau \add},nosep]{\skp{\post bhūmau \add}}
		\rdg[wit={G11}]{samapādayugas tataḥ |
		āndolanaṃ prakurvīta śarīrasya trimārgataḥ |
		punar āsphālanaṃ kaṭyāṃ mahāmerau ca sādhakaḥ |
		karau padadvaye kṛtvā}}}
\pada{\app{\lem[wit={ceteri}]{sphijau}
		\rdg[wit={N23,Jyo}]{sphicau}
		\rdg[wit={C6}]{sphītau}
		\rdg[wit={N19}]{dvijāt}
		\rdg[wit={V15}]{dvijā}}
	\app{\lem[wit={ceteri},alt={saṃtāḍayec}]{saṃtāḍaye\skp{c}} % saṃtāḍanec N23; °ḍaye V3,J10
		\rdg[wit={V1}]{saṃ[c/t]ālayec}% c ac, t pc?
		\rdg[wit={V15}]{nutāḍayec}
		\rdg[wit={J5}]{saṃjāyate}}%
	\app{\lem[wit={ceteri},alt={chanaiḥ}]{\skm{c }chanaiḥ}% ḥ om. P11,V3; channaiḥ J7, chūnaiḥ N19
		\rdg[wit={J5}]{tataḥ}}/}\\+} 
\tl{
\pada{\app{\lem[wit={ceteri}]{puṭadvayaṃ}
		\rdg[wit={J7}]{jaṃghādvayaṃ}% ac; puṭa pc
		\rdg[wit={N23}]{jaṃghāyuṭadvayam}}
	\app{\lem[wit={ceteri}]{samākramya}% +K3
		\rdg[wit={J5,J7}]{samākṛṣya}% +N24,C7
		\rdg[wit={N23}]{ākṛṣya}
		\rdg[wit={Jyo}]{atikramya}}}
\pada{\app{\lem[wit={J7,Gr3,V1,J10,C6,Jyo}]{vāyuḥ}
		\rdg[wit={N3,J5,N23,G11,N19,V15,P11,V3}]{vāyu}}
		sphurati % sphuraṃti V3
	\app{\lem[wit={N3,J5,G11,N19,J10}]{satvaram}% +M1,M3,G7
		\rdg[wit={P11}]{ratvaraṃ}
		\rdg[wit={V3}]{tatvaraṃ}
		\rdg[wit={V1}]{tatparaṃ}
		\rdg[wit={C6}]{tatparaḥ}
		\rdg[wit={Gr2,V19,V15,Jyo}]{madhyagaḥ}
		\rdg[wit={E2}]{madhyamaḥ}% +K3,C7
		}//\versenr}%
	\myfn{After this verse \getsiglum{N23,J7} have an additional line: \devnote{bandhenānena yogīndraḥ sādhayet sarvam īpsitam/} (=\,\emph{Śivasaṃhitā} 4.42ab in the section on \emph{mahābandha})}%
		\\!}  %3.27
\end{tlg}
\commcite\newpage


\begin{tlg}[hp03_027]
\tl{
\pada{somasūryāgni% some? N23
	\app{\lem[wit={N19,Jyo}]{saṃbandho}% +M1,F
		\rdg[wit={V1,V3}]{sambandhā}
		\rdg[wit={N3,J5,Gr2,G11,J10,P11,C6}]{sambandhāj}
		\rdg[wit={Gr3,V15}]{saṃdhānaṃ}}}\marmas
\pada{jāyate % jāyata? V15
	\app{\lem[wit={Gr1,G11,P11,Jyo}]{cāmṛtāya vai}% vau J5
		\rdg[wit={Gr2,N19,V15,C6}]{cāmṛtāyate}
		\rdg[wit={Gr3}]{vāmṛtāyate}
		\rdg[wit={V1}]{cāmṛtāye vaiḥ}
		\rdg[wit={V3}]{ca mṛtāya vai}
		\rdg[wit={J10}]{ca mṛturjayaḥ}}\marma/}\\+}
\tl{
\pada{\app{\lem[wit={ceteri}]{mṛtāvasthā}
		\rdg[wit={N23}]{mṛtāmasthā}}
	\app{\lem[wit={ceteri}]{samutpannā}
		\rdg[wit={G4}]{samunnaṃ<<ta>>t}
		\rdg[wit={N23},alt={\om}]{\skp{\om}}}}
\pada{tato 
	\app{\lem[wit={N3,J5,J7,Gr3,G11,N19,GrB}]{mṛtyubhayaṃ kutaḥ}
		\rdg[wit={V15,V1,J10,Jyo}]{vāyuṃ virecayet}% °cayat V15; +F
		\rdg[wit={G4}]{vāyuṃ\,+\,+\,+\,+}
		\rdg[wit={N23}]{vāyuṃ nirundhayet kumbhakena}}\marma%
	\app{\lem[alt={\post kutaḥ \add},nosep]{\skp{\post kutaḥ \add}}
		\rdg[wit={G11}]{āndoḷanāsphālanau ca | mahāmudrāmahābandhādyor api karttavyam iti saṃprasāyaḥ}}//\versenr}
		\label{somasurya}\\!}  %3.28
\end{tlg}
\commcite\newpage


\begin{tlg}[hp03_028]
\tl{
\pada{\app{\lem[wit={ceteri}]{mahāvedho}
		\rdg[wit={V15}]{mahābaṃdho}
		}'ya%m
	\app{\lem[wit={N3,J5,J7,E2,J10,C6,V3,Jyo},alt={abhyāsān}]{\skm{m }abhyāsā\skp{n}} % V19 °vedhopamanabhyā°?
		\rdg[wit={N23,P11}]{abhyāsāt}
		\rdg[wit={V19}]{anabhyāsān}
		\rdg[wit={V1}]{abhyāso}
		\rdg[wit={G11,N19,V15}]{abhyasto}% +F
		}}%
\pada{\app{\lem[wit={ceteri},alt={mahā}]{\skm{n }mahā}
		\rdg[wit={N23}]{sarva}}siddhi%
	\app{\lem[wit={ceteri}]{pradāyakaḥ} % siddhiḥ J10pc; dāyaka J5,V3
		\rdg[wit={G11}]{vidhāyakaḥ}}/}\\+}
\tl{
\pada{\app{\lem[wit={ceteri}]{valī}% vaḷī V15
		\rdg[wit={J5}]{valiḥ}
		\rdg[wit={N23,V1}]{vali}
		\rdg[wit={J10}]{valīta}
		}%
	\app{\lem[wit={ceteri}]{palita}
		\rdg[wit={J7}]{palīta}}%
	\app{\lem[wit={Jyo}]{vepa}
		\rdg[wit={N3,Gr2,G11,N19,V15,V1,GrB}]{vedha} % J7ac
		\rdg[wit={J5}]{vaidya}
		\rdg[wit={Gr3}]{vega} % corrupt from roga? C8
		\rdg[wit={J10}]{bandha} % J7pc
		}\marma%
	\app{\lem[wit={ceteri}]{ghnaḥ}
		\rdg[wit={N3,V3}]{ghnaṃ}
		\rdg[wit={J5,N23}]{ghna}}}
\pada{sevyate % savyate N19
	\app{\lem[wit={ceteri}]{sādhakottamaiḥ}
		\rdg[wit={V3}]{sādhakottamaṃ}}//\versenr}\\!}  %3.29
\end{tlg}
\commcite\newpage


\begin{tlg}[hp03_029]
\tl{
\pada{\app{\lem[wit={ceteri}]{etat trayaṃ mahā}% °traya N23; etatrayaṃ V3,V19; sahā J10
		\rdg[wit={V15,V1}]{mahāmudrātrayaṃ}% +F; trayatraṃ V1
		\rdg[wit={G4}]{mahavedhābhayaṃ}}%
	\app{\lem[wit={ceteri}]{guhyaṃ}
		\rdg[wit={Gr3}]{guptaṃ}
		\rdg[wit={J10}]{mudrā}}}
\pada{jarāmṛtyu\app{\lem[wit={ceteri}]{vināśanam}% mṛtyū N23; °sanaṃ V19
		\rdg[wit={J10}]{vināśinī}}/}\\+}
\tl{
\pada{\app{\lem[wit={ceteri}]{vahni}
	\rdg[wit={C6}]{buddhi}}vṛddhikaraṃ
	\app{\lem[wit={N3,Gr2,G11,C6}]{caiva}
		\rdg[wit={J5,N19,V15,V1,J10,P11,V3}]{caivam}
		\rdg[wit={Jyo}]{caiva hy}
		\rdg[wit={Gr3}]{caitad}}}
\pada{aṇimādi\app{\lem[wit={ceteri}]{guṇapradam}% animā° N23
		\rdg[wit={N19}]{gaṇapradaṃ}
		\rdg[wit={N23}]{guṇapradī}
		\rdg[wit={J5}]{pradāyakaṃ}}//\versenr}\\!}  %3.30
\end{tlg}
\commcite%\newpage


\begin{tlg}[hp03_030]
\tl{
\pada{\app{\lem[wit={ceteri}]{aṣṭadhā}
		\rdg[wit={C6}]{aṣṭādi}} kriyate
	\app{\lem[wit={N3,G11,C6},alt={caitad}]{caita\skp{d}}% caihad G7?
		\rdg[wit={Gr3,N19,Jyo}]{caiva}% +N24
		\rdg[wit={Gr2,P11}]{caivaṃ}% M3
		\rdg[wit={V1,J10,V3}]{caikaṃ}
		\rdg[wit={V15}]{caika}
		\rdg[wit={J5}]{taitva}}}%
\pada{\app{\lem[wit={ceteri},alt={yāme yāme}]{\skm{d }yāme yāme}
		\rdg[wit={V15}]{yāmayāme}
		\rdg[wit={V1}]{yāmaṃ yamāṃ}
		\rdg[wit={J5}]{yamair niyamai}} % +J14
		dine dine/}\\+}
\tl{
\pada{\app{\lem[wit={ceteri}]{puṇya}
		\rdg[wit={V15}]{puṇyaṃ}% +F
		\rdg[wit={J5}]{puna}
		\rdg[wit={J10}]{sarva}}%
	\app{\lem[wit={N3,J5,J7,Gr3,N19,Jyo}]{saṃbhāra}
		\rdg[wit={G4,G11}]{saṃcaya}
		\rdg[wit={V1,J10}]{saṃcāra}
		\rdg[wit={V15}]{saṃsāra}% +F
		\rdg[wit={P11}]{saṃdhāta}
		\rdg[wit={C6}]{saṃdhāna}
		\rdg[wit={V3}]{sahāra}
		\rdg[wit={N23},alt={\om}]{\skp{\om}}}%
	\app{\lem[wit={G4,Gr2,N19,V3}]{saṃbhāvi}
		\rdg[wit={N3,J5}]{saṃbhāvī}
		\rdg[wit={V1}]{sabhāvī}
		\rdg[wit={V15,C6,Jyo}]{saṃdhāyi}
		\rdg[wit={J10}]{saṃdhāyī}
		\rdg[wit={G11}]{saṃdāyī}
		\rdg[wit={P11}]{saṃdhīra}
		\rdg[wit={Gr3}]{saṃpādi}}}
\pada{\app{\lem[wit={ceteri}]{pāpaugha}% pāpocca J5
		\rdg[wit={J7}]{pāprogha}
		\rdg[wit={N23}]{padhau\,\_\,dhava}}%
	\app{\lem[wit={ceteri}]{bhiduraṃ sadā}% dū V3
		\rdg[wit={J5}]{bhidiraṃ sadā}
		\rdg[wit={G4}]{vidhuraṃ tathā}}//\versenr}\\!}  %3.31
\end{tlg}
\commcite\newpage


\begin{tlg}[hp03_031]
\tl{
\pada{samyak\app{\lem[wit={ceteri},alt={śikṣāvatām}]{śikṣāvatā\skp{m}} % sikṣī V19; °catām N23
		\rdg[wit={C6}]{śikṣavatā}
		\rdg[wit={J5,N19}]{śiṣyāvatām}
		\rdg[wit={J10}]{jijñāsatām}}%
	\app{\lem[wit={J5,Gr2,Gr3,V15,P11},alt={eva}]{\skm{m }eva} % M3?
		\rdg[wit={N3,G11,N19,V1,J10,V3,Jyo}]{evaṃ}
		\rdg[wit={C6}]{bhavyaṃ}}}
\pada{svalpaṃ % svaplaṃ V1; ṃ om. E2
	prathama\app{\lem[wit={J5,N23,Gr3,N19,V15,V1,P11,C6}]{sādhane}
		\rdg[wit={N3}]{sādhanaiḥ}
		\rdg[wit={J7,G11,J10,V3,Jyo}]{sādhanaṃ}}/}\\+} % pratyama N23; sādhana? J10ac
\tl{
\pada{\app{\lem[nolem]{\skp{pāda c}}
	\rdg[wit={Jyo},alt={\om},post=\texteng{(cf. note on the prescript to 1.61)}]{\skp{\om\ (cf. note on the prescript to 1.61)}}}%
vahnistrīpatha% vahna J5; stri N19; paṭha N3
	\app{\lem[wit={ceteri},alt={sevānām}]{sevānā\skp{m}}
		\rdg[wit={N19}]{sevācanām}
		\rdg[wit={J10}]{sevanām}
		\rdg[wit={V1}]{sevanam}
		\rdg[wit={N23}]{sevenam}
		%\rdg[wit={Jyo},alt={\om}]{\skp{\om}}
		}}% °nāṃmādau J7
\pada{\app{\lem[nolem]{\skp{pāda d}}
	\rdg[wit={Jyo},alt={\om}]{\skp{\om}}}%
m ādau varjana%
	\app{\lem[wit={N3,J5,N19,V15,V1,P11,C6},alt={ādiśet}]{\skm{m }ādiśet}
		\rdg[wit={V3}]{ādṛśyet}
		\rdg[wit={Gr2,Gr3,G11,J10}]{ācaret}% +M3,G7,F
		%\rdg[wit={Jyo},alt={\om}]{\skp{\om}}
		}//\versenr}%
	\myfn{%
	After this verse, \getsiglum{Gr2} have three additional lines:
	\devnote{mahāmudrā mahābandho mahāvedhaś ca nityaśaḥ/ % °bandhā N23
	etat trayaṃ prayatnena caturvāraṃ karoti yaḥ/
	ṣaṇmāsābhyantare mṛtyuṃ jayaty eva na saṃśayaḥ//} % °bhyāṃtare, mṛtyu N23
	 (cf. \textit{Śivasaṃhitā} 4.48)}
%	\lineom{cd}{Jyo}
%	\anm{cd after \manuref{1.60} \getsiglum{Jyo}}
	\\!}  %3.32
\end{tlg}
\commcite\newpage

%%%%%%%%%%%%%%%%%%%%%%%%

\begin{ava}[hp03_032a]
\app{\lem[wit={ceteri}]{atha}
\rdg[wit={Gr2,E2},alt={\om}]{\skp{\om}}} khecarī/ % ṣecarī J5
\end{ava}

\teimute{\vspace{-0.5ex}}
\avacite{32a}
\bigskip

\startaltrecension
\begin{alttlg}[hp03_031_1] % MD Numbering problem
\tl{\app{\lem[nolem]{}
	\rdg[wit={Gr6}]{\incl}}%
\pada{nāsanaṃ siddhasadṛśaṃ} % proof that J8 is a copy of V3? Halanta of k in the above line was mistaken for e?(V3=9v4, J8=16r6)
\pada{na \app{\lem[wit={E4pc}]{kumbhaṃ}
		\rdg[wit={E4ac,V3}]{kumbha}
		\rdg[wit={J6}]{kuṃbhaka}}
	\app{\lem[wit={E4pc}]{kevalopamam} % =J15
		\rdg[wit={E4ac,V3}]{kevalokanam}
		\rdg[wit={J6}]{samonilaṃ}}/}\\+}
\tl{
\pada{na khecarīsamā mudrā} \pada{na nādasadṛśo layaḥ//\versenr}
\anm{=\,\manuref{1.43}}\\!}
\end{alttlg}
%\altcommcite\newpage
\endaltrecension
\teimute{
	\vspace{0pt minus 3pt}
	\begin{quote}%
	\textcolor{gray}{\ExecuteMetaData[HP1_comm.tex]{tr43}
	\texteng{(31*1)}}
	\end{quote}
	}
\newpage

%\teimute{\medskip}
\begin{tlg}[hp03_032]
\tl{
\pada{\app{\lem[wit={J5,N23,Gr3,J6,G11,V15,V1,P11,V3,Jyo}]{chedana}% chedata P11
		\rdg[wit={J10,E4}]{chedanaṃ}% +J8
		\rdg[wit={C6}]{chedanaiś}
		\rdg[wit={G4}]{bhedana}
		\rdg[wit={N19}]{vedana}
		\rdg[wit={J7}]{rasanā}
		\rdg[wit={N3},alt={\illeg}]{\skp{\illeg}}}%
	\app{\lem[wit={J7,Gr3,J6,G11,V15,V1,P11,Jyo}]{cālanadohaiḥ}% dohai P11
		\rdg[wit={N23}]{cālajadohaiḥ}
		\rdg[wit={J5}]{cālanaṃ dohaiḥ}
		\rdg[wit={J10,E4,V3}]{cālanaṃ dohau}
		\rdg[wit={C6}]{cālanair dāsyai}
		\rdg[wit={N19}]{cāladohaiḥ}
		\rdg[wit={G4}]{pādanadoṣaiḥ}
		\rdg[wit={N3},alt={\illeg}]{\skp{\illeg}}}
\app{\lem[wit={J7,G11,N19,V1,P11,C6,Jyo}]{kalāṃ}% kalā vardhayet kramaśaḥ G4
		\rdg[wit={N23}]{kalan}
		\rdg[wit={N3}]{kalāḥ}% +N26
		\rdg[wit={J5,G4,E4,V3}]{kalā}
		\rdg[wit={J10}]{kāla}
		\rdg[wit={Gr3,J6}]{jihvāṃ} % jihva V19
		\rdg[wit={V15}]{krameṇa}}
	\app{\lem[wit={N3,G11,N19,V1,J10,E4,GrB}]{krameṇa}
		\rdg[wit={J5}]{kramaṇa}
		\rdg[wit={Jyo}]{krameṇātha}
		\rdg[wit={V15}]{jihvāṃ}
		\rdg[wit={Gr2}]{tu}
		\rdg[wit={Gr3}]{vai}
		\rdg[wit={G4,J6},alt={\om}]{\skp{\om}}}
	\app{\lem[wit={J5,G11,V15,V1,J10,P11},alt={pravardhayet}]{pravardhaye\skp{t}}
		\rdg[wit={Gr2,J6}]{saṃvardhayet}% °ye N23, °me P11
		\rdg[wit={N3,G4,Gr3,N19,E4,C6,V3,Jyo}]{vardhayet}}%
	\app{\lem[wit={J5,J7,Gr3,J6,G11,N19,V15,P11,C6,Jyo},alt={tāvat}]{\skm{t }tāvat}
		\rdg[wit={J10}]{tā<<va>>t}
		\rdg[wit={G4}]{kramaśaḥ}
		\rdg[wit={N3,N23,V1,E4,V3},alt={\om}]{\skp{\om}}}/}\\+}
\tl{
\pada{\app{\lem[wit={N3,J5,G11,V15,V1,J10,P11,C6,Jyo},alt={sā yāvad}]{sā yāva\skp{d}}% mā yāvad P11; yāva N3, pāvad J10
		\rdg[wit={Gr2,Gr3,J6}]{yāvad iyaṃ}% yāvaṃd iya N23
		\rdg[wit={E4,V3}]{yāvad}
		\rdg[wit={G4}]{yā}
		\rdg[wit={N19}]{sā}
		}%
	\app{\lem[wit={ceteri},alt={bhrūmadhyaṃ}]{\skm{d }bhrūmadhyaṃ} % bhū C6,N23, bhṛ N3
		\rdg[wit={V19,V1}]{bhrūmadhya}}
	\app{\lem[wit={cetwG4}]{spṛśati}
		\rdg[wit={N23}]{sparśati}
		\rdg[wit={J5}]{visati}
		\rdg[wit={N3}]{viśa}}
\app{\lem[wit={Gr2,E2,J6,V15,J10,Jyo}]{tadā khecarīsiddhiḥ}% +K3,C7
		\rdg[wit={N3,J5,E4,GrB},post=\texteng{(tadānī \getsiglum{J5,P11,C6})}]{tadānīṃ khecarīsiddhiḥ}% °nī + siddhi P11,C6
		\rdg[wit={G11,N19}]{tadānīṃ hi khecarīsiddhiḥ}% khecarīṃ N19
		\rdg[wit={V1}]{tadānī siddhiḥ}
		\rdg[wit={V19}]{tadā khecarī bhavati}
		}//\versenr}%\myfn{The metre is Āryā.\\
%	\getsiglum{Gr2,V19,K3,C7} (Upagīti): 
%		\devnote{chedanaśalanadohaiḥ kalāṃ tu saṃvardhayet tāvat/
%		yāvad iyaṃ bhrūmadhyaṃ spṛśati tadā khecarīsiddhiḥ//\versenr}\\
%	\getsiglum{G11,N19,V15} (Gīti?):
%		\devnote{chedanacālanadohaiḥ kalāṃ krameṇa (pra)vardhayet tāvat/
%		sā yāvad bhrūmadhyaṃ spṛśati tadānīṃ hi khecarīsiddhiḥ//\versenr}\\
	% \getsiglum{J10} (Āryā):
		% \devnote{chedanacālanadohaiḥ kalāṃ krameṇa pravardhayet tāvat/
		% sā yāvad bhrūmadhyaṃ spṛśati tadā khecarīsiddhiḥ//\versenr}\\
%	\getsiglum{V3} (Anuṣṭubh):
%		\devnote{chedanaṃ cālanaṃ dohau kalākrameṇa vardhayet/
%		yāvad bhrūmadhyaṃ spṛśati tadānīṃ khecarīsiddhiḥ(!)//\versenr}\\
	% Perhaps to read:
	% \devnote{chedanacālanadohaiḥ kalāṃ kramād vardhayet tāvat/
	% sā yāvad bhrūmadhyaṃ spṛśati tadā khecarīsiddhiḥ//\versenr}
%	}
	\\!}  %3.33
\end{tlg}
\commcite\newpage


% Khecaryabhyāsakrama

\startaltrecension

\begin{alttlg}[hp03_032_01] % KhV 1.46
\tl{\app{\lem[nolem]{}
	\rdg[wit={V15,Gr6,Jyo}]{\incl}}%
\pada{\app{\lem[wit={J6,Jyo}]{snuhī}
		\rdg[wit={V15,E4}]{snuhi}
		\rdg[wit={V3}]{śnuhi}}pattranibhaṃ śastraṃ} % putra J6ac
\pada{sutīkṣṇaṃ snigdhanirmalam/}\\+}
\tl{
\pada{samādāya tatas tena}
\pada{romamātraṃ \app{\lem[wit={V3}]{samucchidet} % cci J15
		\rdg[wit={V15,E4,Jyo}]{samucchinet}% +N9
		\rdg[wit={J6}]{samucchiṃdyāt}}//\versenr}
		%\note[type=anmkg,nonum,lem=\mylem{32*1--3},labelb={3-32-1}]{included in \getsiglum{V15,Gr6,Jyo}}
		\sgwit{V15,Gr6,Jyo}
		\\!}
\end{alttlg}
\altcommcite%\newpage

\begin{alttlg}[hp03_032_2] % KhV 1.47
\tl{\app{\lem[nolem]{}
	\rdg[wit={V15,Gr6,Jyo}]{\incl}}%
\pada{\app{\lem[wit={V15,Gr6}]{kṛtvā}
		\rdg[wit={Jyo}]{tataḥ}}
	saindhava\app{\lem[wit={Jyo}]{pathyābhyāṃ}
		\rdg[wit={J6,E4,V3}]{pathyādi}
		\rdg[wit={V15}]{pakṣyādi}}}
\pada{cūrṇitābhyāṃ pragharṣayet/}\\+} % cūṇi° V15
\tl{punaḥ saptadine prāpte
\pada{romamātraṃ \app{\lem[wit={E4,V3}]{samucchidet}% +N9,J15
		\rdg[wit={Jyo}]{samucchinet}
		\rdg[wit={V15}]{punaḥ chidet}
		\rdg[wit={J6}]{samutthiyāt}}//\versenr} 
		\sgwit{V15,Gr6,Jyo}
		\\!}
\end{alttlg}
\altcommcite\newpage

\begin{alttlg}[hp03_032_3] % KhV 1.48
\tl{\app{\lem[nolem]{}
	\rdg[wit={V15,Gr6,Jyo}]{\incl}}%
\pada{evaṃ krameṇa \app{\lem[wit={J6,E4,V3,Jyo}]{ṣaṇmāsaṃ}
		\rdg[wit={V15}]{ṣaṇmāsān}}}
\pada{\app{\lem[wit={J6,V15,E4,V3}]{nitya}
		\rdg[wit={Jyo}]{nityaṃ}}% +J15
	\app{\lem[wit={V15,E4,V3}]{yuktaṃ}
		\rdg[wit={Jyo}]{yuktaḥ}
		\rdg[wit={J6}]{muktaṃ}} samācaret/}\\+}
\tl{
\pada{\app{\lem[wit={Gr6,Jyo},alt={ṣaṇmāsād}]{ṣaṇmāsā\skp{d}}
		\rdg[wit={V15}]{ṣaṇmāse}}d
	rasanā\app{\lem[wit={J6,E4,V3,Jyo}]{mūla}
		\rdg[wit={V15}]{mūlaṃ}}}%
\pada{\app{\lem[wit={E4pc}]{śirābandhaṃ}
		\rdg[wit={V15,Jyo}]{śirābandhaḥ}
		\rdg[wit={E4ac,V3}]{śarābandhaṃ}
		\rdg[wit={J6}]{śarabaṃdhaṃ}
		}
	\app{\lem[wit={Gr6}]{vinaśyati}
		\rdg[wit={V15,Jyo}]{praṇaśyati}}//\versenr} 
		\sgwit{V15,Gr6,Jyo}
		\\!}
\end{alttlg}
\altcommcite%\newpage
%\endaltrecension

%\Anm{16 more verses in \getsiglum{Gr6}}

\begin{alttlg}[hp03_032_4] % KhV 1.49
\tl{\app{\lem[nolem]{}
	\rdg[wit={Gr6}]{\incl}}%
\pada{atha vāgīśvarīdhāma}%
\pada{śiro vastreṇa veṣṭayet/}\\+}
\tl{
\pada{śanair utkarṣayed yogī}
\pada{kālavelāvidhānavit//\versenr} 
	\sgwit{Gr6}
	\\!}
\end{alttlg}
\altcommcite\newpage


\begin{alttlg}[hp03_032_5] % from long recension of the Yogabīja
\tl{\app{\lem[nolem]{}
	\rdg[wit={Gr6}]{\incl}}%
\pada{vitasti\app{\lem[wit={V3,J6}]{pramitaṃ}
		\rdg[wit={E4}]{pratimaṃ}}
	\app{\lem[wit={E4,V3}]{dairghyaṃ}
		\rdg[wit={J6pc}]{dairghye}
		\rdg[wit={J6ac}]{dairghya}}}
\pada{vistāraṃ
	\app{\lem[wit={V3,J6}]{caturaṅgulam}
		\rdg[wit={E4}]{caturāṅgulam}}/}\\+}
\tl{
\pada{mṛdulaṃ dhavalaṃ proktaṃ}
\pada{veṣṭitāmbaralakṣaṇam//\versenr}
	\sgwit{Gr6}
	\anm{=\,\manuref{3.96*2}}\label{III33-5}\\!}
\end{alttlg}
\altcommcite%\newpage

\begin{alttlg}[hp03_032_6] % KhV 1.50
\tl{\app{\lem[nolem]{}
	\rdg[wit={Gr6}]{\incl}}%
\pada{punaḥ ṣaṇmāsamātreṇa}
\pada{\app{\lem[resp=emend]{nitya}
	\rdg[wit={E4,V3,J6}]{punaḥ}}saṃkarṣaṇāt priye/}\\+}
\tl{
\pada{bhrūmadhyāvadhi vardheta}
\pada{tiryakkarṇa% tiryakarṇa V3,E4
	bilāvadhi//\versenr}
	\sgwit{Gr6}
	\\!} % °vadhiḥ E4
\end{alttlg}
\altcommcite\newpage


\begin{alttlg}[hp03_032_7] % KhV 1.51ab, 1.52ab
\tl{\app{\lem[nolem]{}
	\rdg[wit={Gr6}]{\incl}}%
\pada{adhastā\app{\lem[wit={J6},alt={cibukaṃ mūlaṃ}]{\skm{c }cibukaṃ mūlaṃ} % cf. J3
		\rdg[wit={E4}]{cibukaṃ mūla}
		\rdg[wit={V3}]{cibukamūla}}}
\pada{prayāti kramakāritā/}\\+}
\tl{
\pada{\app{\lem[resp=emend]{keśāntam ūrdhvaṃ}
		\rdg[wit={E4,V3,J6}]{krośād ūrdhvaṃ ca}} % °rdhaṃ E4
	\app{\lem[wit={E4,V3}]{kramati}
		\rdg[wit={J6}]{krāmati}}}
\pada{tiryak\app{\lem[resp=emend]{śaṅkhāvadhi}
		\rdg[wit={E4,V3,J6}]{saṃkhyāvadhi}} priye//\versenr}
	\sgwit{Gr6}
	\\!}
\end{alttlg}
\altcommcite%\newpage

\begin{alttlg}[hp03_032_8] % = HP6 and HP10
\tl{\app{\lem[nolem]{}
	\rdg[wit={Gr6}]{\incl}}%
\pada{punaḥ saṃvatsarād devi}
\pada{\app{\lem[resp=emend,alt={dvitīyāc}]{dvitīyā\skp{c}}
	\rdg[wit={E4,V3,J6}]{dvitīyā}}c 
	caiva līlayā/}\\+} % dvitiyā V3,J15; lilayā N9,J15
\tl{
\pada{brahmarandhrāntam āvṛtya} % ābṛttya V3, āvṛttya E4
\pada{tiṣṭhet paramavandite//\versenr} 
\sgwit{Gr6}
\\!}
\end{alttlg}
\altcommcite\newpage


\begin{alttlg}[hp03_032_9]% = HP6 and HP10
\tl{\app{\lem[nolem]{}
	\rdg[wit={Gr6}]{\incl}}%
\pada{svatālumūlaṃ saṃghṛṣya}
\pada{saptavāsara%m 
	\app{\lem[resp=emend,alt={ātmavit}]{\skm{m }ātmavit}
	\rdg[wit={E4,V3,J6}]{ātmani}}/}\\+}
\tl{
\pada{svagurūktaprakāreṇa}
\pada{malaṃ sarvaṃ viśoṣayet//\versenr} % viśeṣayet J8
\sgwit{Gr6}
\\!}
\end{alttlg}
\altcommcite%\newpage

\begin{alttlg}[hp03_032_10]% = HP10 (33.10ab = KhV 1.56cd)
\tl{\app{\lem[nolem]{}
	\rdg[wit={Gr6}]{\incl}}%
\pada{aṅgulyagreṇa saṃghṛṣya}
\pada{jihvāṃ tatra niveśayet/}\\+}
\tl{
\pada{śanaiḥ śanair mastakāc ca}
\pada{mahāvajrakapāṭabhit//\versenr} 
\sgwit{Gr6}
\\!} % bhīt J15
\end{alttlg}
\altcommcite\newpage

\begin{alttlg}[hp03_032_11]% = KhV 1.34
\tl{\app{\lem[nolem]{}
	\rdg[wit={Gr6}]{\incl}}%
\pada{pūrva\app{\lem[wit={E4}]{bīja}% +N26,V17,P1
		\rdg[wit={V3}]{vīya}
		\rdg[wit={J6}]{vīrya}}yutāṃ vidyāṃ}
\pada{vyākhyātām atidurlabhām/}\\+}
\tl{
\pada{asyāḥ ṣaḍaṅgaṃ kurvīta}
\pada{tayā ṣaṭcakrabhinnayā//\versenr} 
\sgwit{Gr6}
\\!}
\end{alttlg}
\altcommcite%\newpage

% metre: Rathoddhatā
\begin{alttlg}[hp03_032_12]% = HP6 and HP10
\tl{\app{\lem[nolem]{}
	\rdg[wit={Gr6}]{\incl}}%
\pada{khe nirastasakalakriyākrame}\\+}
\tl{
\pada{\app{\lem[resp=emend,postwit=\texteng{(cf.\,Yoginīhṛdaya)}]{yā citiś carati} % J15 yojitaś, many other mss °ṇa cittaś (unmetrical)
		\rdg[wit={E4}]{yā citaś carati}% +V17; cittam ācarati N26
		\rdg[wit={V3}]{yā cittaś carati}% +P1
		\rdg[wit={J6}]{°ṇa cittaś carati}
		} śāśvatodaye/}\\+}
\tl{
\pada{sā śivatva\app{\lem[wit={E4,V3}]{samavāya}
		\rdg[wit={J6}]{samavāyi}}%
	\app{\lem[wit={E4,J6}]{kāriṇī}
		\rdg[wit={V3}]{kariṇī}}}\\+}
\tl{
\pada{khecarī
	\app{\lem[wit={E4,V3}]{ca bhava}
		\rdg[wit={J6}]{\{\{ca\}\} bhavati}
		}khedahāriṇī//\versenr} 
		\sgwit{Gr6}
		\\!}
\end{alttlg}
\altcommcite\newpage

\begin{alttlg}[hp03_032_13]% = HP6 and HP10, ~ KhV 1.54
\tl{\app{\lem[nolem]{}
	\rdg[wit={Gr6}]{\incl}}%
\pada{krameṇaiva
	\app{\lem[wit={E4,V3}]{prakartavyā}
		\rdg[wit={J6}]{pravartavyā}}}%
\pada{bhyāsena
	\app{\lem[wit={E4,V3}]{varavarṇini}
		\rdg[wit={J6}]{paravarṇinī}}/}\\+}
\tl{
\pada{yugapa%
	\app{\lem[wit={E4,V3,J6pc},alt={yatate}]{\skm{d }yatate}
		\rdg[wit={J6ac}]{yatete}
		} tasya}
\pada{śarīraṃ vilayaṃ vrajet//\versenr} 
\sgwit{Gr6}
\\!}
\end{alttlg}
\altcommcite%\newpage

\begin{alttlg}[hp03_032_14]% = HP6 and HP10, KhV 1.55ab
\tl{\app{\lem[nolem]{}
	\rdg[wit={Gr6}]{\incl}}%
\pada{tasmāc chanaiḥ śanaiḥ kāryo}%
\pada{'bhyāso na yugapat priye/}\\+}
\tl{
\pada{\app{\lem[wit={J6}]{evaṃ} % +N9
		\rdg[wit={E4,V3}]{eva}} varṣatrayaṃ kṛtvā}
\pada{brahmadvāraṃ
	\app{\lem[wit={E4,J6},alt={viśed}]{viśe\skp{d}}
		\rdg[wit={V3}]{biśe}}d dhruvam//\versenr} 
		\sgwit{Gr6}
		\\!}
\end{alttlg}
\altcommcite\newpage

% metre: Śārdūlavikrīḍita
\begin{alttlg}[hp03_032_15]% = HP6 and HP10
\tl{\app{\lem[nolem]{}
	\rdg[wit={Gr6}]{\incl}}%
\pada{ṣaṭcakrāṇi vibhidya
śakti\app{\lem[wit={E4pc,J6}]{bhujagīṃ}
		\rdg[wit={E4ac,V3}]{bhujaṃgī}
		}
	\app{\lem[wit={E4,V3}]{protthāpya}
		\rdg[wit={J6}]{protthāya}
		} mūlasthitāṃ}\\+}
\tl{
\pada{bhittvā granthitrayaṃ ca % tra of trayaṃ treated as a single consonant?
paścimaśirāprākārarūpaṃ mahat/}\\+}
\tl{
\pada{nītvā prāṇam ataḥ śirobilam alaṃ % anaḥ? J6
	nirmathya cittena tat}\\+} % tal J6,N9
\tl{
\pada{\app{\lem[wit={J6}]{liṅgaṃ}
		\rdg[wit={E4,V3}]{taliṅgaṃ}} yaḥ
\app{\lem[wit={J6}]{pibatī}
		\rdg[wit={E4,V3}]{pibate}}ndumaṇḍalagala% galat* V3,E4,J15
	\app{\lem[wit={E4,V3},alt={muktaḥ sa sākṣācchivaḥ}]{\skm{n }muktaḥ sa sākṣācchivaḥ}
		\rdg[wit={J6}]{muktaś ca sākṣācchivaḥ}}//\versenr} 
		\sgwit{Gr6}
		\\!}
\end{alttlg}
\altcommcite\newpage

% metre: Sragdharā
\begin{alttlg}[hp03_032_16]% = HP6 and HP10
\tl{\app{\lem[nolem]{}
	\rdg[wit={Gr6}]{\incl}}%
\pada{nityaṃ
	\app{\lem[wit={E4,V3}]{yas tūrdhva}
		\rdg[wit={J6}]{yasphūrja}}%
	\app{\lem[wit={E4}]{jihvo yadi}% +N26,V17
		\rdg[wit={V3}]{jihvogradi}% jihvāgrādi P1
		\rdg[wit={J6}]{jihvāgrayā}}
pibati pumān
saptadhārāmṛ\app{\lem[wit={E4,V3,J6pc}]{taughaṃ}
		\rdg[wit={J6ac}]{tauccaṃ}}}\\+}
\tl{
\pada{\app{\lem[wit={E4,V3}]{susvādaṃ}
		\rdg[wit={J6}]{su[kha]daṃ}} śītalāṅgaṃ
	duritabhayaharaṃ kṣutpipāsānivāri/}\\+} % nivāritaṃ J6ac
\tl{
\pada{piṇḍasthairyaṃ hi tasmād bhavati
	\app{\lem[resp=emend]{mṛtapathā} % +N26, mṛttapathā N9
		\rdg[wit={E4,V3}]{mṛtayathā}
		\rdg[wit={J6}]{mṛtaṃ yathā}}
	mṛtyurogā%
	\app{\lem[wit={E4,V3},alt={bhavanti}]{\skm{d }bhavanti}
		\rdg[wit={J6}]{bhavati}
		}}\\+}
\tl{
\pada{\app{\lem[wit={J6}]{daurbhāgyaṃ}
		\rdg[wit={E4,V3}]{daurbhyāgyaṃ}}
yāti nāśaṃ prasarati sakalaṃ yāti % prasar{{i}}ti E4
	\app{\lem[wit={E4,V3}]{kālaṃ}
		\rdg[wit={J6}]{kālo}} bhramitvā//\versenr} 
		\sgwit{Gr6}
		\\!}
\end{alttlg}
\altcommcite\newpage

\begin{alttlg}[hp03_032_17]% = HP6 and HP10
\tl{\app{\lem[nolem]{}
	\rdg[wit={Gr6}]{\incl}}%
\pada{\app{\lem[wit={J6}]{tīkṣṇakaṃ}
		\rdg[wit={E4,V3}]{tīkṣṇake}}
	harate vyādhiṃ} % vyādhiḥ E4
\pada{kaṭukaṃ kuṣṭhanāśanam/}\\+} % u of kaṭuka resembles to a virāma in V3,N9
\tl{
\pada{ghṛtasvādūpamaṃ caiva}   % caivāmaratvaṃ J6
\pada{amaratvaṃ 
	\app{\lem[wit={E4,V3},alt={labhed}]{labhe\skp{d}} % labhe V3
		\rdg[wit={J6}]{labhate}}d dhruvam//\versenr} % +N9
		\sgwit{Gr6}
		\\!}
\end{alttlg}
\altcommcite%\newpage

\begin{alttlg}[hp03_032_18]% = HP6
\tl{\app{\lem[nolem]{}
	\rdg[wit={Gr6}]{\incl}}%
\pada{madhusvādūpamaṃ caiva}
\pada{śāstra%m
	\app{\lem[wit={E4,V3},alt={udgirate}]{\skm{m }udgirate}
		\rdg[wit={J6}]{udgirati}}
	\app{\lem[wit={E4,J6}]{bahu}
		\rdg[wit={V3}]{bahuḥ}}/}\\+}
\tl{
\pada{\crux\app{\lem[resp=emend]{laḍḍu}% +N26
		\rdg[wit={E4,V3,J6}]{laḍu}}%
	\app{\lem[wit={E4,V3}]{ṣaṇḍaka}
		\rdg[wit={J6}]{khaṃḍaka}}%
	\app{\lem[resp=emend]{khādyāni}
		\rdg[wit={E4,V3}]{pādyāni}
		\rdg[wit={J6}]{pādyāni}}}
\pada{\app{\lem[wit={E4,V3}]{pakvānnāni}
		\rdg[wit={J6}]{pakvānnāny apy}} 
		anekaśaḥ\crux//\versenr} 
		\sgwit{Gr6}
		\\!}
\end{alttlg}
\altcommcite\newpage

\begin{alttlg}[hp03_032_19]% = HP6
\tl{\app{\lem[nolem]{}
	\rdg[wit={Gr6}]{\incl}}%
\pada{divyakalpaṃ
	\app{\lem[wit={E4,V3},alt={ramen}]{rame\skp{n}}
		\rdg[wit={J6}]{krīḍen}}n nityam} % rame J8
\pada{utkṛṣṭo jāyate dhruvam/}\\+}
\tl{
\pada{tanmayatvam avāpnoti}
\pada{\app{\lem[wit={J6}]{kośakārīva}% +V17; koṣa N26
		\rdg[wit={V3}]{kauśakārīva}
		\rdg[wit={E4}]{kauṣṭakārīva}% +P10; koṣṭa J15, kāṣṭa P1
		} kīṭakaḥ//\versenr} 
		\sgwit{Gr6}
		\\!}
\end{alttlg}
\endaltrecension
\altcommcite%\newpage


\begin{tlg}[hp03_033]
\tl{\app{\lem[nolem]{}
	\rdg[wit={Jyo},alt=\textapp{found before 3.32}]{\skp{found before 3.32}}}%
\pada{kapāla\app{\lem[wit={ceteri}]{kuhare} % kalāpa C6
		\rdg[wit={J5,P11}]{vivare}} jihvā} % jvihvā N23
\pada{\app{\lem[wit={ceteri}]{praviṣṭā viparītagā}
		\rdg[wit={N3}]{pravi\,+\,+\,+\,+\,+\,+}
		\rdg[wit={C6}]{praviṣṭā viṣa<<taṃ>>tugā}}/}\\+}
\tl{
\pada{bhruvo%r  % bhṛ° N3, bhu° P11, bhrū° V3, bhrūvaur N23
	\app{\lem[wit={ceteri},alt={antargatā}]{\skm{r }antargatā}
		\rdg[wit={N3}]{aṃtagatā}
		\rdg[wit={P11}]{madhagatā}
		\rdg[wit={C6}]{madhye gatā}}
	\app{\lem[wit={J7,V19,E2,N19,V1,J10,C6,V3,Jyo},alt={dṛṣṭir}]{dṛṣṭi\skp{r}}
		\rdg[wit={N3,J5,N23,G11,V15,P11}]{dṛṣṭi}}r}
\pada{mudrā \app{\lem[wit={ceteri}]{bhavati}
		\rdg[wit={G4}]{bhavatu}% +F
		}
	khecarī//\versenr} % ṣecarī J5, <khe>carī V1
%	\anm{before \manuref{3.32} \getsiglum{Jyo}}
	\\!}  %3.34
\end{tlg}
\commcite\newpage


\begin{tlg}[hp03_034]% J5,G4,N23,J7,V1,J10
\tl{\app{\lem[nolem]{}
		\rdg[wit={J5,G4,N23,J7,V1,J10}]{\incl}
		\rdg[wit={N3}]{\om}}%
\pada{\app{\lem[wit={J5,J7,V1}]{kalāṃ}
		\rdg[wit={G4,N23}]{kalā}
		\rdg[wit={J10}]{kālaṃ}} % +J6 sonst wie lemmata
	\app{\lem[wit={V1}]{parāṅmukhīṃ}
		\rdg[wit={G4,Gr2,J10}]{parāṅmukhī}
		\rdg[wit={J5}]{paṅmukhī}
%		\rdg[wit={N24}]{tu prāṅmukhī}
		}
	\app{\lem[wit={G4,V1}]{kṛtvā}
		\rdg[wit={J10}]{kṛtya}
		\rdg[wit={J5,Gr2}]{nītvā}% +N24
		}}
\pada{kṣaṇārdhaṃ % ṃ om. P11,N23
	\app{\lem[wit={G4,Gr2,V1}]{yadi}% padi P11
		\rdg[wit={J5,J10}]{api}
%		\rdg[wit={N24}]{di}
		} tiṣṭhati/}\\+} % tiṣṭhatī P11
\tl{
\pada{\app{\lem[wit={G4,J7,V1,J10}]{kṣaṇena}% +N24
		\rdg[wit={N23}]{kṣaṇe [ca]}
		\rdg[wit={J5}]{viṣaye}
		}
		mucyate
	\app{\lem[wit={G4,N23,V1,J10}]{yogī}
		\rdg[wit={J5}]{yogogī}
		\rdg[wit={J7}]{vyādhi}% +N24
		}}
\pada{\app{\lem[wit={G4,N23,V1,J10}]{vyādhimṛtyujarādibhiḥ}
		\rdg[wit={J5}]{vyādhimṛjarāpahe}
		\rdg[wit={J7}]{janmamṛtyujarādibhiḥ}
%		\rdg[wit={N24}]{jarāmṛtyumayādikaiḥ}
		}//\versenr}%
%	\myfn{The Pādas b--e are not found in \getsiglum{J5,G4,Gr2,V1,J10}, but in \getsiglum{Gr3,G11,N19,V15,GrB,Jyo}.}
%	\NotIn{N3}
	\\!}
\end{tlg}
%\commcite\newpage
\trcite{34}

\startaltrecension
\begin{alttlg}[hp03_034_1]% Gr3,G11,N19,V15,GrB,Jyo (9)
\tl{\app{\lem[alt=\mylem{34*1--2},nonum,nosep]{}
		\rdg[wit={Gr3,G11,N19,V15,GrB,Jyo},postwit=\textapp{(as a substitute for 3.34)}]{\incl}}%
\pada{kalāṃ
	\app{\lem[wit={Gr3,N19,V15,P11,Jyo}]{parāṅmukhīṃ}
		\rdg[wit={G11,V3}]{parāṅmukhī}
		\rdg[wit={C6}]{avāṅmukhī}}
		kṛtvā}
\pada{\app{\lem[wit={Gr3,G11,V15,V3,Jyo}]{tripathe}% tṛpathe V19
		\rdg[wit={N19}]{tripathaṃ}
		\rdg[wit={C6}]{tripatha}
		\rdg[wit={P11}]{tripātha}}
	\app{\lem[wit={V15}]{parivartayet}
		\rdg[wit={V19,N19,GrB}]{parivarjayet} % dv<<ā>>ravarjayet C6
		\rdg[wit={E2}]{parivardhayet}% +K3,C7
		\rdg[wit={Jyo}]{pariyojayet}
		\rdg[wit={G11}]{ko prayojayet}}/}\\+}  % om. in V1,J10
\tl{
\pada{\app{\lem[wit={Gr3,G11,N19,V15,P11,C6,Jyo}]{sā}
		\rdg[wit={N19}]{sāṃ}
		\rdg[wit={V3}]{sa}
		}
	bhavet khecarī mudrā} % bhat ṣecarī V19
\pada{vyomacakraṃ 
	\app{\lem[wit={ceteri}]{tad ucyate}% Gr3,N19,V15,GrB,Jyo
		\rdg[wit={G11}]{praśasyate}
		}//\versenr} 
		\sgwit{Gr3,G11,N19,V15,GrB,Jyo}
%		\note[type=anmkg,nonum,lem=\mylem{34*1--2},labelb={3-34-1}]{included in \getsiglum{Gr3,G11,N19,V15,GrB,Jyo} as a substitute for 3.34.}
		\\!}
\end{alttlg}

\begin{alttlg}[hp03_034_2]% Gr3,G11,N19,V15,GrB,Jyo (9)
\tl{%\app{\lem[nolem]{}
	%\rdg[wit={Gr3,G11,N19,V15,GrB,Jyo}]{\incl}}%
\pada{rasanām ūrdhvagāṃ kṛtvā}
\pada{kṣaṇārdhaṃ % ṃ om. P11,N23
	\app{\lem[wit={ceteri}]{yadi}% padi P11
		\rdg[wit={Jyo}]{api}
		} 
		tiṣṭhati/}\\+} % tiṣṭhatī P11
\tl{
\pada{\app{\lem[wit={G11,N19,V15,V3}]{kṣaṇena}% +N24
		\rdg[wit={Gr3}]{viṣayair}% viṣayai V19
		\rdg[wit={P11,Jyo}]{viṣair vi°}% +F
		\rdg[wit={C6}]{duḥkhair vi°}}
		mucyate yogī}
\pada{vyādhimṛtyujarādibhiḥ//\versenr}
	\sgwit{Gr3,G11,N19,V15,GrB,Jyo}%
%	\myfn{The Pādas b--e are not found in \getsiglum{J5,G4,Gr2,V1,J10}, but in \getsiglum{Gr3,G11,N19,V15,GrB,Jyo}.}
%	\NotIn{N3}
	\\!}
\end{alttlg}
\endaltrecension

\teimute{
	\begin{quote}
	\textcolor{gray}{\ExecuteMetaData[\commfilename]{tr34-2}
	\texteng{(34*1--2)}}
	\end{quote}
%\def\commvnum{34}%
\def\labelvnum{34--34*2}%
\comminfn}
\newpage\ \newpage

\begin{tlg}[hp03_035]
\tl{\app{\lem[nolem]{}
	\rdg[wit={J7},alt=\textapp{found after \ref{III38}}]{\skp{found after 3.37}}}%
\pada{na \app{\lem[wit={ceteri}]{rogo}
		\rdg[wit={V1}]{roga}
		\rdg[wit={J10}]{rogān}}
	maraṇaṃ
	\app{\lem[wit={ceteri}]{tasya}
		\rdg[wit={Jyo}]{tandrā}}}
\pada{na nidrā na 
	\app{\lem[wit={ceteri}]{kṣudhā tṛṣā}
		\rdg[wit={V19,C6}]{tṛṣā kṣudhā}% tṛkhā V19
		}/}\\+}
\tl{
\pada{na
	\app{\lem[wit={ceteri}]{ca}
		\rdg[wit={V3}]{bhra}
		\rdg[wit={J5},alt={\om}]{\skp{\om}}} mūrchā
	\app{\lem[wit={ceteri},alt={bhavet}]{bhave\skp{t}}
		\rdg[wit={J10}]{bhave}}%
	\app{\lem[wit={ceteri},alt={tasya}]{\skm{t }tasya}
		\rdg[wit={N3}]{ta\,+}}}
\pada{\app{\lem[wit={ceteri}]{yo mudrāṃ vetti}% vitti J5, veti G11
		\rdg[wit={N3},alt={\illeg}]{\skp{\illeg}}}
	\app{\lem[wit={G4,J7,Gr3,G11,N19,V15,V1,J10,Jyo}]{khecarīm}
		\rdg[wit={N3,J5,N23,GrB}]{khecarī}% ṣecarī J5
		}//\versenr}
%		\anm{after \ref{III38}  \getsiglum{J7}}
		\\!}  %3.36
\end{tlg}
\commcite%\newpage


\begin{tlg}[hp03_036]
\tl{\app{\lem[nolem]{}
	\rdg[wit={Gr2,Gr3,N19,V15},alt={\om}]{\skp{\om}}}%
\pada{\app{\lem[wit={N3,G4,G11,P11,Jyo}]{pīḍyate}
		\rdg[wit={J5}]{piṃḍaṃte}
		\rdg[wit={J10,C6,V3}]{bādhyate}
		\rdg[wit={V1}]{chādyate}}
	na \app{\lem[wit={V1,J10,V3,Jyo}]{sa}% +N24
		\rdg[wit={Gr1,G11,P11,C6}]{ca}% +F
		} rogeṇa} % rogena V3
\pada{\app{\lem[wit={J5,P11}]{na ca lipyati}
		\rdg[wit={N3}]{na ca lipyata}
		\rdg[wit={G4,Jyo}]{lipyate na ca}% +F; pīḍyate na ca N24
		\rdg[wit={G11,V1,J10,C6,V3}]{lipyate na sa}
		} karmaṇā/}\\+}
\tl{
\pada{\app{\lem[wit={N3,G11,V1,J10,P11,V3,Jyo}]{bādhyate}
		\rdg[wit={J5}]{badhyate}
		\rdg[wit={G4}]{bhidyate}
		\rdg[wit={C6}]{khādyate}}
	na \app{\lem[wit={Gr1,P11}]{ca}
		\rdg[wit={G11,V1,J10,C6,V3,Jyo}]{sa}} kālena}
\pada{\app{\lem[wit={N3,J5,J10,GrB,Jyo}]{yo mudrāṃ vetti}% mudrā N3
		\rdg[wit={G11,V1}]{yasya mudrāsti}% +F,N24
		}
	\app{\lem[wit={N3,J10,Jyo}]{khecarīm}
		\rdg[wit={J5,G11,V1,GrB}]{khecarī}}//\versenr} % ṣecarī J5
%	\NotIn{Gr2,Gr3,N19,V15}
	\\!}
%	\sgwit{Gr1,V1,J10,GrB,Jyo}\\!}  %3.37
\end{tlg}
\commcite\newpage


\begin{tlg}[hp03_037]
\tl{\app{\lem[nolem]{}
	\rdg[wit={N23},alt={\om}]{\skp{\om}}}%
\pada{\app{\lem[wit={ceteri}]{cittaṃ}
		\rdg[wit={N3}]{ci\,+}
		\rdg[wit={V19}]{citte}
		}
	\app{\lem[wit={ceteri}]{carati khe} % ṣe J5, caratī kṣe P11
		\rdg[wit={N3},alt={\illeg}]{\skp{\illeg}}}
	\app{\lem[wit={ceteri},alt={yasmāj}]{yasmā\skp{j}}% yasmāt* V1
		\rdg[wit={N3}]{+\,.āj}
		\rdg[wit={J5}]{yasyā}
		\rdg[wit={V3}]{yasyāt}
		}j}
\pada{jihvā carati % jihva J8; jihvā jihva V19 (ditt.); calati N3
	khe \app{\lem[wit={ceteri}]{gatā}% gatāḥ C8
		\rdg[wit={P11}]{yadā}}/}\\+} 
\tl{
\pada{\app{\lem[wit={N3,J5,G11,V15,P11,Jyo}]{tenaiṣā}
		\rdg[wit={N19,V1,J10,V3}]{tenaiva}
		\rdg[wit={J7,Gr3,C6}]{teneyaṃ}} khecarī
	\app{\lem[wit={N3,J5,G11,V15,P11,Jyo}]{nāma}
		\rdg[wit={ceteri}]{mudrā}}}
\pada{\app{\lem[wit={N3,J5,G11,V15,P11,Jyo}]{mudrā}
		\rdg[wit={J7,Gr3,N19,V1,J10,C6,V3}]{sarva}
		\rdg[wit={G4},alt={\illeg}]{\skp{\illeg}}}
	\app{\lem[wit={ceteri}]{siddhair namaskṛtā}  % śi° P11, siddhai V19,V15, siddhir C8ac
		\rdg[wit={G4}]{siddhir nigadyate}
		\rdg[wit={Jyo}]{siddhair nirūpitā}
		}//\versenr}\label{III38} 
%	\NotIn{N23}
	\anm{=\,\manuref{X4.47}}\\!}% 4.11*7
\end{tlg}
\commcite%\newpage


\begin{tlg}[hp03_038]
\tl{
\pada{\app{\lem[wit={ceteri}]{khecaryā}
		\rdg[wit={V3}]{khecaryāṃ}} 
	\app{\lem[wit={ceteri}]{mudritaṃ}
		\rdg[wit={J5}]{mudrataṃ}
		\rdg[wit={G4,C6}]{mudritā}
		} 
	\app{\lem[wit={ceteri}]{yena}
	\rdg[wit={C6}]{ye tu}}}
\pada{\app{\lem[wit={ceteri}]{vivaraṃ}
		\rdg[wit={V1,P11,C6}]{vicaran/raṃ}} % °ran* C6
	\app{\lem[wit={ceteri}]{lambikordhvataḥ}% +G4; °ta J5
		\rdg[wit={E2}]{lampikordhvataḥ}% +K3,C7
		\rdg[wit={N3},alt={\illeg}]{\skp{\illeg}}}/}\\+}
\tl{
\pada{\app{\lem[wit={J5,G4,N23,G11,N19,V15,V1,J10,GrB}]{tasya na}% naḥ V15
		\rdg[wit={N3}]{+\,[s]ya na}
		\rdg[wit={J7,Gr3,Jyo}]{na tasya}}
		kṣarate 
	\app{\lem[wit={ceteri}]{binduḥ} % bindu P11,N23
		\rdg[wit={G11}]{cittaṃ}}}
\pada{kāminyā\app{\lem[wit={ceteri}]{śleṣitasya}% ślo° J5
		\rdg[wit={P11}]{saṃślitasya}
		\rdg[wit={Gr2}]{liṅgitasya}% +K3
		\rdg[wit={C6}]{liṅgitena}}
		ca//\versenr}\label{III39}\\!}  %3.39
\end{tlg}
\commcite\newpage

% P11,C6 have the pātāla-stanza here.


\begin{tlg}[hp03_039]
\tl{\app{\lem[nolem]{}
	\rdg[wit={C6},alt=\textapp{found between \ref{III42}ab and cd}]{\skp{found between 3.41ab and cd}}}%
\pada{\app{\lem[wit={ceteri}]{calito}
		\rdg[wit={N3}]{calato}
		\rdg[wit={N23}]{calitā}
		\rdg[wit={V19}]{calate}
		\rdg[wit={C6}]{patito}}'pi 
	\app{\lem[wit={ceteri}]{yadā}
		\rdg[wit={J5,G11}]{yathā}
		} binduḥ} % bindu P11, biṃḥ N19, sinduḥ G11
\pada{\app{\lem[wit={N3,E2,G11,N19,V15,V1,P11,C6,Jyo}]{saṃprāpto}% +K3,C7
		\rdg[wit={Gr2,V19,J10,V3}]{saṃprāptaś}
		\rdg[wit={J5}]{saṃprāpte}}
	\app{\lem[wit={J5,G11,V15,V1,Jyo}]{yonimaṇḍalam}% +G7? illeg. G4
		\rdg[wit={N3}]{yogimaṃḍalaṃ}
		\rdg[wit={N19}]{yonimaṃgalaṃ}
		\rdg[wit={P11,C6}]{vahnimaṇḍalaṃ}% +M1,M3(°le)
		\rdg[wit={J10,V3}]{cāgnimaṇḍalaṃ}
		\rdg[wit={Gr2,V19}]{ca hutāśanam}% = VM; °nā J7ac
		\rdg[wit={E2}]{pi hutāśanaṃ}% +K3,C7
		}/}\\+}
\tl{
\pada{\app{\lem[wit={ceteri},alt={vrajaty}]{vraja\skp{ty}}
		\rdg[wit={N3}]{vṛjaṃty}
		\rdg[wit={J5}]{vrajety}
		\rdg[wit={N23}]{jajaty}
%		\rdg[wit={N24}]{nayed}
		}%
	\app{\lem[wit={ceteri},alt={ūrdhvaṃ}]{\skm{ty }ūrdhvaṃ} % ṃ om. J7
		\rdg[wit={N3}]{ū\,+}}
	\app{\lem[resp=emend,postwit=\texteng{(cf.\,VM)}]{hataḥ śaktyā}
		\rdg[wit={Jyo}]{hṛtaḥ śaktyā}
		\rdg[wit={N23}]{hatāchantkā}
		\rdg[wit={J7,G11,N19,V15,V1,J10,GrB}]{haṭhāc chaktyā}% havāt N19
		\rdg[wit={N3,G4},alt={\illeg}]{\skp{\illeg}}
		\rdg[wit={J5}]{haṭhāt saktyā}% G4 illeg.; harec chaktyā F
%		\rdg[wit={N24}]{haṭhavyaktyā}
		\rdg[wit={E2}]{hi tacchaktyā}% +K3,C7
		\rdg[wit={V19}]{hi tadbhuktyā}
		}}\marmas
\pada{\app{\lem[wit={J5,G4,G11,N19,V15,P11,C6,Jyo}]{nibaddho}
		\rdg[wit={V1}]{nibadhno}
		\rdg[wit={Gr2,E2}]{niruddho}% +K3,C7
		\rdg[wit={J10,V3}]{nirodho}
		\rdg[wit={V19}]{viruddhe}
		\rdg[wit={N3},alt={\illeg}]{\skp{\illeg}}}
	\app{\lem[wit={ceteri}]{yoni}
		\rdg[wit={J10,P11,V3}]{yoga}
		}mudrayā//\versenr}
%\anm{betw. \ref{III42}ab and cd \getsiglum{C6}}
\label{calitopi}
\\!}  %3.40
\end{tlg}
\commcite\newpage

%\Anm{\getsiglum{J10,V3} have \ref{III43} here.}


\startaltnormal
\begin{alttlg}[hp03_039_1]
\tl{\app{\lem[nolem]{}
	\rdg[wit={Gr1,Gr2,G11,N19,V15,GrB}]{\incl}}%
\pada{kapālakuhare jihvā} % jīhvā N23
\pada{\app{\lem[wit={J5,G4,G11,V15,V3}]{kalā}
		\rdg[wit={Gr2,N19}]{kāla}
		\rdg[wit={N3}]{kālā}
		\rdg[wit={P11,C6}]{kṛtvā}}%
	\app{\lem[wit={Gr1,G11,V15,GrB}]{saṃdhāna}
		\rdg[wit={N19}]{saṃdhāra}
		\rdg[wit={Gr2}]{saṃhāra}}%
	\app{\lem[wit={Gr1,Gr2,G11,N19,V15,P11,C6}]{mudrayā}
		\rdg[wit={V3}]{varjitā}}//\versenr} % varjitāḥ
%		\NotIn{Gr3,V1,J10,Jyo}
%		\sgwit{Gr1,Gr2,G11,N19,V15,GrB}
		\label{III40_1}\\!}% +F
\end{alttlg}
\altcommcite\newpage
\endaltnormal

% In the 6-chp version it is followed by \devnote{brahmarandhragatā nityaṃ tasya siddhir na dūrataḥ}.


\begin{tlg}[hp03_040]
\tl{
\pada{\app{\lem[wit={ceteri}]{ūrdhva}
		\rdg[wit={J5,G4,N19}]{ūrdhvaṃ}}%
	\app{\lem[wit={J7,V19,Jyo}]{jihvaḥ}% +C7
		\rdg[wit={N23,G11}]{jihva}
		\rdg[wit={N3,J5,E2,N19,V15,V1,J10,GrB}]{jihvā}}
	\app{\lem[wit={N3,J5,E2,G11,J10,V3}]{sthito}
		\rdg[wit={Gr2,V19,V15,V1,P11,C6,Jyo}]{sthiro}% +K3,C7
		\rdg[wit={N19}]{sito}} bhūtvā}
\pada{somapānaṃ
	\app{\lem[wit={ceteri}]{karoti yaḥ}
		\rdg[wit={C6}]{karoti saḥ}
		\rdg[wit={N3}]{karo\,+\,+}}/}\\+}
\tl{
\pada{\app{\lem[wit={ceteri}]{māsārdhena}
		\rdg[wit={G11}]{māsād ūrddhve}
		\rdg[wit={N3},alt={\illeg}]{\skp{\illeg}}} 
\app{\lem[wit={ceteri}]{na}% ni J5
		\rdg[wit={C6}]{tu}
		\rdg[wit={N3},alt={\illeg}]{\skp{\illeg}}
		\rdg[wit={V19},alt={\om}]{\skp{\om}}
		} saṃdeho}
\pada{mṛtyuṃ jayati yogavit//\versenr}\label{III41}\\!}  %3.41
\end{tlg}
\commcite%\newpage


\begin{tlg}[hp03_041]
\tl{\app{\lem[nolem]{}
	\rdg[wit={N3,J5},alt={\om}]{\skp{\om}}}%
\pada{nityaṃ % nitya V19
	somakalā\app{\lem[wit={ceteri}]{pūrṇaṃ}
		\rdg[wit={N23,N19,P11}]{pūrṇa}
		\rdg[wit={J10}]{pūrṇe}}}
\pada{śarīraṃ yasya
	\app{\lem[wit={G4,Gr2,Gr3,N19,P11,C6,Jyo}]{yoginaḥ}
		\rdg[wit={V3}]{yoginaṃ}
		\rdg[wit={G11,V15,V1,J10}]{dehinaḥ}}/}\\+}
\tl{
\pada{takṣakeṇāpi % tat*kṣa° N19; °nādi G4
	\app{\lem[wit={G4,J7,Gr3,G11,V15,J10,C6,Jyo}]{daṣṭasya}
		\rdg[wit={N23,V1,P11,V3}]{dṛṣṭasya}% +J17
		\rdg[wit={N19}]{daṃṣṭrasya}}}
\pada{\app{\lem[wit={ceteri}]{viṣaṃ tasya na}
		\rdg[wit={G4}]{viṣeṇa na hi}}
	\app{\lem[wit={ceteri}]{sarpati}
		\rdg[wit={V3}]{sparśati}
		\rdg[wit={G4,N23}]{pīḍyate}}//\versenr}%
	\myfn{After this verse, \getsiglum{V3} has an additional line:
	\devnote{tasmād idaṃ prakurvīta nityayuktaḥ samāhitaḥ/}}
	\label{III42}
%	\NotIn{N3,J5}
	\\!}  %3.42 ; not in N3,J5,C2 but in G4,N24; possibly interpolation
\end{tlg}
\commcite\newpage


\begin{tlg}[hp03_042]
\tl{\app{\lem[nolem]{}
	\rdg[wit={J10,V3},alt=\textapp{found after \ref{calitopi}}]{\skp{found after 3.39}}}%
\pada{indhanāni % indhānāni N23, yindhanāni V3
	yathā vahni}% s om. P11
\pada{\app{\lem[wit={Gr3,V15,V3,Jyo},alt={tailavart(t)iṃ}]{\skm{s }tailavartiṃ}% varttiṃ V3,V19,V15
		\rdg[wit={N3,N19,P11}]{tailavart(t)i}% tela P11; vartti N3,P11,N19
		\rdg[wit={Gr2,J10,C6}]{tailavart(t)ī}% varttī N23,C6,J10
		\rdg[wit={G11}]{tailaṃ varttiṃ}
		\rdg[wit={J5}]{tailavatti}
		\rdg[wit={V1}]{tailāvṛtti}}
	\app{\lem[wit={ceteri}]{ca}
		\rdg[wit={J5,V1}]{va}}
	\app{\lem[wit={ceteri}]{dīpakaḥ}
		\rdg[wit={V1}]{dīpikaḥ}}/}\\+} % V1 vowel sign i added later
\tl{
\pada{tathā \app{\lem[wit={ceteri}]{soma}
		\rdg[wit={N19}]{sarva}}%
	kalā\app{\lem[wit={ceteri}]{pūrṇaṃ}% mūrṇaṃ J5
		\rdg[wit={J10}]{pūrṇa}
		\rdg[wit={J7,N19}]{pūrṇo}}}
\pada{\app{\lem[wit={N23,J7,V19,E2,N19,V1,J10,V3,Jyo}]{dehī dehaṃ}% dehāṃ V3
		\rdg[wit={G11,V15,C6}]{dehaṃ dehī}
		\rdg[wit={J5}]{deha devaṃ}
		\rdg[wit={P11}]{dehe dehīṃ}
		\rdg[wit={N3},alt={\illeg}]{\skp{\illeg}}}
	\app{\lem[wit={J5,Gr2,Gr3,G11,N19,V1,P11,Jyo}]{na muñcati}
		\rdg[wit={J10}]{na mucyati}
		\rdg[wit={V15}]{na muṃcyati}
		\rdg[wit={C6,V3}]{na mucyate}
		\rdg[wit={N3}]{+\,+\,+\,ti}}//\versenr}\label{III43}%
\myfn{After this verse, \getsiglum{Gr2,V19,E2} have an additional line:
\devnote{rasanāṃ veśayed ūrdhvaṃ pibet tat srā(sra)vitaṃ jalam/}}	
%	\anm{after \ref{calitopi} \getsiglum{J10,V3}}
	\\!}  %3.43
\end{tlg}
\commcite%\newpage

% \startaltrecension
% \begin{alttlg}[hp03_042_1]
% \pada{\app{\lem[wit={J7,Gr3}]{rasanāṃ}
		% \rdg[wit={N23}]{rasānāṃ}}
	% \app{\lem[wit={J7,Gr3}]{veśayed ūrdhvaṃ}
		% \rdg[wit={N23}]{vasayed ūrdhvaṃ}}}
% \pada{pibet ta\app{\lem[wit={Gr3},alt={srāvitaṃ}]{\skm{t }srāvitaṃ}
		% \rdg[wit={Gr2}]{sravitaṃ}} 
	% \app{\lem[wit={ceteri}]{jalam}
		% \rdg[wit={E2}]{jagat}}/} % ta{{sya}}chrāvitaṃ V19
% \sgwit{Gr2,Gr3}\\!
% % \begin{alttlg}[hp03_042_2]
% % \pada{tasmād idaṃ prakurvīta}
% % \pada{nityayuktaḥ samāhitaḥ/}\label{III43_2} \sgwit{V3}\\!
% \end{alttlg}
% \altcommcite\newpage
% \endaltrecension


\begin{tlg}[hp03_043]
\tl{
\pada{\app{\lem[wit={ceteri}]{gomāṃsaṃ}
		\rdg[wit={J7,J10}]{gomāṃsa}
		\rdg[wit={V19}]{gomāṃ}}
		bhakṣayen nityaṃ} % °ye V3, bhakṣat*ye J10
\pada{pibed amara%
	\app{\lem[wit={ceteri}]{vāruṇīm}
		\rdg[wit={J5,N19,V15,V3}]{vāruṇī}}/}\\+}    % V3 bhakṣaye
\tl{
\pada{kulīnaṃ % kulītaṃ P11
	\app{\lem[wit={ceteri},alt={tam}]{ta\skp{m}}
		\rdg[wit={J7}]{tum}}m ahaṃ
	\app{\lem[wit={ceteri}]{manye} % ahamanye N23
%		\rdg[wit={Jyo}]{manya}
		\rdg[wit={V3}]{vidyāṃ}
		\rdg[wit={J10}]{viṃdyāṃ}}}
\pada{\app{\lem[wit={N3,G4,G11,J10,C6,V3,Jyo}]{itare}
		\rdg[wit={V15,V1,P11}]{tv itare}
		\rdg[wit={N19}]{cetare}
		\rdg[wit={J5}]{udhare}
		\rdg[wit={Gr2,Gr3}]{netarān}}
	kula\app{\lem[wit={ceteri}]{ghātakāḥ} % °kā P11
%		\rdg[wit={N3}]{kuṣvaghātakāḥ}
		\rdg[wit={Gr2,Gr3}]{ghātakān}
		\rdg[wit={J5}]{bālakān}
		}//\versenr}\label{III44}\\!}  %3.44
\end{tlg}
\commcite\newpage


\begin{tlg}[hp03_044]
\tl{
\pada{gośabde\app{\lem[wit={ceteri},alt={°noditā jihvā}]{\skp{°}noditā jihvā}% śabdono P11
		\rdg[wit={N23}]{nāditā jihvā}
		\rdg[wit={N3},alt={\illeg}]{\skp{\illeg}}}}
\pada{\app{\lem[wit={ceteri}]{tatpraveśo}
		\rdg[wit={N3}]{+\,[t]praveśo}
		\rdg[wit={J5}]{tatpradeśo}
		\rdg[wit={G11}]{tatpraveśaś}
		\rdg[wit={P11}]{tatrāveśo}}
	\app{\lem[wit={ceteri}]{hi}
		\rdg[wit={P11}]{ha}
		\rdg[wit={N23}]{di}
		\rdg[wit={G11}]{ca}} tāluni/}\\+}
	% \app{\lem[wit={ceteri}]{tāluni}
		% \rdg[wit={G11,P11}]{tālunī}
		% \rdg[wit={J5}]{lālunī}
		% \rdg[wit={J10}]{tāluniṃ}}
\tl{
\pada{\app{\lem[wit={ceteri}]{gomāṃsa}
		\rdg[wit={J5,N19,V15,V1}]{gomāṃsaṃ}
		\rdg[wit={N23}]{gomāsaṃ}}%
	\app{\lem[wit={ceteri}]{bhakṣaṇaṃ}
		\rdg[wit={N3}]{bhakṣaṇe}}
	\app{\lem[wit={ceteri}]{tat tu}% ttat tu J7
		\rdg[wit={N23}]{\_\,rttu}
		\rdg[wit={V15}]{tac ca}
		\rdg[wit={N19}]{caitat}
		\rdg[wit={C6}]{hy etan}}}
\pada{mahāpātakanāśanam//\versenr}\\!}  %3.45
\end{tlg}
\commcite%\newpage


\begin{tlg}[hp03_045]
\tl{\app{\lem[nolem]{}
	\rdg[wit={V15,V1},alt={\om},postwit=\texteng{(cd added in the margin \emph{sec. m.} \getsiglum{V1})}]{\skp{\om}}}%
\pada{jihvāpraveśasaṃbhūta}% jihva V19; pradesasaṃbhūtā J5; bhūtaḥ P11
\pada{\app{\lem[wit={J7,E2,N19,Jyo}]{vahninotpāditaḥ}% +P17,K3
		\rdg[wit={J5,C6},alt={°tpāditā}]{vahninotpāditā}% +F,C7
		\rdg[wit={P11},alt={°tpāditāṃ}]{vahninotpāditāṃ}
		\rdg[wit={N3},alt={°tpādi\,+}]{vahninotpādi\,+}
		\rdg[wit={V3},alt={°nnāpitā}]{vahninonnāpitā}
		\rdg[wit={J10},alt={°ttāpito}]{vahninottāpito}
		\rdg[wit={G4}]{vaṃh[n]inotāpitaḥ}
		\rdg[wit={G11}]{vahniś cotthāpitot}
		\rdg[wit={N23}]{vahnir utpāditaḥ}
		\rdg[wit={V19}]{hninotpāditaṃ}}
	\app{\lem[wit={J5,G4,Gr2,V19,G11,N19,Jyo}]{khalu}% khaluḥ J5; =P17
%		\rdg[wit={V17}]{daralu} % stemma point;
%		N26 vahnino[kṣ]āpitodarāt*, N9 vahni[natpi or: nāpi]tādaraṃ
		\rdg[wit={E2}]{kila}
		\rdg[wit={J10,V3}]{daraṃ}
		\rdg[wit={P11}]{°bare}
		\rdg[wit={C6}]{surāḥ}
		\rdg[wit={N3},alt={\illeg}]{\skp{\illeg}}}/}\\+}
\tl{
\pada{\app{\lem[wit={G4,Gr2,E2,N19,J10,C6,V3,Jyo}]{candrāt sravati} % °drān mravati N23, śravati N19, snavati J10
		\rdg[wit={J5}]{candrā sravati}
		\rdg[wit={P11}]{candra sravati}
		\rdg[wit={G11}]{candrāt sṛjati}
		\rdg[wit={V19}]{candrā dravati}% °d dravatiF
		\rdg[wit={N3}]{+\,+\,+\,+\,[t]i}}
	\app{\lem[wit={J7,Gr3,J10,P11,Jyo}]{yaḥ sāraḥ}
		\rdg[wit={G4,N23}]{yaḥ sāra}% ṭāḥ? N23
		\rdg[wit={N3,N19,C6}]{yat sāraṃ}% V1 sec.m.
		\rdg[wit={G11}]{yat sāraḥ}
		\rdg[wit={J5}]{ya sāraṃ}
		\rdg[wit={V3},prewit=\texteng{(the same as the line inserted after \ref{III42})}]{yaḥ sāraṃ tasmād idam prakurvīta nityayuktaḥ samāhitaḥ}
		}}
\pada{\app{\lem[wit={ceteri}]{sā}
		\rdg[wit={E2}]{sa}} 
		syā\app{\lem[wit={ceteri},alt={amaravāruṇī}]{\skm{d }amaravāruṇī}
		\rdg[wit={J10}]{aṃmavāruṇī}}//\versenr}%
	\myfn{%
%	In \getsiglum{V1} the second half is only added in the margin sec. m.:
%	\devnote{tasmāc carati yat sāraṃ sā syād amaravāruṇī}.\\
	After this verse, \getsiglum{G11} has \emph{Gorakṣaśataka} 60--61.}
%	\NotIn{V15,V1}
	\\!}  % om. in E4
\end{tlg}
\commcite\newpage


\begin{tlg}[hp03_046]
\tl{
\pada{\app{\lem[wit={J7,Gr3,V15,V3,Jyo}]{mūrdhnaḥ}
		\rdg[wit={J10}]{mūrdhneḥ}
		\rdg[wit={G11}]{mūrddhaḥ}
		\rdg[wit={J5,N19}]{mūrddhaṃ}
		\rdg[wit={N3}]{mūrddhvaḥ}
		\rdg[wit={V1}]{mūrddhva}
		\rdg[wit={N23}]{bhūrddhaḥ}
		\rdg[wit={P11,C6}]{ūrdhvaṃ}}
ṣoḍaśa\app{\lem[wit={N3,J7,V19,G11,V1,J10,GrB}]{padmapattra}% +F
		\rdg[wit={J5,E2,V15,Jyo}]{pattrapadma}% +K3,C7
		\rdg[wit={N19}]{patrapatra}
		\rdg[wit={N23}]{patra}}galitaṃ % galitaḥ N19
		prāṇād avāptaṃ % prāṇāvadāttaṃ J5
	\app{\lem[wit={ceteri},alt={haṭhād}]{haṭhā\skp{d}}
		\rdg[wit={V3}]{haṭhāṃ}}-}\\+}
\tl{
\pada{\app{\lem[wit={ceteri},alt={ūrdhvāsyo}]{\skm{d }ūrdhvāsyo}
		\rdg[wit={N23}]{ūrddhosyo}
		\rdg[wit={V3}]{varddhāsyo}}
	\app{\lem[wit={ceteri}]{rasanāṃ}
		\rdg[wit={N19}]{rasanā}
		\rdg[wit={N23}]{ramanā}}
	\app{\lem[wit={N3,J5,J7,Gr3,G11,V15,P11,C6,Jyo}]{niyamya}
		\rdg[wit={N23,N19}]{niyasya}% ti° N19
		\rdg[wit={V1}]{ca yāmya}
		\rdg[wit={J10,V3}]{vidhāya}}
	\app{\lem[wit={ceteri}]{vivare}
		\rdg[wit={N23}]{vicare}
		\rdg[wit={Gr3}]{vivaraṃ}
		\rdg[wit={C6}]{vidhivat}} % °vac chaktiṃ C6
	\app{\lem[wit={ceteri}]{śaktiṃ}% ṃ om. N23
		\rdg[wit={J7}]{śaktiḥ}} parāṃ % parā G4,J10
	\app{\lem[wit={ceteri}]{cintayet} % citayet N19
		\rdg[wit={N23}]{cintayat}
		\rdg[wit={G4,E2,Jyo}]{cintayan}% +K3,C7
		}/}\\+}
\tl{
\pada{\app{\lem[wit={N3,J5,G11,N19,V15,V1,GrB,Jyo}]{utkallola}
		\rdg[wit={J10}]{uttakallola}
		\rdg[wit={J7,Gr3}]{tatkallola}
		\rdg[wit={N23}]{taptalola}
		\rdg[wit={G4}]{hṛt[k]alola}}%
	\app{\lem[wit={ceteri}]{kalājalaṃ}
		\rdg[wit={G4}]{karāṃṛtaṃ}
		\rdg[wit={J10,V3}]{jalākulaṃ}}
	\app{\lem[wit={N3,G4,J7,Gr3,G11,N19,V1,P11,C6,Jyo}]{ca}
		\rdg[wit={J5,J10,V3}]{su}
		\rdg[wit={N23}]{ya}
		\rdg[wit={V15},alt={\om}]{\skp{\om}}}
	\app{\lem[wit={ceteri}]{vimalaṃ}
		\rdg[wit={N3}]{vimala}
		\rdg[wit={C6}]{vimalā}}
	\app{\lem[wit={ceteri}]{dhārāmṛtaṃ}% dhāra° J7ac
		\rdg[wit={Jyo}]{dhārāmayaṃ}
		}\marmas yaḥ pibe-}\\+}% pibe J7,V15, piben N3,P11,C6,N23
\tl{%
\pada{\app{\lem[wit={ceteri},alt={nirdoṣaḥ sa}]{\skm{n }nirdoṣaḥ sa}% +G5
		\rdg[wit={J5,V1,P11}]{nirdoṣaṃ sa}
		\rdg[wit={N19}]{nirdoṣo 'sya}
		\rdg[wit={G11}]{nidoṣaś ca}
		\rdg[wit={Jyo}]{nirvyādhiḥ sa}}
	mṛṇāla\app{\lem[wit={ceteri}]{komala}% mṛnāla V19
		\rdg[wit={N23}]{komale}}%
	\app{\lem[wit={N3,G11,N19,V15,V1,P11,Jyo},alt={vapur}]{vapu\skp{r}}
		\rdg[wit={J5,Gr2,Gr3,J10,C6,V3}]{tanur}}r % tanu N23; +N24
		yogī ciraṃ jīvati//\versenr}% jogī V19, jīviti J5
	\myfn{\getsiglum{Jyo} has a different verse order: \manuref{3.47} \textrightarrow\ \manuref{46} \textrightarrow\ \manuref{48*2} \textrightarrow\ \manuref{48*1} \textrightarrow\ \manuref{48}.}
	\\!}  %3.46
\end{tlg}
\commcite\newpage
\ \newpage

\begin{tlg}[hp03_047]
\tl{
\pada{\app{\lem[wit={ceteri}]{cumbantī}% °ti P11
		\rdg[wit={J5}]{cubaṃti}
		\rdg[wit={N3}]{cubiṃtī}
		\rdg[wit={N23}]{vipitīṃ}}
	yadi \app{\lem[wit={ceteri},alt={lambikāgram}]{lambikāgra\skp{m}}
		\rdg[wit={P11}]{laṃbakārgram}
		\rdg[wit={E2}]{lampikāgram}}m % +K3,C7
		aniśaṃ jihvā % anisaṃ P11
	\app{\lem[wit={ceteri}]{rasa} % also V3mg
		\rdg[wit={J10,V3}]{śiraḥ}}syandinī}\\+}
\tl{
\pada{\app{\lem[wit={J5,N23,J7,V15,V1,C6,V3,Jyo}]{sakṣārā}
		\rdg[wit={N3,Gr3,G11,N19}]{sā kṣārā}
		\rdg[wit={J10}]{sakṣāra}
		\rdg[wit={P11}]{sakṣīro°}}
	\app{\lem[wit={N3,J5}]{kaṭukātha}% = VM
		\rdg[wit={J7,Gr3,V15,Jyo}]{kaṭukāmla}% +V3marg,F; kaṭukāmū N24
		\rdg[wit={N23}]{vaṭukāmla}
		\rdg[wit={G11}]{kaṭukā ca}% +G5
		\rdg[wit={V1}]{kaṭukāsa}
		\rdg[wit={J10,V3}]{kaṭukādya}
		\rdg[wit={N19}]{kaṭutikta}
		\rdg[wit={G4}]{+\,+\,tikta}
		\rdg[wit={C6}]{kaṭutyakta}
		\rdg[wit={P11}]{°dakatikta}
		}
	\app{\lem[wit={ceteri}]{dugdha}
		\rdg[wit={P11}]{dagdha}
		\rdg[wit={J7}]{dugdhaṃ}
		\rdg[wit={N23}]{du}
		\rdg[wit={V19}]{dhugdha}
		}%
	\app{\lem[wit={N3,J5,G11,V1}]{sadṛśā}% = VM
		\rdg[wit={G4}]{sadṛśaṃ}
		\rdg[wit={N23,J10,C6,V3,Jyo}]{sadṛśī}% +K3,C7
		\rdg[wit={J7}]{sadṛśīṃ}
		\rdg[wit={Gr3}]{sādṛśī}
		\rdg[wit={N19,V15}]{lavaṇā}
		\rdg[wit={P11}]{lavaṇo}}
	\app{\lem[wit={ceteri}]{madhvājya} % also V3mg
		\rdg[wit={J10,V3}]{madhvādya}
		\rdg[wit={N19}]{vaddhājya}}%
	\app{\lem[wit={ceteri}]{tulyā} % also V3pc, talyā C6
		\rdg[wit={J10,V3}]{tulyaṃ}
		\rdg[wit={G4}]{tulya}}%
	\app{\lem[wit={J5,Gr2,Gr3},alt={°thavā}]{\skp{°}thavā}% +N24
		\rdg[wit={N3,N19,V1,J10,GrB,Jyo}]{tathā}% +F
		\rdg[wit={G4}]{pradā}
		\rdg[wit={G11}]{sadā}
		\rdg[wit={V15}]{savā}}/} \\+}
\tl{
\pada{vyādhīnāṃ haraṇaṃ % vyādhināṃ V1
	\app{\lem[wit={ceteri}]{jarāntakaraṇaṃ} % °karaṇa V15
		\rdg[wit={V19}]{jvarāntakaraṇaṃ}% +K3
		\rdg[wit={P11,C6}]{jarāpraśamanaṃ}}
	\app{\lem[wit={N3,G11,Jyo}]{śāstrāgamodīraṇaṃ}
		\rdg[wit={J5}]{śastrārthagamodīraṇaṃ}
		\rdg[wit={N23}]{śāstrapramodīraṇaṃ}
		\rdg[wit={J7,V15,J10,C6,V3}]{śāstrāgamodgīraṇaṃ}
		\rdg[wit={V1}]{śastrāṃgamodgīraṇaṃ}
		\rdg[wit={P11}]{śāstrodgamodgīraṇaṃ}
		\rdg[wit={Gr3,N19}]{śāstrāgamoddhāraṇaṃ}% +F
		}} \\+}
\tl{
\pada{\app{\lem[wit={ceteri},alt={tasya syād}]{tasya syā\skp{d}}% syābd N19
		\rdg[wit={N23}]{tasyād}
		\rdg[wit={P11}]{syāt svādam}}d
	a\app{\lem[wit={ceteri},alt={amaratvam}]{\skp{a}maratva\skp{m}}
		\rdg[wit={N23}]{amarakṣam}
		\rdg[wit={V3}]{aramatvam}
		\rdg[wit={Gr3}]{iha siddhir}}m
	aṣṭa\app{\lem[wit={N3,J5,G11,V1,P11},alt={guṇavat}]{guṇava\skp{t}} % guṇava J5
		\rdg[wit={V15}]{guṇāvat}
		\rdg[wit={Gr2,V19,N19,J10,C6,V3,Jyo}]{guṇitaṃ}
		\rdg[wit={E2}]{guṇitā}}% +K3,C7
	\app{\lem[wit={J5,E2,V1,G11,N19,V15,J10,P11,C6,Jyo},alt={siddhāṅganā}]{\skm{t }siddhāṅganā}% +K3,C7
		\rdg[wit={N3,N23,V19,V3},post=\texteng{(ṇā˟ \getsiglum{N3})}]{siddhāṅgaṇā}% +J17
		\rdg[wit={J7}]{siddhāṅgānā}}%
	\app{\lem[wit={ceteri}]{karṣaṇam}
		\rdg[wit={J5}]{karṣaṇaḥ}
		\rdg[wit={N23}]{karṣaṇā}}//\versenr}%
\myfn{After this verse, \getsiglum{G11} has 9 verses from the \emph{Vivekamārtaṇḍa} (119--124, 126--127, and 129).}%\emph{kākacañcuvad ... aṇimādiguṇādayaḥ}
		\\!}  %3.47
\end{tlg}
\commcite\newpage


\begin{tlg}[hp03_048]
\tl{
\pada{\app{\lem[wit={ceteri}]{ekaṃ}
		\rdg[wit={P11}]{ekāṃ}
		\rdg[wit={N23}]{evaṃ}}
\app{\lem[wit={ceteri}]{sṛṣṭi} % śṛṣṭi N3
		\rdg[wit={N19}]{dṛṣṭi}}%
	\app{\lem[wit={ceteri}]{mayaṃ} % maya N23
		\rdg[wit={G11}]{mataṃ}
		\rdg[wit={C6}]{midaṃ}
		\rdg[wit={N19}]{layaṃ}} bījam} 
\pada{ekā mudrā \app{\lem[wit={ceteri}]{ca}% eka J7ac,V15
		\rdg[wit={N19}]{tu}} khecarī/}\\+}
\tl{
\pada{eko
	\app{\lem[wit={ceteri}]{devo}% +G4
		\rdg[wit={N23}]{devā}
		\rdg[wit={N3}]{nirā°}}
	\app{\lem[wit={V1,V3,Jyo}]{nirālamba}% +YCM
		\rdg[wit={J7,Gr3}]{nirālambaś}
		\rdg[wit={N23}]{nirāśambaś}
		\rdg[wit={J5,G11,N19,V15,P11,C6}]{nirālambo}
		\rdg[wit={G4,J10}]{nirālambaṃ}
		\rdg[wit={N3}]{°laṃbo deva}}}
\pada{\app{\lem[wit={N3,G4,N19,V1,J10,C6,V3,Jyo}]{ekā}
		\rdg[wit={Gr3}]{caikā}
		\rdg[wit={N23}]{cakā}
		\rdg[wit={J7}]{caiṣā}
		\rdg[wit={J5,G11,V15,P11}]{hy ekā}}%
	\app{\lem[wit={ceteri},alt={°vasthā}]{vasthā} % °vatsthā? V15
		\rdg[wit={P11}]{mudrā}} manonmanī//\versenr}
	\anm{=\,\manuref{X4.124}}% 4.35*1
	\myfn{After this verse, \getsiglum{J10,V3,Jyo} have the same verse as \manuref{X4.45}. 
	While \getsiglum{J10,V3} have it also in chapter 4, \getsiglum{Jyo} has it only here.}
	\\!}  %3.48
\end{tlg}
\commcite\newpage


\startaltnormal
% \begin{alttlg}[hp03_048_1]
% \tl{
% \pada{\app{\lem[wit={Jyo}]{suṣiraṃ}
		% \rdg[wit={J10}]{sukhiraṃ}
		% \rdg[wit={V3}]{suciraṃ}} jñānajanakaṃ}
% \pada{pañca\app{\lem[wit={J10,Jyo}]{srotaḥ}
		% \rdg[wit={V3}]{śrotaḥ}}samanvitam/}\\+}
% \tl{
% \pada{\app{\lem[wit={Jyo}]{tiṣṭhate}
		% \rdg[wit={V3}]{tiṣṭhaṃti}
		% \rdg[wit={J10}]{tiṣṭhaṃtī}} % tiṣṭhati J10pc
		% khecarī mudrā}
% \pada{tasmin śūnye nirañjane//\versenr}
	% \sgwit{J10,V3,Jyo} \anm{=\,\manuref{4.11*5}}\\!} % 3.52
% \end{alttlg}
%\altcommcite%\newpage


%\Allexcept{N3,Gr3}
\begin{alttlg}[hp03_048_1] % metre: Mandākrāntā
\tl{\app{\lem[nolem]{}
	\rdg[wit={N3,Gr3}]{\excl}}%
\pada{\app{\lem[wit={Gr2}]{pātāle yad viśati}
		\rdg[wit={C6}]{pātālād yad viśati}
%		\rdg[wit={J11}]{pātāle yadvitaya}
		\rdg[wit={G11}]{pātāle yadvitayaḥ}
		\rdg[wit={P11}]{pātāle yadvitayu}
		\rdg[wit={V15}]{pātāle yadvitanta}
		\rdg[wit={G4}]{pātāḷe yadvitadhaya}
		\rdg[wit={N19}]{pātāle yadinaya}
		\rdg[wit={V1}]{pātāle yadvita}
		\rdg[wit={V3}]{yat prāleyaṃ cāpihita}
		\rdg[wit={J5}]{yat prāleyaṃ pihita}
		\rdg[wit={J10}]{yat prāleya pihita}
		\rdg[wit={Jyo}]{yat prāleyaṃ prahita}}
	\app{\lem[wit={G11,V15,Jyo}]{suṣiraṃ}
		\rdg[wit={C6}]{suśiraṃ}
		\rdg[wit={Gr2,J10,V3}]{sukhiraṃ} % sukhira N23
		\rdg[wit={J5}]{sukhire}
		\rdg[wit={N19}]{sukhīraṃ}
		\rdg[wit={V1}]{stuṣime}
		\rdg[wit={P11}]{śubiraṃ}
		}
	meru\app{\lem[wit={G11,N19,V15,P11}]{mūle tad asmin}% mūlai P11
		\rdg[wit={V1}]{mū tad asmi[ṃ]s}% merū V1
		\rdg[wit={C6}]{mūle tad asti}
		\rdg[wit={J7}]{mūle yad asti}
		\rdg[wit={J5}]{mūle yad astī}
		\rdg[wit={N23}]{mūle pakṣasti} % merū N23
		\rdg[wit={J10}]{mūrdhni sthitaṃ}
		\rdg[wit={V3}]{mūrddhyataḥthyaṃ}
		\rdg[wit={Jyo}]{mūrdhāntarasthaṃ}}}\\+}  %
\tl{
\pada{\app{\lem[wit={G11,P11,C6},alt={tattvaṃ caitat}]{tattvaṃ caita\skp{t}}
		\rdg[wit={Gr2,N19}]{tadvac caitat} % tadva caitat J7
%		\rdg[wit={J11}]{tadvac caitā}
		\rdg[wit={V15}]{taddac caitat}
		\rdg[wit={J10,V3,Jyo}]{tasmiṃs tattvaṃ}% =P17
		\rdg[wit={J5}]{tasmitvaṃ}
		\rdg[wit={V1}]{tatvaṃ yat}}t
	pravadati % °vadaṃti J5,G11, prava<<da>>ti J7
	\app{\lem[wit={ceteri},alt={sudhīs}]{sudhī\skp{s}}
%		\rdg[wit={J11}]{sudhī}
		\rdg[wit={N23}]{sudhās}}%
	\app{\lem[wit={ceteri},alt={tan mukhaṃ}]{\skm{s }tan mukhaṃ}
		\rdg[wit={P11,C6}]{tat sukhaṃ}}
%	\app{\lem[wit={ceteri}]{nimnagānām}
%		\rdg[wit={N23}]{niṣagmanāṃ}}
		nimnagānām/}\\+} % J10 gloss? korthaḥ nāḍīnāṃ
\tl{
\pada{\app{\lem[wit={N23,J7,G11,J10,V3,Jyo}]{candrāt sāraḥ}
		\rdg[wit={V15}]{candrā sāraḥ}
		\rdg[wit={J5,C6}]{candrāt sāraṃ}
		\rdg[wit={N19}]{candraḥ sāraḥ}
		\rdg[wit={V1}]{candrasāro[dha]}
		\rdg[wit={P11}]{caṃtaṃ prasāraṃ}}
	\app{\lem[wit={ceteri},alt={sravati/śravati}]{sravati} % śravati J5,V3,N19,V15,J10
		\rdg[wit={P11}]{grasati}
		\rdg[wit={N23}]{rapati}
		\rdg[wit={V1}]{[sra]vaṃtyai}} % one illegible akṣara before
	\app{\lem[wit={ceteri},alt={vapuṣas}]{vapuṣa\skp{s}}
		\rdg[wit={J10}]{vapuṣes}
		\rdg[wit={V3}]{vapayuṣes}
		\rdg[wit={P11,C6}]{vapuṣā}
		\rdg[wit={V15},alt={\om},post=\texteng{(jumps to \ref{Jaala1}a)}]{\skp{\om\ (jumps to 3.66a)}}}%
	\app{\lem[wit={ceteri},alt={tena}]{\skm{s }tena}
		\rdg[wit={P11}]{doṣa}}
	\app{\lem[wit={ceteri},alt={mṛtyur}]{mṛtyu\skp{r}} % mṛtyu P11
		\rdg[wit={J10,V3}]{mṛtyun}
		\rdg[wit={V15},alt={\om}]{\skp{\om}}}r narāṇāṃ}\\+}
\tl{
\pada{\app{\lem[nolem]{\skp{pāda d}}
		\rdg[wit={V15},alt={\om}]{\skp{\om}}}%
	\app{\lem[wit={ceteri}]{taṃ}
		\rdg[wit={G4,J7,Jyo}]{tad}
		\rdg[wit={N23}]{tac}}
	\app{\lem[wit={ceteri},alt={badhnīyāt}]{badhnīyā\skp{t}}% °yāta V3
		\rdg[wit={N23}]{cha\,\_\,yāt}
		%\rdg[wit={V15},alt={\om}]{\skp{\om}}
		}%
	\app{\lem[wit={J5,P11,C6},alt={sukaraṇamṛdā}]{\skm{t }sukaraṇamṛdā}% +J11
		\rdg[wit={N19}]{pakaraṇamṛdā}
		\rdg[wit={G4}]{svakaraṇamṛtaṃ}
		\rdg[wit={G11}]{sukaraṇam amṛtaṃ}
		\rdg[wit={V1}]{kakaraṇam amṛtaṃ}
		\rdg[wit={J10,V3}]{sukaraṇam atho}
		\rdg[wit={Jyo}]{sukaraṇam adho}
		\rdg[wit={J7}]{sukhakaram atho}
		\rdg[wit={N23}]{sukhakaraṇam artho}
		%\rdg[wit={V15},alt={\om}]{\skp{\om}}
		}
	\app{\lem[wit={ceteri}]{nānyathā}
		\rdg[wit={N23}]{nāmarthā}
		%\rdg[wit={V15},alt={\om}]{\skp{\om}}
		}
	\app{\lem[wit={Gr2,G11,N19,J10,C6,V3,Jyo}]{kāya}
		\rdg[wit={J5,G4,V1,P11}]{kārya}% +F,J11
		%\rdg[wit={V15},alt={\om}]{\skp{\om}}
		}siddhiḥ//\versenr}\marmas % siddhi P11,N19
\label{III49-2}
\anm{=\,\manuref{4.9}}
\myfn{%
\getsiglum{P11,C6} have this verse immediately after \ref{III39};
\getsiglum{N3,V19,C7} have it in chapter 4 (\manuref{4.9}), and
%\devnote{pātālād vā vipati śikhare merumūle tadāstā,
%tatvaṃ caitat pravadati susaṃmukhaṃ nimnaśanāṃ/\\
%caṃdrāt srāvaḥ śravati vapuṣas tena mṛtyur narāṇāṃ,
%taṃ badhnīyāt svakaraṇamṛnā nānyathā kāryasiddhi//\versenr}\\
\getsiglum{J5,G4,Gr2} in both chapter 3 and 4.
\getsiglum{E2} omits it in both places.} % but not in N2!
%\myfn{After \textit{sravati} in pāda c \getsiglum{V15} jumps to Jālaṃdharabandha (\ref{Jaala1}).}
\\!}
\end{alttlg}
\def\commvnum{48-1}%
\def\labelvnum{48*1}%
\comminfn\newpage
\endaltnormal


\begin{ava}[hp03_049a]
\app{\lem[wit={ceteri}]{atha mūlabandhaḥ} % bandha J5,V3,N19
\rdg[wit={J7,V19}]{mūlabandhaḥ}% +K3,C7
\rdg[wit={N23,V15},alt={\om}]{\skp{\om}}}/%
\myfn{In \getsiglum{G4}, the description of Uḍḍīyāṇabandha (3.58--65) follows after this heading for Mūlabandha. Similarly, in \getsiglum{V3} and \getsiglum{Jyo} (and also some \texteta\textsubscript{\textomega} manuscripts) Uḍḍīyāṇabandha is described before Mūlabandha, but with the correct heading each.}
\end{ava}

\begin{tlg}[hp03_049]
\tl{\app{\lem[nolem]{}
	\rdg[wit={V15},alt={\om}]{\skp{\om}}}%
\pada{\app{\lem[wit={ceteri}]{pārṣṇi}
		\rdg[wit={N23}]{pādima}}bhāgena % bhāge ca E2
		saṃpīḍya} % saṃpījya N23ac
\pada{yoni% yonam N23, yonīm V1
\app{\lem[wit={ceteri},alt={ākuñcayed}]{\skm{m }ākuñcaye\skp{d}}% +J5
		\rdg[wit={N3}]{ākuṃcaned}
		\rdg[wit={N23}]{ākuṃ<<cya\,+>>}}%
\app{\lem[wit={ceteri},alt={gudam}]{\skm{d }gudam}
		\rdg[wit={G4,V1,J10}]{dṛḍhaṃ}
		\rdg[wit={N23}]{<<+\,ta>>}}/}\\+}
\tl{
\pada{apānam ūrdhvam ākṛṣya} % ūrdhaṃ V3; utkṛṣya J5
\pada{\app{\lem[nolem]{\skp{pāda d}}
	\rdg[wit={J5},alt={\om}]{\skp{\om}}}%
	mūlabandho%
\app{\lem[wit={C6,V3}]{'yam iṣyate}% +F
		\rdg[wit={P11}]{'yam iṣyati}
		\rdg[wit={N3}]{mayiṣyate} % J5 jumps to the next verse: mūlabandhā(page break)di yoginaḥ.
		\rdg[wit={Gr2,Gr3,G11,N19,V1,J10}]{'yam ucyate}% VM
		\rdg[wit={Jyo}]{'bhidhīyate}
%		\rdg[wit={J5},alt={\om}]{\skp{\om}}
		\rdg[wit={G4},alt={\illeg}]{\skp{\illeg}}}//\versenr}
%	\lineom{d}{J5}
%	\unavbl{V15}
	\label{III50}
	\\!}
\end{tlg}

\avacite{49a}
\commcite\newpage


\begin{tlg}[hp03_050]
\tl{\app{\lem[nolem]{}
	\rdg[wit={V15},alt={\om}]{\skp{\om}}}%
\pada{\app{\lem[nolem]{\skp{pāda a}}
	\rdg[wit={J5},alt={\om}]{\skp{\om}}}%
\app{\lem[wit={G11,V1,GrB,Jyo},alt={adhogatim}]{adhogati\skp{m}} % better, =HR
		\rdg[wit={N3,Gr2,Gr3,N19,J10}]{adhogatam}}% lost J5,G4
\app{\lem[wit={Gr2,G11,N19,J10,P11,C6,Jyo},alt={apānaṃ vai}]{\skm{m }apānaṃ vai}
		\rdg[wit={N3}]{apānaṃ vaiḥ}
%		\rdg[wit={Jyo}]{apānaṃ vā}% = vai, due to Sandhi
		\rdg[wit={V3}]{apānaṃ ca}
		\rdg[wit={Gr3}]{apānaṃ tu}
		\rdg[wit={V1}]{apānaivam}}}
\pada{\app{\lem[nolem]{\skp{pāda b}}
	\rdg[wit={J5},alt={\om}]{\skp{\om}}}%
\app{\lem[wit={ceteri}]{ūrdhvagaṃ}
		\rdg[wit={N3}]{mūrddhagaṃ}
		\rdg[wit={P11}]{hy urdhvaṃgaṃ}
		\rdg[wit={N19}]{kurddhagaṃ}
		\rdg[wit={V3}]{vidyūrdhagaṃ}} kurute
\app{\lem[wit={N3,G11,V1,J10,GrB,Jyo}]{balāt}% +J11
		\rdg[wit={Gr2,Gr3}]{haṭhāt}
		\rdg[wit={N19}]{havān}}/}\\+}
\tl{
\pada{\app{\lem[nolem]{\skp{pāda c}}
	\rdg[wit={J5},alt={\om}]{\skp{\om}}}%
\app{\lem[wit={ceteri}]{ākuñcanena}% āku{ta}ñcanena G11
		\rdg[wit={J10}]{ākuñcya tena}}
\app{\lem[wit={ceteri}]{taṃ}
		\rdg[wit={C6}]{tu}}
\app{\lem[wit={ceteri},alt={prāhur}]{prāhu\skp{r}}% prāhu P11,V3,V1,J10, °huḥ C6; +J11 s.l.
		\rdg[wit={N19}]{grāhyaṃ}}}%
\pada{\app{\lem[wit={ceteri},alt={mūlabandhaṃ}]{\skm{r }mūlabandhaṃ}% ṃ om. N19
		\rdg[wit={J10}]{mūlabandho}
		\rdg[wit={J5}]{mūlabandhā}
		\rdg[wit={G4}]{mūlo siddhiṃ}}
\app{\lem[wit={N3,G4,G11,N19,V1,J10,V3,Jyo}]{hi}
		\rdg[wit={Gr2,Gr3,P11,C6}]{tu}% = source
		\rdg[wit={J5}]{di}}
	yoginaḥ//\versenr} %\lineom{abc}{J5}\unavbl{V15}
	\\!}
\end{tlg}
\commcite\newpage


\begin{tlg}[hp03_051]
\tl{\app{\lem[nolem]{}
	\rdg[wit={V15},alt={\om}]{\skp{\om}}}%
\pada{\app{\lem[wit={ceteri}]{gudaṃ}
		\rdg[wit={N19}]{gulpha}
		\rdg[wit={C6}]{pārṣṇi°}}
\app{\lem[wit={N3,J5,Gr3,G11,J10,P11,V3,Jyo}]{pārṣṇyā tu} % prob. V19, pāṣṇyā V3; nu P11
		\rdg[wit={N19,V1}]{pārṣṇyā ca}
		\rdg[wit={G4}]{[p]ārṣṇena}
		\rdg[wit={J7}]{pārśnī tu}
		\rdg[wit={N23}]{pādarmyāṃ tu}
		\rdg[wit={C6}]{°nā gudam}}
	\app{\lem[wit={ceteri}]{saṃpīḍya} % pījya N23ac, °pīḍye V3
		\rdg[wit={C6}]{āpīḍya}}}
\pada{\app{\lem[wit={Gr1,Gr2,G11,V1,J10,C6,V3,Jyo},alt={vāyum ā°}]{vāyum ā\skp{°}}
		\rdg[wit={N19,P11}]{vāyunā}
		\rdg[wit={Gr3}]{yonim ā°}}kuñcayed % °ṣed N23ac
		balāt/}\\+}
\tl{
\pada{vāraṃ vāraṃ % 1st vāra V1; 2nd vāra V3,J10
\app{\lem[wit={N3,G11,N19,V1,J10,GrB,Jyo}]{yathā}
		\rdg[wit={J5,Gr2,Gr3}]{tathā}}
	cordhvaṃ} % ṃ om. N23
\pada{samāyāti samīraṇaḥ//\versenr}%\unavbl{V15}
\\!} % °ṇa V3, °ṇāḥ J5
\end{tlg}
\commcite\newpage


\begin{tlg}[hp03_052]
\tl{\app{\lem[nolem]{}
	\rdg[wit={V15},alt={\om}]{\skp{\om}}}%
\pada{prāṇāpānau % prāṇa° V1
	\app{\lem[wit={N23,J7,V19,E2,V1,C6,Jyo}]{nādabindū}
		\rdg[wit={N3,J5,G11,N19,J10,P11,V3}]{nādabindu}}}
\pada{mūlabandhena
\app{\lem[wit={ceteri}]{caikatām}
		\rdg[wit={N19,C6}]{caikatā}
		\rdg[wit={N23}]{cakataṃ}
		\rdg[wit={V3}]{caikataḥ}}/}\\+}
\tl{
\pada{\app{\lem[wit={ceteri}]{gatvā}
		\rdg[wit={J10}]{tato}} yogasya % yo<<ga>>sya P11, yagasya J10ac
\app{\lem[wit={N3,G4,J7,Gr3,G11,V1,C6,V3,Jyo}]{saṃsiddhiṃ}
		\rdg[wit={P11}]{saṃsiddhi}
		\rdg[wit={J5,N23,N19}]{saṃsiddhir}% +C7
		\rdg[wit={J10}]{saṃsiddhyaiḥ}}}
\pada{\app{\lem[wit={J5,V3,Jyo}]{yacchato}
		\rdg[wit={C6}]{yakṣyato}
		\rdg[wit={N3}]{yichato}
		\rdg[wit={P11}]{pracchato}
		\rdg[wit={V19,G11,N19}]{gacchato} % gakṣato V19; +K3,C7
		\rdg[wit={J7}]{gacchate}
		\rdg[wit={N23}]{gacchatā}
		\rdg[wit={V1}]{prāpnoty e°}
		\rdg[wit={J10}]{pamāta}
		\rdg[wit={G4}]{niścayo}
		\rdg[wit={E2}]{kurute}}
\app{\lem[wit={ceteri}]{nātra}
		\rdg[wit={V1}]{°va na}
		\rdg[wit={J10}]{tra na}} saṃśayaḥ//\versenr} % śaṃsayaḥ V19
	%\unavbl{V15}
	\\!} % V19 written in margin pr.m.
\end{tlg}
\commcite\newpage


\begin{tlg}[hp03_053]
\tl{\app{\lem[nolem]{}
	\rdg[wit={V15},alt={\om}]{\skp{\om}}}%
\pada{apānaprāṇa%
\app{\lem[wit={ceteri},alt={°yor aikyaṃ}]{\skp{°}yor aikyaṃ} % +C7; °tor G4, °yaur J5
		\rdg[wit={N23}]{°yor aikya}
		\rdg[wit={J10}]{°yor aikye}
	}}
\pada{\app{\lem[wit={ceteri}]{kṣayo}
		\rdg[wit={G4}]{kṣayaṃ}
		\rdg[wit={N23}]{kṣayān}} % +J11
		mūtrapurīṣayoḥ/}\\+} % mutra N23; °yo V3
\tl{
\pada{yuvā bhavati vṛddho'pi}
\pada{satataṃ
	mūla\app{\lem[wit={ceteri}]{bandhanāt}
		\rdg[wit={V19}]{bandhataḥ}}//\versenr}\myfn{After this verse, \getsiglum{N23} has an additional verse:
	\vspace{2pt minus 1pt}\\
	\devnote{bandhamūlaṃ yena tena tena vighnāṃ nivāritaḥ/ 
	ajarāmaratāṃ yāti yathā pañcamukho haraḥ//}}
	%\unavbl{V15}
	\\!}
\end{tlg}
\commcite%\newpage


\begin{tlg}[hp03_054]
\tl{\app{\lem[nolem]{}
	\rdg[wit={V15},alt={\om}]{\skp{\om}}}%
\pada{\app{\lem[wit={ceteri}]{apāne}
		\rdg[wit={Jyo}]{apāna}
		\rdg[wit={J7,V3}]{apānaṃ}}
\app{\lem[wit={ceteri}]{cordhvage jāte}% cordhage V3
		\rdg[wit={V19}]{cordhvage yāte} % or yāne? V19
		\rdg[wit={J10}]{cordhvam āpāte}
		\rdg[wit={Jyo}]{ūrdhvage jāte}}}
\pada{\app{\lem[wit={Gr2,Gr3,G11,N19,C6}]{saṃprāpte}% = GŚ keep
		\rdg[wit={P11}]{saṃprāptau}
		\rdg[wit={V3}]{saṃyāte}% +F
		\rdg[wit={N3,J5,V1,J10,Jyo}]{prayāte} % °yātai J5; damaged G4 ##
		}
\app{\lem[wit={N3,J7,P11,V3,Jyo}]{vahnimaṇḍalam}% =J7ac
		\rdg[wit={J5,N23,Gr3,G11,N19,V1,C6}]{vahnimaṇḍale} % +J7pc
		\rdg[wit={J10}]{nābhimaṇḍalaṃ}}/}\\+}
\tl{
\pada{\app{\lem[wit={ceteri}]{tadānala}
		\rdg[wit={N19}]{tadānale}
		\rdg[wit={J5,E2,J10}]{tathānala}% +C7
		\rdg[wit={V1}]{tathānale}
	}śikhā dīrghā} % śiṣā J5,V19; dīryā V1
\pada{\app{\lem[wit={N3,J5,Gr2,G11,N19,P11,V3}]{vardhate vāyunāhatā}
		\rdg[wit={C6},alt={°hatāḥ}]{vardhate vāyunāhatāḥ}
		\rdg[wit={G4},alt={°hataḥ}]{vardhate vāyunāhataḥ}
		\rdg[wit={Gr3}]{baṃdhane vāyunāhatā}
		\rdg[wit={J10}]{kriyate vāyunāhatāḥ}
		\rdg[wit={Jyo}]{jāyate vāyunāhatā}
		\rdg[wit={V1}]{vāyunā preritā tathā}}//\versenr}
		%\unavbl{V15}
		\\!}
\end{tlg}
\commcite\newpage


\begin{tlg}[hp03_055]
\tl{\app{\lem[nolem]{}
	\rdg[wit={V15},alt={\om}]{\skp{\om}}}%
\pada{\app{\lem[wit={ceteri}]{tato}
		\rdg[wit={V1}]{yātā}}
\app{\lem[wit={G11,C6}]{yātau}
		\rdg[wit={J5}]{yāttau}
		\rdg[wit={V1,P11,Jyo}]{yāto}% +M1
		\rdg[wit={J10}]{yāte}
		\rdg[wit={G4}]{yaṃtā}
		\rdg[wit={N3}]{yāmau}
		\rdg[wit={J7,Gr3}]{jātau}
		\rdg[wit={N23}]{jātā}
		\rdg[wit={V3}]{jāto}
		\rdg[wit={N19}]{vahnim}}\marmas
\app{\lem[wit={J7,Gr3,G11,V1,P11,Jyo}]{vahnyapānau}% °panau J7
		\rdg[wit={N3}]{vahnipānau}
		\rdg[wit={J5}]{vahnipātau}
		\rdg[wit={G4}]{[m]ahnyapāne}
		\rdg[wit={J10}]{vahniyonau}
		\rdg[wit={C6}]{bāhyapānau}
		\rdg[wit={N23}]{baṃdhapānau}
		\rdg[wit={V3}]{vardhapānai}
		\rdg[wit={N19}]{apānai ca}}}
\pada{\app{\lem[wit={G4,J7,G11,N19,GrB,Jyo}]{prāṇam uṣṇa}% praṇam P11
		\rdg[wit={N23}]{prāṇamura}
		\rdg[wit={Gr3}]{prāṇamukta}
		\rdg[wit={N3,J5,V1,J10}]{prāṇamūla}}%
\app{\lem[wit={ceteri}]{svarūpakam}
		\rdg[wit={G11}]{surūpakaṃ}
		\rdg[wit={J10}]{svarūpakaḥ}
		\rdg[wit={V1}]{svarūpakau}
		}/}\\+} % °rūpaṃke N23
\tl{
\pada{\app{\lem[wit={Gr2,Gr3,G11,V1,P11,Jyo}]{tenātyanta}
		\rdg[wit={N3}]{tenātyantaṃ}
		\rdg[wit={V3}]{tenābhyanta}
		\rdg[wit={J10}]{tenābhyantaḥ}
		\rdg[wit={C6}]{tenāyaṃna}
		\rdg[wit={J5}]{tenotyataṃ}
		\rdg[wit={N19}]{tailābhyaṃtaḥ}}%
\app{\lem[wit={ceteri}]{pradīptas tu} % °diptas V3
		\rdg[wit={V1}]{pradīpas tu}
		\rdg[wit={J5}]{pradāyas tu}
		\rdg[wit={G11}]{pradīptāsau}% pradīpto'sau G5,F
		\rdg[wit={N19}]{pradīpāsau}
		}}
\pada{\app{\lem[wit={N3,J5,G11,P11,V3}]{jvalano dehajas tadā}% nehajas V3
		\rdg[wit={C6}]{jvalato dehatas tadā}
		\rdg[wit={Gr2,Gr3,N19,V1,Jyo}]{jvalano dehajas tathā} % jvalanā N23
		\rdg[wit={J10}]{kuṃto dehakṣayas tadā}}//\versenr}
		%\unavbl{V15}
		\\!}
\end{tlg}
\commcite\newpage

% V1 adds here "vāmanāya namaḥ" at the end of the folio. The new folio begins with tena.


\begin{tlg}[hp03_056]
\tl{\app{\lem[nolem]{}
	\rdg[wit={V15},alt={\om}]{\skp{\om}}}%
\pada{tena kuṇḍalinī suptā} % kuṃḍalanī N3,J5
\pada{\app{\lem[wit={ceteri}]{saṃtaptā}% sataptā P11
		\rdg[wit={Gr3,N19}]{satataṃ}}
\app{\lem[wit={N3,J5,Gr2,J10,C6,Jyo}]{saṃprabudhyate}% ddhy J7, yudhyate N23; +G5
		\rdg[wit={V1}]{saṃprabudhyati}
		\rdg[wit={N19}]{saṃprabodhyate}
		\rdg[wit={P11}]{samabuddhyate}
		\rdg[wit={V3}]{sa prabudhyate}
		\rdg[wit={E2}]{sā prabodhyate}% +C7, °budhyate K3
		\rdg[wit={V19}]{sānubodhyate}
		\rdg[wit={G11}]{buddhyate tadā}
		}/}\\+}
\tl{
\pada{daṇḍāhatā bhujaṅgīva} % hatya bhū° N23; °hatabhujaṃgevā G4, °gīca V3
\pada{\app{\lem[wit={N3,G4,G11,V1,J10,P11,Jyo}]{niśvasya}% +J11
		\rdg[wit={J5}]{niśvāsya}
		\rdg[wit={C6}]{niḥśvasya}
		\rdg[wit={V3}]{viśvasya}
		\rdg[wit={Gr2,V19,N19}]{niścitam}
		\rdg[wit={E2}]{niścayād}% +C7, °cayam K3
		}\marmas
\app{\lem[wit={ceteri}]{ṛjutāṃ vrajet} % ṛjutā P11, ṛjvatāṃ V3
		\rdg[wit={N3}]{rujutāṃ vṛjet}
		\rdg[wit={J5}]{rujanāṃ vrajet}
		\rdg[wit={J10}]{rijutām iyāt}}//\versenr}
		%\unavbl{V15}
		\\!}
\end{tlg}
\commcite%\newpage


\begin{tlg}[hp03_057]
\tl{\app{\lem[nolem]{}
	\rdg[wit={V15},alt={\om}]{\skp{\om}}}%
\pada{\app{\lem[nolem]{\skp{pāda a}}
	\rdg[wit={Gr3},alt={\om}]{\skp{\om}}}%
bilaṃ % cilaṃ N23, balaṃ G11
\app{\lem[wit={Gr1,J7,G11,N19,V1,Jyo}]{praviṣṭeva}
		\rdg[wit={N23,P11,C6}]{praviṣṭe ca}
		\rdg[wit={V3}]{praviṣṭaṃ ca}
		\rdg[wit={J10}]{praviṣṭaś ca}
		%\rdg[wit={Gr3},alt={\om}]{\skp{\om}}
		}
\app{\lem[wit={ceteri}]{tato}
		\rdg[wit={N23}]{to}
		%\rdg[wit={Gr3},alt={\om}]{\skp{\om}}
		}}
\pada{\app{\lem[nolem]{\skp{pāda b}}
	\rdg[wit={Gr3},alt={\om}]{\skp{\om}}}%
\app{\lem[wit={ceteri}]{brahma}
		\rdg[wit={N23}]{tha\,\_}
		%\rdg[wit={Gr3},alt={\om}]{\skp{\om}}
		}%
\app{\lem[wit={ceteri}]{nāḍyantaraṃ}% nājya° N23
		\rdg[wit={J5,C6}]{nāḍyāntaraṃ}
		\rdg[wit={J10}]{nāḍyantare}
		%\rdg[wit={Gr3},alt={\om}]{\skp{\om}}
		} vrajet/}
%	\lineom{ab}{Gr3}
	\\+}
\tl{
\pada{tasmā%n
\app{\lem[wit={ceteri},alt={nityaṃ}]{\skm{n }nityaṃ}% ṃ om. N23
		\rdg[wit={N19}]{nityo}} mūlabandhaḥ} % ḥ om. J5,P11
\pada{kartavyo yogibhiḥ sadā//\versenr}
%\unavbl{V15}
\\!}
\end{tlg}

\commcite

%\begin{altpostmula}[hp03_057p]
%\app{\lem[nolem]{}
%	\rdg[wit={G11}]{\incl}}%
%	iti mūlabandhaḥ/
%\end{altpostmula}

\newpage

\begin{ava}[hp03_058a]
\app{\lem[wit={N3}]{athoḍḍīyāṇam}
		\rdg[wit={P11}]{athoḍīyāṇāṃ}
		\rdg[wit={V3}]{athoḍḍiyāṇaṃ}
		\rdg[wit={J5}]{athoḍiyāṇaṃ}
		\rdg[wit={C6}]{athoḍiyānaṃ}
		\rdg[wit={G11}]{iti mūlabandhaḥ | atha oḍyāṇaṃ}
		\rdg[wit={V1}]{athoḍyāṇabaṃdhaḥ}% +F
		\rdg[wit={N19}]{atha uḍḍīyāṇabandhaḥ}
		\rdg[wit={E2}]{atha uḍḍīyānabandhaḥ}
		\rdg[wit={N23}]{atha uḍḍiyānabandhaḥ}
		\rdg[wit={J10}]{atha uḍḍiyānaṃ bandhaḥ}
		\rdg[wit={Jyo}]{atha uḍḍīyanabandhaḥ}
		\rdg[wit={J7}]{uḍḍiyāṇaṃ bandhaḥ}
		\rdg[wit={V19,V15},alt={\om}]{\skp{\om}}}/%
%	\myfn{\getsiglum{V3,Jyo} have this section before the Mūlabandha.}
\end{ava}

\begin{tlg}[hp03_058]
\tl{\app{\lem[nolem]{}
	\rdg[wit={V15},alt={\om}]{\skp{\om}}}%
\pada{\app{\lem[wit={N3,J7,V19,C6,V3,Jyo}]{baddho}% +K3
		\rdg[wit={J5,E2,G11,N19,P11}]{bandho}% +C7
		\rdg[wit={V1,J10}]{ūrdhvo}
		\rdg[wit={N23}]{vidrā}}
\app{\lem[wit={ceteri}]{yena suṣumṇāyāṃ} % suṣumtāyaṃ N23; °yā V3
		\rdg[wit={J5}]{yoni suṣumnāyāṃ}
		\rdg[wit={J10}]{kṣitaḥ suṣumṇāyāḥ}}}
\pada{\app{\lem[wit={ceteri},alt={prāṇas}]{prāṇa\skp{s}}% vrāṇas N23
		\rdg[wit={N19,V1,C6}]{prāṇam}
		\rdg[wit={E2}]{prāṇa}}% 
\app{\lem[wit={N3,J7,V19,J10,Jyo},alt={tūḍḍīyate}]{\skm{s }tūḍḍīyate}
		\rdg[wit={N3,J7}]{tuḍḍīyate}
		\rdg[wit={J5}]{taḍḍīyate}
		\rdg[wit={G11}]{sūḍḍīyate}
		\rdg[wit={V3}]{tūḍiyate}
		\rdg[wit={N23}]{tudīyate}
		\rdg[wit={E2,N19}]{uḍḍīyate}
		\rdg[wit={P11}]{kṛḍīyate}
		\rdg[wit={C6}]{uḍiyate}
		\rdg[wit={V1}]{uḍyayate}}
		yataḥ/}\\+}
\tl{
\pada{\app{\lem[wit={ceteri},alt={tasmād}]{tasmā\skp{d}}% rasmād P11
		\rdg[wit={J7}]{tasmātu}% sic
		\rdg[wit={J10}]{tasmāc ca}}%
\app{\lem[wit={G11},post=\texteng{\emph{m.c.\@}},alt={uḍḍīyaṇākhyo}]{\skm{d }uḍḍīyaṇākhyo}
		\rdg[wit={Gr3,Jyo}]{uḍḍīyanākhyo}
		\rdg[wit={N3}]{uḍḍīyanākhye}
		\rdg[wit={N19}]{uḍḍīyāṇākhyo}
		\rdg[wit={V1}]{uḍḍiyāṇākhyo}
		\rdg[wit={J7,J10}]{uḍḍiyānākhyo}
		\rdg[wit={J5,P11}]{uḍiyaṇākhyo}
		\rdg[wit={V3}]{uḍiyāṇākhye}
		\rdg[wit={C6}]{uḍiyānākhyaṃ}
		\rdg[wit={N23}]{uddiyānākhyo}
		}\marma%
\app{\lem[wit={ceteri}]{'yaṃ}
		\rdg[wit={C6}]{tad}
		\rdg[wit={J10},alt={\om}]{\skp{\om}}}}
\pada{yogibhiḥ % °bhi V19
\app{\lem[wit={ceteri}]{samudāhṛtaḥ}
		\rdg[wit={N19,C6,V3}]{samudāhṛtaṃ}}//\versenr}%
		\myfn{Before this verse, \getsiglum{G11} has \emph{Gorakṣaśataka} 57cd.}
		%\unavbl{V15}
		\\!}
\end{tlg}

\avacite{58a}
\commcite\newpage


\begin{tlg}[hp03_059]
\tl{\app{\lem[nolem]{}
	\rdg[wit={V15},alt={\om}]{\skp{\om}}}%
\pada{\app{\lem[wit={N3,J7,G11,P11}]{uḍyāṇaṃ}% +M1,VM(Ms T) ##
		\rdg[wit={J5}]{uḍyāṇāṃ}
		\rdg[wit={C6}]{uḍyānaṃ}
		\rdg[wit={V3}]{uḍīṇaṃ}
		\rdg[wit={Gr3,Jyo}]{uḍḍīnaṃ}
		\rdg[wit={G4,V1}]{uḍḍiyāṇaṃ}
		\rdg[wit={N23,J10}]{uḍḍiyānaṃ}
		\rdg[wit={N19}]{uḍḍīyāṇaṃ}
		}
\app{\lem[wit={ceteri},alt={kurute yasmād}]{kurute yasmā\skp{d}}
		\rdg[wit={J7}]{kṛyate yasmād}
		\rdg[wit={N19}]{kṛte yasmād}
		\rdg[wit={G4}]{tu kurute}}d}
\pada{a\app{\lem[wit={Gr3,G11,N19,V1,J10,C6,V3,Jyo},alt={aviśrāntaṃ}]{\skp{a}viśrāntaṃ}
		\rdg[wit={N3,J5,P11}]{aviśrānta}% VM(Ms T)
		\rdg[wit={J7}]{aviśrānto}% +F,J11
		\rdg[wit={N23}]{aviśrāṃtā}
		\rdg[wit={G4}]{khaviśrāṃtā}} 
		mahākhagaḥ/}\\+} % ṣagaḥ J5, khaga V3, khagam E2
\tl{
\pada{\app{\lem[wit={N3,N19,V1}]{uḍḍīyāṇaṃ}
		\rdg[wit={Gr3}]{uḍḍīyānaṃ}
		\rdg[wit={J5}]{uḍīyāṇaṃ}
		\rdg[wit={G11}]{oḍḍīyāṇaṃ}
		\rdg[wit={J7,J10}]{uḍḍiyānaṃ}
		\rdg[wit={N23}]{uddiyānaṃ}
		\rdg[wit={P11,V3}]{uḍiyāṇaṃ}
		\rdg[wit={C6}]{uḍiyānaṃ}
		\rdg[wit={Jyo}]{uḍḍīyanaṃ}
		} 
ta\app{\lem[wit={ceteri},alt={eva}]{\skm{d }eva}% vadeva J5
		\rdg[wit={V19}]{evaṃ}
		\rdg[wit={N19}]{evaḥ}} syā}% syā V3, sya J5
\pada{\app{\lem[wit={ceteri},alt={tatra}]{\skm{t }tatra}
		\rdg[wit={J10}]{kṣetra}
		\rdg[wit={Gr3}]{mūla}}\marmas 
\app{\lem[wit={ceteri}]{bandho}
		\rdg[wit={J5}]{vedho}}
\app{\lem[wit={J5,J7,C6}]{vidhīyate}% +J11
		\rdg[wit={N3,Gr3,G11,N19,V1,J10,P11,V3,Jyo}]{bhidhīyate} % bhi[dh]īyate V19
		\rdg[wit={N23}]{nigadyate}}//\versenr}
		%\unavbl{V15}
		\\!}
\end{tlg}
\commcite\newpage


\begin{tlg}[hp03_060]
\tl{\app{\lem[nolem]{}
	\rdg[wit={V15},alt={\om}]{\skp{\om}}}%
\pada{% nested! MD
\app{\lem[alt={udare ... tāṇaṃ}]{\skp{udare ... tāṇaṃ}}
	\rdg[wit={G11}]{paścimaṃ tāṇam udare}
	}%
\app{\lem[wit={ceteri}]{udare}
		\rdg[wit={V3}]{udarāt}
		}
\app{\lem[wit={J5,J7,E2,V1,P11,C6,Jyo}]{paścimaṃ}% +C7
		\rdg[wit={N3,N23,J10}]{paścima}
		\rdg[wit={V19,N19,V3}]{paścime}% +K3
		}
\app{\lem[wit={N3,J5,N19}]{tāṇaṃ}
		\rdg[wit={Gr2,Gr3,V1,J10,C6,Jyo}]{tānaṃ}
		\rdg[wit={P11}]{tālaṃ}% +K3
		\rdg[wit={V3}]{bhāge}
		}}\marmas
\pada{nābhe%r nābhed J7
\app{\lem[wit={ceteri},alt={ūrdhvaṃ}]{\skm{r }ūrdhvaṃ}
		\rdg[wit={J10}]{ūrdhve}} % °rddha-akāra° N23
\app{\lem[wit={ceteri}]{ca}
		\rdg[wit={N19,J10}]{tu}} kārayet/}\\+}
\tl{
\pada{\app{\lem[nolem]{\skp{pāda c}}
	\rdg[wit={V3},alt={\om}]{\skp{\om}}}%
\app{\lem[wit={N3,N19}]{uḍḍīyāṇo}
		\rdg[wit={E2}]{uḍḍīyāno}% +K3,C7
		\rdg[wit={G4,G11,V1}]{uḍḍiyāṇo}
		\rdg[wit={Gr2,V19,J10}]{uḍḍiyāno}
		\rdg[wit={P11}]{uḍiyāṇo}
		\rdg[wit={C6}]{uḍiyāno}
		\rdg[wit={J5}]{uḍḍāṇo}
		\rdg[wit={Jyo}]{uḍḍīyano}
		} 
\app{\lem[wit={ceteri}]{hy asau}
		\rdg[wit={V19}]{hy ayaṃ}% +K3
		\rdg[wit={E2}]{hy asam}% +C7
		} bandho} % baṃdhau J5, baṃdhā N23
\pada{\app{\lem[nolem]{\skp{pāda d}}
	\rdg[wit={V3},alt={\om}]{\skp{\om}}}%
mṛtyumātaṅgakesarī//\versenr}% matta for mṛtyu J5; keśarī N19,J10
%		\lineom{cd}{V3}\unavbl{V15}
		\\!}
\end{tlg}
\commcite\newpage


\begin{tlg}[hp03_061]
\tl{\app{\lem[nolem]{}
	\rdg[wit={V15},alt={\om}]{\skp{\om}}}%
\pada{\app{\lem[wit={N3,N19}]{uḍḍīyāṇaṃ}
		\rdg[wit={E2}]{uḍḍīyānaṃ}
		\rdg[wit={J5,G11,V1,P11}]{uḍḍiyāṇaṃ}
		\rdg[wit={J7,V19,J10}]{uḍḍiyānaṃ}
		\rdg[wit={N23}]{uddiyānaṃ}
		\rdg[wit={V3}]{uḍiyāṇaṃ}
		\rdg[wit={C6}]{uḍiyānaṃ}
		\rdg[wit={Jyo}]{uḍḍīyanaṃ}
		} tu
\app{\lem[wit={ceteri}]{sahajaṃ}
		\rdg[wit={P11}]{sahasaṃ}
		\rdg[wit={J7}]{yaḥ sahate}}}
\pada{\app{\lem[wit={ceteri}]{guruṇā}
		\rdg[wit={V3}]{gurūṇāṃ}} kathitaṃ
\app{\lem[wit={ceteri}]{sadā}% N3,J5,G11,N19,V1,J10,GrB,Jyo
		\rdg[wit={Gr2,Gr3}]{yathā}}\marma/}\\+}
\tl{
\pada{\app{\lem[wit={ceteri},alt={abhyased/-set}]{abhyase\skp{d}}%
%	\rdg[wit={N3,N19,V1}]{abhyased}
		\rdg[wit={N23}]{abhyāsen}
		\rdg[wit={J5}]{abhyāsyed}
		\rdg[wit={V3}]{abhyāsāt}
		\rdg[wit={C6}]{abhyāsa°}}%
\app{\lem[wit={J5,G11},alt={astatandras tu}]{\skm{d }astatandras tu}% =source,+M3
		\rdg[wit={N3}]{astatadras tu}
		\rdg[wit={P11}]{asvataṃtras tu}
		\rdg[wit={N19}]{asya taṃtrasya}
		\rdg[wit={C6}]{°taḥ svatantras tu}
		\rdg[wit={J7,Gr3}]{tad atandras tu}
		\rdg[wit={N23}]{na taṃdras tu\,\_}
		\rdg[wit={V1}]{yo hy atandras tu}
		\rdg[wit={J10,V3,Jyo}]{satataṃ yas tu}}\marma} % +M4
\pada{vṛddho'pi % vṛddhā N23
\app{\lem[wit={N3,J5,G11,N19,V1,J10,P11,V3}]{taruṇo bhavet}
		\rdg[wit={Gr2,Gr3,C6,Jyo}]{taruṇāyate}}//\versenr}
		%\unavbl{V15}
		\\!}
\end{tlg}
\commcite\newpage


\begin{tlg}[hp03_062]
\tl{\app{\lem[nolem]{}
	\rdg[wit={V15},alt={\om}]{\skp{\om}}}%
\pada{\app{\lem[alt={\ante nābher \add},nosep]{\skp{\ante nābher \add}}
		\rdg[wit={C6}]{pāṭhāntaram}}%
	nābher ūrdhva%m % nābhed J7; ūrdham V3
\app{\lem[wit={ceteri},alt={adhaś cāpi}]{\skm{m }adhaś cāpi}% aṃdhaś N23
		\rdg[wit={J5}]{adhastāpi}
		\rdg[wit={Gr3}]{adho vāpi}
		\rdg[wit={V1}]{adhaḥkāya}
		\rdg[wit={C6}]{avasthāpya}}}
\pada{\app{\lem[wit={G4,G11,N19,V1,P11,V3}]{tāṇaṃ}
		\rdg[wit={Gr2,Gr3,J10,C6,Jyo}]{tānaṃ}
		\rdg[wit={N3}]{tāpyaṃ}
		\rdg[wit={J5}]{tāruṇaṃ}
		}
	kuryā%t
\app{\lem[wit={ceteri},alt={prayatnataḥ}]{\skm{t }prayatnataḥ}% ḥ om. N23
		\rdg[wit={J10}]{ca yatnataḥ}}/}\\+}
\tl{
\pada{\app{\lem[wit={ceteri},alt={ṣaṇmāsam}]{ṣaṇmāsa\skp{m}}
		\rdg[wit={G4}]{ṣaṇmāsām}
		\rdg[wit={J5}]{ṣaṇmāsād}
		\rdg[wit={V1,J10}]{yogī sam°}}%
\app{\lem[wit={N3,Gr2,J10,P11},alt={abhyasan}]{\skm{m }abhyasa\skp{n}}
		\rdg[wit={G4,Gr3,G11,N19,V1,V3,Jyo}]{abhyasen}% +F
%		\rdg[wit={V19}]{abhyaseni}% nityaṃ ac > mṛtyuṃ pc
		\rdg[wit={C6}]{ca samabhyān}
		\rdg[wit={J5}]{vau mahā}}%
\app{\lem[wit={ceteri},alt={mṛtyuṃ}]{\skm{n }mṛtyuṃ} % mṛtyu J5, matyu N23
		\rdg[wit={C6},alt={\om}]{\skp{\om}}}}
\pada{\app{\lem[wit={ceteri}]{jayaty eva na saṃśayaḥ}% yaryaty P11
%		\rdg[wit={J5}]{jayeva na saṃśayaḥ}
		\rdg[wit={C6}]{mūlaṃ jayaty asaṃśayaḥ}}//\versenr}
		%\unavbl{V15}
		\\!}
\end{tlg}
\commcite\newpage


\begin{tlg}[hp03_063]
\tl{\app{\lem[nolem]{}
	\rdg[wit={V15},alt={\om}]{\skp{\om}}%
	\rdg[wit={Jyo},alt=\textapp{found after \ref{vitasti}}]{\skp{found after 3.95*2}}}%
\pada{sati % sa[t]i V19
\app{\lem[wit={ceteri}]{vajrāsane}% va om. J5, vrajrā° E2
		\rdg[wit={N23}]{vajrāsanau}
		\rdg[wit={N3,P11}]{vajrāsanaṃ}
		} pādau\marma} % +G7,G11; jānu M3,G5
\pada{\app{\lem[wit={ceteri}]{karābhyāṃ}% °bhyā om. V1
		\rdg[wit={N23}]{karā\,\_}
		}
	\app{\lem[wit={E2,G11,N19,P11,C6,Jyo},alt={dhārayed}]{dhāraye\skp{d}}
		\rdg[wit={J7,V19,V3}]{dhāraye}
		\rdg[wit={N23}]{sandhāraye}
		\rdg[wit={V1}]{dhārayaṃ}
		\rdg[wit={J10}]{dhārayad}
		\rdg[wit={N3,J5}]{kāraye}
		}%
	\app{\lem[wit={ceteri},alt={dṛḍham}]{\skm{d }dṛḍham}
		\rdg[wit={N23}]{dṛḍhe}
		}/}\\+}
\tl{
\pada{gulpha\app{\lem[wit={ceteri}]{deśa}% desa V3
		\rdg[wit={J5,E2,G11,N19}]{deśe}
		\rdg[wit={N3}]{deśaṃ}}%
\app{\lem[wit={N3,Gr2,G11,V1,J10,GrB,Jyo}]{samīpe ca}
		\rdg[wit={E2,N19}]{samīpaṃ ca}% +K3,C7
		\rdg[wit={V19}]{samīpaṃ tu}
		\rdg[wit={J5}]{samityeva}}}\marmas
\pada{\app{\lem[wit={ceteri}]{kandaṃ}
		\rdg[wit={V19}]{kaṃdhaṃ}
		\rdg[wit={J5,P11}]{kaṃṭhaṃ}
		}
\app{\lem[wit={ceteri}]{tatra}% tatraḥ J5
		\rdg[wit={J10,V3}]{tacca}
		\rdg[wit={N3}]{tava}
		\rdg[wit={V1}]{tasya}}
\app{\lem[wit={G4,Gr2,Gr3,G11,N19,P11,C6,Jyo}]{prapīḍayet}% pīdayet N23
		\rdg[wit={N3,V1,J10,V3}]{prapīḍyate}
		\rdg[wit={J5}]{pradāyate}}//\versenr}
	%\anm{after \ref{vitasti} in \getsiglum{Jyo}}
	\label{III64}%\unavbl{V15}
	\\!}
\end{tlg}
\commcite\newpage


\begin{tlg}[hp03_064]
\tl{\app{\lem[nolem]{}
	\rdg[wit={V15,Jyo},alt={\om}]{\skp{\om}}}%
\pada{\app{\lem[wit={J5,G4,G11,V1,P11},alt={paścimaṃ tāṇam}]{paścimaṃ tāṇa\skp{m}}
		\rdg[wit={Gr2,Gr3,N19,J10,C6}]{paścimaṃ tānam}
		\rdg[wit={N3,V3}]{paścimatāṇam}}%
\app{\lem[wit={ceteri},alt={udare}]{\skm{m }udare}% Gr1,J7,Gr3,G11,GrB
		\rdg[wit={N23}]{udara}
		\rdg[wit={N19}]{udaraṃ}
		\rdg[wit={V1,J10}]{upari}}}
\pada{\app{\lem[wit={ceteri},alt={kārayed}]{kāraye\skp{d}}
		\rdg[wit={J10}]{pīḍayed}}%
\app{\lem[wit={Gr1,J7,V1,J10,GrB},alt={dhṛdaye gale}]{\skm{d }dhṛdaye gale} % kāraye hṛdaye V3, kārayed-hṛ° V1; +F,J11
		\rdg[wit={G11}]{dhṛtaye gale}
		\rdg[wit={N23}]{dhṛdaye gataiḥ}
		\rdg[wit={V19}]{udare hṛdi}
		\rdg[wit={E2}]{cibukaṃ hṛdi}% +C7, hṛdā K3
		\rdg[wit={N19}]{vṛddhidaṃ śanaiḥ}}\marma/}\\+}
\tl{
\pada{śanaiḥ % śanai P11,V19
\app{\lem[wit={ceteri}]{śanair yathā} % śanai P11
		\rdg[wit={N23},alt={\om}]{\skp{\om}}}
\app{\lem[wit={N3,J5,Gr3,G11,V1,P11,V3},alt={prāṇas}]{prāṇa\skp{s}}
		\rdg[wit={Gr2}]{prāṇās}% praṇos? J7pc
		\rdg[wit={G4}]{strāṇas}
		\rdg[wit={N19,C6}]{prāṇaṃ}
		\rdg[wit={J10}]{prāṇo}}}% 
\pada{\app{\lem[wit={N3,J7,V19,G11,N19,P11,C6},alt={tunda}]{\skm{s }tunda}% +K3
		\rdg[wit={V1,V3}]{tuda}
		\rdg[wit={N23}]{taṃda}
		\rdg[wit={E2}]{tadā}% +C7
		\rdg[wit={J5}]{kaṃda}% misreading of °stuṃda as °skaṃda
		\rdg[wit={J10}]{nāḍī}}%
\app{\lem[wit={N3,Gr2,Gr3,G11,V1,P11}]{saṃdhiṃ}
		\rdg[wit={J5,N19,J10,V3}]{saṃdhi}
		\rdg[wit={C6}]{siddhiṃ}}
\app{\lem[wit={ceteri}]{na}% N3,J5,Gr2,G11,N19,V1,GrB
		\rdg[wit={Gr3}]{ca}
		\rdg[wit={J10}]{ni°}} gacchati\marma//\versenr}%
	\myfn{After this verse, \getsiglum{G11} has \emph{Gorakṣaśataka} 61ab.}
%		\NotIn{Jyo}\unavbl{V15}
		\\!}
\end{tlg}
\commcite\newpage


\begin{tlg}[hp03_065]
\tl{\app{\lem[nolem]{}
	\rdg[wit={V15},alt={\om}]{\skp{\om}}}%
\pada{sarveṣām eva bandhānām} % caiva C6, badhīnām N23
\pada{u\app{\lem[wit={ceteri},alt={uttamo}]{\skp{u}ttamo} % urttamau N23
		\rdg[wit={N19}]{uttamaṃ}}
\app{\lem[wit={J5,G11,V1},post=\texteng{\emph{m.c.\@}}]{hy uḍḍiyāṇakaḥ}
		\rdg[wit={N3}]{hy uḍḍīyāṇakaḥ}
		\rdg[wit={Gr2,Gr3,J10,Jyo}]{hy uḍḍiyānakaḥ}% udvi° N23
		\rdg[wit={P11,V3}]{hy uḍiyāṇakaḥ}
		\rdg[wit={C6}]{hy uḍiyānakaḥ}
		\rdg[wit={N19}]{hy uḍḍīyāṇakaṃ}
		}/}\\+}
\tl{
\pada{\app{\lem[wit={N3,N19}]{uḍḍīyāṇe}
		\rdg[wit={E2}]{uḍḍīyāne}% +K3,C7
		\rdg[wit={G11,V1}]{uḍḍiyāṇe}
		\rdg[wit={Gr2,V19,J10,Jyo}]{uḍḍiyāne}% udviyāna N23
		\rdg[wit={P11,V3}]{uḍiyāṇe} % uḍiṣvāṇe P11
		\rdg[wit={C6}]{uḍiyāne}
		\rdg[wit={J5}]{uḍḍayāṇe}
		}
\app{\lem[wit={ceteri}]{dṛḍhe}
		\rdg[wit={Gr2,Gr3}]{kṛte}} % kṛta N23
	\app{\lem[wit={ceteri}]{bandhe} % baṃdha N23
		\rdg[wit={C6}]{baddhe}
		\rdg[wit={J5}]{jāte}}}
\pada{\app{\lem[wit={N3,J5,N19,J10,P11,V3,Jyo}]{muktiḥ}
		\rdg[wit={G11}]{mukti}
		\rdg[wit={V1}]{muktiṃ}
		\rdg[wit={Gr2,Gr3,C6}]{mūlaṃ}}
\app{\lem[wit={N3,J5,G11,N19,V1,P11,V3,Jyo}]{svābhāvikī}
		\rdg[wit={J10}]{svābhāvakī}
		\rdg[wit={G4}]{svābhāvikir}
		\rdg[wit={Gr3,C6}]{svābhāvikaṃ}
		\rdg[wit={J7}]{svabhāvikaṃ}
		\rdg[wit={N23}]{bhāvikaṃ}} bhavet//\versenr}
	\label{III66}%\unavbl{V15}
	\\!}
\end{tlg}
\commcite\newpage


\begin{ava}[hp03_066a]
\app{\lem[resp=emend]{atha jālandharaḥ}
		\rdg[wit={P11}]{atha jālaṃdhara}
		\rdg[wit={V3}]{atha jālaṃdharaṃ}
		\rdg[wit={N3}]{atha jālāṃdharaḥ}
		\rdg[wit={J5}]{atha jalaṃdhara}
		\rdg[wit={G11}]{atha jālāntaraḥ}
		\rdg[wit={E2,V1,J10,C6,Jyo}]{atha jālaṃdharabandhaḥ}% +C7
		\rdg[wit={N19}]{atha jālaṃdharībaṃdhaḥ}
		\rdg[wit={N23}]{atha nāśaṃdharabaṃdhaḥ}
		\rdg[wit={J7}]{jālaṃdharabandhaḥ}% +K3
		\rdg[wit={V19,V15},alt={\om}]{\skp{\om}}}/
\end{ava}

% V15 resumes with hṛdaye in Pada a. Ca. 18 verses are omitted.
\begin{tlg}[hp03_066]
\tl{
\pada{\app{\lem[wit={ceteri}]{kaṇṭham ākuñcya}
	\rdg[wit={V15},alt={\om}]{\skp{\om}}} 
	hṛdaye} % ākuṃca J5, ākuṃci V3, ākucya N23
\pada{\app{\lem[wit={ceteri},alt={sthāpayed/c}]{sthāpaye\skp{d/c}}
	\rdg[wit={N23},alt={\om}]{\skp{\om}}}% °ye V19,Gr1,V3,J10, °yet* N19, °yed V15
\app{\lem[wit={J5,G4,Gr3,V1,J10,P11,V3},alt={dṛḍham icchayā}]{\skm{d }dṛḍham icchayā}% +G7,M1,M3?
		\rdg[wit={N3}]{dṛḍham īchayā}
		\rdg[wit={N19}]{dṛḍham icchatā}
		\rdg[wit={G11}]{dṛḍhaniścayā}% dṛḍhamūrcchayā G5
		\rdg[wit={V15}]{dṛḍhaniścayāt}
		\rdg[wit={J7,C6,Jyo}]{cibukaṃ dṛḍham}% +G3,G9!
		\rdg[wit={N23},alt={\om}]{\skp{\om}}}/}\\+}
%	\anm{\getsiglum{V15} resumes}
\tl{
\pada{bandho % baṃdha J10
	\app{\lem[wit={ceteri}]{jālandharā}
	\rdg[wit={N3,G11,V3}]{jālāṃdharā}
	}khyo'yam} % °ākṣo N19, °ākṣā J5
\pada{a\app{\lem[wit={V15},alt={amṛtāvyayakārakaḥ}]{\skp{a}mṛtāvyayakārakaḥ}% Marmasthāna
		\rdg[wit={N3}]{amṛtāvayakārakaḥ}
		\rdg[wit={G4}]{amṛtāvyaya\,+\,+\,+}
		\rdg[wit={N19,P11,V3}]{amṛtavyayakārakaḥ} % kāraka V3
		\rdg[wit={J5}]{amṛtākhyopakārakaḥ}
		\rdg[wit={Gr3}]{amṛtākṣayakārakaḥ}
		\rdg[wit={N23}]{mṛtyor mṛtyuḥ paro mṛtaḥ}
		\rdg[wit={J7}]{mṛtyor mṛtyuḥ paro mataḥ}
		\rdg[wit={C6}]{mṛtyumātaṃgakesarī} % mṛtyumṛtyukaro mataḥ G3,G9
		\rdg[wit={G11}]{mṛtyuñjayakaro mataḥ}% mṛtyunāśakaro mataḥ M3
		\rdg[wit={V1,J10,Jyo}]{jarāmṛtyuvināśakaḥ}
		}//\versenr}\label{Jaala1}%
		\myfn{Before this verse, \getsiglum{G11} has \emph{Gorakṣaśataka} 61cd--62ab.}
		\\!}
\end{tlg}

\avacite{66a}
\commcite\newpage


\begin{tlg}[hp03_067]
\tl{
\pada{\app{\lem[wit={N3,N19,V15,V1,J10,GrB,Jyo}]{badhnāti hi} % baddhāti N19; +G5
		\rdg[wit={N23}]{badhnāti ha}
		\rdg[wit={J7,Gr3,G11}]{badhnātīha}
		\rdg[wit={J5}]{badhnāti}
	}
\app{\lem[wit={J7,V19,J10,C6,Jyo}]{śirā}
		\rdg[wit={J5,N23,E2,N19,V15,V1,P11,V3}]{śiro}% +K3,C7
		\rdg[wit={N3}]{śilā˟}
		\rdg[wit={G11}]{ratīrā}}%
\app{\lem[wit={ceteri},alt={jālam}]{jāla\skp{m}}
		\rdg[wit={V3}]{jālāṃ}}}%
\pada{\app{\lem[wit={ceteri},alt={adhogāmi}]{\skm{m }adhogāmi}
		\rdg[wit={N23}]{adhogāmī}
		\rdg[wit={V3}]{madhyegāmi}
		\rdg[wit={V1}]{nādhāyāti}
		}nabhojalam/}\\+} % najo N23ac
\tl{
\pada{tato 
	\app{\lem[wit={ceteri}]{jālandharo}% +V3; °dharā V15
	\rdg[wit={N3}]{jālāṃdharo}
	} bandhaḥ} % baṃdha P11,V3, baṃdho C6,G11
\pada{\app{\lem[wit={N3,J7,G11,N19,V15,GrB,Jyo}]{kaṇṭha}
		\rdg[wit={J5,N23,Gr3,V1,J10}]{kaṇṭhe}}%
	duḥkhaughanāśanaḥ//\versenr}\\!} % duḥkho N23,G11; nāśanaṃ V3
\end{tlg}
\commcite%\newpage


\begin{tlg}[hp03_068]
\tl{
\pada{\app{\lem[wit={ceteri}]{jālandhare} %  °dhara? V15
	\rdg[wit={N3,V3}]{jālāṃdhare}
	\rdg[wit={V19}]{jālādhare}} kṛte bandhe} % badhe N19
\pada{kaṇṭhasaṃkoca\app{\lem[wit={ceteri}]{lakṣaṇe} % kaṃṭhaṃ P11
		\rdg[wit={J5,G4}]{lakṣaṇaṃ}}/}\\+}
\tl{
\pada{na pīyūṣaṃ % yu V3
\app{\lem[wit={ceteri},alt={pataty}]{pata\skp{ty}}
		\rdg[wit={J5}]{pacaty}
		\rdg[wit={V19}]{prayāty}
		\rdg[wit={N23}]{kṣaraty}
		}ty agnau}
\pada{na ca vāyuḥ % vāyu J5,P11,V19
\app{\lem[wit={ceteri}]{pradhāvati}% N3,J5,Gr2,G11,N19,V15,J10,GrB
		\rdg[wit={Gr3,V1,Jyo}]{prakupyati}}//\versenr}\\!}
\end{tlg}
\commcite\newpage


\begin{tlg}[hp03_069]
\tl{
\pada{kaṇṭha\app{\lem[wit={ceteri}]{saṃkocanenaiva}% kaṃṭhaṃ P11, keṭa N23; °canaiva J5
		\rdg[wit={P11}]{saṃkocane caiva}
		\rdg[wit={V1}]{saṃkocane dehe}}}
\pada{\app{\lem[wit={J7,Gr3,G11,V15,J10,Jyo}]{dve nāḍyau}% ddhe J7ac, ddhau J7pc
		\rdg[wit={P11}]{dve nāḍyo}
		\rdg[wit={N3,N19},postwit=\texteng{\getsiglum{J7}\postcorr}]{dvau nāḍyau}
		\rdg[wit={J5}]{dve nāḍyai}
		\rdg[wit={C6}]{dvināḍyau}
		\rdg[wit={N23}]{\_\,nā\,\_}
		\rdg[wit={V1}]{nāḍyau ca}
		\rdg[wit={V3},alt={\lacuna}]{\skp{\lacuna}}}
\app{\lem[wit={N3,J5,G11,V15,V1,J10,P11,C6,Jyo},alt={stambhayed}]{stambhaye\skp{d}}% °bhaye N3,J5,J10, °bhayod P11
		\rdg[wit={Gr2,Gr3,N19}]{stambhite}
		\rdg[wit={V3},alt={\lacuna}]{\skp{\lacuna}}}%
\app{\lem[wit={N3,J5,V1,J10,Jyo},alt={dṛḍham}]{\skm{d }dṛḍham}% +N24
		\rdg[wit={Gr2,Gr3,G11,N19,P11}]{dhruvam}% +N12
		\rdg[wit={V15}]{dhṛvaṃ}
		\rdg[wit={C6}]{dhuram}
		\rdg[wit={V3},alt={\lacuna}]{\skp{\lacuna}}}/}\\+}
\tl{
\pada{\app{\lem[nolem]{\skp{pāda c}}
	\rdg[wit={V1},alt={\om}]{\skp{\om}}}%
\app{\lem[wit={ceteri},alt={madhyacakram}]{madhyacakra\skp{m}}
		\rdg[wit={N23}]{madhyakram}
		\rdg[wit={V3}]{madhye cakram}
		\rdg[wit={J10}]{madhyaṃ cakram}
		\rdg[wit={V1},alt={\om}]{\skp{\om}}
		}m idaṃ
\app{\lem[wit={ceteri}]{jñeyaṃ}% jreyaṃ G11
		\rdg[wit={N23}]{ya}
		\rdg[wit={V1},alt={\om}]{\skp{\om}}}}
\pada{\app{\lem[nolem]{\skp{pāda d}}
	\rdg[wit={V1},alt={\om}]{\skp{\om}}}%
ṣoḍaśādhārabandhanam//\versenr} % ādhāra{ra} N23, ārādha G11
%	\lineom{cd}{V1}
	\\!}
\end{tlg}
\commcite\newpage


\begin{tlg}[hp03_070]
\tl{\app{\lem[nolem]{}
	\rdg[wit={Gr3},alt={\om}]{\skp{\om}}
	\rdg[wit={Jyo},alt=\textapp{found after \ref{III74}}]{\skp{found after \ref{III74}}}}%
\pada{bandhatrayam idaṃ śreṣṭhaṃ} % ida V15
\pada{\app{\lem[wit={N3,J7,N19},alt={mahāsiddhair}]{mahāsiddhai\skp{r}} % siddhai N19
		\rdg[wit={V1,J10,Jyo}]{mahāsiddhaiś}
		\rdg[wit={N23}]{mahāsiddhe}
		\rdg[wit={G11}]{mahāsiddha}% +F
		\rdg[wit={J5,G4,V15,P11,C6}]{mahāsiddhi}
		\rdg[wit={V3}]{mahāsīha}}%
\app{\lem[wit={N3,G4,Gr2,G11,N19,GrB},alt={niṣevitam}]{\skm{r }niṣevitam}% niśe° N3
		\rdg[wit={V1,J10,Jyo}]{ca sevitaṃ}
		\rdg[wit={J5}]{prajāyate}
		\rdg[wit={V15}]{pradāyakaṃ}}/}\\+}
\tl{
\pada{sarveṣāṃ
\app{\lem[wit={N3,J5,J7,GrB,Jyo}]{haṭha}% saṃkaṭatrā + G4
		\rdg[wit={N23,G11,N19,V15,V1,J10}]{yoga} % yoma N19
	}tantrāṇāṃ}
\pada{\app{\lem[wit={ceteri}]{sādhanaṃ}
		\rdg[wit={N23}]{sāranaṃ}
		} yogino viduḥ//\versenr}
%	\NotIn{Gr3} 
%	\anm{after \ref{III74} \getsiglum{Jyo}}
	\\!}
\end{tlg}
\commcite%\newpage


\startaltrecension
\begin{alttlg}[hp03_070_1]
\tl{\app{\lem[nolem]{}
	\rdg[wit={Gr2,G11,V1,J10}]{\incl}}%
\pada{adhastā%t % t om. N23; badhastvā J10ac
\app{\lem[wit={V1,J10},alt={kuñcanenāśu}]{\skm{t }kuñcanenāśu} % +J5,G7
		\rdg[wit={Gr2,G11}]{kuñcanenaiva}% +G3
		}}
\pada{kaṇṭha\app{\lem[wit={V1,J10}]{saṃkocane kṛte} % kuṃṭha N23 % +J5,G7
		\rdg[wit={Gr2,G11}]{saṃkocanena ca}%
		}/}\\+}
\tl{
\pada{\app{\lem[wit={G11,V1}]{madhye}% +J5
		\rdg[wit={Gr2,J10}]{madhya}} 
	paścima\app{\lem[wit={G11}]{tāṇena}% +J5
		\rdg[wit={Gr2,V1,J10}]{tānena}}}
\pada{syāt prāṇo brahmanāḍigaḥ//\versenr} % sātmāṇo J5
%\sgwit{Gr2,G11,V1,J10}% +F
%\anm{=\,\manuref{2.46}}
\myfn{\getsiglum{N3,G4,N19,V15,GrB,Jyo} have this verse as 2.46 in chapter 2;
\getsiglum{J5,Gr2,G11,V1,J10} have it in both chapters 2 and 3 
(\getsiglum{J5} is not collated here because it is obvious that the verse was inserted later. See the description of the manuscript in the introduction). \getsiglum{Gr3} omit it at both places.} 
\\!}
\end{alttlg}
\altcommcite\newpage
\endaltrecension


\begin{tlg}[hp03_071]
\tl{
\pada{\app{\lem[nolem]{\skp{pāda a}}
	\rdg[wit={Gr3,J10},alt={\om}]{\skp{\om}}}%
mūlasthānaṃ
\app{\lem[wit={N3,J5,G11,V15,V1,GrB,Jyo}]{samākuñcya}
		\rdg[wit={Gr2,N19}]{samākṛṣya}
		%\rdg[wit={Gr3,J10},alt={\om}]{\skp{\om}}
		}}
\pada{\app{\lem[nolem]{\skp{pāda b}}
	\rdg[wit={Gr3,J10},alt={\om}]{\skp{\om}}}%
\app{\lem[wit={N3,N19}]{uḍḍīyāṇaṃ}
		\rdg[wit={J5,G11,V1,P11}]{uḍḍiyāṇaṃ}% °ṇāṃ J5
		\rdg[wit={Gr2,V15,Jyo}]{uḍḍiyānaṃ}% udvi° N23
		\rdg[wit={V3}]{uḍiyāṇaṃ}
		\rdg[wit={C6}]{uḍiyānaṃ}
		%\rdg[wit={Gr3,J10},alt={\om}]{\skp{\om}}
		} tu kārayet/} 
%	\lineom{ab}{Gr3,J10}
	\\+}
\tl{
\pada{\app{\lem[nolem]{\skp{pāda c}}
	\rdg[wit={J10},alt={\om}]{\skp{\om}}}%
\app{\lem[wit={G4,J7,Gr3,G11,V15,P11,V3,Jyo}]{iḍāṃ ca piṅgalāṃ} % iḷāṃ G4; piṃgulāṃ V3
		\rdg[wit={N3,N23,N19,C6}]{iḍā ca piṅgalā}
		\rdg[wit={V1}]{ilā piṃgalāṃ}
		\rdg[wit={J5}]{īḍā piṃgalā}
		%\rdg[wit={J10},alt={\om}]{\skp{\om}}
		}
\app{\lem[wit={ceteri}]{baddhvā}
		\rdg[wit={N19}]{baddhā}
		\rdg[wit={G4}]{baṃdhvā}
		%\rdg[wit={J10},alt={\om}]{\skp{\om}}
		}}
\pada{\app{\lem[nolem]{\skp{pāda d}}
	\rdg[wit={J10},alt={\om}]{\skp{\om}}}%
vāhaye%t
\app{\lem[wit={J5,G4,Gr2,Gr3,N19,V1,C6},alt={paścimaṃ}]{\skm{t }paścimaṃ}% pā° J5
		\rdg[wit={P11,V3}]{paścimāṃ}
		\rdg[wit={N3,G11,V15}]{paścimā}
		%\rdg[wit={J10},alt={\om}]{\skp{\om}}
		\rdg[wit={Jyo}]{paścime}}
\app{\lem[wit={ceteri}]{patham}
		\rdg[wit={P11}]{pathāṃ}
		\rdg[wit={Jyo}]{pathi}
		\rdg[wit={G4}]{padaṃ}
		%\rdg[wit={J10},alt={\om}]{\skp{\om}}
		}//\versenr} %\lineom{cd}{J10}
		\\!}
\end{tlg}
\commcite%\newpage


\begin{tlg}[hp03_072]
\tl{
\pada{\app{\lem[wit={ceteri}]{anenaiva vidhānena}% °nai<<va>> P11; pravidhā° J5
		\rdg[wit={J10}]{brahmasthānasthito rodhaḥ}}}
\pada{\app{\lem[wit={N3,J5,Gr3,GrB},alt={sevayet}]{sevaye\skp{t}}
		\rdg[wit={G4}]{śevayet}
		\rdg[wit={N19}]{vaśayet}
		\rdg[wit={Gr2,G11,V15,V1,J10,Jyo}]{prayāti}% +F,N24
		}%
\app{\lem[wit={N3,J5,E2,N19,P11,C6},alt={pavanālayam}]{\skm{t }pavanālayam}% +K3
		\rdg[wit={G4}]{pavanā\,+\,+}
		\rdg[wit={J7,G5,V15,V1,J10,V3,Jyo}]{pavano layam}% +G5,F,N24
		\rdg[wit={N23}]{pavano lagaṃ}
%		\rdg[wit={C7}]{pavanānalam}
		\rdg[wit={G11}]{pavano nalam}
		\rdg[wit={V19}]{paścimānalaṃ}}/}\\+}
\tl{
\pada{tato na jāyate
\app{\lem[wit={ceteri},alt={mṛtyur}]{mṛtyu\skp{r}}
		\rdg[wit={V19,G11}]{mṛtyu}% +K3
		\rdg[wit={N23,P11}]{mṛtyuṃ}}}%
\pada{\app{\lem[wit={Gr2,E2,G11,N19,V1,Jyo},alt={jarārogādikaṃ}]{\skm{r }jarārogādikaṃ}% N24
		\rdg[wit={N3}]{jarārogādikas}
		\rdg[wit={J5,GrB}]{jarārogādikā} % +F ##
		\rdg[wit={V15}]{jarāmohādikaṃ}
		\rdg[wit={V19}]{jvaro rogādikas}
		\rdg[wit={J10}]{nāsya jarādikaṃ}}
\app{\lem[wit={ceteri}]{tathā}% +N24, damaged G4
		\rdg[wit={N3}]{tadā}
		\rdg[wit={J5}]{vyathā}
		\rdg[wit={GrB}]{kathā}% +F
		}//\versenr}\label{III74}\\!}
\end{tlg}
\commcite\newpage


\begin{ava}[hp03_073a]
\app{\lem[wit={ceteri}]{atha}
		\rdg[wit={J7,V19,Jyo},alt={\om}]{\skp{\om}}}
\app{\lem[wit={Gr2,E2,G11,V15,V1,J10,C6,V3}]{viparītakaraṇī}% viyaśeta N23
		\rdg[wit={N3}]{viparītakaraṇīṃ}
		\rdg[wit={P11}]{viparītakaraṇīyaṃ}
		\rdg[wit={J5,N19}]{viparītakaraṇaṃ}
		\rdg[wit={V19,Jyo},alt={\om}]{\skp{\om}}}/
\end{ava}


\begin{tlg}[hp03_073]
\tl{
\pada{\app{\lem[wit={N3,J5,J7,V19,E2,N19,V15,V1,J10,V3},alt={ūrdhvaṃ nābhir}]{ūrdhvaṃ nābhi\skp{r}}
		\rdg[wit={G11,P11}]{ūrdhvanābhir}
		\rdg[wit={N23}]{ūrdhvanābhor}
		\rdg[wit={Jyo}]{ūrdhvanābher}
		\rdg[wit={C6}]{ūrdhvaṃ nābher}
		\rdg[wit={G4}]{+\,+\,nābher}}%
\app{\lem[wit={J5,G4,V19,G11,N19,V15,J10,GrB},alt={adhas tālur}]{\skm{r }adhas tālu\skp{r}}% adhaḥstālur V19, °tālūr G4,V15
		\rdg[wit={N3,J7,V1}]{adhas tālu}% +K3
		\rdg[wit={E2}]{adhas tāla}
		\rdg[wit={N23}]{asāluktar}
		\rdg[wit={Jyo}]{adhastālor}}}%
\pada{\app{\lem[wit={ceteri},alt={ūrdhvaṃ}]{\skm{r }ūrdhvaṃ}
		\rdg[wit={J5,N23,V19}]{ūrdhva}
		\rdg[wit={V1}]{ūrdhvo}}
	bhānur adhaḥ śaśī/}\\+} % adho C6; śaśi V15, śiśī J5, śasī J10
\tl{
\pada{\app{\lem[nolem]{\skp{pāda c}}
	\rdg[wit={V19,N19,J10,P11,C6},alt={\om}]{\skp{\om}}}%
\app{\lem[wit={N3,Gr2,E2,G11,V1,Jyo}]{karaṇī viparītākhyā}% +C7
		\rdg[wit={J5}]{karaṇī viparītākṣaṃ}
		\rdg[wit={G4,V15,V3}]{karaṇaṃ viparītākhyaṃ}
		%\rdg[wit={V19,N19,J10,P11,C6},alt={\om}]{\skp{\om}}
		}}
\pada{\app{\lem[nolem]{\skp{pāda d}}
	\rdg[wit={V19,N19,J10,P11,C6},alt={\om}]{\skp{\om}}}%
guruvākyena
\app{\lem[wit={N3,N23,V15,V1,V3,Jyo}]{labhyate}
		\rdg[wit={J5}]{lakṣate}
		\rdg[wit={G4,G11}]{lakṣayet}
		\rdg[wit={J7,E2}]{gamyate}% +C7
		%\rdg[wit={V19,N19,J10,P11,C6},alt={\om}]{\skp{\om}}
		}//\versenr}\myfn{\getsiglum{Gr2,Gr3,N19,GrB,Jyo} have \manuref{3.77*1--2} before this verse.}
%	\lineom{cd}{V19,N19,J10,P11,C6}
	\label{viparita}\\!}% +K3
\end{tlg}

\avacite{73a}
\commcite\newpage


\begin{tlg}[hp03_074]
\tl{
\pada{\app{\lem[nolem]{\skp{pāda a}}
	\rdg[wit={Jyo},alt={\om}]{\skp{\om}}}%
\app{\lem[wit={ceteri}]{karaṇī}% kāraṇī J7ac
		\rdg[wit={P11}]{karaṇaṃ}% +F
		%\rdg[wit={Jyo},alt={\om}]{\skp{\om}}
		}
\app{\lem[wit={ceteri}]{viparītākhyā}
		\rdg[wit={C6}]{viparītākhyaṃ}% +F
		\rdg[wit={J5,N19}]{viparītākṣaṃ}
		%\rdg[wit={Jyo},alt={\om}]{\skp{\om}}
		}}
\pada{\app{\lem[nolem]{\skp{pāda b}}
	\rdg[wit={Jyo},alt={\om}]{\skp{\om}}}%
sarvavyādhi\app{\lem[wit={N3,Gr2,Gr3,G11,V15,V1,J10}]{vināśinī}
		\rdg[wit={N19,P11}]{vināśanī}
		\rdg[wit={J5,C6,V3}]{vināśanaṃ}% vināśinīm F
		%\rdg[wit={Jyo},alt={\om}]{\skp{\om}}
		}/} 
	%\lineom{ab}{Jyo}
	\\+}
\tl{
\pada{nityam abhyāsa\app{\lem[wit={ceteri}]{yuktasya}
		\rdg[wit={G11}]{yogena}
		\rdg[wit={P11}]{saktasya}}}
\pada{jaṭharāgni\app{\lem[wit={N3,J7,N19,V15,V1}]{vivardhanī}
		\rdg[wit={N23,Gr3,G11,P11,Jyo}]{vivardhinī}
		\rdg[wit={J5,C6,V3}]{vivardhanaṃ}% +F,N24
		\rdg[wit={J10}]{pravardhinī}}//\versenr}\\!}
\end{tlg}
\commcite\newpage


\begin{tlg}[hp03_075]
\tl{
\pada{āhāro bahulas tasya} % bahulo J5
\pada{saṃpādyaḥ sādhakasya % °pādya N3,J5
\app{\lem[wit={ceteri}]{tu}
		\rdg[wit={J5,N23,Jyo}]{ca}% +K3
		}/}\\+}
\tl{
\pada{\app{\lem[wit={Gr1,G11,V1,J10,GrB,Jyo}]{alpāhāro}
		\rdg[wit={Gr2,Gr3,N19,V15}]{anāhāro}
		}
\app{\lem[wit={ceteri},alt={yadi bhaved}]{yadi bhave\skp{d}}
		\rdg[wit={J10}]{nirāhāraḥ}}}%
\pada{\app{\lem[wit={N3,G4,Gr3,G11,V15,V3},alt={agnir dehaṃ}]{\skm{d }agnir dehaṃ}
		\rdg[wit={J5,Gr2,N19,P11}]{agnidehaṃ}
		\rdg[wit={V1}]{deham agnir}
		\rdg[wit={C6}]{agnidāho}
		\rdg[wit={Jyo}]{agnir daha°}
		\rdg[wit={J10}]{kṣudhālasya}}
\app{\lem[wit={N3,J5,Gr2,Gr3,G11,N19,V1,V3},alt={dahet}]{dahe\skp{t}} % dahe J5,V3,G11,N19, daher N23
		\rdg[wit={G4,V15,P11}]{haret}% +F
		\rdg[wit={C6}]{bhavet}
		\rdg[wit={Jyo}]{°ti tat}
		\rdg[wit={J10}]{vaśe}}%
\app{\lem[wit={cetwG4},alt={kṣaṇāt}]{\skm{t }kṣaṇāt}% +G4,J5
		\rdg[wit={N3}]{kramāt}% =HR
		\rdg[wit={J7}]{tataḥ}
		\rdg[wit={J10}]{bhavet}}//\versenr}\\!}
\end{tlg}
\commcite\newpage


\begin{tlg}[hp03_076]
\tl{%
% \pada{adhaḥ\app{\lem[wit={C6,Jyo}]{śirāś cordhva}
		% \rdg[wit={N3,G4,V19,N19,V15,V1,J10,P11,V3}]{śiraś cordhva} % corddhaṃ V3, cordha V19
		% \rdg[wit={J5}]{śira corddha}% adhośira J5
		% \rdg[wit={J7,K3}]{śirā ūrdhva} % arddha J10; śiro J7pc, śirāḥ K3
		% \rdg[wit={N23}]{śīrā ūrdhva}
		% \rdg[wit={C7}]{śira ūrdhva}}%
\pada{adhaḥ\app{\lem[wit={N3,G4,V19,G11,N19,V15,V1,J10,GrB}]{śira\skp{ś}}% +F
		\rdg[wit={J5,E2}]{śira}% śiro J7pc
		\rdg[wit={Jyo},alt={śirāś}]{śirāś}% ardha J10
		\rdg[wit={Gr2}]{śirā}% śirāḥ K3
		}%
	\app{\lem[wit={ceteri},alt={cordhva}]{\skm{ś }cordhva}% cordha V19
		\rdg[wit={V3}]{cordhvaṃ} % corddhaṃ V3
		\rdg[wit={Gr2,E2}]{ūrdhva} % arddha J10 
		}%
\app{\lem[wit={Gr1,J7,Gr3,G11,V15,P11,C6,Jyo}]{pādaḥ}
		\rdg[wit={N23,N19,V3}]{pāda}
		\rdg[wit={V1,J10}]{pādau}}}
\pada{\app{\lem[wit={ceteri},alt={kṣaṇaṃ syāt}]{kṣaṇaṃ syā\skp{t}} % kṣaṇa N3,N19, kṣaṇaḥ C6
		\rdg[wit={V19}]{kṣīṇaṃ syāt}
		\rdg[wit={J10}]{lakṣaṇaṃ}}t 
		prathame 
	\app{\lem[wit={ceteri}]{dine}
		\rdg[wit={E2}]{hani}}/}\\+}
\tl{
\pada{\app{\lem[nolem]{\skp{pāda c}}
	\rdg[wit={V19},alt={\om}]{\skp{\om}}}%
\app{\lem[wit={N3,G11,V15,V1,J10,C6,V3,Jyo}]{kṣaṇāc ca}
		\rdg[wit={J5}]{kṣayāc ca}
		\rdg[wit={Gr2}]{kṣaṇāt tu}% °ārtu N23
		\rdg[wit={P11}]{kṣaṇādyaṃ}
		\rdg[wit={N19}]{kṣaṇārdha}
		\rdg[wit={E2}]{kṣaṇādardhaṃ}
		\rdg[wit={V19},alt={\om}]{\skp{\om}}} 
	kiṃci%d ciddadhi° J10
\app{\lem[wit={ceteri},alt={adhikam}]{\skm{d }adhika\skp{m}}
		\rdg[wit={N23}]{apika}
		\rdg[wit={V19},alt={\om}]{\skp{\om}}}}% kṣaṇārtu N23, apika N23 >> Śāradā?
\pada{\app{\lem[nolem]{\skp{pāda d}}
	\rdg[wit={V19},alt={\om}]{\skp{\om}}}%
\app{\lem[wit={ceteri},alt={abhyasec ca}]{\skm{m }abhyasec ca}% °se P11
		\rdg[wit={J7}]{abhyasetva}
		\rdg[wit={N23}]{bhyarccayec ca}
		\rdg[wit={V19},alt={\om}]{\skp{\om}}} 
		dine dine//\versenr} %\lineom{cd}{V19}
		\\!}
\end{tlg}
\commcite\newpage


\begin{tlg}[hp03_077]
\tl{
\pada{\app{\lem[wit={N3,G11,V1,GrB}]{valiś ca}
		\rdg[wit={J5}]{vali}
		\rdg[wit={E2}]{valiṃ ca}
		\rdg[wit={Gr2,V19,N19,V15,J10,Jyo}]{valitaṃ}% calitaṃ J7ac
		}
	\app{\lem[wit={ceteri}]{palitaṃ}% palite J5
		\rdg[wit={P11,C6}]{palitaś}} caiva}
\marma\pada{\app{\lem[wit={Jyo}]{ṣaṇmāsordhvaṃ na}% +G7!,°sordhvaṃ tu F!!
		\rdg[wit={N3,G11,V15,P11,C6}]{ṣaṇmāsārdhān na}% +M1,G5,6
		\rdg[wit={V3}]{ṣaṇmāsārdhaṃ na}% +N24
		\rdg[wit={J5,Gr2,E2}]{ṣaṇmāsārdhena}
		\rdg[wit={V19}]{ṣaṇmāsārdheṇa}
		\rdg[wit={N19}]{ṣaṇmāsārdhe ca}
		\rdg[wit={V1,J10}]{ṣaṇmāsāt tu na}}
\app{\lem[wit={ceteri}]{dṛśyate}% +M1,G5,6 % N3,J5,G11,V15,V1,J10,GrB,Jyo
		\rdg[wit={Gr2,Gr3,N19}]{naśyati}}/}\\+} % nasyati V19
\tl{
\pada{\app{\lem[wit={ceteri}]{yāmamātraṃ tu}% jāma V19; mātras tu N19
		\rdg[wit={V15}]{yāmamātraṃ ca}
		\rdg[wit={J10}]{māsatrayaṃ tu}} yo
\app{\lem[wit={ceteri},alt={nityam}]{nitya\skp{m}}
		\rdg[wit={N23}]{gnibhyam}}}%
\pada{\app{\lem[wit={ceteri},alt={abhyaset}]{\skm{m }abhyase\skp{t}}
		\rdg[wit={V19}]{aset}}t sa
\app{\lem[wit={ceteri}]{tu}
		\rdg[wit={J7}]{su}
		\rdg[wit={N19}]{ca}}
\app{\lem[wit={ceteri}]{kālajit}% kāra˟jit N3ac
		\rdg[wit={N19,J10}]{kālavit}
		}//\versenr}\label{vali}\\!}
\end{tlg}
\commcite\newpage


\startaltrecension
\begin{alttlg}[hp03_077_1]
\tl{%\app{\lem[nolem]{}
	%\rdg[wit={Gr2,Gr3,G11,N19,GrB,Jyo}]{\incl}}%
\pada{yat kiṃci%t
\app{\lem[wit={J7,Gr3,G11,N19,C6,V3,Jyo},alt={sravate}]{\skm{t }sravate} % śravate N19
		\rdg[wit={N23}]{sravanaṃ}
		\rdg[wit={P11}]{sevate}}
\app{\lem[wit={J7,Gr3,G11,P11,C6,Jyo},alt={candrād}]{candrā\skp{d}}
		\rdg[wit={V3}]{candra}
		\rdg[wit={N19}]{caṃdrāṃn}
		\rdg[wit={N23}]{ced<<am>>}}}%
\pada{d amṛtaṃ % amritaṃ? V19
\app{\lem[wit={C6,Jyo}]{divyarūpiṇaḥ}% =N3
		\rdg[wit={G11,N19}]{divyarūpiṇaṃ}
		\rdg[wit={P11}]{divyarūpiṇī}
		\rdg[wit={Gr2,Gr3}]{divyarūpi ca}% rūpa ca J5
		\rdg[wit={V3}]{divyarūpagaḥ}
		}/}\\+}
\tl{
\pada{tat sarvaṃ % tata sarva N23
\app{\lem[wit={ceteri}]{grasate}
		\rdg[wit={P11}]{sravate}}
\app{\lem[wit={ceteri},alt={sūryas}]{sūrya\skp{s}}% Gr2,Gr3,G11,GrB,Jyo
		\rdg[wit={N19}]{roho}}}%
\pada{s tena \app{\lem[wit={J7,Gr3,N19,C6,V3}]{piṇḍaṃ}
		\rdg[wit={N23}]{piḍaṃ}
		\rdg[wit={G11,P11}]{piṇḍa}
		\rdg[wit={Jyo}]{piṇḍo}}
\app{\lem[wit={G11,GrB}]{jarāyutam}% = N3
		\rdg[wit={Jyo}]{jarāyutaḥ}
		\rdg[wit={Gr2,E2,N19}]{vināśi ca}
		\rdg[wit={V19}]{vinasyati}
		}//\versenr}
%	\NotIn{Gr1,V15,V1,J10}
	\sgwit{Gr2,Gr3,G11,N19,GrB,Jyo} 
%	\anm{=\,\manuref{4.10}}
	\\!} 
\end{alttlg}
\alttrcite%\newpage

%%% G7 omits these two verses too. G11,M3 have them at the end of this section. !!!


\begin{alttlg}[hp03_077_2]
\tl{%\app{\lem[nolem]{}
	%\rdg[wit={Gr2,Gr3,G11,N19,GrB,Jyo}]{\incl}}%
\pada{tatrāsti karaṇaṃ divyaṃ} % kāraṇaṃ E2
\pada{sūryasya mukha%
\app{\lem[wit={ceteri}]{bandhanam} % entry in Marmasthāna
		\rdg[wit={Jyo}]{vañcanam}}\marma/}\\+}
\tl{
\pada{gurūpadeśato % gurupa° V3, guropa° N19; desetā N23
\app{\lem[wit={ceteri}]{jñeyaṃ}% J7,Gr3,G11,N19,GrB,Jyo
		\rdg[wit={N23}]{\_\,yaṃ}}}
\pada{\app{\lem[wit={ceteri}]{na tu}% J7,Gr3,G11,N19,GrB,Jyo
		\rdg[wit={N23}]{rttu}} 
	\app{\lem[wit={ceteri}]{śāstrārtha} % śastrā° N23
		\rdg[wit={E2}]{śāstrāstra}}koṭibhiḥ//\versenr}%
%	\anm{=\,\manuref{4.11}}
%	\NotIn{Gr1,V15,V1,J10}
	\sgwit{Gr2,Gr3,G11,N19,GrB,Jyo}%
	\note[type=anmkg,nonum,lem=\mylem{77*1--2},labelb={hp3-77-2}]{%
	included in \getsiglum{Gr2,Gr3,G11,N19,GrB,Jyo}. The \textepsilon\ group has this pair of verses here, but the other manuscripts have it at the beginning of this section (before \ref{viparita}). 
	The \textalpha\ group has it as 4.10--11 in chapter 4.}
	\\!}
\end{alttlg}
\teimute{\alttrcite\def\commvnum{77-2}%
\def\labelvnum{77*1--2}\comminfn}
\newpage
\endaltrecension


% \begin{altava}[hp03_078a]
% \teimute{\small}
% atratyā vajrolī
% \app{\lem[wit={V19}]{granthānte likhitā}% liṣitā V19
% %		\rdg[wit={K3}]{granthāntare likhitā vartate}
		% \rdg[wit={C7}]{granthāntare tu likhitāsīt}}/
% \app{\lem[wit={C7}]{kramaprāptāpy atra tyaktā} % +K3; oṃ kramaprāptyāpy  C7
		% \rdg[wit={V19},alt={\om}]{\skp{\om}}}/
% \app{\lem[resp=emend]{asādhāraṇa}% =K3
		% \rdg[wit={V19}]{asādhāraṇaṃ}
		% \rdg[wit={C7}]{asāraṇa}}prāṇyanuṣṭheyatvāt tasyāḥ/ % prāṇuṣṭheyatvāt C7
	% \sgwit{V19,C7}%
	% \myfn{In \getsiglum{V19,C7} the Vajrolī section is found as Chpater 5 at the end of the work.\label{vajroli-delta} \getsiglum{E2} does not have this omission note. It is not clear whether the ms from which this ms was copied originally included this section, as the text of \getsiglum{E2} ends in the middle of the last verse of Chapter 4 (4.78).}
% \teimute{\normalsize}
% \end{altava}


\begin{ava}[hp03_078a]
\app{\lem[nolem]{}
	\rdg[wit={J10,C6}]{\textapp{found after the first half of the next verse}}}%
		atha vajrolī/% vajrontī N23, vajroliḥ G5
\end{ava}


\begin{tlg}[hp03_078]
\tl{
\pada{\app{\lem[wit={ceteri}]{svecchayā}% +G5
		\rdg[wit={G11}]{sarvathā}} vartamāno'pi}
\pada{\app{\lem[wit={G4,J7,V19,G5,V1,J10,GrB,Jyo},alt={yogoktair}]{yogoktai\skp{r}}% °ktai P11
		\rdg[wit={N23}]{yogokair}
		\rdg[wit={J5,G11,N19,V15}]{yogokta}% ## = DYŚ
		\rdg[wit={N3}]{yogoktaṃ}
		\rdg[wit={C7}]{niyamair}}%
	\app{\lem[wit={ceteri},alt={niyamair vinā}]{\skm{r }niyamair vinā} % V3 niyamai
		\rdg[wit={C7}]{vividhais tathā}}/}\\+}
\tl{
\pada{\app{\lem[wit={V19,V15,V1,J10,C6,Jyo}]{vajrolīṃ yo}
		\rdg[wit={G11,G5}]{vajroliṃ yo}
		\rdg[wit={J5,Gr2,C7,N19,V3}]{vajrolī yo}
		\rdg[wit={P11}]{vajrolīr yo}
		\rdg[wit={N3}]{vajrālī yo}} % yo (l.br.) yo J10
	\app{\lem[wit={ceteri}]{vijānāti}% +K4; vināti J5
		\rdg[wit={Gr2}]{bhijānāti}}}
\pada{sa yogī % saṃ N23
	\app{\lem[wit={ceteri}]{siddhibhājanam}% °jana J5
		\rdg[wit={N23,G11,G5},alt={°bhājanaḥ}]{siddhibhājanaḥ}
		\rdg[wit={J10}]{siddhimān bhavet}}\marma//\versenr}%
		\myfn{Before this verse, \getsiglum{G11,G5} have \emph{Dattātreyayogaśāstra} 150.}
		\\!}
\end{tlg}

\avacite{78a}
\commcite\newpage

% prob. om. in G4
\begin{tlg}[hp03_079]
\tl{
\pada{tatra % tava N3, tā J5
	\app{\lem[wit={ceteri}]{vastu}
		\rdg[wit={N3}]{castu}
		\rdg[wit={N19}]{bheda}
		}dvayaṃ
	\app{\lem[wit={ceteri},alt={vakṣ(y)e}]{vakṣye}%
%		\rdg[wit={J5,V19,V3}]{vakṣe}
		\rdg[wit={J7}]{manye}
		\rdg[wit={N23}]{api}
		}} % dvayam api N23
\pada{durlabhaṃ 
	\app{\lem[wit={ceteri}]{yasya kasya}
		\rdg[wit={G11,G5}]{yena kena}}%
	\app{\lem[wit={ceteri}]{cit}
		\rdg[wit={V15}]{tu}}/}\\+}
\tl{
\pada{kṣīraṃ \app{\lem[wit={ceteri}]{caikaṃ}% vaikaṃ C7
		\rdg[wit={G5}]{caita}
		\rdg[wit={J10}]{caiva}% +F
		\rdg[wit={Gr2,C6}]{ekaṃ}}
		dvitīyaṃ 
	\app{\lem[wit={ceteri}]{tu}
		\rdg[wit={G11,G5}]{ca}}} % °tiyaṃ V3
\pada{nārī  % nāḍī C6
	\app{\lem[wit={ceteri}]{ca}
		\rdg[wit={J5}]{vā}
		\rdg[wit={C7}]{tu}} 
		vaśavartinī//\versenr}% vaśi° N19, varttanī J5,N23
	\myfn{After this verse, \getsiglum{G11,G5} have \emph{Dattātreyayogaśāstra} 154cd--157cd.}
		\\!}
\end{tlg}
\commcite\newpage

% prob. om. in G4

\begin{tlg}[hp03_080]
\tl{
\pada{\app{\lem[wit={N3,J7,G11,G5,N19,V15,C6,Jyo}]{mehanena}
		\rdg[wit={P11}]{mehanīna}
		\rdg[wit={N23}]{mehanaiva}
		\rdg[wit={V19}]{mohanena}
		\rdg[wit={C7}]{mohanenā}
		\rdg[wit={V3}]{meḍhrenena}
		\rdg[wit={V1}]{meṃḍhreṇa}
		\rdg[wit={J10}]{mahānibhaṃ}% mehanena J10pc1, meḍhra J10pc2
		\rdg[wit={J5}]{hematene}}}
	\app{\lem[wit={ceteri}]{śanaiḥ}
		\rdg[wit={V19}]{sadā}
		\rdg[wit={J5}]{hane}
		} samya% kasyag V15
\pada{\app{\lem[wit={ceteri},alt={ūrdhvākuñcanam}]{\skm{g }ūrdhvākuñcana\skp{m}}
		\rdg[wit={P11}]{ūrdhvāṃkuñcanam}
		\rdg[wit={G11}]{ūrdhvakuñcanam}
		\rdg[wit={J7}]{ūrdhva kiṃcanam}
		\rdg[wit={J10}]{kṛtvā kuñcanam}
		\rdg[wit={V1}]{gudākuñcanam}}m 
		abhyaset/}\\+} % abhyāset N23; kuṃcana<m a>bhyaset V15
\tl{
\pada{puruṣo \app{\lem[wit={ceteri}]{vāpi nārī vā}% vāti N19
		\rdg[wit={J5,C7}]{vāpi vā nārī}
		\rdg[wit={Jyo}]{'py atha vā nārī}}}
\pada{\app{\lem[wit={ceteri}]{vajrolī} % °li V15
		\rdg[wit={V19,V1}]{vajrolīṃ}
		\rdg[wit={J7}]{vajrolīḥ}}%
	\app{\lem[wit={ceteri}]{siddhim āpnuyāt}
		\rdg[wit={J7}]{siddhibhājanam}
		\rdg[wit={N23}]{siddhibhājanaḥ}}//\versenr}
		\\!}  %3.81
\end{tlg}
\commcite\newpage

% om. N23; prob. om. in G4
\begin{tlg}[hp03_081]
\tl{\app{\lem[nolem]{}
	\rdg[wit={N23},alt={\om}]{\skp{\om}}}%
\pada{\app{\lem[wit={ceteri}]{yatnataḥ} % yantrataḥ C7
		\rdg[wit={J7,V1,J10}]{prayatnataḥ}
		\rdg[wit={C6}]{prayatnāt}} % + J10pc
	\app{\lem[wit={N3,J5,V19,C7, G11,G5,GrB}]{śaranālena}% sara° J5,C6
		\rdg[wit={N19}]{śaranolena}
		\rdg[wit={V15}]{śatanārīṇāṃ}
		\rdg[wit={Jyo}]{śastanālena}
		\rdg[wit={J7,V1,J10}]{śironāle}}} % + J10pc
\pada{\app{\lem[wit={N3,C7,N19,V1,Jyo}]{phūtkāraṃ} % + J10pc
		\rdg[wit={V3}]{phutkāraṃ}
		\rdg[wit={V19,G11,V15}]{pūtkāraṃ}
		\rdg[wit={J7,J10}]{phūtkāraḥ}
		\rdg[wit={C6}]{sphūtkāraṃ}
		\rdg[wit={G5}]{sūtkāraṃ}
		\rdg[wit={P11}]{śaraṃ tu}
		\rdg[wit={J5}]{leneraṃ}}
	\app{\lem[wit={ceteri}]{vajra} % +J10pc; vajñā J5
		\rdg[wit={J7,J10}]{kaṃbu}}%
	\app{\lem[wit={N19,V15,GrB,Jyo}]{kandare}
		\rdg[wit={N3,J7,V19,C7,V1,J10}]{kandhare}
		\rdg[wit={G11}]{kandharet}
		\rdg[wit={J5,G5}]{kaṃharet}}/}\\+}
\tl{
\pada{śanaiḥ \app{\lem[wit={ceteri}]{śanaiḥ}% 2 x śanai P11
		\rdg[wit={J10}]{śanaḥ}}
	\app{\lem[wit={ceteri}]{prakurvīta}
		\rdg[wit={J10,C6}]{prakurvaṃti}}}
\pada{\app{\lem[wit={ceteri}]{vāyu} % vāyu{{ḥ}} C7; vāyuma° J10ac
		\rdg[wit={C6}]{vāyoḥ}}saṃcāra%
	\app{\lem[wit={ceteri}]{kāraṇāt} % kāraṇā J5, kāranat V19
		\rdg[wit={G11,G5}]{dhāraṇāt}}//\versenr}
%	\NotIn{N23}
	\\!}
\end{tlg}
\commcite\newpage


\begin{tlg}[hp03_082]
\tl{
\pada{\app{\lem[wit={Gr2,V19,C7,N19,V15,P11,C6}]{nāryā}
		\rdg[wit={J5,Jyo}]{nārī}% +F
		\rdg[wit={N3}]{māryā}
		\rdg[wit={V1,J10,V3}]{bhāryā}
		\rdg[wit={G11}]{tato}
		\rdg[wit={G5}]{bhage}}
	\app{\lem[wit={ceteri}]{bhage}
		\rdg[wit={N3,P11}]{bhāge}
		\rdg[wit={J5}]{bhaga}
		\rdg[wit={G5}]{patat}}
	\app{\lem[wit={ceteri},alt={patad}]{pata\skp{d}} % pata<<t>> N23, patat V1
		\rdg[wit={J7}]{pated}
		\rdg[wit={J5}]{yad}
		\rdg[wit={N19}]{ca tad}
		\rdg[wit={G5}]{tato}}%
	\app{\lem[wit={ceteri},alt={bindum}]{\skm{d}bindu\skp{m}} % biṃduṃm N19
		\rdg[wit={V3}]{bindhuḥm}
		\rdg[wit={V1,J10}]{bindur}
		\rdg[wit={G11,G5}]{vīryam}}}%
\pada{m abhyāsenordhva%m % abhyāsādṛḍham J5
	\app{\lem[wit={ceteri},alt={āharet}]{\skm{m }āharet}
		\rdg[wit={P11}]{ācaret}
		\rdg[wit={C7}]{āruhet}}/}\\+}
\tl{
\pada{\app{\lem[nolem]{\skp{pāda c}}
	\rdg[wit={V19,C7},alt={\om}]{\skp{\om}}}%
	\app{\lem[wit={ceteri}]{calitaṃ}
		\rdg[wit={P11}]{bhavitaṃ}
		%\rdg[wit={V19,C7},alt={\om}]{\skp{\om}}
		}
	\app{\lem[wit={N3,J5}]{ca svakaṃ}% ca svayaṃ C2
		\rdg[wit={G4,Gr2,N19,P11,C6}]{tu svakaṃ}% mvakaṃ N23; +N24
		\rdg[wit={V3}]{tu sukaṃ}
		\rdg[wit={G11,G5,V15,Jyo}]{ca nijaṃ}
		\rdg[wit={V1}]{patitaṃ}
		\rdg[wit={J10}]{calitaṃ}
		%\rdg[wit={V19,C7},alt={\om}]{\skp{\om}}
		} bindu}%m
\pada{\app{\lem[nolem]{\skp{pāda d}}
	\rdg[wit={V19,C7},alt={\om}]{\skp{\om}}}%
	\app{\lem[wit={cetwG4},alt={ūrdhvam ākṛṣya rakṣayet}]{\skm{m }ūrdhvam ākṛṣya rakṣayet}
		\rdg[wit={N3}]{ūrdhvam ākṛ\,+\,+\,+\,+}
		\rdg[wit={V15}]{ūrdhvam āhṛtya rakṣayet}
		\rdg[wit={N19}]{abhyāsenordhvam āharet}
		%\rdg[wit={V19,C7},alt={\om}]{\skp{\om}}
		}//\versenr}
%	\lineom{cd}{V19,C7}
	\\!}
\end{tlg}
\commcite\newpage


\begin{tlg}[hp03_083]
\tl{\app{\lem[nolem]{}
	\rdg[wit={N3}]{\unavbl}}%
\pada{evaṃ
	\app{\lem[wit={J5,G4,Gr2,N19,GrB},alt={tu rakṣayed}]{tu rakṣaye\skp{d}}% +N24
		\rdg[wit={V19,C7,G11,G5,Jyo}]{saṃrakṣayed}% +F
		\rdg[wit={V15}]{surakṣayed}
		\rdg[wit={V1,J10}]{rakṣati yo}}d
		binduṃ} % biṃdu G4,P11,V3,N23,V19,N4
\pada{mṛtyuṃ jayati 
	\app{\lem[wit={ceteri}]{yogavit}
		\rdg[wit={G11,G5}]{tatvataḥ}}/}\\+}
%	\anm{\ref{VuIII88}--\ref{VuIII116}a lost \getsiglum{N3}}\\+}
\tl{
\pada{\app{\lem[nolem]{\skp{pāda c}}
	\rdg[wit={V19},alt={\om}]{\skp{\om}}}%
	maraṇaṃ \app{\lem[wit={ceteri}]{bindu}
		\rdg[wit={N19}]{bida}
		%\rdg[wit={V19},alt={\om}]{\skp{\om}}
		}pātena}
\pada{\app{\lem[nolem]{\skp{pāda d}}
	\rdg[wit={V19},alt={\om}]{\skp{\om}}}%
	\app{\lem[wit={J5,C7,G11,G5,N19,V15,V1,J10,V3}]{jīvitaṃ} % +N24; jīvita J5
		\rdg[wit={P11}]{jīvituṃ}
		\rdg[wit={J7,C6,Jyo}]{jīvanaṃ}% +F
		\rdg[wit={N23}]{jī<<vanaṃ>>}
		%\rdg[wit={V19},alt={\om}]{\skp{\om}}
		}
	\app{\lem[wit={J7,C7,N19,V1,J10,GrB,Jyo}]{bindudhāraṇāt}
		\rdg[wit={G11,G5,V15}]{bindurakṣaṇāt}
		\rdg[wit={J5}]{baṃdhasaṃgrahāt}
		\rdg[wit={N23}]{<<bindudhāraṇam>>}
		%\rdg[wit={V19},alt={\om}]{\skp{\om}}
		}//\versenr}\label{VuIII88}
	%\lineom{cd}{V19}
	%\unavbl{N3}
	\\!}
\end{tlg}
\commcite\newpage

\begin{tlg}[hp03_084]
\tl{\app{\lem[nolem]{}
	\rdg[wit={N3}]{\unavbl}}%
\pada{\app{\lem[nolem]{\skp{pāda a}}
	\rdg[wit={C7,V1,J10},alt={\om}]{\skp{\om}}
	\rdg[wit={V15},alt=\textapp{found after \ref{manayattam}b}]{\skp{found after 3.85b}}}%
\app{\lem[wit={J5,Gr2,V19,V15,GrB}]{sugandhi}% yugaṃndhi N23
		\rdg[wit={G11,G5}]{sugandhaṃ}
		\rdg[wit={N19,Jyo}]{sugandho}% +N24
		%\rdg[wit={C7,V1,J10},alt={\om}]{\skp{\om}}
		}
		yogino
	\app{\lem[wit={J5,P11,V3}]{dehaṃ} % G4 illeg.
		\rdg[wit={V19,G11,G5,V15,Jyo}]{dehe}
		\rdg[wit={Gr2,N19,C6}]{deho}% +N24
		%\rdg[wit={C7,V1,J10},alt={\om}]{\skp{\om}}
		}}
\pada{\app{\lem[nolem]{\skp{pāda b}}
	\rdg[wit={C7,G11,V1,J10},alt={\om}]{\skp{\om}}
	\rdg[wit={V15},alt=\textapp{found after \ref{manayattam}b}]{\skp{found after 3.85b}}}%
	jāyate bindu%
	\app{\lem[wit={J5,N23,J7,V19,G5,N19,P11,V3,Jyo}]{dhāraṇāt}
		\rdg[wit={V15,C6}]{rakṣaṇāt}
		%\rdg[wit={C7,G11,V1,J10},alt={\om}]{\skp{\om}}
		}/}
		%\lineom{ab}{C7,V1,J10} %\lineom{b}{G11} 
		%\anm{after~\ref{manayattam}ab~\getsiglum{V15}}		
		\\+}
\tl{
%		\sgwit{J7,V19,N19,V15,GrB,Jyo}
\pada{\app{\lem[nolem]{\skp{pāda c}}
	\rdg[wit={G11},alt={\om}]{\skp{\om}}}% Haplography due to deha
	\app{\lem[wit={N23,C7,J10,Jyo}]{yāvad binduḥ}% +C6pc; yāva N23
		\rdg[wit={J5,G4,J7,V19,G5,N19,V15,V1,GrB}]{yāvad bindu}
		%\rdg[wit={G11},alt={\om}]{\skp{\om}}
		}
	\app{\lem[wit={J5,G4,Gr2,V19,G5,N19,V1,J10,Jyo}]{sthiro}
		\rdg[wit={C7,V15,GrB}]{sthito}% +F
		%\rdg[wit={G11},alt={\om}]{\skp{\om}}
		}
	\app{\lem[wit={G4,V19,C7,G5,N19,V15,V1,J10,GrB,Jyo}]{dehe}
		\rdg[wit={Gr2}]{deho}
		\rdg[wit={J5}]{hahe}
		%\rdg[wit={G11},alt={\om}]{\skp{\om}}
		}}
\pada{tāva\app{\lem[wit={G4,V19,C7,G11,G5,V1,J10,GrB},alt={mṛtyubhayaṃ kutaḥ}]{\skm{n }mṛtyubhayaṃ kutaḥ}% +N24
		\rdg[wit={J5,Gr2,N19,Jyo}]{kālabhayaṃ kutaḥ}% +K1
		\rdg[wit={V15}]{jīvanam ucyate}}//\versenr}
		%\lineom{c}{G11}
		%\unavbl{N3}
	\\!}
\end{tlg}
\commcite\newpage


\begin{tlg}[hp03_085]
\tl{\app{\lem[nolem]{}
	\rdg[wit={N3}]{\unavbl}}%
\pada{\app{\lem[resp=emend]{manāyattaṃ}
		\rdg[wit={G4}]{anāyattaṃ}% < manāyattaṃ
		\rdg[wit={J5}]{manomayaṃ}
		\rdg[wit={GrB}]{manodhīnaṃ}% °dhinaṃ P11
		\rdg[wit={J7,V19,C7,G11,G5,V15,V1,J10,Jyo}]{cittāyattaṃ}% cittayuktaṃ F
		\rdg[wit={N23}]{cittamattaṃ}
		\rdg[wit={N19}]{cintāyatnaṃ}
		}
	\app{\lem[wit={ceteri}]{nṛṇāṃ}
		\rdg[wit={J5}]{taṃ nṛ}
		\rdg[wit={C6}]{bhavet}}
	\app{\lem[wit={ceteri}]{śukraṃ}
		\rdg[wit={G4,G11,G5,V3}]{śuklaṃ}}}
\pada{\app{\lem[wit={J5,Gr2,V19,C7,N19,V15,V1,J10,Jyo}]{śukrāyattaṃ}% °yatvaṃ J5
		\rdg[wit={G11,G5}]{śuklāyattaṃ}
		\rdg[wit={V3}]{śuklāyataṃ}
		\rdg[wit={G4}]{śuklā\,+\,+}
		\rdg[wit={P11,C6}]{śukrādhīnaṃ}}
	\app{\lem[wit={G11,G5,N19,V1,J10,P11,V3}]{hi}
		\rdg[wit={J5,Gr2,V19,C6}]{tu}
		\rdg[wit={C7,V15,Jyo}]{ca}}
	\app{\lem[wit={ceteri}]{jīvitam}
		\rdg[wit={J7,C6}]{jīvanaṃ}}/}\\+}
\tl{
\pada{tasmā\app{\lem[wit={ceteri},alt={chukraṃ}]{\skp{c }chukraṃ}% V3 correct here; G4 damaged; tasmā<<cchu>kraṃ J7
		\rdg[wit={G11}]{śuklaṃ}
		\rdg[wit={G5}]{tūrṇaṃ}}
	\app{\lem[wit={ceteri}]{manaś caiva} % + J10pc
		\rdg[wit={V1}]{manaś caivaṃ}
		\rdg[wit={J10}]{rajaś caiva}
		\rdg[wit={C7}]{rakṣaṇīyaṃ}}} % + J7pc
\pada{\app{\lem[wit={ceteri}]{rakṣaṇīyaṃ}% °yaḥ C6
		\rdg[wit={C7}]{yogibhiś ca}} 
		prayatnataḥ//\versenr} % prayet tataḥ N23
		\label{manayattam}
		%\unavbl{N3}
	\\!}
\end{tlg}
\commcite%\newpage


\begin{tlg}[hp03_086]
\tl{\app{\lem[nolem]{}
	\rdg[wit={N3}]{\unavbl}}%
\pada{\app{\lem[wit={ceteri}]{ṛtumatyā} % ritu° J5, rutu° V3, ṛtumalyā N23, °matyāṃ G11 ; J5,Gr2,G11,G5,N19,V15,GrB,Jyo
		\rdg[wit={V19,C7,V1,J10}]{bindumadhye}
		} 
	\app{\lem[wit={ceteri}]{rajo}
		\rdg[wit={P11}]{nijo}}% 
	\app{\lem[wit={ceteri}]{'py evaṃ} % +J10pc
		\rdg[wit={J10}]{py eva}
		\rdg[wit={V3}]{thevaṃ}
		\rdg[wit={P11}]{strījaṃ}}}
\pada{\app{\lem[wit={P11}]{svīyaṃ}% dvayaṃ F % ṛtumatyā nijo(rajo?) strījaṃ svīyaṃ biṃdu ca P11
		\rdg[wit={J5}]{vīryaṃ}
		\rdg[wit={V19,G11,N19,V15,V1,J10}]{bījaṃ}
		\rdg[wit={C7}]{jīvaṃ}
		\rdg[wit={Gr2}]{striyā}
		\rdg[wit={G5,C6}]{biṃduṃ}
		\rdg[wit={V3}]{jayaṃ}
		\rdg[wit={Jyo}]{nijaṃ}% + nija/nijaṃ Gr7
		}\marmas % P7 similar to C6
	\app{\lem[wit={ceteri}]{binduṃ}
		\rdg[wit={J10,P11,V3}]{bindu}
		\rdg[wit={G5}]{bījaṃ}
		\rdg[wit={C6}]{rakṣe}}
	\app{\lem[wit={J5,Gr2,V19,V15,J10,P11,V3,Jyo}]{ca}
		\rdg[wit={G11,G5,N19,V1,C6}]{tu}
		\rdg[wit={C7}]{pra°}}
	\app{\lem[wit={ceteri}]{rakṣayet}
		\rdg[wit={V3}]{rakṣayan}
		\rdg[wit={C6}]{yogavit}
		\rdg[wit={V19}]{taṃnnayet}
		\rdg[wit={C7}]{°pālayet}
		}/}\\+}
\tl{
\pada{\app{\lem[wit={J7,G11,N19,V1,C6,V3,Jyo}]{meḍhreṇā} % meṃḍhre° N19,V1
		\rdg[wit={J5,V19,C7,V15}]{meḍhreṇa}
		\rdg[wit={P11}]{meṃḍhraṇā}
		\rdg[wit={N23}]{meḍhrā}
		\rdg[wit={G5}]{meḍhram ā}
		\rdg[wit={J10}]{meḍhrām ā}
		}%
	\app{\lem[wit={ceteri},alt={karṣayed}]{karṣaye\skp{d}}% kardhayed G4
		\rdg[wit={V3}]{karṣayad}
		\rdg[wit={J10}]{kuṃcayed}}d ūrdhvaṃ} % ūrdhaṃ V19, ṃ om. N23
\pada{samyag abhyāsa%
	\app{\lem[wit={G4,G11,G5,N19,V15,P11,V3}]{yogavān}% +F
		\rdg[wit={Gr2,V19,C7,V1}]{yogataḥ}
		\rdg[wit={J10,Jyo}]{yogavit}
		\rdg[wit={J5,C6}]{pāṭavāt}}//\versenr}
		%\unavbl{N3}
	\\!}
\end{tlg}
\commcite\newpage


\startaltnormal
\begin{alttlg}[hp03_086_1]
\tl{\app{\lem[nolem]{}
	\rdg[wit={J5,G4}]{\excl}
	\rdg[wit={N3}]{\unavbl}
	\rdg[wit={V15},alt=\textapp{pādas b and d are transposed}]{\skp{pādas b and d are transposed}}
	\rdg[wit={Jyo},alt=\textapp{found after 3.89}]{\skp{found after 3.89}}
	}%
\pada{ayaṃ yogaḥ puṇyavatāṃ} % arya P11
\pada{\app{\lem[wit={ceteri}]{dhanyānāṃ}
		\rdg[wit={Jyo}]{dhīrāṇāṃ}}
	tattva\app{\lem[wit={J7,V19,C7,G11,G5,V15,V1,P11,C6}]{śālinām} % tacca P11; tattva in V19 unclear
		\rdg[wit={N19,V3}]{śālinaṃ}
		\rdg[wit={N23}]{sattināṃ}
		\rdg[wit={J10,Jyo}]{darśinām}}/} 
		%\Allexcept{J5,G4}
		\\+}
\tl{
\pada{\app{\lem[nolem]{\skp{pāda c}}
	\rdg[wit={C7},alt={\om}]{\skp{\om}}}%
	nirmatsarāṇāṃ % ni<r>° N23, °nāṃ V19, dur° G5
	\app{\lem[wit={N23,V19,G11,G5,N19,V15,V1,P11,V3}]{sidhyeta} % sidhyaita N23
		\rdg[wit={J7}]{siddheta}
		\rdg[wit={J10}]{siddhet*}
		\rdg[wit={Jyo}]{vai sidhyen}
		\rdg[wit={C6}]{siddhānāṃ}}}
\pada{\app{\lem[nolem]{\skp{pāda d}}
	\rdg[wit={C7},alt={\om}]{\skp{\om}}}%
	na tu 
	matsara\app{\lem[wit={Gr2,V19,G11,G5,V15,V1,GrB,Jyo}]{śālinām}
		\rdg[wit={N19}]{śālinaṃ}
		\rdg[wit={J10}]{śīlinām}}//\versenr} 
		%\Allexcept{J5,G4,C7}
		%\myfn{In \getsiglum{V15} Pāda b and d are transposed; 
		%\getsiglum{Jyo} has this verse at the end of the Sahajolī section.}
		%\unavbl{N3}
	\\!}
\end{alttlg}
\altcommcite%\newpage
\endaltnormal


% \begin{altava}[hp03_087a]
% \app{\lem[nolem]{}
	% \rdg[wit={J7,J10,Jyo}]{\incl}}%
% atha \app{\lem[wit={J7,J10}]{sahajolī}
		% \rdg[wit={Jyo}]{sahajoliḥ}}/
	% %\sgwit{J7,J10,Jyo}
% \end{altava}


\begin{tlg}[hp03_087]
\tl{\app{\lem[nolem]{}
	\rdg[wit={G11,G5},alt={\om}]{\skp{\om}}
	\rdg[wit={N3}]{\unavbl}
	}%
\pada{\app{\lem[alt={\ante sahajolī \add},nosep]{\skp{\ante sahajolī \add}}
		\rdg[wit={J7,J10}]{atha sahajolī}
		\rdg[wit={Jyo}]{atha sahajoliḥ}	
		}%
	\app{\lem[wit={J5,V19,C7,V1,J10,C6}]{sahajolī}% °yolī V19
		\rdg[wit={Gr2,N19,V15,V3,Jyo}]{sahajoliś} % Jyo-ed
		\rdg[wit={P11}]{sahajolāṃś}
		\rdg[wit={G4}]{sahajaś}}
	\app{\lem[wit={J5,V19,C7,P11,C6}]{cāmarolī}
%		\rdg[wit={N4}]{cāmarolīr}
		\rdg[wit={N19,V3}]{cāmaroli}
		\rdg[wit={V15,Jyo}]{cāmarolir}
		\rdg[wit={J10}]{vāmarolī}
		\rdg[wit={V1}]{cāmarolī ca}
		\rdg[wit={Gr2}]{cāmaroliś ca}% °rāliś? N23
		\rdg[wit={G4}]{camaronauḷi}}}
\pada{\app{\lem[wit={J5,G4,Gr2,N19,V15,V1,J10,P11,V3,Jyo}]{vajrolyā} % vajrāḷyā V15
		\rdg[wit={V19,C7}]{vajrolyante}
		\rdg[wit={C6}]{vajrolī}}
	\app{\lem[wit={J5,Gr2,N19,V15,V1,GrB}]{eva bhedataḥ}
		\rdg[wit={J10}]{ekabhedataḥ} % eka -> nāma J10pc
		\rdg[wit={Jyo}]{bheda ekataḥ}
		\rdg[wit={V19}]{prakīrtitā}
		\rdg[wit={C7}]{pracodyate}}//\versenr}
%		\NotIn{G11,G5}
%		\unavbl{N3}
	\\!}
\end{tlg}

%\avacite{87a}
\commcite\newpage
		

\begin{tlg}[hp03_088]
\tl{\app{\lem[nolem]{}
	\rdg[wit={N3}]{\unavbl}}%
\pada{\app{\lem[nolem]{\skp{pāda a}}
		\rdg[wit={J5,G4,N23,GrB},alt={\om}]{\skp{\om}}}%
	\app{\lem[wit={J7,V19,N19,V15,V1,J10},alt={jaleṣu bhasma}]{jaleṣu\marmas bhasma}% +F
		\rdg[wit={G11,G5}]{jale tu bhasma}
		\rdg[wit={Jyo}]{jale subhasma}
		\rdg[wit={C7}]{jale bhasmani}}
	\app{\lem[wit={V19,C7,G11,G5,N19,V15,V1,J10,Jyo}]{nikṣipya}
		\rdg[wit={J7}]{niḥkṣipya}}}
\pada{\app{\lem[nolem]{\skp{pāda b}}
		\rdg[wit={J5,G4,N23,GrB},alt={\om}]{\skp{\om}}}%
	\app{\lem[wit={J7,V19,C7,G11,V15,V1,J10,Jyo}]{dagdha}
		\rdg[wit={N19}]{dagdhaṃ}
		\rdg[wit={G5}]{daṇḍa}}gomaya%
	\app{\lem[wit={J7,V19,G11,N19,V15,V1,J10,Jyo}]{saṃbhavam}
		\rdg[wit={C7,G5}]{sambhave}}/}
%		\lineom{ab}{J5,G4,N23,GrB}
%		\sgwit{J7,V19,C7,N19,V15,V1,J10,Jyo}
		\\+}
\tl{
\pada{vajrolī\app{\lem[wit={ceteri},alt={maithunād}]{maithunā\skp{d}}
		\rdg[wit={V15}]{mithunād}% vajroḷi V15
		\rdg[wit={G5}]{madhanād}
		}d ūrdhvaṃ} % ṃ om. G5,J10; ūrdhaṃ V19
\pada{\app{\lem[wit={J7,G5,N19,V1,J10,Jyo}]{strīpuṃsoḥ}% puṃsoṃ J10ac
		\rdg[wit={J5,G11,V3}]{strīpuṃso}
		\rdg[wit={P11}]{puṃsostrī}
		\rdg[wit={N23}]{strīpuṃsā}
		\rdg[wit={V15}]{strīpuṃsau}
		\rdg[wit={V19,C7,C6}]{strīpuṃsoś}}
	\app{\lem[wit={J5,Gr2,G11,N19,V15,V1,J10,P11,Jyo}]{svāṅga}
		\rdg[wit={V3}]{svāṃgu}
		\rdg[wit={G5}]{sāṅga}
		\rdg[wit={V19,C7,C6}]{cāṃga}}lepanam//\versenr} % lepavān J5, lāpanaṃ C6
		%\unavbl{N3}
	\\!}
\end{tlg}
\commcite\newpage


\begin{tlg}[hp03_089]
\tl{\app{\lem[nolem]{}% ab written in margin V19
	\rdg[wit={N3}]{\unavbl}}% 
\pada{\app{\lem[wit={ceteri}]{āsīnayoḥ} % °yot N19, °yo P11
		\rdg[wit={V15}]{anenaiva}}
	\app{\lem[wit={ceteri}]{sukhenaiva}
		\rdg[wit={J10}]{mukhenaiva}}}
\pada{mukta%
	\app{\lem[wit={Gr2,C7,G5,V15,V1,P11,C6,Jyo}]{vyāpārayoḥ}
		\rdg[wit={J5,G11,N19,J10}]{vyāpārayo}
		\rdg[wit={V3}]{vyāpāramo}
		\rdg[wit={V19}]{vyāpārala°}}
	\app{\lem[wit={J5,V19,C7,G5,N19,V15,V1,J10,GrB}]{kṣaṇam}
		\rdg[wit={Gr2,G11,Jyo}]{kṣaṇāt}}/}\\+} % {{ra}}kṣaṇāt N23
\tl{
\pada{\app{\lem[wit={J5,J7,V19,C7,G11,G5,V1,J10,P11},alt={sahajolīr}]{sahajolī\skp{r}}% sahayolīr V19; +r G4
		\rdg[wit={N23,N19,V15,V3,Jyo}]{sahajolir}
		\rdg[wit={C6}]{sahajolī}
		}r iyaṃ proktā} % iva J5
\pada{\app{\lem[wit={G11,P11,V3,Jyo}]{śraddheyā}% = DYŚ
		\rdg[wit={J5,G4,V19,C7,G5,V1,C6}]{śraddhayā}% +M1,M3,G5,G7,F ##
		\rdg[wit={J10}]{sādhyeyā}
		\rdg[wit={V15}]{siddhaye}
		\rdg[wit={Gr2,N19}]{sevyate}} % +C2
		yogibhiḥ sadā//\versenr}
		%\unavbl{N3}
	\\!}
\end{tlg}
\commcite%\newpage

\startaltnormal
\begin{alttlg}[hp03_089_1] % From hier 2.5 verses om. in N23
\tl{\app{\lem[nolem]{}
	\rdg[wit={J5,G4,N23}]{\excl}
	\rdg[wit={N3}]{\unavbl}}%
\grau{\pada{ayaṃ śubhakaro yogo} % śu<bha> G11
\pada{\app{\lem[wit={J7,G11,G5,V15,J10,P11,V3}]{bhoge}
		\rdg[wit={N19,V1,C6,Jyo}]{bhoga}
		\rdg[wit={C7}]{yoga}
		\rdg[wit={V19},alt={\lacuna}]{\skp{\lacuna}}}
	\app{\lem[wit={J7,V15,J10,V3}]{bhukte'pi}
		\rdg[wit={G11}]{bhuktyapi}
		\rdg[wit={P11}]{yuktepi}
		\rdg[wit={Jyo}]{yukto'pi}
		\rdg[wit={C6}]{yogepi}
		\rdg[wit={N19}]{muktepi}
		\rdg[wit={G5}]{muktyapi}
		\rdg[wit={C7,V1}]{muktivi°}
		\rdg[wit={V19}]{muktipra°}
		}
	\app{\lem[wit={ceteri}]{muktidaḥ}
		\rdg[wit={G11}]{muktidā}
		\rdg[wit={V19}]{dāyakaḥ}}//\versenr}}\label{III94}
	%\Allexcept{J5,G4,N23}
	\anm{=\,\ref{III101}cd}%\unavbl{N3}
	\\!}  % M1,F omit too, but M3 has it.
\end{alttlg}
\altcommcite\newpage
\endaltnormal


% \begin{altava}[hp03_090a]
% \app{\lem[nolem]{}
	% \rdg[wit={J7,V19,C7,V15,J10,Jyo}]{\incl}}%
% \app{\lem[wit={J7,J10,Jyo}]{atha amarolī}
		% \rdg[wit={V15}]{āthamāroḷi}
		% \rdg[wit={V19,C7}]{tatrāmarolī}}/ 
	% %\sgwit{J7,V19,C7,V15,J10,Jyo}
% \end{altava}


\begin{tlg}[hp03_090]
\tl{\app{\lem[nolem]{}
	\rdg[wit={N23,G11,G5},alt={\om}]{\skp{\om}}
	\rdg[wit={N3}]{\unavbl}}%
\pada{\app{\lem[alt={\ante pittolbaṇa° \add},nosep]{\skp{\ante pittolbaṇa° \add}}
		\rdg[wit={J7,J10,Jyo}]{atha amarolī}
		\rdg[wit={V15}]{āthamāroḷi}
		\rdg[wit={V19,C7}]{tatrāmarolī}}%
	\app{\lem[wit={J5,G4,V19,C7,V15,V1,P11,V3,Jyo},alt={pittolbaṇatvāt}]{pittolbaṇatvā\skp{t}} % °lvana° V19; °tvān J5
		\rdg[wit={C6}]{pītvā aṇut}
		\rdg[wit={N19}]{virttaṇatvāḍyat}
		\rdg[wit={J10}]{vihāya nityāṃ}
		\rdg[wit={J7}]{vihāya nīv\,..\,ḥ}}%
	\app{\lem[wit={J5,N19,V15,J10,P11,C6},alt={prathamāṃ ca}]{\skm{t }prathamāṃ ca}% +F
		\rdg[wit={G4,J7}]{prathamaṃ ca}
		\rdg[wit={V3}]{prathamaṃ vi}
		\rdg[wit={V19}]{prathamāṃ}
		\rdg[wit={C7,V1,Jyo}]{prathamāmbu}}
	\app{\lem[wit={ceteri}]{dhārāṃ}% ṃ om. P11
		\rdg[wit={J5}]{dhārī}
		\rdg[wit={V19},alt={\om}]{\skp{\om}}}}\\+}
\tl{
\pada{vihāya
	\app{\lem[wit={J5,V19,V15,V1,J10,P11,Jyo}]{niḥsāratayāntya}% °vya P11
		\rdg[wit={C7}]{niḥsārabhayāntya}
		\rdg[wit={J7}]{niḥsāralayāṃtya}
		\rdg[wit={V3}]{niḥsārayāṃtya}
		\rdg[wit={N19}]{niḥsmāratayāṃtya}
		\rdg[wit={C6}]{niḥsāratapāṃśu}
		}dhārām/}\\+}
\tl{
\pada{\app{\lem[wit={ceteri}]{niṣevyate}
		\rdg[wit={C6}]{niṣevite}
		\rdg[wit={V1,P11}]{niḥsevyate}% śe P11
		\rdg[wit={V3}]{nikhyevyate}}
	śītalamadhya\app{\lem[wit={J5,N19,V15,J10,P11,V3,Jyo}]{dhārā}
		\rdg[wit={J7,C7,V1,C6}]{dhārāṃ}
		\rdg[wit={V19}]{dhārāḥ}}}\\+}
\tl{
\pada{\app{\lem[wit={J5,V19,C7,N19,P11,V3}]{kāpālikaiḥ}
		\rdg[wit={J7,V15,V1,J10}]{kapālikaiḥ}
		\rdg[wit={C6}]{kapālakaiḥ}
		\rdg[wit={Jyo}]{kāpālike}}
	\app{\lem[wit={V1,GrB},alt={khaṇḍamatair}]{khaṇḍamatai\skp{r}}
		\rdg[wit={N19}]{khaṃḍamitair}
		\rdg[wit={V15,Jyo}]{khaṃḍamate}
		\rdg[wit={J5}]{ṣaḍamatair}
		\rdg[wit={V19,C7}]{kaṃṭhamaṭhair}
		\rdg[wit={J7,J10}]{kuṃṭhamatair}}% kaṃṭha J10pc
	\app{\lem[wit={V19,C7,N19},alt={amaryāḥ}]{\skm{r }amaryāḥ}
		\rdg[wit={J5,P11,C6}]{amaryā}
		\rdg[wit={V3}]{aryā}
		\rdg[wit={V1,J10}]{amedhyā}
		\rdg[wit={J7}]{amedhyāṃ}
		\rdg[wit={Jyo}]{'marolī}
		\rdg[wit={V15}]{'maroḷi}}//\versenr}%
	\myfn{\getsiglum{J7} seems to have supplied this verse and the next one from a manuscript belonging to the {\texteta} group.}
	\\!}
%	\NotIn{N23,G11,G5}% M3 omits too
\end{tlg}

%\avacite{90a}
\commcite\newpage


\begin{tlg}[hp03_091]
\tl{\app{\lem[nolem]{}
	\rdg[wit={N23,G11,G5},alt={\om}]{\skp{\om}}
	\rdg[wit={N3}]{\unavbl}}%
\pada{\app{\lem[wit={J7,V19,C7,J10,Jyo}]{amarīṃ}
		\rdg[wit={J5,N19,V15,V1,P11,V3}]{amarī}
		\rdg[wit={C6}]{amariṃ}}
	\app{\lem[wit={J5,V19,C7,N19,V15, GrB,Jyo}]{yaḥ} % + J10pc
		\rdg[wit={V1}]{ya[ṃ]}
		\rdg[wit={J7,J10}]{yo}}
	\app{\lem[wit={ceteri},alt={piben}]{pibe\skp{n}} % paven V19
		\rdg[wit={C7}]{piban}}n nityaṃ}
\pada{\app{\lem[wit={V19,C6,V3,Jyo},alt={nasyaṃ kurvan}]{nasyaṃ kurva\skp{n}}% kurvana V3
		\rdg[wit={G4,N19,V15}]{naśyaṃ kurvan}
		\rdg[wit={C7}]{nasaṃ kurvan}
		\rdg[wit={P11}]{tṛśya kurvan}
		\rdg[wit={V1}]{naśyaṃ kuryād}
		\rdg[wit={J5}]{nasya kuryā}
		\rdg[wit={J7}]{tasya kuryā}
		\rdg[wit={J10}]{tasthaṃ kuryād}}n\marmas % nāśāraṃdhrā J10pc
	dine dine/}\\+}
\tl{
\pada{\app{\lem[wit={G4,V19,C7}]{vajrolīṃ cā}
		\rdg[wit={N19,V15,V1,V3}]{vajrolī cā}
		\rdg[wit={P11}]{vajrolī vā}
		\rdg[wit={J7,J10,Jyo}]{vajrolīm a}
		\rdg[wit={C6}]{vajrolī ka}
		\rdg[wit={J5}]{vijrolī sā}}%
	\app{\lem[wit={J5,G4},alt={°bhyased evam}]{bhyased eva\skp{m}}
		\rdg[wit={V19,C7,N19,V15}]{bhyasec ceyam}% ceya ama° N19
		\rdg[wit={V3}]{bhyaset seyam}
		\rdg[wit={P11}]{bhyasevoyam}
		\rdg[wit={V1}]{bhyasen nityaṃ} % nityaṃ a°
		\rdg[wit={J7}]{bhyaset satve}
		\rdg[wit={J10}]{bhyasec chattve} % bhyased eva J10pc
		\rdg[wit={Jyo}]{bhyaset samyak}
		\rdg[wit={C6}]{thyate seyam}}}%
\pada{\app{\lem[wit={ceteri},alt={amarolīti}]{\skm{m }amarolīti}% +G4
		\rdg[wit={Jyo}]{sāmarolīti}
		\rdg[wit={J5}]{amarolī tu}
		\rdg[wit={V15}]{amaroḷīṃ tu}}
	\app{\lem[wit={ceteri}]{kathyate}% °ti P11
		\rdg[wit={V15}]{kalpayet}
		\rdg[wit={J10}]{kasyate}}//\versenr}\label{III96}%
	\myfn{After this verse, \getsiglum{Jyo} has \ref{VuIII98}.}
	\\!}
%	\NotIn{N23,G11,G5} % M3 omits too
\end{tlg}
\commcite\newpage

\startaltnormal
\begin{alttlg}[hp03_091_1]
\tl{\app{\lem[nolem]{}
	\rdg[wit={J5,G4}]{\excl}
	\rdg[wit={N3}]{\unavbl}}%\unavbl{N3}%
\pada{\app{\lem[wit={N23,V19,C7,G11,N19,V15,J10,P11,Jyo}]{puṃso}
		\rdg[wit={J7,C6}]{puṃsor}
		\rdg[wit={G5,V1,V3}]{puṃsāṃ}}
	\app{\lem[wit={J7,V19,C7,G11,G5,V1,C6,Jyo}]{binduṃ}
		\rdg[wit={N23,N19,V15,J10,P11,V3}]{bindu}} % bidu P11
	\app{\lem[wit={Gr2}]{samākṛṣya}
		\rdg[wit={V19,C7,G11,G5,N19,V15,V1,J10,GrB,Jyo}]{samākuñcya}% °cyāt G5
		}}\label{III99} % svam P11
\pada{samyagabhyāsa% N19 om. sa
	\app{\lem[wit={V19,C7,G5,V15,J10,P11,C6,Jyo}]{pāṭavāt}
		\rdg[wit={Gr2,G11,N19,V3}]{pāṭavān}
		\rdg[wit={V1}]{pāravān}}/}\\+}  %3.97
\tl{
\pada{yadi nārī rajo rakṣe}%d % rakṣe N19
\pada{\app{\lem[wit={J7,G11,G5,N19,V1,P11,V3,Jyo},alt={vajrolyā}]{\skm{d }vajrolyā}
		\rdg[wit={V19}]{vajrolyāṃ}
		\rdg[wit={C7}]{vajrolya}
		\rdg[wit={C6}]{vajrolī}
		\rdg[wit={V15}]{vajroḷi}
		\rdg[wit={J10}]{saṃyoge}
		\rdg[wit={N23},alt={\om},postwit=\texteng{(jumps to \ref{III101}b)}]{\skp{\om\ (jumps to 3.93b)}}}
	\app{\lem[wit={G11,G5,V3}]{sā hi}
		\rdg[wit={J7,N19,P11,C6}]{saha}
		\rdg[wit={V19,V15,V1,Jyo}]{sāpi}% +F
		\rdg[wit={C7}]{syāpi}
		\rdg[wit={J10}]{cāpi}
		\rdg[wit={N23},alt={\om}]{\skp{\om}}}
		yoginī//\versenr}\label{III97} 
		%\Allexcept{J5,G4}
		\\!}
\end{alttlg}
\altcommcite\newpage
	%\anm{\getsiglum{N23} om. 97d--100c by haplogr.?}


\begin{alttlg}[hp03_091_2]
\tl{\app{\lem[nolem]{}
	\rdg[wit={J5,G4,N23}]{\excl}
	\rdg[wit={N3}]{\unavbl}}%
\pada{tasyāḥ kiṃcid rajo nāśaṃ} % tasyā P11,N4,J10; nāśa J10
\pada{na gacchati % gakṣati V19; gacchaṃti N19
		na saṃśayaḥ/}\\+} % śaṃsayaḥ V19
\tl{
\pada{\app{\lem[nolem]{\skp{pāda c}}
	\rdg[wit={J7},alt={\om}]{\skp{\om}}}%
\app{\lem[wit={V19,C7,V15,V1,J10,P11,C6,Jyo}]{tasyāḥ}% tasyā P11
		\rdg[wit={G11}]{tasya}
		\rdg[wit={G5}]{tasmāt}
		\rdg[wit={N19}]{yasyāḥ}
		\rdg[wit={V3}]{asyāḥ}
		}
	\app{\lem[wit={V19,G11,N19,V1,J10,C6,V3,Jyo}]{śarīre}
		\rdg[wit={C7,V15}]{śarīra}
		\rdg[wit={P11}]{śarīre pi}
		\rdg[wit={G5}]{tu jarja°}
		}
	\app{\lem[wit={C7,N19,V15,V1,GrB}]{nādas tu}
		\rdg[wit={J10}]{nādas tat}
		\rdg[wit={V19}]{nādātmā}
		\rdg[wit={Jyo}]{nādaś ca}
		\rdg[wit={G11}]{nāstu}
		\rdg[wit={G5}]{°re nādaḥ}
		}}
\pada{\app{\lem[nolem]{\skp{pāda d}}
	\rdg[wit={J7},alt={\om}]{\skp{\om}}}%
\app{\lem[wit={V19,C7,N19,V15,V1,P11,V3,Jyo}]{bindutām eva}
		\rdg[wit={G11}]{bindutām etra}
		\rdg[wit={G5}]{bindutām atra}
		\rdg[wit={J10}]{bindus tam eva}
		\rdg[wit={C6}]{vyaṃjatām eva}
		} gacchati//\versenr}
		%\Allexcept{J5,G4,N23}
	\\!}
\end{alttlg}
\altcommcite\newpage


\begin{alttlg}[hp03_091_3]
\tl{\app{\lem[nolem]{}
	\rdg[wit={J5,G4,N23}]{\excl}
	\rdg[wit={N3}]{\unavbl}}%
\pada{sa bindus tad rajaś caiva} % svabiṃdu? V15, binduṃ G5
\pada{\app{\lem[wit={ceteri}]{ekī}
		\rdg[wit={C7}]{hy ekī}
		}%
	\app{\lem[wit={J7,V19,C7,G11,G5,V15,C6,V3,Jyo}]{bhūya}
		\rdg[wit={N19}]{bhūyaḥ}
		\rdg[wit={V1,J10}]{bhūtaḥ}
		\rdg[wit={P11}]{bhūta}
		}
	\app{\lem[wit={J7,G11,G5,N19,V15,V1,P11,V3}]{svadehajau}
		\rdg[wit={J10,C6}]{svadehajaiḥ}
		\rdg[wit={C7}]{svadehajam}
		\rdg[wit={V19}]{sadehajaṃ}
		\rdg[wit={Jyo}]{svadehagau}
		}\marma/}\\+}
\tl{
\pada{\app{\lem[wit={G11,G5,N19,V15,V1,J10,GrB}]{vajrolyā}
		\rdg[wit={J7,V19,C7,Jyo}]{vajroly-a°}% +F
		}bhyāsayogena} % °geva N19
\pada{\app{\lem[wit={ceteri}]{sarva}
		\rdg[wit={C6}]{sarvāṃ}}% 
	\app{\lem[wit={J7,V19,G5,N19,V15,P11,C6,Jyo}]{siddhiṃ}
		\rdg[wit={G11,V1,V3}]{siddhi}
		\rdg[wit={C7,J10}]{siddhiḥ}
		} % +J10pc,M1
	\app{\lem[wit={G5,N19,V15,P11}]{prakurvataḥ}% +M1
		\rdg[wit={G11}]{prakurvatā}
		\rdg[wit={J7,V19,C7}]{prakurvate}
		\rdg[wit={Jyo}]{prayacchataḥ}
		\rdg[wit={C6}]{prayacchati}
		\rdg[wit={V1,J10,V3}]{prajāyate}% = DYŚ
		}//\versenr}
		%\Allexcept{J5,G4,N23}%\unavbl{N3}
	\\!}
\end{alttlg}
\altcommcite\newpage
\endaltnormal


\begin{tlg}[hp03_092]
\tl{\app{\lem[nolem]{}
	\rdg[wit={N3}]{\unavbl}}%
	\app{\lem[alt=\mylem{92ab},nosep,nonum]{\skp{pāda ab}}
	\rdg[wit={J5,G4,J7,J10,Jyo}]{\incl}
%	\rdg[wit={N23},alt={\om}]{\skp{\om}}%
	\rdg[wit={J7},alt=\textapp{found betw. \ref{III96} and \ref{III99}}]{\skp{found betw. 3.91 and 3.91*1}}}%
\pada{\app{\lem[resp=emend]{rakṣed ākuñcanenordhvaṃ}
		\rdg[wit={G4}]{rakṣe[d ā]kuṃcane\,..\,+}
		\rdg[wit={J5}]{rakṣed ākuṃbhanonordhaṃ}
%	\rdg[wit={N24}]{rakṣaṇe kuṃcanenorddhva}
		\rdg[wit={Jyo}]{rakṣed ākuñcanād ūrdhvaṃ}
		\rdg[wit={J7}]{mehenākuṃcanād ūrdhva}
		\rdg[wit={J10}]{meḍhrām ākuṃcanād ūrdhvaṃ}}}
\pada{\app{\lem[wit={Jyo}]{yā rajaḥ sā hi yoginī}
		\rdg[wit={J5}]{yā rajaḥ saha yoginī}
		\rdg[wit={J10}]{rajasāpi hi yoginī}
		\rdg[wit={J7}]{rajasāpi hi yoginaḥ}
%	\rdg[wit={G4},alt={\illeg}]{\skp{\illeg}}
	}/}
	%\sgwit{J5,G4,J7,J10,Jyo}
%	\anm{betw. \ref{III96} \& \ref{III99} \getsiglum{J7}}
	\\+}
\tl{
% J5: rakṣed ākuṃbhanonordhaṃ yā rajaḥ saha yoginī
% N24: rakṣaṇe kuṃcanenorddhva yo radhaḥ saha yogavit
% G4: (damaged)
\pada{\app{\lem[wit={J5,Gr2,V19,C7,G11,G5,V1,J10,P11,V3,Jyo}]{atītānāgataṃ}
		\rdg[wit={C6}]{atītānāgate}
		\rdg[wit={V15}]{atītānāgatiṃ}
		\rdg[wit={N19}]{atītānāṃ gatiṃ}
		} vetti}
\pada{\app{\lem[wit={J5,Gr2,G11,G5,N19,V15,V1,GrB,Jyo}]{khecarī ca}% +J10pc
		\rdg[wit={J10}]{khecaraś ca}
		\rdg[wit={C7}]{khecarīṃ la°}
		\rdg[wit={V19}]{khecarīṃ}
		}
	\app{\lem[wit={N23,G11,G5,N19,V15,V1,J10,C6,V3,Jyo}]{bhaved dhruvam}% bhave V3,J10
		\rdg[wit={J5}]{bhave druvaṃ}
		\rdg[wit={P11}]{bhaved dṛḍhaṃ}
		\rdg[wit={V19,C7}]{°bhate dhruvam}
		\rdg[wit={J7}]{prajāyate}% +F
		}//\versenr}
		%\unavbl{N3}
	\\!}
\end{tlg}
\commcite\newpage


\begin{tlg}[hp03_093]
\tl{\app{\lem[nolem]{}
	\rdg[wit={J5},alt={\om}]{\skp{\om}}
	\rdg[wit={N3}]{\unavbl}}%
\pada{\app{\lem[nolem]{\skp{pāda a}}
	\rdg[wit={N23},alt={\om}]{\skp{\om}}}%
	dehasiddhiṃ % dehe C6
	\app{\lem[wit={ceteri}]{ca}
		\rdg[wit={V1}]{tu}
		}
	\app{\lem[wit={ceteri}]{labhate}% lābhate V19
		\rdg[wit={C6}]{labhyeta}
		}}
\pada{\app{\lem[wit={J7,V19,C7,Jyo}]{vajrolyabhyāsa}% +N24,F
		\rdg[wit={N23,G11,G5,N19,V15,V1,J10,GrB}]{vajrolyābhyāsa}% ā N23pc
		}yogataḥ/}  % +N24, yogavit G4
		\\+}
\tl{
\pada{\app{\lem[nolem]{\skp{pāda c}}
	\rdg[wit={Gr2},alt={\om}]{\skp{\om}}}%
\app{\lem[wit={G4}]{ayaṃ śubhakaro yogo}% +N24; sukhakaro F(1st)
		\rdg[wit={Jyo}]{ayaṃ puṇyakaro yogo}
		\rdg[wit={GrB}]{yasmād ayaṃ sādhakāya}
		\rdg[wit={V19,C7,G11,G5,N19,V15}]{tasmād ayaṃ sādhakāya}% +F(2nd)
		\rdg[wit={V1}]{tasmād ayaṃ sādhako'yaṃ}
		\rdg[wit={J10}]{tasmād ayaṃ sādhakānāṃ}
	}}
\pada{\app{\lem[nolem]{\skp{pāda d}}
	\rdg[wit={Gr2},alt={\om}]{\skp{\om}}}%
\app{\lem[wit={G11,P11,Jyo}]{bhoge bhukte'pi}
		\rdg[wit={V3}]{bhoge bhukti <<pi>>}
		\rdg[wit={V19}]{bhogabhukti(yogamukti \textit{ac})vi°}
		\rdg[wit={V15}]{bhogayukto pi}% +F; yogabhogavimuktidaḥ F(2nd)
		\rdg[wit={C6}]{bhogayoge pi}
		\rdg[wit={G5}]{bhoge mukte pi}
		\rdg[wit={N19}]{bhogamukte pi}
		\rdg[wit={G4,C7,V1,J10}]{bhogamuktivi°}
%		\rdg[wit={N24}]{bhyāsyayuktasya muktida}
		} muktidaḥ//\versenr}\label{III101}
	\\!}
	%\anm{cf. \ref{III94}cd}
\end{tlg}
\commcite%\newpage


\startaltnormal
\begin{alttlg}[hp03_093_1]
\tl{\app{\lem[nolem]{}
	\rdg[wit={J5,G4,Jyo}]{\excl}
	\rdg[wit={N3}]{\unavbl}}%
\pada{tasmāt puṇyavatā%m 
	\app{\lem[wit={Gr2,G5,J10,C6},alt={eva}]{\skm{m }eva}
		\rdg[wit={V19,C7,G11,N19,V15,V1,P11,V3}]{evam}}}
\pada{\app{\lem[wit={Gr2,G5,N19,V15,V1,GrB}]{ayaṃ yogaḥ}% yoga P11
		\rdg[wit={G11}]{ayaḥ yogaḥ}
		\rdg[wit={V19,C7}]{eṣa yogaḥ}
		\rdg[wit={J10}]{yogo'yaṃ sa<<ṃ>>}
		} prasidhyati}//% +ḥ N23
	\myfn{\getsiglum{N23} has a sub-colophon marking the end of chapter 3 after this verse, numbered 100 in this manuscript. Chapter 4 contains only 29 verses, which are the remaining verses of the usual chapter 3. Chapter 5 corresponds to the usual chapter 4. \mydelim 
	In the \textdelta\ manuscripts, this verse is the final verse, since the Vajrolī section has been moved to the end of the text.}
%Cf. the colophon of \texteng{P23} \devnote{iyaṃ vajrolī trayodaśe patre śakticālanāt pūrvaṃ jñātavyā// // iti śrīātmārāmamunīṃdraviracitāyāṃ haṭhadīpikāyaṃ(!) paṃcamopadeśaḥ// 5 //\versenr}
%\getsiglum{V19} has here: \devnote{iti haṭhayogapradīpikāyāṃ paṃcama upadeśaḥ// 5 // samāptoyaṃ graṃthaḥ// saṃvat 1707 jyeṣṭha kṛṣṇa 4 bhṛgau liṣitam idaṃ// // śubhaṃ// //\versenr}; \getsiglum{C7} \devnote{iti śrīmadātmārāmaviracitāyāṃ pañcamoyam upadeśaḥ// 5 // śubham astu sarvajagatām//\versenr};
%\Allexcept{J5,G4,Jyo}%\unavbl{N3}
	\\!}
\end{alttlg}
\altcommcite\newpage
\endaltnormal


\begin{ava}[hp03_093_2a]
\app{\lem[wit={ceteri}]{atha}% with G4
	\rdg[wit={Gr2,V19,V15},alt={\om}]{\skp{\om}}} % +K3,C7
\app{\lem[wit={ceteri}]{śakticālanam}
		\rdg[wit={N23}]{śaktiyānaṃ}
		\rdg[wit={J10}]{śakti}
		\rdg[wit={V15},alt={\om}]{\skp{\om}}}/
\end{ava}

\teimute{\vspace{-0.5ex}}
\avacite{93-2a}
\bigskip

\startaltnormal
\teimute{\setcounter{poemline}{2}\setcounter{altvnum}{1}}
\begin{alttlg}[hp03_093_2]
\tl{\app{\lem[nolem]{}
	\rdg[wit={J5,G4}]{\excl}
	\rdg[wit={N3}]{\unavbl}}%
\pada{\app{\lem[wit={Gr3,G11,G5,V1,J10,GrB,Jyo}]{kuṭilāṅgī}
		\rdg[wit={J7,N19,V15}]{kuṃḍalāṅgī}
		\rdg[wit={N23}]{kundalīgī}} kuṇḍalinī} % kuḍalinī V15
\pada{bhujaṅgī
\app{\lem[wit={ceteri}]{śaktir īśvarī}
		\rdg[wit={N23}]{śaktir asvarī}
		\rdg[wit={V19}]{śaktir aiśvarī}% +K3
		}/}\\+}
\tl{
\pada{\app{\lem[wit={ceteri},alt={kuṇḍaly}]{kuṇḍaly\skm{a}} % kuṃḍalasaṃdhanī N23
		\rdg[wit={Gr3}]{kuṭily}}%
\app{\lem[wit={ceteri},alt={arundhatī}]{\skp{a}rundhatī}
		\rdg[wit={P11}]{aruṃdhīti}
		\rdg[wit={V1}]{ā[ku]ṃḍalī}
		\rdg[wit={J10}]{āceti ruṃ°}}
\app{\lem[wit={G11,G5,V1,P11}]{ceti} % c[e]ti V1
		\rdg[wit={V3}]{veti}
		\rdg[wit={N19}]{cati}
		\rdg[wit={V15}]{caiva}
		\rdg[wit={C6,Jyo}]{caite}
		\rdg[wit={Gr2,Gr3}]{devī}
		\rdg[wit={J10}]{dhaṃti}}}
\pada{\app{\lem[wit={ceteri}]{śabdāḥ paryāyavācakāḥ} % śabdā V3,V1; °vācakā V3
		\rdg[wit={V19}]{śabdaḥ paryāyavācakaḥ}% +C7
		\rdg[wit={P11}]{śabdā cārvāk vācakāḥ}
		}//\versenr}
		%\Allexcept{J5,G4}%\unavbl{N3}
		\\!}
\end{alttlg}
\altcommcite\newpage

\begin{alttlg}[hp03_093_3]
\tl{\app{\lem[nolem]{}
	\rdg[wit={J5,G4}]{\excl}
	\rdg[wit={N3}]{\unavbl}
	\rdg[wit={N19},alt={transposed with the next verse}]{\skp{transposed with the next verse}}}%
\pada{\app{\lem[wit={ceteri},alt={udghāṭayet}]{udghāṭaye\skp{t}}
		\rdg[wit={N19}]{udghāṭayati}}t %
		kapāṭaṃ % kavāṭaṃ G11,G5
\app{\lem[wit={ceteri}]{tu}
		\rdg[wit={N19},alt={\om}]{\skp{\om}}}}
\pada{yathā
\app{\lem[wit={ceteri}]{kuñcikayā}% kuṃcīkayā V3
		\rdg[wit={C6}]{kuṃcukayā}} haṭhāt/}\\+} % haṭhan G11
\tl{
\pada{kuṇḍalinyā tathā yogī} % kuṃḍilinyā P11, °lilyā N23
\pada{mokṣadvāraṃ
\app{\lem[wit={ceteri}]{vibhedayet}
		\rdg[wit={N23}]{prabhedayet}
		\rdg[wit={J7}]{nirodhayet}}//\versenr}
%		\myfn{This verse and the next one are transposed in \getsiglum{N19}.} 
		%\Allexcept{J5,G4}%\unavbl{N3}
		\\!}
\end{alttlg}
\altcommcite%\newpage

\begin{alttlg}[hp03_093_4]
\tl{\app{\lem[nolem]{}
	\rdg[wit={J5,G4}]{\excl}
	\rdg[wit={N3}]{\unavbl}}%
\pada{yena \app{\lem[wit={G11,G5,N19,V15,V1,J10,GrB,Jyo}]{mārgeṇa}
		\rdg[wit={Gr2,Gr3}]{dvāreṇa}} gantavyaṃ}
\pada{brahmasthānaṃ  % °khyānaṃ N23, sthāna N19
	\app{\lem[wit={ceteri}]{nirāmayam}
	\rdg[wit={G11}]{anāmayaṃ}}/}\\+}
\tl{
\pada{mukhe\app{\lem[wit={ceteri},alt={ācchādya}]{\skm{n}ācchādya}
		\rdg[wit={V19}]{ākṣādya}
		\rdg[wit={N19}]{āvādya}}
\app{\lem[wit={N23,E2,J10}]{taddvāraṃ}
		\rdg[wit={J7,V19,G11,G5,N19,V15,V1,P11,V3,Jyo}]{tadvāraṃ}
		\rdg[wit={C6}]{taṃ dvāraṃ}
		}}
\pada{prasuptā parameśvarī//\versenr} 
	%\Allexcept{J5,G4}%\unavbl{N3}
	\\!}
\end{alttlg}
\altcommcite\newpage


\begin{alttlg}[hp03_093_5]
\tl{\app{\lem[nolem]{}
	\rdg[wit={J5,G4}]{\excl}
	\rdg[wit={N3}]{\unavbl}}%
\pada{\app{\lem[wit={Gr2,G11,N19,V15,P11}]{kandordhvaṃ}
		\rdg[wit={G5,V3}]{kandordhva}
		\rdg[wit={V19,V1,J10,Jyo}]{kandordhve}% V19pc,C7
		\rdg[wit={E2}]{kandhordhve}% V19ac $$
		\rdg[wit={C6}]{kaṃṭhorddhaṃ}
		} 
		kuṇḍalī śaktiḥ} % kuṃḍilī P11; śakti V3
\pada{\app{\lem[wit={G11,G5,V15,V1,J10,GrB,Jyo}]{suptā}
		\rdg[wit={Gr2,N19}]{buddhā}% burddhā N23, +K3
		\rdg[wit={E2}]{buddhvā}
		\rdg[wit={V19}]{baddhā}% +C7
		} 
		mokṣāya yoginām/}\\+} % yoginaṃ N23
\tl{
\pada{bandhanāya ca % °ṇāya N23
\app{\lem[wit={ceteri}]{mūḍhānāṃ}% °ṇāṃ V19
		\rdg[wit={J7}]{mūrkhāṇāṃ}}}
\pada{yas tāṃ vetti sa yogavit//\versenr}  % vartti N23; taṃ P11,J10ac
	%\Allexcept{J5,G4}%\unavbl{N3}
	\\!}\end{alttlg}
\altcommcite%\newpage


\begin{alttlg}[hp03_093_6]
\tl{\app{\lem[nolem]{}
	\rdg[wit={J5,G4,Jyo}]{\excl}
	\rdg[wit={N3}]{\unavbl}}%
\pada{ambhodhi% aṃbhodha N23, ambodhi E2
	\app{\lem[wit={Gr2,G11,G5,V15,V1,J10,P11,V3},alt={śailadvīpānām}]{śailadvīpānā\skp{m}} % aṃbhodha N23
		\rdg[wit={C6}]{śailordvagānām}
		\rdg[wit={N19}]{plauladvīpānām}
		\rdg[wit={Gr3}]{dvīpaśailānām}% dvepa K3
		}}% 
\pada{\app{\lem[wit={ceteri},alt={ādhāraḥ}]{\skm{m }ādhāraḥ}% ḥ om. P11
		\rdg[wit={J7}]{ādharaḥ}
		\rdg[wit={N19}]{ādhāraṃ}} 
		śeṣakuṇḍalī/} % śceṣa N23, syeṣa G5
		%\Allexcept{J5,G4,Jyo}
		\\+}
\tl{
\pada{\app{\lem[nolem]{\skp{pāda c}}
	\rdg[wit={V1},alt={\om}]{\skp{\om}}}%
	aśeṣayoga\app{\lem[wit={ceteri},alt={tantrāṇām}]{tantrāṇā\skp{m}} % +J10pc; tandrā°? G11
		\rdg[wit={J10}]{jagatām}}}%
\pada{\app{\lem[nolem]{\skp{pāda d}}
	\rdg[wit={V1},alt={\om}]{\skp{\om}}}%
	m ādhāraḥ % h. om. P11, °rā G11,G5
\app{\lem[wit={ceteri}]{kuṇḍalī tathā}
		\rdg[wit={V19}]{kuṇḍalī yathā}
		\rdg[wit={V15}]{śeṣakuṇḍalī}}//\versenr}
		%\Allexcept{J5,G4,Jyo}%\unavbl{N3}
		%\anm{cf.\,\manuref{3.1}}
		\\!}
\end{alttlg}
\altcommcite\newpage


\begin{alttlg}[hp03_093_7]
\tl{\app{\lem[nolem]{}
	\rdg[wit={J5,G4}]{\excl}
	\rdg[wit={N3}]{\unavbl}}%
\pada{kuṇḍalī % kuṃḍilī P11, kuṇḍalinī N19
\app{\lem[wit={Gr2,Gr3,G11,G5,N19,GrB,Jyo}]{kuṭilākārā}% °kā<<rā>> J7, °kāra G5
		\rdg[wit={V15}]{kuṃḍilākārā}
		\rdg[wit={V1}]{kuṃḍalākārā}
		\rdg[wit={J10}]{kuṭilākarī}}}
\pada{sarpavat parikīrtitā/}\\+} % sarvavat G11
\tl{
\pada{sā śaktiś cālitā yena} % sa˟ktiś P11; cātītā P11; ye<<na>> J7
\pada{sa mukto nātra saṃśayaḥ//\versenr}% 
	%\Allexcept{J5,G4}%\unavbl{N3}%
	\anm{=\,\manuref{4.62}}%
	\myfn{After this verse, \getsiglum{G11,G5} have \manuref{4.61 and 63--64}.}
	\\!}
\end{alttlg}
\altcommcite%\newpage
\endaltnormal

%\newpage

\begin{tlg}[hp03_094]
\tl{\app{\lem[nolem]{}
	\rdg[wit={N3}]{\unavbl}}%
\pada{gaṅgā\app{\lem[wit={ceteri},alt={yamunayor}]{yamunayo\skp{r}}
		\rdg[wit={J10,V3}]{yamunāyor}}r madhye} % jamunā° V3
\pada{\app{\lem[wit={ceteri}]{bālaraṇḍā}% vā in mss; bā V15,V1,K3
		\rdg[wit={P11,Jyo}]{bālaraṇḍāṃ}
		\rdg[wit={G11}]{bālārundhā}}
\app{\lem[wit={ceteri}]{tapasvinī}% tapasvīnī J5, tapaśvinī N19
		\rdg[wit={V19}]{tapaścānī}
		\rdg[wit={Jyo}]{tapasvinīm}
		\rdg[wit={P11,C6}]{sarasvatī}}/}\\+}
\tl{
\pada{balātkāreṇa gṛhṇīyāt} % gṛhnī° C6,Gr2,V19,N19, grahṇī° G11; °yā V15
\pada{tad viṣṇoḥ paramaṃ padam//\versenr}
%\unavbl{N3}
	\\!} % vidhmauḥ padamaṃ N23
\end{tlg}
\commcite\newpage

\startaltrecension
\begin{alttlg}[hp03_094_1]
\tl{\app{\lem[nolem]{}
	\rdg[wit={J5,J7,V1,J10,C6,V3,Jyo}]{\incl}}%
\pada{iḍā bhagavatī gaṅgā}
\pada{piṅgalā yamunā nadī/}\\+} % jamunā V3
\tl{
\pada{\app{\lem[wit={J7,V1,J10,C6,V3,Jyo},postwit=\texteng{(piṃgalāyor \getsiglum{J10,V3})}]{iḍāpiṅgalayor madhye\skp{ (piṃgalāyor \getsiglum{J10,V3})}}
	\rdg[wit={J5}]{tayor madhye prayāgaṃ tu}}} % ilā V1; piṃgalāyor V3,J10
\pada{\app{\lem[wit={J7,V1,J10,C6,V3,Jyo}]{bālaraṇḍā}
	\rdg[wit={J5}]{yas taṃ veda}}
	\app{\lem[wit={J7,V1,J10,C6,V3}]{sarasvatī}
		\rdg[wit={Jyo}]{ca kuṇḍalī}
		\rdg[wit={J5}]{sa vedavit}}//\versenr}
%\sgwit{J5,J7,V1,J10,C6,V3,Jyo} 
\anm{cf. the verse inserted after \manuref{4.77} in the manuscripts of the \textdelta\ group.}\\!}
\end{alttlg}
\altcommcite%\newpage
%\NotIn{N23,Gr3,G11,N19,V15,P11}
% Mbh Bhīṣmaparvan
%06,040.078d@003A_0041 iḍā bhagavatī gaṅgā piṅgalā yamunā nadī |
%06,040.078d@003A_0042 tayor madhye tṛtīyā tu tat prayāgam anusmaret || (21)


\endaltrecension


\begin{tlg}[hp03_095]
\tl{\app{\lem[nolem]{}
	\rdg[wit={N3}]{\unavbl}}%
\pada{\app{\lem[wit={ceteri}]{pucchaṃ}% puṃcchaṃ V3,N23
		\rdg[wit={G11,G5,J10,Jyo}]{pucche}}
\app{\lem[wit={J5,Gr2,Gr3,N19,J10,GrB,Jyo}]{pragṛhya}% praguhya J5
		\rdg[wit={G11,G5,V15}]{nigṛhya}
		\rdg[wit={V1}]{gṛhya}}
\app{\lem[wit={J7,Gr3,G11,G5,V15,C6,Jyo}]{bhujagīm}
		\rdg[wit={J5,N19,P11}]{bhujagī}
		\rdg[wit={N23,V3}]{bhujaṃgī}
		\rdg[wit={J10}]{bhujaṃgīṃ}
		\rdg[wit={V1}]{bhujaṃgīva}}}
\pada{suptā%m
\app{\lem[wit={J5,J7,Gr3,G11,G5,N19,V15,V1,J10,C6,V3,Jyo},alt={udbodhayed/c}]{\skm{m }udbodhaye\skp{d}}
%	°yed J5,C6,J7,V19,V15; °yec V3,V1,J10,Jyo
		\rdg[wit={G4}]{udyodhayeṃd}
		\rdg[wit={N23}]{udrodhyamed}
		\rdg[wit={P11}]{udbdhoyed (sic!)}}%
\app{\lem[wit={G4,Gr2,G11,G5},alt={abhīḥ}]{\skm{d }abhīḥ}
		\rdg[wit={J5,P11}]{abhī}
		\rdg[wit={N19,V15}]{abhiḥ}
		\rdg[wit={Gr3}]{api}
		\rdg[wit={V1,J10,V3,Jyo}]{ca tām}
		\rdg[wit={C6}]{balāt}}/}\\+}
\tl{
\pada{nidrāṃ vihāya sā
\app{\lem[wit={Gr2,E2,G11,G5,V15,V1,J10}]{ṛjvī}
		\rdg[wit={G4,V19,C6}]{ṛjvīm}% % ṃ not m J7; hiatus-bridger?
		\rdg[wit={P11}]{rījvīm}
		\rdg[wit={V3}]{rujvīm}
		\rdg[wit={N19}]{rajvī}
		\rdg[wit={J5}]{rajvām}
		\rdg[wit={Jyo}]{śaktir}}}
\pada{\app{\lem[wit={ceteri},alt={ūrdhvam}]{ūrdhva\skp{m}}% urddhvam J7, ūrdham V19
		\rdg[wit={N19}]{kurddham}}%
\app{\lem[wit={ceteri},alt={uttiṣṭhate}]{\skm{m }uttiṣṭhate}% urtti° N23
		\rdg[wit={N19}]{ākṛṣyate}}
\app{\lem[wit={ceteri}]{haṭhāt}% ṭhahāt G11
		\rdg[wit={C6}]{kṣaṇāt} % +V23
		}//\versenr}
	%\unavbl{N3}
	\\!}
\end{tlg}
\commcite\newpage


\startaltnormal
\begin{alttlg}[hp03_095_1]
\tl{\app{\lem[nolem]{}
	\rdg[wit={J5,G4,E2}]{\excl}
	\rdg[wit={N3}]{\unavbl}}%
\pada{\app{\lem[resp=emend]{pravistṛtāsyaiva}% pravismṛtāsyaiva F
		\rdg[wit={P11}]{pravistṛtasyava}% pravismṛtāsyaiva F
		\rdg[wit={G11}]{pavisthitasyaiva}
		\rdg[wit={G5}]{pathi sthitasyaiva}
		\rdg[wit={V15}]{paristhitasyaiva}
		\rdg[wit={V3}]{pṛṣṭisthitasyaiva}
		\rdg[wit={V1}]{paristhitā [sai]va}
		\rdg[wit={Gr2,V19,N19,C6}]{paristhitā caiva}
		\rdg[wit={J10}]{avasthitasya}
		\rdg[wit={Jyo}]{avasthitā caiva}}
\app{\lem[wit={ceteri}]{phaṇāvatī sā}% kaṇā° N23; śā V1
		\rdg[wit={G11,G5}]{phaṇāvatī ye}
		\rdg[wit={J10}]{phaṇāryayāṃtīyaṃ}}}\\+}
\tl{
\pada{\app{\lem[wit={ceteri}]{prātaś ca sāyaṃ}
		\rdg[wit={G11}]{prāṇaś ca sāyaṃ}
		\rdg[wit={V15}]{prātas tu sāyaṃ}
		}
	praharārdha\app{\lem[wit={ceteri}]{mātram}% praha<<rā>>° N23
		\rdg[wit={P11,V3}]{rātraṃ}}/}\\+}
\tl{
\pada{\app{\lem[wit={J7,V19,G5,N19,V15,V3,Jyo}]{prapūrya}% +G5
		\rdg[wit={N23}]{prapūrvva}
		\rdg[wit={G11}]{prapūjya}
		\rdg[wit={V1}]{prasūrya}
		\rdg[wit={J10,P11,C6}]{prasārya}}
\app{\lem[wit={Gr2,G5,N19,V15,V1,P11,V3,Jyo},alt={sūryāt}]{sūryā\skp{t}}% +G5
		\rdg[wit={G11}]{sūryo}
		\rdg[wit={V19}]{sauryā}
		\rdg[wit={C6}]{sācāryya}
		\rdg[wit={J10}]{ryāṣṇut}}%
\app{\lem[wit={ceteri},alt={paridhāna}]{\skm{t }paridhāna} % pa[ri] .. na V1
		\rdg[wit={V3}]{paridhāya}
		\rdg[wit={P11}]{mavidhāna}
		\rdg[wit={C6}]{vidhāna}}%
\app{\lem[wit={J10,P11,C6,Jyo}]{yuktyā}
		\rdg[wit={Gr2,N19,V15,V1,V3}]{yuktā}% vyuktā? V15; +F
		\rdg[wit={G11}]{yuktāṃ}
		\rdg[wit={V19,G5}]{muktā}}}\\+} % +K3,C7
\tl{
\pada{pragṛhya
\app{\lem[wit={G5}]{tiryak paricālanīyā}% +N12, tīryak G2; 
		\rdg[wit={G11}]{tirya paracālanīyāṃ}
		\rdg[wit={P11,V3}]{niryāt paricālanīyā}
		\rdg[wit={N19}]{niryāt paricālanīyāt}
		\rdg[wit={V15}]{niryātya paricālanīyā}
		\rdg[wit={N23}]{niyāt* pavicālinī sā}
		\rdg[wit={J7,V19}]{niryāty avicālinī sā}% +K3,C7
		\rdg[wit={V1}]{°te yā paricālanīy[ai]}% te yat F
		\rdg[wit={C6,Jyo}]{nityaṃ paricālanīyā}
		\rdg[wit={J10}]{paricālanīyā}}//\versenr}\marma
	%\unavbl{N3}
	%\Allexcept{J5,G4,E2}
	\\!}
\end{alttlg}
\altcommcite\newpage
%\ \newpage


\begin{alttlg}[hp03_095_2]
\tl{\app{\lem[nolem]{}
	\rdg[wit={J5,G4}]{\excl}
	\rdg[wit={N3}]{\unavbl}}%\unavbl{N3}%
\pada{\app{\lem[wit={G11,G5,N19,V3}]{vitastipramitaṃ dairghyaṃ}
		\rdg[wit={V15,V1}]{vitastipramita-dairghyaṃ}
		\rdg[wit={Gr2,Gr3,J10,C6}]{vitastipramitaṃ dīrghaṃ} % vitasthi N23
		\rdg[wit={P11}]{vitastipramitaṃ divyaṃ}
		\rdg[wit={Jyo}]{ūrdhvaṃ vitastimātraṃ tu}}}
\pada{\app{\lem[wit={N23,Gr3,V15,V1,J10,C6,V3,Jyo}]{vistāraṃ}% +F
		\rdg[wit={G11}]{vistāraś}
		\rdg[wit={J7,G5,N19,P11}]{vistāre}} % +M3
		caturaṅgulam/}\\+}
\tl{
\pada{\app{\lem[wit={ceteri}]{mṛdulaṃ}
		\rdg[wit={G11}]{mṛddalaṃ}
		\rdg[wit={V19}]{mṛlaṃ}}
		dhavalaṃ proktaṃ}
\pada{\app{\lem[wit={G5,V15,V1,J10,P11}]{veṣṭanāmbara}% ceṣṭanā° P11
		\rdg[wit={V3}]{veṣṭatāṃbara}
		\rdg[wit={G11,Jyo}]{veṣṭitāmbara}
		\rdg[wit={J7}]{veṣṭanāṃbala}
		\rdg[wit={C6}]{veṣṭanāṃba}
		\rdg[wit={N23,N19}]{vaṣṭanāṃbara} % °cara N23
		\rdg[wit={Gr3}]{veṣṭanādhāra}}lakṣaṇam//\versenr}%
%	\Allexcept{J5,G4}
	\anm{=\,\manuref{3.32*5}}
	\myfn{After this verse, the \textepsilon\ manuscripts have three additional verses from unknown source (\emph{kṛtvātha dohanaṃ ... gulphau karadvayāt}). \mydelim %
	\getsiglum{Jyo} has \ref{III64} after this verse.}
	\label{vitasti}
	\\!}
\end{alttlg}
\altcommcite\newpage
\endaltnormal


\begin{tlg}[hp03_096]
\tl{\app{\lem[nolem]{}
	\rdg[wit={N3}]{\unavbl}}%
\pada{\app{\lem[wit={ceteri}]{vajrāsana}
		\rdg[wit={C6,Jyo}]{vajrāsane}}sthito yogī}
\pada{\app{\lem[wit={ceteri}]{cālayitvā} % cālaïtvā V19
	\rdg[wit={P11}]{vārayitvā}}
\app{\lem[wit={Gr2,Gr3,G11,G5,N19,GrB}]{tu}
		\rdg[wit={J5,V15,J10,Jyo}]{ca}
		\rdg[wit={V1},alt={\om}]{\skp{\om}}} 
		kuṇḍalīm/}\\+}% °lī J5,N23,G5,N19,GrB
\tl{
\pada{\app{\lem[alt={\ante kuryād \add},nosep]{\skp{\ante kuryād \add}}
		\rdg[wit={N23,E2}]{sūryabhedāt}}% as header E2,C7
\app{\lem[wit={J5,G11,G5,V1,J10,GrB,Jyo},alt={kuryād}]{kuryā\skp{d}}
		\rdg[wit={Gr2,Gr3,N19,V15}]{sūryād}
		}%
\app{\lem[wit={ceteri},alt={anantaraṃ}]{\skm{d }anantaraṃ}
		\rdg[wit={N23}]{vanara}
		}
\app{\lem[wit={G11,V15,J10,P11}]{bhastrīṃ}
		\rdg[wit={J5,N23,Gr3,G5}]{bhastrī}% bhaṃstrī N23
		\rdg[wit={J7}]{bhasrī}
		\rdg[wit={N19,V3}]{bhastri}
		\rdg[wit={C6,Jyo}]{bhastrāṃ}
		\rdg[wit={V1},alt={\illeg}]{\skp{\illeg}}
		}}
\pada{\app{\lem[wit={ceteri}]{kuṇḍalīm āśu bodhayet} % āsu V3, ātu P11
		\rdg[wit={J5},alt={\om}]{\skp{\om}}}//\versenr}
		%\unavbl{N3}
	\\!}
\end{tlg}
\commcite%\newpage


\begin{tlg}[hp03_097]
\tl{
\pada{\app{\lem[wit={ceteri},alt={bhānor}]{bhāno\skp{r}}
		\rdg[wit={P11}]{bhānur}
		\rdg[wit={J5},alt={\om}]{\skp{\om}}
		\rdg[wit={N3},alt={\lost}]{\lost}}%
\app{\lem[wit={Gr2,E2,G11,G5,N19,V15,GrB,Jyo},alt={ākuñcanaṃ kuryāt}]{\skm{r }ākuñcanaṃ kuryā\skp{t}}
		\rdg[wit={G4}]{+\,+\,canaṃ kuryāt}
		\rdg[wit={V19}]{ākuñcanaṃ puryāt}
		\rdg[wit={J10}]{ākuñcanenaiva}
		\rdg[wit={V1}]{ākunacaivaṃ}
		\rdg[wit={J5},alt={\om}]{\skp{\om}}
		\rdg[wit={N3},alt={\lost}]{\lost}}t}
\pada{kuṇḍalīṃ % kuṃḍalī Gr2,V15,N3(+ ḍalī),P11,V3
\app{\lem[wit={cetwG4},alt={cālayet}]{cālaye\skp{t}}
		\rdg[wit={N23}]{cālayan}
		\rdg[wit={N3,J5}]{bodhayet}}%
\app{\lem[wit={ceteri},alt={tataḥ}]{\skm{t }tataḥ}
		\rdg[wit={J10}]{tadā}}/}\\+}
\tl{
\pada{\app{\lem[wit={ceteri}]{mṛtyu}
		\rdg[wit={J10}]{mṛtyor}}%
\app{\lem[wit={cetwG4}]{vaktra}
		\rdg[wit={G11,V3}]{vaktraṃ}
		\rdg[wit={N23}]{vakra}
		\rdg[wit={J5}]{vajra}}gatasyāpi}
\pada{tasya mṛtyubhayaṃ kutaḥ//\versenr}\label{VuIII116}\\!}
\end{tlg}
\commcite\newpage


\startaltnormal
\begin{alttlg}[hp03_097_1]
\tl{\app{\lem[nolem]{}
	\rdg[wit={Gr1}]{\excl}
	\rdg[wit={Jyo},alt=\textapp{found after \ref{VuIII121} without commentary}]{\skp{found after 3.101*1 without commentary}}}%
\pada{nāsā\app{\lem[wit={Gr2,G11,G5,N19,V15,J10,P11,V3,Jyo}]{dakṣiṇamārgavāhi} % nāśā N19; mārge N23
		\rdg[wit={C6}]{dakṣiṇavāhimārga}
		\rdg[wit={Gr3}]{paścimavartmavāhi}
		\rdg[wit={V1}]{da[kṣi]ṇa[n]ā\,..\,mārgeṇa}}%
\app{\lem[wit={G11,G5,N19,V1,J10,P11,V3,Jyo},alt={pavanāt}]{pavanā\skp{t}}% +F
		\rdg[wit={V15}]{pavanot}
		\rdg[wit={J7,Gr3,C6}]{pavano}
		\rdg[wit={N23}]{pavana}}%
\app{\lem[wit={N23,E2},alt={prāṇe}]{\skm{t }prāṇe}
		\rdg[wit={G11,G5,N19,V15,V1,J10,P11,V3,Jyo}]{prāṇo}
		\rdg[wit={J7,C6}]{ghrāṇe}% +K3,C7
		\rdg[wit={V19}]{ghrāṇo}% +F
		}%
\app{\lem[resp=emend]{'tidīrghīkṛte}
		\rdg[wit={J7},alt={°kṛteś}]{tidīrghīkṛteś}
		\rdg[wit={E2,V15,V1,Jyo},alt={°kṛtaś}]{tidīrghīkṛtaś}% +K3
		\rdg[wit={N23}]{tidīrghākṛtaś}
		\rdg[wit={N19,J10}]{tidīrghākṛtiś}
		\rdg[wit={V19}]{tirghīkṛtiś\,(°kṛtaś \emph{pc}?)}
		\rdg[wit={G11}]{tidīrghī tataś}
		\rdg[wit={G5,P11}]{pi dīrghīkṛtaś}% °taḥ P11
		\rdg[wit={V3}]{dīrghīkṛtaḥ}
		\rdg[wit={C6}]{na dīrghīkṛtaḥ}
		}}\\+}
\tl{
\pada{\app{\lem[wit={G11,V15,V1,J10,C6}]{candrāmbhaḥ} % caṃdrāṃbhaḥ; 2nd ṃ unclear V1
		\rdg[wit={Gr2,G5,Jyo}]{candrābhaḥ}% +G5,F
		\rdg[wit={Gr3}]{candrāntaḥ}
		\rdg[wit={P11}]{caṃdrāṃśāt}
		\rdg[wit={V3}]{caṃdrāṃgāt}
		\rdg[wit={N19}]{caṃdrād[vā]}}%
\app{\lem[wit={Gr2,G11,G5,N19,J10,P11,C6,Jyo}]{paripūritāmṛtatanuḥ}% ḥ om. G11, taraḥ G5
		\rdg[wit={V15}]{paripūrṇatāmṛtatanuḥ}
		\rdg[wit={V3}]{paripūritāmṛtyutanuḥ}
		\rdg[wit={V1}]{paripūritā\,..\,..\,..\,..}
		\rdg[wit={Gr3}]{paripūrya pūritatanuḥ}}
\app{\lem[wit={ceteri},alt={prāg}]{prā\skp{g}}% prāghgh° J7,G11,V15; prāk* N19,V1
		\rdg[wit={V19,G5,C6}]{prā}}% °kāyā N23; ghaṇṭi illeg. V1
\app{\lem[wit={N23,Gr3,N19,J10,C6},alt={ghaṇṭikāyās tathā}]{\skm{g }ghaṇṭikāyās tathā}
		\rdg[wit={P11},alt={°kāyāḥ pathā}]{ghaṇṭikāyāḥ pathā}% pathā Instr.
		\rdg[wit={V3},alt={°kāyā[pa]thā}]{ghaṇṭikāyā[pa]thā}% pathāt F
		\rdg[wit={G11},alt={°kāyā yadā}]{ghaṇṭikāyā yadā}
		\rdg[wit={Jyo},alt={°kāyās tataḥ}]{ghaṇṭikāyās tataḥ}
		\rdg[wit={J7},alt={°kāyās tadā}]{ghaṇṭikāyās tadā}
		\rdg[wit={G5,V15},alt={°kāyāḥ sadā}]{ghaṇṭikāyāḥ sadā}% ghāṭikā° G5
		\rdg[wit={V1}]{..\,..\,kāyā\,..\,..}}/}\\+}
\tl{
\pada{\app{\lem[resp=emend,alt={siñcan},postwit=\texteng{(cf.~\emph{Ama\-rau\-gha\-śāsana})}]{siñca\skp{n}}% 
		\rdg[wit={N19,V15}]{chindan}% +M1, chintan F ##
		\rdg[wit={P11}]{chiṃdat}
		\rdg[wit={C6}]{chaṃdaḥ}
		\rdg[wit={J10,V3}]{chinnat}
		\rdg[wit={Jyo}]{chittvā}
		\rdg[wit={J7,Gr3}]{bhindan}
		\rdg[wit={N23}]{bhidan}
		\rdg[wit={G11}]{\{\{bhi\}\}śandan}
		\rdg[wit={G5}]{binduḥ}
		\rdg[wit={V1}]{[piṃ]\,..}}n
	kāla\app{\lem[wit={ceteri}]{viśāla}% [vi]śāla V1
		\rdg[wit={G11}]{vikāla}}% 
\app{\lem[wit={ceteri}]{vahni}
		\rdg[wit={P11}]{vadri}
		\rdg[wit={V15}]{pāśa}
		\rdg[wit={N23},alt={\om}]{\skp{\om}}}%
\app{\lem[wit={Gr2,E2,V1,V3},alt={vaśagān}]{vaśagā\skp{n}}
		\rdg[wit={V19,V15}]{vaśagā}% +C7
		\rdg[wit={G11,G5,J10}]{vaśagāt}
		\rdg[wit={N19}]{vaśanān}
		\rdg[wit={Jyo}]{vaśagaṃ}
		\rdg[wit={P11}]{paramān}
		\rdg[wit={C6}]{pavanān}}%
\app{\lem[wit={ceteri},alt={bhrū}]{\skm{n }bhrū}
		\rdg[wit={V15}]{bhū}
		\rdg[wit={N23}]{tū}
		\rdg[wit={V3}]{bhṛṃ}
		\rdg[wit={G11}]{ku}
		\rdg[wit={J10}]{prāg}
		}randhra% raṃdhranā_n taṃ N23
\app{\lem[wit={ceteri},alt={nāḍīgaṇān/gaṇāṃs}]{nāḍīgaṇān\skp{/gaṇāṃs}} % gaṇān* C6,V3,J7,G11,V1,N19,V15,P23; gaṇāṃs Gr3
		\rdg[wit={G5,J10}]{nāḍīgaṇāt}
		\rdg[wit={P11}]{nāḍīguṇān}
		\rdg[wit={Jyo}]{nāḍīgataṃ}
		\rdg[wit={N23}]{nā\,\_\,n}}}\\+}
\tl{
\pada{\app{\lem[wit={G11,G5,N19,V15,J10,C6,Jyo},alt={tat}]{ta\skp{t}}
		\rdg[wit={Gr2,Gr3,V1,P11,V3}]{taṃ}}% ta(ṃ) V1; +F
	\app{\lem[wit={ceteri},alt={kāyaṃ}]{\skm{t }kāyaṃ}
		\rdg[wit={J10}]{kāryaṃ}} 
		kurute punar navataraṃ % navattaraṃ J10; °tara V3; °karaṃ G5; [kuru](te) puna(r nava)taraṃ V1
\app{\lem[wit={J7,V19,G5,P11,C6}]{jīrṇa}
		\rdg[wit={E2,G11,V3}]{jīrṇaṃ}
		\rdg[wit={N19}]{chiṃjīrṇaṃ}
		\rdg[wit={J10,Jyo}]{chinna}% bhinna F
		\rdg[wit={V15}]{chinnaṃ}
		\rdg[wit={V1}]{kṛnta}
		\rdg[wit={N23}]{bhasma}
		}drumaskandhavat//\versenr}\marma\ % [dru](ma) V1
	%\Allexcept{Gr1}
	%\myfn{\getsiglum{Jyo} has this verse and the next one after \ref{VuIII121}. Brahmānanda does not comment on this verse.}
	\\!}
\end{alttlg}
\altcommcite\newpage
\ \newpage

\begin{alttlg}[hp03_097_2]
\tl{\app{\lem[nolem]{}
	\rdg[wit={Gr1}]{\excl}
	\rdg[wit={Jyo},alt=\textapp{found after \ref{VuIII121} with the previous verse}]{\skp{found after 3.101*1 with the previous verse}}}%
\pada{\app{\lem[wit={Gr3,G5,V1,P11,C6,Jyo}]{kuṇḍalīṃ}
		\rdg[wit={Gr2,G11,N19,V15,V3}]{kuṇḍalī}
		\rdg[wit={J10}]{kuṇḍaliṃ}}
cālayi\app{\lem[wit={J7,Gr3,G11,G5,N19,Jyo},alt={°tvā tu}]{\skp{°}tvā tu} % cāla-i-tvā V19, lācayitvā G5
		\rdg[wit={V15,J10,P11,C6}]{°tvātha}% +K1,F; ca P6
		\rdg[wit={N23}]{°tvācca}
		\rdg[wit={V3}]{°tvādhaḥ}
		\rdg[wit={V1},alt={\illeg}]{\skp{\illeg}}}}
\pada{\app{\lem[wit={E2,G11,G5,V15,V1,P11}]{kuryād bhastrīṃ}% +K3,C7
		\rdg[wit={V19,N19,V3}]{kuryād bhastrī} % kuryā bhastri V3
		\rdg[wit={J10}]{kuryād bhastrāṃ}
		\rdg[wit={Gr2}]{bhasrī kuryād}
		\rdg[wit={C6,Jyo}]{bhastrāṃ kuryād}}
	viśeṣataḥ/}% viśet tataḥ G5
	\\+}
\tl{
\pada{\app{\lem[wit={ceteri},alt={evam}]{eva\skp{m}}% +G5
		\rdg[wit={G11}]{etad}}%
\app{\lem[wit={G5,V15,Jyo},alt={abhyasato}]{\skm{m }abhyasato}% +F,C7
		\rdg[wit={Gr2,V19,G11,N19,C6}]{abhyāsato}% abhyasyayo P6
		\rdg[wit={E2,J10}]{abhyasyato}% +N22
		\rdg[wit={P11}]{abhyasyatāṃ}
		\rdg[wit={V3}]{abhyasyatā}
		\rdg[wit={V1}]{..\,..\,syat.}
		} nityaṃ} % [ni](tyaṃ) V1
\pada{\app{\lem[wit={ceteri}]{yaminaḥ śaṅkate yamaḥ} % śaṃkati P11
		\rdg[wit={G5}]{yamir na kurute yamaḥ}
		\rdg[wit={Jyo}]{yamino yamabhīḥ kutaḥ}}//\versenr} 
		%\Allexcept{Gr1}
		\\!}
\end{alttlg}
\altcommcite\newpage


\begin{alttlg}[hp03_097_3]
\tl{\app{\lem[nolem]{}
	\rdg[wit={Gr1,E2,Jyo}]{\excl}}%
\pada{\app{\lem[wit={Gr2,N19,V1,P11,V3},alt={tadābhyaset}]{tadābhyase\skp{t}}
		\rdg[wit={J10}]{tadābhyasyet}
		\rdg[wit={V19,V15,C6}]{tad abhyaset}
		\rdg[wit={G11,G5}]{tathābhyaset}% tato° F
		}%
\app{\lem[wit={ceteri},alt={sūryabhedam}]{\skm{t }sūryabheda\skp{m}} % °bhedaḥm P11, °bhedan*m J10
		\rdg[wit={V15}]{sūryabhede}
		}}%
\pada{\app{\lem[wit={Gr2,G11,G5,V1,J10,GrB},alt={ujjāyīṃ}]{\skm{m }ujjāyīṃ} % °yī N23,V3,G5; uj-hāyīṃ P11; +K3,C7
		\rdg[wit={N19}]{ujjāī}
		\rdg[wit={V15}]{ujjāyāṃ}
		\rdg[wit={V19}]{ujrākhyām}}
\app{\lem[wit={ceteri}]{cāpi}
		\rdg[wit={V15}]{vāpi}
		\rdg[wit={V1}]{[vā]\,..}
		\rdg[wit={V19}]{api}} 
		śītalīm/}\\+} % śītalī N23,N19,V15; sītalī V3
\tl{
\pada{evam
abhyāsa\app{\lem[wit={ceteri}]{yuktasya}% (yukta)sya V1, muktasya G5
		\rdg[wit={J10}]{yogena}}}
\pada{\app{\lem[wit={C6pc,G11,J10,V3}]{śamano} % +M1,F; śamana means Yama.
		\rdg[wit={C6ac,N19,V1,P11}]{śamino}% śamin. V1
		\rdg[wit={V15}]{śramas tu}% śramaś ca N22,K1, ścamaś ca P6
		\rdg[wit={Gr2,V19}]{yamas tu}
		\rdg[wit={G5}]{yamino}
		}
\app{\lem[wit={ceteri}]{yaminaḥ}% (yami)naḥ
		\rdg[wit={P11}]{yamina}
		\rdg[wit={V3}]{yaminaṃ}} 
\app{\lem[wit={ceteri}]{kutaḥ} % kuta V3
	\rdg[wit={C6}]{kva ca}}//\versenr}
	%\Allexcept{Gr1,E2,Jyo}
	\\!}
\end{alttlg}
\altcommcite\newpage
\endaltnormal


\begin{tlg}[hp03_098]
\tl{
\pada{\app{\lem[nolem]{\skp{pāda a}}
	\rdg[wit={E2},alt={\om}]{\skp{\om}}}%
muhūrtadvayaparyantaṃ} % mahūrtta V3; paryaṃta V15
\pada{\app{\lem[nolem]{\skp{pāda b}}
	\rdg[wit={E2},alt={\om}]{\skp{\om}}}%
\app{\lem[wit={V1,J10,Jyo}]{nirbhayaṃ}% +M1,G7,F; ṃ unsichtbar V1
		\rdg[wit={N19}]{nirbhayaṃś}
		\rdg[wit={N3,J5,G11,G5,V15,GrB}]{nirbhayaś}% +K1
		\rdg[wit={Gr2,V19}]{nirbharaṃ}
		}
\app{\lem[wit={N3,J5,Gr2,N19,V15,V1,C6,V3,Jyo}]{cālanād asau}% +F
		\rdg[wit={V19}]{calanād asau}
		\rdg[wit={P11}]{calanādiṣu}
		\rdg[wit={G11,G5}]{cālayet imāṃ}
		\rdg[wit={J10}]{vā diśodiśa}}/}
	%\lineom{ab}{E2}
	\\+}
\tl{
\pada{\app{\lem[nolem]{\skp{pāda c}}
	\rdg[wit={Gr3},alt={\om}]{\skp{\om}}}%
ūrdhva%m % urdham V3
\app{\lem[wit={ceteri},alt={ākṛṣyate}]{\skm{m }ākṛṣyate}% °ti P11
		\rdg[wit={V15}]{ākṛte}}
		kiṃcit} % kiṃ(cit) V1
\pada{\app{\lem[nolem]{\skp{pāda d}}
	\rdg[wit={Gr3},alt={\om}]{\skp{\om}}}%
\app{\lem[wit={J5,G11,G5,N19}]{suṣumṇā kuṇḍalīgatā}% +M1,M3; °kṛtā F
		\rdg[wit={N3}]{suṣumnā kuṃḍalīgataḥ}% +G7
		\rdg[wit={G4}]{+\,+\,+\,+\,ḍa[l]ī[ga]taḥ}
		\rdg[wit={V3}]{suṣumnāṃ kuṃḍalīgatā}
		\rdg[wit={P11}]{suṣumṇāṃ kuṃḍalīgatāṃ}
		\rdg[wit={Gr2,V15,C6}]{suṣumṇāgatakuṇḍalī}
		\rdg[wit={Jyo}]{suṣumnāyāṃ samudgatā}
		\rdg[wit={J10}]{suṣumṇāyāḥ samuddhṛtaḥ}
		\rdg[wit={V1},alt={\om}]{\skp{\om}}}//\versenr} % haplogr. V1
	%\lineom{cd}{Gr3}
	\\!}
\end{tlg}
\commcite\newpage


\begin{tlg}[hp03_099]
\tl{\app{\lem[nolem]{}
	\rdg[wit={Gr3},alt={\om}]{\skp{\om}}}%
\pada{\app{\lem[wit={ceteri}]{tena kuṇḍalinī} % kuṃḍalanī N3, kuṃḍilinī P11, kuṇḍalīnī J5,N23
		\rdg[wit={V1},alt={\om}]{\skp{\om}}}
	\app{\lem[wit={N3,Gr2,N19,V15,J10,P11,C6,Jyo}]{tasyāḥ}
		\rdg[wit={J5,G11,V3}]{tasyā}
		\rdg[wit={G4}]{tasyāt}
		\rdg[wit={G5}]{tasya}
		\rdg[wit={V1},alt={\om}]{\skp{\om}}}}
\pada{\app{\lem[wit={Gr2,N19,V15,V1,J10}]{suṣumṇāyāḥ}% su(ṣumnā)yāḥ V1
		\rdg[wit={Gr1,G11,GrB,Jyo}]{suṣumṇāyā}
		\rdg[wit={G5}]{suṣumṇāyāṃ}}
\app{\lem[wit={J5,G4,Gr2,N19,V15,V1,J10,V3}]{samuddhṛtā} % tāḥ N19, samudhṛtā J5,G4, samū° N23, sa[m]u[ddhṛ]tā V1; +G3
		\rdg[wit={N3}]{samudbhutā}
		\rdg[wit={P11,C6,Jyo}]{mukhaṃ dhruvam} % = GŚ
		\rdg[wit={G11}]{mukhaṃ dṛḍhaṃ dhruvaṃ}
		\rdg[wit={G5}]{mukhaṃ dṛḍhaṃ}}/}\\+} % +M3
\tl{
\pada{\app{\lem[wit={ceteri}]{jahāti}
		\rdg[wit={J10}]{na yāti}
		} tasmāt prāṇo'yaṃ} % tasyā prāṇo haṃ J5
\pada{suṣumṇāṃ vrajati % suṣumṇā J5,C6,V3,G5; vujati N19
\app{\lem[wit={G11,V15,V1,Jyo}]{svataḥ}
		\rdg[wit={N3,N19,P11,V3}]{svanaḥ} % svana V3
		\rdg[wit={J5,Gr2,C6}]{svayam}% śvayaṃ J5, svaryaṃ N23
		\rdg[wit={G5}]{sthitā}
		\rdg[wit={J10}]{niścalaḥ}}//\versenr} 
%		\NotIn{Gr3}
		\\!}
\end{tlg}
\commcite\newpage


\begin{tlg}[hp03_100]
\tl{\app{\lem[nolem]{}
	\rdg[wit={Gr3},alt={\om}]{\skp{\om}}}%
\pada{\app{\lem[wit={ceteri},alt={tasmāt}]{tasmā\skp{t}}
		\rdg[wit={N23}]{kasmāt}
		}t saṃcālayen nityaṃ} % saṃcāra˟yen N3, °cāra° G5
\pada{\app{\lem[wit={J5,G11,G5,GrB},alt={śabdagarbhām}]{śabdagarbhā\skp{m}}% gabhāṃ C6
		\rdg[wit={N3}]{śabdagaṃdhām}
		\rdg[wit={V1}]{..\,..\,..\,dhām}
		\rdg[wit={Gr2,N19,V15}]{śaṃbhugarbhām}
		\rdg[wit={Jyo}]{sukhasuptām}
		\rdg[wit={J10}]{suṣasuptām}
		}\marma%
\app{\lem[wit={ceteri},alt={arundhatīm}]{\skm{m }arundhatīm} % °tī N23,N19,N3,J5; ddhaṃtī V3
		\rdg[wit={G11,G5,P11,C6}]{sarasvatīṃ}}/}\\+}
\tl{
\pada{\app{\lem[wit={N3,J5,V15,C6,V3,Jyo}]{tasyāḥ}
		\rdg[wit={J10}]{tasyāṃ}
		\rdg[wit={P11}]{tasmāt}
		\rdg[wit={Gr2,G11,G5,N19}]{yasyāḥ}
		\rdg[wit={V1}]{[ya]\,..}}
\app{\lem[wit={N3,J5,GrB,Jyo}]{saṃcālanenaiva}% sacāla° N3
		\rdg[wit={Gr2,G11,G5,V15}]{saṃcālanenāśu}% +F
		\rdg[wit={N19,J10}]{saṃcālayenāśu}
		\rdg[wit={V1}]{..\,..\,lanen.\,..}}}
\pada{yogī \app{\lem[wit={ceteri},alt={rogaiḥ/rogair}]{rogaiḥ\skp{/rogair}}% raugaiḥ C6, yogaiḥ! G5
%	\rdg[wit={J7,N19,V15}]{rogair}
		\rdg[wit={N23}]{[r]. .air}
		\rdg[wit={V3}]{rogoḥ}
		\rdg[wit={J10,P11}]{rogāt}}
\app{\lem[wit={N3,J5,G11,G5,V1,J10,C6,V3,Jyo}]{pramucyate}
		\rdg[wit={P11}]{pramuṃcati}
		\rdg[wit={Gr2,N19,V15}]{vimucyate}}//\versenr} 
%		\NotIn{Gr3}
		\\!}
\end{tlg}
\commcite\newpage


\begin{tlg}[hp03_101]
\tl{\app{\lem[nolem]{}
	\rdg[wit={Gr3},alt={\om}]{\skp{\om}}}%
\pada{yena 
	\app{\lem[wit={ceteri}]{saṃcālitā}
		\rdg[wit={N19}]{saṃcalitā}
		\rdg[wit={N3}]{saṃcalatā}
		\rdg[wit={J7}]{sa cālitā}} śaktiḥ} % śakti V3,G4
\pada{sa yogī 
	\app{\lem[wit={ceteri}]{siddhi}% .. ddhi V1
		\rdg[wit={G4}]{siddha}
		\rdg[wit={Gr2}]{mukti}}% 
	\app{\lem[wit={ceteri}]{bhājanam}% °javaṃ G4
		\rdg[wit={J5}]{bhājana}
		\rdg[wit={G11,G5,C6}]{bhājanaḥ}
		\rdg[wit={V1}]{..\,janaḥ}}/}\\+}
\tl{
\pada{kim atra bahunoktena} % (bahu)no(ktena)
\pada{kālaṃ % kaliṃ G4
	\app{\lem[wit={ceteri}]{jayati}
		\rdg[wit={V1}]{..\,yalati}
		\rdg[wit={J10}]{vrajati}
	} līlayā//\versenr} 
%	\NotIn{Gr3}
	\\!} % jayalati? V1
\end{tlg}
\commcite%\newpage


\startaltnormal
\begin{alttlg}[hp03_101_1]
\tl{\app{\lem[nolem]{}
	\rdg[wit={Gr1,E2}]{\excl}}%
\pada{\app{\lem[wit={G11,G5,V15,V3}]{brahmacaryavratasyaiva}% °vaṃ G5; +F [but usually °vrata is not used as Bahuvrīti]
\rdg[wit={V19,V1,Jyo},alt={°ratasyaiva}]{brahmacaryaratasyaiva} % +K3,C7
		\rdg[wit={N19},alt={°rataś caiva}]{brahmacaryarataś caiva}
		\rdg[wit={P11}]{brahmacaryāvatastaiva}
		\rdg[wit={J7}]{brahmacarye ca tasyaiva}
		\rdg[wit={C6}]{brahmacaryavrataṃ}
		\rdg[wit={N23}]{brahmavatasyaiva}
		\rdg[wit={J10}]{brahmadharmaratasyaiva}}}
\pada{nityaṃ
\app{\lem[wit={J7,Jyo}]{hitamitāśinaḥ}% +C7pc
		\rdg[wit={P11},alt={°śini}]{hitamitāśini}
		\rdg[wit={N23,V19,G11,N19,V3},alt={°śanaḥ}]{hitamitāśanaḥ} % °sanaḥ V19; +C7ac
		\rdg[wit={G5,C6},alt={°śanaṃ}]{hitamitāśanaṃ}
		\rdg[wit={V15},alt={°śanaiḥ}]{hitamitāśanaiḥ}
		\rdg[wit={J10}]{mitahitāśinaḥ}% +K3
		\rdg[wit={V1},alt={\illeg}]{\skp{\illeg}}}/}\\+}
\tl{
\pada{\app{\lem[wit={J7,G11,G5,N19,V15,C6,Jyo},alt={maṇḍalād}]{maṇḍalā\skp{d}}% maṇḍalāt* J7; +K3,C7
		\rdg[wit={N23,J10,P11,V3}]{maṃḍalā}
		\rdg[wit={V19}]{maṃḍalī}
		\rdg[wit={V1},alt={\illeg}]{\skp{\illeg}}}d 
		dṛśyate siddhiḥ} % illeg, V1; siddhiṃ G5,N19; siddhi P11,V3
\pada{\app{\lem[wit={J7,G11,G5,V15,C6,Jyo},alt={kuṇḍalya°}]{kuṇḍalya\skp{°}}
		\rdg[wit={V19,N19,J10,V3}]{kuṇḍalyā}
		\rdg[wit={P11}]{kuṇḍalā}
		\rdg[wit={N23}]{kuṇḍali}
		\rdg[wit={V1},alt={\illeg}]{\skp{\illeg}}}bhyāsa%
\app{\lem[wit={G11,G5,N19,V15,V3,Jyo}]{yoginaḥ}% +F
		\rdg[wit={Gr2,V19,V1,J10,P11,C6}]{yogataḥ}% V1 uncertain
		}//\versenr}\label{VuIII121}
	%\Allexcept{Gr1,E2}
	\\!}
\end{alttlg}
\altcommcite\newpage
\endaltnormal


\begin{tlg}[hp03_102]
\tl{\app{\lem[nolem]{}
	\rdg[wit={Gr3},alt={\om}]{\skp{\om}}
	\rdg[wit={Jyo},alt=\textapp{found after \ref{III96}}]{\skp{found after 3.91}}}%
\pada{\app{\lem[wit={ceteri}]{abhyāsa}
		\rdg[wit={Jyo}]{abhyāsān}
		\rdg[wit={J10}]{abhyāsā}
		\rdg[wit={J5}]{abhasā}}%
\app{\lem[wit={G11,V15,C6,V3,Jyo}]{niḥsṛtāṃ}
		\rdg[wit={V1}]{niḥsṛtā}
		\rdg[wit={J10}]{niḥśritāṃ}
		\rdg[wit={G5,P11}]{nisṛtāṃ}
		\rdg[wit={N3,J5}]{nisṛtā}
		\rdg[wit={N19}]{nibhṛtāṃ}
		\rdg[wit={Gr2}]{sahitaṃ}}
\app{\lem[wit={N3,N19,J10,C6,Jyo}]{cāndrīṃ}
		\rdg[wit={J5,V15,V1,V3}]{cāndrī}
		\rdg[wit={G11,G5}]{cāndriṃ}
		\rdg[wit={P11}]{cāṃdrāṃ}
		\rdg[wit={Gr2}]{candraṃ}}}
\pada{vibhūtyā saha % vibhūtya J5, vibhūbhyā N23, bibhṛtyā P11, vidhṛtyā G5
\app{\lem[wit={G11,N19,V15,V1,J10,Jyo}]{miśrayet}
		\rdg[wit={GrB}]{miśritāṃ}
		\rdg[wit={J5}]{miśritaṃ}
		\rdg[wit={N3}]{mīśritaṃ}
		\rdg[wit={N23}]{micchayet}
		\rdg[wit={J7}]{mūrchayet}
		\rdg[wit={G5}]{kārayet}}/}\\+}
\tl{
\pada{\app{\lem[wit={N19,V15,C6}]{taddhāraṇaṃ}
		\rdg[wit={Gr2}]{taddhāraṇā}
		\rdg[wit={V3}]{tadvāraṇaṃ}
		\rdg[wit={P11}]{yadvāraṇaṃ}
		\rdg[wit={V1}]{tad dh.\,..\,..}
		\rdg[wit={N3,J5,G11}]{tad dhārayed}% +F
		\rdg[wit={J10}]{tāṃ dhārayed}
		\rdg[wit={G5,Jyo}]{dhārayed ut°}} % +G5,M3
\app{\lem[wit={N19,V15,GrB}]{tūttamāṅge} % nūnūmāṃge P11
		\rdg[wit={J7}]{cottamāṅge}
		\rdg[wit={N23}]{cottamāṃga}
		\rdg[wit={N3,J5,G11,J10}]{uttamāṅge}
		\rdg[wit={G5}]{°tamāṅgena}% °ge tu M3
		\rdg[wit={Jyo}]{°tamāṅgeṣu}
		\rdg[wit={V1},alt={\illeg}]{\skp{\illeg}}}}
\pada{\app{\lem[wit={ceteri}]{divya}
		\rdg[wit={G4,GrB}]{dīrgha}}% +G7
\app{\lem[wit={N3,J5,G4,G11,G5,N19,V15,GrB}]{dṛṣṭipradāyakam}
		\rdg[wit={Gr2}]{dṛṣṭipradāyinī}
		\rdg[wit={J10}]{dṛṣṭipradāyinīṃ}
		\rdg[wit={V1,Jyo}]{dṛṣṭiḥ prajāyate}% +F
		}//\versenr}%
	\myfn{After this verse, \getsiglum{J5} has an additional line:
	\devnote{āsanābhyāsanaṃ pūrvā prayuktāṃ manasā caret/}}
	\label{VuIII98} %
%	\NotIn{Gr3}
	\\!}
\end{tlg}
\commcite\newpage


\startaltnormal
\begin{alttlg}[hp03_102_1]
\tl{\app{\lem[nolem]{}
	\rdg[wit={Gr1,V3}]{\excl}}%
\pada{\app{\lem[wit={J7,G11,G5,N19,V15,V1,J10,P11,C6}]{dvi}% +K3
		\rdg[wit={N23,Gr3,Jyo}]{dvā}}saptatisahasrāṇāṃ} % °srānā V19
\pada{nāḍīnāṃ % ṃ om. P11, °ṇāṃ C6
	\app{\lem[wit={J10,Jyo}]{malaśodhane}% +M1 
		\rdg[wit={Gr2,Gr3,G11,G5,N19,V15,V1,P11,C6}]{malaśodhanam}}/}\\+}
\tl{
\pada{\app{\lem[wit={Gr3,G11,V15,Jyo}]{kutaḥ}% K1,N22,P6
		\rdg[wit={N19}]{kṛta}
		\rdg[wit={J7}]{gudaḥ}
		\rdg[wit={V1,J10}]{guda}
		\rdg[wit={C6}]{aṃtaḥ}
		\rdg[wit={P11}]{aṃtaṃ}
		\rdg[wit={G5}]{ātma}
		\rdg[wit={N23},alt={\om}]{\skp{\om}}}
\app{\lem[wit={J7,Gr3,G11,V15,J10,Jyo}]{prakṣālanopāyaḥ}
		\rdg[wit={G5,N19,V1},alt={°pāyaṃ}]{prakṣālanopāyaṃ}
		\rdg[wit={C6}]{prakṣālano vāyuḥ}
		\rdg[wit={P11}]{prajvālano vāyu}
		\rdg[wit={N23},alt={\om}]{\skp{\om}}}}
\pada{\app{\lem[wit={Jyo}]{kuṇḍalyabhyasanād ṛte}
		\rdg[wit={G11,G5,N19,V15,C6}]{kuṇḍalyabhyāsanād ṛte}% °lyā° N19
		\rdg[wit={P11}]{kuṃḍalībhyāsanād ate}
		\rdg[wit={J7,E2}]{kuṇḍalyabhyāsato vinā}
		\rdg[wit={N23,V19}]{kuṇḍalyābhyāsato vinā}% +K3,C7
		\rdg[wit={J10}]{kuṇḍalyabhyāsa iṣyate}
		\rdg[wit={V1}]{ku\,..\,..\,[bhyā]\,..\,[mā]\,..\,..}}//\versenr}
	%\Allexcept{Gr1,V3}
	\\!}
\end{alttlg}
\endaltnormal

\altcommcite

\begin{altpostmula}[hp03_102_1p]
\app{\lem[nolem]{}
	\rdg[wit={G11,G5,N19,V15,V1,J10}]{\incl}}%
	iti śakticālanam/ %\sgwit{G11,G5,N19,V15,V1,J10}
\end{altpostmula}
%\bigskip

\newpage

\begin{tlg}[hp03_103]
\tl{
\pada{iti mudrā
\app{\lem[wit={ceteri}]{daśa}
		\rdg[wit={N3}]{dabhā}
		\rdg[wit={Gr3}]{nava}} proktā}
\pada{\app{\lem[wit={ceteri}]{ādināthena}
		\rdg[wit={G11,G5}]{hy ādināthena}
		} śaṃbhunā/}\\+}
\tl{
\pada{\app{\lem[wit={N3,Jyo}]{ekaikā tāsu}% +K3,C7
		\rdg[wit={J5,G4}]{ekaikaṃ tāsu}% māsu J5
		\rdg[wit={N19,E2}]{ekaika tāsu}
		\rdg[wit={V19}]{ekaiva tāsu}
		\rdg[wit={Gr2}]{ekaikāpi su°}
		\rdg[wit={G11,G5,V15}]{karaṇe sarva}
		\rdg[wit={J10,P11}]{kāraṇe sarva}
		\rdg[wit={C6,V3}]{kāraṇaṃ sarva}
		\rdg[wit={V1}]{k.\,..\,..\,sarva}}
\app{\lem[wit={Gr1,Gr2,Gr3,N19,Jyo}]{yamināṃ}% yaminaṃ J5
		\rdg[wit={V15,V1,J10}]{siddhānām}
		\rdg[wit={G11,G5,GrB}]{siddhīnām}}}
\pada{\app{\lem[wit={Gr2,Gr3,N19,Jyo}]{mahāsiddhipradāyinī}
		\rdg[wit={N3,J5},alt={°pradāyanī}]{mahāsiddhipradāyanī}
		\rdg[wit={G4},alt={°pradā\,+\,+}]{mahāsiddhipradā\,+\,+}
		\rdg[wit={G11,G5,V15,V1,J10,GrB},postwit=\texteng{(\getsiglum{V1} partly illegible)}]{ekaikāpi kṣamaiva sā}% kṣameva P11
%	\rdg[wit={V1}]{e[k]. .. [p]i [kṣamai] .. ..}
	}//\versenr}\label{III124}\myfn{\getsiglum{Jyo} has a different verse order: \ref{III127} \rightarrow\ \ref{III128} \rightarrow\ \ref{III125} \rightarrow\ \ref{III126} \rightarrow\ \ref{III124}}\\!}
\end{tlg}
\commcite\newpage


\begin{tlg}[hp03_104]
\tl{
\pada{rājayogaṃ vinā % yojayogaṃ V19; °yoge N23, yoga J5,N19
\app{\lem[wit={ceteri}]{pṛthvī}
		\rdg[wit={J10,P11}]{pṛthvīṃ}
		\rdg[wit={V15}]{siddhī}
		\rdg[wit={N19}]{vṛddhi}}}
\pada{rājayogaṃ vinā % yoga J5,N19
\app{\lem[wit={ceteri}]{niśā}% illeg. J5
		\rdg[wit={J10}]{niśāṃ}
		\rdg[wit={N23}]{nyathā}}/}\\+}
\tl{
\pada{rājayogaṃ vinā mudrā} % yoga J5,N19
\pada{vicitrāpi na % vicitrāṃva P11
\app{\lem[wit={ceteri}]{rājate} % rojate N23, rocate F
		\rdg[wit={C6,Jyo}]{śobhate}}//\versenr}\label{III125}\\!}
\end{tlg}
\commcite%\newpage


\begin{tlg}[hp03_105]
\tl{\app{\lem[nolem]{}
	\rdg[wit={E2},alt={\om}]{\skp{\om}}}%
\pada{\app{\lem[wit={ceteri}]{mārutasya vidhiṃ}% maru° P11; vidhi J7,P11, vadhiḥ J5
		\rdg[wit={G11,V15,V1}]{mārutābhyasanaṃ}
		\rdg[wit={G5,J10}]{mārutābhyāsanaṃ}
		}
\app{\lem[wit={ceteri}]{sarvaṃ}% sarva J7
		\rdg[wit={N3}]{sarve}
		\rdg[wit={C6}]{sarvāṃ}
%		\rdg[wit={K3,C7}]{siddhiṃ}
		\rdg[wit={J5}]{mano}
		\rdg[wit={J10}]{kiṃcin}
		}}
\pada{\app{\lem[wit={ceteri}]{manoyuktaṃ} % yukta N23
		\rdg[wit={J5}]{sadā yuktaṃ}}
\app{\lem[wit={ceteri}]{samabhyaset}
		\rdg[wit={V1,J10}]{samācaret}}/}\\+}
\tl{
\pada{itaratra na kartavyā} % itiratta J5, italattra N23, ityatatra C6, iti-r-atra G11; karttavyo P11, kartavyaṃ N23,G5
\pada{manovṛtti%r % °vṛttā J5, °vṛtti P11, manārvvartti N23
\app{\lem[wit={ceteri},alt={manīṣiṇā}]{\skm{r }manīṣiṇā}
		\rdg[wit={J5}]{manīṣiṇī}
		\rdg[wit={N19,V3}]{manīṣiṇām}
		\rdg[wit={V1}]{..\,[nī]\,..\,ṇ.}
		\rdg[wit={G5}]{tu yoginām}
		}//\versenr}\label{III126} 
%		\NotIn{E2}
		\\!}
\end{tlg}
\commcite\newpage


\begin{tlg}[hp03_106]
\tl{\app{\lem[nolem]{}
	\rdg[wit={E2},alt={\om}]{\skp{\om}}}%
\pada{\app{\lem[wit={N3,J5,J7,V19,GrB}]{khilāpi}% śilāpi F; +K3,C7
		\rdg[wit={N23}]{sthirāpi}
		\rdg[wit={N19,V15}]{calāpi}
		\rdg[wit={G11}]{vilāpi}
		\rdg[wit={V1,J10}]{vināpi}
		\rdg[wit={G5}]{suṣumnā}
		\rdg[wit={Jyo}]{iyaṃ tu}
		}\marmas
\app{\lem[wit={ceteri}]{madhyamā}
		\rdg[wit={J10}]{madhyamāṃ}
		\rdg[wit={P11}]{madhyanā°}
		\rdg[wit={V1},alt={\illeg}]{\skp{\illeg}}}
\app{\lem[wit={ceteri}]{nāḍī}
		\rdg[wit={V1}]{..\,ḍīṃ}
		\rdg[wit={P11}]{°ḍī ca}}}
\pada{dṛḍhābhyāsena % draḍhā° P11, driḍhā° J10; °bhyasepi J5
\app{\lem[wit={ceteri}]{yoginām}
		\rdg[wit={N3}]{yoginaṃ}
		\rdg[wit={J5,P11,C6}]{yoginā}% yominā J5
		\rdg[wit={J10}]{yoginaḥ}}/}\\+}
\tl{
\pada{\app{\lem[wit={N3,J5,V19,G5,J10,C6,V3,Jyo}]{āsana} % asana N3
		\rdg[wit={P11}]{āsanā}
		\rdg[wit={G11}]{ānasa}
		\rdg[wit={Gr2,N19,V15,V1}]{āsanaṃ}
	}prāṇa%
\app{\lem[wit={N3,J5,Gr2,G11,G5,N19,V15,V1,Jyo}]{saṃyāma}
		\rdg[wit={V3}]{saṃyama}
		\rdg[wit={V19,C6}]{saṃyāmair}% °mai V19,J7pc; +K3,C7
		\rdg[wit={J10}]{saṃyamair}
		\rdg[wit={P11}]{saṃyāme}
	}}%
\pada{mudrābhiḥ % °bhi V3, °niḥ C6,G5
\app{\lem[wit={ceteri}]{saralā}% śaralā J5,V3
		\rdg[wit={G4}]{sakalā}
		\rdg[wit={V15}]{sabalā}
		\rdg[wit={N19}]{śavalā}
		\rdg[wit={V19}]{na calā}} bhavet//\versenr}
%		\NotIn{E2}
		\label{III127}\\!}
\end{tlg}
\commcite%\newpage


\begin{tlg}[hp03_107]
\tl{\app{\lem[nolem]{}
	\rdg[wit={E2,N19},alt={\om}]{\skp{\om}}}%
\pada{\app{\lem[wit={N3,J5}]{upāsane}
		\rdg[wit={Gr2}]{upāsanaṃ}
		\rdg[wit={G4,V19}]{upāsana}% +C7
		\rdg[wit={V1}]{abhyāse\,..}
		\rdg[wit={G11,G5,V15,GrB}]{abhyāseṣu}
		\rdg[wit={J10}]{abhyāsena}% +G3
		\rdg[wit={Jyo}]{abhyāse tu}}
\app{\lem[wit={ceteri}]{vinidrāṇāṃ} % °ṇā N3
%		\rdg[wit={G11}]{<<vi>>nidrāṇām}
		\rdg[wit={J5}]{pi nidrāṇāṃ}
		\rdg[wit={J10}]{hi mudrāṇāṃ}}}
\pada{\app{\lem[wit={Gr1}]{rājayoga}% +F
		\rdg[wit={Gr2,V19}]{rājayogaḥ}
		\rdg[wit={V1}]{anuddhṛta}
		\rdg[wit={V15}]{anuddhata}
		\rdg[wit={C6}]{anudbhūta}
		\rdg[wit={G11,P11}]{anudruta}% +G7
		\rdg[wit={V3}]{manudṛta}
		\rdg[wit={Jyo}]{mano dhṛtvā}
		\rdg[wit={G5}]{atandrita}
		\rdg[wit={J10}]{tad udeti}}%
\app{\lem[wit={N3,G4}]{samudravat}% samudragāṃ F
		\rdg[wit={J7}]{samudrakaḥ}
		\rdg[wit={N23}]{samūdakaḥ}
		\rdg[wit={J5}]{samudbhavān}
		\rdg[wit={V19}]{samāhnakaḥ}
%		\rdg[wit={C7}]{samahnakaḥ}
		\rdg[wit={V15,V1,C6}]{samādhināṃ}
		\rdg[wit={J10,Jyo}]{samādhinā}
		\rdg[wit={G11,G5,P11,V3}]{samādhiṣu}% +G7,G3
		}/}\marma\\+}
\tl{
\pada{\app{\lem[nolem]{\skp{pāda c}}
	\rdg[wit={J5},alt={\om}]{\skp{\om}}}%
\app{\lem[wit={ceteri}]{rudrāṇī}
		\rdg[wit={V1,J10}]{mudrāṇāṃ}}
\app{\lem[wit={N3,V15,V1,J10,GrB}]{cāparā}
		\rdg[wit={G4}]{ca parā}
		\rdg[wit={Jyo}]{vā parā}% +F
		\rdg[wit={G11}]{va parā}
		\rdg[wit={Gr2,V19,G5}]{sāparā}
		\rdg[wit={J5},alt={\om}]{\skp{\om}}
		}
		mudrā} % mudrāṃ P11
\pada{\app{\lem[nolem]{\skp{pāda d}}
	\rdg[wit={J5},alt={\om}]{\skp{\om}}}%
\app{\lem[wit={cetwG4}]{bhadrāṃ} % bhadrā V3,V1, bha[dr]. G4
		\rdg[wit={N23}]{bhavāṃ}
		\rdg[wit={P11}]{mudrāṃ}
		\rdg[wit={N3}]{sadā} 
		\rdg[wit={J5},alt={\om}]{\skp{\om}}
		} siddhiṃ % siddhi N3,V19,J10
\app{\lem[wit={ceteri}]{prayacchati}% °tiḥ J10
		\rdg[wit={V19}]{prayakṣati}
		}//\versenr}\label{III128} 
	%\lineom{cd}{J5} 
%	\NotIn{E2,N19}
	\\!}
\end{tlg}
\commcite\newpage


\startaltnormal
\begin{alttlg}[hp03_107_1]
\tl{\app{\lem[nolem]{}
	\rdg[wit={Gr1}]{\excl}}%
\pada{\app{\lem[wit={ceteri}]{upadeśaṃ}% °dehaṃ C6, °de<<śaṃ>> J7
		\rdg[wit={V1}]{upadeśe}
		\rdg[wit={N19}]{upadeśo}} hi mudrāṇāṃ}
\pada{yo 
	\app{\lem[wit={J7,Gr3,G11,G5,N19,J10,GrB}]{dhatte}% dadyāt J7pc
		\rdg[wit={V15,Jyo}]{datte}
		\rdg[wit={N23}]{dartte}
		\rdg[wit={V1}]{..\,[tte]}}
	\app{\lem[wit={Gr3,G5,V1,J10,P11,V3,Jyo}]{sāṃpradāyikam}% saṃ° P11, sāpra° V3
		\rdg[wit={V15},alt={°yikāṃ}]{sāṃpradāyikāṃ}
		\rdg[wit={Gr2},alt={°yikaḥ}]{sāṃpradāyikaḥ}
		\rdg[wit={N19},alt={°yakaṃ}]{sāṃpradāyakaṃ}% +F,G7
		\rdg[wit={C6},alt={°yakaḥ}]{sāṃpradāyakaḥ}
		\rdg[wit={G11}]{sāṃparāyikaṃ}}/}\\+}
\tl{
\pada{sa \app{\lem[wit={J7,Gr3,G11,G5,N19,V1,P11}]{evāstu}
		\rdg[wit={V3}]{evastu}
		\rdg[wit={V15,J10,Jyo}]{eva śrī}
		\rdg[wit={N23}]{evavāca}
		\rdg[wit={C6}]{vāstava}} 
	\app{\lem[wit={J7,Gr3,N19,V15,C6,V3,Jyo}]{guruḥ svāmī}% <gu>ruḥ V19
		\rdg[wit={N23,V1,J10}]{gurusvāmī}% +F
		\rdg[wit={G11,P11}]{gurusvāmi}
		\rdg[wit={G5}]{kuru svāmī}
		}}
\pada{sākṣādīśvara % sākhyād N23, sakṣād V19
\app{\lem[wit={ceteri}]{eva}
		\rdg[wit={N19}]{eṣa}}
\app{\lem[wit={ceteri}]{saḥ} % sa V3
		\rdg[wit={N23}]{ca}}//\versenr} 
	%\Allexcept{Gr1}
	\\!}
\end{alttlg}
\altcommcite%\newpage

\begin{alttlg}[hp03_107_2]
\tl{\app{\lem[nolem]{}
	\rdg[wit={Gr1,V1}]{\excl}}%
\pada{tasya vākyaparo % °parā N23, °varo G11
	\app{\lem[wit={G11,G5}]{mudrāṃ}% mudrāṃ M1,M3,G7
		\rdg[wit={Gr2,Gr3,V15,Jyo}]{bhūtvā}% +K1,N22,P6
		\rdg[wit={N19,J10,GrB}]{nityaṃ}% nitya V3
		}}
\pada{\app{\lem[wit={Gr3,G11,G5,V15,P11}]{yo'bhyasyati}% +M1,M3,N22
		\rdg[wit={C6}]{yo bhyasati}% +P6
		\rdg[wit={N23}]{yo bhyaset su°}
		\rdg[wit={J7,N19}]{yo bhyaseta}% yobhyāsena G7
		\rdg[wit={V3}]{yomabhyaset}
		\rdg[wit={J10}]{athābhyāsa}
		\rdg[wit={Jyo}]{mudrābhyāse}
		} samāhitaḥ/}\\+} % °hita V3
\tl{
\pada{aṇimādi% ścaṇi° N23
	\app{\lem[wit={G11,G5}]{guṇaiḥ so'yaṃ}
		\rdg[wit={N23}]{guṇaiḥ svaryaṃ}
		\rdg[wit={J7,Gr3,N19,J10,GrB}]{guṇaiśvaryaṃ}% +F
		\rdg[wit={V15}]{guṇai sārdhaṃ}
		\rdg[wit={Jyo}]{guṇaiḥ sārdhaṃ}% +G7
		}}
\pada{\app{\lem[wit={ceteri}]{jāyate}% +K1,N22,P6
		\rdg[wit={G11}]{jayate}% +G7
		\rdg[wit={J10,Jyo}]{labhate}
		}
kāla\app{\lem[wit={G11,N19,V15,J10,C6,V3}]{vañcakaḥ} % +G7
		\rdg[wit={P11}]{vañcakāḥ}
		\rdg[wit={Gr2}]{vañcanāt}% +K1,N22,P6
		\rdg[wit={Gr3,G5,Jyo}]{vañcanam}% +M1,M3,F
		}//\versenr}
	%\Allexcept{Gr1,V1}
	\\!}
\end{alttlg}
\altcommcite\newpage

% aṇimādiguṇaiśvaryaṃ jayate kālavañcakaḥ
%The yogi who has become intent on that [guru's] teaching and practises with a focused mind aquires mastery of the powers beginning with minimisation as cheater of death.

\endaltnormal

\begin{col}[hp03_col]
iti % ity J10
\app{\lem[wit={V3}]{śrīsvātmārāma}
		\rdg[wit={N3}]{śrīsadgurusvātmārāma}
		\rdg[wit={N23,V1}]{svātmārāma}
		\rdg[wit={J10}]{ātmārāma}
		\rdg[wit={J7,N19,V15}]{śrīsahajānandasaṃtānacintāmaṇi(nā \getsiglum{V15})svātmārāma}
		\rdg[wit={P11}]{°svā° (sic!)}
		\rdg[wit={J5,Gr3,G11,G5,C6},alt={\om}]{\skp{\om}}}%
\app{\lem[wit={Gr2,J10,V3}]{yogīndra} % yogiṃdra V3
		\rdg[wit={N3}]{yogeṃdra}
		\rdg[wit={N19,V15,V1}]{yoginā}
		\rdg[wit={J5,Gr3,G11,G5,P11,C6},alt={\om}]{\skp{\om}}}%
\app{\lem[wit={N3,Gr2,N19,V15,V1,J10,V3}]{viracitāyāṃ}% virici° N23
		\rdg[wit={J5,Gr3,G11,G5,P11,C6},alt={\om}]{\skp{\om}}}
\app{\lem[wit={N3,J7,E2,G11,G5,N19,V15,V1,J10,V3}]{haṭhapradīpikāyāṃ}
		\rdg[wit={J5,C6}]{śrīhaṭhapradīpikāyāṃ}% +K3
		\rdg[wit={V19}]{haṭhayogavidyāyāṃ}
		\rdg[wit={P11}]{ha° (sic!)}
		\rdg[wit={N23},alt={\om}]{\skp{\om}}}
\app{\lem[alt={\ante tṛtīyo° \add},nosep]{\skp{\ante tṛtīyo° \add}}
		\rdg[wit={V15}]{mudrāvidhānaṃ}}%
\app{\lem[wit={N3,J5,J7,G11,G5,N19,V15,GrB}]{tṛtīyopadeśaḥ} % tṛtiyo V3
		\rdg[wit={Gr3}]{tṛtīya upadeśaḥ}
%		\rdg[wit={K3,C7}]{tṛtīyoyam upadeśaḥ}
		\rdg[wit={V1,J10}]{tṛtīyo dhyāyaḥ}
		\rdg[wit={N23},postwit=\texteng{(see note on 3.93*1)}]{caturthopadeśa}}//~3//
\end{col}
\colcite

\end{ekdosis}
\end{otherlanguage}
\end{document}

\bigskip
\begin{center}
* * *
\end{center}
%\vfill
%\section*{List of sigla}

% N23,J7, V19,C7, P15(up to 13a),N19,V15, V1,N3,V3,J10,Jyo
% J6,E4 (for the Khecaryabhyāsakrama only)

\teimute{\small}
\begin{tabular}{llp{8cm}}
\multicolumn{3}{l}{\textbf{List of Sigla}} \\
\\
\getsiglum{N3} & N3 & one folio is missing (\ref{VuIII88}--\ref{VuIII116}a)\\
\getsiglum{J5} & J5 \\
\getsiglum{G4} & G4 & damaged; collated only when available\\
%\getsiglum{N24} & N24 \\
\getsiglum{N23} & N23 \\
\getsiglum{J7} & J7 \\
\getsiglum{V19} & V19 \\
\getsiglum{E2} & E2 & the Vajrolī section is not available\\
\getsiglum{C7} & C7 & collated only for the Vajrolī section\\
\getsiglum{J6} & J6 & collated only for \manuref{3.31*1--32*19}\\
\getsiglum{G11} & G11 \\
\getsiglum{G5} & G5 & collated only from Vajrolī till end\\
\getsiglum{P15} & P15 & lost after \manuref{3.13a}\\
\getsiglum{N19} & N19 \\
\getsiglum{V15} & V15 & \ref{III49-2}c--\ref{Jaala1} omitted; contaminated with Gr3?\\
%\getsiglum{J11} & J11 & collated only for \ref{III49-2}--\ref{III66} as substitute for \getsiglum{V15}\\
\getsiglum{V1} & V1 \\
\getsiglum{J10} & J10 \\
\getsiglum{E4} & E4 & Berlin or. fol. 2166; collated only for \manuref{3.31*1--32*19}\\
\getsiglum{P11} & P11 \\
\getsiglum{C6} & C6 & contaminated with Gr3?\\
\getsiglum{V3} & V3 \\
\getsiglum{Jyo} & Jyo &  Brahmānanda's version\\
\end{tabular}
%\vfill
\end{document}


