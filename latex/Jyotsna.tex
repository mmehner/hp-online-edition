\enableregime[utf]
\definepapersize[IeT][width=155mm, height=220mm]
\setuppapersize[IeT][A4]       
\setuplayout[width=fit,height=fit,header=14pt,grid=no,marking=off,backspace=13mm,topspace=15mm,location=middle] 
\setupheader[leftwidth=.8\hsize, rightwidth=.8\hsize,style=italic]
\setupheadertexts[][][][]
\setuppagenumbering[alternative=doublesided,location={footer,center}]
\definefontfeature[default][default][fakecombining=yes,compose=yes]

\starttypescript [serif] [adobetextpro]
  \definefontsynonym[AdobeTextPro-Regular]     [file:AdobeTextPro-Regular] [features=default]
  \definefontsynonym[AdobeTextPro-Italic]          [file:AdobeTextPro-It]  [features=default]
  \definefontsynonym[AdobeTextPro-Bold]           [file:AdobeTextPro-Bold]  [features=default]
\stoptypescript

\starttypescript [serif] [adobetextpro] [name]
  \definefontsynonym [Serif]           [AdobeTextPro-Regular]  
  \definefontsynonym [SerifItalic]     [AdobeTextPro-Italic] 
  \definefontsynonym [SerifBold]       [AdobeTextPro-Bold] 
\stoptypescript

\starttypescript [adobetextpro]
  \definetypeface [adobetextpro] [rm] [serif] [adobetextpro] [default] [features=default]
\stoptypescript

\setupbodyfont[adobetextpro] 
\mainlanguage[sa]  
% Critical Edition Setup: Prose with two layers of apparatus
\setupnotations[alternative=serried]
\setuplinenumbering[step=5,method=page]
\definelinenote[var][paragraph=yes]
\definelinenote[comment][paragraph=yes]   %oder: n=2]
\definelines[sloka][][indenting={no, small},
                   before=\blank\startnarrower,after=\stopnarrower\blank]
\setuptolerance[vertical,verytolerant]
\setuptolerance[horizontal,verytolerant,stretch]
\clubpenalty=10000 % keine Schusterjungen / no clubs
\widowpenalty=10000 % keine Hurenkinder / no widows
\setupwhitespace[3pt]    %  3pt zwischen Paragraphen
\setupblank[big]    %	  \blank[2*line],   \blank[medium]
\def\lem{\kern0em{}]\enskip}                 

\def\Wthree{W\ }  \def\Wthreeac{W\high{ac}\ }  \def\Wthreepc{W\high{pc}\ } 
\def\Tue{T\ }    \def\Tueac{T\high{ac}\ }  \def\Tuepc{T\high{pc}\ }
\def\Msix{M}\def\EdLo{Ed\low{Lon}\ }   \def\EdAd{Ed\low{Adyar}\ }
\def\EdMu{Ed\low{Princ}\ } \def\emend{em.\ }
\def\om{om.}  
  
\starttext

{\tfa\bf  Haṭhapradīpikājyotsnā}



\blank[7mm]

\startlinenumbering
%\begin{vsid}{#hp01_001}
\startsloka
{\bf śrīādināthāya namo 'stu tasmai 
\ \ yenopadiṣṭā haṭhayogavidyā
vibhrājate pronnatarājayogam 
\ \ āroḍhum icchor adhirohiṇīva (1.1)}
\stopsloka

\startsloka
guruṃ natvā śivaṃ sākṣād brahmānandena tanyate
\ \ haṭhapradīpikājyotsnā yogamārgaprakāśikā 

idānīṃtanānāṃ\var{idānīṃtanānāṃ \Msix] idānīṃ janānāṃ \EdLo} subodhārtham asyāḥ 
\ \ suvijñāya gorakṣasiddhāntahārdam 
mayā meruśāstripramukhyābhiyogāt
\ \ sphuṭaṃ kathyate 'tyantagūḍho 'pi bhāvaḥ
\stopsloka

\blank

mumukṣujanahitārthaṃ rājayogadvārā kaivalyaphalāṃ haṭhapradīpikāṃ vi\-dhitsuḥ paramakāruṇikaḥ
svātmārāmayogīndras tatpratyūhanivṛttaye haṭhayogapravartakaśrīmadādināthanamaskāralakṣaṇaṃ
maṅgalaṃ tāvad ācarati śrī\-ādināthetyādinā. tasmai śrīādināthāya namaḥ astu ity anvayaḥ. ādiś
cāsau nāthaś ca ādināthaḥ sarveśvaraḥ śiva ity arthaḥ. śrīmān ādināthaḥ śrīādināthaḥ tasmai
śrīādi\-nāthāya. śrīśabda ādir yasya saḥ śrīādiḥ śrīādiś cāsau nāthaś ca śrīādināthaḥ tasmai
śrīādināthāya śrīnāthāya viṣṇave iti vārthaḥ.  śrīādināthāyety atra yaṇabhāvas tu api māṣaṃ
maṣaṃ\var{māṣaṃ maṣaṃ \lem māsaṃ masaṃ \Msix (not reported in \EdLo)} kuryāc chandobhaṅgaṃ tyajed girām
iti chandovidāṃ sampradāyād uccāraṇasauṣṭhavāc ceti bodhyam. namaḥ prahvī\-bhāvo 'stu. prārthanāyāṃ
loṭ. tasmai. kasmai ity apekṣāyāṃ āha yeneti yena ādināthena upadiṣṭā girijāyai
haṭhayogavidyā. haś ca ṭhaś ca haṭhau sūryacandrau tayor yogo haṭhayogaḥ. etena haṭhaśabdavācyayoḥ
sūryacandrākhyayoḥ prāṇāpānayor aikyalakṣaṇaḥ prāṇāyāmo haṭhayoga iti haṭha\-yogasya lakṣaṇaṃ
siddham. tathā coktaṃ gorakṣanāthena siddha\-siddhāntapaddhatau

\startsloka
hakāraḥ kīrtitaḥ sūryaṣ ṭhakāraś candra ucyate
sūryacandramasor yogād dhaṭhayogo nigadyate 
\stopsloka



iti tatpratipādikā vidyā haṭhayogavidyā haṭhayogaśāstram iti yāvat. girijāyai\var{girijāyai \lem
  girijāyāḥ \Msix} ādināthakṛto haṭhavidyopadeśo mahākālayogaśāstrādau prasiddhaḥ.

prakarṣeṇa unnataḥ pronnataḥ. mantrayogahaṭhayogādīnām adharabhūmī\-nām uttarabhūmitvād rājayogasya
pronnatatvam. rājayogaś ca sarvavṛttini\-rodhalakṣaṇo 'samprajñātayogaḥ. tam icchor mumukṣor
adhirohiṇīva adhi\-ruhyate 'nayety adhirohiṇī niḥśreṇīva vibhrājate viśeṣeṇa bhrājate śobhate. yathā
pronnatasaudham āroḍhum icchor adhirohiṇy anāyāsena saudha\-prāpi\-kā bhavati evaṃ haṭhapradīpikāpi
pronnatarājayogam āroḍhum icchor anā\-yāsena rājayogaprāpikā bhavatīti upamālaṅkāraḥ.
indravajrākhyaṃ vṛttam. 
%\end{vsid}


%\begin{vsid}{#hp01_002}
\startsloka
{\bf praṇamya śrīguruṃ nāthaṃ svātmārāmeṇa yoginā
kevalaṃ rājayogāya haṭhavidyopadiśyate (1.2)}
\stopsloka

evaṃ paramagurunamaskāralakṣaṇaṃ maṅgalaṃ kṛtvā vighnabāhulye maṅga\-labāhulyasyāpy\var{apy \lem om
  \Msix} apekṣitatvāt svagurunamaskārātmakaṃ maṅgalam ācarann asya granthasya viṣayaprayojanādīn
pradarśayati praṇamyeti.\var{praṇamyeti \lem om \Msix} śrīmantaṃ guruṃ śrīguruṃ nāthaṃ śrīgurunāthaṃ
svagurum iti yāvat.  praṇamya pra\-karṣeṇa bhaktipūrvakaṃ natvā svātmārāmeṇa\var{svātmārāmeṇa \lem
  svātmārāmeṇa svātmārāmākhyeṇa \Msix} yoginā yogo 'syāstīti yena\var{yena \lem tena \Msix}. kevalaṃ
rājayogāya kevalaṃ rājayogārthaṃ haṭhavidyopadiśyata ity anvayaḥ. haṭhavidyāyā rājayoga eva mukhyaṃ
phalaṃ na siddhaya iti kevalapadasyā\-bhi\-prāyaḥ. siddhayas tv ānuṣaṅgikyaḥ\var{ānuṣaṅgikyaḥ \lem
  ānuṣaṅgikāḥ \Msix}. etena rājayogaphalasahito haṭhayogo 'sya granthasya viṣayaḥ. rājayogadvārā
ükaivalyaṃ cāsya phalam.  tatkāmaś cādhikārī. granthaviṣayayoḥ pratipādyapratipādakabhāvaḥ
sambandhaḥ. granthasya kaivalyasya ca prayojyaprayojakabhāvaḥ sambandhaḥ. granthābhidheyasya
saphalayogasya kaivalyasya ca sādhyasādhanabhāvaḥ sambandha ity uktam.
%\end{vsid}



%\begin{vsid}{#hp01_003}
\startsloka
{\bf bhrāntyā bahumatadhvānte rājayogam ajānatām
haṭhapradīpikāṃ dhatte svātmārāmaḥ kṛpākaraḥ (1.3)}
\stopsloka

nanu mantrayogasaguṇadhyānanirguṇadhyānamudrādibhir eva rājayogasiddhau kiṃ haṭhavidyopadeśenety
āśaṅkya vyutthitacittānāṃ mantrayogādibhī rājayogāsiddher haṭhayogād eva rājayogasiddhiṃ vadan
granthaṃ pratijānīte bhrāntyeti. mantrayogādibahumatarūpe dhvānte gāḍhāndhakāre yā bhrāntir bhramas
tayā\var{tayā \lem om \Msix}. tais tair upāyai rājayogārthaṃ pravṛttasya tatra tatra
tadalābhāt. vakṣyati ca haṭhaṃ\var{haṭhaṃ \Msix \lem haṭhayogaṃ Ś} vinā rājayogaḥ
ityādinā. tathā\var{tathā \lem tayā \Msix} rājayogam ajānatāṃ na jānantīty ajānantaḥ teṣām ajānatāṃ
puṃsāṃ rājayogajñānāyeti śeṣaḥ. karotīti karaḥ kṛpāyāḥ karaḥ kṛpākaraḥ kṛpāyā ākara iti vā
tādṛśaḥ. anena haṭhapradīpikākaraṇe ajñānukampaiva\var{ajñānukampaiva \lem bhūtānukampaiva \Msix}
hetur ity uktam. svātmany āramate iti svātmārāmaḥ. haṭhasya haṭhayogasya pradīpikeva prakāśakatvāt
haṭhapradīpikā tām. athavā haṭha eva pradīpikā haṭhapradīpika rājayogaprakāśakatvāt tāṃ dhatte
vidhatte karotīti yāvat.

svātmārāma ity anena jñānasya saptamabhūmikāṃ\var{saptama \Wthree \Tuepc \Msix ] sapta \Tueac ,
saptaṃ \EdLo} prāpto brahmavidvariṣṭha ity uktam. tathā ca śrutiḥ ātmakrīḍa ātmaratiḥ kriyāvān eṣa
  brahmavidāṃ variṣṭhaḥ\comment{{\em Muṇḍakopaniṣat} 3.1.4} iti.

saptabhūmayaś coktā yogavāsiṣṭhe 

\startsloka
jñānabhūmiḥ śubhecchākhyā prathamā samudāhṛtā
vicāraṇā dvitīyā syāt tṛtīyā tanumānasā

sattvāpattiś caturthī syāt tato 'saṃsaktināmikā
parārthābhāvinī ṣaṣṭhī saptamī turyagā smṛtā iti\comment{{\em Laghuyogavāsiṣṭha} 9.114–115}
\stopsloka

asyārthaḥ śubhecchā ity ākhyā yasyāḥ sā śubhecchākhyā vivekavairāgyayutā\var{yutā \lem om \Msix}
śamādipūrvikā tīvramumukṣā prathamā jñānasya bhūmiḥ bhūmikā udāhṛtā kathitā yogibhir iti śeṣaḥ.

vicāraṇā śravaṇamananātmikā dvitīyā jñānabhūmiḥ syāt.

anekārthagrāhakaṃ mano yadānekārthān parityajya sadekārthavṛttipra\-vāha\-vad bhavati tadā
tanumānasaṃ yasyāṃ sā tanumānasā nididhyāsanarūpā tṛtīyā jñānabhūmiḥ syād iti śeṣaḥ.

imās tisraḥ sādhanabhūmikāḥ. āsu bhūmiṣu sādhaka ity ucyate. tisṛbhir bhūmikābhiḥ śuddhasattve
'ntaḥkaraṇe 'haṃ brahmāsmīty ākārikā aparokṣa\-vṛttirūpā sattvāpattināmikā caturthī jñānabhūmiḥ
syāt. caturthīyaṃ phalabhūmiḥ. asyāṃ yogī brahmavid ity ucyate. iyaṃ samprajñātayogabhūmikā.

vakṣyamāṇās tisro 'samprajñātayogabhūmayaḥ. sattvāpatter\var{sattvāpatter \lem tataḥ sattvāpatter
  \Msix} anantarā sattvā\-pattisaṃjñikāyāṃ bhūmāv upasthitāsu siddhiṣu asaṃsaktasyāsaṃsaktināmikā
pañcamī jñānabhūmiḥ syāt. asyāṃ yogī svayam eva vyuttiṣṭhate. etāṃ\var{etāṃ \lem enāṃ \Msix} bhūmiṃ
prāpto brahmavidvara ity ucyate.

parabrahmātiriktam arthaṃ na bhāvayati yasyāṃ sā parārthābhāvinī ṣaṣṭhī jñānabhūmiḥ syāt. asyāṃ
yogī paraprabodhita eva vyutthito bhavati. etāṃ\var{etāṃ \lem enāṃ \Msix} prāpto brahmavidvarīyān ity
ucyate.

turyagā nāma saptamī bhūmiḥ\var{bhūmiḥ \lem jñānabhūmiḥ \Msix} smṛtā. asyāṃ yogī svataḥ parato vā na
vyutthānaṃ prāpnoti. etāṃ\var{etāṃ \lem enāṃ \Msix} prāpto brahmavidvariṣṭha ity ucyate.

tatra pramāṇabhūtā śrutir atraivoktā pūrvam. ayam eva jīvanmukta ity ucyate sa evātra
svātmārāmapadenokta ity alaṃ bahūktena.
%\end{vsid}



%\begin{vsid}{#hp01_004}
\startsloka
{\bf haṭhavidyāṃ hi matsyendragorakṣādyā vijānate
svātmārāmo 'thavā yogī jānīte tatprasādataḥ} (1.4)
\stopsloka

mahatsevitatvād dhaṭhavidyāṃ praśaṃsan svasyāpi mahatsakāśād dhaṭha\-vidyālābhād gauravaṃ
dyotayati haṭhavidyāṃ hīti. hīti prasiddham. matsyendraś ca gorakṣaś ca tau ādyau yeṣāṃ te
matsyendragorakṣādyāḥ ādyaśabdena jālandharanāthabhartṛharigopīcandaprabhrtayo grāhyāḥ. te
haṭhavidyāṃ haṭhayogavidyāṃ vijānate viśeṣeṇa sādhanalakṣaṇabhedaphalair jānantīty arthaḥ.

svātmārāmaḥ svātmārāmanāmā. athavāśabdaḥ samuccaye. yogī yogavān tatprasādataḥ gorakṣaprasādāj
jānīta ity anvayaḥ. paramamahatā brahmaṇāpīyaṃ vidyā sevitety atra yogiyājñavalkyasmṛtiḥ\comment{?}

\startsloka
hiraṇyagarbho yogasya vaktā nānyaḥ purātanaḥ
\stopsloka

vaktṛtvaṃ ca mānasavyāpārapūrvakaṃ bhavatīti mānaso vyāpāro 'rthād āgamaḥ\var{āgamaḥ \lem āgataḥ
  \Msix}. tathā ca śrutiḥ yan manasā dhyāyati tad vācā vadati\comment{nṛ.pū.tā.u. 1.1 ?}
iti. bhagavateyaṃ\var{bhagavateyaṃ \lem bhagavatāpīyaṃ \Msix} vidyā bhāgavatān\var{bhāgavatān \lem
  bhāgavatādau \Msix} uddhavādīn praty uktā. śivas tu yogī prasiddha eva. evaṃ ca sarvottamair
brahmaviṣṇuśivaiḥ seviteyaṃ vidyā.  na ca brahmasūtrakṛtā vyāsena yogo nirākṛta iti
śaṅkanīyam. prakṛtisvātantryacidbhedāṃśamātrasya nirākaraṇāt na tu
bhāvanāviśeṣarūpayogasya. bhāva\-nāyāś ca sarvasaṃmatatvāt tāṃ vinā sukhasyāpy
asambhavāt. tathoktaṃ bhagavadgītāsu\comment{Bhagavadgītā 2.66} iti\var{iti \lem om \Msix}

\startsloka
nāsti buddhir ayuktasya na cāyuktasya bhāvanā
na cābhāvayataḥ śāntir aśāntasya kutaḥ sukham
\stopsloka

nārāyaṇatīrthair apy uktam\comment{Yogasiddhāntacandrikā, introductory verses, p.1.}    

\startsloka
svātantryasatyatvamukhaṃ\var{mukhaṃ ] sukhaṃ Yogasiddhāntacandrikā, ed. ??} pradhāne 
  satyaṃ ca cidbhedam abhedavākyaiḥ\comment{cidbhedam abhedavākyaiḥ \Wthree \Msix ] cidbhedagataṃ ca vākyaiḥ in editio princeps
  and most manuscripts}
vyāso nirācaṣṭa na bhāvanākhyaṃ 
yogaṃ svayaṃ nirmitabrahmasūtraiḥ\comment{The ligature {\em bra}, for which I could not find a variant,
  breaks the metre here.}

api cātmapradaṃ yogaṃ vyākaron matimān svayam
bhāṣyādiṣu tatas tatrāpy ācāryapramukhair mataḥ
mato yogo bhagavatā gītāyām adhiko 'nyataḥ
kṛtaḥ śukādibhis tasmād atra santo 'tisādarāḥ

iti 

vedeṣu yajñeṣu tapaḥsu caiva
dāneṣu yat puṇyaphalaṃ\var{puṇyaphalaṃ \lem kurmaphalaṃ \Msix} pradiṣṭam 
atyeti tat sarvam idaṃ viditvā
yogī paraṃ sthānam upaiti cādyam\comment{BhG 8.28}
\stopsloka

iti bhagavadukteḥ. kiṃ bahunā jijñāsur api yogasya śabdabrahmātivartate\comment{BhG 6.44} iti
vadatā bhagavatā yogajijñāsor apy autkṛṣṭyaṃ varṇitaṃ kim uta yoginaḥ. nāradādibhaktaśreṣṭhair
yājñavalkyādijñānimukhyaiś cāsyāḥ sevanād bhaktajñāninām apy aviruddhety uparamyate.
%\end{vsid}



%\begin{vsid}{#hp01_005}
\startsloka
{\bf śrīādināthamatsyendraśābarānandabhairavāḥ
cauraṅgīmīnagorakṣavirūpākṣabileśayāḥ (1.5)}
\stopsloka

haṭhayoge pravṛttiṃ janayituṃ haṭhavidyayā prāptaiśvaryān siddhān āha śrīādināthetyādinā.
ādināthaḥ śivaḥ sarveṣāṃ nāthānāṃ prathamo nāthaḥ. tato nāthasampradāyaḥ pravṛtta iti
nāthasampradāyino vadanti. 

matsyendrākhyaś cādināthaśiṣyaḥ. atraivaṃ kiṃvadantī. kadācid ādi\-nāthaḥ kasmiṃścid dvīpe sthitaḥ
tatra vijanam iti matvā girijāyai yogam upadiṣṭavān. tīrasamīpanīrasthaḥ kaścana\var{kaścana ]
kaścan \EdLo} matsyas taṃ yogopadeśaṃ śrutvā ekāgracitto niścalakāyo 'vatasthe. taṃ tādṛśaṃ
dṛṣṭvānena yogaḥ śrutaḥ iti taṃ matvā kṛpālur ādinātho jalena prokṣitavān. sa ca prokṣaṇamātrād
divyakāyo matsyendraḥ siddho 'bhūt.  tam eva matsyendranātha iti vadanti.

śābaranāmā kaścit siddhaḥ. ānandabhairavanāmānyaḥ. eteṣām itaretaradvandvaḥ. chinnahastapādaṃ
puruṣaṃ hindusthānabhāṣāyāṃ cauraṅgīti vadanti. kadācid ādināthāl labdhayogasya bhuvaṃ paryaṭato
matsyendranāthasya kṛpāvalokanamātrāt kutracid araṇye sthitaś cauraṅgy aṅkuritahastapādo babhūva.
sa ca tat\var{ca tat \lem caitat \Msix}kṛpayā sañjātahastapādo 'ham iti matvā tatpādayoḥ praṇipatya
mamānugrahaṃ kurv iti prārthitavān. matsyendro 'pi tam anugṛhītavān. tasyānugrahāc cauraṅgīti
prasiddhaḥ siddhaḥ so 'bhūt.

mīno mīnanāthaḥ gorakṣo gorakṣanāthaḥ virūpākṣanāmā bileśayanāmā ca. cauraṅgīprabhṛtīnāṃ
dvandvasamāsaḥ. 
%\end{vsid}



%\begin{vsid}{#hp01_006}
\startsloka
{\bf manthāno bhairavo yogī śuddhabuddhaś ca kanthaḍiḥ
koraṇṭakaḥ surānandaḥ siddhapādaś ca carpaṭiḥ (1.6)}
\stopsloka

manthāna iti. manthānaḥ bhairavaḥ. yogīti manthānaprabhṛtīnāṃ sarveṣāṃ viśeṣaṇam.
%\end{vsid}



%\begin{vsid}{#hp01_007}
\startsloka
{\bf kānerī pūjyapādaś ca nityanātho nirañjanaḥ
kapālī bindunāthaś ca kākacaṇḍīśvarāhvayaḥ (1.7)}
\stopsloka

kānerīti.\var{kānerīti \om \Msix} kākacaṇḍīśvara ity āhvayo nāma yasya sa tathā. anye spaṣṭāḥ.
%\end{vsid}


%\begin{vsid}{#hp01_008}
\startsloka
  {\bf allamaḥ\comment{Here and in the commentary mss.\ read short or long ā in various patterns.
      The text is here standardized following the critical edition of the mūla in the HP project.}
                                                   prabhudevaś ca ghoḍācolī ca ṭiṇṭiṇiḥ.
bhālukī nāgadevaś ca khaṇḍakāpālikas tathā (1.8)}
\stopsloka

allama iti.\var{allama iti \lem om \Msix} tathāśabdaḥ samuccaye.
%\end{vsid}

%\begin{vsid}{#hp01_009}
\startsloka
{\bf ityādayo mahāsiddhā haṭhayogaprabhāvataḥ
khaṇḍayitvā kāladaṇḍaṃ brahmāṇḍe vicaranti te (1.9)}
\stopsloka

iti pūrvoktā ādayo yeṣāṃ te tathā. ādiśabdena tārānāthādayo grāhyāḥ. mahāntaś ca te siddhāś ca
apratihataiśvaryā ity arthaḥ. haṭhayogasya prabhāvāt sāmarthyād iti haṭhayogaprabhāvataḥ.
pañcamyās tasil. kālo mṛtyus tasya daṇḍanaṃ daṇḍaḥ dehaprāṇaviyogānukūlo vyāpāraḥ taṃ
khaṇḍayitvā chittvā mṛtyuṃ jitvety arthaḥ. brahmāṇḍe brahmāṇḍamadhye vicaranti
viśeṣeṇāvyāhatagatyā carantīty arthaḥ. tad uktaṃ bhāgavate

\startsloka  % check sources
yogeśvarāṇāṃ gatim āhur antar
bahistrilokyāḥ pavanāntarātmanām \comment{Bhāgavatapurāṇa 2.2.23ab} iti.
%na karmabhis tāṃ gatim āpnuvanti
%vidyātapoyogasamādhibhājām   the rest of the verse was added in the e-text of unknown provenance. 
\stopsloka
%\end{vsid}


%\begin{vsid}{#hp01_010}
\startsloka
{\bf aśeṣatāpataptānāṃ samāśrayamaṭho haṭhaḥ
aśeṣayogayuktānām ādhārakamaṭho haṭhaḥ (1.10)}
\stopsloka

haṭhasyāśeṣatāpanāśakatvam aśeṣayogasādhakatvaṃ ca maṭhakamaṭha\-rūpakeṇāha aśeṣeti. aśeṣāḥ
ādhyātmikādhibhautikādhidaivikabhedena trividhāḥ. tatrādhyātmikaṃ dvividham śārīraṃ mānasaṃ ca.
tatra śārīraṃ duḥkhaṃ vyādhijaṃ mānasaṃ duḥkhaṃ kāmādijam. ādhibhautikaṃ
vyāghrasarpādijanitam. ādhidaivikaṃ grahādijanitam. te ca te tāpāś ca tais taptānāṃ santaptānāṃ
puṃsāṃ haṭho haṭhayogaḥ. samyag āśrīyata iti samāśrayaḥ āśrayabhūto maṭhaḥ maṭha eva. tathā haṭhaḥ
aśeṣayogayuktānāṃ aśeṣayogayuktāḥ\var{aśeṣayogayuktāḥ ] aśeṣā ye yogayuktāḥ \Msix \Wthree (variant
not reported by \EdLo)} mantrayogakarmayogādiyuktās teṣām ādhārabhūtaḥ kamaṭhaḥ.\comment{\EdLo\
    reports an inserted line here in ms. I. The line occurs also in \Tue, but is there deleted. In
    VI: kamaṭha eva yathā taraṇikiraṇatāpataptānāṃ puṃsāṃ āśrayo maṭhaḥ} evaṃ trividhatāpataptānāṃ
  puṃsām\var{puṃsām \lem om \Msix} āśrayo haṭhaḥ. yathā ca viśvādhāraḥ kamaṭha evaṃ
  nikhilayoginām\var{yoginām \lem yogānām \Msix} ādhāro haṭha ity arthaḥ.
%\end{vsid}



%\begin{vsid}{#hp01_011}
\startsloka
{\bf haṭhavidyā paraṃ gopyā yoginā siddhim icchatā
bhaved vīryavatī guptā nirvīryā tu prakāśitā (1.11)}
\stopsloka

athākhilavidyāpekṣayā haṭhavidyāyā atigopyatvam āha haṭhavidyeti. siddhim aṇimādyaiśvaryam icchatā.
yad vā siddhiṃ kaivalyasiddhim icchatā vāñchatā yoginā haṭhavidyā haṭhayogavidyā param atyantaṃ
gopyā gopanīyā\var{gopanīyā \lem om \Msix} gopanārhāstīti. tatra hetum āha yato guptā haṭhavidyā
vīryavatī bhavet apratihataiśvaryajananasamarthā syāt. kaivalyajananasamarthā
kaivalyasiddhijananasamarthā vā syāt\var{syāt \lem bhavet \Msix}.  prakāśitā prasiddhiṃ gamitā tu
nirvīryā. dīrghakālasevitāpi apratihataiśvaryajananāsamarthā kaivalyasiddhijananāsamarthā vā
syāt. atha yogādhikārī\comment{Source unknown. ??}

\startsloka
  jitākṣāya śāntāya saktāya muktau
  vihīnāya doṣair asaktāya bhuktau
  ahīnāya doṣetarair uktakartre
  pradeyo na deyo haṭhaś cetarasmai
\stopsloka

yājñavalkyaḥ

\startsloka
vidhyuktakarmasaṃyuktaḥ kāmasaṅkalpavarjitaḥ 
yamaiś ca niyamair yuktaḥ sarvasaṅgavivarjitaḥ
kṛtavidyo jitakrodhaḥ satyadharmaparāyaṇaḥ
guruśuśrūṣaṇarataḥ pitṛmātṛparāyaṇaḥ
svāśramasthaḥ sadācāro vidvadbhiś ca suśikṣitaḥ\comment{Yogayājñavalkya 5.3–5ab} 
\stopsloka

iti. śiśnodararatāyaiva na deyaṃ veṣadhāriṇe iti kutracit. atra yogacintāmaṇi\-kārāḥ yady api

\startsloka
brāhmaṇakṣatriyaviśāṃ strīśūdrāṇāṃ ca pāvanam 
śāntaye karmaṇām anyad yogān nāsti vimuktaye\comment{Yogacintāmaṇi, p. 57.}
\stopsloka

ityādipurāṇavākyeṣu prāṇimātrasya yoge 'dhikāra upalabhyate tathāpi mokṣarūpaphalavati yoge
viraktasyaivādhikāra ucitaḥ. tathā ca vāyusaṃhitāyāṃ\comment{Śivapurāṇa 37.12}

\startsloka
dṛṣṭe tathānuśravike viraktaṃ viṣaye manaḥ
yasya tasyādhikāro 'smin yoge nānyasya kasyacit
\stopsloka

sureśvarācāryāḥ\comment{?}

\startsloka
ihāmutraviraktasya saṃsāraṃ prajihāsataḥ
jijñāsor eva kasyāpi yoge 'sminn adhikāritā
\stopsloka

ity āhuḥ. vṛddhair apy uktam—

\startsloka
naitad deyaṃ durvinītāya jātu 
jñānaṃ guptaṃ tad dhi samyak phalāya
asthāne hi sthāpyamānaiva vācāṃ
devī kopān nirdahen no 'cirāya\var{no 'cirāya \lem taṃ cirāya \Msix} 
\stopsloka
%\end{vsid}

iti.



%\begin{vsid}{#hp01_012}
\startsloka
{\bf surājye dhārmike deśe subhikṣe nirupadrave
dhanuḥpramāṇaparyantaṃ śilāgnijalavarjite
ekānte maṭhikāmadhye sthātavyaṃ haṭhayoginā (1.12)}
\stopsloka

atha haṭhābhyāsayogyaṃ deśam āha sārdhena surājya iti. rājñaḥ karma bhāvo vā rājyaṃ tac chobhanaṃ
yasmin sa surājyas tasmin surājye. yathā rājā tathā prajāḥ iti mahadukteḥ. rājñaḥ śobhanatvāt
prajānām api śobhanatvaṃ sūcitam. dhārmike dharmavati\var{dharmavati \Wthree \Tue \Msix \lem
  dharmayati \EdLo} anena haṭhābhyāsino 'nukūlāhārādilābhaḥ sūcitaḥ. subhikṣa ity anenānāyāsena
tallābhaḥ sūcitaḥ. nirupadrave cauravyāghrādyupadravarahite. etena deśasya dīrghakālavāsayogyatā
sūcitā. dhanuṣaḥ pramāṇaṃ dhanuḥpramāṇaṃ caturhastamātraṃ tatparyantam. śilāgnijalavarjite śilā
prastaraḥ agnir vahniḥ jalaṃ toyaṃ tair varjite rahite.  yatrāsanaṃ tataś caturhastamātre
śilāgnijalāni na syur ity arthaḥ. tena śītoṣṇādi\var{ śītoṣṇādi \lem śītoṣṇa \Msix}vikārābhāvaḥ
sūcitaḥ. ekānte vijane. anena\var{anena \lem ity anena \Msix} janasamāgamābhāvāt kalahavārtādyabhāvaḥ
sūcitaḥ. janasaṃmarde tu kalahādikaṃ syād eva. tad uktaṃ bhāgavate

\startsloka
  vāse bahūnāṃ kalaho bhaved vārtā dvayor api\comment{Bhāgavatapurāṇa 11.9.10ab} 
\stopsloka

iti. tādṛśe maṭhikāmadhye\var{tādṛśe maṭhikāmadhye \Wthree \EdMu \lem tādṛśe deśe maṭhikāmadhye
  \Tue Bo3 \Msix, om \EdLo}.  alpo maṭho maṭhikā alpīyasi kan. tasyāḥ madhye haṭhayoginā haṭhābhyāsī
yogī haṭhayogī tena.  śākapārthivādivat samāsaḥ. sthātavyaṃ sthātuṃ yogyam. maṭhikāmadhye ity anena
śītātapādijanitakleśābhāvaḥ sūcitaḥ. atra yuktāhāravihārasya\var{vihārasya \lem vihāreṇa \Msix}
haṭhayogasya siddhaye ity ardhaṃ kenacit kṣiptatvān na vyākhyātam. mūlaślokānām evātra
vyākhyānam. evam agre 'pi ye mayā na vyākhyātā ślokā haṭhapradīpikāyām upalabhyeran te sarve kṣiptā
iti boddhavyam\var{boddhavyam \lem bodhyam \Msix}.
%\end{vsid}


%\begin{vsid}{#hp01_013}
\startsloka
{\bf alpadvāram arandhragartavivaraṃ nātyuccanīcāyataṃ
samyaggomayasāndraliptam amalaṃ niḥśesajantūjjhitam.
bāhye maṇḍapavedikūparuciraṃ prākārasaṃveṣṭitaṃ
proktaṃ yogamaṭhasya lakṣaṇam idaṃ siddhair
\hfill haṭhābhyāsibhiḥ (1.13)}
\stopsloka

atha maṭhalakṣaṇam āha alpadvāram iti\var{alpadvāram iti \lem alpadvāreti \Msix}. alpaṃ dvāram iti
yasmin tat tādṛśam. randhro gavākṣādiḥ garto nimnapradeśaḥ vivaro mūṣakāhibilaṃ\var{mūṣakāhi \lem
  mūṣakāhi \Msix} te na santi yasmin tat tādṛśam. atyuccaṃ ca tan nīcaṃ cātyuccanīcaṃ tac ca tad
āyataṃ cātyuccanīcāyataṃ viśeṣaṇaṃ viśeṣyeṇa bahulam\comment{Aṣṭādhyāyī 2.1.57} ity atra
bahulagrahaṇād viśeṣaṇānāṃ karmadhārayaḥ.

nanūccanīcāyataśabdānāṃ bhinnārthakānāṃ kathaṃ karmadhārayaḥ. tatpuruṣaḥ samānādhikaraṇaḥ
karmadhārayaḥ\comment{Aṣṭādhyāyī 1.2.42} iti tallakṣaṇād iti cen na. maṭhe teṣāṃ
sāmānādhikaraṇyāsambhavāt.\var{sāmānādhikaraṇyāsambhavāt II V XII \lem samānādhikāraṇyasambhavāt IV
  samānādhi\-karaṇy\-asambhavāt VIII X} na cātyucca\var{cātyucca \lem atyucca \Msix}nīcāyataṃ
nātyuccanīcāyataṃ. naśabdena samāsān nalopābhāvaḥ. neti pṛthakpadaṃ vā. na\var{na \lem om \Msix}
atyucce ārohaṇe śramaḥ syād atinīce 'varohaṇe śramo bhavet. atyāyate dūraṃ dṛṣṭir gacchet
tannirākaraṇārtham uktaṃ nātyuccanīcāyatanam iti. samyak samīcīnatayā gomayena
gopurīṣeṇa\var{gopurīṣeṇa \lem goḥ purīṣeṇa \Msix} sāndraṃ yathā bhavati tathā liptam. amalaṃ
nirmalam. niḥśesā nikhilā ye jantavo maśakamatkuṇādyās tair ujjhitam tyaktaṃ rahitam\var{tyaktaṃ
  rahitam \lem tyakte rahite \Msix}.

bāhye maṭhād bahiḥpradeśe. maṇḍapaḥ śālāviśeṣaḥ vediḥ pariṣkṛtā bhūmiḥ kūpo jalāśayaviśeṣaḥ.  tai
ruciraṃ ramaṇīyam. prākāreṇa vara\-ṇena samyag veṣṭitaṃ parito\var{veṣṭitaṃ parito \lem parito
  veṣṭitaṃ parito \Msix} bhittiyuktam ity arthaḥ.  haṭhābhyāsibhiḥ haṭha\-yogābhyasanaśīlaiḥ
siddhaiḥ. idaṃ pūrvoktam alpadvārādikaṃ yogamaṭhasya lakṣaṇaṃ svarūpaṃ proktaṃ kathitam.



nandikeśvarapurāṇe tv evaṃ maṭhalakṣaṇam uktam\comment{Quoted via the Yogacintāmaṇi (p.90)}

\startsloka
mandiraṃ ramyavinyāsaṃ manojñaṃ gandhavāsitam
dhūpāmodādisurabhi kusumotkaramaṇḍitam
munitīrthanadīvṛkṣapadminīśailaśobhitam
citrakarmanibaddhaṃ ca citrabhedavicitritam
kuryād yogagṛhaṃ dhīmān suramyaṃ śubhavartmanā
dṛṣṭvā citragatāṃś chāntān munīn yāti manaḥ śamam
siddhān dṛṣṭvā citragatān matir abhyudyame bhavet
madhye yogagṛhasyātha likhet saṃsāramaṇḍalam
śmaśānaṃ ca mahāghoraṃ narakāṃś ca likhet kvacit
tān dṛṣṭvā bhīṣaṇākārān saṃsāre sāravarjite\var{saṃsāre sāravarjite \lem saṃsāraṃ sāravarjitaṃ \Msix} 
anavasādo bhavati yogī siddhyabhilāṣukaḥ
paśyaṃś ca vyādhitān jantūn natān mattāṃś caladvraṇāt
\stopsloka
%\end{vsid}


%\begin{vsid}{#hp01_014}
\startsloka
{\bf evaṃvidhe maṭhe sthitvā sarvacintāvivarjitaḥ
gurūpadiṣṭamārgeṇa yogam eva sadābhyaset (1.14)}
\stopsloka

maṭhalakṣaṇam uktvā maṭhe yat kartavyaṃ tad āha evaṃvidha iti. evaṃ pūrvoktā vidhā prakāro yasya
sa tathā pūrvoktalakṣaṇa ity arthaḥ. tasmin sthitvā sthitiṃ kṛtvā sarvā yāś cintās tābhir viśeṣeṇa
varjito rahito 'śeṣacintārahitaḥ. guruṇopadiṣṭo yo mārgaḥ haṭhābhyāsaprakārarūpas tena sadā nityaṃ
yogam evābhyaset. evaśabdenābhyāsāntarasya yoge vighna\-karatvaṃ sūcitam. gurūpadiṣṭamārgeṇety anena
gurūpadeśaṃ vinā yogo na sidhyatīti sūcitam. tad uktaṃ yogabīje

\startsloka
marujjayo yasya siddhas taṃ seveta guruṃ sadā
guruvaktraprasādena kuryāt prāṇajayaṃ budhaḥ\comment{Yogabīja 91}
\stopsloka

rājayoge\comment{Amanaskayoga 1.40 = Yogabīja 66}

\startsloka
vedāntatarkoktibhir āgamaiś ca
nānāvidhaiḥ śāstrakadambakaiś ca
dhyānādibhiḥ satkaraṇair na gamyaś 
cintāmaṇir\var{cintāmaṇir \lem cintāmaṇiṃ \Msix} hy ekaguruṃ vihāya
\stopsloka

skandapurāṇe

\startsloka
ācāryād yogasarvasvam avāpya sthiradhīḥ svayam
yathoktaṃ labhate tena prāpnoty api ca nirvṛtim\comment{nirvṛtim \Wthree \Tue \EdMu \lem nivṛtim \EdLo}
\stopsloka

sureśvarācāryaḥ

\startsloka
guruprasādāl labhate yogam aṣṭāṅgasaṃyutam
śivaprasādāl labhate yogasiddhiṃ ca śāśvatīm
\stopsloka

śrutiś ca

\startsloka
yasya deve parā bhaktir yathā deve tathā gurau
tasyaite kathitā hy arthāḥ prakāśante mahātmanaḥ\comment{Śvetāśvataropaniṣat 6.23} 
\stopsloka

iti. ācāryavān puruṣo veda\comment{Chāndogyopaniṣat 6.14.2} iti ca.
%\end{vsid}

%\begin{vsid}{#hp01_015}
\startsloka
{\bf atyāhāraḥ prayāsaś ca prajalpo niyamagrahaḥ
janasaṅgaś ca laulyaṃ ca ṣaḍbhir yogo vinaśyati (1.15)}
\stopsloka

atha yogābhyāsapratibandhakān āha atyāhāra iti. atiśayita āhāro 'tyāhāraḥ
kṣudhāpekṣayādhikabhojanam. prayāsaḥ śramajananānukūlo vyāpāraḥ. prakṛṣṭo jalpaḥ prajalpo
bahubhāṣaṇam. śītodakena prātaḥsnānanakta\var{nakta ] om \EdLo}bhojanaphalāhārādirūpaniyamasya
grahaṇaṃ niyamagrahaḥ.  janānāṃ saṅgo janasaṅgaḥ kāmādijanakatvāt. lolasya bhāvaḥ laulyaṃ
cāñcalyam. ṣaḍbhir atyāhārādibhiḥ abhyāsapratibandhād yogo vinaśyati viśeṣeṇa naśyati.
%\end{vsid}


%\begin{vsid}{#hp01_016}
\startsloka
{\bf utsāhāt sāhasād dhairyāt tattvajñānāc ca niścayāt
janasaṅgaparityāgāt ṣaḍbhir yogaḥ prasiddhyati (1.16)}
\stopsloka

atha yogasiddhikarān āha utsāhād iti. viṣayapravaṇam cittaṃ nirotsyāmy\comment{nirotsyāmy \Msix \lem
  nirotsāmy \EdLo} evety udyamam utsāhaḥ.  sādhyatvāsādhyatve aparibhāvya sahasā pravṛttiḥ
sāhasam. yāvajjīvanaṃ setsyaty evety akhedo dhairyam. viṣayā mṛgatṛṣṇājalavad asantaḥ brahmaiva
satyam\var{satyam \Tue \lem sad \Wthree \Msix} iti vāstavikaṃ jñānaṃ tattvajñānaṃ yogāṅgānāṃ
vāstavikaṃ jñānaṃ vā. śāstraguruvākyeṣu viśvāso niścayaḥ śraddheti yāvat. janānāṃ
yogābhyāsa\var{yogābhyāsa \lem yogābhyāsaḥ \EdLo}pratikūlānāṃ yaḥ saṅgas tasya parityāgāt. ṣaḍbhir
ebhir yogaḥ prakarṣeṇāvilambena siddhyatīty arthaḥ.

% atha yamaniyamāḥ—

% [ahiṃsā satyam asteyaṃ brahmacaryaṃ kṣamā dhṛtiḥ.
% dayārjavaṃ mitāhāraḥ śaucaṃ caiva yamā daśa.|*||
% tapaḥ santoṣa āstikyaṃ dānam īśvarapūjanam.
% siddhāntavākyaśravaṇaṃ hrīmatī ca tapo hutam.
% niyamā daśa samproktā yogaśāstraviśāradaiḥ.|*||\lem
%\end{vsid}


    
%\begin{vsid}{#hp01_017}
\startsloka
{\bf haṭhasya prathamāṅgatvād āsanaṃ pūrvam ucyate
kuryāt tad āsanaṃ sthairyam ārogyaṃ cāṅgalāghavam (1.17)}
\stopsloka

ādāv āsanakathane saṅgatiṃ sāmānyatas tatphalaṃ cāha haṭhasyeti. haṭhasya

\startsloka
āsanaṃ kumbhakaṃ citraṃ\var{citraṃ \lem om \EdLo} mudrākhyaṃ karaṇaṃ tathā
atha nādānusandhānam\comment{HP 1.56}
\stopsloka

iti vakṣyamāṇāni catvāry aṅgāni. pratyāhārādisamādhyantānāṃ nādānusandhāne 'ntarbhāvaḥ. tanmadhye
āsanasya prathamāṅgatvāt pūrvam āsanam ucyate iti sambandhaḥ.

tad āsanaṃ sthairyaṃ dehasya\var{dehasya \lem om \EdLo} manasaś ca\var{ca \Wthree \lem om \EdLo}
cāñcalyarūparajodharmanāśakatvena sthiratāṃ kuryāt. āsanena rajo hanti iti vākyāt. ārogyaṃ
cittavikṣepakarogābhāvaḥ\var{rogābhāvaḥ \lem rogābhāvaṃ \Msix}. rogasya cittavikṣepakatvam uktaṃ
pātañjalasūtre vyādhistyānasaṃśayapramādālasyāviratibhrāntidarśanālabdhabhūmikatvānavasthitatvāni
cittavikṣepās te 'ntarāyāḥ\comment{YS 1.30} iti. aṅgānāṃ lāghavaṃ laghutvam.
gauravarūpatamodharmanāśakatvam apy etenoktam. cakārāt kṣudvṛddhyādikam api bodhyam.
%\end{vsid}

%\begin{vsid}{#hp01_018}
\startsloka
{\bf vasiṣṭhādyaiś ca munibhir matsyendrādyaiś ca yogibhiḥ\var{yogibhiḥ \lem munibhiḥ \Msix}
aṅgīkṛtāny āsanāni kathyante kānicin mayā (1.18)}
\stopsloka

vasiṣṭhādisaṃmatāsanamadhye śreṣṭhāni mayocyanta ity āha vasiṣṭhādyair iti. vasiṣṭha ādyo yeṣāṃ
yājñavalkyādīnāṃ tair munibhir mananaśīlaiḥ. cakārān mantrādiparaiḥ. matsyendra ādyo yeṣāṃ
jālandharanāthādīnāṃ taiḥ. yogibhiḥ\var{yogibhiḥ \lem munibhiḥ \Msix} haṭhābhyāsibhiḥ. cakārān
mudrādiparaiḥ. aṅgīkṛtāni caturaśīty āsanāni tanmadhye kānicit śreṣṭhāni mayā kathyante. yady apy
ubhayor api mananahaṭhābhyāsau stas tathāpi vasiṣṭhādīnāṃ mananaṃ mukhyaṃ matsyendrādīnāṃ
haṭhābhyāso mukhya iti pṛthag grahaṇam.
%\end{vsid}

%  * 50728.jpg  

%\begin{vsid}{#hp01_019}
\startsloka
{\bf jānūrvor antare samyak kṛtvā pādatale ubhe
ṛjukāyaḥ samāsīnaḥ svastikaṃ tat pracakṣate (1.19)}
\stopsloka

tatra sukaratvāt prathamaṃ svastikāsanam āha jānūrvor iti. jānu ca ūruś ca. atra jānuśabdena
jānusaṃnihito jaṅghāpradeśo grāhyaḥ. jaṅghorvor iti pāṭhas tu sādhīyān. tayor antare madhye ubhe
pādayos tale talapradeśau kṛtvā ṛjukāyaḥ samakāyaḥ yatra samāsīno bhavet tad āsanaṃ svastikaṃ
svastikākhyaṃ pracakṣate vadanti. yogina iti śeṣaḥ. śrīdhareṇoktam\var{śrīdharenoktaṃ up to budhāḥ
  iti \lem om \Msix}

\startsloka
ūrujaṅghāntar ādhāya prapade jānumadhyage
yogino yad avasthānaṃ svastikaṃ tad vidur budhāḥ iti
\stopsloka
%\end{vsid}



%\begin{vsid}{#hp01_020}
\startsloka
{\bf savye dakṣiṇagulphaṃ tu pṛṣṭhapārśve niyojayet
dakṣiṇe 'pi tathā savyaṃ gomukhaṃ gomukhākṛti (1.20)}
\stopsloka

gomukhāsanam āha savya iti. savye vāme pṛṣṭhasya pārśve sampradāyāt kaṭer adhobhāge dakṣiṇaṃ
gulphaṃ nitarāṃ yojayet. dakṣiṇe 'pi pṛṣṭhapārśve tathā savyavat savyaṃ gulphaṃ niyojayet.
gomukhasyākṛtir iva ākṛtir yasya tat tādṛśaṃ gomukhaṃ gomukhasaṃjñakam āsanaṃ bhavet.
%\end{vsid}


%\begin{vsid}{#hp01_021}
\startsloka
{\bf ekaṃ pādaṃ tathaikasmin vinyased ūruṇi sthiram\var{sthiram \lem sthitam \Msix}
itarasmiṃs tathā coruṃ vīrāsanam itīritam (1.21)}
\stopsloka

vīrāsanam āha ekam iti. ekaṃ dakṣiṇaṃ pādam. tathā pādapūraṇe. ekasmin vāmoruṇi\var{vāmoruṇi \lem
  vāme ūruṇi \Msix} sthiraṃ\var{sthiram \lem sthitam \Msix} vinyaset.  itarasmin vāme pāde ūruṃ
dakṣiṇaṃ vinyaset. tad vīrāsanam itīritaṃ kathitam.
%\end{vsid}


%\begin{vsid}{#hp01_022}
\startsloka
{\bf gudaṃ nirudhya gulphābhyāṃ vyutkrameṇa samāhitaḥ
kūrmāsanaṃ bhaved etad iti yogavido viduḥ (1.22)}
\stopsloka

kūrmāsanam āha gudam iti. gulphābhyāṃ gudaṃ nirudhya niyamya\var{niyamya \lem om \EdLo} vyutkrameṇa
yatra samyag āhitaḥ sthito bhavet etat kūrmāsanaṃ bhavet iti yogavido vidur ity anvayaḥ.
%\end{vsid}


%\begin{vsid}{#hp01_023}
\startsloka
{\bf padmāsanaṃ tu saṃsthāpya jānūrvor antare karau
  niveśya bhūmau saṃsthāpya vyomasthaṃ
  \hfill kukkuṭāsanam (1.23)}
\stopsloka

kukkuṭāsanam āha padmāsanaṃ tv iti. padmāsanaṃ tu ūrvor upari uttānacaraṇasthāpanarūpaṃ samyak
sthāpayitvā. jānupadena jānusaṃnihito jaṅghāpradeśaḥ. tac ca ūruś ca jānūr tayor antare madhye
karau niveśya bhūmau saṃsthāpya. karāv ity atrāpi sambadhyate. vyomasthaṃ khasthaṃ padmāsanasadṛśaṃ
yat tat kukkuṭāsanam.
%\end{vsid}

%\begin{vsid}{#hp01_024}
\startsloka
{\bf kukkuṭāsanabandhastho dorbhyāṃ sambadhya kandharām
bhaved kūrmavad uttāna etad uttānakūrmakam (1.24)}
\stopsloka

uttānakūrmāsanam āha kukkuṭāsaneti. kukkuṭāsanasya yo bandhaḥ pūrvaślokoktas tasmin sthitaḥ
dorbhyāṃ bāhubhyāṃ kandharāṃ grīvāṃ sambadhya kūrmavad uttāno yasmin bhaved etad āsanam
uttānakūrmakaṃm nāma.
%\end{vsid}


%\begin{vsid}{#hp01_025}
\startsloka
{\bf pādāṅguṣṭhau tu pāṇibhyāṃ gṛhītvā śravaṇāvadhi
dhanurākarṣaṇaṃ kuryād dhanurāsanam ucyate (1.25)}
\stopsloka

dhanurāsanam āha pādāṅguṣṭhau tv iti. pāṇibhyāṃ pādayor aṅguṣṭhau gṛhītvā śravaṇāvadhi
karṇaparyantaṃ dhanuṣa ākarṣaṇaṃ yathā bhavati tathā kuryāt. gṛhītāṅguṣṭham ekaṃ pāṇiṃ prasāritaṃ
kṛtvā gṛhītāṅguṣṭham itaraṃ pāṇiṃ\var{pāṇiṃ \lem paṇiṃ \Msix} karṇaparyantam ākuñcitaṃ kuryād ity
arthaḥ. etad dhanurāsanam ucyate.
%\end{vsid}


%\begin{vsid}{#hp01_026}
\startsloka
{\bf vāmorumūlārpitadakṣapādaṃ
jānor bahir veṣṭitavāmapādam
pragṛhya tiṣṭhet parivartitāṅgaḥ 
śrīmatysanāthoditam āsanaṃ syāt (1.26)}
\stopsloka

matyendrāsanam āha vāmorumūleti. vāmorumūle 'rpitaḥ sthāpito yo dakṣapādaḥ taṃ sampradāyāt
pṛṣṭhatogatavāmapāṇinā gulphasyoparibhāge pragṛhya. jānoḥ dakṣiṇapādajānoḥ bahiḥ\var{bahiḥ \lem
  bahiḥ bahiḥ \Msix}pradeśe veṣṭito yo vāmapādaḥ taṃ vāmapādajānoḥ bahir veṣṭitadakṣiṇapāṇinā
aṅguṣṭhe pragṛhya. parivartitāṅgaḥ vāmabhāgena pṛṣṭhatomukhaṃ yathā syād evaṃ parivartitaṃ
parāvartitam \var{parāvartitam \Wthree \Tue \EdMu \Msix \lem parivartitam \EdLo} aṅgaṃ yena sa tathā
tādṛśo yatra tiṣṭhet sthitiṃ kuryāt tad āsanaṃ matysendranāthenoditam kathitaṃ syāt. taduditatvāt
tannāmakam eva vadanti. evaṃ dakṣorumūlārpitavāmapādaṃ pṛṣṭhatogatadakṣiṇapāṇinā pragṛhya vāmajānor
bahirveṣṭitadakṣapādaṃ dakṣiṇapādajānoḥ bahirveṣṭitavāmapāṇinā pragṛhya dakṣabhāgena pṛṣṭhatomukhaṃ
yathā syād evaṃ parivartitāṅgaś cābhyaset.
%\end{vsid}


%\begin{vsid}{#hp01_027}
\startsloka
{\bf matsyendrapīṭhaṃ jaṭharapradīptiṃ
pracaṇḍarugmaṇḍalakhaṇḍanāstram
abhyāsataḥ kuṇḍalinīprabodhaṃ
candrasthiratvaṃ ca dadāti puṃsām (1.27)}
\stopsloka

matsyendrāsanasya phalam āha matsyendreti. pracaṇḍaṃ duḥsahaṃ yat rujāṃ rogāṇāṃ maṇḍalaṃ samūhaḥ
tasya khaṇḍane chedane 'stram astram iva tādṛśaṃ matsyendrapīṭhaṃ matsyendrāsanam. abhyāsataḥ
pratyahamāvartanarūpād abhyāsāt puṃsāṃ jaṭharasya jaṭharāgneḥ\var{jaṭharāgneḥ \lem jaṭhārāgneḥ
  \EdLo} prakṛṣṭāṃ dīptiṃ vṛddhim dadāti.  tathā kuṇḍalinyā ādhāraśakteḥ prabodhaṃ nidrābhāvaṃ tathā
candrasya tālunaḥ uparibhāge sthitasya nityaṃ kṣarataḥ sthiratvaṃ kṣaraṇābhāvaṃ ca dadātīty arthaḥ.
%\end{vsid}


%\begin{vsid}{#hp01_028}
\startsloka
{\bf prasārya pādau bhuvi daṇḍarūpau
dorbhyāṃ padagradvitayaṃ gṛhītvā
jānūparinyastalalāṭadeśo
vased idaṃ paścimatānam āhuḥ (1.28)}
\stopsloka

paścimatānāsanam āha prasāryeti. bhuvi bhūmau daṇḍasya rūpam iva rūpaṃ yayos tau daṇḍākārau
śliṣṭagulphau prasārya prasāritaṃ\var{prasāritaṃ \lem prasāritau \Msix} kṛtvā. dorbhyām
ākuñcitatarjanībhyāṃ bhujābhyāṃ padoḥ pādayoś cāgre\var{cāgre \lem agre \Msix (better??)} agrabhāgau
tayor dvitayaṃ dvayam aṅguṣṭhapradeśayugmaṃ balād ākarṣaṇapūrvakaṃ yathā jānvadhobhāgasya bhūmer
utthānaṃ na syāt tathā gṛhītvā. jānor\var{jānor \lem jānunor \Msix} uparinyasto lalāṭadeśo yena
tādṛśo yatra vaset. idaṃ paścimatānanāmakam āsanam āhuḥ.
%\end{vsid}



%\begin{vsid}{#hp01_029}
\startsloka
{\bf iti paścimatānam āsanāgryaṃ 
pavanaṃ paścimavāhinaṃ karoti
udayaṃ jaṭharānalasya kuryād
udare kārśyam arogatāṃ ca puṃsām (1.29)}
\stopsloka

atha tatphalam itīti. iti pūrvoktam āsaneṣv 'py agryaṃ mukhyaṃ paścimatānam pavanaṃ prāṇaṃ
paścimavāhinaṃ paścimena paścimamārgeṇa suṣumnāmārgeṇa vahatīti paścimavāhī taṃ tādṛśaṃ karoti.
jaṭharānalasya jaṭhare yo 'nalo 'gnis tasyodayaṃ vṛddhiṃ kuryāt. udare madhyapradeśe kārśyaṃ
kṛśatvaṃ kuryāt. arogatām ārogyaṃ cakārān nāḍīvalanādisāmyaṃ kuryāt.
%\end{vsid}


%\begin{vsid}{#hp01_030}
\startsloka
{\bf dharām avaṣṭabhya karadvayena
tatkūrparasthāpitanābhipārśvaḥ
uccāsano daṇḍavad utthitaḥ khe\var{khe \lem syāt \Msix}
māyūram etat pravadanti pīṭham (1.30)}
\stopsloka

atha māyūrāsanam āha dharām iti. karadvayena karayor dvayaṃ yugmaṃ tena dharām bhūmim avaṣṭabhya
avalambya prasāritāṅgulī bhūmisaṃlagnatalau saṃnnihitau karau kṛtvety arthaḥ. tasya karadvayasya
kūrparayoḥ bhujamadhyasandhibhāgayoḥ sthāpite dhṛte nābheḥ pārśve pārśvabhāgau yena sa\var{sa \lem
  om \Msix (better!)} uccāsana uccam unnatam āsanaṃ yasyaitādṛśaḥ. khe śūnye daṇḍavad daṇḍena tulyam
utthita ūrdhvaṃ sthito yatra bhavati tan māyūram mayūrasyedaṃ tatsambandhitvāt tannāmakaṃ
pravadanti. yogina iti śeṣaḥ.
%\end{vsid}


%\begin{vsid}{#hp01_031}
\startsloka
{\bf harati sakalarogān āśu gulmodarādīn
abhibhavati ca doṣān āsanaṃ śrīmayūram
bahu kadaśanabhuktaṃ bhasmakuryād aśeṣaṃ
janayati ca jaṭharāgniṃ jārayet kālakūṭam (1.31)}
\stopsloka

mayūrāsanaguṇān āha haratīti. gulmo rogaviśeṣaḥ udaraṃ jalodaraṃ te ādinī yeṣāṃ plīhādīnāṃ te tathā
tān sakalarogān sakalā ye rogās tān āśu jhaṭiti harati nāśayati. śrīmayūram āsanam iti sarvatra
sambadhyate. doṣān vātapittakaphān ālasyādīṃś cābhibhavati tiraskaroti\var{tiraskaroti \lem
  tiraskāroti \Msix}. bahv atiśayitaṃ kadaśanaṃ kadannaṃ yad bhuktaṃ tad aśeṣaṃ\var{tad aśeṣaṃ \var
  om \EdLo} samastaṃ bhasmakuryāt pācayed ity arthaḥ. jaṭharāgniṃ jaṭharānalaṃ janayati
prādurbhāvayati. kālakūṭam viṣaṃ kālakūṭaśabdo 'tra kālakūṭavad apakārakānnaparaḥ taṃ jārayet
jīrṇaṃ kuryāt pācayed ity arthaḥ.
%\end{vsid}


%\begin{vsid}{#hp01_032}
\startsloka
{\bf uttānaṃ śavavad bhūmau śayanaṃ tac chavāsanam
śavāsanaṃ śrāntiharaṃ cittaviśrāntikārakam (1.32)}
\stopsloka

śavāsanam āhārdhena uttānam iti. śavena mṛtaśarīreṇa tulyaṃ śavavad uttānaṃ
bhūmisaṃlagnaṃ\var{saṃlagnaṃ \lem saṃlagna \Msix} pṛṣṭhaṃ yathā syāt tathā śayanaṃ nidrāyām iva
saṃniveśo yat tac chavāsanaṃ. śavākhyam āsanam.  śavāsanaprayojanam āha uttarārdhena. śavāsanaṃ
śrāntiharaṃ śrāntiṃ haṭhābhyāsaśramaṃ haratīti śrāntiharaṃ. cittasya viśrāntir viśrāmas tasyāḥ
kārakam.
%\end{vsid}

%\begin{vsid}{#hp01_033}
\startsloka
{\bf caturaśīty āsanāni śivena kathitāni ca
tebhyaś catuṣkam ādāya sārabhūtaṃ bravīmy aham (1.33)}
\stopsloka

vakṣyamāṇāsanacatuṣṭayasya śreṣṭhatvaṃ vadann āha caturaśītīti. śiveneśvareṇa
caturadhikāśītisaṅkhyākāny āsanāni kathitāni. cakārāc caturaśītilakṣāṇi ca. tad uktaṃ
gorakṣanāthena

\startsloka
āsanāni ca tāvanti yāvantyo\var{yāvantyo \lem yāvatyo \Msix} jīvajātayaḥ
eteṣām akhilān bhedān\var{bhedān \lem bhogān \Msix} vijānāti maheśvaraḥ
caturāśītilakṣāṇi ekaikaṃ samudāhṛtam
tataḥ śivena pīṭhānāṃ ṣoḍaśonaṃ śataṃ kṛtam\comment{1.89} iti.
\stopsloka

tebhyaḥ śivoktacaturaśītilakṣāsanānāṃ madhye praśastāni yāni caturaśīty āsanāni tebhya ādāya
gṛhītvā sārabhūtaṃ śreṣṭhabhūtaṃ catuṣkam ahaṃ bravīmīty anvayaḥ.
%\end{vsid}

%\begin{vsid}{#hp01_035}
\startsloka
{\bf siddhaṃ padmaṃ tathā siṃhaṃ bhadraṃ ceti catuṣṭayam
śreṣṭhaṃ tatrāpi ca sukhe tiṣṭhet siddhāsane sadā (1.34)}
\stopsloka

tad eva catuṣkaṃ nāmnā nirdiśati siddham iti. siddhaṃ siddhāsanaṃ padmaṃ padmāsanaṃ siṃhaṃ
siṃhāsanaṃ bhadraṃ bhadrāsanaṃ ca iti catuṣṭayam śreṣṭham atiśayena praśasyam. tatrāpi catuṣṭaye
sukhe sukhakare siddhāsane sadā tiṣṭhet\comment{{\em siddhāsane sadā tiṣṭhet} echoes HP
  1.39b} etena siddhāsanaṃ catuṣṭaye 'py utkṛṣṭam iti sūcitam.
%\end{vsid}


%\begin{vsid}{#hp01_035}
\startsloka
{\bf yonisthānakam\var{yoni \lem tatra siddhāsaṃ yoni \Msix} aṅghrimūlaghaṭitaṃ kṛtvā dṛḍhaṃ vinyaset
meḍhre pādam athaikam eva hṛdaye kṛtvā hanuṃ susthiram
sthāṇuḥ saṃyamitendriyo 'caladṛśā paśyed bhruvor antaraṃ
hy etan mokṣakapāṭabhedajanakaṃ siddhāsanaṃ procyate
\hfill (1.35)}
\stopsloka

āsanacatuṣṭaye 'py utkṛṣṭatvāt prathamaṃ siddhāsanam āha\var{āha \lem āha tatreti \Msix}
yonisthānakam iti. yonisthānam eva yonisthānakam svārthe kapratyayaḥ. gudopasthayor
madhyapradeśo\var{madhyapradeśo \EdLo \Msix \lem madhyamapradeśe padaṃ \EdMu} yonisthānaṃ
tat. aṅghrir vāmaś caraṇas tasya mūlena pārṣṇibhāgena ghaṭitaṃ saṃlagnaṃ kṛtvā. athānantaram ekam
pādaṃ dakṣiṇaṃ pādaṃ meḍhre indriyasyoparibhāge\var{meḍhre indriyasyoparibhāge \Wthree \Tue \Msix
  \lem meḍhrendriyasyoparibhāge Ed Mu \EdLo (no variants reported)} dṛḍhaṃ yathā syāt tathā
vinyaset. hṛdaye hṛdayasamīpe hanuṃ cibukaṃ susthiram samyaksthiraṃ kṛtvā hanuhṛdayayoś
caturaṅgulam antaraṃ yathā bhavati tathā kṛtveti rahasyam. saṃyamitāni viṣayebhyaḥ parāvṛttāni
indriyāṇi yena sa tathā. acalā yā dṛk dṛṣṭis tayā bhruvor antaraṃ madhyaṃ paśyet. hi prasiddhaṃ
mokṣasya yat kapāṭaṃ pratibandhakaṃ tasya bhedaṃ nāśaṃ janayatīti tādṛśam. siddhānāṃ yoginām. āste
'tra āsyate 'neneti vā āsanaṃ siddhāsananāmakam idaṃ bhaved ity arthaḥ.
%\end{vsid}

%\begin{vsid}{#hp01_036}
\startsloka
{\bf matāntare tu

meḍhrād upari vinyasya savyaṃ gulphaṃ tathopari
gulphāntaraṃ ca nikṣipya siddhāsanam idaṃ bhavet (1.36)}
\stopsloka

matsyendrasammataṃ siddhāsanam uktvānyasaṃmataṃ vaktum āha matāntare tv iti. tad eva darśayati
meḍhrād iti. meḍhrād upasthād upary ūrdhvabhāge savyaṃ\var{savyaṃ \EdLo \Msix \lem savye \Tue}
vāmagulphaṃ\var{vāmagulphaṃ \Wthree \Tue \EdMu \lem vāmaṃ gulphaṃ \EdLo \Msix} vinyasya tathā savyavad upari
savyapādasyopari na tu savyagulphasya. gulphāntaraṃ dakṣiṇagulphaṃ ca nikṣipya vased iti
śeṣaḥ. idaṃ siddhāsanam matāntarābhimataṃ bhaved ity arthaḥ.
%\end{vsid}


%\begin{vsid}{#hp01_037}
\startsloka
{\bf pūrvoktam eva matsyendramatam

etat siddhāsanaṃ prāhur anye vajrāsanaṃ viduḥ
muktāsanaṃ vadanty eke prāhur guptāsanaṃ pare (1.37)}
\stopsloka

tatra prathamaṃ mahāsiddhasaṃmatam ity āha pūrvoktam iti. asyaiva matabhedān nāmabhedān āha etad
iti. etat pūrvoktaṃ siddhāsanaṃ siddhāsananāmakaṃ prāhuḥ. kecid ity adhyāhāraḥ. anye vajrāsanaṃ
vajrāsanasaṃjñakaṃ viduḥ jānanti. eke muktāsanaṃ muktāsanābhidhaṃ vadanti. pare guptāsanaṃ
guptāsanākhyaṃ prāhuḥ. atrāsanābhijñāḥ. yatra vāmapādapārṣṇiṃ yonisthāne\var{yonisthāne \lem
  mūlasthāne \Msix} niyojya dakṣiṇapādapārṣṇir meṇḍhrād upari sthāpyate tat siddhāsanam. yatra
dakṣiṇapādapārṣṇiṃ\var{dakṣiṇapādapārṣṇiṃ \lem vāmapādapārṣṇiṃ \Msix} yonisthāne niyojya
vāmapādapārṣṇir\var{vāmapādapārṣṇir \lem dakṣiṇapādapārṣṇir \Msix} meḍhrād upari sthāpyate tad
vajrāsanam\var{vajrāsanam \Wthree \EdMu \Msix \lem siddhāsanam \Tue \EdLo}.  yatra tu
dakṣiṇasavyapārṣṇidvayam uparyadhobhāgena saṃyojya yonisthānena\var{yonisthānena \lem yonisthāne
  \Msix} saṃyojyate tan muktāsanam. yatra ca pūrvavat saṃyuktaṃ pārṣṇidvayaṃ meḍhrād upari nidhīyate
tad guptāsanam iti.
%\end{vsid}



%\begin{vsid}{#hp01_038}
\startsloka
{\bf yameṣv iva mitāhāram ahiṃsāṃ niyameṣv iva
mukhyaṃ sarvāsaneṣv ekaṃ siddhāḥ siddhāsanaṃ viduḥ (1.38)}
\stopsloka

atha saptabhiḥ ślokaiḥ siddhāsanaṃ praśaṃsati yameṣvityādibhiḥ. yameṣu mitāhāram iva. mitāhāro
vakṣyamāṇaḥ susnigdhamadhurāhāra\comment{HP 1.58} ityādinā. niyameṣu ahiṃsām iva. sarvāṇi yāny
āsanāni teṣu siddhāḥ ekaṃ siddhāsanaṃ mukhyaṃ vidur iti sambandhaḥ.
%\end{vsid}


%\begin{vsid}{#hp01_039}
\startsloka
{\bf caturaśītipīṭheṣu siddham eva sadābhyaset
dvāsaptatisahasrāṇāṃ nāḍīnāṃ malaśodhanam (1.39)}
\stopsloka

caturaśītīti. caturadhikāśītisaṅkhyākāni yāni pīṭhāni teṣu siddham eva siddhāsanam eva sadā
sarvadābhyaset. siddhāsanasya sadābhyāse hetugarbhaviśeṣaṇaṃ. dvāsaptatisahasrāṇāṃ nāḍīnāṃ
malaśodhanam malasya śodhanaṃ\var{malasya śodhanaṃ \lem om \Msix} śodhakaṃ yataḥ.
%\end{vsid}


%\begin{vsid}{#hp01_040}
\startsloka
{\bf ātmadhyāyī mitāhārī yāvaddvādaśavatsaram
sadā siddhāsanābhyāsād yogī niṣpattim āpnuyāt (1.40)}
\stopsloka

ātmadhyāyīti. ātmānaṃ dhyāyatīty ātmadhyāyī. mita āhāro\var{mita āhāro \lem mitāhāro \Wthree \Msix}
'syāstīti mitāhārī. yāvanto dvādaśa vatsarāḥ\var{yāvanto dvādaśa vatsarāḥ \lem om \EdLo}
yāvaddvādaśavatsaram. yāvad avadhāraṇa\comment{Aṣṭādhyāyī 2.1.8.} ity avyayībhāvaḥ samāsaḥ.
dvādaśavatsaraparyantam ity arthaḥ. sadā sarvadā siddhāsanasyābhyāsād yogī yogābhyāsī niṣpattim
yogasiddhim āpnuyāt prāpnuyāt. yogāṅgāntarābhyāsam antareṇa siddhāsanābhyāsamātreṇa siddhiṃ
prāpnuyād ity arthaḥ.
%\end{vsid}



%\begin{vsid}{#hp01_041}
\startsloka
{\bf kim anyair bahubhiḥ pīṭhaiḥ siddhe siddhāsane sati
prāṇānile sāvadhāne baddhe kevalakumbhake
utpadyate nirāyāsāt svayam evonmanī kalā (1.41)}
\stopsloka

kim anyair iti. siddhāsane siddhe saty anyair bahubhiḥ pīṭhair āsanaiḥ kim. na kim apīty arthaḥ.
sāvadhāne prāṇānile prāṇavāyau kevalakumbhake baddhe sati unmanī unmanyavasthā. sā
kalevāhlādakatvāc candralekheva nirāyāsād anāyāsāt svayam evotpadyata udeti.
%\end{vsid}

%\begin{vsid}{#hp01_042}
\startsloka
{\bf tathaikāsminn eva dṛḍhe baddhe siddhāsane sati
bandhatrayam anāyāsāt svayam evopajāyate (1.42)}
\stopsloka

tatheti. tathoktaprakāreṇaikasminn eva siddhāsane dṛḍhe baddhe sati bandhatrayam
mūlabandhoḍḍīyānabandhajālandharabandharūpam anāyāsāt

\startsloka
pārṣṇibhāgena sampīḍya yonim ākuñcayed gudam\comment{HP 3.6}
\stopsloka

ityādivakṣyamāṇaprakāreṇa\var{prakāreṇa \Wthree \Msix \EdLo \lem om \Tue \EdMu} mūlabandhādiṣu ya
āyāsas taṃ vinaiva svayam evopajāyate svata evotpadyata ity arthaḥ.
%\end{vsid}


%\begin{vsid}{#hp01_043}
\startsloka
{\bf nāsanaṃ siddhasadṛśaṃ na kumbhaḥ kevalopamaḥ
na khecarīsamā mudrā na nādasadṛśo layaḥ (1.43)}
\stopsloka

nāsanam iti. siddhena siddhāsanena sadṛśam āsanaṃ na astīti śeṣaḥ. kevalena kevalakumbhakena
upamīyata iti kevalopamaḥ kumbhaḥ kumbhako nāsti. khecarīmudrāsamā mudrā nāsti. nādasadṛśo\var{nāda
\lem oṃnāda \Wthree} layo layahetur nāsti.
%\end{vsid}


%\begin{vsid}{#hp01_044}
\startsloka
{\bf atha padmāsanam

vāmorūpari dakṣiṇaṃ ca caraṇaṃ saṃsthāpya vāmaṃ tathā
dakṣorūpari paścimena vidhinā dhṛtvā karābhyāṃ dṛḍham
aṅguṣṭhau hṛdaye nidhāya cibukaṃ nāsāgram ālokayed
etad vyādhivināśakāri yamināṃ padmāsanaṃ procyate (1.44)}
\stopsloka

padmāsanaṃ vaktum upakramate atheti. padmāsanam āha vāmorūparīti. vāmo ya ūrus tasyopari dakṣiṇaṃm
caraṇaṃ. cakāraḥ pādapūraṇe. saṃsthāpya samyag uttānaṃ sthāpayitvā vāmaṃ savyaṃ caraṇaṃ tathā
dakṣiṇacaraṇavad dakṣo dakṣiṇo ya ūrus tasyopari saṃsthāpya paścimena bhāgena\var{paścimena bhāgena
  \lem paścimena paścimabhāgena \Msix} pṛṣṭhabhāgeneti.  vidhir vidhānaṃ karayor ity arthāt. tena
karābhyāṃ hastābhyāṃ dṛḍham yathā syāt tathā aṅguṣṭhau pādāṅguṣṭhau dhṛtvā gṛhītvā. dakṣiṇaṃ karaṃ
pṛṣṭhataḥ kṛtvā vāmorusthitadakṣiṇacaraṇāṅguṣṭhaṃ gṛhītvā vāmakaraṃ pṛṣṭhataḥ kṛtvā
dakṣiṇorusthitavāmacaraṇāṅguṣṭhaṃ gṛhītvety arthaḥ. hṛdaye hṛdayasamīpe. sāmīpikādhāre
saptamī. cibukaṃ hanuṃ nidhāya urasaś caturāṅgulāntare cibukaṃ nidhāyeti rahasyam. nāsāgram
nāsikāgram ālokayet paśyet. yatraitad yamināṃ yogināṃ\var{yogināṃ \lem om \Msix} vyādher vināśaṃ
karotīti vyādhivināśakāri padmāsanam etannāmakaṃ procyate siddhair iti śeṣaḥ.
%\end{vsid}


%\begin{vsid}{#hp01_045 #hp01_046 #hp01_047}
\startsloka
{\bf  matāntare
uttānau caraṇau kṛtvā ūrusaṃsthau prayatnataḥ
ūrumadhye tathottānau pāṇī kṛtvā tato dṛśau (1.45)
nāsāgre vinyased rājadantamūle tu jihvayā
uttambhya cibukaṃ vakṣasy utthāpya pavanaṃ śanaiḥ (1.46)
idaṃ padmāsanaṃ proktaṃ sarvavyādhivināśanam
durlabhaṃ yena kenāpi dhīmatā labhyate bhuvi (1.47)}
\stopsloka

matsyendranāthābhimataṃ padmāsanam āha uttānāv iti. uttānau caraṇau ūrusaṃlagnapṛṣṭhabhāgau caraṇau
pādau prayatnataḥ prakṛṣṭād yatnād ūrusaṃsthāv ūrvoḥ samyak tiṣṭhata\var{tiṣṭhata \lem tiṣṭheta
  \Msix} ity ūrusaṃsthau tādṛśau kṛtvā. ūrvor madhye ūrumadhye. tathā cārthe. pāṇī karāv uttānau
kṛtvā.  ūrusaṃsthottānapādobhayapārṣṇisaṃlagnapṛṣṭhaṃ savyaṃ pāṇim uttānaṃ kṛtvā tadupari dakṣiṇaṃ
pāṇiṃ cottānaṃ kṛtvety arthaḥ. tatas tadanantaraṃ. dṛśau dṛṣṭī. nāsāgra iti nāsāgre nāsikāgre
vinyased viśeṣeṇa niścalatayā nyased ity arthaḥ. rājadantānāṃ daṃṣṭrāṇāṃ savyadakṣiṇabhāge
sthitānāṃ mūle ubhe mūlasthāne jihvayā uttambhya ūrdhvaṃ stambhayitvā. gurumukhād avagantavyo 'yaṃ
jihvābandhaḥ. cibukaṃ vakṣasi nidhāyeti śeṣaḥ. śanair mandaṃ mandaṃ pavanaṃ vāyum utthāpya. anena
mūlabandhaḥ proktaḥ.  mūlabandho 'pi gurumukhād evāvagantavyaḥ. vastutas tu jihvābandhenaivāyaṃ
caritārtha iti haṭharahasyavidaḥ. idam iti\var{idam iti \lem om \Msix}. evaṃ yatrāsyate tad idaṃ
padmāsanaṃ padmāsanābhidhānaṃ proktam. āsanajñair iti śeṣaḥ. kīdṛśam sarveṣāṃ vyādhīnāṃ viśeṣeṇa
nāśanaṃ yena kenāpi bhāgyahīnena durlabham. dhīmatā bhuvi bhūmau labhyate prāpyate.
%\end{vsid}



%\begin{vsid}{#hp01_048}
\startsloka
paścād uktam eva matsyendramatam
{\bf kṛtvā sampuṭitau karau dṛḍhataraṃ baddhvā tu padmāsanaṃ
gāḍhaṃ vakṣasi sannidhāya cibukaṃ dhyāyaṃś ca tac cetasi
vāraṃ vāram apānam ūrdhvam anilaṃ protsārayan pūritaṃ
nyañcan\var{nyañcan \lem muñcan \Msix} prāṇam upaiti bodham atulaṃ śaktiprabhāvān naraḥ (1.48)}
\stopsloka

etac ca mahāyogisaṃmatam ity āha paścād iti. anyad api padmāsane kṛtyaviśeṣam āha kṛtveti.
saṃpuṭitau saṃpuṭīkṛtau karau utsaṅgasthāv iti śeṣaḥ. dṛḍhataram atiśayena dṛḍhaṃ susthiraṃ
padmāsanaṃ baddhvā kṛtvety arthaḥ. cibukaṃ hanuṃ gāḍhaṃ dṛḍhaṃ yathā syāt tathā vakṣasi
vakṣaḥsamīpe saṃnnidhāya saṃnnihitaṃ kṛtvā caturaṅgulāntareṇeti yogisampradāyāj jñeyam.
jālandharabandhaṃ kṛtvety arthaḥ. tat svasyeṣṭa\var{svasyeṣṭa \lem svasva \Msix (better)}devatārūpaṃ
brahma vā. oṃ tat sad iti nirdeśo brahmaṇas trividhaḥ smṛtaḥ\comment{Bhagavadgītā 17.23.} iti
bhagavadukteḥ. cetasi citte dhyāyan cintayan. apānam anilam apānavāyum ūrdhvaṃ protsārayan
mūlabandhaṃ kṛtvā suṣumnāmārgeṇa prāṇam\var{prāṇam \lem om \Msix} ūrdhvaṃ nayan pūritaṃ pūrakeṇa
antar dhāritaṃ prāṇaṃ nyañcan nīcair adhaḥ añcan gamayan.  antarbhāvitaṇyartho
'ñcatiḥ. prāṇāpānayor aikyaṃ kṛtvety arthaḥ. naraḥ pumān atulaṃ bodhaṃ nirupamajñānaṃ
śaktiprabhāvāc chaktir ādhāraśaktiḥ kuṇḍalinī tasyāḥ prabhāvāt sāmarthyād upaiti
prāpnoti. prāṇāpānayor aikye kuṇḍalinībodho bhavati. kuṇḍalinībodhe suṣumnāmārgeṇa prāṇo
brahmarandhraṃ gacchati. tatra gate cittasthairyaṃ bhavati. cittasthairye 'saṃśayād\var{'saṃśayād
  \lem saṃyamād \Msix} ātmasākṣātkāro bhavatīty arthaḥ. % (yo.ta. 1.52, yo.cū.u. 40)
%\end{vsid}



%\begin{vsid}{#hp01_049}
\startsloka
{\bf padmāsane sthito yogī nāḍīdvāreṇa pūritam
mārutaṃ dhārayed yas tu sa mukto nātra saṃśayaḥ (1.49)}
\stopsloka

padmāsana iti. padmāsane sthito yo yogī yogābhyāsī pūritaṃ pūrakeṇāntarnītaṃ mārutaṃ vāyuṃ
suṣumnāmārgeṇa mūrdhānaṃ nītveti śeṣaḥ. yaḥ dhārayet sthirīkuryāt sa muktaḥ. atra saṃśayo
nāstīty anvayaḥ.
%\end{vsid}

%\begin{vsid}{#hp01_050}
\startsloka
{\bf atha siṃhāsanam
gulphau ca vṛṣaṇasyādhaḥ sīvanyāḥ pārśvayoḥ kṣipet
dakṣiṇe savyagulphaṃ tu dakṣagulphaṃ tu savyake (1.50)}
\stopsloka

siṃhāsanam āha gulphau ceti. vṛṣaṇasyādhaḥ adhobhāge sīvanyāḥ pārśvayoḥ sīvanyāḥ ubhayabhāgayoḥ
kṣipet prerayet sthāpayed iti yāvat. gulphasthāpanaprakāram evāha dakṣiṇa iti. sīvanyāḥ dakṣiṇe
bhāge savyagulphaṃ sthāpayet. savyake sīvanyāḥ savyabhāge dakṣiṇagulphaṃ sthāpayet.
%\end{vsid}



%\begin{vsid}{#hp01_051}
\startsloka
{\bf hastau tu jānvoḥ saṃsthāpya svāṅgulīḥ samprasārya ca
vyāttavaktro nirīkṣeta nāsāgraṃ susamāhitaḥ (1.51)}   %  read nāsāgre !!! ?????
\stopsloka

hastāv iti. jānvoḥ upari hastau tu saṃsthāpya samyag jānusaṃlagnatalau yathā syātāṃ tathā
sthāpayitvā. svāṅgulīḥ hastāṅgulīḥ samprasārya samyak prasārayitvā\var{prasārayitvā \Msix \Wthree
  \EdMu \lem prasārya \EdLo} vyāttavaktraḥ samprasāritalalajjihvamukhaḥ susamāhitaḥ ekāgracittaḥ
nāsāgraṃ nāsikāgraṃ yasmin nirīkṣeta.   % yasmin means locative, i.e. nāsāgre??? 
%\end{vsid}



%\begin{vsid}{#hp01_052}
\startsloka
{\bf siṃhāsanaṃ bhaved etat pūjitaṃ yogipuṅgavaiḥ
bandhatritayasandhānaṃ kurute cāsanottamam (1.52)}
\stopsloka

etat siṃhāsanaṃ bhavet. kīdṛśam yogipuṅgavaiḥ yogiśreṣṭhaiḥ pūjitaṃ prastutam āsaneṣūttamaṃ
siṃhāsanaṃ bandhānāṃ mūlabandhādīnāṃ tritayaṃ trayaṃ tasya sandhānaṃ tasya sannidhānaṃ kurute.
%\end{vsid}

%\begin{vsid}{#hp01_053}
\startsloka
{\bf atha bhadrāsanam

gulphau ca vṛṣaṇasyādhaḥ sīvanyāḥ pārśvayoḥ kṣipet
savyagulphaṃ tathā savye dakṣagulphaṃ tu dakṣiṇe (1.53)}
\stopsloka

bhadrāsanam āha gulphāv iti. vṛṣaṇasyādhaḥ sīvanyāḥ pārśvayoḥ sīvanyā ubhayataḥ gulphau
pādagranthī kṣipet. kṣepaṇaprakāram evāha savyagulpham iti. savye sīvanyāḥ pārśve savyagulphaṃ
kṣipet. tathā pādapūraṇe. dakṣagulphaṃ tu dakṣiṇe sīvanyāḥ pārśve kṣipet (53)
%\end{vsid}



%\begin{vsid}{#hp01_054}
\startsloka
{\bf pārśvapādau ca pāṇibhyāṃ dṛḍhaṃ baddhvā suniścalam
bhadrāsanaṃ bhaved etat sarvavyādhivināśanam
gorakṣāsanam ity āhur idaṃ vai siddhayoginaḥ (1.54)}
\stopsloka

pārśvapādau ca pārśvasamīpagatau pādau pāṇibhyāṃ bhujābhyāṃ dṛḍhaṃ baddhvā
parasparasaṃlagnāṅgulibhyām udarasaṃlagnatalābhyāṃ pāṇibhyāṃ baddhvety arthaḥ. etad bhadrāsanaṃ
bhavet. kīdṛśam sarveṣāṃ vyādhīnāṃ viśeṣeṇa nāśanam.\var{siddhāś \Msix \lem siddhāś gorakṣeti
  \EdMu} siddhāś ca te yoginaś ca siddhayoginaḥ. idaṃ bhadrāsanaṃ gorakṣāsanam ity āhuḥ.
gorakṣeṇa prāyaśo 'bhyastatvād gorakṣāsanam iti vadanti.
%\end{vsid}


%\begin{vsid}{#hp01_055}
\startsloka
{\bf 
evam āsanabandheṣu yogīndro vigataśramaḥ
abhyasen nāḍikāśuddhiṃ mudrādipavanakriyām (1.55)}
\stopsloka

āsanāny uktvā\var{uktvā \Msix \Wthree \lem uktāni \EdMu \EdLo}. teṣu yat kartavyaṃ tad āha evam
iti. evam ukteṣv āsanānāṃ bandheṣu bandhanaprakāreṣu vigataḥ śramo yasya sa vigataśramaḥ āsanānāṃ
bandheṣu śramarahitaḥ. yoginām indro yogīndraḥ. nāḍikānāṃ nāḍīnāṃ śuddhim. prāṇaṃ ced iḍayā piben
niyamitam\comment{HP 2.10} iti vakṣyamāṇarūpā mudrā ādir yasyāḥ sūryabhedādes tādṛśīm pavanasya
prāṇavāyoḥ kriyāṃ prāṇāyāmarūpāṃ cābhyaset.
%\end{vsid}



%\begin{vsid}{#hp01_056}
\startsloka
{\bf 
āsanaṃ kumbhakaṃ citraṃ mudrākhyaṃ karaṇaṃ tathā
atha nādānusandhānam abhyāsānukramo haṭhe (1.56)}
\stopsloka

atha haṭhābhyasanakramam āha āsanam iti. āsanam uktalakṣaṇaṃ citraṃ nānāvidhaṃ kumbhakaṃ
sūryabhedanam ujjāyī\comment{HP 2.44} ityādivakṣyamāṇam. mudrā ity akhyā yasya tat mudrākhyam
mahāmudrādirūpaṃ\var{rūpaṃ \Msix \lem rūpa \EdLo} karaṇaṃ haṭhasiddhau prakṛṣṭopakārakam. tathā
cārthe. athaitattrayānuṣṭhānānantaraṃ nādasya anāhatadhvaner anusandhānam anucintanaṃ haṭhe
haṭhayoge abhyāso abhyasanaṃ tasya anukramaḥ paurvāparyakramo 'yam.
%\end{vsid}

%\begin{vsid}{#hp01_057}
\startsloka
{\bf 
brahmacārī mitāhārī tyāgī yogaparāyaṇaḥ
abdād ūrdhvaṃ bhavet siddho nātra kāryā vicāraṇā (1.57)}
\stopsloka

haṭhasiddher avadhim āha brahmacāryīti. brahmacārī brahmacaryavān mitāhāro vakṣyamāṇaḥ so 'syāstīti
mitāhārī tyāgī dānaśīlo viṣayaparityāgī vā yogaparāyaṇaḥ yogāṅgābhyasanaparaḥ. abdād varṣād ūrdhvaṃ
siddhaḥ siddhahaṭho bhavet. atra ukte 'rthe\var{atra ukte 'rthe \Wthree \lem atra uktarthe \Msix}
vicāraṇā syān na veti saṃśayaprayuktā na kāryā. etan niścitam evety arthaḥ.
%\end{vsid}



%\begin{vsid}{#hp01_058}
\startsloka
{\bf 
susnigdhamadhurāhāraś caturthāṃśavivarjitaḥ
bhujyate śivasamprītyai mitāhāraḥ sa ucyate (1.58)}
\stopsloka

pūrvaśloke mitāhārīty uktam tatra yogināṃ kīdṛśo mitāhāra ity apekṣāyām āha susnigdheti.  susnigdho
'tisnigdhaḥ sa cāsau madhuraś ca tādṛśa āhāraḥ caturthāṃśavivarjitaḥ caturthabhāgarahitaḥ.  tad
uktam abhiyuktaiḥ


\startsloka
dvau bhāgau pūrayed annais toyenaikaṃ prapūrayet
vāyoḥ sañcaraṇārthāya caturtham avaśeṣayet iti
\stopsloka

śivo jīva īśvaro vā. bhoktā devo maheśvaraḥ iti vacanāt. tasya samprītyai samyakprītyarthaṃ yo
bhujyate sa mitāhāra ity ucyate.
%\end{vsid}



%\begin{vsid}{#hp01_059}
\startsloka
{\bf 
kaṭvāmlatīkṣṇalavaṇoṣṇaharītaśāka-
sauvīratailatilasarṣapamadyamatsyān
ājādimāṃsadadhitakrakulatthakola-
piṇyākahiṅgulaśunādyam apathyam āhuḥ (1.59)}
\stopsloka

atha yoginām apathyam āha dvābhyāṃ kaṭv iti. kaṭu kāravellādi. amlaṃ ciñcāphalādi. tīkṣṇaṃ
marīcādi. lavaṇaṃ prasiddham. uṣṇaṃ guḍādi. harītaśākaṃ patraśākaṃ. sauvīraṃ kāñjikaṃ. tailaṃ
tilasarṣapādisnehaḥ. tilāḥ prasiddhāḥ. sarṣapāḥ siddhārthāḥ. madyaṃ surā. matsyo jhaṣaḥ. eṣām
itaretaradvandvaḥ. etān apathyān āhuḥ. ajasyedam ājaṃ ājaṃ ādir yasya saukarādeḥ tad ājādi tac ca
tanmāṃsaṃ cājādimāṃsam. dadhi dugdhapariṇāmaviśeṣaḥ. takraṃ gṛhītasāraṃ
dadhi. kulatthāḥ\var{kulatthāḥ ? \lem kulitthāḥ \Msix} dvidalaviśeṣāḥ. kolaṃ kolyāḥ phalaṃ
badaram. karkandhūr badarī koliḥ ity\var{koliḥ ity ? \lem kolīty \Msix}
amaraḥ. piṇyākaṃ\var{piṇyākaṃ ? \lem pinyākaṃ \Msix} tilapiṇḍam. hiṅgu rāmaṭhaṃ laśunam. eṣām
itaretaradvandvaḥ. etāny ādyāni yasya tat tathā. ādyaśabdena palāṇḍugṛñjanamādakadravyamāṣānnādikaṃ
grāhyam. apathyam ahitam. yoginām\var{yoginām \lem yiginām \Msix} iti śeṣaḥ. āhur yogina ity
adhyāhāraḥ.
%\end{vsid}



%\begin{vsid}{#hp01_060}
\startsloka
{\bf 
bhojanam ahitaṃ vidyāt punar apy uṣṇīkṛtaṃ rūkṣam
atilavaṇam amlayuktaṃ kadaśanaśākotkaṭaṃ varjyam (1.60)}
\stopsloka

bhojanam iti. pūrvaṃ pācitaṃ paścād agnisaṃyogenoṣṇīkṛtaṃ yad bhojanaṃ sūpaudanaroṭikādi rūkṣaṃ
ghṛtādihīnam. atiśayitaṃ lavaṇaṃ yasmin tad atilavaṇam. yad vā lavaṇam atikrāntam atilavaṇaṃ
cākūvā iti loke prasiddhaṃ śākaṃ yavakṣārādikaṃ ca. lavaṇasya sarvathā varjanīyatvād uttarapakṣaḥ
sādhuḥ. tathā ca dattātreyaḥ

\startsloka
atha varjyāni vakṣyāmi yogavighnakarāṇi ca
lavaṇaṃ sarṣapaṃ cāmlam uṣṇaṃ tīkṣṇaṃ ca rūkṣakam
atīva bhojanaṃ tyājyam atinidrātibhāṣaṇam  iti.
\stopsloka

skandapurāṇe 'pi

\startsloka
tyajet kaṭvamlalavaṇaṃ kṣīrabhojī sadā bhavet iti
\stopsloka

amlayuktam amladravyeṇa yuktam. amladravyeṇa yuktam api tyājyaṃ kim uta sākṣād
amlam.% (āmla? All sources checked have amla)
atra tṛtīyapādaṃ palalaṃ vā tilapiṇḍam iti kecit paṭhanti. tasyāyam arthaḥ palalaṃ māṃsaṃ
tilapiṇḍaṃ piṇyākaṃ\var{piṇyākaṃ \lem pinyākaṃ \Msix} kadaśanaṃ kadannaṃ yāvanālakodravādi śākaṃ
vihitetaraśākamātram. utkaṭaṃ vidāhi miracī iti loke prasiddham miracā iti
hindusthānabhāṣāyām. kadaśanādīnāṃ samāhāradvandvaḥ.  atilavaṇādikaṃ varjyaṃ varjanārham. duṣṭam
iti pāṭhe duṣṭaṃ pūtiparyuṣitādi. ahitam iti yojanīyam.
%\end{vsid}


%\begin{vsid}{#hp01_061}
\startsloka
{\bf 
vahnistrīpathisevānām ādau varjanam ācaret 
tathā hi gorakṣavacanam
varjayed durjanaprāntaṃ vahnistrīpathisevanam
prātaḥsnānopavāsādi kāyakleśavidhiṃ tathā (1.61)}
\stopsloka

evaṃ yogināṃ sadā varjyāny uktvābhyāsakāle varjyāny āha ardhena vahnīti. vahniś ca strī ca panthāś
ca teṣāṃ sevā vahnisevana\var{vahnisevana \EdLo \EdMu vahniseka \Msix \Wthree
  Tue}strīsaṅgatīrthayātrādau\var{yātrādau \Msix Wai \lem yātrādi \Tue} pathigamanādirūpās tāsāṃ
varjanam ādāv abhyāsakāle ācaret. siddhe 'bhyāse tu kadācit śīte vahnisevanaṃ gṛhasthasya ṛtau
svabhāryāgamanaṃ tīrthayātrādau mārgagamanaṃ ca na niṣiddham ity ādipadena sūcyate.

tatra pramāṇaṃ gorakṣavacanam avatārayati tathā hīti. tat paṭhati varjayed iti. durjanaprāntaṃ
durjanasamīpavāsam. durjanaprītim iti kvacit pāṭhaḥ. vahnistrīpathāṃ\var{pathāṃ \Msix \Wthree \Tueac
  \lem pathi \Tuepc}sevanaṃ vyākhyātam.  prātaḥsnānam\lem{prātaḥsnānam \Tue \lem prātaḥsnānaṃ ca
  \Wthree \Msix} upavāsaś cādir yasya phalāhārādeḥ tac ca tayoḥ samāhāradvandvaḥ. prathamābhyāsinaḥ
prātaḥsnāne śītavikārotpatteḥ\var{vikārotpatteḥ \Tue \Wthree \lem kārotpatteḥ \Msix}. upavāsādinā
pittādyutpatteḥ. kāyakleśavidhiṃ kāyakleśakaraṃ vidhiṃ kriyāṃ bahusūryanamaskārādirūpāṃ
bahubhārodvahanādirūpāṃ ca. tathā samuccaye. atra pratipadaṃ varjayed iti kriyāsambandhaḥ.
%\end{vsid}



%\begin{vsid}{#hp01_062}
\startsloka
{\bf godhūmaśāliyavaṣāṣṭikaśobhanānnaṃ\var{ṣāṣṭika \Wthree \Tue \Msix \EdMu \lem ṣaṣṭika \EdLo}
kṣīrājyakhaṇḍanavanītasitāmadhūni
śuṇṭhīpaṭolakaphalādikapañcaśākaṃ
mudgādi divyam udakaṃ ca yamīndrapathyam (1.62)}
\stopsloka


atha yogipathyam āha godhūmetyādinā. godhūmāś ca śālayaś ca yavāś ca ṣāṣṭikāḥ ṣaṣṭhyā dinair ye
pacyante taṇḍulaviśeṣās \var{taṇḍulaviśeṣās \lem taṇḍulaviśeṣās te \EdLo (against all mss.)}
śobhanam annaṃ\var{śobhanam annaṃ \lem śobhanānnaṃ \Msix} pavitrānnaṃ śyāmākanīvārādi tac caiteṣāṃ
samāhāradvandvaḥ. kṣīraṃ dugdham. ājyaṃ ghṛtaṃ. khaṇḍaḥ śarkarā. navanītaṃ mathitadadhisāraḥ. sitā
tīvrapallī\var{tīvrapallī \Msix \EdLo \lem tīvrapadī \EdMu tīvraparṇī \Wthree}
khaḍīsākara\var{khaḍīsākara \Msix \EdLo \lem khaṇḍaśarkareti \EdMu khaḍīśākara \Wthree khaṇḍaśākara
  \Tuepc khaṇḍīśākara \Tueac} iti loke prasiddhā misarī iti
hindusthānabhāṣāyām\var{hindusthānabhāṣāyām \lem hindusthānabhāṣāyām prasiddham \Wthree}. madhu
kṣaudram eṣām itaretaradvandvaḥ. śuṇṭhī prasiddhā paṭolakaphalaṃ paravara iti bhāṣāyāṃ prasiddhaṃ
śākaṃ.  tadādir yasya kośātakyādes\var{kośātakyādes \lem kauśāta\-kyādes \EdLo}
tatpaṭolakaphalādikaṃ śeṣād vibhāṣā\comment{Aṣṭhādhyāyī 5.4.154} iti kapratyayaḥ. pañcānāṃ śākānāṃ
samāhāraḥ pañcaśākam. tad uktaṃ vaidyake

\startsloka
sarvaśākam acākṣuṣyaṃ cākṣuṣyaṃ śākapañcakam
jīvantīvāstumatsyākṣīmeghanādapunarnavāḥ iti.
\stopsloka

mudgā\var{mudgā \lem om \EdLo} dvidalaviśeṣā ādir yasya tan mudgādi. ādipadena āḍhakī grāhyā. divyaṃ
nirdoṣam udakaṃ jalaṃ ca. yama eṣām astīti yaminaḥ teṣv indro iva śreṣṭho\var{iva śreṣṭho \Wthree \Tue
  \lem devaśreṣṭho \EdMu \EdLo} yo yogīndras tasya pathyaṃ hitam.
%\end{vsid}


%\begin{vsid}{#hp01_063}
\startsloka
{\bf puṣṭaṃ sumadhuraṃ snigdhaṃ gavyaṃ dhātuprapoṣaṇam
mano'bhilaṣitaṃ yogyaṃ yogī bhojanam ācaret (1.63)}
\stopsloka

atha yogino bhojananiyamam āha puṣṭam iti. puṣṭaṃ dehapuṣṭikaram odanādi sumadhuraṃ
śarkarādisahitaṃ snigdhaṃ saghṛtaṃ gavyaṃ godugdhaghṛtādiyuktaṃ gavyālābhe māhiṣaṃ dugdhādi
grāhyam.  dhātuprapoṣaṇaṃ laḍḍukāpūpādi. mano'bhilaṣitaṃ puṣṭādiṣu yan manorucikaraṃ tad eva yoginā
bhoktavyam. mano'bhilāṣitam api kim avihitaṃ bhoktavyam nety āha yogyam iti. vihitam evety
arthaḥ. yogī bhojanaṃ pūrvoktaviśeṣaṇaviśiṣṭam ācaret kuryād ity arthaḥ. na tu saktubharjitānnādinā
nirvāhaṃ kuryād iti bhāvaḥ.
%\end{vsid}


%\begin{vsid}{#hp01_064}
\startsloka
{\bf yuvo vṛddho 'tivṛddho vā vyādhito durbalo 'pi vā
abhyāsāt siddhim āpnoti sarvayogeṣv atandritaḥ (1.64)}
\stopsloka

yogābhyāsino vayoviśeṣārogyādyapekṣā nāstīty āha yuveti. yuvā taruṇaḥ vṛddho vṛddhāvasthāṃ prāptaḥ
ativṛddho 'tivārddhakaṃ gato vā. abhyāsād āsanakumbhakādīnām abhyasanāt siddhiṃ
samādhitatphalarūpām āpnoti. abhyāsaprakāram eva vadan viśinaṣṭi sarvayogeṣv iti. sarveṣu yogeṣu
yogāṅgeṣv atandrito 'nalasaḥ. yogāṅgābhyāsāt siddhim āpnotīty arthaḥ. jīvanasādhane kṛṣivāṇijyādau
jīvanaśabdaprayogavat sākṣāt paramparayā vā yogasādhaneṣu yogāṅgeṣu\var{yogāṅgeṣu \lem \om \Msix}
yogaśabdaprayogaḥ.
%\end{vsid}


%\begin{vsid}{#hp01_055_1}
\startsloka
{\bf kriyāyuktasya siddhiḥ syād akriyasya kathaṃ bhavet
na śāstrapāṭhamātreṇa yogasiddhiḥ prajāyate (1.65)}
\stopsloka

abhyāsād eva siddhir bhavatīti draḍhayann āha dvābhyāṃ kriyāyuktasyeti. kriyā yogāṅgānuṣṭhānarūpā
tayā yuktasya siddhir yogasiddhiḥ syāt. akriyasya yogāṅgānuṣṭhānarahitasya kathaṃ bhavet. na
katham apīty arthaḥ. nanu yogaśāstrādhyayanena yogasiddhiḥ syāt nety āha neti. śāstrasya
yogaśāstrasya pāṭhamātreṇa kevalena pāṭhena yogasya siddhir na prajāyate naiva jāyata ity arthaḥ.
%\end{vsid}

%\begin{vsid}{#hp01_055_2}
\startsloka
{\bf na veṣadhāraṇaṃ siddheḥ kāraṇaṃ na ca tatkathā
kriyaiva kāraṇaṃ siddheḥ satyam etan na saṃśayaḥ (1.66)}
\stopsloka

neti. veṣasya kāṣāyavastrādeḥ dhāraṇaṃ siddher yogasiddheḥ kāraṇaṃ na. tasya yogasya kathā vārttā
ca na siddheḥ kāraṇaṃ. kiṃ tarhi siddheḥ kāraṇam ity ata āha kriyaiveti. yogāṅgānuṣṭhānarūpā
kriyaiva siddheḥ kāraṇam etat satyaṃ saṃśayo na atrety adhyāhāraḥ.
%\end{vsid}


%\begin{vsid}{#hp01_0??} %  not in critical edition 
\startsloka
{\bf pīṭhāni kumbhakāś citrā divyāni karaṇāni ca
sarvāṇy api haṭhābhyāse rājayogaphalāvadhi (1.65)}
\stopsloka

atha\var{atha \Msix \Wthree \ \om \EdMu \EdLo} yogāṅgānuṣṭhānasyāvadhim āha pīṭhānīti. pīṭhāny āsanāni
citrā aneka\-vidhāḥ kumbhakāḥ sūryabhedādayaḥ divyāny utkṛṣṭāni karaṇāni mahā\-mudrādīni
haṭhasiddhau\var{haṭhasiddhau \EdMu \Wthree \Tue \lem om \EdLo} haṭhābhyāse haṭhasyābhyāse sarvāṇi
pīṭhakumbhakakaraṇāni rājayogaphalāvadhi rājayoga eva phalaṃ tadavadhi tatparyantaṃ kartavyānīti
śeṣaḥ.
%\end{vsid}
 
%%%%%%%%%%%%%      (2)

%\begin{vsid}{#hp02_001}
\startsloka
{\bf athāsane dṛḍhe yogī vaśī hitamitāśanaḥ
gurūpadiṣṭamārgeṇa prāṇāyāmān samabhyaset (2.1)}
\stopsloka

athāsanopadeśānantaraṃ prāṇāyāmān vaktum upakramate atheti. atheti maṅgalārthaḥe. āsane dṛḍhe sati
vaśī jitākṣaḥ hitaṃ pathyaṃ ca tan mitaṃ ca pūrvopadeśoktalakṣaṇaṃ tat tādṛśam aśanaṃ yasya sa
hitamitāśanaḥ guruṇopadiṣṭo yo mārgaḥ prāṇāyāmābhyāsaprakāras tena prāṇāyāmān vakṣyamāṇān samyag
utsāhasāhasa\var{sāhasa \Tuepc \EdMu\lem sāha \Msix sa \Tueac \EdLo}dhairyādibhir abhyaset. dṛḍhe
sthire kukkuṭādivivarjite siddhāsanādāv iti vā yojanā.
%\end{vsid}


%\begin{vsid}{#hp02_002}
\startsloka
{\bf cale vāte calaṃ cittaṃ niścale niścalaṃ bhavet
yogī sthāṇutvam āpnoti tato vāyuṃ nirodhayet (2.2)}
\stopsloka

prayojanam anuddiśya na mando 'pi pravartate iti mahadukteḥ\var{REF?} prayojanābhāve pravṛttyabhāvāt
prāṇāyāmaprayojanam āha cale vāte iti. vāte cale sati cittaṃ calaṃ bhavet. niścale vāte niścalaṃ
bhavet cittam ity atrāpi sambadhyate. vāte citte ca niścale yogī sthāṇutvaṃ sthiratvaṃ
dīrghajīvitvam iti yāvat. īśatvaṃ vāpnoti. tatas tasmād vāyuṃ prāṇaṃ nirodhayet kumbhayet.
%\end{vsid}


%\begin{vsid}{#hp02_003}
\startsloka
{\bf yāvad vāyuḥ sthito dehe tāvaj jīvanam ucyate
maraṇaṃ tasya niṣkrāntis tato vāyuṃ nirodhayet (2.3)}
\stopsloka

yāvad iti. dehe śarīre yāvatkālaṃ vāyuḥ prāṇaḥ sthitaḥ tāvatkālaparyantaṃ jīvanam ucyate lokaiḥ.
dehaprāṇasaṃyogasyaiva jīvanapadārthatvāt. tasya prāṇasya niṣkrāntir dehād viyogaḥ maraṇam ucyate.
tatas tasmād vāyuṃ nirodhayet.
%\end{vsid}

%\begin{vsid}{#hp02_004}
\startsloka
{\bf malākulāsu nāḍīṣu māruto naiva madhyagaḥ
kathaṃ syād unmanībhāvaḥ kāryasiddhiḥ kathaṃ bhavet (2.4)}
\stopsloka

malaśuddher haṭhasiddhijanakatvaṃ vyatirekeṇāha malākulāsv iti. nāḍīṣu malair ākulāsu vyāptāsu
satīṣu\var{satīṣu \Tuepc \EdMu \lem satsu Ba2 \Msix \Wthree \Tueac \EdLo} mārutaḥ prāṇo madhyagaḥ
suṣumnāmārgavāhī naiva syāt. api tu śuddhamalāsv eva madhyago bhavatīty arthaḥ. unmanībhāva unmanyā
bhāvo bhavanaṃ kathaṃ syāt na katham apīty arthaḥ.  kāryasya kaivalyarūpasya siddhir niṣpattiḥ
kathaṃ bhaven na kathañcid apīty arthaḥ.
%\end{vsid}


%\begin{vsid}{#hp02_005}
\startsloka
{\bf śuddhim eti yadā sarvaṃ nāḍīcakraṃ malākulam
tadaiva jāyate yogī prāṇasaṃgrahaṇe kṣamaḥ (2.5)}
\stopsloka

anvayenāpi malaśuddher haṭhasiddhihetutvam āha śuddhim etīti. yadā yasmin kāle malair ākulam
vyāptaṃ sarvaṃ samastaṃ nāḍīnāṃ cakraṃ samūhaḥ śuddhim malarāhityam eti prāpnoti tadaiva tasminn
eva kāle yogī yogābhyāsī prāṇasya samyag grahaṇe\var{samyag grahaṇe \Wthree \Msix \lem saṃgrahaṇe
  \EdLo grahaṇe \EdMu Ba2} kṣamaḥ samartho jāyate.
%\end{vsid}


%\begin{vsid}{#hp02_006}
\startsloka
{\bf prāṇāyāmaṃ tataḥ kuryān nityaṃ sāttvikayā dhiyā
yathā suṣumṇānāḍīsthā malāḥ śuddhiṃ prayānti ca (2.6)}    %  mala in der suṣumṇā ??
\stopsloka

malaśuddhiḥ kathaṃ bhavatīty ākāṅkṣāyāṃ tacchodhakaṃ prāṇāyāmam āha prāṇāyāmam iti. yato
malaśuddhiṃ vinā prāṇasaṅgrahaṇe kṣamo na bhavati tatas tasmād
īśvarapraṇidhānotsāhasāhasādiprayatnābhibhūtavikṣepālasyādirājasatāmasadharmayā sāttvikayā
prakāśa\-pra\-sādaśīlayā dhiyā buddhyā nityaṃ prāṇāyāmaṃ kuryāt. yathā yena prakāreṇa suṣumnṇānāḍyāṃ
sthitā malāḥ śuddhim apagamaṃ prayānti naśyantīty arthaḥ.
%\end{vsid}


%\begin{vsid}{#hp02_007}
\startsloka
{\bf baddhapadmāsano yogī prāṇaṃ candreṇa pūrayet
dhārayitvā yathāśakti bhūyaḥ sūryeṇa recayet (2.7)}
\stopsloka

malaśodhakaprāṇāyāmaprakāram āha dvābhyāṃ baddhapadmāsana iti. baddhaṃ padmāsanaṃ yena tādṛśo yogī
prāṇaṃ vāyuṃ candreṇa candranāḍyā iḍayā pūrayet. śaktim anatikramya yathāśakti dhārayitvā
kumbhayitvā bhūyaḥ punaḥ sūryeṇa sūryanāḍyā piṅgalayā recayet. bāhyavāyoḥ prayatnaviśeṣād upādānaṃ
pūrakaḥ. jālandharādibandhapūrvakaṃ prāṇanirodhaḥ kumbhakaḥ. kumbhitasya vāyoḥ prayatnaviśeṣād
vamanaṃ recakaḥ. prāṇāyāmāṅgarecakapūrakayor\var{recakapūrakayor \Wthree \lem pūrakarecakayor \Msix}
eveme lakṣaṇe iti. bhastrāval lohakārasya recapūrau sasambhramau\comment{HP 2.35} iti
gauṇarecakapūrakayor nātivyāptiḥ tayor lakṣyatvābhāvāt.
%\end{vsid}


%\begin{vsid}{#hp02_008}
\startsloka
{\bf prāṇaṃ sūryeṇa cākṛṣya pūrayed udaraṃ śanaiḥ
vidhivat kumbhakaṃ kṛtvā punaś candreṇa recayet (2.8)}  % dhṛtvā \Msix
\stopsloka

prāṇam iti. sūryeṇa sūryanāḍyā piṅgalayā sūryeṇa prāṇam ākṛṣya gṛhītvā śanair mandaṃ mandam udaraṃ
jaṭharaṃ pūrayet. vidhivatd bandhapūrvakaṃ kumbhakaṃ kṛtvā punar bhūyaś candreṇeḍayā recayet.
%\end{vsid}


%\begin{vsid}{#hp02_009}
\startsloka
{\bf yena tyajet tena pītvā dhārayed atirodhataḥ
recayec ca tato 'nyena śanair eva na vegataḥ (2.9)}
\stopsloka

ukte prāṇāyāme viśeṣam āha yeneti. yena candreṇa sūryeṇa vā tyajed recayet tena pītvā tenaiva
pūrayitvā. atirodhato 'tiśayitena rodhena svedakampādijananaparyantena. sārvavibhaktikas tasil.
yena pūrakas tato 'nyena śanair eva recayet na tu vegataḥ. vegād recane balahāniḥ syāt. yena
pūrakaḥ kṛtas tena recako na kartavyaḥ. yena recakaḥ kṛtas tenaiva pūrakaḥ kartavya iti bhāvaḥ.
%\end{vsid}


%\begin{vsid}{#hp02_010}
\startsloka
{\bf prāṇaṃ ced iḍayā piben niyamitaṃ bhūyo 'nyayā recayet 
pītvā piṅgalayā samīraṇam atho baddhvā tyajed vāmayā
sūryācandramasor anena vidhinābhyāsaṃ sadā tanvatāṃ     % wrong caesura !?
śuddhā nāḍigaṇā bhavanti yamināṃ māsatrayād ūrdhvataḥ (2.10)}
\stopsloka

baddhapadmāsana ityādyuktam arthaṃ piṇḍīkṛtyānuvadan prāṇāyāma\-syāvāntara\-phalam āha prāṇam iti ced
yadi iḍayā vāmanāḍyā prāṇaṃ pibet pūrayet tarhi niyamitaṃ kumbhitaṃ prāṇaṃ bhūyaḥ punar anyayā
piṅgalayā recayet. piṅgalayā dakṣanāḍyā samīraṇaṃm vāyuṃ pītvā pūrayitvātho pūraṇānantaraṃ baddhvā
kumbhayitvā vāmayā iḍayā tyajed recayet. sūryaś ca candramāś ca sūryācandramasau tayoḥ
devatādvandve ca\comment{Aṣṭādhyāyī 7.3.21} ity ānaṅ. anenoktena vidhinā prakāreṇa sadā nityam
abhyāsaṃ candreṇāpūrya kumbhayitvā sūryeṇa recayet sūryeṇāpūrya kumbhayitvā candreṇa recayed ity
ākārakaṃ tanvatāṃ vistārayatāṃ yamināṃ yamavatāṃ nāḍīgaṇāḥ nāḍīsamūhāḥ māsatrayād ūrdhvato māsānāṃ
trayaṃ māsatrayaṃ tasmād upari śuddhā malarahitā bhavanti.
%\end{vsid}



%\begin{vsid}{#hp02_011}
\startsloka
{\bf prātar madhyandine sāyam ardharātre ca kumbhakān
śanair aśītiparyantaṃ caturvāraṃ samabhyaset (2.11)}
\stopsloka

atha prāṇāyāmābhyāsakālaṃ tadavadhiṃ cāha prātar iti. prātar aruṇodayam ārabhya sūryodayād
ghaṭikātrayaparyante prātaḥkāle madhyandine madhyāhne pañcadhā vibhaktasya dinasya madhyabhāge
sāyaṃsandhyā trināḍīpramitārkam astād\var{pramitārkam astād \EdLo \Msix \Tue \lem pramitārkāstād
  \EdMu} adhastād ūrdhvaṃ cety uktalakṣaṇe sandhyākāle rātrer ardhaṃ ardharātraṃ tasminn ardharātre
rātrer madhye muhūrtadvaye ca śanair aśītiparyantam aśītisaṅkhyāvadhi caturvāraṃ
vāracatuṣṭayaṃ. kālādhvanor atyantasaṃyoge\comment{Aṣṭādhyāyī 2.3.5.} iti dvitīyā.  caturṣu kāleṣv
ekaikasmin kāle aśītiprāṇāyāmāḥ kāryāḥ. ardharātre kartum aśaktaś cet trisandhyaṃ kartavyā iti
sampradāyaḥ. caturvāraṃ kṛtāś ced dine dine viṃśatyadhikaśatatrayaparimitāḥ prāṇāyāmā
bhavanti. vāratrayaṃ kṛtāś cet catvāriṃśadadhikaśatadvayaparimitā bhavanti.
%\end{vsid}

%\begin{vsid}{#hp02_012}
\startsloka
{\bf kanīyasi bhavet svedaḥ kampo bhavati madhyame
uttame sthānam āpnoti tato vāyuṃ nibandhayet (2.12)}
\stopsloka

kaniṣṭhamadhyamottamānāṃ prāṇāyāmānāṃ krameṇa jñāpakaviśeṣān āha kanīyasīti. kanīyasi kaniṣṭhe
prāṇāyāme svedaḥ prasvedo bhaved bhavati. svedānumeyaḥ kaniṣṭhaḥ. madhyame prāṇāyāme kampo
'ṅgamejayo bhavati. kampānumeyo madhyamaḥ. uttame prāṇāyāme sthānaṃ brahmarandhram āpnoti.
sthanaprāptyanumeya uttamaḥ. tatas tasmād vāyuṃ prāṇaṃ nibandhayen nitarāṃ bandhayet.
kaniṣṭhādīnāṃ lakṣaṇam uktaṃ liṅgapurāṇe

\startsloka
prāṇāyāmasya mānaṃ tu mātrādvādaśakaṃ smṛtam
nīco dvādaśamātras tu sakṛd udgdhāta īritaḥ
madhyamas tu dvirudghātaś caturviṃśatimātrakaḥ
mukhyas tu yas trirudghātaḥ ṣaṭtriṃśanmātra ucyate
prasvedakampanotthānajanakaś ca yathākramam
ānando jāyate cātra nidrā ghūrṇis\var{ghūrṇis \EdAd \lem dhūmas \EdMu \Tueac (?) ghūrmas \EdLo \Wthree \Msix \Tuepc } tathaiva ca
romāñco dhvanisaṃvittir aṅgamoṭanakampanam
bhramaṇasvedajalpādyaṃ saṃvinmūrcchā jayed yadā   % bhramaṇa \lem śravaṇa \Tue  ? ādyaṃ \Tue \lem ādya \Msix \Wthree
tadottama iti proktaḥ prāṇāyāmaḥ suśobhana iti\comment{Liṅgapurāṇa 8.46–50}
\stopsloka

ghūrṇiś cittāndolanam. gorakṣo 'pi

\startsloka
adhame dvādaśa proktā madhyame dviguṇāḥ smṛtāḥ
uttame triguṇā mātrāḥ prāṇāyāme dvijottamaiḥ\comment{2.5?}??
\stopsloka

udgdhātalakṣaṇaṃ tu

\startsloka
prāṇenotsāryamāṇena\var{sāryamāṇena \EdLo \lem sārpamāṇena \Msix} apānaḥ pīḍyate yadā
gatvā cordhvaṃ nivarteta etad uddghātalakṣaṇam
\stopsloka

mātrām āha yājñavalkyaḥ

\startsloka
aṅguṣṭhāṅgulimokṣaṃ tris trir jānuparimārjanam
tālatrayam api prājñā mātrāsaṃjñāṃ pracakṣate\comment{?}
\stopsloka

skandapurāṇe

\startsloka
ekaśvāsamayī mātrā prāṇāyāme\var{prāṇāyāme \Msix \Wthree \Tue \lem prāṇāyāmo \EdMu \EdLo} nigadyate 
\stopsloka

etad vyākhyātaṃ yogacintāmaṇau\comment{}

nidrāvaśaṃ gatasya puṃso yāvatā kālenaikaḥ śvāso gacchaty āgacchati ca tāvān kālaḥ prāṇāyāmasya
mātrety ucyata iti. ardhaśvāsādhikadvādaśaśvā\-sā\-vacchinnaḥ kālaḥ prāṇāyāmakālaḥ. ṣaḍbhiḥ śvāsair
ekaṃ palaṃ bhavati. evaṃ ca sārdhaśvāsapaladvayātmakaḥ kālaḥ prāṇāyāmakālaḥ
siddhaḥ. sārdha\-dvādaśamātrāmitaḥ prāṇāyāmo yaḥ sa evottamaḥ prāṇāyāma ity ucyate.  na ca
pūrvodāhṛtaliṅgapurāṇagorakṣavākyavirodhas tatra dvādaśa\-mātrakasya prāṇāyāmasyādhamatvokter iti
śaṅkanīyam.

\startsloka
jānuṃ pradakṣiṇīkuryān na drutaṃ na vilambitam
pradadyāc choṭikāṃ yāvat tāvan mātreti gīyate
\stopsloka

iti skandapurāṇāt.

\startsloka
aṅguṣṭhāṅgulimokṣaṃ ca jānoś ca parimārjanam
pradadyāc choṭikām ekāṃ mātrā saṅkhyāyate tadā
\stopsloka

iti dattātreyavacanāc ca.

liṅgapurāṇagorakṣādivākyeṣv ekacchoṭikāvacchinnasya kālasya mātrātvena vivakṣitatvāt.
yājñavalkyādivākyeṣu ekachoṭikāvacchinnasya kālasya mātrātvena vivakṣitatvāt
triguṇasyādhamasyottamatvaṃ tatrāpy uktam ity avirodhaḥ. sarveṣu yogasādhaneṣu prāṇāyāmo mukhyaḥ
tatsiddhau pratyāhārādīnāṃ siddheḥ. tadasiddhau pratyāhārādyasiddheś ca. vastutas tu prāṇāyāma eva
pratyāhārādiśabdair nigadyate. tathā coktaṃ yogacintāmaṇau\comment{?}

\startsloka
prāṇāyāma evābhyāsakrameṇa vardhamānaḥ pratyāhāradhāraṇādhyāna\-samādhiśabdair ucyata iti.
\stopsloka

tad uktaṃ skandapurāṇe

\startsloka
prāṇāyāmadviṣaṭkena pratyāhāra udāhṛtaḥ
pratyāhāradviṣaṭkena dhāraṇā parikīrtitā
bhaved īśvarasaṅgatyai dhyānaṃ dvādaśadhāraṇam
dhyānadvādaśakenaiva samādhir abhidhīyate
yat samādhau paraṃ jyotir anantaṃ svaprakāśakam
tasmin dṛṣṭe kriyākāṇḍaṃ yātāyātaṃ nivartate iti
\stopsloka

tathā

\startsloka
dhāraṇā pañcanāḍībhir dhyānaṃ syāt ṣaṣṭhināḍikam
dinadvādaśakena syāt samādhiḥ prāṇasaṃyamād  iti ca
\stopsloka

gorakṣādibhir apy evam evoktam. atraivaṃ vyavasthā kiñcid ūnadvicatvāriṃśadvipalātmakaḥ
kaniṣṭhaprāṇāyāmakālaḥ. ayam evaikacchoṭikāvacchinnasya kālasya mātrātvavivakṣayā dvādaśamātrakaḥ
kālaḥ. kiñcid ūnacaturaśītivipalātmako madhyamaprāṇāyāmakālaḥ. ayam ekacchoṭikāvacchinnasya kālasya
mātrātvavivakṣayā caturviṃśatimātrakaḥ kālaḥ. pañcaviṃśatyuttaraśatavipalātmaka uttamaḥ
prāṇāyāmakālaḥ. ayam ekacchoṭikāvacchinnasya kālasya mātrātvavivakṣayā ṣaṭtriṃśanmātrakaḥ kālaḥ.
choṭikātrayāvacchinnasya kālasya mātrātvavivakṣayā tu dvādaśamātraka eva. bandhapūrvakaṃ
pañcaviṃśatyuttaraśatavipalaparyantaṃ yadā prāṇāyāmasthairyaṃ bhavati tadā prāṇo brahmarandhraṃ
gacchati. brahmarandhraṃ gataḥ prāṇo yadā pañcaviṃśatipalaparyantaṃ tiṣṭhati tadā pratyāhāraḥ.
yadā pañcaghaṭikāparyantaṃ tiṣṭhati tadā dhāraṇā. yadā ṣaṣṭhighaṭikāparyantaṃ tiṣṭhati tadā
dhyānam. yadā dvādaśadinaparyantaṃ tiṣṭhati tadā samādhir bhavatīti sarvaṃ ramaṇīyam.
%\end{vsid}

%\begin{vsid}{#hp02_013}
\startsloka
{\bf jalena śramajātena gātramardanam ācaret
dṛḍhatā laghutā caiva tena gātrasya jāyate (2.13)}
\stopsloka

prāṇāyāmān abhyasataḥ svede jāte viśeṣam āha jaleneti. śramāt prāṇāyāmābhyāsaśramāj jātaṃ
śramajātaṃ tena jalena prasvedena gātrasya śarīrasya mardanaṃ tailābhyaṅgavad ācaret kuryāt. tena
mardanena gātrasya dṛḍhatā dārḍhyaṃ laghutā jāḍyābhāvo jāyate prādurbhavati.
%\end{vsid}


%\begin{vsid}{#hp02_014}
\startsloka
{\bf abhyāsakāle prathame śastaṃ kṣīrājyabhojanam
tato 'bhyāse dṛḍhībhūte na tādṛṅniyamagrahaḥ (2.14)}
\stopsloka

atha prathamottarābhyāsayoḥ kṣīrādiniyamāniyamau\var{kṣīrādiniyamāniyamau \Msix \Wthree kṣīrādiniyamān
  \EdLo \EdMu} āha abhyāsakāla iti. kṣīraṃ dugdham ājyaṃ ghṛtaṃ tadyuktaṃ bhojanaṃ
kṣīrājyabhojanam. śākapārthivādivat samāsaḥ. kevale kumbhake siddhe 'bhyāso dṛḍho bhavati. spaṣṭam
anyat.
%\end{vsid}


%\begin{vsid}{#hp02_015}
\startsloka
{\bf yathā siṃho gajo vyāghro bhaved vaśyaḥ śanaiḥ śanaiḥ
tathaiva sevito vāyur anyathā hanti sādhakam (2.15)}
\stopsloka

siṃhādivac chanair eva prāṇaṃ vaśayen na sahasety āha yatheti. yathā yena prakāreṇa siṃho mṛgendro
gajo hastī\var{hastī \Msix \Wthree \lem vanahastī \EdLo} vyāghraḥ śārdūlaḥ śanaiḥ śanair eva vaśyaḥ
svādhīno bhaven na sahasā tathaiva tenaiva prakāreṇa sevito 'bhyasto vāyuḥ prāṇo vaśyo
bhavet. anyathā sahasā gṛhyamāṇaḥ sādhakam abhyāsinaṃ hanti siṃhādivat.
%\end{vsid}


%\begin{vsid}{#hp02_016}
\startsloka
{\bf prāṇāyāmena yuktena sarvarogakṣayo bhavet
ayuktābhyāsayogena sarvarogasamudbhavaḥ (2.16)}
\stopsloka

yuktāyuktayoḥ prāṇāyāmayoḥ phalam āha prāṇāyāmeneti. āhārādiyuktipūrvako
jālandharādibandhayuktiviśiṣṭaḥ prāṇāyāmo yukta ity ucyate. tena sarvarogakṣayaḥ sarveṣāṃ rogāṇāṃ
kṣayo nāśo bhavet. ayukta uktayuktirahito yo 'bhyāsas tadyuktena prāṇāyāmena sarvarogasamudbhavaḥ
sarveṣāṃ rogāṇāṃ samyag udbhava utpattir bhavet.
%\end{vsid}


%\begin{vsid}{#hp02_017}
\startsloka
{\bf hikkā śvāsaś ca kāsaś ca śiraḥkarṇākṣivedanāḥ
bhavanti vividhā rogāḥ pavanasya prakopataḥ (2.17)}
\stopsloka

ayuktena prāṇāyāmena ke rogā bhavantīty apekṣāyām āha hikketi. hikkāśvāsakāsāḥ rogaviśeṣāḥ. śiraś ca
karṇau cākṣiṇī ca śiraḥkarṇākṣi śiraḥkarṇākṣiṇi vedanāḥ śiraḥkarṇākṣivedanāḥ. vividhā nānāvidhā
rogāḥ jvarādayaḥ. pavanasya vāyoḥ prakopato bhavanti.
%\end{vsid}


%\begin{vsid}{#hp02_018}
\startsloka
{\bf yuktaṃ yuktaṃ tyajed vāyuṃ yuktaṃ yuktaṃ ca pūrayet
yuktaṃ yuktaṃ ca badhnīyād evaṃ siddhim avāpnuyāt (2.18)}
\stopsloka

yataḥ pavanasya prakopato vividhā rogā bhavanty ataḥ \var{ataḥ \Wthree \Msix \Tue \EdAd \lem ataḥ kiṃ
  kartavyam ata āha \EdMu \EdLo} yuktaṃ yuktam iti.  vāyuṃ prāṇaṃ yuktaṃ yuktaṃ tyajet. recakakāle
śanaiḥ śanair eva recayen na vegataḥ ity arthaḥ.  yuktaṃ yuktaṃ ca na cālpaṃ\var{na cālpaṃ \EdLo
  \lem nālpaṃ \Msix \Wthree} nādhikaṃ ca pūrayet. yuktaṃ yuktaṃ ca jālandharādibandhayuktaṃ badhnīyāt
kumbhayet. evam abhyasan siddhiṃ haṭhasiddhim avāpnuyāt prāpnuyāt\var{āpnuyāt prāpnuyāt \Msix \Wthree
  \Tue \lem avāpnuyāt \EdMu \EdLo}.
%\end{vsid}

% wwwww 2125


%\begin{vsid}{#hp02_019}
\startsloka
{\bf yadā tu nāḍīśuddhiḥ syāt tathā cihnāni bāhyataḥ
kāyasya kṛśatā kāntis tadā jāyeta niścitam (2.19)}  % tathā  tadā   check Lesarten.
\stopsloka

yuktaṃ prāṇāyāmam abhyasato jāyamānāyā nāḍīśuddher lakṣaṇam āha dvābhyāṃ yadā tv iti. yadā tu
yasmin kāle tu nāḍīnāṃ śuddhir malarāhitayaṃ syāt tadā bāhyato bāhyāni sārvavibhaktikas tasiḥ
cihnāni lakṣaṇāni tathāśabdenāntarāṇy api cihnāni bhavantīty arthaḥ. tāny evāha kāyasyeti.
kāyasya dehasya kṛśatā kārśyaṃ kāntiḥ surucir niścitam jāyeta.
%\end{vsid}

%\begin{vsid}{#hp02_020}
\startsloka
{\bf yatheṣṭaṃ dhāraṇaṃ vāyor analasya pradīpanam
nādābhivyaktir ārogyaṃ jāyate nāḍiśodhanāt (2.20)}
\stopsloka

yatheṣṭam iti. vāyoḥ prāṇasya yatheṣṭaṃ bahuvāraṃ dhāraṇaṃ kumbhakeṣu. analasya jaṭharāgneḥ
pradīpanaṃ prakṛṣṭā dīptir nādasya dhvaner abhivyaktiḥ prākaṭyam ārogyam arogatā nāḍīśodhanāt
nāḍīnāṃ śodhanāt malarāhityāj jāyate.
%\end{vsid}


%\begin{vsid}{#hp02_021}
\startsloka
{\bf medaśleṣmādhikaḥ pūrvaṃ ṣaṭkarmāṇi samācaret
anyas tu nācaret tāni doṣāṇāṃ samabhāvataḥ (2.21)}
\stopsloka

medādyādhikye upāyāntaram āha medaśleṣmādhika iti. medaś ca śleṣmā ca medaśleṣmāṇau tāv adhikau
yasya sa tādṛśaḥ puruṣaḥ. pūrvaṃ prāṇāyāmābhyāsāt prāk na tu prāṇāyāmābhyāsakāle. ṣaṭ karmāṇi
vakṣyamāṇāni samācaret samyag ācaret. anyas tu medaśleṣmādhikyarahitas tu tāni ṣaṭ karmāṇi nācaret.
tatra hetum āha doṣāṇāṃ vātapittakaphānāṃ samasya bhāvaḥ samabhāvaḥ samatvaṃ tasmād doṣāṇāṃ
samatvād ity arthaḥ.
%\end{vsid}

%\begin{vsid}{#hp02_022}
\startsloka
{\bf dhautir bastis tathā netis trāṭakaṃ naulikaṃ tathā
kapālabhātiś caitāni ṣaṭkarmāṇi pracakṣate (2.22)}
\stopsloka

ṣaṭkarmāṇy uddiśati dhautir iti. spaṣṭam.
%\end{vsid}


%\begin{vsid}{#hp02_023}
\startsloka
{\bf karmaṣaṭkam idaṃ gopyaṃ ghaṭaśodhanakārakam
vicitraguṇasandhāyi pūjyate yogipuṅgavaiḥ (2.23)}
\stopsloka

idaṃ rahasyam ity āha karmaṣaṭkam iti. ghaṭasya śarīrasya śodhanaṃ malāpanayanaṃ karotīti
ghaṭaśodhanakārakam idam uddiṣṭaṃ karmaṇāṃ ṣaṭkam dhautyādikaṃ gopyaṃ gopanīyam. yataḥ
vicitraguṇasandhāyi\var{sandhāyi \emend \lem sandhāyī \Wthree \Msix \Tue sandhāyīti \EdLo \EdMu}
vicitraṃ vilakṣaṇaṃ guṇaṃ ṣaṭkarmarūpaṃ sandhātuṃ kartuṃ śīlam asyeti vicitraguṇasandhāyi
yogipuṅgavair yogiśreṣṭhaiḥ pūjyate satkriyate. gopanābhāve tu karmaṣaṭkam anyair api viditaṃ syād
iti yoginaḥ pūjyatvābhāvaḥ prasajyeteti bhāvaḥ. etenedam eva karmaṣaṭkasya mukhyaṃ phalam iti
sūcitam. medaśleṣmādināśasya prāṇāyāmair api sambhavāt. tad uktaṃ ṣaṭkarmayogam āpnoti
pavanābhyāsatatpara iti pūrvottaragranthasyāpy evam eva svārasyāc ca.
%\end{vsid}


%\begin{vsid}{#hp02_024}
\startsloka
{\bf tatra dhautiḥ
caturaṅgulavistāraṃ hastapañcadaśāyatam
gurūpadiṣṭamārgeṇa siktaṃ vastraṃ śanair graset
punaḥ pratyāharec caitad uditaṃ dhautikarma tat (2.24)}
\stopsloka

dhautikarmāha caturaṅgulam iti. caturṇām aṅgulānāṃ samāhāraś caturaṅgulaṃ caturaṅgulaṃ
vistāro\var{caturaṅgulaṃ vistāro \Tue \EdMu \Msix \lem caturaṅgulavistāro \Wthree vistāro \EdLo} yasya
tādṛśaṃ hastānāṃ pañcadaśair āyataṃ dīrghaṃ siktaṃ jalārdraṃ kiñcid uṣṇaṃ vastraṃ paṭaṃ tac ca
sūkṣmaṃ nūtanoṣṇīṣādeḥ khaṇḍaṃ grāhyam. guruṇā upadiṣṭo yo mārgaḥ vastragrasanaprakāras tena śanair
mandaṃ mandaṃ kiñcit kiñcid graset.  prathamadine hastamātraṃ graset.\var{prathamadine hastamātraṃ
  graset \Wthree \Msix \ \om \Tue \EdLo (reports something misssing, but the missing part is not
  mentioned) \EdMu \EdAd} dvitīye dine hastadvayaṃ tṛtīyadine hastatrayam. evaṃ dinavṛddhyā
hastapañcadaśamātraṃ graset. punar iti. tasya prāntaṃ rājadantamadhye dṛḍhaṃ saṃlagnaṃ kṛtvā
naulikarmaṇodarasthaṃ vastraṃ samyak cālayitvā punaḥ śanaiḥ pratyāharec ca etad vastram udgiren
niṣkāsayec ca. tad dhautikarma uditaṃ kathitaṃ siddhaiḥ.
%\end{vsid}



%\begin{vsid}{#hp02_025}
\startsloka
{\bf kāsaśvāsaplīhakuṣṭhaṃ kapharogāś ca viṃśatiḥ
dhautikarmaprabhāveṇa prayānty eva na saṃśayaḥ (2.25)}
\stopsloka

dhautikarmaṇaḥ phalam āha kāsaśvāseti. kāsaś ca śvāsaś ca plīhaś ca kuṣṭhaṃ ca. samāhāradvandvaḥ
kāsādayo rogaviśeṣāḥ viṃśatiḥ viṃśatisaṅkhyākāḥ kapharogāś ca dhautikarmaṇaḥ prabhāveṇa gacchanty
eva na saṃśayaḥ niścitam etad ity arthaḥ.
%\end{vsid}


%\begin{vsid}{#hp02_026}
\startsloka
{\bf atha bastiḥ
nābhidaghnajale pāyau nyastanālotkaṭāsanaḥ
ādhārākuñcanaṃ kuryāt kṣālanaṃ bastikarma tat (2.26)}
\stopsloka

atha bastikarmāha nābhidaghneti. nābhipramāṇaṃ nābhidaghnam. pramāṇe
daghnacpratyayaḥ.\comment{Compare Aṣṭādhyāyī 5.3.37. A vārttika mentions parimāṇa as a possible
  sense on the same Sūtra, but not for daghna. Readings of many manuscripts have parimāṇa for
  pramāṇa.} tasminn nābhidaghne nābhipramāṇe jale nadyāditoye pāyur gudaṃ tasmin nyasto nālo
vaṃśanālo yena kaniṣṭhikāpraveśayogyarandhrayuktaṃ ṣaḍaṅguladīrghaṃ vaṃśanālaṃ gṛhītvā caturaṅgulaṃ
pāyau praveśayet. aṅgulidvayamitaṃ bahiḥ sthāpayet. utkaṭam āsanaṃ yasya sa utkaṭāsanaḥ.
pārṣṇidvaye sphicau vinyasya pādāṅgulibhiḥ sthitir utkaṭāsanam. ādhārasya mūlādhārasya ākuñcanaṃ
yathā jalam antaḥ praviśet tathā saṅkocanaṃ kuryāt. antaḥpraviṣṭaṃ jalaṃ naulikarmaṇā cālayitvā
tyajet. tat kṣālanaṃ bastikarmocyate. dhautibastikarmadvayaṃ bhojanāt prāg eva kartavyam.
tadanantaraṃ bhojane vilambo 'pi na kāryaḥ. kecit tu pūrvaṃ mūlādhāreṇa vāyor ākarṣaṇam abhyasya
jale sthitvā pāyau nālapraveśanam antareṇaiva bastikarmābhyasyanti. tathākaraṇe sarvaṃ jalaṃ bahir
nāyāti. ato nānārogadhātukṣayādisambhavāc ca tathā bastikarma naiva vidheyam. kim anyathā
svātmārāmaḥ pāyau nyastanāla iti brūyāt.
%\end{vsid}


%\begin{vsid}{#hp02_027}
\startsloka
{\bf gulmaplīhodaraṃ cāpi vātapittakaphodbhavāḥ
bastikarmaprabhāveṇa kṣīyante sakalāmayāḥ (2.27)}
\stopsloka

bastikarmaguṇān āha dvābhyāṃ gulmaplīhodaram iti. gulmaś ca plīhaś ca rogaviśeṣāv udaraṃ jalodaraṃ
ca teṣāṃ samāhāradvandvaḥ. vātaś ca pittaṃ ca kaphaś ca tebhyaḥ udbhavāḥ ekaikasmād dvābhyāṃ
sarvebhyo vā jātāḥ sakalāḥ sarve āmayā rogāḥ bastikarmaṇaḥ prabhāvaḥ sāmarthyaṃ śaktiḥ tena
kṣīyante naśyanti.
%\end{vsid}


%\begin{vsid}{#hp02_028}
\startsloka
{\bf dhātvindriyāntaḥkaraṇaprasādaṃ 
dadyāc ca kāntiṃ dahanapradīptim
aśeṣadoṣopacayaṃ nihanyād 
abhyasyamānaṃ jalabastikarma (2.28)}
\stopsloka

dhātv iti. abhyasyamānam anuṣṭhīyamānaṃ jale bastikarma jalabastikarma kartṛ dadyād anuṣṭhātur
iti śeṣaḥ. dhātavo

\startsloka
rasāsṛṅmāṃsamedo'sthimajjāśukrāṇi dhātavaḥ\comment{vāgbhaṭaḥ 1.13?}
\stopsloka

ity uktāḥ.  indriyāṇi vākpāṇipādapāyūpasthāni pañca karmendriyāṇi śrotratvakcakṣurjihvāghrāṇāni
pañca jñānendriyāṇi ca antaḥkaraṇāni manobuddhicittāhaṅkārarūpāṇi teṣāṃ
paritāpavikṣepaśokādi\var{śokādi \Msix \Wthree \Tue \lem śoka
  \EdLo}mohagauravāvaraṇadainyādirājasatāmasadharmavinivartanena
sukhaprakāśalāgha\-vā\-disāttvikadharmāvirbhāvaḥ prasādas taṃ kāntiṃ dyutiṃ dahanasya jaṭharāgneḥ
pradīptiṃ prakṛṣṭāṃ dīptiṃ ca. tathā aśeṣāḥ samastā ye doṣāḥ vātapittakaphās teṣām upacayo
'pacayasyāpy\var{upacayo 'pacayasyāpy \Wthree \Tue \Msix \lem \EdLo inconclusive}
upalakṣaṇam. upacayāpacayau nihanyād nitarāṃ hanyāt.  doṣasāmyarūpam ārogyaṃ kuryād ity arthaḥ.
%\end{vsid}

% Commentary on 2.28:
% Brahmānanda clarifies the construction as: {\em jalabastikarma kartṛ dadyād anuṣṭhātur
% iti śeṣaḥ} ``...''.

%   bis hier mit \Msix kollationiert. 


%\begin{vsid}{#hp02_030}
\startsloka
{\bf
atha netiḥ
sūtraṃ vitasti susnigdhaṃ nāsānāle praveśayet
mukhān nirgamayec caiṣā netiḥ siddhair nigadyate (2.29)}
\stopsloka

atha netikarmāha sūtram iti. vitasti vitastimitaṃ vitastir ity upalakṣaṇam adhikasyāpi. yāvatā
sūtreṇa samyag netikarma bhavet tāvad grāhyam. susnigdhaṃ suṣṭhu snigdhaṃ granthyādirahitaṃ
sūtram. tac ca navadhā dvādaśadhā pañcadaśadhā vā guṇitaṃ sudṛḍhaṃ grāhyam. nāsā nāsikā saiva nālaḥ
sacchidratvāt tasmin praveśayet. mukhān nirgamayen niṣkāsayet.

tatprakāras tv evaṃ. sūtraprāntaṃ nāsānāle praveśyetaranāsāpuṭam aṅgulyā nirudhya pūrakaṃ kuryāt.
punaś ca mukhena recayet. punaḥ punar evaṃ kurvato mukhe sūtraprāntam āyāti. tatsūtraprāntaṃ
nāsābahiḥsthasūtraprāntaṃ ca gṛhītvā śanaiś cālayed iti. cakārād ekasmin nāsānāle praveśyetarasmin
nirgamayed ity uktam. tatprakāras tv ekasmin nāsānāle sūtraprāntaṃ praveśyetaranāsāpuṭam aṅgulyā
nirudhya pūrakaṃ kuryāt paścād itaranāsānālena recayet. punaḥ punar evaṃ kurvata itaranāsānāle
sūtraprāntam āyāti tasya pūrvavac cālanaṃ kuryād iti. ayaṃ prakāras tu bahuvāraṃ kurvataḥ kadācid
bhavati. eṣoktā siddhair aṇimādiguṇasampannaiḥ.

tad uktam

\startsloka
avāptāṣṭaguṇaiśvaryāḥ siddhāḥ sadbhir nirūpitāḥ
\stopsloka

iti. netir nigadyate netir iti kathyate.
%\end{vsid}


%\begin{vsid}{#hp02_031}
\startsloka
{\bf kapālaśodhanī caiva divyadṛṣṭipradāyinī
jatrūrdhvajātarogaughaṃ netir āśu nihanti ca (2.30)}
\stopsloka

netiguṇān āha kapālaśodhanīti. kapālaṃ śodhayati śuddhaṃ malarahitaṃ karotīti kapālaśodhanī.
cakārān nāsānālādīnām api. evaśabdo 'vadhāraṇe. divyāṃ sūkṣmapadārthagrāhiṇīṃ dṛṣṭiṃ prakarṣeṇa
dadātīti divyadṛṣṭipradāyinī. netir netikriyā. jatruṇoḥ skandhasandhyayor ūrdhvam uparibhāge jāto
jatrūrdhvajātaḥ sa cāsau rogāṇām aughaś ca tam āśu jhaṭiti nihanti. cakāraḥ pādapūraṇe.

\startsloka
  skandhe bhujaśiro 'ṃso 'strī sanṃdhī tasyaiva jatruṇī
\stopsloka

ity amaraḥ\comment{Amarakośa 2.6.78}.
%\end{vsid}



%\begin{vsid}{#hp02_032}
\startsloka
{\bf atha trāṭakam
nirīkṣen niścaladṛśā sūkṣmalakṣyaṃ samāhitaḥ
aśrusampātaparyantam ācāryais trāṭakaṃ smṛtam (2.31)}
\stopsloka

trāṭakam āha nirīkṣed iti. samāhitaḥ ekāgracitto 'bhyāsī niścalā acalā cāsau\var{niścalā acalā
  cāsau \Wthree \lem niścalā cāsau \EdMu (and six mss), niścalācalatvācāsau \EdLo} dṛk ca dṛṣṭis tayā
sūkṣmaṃ ca tal lakṣyaṃ ca sūkṣmalakṣyam aśrūṇāṃ samyak pātaḥ patanaṃ tatparyantam.  anena
nirīkṣaṇasyāvadhir uktaḥ. nirīkṣet paśyet. ācāryair matsyendrādibhir idaṃ trāṭakaṃ trāṭakakarma
smṛtaṃ kathitam.
%\end{vsid}


%\begin{vsid}{#hp02_033}
\startsloka
{\bf mocanaṃ netrarogāṇāṃ tandrādīṇāṃ kapāṭakam
yatnatas trāṭakaṃ gopyaṃ yathā hāṭakapeṭakam (2.32)}
\stopsloka

trāṭakaguṇān āha mocanam iti. netrasya rogā netrarogās teṣāṃ mocanaṃ nāśakaṃ tandrā ādir yeṣām
ālasyādīnāṃ teṣāṃ kapāṭakam kapāṭavad antardhāyakam abhibhāvakam ity arthaḥ. tandrā tāmasaś
cittavṛttiviśeṣaḥ. trāṭakaṃm trāṭakākhyaṃm karma yatnataḥ prayatnād gopyaṃ gopanīyam.  gopane
dṛṣṭāntam āha yatheti. hāṭakasya suvarṇasya peṭakaṃ peṭī iti loke prasiddhaṃ. yathā yena prakāreṇa
gopyate tadvat.
%\end{vsid}


%\begin{vsid}{#hp02_034}
\startsloka
{\bf atha nauliḥ
amandāvartavegena tundaṃ savyāpasavyataḥ
natāṃso bhrāmayed eṣā nauliḥ siddhaiḥ pracakṣyate (2.33)}
\stopsloka

naulikarmāha amandeti. natau namrībhūtāv aṃsau skandhau yasya sa natāṃsaḥ pumān amando 'tiśayito ya
āvartas tasyeva\var{āvartas tasyeva \EdMu I II XII XIII \lem āvartasyeva \EdLo \Wthree} jalabhramasyeva
vego javas tena tundam udaram

\startsloka
  picaṇḍakukṣī jaṭharodaraṃ tundaṃ stanau kucau
\stopsloka

ity amaraḥ\comment{Amarakoṣa 2.6.77}. savyaṃ cāpasavyaṃ ca savyāpasavye dakṣiṇavāmabhāgau tayoḥ
savyāpasavyataḥ. saptamyarthe tasiḥ. bhrāmayed bhramantaṃ prerayet. siddhair eṣā nauliḥ
pracakṣyate kathyate.
%\end{vsid}


%\begin{vsid}{#hp02_035}
\startsloka
{\bf mandāgnisandīpanapācanādi-
sandhāyikānandakarī sadaiva
aśeṣadoṣāmayaśoṣaṇī ca 
haṭhakriyāmaulir iyaṃ ca nauliḥ (2.34)}
\stopsloka

nauliguṇān āha mandāgnīti. mandaś cāsāv agniś ca jaṭharāgnis tasya sandīpanaṃ samyag dīpanaṃ ca
pācanaṃ ca bhuktānnaparipākaś ca mandāgnisandīpanapācane te ādinī yasya tan
mandāgnisandīpanapācanādi tasya sandhāyikā vidhātrī. ādiśabdena malaśuddhyādi. sadaiva
sarvadaivānandakarī sukhakarī. aśeṣāḥ samastāś ca te doṣāś ca vātādaya āmayāś ca rogās teṣāṃ śoṣaṇī
śoṣaṇakartrī. haṭhasya kriyāṇāṃ dhautyādīnāṃ maulir maulir ivottamā dhautibastyor
naulisāpekṣatvāt. iyam uktā nauliḥ.
%\end{vsid}

%\begin{vsid}{#hp02_036}
\startsloka
{\bf atha kapālabhātiḥ
bhastrāval lohakārasya recapūrau sasambhramau
kapālabhātir vikhyātā kaphadoṣaviśoṣaṇī (2.35)}
\stopsloka

atha kapālabhātiṃ tadguṇaṃ ca āha bhastrāvad iti. lohakārasya bhastrā agner dhamanasādhanībhūtaṃ
carma tadvat sambhrameṇa saha vartamānau sasambhramāv amandau yau recapūrau recakapūrakau
kapālabhātir iti vikhyātā. kīdṛśī kaphadoṣaviśoṣaṇī. kaphasya doṣā viṃśatibhedabhinnāḥ. tad uktaṃ
nidāne kapharogāś ca viṃśatir iti. teṣāṃ viśoṣaṇī vināśinī.
%\end{vsid}



%\begin{vsid}{#hp02_037}
\startsloka
{\bf ṣaṭkarmanirgatasthaulyakaphadoṣamalādikaḥ
prāṇāyāmaṃ tataḥ kuryād anāyāsena siddhyati (2.36)}
\stopsloka

ṣaṭkarmaṇāṃ prāṇāyāmopakārakatvam āha ṣaṭkarmeti. ṣaṭkarmabhir dhautiprabhṛtibhiḥr nirgatāḥ.
sthaulyaṃ sthūlasya bhāvaḥ sthūlatvam. kaphadoṣā viṃśatisaṅkhyākāḥ malādayaś ca yasya sa tathā.
śeṣād vibhāṣā\comment{Aṣṭādhyāyī 5.4.154.} iti kapratyayaḥ. ādiśabdena pittādayaḥ. prāṇāyāmaṃm
kuryāt. tatas tasmāt ṣaṭkarmapūrvakāt prāṇāyāmād anāyāsenāśrameṇa siddhyati yoga iti
śeṣaḥ. ṣaṭkarmākaraṇe tu prāṇāyāme śramādhikyaṃ syād iti bhāvaḥ.
%\end{vsid}


%\begin{vsid}{#hp02_038}
\startsloka
{\bf prāṇāyāmair eva sarve praśuṣyanti malā iti
ācāryāṇāṃ tu keṣāñcid anyat karma na samṃmatam (2.37)}
\stopsloka

matabhedena ṣaṭkarmaṇām anupayogam āha prāṇāyāmair iti. prāṇāyāmair eva. evaśabdaḥ
ṣaṭkarmavyavacchedārthaḥ. sarve malāḥ praśuṣyanti. malā ity upalakṣaṇaṃ sthaulyakaphapittādīnām.
iti hetoḥ keṣāñcid ācāryāṇāṃ yājñavalkyādīnām anyat karma ṣaṭkarma na samṃmatam nābhimatam.
ācāryalakṣaṇam uktaṃ vāyupurāṇe

\startsloka
ācinoti ca śāstrārtham ācāre sthāpayed api
svayam ācarate yasmād ācāryas tena cocyate iti
\stopsloka
%\end{vsid}


%\begin{vsid}{#hp02_29}
\startsloka
{\bf atha gajakaraṇī
udaragatapadārtham udvamanti 
pavanam apānam udīrya kaṇṭhanāle
kramaparicayavaśyanāḍicakrā
gajakaraṇīti nigadyate haṭhajñaiḥ (2.38)}
\stopsloka

gajakaraṇīm āha udaragateti. apānaṃ pavanam apānavāyuṃ kaṇṭhanāle kaṇṭho nāla iva kaṇṭhanālas
tasminn udīryotkṣipyodare gataḥ prāptaḥ sa cāsau padārthaś ca bhuktapītānnajalādis taṃ yayodvamanti
udgiranti yayā. yogina ity adhyāhāraḥ. krameṇa yaḥ paricayo 'bhyāsas tena vaśyaṃ svādhīnaṃ nāḍīnāṃ
cakraṃ yasyāṃ sā tathā. sā kriyā haṭhajñair haṭhayogābhijñair gajakaraṇīti nigadyate
kathyate. kramaparicayavaśyanāḍimārgā iti kvacit pāṭhaḥ. tasyāyam arthaḥ kramaparicayena vaśyo
nāḍyāḥ śaṅkhinyāḥ mārgaḥ kaṇṭhaparyanto yasyāṃ sā tathā.
%\end{vsid}


%\begin{vsid}{#hp02_039}
\startsloka
{\bf brahmādayo 'pi tridaśāḥ pavanābhyāsatatparāḥ
abhūvann antakabhayāt tasmāt pavanam abhyaset (2.39)}
\stopsloka

prāṇāyāmo 'vaśyam abhyasanīyaḥ sarvottamair abhyastatvān mahāphalatvāc ceti sūcayann āha caturbhiḥ
brahmādaya iti. brahmā ādir yeṣāṃ te brahmādayas te 'pi kim utānye ity arthaḥ. tridaśāḥ devāḥ.
antayatīty antakaḥ kālas tasmād bhayam antakabhayaṃ tasmāt pavanasya prāṇavāyoḥ yo 'abhyāsaḥ
recakapūrakakumbhakabhedabhinnaprāṇāyāmānuṣṭhānarūpas tasmin tatparāḥ avahitā abhūvann āsan.
tasmāt pavanam abhyaset prāṇam abhyaset.
%\end{vsid}


%\begin{vsid}{#hp02_040}
\startsloka
{\bf yāvad baddho marud dehe yāvac cittaṃ nirākulam
yāvad dṛṣṭir bhruvor madhye tāvat kālabhayaṃ kutaḥ (2.40)}
\stopsloka

yāvad iti. yāvad yāvatkālaparyantaṃ marut prāṇānilo dehe śarīre baddhaḥ śvāsocchvāsakriyāśūnyaḥ.
yāvac cittam antaḥkaraṇaṃ nirākulam avikṣiptaṃ samāhitam. yāvad bhruvor madhye dṛṣṭir
antaḥkaraṇavṛttiḥ dṛśir atra jñānasāmānyārthaḥ. tāvat tāvatkālaparyantaṃ kalayatīti kālo 'ntakas
tasmād bhayaṃ kutaḥ. na kuto 'pīty arthaḥ. tathā ca vakṣyati

\startsloka
khādyate na ca kālena bādhyate na ca karmaṇā
sādhyate na sa kenāpi yogī yuktaḥ samādhinā\comment{HP 4.108.} 
\stopsloka

iti svādhīnamṛtyur bhavatīty arthaḥ.
%\end{vsid}


%\begin{vsid}{#hp02_041}
\startsloka
{\bf vidhivat prāṇasaṃyāmair nāḍīcakre viśodhite
suṣumnṇāvadanaṃ bhittvā sukhād viśati mārutaḥ (2.41)}
\stopsloka

vidhivad iti. vidhivat prāṇasaṃyāmair āsanajālandharabandhādividhiyuktaprāṇāyāmair nāḍīcakre
nāḍīnāṃ cakraṃ samūhas tasmin viśodhite viśeṣeṇa śodhite nirmale sati māruto vāyuḥ suṣumṇā
iḍāpiṅgalayor madhyasthā nāḍī tasyā vadanaṃ mukhaṃ bhittvā sukhād anāyāsād viśati suṣumṇnāntar iti
śeṣaḥ.
%\end{vsid}


%\begin{vsid}{#hp02_042}
\startsloka
{\bf mārute madhyasañcāre manaḥsthairyaṃ prajāyate
yo manaḥsusthirībhāvaḥ saivāvasthā manonmanī (2.42)}
\stopsloka

māruta iti. mārute prāṇavāyau madhye suṣumṇnāmadhye sañcāraḥ samyak caraṇaṃ gamanaṃ mūrdhaparyantaṃ
yasya sa madhyasañcāras tasmin sati manasaḥ sthairyaṃ dhyeyākāravṛttipravāho jāyate
prādurbhavati. yo manasaḥ susthirībhāvaḥ suṣṭhu sthirībhavanaṃ saiva manonmanyavasthā.
manonmanīśabda unmanīparyāyaḥ. tathāgre vakṣyati rājayogaḥ samādhiś ca\comment{HP 4.3} ity ādinā.
%\end{vsid}


%\begin{vsid}{#hp02_043}
\startsloka
{\bf tatsiddhaye vidhānajñāś citrān kurvanti kumbhakān
vicitrakumbhakābhyāsād vicitrāṃ siddhim āpnuyāt (2.43)}
\stopsloka

vicitreṣu kumbhakeṣu pravṛttiṃ janayituṃ teṣāṃ mukhyaphalam avāntaraphalaṃ cāha tatsiddhaya iti.
vidhānaṃ kumbhakānuṣṭhānaprakāras taj jānantīti vidhānajñās tatsiddhaye unmanyavasthāsiddhaye
citrān sūryabhedanādibhedena nānāvidhān kumbhakān kurvanti. vicitrāś ca te kumbhakāś ca
vicitrakumbhakās teṣām abhyāsād anuṣṭhānād vicitrām aṇimādibhedena nānāvidhāṃ vilakṣaṇāṃ vā
janmauṣadhimantratapojātābhyāṃ siddhim. tad uktaṃ bhāgavate

\startsloka
janmauṣadhitapomantrair yāvatya iha siddhayaḥ
yogenāpnoti tāḥ sarvā nānyair yogagatiṃ vrajet\comment{Bhāgavatapurāṇa 11.15.34.} 
\stopsloka

iti. āpnuyāt pratyāhārādiparamparayeti bhāvaḥ.
%\end{vsid}


%\begin{vsid}{#hp02_044}
\startsloka
{\bf atha kumbhakabhedāḥ
sūryabhedanam ujjāyī sītkārī śītalī tathā
bhastrikā bhrāmarī mūrcchā plāvinīty aṣṭakumbhakāḥ (2.44)}
\stopsloka

athāṣṭakumbhakān nāmabhir nirdiśati sūryabhedanam iti spaṣṭam.
%\end{vsid}




%\begin{vsid}{#hp02_045}
\startsloka
{\bf pūrakānte tu kartavyo bandho jālandharābhidhaḥ
kumbhakānte recakādau kartavyas tūḍḍiyānakaḥ (2.45)}
\stopsloka

atha haṭhasiddhāv ananyathāsiddhāṃ pāramarahasyāṃ sarvakumbhakasādhāraṇayuktim āha tribhiḥ
pūrakānta iti. jālandhara ity abhidhā nāma yasya sa jālandharābhidho bandho badhnāti prāṇavāyum iti
bandhaḥ. kaṇṭhākuñcanapūrvakaṃ cibukasya hṛdi sthāpanaṃ jālandharabandhaḥ pūrakānte pūrakasyānte
pūrakānantaraṃ jhaṭiti kartavyaḥ. tuśabdāt kumbhakādau. uḍḍiyānakas tu kumbhakānte
kumbhakasyānte kiñcit kumbhakaśeṣe recakasyādau recakādau recakāt pūrvaṃ kartavyaḥ.
prayatnaviśeṣeṇa nābhipradeśasya pṛṣṭhata ākarṣaṇam uḍḍiyānabandhaḥ.
%\end{vsid}


%\begin{vsid}{#hp02_046}
\startsloka
{\bf adhastāt kuñcanenāśu kaṇṭhasaṅkocane kṛte
madhye paścimatānena syāt prāṇo brahmanāḍigaḥ (2.46)}
\stopsloka

adhastād iti. kaṇṭhasya saṅkocanaṃ kaṇṭhasaṅkocanaṃ tasmin kṛte sati jālandharabandhe kṛte satīty
arthaḥ. āśv avyavahitottaram evādhastād adhaḥpradeśād ākuñcanena ādhārākuñcanena mūlabandhenety
arthaḥ.  madhye nābhipradeśe paścimataḥ pṛṣṭhatas tānaṃ tānanam ākarṣaṇaṃ tenoḍḍiyānabandhenety
arthaḥ. uktarītyā kṛtena bandhatrayeṇa prāṇo vāyur brahmanāḍiṃ suṣumnṇāṃ gacchatīti brahmanāḍigaḥ
suṣumṇnānāḍigāmī syād ity arthaḥ.

atredaṃ rahasyaṃ yadi śrīgurumukhāj jihvābandhaḥ samyak parijñātas tarhi jihvābandhapūrvakena
jālandharabandhenaiva prāṇāyāmaḥ siddhyati. vāyuprakopo naiva bhavati.  vapuḥkṛśatvaṃ vadane
prasannatā\comment{HP 2.78} ityādīni sarvāṇi lakṣaṇāni jāyanta iti mūlabandhoḍḍiyānabandhau
nopayuktau. tayor jihvābandhapūrvakeṇa jālandharabandhenānyathāsiddhatvāt. jihvābandho na viditaś
ced adhastāt kuñcanena iti ślokoktarītyā prāṇāyāmāḥ kartavyāḥ. trayo 'pi bandhā gurumukhāj
jñātavyāḥ. mūlabandhas tvasamyagjñāto nānārogotpādakaḥ. tathā hi yadi mūlabandhe kṛte dhātukṣayo
viṣṭambho 'gnimāndyaṃ nādamandyaṃ guṭikāsamūhākāram ajasyeva purīṣaṃ syāt tadā mūlabandhaḥ samyag
na jñāta iti bodhyam. yadi tu dhātupuṣṭiḥ samyag malaśuddhir agnidīptiḥ samyag nādābhivyaktiś ca
syāt tadā jñeyaṃ tadā mūlabandhaḥ samyag jñāta iti.
%\end{vsid}



%\begin{vsid}{#hp02_047}
\startsloka
{\bf āpānam ūrdhvam utthāpya prāṇaṃ kaṇṭhād adho nayet
yogī jarāvimuktaḥ san ṣoḍaśābdavayā bhavet (2.47)}
\stopsloka

āpānam iti. āpānam āpānavāyum ūrdhvam utthāpya ādhārākuñcanena prāṇaṃ prāṇavāyuṃ kaṇṭhād adhaḥ
adhobhāge nayet yogī prāpayed. yaḥ sa yogī yogo 'syāsti abhyasyatveneti yogī yogābhyāsī jarayā
vārdhakyena vimukto viśeṣeṇa muktaḥ san ṣoḍaśānām abdānāṃ samāhāraḥ ṣoḍaśābdam ṣoḍaśābdaṃ vayo
yasya sa tādṛśo bhavet. yady api pūrakānte tu kartavyaḥ\comment{HP 2.45.} ity ādinā trayāṇāṃ
ślokānām eka evārthaḥ paryavasyati tathāpi pūrakānte tu kartavyaḥ ity anena bandhānāṃ kāla
uktaḥ. adhastāt kuñcanenety anena bandhānāṃ svarūpam uktam. apānam ūrdhvam utthāpyety anena
bandhānāṃ phalam uktam iti viśeṣaḥ. jālandharabandhe mūlabandhe ca kṛte uḍḍiyānabandho bhavaty
evety asmin śloke-noktaḥ. tathā coktaṃ jñāneśvareṇa gītāṣaṣṭhādhyāyavyākhyāyāṃ mūlabandhe
jālandharabandhe ca kṛte nābher adhobhāga ākarṣaṇākhyo bandhaḥ svayam eva
bhavati\comment{Jñāneśvarī 6.200?} iti.
%\end{vsid}



%\begin{vsid}{#hp02_048}
\startsloka
{\bf atha sūryabhedanam
āsane sukhade yogī baddhvā caivāsanaṃ tataḥ
dakṣanāḍyā samākṛṣya bahiḥsthaṃ pavanaṃ śanaiḥ (2.48)}
\stopsloka

\startsloka
yogābhyāsakramaṃ vakṣye yogināṃ yogasiddhaye
uṣaḥkāle samutthāya prātaḥkāle 'thavā budhaḥ
guruṃ saṃsmṛtya śirasi hṛdaye sveṣṭadevatām
śaucaṃ kṛtvā dantaśuddhiṃ vidadhyād bhasmadhāraṇam
śucau deśe maṭhe ramye pratiṣṭhāpyāsanaṃ mṛdu 
tatropaviśya saṃsmṛtya manasā gurum īśvaram
deśakālau ca saṅkīrtya saṅkalpavidhipūrvakam
\stopsloka

adyetyādi śrīparameśvaraprasādapūrvakam samādhitatphalasiddhyartham āsanapūrvakān prāṇāyāmādīn kariṣye

\startsloka
  anantaṃ praṇamed devaṃ nāgeśaṃ pīṭhasiddhaye
\stopsloka
  
maṇibhrājatphaṇāsahasravidhṛtaviśvambharāmaṇḍalāyānantāya nāgarājāya namaḥ

\startsloka
tato 'bhyased āsanāni śrame jāte śavāsanam
ante samabhyaset tat tu śramābhāve tu nābhyaset
karaṇīṃ viparītākhyāṃ kumbhakān pūrvam abhyaset
jālandharasya dārḍhyārthaṃ kumbhakeṣūpayogataḥ
vidhāyācamanaṃ kṛtvā karmāṅgaṃ prāṇasaṃyamam 
yogīndrādīn namaskṛtya kaurmāc ca śivavākyataḥ
\stopsloka


kūrmapurāṇe śivavākyam

\startsloka
namaskṛtyātha yogīndrān saśiṣyāṃś ca vināyakam
guruṃ caivātha māṃ yogī yuñjīta susamāhitaḥ
baddhvābhyāse siddhapīṭhaṃ kumbhakān bandhapūrvakam
prathame daśa kartavyāḥ pañcavṛddhyā dine dine
kāryā aśītiparyantaṃ kumbhakāḥ susamāhitaiḥ
yogīndraiḥ prathamaṃ kuryād abhyāsaṃ candrasūryayoḥ
anulomavilomākhyam etaṃ prāhur manīṣiṇaḥ
sūryabhedanam abhyasya bandhapūrvakam ekadhīḥ
ujjāyinaṃ tataḥ kuryāt sītkārīṃ śītalīṃ tataḥ 
bhastrikāṃ ca samabhyasya kuryād anyān na vā parān
mudrāḥ samabhyased buddhā guruvaktrād yathākramam
tataḥ padmāsanaṃ baddhvā kuryān nādānucintanam
abhyāsaṃ sakalaṃ kuryād īśvarārpaṇam ādṛtaḥ
abhyāsād utthitaḥ snānaṃ kuryād uṣṇena vāriṇā
snātvā samāpayen nityaṃ karma saṅkṣepataḥ sudhīḥ
madhyāhne 'pi tathābhyasya kiñcid viśramya bhojanam
kurvīta yogināṃ pathyam apathyaṃ na kadācana
elāṃ vāpi lavaṅgaṃ vā bhojanānte ca bhakṣayet
kecit karpūram icchanti tāmbūlaṃ śobhanaṃ tathā
cūrṇena rahitaṃ śastaṃ pavanābhyāsayoginām
iti cintāmaṇer vākyaṃ svārasyaṃ bhajate na hi
kecit padena yasmāt tu tayoḥ śītauṣṇyahetutā
bhojanānantaraṃ kuryān mokṣaśāstrāvalokanam
purāṇaśravaṇaṃ vāpi nāmasaṃkīrtanaṃ vibhoḥ
sāyaṃ sandhyāvidhiṃ kṛtvā yogaṃ pūrvavad abhyaset
yadā trighaṭikāśeṣo divaso 'bhyāsam ācaret
abhyāsānantaraṃ kāryā sāyaṃsandhyā tadā budhaiḥ
ardharātre haṭhābhyāsaṃ vidadhyāt pūrvavad yamī
viparītāṃ tu karaṇīṃ sāyaṃkālārdharātrayoḥ
nābhyased bhojanād ūrdhvaṃ yataḥ sā na praśasyate
\stopsloka

athoddeśānukrameṇa kumbhakān vivakṣus tatra prathamoditaṃ sūryabhedanaṃ tadguṇāṃś cāha tribhiḥ
āsana iti. sukhaṃ dadātīti sukhadaṃ tasmin sukhade.

\startsloka
śucau deśe pratiṣṭhāpya sthiram āsanam ātmanaḥ
nātyucchritaṃ nātinīcaṃ cailājinakuśottaram\comment{Bhagavadgītā 6.11}
\stopsloka

ity uktalakṣaṇe viviktadeśe ca sukhāsanasthaḥ śuciḥ samagrīvaśiraḥśarīraḥ\comment{kaivalyopaniṣat
  5.} iti śruteś ca. cailājinakuśottara āsane. āste 'sminn ity āsanam āsyate 'neneti vā tasmin yogī
yogābhyāsī.  āsanaṃ svastikavīrasiddhapadmādyanyatamaṃ mukhyatvāt siddhāsanam eva vā baddhvaiva
bandhanena sampādyaiva kṛtvaivety arthaḥ. tata āsanabandhānantaraṃ dakṣā dakṣiṇabhāgasthā yā nāḍī
piṅgalā tayā bahiḥsthaṃ dehād bahir vartamānaṃ pavanaṃ vāyuṃ śanair mandamandam ākṛṣya piṅgalayā
mandamandaṃ pūrakaṃ kṛtvety arthaḥ.
%\end{vsid}



%\begin{vsid}{#hp02_049}
\startsloka
{\bf ākeśād ānakhāgrāc ca nirodhāvadhi kumbhayet
tataḥ śanaiḥ savyanāḍyā recayet pavanaṃ śanaiḥ (2.49)}
\stopsloka


ākeśād iti. keśān maryādīkṛtyety ā keśaṃ tasmāt. nakhāgrān maryādī\-kṛtyety ānakhāgraṃ tasmāc
ca. nirodhasya vāyor avarodhasya avadhir maryādā yasmin karmaṇi tat tathā kumbhayet.  keśaparyantaṃ
nakhāgraparyantaṃ ca vāyor nirodho yathā bhavet tathātiprayatnena kumbhakaṃ kuryād ity arthaḥ.

nanu

\startsloka
haṭhān niruddhaḥ prāṇo 'yaṃ romakūpeṣu niḥsaret
dehaṃ vidārayaty eṣa kuṣṭhādi janayaty api
tataḥ pratyāyitavyo 'sau krameṇāraṇyahastivat
vanyo gajo gajārir vā krameṇa mṛdutām iyāt
karoti śāstṛnirdeśān na ca taṃ parilaṅghayet
tathā prāṇo hṛdistho'yaṃ yogināṃ kramayogataḥ
gṛhītaḥ sevyamānas tu viśrambham upagacchati\comment{Quoted in YCM.}    
\stopsloka

iti vākyaviruddham atiprayatnena kumbhakaṃ kuryād iti katham uktam iti cen na. haṭhān niruddhaḥ
prāṇo 'yaṃ itivākyasya balād acireṇa prāṇajayaṃ kariṣyāmīti buddhyārambhaḥ eva. evaṃ ca
bahvabhyāsāsaktaparatvāt krameṇāraṇyahastivad iti dṛṣṭāntasvārasyāc ca. ata eva sūryācandramasor
abhyāse dhārayitvā yathāśakti\comment{HP 3.21.} nidhārayed atirodhata\comment{HP 2.9.} iti coktaṃ
saṅgacchate.  tasmāt kumbhakas tv atiprayatnapūrvakaṃ kartavyaḥ. yathā yathātiyatnena kumbhakaḥ
kriyate tathā tathā tasmin guṇādhikyaṃ bhavet. yathā yathā ca śithilaḥ kumbhakaḥ syāt tathā tathā
guṇālpatvaṃ syāt. atra yoginām anubhavo 'pi mānam. pūrakas tu śanaiḥ śanaiḥ kāryaḥ vegād vā
kartavyaḥ.  vegād api kṛte pūrake doṣābhāvāt. recakas tu śanaiḥ śanair eva kartavyaḥ. vegāt kṛte
recake balahāniprasaṅgāt. tataḥ śanaiḥ śanair eva recayen na tu vegata ityādy anekadhā
granthakārokteś ca. tato nirodhāvadhi kumbhakānantaraṃ śanaiḥ śanair mandaṃ mandaṃ savye vāmabhāge
sthitā nāḍī savyanāḍī tayā savyanāḍyā iḍayā pavanaṃ vāyuṃ recayed bahir niḥsārayet. punaḥ śanair
ity uktas tu śanair eva recayed ity avadhāraṇārthā. tad uktaṃ 

\startsloka
{\bf vismaye ca viṣāde ca dainye caivāvadhāraṇe
tathā prasādane harṣe vākyam ekaṃ dvir ucyate iti}
\stopsloka
%\end{vsid}

%\begin{vsid}{#hp02_050}
\startsloka
{\bf kapālaśodhanaṃ vātadoṣaghnaṃ kṛmidoṣahṛt
punaḥ punar idaṃ kāryaṃ sūryabhedanam uttamam (2.50)}
\stopsloka

kapālaśodhanam iti. kapālasya mastakasya śodhanaṃ śuddhikaraṃ vātajā doṣā vātadoṣā aśītiprakārās
tān hantīti vātadoṣaghnam. kṛmīṇām udare jātānāṃ doṣo vikāras taṃ haratīti kṛmidoṣahṛt. punaḥ punar
bhūyo bhūyaḥ kāryam. sūryeṇāpūrya kumbhayitvā candreṇa recanam iti rītyedam uttamam utkṛṣṭaṃ
sūryabhedanaṃ sūryabhedanākhyam uktaṃ yogibhir iti śeṣaḥ.
%\end{vsid}


%\begin{vsid}{#hp02_051}
\startsloka
{\bf atha ujjāyī
mukhaṃ saṃyamya nāḍībhyām ākṛṣya pavanaṃ śanaiḥ
yathā lagati kaṇṭhāt tu hṛdayāvadhi sasvanam (2.51)}
\stopsloka

ujjāyinam āha sārdhena mukham iti. mukham āsyaṃ saṃyamya saṃyataṃ kṛtvā mudrayitvety
arthaḥ. kaṇṭhāt tu kaṇṭhād ārabhya hṛdayāvadhi hṛdayam avadhir yasmin karmaṇi tat tathā svanena
sahitaṃ yathā syāt tathā. ubhe kriyāviśeṣaṇe. lagati śliṣyati pavana ity arthāt. tathā tena
prakāreṇa nāḍībhyām iḍāpiṅgalābhyāṃ pavanaṃ vāyuṃ śanair mandam ākṛṣya ākṛṣṭaṃ kṛtvā pūrayitvety
arthaḥ.
%\end{vsid}



%\begin{vsid}{#hp02_052}
\startsloka
{\bf pūrvavat kumbhayet prāṇaṃ recayed iḍayā tataḥ
śleṣmadoṣaharaṃ kaṇṭhe dehānalavivardhanam (2.52)}
\stopsloka

pūrvavad iti. prāṇaṃ pūrvavat pūrveṇa sūryabhedanena tulyaṃ pūrvavat. ākeśād ānakhāgrāc ca
nirodhāvadhi kumbhayed ityuktarītyā kumbhayed rodhayet. tataḥ kumbhakānantaram iḍayā vāmanāḍyā
recayet tyajet.  prāṇaṃ pūrvavat pūrveṇa sūryabhedanena tulyaṃ pūrvavat. ākeśād ānakhāgrāc ca
nirodhāvadhi kumbhayed ityuktarītyā kumbhayed rodhayet. tataḥ kumbhakānantaram iḍayā vāmanāḍyā
recayet tyajet.

ujjāyiguṇān āha sārdhaślokena śleṣmadoṣaharam iti. kaṇṭhe kaṇṭhapradeśe śleṣmaṇo doṣāḥ śleṣmadoṣāḥ
kāsādayas tān haratīti śleṣmadoṣaharas taṃ dehānalasya dehamadhyagatānalasya jāṭharasya vivardhanaṃ
viśeṣeṇa vardhanaṃ dīpanam ity arthaḥ.
%\end{vsid}

%\begin{vsid}{#hp02_053}
\startsloka
{\bf nāḍījalodarādhātugatadoṣavināśanam
gacchatā tiṣṭhatā kāryam ujjāyyākhyaṃ tu kumbhakam (2.53)}
\stopsloka

nāḍīti. nāḍī śirā jalaṃ pītam udakam udaraṃ tundam āsamantād dehe vartamānā dhātava ādhātavaḥ.
eṣām itaretaradvandvaḥ. teṣu gataḥ prāpto yo doṣo vikāras taṃ viśeṣeṇa nāśayatīti
nāḍījalodarādhātugatadoṣavināśanam. gacchatā gamanaṃ kurvatā tiṣṭhatā sthitena vāpi puṃsā
ujjāyyākhyam ujjāyīty ākhyā yasya tat. tu ity anena asya vaiśiṣṭyaṃ dyotayati. kāryaṃ kartavyam.
udyāyīti kvacit pāṭhaḥ. gacchatā tiṣṭhatā tu bandharahitaḥ kartavyaḥ. kumbhakaśabdas
triliṅgaḥ. puṃlliṅgapāṭhe tu viśeṣaṇeṣv api puṃlliṅgaḥ pāṭhaḥ kāryaḥ.
%\end{vsid}


%\begin{vsid}{#hp02_054}
\startsloka
{\bf atha sītkārī
sītkāṃ kuryāt tathā vaktre ghrāṇenaiva vijṛmbhikām
evam abhyāsayogena kāmadevo dvitīyakaḥ (2.54)}
\stopsloka

sītkārīkumbhakam āha sītkām iti. vaktre mukhe sītkāṃ sīd eva sītkā sīd iti śabdaḥ sītkāras tāṃ
kuryāt. oṣṭhayor antare saṃlagnayā jihvayā sītkārapūrvakaṃ mukhena pūrakaṃ kuryād ity arthaḥ.
ghrāṇenaiva nāsikayaivety anenobhābhyāṃ nāsāpuṭābhyāṃ recakaḥ kāryam ity uktam. evaśabdena
vaktrasya vyavacchedaḥ. vaktreṇa vāyor niḥsāraṇaṃ tv abhyāsānantaram api na kāryam.
balahānikaratvāt. vijṛmbhikāṃ recakaṃ kuryād ity atrāpi sambadhyate. kumbhakas tv anukto 'pi
sītkāryāḥ kumbhakatvād evāvagantavyaḥ. atha sītkāryāḥ praśaṃsā. evam uktaprakāreṇābhyāsaḥ
paunaḥpunyenānuṣṭhānaṃ sa eva yogaḥ yogasādhanatvāt tena dvitīya eva dvitīyakaḥ kāmadevaḥ
kandarpaḥ. rūpalāvaṇyātiśayena kāmadevasādṛśyāt.
%\end{vsid}


%\begin{vsid}{#hp02_055}
\startsloka
{\bf yoginīcakrasaṃmānyaḥ sṛṣṭisaṃhārakārakaḥ
na kṣudhā na tṛṣā nidrā naivālasyaṃ prajāyate (2.55)}
\stopsloka

yoginīnāṃ cakraṃ yoginīcakraṃ yoginīsamūhaḥ. tasya samṃmānyaḥ saṃsevyaḥ. sṛṣṭiḥ prapañcotpattiḥ
saṃhāras tallayaḥ tayoḥ kārakaḥ kartā. kṣudhā bhoktum icchā na. tṛṣā jalapānecchā na. nidrā
suṣuptir na. ālasyaṃ kāyacittagauravāt pravṛttyabhāvaḥ. kāyagauravaṃ kaphādinā cittagauravaṃ
tamoguṇena. naiva prajāyate naiva prādurbhavati. evam abhyāsayogeneti prajāyata iti ca prativākyaṃ
sambadhyate.
%\end{vsid}


%\begin{vsid}{#hp02_056}
\startsloka
{\bf bhavet sattvaṃ ca dehasya sarvopadravavarjitaḥ
anena vidhinā satyaṃ yogīndro bhūmimaṇḍale (2.56)}
\stopsloka

bhaved iti. dehasya śarīrasya sattvaṃ balaṃ ca bhavet. anenoktena vidhinābhyāsavidhinā yogīndro
yoginām indra iva yogīndro bhūmimaṇḍale sarvair upadravair varjitaḥ sarvopadravavarjito bhavet
satyam. sarvaṃ vākyaṃ sāvadhāraṇam iti nyāyāt. yad uktaṃ phalaṃ tat satyam evety arthaḥ.
%\end{vsid}


%\begin{vsid}{#hp02_058}
\startsloka
{\bf atha śītalī
jihvayā vāyum ākṛṣya pūrvavat kumbhasādhanam
śanakair ghrāṇarandhrābhyāṃ recayet pavanaṃ sudhīḥ (2.57)}
\stopsloka

śītalīkumbhakam āha jihvayeti. jihvayā oṣṭhayor bahir nirgatayā vihaṅgamādharacañcusadṛśayā vāyum
ākṛṣya śanaiḥ pūrakaṃ kṛtvety arthaḥ. pūrvavat sūryabhedanavat kumbhasya kumbhakasya sādhanaṃm
vidhānaṃ kṛtvety adhyāhāraḥ. sudhīḥ śobhanā dhīr yasya saḥ. ghrāṇasya randhre ghrāṇarandhre tābhyāṃ
nāsāpuṭavivarābhyāṃ śanakaiḥ śanair eva. avyayasarvanāmnām\comment{Aṣṭādhyāyī 5.3.71.} ity
akac. pavanaṃ vāyum recayet.
%\end{vsid}



%\begin{vsid}{#hp02_059}
\startsloka
{\bf gulmaplīhādikān rogān jvaraṃ pittaṃ kṣudhāṃ tṛṣām
viṣāṇi śītalī nāma kumbhikeyaṃ nihanti hi (2.58)}
\stopsloka

śītalīguṇān āha gulmeti. gulmaś ca plīhaś ca gulmaplīhau rogaviśeṣāv ādī yeṣāsāṃ te gulmaplīhādikās
tān rogān āmayān jvaraṃ jvarākhyaṃ rogam pittaṃ pittavikāram kṣudhāṃ bhoktum icchām tṛṣām
jalapānecchām viṣāṇi sarpādiviṣajanitavikārān śītalī nāmeti prasiddhārthakam avyayam iyam uktā
kumbhikā nihanti nitarāṃ hanti. kumbhaśabdaḥ strīliṅgo 'pi. tathā ca śrīharṣaḥ

\startsloka
udasya kumbhīr athaha śātakumbhajāḥ\comment{Naiṣadhīyacarita 5.19.}
\stopsloka

iti svārthe kapratyaye kumbhī eva kumbhikā iti.
%\end{vsid}


%\begin{vsid}{#hp02_060}
\startsloka
{\bf atha bhastrikā
ūrvor upari saṃsthāpya śubhe pādatale ubhe
padmāsanaṃ bhaved etat sarvapāpapraṇāśanam (2.59)}
\stopsloka

bhastrākumbhakasya padmāsanapūrvakam evānuṣṭhānāt tadādau padmāsanam āha ūrvor iti. upary uttāne
śubhe śuddhe ubhe dve pādayos tale 'dhaḥpradeśe ūrvoḥ saṃsthāpya samyak sthāpayitvā vaset. etat
padmāsanaṃ bhavet. kīdṛśam sarveṣāṃ pāpānāṃ prakarṣeṇa nāśanam. atroparīty avyayam
uttānavācakam. tathā ca kārakeṣu manoramāyāṃ upary upari buddhīnām\comment{Prauḍhamanoramā (KSS),
  p. 506.} ity atroparibuddhīnām ity asyottānabuddhīnām iti vyākhyānaṃ kṛtam.
%\end{vsid}


%\begin{vsid}{#hp02_061}
\startsloka
{\bf samyak padmāsanaṃ baddhvā samagrīvodaraḥ sudhīḥ
mukhaṃ saṃyamya yatnena prāṇaṃ ghrāṇena recayet (2.60)}
\stopsloka

bhastrikākumbhakam āha samyag iti. grīvā ca udaraṃ ca grīvodaram. prāṇyaṅgatvād ekavadbhāvaḥ.
samaṃ grīvodaraṃ yasya sa samagrīvodaraḥ. susthitā dhīr yasya sa sudhīḥ. padmāsanaṃ samyak sthiraṃ
baddhvā mukhaṃ saṃyamya saṃyataṃ kṛtvā yatnena prayatnena ghrāṇena ghrāṇasyaikatareṇa randhreṇa
prāṇaṃ śarīrāntaḥsthitaṃ vāyuṃ recayet.  % !! Kumbhakapaddhati Nils
%\end{vsid}


%\begin{vsid}{#hp02_062}
\startsloka
{\bf yathā lagati hṛtkaṇṭhe kapālāvadhi sasvanam
vegena pūrayec cāpi hṛtpadmāvadhi mārutam (2.61)}
\stopsloka

recakaprakāram āha yatheti. hṛc ca kaṇṭhaś ca hṛtkaṇṭhaṃ tasmin hṛtkaṇṭhe. samāhāradvandvaḥ.
kapālāvadhi kapālaparyantaṃ svanena sahitaṃ sasvanaṃm yathā syāt tathā yena prakāreṇa lagati. prāṇa
iti śeṣaḥ. tathā recayet. hṛtpadmam avadhir yasmin karmaṇi tat hṛtpadmāvadhi vegena tarasā
mārutaṃ vāyuṃ pūrayet. cāpīti pādapūraṇārtham.
%\end{vsid}

%\begin{vsid}{#hp02_063}
\startsloka
{\bf punar virecayet tadvat pūrayec ca punaḥ punaḥ
yathaiva lohakāreṇa bhastrā vegena cālyate (2.62)}
\stopsloka

punar iti. tadvat pūrvavat punar virecayet punaḥ punaḥ pūrayec cety anvayaḥ. ukte 'rthe dṛṣṭāntam
āha yathaiveti. lohakāreṇa lohavikārāṇāṃ kartrā bhastrā agner dhamanasādhanībhūtaṃ carma yathaiva
yenaiva prakāreṇa vegena cālyate.
%\end{vsid}


%\begin{vsid}{#hp02_064}
\startsloka
{\bf tathaiva svaśarīrasthaṃ cālayet pavanaṃ dhiyā
yadā śramo bhaved dehe tadā sūryeṇa pūrayet (2.63)}
\stopsloka

tathaiveti. tathaiva tenaiva prakāreṇa svaśarīrasthaṃ svaśarīre sthitaṃ pavanaṃ prāṇaṃ dhiyā
buddhyā cālayet. recakapūrakayor nirantarāvartanena cālanasyāvadhim āha yadā śrama iti. yadā yasmin
kāle dehe śarīre śramo recakapūrakayor nirantarāvartanenāyāso bhavet tadā tasmin kāle sūryeṇa
sūryanāḍyā pūrayet.
%\end{vsid}


%\begin{vsid}{#hp02_065}
\startsloka
{\bf yathodaraṃ bhavet pūrṇam anilena tathā laghu
dhārayen nāsikāṃ madhyātarjanībhyāṃ vinā dṛḍham (2.64)}
\stopsloka

yatheti. yathā yena prakāreṇa pavanena vāyunā laghu kṣipram evodaraṃ pūrṇaṃ bhavet tathā tena
prakāreṇa sūryanāḍyā pūrayet. laghu kṣipram araṃ drutam ity amaraḥ\comment{Amarakośa 1.1.66.}.
pūrakānantaraṃ yat kartavyaṃ tad āha dhārayed iti. madhyātarjanībhyāṃ madhyamātarjanībhyāṃ vinā
aṅguṣṭhānāmikākaniṣṭhikābhir nāsikāṃ dṛḍhaṃ dhārayet. aṅguṣṭhena dakṣiṇanāsāpuṭaṃ
nirudhyānāmikākaniṣṭhikābhyāṃ vāmanāsāpuṭaṃ nirudhya nāsikāṃ dṛḍhaṃ gṛhṇīyād ity arthaḥ.
%\end{vsid}

%\begin{vsid}{#hp02_066,02.067}
\startsloka
{\bf vidhivat kumbhakaṃ kṛtvā recayed iḍayānilam
vātapittaśleṣmaharaṃ śarīrāgnivivardhanam (2.65)
kuṇḍalībodhakaṃ kṣipraṃ pavanaṃ sukhadaṃ hitam
brahmanāḍīmukhe saṃsthakaphādyargalanāśanam (2.66)}
\stopsloka

vidhivad iti. vidhivat bandhapūrvakaṃ kumbhakaṃ kṛtvā iḍayā candranāḍyānilaṃ vāyuṃ recayet.
bhastrākumbhakasyeyaṃ paripāṭī. vāmanāsikāpuṭaṃ dakṣiṇabhujānāmikākaniṣṭhikābhyāṃ nirudhya
dakṣiṇānāsikāpuṭena bhastrāvad vegena recakapūrakāḥ kāryāḥ. śrame jāte tenaiva nāsāpuṭena pūrakaṃ
kṛtvāṅguṣṭhena dakṣiṇaṃ nāsikāpuṭaṃ nirudhya yathāśakti kumbhakaṃ dhārayet. paścād iḍayā
recayet. punar dakṣiṇanāsāpuṭam aṅguṣṭhena nirudhya vāmanāsikāpuṭena bhastrāvaj jhaṭiti
recakapūrakāḥ kartavyāḥ. śrame jāte tenaiva nāsikāpuṭena pūrakaṃ kṛtvānāmikākaniṣṭhikābhyāṃ
vāmanāsikāpuṭaṃ nirudhya yathāśakti kumbhakaṃ kṛtvā piṅgalayā recayed ity ekā
rītiḥ.

vāmanāsikāpuṭam anāmikākaniṣṭhikābhyāṃ nirudhya dakṣiṇanāsikāpuṭena pūrakaṃ kṛtvā jhaṭity
aṅguṣṭhena dakṣiṇanāsikāpuṭaṃ nirudhya vāmanāsāpuṭena recayed. evaṃ śatadhā kṛtvā śrame jāte
tenaiva pūrayet. bandhapūrvakaṃ kṛtveḍayā recayet. punar dakṣiṇanāsāpuṭam aṅguṣṭhena nirudhya
vāmanāsāpuṭena pūrakaṃ kṛtvā jhaṭiti vāmanāsikāpuṭam anāmikākaniṣṭhikābhyāṃ nirudhya piṅgalayā
recayed bhastrāvat. punaḥ punar evaṃ kṛtvā recakapūrakāvṛttiśrame jāte vāmanāsāpuṭenaiva pūrakaṃ
kṛtvānāmikākaniṣṭhikābhyāṃ dhṛtvā kumbhakaṃ kṛtvā piṅgalayā recayed iti dvitīyā rītiḥ.

bhastrikāguṇān āha vātapitteti. vātaś ca pittaṃ ca śleṣmā ca vātapittaśleṣmāṇas tān haratīti tat
tādṛśaṃ śarīre dehe yo 'gnir jaṭharānalas tasya viśeṣeṇa vardhanaṃ dīpanam. kṣipraṃ śīghraṃ
kuṇḍalyāḥ suptāyā bodhakaṃ bodhakartṛ punātīti pavanaṃ pavitrakārakaṃ sukhaṃ dadātīti sukhadaṃ
hitaṃ tridoṣaharatvāt sarveṣāṃ hitaṃ sarvadā ca hitam. sarveṣāṃ kumbhakānāṃ sarvadā hitatve 'pi
sūryabhedanojjāyināv uṣṇau prāyeṇa śīte hitau. sītkārīśītalyau śītale prāyeṇoṣṇe
hite. bhastrākumbhakaḥ samaśītoṣṇaḥ sarvadā hitaḥ. sarveṣāṃ kumbhakānāṃ sarvarogaharatve 'pi
sūryabhedanaṃ prāyeṇa vātaharam. ujjāyī prāyeṇa śleṣmaharaḥ. sītkārīśītalyau prāyeṇa
pittahare. bhastrākhyaḥ kumbhakaḥ tridoṣaharaḥ iti bodhyam. brahmanāḍī suṣumnṇā
brahmaprāpakatvāt. tathā ca śrutiḥ

\startsloka
śataṃ caikā ca hṛdayasya nāḍyas 
tāsāṃ mūrdhānam abhiniḥsṛtaikā
tayordhvam āyann amṛtatvam eti 
viṣvaṅṅ anyā utkramaṇe bhavanti\comment{Chāndogyopaniṣat 8.6.6.}
\stopsloka

tasyā mukhe 'grabhāge saṃsthaḥ samyak sthito yaḥ kaphādirūpo 'rgalaḥ
prāṇagatipratibandhakas tasya nāśanaṃ nāśakartṛ.
%\end{vsid}

%\begin{vsid}{#hp02_068}
\startsloka
{\bf samyaggātrasamudbhūtagranthitrayavibhedakam
viśeṣeṇaiva kartavyaṃ bhastrākhyaṃ kumbhakaṃ tv idam (2.67)}
\stopsloka

samyag iti. samyag dṛḍhībhūtaṃ gātre gātramadhye suṣumṇnāyām eva samyag udbhūtaṃ samudbhūtaṃ jātaṃ
yad granthīnāṃ trayaṃ granthitrayaṃm brahmagranthiviṣṇugranthirudragranthirūpaṃ tasya viśeṣeṇa
bhedajanakam. ata eva idam bhastrā ity ākhyāyata iti bhastrākhyaṃ kumbhakaṃ tu viśeṣeṇaiva
kartavyam avaśyakartavyam ity arthaḥ. sūryabhedanādayas tu yathāsambhavaṃ kartavyāḥ.
%\end{vsid}

%\begin{vsid}{#hp02_069}
\startsloka
{\bf atha bhrāmarī
vegād ghoṣaṃ pūrakaṃ bhṛṅganādaṃ 
bhṛṅgīnādaṃ recakaṃ mandamandam
yogīndrāṇām evam abhyāsayogāc
citte jātā kācid ānandalīlā (2.68)}
\stopsloka

bhrāmarīkumbhakam āha vegād iti. vegāt tarasā ghoṣaṃ saśabdaṃ yathā syāt tathā bhṛṅgasya
bhramarasya nāda iva nādo yasmin karmaṇi tat tathā pūrakaṃ kṛtvā bhṛṅgyā bhramaryā nāda

iva nādo yasmin tat tathā mandaṃ mandaṃ recakaṃ kuryāt. pūrakānantaraṃ kumbhakas tu bhrāmaryāḥ
kumbhakatvād eva siddho 'viśeṣāc ca noktaḥ. pūrakarecakayos tu viśeṣo 'stīti tāv evoktau. evam
uktarītyābhyasanam abhyāsas tasya yogo yuktis tasmād yogīndrāṇāṃ citte kācid anirvācyā ānande līlā
krīḍā ānandalīlā jātotpannā bhavati.
%\end{vsid}


%\begin{vsid}{#hp02_070}
\startsloka
{\bf atha mūrcchā
pūrakānte gāḍhataraṃ baddhvā jālandharaṃ śanaiḥ
recayen mūrcchanākhyeyaṃ manomūrcchā sukhapradā (2.69)}
\stopsloka

mūrcchākumbhakam āha pūrakānta iti. pūrakasyānte 'vasāne 'tiśayena gāḍhaṃ gāḍhataraṃ dṛḍhataraṃ
jālandharākhyaṃ bandhaṃ baddhvā śanair mandaṃ mandaṃ recayet. iyaṃ kumbhikā mūrcchanākhyā mano
mūrcchayatīti mūrcchanā ity ityākhyāyata iti mūrcchanākhyā. kīdṛśī mano mūrcchayati iti
manomūrcchā. etena mūrcchanāyā vigrahadarśanapūrvakaṃ phalam uktam. punaḥ kīdṛśī sukhapradā
sukhaṃ pradadātīti sukhapradā.
%\end{vsid}


%\begin{vsid}{#hp02_071}
\startsloka
{\bf atha plāvinī
antaḥ pravartitodāramārutāpūritodaraḥ
payasy agādhe 'pi sukhāt plavate padmapatravat (2.70)}
\stopsloka

plāvinīkumbhakam āha antar iti. antaḥ śarīrāntaḥ pravartitaḥ pūrita udāro 'tiśayito yo mārutaḥ
samīras tena ā samantāt pūritam udaraṃ yena sa pumān agādhe 'py atalasparśe 'pi payasi jale
padmapatravat padmapatreṇa tulyaṃ sukhād anāyāsāt plavate tarati jalopari gacchati.
%\end{vsid}


%\begin{vsid}{#hp02_072}
\startsloka
{\bf prāṇāyāmas tridhā prokto recapūrakakumbhakaiḥ
sahitaḥ kevalaś ceti kumbhako dvividho mataḥ (2.71)}
\stopsloka

atha prāṇāyāmabhedān āha prāṇāyāma iti. prāṇasya śarīrāntaḥsañcārivāyor āyamanaṃ nirodhanam āyāmaḥ
prāṇāyāmaḥ. prāṇāyāmalakṣaṇam uktaṃ gorakṣanāthena

\startsloka
prāṇaḥ svadehajo vāyur āyāmas tannirodhanam\comment{Gorakṣaśataka 42.}
\stopsloka

iti. recakaś ca pūrakaś ca kumbhakaś ca tair bhedais tridhā triprakārakaḥ recakaprāṇāyāmaḥ
pūrakaprāṇāyāmaḥ kumbhakaprāṇāyāmaś ceti. recakalakṣaṇam āha yājñavalkyaḥ

\startsloka
  bahir yad recanaṃ vāyor udarād recakaḥ smṛtaḥ\comment{Yogayājñavalkya 6.25ab} 
\stopsloka


iti. recakaprāṇāyāmalakṣaṇam


\startsloka
niṣkramya nāsāvivarād aśeṣaṃ 
prāṇaṃ bahiḥ śūnyam ivānilena
nirucchvasaṃs tiṣṭhati ruddhavāyuḥ 
sa recako nāma mahānirodhaḥ\comment{Source unknown.}
\stopsloka

pūrakalakṣaṇam

\startsloka
bāhyād āpūraṇaṃ vāyor udare pūrako hi saḥ\comment{Yogayājñavalkya 6.24ab}
\stopsloka

iti. pūrakaprāṇāyāmalakṣaṇam

\startsloka
bāhye sthitaṃ prāṇapuṭena vāyum 
ākṛṣya tenaiva śanaiḥ samantāt
nāḍīś ca sarvāḥ paripūrayed yaḥ 
sa pūrako nāma mahānirodhaḥ\comment{Source unknown.}
\stopsloka

kumbhakalakṣaṇaṃ

\startsloka
  sampūrya kumbhavad vāyor dhāraṇaṃ kumbhako bhavet\comment{Yogayājñavalkya 6.24cd}
\stopsloka

ayaṃ kumbhakas tu pūrakaprāṇāyāmād abhinnaḥ. bhinnas tu

\startsloka
  na recako naiva ca pūrako 'tra
  nāsāpuṭāntasthitam eva vāyum
  suniścalaṃ dhārayate krameṇa
  kumbhākhyam etat pravadanti tajjñāḥ
\stopsloka

atha prakārāntareṇa prāṇāyāmaṃ vibhajate sahita iti. kumbhako dvividhaḥ sahitaḥ kevalaś ceti.  mato
'bhimato yoginām iti śeṣaḥ. tatra sahito dvividhaḥ recakapūrvakaḥ pūrakapūrvakaś ca. tad
uktaṃ


%\begin{vsid}{#hp02_073}
\startsloka     
  ārecyāpūrya vā kuryāt sa vai sahitakumbhakaḥ\comment{This line was originally part of the
  HP and was complemented by 2.72ab (yāvat kevalasiddhiḥ ...)} 
\stopsloka
%\end{vsid}

tatra recakapūrvako recakaprāṇāyāmād abhinnaḥ. pūrakapūrvakaḥ kumbhakaḥ pūrakaprāṇāyāmād
abhinnaḥ. kevalakumbhakaḥ kumbhaka\-prāṇā\-yāmād abhinnaḥ. prāg uktāḥ sūryabhedanādayaḥ
pūraka\-pūrvakasya kumbhakasya bhedā jñātavyāḥ.
%\end{vsid}



%\begin{vsid}{#hp02_073-074}
\startsloka
{\bf yāvat kevalasiddhiḥ syāt sahitaṃ tāvad abhyaset
recakaṃ pūrakaṃ muktvā sukhaṃ yad vāyudhāraṇam (2.72)}
\stopsloka

sahitakumbhakābhyāsasyāvadhim āha yāvad iti. kevalasya kevalakumbha\-kasya siddhiḥ kevalasiddhir
yāvat yāvatparyantaṃ syāt tāvat tāvatparyantaṃ sahitaṃ sahitakumbhakaṃ sūryabhedādikam abhyased
anutiṣṭhet. suṣumṇnābhedānantaraṃ yadā suṣumṇnāntaḥ ghaṭaśabdā\var{ghaṭa \Tue \EdLo etc. \lem ṣaṭ \Wthree
  etc. (neither reading seems interpretable)} bhavanti tadā kevalakumbhakaḥ siddhyati. tadanantaraṃ
sahitakumbhakā daśa viṃśatir vā kāryāḥ. aśītisaṅkhyāpūrtiḥ kevalakumbhakair eva kartavyā. sati
sāmarthye kevalakumbhakā aśīter adhikāḥ kāryāḥ.  kevalakumbhakasya lakṣaṇam āha recakam
iti. recakaṃ pūrakaṃ muktvā tyaktvā sukham anāyāsaṃ yathā syāt tathā vāyor dhāraṇaṃm vāyudhāraṇaṃm
yat.
%\end{vsid}



%\begin{vsid}{#hp02_074-075}
\startsloka
{\bf prāṇāyāmo 'yam ity uktaḥ sa vai kevalakumbhakaḥ
kumbhake kevale siddhe recapūrakavarjite (2.73)}
\stopsloka

prāṇāyāma iti. sa vai iti niścitaṃ. kevalakumbhakaḥ prāṇāyāmaḥ ity ayam uktaḥ. kevalaṃ
praśaṃsanti kevala iti. reco recakaḥ. recaś ca pūrakaś ceti recapūrakau tābhyāṃ varjite rahite
kevale kumbhake siddhe sati.
%\end{vsid}


%\begin{vsid}{#hp02_075-076}
\startsloka
{\bf na tasya durlabhaṃ kiñcit triṣu lokeṣu vidyate
śaktaḥ kevalakumbhena yatheṣṭaṃ vāyudhāraṇāt (2.74)}
\stopsloka

neti. tasya yoginas triṣu lokeṣu durlabhaṃ duṣprāpaṃ kim api kiñcit kim api na vidyate. tasya
sarvaṃ sulabham ity arthaḥ. śakta iti kevalakumbhakena kevalakumbhakābhyāsena śaktaḥ samartho
yatheṣṭaṃ yathecchaṃ vāyor dhāraṇaṃ vāyudhāraṇaṃ tasmād vāyudhāraṇāt.
%\end{vsid}


%\begin{vsid}{#hp02_076,02.079}
\startsloka
{\bf rājayogapadaṃ cāpi labhate nātra saṃśayaḥ
kumbhakāt kuṇḍalībodhaḥ kuṇḍalībodhato bhavet
anargalā suṣumṇā ca haṭhasiddhiś ca jāyate (2.75)}
\stopsloka

rājeti. rājayogapadaṃ rājayogātmakaṃ padaṃ labhate. atra saṃśayo na. niścitam etad ity arthaḥ.
kumbhakābhyāsasya paramparayā kaivalyahetumatvam āha kumbhakād iti. kumbhakāt kumbhakābhyāsāt
kuṇḍalī ādhāraśaktis tasyā bodho nidrābhaṅgo bhavet. kuṇḍalyā bodhaḥ kuṇḍalībodhas tasmāt
kuṇḍalībodhataḥ anargaleti suṣumṇnā madhyanāḍī anargalā kaphādyargalarahitā bhavet. haṭhasya
haṭhābhyāsasya siddhiḥ pratyāhārādiparamparayā kaivalyarūpā siddhir jāyate.
%\end{vsid}


%\begin{vsid}{#hp02_077}
\startsloka
{\bf haṭhaṃ vinā rājayogo rājayogaṃ vinā haṭhaḥ
na sidhyati tato yugmam āniṣpatteḥ samabhyaset (2.76)}
\stopsloka

haṭhayogarājayogasādhanayoḥ parasparopakāryopakārakatvam āha haṭhaṃ vineti. haṭhaṃ haṭhayogaṃ vinā
rājayogo na siddhyati rājayogaṃ vinā haṭho na siddhyati. tataḥ yato 'nyatarasya siddhir nāsti
tasmān niṣpattiṃ rājayogasiddhim ā maryādīkṛtya yā niṣpattis tasyā rājayogasiddhiparyantaṃ yugmaṃ
haṭhayogarājayogadvayam abhyased anutiṣṭhet. haṭhātirikte sākṣāt paramparayā vā rājayogasādhane
'tra rājayogaśabdaḥ jīvanasādhane lāṅgale jīvanaśabdaprayogavat. rājayogasādhanaṃ ca caturthopadeśe
vakṣyamāṇam unmanīśāmbhavīmudrādirūpam aparokṣānubhūtāv\comment{REF?} uktaṃ pañcadaśāṅgarūpaṃ
ca. vākyasudhāyām\comment{REF?} uktaṃ dṛśyānuviddhādirūpaṃ ca.
%\end{vsid}



%\begin{vsid}{#hp02_078}
\startsloka
{\bf kumbhakaprāṇarodhānte kuryāc cittaṃ nirāśrayam
evam abhyāsayogena rājayogapadaṃ vrajet (2.77)}
\stopsloka

haṭhābhyāsād rājayogaprāptiprakāram āha kumbhaketi. kumbhakena prāṇasya yo rodhas tasyānte madhye
cittam antaḥkaraṇaṃ nirāśrayaṃ kuryāt. samprajñātasamādhau jātāyāṃ brahmākāravṛtteḥ paravairāgyeṇa
vilayaṃ kuryād ity arthaḥ. evam uktarītyābhyāsasya yogo yuktis tena. yogaḥ
saṃnahanopāyadhyānasaṅgatiyuktiṣu iti kośaḥ\comment{Amarakośa 3.3.22?}. rājayogapadaṃ
rājayogātmakaṃ padaṃ vrajet prāpnuyāt.
%\end{vsid}


%\begin{vsid}{#hp02_080}
\startsloka
{\bf vapuḥkṛśatvaṃ vadane prasannatā
nādasphuṭatvaṃ nayane sunirmale
arogatā bindujayo 'gnidīpanaṃ
nāḍīviśuddhir haṭhasiddhilakṣaṇam (2.78)}
\stopsloka

haṭhasiddhijñāpakam āha vapuḥkṛśatvam iti. vapuṣo dehasya kṛśatvaṃ kārśyaṃ. vadane mukhe
prasannatā prasādaḥ. nādasya dhvaneḥ sphuṭatvaṃ prākaṭyaṃm. nayane netre suṣṭhu nirmale.
arogasya bhāvo 'rogatā ārogyam. bindor dhātor jayaḥ kṣayābhāvarūpaḥ. agner udaryasya dīpanaṃ
dīptiḥ. nāḍīnāṃ viśeṣeṇa śuddhir malāpagamaḥ. etad dhaṭhasya haṭhābhyāsasiddher bhāvinyā
lakṣyate 'neneti lakṣaṇaṃ jñāpakam.
%\end{vsid}


%%%%%%%%%%%%%%%%%%%%%%%%%%%%5


%\begin{vsid}{#hp03_001}
\startsloka
{\bf saśailavanadhātrīṇāṃ yathādhāro 'hināyakaḥ
sarveṣāṃ yogatantrāṇāṃ tathādhāro hi kuṇḍalī (3.1)}
\stopsloka

atha kuṇḍalyāḥ sarvayogāśrayatvam āha saśaileti. śailāś ca vanāni ca śailavanāni taiḥ saha
vartamānāḥ saśailavanāḥ tāś ca tā dhātryaś ca bhūmayas tāsām. dhātryā ekatve 'pi deśabhedād bhedam
ādāya bahuvacanam. ahīnāṃ sarpāṇāṃ nāyako netāhināyakaḥ śeṣo yathā yadvad ādhāra āśrayas tathā
tadvat. sarveṣāṃ yogasya tantrāṇi yogatantrāṇi yogopāyās teṣām. kuṇḍālyādhāraśaktir
āśrayaḥ. kuṇḍalībodhaṃ vinā sarvayogopāyānāṃ vaiyarthyād iti bhāvaḥ.
%\end{vsid}


%\begin{vsid}{#hp03_002}
\startsloka
{\bf suptā guruprasādena yadā jāgarti kuṇḍalī
tadā sarvāṇi padmāni bhidyante granthayo 'pi ca (3.2)}
\stopsloka

kuṇḍalībodhasya phalam āha dvābhyāṃ supteti. suptā kuṇḍalī guroḥ prasādena yadā jāgarti budhyate
tadā sarvāṇi padmāni ṣaṭcakrāṇi bhidyante bhinnāni bhavanti. granthayo 'pi ca
brahmagranthiviṣṇugranthirudragranthayo bhidyante bhedaṃ prāpnuvantīty anvayaḥ.
%\end{vsid}


%\begin{vsid}{#hp03_003}
\startsloka
{\bf prāṇasya śūnyapadavī tadā rājapathāyate
tadā cittaṃ nirālambaṃ tadā kālasya vañcanam (3.3)}
\stopsloka

prāṇasyeti. tadā śūnyapadavī suṣumnā prāṇasya vāyo rājñāṃ panthā rājapathaḥ rājapatha ivācarati
rājapathāyate rājamārgāyate. sukhena gamanasambhavāt. tadā cittam ālambanam āśrayas tasmān nirgataṃ
nirālambaṃ nirviṣayaṃ bhavati. tadā kālasya mṛtyor vañcanaṃ pratāraṇaṃ bhavati.
%\end{vsid}


%\begin{vsid}{#hp03_004}
\startsloka
{\bf suṣumṇā śūnyapadavī brahmarandhraṃ mahāpathaḥ
śmaśānaṃ śāmbhavī madhyamārgaś cety ekavācakāḥ (3.4)}
\stopsloka

suṣumnāparyāyān āha suṣumneti. ity uktāḥ śabdā ekasya ekārthasya vācakāḥ ekavācakāḥ. paryāyā ity
arthaḥ. spaṣṭaḥ ślokārthaḥ.
%\end{vsid}



%\begin{vsid}{#hp03_005}
\startsloka
{\bf tasmāt sarvaprayatnena prabodhayitum īśvarīm
brahmadvāramukhe suptāṃ mudrābhyāsaṃ samācaret (3.5)}
\stopsloka

tasmād iti. yasmāt kuṇḍalībodhanenaiva ṣaṭcakrabhedādikaṃ bhavati tasmāt sarvaprayatnena sarveṇa
prayatnena brahma saccidānandalakṣaṇaṃ tasya dvāraṃ prāptyupāyaḥ suṣumnā tasya mukhe 'grabhāge
mukhena suṣumnādvāraṃ pidhāya suptām īśvarīṃ kuṇḍalīṃ prabodhayituṃ prakarṣeṇa bodhayituṃ mudrāṇāṃ
mahāmudrādīnām abhyāsam āvṛttiṃ samācaret samyag ācaret.
%\end{vsid}


%\begin{vsid}{#hp03_006,03.007}
\startsloka
{\bf mahāmudrā mahābandho mahāvedhaś ca khecarī
uḍḍyānaṃ mūlabandhaś ca bandho jālandharābhidhaḥ (3.6)
karaṇī viparītākhyā vajrolī śakticālanam
idaṃ hi mudrādaśakaṃ jarāmaraṇanāśanam (3.7)}
\stopsloka

mudrā uddiśati mahāmudretyādinā sārdhena. sārdhārthaḥ spaṣṭaḥ. mudrāphalam āha sārdhadvābhyām idam
iti. idam uktaṃ mudrāṇāṃ daśakaṃ jarā ca maraṇaṃ ca jarāmaraṇe tayor nāśanam nivārakam.
%\end{vsid}


%\begin{vsid}{#hp03_08}
\startsloka
{\bf ādināthoditaṃ divyam aṣṭaiśvaryapradāyakam
vallabhaṃ sarvasiddhānāṃ durlabhaṃ marutām api (3.8)}
\stopsloka

ādināthena śambhunoditaṃ kathitam. divi bhavaṃ divyam uttamam. aṣṭau ca tāny aiśvaryāṇi
cāṣṭaiśvaryāṇi aṇimamahimagarimalaghimaprāptiprākāmyeśatāvaśitākhyāni. tatrāṇimā saṅkalpamātreṇa
prakṛtyapagame paramāṇuvad dehasya sūkṣmatā. mahimā prakṛtyāpūreṇā\-kāśādivan mahadbhāvaḥ. garimā
laghutarasyāpi tūlādeḥ parvatādivad gurubhāvaḥ. laghimā gurutarasyāpi parvatādes tūlādival
laghubhāvaḥ. prāptiḥ sarvabhāvasānnidhyam. yathā bhūmistha evāṅgulyagreṇa spṛśati
candramasam. prākāmyam icchānabhighātaḥ. yathā udaka iva bhūmau nimajjaty unmajjati ca. īśatā
bhūtabhautikānāṃ prabhavāpyayasaṃsthānaviśeṣasāmarthyam. vaśitvaṃ bhūtabhautikānāṃ
svādhī\-nī\-karaṇam. teṣāṃ pradāyakaṃ prakarṣeṇa dadātīti tathā taṃ sarve ca te siddhāś ca kapilādayas
teṣāṃ vallabhaṃ priyaṃ marutāṃ devānām api durlabhaṃ duṣprāpaṃ kim utānyeṣām ity arthaḥ.
%\end{vsid}


%\begin{vsid}{#hp03_09}
\startsloka
{\bf gopanīyaṃ prayatnena yathā ratnakaraṇḍakam
kasyacin naiva vaktavyaṃ kulastrīsurataṃ yathā (3.9)}
\stopsloka

gopanīyam iti. prayatnena prakṛṣṭena yatnena gopanīyaṃ gopanārham. gopane dṛṣṭāntam
āha yatheti. ratnānāṃ hīrakādīnāṃ karaṇḍakaṃ ratnakaraṇḍakaṃ yathā yena prakāreṇa gopyate
tadvat. kasyāpi janamātrasya yad vā kasyāpi brahmaṇo 'pi naiva vaktavyaṃ naiva vācyaṃ kim
utānyasya. tatra dṛṣṭāntaḥ kulastriyaḥ surataṃ kulastrīsurataṃ saṅgamanaṃ yathā yadvat.
%\end{vsid}


%\begin{vsid}{#hp03_10}
\startsloka
{\bf atha mahāmudrā
pādamūlena vāmena yoniṃ sampīḍya dakṣiṇam
prasāritaṃ padaṃ kṛtvā karābhyāṃ dhārayed dṛḍham (3.10)}
\stopsloka

mudrāsu\var{mudrāsu \Wthree \EdAd \lem mudrādiṣu \EdLo daśavidhamudrādiṣu \EdMu} prathamoddiṣṭatvena
mahāmudrāṃ tāvad āha pādamūleneti. vāmena savyena pādasya mūlaṃ pādamūlaṃ pārṣṇis tena pādamūlena
vāmapādapārṣṇinety arthaḥ. yoniṃ yonisthānaṃ gudameḍhrayor madhyabhāgaṃ
sampīḍyākuñcitavāmapādapārṣṇinā yonisthānaṃ dṛḍhaṃ saṃyojyety arthaḥ. dakṣiṇaṃ savyetaraṃ padaṃ
caraṇaṃ prasāritaṃ bhūmisaṃlagnapārṣṇikam ūrdhvāṅgulikaṃ daṇḍavat kṛtvā karābhyāṃ sampradāyād
ākuñcitakaratarjanībhyāṃ dṛḍhaṃ gāḍhaṃ dhārayed aṅguṣṭhapradeśe gṛhṇīyāt.
%\end{vsid}


%\begin{vsid}{#hp03_11}
\startsloka
{\bf kaṇṭhe bandhaṃ samāropya dhārayed vāyum ūrdhvataḥ
yathā daṇḍahataḥ sarpo daṇḍākāraḥ prajāyate (3.11)}
\stopsloka

kaṇṭha iti. kaṇṭhe kaṇṭhadeśe bandhaṃ bandhanaṃ samyag āropya kṛtvā. jālandharabandhaṃ kṛtvety
arthaḥ. vāyuṃ pavanam ūrdhvata upari suṣumnāyāṃ dhārayet. anena mūlabandhaḥ sūcitaḥ. sa tu
yonisampīḍanena jihvābandhena ca caritārtha iti sāmpradāyikāḥ. yathā daṇḍena hatas tāḍito
daṇḍahataḥ sarpaḥ kuṇḍalī daṇḍākāraḥ daṇḍasyākāra ivākāro yasya sa tādṛśaḥ. kuṇḍalākāraṃ tyaktvā
sarala ity arthaḥ. prakarṣeṇa jāyate bhavati.
%\end{vsid}


%\begin{vsid}{#hp03_12}
\startsloka
{\bf ṛjvībhūtā tathā śaktiḥ kuṇḍalī sahasā bhavet
tadā sā maraṇāvasthā jāyate dvipuṭāśrayā (3.12)}
\stopsloka

ṛjvīti. tathā kuṇḍaly ādhāraśaktiḥ sahasā śīghram eva ṛjvī sampadyate tathābhūtā ṛjvībhūtā saralā
bhavet. tadā seti. dve puṭe iḍāpiṅgale āśrayo yasyāḥ sā maraṇāvasthā jāyate. kuṇḍalībodhe sati
suṣumnāyāṃ praviṣṭe prāṇe dvayoḥ puṭayoḥ prāṇaviyogāt.
%\end{vsid}


%\begin{vsid}{#hp03_13}
\startsloka
{\bf tataḥ śanaiḥ śanair eva recayen na tu vegataḥ
iyaṃ khalu mahāmudrā mahāsiddhaiḥ pradarśitā (3.13)}
\stopsloka

tatas tadanantaraṃ śanaiḥ śanair eva recayet. vāyum iti sambadhyate. vegatas tu vegān na
recayet. vegato recane balahāniprasaṅgāt. khalv iti vākyālaṅkāre. iyaṃ mahāmudrā mahāsiddhair
ādināthādibhiḥ pradarśitā prakarṣeṇa darśitā.
%\end{vsid}


%\begin{vsid}{#hp03_14}
\startsloka
{\bf mahākleśādayo doṣāḥ kṣīyante maraṇādayaḥ
mahāmudrāṃ ca tenaiva vadanti vibudhottamāḥ (3.14)}
\stopsloka

mahāmudrāyā anvarthatām āha mahākleśeti. mahāntaś ca te kleśāś ca mahākleśā
avidyāsmitārāgadveṣābhiniveśāḥ pañca. ta ādayo yeṣāṃ tatkāryāṇāṃ śokamohādīnāṃ te doṣāḥ
kṣīyante. maraṇam ādir yeṣāṃ jarādīnāṃ te 'pi ca kṣīyante naśyanti. yatas tenaiva hetunā viśiṣṭā
budhā vibudhās teṣūttamā vibudhottamā mahāmudrāṃ vadanti. mahākleśān maraṇādīṃś ca doṣāṇ mudrayati
śamayatīti mahāmudreti vyutpatter ity arthaḥ.
%\end{vsid}


%\begin{vsid}{#hp03_15}
\startsloka
{\bf candrāṅge tu samabhyasya sūryāṅge punar abhyaset
yāvat tulyā bhavet saṅkhyā tato mudrāṃ visarjayet (3.15)}
\stopsloka

mahāmudrābhyāsakramam āha candrāṅga iti. candreṇa candranāḍyopalakṣitam aṅgaṃ candrāṅgaṃ tasmin
candrāṅge vāmāṅge. tuśabdaḥ pādapūraṇe. samyag abhyasya sūryeṇa piṅgalayopalakṣitam aṅgaṃ sūryāṅgaṃ
tasmin sūryāṅge dakṣāṅge punar vāmāṅgābhyāsānantaraṃ yāvad yāvatkālaparyantaṃ tulyā vāmāṅge
kumbhakakābhyāsa\-saṅkhyā\-samā saṅkhyā bhavet tāvad abhyaset. tataḥ saṅkhyāsāmyānantaraṃ mudrāṃ
mahāmudrāṃ visarjayet. atrāyaṃ kramaḥ. ākuñcitavāma\-pāda\-pārṣṇiṃ yonisthāne saṃyojya
prasāritadakṣiṇapādāṅguṣṭham ākuñicitatarjanībhyāṃ gṛhītvābhyāso vāmāṅge 'bhyāsaḥ. asminn abhyāse
pūrito vāyur vāmāṅge tiṣṭhati. ākuñcitadakṣapādapārṣṇiṃ yonisthāne saṃyojya
prasāritavāmapādāṅguṣṭham ākuñcitatarjanībhyāṃ gṛhītvābhyāso dakṣāṅge 'bhyāsaḥ. asminn abhyāse
pūrito vāyur dakṣāṅge tiṣṭhati.
%\end{vsid}


%\begin{vsid}{#hp03_00}
\startsloka
{\bf na hi pathyam apathyaṃ vā rasāḥ sarve 'pi nīrasāḥ
api bhuktaṃ viṣaṃ ghoraṃ pīyūṣam api jīryati (3.16)}
\stopsloka

mahāmudrāguṇān āha na hīti. hi yasmād mahāmudrābhyāsina ity adhyāhāraḥ. pathyam apathyaṃ vā
na. pathyāpathyavicāro nāstīty arthaḥ. tasmāt sarve 'pi bhuktā rasāḥ kaṭvamlādayo jīryanta ti
vibhaktivipariṇāmenānvayah.\var{vibhakti cett. \lem vacana \EdAd} nīrasā nirgato raso yebhyas te
yātayāmāḥ padārthā jīryanti. ghoram iti durjaraṃ bhuktam annaṃ viṣaṃ kṣveḍam api pīyūṣam ivāmṛtam
iva jīryati jīrṇaṃ bhavati. kim utānyad iti bhāvaḥ.


%\begin{vsid}{#hp03_00}
\startsloka
{\bf kṣayakuṣṭhagudāvartagulmājīrṇapurogamāḥ
tasya doṣāḥ kṣayaṃ yānti mahāmudrāṃ tu yo 'bhyaset (3.17)}
\stopsloka

yaḥ pumān mahāmudrām abhyaset tasya kṣayo rājarogaḥ kuṣṭhagudāvartagulmā rogaviśeṣāḥ. ajīrṇaṃ
bhuktānnāparipākas tāni purogamāny agresarāṇi yeṣāṃ mahodarajvarādīnāṃ te tathā tādṛśā doṣā
doṣajanitā rogāḥ kṣayaṃ nāśaṃ yānti prāpnuvanti.
%\end{vsid}


%\begin{vsid}{#hp03_00}
\startsloka
{\bf kathiteyaṃ mahāmudrā mahāsiddhikarī nṝṇām
gopanīyā prayatnena na deyā yasya kasyacit (3.18)}
\stopsloka

mahāmudrām upasaṃharan tasyā gopyatvam āha kathiteti. iyam eṣā mahāmudrā kathitoktā. mayā iti
śeṣaḥ. kīdṛśī nṝṇām abhyasyatāṃ narāṇāṃ mahatyaś ca tāḥ siddhayaś cāṇimādyās tāsāṃ karī
kartrīyam. prakṛṣṭo yatnaḥ prayatnas tena prayatnena gopanīyā gopanārhā yasya kasyacid yasya
kasyāpy anadhikāriṇaḥ sambandhasāmānye ṣaṣṭhī. na deyā dātuṃ yogyā na bhavatīty arthaḥ.
%\end{vsid}

%\begin{vsid}{#hp03_00}
\startsloka
{\bf atha mahābandhaḥ
pārṣṇiṃ vāmasya pādasya yonisthāne niyojayet 
vāmorūpari saṃsthāpya dakṣiṇaṃ caraṇaṃ tathā (3.19)}
\stopsloka

mahābandham āha pārṣṇim iti. vāmasya savyasya pādasya caraṇasya pārṣṇiṃ gulphayor adhobhāgam.

\startsloka
  tad granthī ghuṭike gulphau pumān pārṣṇis tayor adhaḥ\comment{Amarakośa 2.6.72}
\stopsloka

ity amaraḥ. yonisthāne gudameṇḍhrayor antarāle niyojayen nitarāṃ yojayet. vāmaḥ savyo ya ūrus
tasyopari dakṣiṇaṃ caraṇaṃ pādaṃ saṃsthāpya samyak sthāpayitvā. tathāśabdaḥ pādapūraṇe.
%\end{vsid}


%\begin{vsid}{#hp03_00}
\startsloka
{\bf pūrayitvā tato vāyuṃ hṛdaye cibukaṃ dṛḍham 
niṣpīḍya yonim ākuñcya mano madhye niyojayet (3.20)}
\stopsloka

pūrayitveti. tatas tadanantaraṃ vāyuṃ pūrayitvā hṛdaye cibukaṃ dṛḍham niṣpīḍya gāḍhaṃ
saṃsthāpya. etena jālandharabandhaḥ proktaḥ. yoniṃ gudameṇḍhrayor antarālam ākuñcya. anena
mūlabandhaḥ sūcitaḥ. sa tu jihvābandhena gatārthatvān na kartavyaḥ. manaḥ svāntaṃ madhye
madhyanāḍyāṃ niyojayet pravartayet.
%\end{vsid}


%\begin{vsid}{#hp03_00}
\startsloka
{\bf dhārayitvā yathāśakti recayed anilaṃ śanaiḥ
savyāṅge tu samabhyasya dakṣāṅge punar abhyaset (3.21)}
\stopsloka

dhārayitveti. śaktim anatikramya yathāśakti. dhārayitvā kumbhayitvā. śanair mandaṃ mandam anilaṃ
vāyuṃ recayet. savyāṅge vāmāṅge samabhyasya samyag āvartya dakṣāṅge dakṣiṇāṅge punar yāvat tulyā
bhavet saṅkhyā tāvad abhyaset.
%\end{vsid}


%\begin{vsid}{#hp03_00}
\startsloka
{\bf matam atra tu keṣāñcit kaṇṭhabandhaṃ vivarjayet
rājadantasthajihvāyāṃ bandhaḥ śasto bhaved iti (3.22)} 
\stopsloka

atha jālandharabandhe kaṇṭhasaṅkocanasyānupayogam āha matam iti. keṣāñcit tv ācāryāṇām idaṃ
matam. kiṃ tat ity āha atra jālandharabandhe kaṇṭhabandhaṃ\var{kaṇṭhabandhaṃ \Wthree}{kaṇṭhabandhe \EdLo}
kaṇṭhasya bandhanaṃ bandhaḥ saṅkocas taṃ vivarjayet viśeṣeṇa varjayet. kutaḥ yato dantānāṃ rājāno
rājadantā rājadanteṣu tiṣṭhatīti rājadantasthāḥ rājadantasthā cāsau jihvā ca tasyāṃ
rājadantasthajihvāyāṃ bandhaḥ taduparibhāgasya sambandhaḥ śastaḥ. kaṇṭhākuñcanāpekṣayā praśasto
bhaved iti hetoḥ.
%\end{vsid}


%\begin{vsid}{#hp03_00}
\startsloka
{\bf ayaṃ tu sarvanāḍīnām ūrdhvaṃgatinirodhakaḥ\var{ūrdhvaṃ \Tue \EdMu \lem ūrdhva \Wthree \EdLo (unmetrical)}
ayaṃ khalu mahābandho mahāsiddhipradāyakaḥ (3.23)}
\stopsloka

ayaṃ tv iti. ayaṃ tu rājadantasthajihvāyāṃ bandhas tu sarvāś ca tā nāḍyaś ca sarvanāḍyo
dvāsaptatisahasrasaṅkhyākās tāsāṃ suṣumnātiriktānām ūrdhvam upari vāyor gatir ūrdhvagatiḥ tasyā
nirodhakaḥ pratibandhakaḥ. etena

\startsloka
badhnāti hi sirājālam adhogāmi nabhojalam\comment{HP 3.71}
\stopsloka

iti jālandharoktaṃ phalam anenaiva siddham iti sūcitam. mahābandhasya phalam āha ayaṃ khalv
iti. ayam uktaḥ khalu prasiddhaḥ mahābandho mahāsiddhīḥ prakarṣeṇa dadātīti tathā.
%\end{vsid}


%\begin{vsid}{#hp03_00}
\startsloka
{\bf kālapāśamahābandhavimocanavicakṣaṇaḥ
triveṇīsaṅgamaṃ dhatte kedāraṃ prāpayen manaḥ (3.24)}
\stopsloka

kālasya mṛtyoḥ pāśo vāgurā tena yo mahābandho bandhanaṃ tasya viśeṣeṇa mocane mokṣaṇe vicakṣaṇaḥ
pravīṇaḥ. tisṛṇāṃ nadīnāṃ veṇī samudāyaḥ sa eva saṅgamaḥ prayāgas taṃ dhatte vidhatte. kedāraṃ
bhruvor madhye śivasthānaṃ kedāraśabdavācyaṃ taṃ manaḥ svāntaṃ
prāpayet. gatibuddhītyādinā\comment{Aṣṭādhyāyī 1.4.52.} aṇau kartur manaso ṇau karmatvam.
%\end{vsid}



%\begin{vsid}{#hp03_00}
\startsloka
{\bf rūpalāvaṇyasampannā yathā strī puruṣaṃ vinā 
mahāmudrāmahābandhau niṣphalau vedhavarjitau (3.25)}
\stopsloka

mahāvedham vaktukāmas tasyotkarṣaṃ tāvad āha rūpeti. rūpaṃ saundaryaṃ cakṣuḥpriyo guṇaḥ. lāvaṇyaṃ
kāntiviśeṣaḥ. tad uktaṃ



\startsloka
  muktāphaleṣu chāyāyās taralatvam ivāntarā
  pratibhāti yad aṅgeṣu lāvaṇyaṃ tad ihocyate (3.26)\comment{This definition of
    {\em lāvaṇya} is regularly quoted, as for instance,
    in {\em Rasaratnasamuccayabodhinī} (4.15)} 
\stopsloka


tābhyāṃ sampannā viśiṣṭā strī yuvatī puruṣaṃ bhartāraṃ vinā yathā yādṛśī niṣphalā tathā mahāmudrā
ca mahābandhaś ca tau vedhena mahāvedhena vināpi. pratyayapūrvottarapadayor lopo vaktavyaḥ iti
bhāṣyakārokter\comment{Quotation not traced. \EdMu reads {\em vināpi} as part of the quotation.}
mahacchabdasya lopaḥ. varjitau rahitau niṣphalau vyarthāv ity arthaḥ.
%\end{vsid}


%\begin{vsid}{#hp03_00}
\startsloka
  {\bf atha mahāvedhaḥ
    mahābandhasthito yogī kṛtvā pūrakam ekadhīḥ
vāyūnāṃ gatim āvṛtya nibhṛtaṃ kaṇṭhamudrayā (3.26)}
\stopsloka

mahāvedham āha mahābandheti. mahābandhe mahābandhamudrāyāṃ sthito mahābandhasthitaḥ. ekā ekāgrā
dhīr yasya sa ekadhīḥ. yogī yogābhyāsī pūrakaṃ nāsāpuṭābhyāṃ vāyor grahaṇaṃ kṛtvā kaṇṭhe mudrā
kaṇṭhamudrā tayā jālandharamudrayā vāyūnāṃ prāṇādīnāṃ gatim ūrdhvādhogamanādirūpāṃ nibhṛtaṃ
niścalaṃ yathā bhavati tathāvṛtya nirudhya kumbhakaṃ kṛtvety arthaḥ.
%\end{vsid}


%\begin{vsid}{#hp03_00}
\startsloka
{\bf samahastayugo bhūmau sphicau santāḍayec chanaiḥ 
puṭadvayam atikramya vāyuḥ sphurati madhyagaḥ (3.27)}
\stopsloka

samahasteti. bhūmau bhuvi hastayor yugaṃ hastayugaṃ samaṃ hastayugaṃ yasya sa samahastayugaḥ
bhūmisaṃlagnatalau saralau hastau yasya tādṛśaḥ sann ity arthaḥ. sphicau kaṭiprothau. striyām
sphicau kaṭiprothau ity amaraḥ.\comment{Amarakośa ?} bhūmisaṃlagnatalayor hastayor avalambanena
yonisthānasaṃlagnapārṣṇinā vāmapādena saha bhūmeḥ kiñcid utthāpitau śanair mandaṃ santāḍayet samyak
tāḍayet bhūmāv eva. puṭayor dvayam iḍāpiṅgalayor yugmam atikramyollaṅghya madhye suṣumnāmadhye
gacchatīti madhyago vāyuḥ sphurati.  %  Misprints \EdLo avalambena and vāyu 
%\end{vsid}


%\begin{vsid}{#hp03_028}
\startsloka
{\bf somasūryāgnisambandho jāyate cāmṛtāya vai
mṛtāvasthā samutpannā tato vāyuṃ virecayet (3.28)}
\stopsloka

someti. somaś ca sūryaś cāgniś ca somasūryāgnayaḥ somasūryāgniśabdais tadadhiṣṭhitā nāḍya
iḍāpiṅgalāsuṣumnā grāhyās teṣāṃ sambandhaḥ. tadvāyusambandhāt teṣāṃ sambandhaḥ. amṛtāya mokṣāya
jāyate. vai iti niścaye 'vyayam. mṛtasya prāṇaviyuktasyāvasthā mṛtāvasthā samutpannā bhavati
iḍāpiṅgalayoḥ prāṇasañcārābhāvāt. tatas tadanantaraṃ vāyuṃ virecayen nāsikāpuṭābhyāṃ śanais tyajet.
%\end{vsid}


%\begin{vsid}{#hp03_029}
\startsloka
{\bf mahāvedho 'yam abhyāsān mahāsiddhipradāyakaḥ
valīpalitavepaghnaḥ sevyate sādhakottamaiḥ (3.29)}
\stopsloka

mahāvedha iti. ayaṃ mahāvedho 'bhyāsāt punaḥ punar āvartanāt mahāsiddhayo 'ṇimādyās tāsāṃ
pradāyakaḥ prakarṣeṇa samarpakaḥ. valī jarayā dehe carmasaṅkocaḥ palitaṃ jarasā keśeṣu śauklyaṃ
vepaḥ kampaḥ tān hantīti valīpalitavepaghnaḥ. ata eva sādhakeṣv abhyāsiṣūttamāḥ sādhakottamās taiḥ
sevyate 'bhyasyata ity arthaḥ.
%\end{vsid}


%\begin{vsid}{#hp03_030}
\startsloka
{\bf etat trayaṃ mahāguhyaṃ jarāmṛtyuvināśanam
vahnivṛddhikaraṃ caiva hy aṇimādiguṇapradam (3.30)}
\stopsloka

mahāmudrādīnāṃ tisṛṇām atigopyatvam āha etad iti. etat trayaṃ\var{trayaṃ \lem jayaṃ \EdLo}
mahāmudrāditrayaṃ mahāguhyam atirahasyam. atra hetugarbhāṇi viśeṣaṇāni. hi yasmāt. jarā vārdhakaṃ
mṛtyuś caramaḥ prāṇadehaviyogaḥ tayor viśeṣeṇa nāśanam. vahner jāṭharasya vṛddhir dīptis tasyāḥ
karaṃ kartṛ aṇimā ādir yeṣāṃ te 'ṇimādayas te ca te guṇāś ca tān prakarṣeṇa dadātīty
aṇimādiguṇapradam. cakāra ārogyabindujayādisamuccayārthaḥ. evaśabdo 'vadhāraṇārthaḥ.
%\end{vsid}


%\begin{vsid}{#hp03_031}
\startsloka
{\bf aṣṭadhā kriyate caiva yāme yāme dine dine
puṇyasaṃbhārasandhāyi pāpaughabhiduraṃ sadā
samyakśikṣāvatām evaṃ svalpaṃ prathamasādhanam (3.31)}
\stopsloka

athaitat trayasya pṛthak sādhanaviśeṣam āha aṣṭadheti. dine dine pratidinam. yāme yāme prahare
prahare. paunaḥpunye dvirvacanam. aṣṭabhiḥ prakārair aṣṭadhā kriyate. caśabdo viśeṣe evaśabdo
'vadhāraṇe. etat trayam ity atrāpi sambadhyate. kīdṛśam puṇyasya\var{puṇyasya \Wthree P14 \lem puṇyāṇāṃ
  \EdLo \EdAd, puṇya \Tue \EdMu} sambhāraḥ samūhas tasya sandhāyi vidhāyi. punaḥ kīdṛśam. pāpānām
oghaḥ pūraḥ samūha iti yāvat. tasya bhiduraṃ kuliśam iva nāśanaṃ sadā sarvadā yadābhyastaṃ tadaiva
pāpanāśanam. samyak sāmpradāyikī śikṣā gurūpadeśo vidyate yeṣāṃ te tathā teṣām. evaṃ dine dine yāme
yāme 'ṣṭadhety uktarītyā pūrvaṃ prathamaṃ sādhanaṃ svalpasvalpam eva kāryam.
%\end{vsid}



%\begin{vsid}{#hp03_034}
\startsloka
{\bf atha khecarī
kapālakuhare jihvā praviṣṭā viparītagā
bhruvor antargatā dṛṣṭir mudrā bhavati khecarī (3.32)}
\stopsloka

khecarīṃ vivakṣur ādau tatsvarūpam āha kapāleti. kapāle mūrdhni kuharaṃ suṣiraṃ tasmin kapālakuhare
viparītaṃ pratīpaṃ gacchatīti viparītagā parāṅmukhībhūtā jihvā rasanā syāt. bhruvor antargatā
bhruvor madhye praviṣṭā dṛṣṭir darśanaṃ syāt. sā khecarīmudrā bhavati. kapālakuhare
jihvāpraveśapūrvakaṃ bhruvor antardarśanaṃ khecarīti lakṣaṇaṃ siddham.
%\end{vsid}


%\begin{vsid}{#hp03_033}
\startsloka
{\bf chedanacālanadohaiḥ kalāṃ krameṇātha vardhayet tāvat
sā yāvad bhrūmadhyaṃ spṛśati tadā khecarīsiddhiḥ  (3.33)}
\stopsloka

khecarīsiddher lakṣaṇam āha chedaneti. chedanam anupadam eva vakṣyamāṇam. cālanaṃ hastayor
aṅguṣṭhatarjanībhyāṃ rasanāṃ gṛhītvā savyāpasavyataḥ parivartanam. dohaḥ karayor
aṅguṣṭhatarjanībhyāṃ godohanavat dohanam. taiḥ kalāṃ jihvāṃ tāvat vardhayet dīrghāṃ kuryāt. tāvat
kiyat yāvat sā kalā bhrūmadhyaṃ bahir bhruvor madhyaṃ spṛśati yadā tadā khecaryāḥ siddhiḥ
khecarīsiddhir bhavati.
%\end{vsid}


%\begin{vsid}{#hp03_00}
\startsloka
{\bf snuhīpatranibhaṃ śastraṃ sutīkṣṇaṃ snigdhanirmalam
samādāya tatas tena romamātraṃ samucchinet (3.34)}
\stopsloka

tatsādhanam āha snuhīti. snuhī guḍā tasyāḥ patraṃ dalaṃ snuhīpatram. snuhīpatreṇa sadṛśaṃ
snuhīpatranibhaṃ sutīkṣṇam atitīkṣṇaṃ snigdhaṃ ca tan nirmalam ca snigdhanirmalam śastraṃ
chedanasādhanaṃ samādāya samyag ādāya gṛhītvā tataḥ śastragrahaṇānantaraṃ tena śastreṇa
romapramāṇaṃ romamātraṃ samucchinet samyag ucchinec chindyāt. rasanāmūlaśirām iti
karmādhyāhāraḥ.

\startsloka
miśre yo 'py atha sīhuṇḍo vajradruḥ snuk snuhī guḍā
\stopsloka

ity amaraḥ\comment{?? CHECK}.
%\end{vsid}

%\begin{vsid}{#hp03_00}
\startsloka
{\bf tataḥ saindhavapathyābhyāṃ cūrṇitābhyāṃ pragharṣayet
punaḥ saptadine prāpte romamātraṃ samucchinet (3.35)}
\stopsloka

tataś chedanānantaraṃ cūrṇitābhyāṃ cūrṇīkṛtābhyāṃ saindhavaṃ sindhudeśodbhavaṃ lavaṇaṃ pathyā
harītakī tābhyāṃ pragharṣayet prakarṣeṇa gharṣayet chinnaṃ śirāpradeśam. saptadinaparyantaṃ
chedanaṃ saindhavapathyābhyāṃ gharṣaṇaṃ ca sāyaṃ prātar vidheyam. yogābhyāsino lavaṇaniṣedhāt
khadirapathyācūrṇaṃ gṛhṇanti. mūle saindhavoktis tu haṭhābhyāsāt pūrvaṃ
khecarīsādhanābhiprāyeṇa. saptānāṃ dinānāṃ samāhāraḥ saptadinaṃ tasmin prāpte gate sati aṣṭame dina
ity arthāt. ye prāptyarthās te gatyarthāḥ. punaḥ pūrvachedanāpekṣayādhikaṃ romamātraṃ samucchinet.
%\end{vsid}


%\begin{vsid}{#hp03_00}
\startsloka
{\bf evaṃ krameṇa ṣaṇmāsaṃ nityaṃ yuktaḥ samācaret
ṣaṇmāsād rasanāmūlaśirābandhaḥ praṇaśyati (3.36)}
\stopsloka

evam iti. evaṃ krameṇa pūrvaṃ romamātracchedanaṃ saptadinaparyantaṃ tāvad eva sāyaṃprātaś chedenaṃ
gharṣaṇaṃ ca. aṣṭame dine 'dhikaṃ chedanam ity uktakrameṇa ṣaṇmāsaṃ ṣaṇmāsaparyantaṃ nityaṃ yuktaḥ
san samācaret samyag ācaret. chedanagharṣaṇe iti karmādhyāhāraḥ. ṣaṇmāsād anantaraṃ rasanā jihvā
tasyā mūlam adhobhāgo rasanāmūlaṃ tatra yā śirā kapālakuhararasanāsaṃyogapratibandhakībhūtā nāḍī
tayā bandho bandhanaṃ praṇaśyati prakarṣeṇa naśyati.
%\end{vsid}


%\begin{vsid}{#hp03_035}
\startsloka
{\bf kalāṃ parāṅmukhīṃ kṛtvā tripathe pariyojayet
sā bhavet khecarīmudrā vyomacakraṃ tad ucyate (3.37)}
\stopsloka

chedanādinā jihvāvṛddhau yat kartavyaṃ tad āha kalām iti. kalāṃ jihvāṃ parāṅmukham agraṃ yasyāḥ sā
tathā tāṃ parāṅmukhīṃ pratyaṅmukhīṃ kṛtvā tisṛṇāṃ nāḍīnāṃ panthāḥ tripathas tasmiṃstripathe
kapālakuhare pariyojayet saṃyojayet. sā tripathe rasanāpariyojanarūpā khecarīmudrā tad vyomacakram
ity ucyate vyomacakraśabdenocyate.
%\end{vsid}



%\begin{vsid}{#hp03_035}
\startsloka
{\bf rasanām ūrdhvagāṃ kṛtvā kṣaṇārdham api tiṣṭhati
viṣair vimucyate yogī vyādhimṛtyujarādibhiḥ (3.38)}
\stopsloka

atha khecarīguṇāḥ rasanām iti. ūrdhvaṃ tālūpari vivaraṃ gacchatīti ūrdhvagā tāṃ tādṛśīṃ rasanām
jihvāṃ kṛtvā kṣaṇārdham kṣaṇasya muhūrtasya ardhaṃ kṣaṇārdham ghaṭikāmātram api khecarīmudrā
tiṣṭhati cet tarhi yogī viṣaiḥ sarpavṛścikādiviṣair vimucyate viśeṣeṇa mucyate. vyādhir
dhātuvaiṣamyaṃ mṛtyuś caramaḥ prāṇadehaviyogaḥ jarā vṛddhāvasthā tā ādayo yeṣāṃ valyādīnāṃ taiś ca
vimucyate.

\startsloka
utsave ca prakoṣṭhe ca muhūrte niyame tathā
kṣaṇaśabdo vyavasthāyāṃ samaye 'pi nigadyate 
\stopsloka

iti nānārthaḥ\comment{REF?}
%\end{vsid}


%\begin{vsid}{#hp03_036}
\startsloka
{\bf na rogo maraṇaṃ tandrā na nidrā na kṣudhā tṛṣā
na ca mūrcchā bhavet tasya yo mudrāṃ vetti khecarīm  (3.39)}
\stopsloka

na roga iti. yaḥ khecarīṃ mudrāṃ vetti tasya rogo na maraṇaṃ na tandrā
tāmasāntaḥkaraṇavṛttiviśeṣo na nidrā na kṣudhā tṛṣā pipāsā na mūrcchā cittasya
tamasābhibhūtāvasthāviśeṣaś ca na bhavet.
%\end{vsid}


%\begin{vsid}{#hp03_40}
\startsloka
{\bf pīḍyate na sa rogeṇa lipyate na ca karmaṇā
bādhyate na sa kālena yo mudrāṃ vetti khecarīm (3.40)}
\stopsloka

pīḍyata iti. yaḥ khecarīṃ mudrāṃ vetti sa rogeṇa jvarādinā na pīḍyate pramādāj jātenāśubhena
karmaṇā lokasaṅgrahārthakṛtaśubhena ca karmaṇā na lipyate. kālena mṛtyunā sa na bādhyate na
hanyate.
%\end{vsid}

%\begin{vsid}{#hp03_00}
\startsloka
{\bf cittaṃ carati khe yasmāj jihvā carati khe gatā
tenaiṣā khecarī nāma mudrā siddhair nirūpitā (3.41)}
\stopsloka

cittam iti. yasmād dhetoś cittam antaḥkaraṇaṃ khe bhruvor antaravakāśe carati jihvā khe tatraiva
gatā satī carati. tena hetunā eṣā kathitā mudrā khecarī nāma khecarīti prasiddhā. nāmeti prasiddhāv
avyayam. siddhaiḥ kapilādibhir nirūpitā. khe bhruvor antar vyomni carati gacchati cittaṃ jihvā ca
yasyāṃ sā khecarīty avayavaśaktyā vyutpāditā. ukteṣu triṣu ślokeṣu vyādhyādīnāṃ punaruktis tu teṣāṃ
ślokānāṃ saṃgṛhītatvān na doṣāya.
%\end{vsid}


%\begin{vsid}{#hp03_00}
\startsloka
{\bf khecaryā mudritaṃ yena vivaraṃ lambikordhvataḥ
na tasya kṣarate binduḥ kāminyāśleṣitasya ca (3.42)}   
\stopsloka

khecaryeti. yena yoginā khecaryā mudrayā lambikāyā ūrdhvam iti lambikordhvataḥ. sārvibhaktikas
tasiḥ. lambikā tālu tasyā ūrdhvata uparibhāge sthitaṃ vivaraṃ chidraṃ mudritaṃ pihitam. kāminyā
yuvatyāśleṣitasyāliṅgitasyāpi. caśabdo 'pyarthe. tasya binduḥ vīryaṃ na kṣarate na skhalati.
%\end{vsid}

%\begin{vsid}{#hp03_00}
\startsloka
{\bf calito 'pi yadā binduḥ samprāpto yonimaṇḍalam
vrajaty ūrdhvaṃ hṛtaḥ śaktyā nibaddho yonimudrayā (3.43)}
\stopsloka

calita iti. calito 'pi skhalito 'pi bindur yadā yasmin kāle yonimaṇḍalam yonisthānaṃ samprāptaḥ
saṃgatas tadaiva yonimudrayā meḍhrākuñcanarūpayā. etena vajrolīmudrā sūcitā. nibaddho nitarāṃ
baddhaḥ śaktyākarṣaṇaśaktyā hṛtaḥ ākṛṣṭa ūrdhvaṃ vrajati. suṣumnāmārgeṇa bindusthānaṃ gacchati.
%\end{vsid}

%\begin{vsid}{#hp03_00}
\startsloka
{\bf ūrdhvajihvaḥ sthiro bhūtvā somapānaṃ karoti yaḥ
māsārdhena na sandeho mṛtyuṃ jayati yogavit (3.44)}
\stopsloka

ūrdhvajihva iti. ūrdhvā lambikordhvavivaronmukhā jihvā yasya sa ūrdhvajihvaḥ sthiro niścalo bhūtvā
somasya lambikordhvavivaragalitacandrāmṛtasya pānaṃ somapānaṃ yaḥ pumān karoti yogaṃ vettīti
yogavit sa māsasyārdhaṃ māsārdhaṃ tena māsārdhena pakṣeṇa mṛtyuṃ maraṇaṃ jayati abhibhavati. na
sandehaḥ niścitam etad ity arthaḥ.
%\end{vsid}


%\begin{vsid}{#hp03_00}
\startsloka
{\bf nityaṃ somakalāpūrṇaṃ śarīraṃ yasya yoginaḥ
takṣakeṇāpi daṣṭasya viṣaṃ tasya na sarpati (3.45)}
\stopsloka

nityam iti. yasya yoginaḥ śarīraṃ nityaṃ pratidinaṃ somakalāpūrṇaṃ candrakalāmṛtapūrṇaṃ tasya
takṣakeṇa sarpaviśeṣeṇāpi daṣṭasya daṃśitasya yoginaḥ śarīre viṣaṃ garalaṃ tajjanyaṃ duḥkhaṃ na
sarpati na prasarati.
%\end{vsid}


%\begin{vsid}{#hp03_00}
\startsloka
{\bf indhanāni yathā vahnis tailavartiṃ ca dīpakaḥ
tathā somakalāpūrṇaṃ dehī dehaṃ na muñcati (3.46)}
\stopsloka

yathā vahnir agniḥ indhanāni kāṣṭhādīni na muñcati dīpako dīpaḥ tailavartiṃ ca tailayuktāṃ vartiṃ na
muñcati tathā somakalāpūrṇaṃ candrakalāmṛtapūrṇaṃ dehaṃ śarīraṃ dehī jīvo na muñcati na tyajati.
%\end{vsid}

% anti-Tantric, for B. it is by Svātmārāma. 

%\begin{vsid}{#hp03_00}
\startsloka
{\bf gomāṃsaṃ bhakṣayen nityaṃ pibed amaravāruṇīm
kulīnaṃ tam ahaṃ manye itare kulaghātakāḥ (3.47)}
\stopsloka

gomāṃsam iti. gomāṃsaṃ pāribhāṣikaṃ vakṣyamāṇaṃ yo bhakṣayen nityaṃ pratidinam amaravāruṇīm api
vakṣyamāṇāṃ pibet taṃ yoginam. aham iti granthakāroktiḥ. kule jātaḥ kulīnaḥ tam satkulotpannaṃ
manye. tad uktaṃ brahmavaivarte\comment{???}


\startsloka
kṛtārthau pitarau tena dhanyo deśaḥ kulaṃ ca tat 
jāyate yogavān yatra dattam akṣayyatāṃ vrajet
dṛṣṭaḥ sambhāṣitaḥ spṛṣṭaḥ puṃprakṛtyor vivekavān
bhavakoṭiśatopāttaṃ punāti vṛjinaṃ nṛṇām iti 
\stopsloka

brahmāṇḍapurāṇe\comment{???}

\startsloka
gṛhasthānāṃ sahasreṇa vānaprasthaśatena ca
brahmacārisahasreṇa yogābhyāsī viśiṣyate
\stopsloka

rājayoge vāmadevaṃ prati śivavākyaṃ\comment{???}

\startsloka
rājayogasya māhātmyaṃ ko vijānāti tattvataḥ
tajjñānī vasate yatra sa deśaḥ puṇyabhājanam
darśanād arcanād asya triḥsaptakulasaṃyutāḥ 
ajñā muktipadaṃ yānti kiṃ punas tatparāyaṇāḥ
antaryogaṃ bahiryogaṃ yo jānāti viśeṣataḥ
tvayā mayāpy asau vandyaḥ śeṣair vandyās tu kiṃ punaḥ 
\stopsloka

iti.

kūrmapurāṇe\comment{???}

\startsloka
ekakālaṃ dvikālaṃ vā trikālaṃ nityam eva vā
ye yuñjate mahāyogaṃ vijñeyās te maheśvarāḥ
\stopsloka

iti. itare vakṣyamāṇagomāṃsabhakṣaṇāmaravāruṇīpānarahitā ayoginas te kulaghātakāḥ
kulanāśakāḥ satkule jātasya janmano vaiyarthyāt.
%\end{vsid}


%\begin{vsid}{#hp03_00}
\startsloka
{\bf gośabdenoditā jihvā tat praveśo hi tāluni
gomāṃsabhakṣaṇaṃ tat tu mahāpātakanāśanam  (3.48)}
\stopsloka

gomāmṣaśabdārtham āha gośabdeneti. gośabdena go ityākārakeṇa śabdena gopadenety arthaḥ. jihvā
rasanoditā kathitā. tālunīti sāmīpikādhāre saptamī. tālusamīpordhvavivare tasyā jihvāyāḥ praveśo
gomāṃsabhakṣaṇaṃ gomāṃsabhakṣaṇaśabdavācyam. tat tu tādṛśaṃ gomāṃsabhakṣaṇaṃ tu mahāpātakānāṃ
svarṇasteyādīnāṃ nāśanam.
%\end{vsid}


%\begin{vsid}{#hp03_00}
\startsloka
{\bf jihvāpraveśasambhūtavahninotpāditaḥ khalu
candrāt sravati yaḥ sāraḥ sā syād amaravāruṇī  (3.49)}
\stopsloka

amaravāruṇīśabdārtham āha jihveti. jihvāyāḥ praveśo lambikordhvavivare praveśanaṃ tasmāt sambhūto
jāto yo vahnir ūṣmā tenotpādito niṣpāditaḥ. atra vahniśabdenauṣṇyam upalakṣyate. yaḥ sāraḥ candrād
bhruvor antarvāmabhāgasthāt somāt sravati galati sā amaravāruṇī syād amaravāruṇīpadavācyā bhavet.
%\end{vsid}


%\begin{vsid}{#hp03_00}
\startsloka
{\bf cumbantī yadi lambikāgram aniśaṃ jihvā rasasyandinī 
sakṣārā kaṭukāmladugdhasadṛśī madhvājyatulyā tathā
vyādhīnāṃ haraṇaṃ jarāntakaraṇaṃ śastrāgamodīraṇaṃ
tasya syād amaratvam aṣṭaguṇitaṃ siddhāṅganākarṣaṇam  (3.50)}
\stopsloka

cumbantīti. yadi cel lambikāgram lambikordhvavivaraṃ cumbantī spṛśantī. aniśaṃ nirantaram. ata eva
rasasya somakalāmṛtasya syandaḥ syandanaṃ prasravaṇam asyām astīti rasasyandinī yasya
jihvā. kṣāreṇa lavaṇarasena sahitā sakṣārā. kaṭukaṃ marīcādi. āmlaṃ ciñcāphalādi. dugdhaṃ payas.
taiḥ sadṛśī samānā. madhu kṣaudram. ājyaṃ ghṛtaṃ. tābhyāṃ tulyā samā. tathāśabdaḥ samuccaye. etair
viśeṣaṇai rasasyānekarasavattvād madhuratvāt snigdhatvāc ca jihvāyā api rasasyandane tathātvam
uktam. tarhi tasya vyādhīnāṃ rogāṇāṃ haraṇam apagamo jarāyā vṛddhāvasthāyā antaḥkaraṇaṃ nāśanaṃ
śastrāṇām āyudhānām āgamaḥ svābhimukhāgamanaṃ tasyodīraṇaṃ nivāraṇam. aṣṭau guṇā aṇimādayas te asya
sañjātā ity aṣṭaguṇitam amaratvam amarabhāvaḥ siddhānām aṅganāḥ siddhāṅganāḥ siddhāś ca tā
aṅganāś ceti vā tāsām ākarṣaṇam ākarṣaṇaśaktiḥ syāt.
%\end{vsid}

% sequence of tastes acc. to Jim. not presence of all tastes. Khecari, Intro, p. 22. 

%\begin{vsid}{#hp03_00}
\startsloka
{\bf mūrdhnaḥ ṣoḍaśapatrapadmagalitaṃ prāṇād avāptaṃ haṭhād
ūrdhvāsyo rasanāṃ niyamya vivare śaktiṃ parāṃ cintayan 
utkallolakalājalaṃ ca vimalaṃ dhārāmayaṃ yaḥ piben 
nirvyādhiḥ sa mṛṇālakomalavapur yogī ciraṃ jīvati (3.51)}
\stopsloka

mūrdhna iti. rasanāṃ jihvāṃ vivare kapālakuhare niyamya saṃyojya. ūrdhvam uttānam āsyaṃ yasya saḥ
ūrdhvāsya ity anena viparītakaraṇī sūcitā. parāṃ śaktiṃ kuṇḍalinīṃ cintayan dhyāyan san prāṇāt
sādhanabhūtāt. ṣoḍaśa patrāṇi dalāni yasya tat ṣoḍaśapatraṃ tac ca tatpadmaṃ ca kaṇṭhasthāne
vartamānaṃ tasmin galitaṃ haṭhād haṭhayogād avāptaṃ prāptaṃ vimalaṃ nirmalam dhārāmayaṃ dhārārūpam
utkallolam uttaraṅgaṃ ca tat kalājalaṃ somakalārasaṃ yaḥ pumān pibet dhayet sa yogī nirgatā vyādhayo
jvarādayo yasmāt sa nirvyādhiḥ san yad vā nirgatā vividhā trividhā ādhayo mānasī vyathā yasmāt sa
tādṛśaḥ san mṛṇālaṃ bisam iva komalaṃ mṛdulaṃ vapuḥ śarīraṃ yasya sa mṛṇālakomalavapuś ca san ciraṃ
dīrghakālaṃ jīvati prāṇān dhārayati. haṭhād dhaṭhayogāt. prāṇāt sādhanabhūtād avāptam iti vā
yojanā. prāṇair iti kvacit pāṭhaḥ.
%\end{vsid}


%\begin{vsid}{#hp03_00}
\startsloka
{\bf yat prāleyaṃ prahitasuṣiraṃ merumūrdhāntarasthaṃ
tasmiṃs tattvaṃ pravadati sudhīs tanmukhaṃ nimnagānām
candrāt sāraḥ sravati vapuṣas tena mṛtyur narāṇāṃ
tad badhnīyāt sukaraṇam adho nānyathā kāyasiddhiḥ  (3.52)}
\stopsloka

yat prāleyam iti. meruvat sarvonnatā suṣumnā merus tasya mūrdhā uparibhāgaḥ tasyāntare madhye
tiṣṭhatīti merumūrdhāntarasthaṃ yat prāleyaṃ somakalājalaṃ prahitaṃ nihitaṃ yasmin tat tathā tac
ca tat suṣiraṃ vivaraṃ tasmin vivare sudhīḥ śobhanā rajastamobhyām anabhibhūtasattvā dhīr yasya
saḥ tattvam ātmatattvaṃ pravadati prakarṣeṇa vadati.

\startsloka
tasyāḥ śikhāyā madhye paramātmā vyavasthitaḥ\comment{ma.nā.u. 11.13\lem ??}
\stopsloka

iti śruteḥ.

ātmano vibhutve khecarīmudrāyāṃ tatrābhivyaktis tasmiṃs tattvam ity uktam. nimnagānām
gaṅgāyamunāsarasvatīnarmadādiśabdavācyānām iḍāpiṅgalāsuṣumnāgāndhārīprabhṛtīnāṃ tat tasmin vivare
tatsamīpe mukham agram asti candrāt somād vapuṣaḥ śarīrasya sāraḥ sārabhūto rasaḥ sravati kṣarati
tena candrasārakṣaraṇena narāṇāṃ manuṣyāṇāṃ mṛtyur maraṇaṃ bhavati. ato hetos tatpūrvoditaṃ
sukaraṇaṃ śobhanaṃ karaṇaṃ khecarīmudrākhyaṃ badhnīyāt. sukaraṇe baddhe candrasārasravaṇābhāvān
mṛtyur na syād iti bhāvaḥ. anyathā sukaraṇabandhanābhāve kāyasya dehasya siddhī
rūpalāvaṇyabalavajrasaṃhananarūpā na syāt.
%\end{vsid}


%\begin{vsid}{#hp03_00}
\startsloka
{\bf suṣiraṃ jñānajanakaṃ pañcasrotaḥsamanvitam
tiṣṭhate khecarīmudrā tasmin śūnye nirañjane  (3.53)}
\stopsloka

suṣiram iti. pañca yāni srotāṃsi iḍādīnāṃ pravāhāḥ taiḥ samanvitam samyag anugatam
saptasrotaḥsamanvitam iti kvacit pāṭhaḥ. jñānajanakam alaukikābādhitātmasākṣātkārajanakaṃ yat
suṣiraṃ vivaraṃ tasmin suṣire 'ñjanam avidyā tatkāryaṃ śokamohādi ca nirgataṃ yasmāt tan nirañjanaṃ
tasmin nirañjane śūnye suṣirāvakāśe khecarīmudā tiṣṭhate sthirībhavati. prakāśanastheyākhyayoś ca
\comment{Aṣṭādhyāyī 1.3.23.} ity ātmanepadam.
%\end{vsid}


%\begin{vsid}{#hp03_00}
\startsloka
{\bf ekaṃ sṛṣṭimayaṃ bījam ekā mudrā ca khecarī
eko devo nirālamba ekāvasthā manonmanī  (3.54)}
\stopsloka

ekam iti. sṛṣṭimayaṃ sṛṣṭirūpaṃ praṇavākhyaṃ bījam ekaṃ mukhyam. tad uktaṃ māṇḍūkyopaniṣadi

om ity etad akṣaram idaṃ sarvam\comment{Māṇḍūkyopaniṣad 1.}

iti.

khecarīmudrā ekā mukhyā. nirālamba ālambanaśūnya eko mukhyo devaḥ. ālambanaparityāgenātmanaḥ
svarūpāvasthānāt. unmanyavasthaikā mukhyā. eke mukhyānyakevalāḥ\comment{Amarakośa ??} ity
amaraḥ. bījādiṣu praṇavādivan mudrāsu khecarī mukhyety arthaḥ.
%\end{vsid}


%\begin{vsid}{#hp03_00}
\startsloka
{\bf atha uḍḍīyānabandhaḥ
baddho yena suṣumṇāyāṃ prāṇas tūḍḍīyate yataḥ
tasmād uḍḍīyanākhyo 'yaṃ yogibhiḥ samudāhṛtaḥ (3.55)}
\stopsloka

uḍḍīyānabandhaṃ vivakṣus tāvad uḍḍīyānaśabdārtham āha baddha iti. yato yasmād dhetor yena bandhena
baddho niruddhaḥ prāṇaḥ suṣumnāyāṃ madhyanāḍyām uḍḍīyate suṣumnāṃ vihāyasā gacchati tasmāt kāraṇād
ayaṃ bandho yogibhir matsyendrādibhir ? check uḍḍīya(ā)?  uḍḍīyānam ākhyābhidhā yasya sa
uḍḍīyānākhyaḥ samudāhṛtaḥ samyag vyutpattyodāhṛtaḥ kathitaḥ. suṣumnāyām uḍḍīyate 'nena baddhaḥ
prāṇaḥ ity uḍḍīyānam. utpūrvāḍ ḍīṅ vihāyasā gatāv ity asmāt karaṇe lyuṭ.
%\end{vsid}


%\begin{vsid}{#hp03_00}
\startsloka
{\bf uḍḍīnaṃ kurute yasmād aviśrāntaṃ mahākhagaḥ
uḍḍīyānaṃ tad eva syāt tatra bandho 'bhidhīyate (3.56)}
\stopsloka

uḍḍīṇam iti. mahāṃś cāsau khagaś ca mahākhagaḥ prāṇaḥ. sarvadā dehāvakāśe gatimattvāt. yasmād
bandhād aviśrāntaṃ yathā syāt tathoḍḍīnaṃ vihaṅgamagatiṃ kurute. suṣumnāyām ity adhyāhāryam. tad
eva bandhaviśeṣa eva uḍḍīyānam uḍḍīyānanāmakaṃ syāt. tatra tasmin viṣaye bandho 'bhidhīyate
bandhasvarūpaṃ kathyate mayeti śeṣaḥ.
%\end{vsid}


%\begin{vsid}{#hp03_00}
\startsloka
{\bf udare paścimaṃ tānaṃ nābher ūrdhvaṃ ca kārayet
uḍḍīyāno hy asau bandho mṛtyumātaṅgakesarī (3.57)}
\stopsloka

uḍḍīyānabandham āha udara iti. udare tunde nābher ūrdhvaṃ cakārād adha uparibhāge 'dhobhāge ca
paścimaṃ tānaṃ paścimam ākarṣaṇaṃ nābher ūrdhvādhobhāgau yathā pṛṣṭhasaṃlagnau syātāṃ tathā tānaṃ
tānanaṃ nāmākarṣaṇaṃ kārayet kuryāt. ṇijartho 'vivakṣitaḥ. asau nābher ūrdhvādhobhāgayos tānanarūpa
uḍḍīyāna uḍḍīyānākhyo bandhaḥ. kīdṛśaḥ mṛtyur eva mātaṅgo gajas tasya kesarī siṃhaḥ siṃha iva
nivartakaḥ.
%\end{vsid}


%\begin{vsid}{#hp03_00}
\startsloka
{\bf uḍḍīyānaṃ tu sahajaṃ guruṇā kathitaṃ sadā
abhyaset satataṃ yas tu vṛddho 'pi taruṇāyate (3.58)}
\stopsloka

uḍḍīyānaṃ tv iti. gurur hitopadeṣṭā tena guruṇā uḍḍīyānaṃ tu sadā sarvadā sahajaṃ svābhāvikaṃ
kathitam. prāṇasya bahirgamanaṃ sarvadā sarvasyaiva jāyamānatvāt. yas tu yaḥ puruṣas tu satataṃ
nirantaram abhyaset. uḍḍīyānam ity atrāpi sambadhyate. sa tu vṛddho 'pi sthaviro 'pi taruṇāyate
taruṇa ivācarati taruṇāyate.
%\end{vsid}

% Rewritten by Haṭharatnāvalī  guruṇā sahajaṃ proktam vṛddho 'pi .. (check)


%\begin{vsid}{#hp03_00}
\startsloka
{\bf nābher ūrdhvam adhaś cāpi tānaṃ kuryāt prayatnataḥ 
ṣaṇmāsam abhyasen mṛtyuṃ jayaty eva na saṃśayaḥ (3.59)}
\stopsloka

nābher iti. nābher ūrdhvam uparibhāge 'dhaś cāpy adhobhāge 'pi prayatnataḥ prakṛṣṭo yatnaḥ
prayatnas tasmāt prayatnataḥ. yatnaviśeṣāt tānaṃ paścimatānaṃ kuryāt. pūrvārdhena uḍḍīyānasvarūpam
uktam. atha tatpraśaṃsā yaḥ ṣaṇmāsam ṣaṇmāsaparyantam uḍḍīyānam ity adhyāhāraḥ abhyaset punaḥ
punar anutiṣṭhet. sa mṛtyuṃ jayaty eva saṃśayo na atra sandeho nāstīty arthaḥ.
%\end{vsid}

%\begin{vsid}{#hp03_00}
\startsloka
{\bf sarveṣām eva bandhānām uttamo hy uḍḍīyānakaḥ
uḍḍiyāne dṛḍhe bandhe muktiḥ svābhāvikī bhavet (3.60)}
\stopsloka

sarveṣām iti. sarveṣāṃ bandhānām ṣoḍaśādhārabandhānāṃ madhye uḍḍīyānaka eva uḍḍīyānabandha eva
svārthe kapratyayaḥ uttama utkṛṣṭaḥ hi yasmād uḍḍīyāne bandhe dṛḍhe sati svābhāvikī
svabhāvasiddhaiva muktir bhavet. uḍḍīyānabandhe kṛte vihaṅgamagatyā suṣumnāyāṃ prāṇasya mūrdhni
gamanāt samādhau mokṣam āpnoti iti vākyāt\comment{Gorakṣaśataka 113c.} sahajaiva muktiḥ syād iti
bhāvaḥ.
%\end{vsid}


%\begin{vsid}{#hp03_00}
\startsloka
{\bf atha mūlabandhaḥ
pārṣṇibhāgena sampīḍya yonim ākuñcayed gudam
apānam ūrdhvam ākṛṣya mūlabandho 'bhidhīyate (3.61)}
\stopsloka

mūlabandham āha pārṣṇibhāgeneti. pārṣṇibhāgaḥ pārṣṇer bhāgo gulphayor adhaḥpradeśas tena yoniṃ
yonisthānam gudameḍhrayor madhyabhāgaṃ sampīḍya samyak pīḍayitvā gudam pāyum ākuñcayed
saṅkocayet. apānam adhogatiṃ vāyum ūrdhvam upary ākṛṣyākṛṣṭaṃ kṛtvā mūlabandho 'bhidhīyate
kathyate. pārṣṇibhāgena yonisthānasampīḍanapūrvakaṃ gudasyākuñcanaṃ mūlabandha ity ucyate ity
arthaḥ.
%\end{vsid}


%\begin{vsid}{#hp03_00}
\startsloka
{\bf adhogatim apānaṃ vai ūrdhvagaṃ kurute balāt
ākuñcanena taṃ prāhur mūlabandhaṃ hi yoginaḥ (3.62)}
\stopsloka

adhogatim iti. yaḥ adhogatim adho 'rvāggatir yasya sa tathā tam apānam apānavāyum ākuñcanena
mūlādhārasya saṅkocanena balād haṭhād ūrdhvaṃ gacchatīty ūrdhvagaḥ tam ūrdhvagaṃ suṣumnāyām
ūrdhvagamanaśīlaṃ kurute. vai iti niścaye 'vyayam. yogino yogābhyāsinas taṃ mūlabandhaṃ mūlasya
mūlasthānasya bandhanaṃ mūlabandhas taṃ mūlabandham ity anvarthaṃ prāhuḥ. anena mūlabandhaśabdārtha
uktaḥ. pūrvaślokena tu tasya bandhanaprakāra ukta ity apaunaruktyam.
%\end{vsid}


%\begin{vsid}{#hp03_00}
\startsloka
{\bf gudaṃ pārṣṇyā tu sampīḍya vāyum ākuñcayed balāt
vāraṃ vāraṃ yathā cordhvaṃ samāyāti samīraṇaḥ  (3.63)}
\stopsloka

% B. knows the Southern recension of the Yogabīja (16th) according to Birch.

atha yogabījoktarītyā\comment{Yogabīja 103 (Southern recension).} mūlabandham āha gudam
iti. pārṣṇyā gulphayor adhobhāgena gudaṃ pāyuṃ sampīḍya samyak pīḍayitvā saṃyojyety
arthaḥ. tuśabdaḥ pūrvasmād asya viśeṣatvadyotakaḥ. yathā yena prakāreṇa samīraṇo vāyur ūrdhvaṃ
suṣumnāyā uparibhāge yāti gacchati tathā tena prakāreṇa balād haṭhād vāraṃ vāraṃ punaḥ punar vāyum
apānam ākuñcayed gudasyākuñcanenākarṣayet. ayaṃ mūlabandha iti vākyādhyāhāraḥ.
%\end{vsid}


%\begin{vsid}{#hp03_00}
\startsloka
{\bf prāṇāpānau nādabindū mūlabandhena caikatām
gatvā yogasya saṃsiddhiṃ yacchato nātra saṃśayaḥ (3.64)}
\stopsloka

atha mūlabandhaguṇān āha prāṇāpānau iti. prāṇaś cāpānaś ca prāṇāpānau ūrdhvādhogatī vāyū. nādo
'nāhatadhvaniḥ bindur anusvāras tau mūlabandhenaikatām gatvaikībhūya yogasya saṃsiddhiṃ samyak
siddhis tāṃ yogasiddhiṃ yacchato dattaḥ. abhyāsina iti śeṣaḥ. atrāsminn arthe saṃśayo na sandeho
nāstīty arthaḥ. ayaṃ bhāvaḥ mūlabandhe kṛte 'pānaḥ prāṇena sahaikībhūya suṣumnāyāṃ praviśati. tato
nādābhivyaktir bhavati. tato nādena saha prāṇāpānau hṛdayopari gatvā nādasya bindunā sahaikyam
vidhāya mūrdhni gacchataḥ. tato yogasiddhiḥ.
%\end{vsid}


%\begin{vsid}{#hp03_00}
\startsloka
{\bf apānaprāṇayor aikyaṃ kṣayo mūtrapurīṣayoḥ
yuvā bhavati vṛddho 'pi satataṃ mūlabandhanāt (3.65)}
\stopsloka

apānaprāṇayor iti. satataṃ santataṃ mūlabandhanān mūlabandhamudrākaraṇād apānaprāṇayor aikyaṃ
bhavati. mūtrapurīṣayoḥ sañcitayoḥ kṣayaḥ patanaṃ bhavati. vṛddho 'pi sthaviro 'pi yuvā taruṇo
bhavati.
%\end{vsid}


%\begin{vsid}{#hp03_00}
\startsloka
{\bf apāna ūrdhvage jāte prayāte vahnimaṇḍalam
tadānalaśikhā dīrghā jāyate vāyunāhatā (3.66)}
\stopsloka

apāna iti. mūlabandhanād apāne adhogamanaśīle vāyau ūrdhvage ūrdhvaṃ gacchatīty ūrdhvagaḥ tasmin
tādṛśe sati vahnimaṇḍalam vahner maṇḍalam trikoṇaṃ nābher adhobhāge 'sti. tad uktaṃ
yājñavalkyena\comment{Yogayājñyavalkya  4.11cd–131b}
%\end{vsid}

\startsloka
dehamadhye śikhisthānaṃ taptajāmbūnadaprabham
trikoṇaṃ tu manuṣyānāṃ  caturasraṃ catuṣpadām
maṇḍalaṃ tu pataṅgānāṃ  satyam etad bravīmi te
tanmadhye tu śikhā tanvī sadā tiṣṭhati pāvake
\stopsloka

iti.  tadā tasmin kāle vāyunā apānenāhatā saṅgatā saty analaśikhā jaṭharāgniśikhā dīrghā āyatā
jāyate. vardhata iti kvacit pāṭhaḥ.
%\end{vsid}


%\begin{vsid}{#hp03_00}
\startsloka
{\bf tato yāto vahnyapānau prāṇam uṣṇasvarūpakam
tenātyantapradīptas tu jvalano dehajas tathā  (3.67)}
\stopsloka

tata iti. tatas tadanantaraṃ vahniś cāpānaś ca vahnyapānau. uṣṇaṃ svarūpaṃ yasya sa tathā tam
analaṃ śikhādairghyād uṣṇasvarūpaṃ prāṇam ūrdhvagatim anilaṃ yāto gacchataḥ. tato
'nalaśikhādairghyād uṣṇasvarūpakam iti vā yojanā. tena prāṇasaṅgamanena dehe jāto dehajo jvalano
'gnir atyantam adhikaṃ dīpto bhavati. tatheti pādapūraṇe. apānasyordhvagamanena dīpta eva jvalanaḥ
prāṇasaṅgatyātyantaṃ pradīpto bhavatīty arthaḥ.
%\end{vsid}


%\begin{vsid}{#hp03_00}
\startsloka
{\bf tena kuṇḍalinī suptā santaptā samprabudhyate
  daṇḍāhatā bhujaṅgīva niśvasya\comment{\EdLo reads {\em niḥśvasya} against
    all sources checked.} ṛjutāṃ vrajet (3.68)}
\stopsloka

teneti. tena jvalanasyātyantapradīpanena santaptā samyak taptā satī suptā nidritā kuṇḍalinī śaktiḥ
samprabudhyate samyak prabuddhā bhavati. daṇḍenāhatā daṇḍāhatā. sā cāsau bhujaṅgīva sarpiṇīva
niśvasya niśvāsaṃ kṛtvā ṛjutāṃ saralatāṃ vrajed gacchet.
%\end{vsid}


%\begin{vsid}{#hp03_00}
\startsloka
{\bf bilaṃ praviṣṭeva tato brahmanāḍyantaraṃ vrajet
tasmān nityaṃ mūlabandhaḥ kartavyo yogibhiḥ sadā (3.69)}
\stopsloka

bilaṃ praviṣṭeti. tataḥ ṛjutāprāptyanantaraṃ bilaṃ vivaraṃ praviṣṭā bhujaṅgīva brahmanāḍī suṣumnā
tasyā antaraṃ madhyaṃ vrajet gacchet. tasmād dhetor yogibhir yogābhyāsibhir mūlabandho nityaṃ
pratidinaṃ sadā sarvasmin kāle kartavyaḥ kartuṃ yogyaḥ.
%\end{vsid}

%\begin{vsid}{#hp03_00}
\startsloka
{\bf atha jālandharabandhaḥ
kaṇṭham ākuñcya hṛdaye sthāpayec cibukaṃ dṛḍham
bandho jālandharākhyo 'yaṃ jarāmṛtyuvināśakaḥ (3.70)}
\stopsloka

jālandharabandham āha kaṇṭham iti. kaṇṭham galabilam ākuñcya saṅkocya hṛdaye vakṣaḥsamīpe
caturaṅgulāntaritapradeśe cibukaṃ hanuṃ dṛḍham sthiraṃ sthāpayet sthitaṃ kuryāt. ayaṃ
kaṇṭhākuñcanapūrvakaṃ caturaṅgulāntaritahṛdayasamīpe 'dhonamanayatnapūrvakaṃ cibukasthāpanarūpo
jālandhara ity ākhyayata iti jālandharākhyo jālandharanāmā bandhaḥ. kīdṛśaḥ jarā vṛddhāvasthā
mṛtyur maraṇaṃ tayor vināśako viśeṣeṇa nāśayatīti vināśako vināśakartā.
%\end{vsid}


%\begin{vsid}{#hp03_00}
\startsloka
{\bf badhnāti hi śirājālam adhogāmi nabhojalam
tato jālandharo bandhaḥ kaṇṭhaduḥkhaughanāśanaḥ (3.71)}
\stopsloka

jālandharapadasyārtham āha badhnātīti. hi yasmāt. śirāṇāṃ nāḍīnāṃ jālam samudāyaṃ badhnāti. adho
gantuṃ śīlam asyety adhogāmi nabhasaḥ kapālakuharasya jalam amṛtaṃ ca badhnāti pratibadhnāti. tatas
tasmāj jālandharo jālandharanāmako 'nvartho bandhaḥ. jālaṃ sirājālaṃ (not śirā??) jalānāṃ samūho
jālaṃ ca dharatīti jālandharaḥ. kīdṛśaḥ kaṇṭhe galapradeśe yo duḥkhaugho vikārajāto duḥkhasamūhas
tasya nāśano nāśakartā.
%\end{vsid}


%\begin{vsid}{#hp03_00}
\startsloka
{\bf jālandhare kṛte bandhe kaṇṭhasaṃkocalakṣaṇe
na pīyūṣaṃ pataty agnau na ca vāyuḥ prakupyati (3.72)}
\stopsloka

jālandharaguṇān āha jālandhara iti. kaṇṭhasya galabilasya saṃkocanaṃ saṅkoca ākuñcanaṃ tad eva
lakṣaṇaṃ svarūpaṃ yasya saḥ kaṇṭhasaṃkocalakṣaṇaḥ tasmin tādṛśe jālandhare jālandharasaṃjñake
bandhe kṛte sati pīyūṣam amṛtam agnau jāṭhare 'nale na patati na sarati. vāyuś ca prāṇaś ca na
prakupyati nāḍyantare vāyor gamanaṃ prakopas taṃ na karotīty arthaḥ.
%\end{vsid}



%\begin{vsid}{#hp03_00}
\startsloka
{\bf kaṇṭhasaṃkocanenaiva dve nāḍyau stambhayed dṛḍham
madhyacakram idaṃ jñeyaṃ ṣoḍaśādhārabandhanam (3.73)}
\stopsloka

kaṇṭhasaṃkocaneneti. dṛḍham gāḍhaṃ kaṇṭhasaṃkocanenaiva kaṇṭha\-saṃkoca\-namātreṇa dve nāḍyau
iḍāpiṅgale stambhayed bandhayet. ayaṃ jālandhara iti kartṛpadādhyāhāraḥ. idaṃ kaṇṭhasthāne sthitaṃ
viśuddhākhyaṃ cakram madhyacakram madhyamaṃ cakram jñeyam. kīdṛśam ṣoḍaśādhārabandhanam
ṣoḍaśasaṅkhyākā ye ādhārā aṅguṣṭhā\-dhārādi\-brahma\-randhrāntās teṣāṃ bandhanaṃ bandhanakārakam.

\startsloka
aṅguṣṭhagulphajānūrusīvanīliṅganābhayaḥ
hṛd grīvā kaṇṭhadeśaś ca lambikā\var{lambikā \Wthree \Tue \EdMu \lem lambhikā \EdLo \EdAd} nāsikā tathā
bhrūmadhyaṃ ca lalāṭaṃ ca mūrdhā ca brahmarandhrakam
ete hi ṣoḍaśādhārāḥ kathitā yogipuṅgavaiḥ 
\stopsloka

teṣv ādhareṣu dhāraṇāyāḥ phalaviśeṣas tu gorakṣasiddhāntād\comment{siddhasiddhāntapaddhatiḥ 2.1025
  ??} avagantavyaḥ.
%\end{vsid}


%\begin{vsid}{#hp03_00}
\startsloka
{\bf mūlasthānaṃ samākuñcya uḍḍiyānaṃ tu kārayet
iḍāṃ ca piṅgalāṃ baddhvā vāhayet paścime pathi (3.74)}
\stopsloka

uktasya bandhatrayasyopayogam āha mūlasthānam iti. mūlasthānam ādhārabhūtam ādhārasthānaṃ
samākuñcya samyag ākuñcya uḍḍiyānaṃ nābheḥ paścimatānarūpaṃ bandhaṃ kārayet kuryāt. ṇijartho
'vivakṣitaḥ. iḍāṃ gaṅgāṃ piṅgalāṃ yamunāṃ ca baddhvā jālandharabandhenety
arthaḥ. kaṇṭhasaṅkocanenaiva dve nāḍyau stambhayed dṛḍham ity ukteḥ. paścime pathi suṣumnāmārge
vāhayet gamayet prāṇam iti śeṣaḥ.
%\end{vsid}


%\begin{vsid}{#hp03_00}
\startsloka
{\bf anenaiva vidhānena prayāti pavano layam
tato na jāyate mṛtyur jarārogādikaṃ tathā (3.75)}
\stopsloka

aneneti. anenaivoktenaiva vidhānena vidhinā pavanaḥ prāṇo layam sthairyaṃ
prayāti. gatyabhāvapūrvakaṃ brahmarandhre sthitiḥ prāṇasya layaḥ. tataḥ prāṇasya layān mṛtyur
jarārogādikaṃ. tathā cārthe. na jāyate nodbhavati. ādipadena valīpalitatandrālasyādikaṃ grāhyam.
%\end{vsid}



%\begin{vsid}{#hp03_00}
\startsloka
{\bf bandhatrayam idaṃ śreṣṭhaṃ mahāsiddhaiś ca sevitam
sarveṣāṃ haṭhatantrāṇāṃ sādhanaṃ yogino viduḥ (3.76)}
\stopsloka

bandhatrayam iti. idaṃ pūrvoktaṃ bandhatrayam śreṣṭhaṃ ṣoḍaśādhārabandheṣv atipraśastaṃ
mahāsiddhair matsyendrādibhiḥ cakārād vasiṣṭhādimunibhiḥ sevitam. sarveṣāṃ haṭhatantrāṇāṃ
haṭhopāyānāṃ sādhanaṃ siddhijanakaṃ yogino gorakṣādyā vidur jānanti.
%\end{vsid}


%\begin{vsid}{#hp03_00}
\startsloka
{\bf yat kiñcit sravate candrād amṛtaṃ divyarūpiṇaḥ
tat sarvaṃ grasate sūryas tena piṇḍo jarāyutaḥ (3.77)}
\stopsloka

viparītakaraṇīṃ vivakṣus tadupodghātatvena piṇḍasya jarākaraṇaṃ tāvad āha yat kiñcid iti. divyam
utkṛṣṭaṃ sudhāmayaṃ rūpaṃ yasya sa tathā tasmād divyarūpiṇaś candrāt somāt tālumūlasthād yat kiñcit
yat kim apy amṛtaṃ pīyūṣaṃ sravate patati tat sarvaṃ tat pīyūṣaṃ sūryo nābhistho 'nalātmakaḥ
grasate grāsīkaroti. tad uktaṃ gorakṣanāthena

\startsloka
nābhideśe sthito nityaṃ  bhāskaro dahanātmakaḥ
amṛtātmā sthito nityaṃ tālumūle ca candramāḥ
varṣaty adhomukhaś candro grasaty ūrdhvamukho raviḥ
karaṇaṃ tac ca kartavyaṃ  yena pīyūṣam āpyate\comment{go.śa. 5758, go.pa. 1.3233 ??} 
\stopsloka

iti. tena sūryakartṛkāmṛtagrasanena piṇḍo deho jarāyutaḥ jarasā yukto bhavati.
%\end{vsid}


%\begin{vsid}{#hp03_00}
\startsloka
{\bf tatrāsti karaṇaṃ divyaṃ sūryasya mukhavañcanam
gurūpadeśato jñeyaṃ na tu śāstrārthakoṭibhiḥ (3.78)}
\stopsloka

tatreti. tatra tadviṣaye sūryasya nābhisthānalasya mukhaṃ vañcyate 'neneti tādṛśaṃ divyam uttamaṃ
karaṇaṃ vakṣyamāṇamudrākhyam asti. tad gurūpadeśato gurūpadeśāj jñeyaṃ jñātuṃ śakyam. śāstrārthānāṃ
koṭibhiḥ na tu naiva jñātuṃ śakyam.
%\end{vsid}


%\begin{vsid}{#hp03_00}
\startsloka
{\bf ūrdhvanābher adhastālor ūrdhvaṃ bhānur adhaḥ śaśī
karaṇī viparītākhyā guruvākyena labhyate  (3.79)}
% \Wthree add: karaṇaṃ viparītākhyaṃ sarvaṃ vyādhivināśanam
% commentary ends abruptly on 208, folio lost?
\stopsloka

viparītakaraṇīm āha ūrdhvanābher iti. ūrdhvam uparibhāge nābhir yasya sa ūrdhvanābhiḥ
tasyordhvanābher adhaḥ adhobhāge tālu tālusthānaṃ yasya so 'dhastālus tasyādhastālor yogina ūrdhvam
uparibhāge bhānur dahanātmakaḥ sūryo bhavati. adhaḥ adhobhāge śaśy amṛtātmā candro
bhavati.

prathamāntapāṭhe tu yadā ūrdhvanābhir adhastālur yogī bhavati tadordhvaṃ bhānur adhaḥ śaśī bhavati.
yadā tadā padayor adhyāhāreṇānvayaḥ. iyaṃ viparītākhyā viparītanāmikā karaṇī. ūrdhvādhaḥsthitayoś
candrasūryayor adhaūrdhvakaraṇenānvarthā guruvākyena guror vākyenaiva labhyate prāpyate nānyathā.
%\end{vsid}



%\begin{vsid}{#hp03_00}
\startsloka
{\bf nityam abhyāsayuktasya jaṭharāgnivivardhinī
āhāro bahulas tasya sampādyaḥ sādhakasya ca (3.80)}
\stopsloka

nityam iti. nityam pratidinaṃ abhyāso 'bhyasanaṃ tasmin yuktasyāvahitasya jaṭharāgnir udarāgnis
tasya vivardhinī viśeṣeṇa vardhinīti viparītakaraṇīviśeṣaṇam. tasya sādhakasya
viparītakaraṇyabhyāsina āhāro bhojanaṃ bahulo yathecchaḥ sampādyaḥ sampādanīyaḥ. caḥ pādapūraṇe
%\end{vsid}


%\begin{vsid}{#hp03_00}
\startsloka
{\bf alpāhāro yadi bhaved agnir dahati tatkṣaṇāt
  adhaḥśirāś cordhvapādaḥ kṣaṇaṃ syāt prathame dine
  kṣaṇāc ca kiñcid adhikam abhyasec ca dine dine   (3.81)}\var{\Wthree, \Tue and \EdAd
  read the ṣaṭpadaśloka here.}
\stopsloka

alpāhāra iti. yady alpāhāraḥ alpo bhoktum iṣṭād ūna āhāro bhojanaṃ yasya tādṛśo bhavet syāt
tadāgnir jaṭharānalo dehaṃ tatkṣaṇāt kṣaṇamātrād dahet. śīghraṃ dahed ity arthaḥ.

ūrdhvādhaḥsthitayoś candrasūryayor adhaūrdhvakaraṇakriyām āha adhaḥśirā iti. adhaḥ adhobhāge bhūmau
śiro yasya so 'dhaḥśirāḥ. karābhyāṃ kaṭipradeśam avalambya bāhumūlād ārabhya kūrparaparyantābhyāṃ
bāhubhyāṃ skandhābhyāṃ galapṛṣṭhabhāgaśiraḥpṛṣṭha\-bhāgābhyāṃ ca bhūmim avaṣṭabhyādhaḥśirā
bhavet. ūrdhvam upary antarikṣe pādau yasya sa ūrdhvapādaḥ. prathame dine ārambhadine kṣaṇaṃ
kṣaṇamātraṃ syāt. dine dine pratidinaṃ kṣaṇāt kiñcid adhikaṃ dvikṣaṇaṃ trikṣaṇam evaṃ
dinakramavṛddhyābhyased abhyāsaṃ kuryāt.
%\end{vsid}

%\begin{vsid}{#hp03_00}
\startsloka
{\bf  valitaṃ palitaṃ caiva ṣaṇmāsordhvaṃ na dṛśyate      % Pāda d Jyo selects reading of testimonia
yāmamātraṃ tu yo nityam abhyaset sa tu kālajit (3.82)}
\stopsloka

viparītakaraṇīguṇān āha valitam iti. valitaṃ carmasaṅkocaḥ palitaṃ keśeṣu śāuklyaṃ ca. ṣaṇṇāṃ
māsānāṃ samāhāraḥ ṣaṇmāsaṃ. tasmād ūrdhvam upari naiva dṛśyate naivāvalokyate. sādhakasya deha iti
vākyādhyāhāraḥ. yas tu sādhako yāmamātraṃ praharamātraṃ nityam abhyaset sa tu kālajit kālaṃ mṛtyuṃ
jayatīti kālajin mṛtyujetā bhavet. etena yogasya prārabdha\-karmapratibandhakatvam api sūcitam. tad
uktaṃ viṣṇudharme\comment{Viṣṇudharma 100.141}

\startsloka
svadehārambhakasyāpi karmaṇaḥ saṃkṣayāvahaḥ
yo yogaḥ pṛthivīpāla śṛṇu tasyāpi lakṣaṇam iti 
\stopsloka

vidyāraṇyair api jīvanmuktāv uktam yathā prārabdhakarma tattvajñānāt prabalaṃ tathā tasmād api
karmaṇo yogābhyāsaḥ prabalaḥ. ata eva yoginām uddālakavītahavyādīnāṃ svecchayā dehatyāga upapadyate
iti. bhāgavate 'py uktaṃ daivaṃ\comment{\EdLo reads dehaṃ, as does \EdMu, but according to the
  apparatus in \EdLo most mss. read daivaṃ.} jahyāt samādhinā\comment{Bhāgavatapurāṇa 7.15.24b.}
iti.
%\end{vsid}


%\begin{vsid}{#hp03_00}
\startsloka
{\bf atha vajrolī
svecchayā vartamāno 'pi yogoktair niyamair vinā
vajrolīṃ yo vijānāti sa yogī siddhibhājanam (3.83)}
\stopsloka

vajrolyāṃ pravṛttiṃ janayitum ādau tatphalam āha svecchayeti. yo 'bhyāsī vajrolīṃ vajrolīmudrāṃ
vijānāti viśeṣeṇa svānubhavena jānāti sa yogī yoge yogaśāstre uktā yogoktās tair yogoktair niyamair
brahmacaryādibhir vinā ṛte svecchayā nijecchayā vartamāno 'pi vyavaharann api siddhibhājanam
siddhīnām aṇimādīnāṃ bhājanaṃ pātraṃ bhavati.
%\end{vsid}

%\begin{vsid}{#hp03_00}
\startsloka
{\bf tatra vastudvayaṃ vakṣye durlabhaṃ yasya kasyacit
kṣīraṃ caikaṃ dvitīyaṃ tu nārī ca vaśavartinī (3.84)}
\stopsloka

tatsādhanopayogi vastudvayam āha tatreti. tatra vajrolyabhyāse vastunor dvayaṃ vastudvayaṃ
padārthayugmaṃ vakṣye kathayiṣye. kīdṛśaṃ vastudvayam yasya kasyacit yasya kasyāpi dhanahīnasya
durlabhaṃ duḥkhena labdhum śakyaṃ duḥkhenāpi labdhum aśakyam durlabham iti vā. duḥ syāt
kaṣṭaniṣedhayoḥ iti kośāt\comment{Not indentified.}. kiṃ tad vastudvayam ity apekṣāyām āha kṣīram iti. ekaṃ vastu
kṣīraṃ dugdhaṃ pānārthaṃ mehanānantaram indriyanairbalyāt tadbalārthaṃ kṣīrapānaṃ yuktam. kecit tu
abhyāsakāle ākarṣaṇārtham ity āhuḥ.
%\comment{Compare Haṭhasaṅketacandrikā f. 39r liṃgachidre ['tha  vivṛte kṣīrākṛṣṭiṃ tato bhajed ???}
tad ayuktam. tasyāntargatasya ghanībhāve nirgamanāsambhavāt. dvitīyaṃ tu vastu vaśavartinī
svādhīnā nārī vanitā.
%\end{vsid}


%\begin{vsid}{#hp03_85}
\startsloka
{\bf mehanena śanaiḥ samyag ūrdhvākuñcanam abhyaset
puruṣo 'py athavā nārī vajrolīsiddhim āpnuyāt (3.85)}
\stopsloka

vajrolīmudrāprakāram āha mehaneneti. mehanena strīsaṅgānantaraṃ bindoḥ \var{kṣaraṇena \Tue \lem
  rakṣaṇena \Wthree} sādhanabhūtena puruṣaḥ pumān athavā nāry api yoṣid api śanair mandaṃ samyak
yatnapūrvakam ūrdhvākuñcanam ūrdhvam upary ākuñcanaṃ meṇḍhrākuñcanena bindor upary ākarṣaṇam
abhyased vajrolīmudrāsiddhim āpnuyāt siddhiṃ gacchet.
%\end{vsid}

%\begin{vsid}{#hp03_00}
\startsloka
{\bf yatnataḥ śastanālena phūtkāraṃ vajrakandare
śanaiḥ śanaiḥ prakurvīta vāyusañcārakāraṇāt (3.86)}
\stopsloka

atha vajrolyāḥ pūrvāṅgaprakriyām āha yatnata iti. śastaḥ praśasto yo nālas tena śastanālena
sīsakādinirmitena nālena śanaiḥ śanaiḥ mandaṃ mandaṃ yathāgner dhamanārthaṃ phūtkāraḥ kriyate
tādṛśaṃ phūtkāraṃ vajrakandare meṇḍhravivare vāyoḥ sañcāraḥ samyag vajrakandare caraṇaṃ gamanaṃ
tatkāraṇāt taddhetoḥ prakurvīta prakarṣeṇa punaḥ punaḥ kurvīta.

atha vajrolīsādhanaprakriyā. sīsakanirmitāṃ snigdhāṃ meṇḍhrapraveśa\-yogyāṃ caturdaśāṅgulamātrāṃ
śalākāṃ kārayitvā tasyā meṇḍhre praveśanam abhyaset. prathamadine ekāṅgulamātrāṃ praveśayet
dvitīyadine dvyaṅgulamātrāṃ tṛtīyadine tryaṅgulamātrām. evaṃ krameṇa vṛddhyā\var{vṛddhyā \lem
  vṛddhau \EdLo (unclear variants)} dvādaśāṅgulamātrāpraveśe meṇḍhramārgaḥ śuddho bhavati. punas
tādṛśīṃ caturdaśāṅgulamātrāṃ dvyaṅgulamātrā\var{mātrā \lem mātre mātra}vakrām ūrdhvamukhāṃ
kārayitvā tāṃ dvādaśāṅgulamātrāṃ praveśayet. vakram ūrdhvamukhaṃ dvyaṅgulamātraṃ bahiḥ
sthāpayet. tataḥ suvarṇakārasya agnidhamanasādhanībhūtanālasadṛśaṃ nālaṃ gṛhītvā tadagraṃ
meṇḍhrapraveśitadvādaśāṅgulasya nālasya vakrordhvamukhadvyaṅgulamadhye praveśya phūtkāraṃ
kuryāt. tena samyak mārgaśuddhir bhavati. tataḥ koṣṇasya jalasya meṇḍhreṇākarṣaṇam
abhyasyet. jalākarṣaṇe siddhe pūrvoktaślokarītyā bindor ūrdhvākarṣaṇam abhyasyet. bindvākarṣaṇe
siddhe vajrolīmudrāsiddhiḥ. iyaṃ jitaprāṇasyaiva sidhyati nānyasya.
khecarīmudrāprāṇajayobhayasiddhau tu samyag bhavati.
%\end{vsid}


%\begin{vsid}{#hp03_00}
\startsloka
{\bf nārībhage patadbindum abhyāsenordhvam āharet
calitaṃ ca nijaṃ bindum ūrdhvam ākṛṣya rakṣayet (3.87)}
\stopsloka

evaṃ vajrolyabhyāse siddhe taduttaraṃ sādhanam āha nārībhage iti. nārībhage strīyonau patatīti
patan pataṃś cāsau binduś patadbindus taṃ patadbindum ratikāle patantaṃ bindum abhyāsena
vajrolīmudrābhyāsenordhvam upary āhared ākarṣayet patanāt pūrvam eva. yadi patanāt pūrvaṃ bindor
ākarṣaṇaṃ na syāt tarhi patitam ākarṣayed ity āha calitam iti. calitaṃ nārībhage patitaṃ nijaṃ
svakīyaṃ bindum cakārāt tadrajaḥ ūrdhvam upary ākṛṣyāhṛtya rakṣayet sthāpayet.
%\end{vsid}



%\begin{vsid}{#hp03_00}
\startsloka
{\bf evaṃ saṃrakṣayed binduṃ mṛtyuṃ jayati yogavit
maraṇaṃ bindupātena jīvanaṃ bindudhāraṇāt (3.88)}
\stopsloka

vajrolīguṇān āha evam iti. evam uktarītyā binduṃ yaḥ saṃrakṣayet samyak rakṣayet sa yogavit
yogābhijño mṛtyuṃ jayaty abhibhavati. yato bindoḥ śukrasya pātena patanena maraṇaṃ bhavati. bindor
dhāraṇaṃ bindudhāraṇaṃ tasmād bindudhāraṇāj jīvanaṃ bhavati. tasmād binduṃ saṃrakṣayed ity arthaḥ.
%\end{vsid}

%\begin{vsid}{#hp03_00}
\startsloka
{\bf sugandho yogino dehe jāyate bindudhāraṇāt
yāvad binduḥ sthiro dehe tāvat kālabhayaṃ kutaḥ (3.89)}
\stopsloka

sugandha iti. yogino vajrolyabhyāsino dehe bindoḥ śukrasya dhāraṇaṃ bindudhāraṇaṃ tasmāt sugandhaḥ
śobhano gandho jāyate prādurbhavati. dehe yāvad binduḥ sthiras tāvat kālabhayaṃ mṛtyubhayaṃ kutaḥ
na kuto 'pīty arthaḥ.
%\end{vsid}

%\begin{vsid}{#hp03_00}
\startsloka
{\bf cittāyattaṃ nṝṇāṃ śukraṃ śukrāyattaṃ ca jīvitam
tasmāc chukraṃ manaś caiva rakṣaṇīyaṃ prayatnataḥ (3.90)}
\stopsloka

cittāyattam iti. hi yasmāt. nṝṇāṃ śukraṃ vīryaṃ cittāyattam. citte cale calatvāt citte sthire
sthiratvāc cittādhīnam. jīvitaṃ jīvanaṃ śukrāyattaṃ śukre sthire jīvanāc chukre naṣṭe maraṇāt
śukrādhīnam. tasmāc chukraṃ binduṃ manaś ca mānasaṃ ca prakṛṣṭād yatnād iti prayatnataḥ rakṣaṇīyam
eva. avaśyaṃ rakṣaṇīyam ity arthaḥ. evaśabdo bhinnakramaḥ.
%\end{vsid}



%\begin{vsid}{#hp03_00}
\startsloka
{\bf ṛtumatyā rajo 'py evaṃ nijaṃ binduṃ ca rakṣayet
meḍhreṇākarṣayed ūrdhvaṃ samyag abhyāsayogavit (3.91)}
\stopsloka

ṛtumatyā iti. evaṃ pūrvoktenābhyāsena ṛtur vidyate yasyāḥ sā ṛtumatī tasyā ṛtumatyā ṛtusnātāyāḥ
striyāḥ\var{striyāḥ \EdAd \lem striyoḥ \Tue \EdLo \EdMu} rajaḥ nijaṃ svakīyaṃ binduṃ ca
rakṣayet. pūrvoktābhyāsaṃ darśayati meṇḍhreṇeti. abhyāso vajrolyabhyāse sa eva yogo yogasādhanatvāt
taṃ vettīty abhyāsayogavit meḍhreṇa guhyendriyeṇa samyag yatnapūrvakam ūrdhvam upary
ākarṣayet. rajo binduṃ ceti karmādhyāhāraḥ. ayaṃ ślokaḥ prakṣiptaḥ. % warum?  \Tue liest kṣisaḥ, not reported.
%\end{vsid}


%\begin{vsid}{#hp03_00}
\startsloka
{\bf atha sahajoliḥ
sahajoliś cāmarolir vajrolyā bheda ekataḥ
jale subhasma nikṣipya dagdhagomayasambhavam (3.92)}
\stopsloka

sahajolyamarolyau vivakṣus tayor vajrolīviśeṣatvam āha sahajoliś ceti. vajrolyā bhedo viśeṣaḥ
sahajolir amaroliś ca. tatra hetuḥ ekata ekatvād ekaphalatvād ity arthaḥ. ekaśabdād bhāvapradhānāt
pañcamyās tasil. sahajolim āha jala iti. goḥ purīṣāni gomayāni dagdhāni ca tāni gomayāni
dagdhagomayāni teṣu sambhava utpattir yasya tad dagdhagomayasambhavam śobhanaṃ bhasma subhasma
vibhūtiḥ tat jale toye nikṣipya toyamiśraṃ kṛtvety uttaraślokenānveti.
%\end{vsid}

%\begin{vsid}{#hp03_00}
\startsloka
{\bf vajrolīmaithunād ūrdhvaṃ strīpuṃsoḥ svāṅgalepanam
āsīnayoḥ sukhenaiva muktavyāpārayoḥ kṣaṇāt (3.93)}
\stopsloka

vajrolīti. vajrolīmudrārthaṃ maithunaṃ tasmād ūrdhvam anantaraṃ sukhenaivānandenaivāsīnayor
upaviṣṭayoḥ kṣaṇād ratyutsavān muktas tyakto vyāpāro ratikriyā yābhyāṃ tau muktavyāpārau tayor
muktavyāpārayoḥ strī ca pumāṃś ca strīpuṃsau tayoḥ strīpuṃsayoḥ svāṅgalepanam śobhanāny aṅgāni
mūrdhalalāṭanetrahṛdayaskandhabhujādīni teṣu lepanam. kuryād iti śeṣaḥ.
%\end{vsid}

%\begin{vsid}{#hp03_00}
\startsloka 
{\bf sahajolir iyaṃ proktā śraddheyā yogibhiḥ sadā
ayaṃ śubhakaro yogo bhogayukto 'pi muktidaḥ (3.94)}
\stopsloka

sahajolir iti. iyam uktā kriyā sahajolir iti proktā kathitā yogibhir matsyendrādibhiḥ. kīdṛśī sadā
śraddheyā sarvadā śraddhātuṃ yogyā. ayaṃ sahajolyākhyo yoga upāyaḥ śubhakaraḥ śubhaṃ śreyaḥ
karotīti śubhakaraḥ. yogaḥ saṃnahanopāyadhyānasaṃgatiyuktiṣu ity abhidhānāt.\comment{Amarakośa
  3.3.22.} kīdṛśo yogaḥ bhogena strīsaṅgena yukto 'pi muktido mokṣadaḥ.
%\end{vsid}


%\begin{vsid}{#hp03_00}
\startsloka
{\bf ayaṃ yogaḥ puṇyavatāṃ dhīrāṇāṃ tattvadarśinām
nirmatsarāṇāṃ vai sidhyen na tu matsaraśālinām (3.95)}
\stopsloka

ayaṃ yoga iti. ayam ukto yogaḥ puṇyaṃ vidyate yeṣāṃ te puṇyavantaḥ sukṛtinas teṣāṃ puṇyavatāṃ
dhīrāṇāṃ dhairyavatāṃ tattvaṃ vāstavikaṃ draṣṭuṃ śīlaṃ yeṣāṃ te\var{draṣṭuṃ śīlaṃ yeṣāṃ te \EdAd
  \lem paśyantīti \EdLo \Wthree} tattvadarśinas teṣāṃ tattvadarśinām. matsarān niṣkrāntā nirmatsarās
teṣāṃ nirmatsarāṇām anyaguṇadveṣarahitānām. matsaro 'nyaguṇadveṣa ity amaraḥ\comment{Amarakośa
  3.3.172}. tādṛśānāṃ puṃsāṃ sidhyet siddhiṃ gacchet. matsaraśālinām matsaravatāṃ tu na sidhyet.
%\end{vsid}



%\begin{vsid}{#hp03_00}
\startsloka
{\bf atha amarolī
pittolbaṇatvāt prathamāmbudhārāṃ 
vihāya niḥsāratayāntyadhārām
niṣevyate śītalamadhyadhārā
kāpālike khaṇḍamate 'marolī  (3.96)}
\stopsloka

amarolīm āha pittolbaṇatvād iti. pittenolbaṇotkaṭā pittolbaṇā tasyā bhāvaḥ pittolbaṇatvaṃ
tasmāt. yathā prathamā pūrvā yāmbunaḥ śivāmbuno dhārā tāṃ vihāya śivāmbunirgamanasamaye kiñcit
pūrvāṃ dhārāṃ tyaktvā. nirgataḥ sāro yasyāḥ sā niḥsārā tasyā bhāvo niḥsāratā tayā niḥsāratayā
niḥsāratvenāntyadhārā antyā caramā yā dhārā tāṃ vihāya kiñcid antyāṃ dhārāṃ tyaktvā. śītalā
pittādidoṣaniḥsāratvarahitā yā madhyadhārā madhyamā dhārā sā niṣevyate nitarāṃ sevyate. khaṇḍo
yogaviśeṣo mato 'bhimato yasya sa khaṇḍamatas tasmin khaṇḍamate kāpālikasyāyaṃ kāpālike
khaṇḍakāpālikasampradāya ity arthaḥ. amarolī prasiddheti śeṣaḥ.
%\end{vsid}


%\begin{vsid}{#hp03_00}
\startsloka
{\bf amarīṃ yaḥ piben nityaṃ nasyaṃ kurvan dine dine
vajrolīm abhyaset samyak sāmarolīti kathyate  (3.97)}
\stopsloka

amarīm iti. amarīṃ śivāmbu yaḥ pumāṇ nityaṃ pibet nasyaṃ kurvan śvāsenāmaryā ghrāṇāntar grahaṇaṃ
kurvan san dine dine pratidinaṃ mehanena śanaiḥ\comment{HP 3.85.} iti ślokenoktāṃ samyag abhyaset
sāmarolīti kathyate. kāpālikair iti śeṣaḥ. amarīpānāmarīnasyapūrvikā vajroly amarolīśabdenocyate
ity arthaḥ.
%\end{vsid}

%\begin{vsid}{#hp03_00}
\startsloka
{\bf abhyāsān niḥsṛtāṃ cāndrīṃ vibhūtyā saha miśrayet
dhārayed uttamāṅgeṣu divyadṛṣṭiḥ prajāyate (3.98)}
\stopsloka

abhyāsād iti. abhyāsād amarolyabhyāsān niḥsṛtāṃ nirgatāṃ cāndrīṃ candrasyeyaṃ cāndrī tāṃ cāndrīṃ
sudhāṃ vibhūtyā bhasmanā saha sākaṃ miśrayet saṃyojayet uttamāṅgeṣu
śiraḥkapālanetraskandhakaṇṭhahṛdayabhujādiṣu dhārayet. bhasmamiśritāṃ cāndrīm iti śeṣaḥ. divyā
atītānāgatavartamānavyavahitaviprakṛṣṭapadārthadarśanayogyā dṛṣṭir yasya sa divyadṛṣṭir divyadṛk
prajāyate prakarṣeṇa jāyate. amarīsevanaprakāraviśeṣāḥ śivāmbukalpād\comment{???} avagantavyāḥ.
%\end{vsid}


%\begin{vsid}{#hp03_00}
\startsloka
{\bf puṃso binduṃ samākuñcya samyagabhyāsapāṭavāt
yadi nārī rajo rakṣed vajrolyā sāpi yoginī (3.99)}
\stopsloka

puṃso vajrolīsādhanam uktvā nāryās tad āha puṃso bindum iti. samyagabhyāsasya samyagabhyasanasya
pāṭavaṃ paṭutvaṃ tasmāt puṃsaḥ puruṣasya binduṃ vīryaṃ samākuñcya samyag ākṛṣya nārī strī yadi rajo
vajrolyā vajrolīmudrayā rakṣet sāpi nārī yoginī praśastayogavatī jñeyā. puṃso bindusamāyuktam iti
pāṭhe tu etad rajaso viśeṣaṇam.
%\end{vsid}

%yoginī in the sense of Yoga practioner !

%\begin{vsid}{#hp03_00}
\startsloka
{\bf tasyāḥ kiñcid rajo nāśaṃ na gacchati na saṃśayaḥ
tasyāḥ śarīre nādaś ca bindutām eva gacchati  (3.100)}
\stopsloka

nārīkṛtāyā vajrolyāḥ phalam āha tasyā iti. tasyā vajrolyabhyasanaśīlāyā nāryā rajaḥ kiñcit kim api
svalpam api nāśaṃ na gacchati naṣṭaṃ na bhavati patanaṃ na prāpnotīty arthaḥ. atra saṃśayo na
sandeho na. tasyā nāryāḥ śarīre nādaś ca bindutām eva gacchati mūlādhārād utthito nādo hṛdayopari
bindubhāvaṃ gacchati. bindunā sahaikībhavatīty arthaḥ. amṛtasiddhau\comment{Amṛtasiddhi 7.8cd–7.9,
  7.12 (only a–c), 7.5.ab, 7.2ab}

% JH  nāda becomes bindu ?  usage unclear. 

\startsloka
bījaṃ ca pauruṣaṃ proktaṃ rajaś ca strīsamudbhavam 
anayor bāhyayogena sṛṣṭiḥ sañjāyate nṛṇām
yadābhyantaryogaḥ syāt tadā yogīti gīyate
binduś candramasaḥ prokto rajaḥ sūryamayaṃ tathā
anayoḥ saṅgamād eva jāyate paramaṃ padam
svargado mokṣado bindur dharmado 'dharmadas tathā
tanmadhye devatāḥ sarvās tiṣṭhante sūkṣmarūpataḥ
\stopsloka

iti.
%\end{vsid}



%\begin{vsid}{#hp03_00}
\startsloka
{\bf sa bindus tad rajaś caiva ekībhūya svadehagau
vajrolyabhyāsayogena sarvasiddhiṃ prayacchataḥ (3.101)}
\stopsloka

sa bindur iti. sa puṃso bindus tad rajo nāryā rajaś caiva vajrolīmudrāyā abhyāso vajrolyabhyāsaḥ sa
eva yogas tenaikībhūya militvā svadehagau svadehe gatau sarvasiddhiṃ prayacchataḥ dattaḥ.
%\end{vsid}

% JH again singular readings of Jyotsnā prayacchataḥ 


%\begin{vsid}{#hp03_00}
\startsloka
{\bf rakṣed ākuñcanād ūrdhvaṃ yā rajaḥ sā hi yoginī
atītānāgataṃ vetti khecarī ca bhaved dhruvam (3.102)}
\stopsloka

rakṣed iti. yā nāry ākuñcanād yonisaṅkocanād ūrdhvam uparisthāne nītvā rajo rakṣet hīti prasiddhaṃ
yogaśāstre sā yoginī atītānāgataṃ bhūtaṃ bhaviṣyaṃ ca vastu vetti jānāti. dhruvam iti
niścitam. khe 'ntarikṣe caratīti khecary antarikṣacarī ca bhavet.
%  reading of the HP mss. is meḍhraṃ, which does not make sense. 
%   women emphasized by Jyotsnā?  not in early versions in this constellation    ????
%\end{vsid}


%\begin{vsid}{#hp03_00}
\startsloka
{\bf dehasiddhiṃ ca labhate vajrolyabhyāsayogataḥ
ayaṃ puṇyakaro yogo bhoge bhukte 'pi muktidaḥ (3.103)}
\stopsloka

dehasiddhim iti. vajrolyā abhyāsasya yogo yuktis tasmād dehasya siddhiṃ
rūpalāvaṇyabalavajrasaṃhananatvarūpāṃ\var{tvarūpam \EdLo \Tue \lem svarūpāṃ ??} labhate. ayaṃ yogo
vajrolyabhyāsayogaḥ puṇyakaro 'dṛṣṭaviśeṣajanakaḥ. kīdṛśo yogaḥ  bhujyata iti bhogo viṣayas tasmin
bhukte 'pi muktido mokṣadaḥ.
%\end{vsid}


%\begin{vsid}{#hp03_00}
\startsloka
{\bf atha śakticālanam
kuṭilāṅgī kuṇḍalinī bhujaṅgī śaktir īśvarī
kuṇḍaly arundhatī caite śabdāḥ paryāyavācakāḥ (3.104)}
\stopsloka

śakticālanaṃ vivakṣus tadupodghātatayā kuṇḍalīparyāyān tayā mokṣadvāravibhedanādikaṃ cāha
saptabhiḥ kuṭilāṅgīti. kuṭilāṅgī kuṇḍalinī bhujaṅgī śaktiḥ īśvarī kuṇḍalī arundhatī caite
sapta śabdāḥ paryāyavācakā ekārthavācakāḥ.
%\end{vsid}




%\begin{vsid}{#hp03_00}
\startsloka
{\bf udghāṭayet kapāṭaṃ tu yathā kuñcikayā haṭhāt
kuṇḍalinyā tathā yogī mokṣadvāraṃ vibhedayet (3.105)}
\stopsloka

udghāṭayed iti. yathā yena prakāreṇa pumān kuñcikayā kapāṭārgalotsāraṇasādhanībhūtayā haṭhād balāt
kapāṭam araram udghāṭayed utsārayet. haṭhād iti dehalīdīpanyāyenobhayatra sambadhyate. tathā tena
prakāreṇa yogī haṭhād haṭhābhyāsāt kuṇḍalinyā śaktyā mokṣadvāraṃ mokṣasya dvāraṃ prāpakaṃ
suṣumnāmārgaṃ vibhedayed viśeṣeṇa bhedayet. tayordhvam āyann amṛtatvam eti\comment{Chāndogyopaniṣat
  8.6.6?} iti śruteḥ.
%\end{vsid}


%\begin{vsid}{#hp03_00}
\startsloka
{\bf yena mārgeṇa gantavyaṃ brahmasthānaṃ nirāmayam
mukhenācchādya tad dvāraṃ prasuptā parameśvarī (3.106)}
\stopsloka

yeneti. āmayo rogajanyaṃ duḥkhaṃ duḥkhamātropalakṣaṇaṃ tasmān nirgataṃ nirāmayaṃ duḥkhamātrarahitaṃ
brahmasthānaṃ brahmāvirbhāvajanakaṃ sthānaṃ brahmasthānaṃ brahmarandhram. tasyāḥ śikhāyā madhye
paramātmā vyavasthitaḥ\comment{Mahānārāyaṇopaniṣat 11.13.} iti śruteḥ. yena mārgeṇa suṣumnāmārgeṇa
gantavyaṃ gamanārham asti tad dvāraṃ tasya mārgasya dvāraṃ praveśamārgaṃ mukhenāsyenācchādya
ruddhvā parameśvarī kuṇḍalinī prasuptā nidritāsti.
%\end{vsid}

%\begin{vsid}{#hp03_00}
\startsloka
{\bf kandordhve kuṇḍalī śaktiḥ suptā mokṣāya yoginām
bandhanāya ca mūḍhānāṃ yas tāṃ vetti sa yogavit (3.107)}
\stopsloka

kandordhva iti. kuṇḍalī śaktiḥ kandordhve kandasyoparibhāge yogināṃ mokṣāya suptā mūḍhānāṃ
bandhanāya suptā. yoginas tāṃ cālayitvā muktā bhavanti. mūḍhās tadajñānād baddhās tiṣṭhantīti
bhāvaḥ. tāṃ kuṇḍalinīṃ yo vetti sa yogavit. sarveṣāṃ yogatantrāṇāṃ kuṇḍalyāśrayatvād ity arthaḥ.
%\end{vsid}

% Brahmānanda might have elided the verse 3.106 of the critical edition (ambodhiśaila ...)
% because it duplicates 3.1  


%\begin{vsid}{#hp03_00}
\startsloka
{\bf kuṇḍalī kuṭilākārā sarpavat parikīrtitā
sā śaktiś cālitā yena sa mukto nātra saṃśayaḥ (3.108)}
\stopsloka

kuṇḍalīti. kuṇḍalī śaktiḥ sarpavat bhujaṅgavat kuṭila ākāraḥ svarūpaṃ yasyāḥ sā kuṭilākārā
parikīrtitā kathitā yogibhiḥ. sā kuṇḍalī śaktir yena puṃsā cālitā mūlādhārād ūrdhvaṃ nītā sa mukto
'jñānabandhān nivṛttaḥ. atrāsminn arthe saṃśayo na sandeho nāstīty arthaḥ. tayordhvam āyann
amṛtatvam eti\comment{Chāndogyopaniṣat 8.6.6.} iti śruteḥ.
%\end{vsid}


%\begin{vsid}{#hp03_00}
\startsloka
{\bf gaṅgāyamunayor madhye bālaraṇḍāṃ tapasvinīm
balātkāreṇa gṛhṇīyāt tad viṣṇoḥ paramaṃ padam (3.109)}
\stopsloka

gaṅgāyamunayor iti. gaṅgāyamunayor ādhārādheyabhāvena tayor bhāvanād gaṅgāyamunayor abhedena
bhāvanād vā gaṅgāyamune iḍāpiṅgale tayor madhye suṣumnāmārge tapasvinīṃ niraśanasthiteḥ bālaraṇḍāṃ
bālaraṇḍāśabdavācyāṃ kuṇḍalīṃ balātkāreṇa haṭhena gṛhṇīyāt. tat tasyā gaṅgāyamunayor madhye
grahaṇaṃ viṣṇor harer vyāpakasyātmano vā paramaṃ padam paramapadaprāpakam.
%\end{vsid}



%\begin{vsid}{#hp03_00}
\startsloka
{\bf iḍā bhagavatī gaṅgā piṅgalā yamunā nadī
iḍāpiṅgalayor madhye bālaraṇḍā ca kuṇḍalī (3.110)}
\stopsloka

gaṅgāyamunādipadārtham āha iḍeti. iḍā vāmaniḥśvāsā nāḍī bhagavaty aiśvaryādisampannā gaṅgā
gaṅgāpadavācyā piṅgalā dakṣiṇaniḥśvāsā yamunā yamunāśabdavācyā nadī. iḍāpiṅgalayor madhye
madhyagatā yā kuṇḍalī sā bālaraṇḍā bālaraṇḍāśabdavācyā.
%\end{vsid}


%\begin{vsid}{#hp03_00}
\startsloka
{\bf pucche pragṛhya bhujagīṃ suptām udbodhayec ca tām
nidrāṃ vihāya sā śaktir  ūrdhvam uttiṣṭhate haṭhāt (3.111)}
\stopsloka

śakticālanam āha puccha iti. suptām nidritāṃ tāṃ bhujagīṃ tāṃ kuṇḍalinīṃ pucche pragṛhya
pragṛhītvodbodhayet prabodhayet yā śaktiḥ kuṇḍalī nidrāṃ vihāya haṭhād ūrdhvam uttiṣṭhata ity
anvayaḥ. etad rahasyaṃ tu gurumukhād avagantavyam.
%\end{vsid}

% Only the Jyotsnā reads śakti, even YCM has ṛjvī as most manuscripts. 


%\begin{vsid}{#hp03_00}
\startsloka
{\bf avasthitā caiva phaṇāvatī sā 
prātaś ca sāyaṃ praharārdhamātram
prapūrya sūryāt paridhānayuktyā
pragṛhya nityaṃ paricālanīyā (3.112)}
\stopsloka

avasthiteti. avasthitārvāk sthitā mūlādhārasthitā phaṇāvatī bhujaṅgī sā kuṇḍalī sūryād āpūrya
sūryāt pūraṇaṃ kṛtvā paridhānasya yuktiḥ paridhānayuktiḥ tayā paridhānayuktyā pragṛhya
gṛhītvā. sāyaṃ sūryāstasamaye prātaḥ sūryodayavelāyāṃ nityam aharahaḥ praharasya yāmasyārdhaṃ
praharārdhaṃ praharārdham eva praharārdhamātram muhūrtadvayamātraṃ paricālanīyā paritaś cālayituṃ
yogyā. paridhānayuktir deśikād bodhyā.


%\begin{vsid}{#hp03_00}
\startsloka
{\bf ūrdhvaṃ vitastimātraṃ tu vistāraṃ caturaṅgulam
mṛdulaṃ dhavalaṃ proktaṃ veṣṭitāmbaralakṣaṇam (3.113)}
\stopsloka

% Strange interpretation by B. because of a misplaced next verse (different in ....) 


kandasampīḍanena śakticālanaṃ vivakṣur ādau kandasya sthānaṃ svarūpaṃ cāha ūrdhvam iti. mūlasthānād
vitastimātraṃ vitastipramāṇam ūrdhvam upari nābhimeṇḍhrayor madhye. etena kandasya sthānam
uktam. tathā coktaṃ gorakṣaśatake

\startsloka
  ūrdhvaṃ meḍhrād adho nābheḥ kandayoniḥ khagāṇḍavat
  tatra nāḍyaḥ samutpannāḥ sahasrāṇāṃ dvisaptatiḥ\comment{Vivekamārtaṇḍa 16} %  yo.ta. 1.25?} 
\stopsloka

iti. yājñavalkyaḥ\comment{Yogayājñavalkya 4.14, 16–17.}

\startsloka
gudāt tu dvyaṅgulād ūrdhvam adho meḍhrāc ca dvyaṅgulāt
dehamadhyaṃ tayor madhyaṃ manuṣyāṇām itīritam
kandasthānaṃ manuṣyāṇāṃ dehamadhyān navāṅgulam 
caturaṅgulam utsedam āyāmaṃ ca tathāvidham  % āyāmaḥ ? so etext
aṇḍākṛtivad ākāraṃ bhūṣitaṃ tat tvagādibhiḥ
catuṣpadāṃ tiraścāṃ ca dvijānāṃ tundamadhyame
\stopsloka

iti.

gudād dvyaṅgulopary ekāṅgulaṃ madhyaṃ tasmān navāṅgulaṃ kandasthānaṃ militvā dvādaśāṅgulapramāṇaṃ
vitastimātraṃ jātam. caturṇām aṅgulānāṃ samāhāraś caturaṅgulaṃ caturaṅgulapramāṇaṃ
vistāram. vistāro dairghyasyāpy upalakṣaṇam. caturaṅgulaṃ dīrghaṃ ca mṛdulaṃ komalaṃ dhavalaṃ
śubhraṃ veṣṭitaṃ veṣṭanena khagāṇḍākārīkṛtaṃ yad ambaraṃ vastraṃ tasya lakṣaṇaṃ svarūpam iva
lakṣaṇaṃ svarūpaṃ yasya tādṛśaṃ proktaṃ kathitam. kandasvarūpaṃ yogibhir iti śeṣaḥ.


%\begin{vsid}{#hp03_00}
\startsloka
{\bf sati vajrāsane pādau karābhyāṃ dhārayed dṛḍham
gulphadeśasamīpe ca kandaṃ tatra prapīḍayet (3.114)}
\stopsloka

satīti. vajrāsane\var{vajrāsane \Tue \lem vajrāsane siddhāsane \Wthree} kṛte sati karābhyāṃ hastābhyāṃ
gulphau pādagranthī tayor deśau pradeśau tayoḥ samīpe gulphābhyāṃ kiñcid upari. tadgranthī ghuṭike
gulphau ity amaraḥ.\comment{Amarakośa ??} pādau caraṇau dṛḍham gāḍham dhārayed gṛhṇīyāt. cakārād
dhṛtābhyāṃ pādābhyāṃ tatra kandasthāne kandaṃ prapīḍayet prakarṣeṇa pīḍayet. gulphād ūrdhvaṃ
karābhyāṃ pādau gṛhītvā nābher adhobhāge kandaṃ pīḍayed ity arthaḥ.

% Brahmānanda has redacted the text by lifting out 3.114 from its position in the description of
% Uḍḍiyānabandha (where it occurs in all manuscripts available to us) and inserted it in
% Śakticālana, where it fits perfectly.  Furthermore the Jyotsnā completely omits the verse
% following in all manuscripts (3.65 of our critical edition). 


%\begin{vsid}{#hp03_00}
\startsloka
{\bf vajrāsane sthito yogī cālayitvā ca kuṇḍalīm
kuryād anantaraṃ bhastrāṃ kuṇḍalīm āśu bodhayet (3.115)}
\stopsloka

vajrāsana iti. vajrāsane sthito yogī kuṇḍalīm cālayitvā śakticālanamudrāṃ kṛtvety arthaḥ.
anantaraṃ śakticālanānantaraṃ bhastrāṃ bhastrākhyaṃ kumbhakaṃ kuryāt. evaṃrītyā kuṇḍalīm śaktim
āśu śīghraṃ bodhayet prabuddhāṃ kuryāt. vajrāsane śakticālanasya pūrvaṃ vidhāne 'pi punar
vajrāsanopādānaṃ\var{vajrāsanopādānaṃ \Tue \lem vajrāsanopapādānaṃ \EdLo} śakticālanānantaraṃ
bhastrāyāṃ vajrāsanam eva kartavyam iti niyamārtham. 


%\begin{vsid}{#hp03_00}
\startsloka
{\bf bhānor ākuñcanaṃ kuryāt kuṇḍalīṃ cālayet tataḥ
mṛtyuvaktragatasyāpi tasya mṛtyubhayaṃ kutaḥ (3.116)}
\stopsloka

bhānor iti. bhānor nābhideśasthasya sūryasyākuñcanaṃ kuryāt. nābher ākuñcanenaiva
tasyākuñcanaṃ bhavati. tato bhānor ākuñcanāt kuṇḍalīṃ śaktiṃ cālayet. mṛtyor vaktraṃ mukhaṃ
gatasyāpi prāptasyāpi tasya puṃso mṛtyubhayaṃ kālabhayaṃ kutaḥ.  na kuto 'pīty arthaḥ.



%\begin{vsid}{#hp03_00}
\startsloka
{\bf muhūrtadvayaparyantaṃ nirbhayaṃ cālanād asau
ūrdhvam ākṛṣyate kiñcit suṣumṇāyāṃ samudgatā (3.117)}
\stopsloka

muhūrtadvayeti\var{muhūrtadvayeti \Wthreepc \EdAd \lem muhūrtadvayam iti \Wthreeac \Tue \EdLo \EdMu}. muhūrtayor
dvayaṃ yugmaṃ ghaṭikācatuṣṭayātmakaṃ tatparyantaṃ tadavadhi nirbhayaṃ niḥśaṅkaṃ cālanād asau śaktiḥ
suṣumnāyāṃ samudgatā satī kiñcid ūrdhvam ākṛṣyate ākṛṣṭā bhavati.


%\begin{vsid}{#hp03_00}
\startsloka
{\bf tena kuṇḍalinī tasyāḥ suṣumṇāyā mukhaṃ dhruvam
jahāti tasmāt prāṇo 'yaṃ suṣumṇāṃ vrajati svataḥ (3.118)}
\stopsloka

teneti. tenordhvākarṣaṇena kuṇḍalinī tasyāḥ prasiddhāyāḥ suṣumnāyā mukhaṃ praveśamārgaṃ dhruvaṃ
niścitaṃ jahāti tyajati. tasmān mārgatyāgād ayaṃ prāṇo vāyuḥ svataḥ svayam eva suṣumnāṃ vrajati
gacchati. suṣumnāmukhāt prāg eva kuṇḍalinyā nirgatatvād iti bhāvaḥ.



%\begin{vsid}{#hp03_00}
\startsloka
{\bf tasmāt sañcālayen nityaṃ sukhasuptām arundhatīm
tasyāḥ sañcālanenaiva yogī rogaiḥ pramucyate (3.119)}
\stopsloka

tasmād iti. yasmāc chakticālanena prāṇaṃ suṣumnāṃ vrajati tasmāt sukhena suptā sukhasuptā taṃ
sukhasuptām arundhatīm śaktiṃ nityaṃ pratidinaṃ sañcālayet samyak cālayet. tasyāḥ śakteḥ
sañcālanenaiva sañcālanamātreṇa yogī rogaiḥ kāsaśvāsajvarā\var{jvarā \Tue \EdAd \lem jarā EdMU
  \EdLo}dhibhiḥ pramucyate prakarṣeṇa mukto bhavati.

%\begin{vsid}{#hp03_00}
\startsloka
{\bf yena sañcālitā śaktiḥ sa yogī siddhibhājanam 
kim atra bahunoktena kālaṃ jayati līlayā (3.120)}
\stopsloka

yeneti. yena yoginā śaktiḥ kuṇḍalī sañcālitā sa yogī siddhīnām aṇimādīnāṃ bhājanam pātraṃ
bhavati. atrāsminn arthe bahunoktena bahupraśaṃsanena kim  na kim apīty arthaḥ. kālaṃ mṛtyuṃ
līlayā krīḍayānāyāsena jayaty abhibhavatīty arthaḥ.



%\begin{vsid}{#hp03_00}
\startsloka
{\bf brahmacaryaratasyaiva nityaṃ hitamitāśinaḥ
maṇḍalād dṛśyate siddhiḥ kuṇḍalyabhyāsayoginaḥ (3.121)}
\stopsloka

brahmacaryeti. brahmacaryaṃ śrotrādibhiḥ sahopasthasaṃyamas tasmin ratasya tatparasya nityaṃ
sarvadā hitaṃ pathyaṃ mitaṃ caturthāṃśavarjitam aśnātīti tasya kuṇḍalyabhyāsaḥ śakticālanābhyāsaḥ
sa eva yogaḥ so 'syāstīti sa tathā tasya maṇḍalāc catvāriṃśaddinātmakād anantaraṃ siddhiḥ
prāṇāyāmasiddhir dṛśyate.


\startsloka
{\bf nāsādakṣiṇamārgavāhipavanāt prāṇo 'tidīrghīkṛtaś
candrābhaḥ paripūritāmṛtatanuḥ prāg ghaṇṭikāyās tataḥ 
chittvā kālaviśālavahnivaśagaṃ bhrūrandhranāḍīgataṃ
tat kāyaṃ kurute punar navataraṃ chinnadrumaṃ skandhavat}\comment{drumaṃ neutre?
         Amaraughaśāsana (part of the text in all manuscripts 4.114) check marmasthānas?? check with V10} 
\stopsloka



%\begin{vsid}{#hp03_00}
\startsloka
{\bf kuṇḍalīṃ cālayitvā tu bhastrāṃ kuryād viśeṣataḥ
evam abhyasato nityaṃ yamino yamabhīḥ kutaḥ (3.122)}
\stopsloka

% abhyasya vs abhyasa

kuṇḍalīm iti. kuṇḍalīṃ cālayitvā śakticālanaṃ kṛtvā. athānantaram eva bhastrāṃ bhastrākhyaṃ
kumbhakaṃ kuryāt. nityaṃ pratidinam. evam uktaprakāreṇābhyasato yamino yogino yamabhīr yamād bhayaṃ
kutaḥ na kuto 'pīty arthaḥ. yogino dehatyāgasya svādhīnatvād iti tātparyam.


%\begin{vsid}{#hp03_00}
\startsloka
{\bf dvāsaptatisahasrāṇāṃ nāḍīnāṃ malaśodhane
kutaḥ prakṣālanopāyaḥ kuṇḍalyabhyasanād ṛte (3.123)}
\stopsloka

dvāsaptatīti. dvābhyām adhikā saptatiḥ dvāsaptatisaṅkhyākāni sahasrāṇi dvāsaptatisahasrāṇi
teṣāṃ tatsaṅkhyākānāṃ nāḍīnāṃ malaśodhane kartavye sati kuṇḍalyabhyasanāc chakticālanābhyāsād
ṛte vinā kutaḥ prakṣālanopāyaḥ na kuto 'pi. śakticālanābhyāsenaiva sarvāsāṃ nāḍīnāṃ
malaśodhanaṃ bhavatīty abhiprāyaḥ.


%\begin{vsid}{#hp03_00}
\startsloka
{\bf iyaṃ tu madhyamā nāḍī dṛḍhābhyāsena yoginām
āsanaprāṇasaṃyāmamudrābhiḥ saralā bhavet (3.124)}
\stopsloka

iyaṃ tv iti. iyaṃ madhyamā nāḍī suṣumnā yoginām dṛḍhābhyāsenāsanaṃ svastikādi prāṇasaṃyāmaḥ
prāṇāyāmo mudrā mahāmudrādikās taiḥ saralā ṛvī bhavet.


%\begin{vsid}{#hp03_00}
\startsloka
{\bf abhyāse tu vinidrāṇāṃ mano dhṛtvā samādhinā
rudrāṇī vāparā mudrā bhadrāṃ siddhiṃ prayacchati (3.125)}
\stopsloka

abhyāsa iti. samādhinetaravṛttinirodharūpeṇaikāgryeṇa mano dhṛtvāntaḥkaraṇaṃ dhāraṇāniṣṭhaṃ
kṛtvābhyāse manaḥsthitau yatne vigatā nidrā yeṣāṃ te tathā teṣām. nidrāpadam ālasyopalakṣaṇam
analasānām ity arthaḥ. rudrāṇī śāmbhavī mudrā vā athavāparānyā unmanyādikā bhadrāṃ śubhāṃ siddhiṃ
yogasiddhiṃ prayacchati dadāti. etena haṭhayogopakārako rājayogaḥ proktaḥ.

%\begin{vsid}{#hp03_00}
\startsloka
{\bf rājayogaṃ vinā pṛthvī rājayogaṃ vinā niśā
rājayogaṃ vinā mudrā vicitrāpi na rājate (3.126)}
\stopsloka

rājayogaṃ vinā āsanādīnāṃ vaiyarthyam aupacārikaślokenāha rājayogam.
iti. vṛttyantaranirodhapūrvakātmagocaradhārāvāhikanirvikalpavṛttī rājayogaḥ. haṭhaṃ vinā
rājayogaḥ\comment{HP 2.76} ity atra sūcitas tatsādhanābhyāso vā taṃ vinā tam ṛte. pṛthvīśabdena
sthairyaguṇayogād āsanaṃ lakṣyate. tan na rājate na śobhate. rājayogaṃ vinā
paramapuruṣārthaphalāsiddher iti hetur agre 'pi yojanīyaḥ.  rājayogaṃ vinā niśeva niśā kumbhako na
rājate niśāyāṃ prāyeṇa janasañcārābhāvāt. niśāśabdena prāṇasañcārābhāvalakṣaṇaḥ kumbhako
lakṣyate. rājayogaṃ vinā mudrā mahāmudrādirūpā vicitrāpi vividhāpi vilakṣaṇāpi vā na rājate na
śobhate.

pakṣāntare rājño nṛpasya yogo rājayogo rājasambandhas taṃ vinā pṛthvī bhūmir na rājate. śāstāraṃ 
vinā bhūmau nānopadravasambhavāt. rājā candraḥ. somo 'smākaṃ brāhmaṇānāṃ
rājā\comment{Taittirīyasaṃhitā 1.8.10.2.} iti śruteḥ. tasya yogaṃ sambandhaṃ vinā niśā rātrir na
rājate. rājayogaṃ vinā nṛpasambandhaṃ vinā mudrā rājabhiḥ patreṣu kriyamāṇaś
cihnaviśeṣaḥ. vicitrāpi. pṛthvīpakṣe ratnādijanakatvena vilakṣaṇāpi niśāpakṣe grahanakṣatrādibhir
vicitrāpi mudrāpakṣe rekhābhir vicitrāpi na rājate.


%\begin{vsid}{#hp03_00}
\startsloka
{\bf mārutasya vidhiṃ sarvaṃ manoyuktaṃ samabhyaset
itaratra na kartavyā manovṛttir manīṣiṇā (3.127)}
\stopsloka

mārutasyeti. mārutasya vāyoḥ sarvaṃ vidhiṃ kumbhakamudrāvi\-dhānaṃ manoyuktaṃ manasā yuktaṃ
samabhyaset samyag abhyaset manīṣiṇā buddhimatā puṃsā. itaratra mārutasya vidher anyasmin viṣaye
manovṛttir manaso vṛttir na kartavyā na kāryā.

%\begin{vsid}{#hp03_00}
\startsloka
{\bf iti mudrā daśa proktā ādināthena śambhunā
ekaikā tāsu yamināṃ mahāsiddhipradāyinī (3.128)}
\stopsloka

mudrā upasaṃharati itīti. ādināthena sarveśvareṇa śambhunā śaṃ sukhaṃ bhavaty asmād iti śambhuḥ
tena. ity uktarītyā daśa daśasaṅkhyākā mudrāḥ proktāḥ kathitāḥ. tāsu mudrāsu madhye ekaikāpi
pratyekāpi yā kācana mudrā yamināṃ yamavatāṃ yogināṃ mahāsiddhipradāyinī aṇimādipradātrī
kaivalyapradātrī vā.

%\begin{vsid}{#hp03_00}
\startsloka
{\bf upadeśaṃ hi mudrāṇāṃ yo datte sāmpradāyikam
sa eva śrīguruḥ svāmī sākṣād īśvara eva saḥ (3.129)}
\stopsloka

mudropadeṣṭāraṃ guruṃ praśaṃsati upadeśam iti. yaḥ pumān mudrāṇāṃ mahāmudrādīnāṃ sampradāyāt
yogināṃ guruparamparārūpād āgataṃ sāmpradāyikam upadeśaṃ datte dadāti sa eva sa pumān eva śrīguruḥ
śrīmān guruḥ sarvagurubhyaḥ śreṣṭha ity arthaḥ. svāmī prabhuḥ sa eva sākṣāt pratyakṣa īśvara eva
saḥ īśvarābhinna eva sa ity arthaḥ.

%\begin{vsid}{#hp03_00}
\startsloka
{\bf tasya vākyaparo bhūtvā mudrābhyāse samāhitaḥ
aṇimādiguṇaiḥ sārdhaṃ labhate kālavañcanam (3.130)}
\stopsloka

tasya mudrāṇām upadeṣṭur guror vākyaparo vākyam āsanakumbhakādyanuṣṭhānaviṣayakaṃ
ca tasmin paras tatparaḥ tatparaś cādaravān. ādaraś cāvahitatayā karaṇam. bhūtvā sambhūya
mudrāṇāṃ mahāmudrādīnāṃ abhyāsaḥ paunaḥpunyenāvartanaṃ tasmin mudrābhyāse samāhitaḥ sāvadhānaḥ
puruṣo 'ṇimādiguṇair aṇimādisiddhibhiḥ sārdhaṃ sākaṃ kālasya mṛtyor vañcanaṃ pratāraṇaṃ
labhate prāpnoti.


% iti śrīsvātmārāmayogīndraviracitāyāṃ haṭhayogapradīpikāyāṃ mudrāvidhānaṃ nāma tṛtīyopadeśaḥ.


%\begin{vsid}{#hp04_00}
\startsloka
{\bf namaḥ śivāya gurave nādabindukalātmane
nirañjanapadaṃ yāti nityaṃ tatra parāyaṇaḥ (4.1)}
\stopsloka

prathamadvitīyatṛtīyopadeśoktānām āsanakumbhakamudrāṇāṃ phala\-bhūtaṃ rājayogaṃ vivakṣuḥ svātmārāmaḥ
śreyāṃsi bahuvighnānīti tatra vighnabāhulyasya sambhavāt tannivṛttaye śivābhinnagurunamas\-kārātmakaṃ
maṅgalam ācarati nama iti.

śivāya sukharūpāyeśvarābhinnāya vā. tad uktaṃ

\startsloka
  namas te nātha bhagavan śivāya gururūpiṇe\comment{The verse occurs in many texts, e.g.\ {\em
      Śrīvidyārṇavatantra} 1.383}
\stopsloka

iti. gurave deśikāya yad vā gurave sarvāntaryāmitayā nikhilopadeṣṭre śivāyeśvarāya. tathā ca
pātañjalasūtram

\startsloka
  pūrveṣām api guruḥ kālenānavacchedāt\comment{Yogasūtra 1.26.} iti.
\stopsloka

namaḥ prahvībhāvo 'stu. kīdṛśāya śivāya gurave nādabindukalātmane. kāṃsyaghaṇṭānirhrādavad
anuraṇanaṃ nādaḥ bindur anusvārottarabhāvī dhvaniḥ kalā nādaikadeśas tā ātmā svarūpaṃ yasya sa
tathā tasmai. nādabindukalātmanā vartamānāyety arthaḥ. tatra nādabindukalātmani śive gurau nityaṃ
pratidinaṃ parāyaṇo 'vahitaḥ pumān. etena nādānusandhānaparāyaṇa ity uktam. pūrvapādena guruśivayor
abhedaḥ sūcitaḥ. añjanaṃ māyopādhis tadrahitaṃ nirañjanaṃ śuddham padyate gamyate yogibhir iti
padaṃ brahma yāti prāpnoti. tathā ca vakṣyati nādānusandhānasamādhibhājām\comment{HP 4.81.}
ityādinā.


%\begin{vsid}{#hp04_00}
\startsloka
{\bf athedānīṃ pravakṣyāmi samādhikramam uttamam
mṛtyughnaṃ ca sukhopāyaṃ brahmānandakaraṃ param (4.2)}
\stopsloka

samādhikramaṃ pratijānīte atheti. athāsanakumbhakamudrākathanānantar\-am idānīm asminn avasare
samādhikramam pratyāhārādirūpaṃ pravakṣyāmi prakarṣeṇa vivicya vakṣyāmīty anvayaḥ. kīdṛśaṃ
samādhikramam uttamam śrīādināthoktasapādakoṭisamādhiprakāreṣūtkṛṣṭam.

punaḥ kīdṛśam mṛtyuṃ kālaṃ hanti nivārayatīti mṛtyughnaṃ svecchayā dehatyāgajanakaṃ
tattvajñānodayamanonāśavāsanākṣayaiḥ sukhasya jīvanmuktisukhasyopāyaṃ prāptisādhanam.

punaḥ kīdṛśam paraṃ brahmānandakaraṃ prārabdhakarmakṣaye sati jīvabrahmaṇor
abhedenātyantikabrahmānandaprāptirūpavidehamuktikaram. tatra nirodhasamādhinā cittasya
sasaṃskārāśeṣavṛttinirodhe śāntaghoramūḍhāvasthānivṛttau

\startsloka
jīvann eva hi vidvān harṣaśokābhyāṃ vimucyate\comment{quotation untraced}
\stopsloka

ity ādiśrutyuktanirvikārasvarūpāvasthitirūpā jīvanmuktir bhavati. parama\-muktis tu
prārabdha\var{prārabdha \lem prānta \Tue \Wthree prāpta \EdMu \EdLo}bhogānte 'ntaḥkaraṇaguṇānāṃ
pratiprasavenaupādhikarūpātyantikanivṛttāv ātyantikasvarūpāvasthānaṃ
pratiprasava\-siddham. vyutthānanirodhasamādhisaṃskārā manasi līyante. mano 'smitāyām asmitā mahati
mahān pradhāna iti cittaguṇānāṃ pratiprasavaḥ pratisargaḥ svakāraṇe layaḥ.

nanu jīvanmuktasya vyutthāne brāhmaṇo 'haṃ manuṣyo 'ham ity ādi vyavahāradarśanāc cittādibhir
aupādhikabhāvajananād amlena dugdhasyeva svarūpacyutiḥ syād iti cen na samprajñātasamādhau
anubhūtātmasaṃskārasya nirodhasaṃskārasya ca tadānīṃ sattvāt. tābhyāṃ ca vyutthānasaṃskārasya
dagdhabījakalpatvād vyutthānavyavahārasyā\-tāttvi\-ka\-tvaniścayāt. atāttvikānyathābhāvasya
vikāritvāprayojakatvāt. amlena dugdhasya dadhibhāvas tu tāttvika iti dṛṣṭāntavaiṣamyāc
ca. puruṣasya tv antaḥkaraṇopādhiko 'haṃ brāhmaṇa ityādivyavahāraḥ sphaṭikasya
japākusumasaṃnidhānopādhir aruṇimeva na tāttvikaḥ. japākusumāpagame sphaṭikasya svasvarūpasthitivad
antaḥkaraṇasya sakalavṛttinirodhe svasvarūpāvasthitir acyutaiva puruṣasya.



%\begin{vsid}{#hp04_00}
\startsloka
{\bf rājayogaḥ samādhiś ca unmanī ca manonmanī
amaratvaṃ layas tattvaṃ śūnyāśūnyaṃ paraṃ padam (4.3)
amanaskaṃ tathādvaitaṃ nirālambaṃ nirañjanam
jīvanmuktiś ca sahajā turyā cety ekavācakāḥ (4.4)}
\stopsloka

samādhiparyāyān viśeṣeṇāha rājayoga ityādinā ślokadvayena. spaṣṭam.

%\begin{vsid}{#hp04_00}
\startsloka
{\bf salile saindhavaṃ yadvat sāmyaṃ bhajati yogataḥ
tathātmamanasor aikyaṃ samādhir abhidhīyate (4.5)}
\stopsloka

tribhiḥ samādhim āha salila iti. yadvat yathā saindhavaṃ sindhudeśodbhavaṃ lavaṇaṃ salile jale
yogataḥ saṃyogāt sāmyaṃ salilasāmyaṃ salilaikyaṃ bhajati prāpnoti tathā tadvad ātmā ca manaś
cātmamanasī tayor ātmamanasor aikyam ekākāratā. ātmani dhāritaṃ mana ātmākāraṃ sadātmasāmyaṃ
bhajati tādṛśam ātmamanasor aikyaṃ samādhir abhidhīyate samādhiśabdenocyate ity arthaḥ.

%\begin{vsid}{#hp04_00}
\startsloka
{\bf yadā saṃkṣīyate prāṇo mānasaṃ ca pralīyate
tadā samarasatvaṃ ca samādhir abhidhīyate (4.6)}
\stopsloka

yadā saṃkṣīyate iti. yadā yasmin kāle jitaprāṇasya yoginaḥ kumbhakakāle prāṇaḥ śarīrāntarvartī
vāyuḥ kumbhakena niruddhavṛttikaḥ prāṇaḥ kṣīṇa ity ucyate saṃkṣīyate samyak kṣīṇo bhavati.

mānasaṃ ca manaś ca pralīyate prakarṣeṇa layaṃ prāpnoti ātmani sthitasya manasaḥ ātmākāratā layaḥ
tadā tasmin kāle samarasatvam ekākāratvaṃ manasaś ca ātmani sthitasya
ātmākārapariṇāmenātmākāratvam. japākusumasthasya sphaṭikamaṇer japākusumākāratvavat. tathā ca
pātañjalaṃ sūtraṃ

% Strange interpretation: see also sāmarasyamekībhāvaḥ (Nityaṣoḍaśik), vedyavedakasāmarasyaṃ
% (śitikaṇṭha),evākhaṇḍacidābhāsarūpo'nākhyakramamayo bhedavigalanena sāmarasyamayo, ekībhūtaṃ
% sāmarasyaṃ prāptaṃ (Sambapancasika), teṣāṃ mātṛmānameyānāṃ śaṅghaṭṭaḥ" sāmarasyaṃ tataḥ TĀV 6.103



\startsloka
  kṣīṇavṛtter abhijātasyeva maṇer grahītṛgrahaṇagrāhyeṣu
  tatsthatadañjanatā samāpattiḥ\comment{Yogasūtra 1.41}
\stopsloka

cakāreṇa samarasatvasya sthairyaṃ samuccīyate. ata eva pātañjalabhāṣye

\startsloka
  vikṣipte cetasi vikṣepopasarjanībhūtaḥ samādhir na
  yogapakṣe vartate\comment{Yogasūtrabhāṣya 1.1}
\stopsloka

ity uktam. itthaṃ cātmākāramanovṛtteḥ sthirībhāvaḥ samādhir abhidhīyate ity arthaḥ. uktābhyāṃ
dvābhyāṃ ślokābhyāṃ samprajñātaḥ samādhir uktaḥ. samyak saṃśayaviparyayarahitatvena prajñāyate
prakarṣeṇāparokṣatayānubhūyate bhāvyasya svarūpam aneneti samprajñātaḥ samādhiḥ
bhāvanāviśeṣaḥ. bhāvanā ca bhāvyasya viṣayāntaraparihāreṇa cetasi punaḥ punar
niveśanam samprajñātaḥ sālambanaḥ sabījo layaḥ. ete samprajñātasya prasiddhāḥ paryāyāḥ.



%\begin{vsid}{#hp04_00}
\startsloka
{\bf tatsamaṃ ca dvayor aikyaṃ jīvātmaparamātmanoḥ
pranaṣṭasarvasaṅkalpaḥ samādhiḥ so 'bhidhīyate (4.7)}   % pranaṣṭa conj Jyo  all VM read samastanaṣṭa-
\stopsloka

tatsamam iti. tatsamaṃ pūrvoktasamaṃ dvayor ubhayor jīvātmā antaḥkaraṇopahitaṃ caitanyam paramātmā
saccidānandalakṣaṇaṃ deśakālavastuparicchedaśūnyaṃ caitanyaṃ tayor aikyam abhedaḥ. aikyaṃ
viśinaṣṭi pranaṣṭasarvasaṅkalpa iti. prakarṣeṇa naṣṭā niruddhāḥ sarve dhyātṛdhyānadhyeyādirūpāḥ
saṅkalpāḥ samyak kalpanāḥ yasmin tat tādṛśam aikyam. puṃliṅgapāṭhe samādher viśeṣaṇam. sa
yogaśāstraprasiddhaḥ samādhiḥ nirbījaḥ samādhiḥ abhidhīyate kathyate. tathā ca pātañjalaṃ
sūtram

\startsloka
tasyāpi nirodhe sarvanirodhān nirbījaḥ samādhiḥ\comment{Yogasūtra 1.51}
\stopsloka

iti anenāsamprajñātaḥ samādhir uktaḥ. anvarthaś cāyaṃ na kim api samyag dhyātṛdhyānadhyeyatvādinā
prajñāyate 'sminn ity asamprajñātaḥ. asamprajñāto nirālambo nirbījo rājayogo nirodhaś caite
asamprajñātasya prasiddhāḥ paryāyāḥ.

%\begin{vsid}{#hp04_00}
\startsloka
{\bf rājayogasya māhātmyaṃ ko vā jānāti tattvataḥ
jñānaṃ muktiḥ sthitiḥ siddhir guruvākyena labhyate (4.8)}
\stopsloka

atha rājayogapraśaṃsā rājayogasyeti. rājayogasyānantaram evoktasya māhātmyaṃ prabhāvaṃ tattvato
vastutaḥ ko vā jānāti. na ko 'pi jānātīty arthaḥ. tattvato vaktum aśakyatve 'py ekadeśena
rājayogaprabhāvam āha jñānam iti. jñānam svasvarūpāparokṣānubhavo muktir videhamuktiḥ sthitir
nirvikārasvarūpāvasthitirūpā jīvanmuktiḥ siddhir aṇimādir guruvākyena guruvacasā
labhyate. rājayogād iti śeṣaḥ.

%\begin{vsid}{#hp04_00}
\startsloka
{\bf durlabho viṣayatyāgo durlabhaṃ tattvadarśanam
durlabhā sahajāvasthā sadguroḥ karuṇāṃ vinā (4.9)}
\stopsloka

durlabha iti. viśeṣeṇa sinvanty avabadhnanti pramātāraṃ svasaṅgeneti\var{svasaṅgeneti \Wthree \Tue
  \EdAd \EdMu \lem svasaṃvegena \EdLo (no variants given)} viṣayā aihikā dārādaya āmuṣmikāḥ
sudhādayas teṣāṃ tyāgo bhogecchābhāvo durlabhaḥ. tattvadarśanam ātmāparokṣānubhavaḥ
durlabham. sahajāvasthā turyāvasthā. sadguroḥ

\startsloka
dṛṣṭiḥ sthirā yasya vinaiva dṛśyam\comment{Amanaska 44}
\stopsloka

iti vakṣyamāṇalakṣaṇasya karuṇāṃ dayāṃ vineti sarvatra sambadhyate. durlabhā labdhum aśakyā duḥ
syāt kaṣṭaniṣedhayoḥ iti koṣaḥ. gurukṛpayā tu sarvaṃ sulabham iti bhāvaḥ.

%\begin{vsid}{#hp04_00}
\startsloka
{\bf vividhair āsanaiḥ kumbhair vicitraiḥ karaṇair api
prabuddhāyāṃ mahāśaktau prāṇaḥ śūnye pralīyate (4.10)}
\stopsloka

vividhair iti. vividhair anekavidhair āsanaiḥ matsyendrādipīṭhaiḥ vicitrair nānāvidhaiḥ kumbhaiḥ
kumbhakaiḥ. vicitrair iti kākākṣigolakanyāyenobhayatra sambadhyate. vicitrair anekaprakārakaiḥ
karaṇair haṭhasiddhau prakṛṣṭopakārakair mahāmudrādibhir mahāśaktau kuṇḍalinyāṃ prabuddhāyāṃ
gatanidrāyāṃ satyāṃ prāṇo vāyuḥ śūnye brahmarandhre pralīyate pralayaṃ prāpnoti. vyāpārābhāvaḥ
prāṇasya pralayaḥ.

%\begin{vsid}{#hp04_00}
\startsloka
{\bf utpannaśaktibodhasya tyaktaniḥśeṣakarmaṇaḥ
yoginaḥ sahajāvasthā svayam eva prajāyate (4.11)}
\stopsloka

utpanneti. utpanno jātaḥ śaktibodhaḥ kuṇḍalībodho yasya tasya tyaktāni parihṛtāni niḥśeṣāṇi
samagrāṇi karmāṇi yena tasya yoginaḥ āsanena kāyikavyāpāre tyakte prāṇendriyeṣu vyāpāras
tiṣṭhati. kumbhakena prāṇanirodhe prāṇendriyavyāpāre tyakte manasi vyāpāras
tiṣṭhati. pratyāhāradhāraṇādhyānasamprajñātasamādhibhir mānasikavyāpāre tyakte buddhau vyāpāras
tiṣṭhati. asaṅgo hy ayaṃ puruṣaḥ\comment{Bṛhad\-āraṇya\-kopani\-ṣat 4.3.15} iti śruter apariṇāmī śuddhaḥ
puruṣaḥ sattvaguṇātmikā pariṇāminī buddhir iti. paravairāgyeṇa dīrghakālasamprajñātābhyāsenaiva vā
buddhivyāpāre parityakte nirvikārasvarūpāvasthitir bhavati. saiva sahajāvasthā turyāvasthā
jīvanmuktiḥ svayam eva prayatnāntaraṃ vinaiva prajāyate prādurbhavati. yena tyajasi tat tyaja iti
niḥsaṅgaḥ prajñayā bhavet\comment{Māṇḍūkyakārikā 3.45??} iti ca śruteḥ.

%\begin{vsid}{#hp04_00}
\startsloka
{\bf suṣumṇāvāhini prāṇe śūnye viśati mānase
tadā sarvāṇi karmāṇi nirmūlayati yogavit (4.12)}
\stopsloka

suṣumṇeti. prāṇe vāyau suṣumṇāvāhini madhyanāḍīpravāhiṇi sati mānase antaḥkaraṇe śūnye
deśakālavastuparicchedahīne brahmaṇi viśati sati tadā tasmin kāle yogavit cittavṛttinirodhajñaḥ
sarvāṇi karmāṇi saprārabdhāni nirmūlāni karoti nirmūlayati. nirmūlaśabdāt tat
karoti\comment{gaṇasūtra?}  iti ṇic.

%\begin{vsid}{#hp04_00}
\startsloka
{\bf amarāya namas tubhyaṃ so 'pi kālas tvayā jitaḥ
patitaṃ vadane yasya jagad etac carācaram (4.13)}
\stopsloka

samādhyabhyāsena prārabdhakarmaṇo py abhibhavāj jitakālaṃ yoginaṃ namaskaroti amarāyeti. na mriyate
ity amaras tasmai amarāya cirajīvine tubhyaṃ yogine namaḥ. so 'pi durvāro 'pi kālo mṛtyus tvayā
yoginā jito 'bhibhūtaḥ. idaṃ vākyaṃ namaskaraṇe hetuḥ. sa kaḥ yasya kālasya vadane mukhe etad
dṛśyamānaṃ carācaraṃ sthāvarajaṅgamaṃ jagat saṃsāraḥ patitaḥ so 'pi jagadbhakṣako 'pīty arthaḥ.


%\begin{vsid}{#hp04_00}
\startsloka
{\bf citte samatvam āpanne vāyau vrajati madhyame
tadāmarolī vajrolī sahajolī prajāyate (4.14)}
\stopsloka

pūrvoktam amarolyādikaṃ samādhisiddhāv eva siddhyatīti samādhinirūpa\-ṇānantaraṃ samādhisiddhau
tatsiddhir ity āha citta iti. citte 'ntaḥkaraṇe samatvam dhyeyākāravṛttipravāhavattvam āpanne
prāpte sati vāyau prāṇe madhyame suṣumnāyāṃ vrajati satīti cittasamatve hetuḥ. tadā tasmin kāle
amarolī vajrolī sahajolī ca pūrvoktāḥ prajāyante. nājitaprāṇasya na cājitacittasya siddhyantīti
bhāvaḥ.

%\begin{vsid}{#hp04_00}
\startsloka
{\bf jñānaṃ kuto manasi sambhavatīha tāvat
prāṇo 'pi jīvati mano mriyate na yāvat
prāṇo mano dvayam idaṃ vilayaṃ nayed yo
mokṣaṃ sa gacchati naro na kathañcid anyaḥ (4.15)}
\stopsloka

haṭhābhyāsaṃ vinā jñānaṃ mokṣaś ca na siddhyatīty āha jñānam iti. yāvat prāṇo jīvati apiśabdād
indriyāṇi jīvanti na tu mriyante yāvan mano na mriyate kintu jīvaty eva. iḍāpiṅgalābhyāṃ vahanaṃ
prāṇasya jīvanam svasvaviṣayagrahaṇam indriyāṇāṃ jīvanam nānāviṣayākāravṛttyutpādanaṃ manaso
jīvanam tattadabhāvaḥ tattanmaraṇam atra vivakṣitam na tu svarūpatas teṣāṃ nāśas. tāvan manasy
antaḥkaraṇe jñānam ātmāparokṣānubhavaḥ kutaḥ sambhavati na kuto'pi. prāṇendriyamanovṛttīnāṃ
jñānapratibandhakatvād iti bhāvaḥ. prāṇo manaḥ idaṃ dvayaṃ yo yogī vilayaṃ nāśaṃ nayet sa mokṣam
ātyantikasvarūpāvasthānalakṣaṇaṃ gacchati prāpnoti. brahmarandhre nirvyāpārasthitiḥ prāṇasya
layaḥ. dhyeyākārāveśāt viṣayāntareṇāpariṇamanaṃ manaso layaḥ. anyaḥ alīnaprāṇo 'līnamanāś ca
kathañcid upāyaśatenāpi na mokṣaṃ prāpnotīty arthaḥ. tad uktaṃ yogabīje


\startsloka
nānāvidhair vicārais tu na sādhyaṃ jāyate manaḥ
tasmāt tasya jayopāyaḥ prāṇasya jaya eva hi \  iti\comment{Yogabīja 80?}
nānāmārgaiḥ suduṣprāpaṃ kaivalyam paramaṃ padam
siddhamārgeṇa labhyeta nānyathā śivabhāṣitam \ iti ca \comment{Yogabīja 8?}
\stopsloka

siddhamārgo yogamārgaḥ. etena yogaṃ vinā jñānaṃ mokṣaś ca na siddhyatīti
siddham. śrutismṛtītihāsapurāṇādiṣu cedaṃ prasiddham. tathā hi atha taddarśanābhyupāyo yogaḥ
iti\comment{atha taddarśanābhyopāyo yogaḥ (quotation not traced)} taddarśanam ātmadarśanam.


\startsloka
adhyātmayogādhigamena devaṃ
matvā dhīro harṣaśokau jahāti \comment{Kaṭhopaniṣat 1.2.12 ?} iti.

śraddhābhaktidhyānayogād avehi \comment{Kaivalyopaniṣat 2 ?} iti.

yadā pañcāvatiṣṭhante jñānāni manasā saha
buddhiś ca na viceṣṭati tām āhuḥ paramāṃ gatim
tāṃ yogam iti manyante sthirām indriyadhāraṇām
apramattas tadā bhavati iti \comment{Kaṭhopaniṣat 2.3–10 ?}

yad ātmatattvena tu brahmatattvaṃ
dīpopameneha yuktaḥ prapaśyet
ajaṃ dhruvaṃ sarvatattvair viśuddhaṃ
jñātvā devaṃ mucyate sarvapāśaiḥ \comment{Śvetaśvataropaniṣat 2.15 ?}

brahmaṇe tvā mahasa om ity ātmānaṃ yuñjīta \comment{Mahā\-nārāyaṇo\-pani\-ṣat 24.2 ?}

trir unnataṃ sthāpya samaṃ śarīraṃ
hṛdīndriyāṇi manasā sannirudhya
brahmoḍupena pratareta vidvān
srotāṃsi sarvāṇi bhayāvahāni \comment{Śvetaśvataropaniṣat 2.8 ?}

om ity evaṃ dhyāyatha ātmānaṃ \comment{Muṇḍakopaniṣat 2.2.6 ?} ityādyāḥ śrutayaḥ.
\stopsloka

yatidharmaprakaraṇe\comment{??} manuḥ

\startsloka
bhūtabhāvyān avekṣeta yogena paramātmanaḥ
dehadvayaṃ vihāyāśu mukto bhavati bandhanāt
\stopsloka

yājñavalkyasmṛtau\comment{Yājñavalkyasmṛti 1.1.8}

\startsloka
ijyācāradamāhiṃsādānasvādhyāyakarmaṇām
ayaṃ tu paramo dharmo yad yogenātmadarśanam 
\stopsloka

maharṣimātaṅgaḥ\comment{??}

\startsloka
agniṣṭomādikān sarvān vihāya dvijasattamaḥ
yogābhyāsarataḥ śāntaḥ paraṃ brahmādhigacchati
brāhmaṇakṣatriyaviśāṃ strīśūdrāṇāṃ ca pāvanam
śāntaye karmaṇām anyad yogān nāsti vimuktaye
\stopsloka

dakṣasmṛtau vyatirekamukhenoktam\comment{7.25 / 8.25}

\startsloka
svasaṃvedyaṃ hi tad brahma kumārī strīsukhaṃ yathā
ayogī naiva jānāti jātyandho hi yathā ghaṭam ityādyāḥ smṛtayaḥ.
\stopsloka


mahābhārate yogamārgaprasaṅge vyāsaḥ

\startsloka
api varṇāvakṛṣṭas tu nārī vā dharmakāṅkṣiṇī  % ?
tāv apy etena mārgeṇa gacchetāṃ paramāṃ gatim\comment{Mahābhārata 12.232.32 (246.34)?}
yadi vā sarvadharmajño yadi vāpy akṛtī pumān
yadi vā dhārmikaḥ śreṣṭho yadi vā pāpakṛttamaḥ 
yadi vā puruṣavyāghro yadi vā klaibyadhārakaḥ
taraty eva mahāduḥkhaṃ jarāmaraṇasāgaram 
api jijñāsamāno 'pi śabdabrahmātivartate\comment{Mahābhārata 12.228.68} iti.
\stopsloka


bhagavadgītāyām

\startsloka
yuñjann evaṃ sadātmānaṃ yogī niyatamānasaḥ
śāntiṃ nirvāṇaparamāṃ matsaṃsthām adhigacchati \comment{gītā 6.15}

yat sāṅkhyaiḥ prāpyate sthānam \comment{gītā 5.5} ity ādi ca. 
\stopsloka


ādityapurāṇe

\startsloka
yogāt sañjāyate jñānaṃ yogo mayy ekacittatā 
\stopsloka


skandapurāṇe \comment{?}
\startsloka
ātmajñānena muktiḥ syāt tac ca yogād ṛte nahi
sa ca yogaś ciraṃ kālam abhyāsād eva siddhyati
\stopsloka


kūrmapurāṇe śivavākyam \comment{?}

\startsloka
ataḥ paraṃ pravakṣyāmi yogaṃ paramadurlabham
yenātmānaṃ prapaśyanti bhānum antam iveśvaram
yogāgnir dahati kṣipram aśeṣaṃ pāpapañjaram
prasannaṃ jāyate jñānaṃ jñānān nirvāṇam ṛcchati
\stopsloka

garuḍapurāṇe \comment{?}

\startsloka
tathā yateta matimān yathā syān nirvṛtitḥ parā
yogena labhyate sā tu na cānyena tu kenacit
bhavatāpena taptānāṃ yogo hi paramauṣadham
parāvaraprasaktā dhīr yasya nirvedasambhavā
sa ca yogāgninā dagdhasamastakleśasañcayaḥ
nirvāṇaṃ paramaṃ nityaṃ prāpnoty eva na saṃśayaḥ
samprāptayogasiddhis tu pūrṇo yas tv ātmadarśanāt
na kiñcid dṛśyate kāryaṃ tenaiva sakalaṃ kṛtam
ātmārāmaḥ sadā pūrṇaḥ sukham ātyantikaṃ gataḥ
atas tasyāpi nirvedaḥ parānandamayasya ca
tapasā bhāvitātmāno yoginaḥ saṃyatendriyāḥ
prataranti mahātmāno yogenaiva mahārṇavam 
\stopsloka


viṣṇudharmeṣu \comment{?}

\startsloka
yac chreyaḥ sarvabhūtānāṃ strīṇām atyupakārakam
api kīṭapataṅgānāṃ tan naḥ śreyaḥ paraṃ vada
ity uktaḥ kapilaḥ pūrvaṃ devair devarṣibhis tathā
yoga eva paraṃ śreyas teṣām ity uktavān purā
\stopsloka

vāsiṣṭhe

\startsloka
duḥsahā rāma saṃsāraviṣavegaviṣūcikā
yogagāruḍamantreṇa pāvanenopaśāmyati iti\comment{Laghuyogavāsiṣṭha 2.51}
\stopsloka

nanu tattvamasyādivākyair apy aparokṣapramā sambhavatīti kimartham atiśramasādhye yoge prayāsaḥ
kāryaḥ. na ca vākya\-janya\-jñānasyāpa\-rokṣatve pramāṇāsambhava iti
vācyam. tattvama\-syādi\-vākya\-janyaṃ jñānam aparokṣaṃ aparokṣaviṣayakatvāt
cākṣuṣaghaṭādi\-pratyakṣa\-vad ity anumānasya pramāṇatvāt. na ca viṣayagatāparokṣatvasya
durnirūpatvād dhetvasiddhir iti vācyam.
ajñānāviṣaya\-cittva\-tattādātmyā\-panna\-tvānya\-tara\-rūpasya tasya sunirūpatvāt. yathā hi
ghaṭādau cakṣuḥsaṃnikarṣeṇāntaḥkaraṇavṛttidaśāyāṃ tadadhiṣṭhānacaitanyājñānanivṛttau tac
caitanyasyājñānāviṣayatvaṃ tadghaṭasyājñānāviṣayacaitanyatādātmyāpannatvaṃ cāparokṣatvam. tathā
tattvamasyādivākyena śuddhacaitanyākārāntaḥkaraṇavṛttyutthāpane sati tadajñānasya nivṛttatvenaiva
caitanyasyājñānāviṣayatvāc caitanyasyāparokṣatvam iti na hetvasiddhiḥ. na
cāprayojakatvam. jñānagatāparokṣatvaṃ praty aparokṣaviṣayakatvena prayojakatvāt.

nanv indriyajanyatvaṃ manasa indriyatvābhāvena sukhādipratyakṣe
vyabhicārāt. athavābhivyaktacaitanyābhinnatayā bhāsamānatvaṃ viṣayasyāparokṣatvam. abhivyaktatvaṃ
ca nivṛttāvaraṇakatvaṃ parokṣavṛttisthale cāvaraṇanivṛttyabhāvān
nātivyāptiḥ. sarpādibhramajanakadoṣavatas tu nāyaṃ sarpaḥ kintu rajjur iti vākyena jāyamānā vṛttis
tu nāvaraṇam nivartayatīti tatra parokṣa eva viṣayaḥ. vedāntavākyajanyaṃ ca jñānam
āvaraṇanivartakatvād aparokṣam eva. mananādeḥ pūrvam utpannaṃ jñānam ajñānānivartakam.
pramāṇāsambhāvanādidoṣasāmānyābhāvaviśiṣṭasyaiva tasyājñānanivartakatvāt.

kiṃ ca taṃ tv aupaniṣadaṃ puruṣaṃ pṛcchāmi \comment{Bṛhad\-āraṇya\-kopani\-ṣat 3.9.26} iti
śrutipratipannam upaniṣanmātragamyatvaṃ yogagamyatve nopapannaṃ syāt. tasmāt tattvamasyādivākyād
evāparokṣam iti cen na. anumānasyāprayojakatvāt. na ca pratyakṣaṃ prati niruktāparokṣaviṣayakatvena
prayojakatvāt tanmūlakatarkeṇāprayojakaśaṅkānivṛttir iti vācyam. lāghavāj janyapratyakṣasāmānyaṃ
prati indriyatvena kāraṇatayā tajjanyatvasyaiva prayojakatvāt. nityānityasādhāraṇapratyakṣatve tu
na kiñcit prayojakam iti. tanmate\comment{read tvanmate \Wthree \Tue etc} tu pratyakṣaviśeṣe indriyaṃ
kāraṇaṃ tadviśeṣe ca śabdaviśeṣa iti kāryakāraṇabhāvadvayaṃ syāt. na ca manaso 'nindriyatvaṃ manasa
indriyatve bādhakābhāvāt indriyāṇāṃ mano nāthaḥ \comment{HP 4.29} iti tu manuṣyam evoddiśya
manuṣyāṇām ayaṃ rājetyādivad indriyeṣv eva kiñcid utkarṣaṃ bravīti. na tu tasyānindriyatvaṃ tattvaṃ
ca ṣaṭsv akhaṇḍopādhiviśeṣa eva. ata eva karmendriyaṃ tu pāyvādi manonetrādi dhīndriyam
\comment{Amarakośa 1.5.8} iti pratyakṣaṃ syād aindriyakam apratyakṣam atīndriyam \comment{Amarakośa
  3.1.79} iti ca śaktipramāṇabhūtakośo 'pīndriyāpramāṇakajñānasyāpratyakṣatvaṃ vadan manasa
indriyatvajñāpakaḥ saṅgacchate. indriyāṇi daśaikaṃ ca \comment{Bhagavadgītā 13.5} iti gītāvacanam
api manasa indriyatve pramāṇam.

kiṃ ca tattvamasyādivākyajanyaṃ jñānaṃ śābdam. śabdajanyatvād yajetetyādivākyajanyajñānavad ity
anenāparokṣatvavirodhiśābdatvasādhakena satpratipakṣaḥ. na cedam aprayojakam. śābdaṃ praty eva
śabdasya janakatvena lāghavamūlakānukūlatarkāt. tvanmate tu śabdād api pratyakṣasvīkāreṇa
kāryakāraṇabhāvadvayakalpane gauravam. api ca manananididhyāsanābhyāṃ pūrvam apy utpannaṃ jñānaṃ
tava mate 'parokṣam api nājñānanivartakam ity ajñānanivṛttiṃ praty apratibaddhajñānatvenaiva
hetutvam iti gauravam. mama tu        % \W 261
samādhyabhyāsaparipākenāsambhāvanādisakalamalarahitenāntaḥkaraṇenātmani dṛṣṭe sati darśanamātrād
evājñānanivṛtter na kaścid gauravāvakāśaḥ.

\startsloka
eṣa sarveṣu bhūteṣu gūḍho 'tmā na prakāśate
dṛśyate tv agryayā (\Wthree agrayā) buddhyā sūkṣmayā sūkṣmadarśibhiḥ
yacched vāṅmanasī prājñaḥ \comment{Kaṭhopaniṣat 1.3.12–13 ?} 
\stopsloka

ity ārabhyājñānanivṛttyarthakena mṛtyumukhāt pramucyate \comment{Kaṭhopaniṣat 1.3.16} ityantena
kaṭhavallīsthamṛtyūpadeśena saṃmato 'yam artha iti na kaścid atra vivāda iti. yadi tu mananādeḥ
pūrvam utpannaṃ jñānaṃ parokṣam eveti nāpratibaddhatvakṛtagauravam iti matam ādriyate. tadāpi
śravaṇādibhir manaḥsaṃskāre siddhe 'vyavahitottaram ātmadarśanasambhavāt taduttaraṃ
vākyasmaraṇādikalpanaṃ mahadgauravāpādakam eva.

nanu na vayaṃ kevalena tarkeṇa śabdajanyajñānasyāparokṣatvaṃ vadāmaḥ\var{vadāma \lem vadāma \Wthree}
kintu śrutyāpi. tathā hi taṃ tv aupaniṣadaṃ puruṣaṃ pṛcchāmi \comment{Bṛhad\-āraṇya\-kopani\-ṣat 3.9.26}
iti śrutyā\var{śrutyā ca\lem śrutau \Wthree} caupaniṣadatvaṃ puruṣasya
nopaniṣajjanyabuddhiviṣayatvamātraṃ pratyakṣādigamye 'py aupaniṣadatvavyavahārāpatteḥ. yathā hi
dvādaśakapāleṣv aṣṭānāṃ kapālānāṃ sattvād\var{sattvād \lem sattve 'pi \Wthree} dvādaśakapāla\var{kapāla
  \lem om \Wthree}saṃskṛtenāṣṭā\-kapālādivyavahāraḥ yathā dviputrādāv ekaputrādivyavahāraḥ tathātrāpi
nānyatra tathā vyavahāra iti upaniṣanmātragamyatvam eva pratyayārthaḥ.

tac ca manogamyatve' nupapannam iti cen na. na hi pratyayenopaniṣadbhinnaṃ sarvaṃ
kāraṇatvena\var{kāraṇatvena \EdLo \Wthree \lem kāraṇatvād \EdAd} vyāvartyate. śābdāparokṣavādinā tvayāpy
ātmāparokṣe manaādīnāṃ karaṇatvasyāṅgīkārāt. kiṃ tu purāṇādiśabdāntaram eva śrotavyaḥ
śrutivākyebhyaḥ\comment{???} iti smaraṇāt. sa cārtho mamāpi saṃmata iti na kiñcid
etat. pramāṇāntaravyāvṛttau tātparyakalpanaṃ cātmāparokṣe śabdasya pramāṇatve siddha eva vaktum
ucitam. śabdāntaravyāvṛttitātparyaṃ tu śrutyādisaṃmatatvāt kalpayitum ucitam eva. evaṃ sthite
manasaivānudraṣṭavyam \comment{Bṛhad\-āraṇya\-kopani\-ṣat 4.4.19} manasaivedam āptavyam
\comment{Kaṭhopaniṣat 2.1.11} ityādiśrutayo 'py āñjasyena pratipāditā bhaveyuḥ.

% wwwwwwwwwww

yat tu kaiścid uktaṃ darśanavṛttiṃ prati manomātrasyopādānatvaparāyattāḥ\var{parāyattāḥ \lem parā
  etā \Wthree} śrutayo na virudhyanta iti tad atīva vicārāsaham. yataḥ pramāṇākāṅkṣāyāṃ pravṛttās tāḥ
katham upādānaparā bhaveyuḥ. kāmaḥ saṅkalpo vicikitsā \comment{Bṛhad\-āraṇya\-kopani\-ṣat 1.5.3}
ityādiśrutyā sāvadhāraṇayā sarvāsāṃ vṛttīnāṃ manomātropādānakatve bodhite
ākāṅkṣābhāvenopādānatātparyakatvena varṇayituṃ kathaṃ śakyeran. pūrvaṃ dvitīyavallyāṃ praṇavasya
brahmabodhakatvenoktes tasyāpy aparokṣahetutvam iti śaṅkāṃ nivārayituṃ manasaivānudraṣṭavyam
ityādisāvadhāraṇavākyānīty evaṃ varṇayituṃ śakyāni syur ity alam ativāgjālena.

vastutas tu yogināṃ samādhau dūraviprakṛṣṭapadārthajñānaṃ sarvaśāstraprasiddhaṃ na
parokṣam. tadānīṃ parokṣasāmagryabhāvāt. nāpi smaraṇaṃ teṣāṃ pūrvaṃ viśiṣyānanubhavāt. nāpi
sukhādijñānavat sākṣirūpam apasiddhāntāt. nāpy apramāṇakaṃ pramāsāmānye karaṇaniyamāt. nāpi
cakṣurādijanyaṃ teṣām asaṃnikarṣāt. tasmān mānasikī pramaiva sā vācyeti manasa indriyatvaṃ
pramāṇatvaṃ ca durapahnavam eveti. ye 'pi yogaśrutyoḥ samuccayaṃ kalpayanti teṣām api
pūrvoktadūṣaṇagaṇas tadavastha eva. tasmād yogajanyasaṃskārasacivamanomātragamya ātmeti siddham.

na ca kāminīṃ bhāvayato vyavahitakāminīsākṣātkārasyeva
bhāvanājanyatvenātmasākṣātkārasyāpramātvaprasaṅgaḥ. abādhitaviṣayatvād doṣajanyatvābhāvāc
ca. kāminīsākṣātkārasya tu bādhitaviṣayatvād doṣajanyatvāc cāprāmāṇyaṃ na bhāvanājanyatvāt. na ca
bhāvanāsamādher jñānajanakatve pramāṇāntarāpātaḥ\var{āntarāpātaḥ \Wthree \lem āntartvāpātaḥ ?}. tasyā
manaḥsahakāritvāt pramāṇanirūpaṇanpuṇair naiyāyikādibhir api yogaja\var{yogaja \Wthree \lem yoga
  ?}pratyakṣasyālaukikapratyakṣe 'ntarbhāvaḥ kṛtaḥ. yogajālaukikasaṃni\-karṣeṇa yogino
vyavahitaviprakṛṣṭasūkṣmārtham ātmānam api yathārtham paśyanti.

tathā ca pātañjale sūtre ṛtambharā tatra prajñā śrutānumānaprajñābhyām anyaviṣayā viśeṣārthatvāt.
\comment{Yogasūtra 1.48–49} tatra samādhau yā prajñā sā ṛtaṃ satyam eva puruṣayāthārthyaṃ
bibhartīti ṛtambhareti prathamasūtrārthaḥ. sā samādhiprajñā śrutānumānaprajñābhyāṃ śrutaṃ śravaṇaṃ
śābdabodhaḥ anumananam anumānaṃ yauktika\var{yauktika \lem yauktikaṃ \Wthree}jñānaṃ tadrūpaprajñābhyām
anyaviṣayā atiriktaviṣayā. kutaḥ viśeṣārthatvāt. viśeṣo nirvikalpo 'rtho viṣayo yasyāḥ sā tathā
tasyā bhāvas tathātvaṃ tasmāc chabdasya padārthatāvacchedakapuraskāreṇaivānumānasya
vyāpakatāvacchedakapuraskāreṇaiva dhījanakatvaniyamena tadgrahaṇāyogyaviśeṣyamātraparatvād ity
arthaḥ.

atra bādarāyaṇakṛtaṃ bhāṣyam śrutam āgamavijñānaṃ tatsāmānyaviṣayam. na hy āgamena śakyo viśeṣo
'bhidhātum. kasmāt na hi viśeṣeṇa kṛtasaṅketaḥ śabda ity ārabhya
samādhiprajñānirgrāhya\var{nirgrāhya \EdLo \lem nirgrāhya \Wthree} eva sa viśeṣo bhavati\var{viśeṣo
  bhavati ?  \lem viśeṣo \EdLo}. bhūtasūkṣmagato vā puruṣagato vā \comment{Vyāsabhāṣya 1.49} iti.

yogabīje

\startsloka
jñānaniṣṭho virakto 'pi dharmajño 'pi jitendriyaḥ
vinā yogena devo 'pi na mokṣaṃ labhate priye \comment{Yogabīja ?}
\stopsloka

kiṃ ca tad eva saktaḥ saha karmaṇa iti liṅgaṃ mano yatra niṣaktam asya
\comment{Bṛhad\-āraṇya\-kopani\-ṣat 4.4.6} iti śruteḥ.  kāraṇaṃ guṇasaṅgo 'sya sadasadyonijanmasu
\comment{Bhagavadgītā 13.21} iti smṛteś ca dehāvasānasamaye yatra rāgādyudbuddho bhavati tām eva
yoniṃ jīvaḥ prāpnotīti yogahīnasya janmāntaraṃ syād eva. maraṇasamaye samudbhūtavaiklavyasyāyoginā
vārayitum aśakyatvāt. tad uktaṃ yogabīje

\startsloka
dehāvasānasamaye citte yad yad vibhāvayet
tat tad eva bhavej jīva ity evaṃ janmakāraṇam
dehānte kiṃ bhavej janma tan na jānanti mānavāḥ
tasmāj jñānaṃ ca vairāgyaṃ japaś ca kevalaṃ śramaḥ
pipīlikā yadā lagnā dehe jñānād vimucyate
asau kiṃ vṛścikair daṣṭo dehānte vā kathaṃ sukhī
\stopsloka

iti. yogināṃ tu yogabalenāntakāle 'py ātmabhāvanayā mokṣa eveti na syāj janmāntaram. tad uktaṃ
bhagavatā

\startsloka
prayāṇakāle manasācalena
bhaktyā yukto yogabalena caiva\comment{Bhagavadgītā 8.10}
\stopsloka

ityādinā.

\startsloka
  śataṃ caikā ca hṛdayasya nāḍyaḥ \comment{Chāndogyopaniṣat 8.6.6}
\stopsloka

ityādiśruteś ca. na ca tattvamasyādivākyasyāparokṣajñānājanakatve\var{jñānājanakatve \EdLo \EdAd
  \lem jñānajanakatve \Wthree} tadvicārasya vaiyarthyam eveti śaṅkyam. vākyavicārajanyajñānasya
yogadvārāparokṣajñānasādhanatvāt.

atra ca yogabīje gaurīśvarasaṃvādo mahān asti tataḥ kiñcil likhyate

\startsloka
devy uvāca 
jñāninas tu mṛtā ye vai teṣāṃ bhavati kīdṛśī
gatiḥ kathaya deveśa kāruṇyāmṛtavāridhe \comment{Yogabīja ???}
\stopsloka

\startsloka
īśvara uvāca
dehānte jñāninā puṇyāt pāpāt tatphalam āpyate
yādṛśaṃ tu bhavet tat tat bhuktvā jñānī punar bhavet
puṇyāt puṇyena labhate siddhena saha saṅgatim
tataḥ siddhasya kṛpayā yogī bhavati nānyathā
tato naśyati saṃsāro nānyathā śivabhāṣitam\comment{Yogabīja ???}
\stopsloka

\startsloka
devy uvāca
jñānād eva hi mokṣaṃ ca vadanti jñāninaḥ sadā
na kathaṃ siddhayogena yogaḥ kiṃ mokṣado bhavet\comment{Yogabīja ???}
\stopsloka

\startsloka
īśvara uvāca
jñānenaiva hi mokṣo hi teṣāṃ vākyaṃ tu nānyathā
sarve vadanti khaḍgena jayo bhavati tarhi kim
vinā yuddhena vīryeṇa kathaṃ jayam avāpnuyāt
tathā yogena rahitaṃ jñānaṃ mokṣāya no bhavet
\stopsloka

nanu janakādīnāṃ yogam antareṇāpy apratibaddhajñānamokṣayoḥ śravaṇāt kathaṃ yogād
evāpratibaddhajñānaṃ mokṣaś ceti cet ucyate teṣāṃ pūrvajanmānuṣṭhitayogasaṃskārāj jñānaprāptir iti
purāṇādau śrūyate. tathā hi

\startsloka
jaigīṣavyo yathā vipro yathā caivāsitādayaḥ
kṣatriyā janakādyās tu tulādhārādayo viśaḥ
samprāptāḥ paramāṃ siddhiṃ pūrvābhyastasvayogataḥ
dharmavyādhādayaḥ sapta śūdrāḥ pailavakādayaḥ
maitreyī sulabhā śārṅgī śāṇḍilī ca tapasvinī
ete cānye ca bahavo nīcayonigatā api
jñānaniṣṭhāṃ parāṃ prāptāḥ pūrvābhyastasvayogataḥ
\stopsloka

iti. kiṃ ca pūrvajanmānuṣṭhitayogābhyāsapuṇyatāratamyena kecid brahmatvaṃ kecid brahmaputratvaṃ
kecid devarṣitvaṃ kecid brahmarṣitvaṃ kecin munitvaṃ kecid bhaktatvaṃ ca prāptāḥ
santi. tatropadeśam antareṇaivātmasākṣātkāravanto babhūvuḥ. tathā hi
hiraṇyagarbhavasiṣṭhanāradasanatkumāravāmadevaśukādayo janmasiddhā ity eva purāṇādiṣu śrūyante. yat
tu brāhmaṇa eva mokṣādhikārīti śrūyate purāṇādau tadayogiparam. tad uktaṃ garuḍapurāṇe

\startsloka
yogābhyāso nṛṇāṃ yeṣāṃ nāsti janmāntarād ṛtaḥ
yogasya prāptaye teṣāṃ śūdravaiśyādikakramaḥ
strītvāc chūdratvam abhyeti tato vaiśyatvam āpnuyāt
tataś ca kṣatriyo vipraḥ kriyāhīnas tato bhavet
anūcānaḥ smṛto yajvā karmanyāsī tataḥ param
tato jñānitvam abhyeti yogī muktiṃ kramāl labhet
\stopsloka

iti. śūdravaiśyādikramād yogī bhūtvā muktiṃ labhetety arthaḥ. itthaṃ ca yoge sarvādhikāraśravaṇād
yogotpannatattvajñānena sarva eva mucyanta iti siddham. yoginas tu bhraṣṭasyāpi na śūdrādikramaḥ

\startsloka
 śucīnāṃ śrīmatāṃ gehe yogabhraṣṭo 'bhijāyate
\stopsloka

atha vā yoginām eva ityādibhagavadvacanād\comment{Bhagavadgītā 6.41--42} ity alam.



%\begin{vsid}{#hp04_00}
\startsloka
{\bf jñātvā suṣumṇāsadbhedaṃ kṛtvā vāyuṃ ca madhyagam
sthitvā sadaiva susthāne brahmarandhre nirodhayet (4.16)}
\stopsloka

prāṇamanasor layaṃ vinā mokṣo na sidhyatītyuktam. tatra prāṇalayena manaso 'pi layaḥ sidhyatīti
tallayarītim āha jñātveti. sadaiva sarvadaiva susthāne śobhane sthāne surājye dhārmike deśe
\comment{HP 1.12} ityādyuktalakṣaṇe sthitvā sthitiṃ kṛtvā vasatiṃ kṛtvety arthaḥ. suṣumṇā
madhyanāḍī tasyāḥ sadbhedaṃ śobhanaṃ bhedanaprakāraṃ jñātvā gurumukhād viditvā vāyuṃ prāṇaṃ
madhyagam madhyanāḍīsañcāriṇaṃ kṛtvā brahmarandhre mūrdhāvakāśe nirodhayet nitarāṃ ruddhaṃ
kuryāt. prāṇasya brahmarandhre nirodho layaḥ. prāṇalaye jāte mano 'pi līyate. tad uktaṃ vāsiṣṭhe

\startsloka
abhyāsena parispande prāṇānāṃ kṣayam āgate  \comment{Laghuyogavāsiṣṭha 9.88}
manaḥ praśamam āyāti nirvāṇam avaśiṣyate 
\stopsloka

iti. prāṇamanasor laye sati bhāvanāviśeṣarūpasamādhisahakṛtenāntaḥ\-kara\-ṇena abādhitātmasākṣātkāro bhavati
tadā jīvann eva muktaḥ puruṣo bhavati.


%\begin{vsid}{#hp04_00}
\startsloka
{\bf sūryācandramasau dhattaḥ kālaṃ rātrindivātmakam
  bhoktrī suṣumnā kālasya guhyam etad udāhṛtam (4.17)}
\stopsloka

prāṇalaye kālajayo bhavatīty āha sūryacandramasau iti. sūryaś ca candramāś ca
sūryācandramasau. devatādvandve ca \comment{Aṣṭādhyāyī 6.3.26} ity ānaṅ. rātriś ca divā ca
rātrindivam. acatura \comment{Aṣṭādhyāyī 5.4.77} ityādinā nipātitaḥ. rātrindivam ātmā svarūpaṃ
yasya sa rātrindivātmakas taṃ rātrindivātmakaṃ kālaṃ samayaṃ dhattaḥ vidhattaḥ kurutaḥ. suṣumnā
sarasvatī. kālasya sūryācandramobhyām kṛtasya rātrindivātmakasya samayasya bhoktrī bhakṣikā
vināśikā. etad guhyam rahasyam udāhṛtam kathitam.

ayaṃ bhāvaḥ sārdhaghaṭikādvayaṃ sūryo vahati sārdhaghaṭikādvayaṃ candro vahati. yadā sūryo vahati
tadā dinam ucyate. yadā candro vahati tadā rātrir ucyate. pañcaghaṭikāmadhye rātrindivātmakaḥ kālo
bhavati. laukikāhorātramadhye yogināṃ dvādaśāhorātrātmakaḥ kālavyavahāro bhavati. tādṛśakālamānena
jīvānām āyurmānam asti. yadā suṣumnāmārgeṇa vāyur brahmarandhre līno bhavati tadā
rātrindivātmakasya kālasyābhāvād uktaṃ bhoktrī suṣumnā kālasyeti. yāvad brahmarandhre vāyur līyate
tāvad yogina āyur vardhate. dīrghakālābhyastasamādhir yogī pūrvam eva maraṇakālaṃ jñātvā
brahmarandhre vāyuṃ nītvā kālaṃ nivārayati svecchayā dehatyāgaṃ ca karotīti.

%\begin{vsid}{#hp04_00}
\startsloka
{\bf dvāsaptatisahasrāṇi nāḍīdvārāṇi pañjare
suṣumṇā śāmbhavī śaktiḥ śeṣās tv eva nirarthakāḥ (4.18)}
\stopsloka

dvāsaptatīti. pañjare pañjaravat śirāsthibhir baddhe śarīre dvābhyām adhikā saptatiḥ dvāsaptatiḥ
dvāsaptatisaṅkhyākāni sahasrāṇi dvāsaptatisahasrāṇi nāḍīnāṃ śirāṇāṃ dvārāṇi vāyupraveśamārgāḥ
santi. suṣumṇā madhyanāḍī śāmbhavī śaktir asti. śaṃ sukhaṃ bhavaty asmād bhaktānām iti śambhur
īśvaras tasyeyaṃ śāmbhavī dhyānena śambhuprāpakatvāt. śambhor āvirbhāvajanakatvād vā śāmbhavī. yad
vā śaṃ sukharūpo bhavati tiṣṭhatīti śambhur ātmā tasyeyaṃ śāmbhavī. cidabhivyaktisthānatvād
dhyānenātmasākṣātkārahetutvāc ca. śeṣā iḍāpiṅgalādayas tu nirarthakā eva nirgato 'rthaḥ prayojanaṃ
yāsāṃ tā nirarthakāḥ pūrvoktaprayojanābhāvāt.

%\begin{vsid}{#hp04_00}
\startsloka
{\bf vāyuḥ paricito yasmād agninā saha kuṇḍalīm
bodhayitvā suṣumṇāyāṃ praviśed anirodhataḥ (4.19)}
\stopsloka

vāyur iti. yasmāt paricito 'bhyasto vāyus tasmād agninā jaṭharāgninā saha kuṇḍalīm śaktiṃ
bodhayitvā anirodhato 'pratibandhāt suṣumṇāyāṃ sarasvatyāṃ praviśet vāyoḥ suṣumnāpraveśārtham
abhyāsaḥ kartavya ity arthaḥ.

%\begin{vsid}{#hp04_00}
\startsloka
  {\bf suṣumṇāvāhini prāṇe siddhyaty eva manonmanī
    anyathā tv itarābhyāsāḥ prayāsāyaiva yoginām (4.20)}
\stopsloka

suṣumneti. prāṇe suṣumṇāvāhini sati manonmanī unmanyavasthā siddhyaty eva. anyathā prāṇe
suṣumnāvāhiny asati tu itarābhyāsāḥ suṣumnetaranāḍyabhyāsāḥ yoginām yogābhyāsināṃ prayāsāyaiva
śramāyaiva bhavantīty arthaḥ.


%\begin{vsid}{#hp04_00}
\startsloka
{\bf pavano badhyate yena manas tenaiva badhyate
manaś ca badhyate yena pavanas tena badhyate (4.21)}
\stopsloka

pavana iti. yena yoginā pavanaḥ prāṇavāyur badhyate baddhaḥ kriyate tenaiva yoginā mano
badhyate. yena mano badhyate tena pavano badhyate. manaḥpavanayor ekatare baddhe ubhayaṃ baddhaṃ
bhavatīty arthaḥ.

%\begin{vsid}{#hp04_00}
\startsloka
{\bf hetudvayaṃ tu cittasya vāsanā ca samīraṇaḥ
tayor vinaṣṭa ekasmin tau dvāv api vinaśyataḥ (4.22)}
\stopsloka

hetudvayaṃ tu cittasyeti. cittasya pravṛttau hetudvayaṃ kāraṇadvayam asti. kiṃ tad ity āha vāsanā
bhāvanākhyaḥ saṃskāraḥ samīraṇaḥ prāṇavāyuś ca. tayor vāsanāsamīraṇayor ekasmin vinaṣṭe sati kṣīṇe
sati tau dvāv api vinaśyataḥ. ayam āśayaḥ vāsanākṣaye samīraṇacitte kṣīṇe bhavataḥ. samīraṇe kṣīṇe
cittavāsane kṣīṇe bhavataḥ. citte kṣīṇe samīraṇavāsane kṣīṇe bhavataḥ. tad uktaṃ vāsiṣṭhe

\startsloka
dve bīje rāma cittasya prāṇaspandanavāsane  \comment{Laghuyogavāsiṣṭha 10.64, 110–113, 15}
ekasmiṃś ca tayor naṣṭe kṣipraṃ dve api naśyataḥ (10.64)
\stopsloka

tatraiva vyatirekeṇoktam

\startsloka
yāvad vilīnaṃ na mano na tāvad vāsanākṣayaḥ 
na kṣīṇā vāsanā yāvac cittaṃ tāvan na śāmyati 
na yāvat tattvavijñānaṃ na tāvac cittasaṃkṣayaḥ
yāvan na cittopaśamo na tāvat tattvavedanam
yāvan na vāsanānāśas tāvat tattvāgamaḥ kutaḥ
yāvan na tattvasamprāptir na tāvad vāsanākṣayaḥ
tattvajñānaṃ manonāśo vāsanākṣaya eva ca
mithaḥ kāraṇatāṃ gatvā duḥsādhyāni sthitāny ataḥ (10.110–113) 

traya ete samaṃ yāvan na svabhyastā muhur muhuḥ
tāvan na tattvasamprāptir bhavaty api samaiḥ śataiḥ (10.15)
\stopsloka

%\begin{vsid}{#hp04_00}
\startsloka
{\bf mano yatra vilīyeta pavanas tatra līyate 
pavano līyate yatra manas tatra vilīyate (4.23)}
\stopsloka

mana iti. yatra yasminn ādhāre mano vilīyeta pavanas tatra tasminn ādhāre līyate. yatra yasminn
ādhāre pavano līyate tatra tasminn ādhāre mano vilīyata ity anvayaḥ.

%\begin{vsid}{#hp04_00}
\startsloka
{\bf dugdhāmbuvat saṃmilitāv ubhau tau
tulyakriyau mānasamārutau hi
yato marut tatra manaḥpravṛttir
yato manas tatra marutpravṛttiḥ (4.24)}
\stopsloka

dugdhāmbuvad iti. dugdhāmbuvat kṣīranīravat saṃmilitau samyak militau tāv ubhau dvāv api
mānasamārutau mānasaṃ ca mārutaś ca mānasamārutau cittaprāṇau. tulyakriyau tulyā samā kriyā
pravṛttir yayos tādṛśau bhavataḥ. tulyakriyatvam evāha yata iti. yataḥ yatra sārvavibhaktikas
tasiḥ. yasmin cakre marud vāyuḥ pravartate tatra tasmin cakre manaḥpravṛttiḥ manasaḥ pravṛttir
bhavati. yato yasmin cakre manaḥ pravartate tatra tasmin cakre marutpravṛttiḥ vāyoḥ pravṛttir
bhavatīty arthaḥ. tad uktaṃ vāsiṣṭhe

\startsloka
avinābhāvinī nityaṃ jantūnāṃ prāṇacetasī
kusumāmodavan miśre tilataile iva sthite (7.21)
kurutaś ca vināśena kāryaṃ mokṣākhyam uttamam (7.22cd)
\stopsloka

iti.


%\begin{vsid}{#hp04_00}
\startsloka
{\bf tatraikanāśād aparasya nāśa
ekapravṛtter aparapravṛttiḥ
adhvastayoś cendriyavargavṛttiḥ
pradhvastayor mokṣapadasya siddhiḥ (4.25)}
\stopsloka

tatreti. tatra tayor mānasamārutayor madhye ekasya mānasasya mārutasya vā nāśāl layād
aparasyānyasya mārutasya mānasasya vā nāśo layo bhavati. ekapravṛtter ekasya mānasasya mārutasya vā
pravṛtter vyāpārād aparapravṛttiḥ aparasya mārutasya mānasasya pravṛttir vyāpāro
bhavati. adhvastayor alīnayor mānasamārutayoḥ sator indriyavargavṛttir indriyasamudāyasya
svasvaviṣaye pravṛttir bhavati. pradhvastayoḥ pralīnayos tayoḥ sator mokṣapadasya mokṣākhyapadasya
siddhir niṣpattir bhavati. tayor laye puruṣasya svarūpe 'vasthānād ity arthaḥ.

\startsloka
tatrāpi sādhyaḥ pavanasya nāśaḥ
ṣaḍaṅgayogādiniṣevaṇena
manovināśas tu guroḥ prasādān
nimeṣamātreṇa susādhya eva
\stopsloka

yogabīje mūlaślokasyāyam uttaraḥ ślokaḥ.\comment{Yogabīja ??}


%\begin{vsid}{#hp04_00}
\startsloka
{\bf rasasya manasaś caiva cañcalatvaṃ svabhāvataḥ
raso baddho mano baddhaṃ kiṃ na siddhyati bhūtale (4.26)}
\stopsloka

rasasyeti. rasasya pāradasya manaso mānasasya svabhāvataḥ svabhāvāc cañcalatvaṃ cāñcalyam
asti. rasaḥ pārado baddhaś cen manaś cittaṃ baddhaṃ bhavati. tato bhūtale pṛthivītale kiṃ na
siddhyati. sarvaṃ siddhyatīty arthaḥ.



%\begin{vsid}{#hp04_00}
\startsloka
{\bf mūrcchito harate vyādhīn mṛto jīvayati svayam
baddhaḥ khecaratāṃ dhatte raso vāyuś ca pārvati (4.27)}
\stopsloka

tad evāha mūrcchita iti. oṣadhiviśeṣayogena gatacāpalo raso mūrcchitaḥ. kumbhakānte recakanivṛttau
vāyur mūrcchita ity ucyate. he pārvatīti sambodhyeśvaravākyam. mūrcchito rasaḥ pāradaḥ vāyuḥ prāṇaś
ca vyādhīn rogān harate nāśayati. bhasmībhūto rasaḥ brahmarandhre līno vāyuś ca mṛtaḥ svayam
ātmanā svasāmarthyenety arthaḥ jīvayati dīrghakālaṃ jīvanaṃ karoti. kriyāviśeṣeṇa guṭikākārakṛto
raso baddhaḥ bhrūmadhyādau dhāraṇāviśeṣeṇa dhṛto vāyuś ca baddhaḥ khecaratām ākāśagatiṃ dhatte
vidhatte karotīty arthaḥ. tad uktaṃ gorakṣaśatake

\startsloka
yad bhinnāñjanapuñjasannibham idaṃ vṛttaṃ bhruvor antare
tattvaṃ vāyumayaṃ yakārasahitaṃ tatreśvaro devatā
prāṇaṃ tatra vilāpya\var{vilāpya \lem vilīya (preferred by \EdLo)} pañcaghaṭikaṃ cittānvitaṃ dhārayed
eṣā khe gamanaṃ karoti yamināṃ syād vāyavī dhāraṇā \comment{go.śa. 72, yo.ta. 2.57} iti.
\stopsloka

%\begin{vsid}{#hp04_00}
\startsloka
{\bf manaḥsthairye sthiro vāyus tato binduḥ sthiro bhavet
bindusthairyāt sadā sattvaṃ piṇḍasthairyaṃ prajāyate (4.28)}
\stopsloka

manaḥsthairya iti. manasaḥ sthairye sati vāyuḥ prāṇaḥ sthiro bhavet. tato vāyusthairyād bindur
vīryaṃ sthiro bhavet. bindoḥ sthairyāt sadā sarvadā sattvaṃ balaṃ piṇḍasthairyaṃ dehasthairyaṃ
prajāyate.

%\begin{vsid}{#hp04_00}
\startsloka
{\bf indriyāṇāṃ mano nātho manonāthas tu mārutaḥ
mārutasya layo nāthaḥ sa layo nādam āśritaḥ (4.29)}
\stopsloka

indriyāṇām iti. indriyāṇāṃ śrotrādīnāṃ mano 'ntaḥkaraṇaṃ nāthaḥ pravartakaḥ. manonātho manaso nātho
mārutaḥ prāṇaḥ. mārutasya prāṇasya layo manovilayo nāthaḥ. sa layo manolayaḥ nādam āśrito nāde mano
līyata ity arthaḥ.

%\begin{vsid}{#hp04_00}
\startsloka
{\bf so 'yam evāstu mokṣākhyo māstu vāpi matāntare
manaḥprāṇalaye kaścid ānandaḥ sampravartate (4.30)}
\stopsloka

so 'yam iti. so 'yam eva cittavilaya eva mokṣākhyo mokṣapadavācyaḥ. matāntare 'nyamate māstu
vā. cittalayasya suṣuptāv api sattvān manaḥprāṇayor laye sati kaścid anirvācya ānandaḥ
sampravartate samyak pravṛtto bhavati. anirvācyānandāvirbhāve jīvanmuktisukhaṃ bhavaty eveti.

%\begin{vsid}{#hp04_00}
\startsloka
{\bf pranaṣṭaśvāsaniśvāsaḥ pradhvastaviṣayagrahaḥ
niśceṣṭo nirvikāraś ca layo jayati yoginām (4.31)}
\stopsloka

praṇaṣṭeti. śvāsaś ca niśvāsaś ca śvāsaniśvāsau pranaṣṭau pralīnau śvāsaniśvāsau yasmin sa tathā
bāhyavāyor antaḥpraveśanaṃ śvāsaḥ antaḥsthitasya vāyor bahirniḥsaraṇaṃ niśvāsaḥ. pradhvastaḥ
prakarṣeṇa dhvasto naṣṭo viṣayāṇāṃ śabdādīnāṃ graho grahaṇaṃ yasmin nirgatā ceṣṭā kāyakriyā
yasmin nirgato vikāro 'ntaḥkaraṇakriyā yasmin etādṛśo yoginām layo 'ntaḥkaraṇavṛtter
dhyeyākārāpattir jayati sarvotkarṣeṇa vartate.

%\begin{vsid}{#hp04_00}
\startsloka
{\bf ucchinnasarvasaṅkalpo niḥśeṣāśeṣaceṣṭitaḥ
svāvagamyo layaḥ ko'pi jāyate vāgagocaraḥ (4.32)}
\stopsloka

ucchinneti. ucchinnā naṣṭāḥ sarve saṅkalpā manaḥpariṇāmā yasmin sa tathā nirgataḥ śeṣo yebhyas tāni
niḥśeṣāṇy aśeṣāṇi ceṣṭitāni yasmin sa tathā svenaivāvagantuṃ boddhuṃ śakyaḥ svāvagamyaḥ vācām
agocaro 'viṣayaḥ ko 'pi vilakṣaṇo layaḥ jāyate yogināṃ prādurbhavati.

%\begin{vsid}{#hp04_00}
\startsloka
{\bf yatra dṛṣṭir layas tatra bhūtendriyasanātanī
sā śaktir jīvabhūtānāṃ dve alakṣye layaṃ gate (4.33)}
\stopsloka

yatra dṛṣṭir iti. yatra yasmin viṣaye brahmaṇi dṛṣṭir antaḥkaraṇavṛttis tatraiva layo
bhavati. bhūtāni pṛthivyādīni indriyāṇi śrotrādīni sanātanāni śāśvatāni yasyāṃ sā
satkāryavāde 'vidyāyāṃ kāryajātasya sattvāt. jīvabhūtānāṃ prāṇināṃ śaktir avidyā ime dve alakṣye
brahmaṇi layaṃ gate yoginām iti śeṣaḥ.

\startsloka
{\bf layo laya iti prāhuḥ kīdṛśaṃ layalakṣaṇam
apunarvāsanotthānāl layo viṣayavismṛtiḥ (4.34)}
\stopsloka

laya iti. layo laya iti prāhur vadanti bahavaḥ. layasya lakṣaṇaṃ layasvarūpaṃ kīdṛśam iti
praśnapūrvakaṃ layasvarūpam āha apunar iti. apunarvāsanotthānāt punarvāsanotthānābhāvāt
viṣayavismṛtiḥ viṣayāṇāṃ śabdādīnāṃ dhyeyākārasya viṣayasya vā vismṛtir layo layaśabdārtha ity
arthaḥ. 

%\begin{vsid}{#hp04_00}
\startsloka
{\bf vedaśāstrapurāṇāni sāmānyagaṇikā iva
ekaiva śāmbhavī mudrā guptā kulavadhūr iva (4.35)}
\stopsloka

vedeti. vedāś catvāraḥ śāstrāṇi ṣaṭ purāṇāny aṣṭādaśa sāmānyagaṇikā iva veśyā
iva bahupuruṣagamyatvāt. ekā śāmbhavī mudraiva kulavadhūr iva kulastrīva guptā
puruṣaviśeṣagamyatvāt.


%\begin{vsid}{#hp04_00}
\startsloka
{\bf atha śāmbhavī
antarlakṣyaṃ bahir dṛṣṭir nimeṣonmeṣavarjitā
eṣā sā śāmbhavī mudrā vedaśāstreṣu gopitā (4.36)}
\stopsloka

cittalayāya prāṇalayasādhanībhūtāṃ mudrāṃ vivakṣus tatra śāmbhavīm mudrām āha antarlakṣyam
iti. antar ādhārādibrahmarandhrānteṣu cakreṣu madhye svābhimate cakre lakṣyam
antaḥkaraṇavṛttiḥ. bahir dehād bahiḥ pradeśe dṛṣṭiḥ cakṣuḥsambandhaḥ. kīdṛśī dṛṣṭiḥ
nimeṣonmeṣavarjitā nimeṣaḥ pakṣasaṃyogaḥ unmeṣaḥ pakṣmasaṃyogaviśleṣaḥ tābhyāṃ varjitā rahitā
cittasya dhyeyākārāveśe nimeṣonmeṣavarjitā dṛṣṭir bhavati. soktaiṣā mudrā śāmbhavī śabhor iyaṃ
śāmbhavī śivapriyā śivāvirbhāvajanikā vā bhavati. kīdṛśī vedaśāstreṣu gopitā vedeṣu ṛgādiṣu
śāstreṣu sāṅkhyapātañjalādiṣu gopitā rakṣitā.

%\begin{vsid}{#hp04_00}
\startsloka
{\bf antar lakṣyavilīnacittapavano yogī yadā vartate
dṛṣṭyā niścalatārayā bahir adhaḥ paśyann apaśyann api
mudreyaṃ khalu śāmbhavī bhavati sā labdhā prasādād guroḥ
śūnyāśūnyavilakṣaṇaṃ sphurati tat tattvaṃ padaṃ śāmbhavam (4.37)}
\stopsloka

śāmbhavīṃ mudrām abhinīya darśayati antarlakṣyeti. yadā yasyām avasthāyām antaḥ anāhatapadmādau yal
lakṣyaṃ saguṇeśvaramūrtyādikaṃ tattvamasyādivākyalakṣyaṃ jīveśvarābhinnam ahaṃ brahmāsmīti
vākyārthabhūtaṃ brahma vā tasmin vilīnau viśeṣeṇa līnau cittapavanau manomārutau yasya sa tathā
yogī vartate. niścalatārayā niścalā sthirā tārā kanīnikā yasyāṃ tādṛśyā dṛṣṭyā bahir dehād
bahiḥpradeśe paśyann api cakṣuḥsambandhaṃ kurvann api apaśyan bāhyaviṣayagrahaṇam akurvan vartate
āste. khalv iti vākyālaṅkāre. iyam uktā śāmbhavī mudrā śāmbhavīnāmikā mudrayati kleśān iti mudrā
[bhavati sā śāmbhavī mudrā??? \lem \comment{addition in brackets occurs in some manuscripts as in \Wthree}
guror deśikasya prasādāt prītipūrvakād anugrahāl labdhā prāptā tad idam iti vaktum aśakyaṃ
śāmbhavam śāmbhavyā idaṃ śāmbhavam śāmbhavīmudrāyāṃ bhāsamānaṃ padaṃ padyate gamyate yogibhir iti
padam ātmasvarūpaṃ śūnyāśūnyavilakṣaṇaṃ dhyeyākāravṛtteḥ sadbhāvāc chūnyavilakṣaṇaṃ tasyā api
bhānābhāvād aśūnyavilakṣaṇaṃ tattvaṃ vāstavikaṃ vastu sphurati pratīyate. tathā coktam\comment{source?}

\startsloka
antarlakṣyam ananyadhīr avirataṃ paśyan mudā saṃyamī
dṛṣṭyunmeṣanimeṣavarjitam iyaṃ mudrā bhavec chāmbhavī
gupteyaṃ giriśena tantraviduṣā tantreṣu tattvārthinām
eṣā syād yamināṃ manolayakarī muktipradā durlabhā

ūrdhvadṛṣṭir adhodṛṣṭir ūrdhvavedho hy adhaḥśirā
rādhāyantravidhānena jīvanmukto bhavet kṣitau
\stopsloka


%\begin{vsid}{#hp04_00}
\startsloka
{\bf śrīśāmbhavyāś ca khecaryā avasthādhāmabhedataḥ
bhavec cittalayānandaḥ śūnye citsukharūpiṇi (4.38)}
\stopsloka

śrīśāmbhavyā iti. śrīśāmbhavyāḥ śrīmatyāḥ śāmbhavīmudrāyāḥ khecarīmudrāyāś cāvasthādhāmabhedataḥ
avasthā avasthitiḥ dhāma sthānaṃ tayor bhedāc chāmbhavyāṃ bahirdṛṣṭyā avasthitiḥ khecaryāṃ
bhrūmadhyadṛṣṭyāvasthitiḥ. śāmbhavyāṃ hṛdayaṃ bhāvanādeśaḥ khecaryāṃ bhrūmadhya eva deśaḥ tayor
bhedābhyāṃ śūnye deśakālavastuparicchedaśūnye sajātīyavijātīyasvagatabhedaśūnye vā citsukharūpiṇi
cidānandasvarūpiṇy ātmani cittalayānando bhavet syāt. śrīśāmbhavīkhecaryor
avasthādhāmarūpasādhanāṃśe bhedaḥ na tu cittalayānandarūpaphalāṃśa iti bhāvaḥ.

%\begin{vsid}{#hp04_00}
\startsloka
{\bf tāre jyotiṣi saṃyojya kiñcid unnamayed bhruvau
pūrvayogaṃ mano yuñjann unmanīkārakaḥ kṣaṇāt (4.39)}
\stopsloka

unmanīmudrām āha tāra iti. tāre netrayoḥ kanīnike jyotiṣi tārayor nāsāgre yojanāt prakāśamāne
tejasi saṃyojya saṃyukte kṛtvā bhruvau kiñcit svalpam unnamayed ūrdhvaṃ namayet. pūrvaḥ pūrvoktaḥ
antarlakṣyaṃ bahirdṛṣṭiḥ\comment{HP 4.36} ity ākārako yogo yuktir yasmin tat tādṛśaṃ mano
'ntaḥkaraṇaṃ yuñjan yuktaṃ kurvan yogī kṣaṇān muhūrtād unmanīkāraka unmanyavasthākārako bhavati.

%  read pūrvayogamano acc. to comm.?  -kārakaḥ does not work well. 

%\begin{vsid}{#hp04_00}
\startsloka
{\bf kecid āgamajālena kecin nigamasaṅkulaiḥ
kecit tarkeṇa muhyanti naiva jānanti tārakam (4.40)}
\stopsloka

unmanīm antarā anyastareṇopāyo nāstīty āha kecid iti. kecic chāstratantrādividaḥ āgacchanti buddhim
ārohanti arthā ebhya ity āgamāḥ śāstratantrādayaḥ teṣāṃ jālair jālavad bandhanasādhanais taduktaiḥ
phalair muhyanti mohaṃ prāpnuvanti. tatrāsaktā badhyanta iti bhāvaḥ. kecid vaidikā nigamasaṅkulair
nigamānāṃ nigamoktānāṃ saṅkulaiḥ phalabāhulyair muhyanti. kecid vaiśeṣikādayas\var{vaināśikādayas
  \Wthree \Tue \EdLo \lem vaiśeṣikādayas \EdAd} tarkeṇa svakalpitayuktiviśeṣeṇa muhyanti. tārayatīti
tārakas taṃ tārakaṃ taraṇopāyaṃ naiva jānanti. uktonmany eva taraṇopāyas taṃ na jānantīty arthaḥ.


%\begin{vsid}{#hp04_00}
\startsloka
{\bf ardhonmīlitalocanaḥ sthiramanā nāsāgradattekṣaṇaś
candrārkāv api līnatām upanayan nispandabhāvena yaḥ
jyotīrūpam aśeṣabījam akhilaṃ dedīpyamānaṃ paraṃ
tattvaṃ tat padam eti vastu paramaṃ vācyaṃ kim atrādhikam (4.41)}
\stopsloka

ardhonmīliteti. ardham unmīlite ardhonmīlite ardhonmīlite locane yena saḥ
ardhonmīlitalocanaḥ. ardhodghāṭitalocana ity arthaḥ. sthiraṃ niścalaṃ mano yasya sa sthiramanāḥ
nāsāyā nāsikāyā agre 'grabhāge nāsikāyāṃ dvādaśāṅgulaparyante vā datte prahite īkṣaṇe netre yena sa
nāsāgradattekṣaṇaḥ. tad āha vasiṣṭhaḥ\comment{Laghuyogavāsiṣṭha 9.81}

%  Jyotsnā gives meaning with the majority reading:  ardhodghāṭitalocana


\startsloka
dvādaśāṅgulaparyante nāsāgre vimale 'mbare
saṃviddṛśoḥ praśāmyantyoḥ prāṇaspando nirudhyate  
\stopsloka

iti. nispandasya niścalasya bhāvo nispandabhāvaḥ kāyendriyamanasāṃ niścalatvaṃ tena candrārkau
candrasūryāv api līnatāṃ līnasya bhāvo līnatā layas tām upanayan prāpayan kāyendriyamanasāṃ
niścalatvena prāṇasañcāram api stambhayann ity arthaḥ. tad uktaṃ prāk mano yatra vilīyate
\comment{HP 4.23} ityādipūrvoktaviśeṣaṇasampanno yogī jyotīrūpaṃ jyotir ivākhilaprakāśakaṃ rūpaṃ
yasya sa tathā tam aśeṣabījam ākāśādyutpattidvārā sarvakāraṇam akhilaṃ pūrṇaṃ dedīpyamānam
atiśayena dīpyata iti dedīpyamānaḥ taṃ tathā svaprakāśaṃ paraṃ kāyendriyamanasāṃ sākṣiṇaṃ tattvam
anāropitaṃ vāstavikam ity arthaḥ. tad idam iti vaktum aśakyaṃ padyate gamyate yogibhir iti padaṃ
paramaṃ sarvotkṛṣṭaṃ vastu ātmasvarūpam eti prāpnoti. unmanyavasthāyāṃ svasvarūpāvasthito yogī
bhavatīty arthaḥ. atrādhikaṃ kiṃ vācyam. aparaṃ vastu na prāpnotīty atra kiṃ vaktavyam ity arthaḥ.

%\begin{vsid}{#hp04_00}
\startsloka
{\bf divā na pūjayel liṅgaṃ rātrau caiva na pūjayet
sarvadā pūjayel liṅgaṃ divārātrinirodhataḥ (4.42)}
\stopsloka

unmanībhāvanāyāḥ kālaniyamābhāvam āha divā neti. divā sūryasañcāre liṅgaṃ sarvakāraṇam ātmānam
etasmād ātmana ākāśaḥ sambhūtaḥ \comment{Taittirīyopaniṣat 2.1} ityādiśruteḥ. na pūjayet na
bhāvayet. dhyānam evātmapūjanam. tad uktaṃ vāsiṣṭhe\comment{Laghuyogavāsiṣṭha 2.90}

\startsloka
dhyānopahāra evātmā dhyānam asya mahārcanam
vinā tenetareṇāyam ātmā labhyate eva no iti
\stopsloka

rātrau candrasañcāre ca naiva pūjayen naiva bhāvayet. candrasūryasañcāre cittasthairyābhāvāt. cale
vāte calaṃ cittam \comment{HP 2.2} ity uktatvāt. divārātrinirodhataḥ sūryacandrau
nirudhya. lyablope pañcamī. tasyās tasil. sarvadā sarvasmin kāle liṅgam ātmānaṃ pūjayed
bhāvayet. sūryacandrayor nirodhe kṛte suṣumnāntargate prāṇe manaḥsthairyāt. tad uktaṃ

\startsloka
suṣumnāntargate vāyau manaḥsthairyaṃ prajāyate \comment{quotation untraced.}
\stopsloka

iti.

%\begin{vsid}{#hp04_00}
\startsloka
{\bf atha khecarī
savyadakṣiṇanāḍistho madhye carati mārutaḥ
tiṣṭhate khecarīmudrā tasmin sthāne na saṃśayaḥ (4.43)}
\stopsloka

khecarīm āha savyeti. savyadakṣiṇanāḍistho vāmataditaranāḍistho māruto vāyur yatra madhye carati
yasmin madhyapradeśe gacchati tasmin sthāne tasmin pradeśe khecarī mudrā tiṣṭhate sthirā
bhavati. prakāśanastheyākhyayoś ca \comment{Aṣṭādhyāyī 1.3.32} ity ātmanepadam. na saṃśayaḥ ukte
'rthe sandeho nāstīty arthaḥ.

%\begin{vsid}{#hp04_00}
\startsloka
{\bf iḍāpiṅgalayor madhye śūnyaṃ caivānilaṃ graset
tiṣṭhate khecarīmudrā tatra satyaṃ punaḥ punaḥ (4.44)}
\stopsloka

iḍāpiṅgalayor iti. iḍāpiṅgalayoḥ savyadakṣiṇanāḍyor madhye yac chūnyaṃ khaṃ kartṛ anilaṃ prāṇavāyuṃ
yatra graset. śūnye prāṇasya sthirībhāva eva grāsaḥ. tatra tasmin śūnye khecarī mudrā
tiṣṭhate. punaḥ punaḥ satyam iti yojanā.

%\begin{vsid}{#hp04_00}
\startsloka
{\bf sūryācandramasor madhye nirālambāntare punaḥ
saṃsthitā vyomacakre yā sā mudrā nāma khecarī (4.45)}
\stopsloka

sūryācandramasor iti. sūryācandramasor iḍāpiṅgalayor madhye nirālambaṃ yadantaram avakāśas
tatra. punah pādapūraṇe. vyomnāṃ khānāṃ cakre samudāye. bhrūmadhye sarvakhānāṃ samanvayāt. tad
uktaṃ pañcasrotaḥsamanvitam \comment{3.53} iti. yā saṃsthitā sā mudrā khecarī nāma.

%\begin{vsid}{#hp04_00}
\startsloka
{\bf somād yatroditā dhārā sākṣāt sā śivavallabhā
pūrayed atulāṃ divyāṃ suṣumṇāṃ paścime mukhe (4.46)}
\stopsloka

somād iti. somāc candrād yatra yasyāṃ khecaryāṃ dhārā amṛtadhārā uditodgatā sā khecarī sākṣāc
chivavallabhā śivasya priyeti pūrvenānvayaḥ. atulāṃ nirupamāṃ divyāṃ sarvanāḍyuttamāṃ suṣumṇāṃ
paścime mukhe pūrayet jihvayā iti śeṣaḥ. 

%\begin{vsid}{#hp04_00}
\startsloka
{\bf purastāc caiva pūryeta niścitā khecarī bhavet
abhyastā khecarī mudrāpy unmanī samprajāyate (4.47)}
\stopsloka

purastāc caiveti. purastāc caiva pūrvata eva pūryeta suṣumnā prāṇeneti śeṣaḥ. yadi tarhi
niścitāsandigdhā khecarī khecaryākhyā mudrā bhaved iti. yadi tu purastāt prāṇena na pūryeta
jihvāmātreṇa paścimataḥ pūryeta tarhi mūḍhāvasthājanikā na niścitā khecarī syād iti bhāvaḥ. khecarī
mudrāpy abhyastā satī unmanī samprajāyate cittasya dhyeyākārāveśāt turyāvasthā bhavatīty arthaḥ.

%\begin{vsid}{#hp04_00}
\startsloka
{\bf bhruvor madhye śivasthānaṃ manas tatra vilīyate
jñātavyaṃ tatpadaṃ turyaṃ tatra kālo na vidyate (4.48)}
\stopsloka

bhruvor iti. bhruvor madhye bhruvor antarāle śivasthānaṃ śivasya īśvarasya sthānaṃ śivasya
sukharūpasya ātmano vā sthānam astīti śeṣaḥ. tatra tasmin śive mano līyate. śivākāravṛttipravāhavad
bhavati. tac cittalayarūpaṃ turyaṃ padaṃ jāgratsvapnasuṣuptibhyaś caturthāvasthākhyaṃ
jñātavyam. tatra tasmin pade kālo mṛtyur na vidyate. yad vā sūryacandrayor nirodhād
āyuḥkṣayakārakaḥ kālaḥ samayo na vidyata ity arthaḥ. tad uktaṃ bhoktrī suṣumnā kālasya
\comment{4.17} iti.

%\begin{vsid}{#hp04_00}
\startsloka
{\bf abhyaset khecarīṃ tāvad yāvat syād yoganidritaḥ
samprāptayoganidrasya kālo nāsti kadācana (4.49)}
\stopsloka

abhyased iti. tāvat khecarīṃ mudrām abhyaset yāvad yoganidritaḥ. yogaḥ sarvavṛttinirodhaḥ saiva
nidrā yoganidrā yoganidrā asya\var{asya \lem ālasya \Wthree} sañjātā iti yoganidritaḥ tādṛśaḥ
syāt. samprāptā yoganidrā yena sa samprāptayoganidraḥ tasya kadācana kasmiṃścid api samaye kālo
mṛtyur nāsti.

%\begin{vsid}{#hp04_00}
\startsloka
{\bf nirālambaṃ manaḥ kṛtvā na kiñcid api cintayet
sa bāhyābhyantaraṃ vyomni ghaṭavat tiṣṭhati dhruvam (4.50)}
\stopsloka

nirālambam iti. yo nirālambam ālambanaśūnyaṃ manaḥ kṛtvā kiñcid api na cintayet khecarīmudrāyāṃ
jāyamānāyāṃ brahmākārām api vṛttiṃ paravairāgyeṇa parityajed ity arthaḥ. sa yogī bāhyābhyantare
bāhye bahirbhave ābhyantare 'bhyantarbhave ca vyomny ākāśe ghaṭavat ghaṭa iva tiṣṭhati. dhruvam
niścitam ity arthaḥ. yathākāśe ghaṭo bahirantaś cākāśapūrṇo bhavati tathā khecaryām
ālambanaparityāgena yogī brahmaṇā pūrṇas tiṣthatīty arthaḥ.

%\begin{vsid}{#hp04_00}
\startsloka
{\bf bāhyavāyur yathā līnas tathā madhyo na saṃśayaḥ
svasthāne sthiratām eti pavano manasā saha (4.51)}
\stopsloka

bāhyeti. bāhyo dehād bahirbhavo vāyur yathā līno bhavati khecaryām tasyāntaḥpravṛttyabhāvāt. tathā
madhyo dehamadhyavartī vāyur līno bhavati. tasya bahiḥpravṛttyabhāvāt. na saṃśayaḥ asminn arthe
sandeho nāstīty arthaḥ. sthīyate sthirībhūyate 'sminn iti sthānaṃ svasya prāṇasya sthānaṃ
sthairyādhiṣṭhānaṃ brahmarandhraṃ tatra manasā cittena saha pavanaḥ prāṇaḥ sthiratām niścalatām eti
prāpnoti.

%\begin{vsid}{#hp04_00}
\startsloka
{\bf evam abhyasyamānasya vāyumārge divāniśam
abhyāsāj jīryate vāyur manas tatraiva līyate (4.52)}
\stopsloka

evam uktaprakāreṇa vāyumārge prāṇamārge suṣumnāyām ity arthaḥ. divāniśam rātriṃdivam abhyasamanasya
abhyāsaṃ kurvato yogino 'bhyāsād yatra yasminn ādhāre vāyuḥ prāṇo jīryate kṣīyate līyate ity arthaḥ.
tatraiva vāyor layādhiṣṭhāne manaś cittaṃ līyate jīryata ity arthaḥ.

%\begin{vsid}{#hp04_00}
\startsloka
{\bf amṛtaiḥ plāvayed deham āpādatalamastakam
siddhyaty eva mahākāyo mahābalaparākramaḥ (4.53)}
\stopsloka

amṛtair iti. amṛtaiḥ suṣiranirgataiḥ pādatalaṃ ca mastakaṃ ca pādatalamastakam. dvandvaś ca
prāṇitūryasenāṅgānāṃ \comment{Aṣṭādhyāyī 2.4.2} ity ekavadbhāvaḥ. pādatalamastakam abhivyāpyety
āpādatalamastakam deham āplāvayed āplāvitam kuryāt. mahān utkṛṣṭaḥ kāyo yasya sa mahākāyaḥ mahāntau
balaparākramau yasyety etādṛśo yogī siddhyaty eva amṛtāplāvanena siddho bhavaty eva.

%\begin{vsid}{#hp04_00}
\startsloka
{\bf śaktimadhye manaḥ kṛtvā śaktiṃ mānasamadhyagām
manasā mana ālokya dhārayet paramaṃ padam (4.54)}
\stopsloka

śaktir iti. śaktiḥ kuṇḍalinī tasyā madhye śaktimadhye manaḥ kṛtvā tasyāṃ mano dhṛtvā tadākāraṃ
manaḥ kṛtvety arthaḥ. śaktiṃ mānasamadhyagām kṛtvā. śaktidhyānāveśāc chaktimanasor atyantaikyaṃ
kṛtvā tena kuṇḍalīṃ bodhayitveti yāvat.

\startsloka
  prabuddhā vahniyogena manasā marutā saha \comment{Gorakṣaśataka 31, yo.ta. 1.49}
\stopsloka

iti gorakṣokteḥ. manasāntaḥkaraṇena mana ālokya buddhiṃ manaso'valokanena sthirīkṛtyety
arthaḥ. paramaṃ padam sarvotkṛṣṭaṃ svarūpaṃ dhārayed dhāraṇāviṣayaṃ kuryād ity arthaḥ.


%\begin{vsid}{#hp04_00}
\startsloka
{\bf khamadhye kuru cātmānam ātmamadhye ca khaṃ kuru
sarvaṃ ca khamayaṃ kṛtvā na kiñcid api cintayet (4.55)}
\stopsloka

khamadhya iti. kham iva pūrṇaṃ brahma khaṃ tanmadhye ātmānam svasvarūpaṃ kuru. brahmāham iti
bhāvayety arthaḥ. ātmamadhye svasvarūpe ca khaṃ pūrṇaṃ brahma kuru. ahaṃ brahmeti ca bhāvayety
arthaḥ. sarvaṃ ca khamayaṃ kṛtvā brahmamayaṃ vibhāvya kim api na cintayet. ahaṃ brahmeti dhyānam
api parityajety arthaḥ.

%\begin{vsid}{#hp04_00}
\startsloka
{\bf antaḥ śūnyo bahiḥ śūnyaḥ śūnyaḥ kumbha ivāmbare
antaḥ pūrṇo bahiḥ pūrṇaḥ pūrṇaḥ kumbha ivārṇave (4.56)}
\stopsloka

evaṃ samāhitasya svarūpe sthitim āha antaḥśūnya iti. antaḥ antaḥkaraṇe śūnyaḥ. brahmātiriktavṛtter
abhāvād dvitīyaśūnyaḥ. bahir antaḥkaraṇād bahir api śūnyaḥ. dvitīyādarśanāt. ambare ākāśe kumbho
ghaṭo yathāntarbahiḥśūnyas tadvad antaḥ antaḥkaraṇe hṛdākāśe vā pūrṇo vyāptatvād brahmākāravṛtteḥ
sadbhāvāt brahmasattvād vā. bahiḥ pūrṇaḥ antaḥkaraṇād bahir hṛdākāśād bahir vā pūrṇaḥ. sattayā
brahmātiriktavṛtter abhāvād brahmapūrṇatvād vā. arṇave samudre kumbho ghaṭo yathā sarvato jalapūrṇo
bhavaty evaṃ samādhiniṣṭho yogī brahmapūrṇo bhavatīty arthaḥ.


%\begin{vsid}{#hp04_00}
\startsloka
{\bf bāhyacintā na kartavyā tathaivāntaracintanam
sarvacintāṃ parityajya na kiñcid api cintayet (4.57)}
\stopsloka

bāhyacinteti. samāhitena yoginety adhyāhāraḥ. bāhyacintā bāhyaviṣayacintā na kartavyā. tathaiva
bāhyacintākaraṇavad āntaracintanam āntarāṇāṃ manasā parikalpitānām āśāmodakasaudhavāṭikādīnāṃ
cintanaṃ na kartavyam iti liṅgavipariṇāmenānvayaḥ. sarvacintāṃ bāhyābhyantaracintanaṃ parityajya
kiñcid api na cintayet paravairāgyeṇātmākāravṛttim api parityajet. tattyāge svarūpāvasthitirūpā
jīvanmuktir bhavatīti bhāvaḥ.



%\begin{vsid}{#hp04_00}
\startsloka
{\bf saṅkalpamātrakalanaiva jagat samagraṃ
saṅkalpamātrakalanaiva manovilāsaḥ
saṅkalpamātramatim utsṛja nirvikalpam
āśritya niścayam avāpnuhi rāma śāntim (4.58)}    % MU/LYV
\stopsloka

bāhyābhyantaracintāparityāge śāntiś ca bhavatīty atra vasiṣṭhavākyaṃ\comment{The verse is an
  adaptation of Laghuyogavāsiṣṭha 7.27} pramāṇayati saṅkalpeti. saṅkalpo mānasiko vyāpāraḥ sa eva
saṅkalpamātraṃ tasya kalanaiva racanaivedaṃ dṛśyamānaṃ samagraṃ jagat
bāhyaprapañco\var{bāhyaprapañco \Wthree \EdAd \lem bāhyaḥ samagraḥ prapañco Ed Lo \Tue} manomātrakalpita
ity arthaḥ. manaso mānasasya vilāso nānāviṣayākārakalpanā āśāmodakasaudhavāṭikādikalpanārūpo
vilāsaḥ saṅkalpamātrakalanaiva. mānasaḥ prapañco 'pi saṅkalpamātraracanaivety arthaḥ. saṅkalpamātre
bāhyābhyantaraprapañce yā matiḥ satyatvabuddhis tām utsṛja tyaja. tarhi kiṃ kartavyam ity ata āha
nirvikalpam iti. viśiṣṭā kalpanā\var{viśiṣṭā kalpanā \Tue \Wthree \lem viśiṣṭakalpanā \EdMu \EdLo viśiṣṭaḥ
  kalpo \EdAd} vikalpaḥ. ātmani
kartṛtvabhoktṛtvasukhitvaduḥkhitvasajātīyavijātīyasvagatabhedadeśakālavastuparicchedakalpanārūpaḥ
tasmān niṣkrānto nirvikalpaḥ ātmā tam āśritya dhāraṇādiviṣayaṃ kṛtvā he rāma niścayam asandigdhaṃ
śāntim paramoparatim avāpnuhi. tataḥ sukham api prāpsyasīti bhāvaḥ. tad uktaṃ bhagavatā
vyatirekeṇa

\startsloka
  na cābhāvayataḥ śāntir aśāntasya kutaḥ sukham \comment{Bhagavadgītā 2.66}
\stopsloka

iti.

%\begin{vsid}{#hp04_00}
\startsloka
{\bf karpūram anale yadvat saindhavaṃ salile yathā
tathā sandhīyamānaṃ ca manas tattve vilīyate (4.59)}
\stopsloka

karpūram iti. yad yad yathā anale 'gnau sandhīyamānaṃ saṃyojyamānaṃ karpūram vilīyate viśeṣeṇa
līyate līnaṃ bhavati agnyākāraṃ bhavati. yathā salile jale sandhīyamānaṃ saindhavaṃ lavaṇaṃ
vilīyate lavaṇākāraṃ parityajya jalākāraṃ bhavati tathā tadvat tattve ātmani sandhīyamānaṃ
dhāryamāṇaṃ mano vilīyate ātmākāraṃ bhavati.


%\begin{vsid}{#hp04_00}
\startsloka
{\bf jñeyaṃ sarvaṃ pratītaṃ ca jñānaṃ ca mana ucyate
jñānaṃ jñeyaṃ samaṃ naṣṭaṃ nānyaḥ panthā dvitīyakaḥ (4.60)}
\stopsloka

manaso vilaye jāte dvaitam api līyata ity āha tribhiḥ jñeyam iti. sarvaṃ sakalaṃ jñeyaṃ jñānārhaṃ
pratītaṃ jñātaṃ ca jñānaṃ ca idaṃ sarvaṃ mana ucyate. sarvasya manaḥkalpanāmātratvān
manaḥśabdenocyate. jñānaṃ jñeyaṃ ca samaṃ manaso vilaye manasā sārdhaṃ naṣṭaṃ yadi tarhi dvitīyakaḥ
dvitīya eva dvitīyakaḥ panthā manoviṣayo nāsti. dvaitaṃ nāstīti phalitārthaḥ.

%\begin{vsid}{#hp04_00}
\startsloka
{\bf manodṛśyam idaṃ sarvaṃ yat kiñcit sacarācaram
manaso hy unmanībhāvād dvaitaṃ naivolabhyate (4.61)}
\stopsloka

manodṛśyam iti. idam upalabhyamānaṃ yat kiñcit yat kim api caraṃ jaṅgamam acaraṃ sthāvaraṃ caraṃ
cācaraṃ ca carācare tābhyāṃ saha vartata iti sacarācaram yaj jagat tat sarvaṃ manodṛśyaṃ manasā
dṛśyam manaḥsaṅkalpamātram ity arthaḥ. manaḥkalpanāsattve pratītes tadabhāve cāpratīter bhrama eva
sarvaṃ jagat bhramasya prātītikaśarīratvāt. na ca bauddhamataprasaṅgaḥ. bhramādhiṣṭhānasya
brahmaṇaḥ satyatvābhyupagamāt. manasa unmanībhāvād vilayād dvaitaṃ bhedaḥ naivolabhyate naiva
pratīyate dvaitabhramahetor manaḥsaṅkalpasyābhāvāt. hīti hetāv avyayam.

%\begin{vsid}{#hp04_00}
\startsloka
{\bf jñeyavastuparityāgād vilayaṃ yāti mānasam
manaso vilaye jāte kaivalyam avaśiṣyate (4.62)}
\stopsloka

jñeyeti. jñeyaṃ jñānaviṣayaṃ yad vastu sarvaṃ carācaraṃ yad dṛśyaṃ\var{yad dṛśyaṃ \Wthree \Tue \EdMu \lem
  om \EdLo (no variants given)} tasya parityāgān nāmarūpātmakasya tasya parivarjanād mānasaṃ vilayaṃ
yāti\var{mānasaṃ vilayaṃ yāti \EdAd \lem vilayaṃ \Tue \Wthree \EdLo \EdMu}
saccidānandarūpātmākāraṃ\var{saccidānandātmakaṃ \Wthree \lem saccidānandarūpātmākāraṃ \Tue \EdLo \EdAd
  \EdMu} bhavati. manaso vilaye jāte sati kaivalyam kevalasyātmano bhāvaḥ kaivalyam
avaśiṣyate. advitīyātmasvarūpam avaśiṣṭaṃ bhavatīty arthaḥ.

%\begin{vsid}{#hp04_00}
\startsloka
{\bf evaṃ nānāvidhopāyāḥ samyak svānubhavānvitāḥ
samādhimārgāḥ kathitāḥ pūrvācāryair mahātmabhiḥ (4.63)}
\stopsloka

evam iti. evaṃ antarlakṣyaṃ bahirdṛṣṭiḥ\comment{HP 4.36} ityādyuktaprakāreṇa mahān
samādhipariśīlanaśuddha ātmāntaḥkaraṇaṃ yeṣāṃ te mahātmānas tair mahātmabhiḥ pūrve ca te ācāryāś ca
pūrvācāryā matsyendrādayas taiḥ samādheś cittavṛttinirodhasya mārgāḥ prāptyupāyāḥ kathitāḥ. kīdṛśāḥ
samādhimārgāḥ nānāvidhopāyāḥ nānāvidhā upāyāḥ sādhanāni yeṣāṃ te tathā samyak samīcīnatayā
saṃśayaviparyayarāhityena yaḥ svānubhava ātmānubhavas tenānvitā yuktāḥ.

%\begin{vsid}{#hp04_00}
\startsloka
{\bf suṣumṇāyai kuṇḍalinyai sudhāyai candrajanmane
manonmanyai namas tubhyaṃ mahāśaktyai cidātmane (4.64)}
\stopsloka

suṣumṇādibhyaḥ kṛtakṛtyas tāḥ praṇamati suṣumṇāyai iti. suṣumṇā madhyanāḍī tasyai kuṇḍalinyai
ādhāraśaktyai candrād bhrūmadhyasthāj janma yasyās tasyai sudhāyai pīyūṣāya manonmanyai
turyāvasthāyai cic caitanyam ātmā svarūpaṃ yasyāḥ sā tathā tasyai mahatī jaḍānāṃ kāyendriyamanasāṃ
caitanyasampādakatvāt sarvottamā yā śaktiś citiśaktiḥ puruṣarūpā tasyai. tubhyam iti pratyekaṃ
sambadhyate. namaḥ prahvībhāvo 'stu.

%\begin{vsid}{#hp04_00}
\startsloka
{\bf aśakyatattvabodhānāṃ mūḍhānām api saṃmatam
proktaṃ gorakṣanāthena nādopāsanam ucyate (4.65)}
\stopsloka

nānāvidhān samādhyupāyān uktvā nādānusandhānarūpaṃ mukhyopāyaṃ pratijānīte aśakyeti. avyutpannatvād
aśakyas tattvabodhas tattvajñānaṃ yeṣāṃ te tathā teṣāṃ mūḍhānām anadhītānāṃ saṃmatam. apiśabdāt
kimutādhītānām iti gamyate. gorakṣanāthena proktam ity anena mahaduktatvād upādeyatvaṃ
gamyate. nādasyānāhatadhvaner upāsanam anusandhānarūpaṃ sevanam ucyate kathyate.

%\begin{vsid}{#hp04_00}
\startsloka
{\bf śrīādināthena sapādakoṭi-
layaprakārāḥ kathitā jayanti
nādānusandhānakam ekam eva
manyāmahe mukhyatamaṃ layānām (4.66)}
\stopsloka

śrīādinātheneti. śrīādināthena śivena kathitāḥ proktāḥ pādena caturthāṃśena saha vartamānāḥ
koṭisaṅkhyākā layaprakārāś\var{layaprakārāś \EdMu \EdAd \lem ye layaprakārāś \Tue \Wthree \EdLo}
cittalayasādhanabhedā jayanti utkarṣeṇa vartante. vayaṃ tu nādānusandhānakam nādānucintanam eva
ekam kevalaṃ layānām layasādhanānāṃ madhye mukhyatamam atiśayena mukhyaṃ manyāmahe
jānīmahe. utkṛṣṭānāṃ layasādhanānāṃ madhye utkṛṣṭatamatvād gorakṣābhimatatvād asmadabhimatatvāc ca
nādānusandhānam eva avaśyaṃ vidheyam iti bhāvaḥ.

%\begin{vsid}{#hp04_00}
\startsloka
{\bf muktāsane sthito yogī mudrāṃ sandhāya śāmbhavīm
śṛṇuyād dakṣiṇe karṇe nādam antastham ekadhīḥ (4.67)}
\stopsloka

śāmbhavīmudrayā nādānusandhānam āha muktāsana iti. muktāsane siddhāsane sthito yogī śāmbhavīm
mudrāṃ antarlakṣyaṃ bahirdṛṣṭiḥ \comment{HP 4.36} ityādinoktāṃ sandhāya kṛtvā ekadhīr ekāgracittaḥ
san dakṣiṇe karṇe 'ntasthaṃ suṣumnānāḍyāṃ santam eva nādaṃ śṛṇuyāt. tad uktaṃ tripurāsārasamuccaye

ādau mattālimālājanitaravasamastārajhaṅkārakārī
nādo 'sau vāṃśikasyānilabharitasadvaṃśaniḥsvānatulyaḥ
ghaṇṭānādānukārī tad anu ca jaladhidhvānadhīro gabhīro
garjatparjanyaghoṣaḥ para iha kuhare vartate brahmanāḍyāḥ \comment{Tripurāsārasamuccaye 3.50} iti.

%\begin{vsid}{#hp04_00}
\startsloka
{\bf śravaṇapuṭanayanayugalaghrāṇamukhānāṃ nirodhanaṃ kāryam
śuddhasuṣumṇāsaraṇau sphuṭam amalaḥ śrūyate nādaḥ (4.68)}  % metre? 
\stopsloka

ṣaṇmukhīmudrayā nādānusandhānam āha śravaṇeti. śravaṇapuṭe śrotrapuṭe nayanayor netrayor yugalaṃ
yugmaṃ ghrāṇaśabdena ghrāṇapuṭe mukham āsyam eṣāṃ dvandve prāṇyaṅgatvād ekavadbhāve prāpte 'pi
sarvasyāpi dvandvaikavadbhāvasya vaikalpikatvān na bhavati. teṣāṃ nirodhanaṃ karāṅgulibhiḥ
kāryam. nirodhanaṃ cettham

\startsloka
aṅguṣṭhābhyām ubhau karṇau tarjanībhyāṃ ca cakṣuṣī
nāsāpuṭau tathānyābhyāṃ pracchādya karaṇāni ca \comment{quotation untraced}
\stopsloka

iti cakārāt tadanyābhyāṃ mukhaṃ pracchādyeti samuccīyate. śuddhā prāṇāyāmair malarahitā yā
suṣumnāsaraṇiḥ suṣumṇāpaddhatis tasyām amalo nādaḥ sphuṭam vyaktaṃ śrūyate.

%\begin{vsid}{#hp04_00}
\startsloka
{\bf ārambhaś ca ghaṭaś caiva tathā paricayo 'pi ca
niṣpattiḥ sarvayogeṣu syād avasthācatuṣṭayam (4.69)}
\stopsloka

atha nādasya catasro 'vasthāḥ prāha ārambhaś ceti. ārambhāvasthā ghaṭāvasthā paricayāvasthā
niṣpattyavasthā iti. sarvayogeṣu sarveṣu cittavṛttinirodhopāyeṣu śāmbhavyādiṣu avasthācatuṣṭayam
syāt. cacaivatathāpicāḥ pādapūraṇārthāḥ.

%\begin{vsid}{#hp04_00}
\startsloka
{\bf atha ārambhāvasthā
brahmagranther bhaved bhedo hy ānandaḥ śūnyasambhavaḥ
vicitraḥ kvaṇako dehe 'nāhataḥ śrūyate dhvaniḥ (4.70)}
\stopsloka

tatrārambhāvasthām āha brahmagranther iti. brahmagranther anāhatacakre vartamānāyā bhedaḥ
prāṇāyāmābhyāsena bhedanaṃ yadā bhavet tadeti yattador adhyāhāraḥ. ānandayatīty ānandaḥ
ānandajanakaḥ śūnye hṛdākāśe sambhavatīti śūnyasambhavo hṛdākāśotpanno vicitro nānāvidhaḥ kvaṇo
bhūṣaṇaninadaḥ sa eva kvaṇakaḥ. bhūṣaṇaninadasadṛśa ity arthaḥ.

\startsloka
  \ \ \ \ \ \ \ \ \ \ \ \ \ bhūṣaṇānāṃ tu śiñjitam
  nikvāṇo nikvaṇaḥ kvāṇaḥ kvaṇaḥ kvaṇanam ity api
\stopsloka


ity amaraḥ.\comment{Amarakośa 1.7.3} anāhataḥ dhvanir anāhato nirhrādo dehe dehamadhye śrūyate
śravaṇaviṣayo bhavatīty arthaḥ.

%\begin{vsid}{#hp04_00}
\startsloka
{\bf divyadehaś ca tejasvī divyagandhas tv arogavān
sampūrṇahṛdayaḥ śūnya ārambhe yogavān bhavet (4.71)}
\stopsloka

divyadeha iti. śūnye hṛdākāśe ya ārambho nādārambhas tasmin sati
hṛdākāśaviśuddhākāśa\comment{viśuddhākāśa \Wthree \Tue \EdLo \EdMu \lem viśuddhyākāśa \EdAd}bhrūmadhyākāśāḥ
śūnyātiśūnyamahāśūnyaśabdair vyavahriyante yogibhiḥ. sampūrṇahṛdayaḥ prāṇavāyunā samyak pūrṇaṃ
hṛdayaṃ yasya sa tathā ānandena vā pūrṇahṛdayaḥ yogavān yogī divyo rūpalāvaṇyabalasampanno deho
yasya sa divyadehaḥ tejasvī pratāpavān divyagandho divya uttamo gandho yasya sa tathā arogavān
rogarahito bhaved iti sambandhaḥ.

%\begin{vsid}{#hp04_00}
\startsloka
{\bf atha ghaṭāvasthā
dvitīyāyāṃ ghaṭīkṛtya vāyur bhavati madhyagaḥ
dṛḍhāsano bhaved yogī jñānī devasamas tadā (4.72)}
\stopsloka

ghaṭāvasthām āha dvitīyāyām iti. dvitīyāyāṃ ghaṭāvasthāyāṃ vāyuḥ prāṇaḥ ghaṭīkṛtya ātmanā sahāpānaṃ
nādabindū caikīkṛtya madhyago madhyacakragataḥ kaṇṭhasthāne madhyacakram. tad uktam atraiva
jālandharabandhe

\startsloka
madhyacakram idaṃ jñeyaṃ ṣoḍaśādhārabandhanam \comment{HP 3.73}
\stopsloka

iti. yadā bhaved ity adhyāhāraḥ. tadā tasyām avasthāyāṃ yogī yogābhyāsī dṛḍham āsanaṃ yasya sa
dṛḍhāsanaḥ sthirāsano jñānī pūrvāpekṣayā kuśalabuddhir devasamo rūpalāvaṇyādhikyād devatulyo
bhavet. tad uktam īśvarokte rājayoge\comment{Yogabīja ????}

\startsloka
prāṇāpānau nādabindū jīvātmaparamātmanoḥ
militvā ghaṭate yasmāt tasmāt sa ghaṭa ucyate
\stopsloka

iti.



%\begin{vsid}{#hp04_00}
\startsloka
{\bf viṣṇugranthes tato bhedāt paramānandasūcakaḥ
atiśūnye vimardaś ca bherīśabdas tadā bhavet}
\stopsloka

viṣṇugranther iti. tato brahmagranthibhedanānantaraṃ viṣṇugrantheḥ kaṇṭhe vartamānāyā bhedāt
kumbhakair bhedanāt paramānandasya bhāvino brahmānandasya sūcako jñāpakaḥ. atiśūnye kaṇṭhāvakāśe
vimardo 'nekanādasaṃmardo bheryāḥ śabda iva śabdo bherīśabdo bherīnādaś ca tadā tasmin kāle bhavet.

%  saṃmarda, the explanation in the Jyotsnā is again, the reading of another ms. 



%\begin{vsid}{#hp04_00}
\startsloka
atha paricayāvasthā  
{\bf tṛtīyāyāṃ tu vijñeyo vihāyomardaladhvaniḥ
mahāśūnyaṃ tadā yāti sarvasiddhisamāśrayam (4.74)}
\stopsloka

paricayāvasthām āha sārdhadvābhyāṃ tṛtīyāyām iti. tṛtīyāyāṃ paricayāvasthāyāṃ vihāyomardaladhvanir
vihāyasi bhrūmadhyākāśe mardalasya vādyaviśeṣasya dhvanir iva dhvanir vijñeyo viśeṣeṇa jñānārho
bhavati. tadā tasyām avasthāyāṃ sarvasiddhisamāśrayam sarvāsāṃ siddhīnām aṇimādīnāṃ samāśrayam
sthānam tatra saṃyamād aṇimādiprāpteḥ. mahāśūnyaṃ bhrūmadhyākāśaṃ yāti gacchati prāṇa iti śeṣaḥ.

%\begin{vsid}{#hp04_00}
\startsloka
{\bf cittānandaṃ tadā jitvā sahajānandasambhavaḥ
doṣaduḥkhajarāvyādhikṣudhānidrāvivarjitaḥ (4.75)}
\stopsloka

cittānandam iti. cittānandaṃ nādaviṣayāntaḥkaraṇavṛttijanyaṃ sukhaṃ jitvābhibhūya
sahajānandasambhavaḥ sahajānandaḥ svābhāvikam ātmasukhaṃ tasya sambhava āvirbhāvaḥ saḥ. doṣā
vātapittakaphā duḥkhaṃ tajjanyā vedanā ādhyātmikādi ca jarā vṛddhāvasthā vyādhir jvarādiḥ kṣudhā
bubhukṣā nidrā svāpa etābhir vivarjito rahitas tadā yogī bhavatīti.

%\begin{vsid}{#hp04_00}
\startsloka
{\bf rudragranthiṃ yadā bhittvā śarvapīṭhagato 'nilaḥ
niṣpattau vaiṇavaḥ śabdaḥ kvaṇadvīṇākvaṇo bhavet (4.76)}
\stopsloka

%  Interesting singular reading of the Jyotsna: śarva for sarva  

tadā. kadā ity apekṣāyām āha rudreti. yadā rudragranthiṃ bhittvā ājñācakre rudragranthiḥ
śarvasyeśvarasya pīṭhaṃ sthānaṃ bhrūmadhyaṃ tatra gataḥ prāpto 'nilaḥ prāṇo bhavati tadā.

niṣpattyavasthām āha niṣpattāv iti. niṣpattau niṣpattyavasthāyām brahmarandhre gate prāṇe
niṣpattyavasthā bhavati. vaiṇavaḥ veṇor ayaṃ vaiṇavo vaṃśasambandhī śabdo ninādaḥ kvaṇantī
śabdāyamānā yā vīṇā tasyāḥ kvaṇaḥ śabdo bhavet.


%\begin{vsid}{#hp04_00}
\startsloka
{\bf ekībhūtaṃ tadā cittaṃ rājayogābhidhānakam
sṛṣṭisaṃhārakartāsau yogīśvarasamo bhavet (4.77)}
\stopsloka

tadā tasyām avasthāyāṃ cittam antaḥkaraṇam ekībhūtam ekaviṣayībhūtam viṣayaviṣayiṇor
abhedopacārāt. tad rājayogābhidhānakam rājayoga ity abhidhānaṃ yasya tad rājayogābhidhānakam
cittasyaikāgrataiva rājayoga ity arthaḥ. sṛṣṭisaṃhāreti asau nādānusandhānaparo yogī
sṛṣṭisaṃhārakartā sṛṣṭiṃ saṃhāraṃ ca karotīti tādṛśaḥ. ata eveśvarasama īśvaratulyo bhavet.

%\begin{vsid}{#hp04_00}
\startsloka
{\bf astu vā māstu vā muktir atraivākhaṇḍitaṃ sukham
layodbhavam idaṃ saukhyaṃ rājayogād avāpyate (4.78)
rājayogam ajānantaḥ kevalaṃ haṭhakarmiṇaḥ
etān abhyāsino manye prayāsaphalavarjitān (4.79)}
\stopsloka

astu veti. rajayogam iti. ubhau prāg vyākhyātau. 

%\begin{vsid}{#hp04_00}
\startsloka
{\bf unmanyavāptaye śīghraṃ bhrūdhyānaṃ mama saṃmatam
rājayogapadaṃ prāptuṃ sukhopāyo 'lpacetasām
sadyaḥpratyayasandhāyī jāyate nādajo layaḥ (4.80)}
\stopsloka

% constellation of lines inconclusive in the collation of the original

unmanyavāptaya iti. śīghraṃ tvaritam unmanyā unmanyavasthāyā avāptaye prāptyarthaṃ bhrūdhyānaṃ
bhruvor dhyānaṃ bhrūmadhye dhyānaṃ mama svātmārāmasya saṃmatam. rājayogo yogānāṃ rājā tad eva padaṃ
rājayogapadaṃ turyāvasthākhyaṃ prāptuṃ labdhuṃ pūrvoktabhrūdhyānarūpaḥ sukhopāyaḥ sukhasādhya
upāyaḥ sukhopāyaḥ alpacetasām alpabuddhīnām api. kim utānyeṣām ity abhiprāyaḥ. nādajaḥ nādāj jāto
layaś cittavilayaḥ sadyaḥ śīghraṃ pratyayaṃ pratītiṃ sandadhātīti pratyayasandhāyī pratītikaro
jāyate prādurbhavati.


%\begin{vsid}{#hp04_00}
\startsloka
{\bf nādānusandhānasamādhibhājāṃ
yogīśvarāṇāṃ hṛdi vardhamānam
ānandam ekaṃ vacasām agamyaṃ
jānāti taṃ śrīgurunātha ekaḥ (4.81)}
\stopsloka

nādānusandhāneti. nādasya anāhatadhvaner anusandhānam anucintanaṃ tena samādhiś cittaigāgryaṃ taṃ
bhajantīti nādānusandhānasamādhibhājaḥ teṣām yogiṣu yogayukteṣu īśvarāḥ samarthāḥ teṣām hṛdi
hṛdaye vardhata iti vardhamānas taṃ vardhamānam vacasāṃ vācām agamyam idam iti vaktum aśakyaṃ taṃ
yogaśāstraprasiddham ekaṃ mukhyam ānandam āhlādam eko 'nanyaḥ śrīgurunāthaḥ śrīmān gurur eva nātho
jānāti vetti. etena nādānusandhānānando gurugamya eveti sūcitam.

%\begin{vsid}{#hp04_00}
\startsloka
{\bf karṇau pidhāya hastābhyāṃ yaṃ śṛṇoti dhvaniṃ muniḥ
tatra cittaṃ sthirīkuryād yāvat sthirapadaṃ vrajet (4.82)}
\stopsloka

nādānusandhānāt pratyāhārādikrameṇa samādhim āha karṇāv ity ādibhiḥ munir mananaśīlo yogī
hastābhyām ity anena hastāṅguṣṭhau lakṣyete tābhyāṃ karṇau śrotre pidhāya hastāṅguṣṭhau
śrotravivarayoḥ kṛtvety arthaḥ yaṃ dhvanim anāhataniḥsvanaṃ śṛṇoty ākarṇayati tatra tasmin
dhvanau cittaṃ sthirīkuryād asthiraṃ sthiraṃ sampadyamānaṃ kuryāt. yāvat sthiraṃ padaṃ sthirapadaṃ
turyākhyaṃ gacchet. tad uktaṃ turyāvasthā cidabhivyañjakanādasya vedanaṃ proktam iti
nādānusandhānena vāyusthairyam aṇimādayo 'pi bhavanti. ity uktaṃ ca tripurāsārasamuccaye

\startsloka
vijito bhavatīha tena vāyuḥ sahajo yasya samutthitaḥ praṇādaḥ
aṇimādiguṇā bhavanti tasyāmitapuṇyasya mahāguṇodayasya
surarājatanūjavairirandhre vinirudhya svakarāṅgulidvayena
jaladher iva dhīranādam antaḥ prasarantaṃ sahasā śṛṇoti martyaḥ \comment{Tripurāsārasamuccaya 3.5253} 
\stopsloka

iti. surarāja indras tasya tanūjo 'rjunas tasya vairī karṇas tadrandhre. spaṣṭam anyat.

%\begin{vsid}{#hp04_00}
\startsloka
{\bf abhyasyamāno nādo 'yaṃ bāhyam āvṛṇute dhvanim
pakṣād vikṣepam akhilaṃ jitvā yogī sukhī bhavet (4.83)}
\stopsloka

abhyasyamāna iti. abhyasyamāno 'nusandhīyamāno 'yaṃ nādo 'nāhatākhyo bāhyam dhvanim bahirbhavaṃ
śabdam āvṛṇute śrutyor aviṣayaṃ karoti. yogī nādābhyāsī pakṣān māsārdhād akhilaṃ sarvaṃ vikṣepam
cittacāñcalyaṃ jitvābhibhūya sukhī svānandī bhavet.

%\begin{vsid}{#hp04_00}
\startsloka
{\bf śrūyate prathamābhyāse nādo nānāvidho mahān
tato 'bhyāse vardhamāne śrūyate sūkṣmasūkṣmakaḥ (4.84)}
\stopsloka

śrūyata iti. prathamābhyāse pūrvābhyāse nānāvidho 'nekavidho mahān jaladhijīmūtabheryādisadṛśo
nādo 'nāhatasvanaḥ śrūyate ākarṇyate. tato 'nantaram abhyāse nādānusandhānābhyāse vardhamāne sati
sūkṣmasūkṣmakaḥ sūkṣmaḥ sūkṣma eva śrūyate śravaṇaviṣayo bhavati.

%\begin{vsid}{#hp04_00}
\startsloka
{\bf ādau jaladhijīmūtabherījharjharasambhavāḥ
madhye mardalaśaṅkhotthā ghaṇṭākāhalajās tathā (4.85)}
\stopsloka

nānāvidhaṃ nādam āha dvābhyāṃ ādāv iti. ādau vāyor brahmarandhragamanasamaye jaladhiḥ samudro
jīmūto megho bherī vādyaviśeṣaḥ bherī strī dundubhiḥ pumān ity amaraḥ\comment{Amarakośa
  ??}. jharjharo vādyaviśeṣaḥ

\startsloka
vādyaprabhedā ḍamarumaḍḍuḍiṇḍimajharjharāḥ
mardalaḥ paṇavo 'nye ca\comment{Amarakośa
  ??} 
\stopsloka

ity amaraḥ. jaladhipramukhebhyaḥ sambhava iva sambhavo yeṣāṃ te tathā madhye brahmarandhre
vāyoḥ sthairyānantaraṃ mardalo vādyaviśeṣaḥ śaṅkho jalajas tābhyām utthā iva
mardalaśaṅkhotthāḥ. ghaṇṭākāhalau vādyaviśeṣau tābhyāṃ jātā iva ghaṇṭākāhalajāḥ.


%\begin{vsid}{#hp04_00}
\startsloka
{\bf ante tu kiṅkiṇīvaṃśavīṇābhramaraniḥsvanāḥ
iti nānāvidhā nādāḥ śrūyante dehamadhyagāḥ (4.86)}
\stopsloka

ante tv iti. ante tu prāṇasya brahmarandhre bahusthairyānantaraṃ tu kiṅkiṇī kṣudraghaṇṭikā
vaṃśo veṇuḥ vīṇā tantrī bhramaro madhupaḥ teṣāṃ niḥsvanā iva niḥsvanā iti pūrvoktā nānāvidhā
anekaprakārakā dehasya madhye gatāḥ prāptāḥ śrūyante.

%\begin{vsid}{#hp04_00}
\startsloka
{\bf mahati śrūyamāṇe 'pi meghabheryādike dhvanau
tatra sūkṣmāt sūkṣmataraṃ nādam eva parāmṛśet (4.87)}
\stopsloka

mahatīti. meghaś ca bherī ca te ādī yasya sa meghabheryādikaḥ tasmin. meghabherīśabdau
tajjanyanirghoṣaparau. mahati bṛhati dhvanau nināde śrūyamāṇe ākarṇyamāne saty api tatra teṣu
nādeṣu sūkṣmāt sūkṣmataram atisūkṣmaṃ nādam eva parāmṛśet cintayet. sūkṣmasya nādasya
cirasthāyitvāt tatrāsaktaṃ cittaṃ ciraṃ sthirībhaved iti bhāvaḥ.

%\begin{vsid}{#hp04_00}
\startsloka
{\bf ghanam utsṛjya vā sūkṣme sūkṣmam utsṛjya vā ghane
ramamāṇam api kṣiptaṃ mano nānyatra cālayet (4.88)}      % = Nādabindūpaniṣat 37
\stopsloka

ghanam iti. ghanam mahāntaṃ nādaṃ meghabheryādikam utsṛjya tyaktvā sūkṣme kiṅkiṇībhramarādisvane vā
sūkṣmam utsṛjya ghane vā nāde ramamāṇam ghanasūkṣmānyataranādagrahaṇaparityāgābhyāṃ krīḍantam api
kṣiptaṃ rajasātyantacañcalaṃ mano 'nyatra viṣayāntare na cālayet na prerayet. kṣiptaṃ mano
viṣayāntarāsaktaṃ na samādhīyate nādeṣv eva ramamāṇaṃ tu samādhīyate iti bhāvaḥ.

%\begin{vsid}{#hp04_00}
\startsloka
{\bf yatra kutrāpi vā nāde lagati prathamaṃ manaḥ
tatraiva susthirībhūya tena sārdhaṃ vilīyate (4.89)}
\stopsloka

yatreti. vā athavā yatra kutrāpi nāde yasmin kasmiṃścid ghane sūkṣme vā nāde prathamaṃ pūrvaṃ mano
lagati lagnaṃ bhavati tatraiva tasminn eva nāde susthirībhūya samyak sthiraṃ bhūtvā tena nādena
sārdhaṃ sākaṃ vilīyate līnaṃ bhavatīty arthaḥ. atra pūrvavākyena pratyāhāro dvitīyena dhāraṇā
tṛtīyena dhyānadvārā samādhir uktaḥ.

%  Here Patañjali's aṅgas are used for commentation instead of the four {\em avasthā}s. 
% Could be Svātmārāma!

%\begin{vsid}{#hp04_00}
\startsloka
{\bf makarandaṃ piban bhṛṅgo gandhaṃ nāpekṣate yathā
nādāsaktaṃ tathā cittaṃ viṣayān na hi kāṅkṣate (4.90)}
\stopsloka

makarandam iti. makarandaṃ puṣparasaṃ piban dhayan bhṛṅgo bhramaro gandhaṃ yathā nāpekṣate necchati
tathā nādāsaktaṃ nāde āsaktaṃ cittam antaḥkaraṇaṃ viṣayān visinvanty avabadhnanti pramātāraṃ
svasaṅgeti viṣayāḥ srakcandanavanitādayas tān na kāṅkṣate necchati. hīti niścaye.

%\begin{vsid}{#hp04_00}
\startsloka
{\bf manomattagajendrasya viṣayodyānacāriṇaḥ
niyamane samartho 'yaṃ ninādaniśitāṅkuśaḥ (4.91)}
\stopsloka

mana iti. viṣayaḥ śabdādir evodyānaṃ vanaṃ tatra caratīti viṣayodyānacārī tasya mana eva
mattagajendro durnivāratvāt tasya nināda evānāhatadhvanir eva niśitāṅkuśaḥ tīkṣṇāṅkuśaḥ niyamane
parāvartane samarthaḥ śaktaḥ. etaiḥ ślokaiḥ indriyāṇāṃ viṣayebhyaḥ pratyāharaṇaṃ pratyāhāraḥ

\startsloka
caratāṃ cakṣurādīnāṃ viṣayeṣu yathākramam
yat pratyāharaṇaṃ teṣāṃ pratyāhāraḥ prakīrtitaḥ \comment{source ?}
\stopsloka

ityuktalakṣaṇaḥ pratyāhāraḥ proktaḥ.


%\begin{vsid}{#hp04_00}
\startsloka
{\bf baddhaṃ tu nādabandhena manaḥ santyaktacāpalam
prayāti sutarāṃ sthairyaṃ chinnapakṣaḥ khago yathā (4.92)}
\stopsloka

baddhaṃ tv iti. nāda eva bandhaḥ badhyate 'neneti bandhaḥ bandhanasādhanaṃ tena svaśaktyā
svādhīnakaraṇena baddhaṃ bandhanam iva prāptam. nādadhāraṇādāv āsaktam ity arthaḥ. ata eva samyak
tyaktaṃ santyaktaṃ cāpalam kṣaṇe kṣaṇe viṣayagrahaṇaparityāgarūpaṃ yena tat tathā manaḥ sutarāṃ
sthairyaṃ prayāti nitarāṃ dhāraṇām eti. tatra dṛṣṭāntam āha chinnau pakṣau yasya tādṛśaḥ khe
gacchatīti khagaḥ pakṣī yathā. etena

\startsloka
prāṇāyāmena pavanaṃ pratyāhāreṇa cendriyam
vaśīkṛtya tataḥ kuryāc cittasthānaṃ śubhāśraye \comment{source ??}
\stopsloka

śubhāśraye cittasthāpanaṃ dhāraṇety uktalakṣaṇā dhāraṇā proktā.

%\begin{vsid}{#hp04_00}
\startsloka
{\bf sarvacintāṃ parityajya sāvadhānena cetasā
nāda evānusandheyo yogasāmrājyam icchatā (4.93)}
\stopsloka

sarvacintām iti. sarveṣāṃ bāhyabhyantaraviṣayāṇāṃ yā cintā cintanaṃ tāṃ parityajya tyaktvā
sāvadhānena ekāgreṇa cetasā yogānāṃ sāmrājyam samrājo bhāvaḥ. yogaśabdo 'rśa
ādyajantaḥ.\comment{explains sequence??} rājayogitvam iti yāvat. icchatā vāñchatā puṃsā nāda
evānāhatadhvanir evānusandheyo 'nucintanīyaḥ. nādākāravṛttipravāhaḥ kartavya ity arthaḥ. etena

\startsloka
tadrūpapratyayaikāgryasantatiś cānyanispṛhā
tad dhyānaṃ prathamair aṅgaiḥ ṣaḍbhir niṣpādyate nṛpa\comment{Viṣṇupurāṇam 6.7.89 ?}
\stopsloka

tatra pratyayaikatānatā dhyānam \comment{Yogasūtra 3.2} ityuktalakṣaṇaṃ dhyānam uktam.



%\begin{vsid}{#hp04_00}
\startsloka
{\bf nādo 'ntaraṅgasāraṅgabandhane vāgurāyate
antaraṅgakuraṅgasya vadhe vyādhāyate 'pi ca (4.94)}
\stopsloka

nādo 'ntaraṅgeti. nādaḥ antaraṅgaṃ mana eva sāraṅgo mṛgas tasya bandhane cāñcalyaharaṇe vāgurāyate
vāgurevācarati. vāgurā jālam. yathā vāgurā bandhanena sāraṅgasya cāñcalyaṃ harati tathā nādo
'ntaraṅgasya svaśaktyā cāñcalyaṃ haratīty arthaḥ. antaraṅgaṃ mana eva kuraṅgo hariṇas tasya vadhe
nānāvṛttyutpādanāpanayanam eva manaso vadhas tasmin vyādhāyate vyādha ivācarati. yathā vyādho
vāgurābaddhaṃ mṛgaṃ hanti evaṃ nādo 'pi svāsaktaṃ mano hantīty arthaḥ.

%\begin{vsid}{#hp04_00}
\startsloka
{\bf antaraṅgasya yamino vājinaḥ parighāyate
nādopāstirato nityam avadhāryā hi yoginā (4.95)}
\stopsloka

antaraṅgasyeti. yamino yogino 'ntaraṅgaṃ manas tasya capalatvād\var{capalatvād \Tue \EdAd\lem ca
  palāyatvād \Wthree cāpalatvād \EdLo} vājino 'śvasya parighāyate vājiśālādvāraparigha ivācarati
nāda iti śeṣaḥ. yathā vājiśālāparigho vājino 'nyatra gatiṃ ruṇaddhi tathā nādo 'ntaraṅgasyety
arthaḥ. ataḥ kāraṇāt\var{ataḥ kāraṇāt \Wthree \Tue \EdMu \lem antaḥkāraṇād \EdLo (variants
  unclear), antaḥkaraṇāt} \EdAd yoginā nādasyopāstir upāsanā nityaṃ pratyaham
avadhāryāvadhāraṇīyā. hīti niścaye 'vyayam.

%\begin{vsid}{#hp04_00}
\startsloka
{\bf baddhaṃ vimuktacāñcalyaṃ nādagandhakajāraṇāt
manaḥpāradam āpnoti nirālambākhyakhe 'ṭanam (4.96)}
\stopsloka

baddham iti. nāda eva gandhaka upadhātuviśeṣas tena jāraṇaṃ jīrṇīkaraṇaṃ nādagandhakasambandhena
cāñcalyaharaṇaṃ tasmād baddhaṃ nādaikāsaktaṃ pakṣe guṭikākṛtiṃ prāptaṃ ata eva vimuktaṃ tyaktaṃ
cāñcalyam anekaviṣayākārapariṇāmarūpaṃ yena. pakṣe vimuktalaulyaṃ manaḥ pāradaṃ mana eva pāradaṃ
cañcalaṃ nirālambaṃ brahma tad evākhyā yasya tan nirālambākhyaṃ tad eva kham aparicchinnatvāt
tasminn aṭanam gamanaṃ tadākāravṛttipravāham. pakṣe ākāśagamanaṃ prāpnoti. yathā baddhaṃ pāradam
ākāśagamanaṃ karoti evaṃ baddhaṃ mano brahmākāravṛttipravāham avicchinnaṃ karotīty arthaḥ.


%\begin{vsid}{#hp04_00}
\startsloka
{\bf nādaśravaṇataḥ kṣipram antaraṅgabhujaṅgamam
vismṛtya sarvam ekāgraḥ kutracin na hi dhāvati (4.97)}
\stopsloka

nādeti. nādasyānāhatasvanasya śravaṇataḥ śravaṇāt kṣipram drutam antaraṅgaṃ mana eva bhujaṅgamaḥ
sarpaś capalatvān nādapriyatvāc ca bhujaṅgamena rūpakaṃ manasaḥ. sarvaṃ viśvam vismṛtya
vismṛtiviṣayaṃ kṛtva ekāgro nādākāravṛttipravāhavān san kutrāpi viṣayāntare nahi dhāvati naiva
dhāvanaṃ karoti. dhyānottaraiḥ ślokaiḥ

\startsloka
tasyaiva kalpanāhīnaṃ svarūpagrahaṇaṃ hi yat
manasā dhyānaniṣpādyaṃ samādhiḥ so 'bhidhīyate \comment{Viṣṇupurāṇa 6.7.92}
\stopsloka

iti viṣṇupurāṇoktalakṣaṇaḥ tad evārthamātranirbhāsaṃ svarūpaśūnyam iva samādhiḥ\comment{Yogasūtra
  3.3} iti pātañjalasūtroktalakṣaṇaś ca samprajñātalakṣaṇaḥ samādhir uktaḥ.

%\begin{vsid}{#hp04_00}
\startsloka
{\bf kāṣṭhe pravartito vahniḥ kāṣṭhena saha śāmyati
nāde pravartitaṃ cittaṃ nādena saha līyate (4.98)}
\stopsloka

kāṣṭha iti. kāṣṭhe dāruṇi pravartitaḥ prajvālito vahniḥ kāṣṭhena saha śāmyati jvālārūpaṃ parityajya
tejomātra\var{tejomātra \Wthree \Tue \lem tanmātra \EdAd}rūpeṇāvatiṣṭhate yathā tathā nāde pravartitaṃ
cittaṃ nādena saha līyate. rājasatāmasavṛttināśāt sattvamātrāvaśeṣaṃ saṃskāraśeṣaṃ ca
bhavati. tatra ca maitrāyaṇīmantraḥ

\startsloka
yathā nirindhano vahniḥ svayonāv upaśāmyati
tathā vṛttikṣayāc cittaṃ svayonāv upaśāmyati \comment{Maitrāyanīyopaniṣat 6.1} iti.
\stopsloka

% 
%\begin{vsid}{#hp04_00}
\startsloka
{\bf ghaṇṭādinādasaktastabdhāntaḥkaraṇahariṇasya
  praharaṇam api sukaraṃ syāc charasandhānapravīṇaś
  \hfill cet (4.99)} \comment{{\em syāt} is missing in all
sources except \EdAd.}
\stopsloka

ghaṇṭādīti. ghaṇṭā ādir yeṣāṃ śaṅkhamardalajharjharadundubhi\-jīmūtā\-dīnāṃ te ghaṇṭādayaḥ teṣāṃ nādāḥ
teṣu saktaḥ. ata eva stabdho niścalo yo 'ntaḥkaraṇam eva hariṇo mṛgaḥ tasya praharaṇam
nānāvṛttyutpattipratibandhanam\var{utpatti \Wthree \Tueac \EdLo \ om \Tuepc \EdAd} antaḥkaraṇapakṣe.

hariṇapakṣe tu praharaṇaṃ hananam api śaravad drutagāmino vāyoḥ sandhānaṃ
suṣumnāmārgeṇa brahmarandhre nirodhanam\var{nirodhanam \Tue \Wthree \lem nirodhana (probably misprint)
  \EdLo}. pakṣe śarasya bāṇasya sandhānaṃ dhanuṣi yojanaṃ tasmin pravīṇaḥ kuśalaś cet sukaraṃ
sukhena kartuṃ śakyaṃ syāt.



%\begin{vsid}{#hp04_00}
\startsloka
{\bf anāhatasya śabdasya dhvanir ya upalabhyate
dhvaner antargataṃ jñeyaṃ jñeyasyāntargataṃ manaḥ
manas tatra layaṃ yāti tad viṣṇoḥ paramaṃ padam (4.100)}
\stopsloka

anāhatasyeti. anāhatasya śabdasyānāhatasvanasya yo dhvanir nirhrāda upalabhyate śrūyate tasya
dhvaner antargataṃ jñeyaṃ jyotiḥ svaprakāśacaitanyam. jñeyasyāntargataṃ jñeyākāratām āpannaṃ
mano 'ntaḥkaraṇaṃ tatra jñeye mano vilayaṃ yāti paravairāgyeṇa sakalavṛttiśūnyaṃ saṃskāraśeṣaṃ
bhavati. tad viṣṇoḥ vibhor ātmanaḥ paramam antaḥkaraṇavṛttyupādhirāhityān nirupādhikaṃ padyate
gamyate yogibhir iti padam svarūpam.

%\begin{vsid}{#hp04_00}
\startsloka
{\bf tāvad ākāśasaṅkalpo yāvac chabdaḥ pravartate
niḥśabdaṃ tat paraṃ brahma paramātmeti gīyate (4.101)}
\stopsloka

tāvad iti. yāvac chabdo 'nāhatadhvaniḥ pravartate śrūyate tāvad ākāśasya samyak kalpanaṃ
bhavati. śabdasyākāśaguṇatvād guṇaguṇinor abhedād vā manasā saha śabdasya vilayān niḥśabdaṃ
śabdarahitaṃ yat paraṃ brahma parabrahmaśabdavācyaṃ paramātmeti gīyate paramātmaśabdena sa
ucyate. sarvavṛttivilaye yaḥ svarūpeṇāvasthitaḥ sa eva parabrahmaparamātmaśabdābhyām ucyate iti
bhāvaḥ.



%\begin{vsid}{#hp04_00}
\startsloka
{\bf yat kiñcin nādarūpeṇa śrūyate śaktir eva sā
yas tattvānto nirākāraḥ sa eva parameśvaraḥ (4.102)
 iti nādānusandhānam \var{iti nādānusandhānam \EdAd \Wthree \lem om \Tue \EdLo}}
\stopsloka

yat kiñcid iti. nādarūpeṇa anāhatadhvanirūpeṇa yat kiñcic chrūyate ākarṇyate sā śaktir eva. yas
tattvāntas tattvānām anto layo yasmin saḥ. tathā nirākāra ākārarahitaḥ sa eva parameśvaraḥ
sarvavṛttikṣaye svarūpāvasthito yaḥ sa ātmety arthaḥ kāṣṭhe pravartito vahniḥ \comment{4.98} ity
ādibhiḥ ślokaiḥ rājayogāparaparyāyo 'samprajñātaḥ samādhir uktaḥ.


%\begin{vsid}{#hp04_00}
\startsloka
{\bf sarve haṭhalayopāyā rājayogasya siddhaye
rājayogasamārūḍhaḥ puruṣaḥ kālavañcakaḥ (4.103)}
\stopsloka

sarva iti. haṭhaś ca layaś ca haṭhalayau tayor upāyā haṭhalayopāyā haṭhopāyā āsanakumbhakamudrārūpā
layopāyā nādānusandhānaśāmbhavīmudrādayaḥ. rājayogasya manasaḥ sarvavṛttinirodhalakṣaṇasya
siddhaye niṣpattaye proktā iti śeṣaḥ. rājayogaṃ samārūḍhaḥ samyag ārūḍhaḥ prāptavān ya puruṣaḥ sa
kālavañcakaḥ kālaṃ mṛtyuṃ vañcayati jayatīti tādṛśaḥ syād iti śeṣaḥ.

%\begin{vsid}{#hp04_00}
\startsloka
{\bf tattvaṃ bījaṃ haṭhaḥ kṣetram audāsīnyaṃ jalaṃ tribhiḥ
unmanīkalpalatikā sadya eva pravartate (4.104)}
\stopsloka

tattvam iti. tattvaṃ cittaṃ bījaṃ bījavad unmanyavasthāṅkurākāreṇa pariṇamamānatvāt. haṭhaḥ
prāṇāpānayor aikyalakṣaṇaḥ prāṇāyāmaḥ kṣetram\var{kṣetraṃ \EdAd \lem om \Wthree \EdLo} kṣetra iva
prāṇāyāme unmanīkalpalatikotpatteḥ audāsīnyaṃ paraṃ vairāgyaṃ jalaṃ tasyā
utpattikāraṇatvāt. paravairāgyahetukaḥ saṃskāraviśeṣaś cittasyāsamprajñātaḥ iti tallakṣaṇāt. etais
tribhir unmanyasamprajñātāvasthā saiva kalpalatikā sakaleṣṭasādhanatvāt sadya eva śīghram eva
pravartate pravṛttā bhavati utpannā bhavati.

%\begin{vsid}{#hp04_00}
\startsloka
{\bf sadā nādānusandhānāt kṣīyante pāpasañcayāḥ
nirañjane vilīyete niścitaṃ cittamārutau (4.105)}
\stopsloka

sadeti. sadā sarvadā nādānusandhānāt nādānucintanāt pāpasañcayāḥ pāpasamūhāḥ kṣīyante
naśyanti nirañjane nirguṇe caitanye niścitaṃ dhruvaṃ cittamārutau manaḥprāṇau vilīyete vilīnau
bhavataḥ.



%\begin{vsid}{#hp04_00}
\startsloka
{\bf śaṅkhadundhubhinādaṃ ca na śṛṇoti kadācana
kāṣṭhavaj jāyate deha unmanyāvasthayā dhruvam (4.106)}
\stopsloka

unmanyavasthāṃ prāptasya yoginaḥ sthitim āhāṣṭabhiḥ śaṅkhadundhubhīti. śaṅkho jalajo dundhubhir
vādyaviśeṣaḥ tayor nādaṃ ghoṣaṃ kadācana kasmiṃścid api samaye na śṛṇoti. śaṅkhadundhubhīty
upalakṣaṇaṃ nādamātrasya. unmanyāvasthayā deho dhruvam kāṣṭhavaj jāyate niśceṣṭatvād ity arthaḥ.

%\begin{vsid}{#hp04_00}
\startsloka
{\bf sarvāvasthāvinirmuktaḥ sarvacintāvivarjitaḥ
mṛtavat tiṣṭhate yogī sa mukto nātra saṃśayaḥ (4.107)}
\stopsloka

jāgratsvapnasuṣuptimūrcchāmaraṇalakṣaṇāḥ pañca vyutthānāvasthās tābhir viśeṣeṇa mukto rahitaḥ sarvā
yāś cintāḥ smṛtayaḥ tābhir vivarjito virahito yaḥ yogaḥ sakalavṛttinirodho 'syāstīti yogī
turyāvasthāvān sa mukto jīvann eva muktaḥ. sakalavṛttinirodhe ātmanaḥ svarūpāvasthānāt. tad uktaṃ
pātañjale sūtre tadā draṣṭuḥ svarūpe'vasthānam \comment{Yogasūtra 1.3} iti. spaṣṭam anyat.

%\begin{vsid}{#hp04_00}
\startsloka
{\bf khādyate na ca kālena bādhyate na ca karmaṇā
sādhyate na sa kenāpi yogī yuktaḥ samādhinā (4.108)}
\stopsloka

khādyata iti. samādhinā yukto yogī kālena mṛtyunā na khādyate na bhakṣyate na hanyata ity
arthaḥ. karmaṇā kṛtena śubhenāśubhena vā na bādhyate janmamaraṇādijananena na kleśyate. tathā ca
samādhiprakaraṇe pātañjalasūtraṃ tataḥ kleśakarmanivṛttiḥ \comment{Yogasūtra 4.30} iti. kenāpi
puruṣāntareṇa yantramantrādinā vā na sādhyate sādhayituṃ na śakyate.



%\begin{vsid}{#hp04_00}
\startsloka
{\bf na gandhaṃ na rasaṃ rūpaṃ na ca sparśaṃ na niḥsvanam
nātmānaṃ na paraṃ vetti yogī yuktaḥ samādhinā (4.109)}
\stopsloka

na gandham iti. samādhinā yukto yogī na gandhaṃ surabhim asurabhiṃ vā na rasaṃ
madhurāmlalavaṇakaṭukaṣāyatiktabhedāt ṣaḍvidhaṃ na rūpaṃ śuklanīlapītaraktaharitakapiśacitrabhedāt
saptavidhaṃ na sparśaṃ śītam uṣṇam anuṣṇaśītaṃ vā na niḥsvanam
śaṅkhadundubhijaladhijīmūtādininādaṃ bāhyam ābhyantaraṃ vā nātmānaṃ dehaṃ na paraṃ puruṣāntaraṃ
vettīti sarvatrānveti. ātmā dehe ghṛtau jīve svabhāve paramātmani iti vaijayantī.

%\begin{vsid}{#hp04_00}
\startsloka
{\bf cittaṃ na suptaṃ no jāgratsmṛtivismṛtivarjitam
na cāstam eti nodeti yasyāsau mukta eva saḥ (4.110)}
\stopsloka

cittam iti. yasya yoginaś cittam antaḥkaraṇaṃ na suptaṃ āvarakasya tamaso 'bhāvāt triguṇe
'ntaḥkaraṇe yadā sattvarajasī abhibhūya samastakaraṇāvarakaṃ tama āvirbhavati tadāntaḥkaraṇasya
viṣayākārapariṇāmābhāvāt tat suptam ity ucyate. no jāgrad indriyair arthagrahaṇābhāvāt. smṛtiś ca
vismṛtiś ca smṛtivismṛtī tābhyāṃ varjitam. vṛttisāmānyābhāvād udbodhakābhāvāc ca
smṛtivarjitam. smṛtyanukūlasaṃskārābhāvād\var{bhāvād \Wthree \EdLo \lem bhāvābhāvād \Tue}
vismṛtivarjitam. na cāstam nāśam eti prāpnoti saṃskāraśeṣasya cittasya sattvāt. nodety
udbhavati vṛttyanutpādanāt. so 'sau mukta eva jīvanmukta eva.

%\begin{vsid}{#hp04_00}
\startsloka
{\bf na vijānāti śītoṣṇaṃ na duḥkhaṃ na sukhaṃ tathā
na mānaṃ nāpamānaṃ ca yogī yuktaḥ samādhinā (4.111)}
\stopsloka

na vijānātīti. samādhinā yukto yogī śītaṃ ca uṣṇaṃ ca śītoṣṇaṃ samāhāradvandvaḥ śītam uṣṇaṃ vā
padārthaṃ na duḥkhaṃ duḥkhajanakam parakṛtaṃ tāḍanādikaṃ na sukhaṃ sukhasādhanaṃ
surabhicandanādyanulepanādikaṃ na tathā cārthe mānaṃ parakṛtaṃ satkāraṃ na apamānam anādaraṃ ca
na vijānātīti kriyāpadaṃ prativākyam anveti.

%\begin{vsid}{#hp04_00}
\startsloka
{\bf svastho jāgradavasthāyāṃ suptavad yo 'vatiṣṭhate
niḥśvāsocchvāsahīnaś ca niścitaṃ mukta eva saḥ (4.112)}
\stopsloka

svastha iti. svasthaḥ prasannendriyāntaḥkaraṇaḥ etena tandrāmūrcchādi\-vyāvṛttiḥ jāgradavasthāyām ity
anena svapnasuṣuptyor nivṛttiḥ suptavat suptena tulyaṃ kāyendriyavyāpāraśūnyo yo yogī avatiṣṭhate
sthito bhavati samavapravibhya sthaḥ \comment{Aṣṭādhyāyī 1.3.22} ity
ātmanepadam. niśvāsocchvāsahīnaḥ bāhyavāyoḥ koṣṭhe grahaṇaṃ niḥśvāsaḥ koṣṭhasthitasya vāyor
bahirniḥsaraṇam ucchvāsas tābhyāṃ hīnaś cāvatiṣṭhata ity atrāpi sambadhyate. sa niścitaṃ
niḥsandigdhaṃ mukta eva jīvanmukta eva. tat uktaṃ dattatreyeṇa

\startsloka
nirguṇadhyānasampannaḥ samādhiṃ ca tato 'bhyaset
dinadvādaśakenaiva samādhiṃ samavāpnuyāt
vāyuṃ nirudhya medhāvī jīvanmukto bhaved dhruvam \comment{??} iti
\stopsloka

%\begin{vsid}{#hp04_00}
\startsloka
{\bf avadhyaḥ sarvaśastrāṇām aśakyaḥ sarvadehinām
agrāhyo mantrayantrāṇāṃ yogī yuktaḥ samādhinā (4.113)}
\stopsloka

avadhya iti. samādhinā yukto yogī. sarvaśastrāṇām iti sambandhasāmānye ṣaṣṭhī. sarvaśāstrair ity
arthaḥ. avadhyo hantum aśakya ity arthaḥ. sarvadehinām ity atrāpi sambandhamātravivakṣāyāṃ
ṣaṣṭhī. aśakyaḥ sarvadehibhiḥ balena śakyo na bhavatīty arthaḥ. mantrayantrāṇāṃ
vaśīkaraṇamāraṇoccāṭanādiphalair mantrayantrair agrāhyaḥ vaśīkartum\var{vaśīkartum \EdAd \lem
  vaśyādikaḥ kartum Edlo \Tue} aśakyaḥ. evaṃ prāptayogasya yogino vighnā bahavaḥ
samāyānti. tannivāraṇārthaṃ tajjñānasyāpekṣitatvāt te 'pi pradarśyante. dattātreyaḥ

\startsloka
ālasyaṃ prathamo vighno dvitīyas tu prakathyate
pūrvoktadhūrtagoṣṭhī ca tṛtīyo mantrasādhanam
caturtho dhātuvādaḥ syād iti yogavido viduḥ
\stopsloka

mārkaṇḍeyapurāṇe\var{mārkaṇḍeyapurāṇe \Wthree \lem iti mārkaṇḍeyapurāṇe \Tue}

\startsloka
upasargāḥ pravartante dṛṣṭā hy ātmani yoginaḥ
ye tāṃs te sampravakṣyāmi samāsena nibodha me
kāmyāḥ kriyās tathā kāmān manuṣyo yo 'bhivāñchati\var{manuṣyo yo 'bhivāñchati \EdLo \lem manuṣyān abhivāñchati \Wthree}
striyo dānaphalaṃ vidyāṃ māyāṃ kupyaṃ dhanaṃ vasu
devatvam amareśatvaṃ rasāyanavayaḥ\var{vayaḥ \Wthree \lem cayaḥ \EdLo} kriyāḥ\var{kriyāḥ \EdAd \lem kriyām \EdLo \Wthree}
meruprapatanaṃ yajñaṃ jalāgnyāveśanaṃ tathā
śrāddhānāṃ sarva\var{sarva \EdAd\lem śakti \EdLo}dānānāṃ phalāni niyamāṃs tathā
tathopavāsāt pūrtāc ca devapitrarcanād\var{devapitrarcanād \EdLo \lem devatābhyarcanād \EdAd} api
tebhyas tebhyaś\var{tebhyas tebhyaś \EdAd \lem atitibhyaś \EdLo \Wthree} ca karmabhya upasṛṣṭo 'bhivāñchati 
cittam\var{cittam \EdAd \lem vighnam \EdLo} itthaṃ vartamānaṃ\var{vartamānaṃ \EdAd \lrm pravarte taṃ } yatnād yogī nivartayet
brahmasaṅgi manaḥ kurvann upasargaiḥ\var{upasargaiḥ \EdLo \lem upasargāt \EdAd} pramucyate %
\comment{Mārkaṇḍeyapurāṇa 40.16} 
\stopsloka

iti.

skandapurāṇe

\startsloka
yadaibhir antarāyair na kṣipyate 'sya hi mānasam
tadāgre tam avāpnoti paraṃ brahmātidurlabham 
\stopsloka

yogabhāskare

\startsloka
sāttvikīṃ dhṛtim ālambya yogī sattvena susthiraḥ
nirguṇaṃ manasā dhyāyann upasargaiḥ pramucyate 
evaṃ yogam upāsīnaḥ śakrādipadanispṛhaḥ
siddhyādivāsanātyāgī jīvanmukto bhaven muniḥ
vistarasya bhiyā notkaḥ santi vighnā hy anekaśaḥ
dhyānena viṣṇuharayor vāraṇīyā hi yoginā iti
\stopsloka


%\begin{vsid}{#hp04_00}
\startsloka
{\bf
yāvan naiva praviśati caran māruto madhyamārge
yāvad bindur na bhavati dṛḍhaḥ prāṇavātaprabandhāt
yāvad dhyāne sahajasadṛśaṃ jāyate naiva tattvaṃ
tāvaj jñānaṃ vadati tad idaṃ dambhamithyāpralāpaḥ (4.114)}
\stopsloka

iti śrīsahajānandasantānacintāmaṇisvātmārāmayogīndraviracitāyāṃ haṭhapradīpikāyāṃ
samādhilakṣaṇaṃ nāma caturthopadeśaḥ.

ayogināṃ jñānaṃ nirākurvan yoginām eva jñānaṃ bhavatīty āha yāvad iti. madhyamārge suṣumnāyāṃ caran
gacchan mārutaḥ prāṇavāyuḥ yāvat yāvatkālaparyantaṃ na praviśati prakarṣeṇa brahmarandhraparyantaṃ
na viśati. brahmarandhraṃ gatasya sthairyād brahmarandhraṃ gatvā na sthirībhavatīty
arthaḥ. suṣumnāyām asañcaran vāyur asiddha ity ucyate. tad uktam amṛtasiddhau

\startsloka
yāvad vimārgago vāyur niścalo naiva madhyagaḥ
asiddhaṃ taṃ vijānīyād vāyuṃ karmavaśānugam 
\stopsloka

iti. prāṇayati jīvayati iti prāṇaḥ. sa cāsau vātaś ca prāṇavātaḥ tasya prabandhāt kumbhakena
sthirīkaraṇād bindur vīryaṃ dṛḍhaḥ sthiro na bhavati. prāṇavātasthairye bindusthairyam uktam
atraiva prāk.

\startsloka
manaḥsthairye sthiro vāyus tato binduḥ sthiro bhavet \comment{4.28}
\stopsloka

iti. tadabhāve tv asiddhatvaṃ yoginaḥ. uktam amṛtasiddhau

\startsloka
tāvad baddho 'py asiddho 'sau naraḥ sāṃsāriko mataḥ
yāvad bhavati dehastho rasendro brahmarūpakaḥ
asiddhaṃ taṃ vijānīyān naram abrahmacāriṇam
jarāmaraṇasaṅkīrṇaṃ sarvakleśasamāśrayam 
\stopsloka

iti. yāvat tattvaṃ cittaṃ dhyāne dhyeyacintane sahajasadṛśaṃ svābhāvikadhyeyākāravṛttipravāhavat
naiva jāyate naiva bhavati prāṇavātaprabandhād iti dehalīdīpanyāyenātrāpi
sambadhyate. vāyusthairye cittasthairyam uktam amṛtasiddhau

\startsloka
yadāsau mriyate\var{mriyate \lem śriyate check} vāyur madhyamāṃ madhyayogataḥ
tadā binduś ca cittaṃ ca mriyate vāyunā saha
\stopsloka

tadabhāve 'hy asiddhatvam uktam amṛtasiddhau

\startsloka
yāvat prasyandate cittaṃ bāhyābhyantaravastuṣu
asiddhaṃ tad vijānīyāc cittaṃ karmaguṇānvitam 
\stopsloka

iti. tāvad yaj jñānaṃ śābdaṃ vadati kaścit tad idaṃ jñānakathanaṃ dambhamithyāpralāpaḥ dambhena
jñānakathanenāhaṃ loke pūjyo bhaviṣyāmīti dhiyā mithyāpralāpo mithyābhāṣaṇaṃ dambhapūrvakaṃ
mithyābhāṣaṇam ity arthaḥ. prāṇabinducittānāṃ jayābhāve jñānasyābhāvāt saṃsṛtir durvārā. tad uktam
amṛtasiddhau

\startsloka
calaty eṣa yadā vāyus tadā binduś calaḥ smṛtaḥ
binduś calati yasyāṅge cittaṃ tasyaiva cañcalam
cale bindau cale citte cale vāyau ca sarvadā
jāyate mriyate lokaḥ satyaṃ satyam idaṃ vacaḥ 
\stopsloka

iti. yogabīje 'py uktam

\startsloka
cittaṃ pranaṣṭaṃ yadi bhāsate vai
tatra pratīto maruto 'pi nāśaḥ
na vā yadi syān na tu tasya śāstraṃ
nātmapratītir na gurur na mokṣaḥ \comment{Yogabīja 138}  iti.
\stopsloka

etena prāṇabindumanasāṃ jaye tu jñānadvārā yogino muktiḥ syād eveti sūcitam. tad uktam
amṛtasiddhau\comment{Amṛtasiddhi 7.6cd, 7.16ab, 7.7ab, 7.7cd, 7.16cd}

\startsloka
yām avasthāṃ vrajed vāyur bindus tām adhigacchati
yathā hi sādhyate vāyus tathā binduprasādhanam
mūrcchito harati vyādhiṃ vṛddhaḥ khecaratāṃ nayet
sarvasiddhikaro līno niścalo muktidāyakaḥ
yathāvasthā bhaved bindoś cittāvasthā tathā tathā 
\stopsloka

iti. nanu

\startsloka
yogās trayo mayā proktā nṝṇāṃ śreyovidhitsayā
jñānaṃ karma ca bhaktiś ca nopāyo 'nyo 'sti kutracit \comment{Bhāgavatapurāṇa 11.20.6}
\stopsloka

iti bhagavaduktās trayo mokṣopāyās teṣu satsu kathaṃ yoga eva mokṣopāyatvenokta iti cen na teṣāṃ
yogāṅgeṣv antarbhāvāt. tathā hi


\startsloka
ātmā vā are draṣṭayaḥ śrotavyo mantavyo nididhyāsitavyaḥ \comment{Bṛhad\-āraṇya\-kopani\-ṣat 2.4.5}
\stopsloka

iti śrutyā paramapuruṣārthasādhanātmasākṣātkārahetutayā śravaṇamanananididhyāsanāny uktāni. tatra
śravaṇamanane niyamāntargate svādhyāye 'ntarbhavataḥ. svādhyāyaś ca mokṣaśāstrāṇām adhyayanam. sa
ca tātparyārthanirṇayaparyavasāyī grāhyaḥ. tātparyārthanirṇayaś ca śravaṇamananābhyāṃ bhavatīti
śravaṇamananayoḥ svādhyāye 'ntarbhāvaḥ. niyamavivaraṇe yājñavalkyena

\startsloka
  siddhāntaśravaṇaṃ proktaṃ vedāntaśravaṇaṃ budhaiḥ \comment{Yajñavalkya 2.8}
\stopsloka

iti spaṣṭam eva śravaṇasya niyamāntargatir uktā.

\startsloka
adhītya vedaṃ sūtraṃ vā purāṇaṃ setihāsakam
padeṣv adhayayanaṃ yac ca sadābhyāso japaḥ smṛtaḥ \comment{Yajñavalkya 2.13} 
\stopsloka

iti yuktibhir anavaratam anucintanalakṣaṇasya sadābhyāsarūpasya mananasyāpi niyamāntargatir
uktā. vijātīyapratyayanirodhapūrvakasajātīyapra\-tyaya\-pravāharūpasya nididhyāsanasya uktalakṣaṇe
dhyāne 'ntarbhāvaḥ. tasyāpi tatparipākarūpasamādhinātmasākṣātkāradvārā mokṣahetutvam
īśvarārpaṇabuddhyā niṣkāmakarmānuṣṭhānalakṣaṇasya karmayogasya tapaḥsvādhyāyeśvarapraṇidhānāni
kriyāyogaḥ\comment{Yogasūtra 2.1} iti patañjaliprokte niyamāntargate kriyāyoge 'ntarbhāvaḥ. tatra
tapa uktam īśvaragītāyāṃ

\startsloka
upavāsaparākādi kṛcchracāndrāyaṇādibhiḥ
śarīraśoṣaṇaṃ prāhus tāpasās tapa uttamam \comment{Īśvaragītā 11.21} 
\stopsloka

iti. svādhyāyo 'pi tatroktaḥ

\startsloka
vedāntaśatarudrīyapraṇavādijapaṃ budhāḥ
sattvaśuddhikaraṃ puṃsāṃ svādhyāyaṃ paricakṣate \comment{Īśvaragītā 11.22} iti.
\stopsloka

īśvarapraṇidhānaṃ ca tatroktam

\startsloka
stutismaraṇapūjābhir vāṅmanaḥkāyakarmabhiḥ
suniścalā bhaved bhaktir etad īśvarapūjanam\comment{Īśvaragītā 11.29} 
\stopsloka

iti. kriyāyogaś ca paramparayā samādhinātmasākṣātkāradvāraiva mokṣahetur iti

\startsloka
  samādhibhāvanārthaḥ kleśatanūkaraṇārthaś ca \comment{Yogasūtra 2.2}
\stopsloka

ity uttarasūtreṇa spaṣṭīkṛtaṃ patañjalinā. bhajyate sevyate bhagavadākāram antaḥkaraṇaṃ
kriyate 'nayeti bhaktir iti karaṇavyutpattyā

\startsloka
śravaṇaṃ kīrtanaṃ viṣṇoḥ smaraṇaṃ pādasevanam
arcanaṃ vandanaṃ dāsyaṃ sakhyam ātmanivedanam \comment{Bhāgavatapurāṇa 7.5.23}
\stopsloka

iti navadhoktā sādhanabhaktir abhidhīyate. tasyā īśvarapraṇidhānarūpe niyame 'ntarbhāvaḥ. tasyāś ca
samādhihetutvaṃ coktam patañjalinā īśvarapraṇidhānād vā \comment{Yogasūtra 1.23} iti
īśvaraviṣayakāt praṇidhānād bhaktiviśeṣāt samādhilābhaḥ samādhiphalaṃ ca bhavatīti 
sūtrārthaḥ. bhajanam antaḥkaraṇasya bhagavadākāratārūpaṃ bhaktir iti bhāvavyutpattyā phalabhūtā
bhaktir ity abhidhīyate. saiva premabhaktir ity ucyate. tallakṣaṇam uktaṃ
nārāyaṇatīrthaiḥ

\startsloka
premabhaktiyogas tu īśvaracaraṇāravindaviṣayakaikāntikātyantikapremapravāho 'vicchinnaḥ
\stopsloka

iti. madhusūdanasarasvatībhis tu

\startsloka
dravībhāvapūrvikā manaso bhagavadākāratārūpā savikalpavṛttir bhaktiḥ \comment{bhaktirasāyaṇa 1.1 (vṛtti)}
\stopsloka

iti. tasyās tu śraddhābhaktidhyānayogād avehi \comment{Kaivalyopaniṣat 2} iti śruteḥ. bhaktyā mām
abhijānāti \comment{Bhagavadgītā 18.55} iti smṛteś ca ātmasākṣātkāradvārā mokṣahetutvam. bhaktās
tu sukhasyaiva puruṣārthatvād duḥkhāsambhinnaniratiśayasukhadhārārūpā premabhaktir eva puruṣārtha
ity āhuḥ. tasyās tu samprajñātasamādhāv antarbhāvaḥ. evaṃ ca aṣṭāṅgayogātiriktaṃ kim api
paramapuruṣārthasādhanaṃ nāstīti siddham.

\startsloka
  grāhyam eva viduṣāṃ hitaṃ yato
  bhāṣaṇaṃ mama yad apy asaṃskṛtam
  rakṣa gacchati payo 'nalāhitaṃ
  hy amba ity abhihitaṃ śiśor yathā

  sadarthadyotanakarī tamaḥstomavināśinī
  brahmānandena jyotsneyaṃ śivāṅghriyugale 'rpitā
\stopsloka

%iti śrīhaṭhayogapradīpikāvyākhyāyāṃ brahmānandakṛtāyāṃ jyotsnābhidhāyāṃ samādhinirūpaṇaṃ nāma
%caturthopadeśaḥ.

\stoplinenumbering\stoptext



%%% Local Variables:
%%% mode: context
%%% TeX-master: t
%%% End:
