\documentclass[10pt]{memoir}
\setstocksize{220mm}{155mm} 	        
\settrimmedsize{220mm}{155mm}{*}	
\settypeblocksize{170mm}{116mm}{*}	
\setlrmargins{18mm}{*}{*}
\setulmargins{*}{*}{1.2}
% \setlength{\headheight}{5pt}
\checkandfixthelayout[lines]
\linespread{1}
\setlength{\parskip}{0.3em}
\setlength\parindent{0pt}

\makepagestyle{HPed}
\makeoddhead{HPed}{\small{HP Transl. \& Comm.}}{}{\small{\today}}
\makeevenhead{HPed}{\small{HP Transl. \& Comm.}}{}{\small{\today}}
\makeoddfoot{HPed}{}{\small{\thepage}}{}
\makeevenfoot{HPed}{}{\small{\thepage}}{}

\usepackage[teiexport=tidy,poetry=verse]{ekdosis}
\usepackage{sanskrit-poetry,libertine,xcolor}
\usepackage[english]{babel}
\setlength{\vindent}{0pt}
\setvnum{}




%%%%%%%%%%%%%%%%%%%% THE  MSS         %%%%%%%%%%%%%%%%%%%%%%%%%%%

%%% Versions
\DeclareWitness{Vu}{\selectlanguage{english}Vulg}{Vulgate, i.e. Brahmānanda's version}[]           
\DeclareWitness{X}{\selectlanguage{english}X}{TenChapter Version, Jodhpur 02228 and 02225 (ed. Lonavla)}[]
\DeclareWitness{Six}{\selectlanguage{english}Ṣ}{SixChapterVersion, ``6ChapterHPms'', fragment of enlarged text, Jodhpur}[]
% Mss. in Geographical Groups
%%%% Varanasi mss (Sampūrṇānanda mss). V1 is Important
\DeclareWitness{V1}{\selectlanguage{english}V\textsubscript{1}}{Sampurnananda Library Sarasvati Bhavan 30109}[]
        \DeclareHand{V1ac}{V1}{\selectlanguage{english}V\rlap{\textsubscript{1}}\textsuperscript{ac}}[] % added by MD
        \DeclareHand{V1pc}{V1}{\selectlanguage{english}V\rlap{\textsubscript{1}}\textsuperscript{pc}}[] % added by MD
\DeclareWitness{V2}{\selectlanguage{english}V\textsubscript{2}}{Sampurnananda Library Sarasvati Bhavan 29869}[]
\DeclareWitness{V3}{\selectlanguage{english}V\textsubscript{3}}{Sampurnananda Library Sarasvati Bhavan 29899}[]
\DeclareWitness{V4}{\selectlanguage{english}V\textsubscript{4}}{Sampurnananda Library Sarasvati Bhavan 29937}[]
\DeclareWitness{V5}{\selectlanguage{english}V\textsubscript{5}}{Sampurnananda Library Sarasvati Bhavan 29938}[]
\DeclareWitness{V6}{\selectlanguage{english}V\textsubscript{6}}{Sampurnananda Library Sarasvati Bhavan 29991}[]
\DeclareWitness{V8}{\selectlanguage{english}V\textsubscript{8}}{Sampurnananda Library Sarasvati Bhavan 30014}[]
\DeclareWitness{V11}{\selectlanguage{english}V\textsubscript{11}}{Sampurnananda Library Sarasvati Bhavan 30029}[]
\DeclareWitness{V12}{\selectlanguage{english}V\textsubscript{12}}{Sampurnananda Library Sarasvati Bhavan 30030}[]
\DeclareWitness{V13}{\selectlanguage{english}V\textsubscript{13}}{Sampurnananda Library Sarasvati Bhavan 30031}[]
\DeclareWitness{V14}{\selectlanguage{english}V\textsubscript{14}}{Sampurnananda Library Sarasvati Bhavan 30050}[]
\DeclareWitness{V15}{\selectlanguage{english}V\textsubscript{15}}{Sampurnananda Library Sarasvati Bhavan 30051}[]
\DeclareWitness{V15pc}{\selectlanguage{english}V\rlap{\textsubscript{15}}\textsuperscript{pc}\space}{}[]
\DeclareWitness{V16}{\selectlanguage{english}V\textsubscript{16}}{Sampurnananda Library Sarasvati Bhavan 30052}[]
\DeclareWitness{V17}{\selectlanguage{english}V\textsubscript{17}}{Sampurnananda Library Sarasvati Bhavan 30053}[] % added by MD
\DeclareWitness{V16pc}{\selectlanguage{english}V\rlap{\textsubscript{16}}\textsuperscript{pc}\space}{}[]
\DeclareWitness{V18}{\selectlanguage{english}V\textsubscript{18}}{Sampurnananda Library Sarasvati Bhavan 30064}[]
\DeclareWitness{V19}{\selectlanguage{english}V\textsubscript{19}}{Sampurnananda Library Sarasvati Bhavan 30069}[]
\DeclareWitness{V21}{\selectlanguage{english}V\textsubscript{21}}{Sampurnananda Library Sarasvati Bhavan 30104}[]
\DeclareWitness{V22}{\selectlanguage{english}V\textsubscript{22}}{Sampurnananda Library Sarasvati Bhavan 30110}[]
\DeclareWitness{V25}{\selectlanguage{english}V\textsubscript{25}}{Sampurnananda Library Sarasvati Bhavan 30122}[]
\DeclareWitness{V26}{\selectlanguage{english}V\textsubscript{26}}{Sampurnananda Library Sarasvati Bhavan 30123}[]
\DeclareWitness{V28}{\selectlanguage{english}V\textsubscript{28}}{Sampurnananda Library Sarasvati Bhavan 30136}[]
\DeclareWitness{W4}{\selectlanguage{english}W\textsubscript{4}}{Wai 399-6171}[]

%%%%%%%%%%%%%%%%%%%%%%%%%%%%%%%%%
%%% Jammu & Kaschmir
\DeclareWitness{K1}{\selectlanguage{english}K\textsubscript{1}}{Raghunātha Temple Library 4383}[settlement=Jammu]
        \DeclareWitness{K1ac}{\selectlanguage{english}K\rlap{\textsubscript{1}}\textsuperscript{ac}\space}{}[]
        \DeclareWitness{K1pc}{\selectlanguage{english}K\rlap{\textsubscript{1}}\textsuperscript{pc}\space}{}[]
\DeclareWitness{L1}{\selectlanguage{english}L\textsubscript{1}}{SOAS RE 43454}[settlement=Jammu]
% More details? Catalogue number? L1 And C1 very close (and come from same region)
%%%%%%%%%%%%%%%%%%%%%%%%%%%%%%%%
% Jodhpur
% J10 is important
\DeclareWitness{J10}{\selectlanguage{english}J\textsubscript{10}}{MSPP Jodhpur 2230}[]
        \DeclareHand{J10ac}{J10}{\selectlanguage{english}J\rlap{\textsubscript{10}}\textsuperscript{ac}}[] % modified by MD
        \DeclareHand{J10pc}{J10}{\selectlanguage{english}J\rlap{\textsubscript{10}}\textsuperscript{pc}}[] % modified by MD
\DeclareWitness{J1}{\selectlanguage{english}J\textsubscript{1}}{Jodhpur 02231}[]
\DeclareWitness{J2}{\selectlanguage{english}J\textsubscript{2}}{Jodhpur 02232}[]   
\DeclareWitness{J3}{\selectlanguage{english}J\textsubscript{3}}{Jodhpur 02233}[]
\DeclareWitness{J4}{\selectlanguage{english}J\textsubscript{4}}{Jodhpur 02234}[]
        \DeclareWitness{J4ac}{\selectlanguage{english}J\rlap{\textsubscript{4}}\textsuperscript{ac}\space}{MSPP Jodhpur 02234}[]
        \DeclareWitness{J4pc}{\selectlanguage{english}J\rlap{\textsubscript{4}}\textsuperscript{pc}\space}{MSPP Jodhpur 02234}[]
\DeclareWitness{J5}{\selectlanguage{english}J\textsubscript{5}}{Jodhpur 02235}[]  % 4 chapters, 34 jpgs,   long colophon, missing lines in the beginning.
\DeclareWitness{J6ac}{\selectlanguage{english}J\rlap{\textsubscript{6}}\textsubscript{ac}}{Jodhpur 02237}[]  % 4 chapters, 49 jpgs,   1st folio: idaṃ gulābarāyasya
% tulasīrāmaśarmmaṇaḥ putrasya pustakaṃ ...        End: iti śrīsahajānandasantānacintāmaṇisvātmārāmaviracitāyāṃ ..
% saṃvat 1802   (more consistent text)
\DeclareWitness{J6pc}{\selectlanguage{english}J\rlap{\textsubscript{6}}\textsubscript{pc}}{Jodhpur 02237}[] 
\DeclareWitness{J7}{\selectlanguage{english}J\textsubscript{7}}{Jodhpur 02241}[]  % 4 chapters, 41 jpgs
\DeclareWitness{J8}{\selectlanguage{english}J\textsubscript{8}}{Jodhpur 23709}[]  % 4 chapters,  87 jpgs.   saṃvat 1724
\DeclareHand{J8ac}{J8}{\selectlanguage{english}J\rlap{\textsubscript{8}}\textsuperscript{ac}}[]  % changed by MD
\DeclareHand{J8pc}{J8}{\selectlanguage{english}J\rlap{\textsubscript{8}}\textsuperscript{pc}}[]  % changed by MD
\DeclareWitness{J9}{\selectlanguage{english}J\textsubscript{9}}{Jodhpur 02224}[]  %  fragment, 20 jpgs.
\DeclareWitness{J11}{\selectlanguage{english}J\textsubscript{11}}{Jodhpur 23532}[]
\DeclareWitness{J12}{\selectlanguage{english}J\textsubscript{12}}{Jodhpur 18552}[] 
\DeclareWitness{J13}{\selectlanguage{english}J\textsubscript{13}}{Jodhpur 02229}[]  %  5 chapters, 93 jpgs.
\DeclareWitness{J14}{\selectlanguage{english}J\textsubscript{14}}{Jodhpur 02239}[]  %  4 chapters
\DeclareWitness{J15}{\selectlanguage{english}J\textsubscript{15}}{Jodhpur 9732A}[]
\DeclareWitness{J17}{\selectlanguage{english}J\textsubscript{17}}{Jodhpur 3013}[]
% Haṭhapradīpikā with (non-Sanskrit) Bhāṣya RORI Jodhpur ACC.NO.18552
%  Haṭhapradīpikā with (non-Sanskrit) commentary, RORI Alwar 952, 4 chapters,  colophon of the comm:
% iti śrīlāhorīmiśravrajabhūṣanaviracitāyāṃ bhāvārthadīpikāyāṃ caturthodhyāya ..    
%  Haṭhapradīpikā (5 chapter) MSPP Jodhpur ACC.NO.02229/

%%%%%%%%%%        Bodleian, Oxford
\DeclareWitness{B1}{\selectlanguage{english}B\textsubscript{1}}{Bodleian Library No. d.457(8)}[settlement=Oxford]
\DeclareWitness{B2}{\selectlanguage{english}B\textsubscript{2}}{Bodleian Library No. d.458(1)}[settlement=Oxford]
\DeclareWitness{B3}{\selectlanguage{english}B\textsubscript{3}}{Bodleian Library No. d.458(9)}[settlement=Oxford]

%%%%%%%%%%%   Chandigarh
\DeclareWitness{C1}{\selectlanguage{english}C\textsubscript{1}}{Lalchand M-2080}[]%L1 And C1 very close (and come from same region)
\DeclareWitness{C2}{\selectlanguage{english}C\textsubscript{2}}{Lalchand M-6065}[]
\DeclareWitness{C3}{\selectlanguage{english}C\textsubscript{3}}{Lalchand M-1293}[]
\DeclareWitness{C4}{\selectlanguage{english}C\textsubscript{4}}{Lalchand M-2081}[]
\DeclareWitness{C4ac}{\selectlanguage{english}C\rlap{\textsubscript{4}}\textsuperscript{ac}\space}{}[]
\DeclareWitness{C4pc}{\selectlanguage{english}C\rlap{\textsubscript{4}}\textsuperscript{pc}\space}{}[]
\DeclareWitness{C5}{\selectlanguage{english}C\textsubscript{5}}{Lalchand M-2082}[]%doesn't have chapter 1
\DeclareWitness{C6}{\selectlanguage{english}C\textsubscript{6}}{Lalchand M-2089}[]
\DeclareWitness{C7}{\selectlanguage{english}C\textsubscript{7}}{Lalchand M-6494}[]
\DeclareWitness{C8}{\selectlanguage{english}C\textsubscript{8}}{Lalchand M-2091}[]
\DeclareWitness{C8pc}{\selectlanguage{english}C\rlap{\textsubscript{8}}\textsuperscript{pc}\space}{}[]
\DeclareWitness{C9}{\selectlanguage{english}C\textsubscript{9}}{Lalchand M-4530}[]

% %%%%%%%%%%        Nepalese
\DeclareWitness{N1}{\selectlanguage{english}N\textsubscript{1}}{NGMPP A1400-2}[]
\DeclareWitness{N2}{\selectlanguage{english}N\textsubscript{2}}{NGMPP B 39-19}[]
\DeclareWitness{N3}{\selectlanguage{english}N\textsubscript{3}}{NGMPP B 62-20}[]
\DeclareWitness{N5}{\selectlanguage{english}N\textsubscript{5}}{NGMPP A60-15 + A61-1}[]
\DeclareWitness{N6}{\selectlanguage{english}N\textsubscript{6}}{NGMPP A61-6}[]
\DeclareWitness{N9}{\selectlanguage{english}N\textsubscript{9}}{NGMPP A62-33}[]
\DeclareWitness{N10}{\selectlanguage{english}N\textsubscript{10}}{NGMPP A62-37}[]
\DeclareWitness{N11}{\selectlanguage{english}N\textsubscript{11}}{NGMPP A63-15}[]
\DeclareWitness{N12}{\selectlanguage{english}N\textsubscript{12}}{NGMPP A939-19}[]
\DeclareWitness{N13}{\selectlanguage{english}N\textsubscript{13}}{NGMPP A1378-18}[]
\DeclareWitness{N16}{\selectlanguage{english}N\textsubscript{16}}{NGMPP B39-20}[]
\DeclareWitness{N17}{\selectlanguage{english}N\textsubscript{17}}{NGMPP B 111-10}[]
\DeclareWitness{N18}{\selectlanguage{english}N\textsubscript{18}}{NGMPP E 929-3}[]
\DeclareWitness{N19}{\selectlanguage{english}N\textsubscript{19}}{NGMPP E-1528-1 / E-1527-7(4)}[]
\DeclareWitness{N20}{\selectlanguage{english}N\textsubscript{20}}{NGMPP E 2037-13 }[]
\DeclareWitness{N21}{\selectlanguage{english}N\textsubscript{21}}{NGMPP E 2097-31}[]
\DeclareWitness{N22}{\selectlanguage{english}N\textsubscript{22}}{NGMPP G 4-4}[]
\DeclareWitness{N23}{\selectlanguage{english}N\textsubscript{23}}{NGMPP G 25-2}[]
\DeclareWitness{N24}{\selectlanguage{english}N\textsubscript{24}}{NGMPP G 190-16}[]
\DeclareWitness{N24ac}{\selectlanguage{english}N\rlap{\textsubscript{24}}\textsuperscript{ac}\space}{}[]
\DeclareWitness{N24pc}{\selectlanguage{english}N\rlap{\textsubscript{24}}\textsuperscript{pc}\space}{}[]

\DeclareWitness{P28}{\selectlanguage{english}P\textsubscript{28}}{BORI 399-1895-1902}[]

%%%%%   Mysore
\DeclareWitness{M1}{\selectlanguage{english}M\textsubscript{1}}{P-5682/4}[]
%%%%%   Tübingen
\DeclareWitness{Tü}{\selectlanguage{english}Tü}{Ma I 339}[]
%%%%%%%%%%
\DeclareWitness{YC}{\selectlanguage{english}YC}{Yogacintāmaṇi}[]
\DeclareWitness{ceteri}{\selectlanguage{english}cett.}{ceteri}[]

%%%%%%%%%% Mss with Commentary
\DeclareWitness{A1}{\selectlanguage{english}A\textsubscript{1}}{Alwar 952}[]


%%%%%%%%%%%%%%%%%%%%%%%%%%%%%%%%%%%%%%%%%%%
%List of all Sigla:
%A1,B1,B2,B3,C1,C2,C3,C4,C6,C7,C8,C9,J1,J2,J3,J4,J10,J13,J14,J15,J17,L1,M1,N3,N5,N6,N9,N10,N11,N12,N13,N16,N17,N19,N20,N21,N22,N23,N24,Tü,V1,V2,V3,V4,V5,V6,V8,V11,V19,V22,V26,Vu
%%%%%%%%%%%%%%%%%%%%%%%%%%%%%%%%%%%%%%%%%%%

\DeclareShorthand{x}{\selectlanguage{english}δ}{J10,J17,N17,P28,W4}


%%% Local Variables:
%%% mode: latex
%%% TeX-master: t
%%% End:

%
%%%%%                   Abbreviation for the printed apparatus,        xml interface needed
%%%%%                   (synonyms in same line)

% Macro for Editing Abbrevs.
%\def\om{\textrm{\footnotesize \textit{omitted in}\ }} %prints om. for omitted in apparatus
%\def\korr{\textrm{\footnotesize \textit{em.}\ }} %prints em. for emended in apparatus
%\def\conj{\textrm{\footnotesize \textit{conj.}\ }} %prints conj. for conjectured in apparatus


\def\eyeskip{\textrm{{ab.\,oc. }}}   
\def\aberratio{\textrm{{ab.\,oc. }}}
\def\ad{\textrm{{ad}}}   
\def\add{\textrm{{add.\ }}}
\def\ann{\textrm{{ann.\ }}}
\def\ante{\textrm{{ante }}}
\def\post{\textrm{{post }}}
%\def\ceteri{cett.\,}             % for simplifying the apparatus in print                  
\def\codd{\textrm{{codd.\ }}}   %  the same
\def\conj{\textrm{{coni.\ }}}  
\def\coni{\textrm{{coni.\ }}}
\def\contin{\textrm{{contin.\ }}}
\def\corr{\textrm{{corr.\ }}}
\def\del{\textrm{{del.\ }}}
\def\dub{\textrm{{ dub.\ }}}
\def\emend{\textrm{{emend.\ }}}
\def\expl{\textrm{{explic.\ }}}   
\def\explicat{\textrm{{explic.\ }}}
\def\fol{\textrm{{fol.\ }}}         
\def\foll{\textrm{{foll.\ }}}
\def\gloss{\textrm{{glossa ad }}}
\def\ins{\textrm{{ins.\ }}}          \def\inseruit{\textrm{{ins.\ }}}
\def\im{{\kern-.7pt\lower-1ex\hbox{\textrm{\tiny{\emph{i.m.}}}\kern0pt}}}
\def\inmargine{{\kern-.7pt\lower-.7ex\hbox{\textrm{\tiny{\emph{i.m.}}}\kern0pt}}}
\def\intextu{{\kern-.7pt\lower-.95ex\hbox{\textrm{\tiny{\emph{i.t.}}}\kern0pt}}}%\textrm{\scriptsize{i.t.\ }}}               
\def\indist{\textrm{{indis.\ }}}          \def\indis{\textrm{{indis.\ }}}
\def\iteravit{\textrm{{iter.\ }}}          \def\iter{\textrm{{iter.\ }}}  
\def\lectio{\textrm{{lect.\ }}}             \def\lec{\textrm{{lect.\ }}}
\def\leginequit{\textrm{{l.n. }}}         \def\legn{\textrm{{l.n. }}}         \def\illeg{\textrm{{l.n. }}}
\def\om{\textrm{{om. }}}
\def\primman{\textrm{{pr.m.}}}
\def\prob{\textrm{{prob.}}}
\def\rep{\textrm{{repetitio }}}
% \def\secundamanu{\textrm{\scriptsize{s.m.}}}
% \def\secm{{\kern-.6pt\lower-.91ex\hbox{\textrm{\tiny{\emph{s.m.}}}\kern0pt}}}%   \textrm{\scriptsize{s.m.}}}
\def\sequentia{\textrm{{seq.\,inv.\ }}}         \def\seqinv{\textrm{{seq.\,inv.\ }}} \def\order{\textrm{{seq.\,inv.\ }}}
\def\supralineam{{\kern-.7pt\lower-.91ex\hbox{\textrm{\tiny{\emph{s.l.}}}\kern0pt}}} %\textrm{\scriptsize{s.l.}}}
\def\interlineam{{\kern-.7pt\lower-.91ex\hbox{\textrm{\tiny{\emph{s.l.}}}\kern0pt}}}   %\textrm{\scriptsize{s.l.}}}
\def\vl{\textrm{v.l.}}   \def\varlec{\textrm{v.l.}} \def\varialectio{\textrm{v.l.}}
\def\vide{\textrm{{cf.\ }}}           \def\cf{\textrm{{cf.\ }}}
\def\videtur{\textrm{{vid.\,ut}}}
\def\crux{\textup{[\ldots]} }
\def\cruxx{\textup{[\ldots]}}
\def\unm{\textit{unm.}}        % unmetrical
%%%%%%%%%%%%%%%%%%%%%%%%%%%%%%%%%%%%



%%% Local Variables:
%%% mode: latex
%%% TeX-master: t
%%% End:

% additions/changes 2024-07-04 mm:
\TeXtoTEIPat{\lb}{<lb/>}
\TeXtoTEIPat{\begin {quote}}{<q>}
  \TeXtoTEIPat{\end {quote}}{</q>}
\TeXtoTEIPat{\begin {enumerate}}{<list rend="numbered">}
  \TeXtoTEIPat{\end {enumerate}}{</list>}
\TeXtoTEI{item}{item}

% additions/changes 2024-07-01 mm:
\TeXtoTEIPat{\unavbl {#1}}{<note type="foliolost">Folio lost in <ref>#1</ref></note>}
\TeXtoTEIPat{\NotIn {#1}}{<note type="omission">Omitted in <ref>#1</ref></note>}
\TeXtoTEI{graus}{span}[type="altrec"]
\TeXtoTEI{grau}{span}[type="altrec"]

% addition 2024-03-15 MD
\TeXtoTEI{manuref}{}

\TeXtoTEIPat{\alphaOne}{α<hi rend="sub">1</hi>}% N3
\TeXtoTEIPat{\alphaTwo}{α<hi rend="sub">2</hi>}% J5
\TeXtoTEIPat{\alphaThree}{α<hi rend="sub">3</hi>}% G4
\TeXtoTEIPat{\betaOne}{β<hi rend="sub">1</hi>}% P11
\TeXtoTEIPat{\betaTwo}{β<hi rend="sub">2</hi>}% C6
\TeXtoTEIPat{\betaOmega}{β<hi rend="sub">ω</hi>}% V3
\TeXtoTEIPat{\gammaOne}{γ<hi rend="sub">1</hi>}% N23
\TeXtoTEIPat{\gammaTwo}{γ<hi rend="sub">2</hi>}% J7
\TeXtoTEIPat{\deltaOne}{δ<hi rend="sub">1</hi>}% V19
\TeXtoTEIPat{\deltaTwo}{δ<hi rend="sub">2</hi>}% K3
\TeXtoTEIPat{\deltaThree}{δ<hi rend="sub">3</hi>}% C7
\TeXtoTEIPat{\deltaOmega}{δ<hi rend="sub">ω</hi>}% J6
\TeXtoTEIPat{\epsilonOne}{ε<hi rend="sub">1</hi>}% P15
\TeXtoTEIPat{\epsilonTwo}{ε<hi rend="sub">2</hi>}% N19
\TeXtoTEIPat{\epsilonThree}{ε<hi rend="sub">3</hi>}% V15
\TeXtoTEIPat{\epsilonFour}{ε<hi rend="sub">4</hi>}% J11
\TeXtoTEIPat{\epsilonOmega}{ε<hi rend="sub">ω</hi>}% N26
\TeXtoTEIPat{\etaOne}{η<hi rend="sub">1</hi>}% V1
\TeXtoTEIPat{\etaTwo}{η<hi rend="sub">2</hi>}% J10
\TeXtoTEIPat{\etaOmega}{η<hi rend="sub">ω</hi>}% N9

% addition 2023-12-11 MD:
\TeXtoTEIPat{\begin {metre}[#1]}{<note type="metre" target="##1">}
\TeXtoTEIPat{\end {metre}}{</note>}
\TeXtoTEIPat{\texttheta}{θ}

% change 2023-12-05 mm
\TeXtoTEI{teimute}{} 

% changes/additions 2023-11-27 MM:
\TeXtoTEIPat{\medialink {#1}{#2}}{<ref target="resources/#2">#1</ref>}

% changes/additions 2023-10-25 MM:
% new Sigla
\TeXtoTEIPat{\textAlpha}{Α}
\TeXtoTEIPat{\textalpha}{α}
\TeXtoTEIPat{\textBeta}{Β}
\TeXtoTEIPat{\textbeta}{β}
\TeXtoTEIPat{\textGamma}{Γ}
\TeXtoTEIPat{\textgamma}{γ}
\TeXtoTEIPat{\textDelta}{Δ}
\TeXtoTEIPat{\textdelta}{δ}
\TeXtoTEIPat{\textEpsilon}{Ε}
\TeXtoTEIPat{\textepsilon}{ε}
\TeXtoTEIPat{\textEta}{Η}
\TeXtoTEIPat{\texteta}{η}
\TeXtoTEIPat{\textChi}{Χ}
\TeXtoTEIPat{\textchi}{χ}
\TeXtoTEIPat{\textOmega}{Ω}
\TeXtoTEIPat{\textomega}{ω}

%new environments
\TeXtoTEIPat{\begin {postmula}[#1]}{<div type="postmula" xml:id="#1">} %%% changed 2024-07-01 mm
  \TeXtoTEIPat{\end {postmula}}{</div>}  %%% changed 2024-07-01 mm
  
\TeXtoTEIPat{\begin {altpostmula}[#1]}{<div type="altrec"><div type="postmula" xml:id="#1">} %%% added 2024-07-03 md
  \TeXtoTEIPat{\end {altpostmula}}{</div></div>} %%% added 2024-07-03 md

\TeXtoTEIPat{\begin {altava}[#1]}{<div type="altrec"><div type="avataranika" xml:id="#1">} %%% changed 2024-07-01 mm
  \TeXtoTEIPat{\end {altava}}{</div></div>} %%% changed 2024-07-01 mm

\TeXtoTEIPat{\sgwit {#1}}{<note type="inlineref"><ref>#1</ref></note>}

% changes/additions 2023-10-12 MM:
\TeXtoTEIPat{\\.}{}

% changes/additions 2023-08-15 MD:
\TeXtoTEIPat{\lineom {#1}{#2}}{<note type="omission">#1 omitted in <ref>#2</ref></note>}
%\TeXtoTEIPat{\startgray}{} %%% changed 2023-12-05 mm; not used 2024-03-26 MD
%\TeXtoTEIPat{\endgray}{} %%% changed 2023-12-05 mm; not used 2024-03-26 MD

% additions/changes 2023-06-05 mm:
%\TeXtoTEIPat{\lineom {#1}}{<note type="omission">Line omitted in <ref>#1</ref></note>}

% additions 2023-04-16 MD:
\TeXtoTEIPat{\,}{ }

% additions 2023-04-13 mm:
\TeXtoTEIPat{\begin {versinnote}}{<lg>}
  \TeXtoTEIPat{\end {versinnote}}{</lg>}

% additions 2023-04-05 MD:
\TeXtoTEIPat{\begin {testimonia}[#1]}{<note type="testimonia" target="##1">}
  \TeXtoTEIPat{\end {testimonia}}{</note>}
\TeXtoTEI{devnote}{s}[xml:lang="sa-deva"]

% app in philcomm und testimonia %%% added MM 2023-12-02
\TeXtoTEI{var}{note}[type="appinnote"]


\TeXtoTEI{anm}{note}[type="memo"] %% change 2023-04-16 MD
\TeXtoTEI{Anm}{note}[type="memo"] %% change 2023-12-05 MM
\TeXtoTEIPat{\startverse}{} %%% marked for change 2023-04-13 mm
\TeXtoTEIPat{\endverse}{} %%% marked for change 2023-04-13 mm
\TeXtoTEIPat{\newpage}{}
\TeXtoTEIPat{\marmas}{ } % changed 2024-03-17 MD
\TeXtoTEIPat{\marma}{}
\TeXtoTEIPat{\vin}{} % added by MD 2023-11-14

%%% modify environments and commands
%%% TEI mapping
% additions/changes 2022-06-07 mm:
\TeXtoTEIPat{ \& }{ &amp; }

% additions/changes 2022-06-01 mm:
\TeXtoTEI{skp}{seg}[type="deva-ignore"]
\TeXtoTEI{skm}{seg}[type="ltn-ignore"]

\TeXtoTEIPat{\rlap {#1}}{#1}

% additions/changes 2022-04-06 mm:
%\TeXtoTEI{sgwit}{ref}
\TeXtoTEI{textdev}{s}[xml:lang="sa-deva"]
\TeXtoTEIPat{\begin {col}[#1]}{<div type="colophon" xml:id="#1">}
  \TeXtoTEIPat{\end {col}}{</div>}
\TeXtoTEIPat{\begin {ava}[#1]}{<div type="avataranika" xml:id="#1">} %%% changed 2024-07-01 mm
  \TeXtoTEIPat{\end {ava}}{</div>} %%% changed 2024-07-01 mm
												   
\TeXtoTEIPat{\outdent}{}
\TeXtoTEIPat{\startaltrecension}{} %%% changed 2023-12-05 mm
\TeXtoTEIPat{\endaltrecension}{} %%% changed 2023-12-05 mm
\TeXtoTEIPat{\startaltnormal}{} % added by MD 2023-11-14 %%% changed 2023-12-05 mm
\TeXtoTEIPat{\endaltnormal}{} % added by MD 2023-11-14 %%% changed 2023-12-05 mm
\TeXtoTEIPat{\begin {alttlg}[#1]}{<div type="altrec"><lg xml:id="#1">}
  \TeXtoTEIPat{\end {alttlg}}{</lg></div>}



% additions/changes 2022-03-12 mm:
\TeXtoTEIPat{\begin {tlg}[#1]}{<lg xml:id="#1">}
  \TeXtoTEIPat{\end {tlg}}{</lg>}

\TeXtoTEIPat{\begin {translation}[#1]}{<note type="translation" target="##1">}
  \TeXtoTEIPat{\end {translation}}{</note>}
\TeXtoTEIPat{\begin {philcomm}[#1]}{<note type="philcomm" target="##1">}
  \TeXtoTEIPat{\end {philcomm}}{</note>}
\TeXtoTEIPat{\begin {sources}[#1]}{<note type="sources" target="##1">}
  \TeXtoTEIPat{\end {sources}}{</note>}


\TeXtoTEIPat{\begin {marma}[#1]}{<note type="marma" target="##1">}
  \TeXtoTEIPat{\end {marma}}{</note>}

\TeXtoTEIPat{\begin {jyotsna}[#1]}{<note type="jyotsna" target="##1">}
  \TeXtoTEIPat{\end {jyotsna}}{</note>}

\EnvtoTEI{description}{list}
\EnvtoTEI{itemize}{list}
\TeXtoTEIPat{\item [#1]}{<label>#1</label>\item}

\TeXtoTEI{tl}{l}
\TeXtoTEI{myfn}{note}[type="myfn"]
\TeXtoTEIPat{\getsiglum {#1}}{<ref target="##1"/>}

\TeXtoTEI{SetLineation}{}
\TeXtoTEI{noindent}{}
\TeXtoTEI{subsection*}{}

\TeXtoTEI{rlap}{}

% end additions/changes
% \TeXtoTEIPat{\skp {#1}}{#1}
% \TeXtoTEIPat{\skm {#1}}{}

\TeXtoTEIPat{\begin {prose}}{<p>}
  \TeXtoTEIPat{\end {prose}}{</p>}

\TeXtoTEIPat{\begin {tlate}}{<p>}
  \TeXtoTEIPat{\end {tlate}}{</p>}

\TeXtoTEI{emph}{hi}
\TeXtoTEI{bigskip}{}
% \TeXtoTEI{/}{|}
\TeXtoTEI{tl}{l}
\TeXtoTEIPat{english}{}
%\TeXtoTEIPat{-}{ } %% change 2023-04-16 MD
%\TeXtoTEIPat{°}{} %% change 2023-04-16 MD
\TeXtoTEIPat{\textcolor {#1}{#2}}{<hi rend="#1">#2</hi>}

% \TeXtoTEIPat{\eyeskip}{}
% \TeXtoTEIPat{\aberratio}{}
% \TeXtoTEIPat{\ad}{}
\TeXtoTEIPat{\add}{<hi rend="italic">add.</hi>} %% change 2023-04-16 MD
% \TeXtoTEIPat{\ann}{}
\TeXtoTEIPat{\ante}{<hi rend="italic">ante</hi> } %% change 2023-04-16 MD
\TeXtoTEIPat{\post}{<hi rend="italic">post</hi> } %% change 2023-04-16 MD
% \TeXtoTEIPat{\codd}{}
% \TeXtoTEIPat{\conj }{}
% \TeXtoTEIPat{\contin}{}
% \TeXtoTEIPat{\corr}{}
% \TeXtoTEIPat{\del}{}
% \TeXtoTEIPat{\dub}{}
% \TeXtoTEIPat{\emend }{}
% \TeXtoTEIPat{\expl}{}
% \TeXtoTEIPat{\ȩxplicat}{}
% \TeXtoTEIPat{\fol}{}
% \TeXtoTEIPat{\gloss}{}
% \TeXtoTEIPat{\ins}{}
% \TeXtoTEIPat{\im}{}
% \TeXtoTEIPat{\inmargine}{}
% \TeXtoTEIPat{\intextu}{}
% \TeXtoTEIPat{\indist}{}
% \TeXtoTEIPat{\iteravit}{}
% \TeXtoTEIPat{\lectio}{}
% \TeXtoTEIPat{\leginequit}{}
% \TeXtoTEIPat{\legn}{}
% \TeXtoTEIPat{\illeg}{<hi rend="italic">illeg.</hi>}
\TeXtoTEIPat{\illeg}{<gap reason="illeg."/>} %%% change 2023-04-11 mm
% \TeXtoTEIPat{\om}{<hi rend="italic">om.</hi>}
\TeXtoTEIPat{\om}{<gap reason="om."/>} %%% change 2023-04-11 mm
% \TeXtoTEIPat{\primman}{}
% \TeXtoTEIPat{\prob}{}
% \TeXtoTEIPat{\rep}{}
% \TeXtoTEIPat{\sequentia}{}
% \TeXtoTEIPat{\supralineam}{}
% \TeXtoTEIPat{\interlineam}{}
\TeXtoTEIPat{\vl}{<hi rend="italic">v.l.</hi>}
% \TeXtoTEIPat{\vide}{}
% \TeXtoTEIPat{\videtur}{}
% \TeXtoTEIPat{\crux}{}
% \TeXtoTEIPat{\cruxxx}{}
\TeXtoTEIPat{\unm}{<hi rend="italic">unm.</hi>}
\TeXtoTEIPat{\lacuna}{<gap reason="lac."/>} % addition 2024-03-24 MD
\TeXtoTEIPat{\lost}{<gap reason="lost"/>} % addition 2024-06-24 MD

% List of Scholars
\DeclareScholar{nos}{nos}[
forename=HPP,
surname=Team]

% Nullify \selectlanguage in TEI as it has been used in
% \DeclareWitness but should be ignored in TEI.
\TeXtoTEI{selectlanguage}{}


\SetTEIxmlExport{autopar=false}

%%%%%%%%%%%

\SetTEIxmlExport{autopar=false}
\NewDocumentEnvironment{translation}{O{}}{\textcolor{blue}{\textbf{Translation:}}}{}
\NewDocumentEnvironment{philcomm}{O{}}{
	\textcolor{blue}{\textbf{Commentary:}}}{}
\NewDocumentEnvironment{metre}{O{}}{
	\textcolor{blue}{\textbf{Metre:}}}{} % added MD 2023-12-11
\NewDocumentEnvironment{sources}{O{}}{
	\textcolor{blue}{\textbf{Sources:}}\linebreak}{}
\NewDocumentEnvironment{testimonia}{O{}}{
	\textcolor{blue}{\textbf{Testimonia:}}\linebreak}{}
\NewDocumentEnvironment{versinnote}{O{}}{\begin{ekdverse}}{\end{ekdverse}}
%\newcommand{\var}[1]{\footnotesize\textup{#1}}
\newcommand{\medialink}[2]{\textcolor{green}{\underline{#1}}}
%\TeXtoTEIPat{\medialink {#1}{#2}}{<ref target="/images/#2">#1</ref>}

\NewDocumentCommand{\tl}{m}{#1}

\def\vl{\textit{v.l.}}
\def\var#1{{\footnotesize #1}}
\def\sl#1{\emph{#1}}

%%%%%%%%%%%%

\usepackage{textgreek}

\newcommand{\alphaOne}{\textalpha\textsubscript{1}}% N3
\newcommand{\alphaTwo}{\textalpha\textsubscript{2}}% J5
\newcommand{\alphaThree}{\textalpha\textsubscript{3}}% G4
\newcommand{\betaOne}{\textbeta\textsubscript{1}}% P11
\newcommand{\betaTwo}{\textbeta\textsubscript{2}}% C6
\newcommand{\betaOmega}{\textbeta\textsubscript{\textomega}}% V3
\newcommand{\gammaOne}{\textgamma\textsubscript{1}}% N23
\newcommand{\gammaTwo}{\textgamma\textsubscript{2}}% J7
\newcommand{\deltaOne}{\textdelta\textsubscript{1}}% V19
\newcommand{\deltaTwo}{\textdelta\textsubscript{2}}% K3
\newcommand{\deltaThree}{\textdelta\textsubscript{3}}% C7
\newcommand{\deltaOmega}{\textdelta\textsubscript{\textomega}}% J6
\newcommand{\epsilonOne}{\textepsilon\textsubscript{1}}% P15
\newcommand{\epsilonTwo}{\textepsilon\textsubscript{2}}% N19
\newcommand{\epsilonThree}{\textepsilon\textsubscript{3}}% V15
\newcommand{\epsilonFour}{\textepsilon\textsubscript{4}}% J11
\newcommand{\epsilonOmega}{\textepsilon\textsubscript{\textomega}}% N26
\newcommand{\etaOne}{\texteta\textsubscript{1}}% V1
\newcommand{\etaTwo}{\texteta\textsubscript{2}}% J10
\newcommand{\etaOmega}{\texteta\textsubscript{\textomega}}% N9

%%%%%%%%%%%%%%

\babelhyphenation{%
	Dattā-treya-yoga-śāstra
	Gorakṣa-śataka
	Haṭha-pra-dī-pikā
	Hātha-ratnā-valī
	Svātmā-rāma
	Śiva-saṃhitā
	Vasiṣṭha-saṃhitā
	Viveka-mārtaṇḍa
	Yukta-bhava-deva
	Yoga-cintā-maṇi
	Yoga-yājña-valkya}

\begin{document}
\pagestyle{HPed}
\begin{ekdosis}
\SetLineation{lineation = none,}

%\chapter*{Translation \& philological commentary}
%%%%%%%%%%
\subsection*{2.1}
\begin{translation}[hp02_001] 
Now, when [his] posture is steady, the disciplined yogi whose diet is good and measured should practise breath control in the way taught by [his] teacher.
\end{translation}

\begin{sources}[hp02_001]
\end{sources}

\begin{testimonia}[hp02_001]
\emph{Haṭharatnāvalī} 3.78

\begin{versinnote}
\tl{atha prāṇāyāmaḥ--\\+}
\tl{athāsane dṛḍhe yogī vaśī hitamitāśanaḥ |\\+}
\tl{gurūpadiṣṭamārgeṇa prānāyāmān samabhyaset ||\\!}
\end{versinnote}

\emph{Haṭhatattvakaumudī} 36.1

\begin{versinnote}
\tl{atha nāḍīśuddhiḥ--\\+}
\tl{tathā coktaṃ yogacandrikāyām--\\+}
\tl{athāsane dṛḍhībhūte vaśī hitamitāśanaḥ |\\+}
\tl{gurūpadiṣṭamārgeṇa prānāyāmān samabhyaset ||\\!}
\end{versinnote}

%Cf. Yogamārgaprakāśikā

%\begin{versinnote}
%\tl{athāsane dṛḍhībhūte suśobhanamaṭhe yadā |\\+}
%\tl{guruṃ natvā śivaṃ caiva prāṇāyāmaṃ tato 'bhyaset || 47 ||\\!}
%\end{versinnote}
\end{testimonia}

\begin{philcomm}[hp02_001]
%\emph{dṛḍho yogī} seems unlikely as \emph{dṛḍha} usually qualifies a technique, the body, etc. (not worth commenting on as alpha has dṛḍhe).
The \emph{Jyotsnā} (2.1) has the plural \emph{prāṇāyāmān}, which is supported by some manuscripts of the \textbeta, \textgamma\ and \textepsilon\ groups. The plural is possible here as it could refer to the different techniques of retention (\emph{kumbhaka}) taught later in the chapter. This is how Brahmānanda understands it (\emph{prāṇāyāmān vakṣyamāṇān}). The variation between singular and plural recurs through this chapter, and we have followed the \textalpha\ readings, which make good sense. In this case, since the verse is introducing the topic of breath control, the more general sense of the singular is appropriate.
\end{philcomm}

%%%%%%%%%%
\subsection*{2.2}
\begin{translation}[hp02_002]
When the wind moves, everything moves and, when it is still, everything is still, so the yogi attains motionlessness through restraining the breath.
\end{translation}

\begin{sources}[hp02_002]
\emph{Vivekamārtaṇḍa} 71

\begin{versinnote}
\tl{cale vāte calaṃ sarvaṃ niścale niścalaṃ tathā |\\+}
\tl{yogī sthāṇutvam āpnoti tato vāyunibandhanāt ||\\+}
\tl{\var{71d vāyunibandhanāt ] VT; vāyuṃ nibaṃdhayet A, vāyuṃ nirundhayet G}\\!}
\end{versinnote}

Cf.~\emph{Amanaska} 2.92

\begin{versinnote}
\tl{citte calati saṃsāro 'cale mokṣaḥ prajāyate |\\+}
\tl{tasmāc cittaṃ sthirīkuryād audāsīnyaparāyaṇaḥ ||\\!}
\end{versinnote}
\end{sources}

\begin{testimonia}[hp02_002]
\emph{Haṭharatnāvalī} 3.79 

\begin{versinnote}
\tl{cale vāte calaṃ cittaṃ niścale niścalaṃ tathā |\\+}
\tl{yogī sthāṇutvam āpnoti tato vāyuṃ nirundhayet ||\\!}
\end{versinnote}

\emph{Yogacintāmaṇi} f.~17r

\begin{versinnote}
\tl{tathā ca skandapurāṇe--\\+}
\tl{cale vāte calaṃ cittaṃ niścalaṃ niścale tathā |\\+}
\tl{yogī sthāṇutvam āpnoti tato vāyuṃ nirodhayet ||\\!}
\end{versinnote}

\end{testimonia}

\begin{philcomm}[hp02_002]
%The \textalpha\ manuscripts, as well as those of various other important groups, have \emph{sarvaṃ} instead of \emph{cittaṃ} in the first verse quarter. The stemma aside, the former makes better sense as a general statement (i.e., when the wind moves, everything moves, etc.). The reading \emph{citta} only seems to make sense if one understands \emph{sthāṇutvam} as referring to \emph{samādhi} (i.e., `motionlessness [of the mind]'). As far as we know, such a meaning of \emph{sthāṇutvam} is not attested elsewhere. Nonetheless, Brahmānanda glosses \emph{sthāṇutvam} as \emph{kāṣṭhavat} (`like a piece of wood'), implying that it refers to a yogi in \emph{samādhi} as this simile is found elsewhere in contexts of meditative absorption (e.g., \emph{Amanaska} 1.27).
%(JM: but we have sarvam now and cittam is clearly wrong; I think this note is unnecessary — sthāṇu can mean a piece of wood or a post and is a byword for motionlessness (but never means samādhi), like kāṣṭha, and we don't want to put people off reading the notes by including too many unedifying ones.) 
% JB: Since we often read against the stemma, I dont think it's very convincing to simply say we are reading with the stemma in cases where there are two well-attested plausible readings. In this particular case, cittaṃ is a plausible reading that's well attested among the collated mss., so I think we should firstly bring it to the reader's attention and then explain why we think the adopted reading is better on semantic grounds. Perhaps, my attempts to do the latter (by tidying up a previous note) can be improved?
%JM: should we mention that sthāṇu can mean Śiva?
In the second verse quarter, most of the \textalpha, \textbeta\ and \textgamma\ manuscripts have \emph{dṛḍhabandhanaṃ} instead of the reading we have adopted, \emph{niścalaṃ tathā}, which is supported by some \textbeta, \textdelta\ and \texteta\ manuscripts, and also the source text (i.e. the \emph{Vivekamārtaṇḍa}) and testimonia. The reading \emph{dṛḍhabandhanaṃ} appears to be a dittographical error and does not make sense when read with either \emph{sarvaṃ} (\textalpha) or \emph{cittaṃ} (\textgamma).

As noted by Brahmānanda, in addition to its primary meaning of immobility, \emph{sthāṇutvam} can also mean the state of being Śiva, for whom Sthāṇu is another name.

%The \alpha\ group is split between the reading we have adopted \emph{vāyunibandhanāt} (\alphaThree), which is close to \alphaOne\ (\emph{vāyuṃ nibandhanāt}), and the widely attested readings with a finite verb, including \alphaTwo \ (\emph{vāyuṃ nirundhayet}). 

In the fourth verse quarter, most witnesses have something similar to either \emph{vāyuṃ nibandhayet} and \emph{vāyuṃ nirodhayet}, which are likely to have arisen through confusion with 2.3d. The reading we have adopted (\emph{vāyu\-nibandhanāt}) makes better sense with the finite verb in the second verse quarter.%JM: again I wonder whether all this is worth putting in the edition, online or printed. It is our musings that got us to the text we have constituted, but is not very edifying for the reader. JM later note: if any are necessary, it's only the last sentence as the rest is info found in the apparatus
%JB: If edifying means instructive, then I rate our musings highly :) In 2.2b, we are reading against the stemmea (\textalpha\ dṛdhabandhanaṃ) and in 2.2d most witnesses have a finite verb, including \alphaTwo. But perhaps the note can be improved (I dont find it particularly easy to write this one).

\end{philcomm}

%%%%%%%%%%
\subsection*{2.3}
\begin{translation}[hp02_003]
As long as breath is found in the body, there is said to be life. Its leaving is death, so one should restrain the breath.
\end{translation}

\begin{sources}[hp02_003]
\emph{Vivekamārtaṇḍa} 72

\begin{versinnote}
\tl{yāvad vāyuḥ sthito dehe tāvaj jīvitam ucyate |\\+}
\tl{maraṇaṃ tasya niḥkrāntau tato vāyuṃ nirodhayet ||\\+}
\tl{\var{72a sthito ] sthiro A 72b jīvitam ] VAGBGPT; jīvanam GLGPk 72c niṣkrāntau ] VTA; niḥkrāṃtaṃ GBGP, niṣkrāntis GLGPk 72d vāyuṃ nirodhayet ] T; vāyunibandhanāt V, vāyuṃ nibandhayet ATvl, vāyuṃ nirundhayet G}\\!}
\end{versinnote}

Cf.~\emph{Mṛgendratantra} 1.11.20cd--22ab

\begin{versinnote}
\tl{vyāpārād yasya ceṣṭante śārīrāḥ pañca vāyavaḥ ||\\+}
\tl{prāṇāpānād ayas te tu bhinnā vṛtter na vastutaḥ |\\+}
\tl{vṛttiṃ leśān nigadato bharadvāja nibodha me ||\\+}
\tl{vṛttiḥ praṇayanaṃ nāma yat taj jīvanam ucyate |\\!}
\end{versinnote}
\end{sources}

\begin{testimonia}[hp02_003]

\emph{Haṭharatnāvalī} 3.80

\begin{versinnote}
\tl{yāvad vāyuḥ sthito dehe tāvaj jīvitam ucyate |\\+}
\tl{maraṇaṃ tasya niṣkrāntis tato vāyuṃ nirodhayet ||\\!}
\end{versinnote}

\emph{Yuktabhavadeva} 11.150

\begin{versinnote}
\tl{yāvad vāyuḥ sthito dehe tāvad dehaṃ na muñcati |\\+}
\tl{maraṇaṃ tasya niṣkrāntis tato vāyuṃ nirundhayet ||\\!}
\end{versinnote}

\end{testimonia}

\begin{philcomm}[hp02_003]
%Both \emph{jīvitaṃ} and \emph{jīvanaṃ} are possible in the second verse quarter. The former is well supported by the manuscripts of several important groups (\alphaOne's \emph{jītaviṃ} seems to be the result of a transposition of the \emph{vi} and \emph{ta}), the source text (the \emph{Vivekamārtaṇḍa}) and the testimonia.  

We believe that \emph{nirundhayet} was probably the original reading and that it was changed to the more correct form of \emph{nirodhayet} and, in some cases, confused with the final verse quarter of the previous verse and changed to \emph{nibandhanāt} (\textalpha) or \emph{nibandhayet} (\textbeta).%JM: necessary note?
%JB nirundhayet is not supported by alpha. 
% JB I've commmented on jīvitaṃ/jīvanam because it looks like its unsupported by alpha (alpha one has jītaviṃ). I now see that J5 has jīvitaṃ, so we could eliminate the note if we add this to the collation. But a note might still be helpful here because G4 is damaged at this place, so jīvitaṃ is not so well attested and jīvanaṃ is possible. 
\end{philcomm}

%%%%%%%%%%
\subsection*{2.4}
\begin{translation}[hp02_004]
When the channels are full of impurities, the breath does not go into the middle. How would the state beyond mind occur? How would perfection of the body arise?
\end{translation}

%\begin{sources}[hp02_004]
%\end{sources}

\begin{testimonia}[hp02_004]
\emph{Haṭharatnāvalī} 3.81

\begin{versinnote}
\tl{malākulāsu nāḍīṣu māruto naiva madhyagaḥ |\\+}
\tl{kathaṃ syād unmanībhāvaḥ kāyasiddhiḥ kathaṃ bhavet ||\\!}
\end{versinnote}
\end{testimonia}

%\begin{philcomm}[hp02_004]
%The reading \emph{kāyasiddhi} in the final verse quarter is supported by manuscripts of the \textalpha, \textbeta, \textepsilon\ and \texteta\ groups whereas other manuscripts have \emph{kāryasiddhi}. In the context of physical yoga, \emph{kāyasiddhi} makes better sense, as \emph{kāryasiddhi} can refer more generally to accomplishing anything.
%JM: necessary for the reader? Goes with the stemma, makes better sense, variant not particularly remarkable. But maybe I have become too parsimony-partial.
% JB. In this case I agree!
%\end{philcomm}

%%%%%%%%%%
\subsection*{2.5}
\begin{translation}[hp02_005]
Only when the entire impure network of channels is cleansed is the yogi able to control the breath.
\end{translation}

\begin{sources}[hp02_005]
\emph{Vivekamartāṇḍa} 76

\begin{versinnote}
\tl{śuddhim eti yadā sarvaṃ nāḍīcakraṃ malākulam |\\+}
\tl{tadaiva jāyate yogī prāṇasaṃgrahaṇe kṣamaḥ ||\\!}
\end{versinnote}
\end{sources}

\begin{testimonia}[hp02_005]
\emph{Yogacintāmaṇi} f.~90r

\begin{versinnote}
\tl{skandapurāṇe--\\+}
\tl{śuddhim eti yadā sarvaṃ nāḍīcakraṃ malākulam |\\+}
\tl{tadaiva jāyate yogī kṣamaḥ prāṇanibandhane ||\\!}
\end{versinnote}

\emph{Yuktabhavedeva} 7.11
\begin{versinnote}
\tl{śuddhim eti yadā sarvaṃ nāḍīcakraṃ malākulam |\\+}
\tl{tadaiva jāyate yogī prāṇasaṃgrahaṇe kṣamaḥ ||\\!}
\end{versinnote}

\end{testimonia}

\begin{philcomm}[hp02_005]
%We have understood the term \emph{nāḍīcakra} as the network of channels that originate from the bulb in the lower abdomen. This interpretation is consistent with \emph{Sārdhatriśatikālottara} 10.1–3:%JM there are many many possible refs for this, do we need one? And some systems have them coming from the navel, and some from the heart (Shaman's new article).

%\begin{versinnote}
%\tl{nāḍīcakraṃ paraṃ sūkṣmaṃ pravakṣyāmy anupūrvaśaḥ |\\+}
%\tl{nābher adhastād yat kandam aṅkurās tatra nirgatāḥ ||\\+}
%\tl{dvāsaptatisahasrāṇi nābhimadhye vyavasthitāḥ |\\+}
%\tl{tiryag ūrdhvam adhaś caiva vyāptaṃ nābheḥ samantataḥ ||\\+}
%\tl{cakravatsaṃsthitā nāḍyaḥ pradhānā daśa tāsu yāḥ |\\+}
%\tl{iḍā ca piṅgalā caiva suṣumnā ca tṛtīyakā ||\\!}
%\end{versinnote}

\end{philcomm}

%%%%%%%%%%
\subsection*{2.6}
\begin{translation}[hp02_006]
Therefore [the yogi] should regularly practise breath [retention] with a resolute mind, so that the Suṣumṇā is in good condition and the impurities dry up.%pure (sāttvika) doesn't work with impurities (malāḥ); resolute? grounded in resolve?
\end{translation}

\begin{sources}[hp02_006]
Cf. \emph{Gorakṣaśataka} 73cd–74ab

\begin{versinnote}
\tl{prāṇābhyāsas tataḥ kāryo nityaṃ sattvāsthayā dhiyā |\\+}
\tl{suṣumnāṃ layate cittaṃ na ca vāyuḥ pradhāvati ||\\!}
\end{versinnote}
\end{sources}

\begin{testimonia}[hp02_006]
%\emph{Jyotsnā}
%\begin{versinnote}
%\tl{prāṇāyāmaṃ tataḥ kuryān nityaṃ sāttvikayā dhiyā |\\+}
%\tl{yathā suṣumṇānāḍīsthā malāḥ śuddhiṃ prayānti ca || 6 ||\\!}
%\end{versinnote}
\emph{Yogakarṇikā} 58 (attr.~to the \emph{Haṭhapradīpa})

\begin{versinnote}
\tl{prāṇāyāmaṃ tataḥ kuryān nityaṃ sāttvikayā dhiyā |\\+}
\tl{suṣumnā cāntarālasthā malāḥ śoṣaṃ prayānti ca ||\\!}
\end{versinnote}

%\emph{Prāṇatoṣaṇī} p.788

%\begin{versinnote}
%\tl{prāṇāyāmaṃ tataḥ kuryān nityaṃ  sāttvikayā dhiyā | \\+}
%\tl{tathā suṣumnāpārśvasthā malāḥ śoṣaṃ prayānti hi |\\!}
%\end{versinnote}
\end{testimonia}

\begin{philcomm}[hp02_006]
%There is a problem in the third verse quarter. Among the divergent readings, the delta manuscripts convey the idea that the impurities are in the central channel  (\emph{suṣumṇāntarasthā}), which makes a defective \emph{ra-vipulā} (there is no caesura after the fourth syllable). This is similar in sense to V2(siglum), which has the reading \emph{suṣumṇā\-madhya\-sthā}. This meaning was accepted by Brahmānanda, who adopted \emph{suṣumṇā\-nāḍī\-sthā}. We are not aware of a reference in another text to impurities in the central channel, however, so we have adopted the reading of the \gamma \ group and \betaTwo, which states that the impurites are situated at the sides of Suṣumṇā, understanding this to refer to the secondary channels of \emph{iḍā} and \emph{piṅgalā}. 

%The \emph{ca} at the end of the fourth verse quarter would suggest that two statements are being made in the second half of the verse. The readings of some manuscripts, such as \emph{suṣumṇā susnigdhā} in \epsilonThree\ and \etaOne, appear to be attempts to make sense of the \emph{ca}. However, the meaning of \emph{suṣumṇā susnigdhā} (`\emph{suṣumṇā} becomes well-lubricated') seems implausible. In light of the reading in the \gamma \ group and \betaTwo, it seems likely that \emph{ca} was simply a verse-filler.
%JM: I think we should adopt alpha1 (and others): suṣumṇā susvasthā. It works, and makes sense of ca. But we'd not be reading alpha1's yāsthā, which looks to be a mistake.
\end{philcomm}

\begin{metre}[hp02_006]
Anuṣṭubh (c: ma-vipulā)
\end{metre}

%%%%%%%%%%
\subsection*{2.7}
\begin{translation}[hp02_007]
Seated in the lotus pose, the yogi should fill himself up with air via the moon [channel], hold it for as long as he can, then expel it through the sun [channel].
\end{translation}
% adopt yathāśaktyā and punaḥ = alpha one readings.

\begin{sources}[hp02_007]
\emph{Vivekamārtaṇḍa} 77

\begin{versinnote}
\tl{baddhapadmāsano yogī prāṇaṃ candreṇa pūrayet |\\+}
\tl{dhārayitvā yathāśakti bhūyaḥ sūryeṇa recayet ||\\+}
\tl{\var{77c yathāśakti ] GT; yathāśaktyā VA 77d bhūyaḥ ] VAG; punaḥ T}}
\end{versinnote}
\end{sources}

\begin{testimonia}[hp02_007]
\emph{Haṭharatnāvalī} 3.84ab

\begin{versinnote}
\tl{baddhapadmāsano yogī prāṇaṃ candreṇa pūrayet |\\!}
\end{versinnote}

\emph{Yuktabhavadeva} 7.12

\begin{versinnote}
\tl{baddhapadmāsano yogī prāṇaṃ candreṇa pūrayet |\\+}
\tl{dhārayitvā yathāśakti bhūyaḥ sūryeṇa recayet ||\\!}
\end{versinnote}
\end{testimonia}

%\begin{philcomm}[hp02_007]
%The hemistich beginning with \emph{dhārayitvā} appears to have dropped out of the transmission of the \emph{Haṭharatnāvalī}, thus rendering this passage unintelligible in that work (see Gharote et al. 2002: 125).%JM: necessary?
%\end{philcomm}

%%%%%%%%%%
\subsection*{2.8}
\begin{translation}[hp02_008]
And, drawing the breath through the sun [channel], he should gradually fill the abdomen. Having performed the retention as prescribed, he should then exhale through the moon [channel].
\end{translation}

\begin{sources}[hp02_008]
\emph{Vivekamārtaṇḍa} 79

\begin{versinnote}
\tl{prāṇaṃ sūryeṇa cākṛṣya pūrayed udaraṃ śanaiḥ |\\+}
\tl{vidhivat kuṃbhakaṃ kṛtvā punaś candreṇa recayet ||\\!}
\end{versinnote}
\end{sources}

\begin{testimonia}[hp02_008]
\emph{Haṭharatnāvalī} 3.84cd–85ab
\begin{versinnote}
\tl{prāṇaṃ sūryeṇa cākṛṣya pūrayed udaraṃ śanaiḥ |\\+}
\tl{vidhivat kumbhakaṃ kṛtvā punaś candreṇa recayet||\\!}
\end{versinnote}

\emph{Yukabhavadeva} 7.14

\begin{versinnote}
\tl{prāṇaṃ sūryeṇa cākṛṣya pūrayed udaraṃ śanaiḥ |\\+}
\tl{kumbhayitvā vidhānena bhūyaś candreṇa recayet ||\\!}
\end{versinnote}
\end{testimonia}

%\begin{philcomm}[hp02_008]
%\emph{udare} is well-attested (V1 and J10) but the verb needs an object so \emph{udaraṃ} has been adopted. (not necessary as alpha and most collated witnesses have udaraṃ)
%\end{philcomm}

%%%%%%%%%%
\subsection*{2.9}
\begin{translation}[hp02_009]
[The yogi] should inhale through the [channel] by which he has exhaled and hold [the breath] without discomfort. And then he should exhale through the other [channel] slowly, not quickly.
\end{translation}

\begin{sources}[hp02_009]
Cf. \emph{Dattātreyayogaśāstra} 61 

\begin{versinnote}
\tl{yathāśaktyāvirodhena tataḥ kuryāc ca kumbham |\\+}
\tl{punas tyajet piṅgalayā śanair eva na vegataḥ || \\!}
\end{versinnote}
\end{sources}

\begin{testimonia}[hp02_009]
\emph{Haṭharatnāvalī} 3.85cd%–86ab

\begin{versinnote}
\tl{yena tyajet tenāpūrya dhārayed avirodhataḥ ||\\+}
\tl{\var{85d avirodhataḥ ] anirodhataḥ P}\\!}
%prāṇaṃ ced iḍayā piben niyamitaṃ bhūyo 'nyayā recayet |
\end{versinnote}
\end{testimonia}

\begin{philcomm}[hp02_009]
The meaning of \emph{avirodhataḥ} (‘without harm/discomfort’) makes better sense in this context than \emph{anirodhataḥ} (‘without cessation’). One might try to construe \emph{anirodhataḥ} as ‘without stopping the breath,’ but verse 2.7 clearly states that the breath should be held as long as possible (\emph{yathāśakti}). According to the apparatus of the critical edition of the \emph{Haṭharatnāvalī}, \emph{avirodhataḥ} is well-attested for the parallel hemistich. Furthermore, \emph{avirodhata} is attested, as well as \emph{virodhahīna}, in a passage of the \emph{Haṭhatattvakaumudī} (36.6–9) that appears to have been loosely based on \emph{Haṭhapradīpikā} 2.7–9:

\begin{versinnote}
\tl{prāṇāyāme padmapīṭhe svadakṣāṃ-\\+}
\tl{guṣṭhenādau sanniruddhyendunāḍīm |\\+}
\tl{vāyuṃ nātidrāk śanair nātiyuktyā\\+} 
\tl{vyākṛṣyordhvaṃ pūrayet svodarānte ||\\+}
\tl{yathā svaśaktyā laghu dhārayitvā\\+} 
\tl{nāḍyā tataḥ piṅgalayā virecayet |\\+}
\tl{virodhahīnaṃ viratītamadhyaṃ\\+} 
\tl{hṛdā sthireṇābhyasanaṃ muniś caret ||\\+}
\tl{yena tyajet tena virodhahīnaṃ\\+} 
\tl{dhṛtvā purānyena virecayec chanaiḥ |\\+}
\tl{yānty evam abhyāsaratasya puṃsaḥ\\+} 
\tl{sthitiṃ svalakṣye calacittavṛttayaḥ ||\\+}
\tl{yathāśaktyākṛṣya khagaṃ pūrayed udaraṃ śanaiḥ |\\+}
\tl{yathāśaktyā dhṛtaṃ paścād recayed avirodhataḥ ||\\!}
\end{versinnote}
\end{philcomm}

\begin{metre}[hp02_009]
Anuṣṭubh (a: ra-vipulā)
\end{metre}

%%%%%%%%%%
\subsection*{2.10}
\begin{translation}[hp02_010]
If [the yogi] breathes in through Iḍā, he should then exhale the restrained [breath] through the other [channel, i.e. Piṅgalā]. Next, he should inhale through Piṅgalā, hold the breath and release it through the left [channel]. The channels of ascetics meditating on the two orbs of the sun and moon using this method are purified after three months.
\end{translation}

\begin{sources}[hp02_010]
\emph{Vivekamārtaṇḍa} 81

\begin{versinnote}
\tl{prāṇaṃ ced iḍayā pibet niyamitaṃ bhūyo 'nyayā recayet \\+}
\tl{pītvā piṅgalayā samīraṇam alaṃ baddhvā tyajed vāmayā |\\+}
\tl{sūryācandramasor anena vidhinā bimbadvayaṃ dhyāyatāṃ \\+}
\tl{śuddhā nāḍigaṇā bhavanti yaminā māsatrayād ūrdhvataḥ ||\\!}
\end{versinnote}
\end{sources}

\begin{testimonia}[hp02_010]
\emph{Yogacintāmaṇi} f.~90v

\begin{versinnote}
\tl{haṭhayoge 'pi— \\+}
\tl{prāṇaṃ ced iḍayā piben niyamito bhūyo 'nyayā recayet\\+}
\tl{pītvā piṅgalayā samīraṇam atho baddhvā tyajed vāmayā |\\+}
\tl{sūryācandramasor anena vidhinā bimbadvayaṃ dhyāyatām\\+}
\tl{śuddhā nāḍigaṇā bhavanti yamināṃ māsatrayād ūrdhvataḥ ||\\+}
\end{versinnote}

\emph{Haṭharatnāvalī} 3.86

\begin{versinnote}
\tl{prāṇaṃ ced iḍayā piben niyamitaṃ bhūyo 'nyayā recayet\\+}
\tl{pītvā piṅgalayā samīraṇaṃ atho baddhvā tyajed vāmayā |\\+}
\tl{sūryācandramasor anena vidhinā bimbadvayaṃ dhyāyatāṃ\\+}
\tl{śuddhā nāḍigaṇā bhavanti yamināṃ māsatrayād ūrdhvataḥ ||\\!}
\end{versinnote}

\emph{Yuktabhavadeva} 7.16 (attr.~to Gorakṣanātha)

\begin{versinnote}
\tl{prāṇaṃ ced iḍayā pibet parimitaṃ bhūyo 'nyayā recayet\\+}
\tl{pītvā piṅgalayā samīraṇam amalaṃ baddhvā tyajed vāmayā |\\+}
\tl{sūryācandramasor anena vidhinā bimbadvayaṃ dhyāyatām\\+}
\tl{śuddhā nāḍigaṇā bhavanti yamināṃ māsatrayād ūrdhvataḥ ||\\+}
\end{versinnote}
\end{testimonia}

\begin{philcomm}[hp02_010]
%We have translated \emph{yamināṃ} as `yogis' because in this context it could refer to those practising the observances (\emph{yama}) of yoga rather than ascetics more generally. In \emph{Jyotsnā} 2.44, Brahmānanda understands it this way (\emph{yamināṃ yogināṃ}).%JM: It says ascetics, so why not translate thus?
%The reading \emph{vidhinābhyāsaṃ} in the third pāda of various witnesses is unmetrical because it lacks the caesura. (not well attested in the new collation). 
\end{philcomm}

\begin{metre}[hp02_010]
Śārdūlavikrīḍita 
\end{metre}

%%%%%%%%%%
\subsection*{2.11}
\begin{translation}[hp02_011]
[The yogi] should gently practise [twenty] retentions  four times [a day], at sunrise, midday, sunset and midnight, making a total of eighty.
\end{translation}

%\begin{sources}[hp02_011]
%\end{sources}

\begin{testimonia}[hp02_011]
\emph{Haṭharatnāvalī} 3.87

\begin{versinnote}
\tl{prātar madhyadine sāyam ardharātre ca kumbhakān ||\\+}
\tl{śanair aśītiparyantaṃ caturvāraṃ samabhyaset ||\\!}
\end{versinnote}

\emph{Yogacintāmaṇi} f.~90v (attr.~to the \emph{Haṭhayoga})
\begin{versinnote}
\tl{prātar madhyaṃ dine sāyam ardharātre ca kumbhakān ||\\+}
\tl{śanair aśītiparyantaṃ caturvāraṃ samabhyaset ||\\!}
\end{versinnote}

\end{testimonia}

\begin{philcomm}[hp02_011]
This verse is summarizing the following passage in the \emph{Dattātreyayogaśāstra} (63cd--65ab):

\begin{versinnote}
\tl{evaṃ prātaḥ samāsīnaḥ kuryād viṃśati kumbhakān || 63 ||\\+}
\tl{evaṃ madhyāhnasamaye kuryād viṃśati kumbhakān |\\+}
\tl{evaṃ sāyaṃ prakurvīta punar viṃśati kumbhakān || 64 ||\\+}
\tl{evam evārdharātre 'pi kuryād viṃśati kumbhakān |\\!}
\end{versinnote}

Without reference to the \emph{Dattātreyayogaśāstra}, the meaning of the second half of the verse is ambiguous because it could be understood as saying up to eighty retentions four times a day. In his \emph{Jyotsnā} (2.11), Brahmānanda understands it this way, and takes \emph{śanaiḥ} to mean `gradually' building up to the eighty retentions. However the parallel verses in the \emph{Dattātreyayogaśāstra} make it clear that twenty retentions (\emph{kumbhaka}) are to be practised four times a day. 
\end{philcomm}

%%%%%%%%%%
\subsection*{2.12}
\begin{translation}[hp02_012]
In the lesser cessation of the breath sweating arises, in the middle [cessation], shaking, and in the highest [the yogi] repeatedly rises up in the lotus pose.
\end{translation}

\begin{sources}[hp02_012]
Cf. \emph{Vivekamārtaṇḍa} 87

\begin{versinnote}
\tl{adhame ca ghano gharmaḥ kampo bhavati madhyame |\\+}
\tl{uttiṣṭhaty uttame deho baddhapadmāsano muhuḥ ||\\+}
\tl{\var{87c uttiṣṭhaty uttame deho ] T; uttiṣṭhaty uttamo deho V, uttame nu guṇam āpnoti A, uttame sthāṇum āpnoti GB, uttame sthānam āpnoti GL, uttiṣṭhaṃty uttame prāṇā GP 87d baddhapadmāsano muhuḥ ] VT; tato vāyu nibandhayet A, tato vāyuṃ nirundhayet GBGL, vaddhapadmāsane muhūḥ GP}\\!}%How to change sigla of GB, GL, GP so that second letters are subscript? Do it in PreambleComm file
\end{versinnote}
\end{sources}
% 87c uttiṣṭhaty uttame deho ] T; uttiṣṭhaty uttamo deho V, uttame nu guṇam āpnoti A, uttame sthāṇum āpnoti GB, uttame sthānam āpnoti GL, uttiṣṭhaṃty uttame prāṇā GP 87d baddhapadmāsano muhuḥ ] VT; tato vāyu nibandhayet A, tato vāyuṃ nirundhayet GBGL, vaddhapadmāsane muhūḥ GP

\begin{testimonia}[hp02_012]
\emph{Haṭharatnāvalī} 3.88

\begin{versinnote}
\tl{kanīyasi bhavet svedaḥ kampo bhavati madhyame |\\+}
\tl{uttiṣṭhaty uttame prāṇarodhe padmāsane muhuḥ ||\\!}
\end{versinnote}

\emph{Yogacintāmaṇi} 90v (attr.~to the \emph{Haṭhayoga})

\begin{versinnote}
\tl{kanīyasi bhavet svedaḥ kampo bhavati madhyame |\\+}
\tl{uttiṣṭhaty uttame prāṇarodhe padmāsanasthitaḥ ||\\!}
\end{versinnote}
\end{testimonia}

\begin{philcomm}[hp02_012]
The manuscript readings diverge greatly in the second hemistich. In the third verse quarter, all of the manuscripts have \emph{prāṇa} in some form (instead of \emph{deha} in the \emph{Vivekamārtaṇḍa}). \textalpha\ and some of the \textbeta, \textepsilon\ and \texteta\ manuscripts seem to be stating that it is \emph{padmāsana} that rises up again and again in the highest stage of holding the breath (\emph{uttiṣṭhaty uttame prāṇarodhe padmāsanaṃ muhuḥ}). We have adopted a similar reading but with \emph{padmāsane} (\betaTwo, \deltaOne\ and \deltaThree, and \etaTwo) because it makes better sense that the yogi rises up while seated in lotus pose. Another version is seen in \etaTwo\ (and others), which appears to say that the breaths rise up again and again when one is seated in the lotus pose (\emph{uttiṣṭhanty uttame prāṇā baddhe padmāsane muhuḥ}). However, the verse is about the external signs that might arise in \emph{prāṇāyāma} rather than internal processes. Such confusion has arisen because the verse was taken from the \emph{Vivekamārtaṇḍa} without its context, which is a classification of different levels of \emph{prāṇāyāma}, so Svātmārāma needed to include \emph{prāṇarodhe} meaning \emph{prāṇāyāma} in order for the different adjectives to have something with which to agree, and he did so despite the infelicity of \emph{prāṇarodhe} crossing the \emph{pāda} break.
\end{philcomm}

%%%%%%%%%%
\subsection*{2.13}
\begin{translation}[hp02_013]
[The yogi] should rub the limbs with the sweat produced through exertion. As a result the body becomes firm and lithe.%JM how about "strong and lithe"? "Firm" sounds a bit odd to me.
\end{translation}

\begin{sources}[hp02_013]
Cf. \emph{Dattātreyayogaśāstra} 75

\begin{versinnote}
\tl{prasvedo jāyate pūrvaṃ mardanaṃ tena kārayet |\\+}
\tl{tato ’tidhāraṇād vāyoḥ krameṇaiva śanaiḥ śanaiḥ ||\\!}
\end{versinnote}
\end{sources}

\begin{testimonia}[hp02_013]
\emph{Haṭharatnāvalī} 3.89

\begin{versinnote}
\tl{jalena śramajātena aṅgamardanam ācaret |\\+}
\tl{dṛḍhatā laghutā cāpi tathā gātrasya jāyate || 3.89 ||\\!}
\end{versinnote}

Cf. \emph{Śivasaṃhitā} 3.46

\begin{versinnote}
\tl{svedaḥ saṃjāyate dehe yoginaḥ prathamodyame |\\+}
\tl{yadā saṃjāyate svedo mardanaṃ kārayet sudhīḥ |\\+}
\tl{anyathā vigrahe dhātur naṣṭo bhavati yoginaḥ ||\\!}
\end{versinnote}

\emph{Yogacintāmaṇi} 90v (attr.~to the \emph{Haṭhayoga})

\begin{versinnote}
\tl{jalena śramajātena gātramardanam ācaret |\\+}
\tl{dṛḍhatā laghutā cāpi tena gātrasya jāyate ||\\!}
\end{versinnote}
\end{testimonia}

\begin{philcomm}[hp02_013]
\emph{Śivasaṃhitā} 3.46 adds that if this practice is not done, the body's constituents (\emph{dhātu}s) are lost.
\end{philcomm}


%%%%%%%%%%
\subsection*{2.14}
\begin{translation}[hp02_014]
At the beginning of the practice, food with milk and ghee is recommended. After that, when the practice has become well established, there is no need to adopt such regulations.
\end{translation}

\begin{sources}[hp02_014]
\emph{Śivasaṃhitā} 3.43

\begin{versinnote}
\tl{abhyāsakāle prathamaṃ kuryāt kṣīrājyabhojanam\\+}
\tl{tato'bhyāse sthirībhūte na tādṛṅniyamagrahaḥ 3.43\\!}
\end{versinnote}
\end{sources}

\begin{testimonia}[hp02_014]
\emph{Haṭharatnāvalī} 1.24

\begin{versinnote}
\tl{abhyāsakāle prathame śastaṃ kṣīrādibhojanam |\\+}
\tl{tato 'bhyāse dṛḍhībhūte na tāvan niyamagrahaḥ ||\\!}
\end{versinnote}

\emph{Yuktabhavadeva} 4.27 (attr.~to the \emph{Śivayoga})
\begin{versinnote}
\tl{abhyāsakāle prathame śastaṃ kṣīrādibhojanam |\\+}
\tl{tato 'bhyāse dṛḍhībhūte na tādṛṅniyamāgrahaḥ ||\\!}
\end{versinnote}

\end{testimonia}

\begin{philcomm}[hp02_014]
\end{philcomm}

\begin{metre}[hp02_014]
Anuṣṭubh (a: bha-vipulā)
\end{metre}

%%%%%%%%%%
\subsection*{2.15}
\begin{translation}[hp02_015]
Just as a lion, an elephant [or] a tiger is tamed gradually, so the breath is cultivated [gradually], otherwise it kills the practitioner.
\end{translation}

\begin{sources}[hp02_015]
\emph{Vivekamārtaṇḍa} 101

\begin{versinnote}
\tl{yathā siṃho gajo vyāghro bhaved vaśyaḥ śanaiḥ śānaiḥ |\\+}
\tl{anyathā hanti yantāraṃ tathā vāyur asevitaḥ ||\\!}
\end{versinnote}
\end{sources}

\begin{testimonia}[hp02_015]
\emph{Haṭharatnāvalī} 3.90

\begin{versinnote}
\tl{yathā siṃho gajo vyāghro bhaved vaśyaḥ śanaiḥ śanaiḥ |\\+}
\tl{tathaiva sevito vāyur bhaved vaśyaḥ śanaiḥ śanaiḥ ||\\!}
\end{versinnote}

\emph{Yuktabhavadeva} 7.28 (attr.~to Gorakṣanātha)
\begin{versinnote}
\tl{yathā siṃho gajo vyāghro bhaved vaśyaḥ śanaiḥ śanaiḥ |\\+}
\tl{tathaiva sevito vāyur bhaved vaśyaḥ śanaiḥ śanaiḥ ||\\!}
\end{versinnote}

\end{testimonia}

\begin{philcomm}[hp02_015]
The second hemistich of this verse has been rewritten to express the same idea (and simile) as that found in \emph{Viveka\-mārtaṇḍa} 101, but the author of the \emph{Viveka\-mārtaṇḍa} expresses it more clearly.
\end{philcomm}

%%%%%%%%%%
\subsection*{2.16}
\begin{translation}[hp02_016]
All diseases are destroyed by correct \emph{prāṇāyāma}. As a result of incorrect practice any disease may arise.
\end{translation}

\begin{sources}[hp02_016]
\emph{Vivekamārtaṇḍa} 99

\begin{versinnote}
\tl{prāṇāyāmena yuktena sarvarogakṣayo bhavet |\\+}
\tl{ayuktābhyāsayogena sarvarogasamudbhavaḥ ||\\!}
\end{versinnote}
\end{sources}

\begin{testimonia}[hp02_016]
\emph{Haṭharatnāvalī} 3.90

\begin{versinnote}
\tl{prāṇāyāmena yuktena sarvarogakṣayo bhavet |\\+}
\tl{ayuktābhyāsayogena sarvarogasamudbhavaḥ ||\\!}
\end{versinnote}

\emph{Yogacintāmaṇi} 91v–92r (attr.~to the \emph{Skandapurāṇa})

\begin{versinnote}
\tl{prāṇāyāmena yuktena sarvavyādhikṣayo bhavet |\\+}
\tl{ayuktābhyāsayogena sarvavyādhisamudbhavaḥ ||\\!}
\end{versinnote}

\emph{Yuktabhavadeva} 7.26 (attr.~to Gorakṣanātha)
\begin{versinnote}
\tl{prāṇāyāmena yuktena sarvarogasya saṃkṣayaḥ |\\+}
\tl{ayuktābhyāsayogena sarvarogasya sambhavaḥ ||\\!}
\end{versinnote}
\end{testimonia}

%\begin{philcomm}[hp02_016]
%\end{philcomm}

%%%%%%%%%%
\subsection*{2.17}
\begin{translation}[hp02_017]
Hiccups, wheezing, cough, pains in the head, ears and eyes: various diseases arise as a result of the breath going awry.
\end{translation}

\begin{sources}[hp02_017]
\emph{Vivekamārtaṇḍa} 100

\begin{versinnote}
\tl{hikkā śvāsaś ca kāsaś ca śiraḥkarṇākṣivedanā |\\+}
\tl{bhavanti vividhā doṣāḥ pavanasya vyatikramāt ||\\!}
\end{versinnote}
\end{sources}

\begin{testimonia}[hp02_017]
\emph{Haṭharatnāvalī} 3.92

\begin{versinnote}
\tl{hikkā śvāsaś ca kāsaś ca śiraḥkarṇākṣivedanāḥ |\\+}
\tl{bhavanti vividhā rogāḥ pavanasya vyatikramāt ||\\!}
\end{versinnote}

\emph{Yogacintāmaṇi} 92r (attr.~to the \emph{Skandpurāṇa})

\begin{versinnote}
\tl{hikkā śvāsaś ca kāsaś ca śiraḥkarṇākṣivedanāḥ |\\+}
\tl{bhavanti vividhā rogāḥ pavanasya vyatikramāt ||\\!}
\end{versinnote}

\emph{Yuktabhavadeva} 7.27 (attr.~to Gorakṣanātha)
\begin{versinnote}
\tl{hikkā śvāsas tathā kāsaḥ śiraḥkarṇākṣivedanā |\\+}
\tl{bhavanti vividhā rogāḥ pavanasya vyatikramāt ||\\!}
\end{versinnote}

\end{testimonia}

\begin{philcomm}[hp02_017]
This verse has parallels in verses on the illnesses caused by incorrect breathing in earlier Śaiva works, two examples of which are:

\emph{Śivadharmottara} 10.124cd–125
\begin{versinnote}
\tl{hikkāśvāsapratiśyāyaḥ karṇadantākṣivedanāḥ ||\\+}
\tl{mūkatā jaḍatā kāsaḥ śirorogaḥ śramakṣaraḥ |\\+}
\tl{ityevamādayo doṣā jāyante vyutkrameṇa tu ||\\!}
\end{versinnote}

\emph{Dharmaputrikā} 10.265–266ab
\begin{versinnote}
\tl{kaphakoṣṭhe yadā vāyur granthir bhūtvāvatiṣṭhate |\\+}
\tl{hṛllāsahikkikāśvāsaśiraḥśūlādayo rujāḥ ||\\+}
\tl{jāyante dhātuvaiṣamyāt tadā kuryāt pratikriyāṃ |\\!}
\end{versinnote}

\end{philcomm}


%%%%%%%%%%
\subsection*{2.18}
\begin{translation}[hp02_018]
[The yogi] should exhale correctly, inhale correctly and hold the breath correctly. He thus becomes purified.
\end{translation}

\begin{sources}[hp02_018]
\emph{Vivekamārtaṇḍa} 102

\begin{versinnote}
\tl{yuktaṃ yuktaṃ tyajed vāyuṃ yuktaṃ yuktaṃ ca pūrayet |\\+}%122ab ; HP2.18ab
\tl{yuktaṃ yuktaṃ ca badhnīyād evaṃ siddhim avāpnuyāt ||\\!}
\end{versinnote}
\end{sources}

\begin{testimonia}[hp02_018]
\emph{Haṭharatnāvalī} 3.93

\begin{versinnote}
\tl{yuktaṃ yuktaṃ tyajed vāyuṃ yuktaṃ yuktaṃ prapūrayet |\\+}
\tl{yuktaṃ yuktaṃ ca badhnīyād evaṃ siddhim avāpnuyāt ||\\!}
\end{versinnote}

\emph{Yogacintāmaṇi} 92v (attr.~to the \emph{Skandpurāṇa})

\begin{versinnote}
\tl{yuktaṃ yuktaṃ tyajed vāyuṃ yuktaṃ yuktaṃ ca pūrayet |\\+}
\tl{yuktaṃ yuktaṃ ca badhnīyād itthaṃ siddhyati yogavit ||\\!}
\end{versinnote}

\emph{Yuktabhavadeva} 7.29 (attr.~to Gorakṣanātha)
\begin{versinnote}
\tl{yuktaṃ yuktaṃ tyajed vāyuṃ yuktaṃ yuktaṃ tu pūrayet |\\+}
\tl{yuktaṃ yuktaṃ tu badhnīyād evaṃ siddhim avāpnuyāt ||\\!}
\end{versinnote}
\end{testimonia}

%\begin{philcomm}[hp02_018]
%Some of the alpha, epsilon and eta manuscripts have \emph{śuddhiṃ} instead of \emph{siddhiṃ} in the fourth verse quarter but this is likely a later change, or even an error, by someone who was anticipating the subject of the next verse. The reading \emph{siddhiṃ} is attested across all branches of the stemma, as well as the source text and most of the testimonia. 

%AGS: the repetition of yuktaṃ suggests a southern origin for the verse. [search for Proof]
%\end{philcomm}

%%%%%%%%%%
\subsection*{2.19}
\begin{translation}[hp02_019]
When the channels are pure, external signs occur. Leanness and lustre of the body are certain to arise.
\end{translation}

\begin{sources}[hp02_019]
Cf. \emph{Dattātreyayogaśāstra} 67cd--69ab

\begin{versinnote}
\tl{yadā tu nāḍiśuddhiḥ syāt tadā cihnāni bāhyataḥ ||\\+}
\tl{jāyante yogino dehe tāni vakṣyāmy aśeṣataḥ |\\+}
\tl{śarīralaghutā dīptir jaṭharāgnivivardhanam ||\\+}
\tl{kṛśatvaṃ ca śarīrasya tadā jāyeta niścitam |\\!}
\end{versinnote}
\end{sources}

\begin{testimonia}[hp02_019]
\emph{Haṭharatnāvalī} 3.94

\begin{versinnote}
\tl{yadā tu nāḍīśuddhiḥ syāt tadā cihnāni bāhyataḥ |\\+}
\tl{kāyasya kṛśatā kāntir jāyate tasya niścitam ||\\!}
\end{versinnote}

\emph{Yogacintāmaṇi} 90v (attr.~to the \emph{Hathayoga})

\begin{versinnote}
\tl{yadā nāḍīviśuddhiḥ syāt tadā cintānirākṛtā |\\+}
\tl{kāyasya kṛśatā kāntis tadā jāyeta niścitam  ||\\!}
\end{versinnote}
\end{testimonia}

\begin{philcomm}[hp02_019]
The idea that \emph{prāṇāyāma} is done to purify the channels (\emph{nāḍī}) can be found in discussions of \emph{prāṇāyāma} in early Śaiva tantras. For example, the \emph{Nayasūtra} of the \emph{Niśvāsatattvasaṃhitā} (4.110) and the \emph{Svacchandatantra} (7.294cd–7.295ab) specifically refer to purifying the channels by inhaling through the left nostril and exhaling through the right, as stated in the latter:

\begin{versinnote}
\tl{apasavyena pūryeta savyenaiva virecayet |\\+} 
\tl{nāḍīsaṃśodhanaṃ caitan mokṣamārgapathasya ca ||\\!}
\end{versinnote}
\end{philcomm}

\begin{metre}[hp02_019]
Anuṣṭubh (a: ma-vipulā)
\end{metre}

%%%%%%%%%%
\subsection*{2.20}
\begin{translation}[hp02_020]
The ability to hold the breath as long as one desires, stimulation of the [digestive] fire, manifestation of the inner sound [and] freedom from disease occur as a result of purifying the channels.%JM: I've changed "resonance" to "sound" for nāda. "resonance" isn't right as it's a quality of sound, not sound itself; furthermore, the quality is one of deep reverberation, which not all internal sounds have.
\end{translation}

\begin{sources}[hp02_020]
\emph{Vivekamārtaṇḍa} 101

\begin{versinnote}
\tl{yatheṣṭaṃ dhāraṇāṃ vāyor analasya pradīpanam |\\+}
\tl{nādābhivyaktir ārogyaṃ jāyate nāḍīśodhanāt ||\\!}
\end{versinnote}
\end{sources}

\begin{testimonia}[hp02_020]
\emph{Haṭharatnāvalī} 3.95

\begin{versinnote}
\tl{yatheṣṭaṃ dhāraṇaṃ vāyor analasya pradīpanam |\\+}
\tl{nādābhivyaktir ārogyaṃ jāyate nāḍiśodhanāt ||\\!}
\end{versinnote}

\emph{Yogacintāmaṇi} 90v (attr.~to the \emph{Skandapurāṇa})

\begin{versinnote}
\tl{yatheṣṭaṃ dhāraṇaṃ vāyor analasya pradīpanam |\\+}
\tl{nādābhivyaktir ārogyaṃ bhaven nāḍīviśodhanāt ||\\!}
\end{versinnote}

\emph{Yuktabhavadeva} 7.17 (attr.~to Gorakṣanātha)

\begin{versinnote}
\tl{yatheṣṭaṃ dhāraṇaṃ vāyor analasya pradīpanam |\\+}
\tl{nādābhivyaktir ārogyaṃ jāyate nāḍīśodhanāt ||\\!}
\end{versinnote}

\end{testimonia}

\begin{philcomm}[hp02_020]
Similar signs (\emph{cihna}) arising from the purification of the channels are mentioned in the \emph{Vasiṣṭhasaṃhitā} (2.68--69) and subsequent works related to it:

\begin{versinnote}
\tl{nāḍīśuddhim avāpnoti pṛthak cihnopalakṣitām |\\+}
\tl{śarīralaghutā dīptir jaṭharāgnivivardhanam ||\\+}
\tl{nādābhivyaktir ity etac cihnaṃ tacchuddhisūcakam |\\+}
\tl{yāvad etāni saṃpaśyet tāvad evaṃ samācaret ||\\!}
\end{versinnote}

\end{philcomm}

%%%%%%%%%%
\subsection*{2.21}
\begin{translation}[hp02_021]
The person who has an excess of fat and phlegm should first practise the six therapeutic interventions, but anyone else, because their humours are in balance, should not practise them.
\end{translation}

\begin{sources}[hp02_021]
\end{sources}

\begin{testimonia}[hp02_021]
\emph{Yogacintāmaṇi} 8v (attr.~to Ātmārāma)

\begin{versinnote}
\tl{medaḥśleṣmanivṛtyarthaṃ ṣaṭkarmāṇi samācaret |\\+}
\tl{anyathā nācaret tāni doṣāṇāṃ samatā yataḥ ||\\!}
\end{versinnote}

\emph{Yuktabhavadeva} 7.147 (attr.~to the \emph{Haṭhapradīpikā})

\begin{versinnote}
\tl{medaślemādisampūrṇaḥ ṣaṭkarmāṇi samācaret |\\+}
\tl{anyas tu nācaret tāni doṣāṇāṃ samabhāgikaḥ ||\\!}
\end{versinnote}

\end{testimonia}

\begin{philcomm}[hp02_021]
Manuscripts of the delta group, as well as \etaTwo\ and \epsilonThree, also have the valid readings of \emph{medaḥśleṣmādināśārthaṃ} and \emph{anyathā} in the first and third verse quarters, respectively. This version of the verse states that one should practise the six therapeutic interventions to remove fat, phlegm and the like, otherwise one should not practise them when the humours are in balance. However, the \emph{pūrvam} (`first'), which is attested in two \textalpha\ manuscripts, as well as manuscripts of the \textbeta\ and \textgamma\ groups, fits the context of these interventions being preliminary practices for \emph{prāṇāyāma}.
\end{philcomm}

%%%%%%%%%%
\subsection*{2.22}
\begin{translation}[hp02_022]
\emph{Dhauti, basti, neti, trāṭaka, naulī} and \emph{kapālabhātī}. These are said to be the six [therapeutic] techniques.%JM: "therapeutic interventions" adds a lot to karmāṇi. I usually translate it with "[cleansing] techniques". We have techniques everywhere else.
%JM: note on alphaone's laulikam? And should we have naulī here: that's what it is everywhere else so this is prob for metre.
\end{translation}

\begin{sources}[hp02_022]
\end{sources}

\begin{testimonia}[hp02_022]
\emph{Haṭharatnāvalī} 1.27

\begin{versinnote}
\tl{haṭhapradīpikāyām-\\+}
\tl{dhautir bastis tathā netis trāṭakaṃ naulikaṃ tathā |\\+}
\tl{kapālabhrāntir etāni ṣaṭkarmāṇi pracakṣate  ||\\!}
\end{versinnote}

\emph{Yogacintāmaṇi} 71r (attr.~to the \emph{Haṭhapradīpikā})

\begin{versinnote}
\tl{atha ṣaṭkarmāṇi | haṭhapradīpikāyām ||\\+}
\tl{dhautī bastī tathā netī trāṭakaṃ naulikaṃ tathā |\\+}
\tl{kapālabhātī caitāni ṣaṭkarmāṇi pracakṣate ||\\!}
\end{versinnote}

\emph{Yuktabhavadeva} 7.148 (attr.~to the \emph{Haṭhapradīpikā})

\begin{versinnote}
\tl{dhautir bastiś ca netiś ca trāṭakaṃ naulikaṃ tathā |\\+}
\tl{kapālabhāti caitāni ṣaṭkarmāṇi pracakṣate ||\\!}
\end{versinnote}

\end{testimonia}

\begin{philcomm}[hp02_022]
Manuscripts across all the groups contain many different spellings of the names of these techniques. As well as the requirements of the metre, the spellings we have favoured take into account the occurrences of each name in subsequent verses. 
\end{philcomm}

\begin{metre}[hp02_022]
Anuṣṭubh (c: ma-vipulā)
\end{metre}

%%%%%%%%%%
\subsection*{2.23}
\begin{translation}[hp02_023]
This set of six techniques should be kept secret. Bringing about purification of the body [and] bestowing various good qualities, it is worshipped by the best yogis.
\end{translation}

\begin{sources}[hp02_023]
\end{sources}

\begin{testimonia}[hp02_023]
\emph{Haṭharatnāvalī} 1.28

\begin{versinnote}
\tl{karmāṣṭakam idaṃ gopyaṃ ghaṭaśodhanakārakam |\\+}
\tl{kasya cin naiva vaktavyaṃ kulastrīsurataṃ yathā  ||\\!}
\end{versinnote}

\emph{Yogacintāmaṇi} 71r (attr.~to the \emph{Haṭhapradīpikā})

\begin{versinnote}
\tl{karmaṣaṭkam idaṃ gopyaṃ ghaṭaśodhanakārakam |\\+}
\tl{vicitraguṇasaṃdhāyi pūjyate yogipuṃgavaiḥ ||\\!}
\end{versinnote}

\emph{Yuktabhavadeva} 7.149 (attr.~to the \emph{Haṭhapradīpikā})

\begin{versinnote}
\tl{karmaṣaṭkam idaṃ gopyaṃ ghaṭaśodhanakāraṇam |\\+}
\tl{vicitraguṇasandhāyī kriyate yogibhiḥ sadā ||\\!}
\end{versinnote}

\end{testimonia}

\begin{philcomm}[hp02_023]
\end{philcomm}

%%%%%%%%%%
\subsection*{2.24}
\begin{translation}[hp02_024]
Among them is dhauti:\\ 
{}[The yogi] should slowly swallow a moistened cloth  four finger-breadths in width and then draw it out. This ejection [of it] from the mouth is the dhauti technique.
\end{translation}

%\begin{sources}[hp02_024]
%\end{sources}

\begin{testimonia}[hp02_024]
\emph{Haṭharatnāvalī} 1.37–38ab

\begin{versinnote}
\tl{atha dhautiḥ--\\+}
\tl{viṃśaddhastapramāṇena dhautavastraṃ sudīrghitam |\\+}
\tl{caturaṅgulavistāraṃ siktaṃ caiva śanaiḥ graset ||\\+}
\tl{tataḥ pratyāharec caitad abhyāsād dhautir ucyate |\\!}
\end{versinnote} 

\emph{Yogacintāmaṇi} f.~71r (attr.~to the \emph{Haṭhapradīpikā})

\begin{versinnote}
\tl{atha dhautī |\\+}
\tl{caturaṅgulavistāraṃ siktaṃ vastraṃ śanair graset |\\+}
\tl{punaḥ pratyāhared etad abhyāsād dhautikarmavit ||\\!}
\end{versinnote}

\emph{Yuktabhavadeva} 7.150 (attr.~to the \emph{Haṭhapradīpikā})

\begin{versinnote}
\tl{caturaṃgulavistāraṃ siktaṃ vastraṃ śanair graset |\\+}
\tl{tataḥ pratyāharec caitad ākṣālaṃ dhautikarma tat ||\\!}
\end{versinnote}

Cf. \emph{Satkarmasaṅgraha} 56–57

\begin{versinnote}
\tl{atha dhautī |\\+}
\tl{mṛdulaṃ dhavalaṃ śuddhaṃ caturaṅgulavistṛtam |\\+}
\tl{tithihastamitāyāmaṃ dhautīvastrasya lakṣaṇam ||\\+}
\tl{toyasiktaṃ grased vastraṃ ghrāṇābhyāṃ vāyum utsṛjan |\\+}
\tl{śanaiḥ sanais tu sakalaṃ punaḥ pratyāharec chanaiḥ |\\+}
\tl{dhautīkarmedam ākhyātaṃ yatra gaṅgādhidaivatam ||\\!}
\end{versinnote}

\end{testimonia}

\begin{philcomm}[hp02_024]
The manuscripts contain many different readings for the fourth \emph{pāda}. \alphaThree\ and \etaOne\ have the term \emph{udgāraṃ}, which rarely occurs in yoga texts. The basic meaning of \emph{udgāra} is the act of discharging something from the mouth, which fits the context of \emph{dhauti} in so far as the cloth swallowed into the stomach is drawn back out through the mouth. Some of the other readings, such as \emph{uditaṃ}, \emph{utthānaṃ}, etc., appear to be mistakes or patches that arose possibly because \emph{udgāra} is not normally neuter but masculine.

Many manuscripts have added verse quarters on the length of the cloth (\emph{hastapañcadaśena tu}) and doing the practice according to the guru’s teachings (\emph{gurūpadiṣṭa\-mārgeṇa}). These additional comments are absent in the \textalpha, \textgamma\ and \textepsilon\ groups. Furthermore, the compound \emph{hastapañcadaśena} does not seem to fit the syntax of the sentence. The other addition, on the guru's teaching, is a cliché that is probably being used here as a verse filler.
\end{philcomm}


%%%%%%%%%%
\subsection*{2.25}
\begin{translation}[hp02_025]
Coughing, wheezing, splenitis and skin diseases, as well as the twenty phlegmatic diseases, are sure to disappear through the power of the dhauti technique.
\end{translation}

\begin{sources}[hp02_025]
\end{sources}

\begin{testimonia}[hp02_025]
\emph{Haṭharatnāvalī} 1.39

\begin{versinnote}
\tl{kāsaśvāsaplīhakuṣṭhaṃ kapharogāś ca viṃśatiḥ |\\+}
\tl{dhautikarmaprabhāvena dhāvanty eva na saṃśayaḥ ||\\!}
\end{versinnote}

\emph{Yogacintāmaṇi} f.~71r (attr.~to the \emph{Haṭhapradīpikā})

\begin{versinnote}
\tl{kāsaśvāsaplīhakuṣṭhaṃ kapharogāś ca vidradhiḥ |\\+}
\tl{dhautīkarmaprabhāvena prayānty eva na saṃśayaḥ ||\\!}
\end{versinnote}

\emph{Yuktabhavadeva} 7.151 (attr.~to the \emph{Haṭhapradīpikā})

\begin{versinnote}
\tl{plīhā śvāsaś ca kuṣṭhaṃ ca kapharogāś ca viṃśatiḥ |\\+}
\tl{dhautikarmaprabhāvena gacchanty eva na saṃśayaḥ ||\\!}
\end{versinnote}

Cf. \emph{Satkarmasaṅgraha} 58

\begin{versinnote}
\tl{kāsasvāsaplīhakuṣṭhādināśam\\+} 
\tl{vahner māndyaṃ viṃśatiḥ śleṣarogān |\\+}
\tl{dūrīkuryāt karṇabādhir tam uccair\\+} 
\tl{dhautīkarma praditaṃ śaṅkareṇa ||\\!}
\end{versinnote}

\end{testimonia}

\begin{philcomm}[hp02_025]
The verb \emph{dhāvanti} is a play on words, using a different root \emph{dhāv}, “run”, from that of \emph {dhauti}, which is derived from \emph{dhāv}, “purify”.

Twenty phlegmatic diseases are enumerated in the \emph{Carakasaṃhitā}, \emph{sūtrasthāna} 20.17, a chapter on major diseases (\emph{mahāroga}). %The other diseases mentioned in the \emph{Haṭhapradīpikā}'s verse on \emph{dhauti}, namely cough (\emph{kāsa}), wheezing (\emph{śvāsa}), splenitis (\emph{plīha}) and skin diseases (\emph{kuṣṭha}), are not among them. The likely reason for this is that these particular diseases are not solely caused by phlegm disorders. For example, \emph{Carakasaṃhitā}, \emph{sūtrasthāna} 19.4 states that the five types of spleen diseases are caused by wind, bile, phlegm, commingling of the humours (\emph{sannipāta}) and blood, and the five types of cough arise from wind, bile, phelgm, injury, and wasting.%JM: I don't understand the import of the second part of the note — the text doesn't imply that the first ailments listed should be among the twenty kapha diseases.
\end{philcomm}

\begin{metre}[hp02_025]
Anuṣṭubh (a: ra-vipulā)
\end{metre}

%%%%%%%%%%
\subsection*{2.26}
\begin{translation}[hp02_026]
Now the \emph{basti} technique.\\
Squatting in water up to the navel with a reed inserted in the anus, [the yogi] should contract the perineal region (\emph{ādhāra°}). The [resultant] flushing is the basti technique.
\end{translation}
% Adopt pakhālaṃ — not done in latest version

\begin{sources}[hp02_026]
\end{sources}

\begin{testimonia}[hp02_026]
\emph{Haṭharatnāvalī} 1.45–47

\begin{versinnote}
\tl{nābhidaghne jale sthitvā pāyunāle sthitāṅguliḥ |\\+}
\tl{cakrimārgeṇa jaṭharaṃ pāyunālena pūrayet ||\\+}
\tl{vicitrakaraṇīm kṛtvā nirbhītaḥ recayej jalam |\\+}
\tl{yāvad balaṃ prapūryaiva kṣaṇaṃ sthitvā virecayet ||\\+}
\tl{ghaṭītrayaṃ na bhoktavyaṃ bastim abhyasatā dhruvam |\\+}
\tl{nivātabhūmau santiṣṭhed vaśī hitamitāśanaḥ ||\\!}
\end{versinnote} 

\emph{Yogacintāmaṇi} f.~71r (attr.~to the \emph{Haṭhapradīpikā})

\begin{versinnote}
\tl{atha vastī |\\+}
\tl{nābhidaghne jale pāyunyastanālotkaṭāsanaḥ |\\+}
\tl{ādhārākuñcanaṃ kuryād abhyāsād vastikarmavit ||\\!}
\end{versinnote}

\emph{Yuktabhavadeva} 7.152 (attr.~to the \emph{Haṭhapradīpikā})

\begin{versinnote}
\tl{nābhidaghne jale pāyau nyastanālotkaṭāsanaḥ |\\+}
\tl{ādhārā kuñcanaṃ kuryāt kṣālanaṃ bastikarma tat ||\\!}
\end{versinnote}

Cf. \emph{Satkarmasaṅgraha} 132

\begin{versinnote}
\tl{naulīkriyāsusaṃpannas tyaktamūtramalaḥ sudhīḥ |\\+}
\tl{jānudaghne jale kuryād bastiṃ bastividhānavit ||\\!}
\end{versinnote}

\end{testimonia}

\begin{philcomm}[hp02_026]
% MD: kṣālanaṃ or prakṣālaṃ? Adopt pakhālaṃ [MD: done]
The reading \emph{pakhālaṃ} that we have adopted in the fourth \emph{pāda} is found in \betaOmega \ and \alphaOne \ and reflects vernacular usage as found in the old Hindi \emph{Aṣṭāṅgayoga} of Caraṇadāsa (6.71ab). Some other witnesses have the Sanskritised  form \emph{prakṣālaṃ}, which is very rare in Sanskrit sources, while others have the more common \emph{kṣālanaṃ}.

\end{philcomm}

%%%%%%%%%%
\subsection*{2.27}
\begin{translation}[hp02_027]
By the power of the basti technique, swelling, splenitis, stomach disorders and all diseases arising from wind, bile and phlegm are removed.
\end{translation}

%\begin{sources}[hp02_027]
%\end{sources}

\begin{testimonia}[hp02_027]
\emph{Haṭharatnāvalī} 1.48

\begin{versinnote}
\tl{gulmaplīhodaraṃ vāpi vātapittakaphādikam |\\+}
\tl{bastikarmaprabhāvena dhāvanty eva saṃśayaḥ ||\\!}
\end{versinnote}

\emph{Yogacintāmaṇi} 71r (attr.~to the \emph{Haṭhapradīpikā})

\begin{versinnote}
\tl{gulmodaraṃ cāpi vātaplīhapittakaphodbhavāḥ |\\+}
\tl{vastikarmaprabhāvena bādhyante sakalāmayāḥ ||\\!}
\end{versinnote}

\emph{Yuktabhavadeva} 7.153 (attr.~to the \emph{Haṭhapradīpikā})

\begin{versinnote}
\tl{gulmaplīhodaraṃ cāpi vātapittakaphodbhavāḥ |\\+}
\tl{bastikarmaprabhāvena naśyanti sakalāmayāḥ ||\\!}
\end{versinnote}

Cf. \emph{Satkarmasaṅgraha} 135, 140–141

\begin{versinnote}
\tl{yāvan malā vinaśyanti vātapittakaphodbhāvāḥ |\\+}
\tl{trivāraṃ vā caturvāraṃ kṛtvā bastim virecayet ||\\+}
\tl{mahojasvī mahajjyotir jaṭharāgnipradīpanam |\\+}
\tl{gulmaplīhodarādīnāṃ nāśanaṃ sukhavardhanam ||\\+}
\tl{vātapittakaphottānāṃ doṣāṇāṃ nāśanaṃ paraṃ |\\+}
\tl{kuṣṭhānāṃ nāśanaṃ cāpi bastisiddhe prajāyate ||\\!}
\end{versinnote}
\end{testimonia}

\begin{philcomm}[hp02_027]
%The J10 group has \emph{°bhavaṃ}, which would qualify \emph{gulmaplīhodaram}. (seems to be an improbable reading because then sakalāmayāḥ is unqualified, thus rendering the statement that basti cures all diseases.
We have assumed that the compound \emph{gulmaplīhodara} is referring generally to swelling (\emph{gulma}), spleen disorders (\emph{plīhan}) and stomach diseases (\emph{udara}). However, the terms \emph{gulma} and \emph{plīhodara} can be understood as more specific diseases. In \emph{Suśrutasaṃhitā, uttaratantra} 42.4, \emph{gulma} is defined as a movable or immovable round lump (\emph{granthi}) that might arise between the heart and lower abdomen (\emph{basti}) and might grow or shrink:

\begin{versinnote}
\tl{hṛdbastyorantare granthiḥ saṃcārī yadi vā 'calaḥ |\\+}
\tl{cayāpacayavān vṛttaḥ sa gulma iti kīrtitaḥ ||\\!}
\end{versinnote}

The compound \emph{plīhodara} is the name of a specific disease, which is defined in the \emph{Suśrutasaṃhitā} (\emph{nidānasthāna} 7.14–15) as enlargement of the spleen (\emph{plīhābhivṛddhi}) so \emph{gulmaplīhodara} might be referring more specifically to abdominal lumps and splenomegaly.
%Diwakar thinks that \emph{gulmodara} and \emph{plīhodara} should be read. Search āyurvedic commentaries on this. (homework) JB: I can find strong evidence for a disease called plīhodara, but not gulmodara (which usually occurs in longer compounds and is translated by P.V. Sharma as gulma and udararoga: e.g., Suśruta 1.45.221ab, 1.46.196ef, 6.39.217, 6.40.180) 
\end{philcomm}

%%%%%%%%%%
\subsection*{2.28}
\begin{translation}[hp02_028]
When practised repeatedly, the water enema (\emph{jalabasti}) technique bestows clarity of the bodily constituents, senses and mind, radiance, [and] stimulation of the digestive fire, and removes [excessive] accumulation of all humours.
\end{translation}

%\begin{sources}[hp02_028]
%\end{sources}

\begin{testimonia}[hp02_028]
\emph{Haṭharatnāvalī} 1.49 

\begin{versinnote}
\tl{dhātvindriyāntaḥkaraṇaprasādaṃ \\+}
\tl{dadyāc ca kāntiṃ dahanapradīptim |\\+}
\tl{aśeṣadoṣopacayaṃ nihanyād \\+}
\tl{abhyasyamānaṃ jalabastikarma ||\\!}
\end{versinnote}

\emph{Yogacintāmaṇi} 71r (attr.~to the \emph{Haṭhapradīpikā})

\begin{versinnote}
\tl{dhātvindriyāntaḥkaraṇaprasādaṃ\\+}
\tl{dadhyāc ca kāntiṃ dahanapradīptim |\\+}
\tl{aśeṣadoṣopacayaṃ nihanyād\\+}
\tl{abhyasyamānaṃ jalavastikarma ||\\!}
\end{versinnote}

\emph{Yuktabhavadeva} 7.154 (attr.~to the \emph{Haṭhapradīpikā})

\begin{versinnote}
\tl{dhātvīndrintaḥ karaṇaprabodhaṃ\\+}
\tl{dadāti kāntiṃ dahanapradīptim |\\+}
\tl{aśeṣadoṣopacayaṃ nihanyād\\+}
\tl{abhyasyamānaṃ jalavastikarma ||\\!}
\end{versinnote}

Cf. \emph{Satkarmasaṅgraha} 139–140ab

\begin{versinnote}
\tl{tiṣṭhed vaśī mitāhāraḥ sarvāṅgaṃ tena śudhyati |\\+}
\tl{dhātvindriyāntaḥkaraṇaprasādo dehalāghavam ||\\+}
\tl{mahojasvī mahajjyotir jaṭharāgnipradīpanam |\\!}
\end{versinnote}
\end{testimonia}

%\begin{philcomm}[hp02_028]
%\end{philcomm}

\begin{metre}[hp02_028]
Upajāti
\end{metre}

%%%%%%%%%%
\subsection*{2.29}
\begin{translation}[hp02_029]
Raising the \emph{apāna} wind into the oesophagus (\emph{kaṇṭhanāle}) and ejecting the contents of the stomach from the windpipe, which has been brought under control by cumulative practice, is called the elephant technique by experts in Haṭha.
\end{translation}

\begin{sources}[hp02_029]
\end{sources}

\begin{testimonia}[hp02_029]
\emph{Haṭharatnāvalī} 1.51

\begin{versinnote}
\tl{udaragatapadārtham udvamantī \\+}
\tl{pavanam apānam udīrya kaṇṭhanāle |\\+}
\tl{kramaparicayatas tu vāyumārge \\+}
\tl{gajakaraṇīti nigadyate haṭhajñaiḥ || 1.51 ||\\!}
\end{versinnote}

\emph{Yuktabhavadeva} 7.154 (attr.~to the \emph{Haṭhapradīpikā})

\begin{versinnote}
\tl{udaragatapadārtham udvamantī\\+}
\tl{pavanamapānam udīrya kaṇṭhanāle |\\+}
\tl{kramaparicayavaśyavāyumārgā\\+}
\tl{gajakaraṇīti nigadyate haṭhajñaiḥ ||\\!}
\end{versinnote}

\emph{Haṭhatattvakaumudī} 8.8

\begin{versinnote}
\tl{udaragatapadārthān udvamed eva nityaṃ\\+}
\tl{pavanagamanamārgāt kaṇṭhanālapraveśāt ||\\+}
\tl{kramaparicayavaśyaṃ syāc ca gargādayo hi\\+}
\tl{gajakaraṇam itīha prahur āryā munīndrāḥ ||\\!}
\end{versinnote}

Cf. \emph{Satkarmasaṅgraha} 108–109

\begin{versinnote}
\tl{atha gajakaraṇī\\+}
\tl{śuddhaṃ toyaṃ nārikelodbhavaṃ vā \\+}
\tl{pītvākaṇṭhaṃ dugdhamiśraṃ jalaṃ vā |\\+}
\tl{vāraṃ vāraṃ māṇibandhaṃ tu kurvan \\+}
\tl{nodgāreṇa prakṣiped bhūmibhāge  ||\\+}
\tl{eṣā proktā kaphapittāmayeṣu \\+}
\tl{medoghnīva kariṇī hastipūrvā ||\\!}
\end{versinnote}
\end{testimonia}

\begin{philcomm}[hp02_029]
Some manuscripts, including \etaTwo\ and \betaOmega, have an alternative reading for the third verse quarter that appears to be explaining the name of the practice. In other words, it is called the elephant technique `because the speed of the breath is like that of water [propelled] by elephants’ (\emph{karibhir iva jalasya vāyuvegāt}). The syntax of this reading is not so easy to construe with the rest of the verse, which suggests that it was not original. The reading we have adopted (i.e., \emph{kramaparicayavaśya}...) is attested by manuscripts of the most important groups (\alphaOne\ and \alphaTwo, \betaTwo, the \textgamma\ group, etc.) and the same witnesses preserve \emph{mārga} (rather than \emph{vega}). 

%The \emph{Jyotsnā} and printed editions have a finite verb in the first hemistich and assume a plural subject (i.e., \emph{yoginaḥ}) instead of the feminine singular present participle (agreeing with \emph{gajakaraṇī}). (JB not necessary as alpha, etc. support udvamantī)
\end{philcomm}

\begin{metre}[hp02_029]
Puṣpitāgrā
\end{metre}

%%%%%%%%%%
\subsection*{2.30}
\begin{translation}[hp02_030]
Now \emph{neti}.\\
{[}The yogi] should insert a very smooth thread one handspan [in length] into the nasal passage and take it out through the mouth. This is called neti by the Siddhas.
\end{translation}

%\begin{sources}[hp02_030]
%\end{sources}

\begin{testimonia}[hp02_030]
Cf.~\emph{Haṭharatnāvalī} 1.40–41

\begin{versinnote}
\tl{atha netikarma--\\+}
\tl{ākhupucchākāranibhaṃ sūtraṃ susnigdhanirmitam |\\+}
\tl{ṣaḍvitastimitaṃ sūtraṃ netisūtrasya lakṣanam || 1.40 ||\\+}
\tl{nāsānāle praviśyainaṃ mukhān nirgamayet kramāt |\\+}
\tl{sūtrasyāntaṃ prabaddhvā tu bhrāmayen nāsanālayoḥ |\\!}
\end{versinnote}

\emph{Yogacintāmaṇi} 71r–71v (attr.~to the \emph{Haṭhapradīpikā})

\begin{versinnote}
\tl{atha netī |\\+}
\tl{sūtraṃ vitastisusnigdhaṃ nāsānāle praveśayet |\\+}
\tl{mukhān nirgamayet sā hi netī siddhair nigadyate ||\\!}
\end{versinnote}

\emph{Yuktabhavadeva} 7.156 (attr.~to the \emph{Haṭhapradīpikā})

\begin{versinnote}
\tl{sūtraṃ vitastisusnigdhaṃ nāsānāle praveśayet |\\+}
\tl{mukhān nirgamayed eṣā netiḥ siddhair nigadyate ||\\!}
\end{versinnote}

Cf.~\emph{Satkarmasaṅgraha} 67

\begin{versinnote}
\tl{atha netī\\+}
\tl{mṛdu ślakṣṇaṃ sitaṃ sūtraṃ nāsānāle praveśayet |\\+}
\tl{mukhān nirgamayed dasrau cintayen netikā smṛtā ||\\!}
\end{versinnote}
\end{testimonia}

\begin{philcomm}[hp02_030]
According to Turner’s Comparative and Etymological Dictionary (1966: 427, entry 7588), the word \emph{netī} in Hindi refers to the cord of a churning stick and is cognate with the Sanskrit \emph{netra}. The action of pulling the cord of a churning stick is similar to the way the thread can be pulled back and forth, from side to side, through the nostril and mouth. 

The reading we have adopted in the third verse quarter (\emph{mukhān nirgamayec caiṣā}) is well attested but may not be original because the first hemistich has a finite verb and the \emph{eṣā} must be construed with \emph{netiḥ} in the fourth verse quarter. The alternative reading \emph{mukhanirgamanād eva} (\etaOne\ and \etaTwo) may be original but is difficult to construe and \emph{mukhān nirgamayet} makes better sense in terms of describing the final part of the practice. 
\end{philcomm}
%%%%%%%%%%
\subsection*{2.31}
\begin{translation}[hp02_031]
[The technique] with a small thread purifies the skull, bestows divine sight and quickly cures a multitude of diseases that arise above the jaw.
\end{translation}
%?? expand note on possibility of niti/nīti being original (found in alpha, betaone, epsilonone, delta i.e. in good manuscripts in most groups; vernacular?) Only niti works well in final pāda. Brahmānanda rewrites to have netī as subject, which would be best. sūtrikā is old — eta1 = V1. In 22 alphatwo has nitiś caiva...
% adopt hanūrdhva
% adopt jayaty āśu sa-sūtrikā

% MD: more mss for sūtrikā:
% jayaṃ(one syllable om.) sā sasūtrikā (K1), jayaṃtī sārasūtrikā (P27), jayate sā ca sūtrikā (B2) [gamma]; jayaṃti sārasūtrikā (P6) [contaminated epsilon]
% Perhaps better to read jayaty āśu sa-sūtrikā (= neti)?
% Or: jayate (Middle)/jayantī (Pres. Part.) sā sa-sūtrikā? 


%\begin{sources}[hp02_031]
%\end{sources}

\begin{testimonia}[hp02_031]
\emph{Haṭharatnāvalī} 1.42

\begin{versinnote}
\tl{kapālaśodhinī kāryā divyadṛṣṭipradāyinī | [caiva -P]\\+}
\tl{jatrūrdhvajātarogaghnī jāyate netir uttamā || 1.42 ||\\!} % cf. M1
\end{versinnote}

\emph{Yogacintāmaṇi} 71v (attr.~to the \emph{Haṭhapradīpikā})

\begin{versinnote}
\tl{kapālaśodhanī caiva divyadṛṣṭipradīpinī |\\+}
\tl{jatrūrdhvajātarogaughān jarayaty āśu netivit ||\\!} % cf. J6
\end{versinnote}

\emph{Yuktabhavadeva} 7.156 (attr.~to the \emph{Haṭhapradīpikā})

\begin{versinnote}
\tl{kapālaśodhinī caiva divyadṛṣṭipradāyinī |\\+}
\tl{jatrūrdhvajātarogādyaiḥ jayatyeva suniścitam ||\\!}
\end{versinnote}

\end{testimonia}

\begin{philcomm}[hp02_031]
We have adopted the reading \emph{caiva} in the first verse quarter, which is attested by manuscripts of the \textbeta, \textgamma\ and \textdelta\ groups and is easy to construe. The \textalpha\ manuscripts have \emph{kaṇṭhā} and \emph{vaṭyā}, which may derive from \emph{kaṇṭhyā} (\epsilonOne), but none of these readings makes sense here. \etaOne\ and \etaTwo\ have \emph{kāryā}, which is unnecessary because of the main verb in the final verse quarter.
% MD: kaṇṭhyā (N19,G5,G11 - all epsilon)

Most witnesses, including the \emph{Jyotsnā}, read \emph{jatrūrdhva°} in the third verse quarter, which is generally understood as `above the collar bones' or, as Brahmānanda says, the area above the shoulder joints (\emph{jatruṇoḥ skandhasandhyor ūrdhvam uparibhāge}). On problems concerning the interpretation of \emph{jatru}, see Meulenbeld 1974: 465. We have adopted the reading of \textalpha, \emph{hanūrdhva°}, ‘above the jaw’, which makes good sense in the context of \emph{neti} because it is supposed to cleanse the skull (\emph{kapāla}) or, in other words, the head.

Different readings exist for the last verse quarter. \etaOne\ has \emph{jayati sā tu sūtrikā}, which is unmetrical, but the word \emph{sūtrikā} may have been original because it explains the problematic readings that arose in the other witnesses when attempts were made to replace \emph{sūtrikā} with \emph{neti}. The variants with \emph{netivit} seem implausible because of the epithets in the first hemistich, which require a feminine noun to be understood as the subject of the sentence. The reading \emph{netir āśu nihanti ca} looks like a patch that was adopted later in the transmission. The reading we have conjectured \emph{jayaty āśu tu sūtrikā} retains \emph{sūtrikā} and corrects the metrical fault of \etaOne\ by adopting \emph{jataty āśu}, which is well attested across the stemma.
\end{philcomm}

%%%%%%%%%%
\subsection*{2.32}
\begin{translation}[hp02_032]
Now \emph{trāṭaka}:\\
{}[The yogi] should concentrate and look at a small focal point with a fixed gaze until tears fall. The experts consider this to be trāṭaka.
\end{translation}
% adopt matam

%\begin{sources}[hp02_032]
%\end{sources}

\begin{testimonia}[hp02_032]
\emph{Haṭharatnāvalī} 1.54

\begin{versinnote}
\tl{atha trāṭakam-\\+}
\tl{nirīkṣya niścaladṛśā sūkṣmalakṣyaṃ samāhitaḥ |\\+}
\tl{aśrusampātaparyantam ācāryais trāṭakaṃ smṛtam ||\\!}
\end{versinnote}

\emph{Yogacintāmaṇi} 71v (attr.~to the \emph{Haṭhapradīpikā})

\begin{versinnote}
\tl{atha trāṭakam |\\+}
\tl{nirīkṣen niścaladṛśā sūkṣmalakṣyaṃ samāhitaḥ |\\+}
\tl{aśruprapātaparyantam āryais tat trāṭakaṃ matam ||\\!}
\end{versinnote}

\emph{Yuktabhavadeva} 7.158 (attr.~to the \emph{Haṭhapradīpikā})

\begin{versinnote}
\tl{atha trāṭakam |\\+}
\tl{vīkṣeta niścaladṛśā sulakṣyaṃ ca samāhitaḥ |\\+}
\tl{aśrusampātaparyantam ācāryais trāṭakaṃ smṛtam ||\\!}
\end{versinnote}

Cf. \emph{Satkarmasaṅgraha} 40cd–41ab

\begin{versinnote}
\tl{atha trāṭakam\\+}
\tl{sūkṣmalakṣye dṛśau sthāpya nirnimeṣaś ciraṃ bhavet |\\+}
\tl{aśrusampātaparyantaṃ karma trāṭakam īritam  ||\\!}
\end{versinnote}

\end{testimonia}

%\begin{philcomm}[hp02_032]

%\end{philcomm}

\begin{metre}[hp02_032]
Anuṣṭubh (a: na-vipulā)
\end{metre}

%%%%%%%%%%
\subsection*{2.33}
\begin{translation}[hp02_033]
It is the destroyer of eye diseases and the door [shutting out] sloth and so forth. Trāṭaka should be carefully concealed like a chest of gold.
\end{translation}

%\begin{sources}[hp02_033]
%\end{sources}

\begin{testimonia}[hp02_033]
\emph{Haṭharatnāvalī} 1.55

\begin{versinnote}
\tl{sphoṭanaṃ netrarogāṇāṃ tandrādīnāṃ kapāṭakam |\\+}
\tl{prayatnāt trāṭakaṃ gopyaṃ yathā ratnasupeṭakam ||\\!}
\end{versinnote}

\emph{Yogacintāmaṇi} 71v (attr.~to the \emph{Haṭhapradīpikā})

\begin{versinnote}
\tl{moṭanaṃ netrarogānāṃ tandrādīnāṃ kapāṭakam |\\+}
\tl{etac ca trāṭakaṃ gopyaṃ yathā hāṭakapeṭakam ||\\!}
\end{versinnote}

\emph{Yuktabhavadeva} 6.159 (attr.~to the \emph{Haṭhapradīpikā})

\begin{versinnote}
\tl{moṭakaṃ sarvarogāṇāṃ tandrādīnāṃ kapāṭanam ||\\+}
\tl{yatnatas trāṭakaṃ gopyaṃ yathā hāṭakapeṭakam ||\\!}
\end{versinnote}

Cf. \emph{Satkarmasaṅgraha} 41cd–42ab

\begin{versinnote}
\tl{atha trāṭakam\\+}
\tl{vaṅglāvikaraṇasthe 'sminn antarjyotiḥ prakāśyate |\\+}
\tl{netrarogās tathā tandrā naśyantīty āha dhūrjatiḥ ||\\!}
\end{versinnote}
\end{testimonia}

\begin{philcomm}[hp02_033]
The witnesses have many different readings for the first word of this verse. The \textalpha\ manuscripts are split between \emph{modaka} (\alphaOne), \emph{mocaka} and \emph{mohana}. Of these, \emph{modaka} is the most likely, if it is understood as a medicinal pill. However this meaning is rare, even in medical literature, as \emph{modaka} is generally used to refer to a small sweet. We have adopted \emph{moṭakaṃ}, which is attested by \etaOne\ and \deltaTwo, as well as the \emph{Yuktabhavadeva}. We understand it to mean “destroyer.” Bohtlingk and Roth (s.v.) and Monier-Williams (s.v.) give medicinal pill as a possible meaning of \emph{moṭaka} (cf. \emph{modaka}) but it appears that this is mainly an inference drawn only from this verse, where the pill is merely a comparison. Several witnesses have \emph{sphoṭanaṃ} (‘destroying’), which is also possible. 

The reading \emph{kapāṭakam} in the second verse quarter is found in most of the witnesses and testimonia and we have adopted it accordingly. The context indicates that it means “shutter” (in the sense of shutting out something), but we have not found any parallel usages of it in this sense.
\end{philcomm}

%%%%%%%%%%
\subsection*{2.34}
\begin{translation}[hp02_034]
Now \emph{naulī}:\\
With the shoulders lowered, [the yogi] should rotate the stomach to the left and right with the speed of a rapid whirlpool. This is called naulī by people from Gauḍa.%
\end{translation}

%\begin{sources}[hp02_034]
%\end{sources}

\begin{testimonia}[hp02_034]
\emph{Haṭharatnāvalī} 1.34 

\begin{versinnote}
\tl{atha nauliḥ--\\+}
\tl{amandāvartavegena tundaṃ savyāpasavyataḥ |\\+}
\tl{natāṃso bhrāmayed eṣā nauliḥ gauḍaiḥ praśasyate ||\\!}
\end{versinnote}

\emph{Yogacintāmaṇi} 71v (attr.~to the \emph{Haṭhapradīpikā})

\begin{versinnote}
\tl{atha naulī\\+}
\tl{amandāvartavegena tundaṃ savyāpasavyayoḥ |\\+}
\tl{natāṃso bhrāmayed eṣā naulī yoge pracakṣate ||\\!}
\end{versinnote}

\emph{Yuktabhavadeva} 6.162 (attr.~to the \emph{Haṭhapradīpikā})

\begin{versinnote}
\tl{atha naulī\\+}
\tl{amandāvartavegena tundaṃ savyāpasavyataḥ |\\+}
\tl{natāṃso bhrāmayed eṣā naulir gauḍaiḥ praśasyate ||\\!}
\end{versinnote}

Cf. \emph{Satkarmasaṅgraha} 110cd–111

\begin{versinnote}
\tl{atha naulī\\+}
\tl{amandāvartavegena jaṭharaṃ dakṣavāmayoḥ |\\+}
\tl{cālayec chaṃbhunā proktaṃ tatra lakṣmyadhidevatā  |\\+}
\tl{bāhyanaulir iyaṃ proktā jaṭharānaladīpinī ||\\!}
\end{versinnote}

\end{testimonia}

\begin{philcomm}[hp02_034]
%\emph{amandācakravegena} could mean the speed of a slightly slow wheel (Jürgen’s suggestion). But this is strange so we have adopted the usual metaphor of a fast moving whirlpool. (JB: varta is overwhelming supported, including alpha)
It is worth noting that \alphaOne\ has \emph{laulī} (instead of \emph{naulī}) as the name of this practice in verses 2.34–35 and \emph{laulikaṃ} in 2.22. The vast majority of manuscripts, including \alphaTwo, and the most important testimonia support \emph{naulī} or \emph{nauliḥ} but the names \emph{laulika} and \emph{laulikī} do occur in some more recent works, such as the \emph{Haṭhayogasaṃhitā} (p. 4), \emph{Gheraṇḍasaṃhitā} (1.12), \emph{Yogasārasaṅgraha} (pp. 54–55) and \emph{Yogakarṇikā} (p. 56). 

%??% JM to supply Babaji video to be uploaded to our host. Send to Mitsuyo.

The reading \emph{gauḍaiḥ} is found in only one collated manuscript (\etaTwo) but it is close to the readings of two \textalpha\ manuscripts, \emph{gaulaiḥ} (\alphaOne) and \emph{golaiḥ} (\alphaThree), and is also attested by the \emph{Haṭharatnāvalī} and \emph{Yuktabhavadeva}. This version of the verse, which is likely original, suggests that it was thought that the name \emph{naulī} came from the region known today as Bengal. The reading \emph{gauḍaiḥ} explains many of the mistakes in other manuscripts, such as \emph{gaulīḥ}, \emph{maulaiḥ}, etc.
\end{philcomm}

%%%%%%%%%%
\subsection*{2.35}
\begin{translation}[hp02_035]
Naulī brings about stimulation of the fire in the stomach, [good] digestion and the like, always brings bliss, and makes all humoural disorders and diseases wither away. This naulī is the best of all Haṭha techniques.
\end{translation}
%adopt tundāgni


\begin{sources}[hp02_035]
\end{sources}

\begin{testimonia}[hp02_035]
\emph{Haṭharatnāvalī} 1.35

\begin{versinnote}
\tl{tundāgnisandīpanapācanādisandīpikānandakarī sadaiva |\\+}
\tl{aśeṣadoṣāmayaśoṣaṇī ca haṭhakriyāmaulir iyaṃ ca nauliḥ ||\\!}
\end{versinnote}

\emph{Yogacintāmaṇi} 71v (attr.~to the \emph{Haṭhapradīpikā})

\begin{versinnote}
\tl{mandāgnisaṃdīpanapācanāgnisaṃdhāyikānandakarī tathaiva |\\+}
\tl{aśeṣadoṣāmayaśoṣinī ca haṭhakriyāmaulir iyaṃ hi naulī ||\\!}
\end{versinnote}

\emph{Yuktabhavadeva} 7.163 (attr.~to the \emph{Haṭhapradīpikā})

\begin{versinnote}
\tl{mandāgnisandīpanapācanādisandhāvanānandakarī sadaiva |\\+}
\tl{aśeṣadoṣāmayaśoṣaṇīyaṃ haṭhakriyāmaulir iyaṃ hi nauliḥ ||\\!}
\end{versinnote}

\emph{Haṭhatattvakaumudī} 8.12

\begin{versinnote}
\tl{mandāgnisandīpanapācanādisandhāyikānandakarī sadaiva |\\+}
\tl{aśeṣadoṣopacayaśoṣaṇīva haṭhakriyā 'sau jayatīha nauliḥ ||\\!}
\end{versinnote}


\end{testimonia}

\begin{philcomm}[hp02_035]
\alphaOne\ and \alphaTwo, \betaTwo, and \epsilonOne\ have \emph{tundāgni°} in the first verse quarter. This reading is also supported by all of the manuscripts collated for the published edition of the \emph{Haṭharatnāvalī}. The compound \emph{tundāgni} is rare in Sanskrit literature but it was likely used here because the term \emph{tunda} appears in the previous verse, which explains how naulī is done. The compound \emph{tundāgni} seems synonymous with \emph{udarāgni}, \emph{jaṭharāgni}, \emph{śarīrāgni}, etc., which are commonly used in yoga texts to refer to the body's digestive fire. The alternative reading \emph{mandāgni°} (`sluggish fire') is reasonably common in contexts of stimulating poor digestion, and may have been introduced early in the transmission to replace the more unusual \emph{tundāgni°}. 

%We should adopt \emph{pācana} in the sense of digestion (cooking the food in the intestines). \emph{āmaya} makes better sense in the third pāda, and \emph{mala} might have crept in because of the association of this word with \emph{śoṣiṇī}. (JB: Both pācana and āmaya are well attested, so no need to comment).  

Most witnesses and the testimonia have \emph{maulir iyaṃ} in the fourth verse quarter, which expresses the idea that naulī was thought to be the best of the \emph{ṣaṭkarma}, and the assonance of \emph{naulī} and \emph{mauli} may have been intended. The alternative reading of \emph{mūlam iyaṃ} in \etaOne\ and \epsilonOne\ would suggest that \emph{nauli} is necessary for the other practices, which does not seem to be the case because, according to \emph{Haṭhapradīpikā} 2.21, the \emph{ṣaṭkarma} are more like therapeutic interventions.
\end{philcomm}

\begin{metre}[hp02_035]
Upajāti
\end{metre}

%%%%%%%%%%
\subsection*{2.36}
\begin{translation}[hp02_036]
Now kapālabhātī:\\
Rapid inhalation and exhalation like the bellows of a blacksmith is called kapālabhātī, the skull bellows. It dries up imbalances of phlegm.
\end{translation}

%\begin{sources}[hp02_036]
%\end{sources}

\begin{testimonia}[hp02_036]
\emph{Haṭharatnāvalī} 1.56

\begin{versinnote}
\tl{atha kapālabhastrikā--\\+}
\tl{bhastrival lohakārāṇāṃ recapūrasusambhramau |\\+}
\tl{kapālabhastrī vikhyātā sarvarogaviśoṣaṇī ||\\!}
\end{versinnote}

\emph{Yogacintāmaṇi} 71v (attr.~to the Haṭhapradīpikā)

\begin{versinnote}
\tl{atha kapālabhātī |\\+}
\tl{bhastreva lohakārāṇāṃ recapūrau sasaṃbhramau |\\+}
\tl{kapālabhātī vikhyātā kaphadoṣaviśoṣiṇī ||\\!}
\end{versinnote}

\emph{Yuktabhavadeva} 7.163 (attr.~to the \emph{Haṭhapradīpikā})

\begin{versinnote}
\tl{atha kapālabhātiḥ |\\+}
\tl{bhastrāval lohakārāṇāṃ recapūrau sasambhramau |\\+}
\tl{kapālabhātir vikhyātā kaphadoṣaviśoṣiṇī ||\\!}
\end{versinnote}

Cf. \emph{Satkarmasaṅgraha} 50cd–51

\begin{versinnote}
\tl{atha bhastrā\\+}
\tl{lohakārasya bhastreva recapūrau tu vegataḥ ||\\+}
\tl{punaḥ punaḥ prakurvīta sthiramūrdhnā prayatnataḥ |\\+}
\tl{sthirabhastreti ca khyāta yogināṃ siddhidāyakā ||\\!}
\end{versinnote}

\end{testimonia}

\begin{philcomm}[hp02_036]
The word \emph{bhātī} is derived from \emph{bhastrī} (Turner 1966: 537, entry 9424).

\etaTwo\ and other manuscripts have \emph{kuryāt savyāpasavyataḥ} (‘left and right’) instead of \emph{recapūrau sasambhramau}. Although one might infer that \emph{savyāpasavyataḥ} is referring to performing the skull bellows breathing alternately through the left and right nostrils, it appears to be a secondary reading because there is no indication of what is moving to the left and right. Such a method of alternate nostril breathing is explained as a variation of kapālabhāti in the \emph{Haṭhayogasaṃhitā} (p. 14):
\begin{versinnote}
\tl{iḍayā pūrayed vāyuṃ recayet piṅgalākhyayā |\\+}
\tl{piṅgalayā pūrayitvā punaś candreṇa recayet |\\+}
\tl{pūrakaṃ recakaṃ kṛtvā vegena na tu cālayet ||\\!}
\end{versinnote}

In the second verse quarter, the \textalpha\ manuscripts have \emph{susambhramau} ('great speed') instead of the reading we have adopted \emph{sasambhramau} (`fast'). The syntax requires \emph{sasambhramau} because it is an adjectival compound that must agree with \emph{recapūrau} (`the exhalation and inhalation').  
\end{philcomm}

\begin{metre}[hp02_036]
Anuṣṭubh (c: ma-vipulā)
\end{metre}

%%%%%%%%%%
\subsection*{2.37}
\begin{translation}[hp02_037]
The person whose excess weight, phlegm, fat, impurities and the like have been removed by the six techniques should then perform breath-control. It succeeds without effort.
\end{translation}

%\begin{sources}[hp02_037]
%\end{sources}

\begin{testimonia}[hp02_037]
\emph{Haṭharatnāvalī} 1.60

\begin{versinnote}
\tl{karmāṣṭabhir gatasthaulyaṃ kaphamedomalādikam |\\+}
\tl{prāṇāyāmaṃ tataḥ kuryād anāyāsena siddhyati  ||\\!}
\end{versinnote} 

\emph{Yogacintāmaṇi} f.~8v

\begin{versinnote}
\tl{tathā cātmārāmaḥ\\+}
\tl{ṣaṭkarmanirgatasthaulyakaphamedogadādikaḥ |\\+}
\tl{prāṇāyāmaṃ tataḥ kuryād anāyāsena sidhyati ||\\!}
\end{versinnote}

\emph{Yuktabhavadeva} 7.165 (attr.~to the \emph{Haṭhapradīpikā})

\begin{versinnote}
\tl{ṣaṭkarmabhir gatasthaulyaṃ kaphamedomalātigaḥ |\\+}
\tl{prāṇāyāmaṃ tataḥ kuryād anāyāsena sidhyati ||\\!}
\end{versinnote}
\end{testimonia}

%\begin{philcomm}[hp02_037]
%Most witnesses support \emph{ṣaṭkarmabhir gata°}, which is somewhat unconventional syntax. It seems more likely that it was corrected to \emph{ṣaṭkarmanirgata°} than the latter being original. (JB: \emph{ṣaṭkarmabhir gata°} is overwhelmingly supported. No need to justify it.).
%The variant reading \emph{°ādhikaḥ} in the second \emph{pāda} is inferior because one would assume that the \emph{ṣaṭkarma} should remove all excess weight (\emph{sthaulya}) and impurities (\emph{mala}). 
%\end{philcomm}

%%%%%%%%%%
\subsection*{2.38}
\begin{translation}[hp02_038]
Some teachers say that all impurities are dried up by means of breath-controls alone and do not recommend any other practice.
\end{translation}

%\begin{sources}[hp02_038]
%\end{sources}

\begin{testimonia}[hp02_038]
\emph{Yogacintāmaṇi} ff.~8v–9r (attr.~to Ātmārāma)

\begin{versinnote}
\tl{prāṇāyāmair eva sarvaiḥ praśuṣyanti malā yataḥ iti |\\+}
\tl{ācāryāṇāṃ tu keṣāṃ cid anya krama na saṃmatam iti ||\\!}
\end{versinnote}

\emph{Yuktabhavadeva} 7.166 (attr.~to the \emph{Haṭhapradīpikā})

\begin{versinnote}
\tl{prāṇāyāmair eva sarve praśuṣyanti malā iti |\\+}
\tl{ācāryāṇāṃ tu keṣāñ cid anyat karma na sammatam ||\\!}
\end{versinnote}
\end{testimonia}

\begin{philcomm}[hp02_038]
The plural of \emph{prāṇāyāma}, which we have translated here as ‘breath-controls’, probably refers to practising multiple repetitions of breath retentions with alternate nostril breathing. 

%The verb \emph{praśuṣyanti} is well attested and makes good sense in the context of impurities (\emph{mala}), which usually supposed to dry up and wither away because of the practice of yoga. (JB: \emph{praśuṣyanti}  is so well attested and straight forward that there's no need to comment on it) 
Many witnesses lower on the stemma have \emph{malāśaya} (instead of \emph{malā api} or \emph{malā iti}) in the second verse quarter but this usually has the more specific meaning of bowels or bladder and so seems inappropriate in a general statement. The witnesses that have \emph{malāśaya} also have the verb \emph{pra+śudh}, which connotes that the place where the impurities accumulate is cleaned (rather than the impurities themselves). 
\end{philcomm}

\begin{metre}[hp02_038]
Anuṣṭubh (a: ra-vipulā)
\end{metre}

%%%%%%%%%%
\subsection*{2.39}
\begin{translation}[hp02_039]
Even Brahmā and the other gods became devoted to breath practice through fear of death, so one should perform breath practice.
\end{translation}

%\begin{sources}[hp02_039]
%\end{sources}

\begin{testimonia}[hp02_039]
\emph{Haṭharatnāvalī} 3.82

\begin{versinnote}
\tl{brahmādayo 'pi tridaśāḥ pavanābhyāsatatparāḥ |\\+}
\tl{abhūvan mṛtyurahitā tasmāt pavanam abhyaset ||\\!}
\end{versinnote}

\emph{Haṭhatattvakaumudī} 8.19

\begin{versinnote}
\tl{brahmādayo 'pi tridaśāḥ pavanābhyāsatatparāḥ |\\+}
\tl{tena siddhiṃ gatā yoge tasmāt pavanam abhyaset ||\\!}
\end{versinnote}
\end{testimonia}

\begin{philcomm}[hp02_039]
This verse has been rewritten in \etaTwo\ and the delta group of manuscripts. \etaTwo's reading attempts, somewhat unsuccessfully, to connect this verse more directly to the \emph{ṣaṭkarma}:
\begin{versinnote}
\tl{ṣaṭkarmayogam āpnoti pavanābhyāsatatparaḥ |\\+}
\tl{sumanaskāṃtako bhavya[s] tasmāt pavanam abhyaset ||\\!}
\end{versinnote}
Different versions of this verse occur in \etaTwo\ and \epsilonThree. Neither of these appear to be original as the compound \emph{sumanaskāntaka} (`dying with the mind active'?) is rather odd. The \delta \ group of manuscripts has another reading for the third quarter (\emph{tena siddhiṃ gatās te ca}), which puts a more positive spin on the verse in as much as one should practise \emph{prāṇāyāma} because through it the gods attained perfection. 
%\epsilonThree\ has a similar version of the third quarter (\emph{samanaskāntakabhayaṃ}), which might refer to the fear of dying with the mind active as opposed to the preferable situation of dying while in \emph{samādhi}.
\end{philcomm}

\begin{metre}[hp02_039]
Anuṣṭubh (a: bha-vipulā; c: na-vipulā)
\end{metre}

%%%%%%%%%%
\subsection*{2.40}
\begin{translation}[hp02_040]
So long as the breath is bound in the body, so long as the mind is without support, so long as the gaze is on the middle of the brow, where is the fear of death?
\end{translation}
% adopt vīkṣā from Six-chapter VM

\begin{sources}[hp02_040]
\emph{Vivekamārtaṇḍa} 73

\begin{versinnote}
\tl{yāvad baddho marud dehe tāvac cittaṃ nirāśrayam |\\+}
\tl{yāvad vīkṣā bhruvor madhye tāvat kālabhayaṃ kutaḥ ||\\+}
\tl{\var{vīkṣā] T; dṛṣṭir \emph{cett.}}}

\end{versinnote}
\end{sources}

\begin{testimonia}[hp02_040]
\emph{Yogacintāmaṇi} f.~92r (attr.~to the Skandapurāṇa)

\begin{versinnote}
\tl{yāvad baddho marud dehe yāvad vṛttau nirāśrayam |\\+}
\tl{yāvad dṛṣṭir bhruvor madhye tāvat kālabhayaṃ kutaḥ ||\\!}
\end{versinnote}

\emph{Yuktabhavadeva} 7.8 (attr.~to Gorakṣanātha)

\begin{versinnote}
\tl{yāvad baddho marud dehe yāvac cittaṃ nirāmayam |\\+}
\tl{yāvad dṛṣṭir bhruvor madhye tāvat kālabhayaṃ kutaḥ ||\\!}
\end{versinnote}
\end{testimonia}

\begin{philcomm}[hp02_040]
We have adopted the reading \emph{vīkṣā} in the third verse quarter, which is an emendation of \alphaOne\ and \alphaTwo's \emph{vīkṣed}. The verb \emph{vīkṣed} is likely an error as the \emph{yāvat} clauses appear to have been written as nominal phrases. The reading \emph{vīkṣā} occurs in the same verse of the six-chapter version of the \emph{Vivekamārtaṇḍa} (siglum T), which sometimes preserves old readings of that text.  

%The verb \emph{paśyet} in the third \emph{pāda} is well attested among the manuscripts (V1, J10, etc.). However, it often occurs with \emph{bhruvor madhye}, which is rather odd and suggests that the alternative reading \emph{dṛṣṭir} was original. In fact, \emph{dṛṣṭir} is supported by the manuscripts of the source text, the \emph{Vivekamārtaṇḍa}, and the testimonia. (JB: This is an old note).
\end{philcomm}

%%%%%%%%%%
\subsection*{2.41}
\begin{translation}[hp02_041]
When the network of channels has been purified by breath-controls as prescribed, the breath pierces the mouth of \emph{suṣumṇā} and enters it with ease.
\end{translation}

%\begin{sources}[hp02_041]
%\end{sources}

\begin{testimonia}[hp02_041]
\emph{Haṭharatnāvalī} 2.2

\begin{versinnote}
\tl{vidhivat prāṇasaṃyāmaiḥ nāḍicakre viśodhite |\\+}
\tl{suṣumnāvadanaṃ bhitvā sukhād viśati mārutaḥ ||\\!}
\end{versinnote}

\emph{Yogacintāmaṇi} f.~18r

\begin{versinnote}
\tl{haṭhapradīpikāyām—\\+}
\tl{vividhaiḥ prāṇasaṃyāmaiḥ nāḍīcakre viśodhite |\\+}
\tl{suṣumnāvadanaṃ bhitvā sukhād viśati mārutaḥ ||\\!}
\end{versinnote}


\end{testimonia}

\begin{philcomm}[hp02_041]
The compound \emph{prāṇasaṃyāmaiḥ} (‘breath-controls’) should be understood here as a synonym for \emph{prāṇāyāmaiḥ} as found in 2.38 (on which see the note on this verse).
\end{philcomm}

%%%%%%%%%%
\subsection*{2.42}
\begin{translation}[hp02_042]
When the breath moves in the middle, stillness of the mind arises. The mind’s becoming very still is the ‘mind beyond mind’ (\emph{manonmanī}) state.
\end{translation}

%\begin{sources}[hp02_042]
%\end{sources}

\begin{testimonia}[hp02_042]
\emph{Haṭharatnāvalī} 2.3

\begin{versinnote}
\tl{mārute madhyame jāte manaḥsthairyaṃ prajāyate |\\+}
\tl{manasaḥ susthirībhāvaḥ saivāvasthā manonmanī ||\\!}
\end{versinnote}

\emph{Yogacintāmaṇi} f.~18a (attr.~to Haṭhapradīpikā)

\begin{versinnote}
\tl{mārute madhyasaṃcāre manaḥsthairyaṃ prajāyate |\\+}
\tl{yo manaḥsusthirībhāvaḥ saivāvasthā manonmanī ||\\!}
\end{versinnote}

\end{testimonia}

%\begin{philcomm}[hp02_042]
%\end{philcomm}

%%%%%%%%%%
\subsection*{2.43}
\begin{translation}[hp02_043]
In order to achieve that, he who knows [their] methods should perform various retentions. As a result of the practice of the various retentions, [the yogi] obtains various results.
\end{translation}
% adopt the singular °jñaś .. kurvīta

%\begin{sources}[hp02_043]
%\end{sources}

\begin{testimonia}[hp02_043]
\emph{Haṭharatnāvalī} 2.4
\begin{versinnote}
\tl{tatsiddhaye vidhānajñaḥ sadā kurvīta kumbhakān |\\+}
\tl{vicitrakumbhakābhyāsād vicitrāṃ siddhim āpnuyāt ||\\!}
\end{versinnote}

\emph{Yuktabhavadeva} 7.92 (attr.~to the \emph{Yājñavalkyagītā})

\begin{versinnote}
\tl{tatsiddhaye vidhānajñāś citrān kurvanti kumbhakān |\\+}
\tl{vicitrakumbhakābhyāsād vicitrāṃ siddhim āpnuyāt ||\\!}
\end{versinnote}
\end{testimonia}

\begin{philcomm}[hp02_043]
The majority of witnesses, including \alphaTwo, have a plural subject (\emph{vidhānajñāḥ}) and verb (\emph{kurvanti}) in the first hemistich. We have adopted the singular, which is attested by \alphaOne, \alphaThree\ and manuscripts of the \emph{Haṭharatnāvalī}, because it corresponds with the singular subject of the second hemistich. 
\end{philcomm}

%%%%%%%%%%
\subsection*{2.44}
\begin{translation}[hp02_044]
Sūryabhedana, ujjāyī, śītkā, śītalī, bhastrikā, bhrāmarī, mūrcchā and plāvanī: these are the eight kumbhakas.
\end{translation}
% adopt śītka
% adopt plāvanī

\begin{sources}[hp02_044]
\end{sources}

\begin{testimonia}[hp02_044]

\emph{Haṭharatnāvalī} 2.6
\begin{versinnote}
\tl{sūryabhedanam ujjayī tathā sītkāraśītalī |\\+}
\tl{bhastrikā bhrāmarī mūrcchā kevalaś cāṣṭa kumbhakāḥ ||\\!}
\end{versinnote}

\emph{Yogalakṣaṇāvalī} f.~32r (attrib. to the \emph{Haṭhapradīpikā})
\begin{versinnote}
\tl{sūryabhedanam ujjayī tathā sītkā ca sītalī |\\+}
\tl{bhastrikā bhrāmarī mūrcchā kevalāś cāṣṭa kumbhakāḥ ||\\!}
\end{versinnote}

\emph{Yogacintāmaṇi} f.~101r

\begin{versinnote}
\tl{haṭhayoge—\\+}
\tl{sūryabhedanam ujjāyī tathā sītkāraśītalī |\\+}
\tl{bhastrikā bhramarī mūrcchā sahitaṃ cāṣṭa kumbhakāḥ ||\\!}
\end{versinnote}

\emph{Yuktabhavadeva} 7.93 (attr.~to the \emph{Haṭhapradīpikā})
\begin{versinnote}
\tl{sūryabhedanamujjāyī sītkārī śītalī tathā |\\+}
\tl{bhastrikā bhrāmarī mūrcchā kevalāś cāṣṭa kumbhakāḥ ||\\!}    
\end{versinnote}

\end{testimonia}

\begin{philcomm}[hp02_044]
In the fourth verse quarter, only some manuscripts of the \textepsilon\ and \textgamma\ groups have the adopted reading \emph{plāvanīty aṣṭa}. The name \emph{plāvanī} is necessary in this list because most manuscript groups, including \textalpha, and the testimonia contain a verse on \emph{plāvinīkumbhaka} as one of the eight \emph{kumbhaka}s. The alternative reading \emph{kevalī/kevalaś}, which is supported by the \textalpha\ group and other manuscripts, appears to be a mistake because it is not consistent with the idea of \emph{kevalakumbhaka} standing outside the category of \emph{sahitakumbhaka}, as stated in 2.72–75. These witnesses are among those which include a verse on \emph{plāvinī} later in the chapter. 
%
%This idea of two types of kumbhaka is found in the source text from which Svātmārāma borrowed four of the kumbhakas, namely, the \emph{Gorakṣaśataka}. 
%Therefore, it seems likely the word \emph{plāvinī} dropped out of some manuscripts and was replaced by \emph{kevala} because \emph{kevala} is discussed later in the chapter. In fact, \emph{plāvinī} may have been removed by some scribes because its heading is omitted in many witnesses. The reading in V19 and the \emph{Yogacintāmaṇi} (\emph{sahitāś cāṣṭa}) was probably an attempt to remove \emph{kevalī/kevalaś} from the list.
% mention Haṭharatnāvalī (absence of plāvinī)
% [JB I think the second paragraph is unnecessary]
\end{philcomm}

%%%%%%%%%%
\subsection*{2.45}
\begin{translation}[hp02_045]
At the end of the inhalation, the lock called \emph{jālandhara} is to be performed, while at the end of the retention and beginning of the exhalation, \emph{uḍḍiyāna} is to be performed.
\end{translation}

\begin{sources}[hp02_045]
\emph{Gorakṣaśataka} 62ab

\begin{versinnote}
\tl{pūrakānte tu kartavyo bandho jālandharābhidhaḥ |\\!}
\end{versinnote}
\emph{Gorakṣaśataka} 58ab

\begin{versinnote}
\tl{kuṃbhakānte recakādau kartavyoḍḍiyaṇābhidhaḥ |\\!}
\end{versinnote}
\end{sources}

\begin{testimonia}[hp02_045]
\emph{Haṭharatnāvalī} 2.7

\begin{versinnote}
\tl{pūrakānte tu kartavyo bandho jālandharābhidhaḥ |\\+}
\tl{kumbhakānte recakādau kartavyas tūḍḍiyānakaḥ ||\\!}
\end{versinnote} 

\emph{Yogacintāmaṇī} f.~80r (attr.~to the \emph{Yogabīja})

\begin{versinnote}
\tl{pūrakānte tu kartavyo bandho jālandharābhidhaḥ |\\+}
\tl{kumbhakānte recakādau kartavyas tūḍḍiyānakaḥ ||\\!}
\end{versinnote}

\emph{Yuktabhavadeva} 7.94 (attrib. to the \emph{Haṭhapradīpikā})

\begin{versinnote}
\tl{pūrakānte ca karttavyo bandho jālandharābhidhaḥ |\\+}
\tl{kumbhakānte recakādau karttavyas tūḍḍiyānakaḥ || \\!}
\end{versinnote}
\end{testimonia}

%\begin{philcomm}[hp02_045]
%\end{philcomm}

\begin{metre}[hp02_045]
Anuṣṭubh (c: ra-vipulā)
\end{metre}

%%%%%%%%%%
\subsection*{2.46}
\begin{translation}[hp02_046]
By quickly contracting the lower region when the throat has been constricted and stretching back the middle [of the body] the breath goes into the channel of Brahman.
\end{translation}

\begin{sources}[hp02_046]
\emph{Gorakṣaśataka} 62cd–63ab
\begin{versinnote}
\tl{adhastāt kuñcanenaiva kaṇṭhasaṃkocane kṛte |\\+}
\tl{madhye paścimatāṇena syāt prāṇo brahmanāḍigaḥ ||\\!}
\end{versinnote}
\end{sources}

\begin{testimonia}[hp02_046]
\emph{Haṭharatnāvalī} 2.8
\begin{versinnote}
\tl{adhastāt kuñcanenāśu kaṇṭhasaṅkocane kṛte |\\+}
\tl{madhye paścimatānena syāt prāṇo brahmanāḍigaḥ ||\\!}
\end{versinnote}

\emph{Yogacintāmaṇī} f.80r (attr.~to the \emph{Yogabīja})

\begin{versinnote}
\tl{adhas tv ākuñcanenāśu kaṇṭhasaṅkocanena ca |\\+}
\tl{madhye paścimatānena syāt prāṇo brahmanāḍigaḥ ||\\!}
\end{versinnote}

\emph{Yogabīja} 110 (southern recension)
\begin{versinnote}
\tl{adhastāt kuñcanenāśu kaṇṭhasaṅkocane kṛte |\\+}
\tl{madhye paścimatānena syāt prāṇo brahmanāḍigaḥ ||\\!}
\end{versinnote}

\emph{Yuktabhavadeva} 7.95 (attr.~to the \emph{Haṭhapradīpikā})

\begin{versinnote}
\tl{adhastāt kuñcanenāśu kaṇṭhasaṃkocane kṛte |\\+}
\tl{madhye paścimatānena syāt prāṇo madhyanāḍiga ||\\!}
\end{versinnote}

\emph{Haṭhatattvakaumudī} 15.25–27
\begin{versinnote}
\tl{adhastāt kuñcanenaiva kaṇṭhasaṃkocanena ca |\\+}
\tl{madhye paścimatānena syāt prāṇo brahmarandhragaḥ ||\\+}
\tl{prāṇaḥ prāṇavāyuḥ brahmarandhragaḥ suṣumnāpathacārī syāt | madhyago bhavet | \\!}
\end{versinnote}
\end{testimonia}

\begin{philcomm}[hp02_046]
%\emph{madhyapaścimatānena} is possible and well attested.
As is clear in the source text, the \emph{Gorakṣaśataka}, the three techniques alluded to here are \sl{mūlabandha, jālandharabandha} and \emph{uḍḍiyānabandha} respectively (on which see chapter three).
\end{philcomm}

%%%%%%%%%%
\subsection*{2.47}
\begin{translation}[hp02_047]
The yogi should raise up \emph{apānavāyu} and lead \emph{prāṇa} down from the throat. Freed from ageing, he becomes sixteen years old.
\end{translation}

%\begin{sources}[hp02_047]
%\end{sources}

\begin{testimonia}[hp02_047]
\emph{Haṭharatnāvalī} 2.9
\begin{versinnote}
\tl{apānam ūrdhvam utthāpya prāṇaṃ kaṇṭhād adho nayet |\\+}
\tl{yogī jarāvimuktaḥ syāt ṣoḍaśo vayasā bhavet ||\\!}
\end{versinnote} 

\emph{Yogacintāmaṇī} f.~80r (attr.~to the \emph{Yogabīja})

\begin{versinnote}
\tl{apānam ūrdhvam utthāpya prāṇaṃ kaṇṭhād adho nayet |\\+}
\tl{yogī jarāvimuktaḥ san vayasā ṣoḍaśo bhavet ||\\!}
\end{versinnote}

\emph{Yuktabhavadeva} 7.96 (attr.~to the \emph{Haṭhapradīpikā})

\begin{versinnote}
\tl{apānam ūrdhvam utthāpya prāṇaṃ kaṇṭhād adho nayet |\\+}
\tl{yogī jarāvinirmuktaḥ ṣoḍaśo vayasā bhavet ||\\!}
\end{versinnote}
\end{testimonia}

%\begin{philcomm}[hp02_047]
%\end{philcomm}

%%%%%%%%%%
\subsection*{2.48}
\begin{translation}[hp02_048]
Now, piercing the sun---\\
The yogi should sit in \emph{vajrāsana} on a comfortable mat, slowly draw in external air through the right nostril, [...]
\end{translation}

\begin{sources}[hp02_048]
Cf. \emph{Gorakṣaśataka} 33–34ab
\begin{versinnote}
\tl{pavitre nātyuccanīce hy āsane sukhade śubhe |\\+}
\tl{baddhvā vajrāsanaṃ kṛtvā sarasvatyāś ca cālanam ||\\+}
\tl{dakṣanāḍyāṃ samākṛṣya bahiṣṭhaṃ pavanaṃ śanaiḥ |\\!}
\end{versinnote} 
\end{sources}

\begin{testimonia}[hp02_048]
\emph{Yogalakṣaṇāvalī} f.~32r (attrib. to the \emph{Haṭhapradīpikā})
\begin{versinnote}
\tl{baddhavajrāsano dakṣanāḍyākṛṣyānilaṃ śanaiḥ |\\!}
\end{versinnote}

\emph{Yogacintāmaṇi} f.~101v (attr.~to the \emph{Yogabīja})

\begin{versinnote}
\tl{āsane sukhade yogī baddhavajrāsanas tataḥ |\\+}
\tl{dakṣanāḍyā samākṛṣya bahiḥsthaṃ pavanaṃ śanaiḥ ||\\!}
\end{versinnote} 

\emph{Yuktabhavadeva} 7.98 (attr.~to the \emph{Haṭhapradīpikā})

\begin{versinnote}
\tl{āsane sukhade yogī baddhvā padmāsanaṃ tataḥ |\\+}
\tl{dakṣanāḍyā samākṛṣya bahiḥsthaṃ pavanaṃ śanaiḥ ||\\!}
\end{versinnote}
\end{testimonia}

\begin{philcomm}[hp02_048]
%\emph{āsane sukhade} is supported by the [different but parallel] reading in the Gorakṣa\-śataka. (JB: I've put the GŚ verses in the source section)
%After citing this verse with attribution to the \emph{Haṭhapradīpikā} in his \emph{Haṭhasaṅketacandrikā} (f.~78v), Sundaradeva remarks that \emph{sukhade} is a \emph{hetugarbhaviśeṣaṇam}, which indicates that he understood the first hemistich as saying that the yogi assumes \emph{vajrāsana} on a mat (\emph{āsana}) because it is comfortable.%JM:needs more explanation.
\end{philcomm}

%%%%%%%%%%
\subsection*{2.49}
\begin{translation}[hp02_049]
and hold the breath as far as the tips of the hair and nails until cessation [of the breath]. The wise man should then exhale the breath slowly through the left nostril.
\end{translation}
% adopt ā keśāgrān nakhāgrāc ca

%\begin{sources}[hp02_049]
%\end{sources}

\begin{testimonia}[hp02_049]
\emph{Yogalakṣaṇāvalī} f.~32r (attrib. to the \emph{Haṭhapradīpikā})
\begin{versinnote}
\tl{ā nakhāgrālakāgrāntaṃ kumbhayitvā yathāsukham |\\+}
\tl{savyanāḍyā tato mandaṃ recayet pavanaṃ sudhīḥ ||\\!}
\end{versinnote}

\emph{Yogacintāmaṇi} f.~101v (attr.~to the \emph{Yogabīja})
\begin{versinnote} 
\tl{ā keśāgraṃ nakhāgraṃ ca śirodhāvadhi kumbhakam |\\+}
\tl{tataḥ śanaiḥ savyanāḍyā recayet pavanaṃ sudhīḥ ||\\!}
\end{versinnote}

\emph{Yuktabhavadeva} 7.99 (attr.~to the \emph{Haṭhapradīpikā})

\begin{versinnote}
\tl{ā keśād ā nakhāgrāc ca nirodhāvadhi kumbhayet |\\+}
\tl{tataḥ śanaiḥ savyanāḍyā recayet pavanaṃ sudhīḥ ||\\!}
\end{versinnote}
\end{testimonia}

\begin{philcomm}[hp02_049]
%Three versions of  the first pāda are attested by the manuscripts:\\ 
%1. ā keśāgraṃ nakhāgraṃ ca (Gr3,Gr4c,V1).\\
%2. ā keśāgrān nakhāgrāc ca (N3,V3,J10).\\
%3. ā keśād ā nakhāgrāc ca (Gr2,Jyo).\\
%The second reading is supported by \textalpha\ and 
% (We've read with the stemma)

%When explaining \emph{ā keśād ā nakhāgrāc ca} in his \emph{Jyotsnā} (2.49), Brahmānanda cites a verse stating that if the breath is held forcefully, it escapes through the hair follicles or nails and destroys the body. The ablative with \emph{ā} (\emph{a keśād nakhāgrāc ca}) is also well-attested (group 2 manuscripts), and it would mean the same.
% [JB: this ref is in the discussion below]

An antecedent to the idea of \emph{prāṇāyāma} affecting the whole body (i.e., as far as the tips of the hair and nails) occurs in the \emph{Baudhāyanadharmasūtra} (4.1.23):

\begin{versinnote}
{}[The yogi] who is constantly engaged [in practice] should repeat breath retentions again and again. Extreme heat burns as far as the tips of the hair and nails. 
\end{versinnote}

\begin{versinnote}
\tl{āvartayet sadā yuktaḥ prāṇāyāmān punaḥ punaḥ |\\+}
\tl{ā keśāntān nakhāgrāc ca tapas tapyata uttamam ||\\!}
\end{versinnote}

The meaning of \emph{nirodhāvadhi} is not entirely clear, but all sources and the \emph{Jyotsnā} (2.49) agree on this reading. The problem is that to practise \emph{kumbhaka} ``up to cessation \emph{nirodha}'' seems to suggest that cessation is not that of the physical breath, which by definition ceases in \emph{kumbhaka}, but of the vital wind (\emph{prāṇa}) within the body. By citing a verse from an unnamed text, Brahmānanda seems to understand this verse as saying that the breath should very carefully (\emph{atiprayatnena}) be held as far as the extremities of the body so that it does not damage the body by exiting through the hair follicles:


\begin{versinnote}
When the breath has been stopped forcefully, it flows out through the hair follicles. This destroys the body and also causes skin diseases and the like.
\end{versinnote}

\begin{versinnote}
\tl{haṭhān niruddhaḥ prāṇo 'yaṃ romakūpeṣu niḥsaret |\\+}
\tl{dehaṃ vidārayaty eṣa kuṣṭhādi janayaty api ||\\!}%
\end{versinnote}
%??JM: how about emending to virodhāvadhi, “until it is uncomfortable”, which would make much better sense. Yes, to add to note but not emend.
\end{philcomm}

\begin{metre}[hp02_049]
Anuṣṭubh (c: ra-vipulā)
\end{metre}

%%%%%%%%%%
\subsection*{2.50}
\begin{translation}[hp02_050]
This purifies the skull, cures [imbalances] of the wind humour [and] gets rid of diseases caused by worms [so] should be done repeatedly. It is called the piercing of the sun.
\end{translation}

\begin{sources}[hp02_050]
\emph{Gorakṣaśataka} 35–36ab

\begin{versinnote}
\tl{kapālaśodhane vāpi recayet pavanaṃ sudhīḥ |\\+}
\tl{tundasya vātadoṣaghnaḥ kṛmidoṣaṃ nihanti ca ||\\+}
\tl{punaḥ punar idaṃ kāryaṃ sūryābhedam udāhṛtam |\\!}
\end{versinnote}
\end{sources}

\begin{testimonia}[hp02_050]
\emph{Haṭharatnāvalī} 2.11cd–12

\begin{versinnote}
\tl{kapālaṃ śodhanaṃ cāpi recayet pavanaṃ śanaiḥ ||\\+}
\tl{kapālaṃ ... śanaiḥ ] kapālaśodhanaṃ vātadoṣaghnaṃ kṛmināśanaṃ N,n1,n4.\\+}
\tl{ālasyaṃ vātadoṣaghnaṃ kṛmikīṭaṃ nihanti ca |\\+}
\tl{punaḥ punar idaṃ kāryaṃ sūryabhedākhyakumbhakam ||\\!}
\end{versinnote} 

\emph{Yogalakṣaṇāvalī} f.~32r (attrib. to the \emph{Haṭhapradīpikā})
\begin{versinnote}
\tl{kapālaśodhanaṃ caitad vātaghnaṃ kṛmidoṣanut ||\\!}
\end{versinnote}

\emph{Yogacintāmaṇi} f.~101v (attr.~to the \emph{Yogabīja})
\begin{versinnote} 
\tl{kapālaśodhanaṃ vātadoṣaghnaṃ kṛmidoṣahṛt |\\+}
\tl{punaḥ punar idaṃ kuryāt sūryabhedanam uttamam ||\\!}
\end{versinnote} 

\emph{Yuktabhavadeva} 7.100 (attr.~to the \emph{Haṭhapradīpikā})

\begin{versinnote}
\tl{kapālaśodhanaṃ vātadoṣaghanaṃ kṛmidoṣaham |\\+}
\tl{punaḥ punar idaṃ kāryaṃ sūryabhedam udāhṛtam ||\\!}
\end{versinnote}

\end{testimonia}

\begin{philcomm}[hp02_050]
%The \emph{Gorakṣaśataka} and \emph{Yogacintāmaṇi} support \emph{kṛmidoṣa}. [JB: Not an issue in the new collation]
The compound \emph{sūryabheda} is metri causa, as the name given in 2.44 and the heading of 2.48 is \emph{sūryabhedana}. Some witnesses, such as the delta group and \getsiglum{V19}, have attempted to reinstate the name \emph{sūryabhedana}.

Both \emph{°doṣaham} and \emph{°doṣahṛt} are well attested and possible. We have favoured the former because the \textalpha\ reading (\emph{doṣajam}) appears to be a corruption of it.
\end{philcomm}

%%%%%%%%%%
\subsection*{2.51}
\begin{translation}[hp02_051]
Now ujjāyī:

[The yogi] should close the mouth and gradually draw in the breath through the nostrils so that it comes into contact [with the region] from the throat to the chest and makes a sound.\end{translation}

\begin{sources}[hp02_051]
\emph{Gorakṣaśataka} 36c–37b

\begin{versinnote}
\tl{mukhaṃ saṃyamya nāḍībhyāṃ ākṛṣya pavanaṃ śanaiḥ |\\+}
\tl{yathā lagati kaṇṭhāt tu hṛdayāvadhi sasvanam || \\+}
\tl{\var{kaṇṭhāt tu] kaṇṭhaṃ tu T}\\!}
\end{versinnote}
\end{sources}

\begin{testimonia}[hp02_051]
\emph{Haṭharatnāvalī} 2.13
\begin{versinnote}
\tl{mukhaṃ saṃyamya nāḍībhyām ākṛṣya pavanaṃ śanaiḥ |\\+}
\tl{yathā lagati hṛtkaṇṭhaṃ hṛdayāvadhi svasvanaḥ ||\\+}
\tl{\var{hṛtkaṇṭhaṃ ] hṛtkaṇṭhe N, n1, n4, J} \\!}%
\end{versinnote}

\emph{Yogalakṣaṇāvalī} f.~32r (attrib.~to the \emph{Haṭhapradīpikā})
\begin{versinnote}
\tl{mukhaṃ saṃyamya nāsābhyāṃ ākṛṣya pavanaṃ śanaiḥ |\\+}
\tl{yathā lagati kaṇṭhe suḥ hṛdayāvadhi sasvanam ||\\!}
\end{versinnote}

\emph{Yogacintāmaṇi} f.~101v (attr.~to the \emph{Yogabīja})

\begin{versinnote}
\tl{mukhaṃ saṃyamya nāḍībhyām ākṛṣya pavanaṃ punaḥ |\\+}
\tl{yathā lagati hṛtkaṇṭhād dhṛdayāvadhi sasvanaḥ ||\\!}
\end{versinnote}

\emph{Yuktabhavadeva} 7.101 (attr.~to the \emph{Haṭhapradīpikā})

\begin{versinnote} 
\tl{mukhaṃ niyamya nāḍībhyām ākṛṣya pavanaṃ śanaiḥ |\\+}
\tl{yathā lagati kaṇṭhāt tu hṛdayāvadhi pūraṇam ||\\!}
\end{versinnote}

Cf.~\emph{Haṭhatattvakaumudī} 10.7
\begin{versinnote}
\tl{athojjāyī kumbhakaḥ\\+}
\tl{āsyaṃ saṃyamya nāsāpuṭayugasuṣirābhyāṃ samākṛṣya vāyuṃ\\+}
\tl{mandaṃ mandaṃ yathāsau lagati galataṭād āhṛdantaḥ saśabdaḥ |\\+}
\tl{ruddhvā keśān nakhāgrāvadhi pavanam amuṃ recayed vāmanāḍyā\\+}
\tl{proktojjāyīti kumbhaḥ kaphagadadalano dīpti kṛjjāṭharāgne ||\\+}
\end{versinnote}
\end{testimonia}

\begin{philcomm}[hp02_051]
The use of \emph{lagati} without a locative or direct object (as found in the source text, the \emph{Gorakṣaśataka}) is supported by the paraphrase of the verse in the \emph{Haṭhatattvakaumudī} (10.7). Most witnesses (including \textalpha) have \emph{kaṇṭhāt tu hṛdayāvadhi}, which we have understood in the sense of a locative as it specifies the place within the body where the contact occurs. 
\end{philcomm}

%%%%%%%%%%
\subsection*{2.52}
\begin{translation}[hp02_052]
As before, he should hold the breath and then exhale through Iḍā. [Because] it cures disorders caused by phlegm in the throat and increases the body’s fire, [\dots]
\end{translation}

\begin{sources}[hp02_052]
\emph{Gorakṣaśataka} 37c–38b
\begin{versinnote}
\tl{pūrvavat kumbhayet prāṇaṃ recayed iḍayā tataḥ |\\+}
\tl{śīrṣotthitānalaharaṃ galaśleṣmaharaṃ paraṃ ||\\!}
\end{versinnote}
\end{sources}

\begin{testimonia}[hp02_052]
\emph{Haṭharatnāvalī} 2.14

\begin{versinnote}
\tl{pūrvavat kumbhayet prāṇaṃ recayed iḍayā tataḥ |\\+}
\tl{gale śleṣmaharaṃ proktaṃ dehānalavivardhanam ||\\!}
\end{versinnote}

\emph{Yogalakṣaṇāvalī} f.~32r (attrib. to the \emph{Haṭhapradīpikā})
\begin{versinnote}
\tl{pūrvavat kuṃbhayet prāṇān īḍayā recayet tataḥ |\\+}
\tl{śleṣmadoṣaharaṃ caitad dhāturogavināśanam ||\\!}
\end{versinnote}

\emph{Yogacintāmaṇi} f.~102r (attr.~to the \emph{Yogabīja})

\begin{versinnote}
\tl{pūrvavat kumbhayet prāṇaṃ recayed iḍayā tataḥ |\\+}
\tl{śleṣmadoṣaharaṃ kaṇṭhe dehānalavivardhanam ||\\!}
\end{versinnote}

\emph{Yuktabhavadeva} 7.102 (attr.~to the \emph{Haṭhapradīpikā})

\begin{versinnote} 
\tl{pūrvavat kumbhayet prāṇān recayediḍayā tataḥ |\\+}
\tl{śleṣmoṣaharaṃ dehānaladīptipravardhanam ||\\!}
\end{versinnote}

\end{testimonia}

\begin{philcomm}[hp02_052]
In the second hemistich of this verse, many of the readings in the oldest manuscripts, such as \emph{dehād analadīptivardhanam} (\getsiglum{V1}), \emph{dehānaladīptivivardhanam} (\getsiglum{J10}) and \emph{dehe [’]naladīptivivardhanam} (\getsiglum{P28}), are unlikely to be original because both the source and testimonia indicate that \emph{Ujjāyī} is supposed to remove phlegm from the throat. These versions may have arisen from attempts to remove \emph{kaṇṭhe} in the third \emph{pāda}, which was thought to be hanging. It appears that \emph{kaṇṭhe dehānalavardhanam} is the better reading and it is well attested among the manuscripts (including \textalpha).
\end{philcomm}

%%%%%%%%%%
\subsection*{2.53}
\begin{translation}[hp02_053]
[and] cures diseases in the bodily constituents inside the network of channels,
the retention called \emph{ujjāyī} should be done when [the yogi] is moving or remaining still.
\end{translation}

\begin{sources}[hp02_053]
\emph{Gorakṣaśataka} 38

\begin{versinnote}
\tl{nāḍījalodarādhātugatadoṣavināśanam |\\+}
\tl{gacchatas tiṣṭhataḥ kāryam ujjāyyākhyaṃ ca kumbhakam ||\\!}
\end{versinnote}
\end{sources}

\begin{testimonia}[hp02_053]
\emph{Haṭharatnāvalī} 2.15

\begin{versinnote}
\tl{nāḍījālodarādhātugatadoṣavināśanam |\\+}
\tl{nāḍījālodarādhātu° ] nāḍījalodaradhātu° \textup{J,P;} nāḍījalodaraṃ dhātu° \textup{N, n1, n4}\\+}
\tl{gacchatā tiṣṭhatā kāryam ujjāyyākhyaṃ hi kumbhakam || \\!}
\end{versinnote}

\emph{Yogalakṣaṇāvalī} f.~32r (attrib. to the \emph{Haṭhapradīpikā})
\begin{versinnote}
\tl{dehānaloddīptikaraṃ jalodaravighātakṛt |\\+}
\tl{gachatā tiṣṭhatā kāryaś cojjāyākhyas tu kumbhakaḥ ||\\!}
\end{versinnote}

\emph{Yogacintāmaṇi} f.~102r (attr.~to the \emph{Yogabīja})

\begin{versinnote}
\tl{nāḍījalodaradhātugatadoṣavināśanam |\\
gacchatas tiṣṭhataḥ kāryam ujjāyyākhyaṃ ca kumbhakam ||\\!}
\end{versinnote}

\emph{Yuktabhavadeva} 7.103 (attr.~to the \emph{Haṭhapradīpikā})

\begin{versinnote} 
\tl{nāḍījalodarādhātugatadoṣanivāraṇam |\\+}
\tl{gacchatā tiṣṭhatā kāryam ujjākhyaṃ kumbhakaṃ tv idam ||\\!}
\end{versinnote}

\end{testimonia}

\begin{philcomm}[hp02_053]
Nearly all the manuscripts have \emph{nāḍījalodara}°, which does not make sense in this context because \emph{jalodara} is the disease ascites and \emph{°gata°} requires a location. This problem is also present in the transmission of the source text for the verse, the \emph{Gorakṣaśataka}. A solution can be found in some of the manuscripts of the \emph{Haṭharatnāvalī}, which read \emph{nāḍījālodarā°} (‘in the network of channels and stomach’). The other problem is \emph{°darādhātu°}. In spite of Brahmānanda’s efforts to explain it as \emph{°dara}, \emph{ā}, and \emph{dhātu°}, the \emph{ā} before \emph{dhātu°} appears to have been inserted metri causa. We have adopted \emph{nāḍījālodare} (‘inside the network of channels’), which is close to \getsiglum{V3} and \getsiglum{J10} (\emph{nāḍījalodare}). It is likely that \emph{°jālodare} was changed to \emph{°jalodara} in the transmission because of confusion with the disease of a similar name.

For the idea of \emph{doṣa}s being in \emph{dhātu}s see \emph{Tantrāloka} 28.283cd, where worldly concepts are said to arise from it (\emph{dhātudoṣāc ca saṃsārasaṃskārās te ...}), but it is also the source of physical disorders (\emph{dhātudoṣakṛtaṃ mūrcchā} ĪPV on 2.15).

The name \emph{ujjāyī} may be a Prakrit form from \emph{uddhmāyī} from the verb \emph{ud-dhmā}, “to blow out”. We thank Diwakar Acharya for this suggestion.

\end{philcomm}

%%%%%%%%%%
\subsection*{2.54}
\begin{translation}[hp02_054]
Now sītkā---

{}[The yogi] should continuously make \emph{sīt} sound in the mouth and flare his nostrils. By practising in this way he becomes a second god of love.
\end{translation} 

\begin{sources}[hp02_054]
Cf.~\emph{Kaulajñānanirṇaya} 14.54

\begin{versinnote}
\tl{cittan dadyāt tu vaktreṇa nāse dadyād vijṛmbhikā[m] |\\+}
\tl{vācāsiddhir bhavaty eva kāmadevo ’paraḥ priyaḥ ||\\!}
\end{versinnote}

Cf. \emph{Jñānasāra} 2.13

\begin{versinnote}
\tl{hikkā dadyāt sadā vaktre prāyaś caiva vijṛmbhikām |\\+}
\tl{evam abhyasyamānas tu kāmadevo dvitīyakaḥ ||\\!}
\end{versinnote}

\emph{Prāṇatoṣiṇī} (part 6) p. 851 (citing the \emph{Jñānasāra})

\begin{versinnote}
\tl{hikkāṃ dadyāt sadā vaktre ghrāṇañ caiva vijṛmbhate |\\+}
\tl{evam abhyāsayogena kāmadevo dvitīyakaḥ ||\\!}
\end{versinnote}
\end{sources}

\begin{testimonia}[hp02_054]
\emph{Haṭharatnāvalī} 2.16

\begin{versinnote}
\tl{sītkāṃ kuryāt tathā vaktre ghrāṇenaiva visarjayet |\\+}
\tl{evam abhyāsayogena kāmadevo dvitīyakaḥ || 2.16 ||\\!}
\end{versinnote} 

\emph{Yogalakṣaṇāvalī} f.~32r (attrib. to the \emph{Haṭhapradīpikā})
\begin{versinnote}
\tl{sītkāṃ dadyāt sadā vaktre ghrāṇe caiva vijṛṃbhitām |\\+}
\tl{evam abhyasato na kṣuttṛṭ cālasyādi jāyate ||\\!}
\end{versinnote}

\emph{Yogacintāmaṇi} f.~101v (attr.~to the \emph{Haṭhayoga})

\begin{versinnote}
\tl{sītkāṃ kuryāt tathā vaktre ghrāṇenaiva visarjayet |\\+}
\tl{evam abhyāsayogena kāmadevo dvitīyakaḥ ||\\!}
\end{versinnote}

\emph{Yuktabhavadeva} 7.104 (attr.~to the \emph{Haṭhapradīpikā})

\begin{versinnote} 
\tl{sītkāṃ dadyāt sadā vaktre ghrāṇe caiva vijṛmbhikām |\\+}
\tl{evam abhyāsayogena kāmadevo dvitīyakaḥ ||\\!}
\end{versinnote}


\end{testimonia}

\begin{philcomm}[hp02_054]
%The source texts and V1 have \emph{dadyāt} in the first \emph{pāda}, which seems to be the lectio difficilior. %JB: agrees with the Stemma

%Many of the old manuscripts have \emph{kumbhaṃ} instead of \emph{sītkāṃ}. The latter reading is supported by some manuscripts of the \emph{Haṭhapradīpikā}, the testimonia and the name of the \emph{kumbhaka}, which is stated in 2.44 and the heading of this verse. %JB this is not an issue with the new collation

There is division between \emph{śītkāṃ} and \emph{sītkāṃ} in all the manuscript groups of the \emph{Haṭhapradīpikā} (note that the likely reading in the source texts was \emph{hikkāṃ}). The result of becoming a second god of love may be connected with the sound \emph{sīt}, which is said to be made during sex in the \emph{Kāmasūtra} (2.7.4–19). 

This verse’s source texts are from Kaula tantric milieus and this is reflected in the result of becoming one with the circle of yoginīs described in the next verse.

\end{philcomm}

%%%%%%%%%%
\subsection*{2.55}
\begin{translation}[hp02_055]
He joins the circle of yoginis and brings about creation and destruction. Neither hunger nor thirst [nor] sleep nor indolence arise [for him].
\end{translation}

\begin{sources}[hp02_055]
Cf.~\emph{Kaulajñānanirṇaya} 7.18ab

\begin{versinnote}
\tl{yoginīgaṇasāmānyā sṛṣṭisaṃhārakārakaḥ |\\!}
\end{versinnote}

\emph{Jñānasāra} 2.13cd–14ab

\begin{versinnote}
\tl{yoginīguṇasāmānyaḥ sṛṣṭisaṃhārakārakaḥ ||\\+}
\tl{na kṣudhā na ca tṛṇnidrā naiva murchā prajāyate |\\!}
\end{versinnote}
\end{sources}

\begin{testimonia}[hp02_055]
\emph{Haṭharatnāvalī} 2.17

\begin{versinnote}
\tl{yoginīcakrasaṃsevyaḥ sṛṣṭisaṃhārakārakaḥ |\\+}
\tl{na kṣudhā na tṛṣā nidrā naivālasyaṃ prajāyate ||\\!}
\end{versinnote}

\emph{Yogacintāmaṇi} f.~101v (attr.~to the \emph{Haṭhayoga})

\begin{versinnote}
\tl{yoginīcakrasaṃsevyaḥ sṛṣṭisaṃhārakārakaḥ |\\+}
\tl{na kṣudhā na tṛṣṇā nidrā tandrālasyaṃ na jāyate ||\\!}
\end{versinnote}

\emph{Yuktabhavadeva} 7.105 (attr.~to the \emph{Haṭhapradīpikā})

\begin{versinnote}
\tl{yoginīcakrasāmānyaḥ sṛṣṭisthityantakārakaḥ |\\+}
\tl{na kṣudhā na tṛṣā nidrā nālasya ca prajāyate ||\\!}
\end{versinnote}
\end{testimonia}

%\begin{philcomm}[hp02_055]
%There are various possible readings for the first \emph{pāda}, namely, \emph{yoginīcakrasāmānyaḥ} (‘one who is one among the circle of yoginīs'), \emph{yoginīcakram āsādya} (‘having reached the circle of yoginis’) and \emph{yoginīcakrasaṃsevyaḥ} (‘one worshipped by the circle of yoginis’). The first of these is closer to the source texts and is a cliche in Kaula literature.
%JB I think the above note is informative but alpha one and three support the reading we have adopted, so maybe this note is not needed. But Alpha two omits this verse (should we note this omission?) JM: I don't think we need to.
%\end{philcomm}

%%%%%%%%%%
\subsection*{2.56}
\begin{translation}[hp02_056]
His body is as he wishes, and he is free from all misfortune. By means of this technique, he truly becomes a lord of yogis in the world.
\end{translation}
% We should adopt bhuvimaṇḍale (alpha 3, eta 2) no? 

\begin{sources}[hp02_056]
\emph{Jñānasāra} 2.14cd–15ab

\begin{versinnote}
\tl{bhavet svacchandadehas tu sarvopadravavarjitaḥ ||\\+}
\tl{anena vidhinā devi yogīndro bhūmimaṇḍale |\\!}
\end{versinnote}

\emph{Śivasaṃhitā} 3.94
\begin{versinnote}
\tl{anenaiva vidhānena yogīndro 'vanimaṇḍale |\\+}
\tl{bhavet svacchandacārī ca sarvāpatparivarjitaḥ ||\\!}
\end{versinnote}
\end{sources}

\begin{testimonia}[hp02_056]
\emph{Haṭharatnāvalī} 2.18 

\begin{versinnote}
\tl{bhavet svacchandadehas tu sarvopadravavarjitaḥ |\\+}
\tl{anena vidhinā satyaṃ yogīndro bhāti bhūtale ||\\+}
\tl{°dehas tu ] °dehaḥ syāt P; °dehasyāt T,t1\\!}
\end{versinnote}

\emph{Yogacintāmaṇi} f.~101v (attr.~to the \emph{Haṭhayoga})

\begin{versinnote}
\tl{bhavet svachandadehas tu sarvopadravavarjitaḥ |\\+}
\tl{anena vidhinā yas tu yogīndro bhūmimaṇḍale ||\\!}
\end{versinnote}

\emph{Yuktabhavadeva} 7.106 (attr.~to the \emph{Haṭhapradīpikā})

\begin{versinnote}
\tl{bhavet svacchandadehaś ca sarvopadravavarjitaḥ |\\+}
\tl{anena vidhinā satyaṃ yogīndro bhuvimaṇḍale ||\\!}
\end{versinnote}

\end{testimonia}

\begin{philcomm}[hp02_056]
The aiśa compound \emph{bhuvimaṇḍale}, which is attested at \emph{Mañjuśrīmūlakalpa} 45.221, is likely the original reading here. The word  \emph{bhuvi} as the first member of a compound is attested elsewhere. The alternative \emph{bhūmi}° is well-attested and so the change may have happened early in the transmission. 
\end{philcomm}

%%%%%%%%%%
%\subsection*{2.56*1}
%\begin{translation}[hp02_056_1]
%One who always takes in the breath through the aperture at the roots of the tongue, undoubtedly becomes a receptacle of all \emph{siddhi}s.
%\end{translation}
%\begin{philcomm}[hp02_056_1]
%This verse appears to be a derivative of 2.57, and was not original to the text. Cf. Ten-chapter \emph{Haṭhapradīpikā} 4.46
%\begin{versinnote}
%\tl{jihvāmūlena randhreṇa yaḥ prāṇaṃ satataṃ pibet |\\+}
%\tl{sa bhavet sarvasiddhānāṃ bhājanaṃ nātra saṃśayaḥ ||\\!}
%\end{versinnote}
%\end{philcomm}

%%%%%%%%%%
\subsection*{2.57}
\begin{translation}[hp02_057]
And the very same has been taught [as follows]:

He who continuously takes in the breath through the tongue and the root of the palate has all his diseases cured in half a year.
\end{translation}

\begin{sources}[hp02_057]
\emph{Kauljñānanirṇaya} 6.19
\begin{versinnote}
\tl{rasanātālumūle tu kṛtvā vāyuṃ pibec chanaiḥ |\\+}
\tl{ṣaṇmāsād abhyased devi mahārogaiḥ pramucyate ||\\!}
\end{versinnote}

\emph{Vivekamārtaṇḍa} 120

\begin{versinnote}
\tl{rasanātālumūlena yaḥ prāṇam anilaṃ pibet |\\+}
\tl{abdārdhena bhavet tasya sarvarogaparikṣayaḥ ||\\!}
\end{versinnote}

\emph{Śivasaṃhitā} 3.80

\begin{versinnote}
\tl{rasanāṃ tālumūle yaḥ sthāpayitvā vipaścitaḥ | \\+}
\tl{pibet prāṇānilaṃ tasya rogāṇāṃ saṃkṣayo bhavet ||\\!}
\end{versinnote}
\end{sources}

\begin{testimonia}[hp02_057]
\emph{Yogacintāmaṇi} f.~101v (attr.~to the \emph{Haṭhayoga})

\begin{versinnote}
\tl{rasanātāluyogena yaḥ prāṇaṃ satataṃ pibet |\\+}
\tl{abdārdhena bhavet tasya sarvarogaparikṣayaḥ ||\\!}
\end{versinnote}

\emph{Yuktabhavadeva} 7.107 (attr.~to Gorakṣanātha)

\begin{versinnote}
\tl{etad evoktaṃ gorakṣanāthena-\\+}
\tl{rasanātālumūlena yaḥ prāṇaṃ satataṃ pibete |\\+}
\tl{abdārdhena bhavet tasya sarvarogaparikṣayaḥ ||\\!}
\end{versinnote}

Cf. \emph{Ānandakanda} 1.20.137

\begin{versinnote}
\tl{jihvayā tālumūlena prāṇaṃ yaḥ pibati priye |\\+}
\tl{tasya ṣaṇmāsataḥ sarve rogā naśyanti yoginaḥ || \\!}
\end{versinnote}
\end{testimonia}

\begin{philcomm}[hp02_057]
Verse 2.57 seems to be describing an alternative method of \emph{sītkā kumbhaka}. The introductory phrase \emph{evam eva uktaṃ ca} suggests that the teaching in this verse is consistent with what preceded it. However, the first version of \emph{sītkā} appears to be done at the front of the mouth (\emph{vaktra}), whereas the next version is done at the back of the mouth (see below).%JM Not sure what the note is saying. This practice could be consistent with 2.54 although no mention is made of a sound.

The compound \emph{rasanātālumūlena} is difficult to understand. In his \emph{Haṭhasaṅketacandrikā} (f.~79r–79v), Sundaradeva says that the external air strikes the root of the tongue and palate and the upper part of the uvula (\emph{atra muhū (mūhū codex) rasanātālumūlāhataṃ ghaṇṭikordhvabhāgāhataṃ bahiḥsthavāyuṃ vidhāya pibed ity arthaḥ}), which could make the sound \emph{sīt}. More helpful are the remarks of the commentator of the \emph{Yogataraṅgiṇī} (2.39). He says that a hole or cavity (\emph{vivara}) is made by the root of the palate with the help of the tongue. The yogi breathes through it (\emph{evaṃ rasanātālumūlena rasanā jihvā tatsahāyabhūtatālumūlena kṛtaṃ yad vivaraṃ, tena kṛtvā yaḥ yogī prāṇam anilaṃ prāṇavāyuṃ pibet pūrayet, tasya yogino ’bdārdhena ṣaṇmāsena sarvarogāṇāṃ nāśaḥ kṣayo bhavet} |).

The idea of breathing through a hole between the root of the palate and tongue might have been intended by the parallel reading of the \emph{Yogacintāmaṇi}: \emph{rasanā\-tālu\-yogena} (‘by joining the tongue and palate'). We have thus translated \emph{rasanā\-tālu\-mūlena} as ‘through the tongue and root of the palate’. It could also imply that the tongue is turned up and back to touch the root of the palate to make a hole that one breathes through (when the breath is taken in through the mouth). The \emph{Kumbhaka\-paddhati} (137ab) states this more clearly:

\begin{versinnote}
{}[The yogi] turns the tongue upwards and takes in the breath while making a \emph{sīt} sound.\\ 
\tl{rasanām unmukhīkṛtya sītkāraṃ kurvatā marut |\\!}%JM: could kumbhake tasmin (yasmin prob ok) refer to the practice as a whole, not the retention part of it? I.e. "In this kumbhaka" Otherwise it doesn't make sense.
\end{versinnote}

A similar practice is also described in \emph{Śivasaṃhitā} 3.80:

\begin{versinnote}
When the wise [yogi] places the tongue at the root of the palate and takes in the Prāṇa breath, his diseases are cured.
\end{versinnote}
\begin{versinnote}
\tl{rasanāṃ tālumūle yaḥ sthāpayitvā vipaścitaḥ |\\+}
\tl{pibet prāṇānilaṃ tasya rogāṇāṃ saṃkṣayo bhavet ||\\!}
\end{versinnote}

\end{philcomm}

%%%%%%%%%%
\subsection*{2.58}
\begin{translation}[hp02_058]
Now śītalī:\\
The wise man should draw in air through the tongue and after retaining the breath as before gradually exhale through the nostrils .
\end{translation}

\begin{sources}[hp02_058]
\emph{Gorakṣaśataka} 39cd–40ab

\begin{versinnote}
\tl{jihvayā vāyum ākṛṣya pūrvavat kuṃbhakād anu |\\+}
\tl{śanais tu ghrāṇarandhrābhyāṃ recayed anilaṃ sudhīḥ ||\\!}
\end{versinnote}

Cf. \emph{Vivekamārtaṇḍa} 139
\begin{versinnote}
\tl{kākacañcuvad āsyena śītalaṃ salilaṃ pibet |\\+}
\tl{prāṇaṃ prāṇavidhānajño yogī bhavati nirjaraḥ ||\\!}
\end{versinnote}
\end{sources}

\begin{testimonia}[hp02_058]
\emph{Haṭharatnāvalī} 2.19
\begin{versinnote}
\tl{jihvayā vāyum ākṛṣya pūrvavat kumbhakād anu |\\+}
\tl{śanair aśītiparyantaṃ recayed anilaṃ sudhīḥ ||\\!}
\end{versinnote}

\emph{Yogacintāmaṇi} f.~102v (attr.~to the \emph{Yogabīja})
\begin{versinnote}
\tl{jihvayā vāyum ākṛṣya pūrvavat kumbhakād anu |\\+}
\tl{śanais tu ghrāṇārandhrābhyāṃ recayed anilaṃ suddhīḥ ||\\!}
\end{versinnote}

\emph{Yuktabhavadeva} 7.108 (attr.~to Gorakṣanātha)
\begin{versinnote}
\tl{jihvayā vāyum ākṛṣya pūrvavat kumbhakād anu |\\+}
\tl{śanais tu ghrāṇarandhrābhyāṃ recayed anilaṃ sudhīḥ ||\\!}
\end{versinnote}

Cf. \emph{Ānandakanda} 1.20.135–136ab
\begin{versinnote}
\tl{kākacañcuvad āsyaṃ ca kṛtvā vāyuṃ sasūtkṛtam |\\+}
\tl{ādāya nāsārandhreṇa punas taṃ śvasanaṃ tyajet ||\\+}
\tl{śītalīkaraṇākhyo 'yaṃ yogas tu jvarapittahṛt |\\!}
\end{versinnote}
\end{testimonia}

%\begin{philcomm}[hp02_058]
%\end{philcomm}

%%%%%%%%%%
\subsection*{2.59}
\begin{translation}[hp02_059]
This retention called śītalī cures diseases such as swelling and enlargement of the spleen, fever, [excess] bile, hunger and thirst.
\end{translation}

\begin{sources}[hp02_059]
\emph{Gorakṣaśataka} 41

\begin{versinnote}
\tl{gulmaplīhādikā doṣāḥ kṣayaṃ yānti pittaṃ jvaraṃ |\\+}
\tl{viṣāṇi śītalī nāma kuṃbhako 'yaṃ nihanti ca ||\\!}
\end{versinnote}
\end{sources}

\begin{testimonia}[hp02_059]
\emph{Haṭharatnāvalī} 2.20

\begin{versinnote}
\tl{gulmaplīhodaraṃ doṣaṃ jvarapittakṣudhātṛṣāḥ |\\+}
\tl{viṣāṇi śītalī nāma kumbhako 'yaṃ nihanti ca ||\\+}
\tl{\var{°tṛṣāḥ ] °tṛṣā J,n1; °tṛṣaḥ T,t1}\\!}%
\end{versinnote}

\emph{Yogacintāmaṇi} ms.~L, f.~70r
% this verse is incomplete in N (gulmaplīhodaraṃ vāpi ||)

\begin{versinnote}
\tl{gulmaplīhodaraṃ cāpi vātapittaṃ kṣudhāṃ tṛṣām |\\+}
\tl{viṣāṇi śītalī nāma kumbhako vinihanti ca ||\\!}
\end{versinnote}

\emph{Yuktabhavadeva} 7.109 (attr.~to Gorakṣanātha)
\begin{versinnote}
\tl{gulmaplīhādikān doṣān jvaraṃ pittaṃ kṣudhāṃ tṛṣām |\\+}
\tl{anyāṃś ca śītalī nāma kumbhako 'yaṃ nihanti hi ||\\!}
\end{versinnote}

\end{testimonia}

\begin{philcomm}[hp02_059]
An antecedent to a cooling practice involving the tongue can be found in the \emph{Kaulajñānanirṇaya} (6.23–24), which mentions a point between the two front teeth that is cool to touch with the tongue:

\begin{versinnote}
There is a point located between the two ‘royal teeth’(\emph{rājadanta}). One should know this to be [the place of] nectar that destroys wrinkles and grey hair. Putting the tongue in the place cool to the touch, the wise man becomes free of wrinkles and grey hair and devoid of all diseases.
\tl{dvaurājadantamadhyasthaṃ bindurūpaṃ vyavasthitam |\\+}
\tl{amṛtaṃ taṃ vijānīyād valīpalitanāśanam ||\\+}
\tl{śītalasparśasaṃsthāne rasanāṃ kṛtvā tu buddhimān |\\+}
\tl{valīpalitanirmuktaḥ sarvavyādhivivarjitaḥ ||\\!}   
\end{versinnote}
We wish to thank Shaman Hatley for the reference and translation.
\end{philcomm}

%%%%%%%%%%
\subsection*{2.60}
\begin{translation}[hp02_060]
Now bhastrikā:

If [the yogi] places the soles of both feet on the thighs, the lotus pose, which destroys all bad deeds, duly arises.
\end{translation}

\begin{sources}[hp02_060]
\emph{Gorakṣaśataka} 14

\begin{versinnote}
\tl{ūrvor upari ced dhatte ubhe pādatale tathā |\\+}
\tl{padmāsanaṃ bhavet samyak sarvapāpapraṇāśanam ||\\!}
\end{versinnote}
\end{sources}

\begin{testimonia}[hp02_060]
%\emph{Haṭharatnāvalī} 2.21
%\begin{versinnote}
%\tl{atha bhastrikā---\\+}
%\tl{recakaḥ pūrakaś caiva kumbhakaḥ praṇavātmakaḥ |\\+}
%\tl{recako 'jasraniḥśvāsaḥ pūrakas tannirodhakaḥ |\\+}
%\tl{samānasaṃsthito yo 'sau kumbhakaḥ parikīrtitaḥ ||\\!}
%\end{versinnote}

\emph{Yogacintāmaṇi} f.~102r (attr.~to the \emph{Yogabīja})

\begin{versinnote}
\tl{bhastrikā\\+}
\tl{ūrvor upari saṃsthāpya ubhe pādatale tathā |\\+}
\tl{padmāsanaṃ bhavet samyak sarvapāpapraṇāśanam ||\\!}
\end{versinnote}

\emph{Yuktabhavadeva} 7.110 (attr.~to Gorakṣanātha)

\begin{versinnote}
\tl{atha bhastrikā\\+}
\tl{ūrvor upari cādhatte ubhe pādatale tathā |\\+}
\tl{padmāsanaṃ bhavet samyak sarvapāpapraṇaśanam ||\\!}
\end{versinnote}
\end{testimonia}

%\begin{philcomm}[hp02_060]
%The source text, the \emph{Gorakṣaśataka}, has \emph{ced} in the first pāda, and this has been dropped in nearly all of the available manuscripts of the \emph{Haṭhapradīpikā}, as well as the testimonia. It seems likely that \emph{ced dhatte} was the reading adopted by Svatmārāma because the \emph{cet} makes sense of the two finite verbs in the description. The first finite verb \emph{dhatte} is also supported by some of the old manuscripts, such as V1 and J10. At some stage, the verse was changed to read \emph{saṃsthāpya} to remove the awkward syntax posed by \emph{saṃdhatte} and \emph{vai dhatte}.
%[JB: alpha one and two supports ced, so there's no problem here]
%  JB add alpha two (ced dhatte) to the collation 
%\end{philcomm}

%%%%%%%%%%
\subsection*{2.61}
\begin{translation}[hp02_061]
Having correctly adopted the lotus pose, with his neck and torso straight the wise man should close the mouth and forcefully exhale the breath through the nose
\end{translation}

\begin{sources}[hp02_061]
\emph{Gorakṣaśataka} 41cd–42ab

\begin{versinnote}
\tl{tataḥ padmāsanaṃ baddhvā samagrīvodaraḥ sudhīḥ |\\+}
\tl{mukhaṃ saṃyamya yatnena prāṇaṃ ghrāṇena recayet ||\\!}
\end{versinnote}
\end{sources}

\begin{testimonia}[hp02_061]
\emph{Yogacintāmaṇi} f.~102r (attr.~to the \emph{Yogabīja})

\begin{versinnote}
\tl{samyak padmāsanaṃ badhvā samagrīvodaraḥ sudhīḥ |\\+}
\tl{mukhaṃ saṃyamya yatnena prāṇaṃ ghrāṇena recayet ||\\!}
\end{versinnote}

\emph{Yuktabhavadeva} 7.111 (attr.~to Gorakṣanātha)

\begin{versinnote}
\tl{samyak padmāsanaṃ baddhvā samagrīvodaraḥ śanaiḥ |\\+}
\tl{mukhaṃ saṃyamya yatnena prāṇaṃ ghrāṇena recayet ||\\!}
\end{versinnote}
\end{testimonia}

%\begin{philcomm}[hp02_061]
%\end{philcomm}

%%%%%%%%%%
\subsection*{2.62}
\begin{translation}[hp02_062]
in such a way that it comes into contact with the chest, throat and skull, making a sound. He should then quickly inhale a small amount of breath as far as the heart lotus.
\end{translation}

\begin{sources}[hp02_062]
\emph{Gorakṣaśataka} 42cd–43ab

\begin{versinnote}
\tl{yathā lagati kaṇṭhāt tu kapāle sasvanaṃ tataḥ |\\+}
\tl{vegena pūrayet kiṃ cit hṛtpadmāvadhi mārutam ||\\!}
\end{versinnote}
\end{sources}

\begin{testimonia}[hp02_062]
\emph{Yogacintāmaṇi} f.~102r (attr.~to the \emph{Yogabīja})

\begin{versinnote}
\tl{yathā lagati hṛtkaṇṭhe kapālāvadhi pūrayet |\\+}
\tl{vegena pūrayet samyag hṛtpadmāvadhi mārutam ||\\!}
\end{versinnote}

\emph{Yuktabhavadeva} 7.112 (attr.~to Gorakṣanātha)

\begin{versinnote}
\tl{yathā lagati hṛtkaṇṭhakapāleṣu ca sasvanam |\\+}
\tl{vegena pūrayet kiñ cit hṛtpadmāvadhi mārutam ||\\!}
\end{versinnote}
\end{testimonia}

\begin{philcomm}[hp02_062]
%First hemistich is tricky. None of the old mss preserve \emph{sasvanaṃ}, which is in the GŚ and makes good sense. 
% MD: Now we have at least two good witnesses P11(β1),P15(ε1) with the reading sasvanaṃ.

Most witnesses have \emph{hṛtkaṇṭhe}, which we have understand as a dual accusative. One would expect \emph{kapāla} also to be in the accusative, but we have understood it as a locative sg. with \emph{sasvanam}.
\end{philcomm}

%%%%%%%%%%
\subsection*{2.63}
\begin{translation}[hp02_063]
He should then exhale and inhale in that way over and over again. In the very same way as blacksmiths’ bellows are operated forcefully, [...]
\end{translation}

\begin{sources}[hp02_063]
\emph{Gorakṣaśataka} 43cd–44ab

\begin{versinnote}
\tl{punar virecayet tadvat pūrayec ca punaḥ punaḥ |\\+}
\tl{yathaiva lohakārāṇāṃ bhastrā vegena cālyate ||\\!}
\end{versinnote}
\end{sources}

\begin{testimonia}[hp02_063]
\emph{Haṭharatnāvalī} 2.22ab

\begin{versinnote}
\tl{yathaiva lohakārāṇāṃ bhastrī vegena cālyate |\\+}
\tl{\var{bhastrī ] bhastrā n4}\\!}
\end{versinnote} 

\emph{Yogacintāmaṇi} f.~102r (attr.~to the \emph{Yogabīja})

\begin{versinnote}
\tl{punar virecayet tadvat pūrayitvā punaḥ punaḥ |\\+}
\tl{yathaiva lohakārāṇāṃ bhastrā vegena cālyate ||\\!}
\end{versinnote}


\emph{Yuktabhavadeva} 7.113 (attr.~to Gorakṣanātha)

\begin{versinnote}
\tl{punar virecayet tadvat pūrayec ca punaḥ punaḥ |\\+}
\tl{yathaiva lohakārāṇāṃ bhastrā vegena cālyate || 113 ||\\!}
\end{versinnote}
\end{testimonia}

%\begin{philcomm}[hp02_063]
%The \getsiglum{V1} reading of \emph{lohakāreṇa} fits well with the passive verb, but it is the only witness to have this, and therefore appears to be an attempt to improve what was probably the original reading \emph{lohakārāṇāṃ} (as attested by the manuscripts, the source and testimonia).
%\end{philcomm}

%%%%%%%%%%
\subsection*{2.64}
\begin{translation}[hp02_064]
[...] the wise man should move the breath in his body. When fatigue arises in the body he should inhale through the sun [channel]
\end{translation}

\begin{sources}[hp02_064]
\emph{Gorakṣaśataka} 44cd–45ab

\begin{versinnote}
\tl{tathaiva svaśarīrasthaṃ cālayet pavanaṃ sudhīḥ |\\+}
\tl{yadā śramo bhaved dehe tadā sūryeṇa pūrayet |\\!}
\end{versinnote}
\end{sources}

\begin{testimonia}[hp02_064]
\emph{Haṭharatnāvalī} 2.22cd–23ab

\begin{versinnote}
\tl{tathaiva svaśarīrasthaṃ cālayet pavanaṃ sudhīḥ ||\\+}
\tl{yathā śramo bhaved dehe tathā sūryeṇa pūrayet |\\!}
\end{versinnote}

\emph{Yogacintāmaṇi} f.~102r (attr.~to the \emph{Yogabīja})

\begin{versinnote}
\tl{tathaiva svaśarīrasthaś cālyate pavano dhiyā |\\+}
\tl{yathā śramo bhaved dehe tathā vegena pūrayet ||\\!}
\end{versinnote}

\emph{Yuktabhavadeva} 7.114 (attr.~to Gorakṣanātha)

\begin{versinnote}
\tl{tathaiva svaśarīrasthaṃ cālayet pavanaṃ dhiyā |\\+}
\tl{yadā śramo bhaved dehe tadā sūryeṇa recayet ||\\!}
\end{versinnote}
\end{testimonia}

\begin{philcomm}[hp02_064]
Most of the witnesses (including \textalpha) have \emph{dhiyā} at the end of the second verse quarter, but the manuscripts of the source text and the \emph{Haṭharatnāvalī} have \emph{sudhīḥ}. Since the subject of the simile is \emph{bhastrā}, one would expect the subject of \emph{cālayet}, which must be different, to be stated (as is the case with \emph{sudhīḥ}). %Also, one would expect the instrumental of \emph{dhī} to be qualified by some adjective, such as in the case of \emph{sattvāsthayā dhiyā} (\emph{Gorakṣaśataka} 74b) and \emph{sāttvikayā dhiyā} (\emph{Haṭhapradīpikā} 2.6b).
%  JB could the alphaTwo reading dhiyaḥ (as some sort of aiśa masc. sing.) explain dhiyā?
%JM: dhiyā on its own is fine. I find lots of refs. I've removed that part of the note. 
\end{philcomm}

%%%%%%%%%%
\subsection*{2.65}
\begin{translation}[hp02_065]
in such a way that the abdomen is filled by the breath, and hold the nose quickly [and] firmly without using the middle and index fingers.
\end{translation}

\begin{sources}[hp02_065]
\emph{Gorakṣaśataka} 45cd–46ab

\begin{versinnote}
\tl{yathodaraṃ bhavet pūrṇaṃ pavanena tathā laghu |\\+}
\tl{dhārayan nāsikā madhyātarjanībhyāṃ vinā dṛḍhaṃ |\\!}
%\tl{madhyā° ] em.; °madhyaṃ GU, °madhye T\\!}
\end{versinnote}
\end{sources}

\begin{testimonia}[hp02_065]
\emph{Haṭharatnāvalī} 2.23cd–24ab

\begin{versinnote}
\tl{yathodaraṃ bhavet pūrṇaṃ pavanena tathā laghu ||\\+}
\tl{dhārayen nāsikāṃ madhyātarjanībhyāṃ vinā dṛḍham |\\+}
\tl{\var{23c madhyā ] madhye N,n1,n3,n4,J,T,t1}\\!}%
\end{versinnote}

\emph{Yogacintāmaṇi} f.~102r (attr.~to the \emph{Yogabīja})

\begin{versinnote}
\tl{yathodaraṃ bhavet pūrṇaṃ pavanena tathā laghu |\\+}
\tl{dhārayen nāsikāṃ madhyatarjanībhyāṃ vinā dṛḍham ||\\!}
\end{versinnote}

\emph{Yuktabhavadeva} 7.115 (attr.~to Gorakṣanātha)
\begin{versinnote}
\tl{yathodaraṃ bhavet pūrṇaṃ pavanena tathā laghu |\\+}
\tl{dhārayen nāsikāmadhye tarjanībhyāṃ vinā dṛḍham ||\\!}
\end{versinnote}
\end{testimonia}

\begin{philcomm}[hp02_065]
Only two of the collated witnesses (\getsiglum{J7}, \getsiglum{V15}) have \emph{madhyātarjanībhyāṃ} (`with the middle and index fingers') and this reading is not well attested by the manuscripts of the source text and testimonia. To hold the nose without the middle and index fingers is consistent with the way alternate nostril breathing is done in modern yoga (e.g., Iyengar 1991: 443–444). However, the reading of many manuscripts suggests  that the nose was held by all the fingers of both hands, except the index fingers (\emph{nāsikāmadhye tarjanībhyāṃ vinā}), which seems highly impracticable, or that the nose was held by only the index fingers of both hands (\emph{nāsikāmadhye tarjanībhyāṃ tathā}). It is likely that scribes changed \emph{madhyātarjanībhyāṃ} to \emph{madhye tarjanībhyāṃ} or \emph{madhyaṃ tarjanībhyāṃ} because of the \emph{pāda} break. 
\end{philcomm}
%Iyengar BKS. Light on Yoga: Yoga Dipika. Great Britain: Mandala, 1991.

%%%%%%%%%%
\subsection*{2.66}
\begin{translation}[hp02_066]
[The yogi] should hold the breath as before then exhale through Iḍā. [Because] it removes [imbalances] in wind, bile and phlegm, increases the body’s fire, [\dots]

\end{translation}

\begin{sources}[hp02_066]
\emph{Gorakṣaśataka} 46cd–47ab

\begin{versinnote}
\tl{kumbhakaṃ pūrvavat kṛtvā recayed iḍayānilam ||\\+}
\tl{kaṇṭhotthitānalaharaṃ śarīrāgnivivardhanam |\\!}
\end{versinnote}
\end{sources}

\begin{testimonia}[hp02_066]
\emph{Haṭharatnāvalī} 2.24cd–25ab

\begin{versinnote}
\tl{kumbhakaṃ pūrvavat kṛtvā recayed iḍayānilam ||\\+}
\tl{vātapittaśleṣmaharaṃ śarīrāgnivivardhanam |\\!}
\end{versinnote}

\emph{Yogacintāmaṇi} f.~102r (attr.~to the \emph{Yogabīja})

\begin{versinnote}
\tl{kumbhakaṃ pūrvavat kṛtvā recayed iḍayā tataḥ |\\+}
\tl{vātapittaśleṣmaharaṃ śarīrāgnivivardhanam ||\\!}
\end{versinnote}

\emph{Yuktabhavadeva} 7.116 (attr.~to Gorakṣanātha)
\begin{versinnote}
\tl{kumbhakaṃ pūrvavat kṛtvā recayed iḍayā śanaiḥ |\\+}
\tl{vātapittaśleṣmaharaṃ śarīrāgnivivardhanam ||\\!}
\end{versinnote}

\end{testimonia}

%\begin{philcomm}[hp02_066]
%\end{philcomm}

\begin{metre}[hp02_066]
Anuṣṭubh (c: bha-vipulā)
\end{metre}

%%%%%%%%%%
\subsection*{2.67}
\begin{translation}[hp02_067]
it is an auspicious thunderbolt that awakens \emph{kuṇḍalinī}, destroys bad deeds, bestows happiness, and destroys the blockage of phlegm, etc., situated at the mouth of the central channel,[\dots]
\end{translation}

\begin{sources}[hp02_067]
\emph{Gorakṣaśataka} 47cd–48ab

\begin{versinnote}
\tl{kuṇḍalībodhakaṃ vajraṃ pāpaghnaṃ śubhadaṃ sukham |\\+}
\tl{brahmanāḍīmukhāntaḥsthakaphādyargalanāśanam ||\\!}
\end{versinnote}
\end{sources}

\begin{testimonia}[hp02_067]
\emph{Haṭharatnāvalī} 2.25cd

\begin{versinnote}
\tl{brahmanāḍīmukhe saṃsthakaphādyargalanāśanam |\\!}
\end{versinnote}

\emph{Yogacintāmaṇi} f.~102r (attr.~to the \emph{Yogabīja})

\begin{versinnote}
\tl{kuṇḍalībodhanaṃ kuryāt pāpaghnaṃ sukhadaṃ śubham |\\+}
\tl{brahmanāḍīmukhe saṃsthaṃ kapāṭārgalanāśanam ||\\!}
\end{versinnote}


\emph{Yuktabhavadeva} 7.117 (attr.~to Gorakṣanātha)

\begin{versinnote}
\tl{kuṇḍalībodhanaṃ sarvadoṣaghnaṃ sukhadaṃ śubham |\\+}
\tl{brahmanāḍīmukhāntasthakaphādyargalanāśanam ||\\!}
\end{versinnote}
\end{testimonia}

\begin{philcomm}[hp02_067]
The reading \emph{vajraṃ} in the first verse quarter is an emendation based on the manuscripts of the \emph{Gorakṣaśataka}, the source text for this verse. Some manuscripts of the \emph{Haṭhapradīpikā} have readings close to \emph{vajraṃ}, such as \emph{vipra} (\getsiglum{V1}, \getsiglum{P15}) and \emph{vakra} (G11), which suggest that \emph{vajraṃ} was changed at an early stage of the transmission. The \textalpha\ group have \emph{cakraṃ}, which could be understood as a weapon (i.e., the discus often associated with Viṣṇu), but such a meaning would be rather unusual in yoga texts, where the term \emph{cakra} is used so frequently in contexts of the yogic body.
%The J10 group has rewritten this verse in the masc sg. with \emph{kumbhaḥ}. And this is possible without the \emph{viśeṣeṇaiva [...] kumbhakaṃ tv idaṃ} line. However, it seems that the \emph{viśeṣeṇaiva} line has dropped out because it is in the Gorakṣaśataka. Therefore, the neuter was probably original.
\end{philcomm}

%%%%%%%%%%
\subsection*{2.68}
\begin{translation}[hp02_068]
[and] pierces the three knots that have arisen from the three \emph{guṇa}s, it is particularly important to perform this retention called ‘the bellows’.
\end{translation}
% Adopt \item[guṇatrayasamudbhūta] N21, Bo1. Tr reflects this but ed has samyaggātra

\begin{sources}[hp02_068]
\emph{Gorakṣaśataka} 48cd–49ab

\begin{versinnote}
\tl{guṇatrayasamudbhūtagranthitrayavibhedakam |\\+}
\tl{viśeṣeṇaiva kartavyaṃ bhastrākhyaṃ kuṃbhakaṃ tv idam ||\\!}
\end{versinnote}
\end{sources}

\begin{testimonia}[hp02_068]
\emph{Haṭharatnāvalī} 2.25cd

\begin{versinnote}
\tl{viśeṣenaiva kartavyaṃ bhastrākhyaṃ kumbhakaṃ tv idam ||\\!}
\end{versinnote}

\emph{Yogacintāmaṇi} f.~102r–102v (attr.~to the \emph{Yogabīja})

\begin{versinnote}
\tl{samyaggātrasamudbhūtagranthitrayavibhedanam |\\+}
\tl{viśeṣeṇaiva kartavyaṃ bhastrākhyaṃ kumbhakaṃ tv idam ||\\!}
\end{versinnote}

\emph{Yuktabhavadeva} 7.118 (attr.~to Gorakṣanātha)

\begin{versinnote}
\tl{samyaggātrasamudbhūtagranthitrayavibhedanam |\\+}
\tl{viśeṣeṇaiva karttavyaṃ bhastrākhyaṃ kumbhakaṃ tv idam ||\\!}
\end{versinnote}

\emph{Haṭhasaṅketacandrikā} f.~80v (attr.~to Gorakṣanātha)
\begin{versinnote}
\tl{guṇatrayasamudbhūtagranthitrayavibhedakam |\\+}
\tl{viśeṣeṇaiva karttavyaṃ bhastrākhyaṃ kumbhakaṃ svayam ||\\!}
\end{versinnote}

\end{testimonia}

\begin{philcomm}[hp02_068]
In the first verse quarter, nearly all of the manuscripts, including the \textalpha\ group, have \emph{°gātrasamudbhūta°} (`arisen in the limbs/body'), which is rather meaningless here since the three knots are in the central channel. It is probable that \emph{samyaggātra°} is a corruption of \emph{guṇatraya°}, which is attested by the manuscripts of the source text, the \emph{Gorakṣaśataka}, two of the available manuscripts of the \emph{Haṭhapradīpikā} (\getsiglum{N21}, Bo1) and the \emph{Haṭhasaṅketacandrikā}. In the \emph{Gorakṣaśataka}, it is stated clearly that each of the knots arise from one of the three \emph{guṇa}s: \emph{brahmagranthi} from \emph{rajas} (78cd), \emph{viṣṇugranthi} from \emph{sattva} (79cd) and \emph{rudragranthi} from \emph{tamas} (80cd). When the verses on \emph{bhastrā} were extracted from the \emph{Gorakṣaśataka} without the context of the knots and \emph{guṇa}s, the meaning of \emph{guṇa\-traya\-sam\-udbhūta\-granthi\-traya°} (`the three knots that have arisen from the three \emph{guṇa}s') was lost early in the transmission of the \emph{Haṭhapradīpikā}.  
\end{philcomm}

%%%%%%%%%%
\subsection*{2.69}
\begin{translation}[hp02_069]
Now bhrāmarī:

Forcibly loud inhalation with the sound of a male bee; very slow exhalation with the sound of a female bee: as a result of practising thus, there arises in the minds of the best yogis an extraordinary blissful playfulness.
\end{translation}

\begin{testimonia}[hp02_069]
\emph{Haṭharatnāvalī} 2.26

\begin{versinnote}
\tl{atha bhrāmarī---\\+}
\tl{vegodghoṣaṃ pūrakaṃ bhṛṅganādaṃ \\+}
\tl{bhṛṅgīnādaṃ recakaṃ mandamandam |\\+}
\tl{yogīndrāṇāṃ nityam abhyāsayogāc \\+}
\tl{citte jātā kā cid ānandalīlā || \\+}
\tl{līlā ]°mūrcchā N,n1,n2,n3,n4,J,T,t1\\!}
\end{versinnote}


\emph{Yogacintāmaṇi} f.~101v (attr.~to the \emph{Haṭhayoga})

\begin{versinnote}
\tl{bhramarī—\\+}
\tl{vegodghoṣaṃ pūrakaṃ bhṛṅganādaṃ \\+}
\tl{bhṛṅgīnādaṃ recakaṃ mandamandam |\\+}
\tl{yogīndrāṇāṃ nityam abhyāsayogāc \\+}
\tl{citte jātā kācid ānandalīlā ||\\!}
\end{versinnote}

\emph{Yuktabhavadeva} 7.119 (attr.~to Gorakṣanātha)

\begin{versinnote}
\tl{atha bhrāmarī---\\+}
\tl{vegodghoṣaṃ pūrakaṃ bhṛṃganādaṃ \\+}
\tl{recakaṃ mandamandam |\\+}
\tl{yogīndrāṇāmevamabhyāsayogāc\\+}
\tl{citte jātā kācid ānandalīlā || \\!}
\end{versinnote}

\emph{Yogaprakāśikā} 4.59 (ten-chapter \emph{Haṭhapradīpikā})

\begin{versinnote}
bhrāmarīkumbhakaṃ lakṣayaty atheti | vegena sañjāta udghoṣo yasmin pūrake taṃ bhṛṅganādatulyaṃ bhṛṅgīnādatulyaṃ recakaṃ kuryād ānandalīleti |
\end{versinnote}

\emph{Haṭhasaṅketacandrikā} f.~80v 

\begin{versinnote}
\tl{vegākṛṣṭiṃ pūrakaṃ bhṛṅganādaṃ \\+}
\tl{bhaṅgīnādaṃ recakaṃ mandaṃ mandaṃ |\\+}
\tl{yogīdrāṇām evam abhyāsayogac \\+}
\tl{cite jātā kācid ānandamūrchā ||\\+}
\tl{vegodghoṣam iti vā pāṭhaḥ |\\!}
\end{versinnote}

Cf. \emph{Kumbhakapaddhati} 169

\begin{versinnote}
\tl{aliśabdayutaṃ vegāt pūrayet kumbhayet tataḥ |\\+}
\tl{sāliśabdāc chanai rekāt bhrāmarīkumbhako muneḥ ||\\+}
\tl{ānandalīlāṃ kurute bhrāmarīkumbhako muneḥ || 169 ||\\!}
\end{versinnote}

Cf. \emph{Gheraṇḍasaṃhitā} 7.10--11

\begin{versinnote}
\tl{anilaṃ mandavegena bhrāmarīkumbhakaṃ caret |\\+}
\tl{mandaṃ mandaṃ recayed vāyuṃ bhṛṅganādaṃ tato bhavet || 7.10 ||\\+}
\tl{antaḥsthaṃ bhramarīnādaṃ śrutvā tatra mano nayet |\\+}
\tl{samādhir jāyate tatra ānandaḥ so 'ham ity ataḥ || 7.11 ||\\!}
\end{versinnote}
\end{testimonia}

\begin{philcomm}[hp02_069]
In the first two \emph{pāda}s we are understanding the usually masculine \emph{pūraka} and \emph{recaka} to be neuter nominatives. When explaining this verse, Brahmānanda (\emph{Jyotsnā} 2.68) supplies the gerund \emph{kṛtvā} with \emph{pūrakaṃ} and \emph{kuryāt} with \emph{recakaṃ}, but it is unlikely that the author intended this. 

The compound \emph{vegodghoṣaṃ} (close to \emph{vegoghoṣaṃ} in \alphaOne) is rather unusual. We have understood it according to the commentator Bālakṛṣṇa’s gloss: `an inhalation in which sound is produced by force' (\emph{vegena sañjāta udghoṣo yasmin pūrake}). 

Many manuscripts of the \emph{Haṭhapradīpikā}, including the \textalpha\ group, have \emph{ānandamūrcchā} at the end of the verse, instead of \emph{ānandalīlā}, which we have adopted and which is supported by \betaTwo\ and the \textgamma\ group, as well as the most important testimonia (i.e.~the \emph{Haṭharatnāvalī}, \emph{Yogacintāmaṇi} and \emph{Yuktabhavadeva}). The reading \emph{ānanda\-mūrcchā} is probably the result of dittography as the heading \emph{atha mūrcchā} follows this verse. It has persisted in the transmission because  \emph{ānandamūrcchā} makes some sense by itself (`swooning through bliss'), although when it is read with the rest of the verse quarter, the meaning is somewhat odd (`swooning through bliss arises in the mind').
\end{philcomm}

\begin{metre}[hp02_069]
Śālinī 
\end{metre}

%%%%%%%%%%
\subsection*{2.70}
\begin{translation}[hp02_070]
Now mūrcchā:

At the end of inhalation [the yogi] should tightly apply the Jālandhara [lock] and exhale slowly. This, which is called \emph{mūrcchā}, bestows the bliss of  fainting.
\end{translation}
% adopt feminine in 2nd hemistich

\begin{sources}[hp02_070]
\emph{Haṭharatnāvalī} 2.27

\begin{versinnote}
\tl{atha mūrcchā---\\+}
\tl{pūrakānte gāḍhataraṃ baddhva jālandharaṃ śanaiḥ |\\+}
\tl{recayen mūrcchanākhyo 'yaṃ manomūrcchāsukhapradā ||\\!}
\end{versinnote}

\emph{Yogacintāmaṇi} f.~101v (attr.~to the \emph{Haṭhayoga})

\begin{versinnote}
\tl{mūrchā—\\+}
\tl{pūrakānte gāḍhataraṃ bandho jālandharaḥ śanaiḥ |\\+}
\tl{recayen mūrchanākhyo 'yaṃ manomūrchāsukhapradā ||\\!}
\end{versinnote}

\emph{Yuktabhavadeva} 7.120 (attr.~to Gorakṣanātha)

\begin{versinnote}
\tl{atha mūrcchā---\\+}
\tl{pūrakānte gāḍhataraṃ baddhvā jālandharaṃ śanaiḥ ||\\+}
\tl{recayen mūrcchanākhyeyaṃ manomūrcchāsukhapradā || 120 ||\\!}
\end{versinnote}

\emph{Kumbhakapaddhati} 170

\begin{versinnote}
\tl{āpūrya kumbhitaṃ prāṇaṃ badhvā jālandharaṃ śanaiḥ |\\+}
\tl{recayen mūrcchanākumbho manomūrcchāsukhapradā ||\\!}
\end{versinnote}
\end{sources}

\begin{philcomm}[hp02_070]
The Jālandhara lock is explained at 3.67–70. 
\end{philcomm}

\begin{metre}[hp02_070]
Anuṣṭubh (a: bha-vipulā)
\end{metre}

%%%%%%%%%%
\subsection*{2.71}
\begin{translation}[hp02_071]
Now Plāvanī:\\
With his abdomen completely filled with the wind of eructation, which has been turned inwards, [the yogi] floats easily even in deep water, like a lotus leaf.
\end{translation}

\begin{testimonia}[hp02_071]
%{[Not in \emph{Yogacintāmaṇi}, \emph{Haṭharatnāvalī}]}

\emph{Yuktabhavadeva} 7.121 (attr.~to Gorakṣanātha)

\begin{versinnote}
\tl{antaḥpravarttitādhāramārutāpūritodaraḥ |\\+}
\tl{payasy agādhe 'pi sukhāt plavate padmapatravat ||\\+}
\tl{ayam eva plāvinī kumbhako'pi |\\!}
\end{versinnote}

Cf.~\emph{Kumbhakapaddhati} 171

\begin{versinnote}
\tl{yatheṣṭaṃ pūrayed vāyuṃ baddhe jālandhare dṛḍhe |\\+}
\tl{hṛdi dhṛtvā jale suptvā plāvinīkumbhako bhavet || 171 ||\\!}
\end{versinnote}

\emph{Yogaprakāśikā} 4.61 (Ten-chapter \emph{Haṭhapradīpikā})

\begin{versinnote}
\tl{plāvanīkumbhakaṃ lakṣayati antar iti |\\+}
\tl{antaḥsañcāritenāpānavāyunā pūritam udaraṃ yasyeti vigrahaḥ |\\!}
\end{versinnote}
\end{testimonia}

\begin{philcomm}[hp02_071]
The heading \emph{atha plāvanī} is not found in \alphaOne \ (which follows the description with \emph{plāvanīkumbhakaṃ}), but we have adopted it to be consistent with the other \emph{kumbhaka} descriptions.

We have understood \emph{udgāramāruta} to refer to the breath of eructation, i.e.~the nāga breath as described in e.g.~\emph{Vivekamārtaṇḍa} 36.
% We have adopted \emph{payasy agādhe pi sukham} in pāda c because \emph{sukham} gives a better meaning and it is a bha-vipulā

\getsiglum{V19}, \getsiglum{P28}, the \emph{Yogacintāmaṇi} and the \emph{Haṭharatnāvalī} omit this verse and accordingly do not mention \emph{plāvinī} in verse 2.44, substituting it with \emph{kevala}.

\end{philcomm}

\begin{metre}[hp02_071]
Anuṣṭubh (c: bha-vipulā)
\end{metre}

%%%%%%%%%%
\subsection*{2.72}
\begin{translation}[hp02_072]
Now \emph{kevalakumbhaka}:

Breath control is said to be threefold, with exhalation, inhalation, and retention. Retention is considered to be twofold: \emph{sahita} and \emph{kevala}.

\end{translation}

\begin{sources}[hp02_072]
\emph{Gorakṣaśataka} 29

\begin{versinnote}
\tl{prāṇaś ca dehajo vāyur āyāmaḥ kumbhakaḥ smṛtaḥ |\\+}
\tl{sa eva dvividhaḥ proktaḥ sahitaḥ kevalas tathā ||\\!}
\end{versinnote}

\emph{Vasiṣṭhasaṃhitā} 3.2cd

\begin{versinnote}
\tl{prāṇāyāmas tribhiḥ prokto recapūrakakumbhakaiḥ || 2 ||\\!}
\end{versinnote}
\end{sources}

\begin{testimonia}[hp02_072]
\emph{Yogalakṣaṇāvalī} f.~32r (attrib. to the \emph{Haṭhapradīpikā})
\begin{versinnote}
\tl{prāṇāyāmas tridhā prokto recapūrakakumbhakaiḥ |\\+}
\tl{bahir virocanaṃ vāyor udarād recakaḥ smṛtaḥ ||\\+}
\tl{bāhyād āpūraṇaṃ vāyor udare pūrakaḥ smṛtaḥ |\\+}
\tl{saṃpūrṇakuṃbhavad vāyor dhāraṇaṃ kumbhako bhavet |\\+}
\tl{sahitaḥ kevalaś ceti kumbhako dvividho mataḥ ||\\!}
\end{versinnote}

\emph{Yuktabhavadeva} 7.122 (attr.~to Gorakṣanātha)

\begin{versinnote}
\tl{atha kevalaḥ\\+}
\tl{prāṇāyāmas tridhā prokto recapūrakakumbhakaiḥ |\\+}
\tl{sahitaḥ kevalaś ceti kumbhako dvividho mataḥ ||\\!}
\end{versinnote}
\end{testimonia}

\begin{philcomm}[hp02_072]
The import of the name \emph{sahita}, “accompanied”, is that \emph{kumbhaka} is accompanied by inhalation and exhalation, and of \emph{kevala}, “isolated”, that it is not.
\end{philcomm}

%%%%%%%%%%
\subsection*{2.73}
\begin{translation}[hp02_073]
The [breath retention] that [the yogi] performs with exhalation and inhalation is \emph{sahita}. He should practice \emph{sahita} until \emph{kevala} is perfected.
\end{translation}


\begin{sources}[hp02_073]
\emph{Vasiṣṭhasaṃhitā} 3.28

\begin{versinnote}
\tl{virecyāpūrya yaṃ kuryāt sa vai sahitakumbhakaḥ \\+}
\tl{sahitaṃ kevalaṃ cātha kumbhakaṃ nityam abhyaset ||\\+}
\tl{yāvat kevalasiddhiḥ syāt tāvat sahitam abhyaset |\\+}
\tl{\var{28a virecyāpūrya ] recyavāpūrya yat \emph{la}, ārecyāpūrya yaḥ \emph{śa}}\\!}
\end{versinnote}

\emph{Yogayājñavalkya} 6.31cd–32

\begin{versinnote}
\tl{recya cāpūrya yaḥ kuryāt sa vai sahitakumbhakaḥ ||\\+}
\tl{sahitaṃ kevalaṃ cātha kumbhakaṃ nityam abhyaset |\\+}
\tl{yāvat kevalasiddhiḥ syāt tāvat sahitam abhyaset ||\\!}
\end{versinnote}

Cf.~\emph{Dattātreyayogaśāstra} 66ab

\begin{versinnote}
\tl{sahito recapūrābhyāṃ tasmāt sahitakumbhakaḥ |\\!}
\end{versinnote}

Cf. \emph{Gorakṣaśataka} 30ab

\begin{versinnote}
\tl{yāvat kevalasiddhiḥ syāt tāvat sahitam abhyaset |\\!}
\end{versinnote}


\end{sources}

\begin{testimonia}[hp02_073]
\emph{Yogacintāmaṇi} f.~96v (attr.~to Yājnavalkya)
\begin{versinnote}
\tl{sahitaṃ kevalaṃ vātha kuṃbhakaṃ nityam abhyaset |\\+}
\tl{yāvat kevalasiddhis syāt tāvat sahitam abhyaset ||\\!}
\end{versinnote}

\emph{Yuktabhavadeva} 7.123 (attr.~to Gorakṣanātha)

\begin{versinnote}
\tl{recya vā pūrakaḥ kāryaḥ śanaiḥ sahitakumbhakaḥ |\\+}
\tl{yāvat kevalasiddhiḥ syāt sahitaṃ tāvad abhyaset ||\\!}
\end{versinnote}
\end{testimonia}

\begin{philcomm}[hp02_073]
We have adopted the \emph{Vasiṣṭhasaṃhitā}’s reading, which is very close to that of \deltaOne \ (\emph{ārecyāpūrya yat kuryāt}) and \deltaThree \ (\emph{ārecyāpūrya yaḥ kuryāt}) and similar to \alphaTwo\ (\emph{recapūrya y\emph{[}a\emph{]}t kāryaḥ}). It is the only one that makes sense of 2.73ab. It appears that the relative pronoun dropped out of the first verse quarter early in the transmission of the \emph{Haṭhapradīpikā} and scribes have tried in various unsuccessful ways to restore some sense. 

Cf.~Marcinowska-Rosól \& \ Sellmer 2021, p.\,102f.

% MD: Perhaps \emph{recakaḥ pūrakaḥ kāryaḥ} which is closer to the readings of N3 and V3?  Cf. 2.74a \emph{recakaṃ pūrakaṃ muktvā}.
% JB: We need the relative pronoun to pick up on the sa in the second pāda
% MD: Not always. Cf. 1.58, 2.24, 2.26.
\end{philcomm}



%%%%%%%%%%
\subsection*{2.74}
\begin{translation}[hp02_074]
Holding the breath comfortably without exhalation and inhalation is \emph{kevalakumbhaka}. This is said to be [the true] breath control. 
\end{translation}

\begin{sources}[hp02_074]
\emph{Vasiṣṭhasaṃhitā} 3.27

\begin{versinnote}
\tl{recanaṃ pūraṇaṃ muktvā sukhaṃ yad vāyudhāraṇam |\\+}
\tl{prāṇāyāmo 'yam ity uktaḥ sa vai kevalakumbhakaḥ ||\\!}
\end{versinnote}

\emph{Yogayājñavalkya} 6.30cd–6.31ab

\begin{versinnote}
\tl{recakaṃ pūrakaṃ muktvā sukhaṃ yad vāyudhāraṇam |\\+}
\tl{prāṇāyāmo 'yam ity uktaḥ sa vai kevalakumbhakaḥ ||\\!}
\end{versinnote}
\end{sources}

\begin{testimonia}[hp02_074]
\emph{Haṭharatnāvalī} 2.28

\begin{versinnote}
\tl{atha kevalaḥ---\\+}
\tl{recakaṃ pūrakaṃ muktvā sukhaṃ yad vāyudhāraṇam |\\+}
\tl{prāṇāyāmo 'yam ity uktaḥ sa vai kevalakumbhakaḥ ||\\!}
\end{versinnote}

\emph{Yogacintāmaṇi} f.~94v (attr.~to `\emph{tajjñāḥ}')

\begin{versinnote}
\tl{recakaṃ pūrakaṃ muktvā yat sukhaṃ vāyudhāraṇam |\\+}
\tl{prāṇāyāmo 'yam ity uktaḥ sa vai kevalakumbhakaḥ ||\\!}
\end{versinnote}


\emph{Yuktabhavadeva} 7.124 (attr.~to Gorakṣanātha)

\begin{versinnote}
\tl{recakaṃ pūrakaṃ muktvā yad vāyudhāraṇam |\\+}
\tl{prāṇāyāmo 'yam ity uktaḥ sa vai kevalakumbhakaḥ ||\\!}
\end{versinnote}

%Yogadīpikā 77cd--78ab

%\begin{versinnote}
%\tl{recakaṃ pūrakaṃ muktvā susukhaṃ vāyudhāraṇaṃ\\+}
%\tl{prāṇāyāmo 'yam ity uktaḥ sa vai kevalakuṃbhakaḥ\\!}
%\end{versinnote}
\end{testimonia}

\begin{philcomm}[hp02_074]
%The relative pronoun is omitted in V1 and the J10 group, which have \emph{vāyunirodhanam}. But the relative is needed to connect the description of kevalakumbhaka in the first hemistich to the last pāda (\emph{sa vai kevalakumbhakaḥ}).
On this verse, see Marcinkowska-Rosół and Sellmer 2020: 102–105.
\end{philcomm}
%Maria Marcinkowska-Rosół & Sven Sellmer. “Notes on some difficult passages of the Haṭhapradīpikā”. Zeitschrift der Deutschen Morgenländischen Gesellschaft 171 (2021), pp. 101–21.

%%%%%%%%%%
\subsection*{2.75}
\begin{translation}[hp02_075]
When breath retention is mastered on its own, without exhalation and inhalation, nothing in the three worlds  is impossible for [the yogi].
\end{translation}


\begin{sources}[hp02_075]
\emph{Dattātreyayogaśāstra} 74

\begin{versinnote}
\tl{kevale kumbhake siddhe recapūrakavarjite |\\+}
\tl{na tasya durlabhaṃ kiṃ cit triṣu lokeṣu vidyate ||\\!}
\end{versinnote}

\emph{Vasiṣṭhasaṃhitā} 3.30

\begin{versinnote}
\tl{kevale kumbhake siddhe recapūraṇavarjite |\\+}
\tl{na tasya durlabhaṃ kiṃ cit triṣu lokeṣu vidyate ||\\!}
\end{versinnote}
\end{sources}

\begin{testimonia}[hp02_075]
\emph{Haṭharatnāvalī} 2.29

\begin{versinnote}
\tl{kevale kumbhake siddhe recapūrakavarjite |\\+}
\tl{na tasya durlabhaṃ kiñ cit triṣu lokeṣu vidyate ||\\!}
\end{versinnote}

\emph{Yogacintāmaṇi} f.~97r (attr.~to Yājñavalkya)

\begin{versinnote}
\tl{kevale kumbhake siddhe recapūraṇavarjite |\\+}
\tl{na tasya durlabhaṃ kiṃ cit triṣu lokeṣu vidyate ||\\!}
\end{versinnote}

\emph{Yuktabhavadeva} 7.125 (attr.~to Gorakṣanātha)

\begin{versinnote}
\tl{kevale kumbhake siddhe recapūrakavarjite |\\+}
\tl{na tasya durlabhaṃ kiñcit triṣu lokeṣuṃ vidyate ||\\!}
\end{versinnote}

\end{testimonia}

%%%%%%%%%%
\subsection*{2.76}
\begin{translation}[hp02_076]
He who is capable of \emph{kevalakumbhaka} undoubtedly attains [the ability to] hold the breath as long as he wants and the state of Rājayoga.
\end{translation}

\begin{testimonia}[hp02_076]
\emph{Haṭharatnāvalī} 2.30

\begin{versinnote}
\tl{śaktaḥ kevalakumbhena yatheṣṭaṃ vāyudhāraṇam |\\+}
\tl{etādṛśo rājayogo kathito nātra saṃśayaḥ ||\\!}
\end{versinnote}

\emph{Yuktabhavadeva} 7.126 (attr.~to Gorakṣanātha)

\begin{versinnote}
\tl{śaktaḥ kevalakumbhena yatheṣṭaṃ vāyudhāraṇam |\\+}
\tl{rājayogapadaṃ samyak labhate nātra saṃśayaḥ ||\\!}
\end{versinnote}

\emph{Haṭhatattvakaumudī} 44.59

\begin{versinnote}
\tl{haṭhapradīpikāyām–\\+}
\tl{śaktaḥ kevalakumbhena yatheṣṭaṃ vāyudhāraṇe |\\+}
\tl{rājayogapadaṃ caiva labhate nātra saṃśayaḥ || 59 ||\\!}
\end{versinnote}
\end{testimonia}

%\begin{philcomm}[hp02_076]
%\end{philcomm}

%%%%%%%%%%
\subsection*{2.77}
\begin{translation}[hp02_077]
Rājayoga does not succeed without Haṭha nor Haṭha without Rājayoga so one should practise them both together until [the] niṣpatti [stage].
\end{translation}

\begin{sources}[hp02_077]
\emph{Śivasaṃhitā} 5.222

\begin{versinnote}
\tl{haṭhaṃ vinā rājayogo rājayogaṃ vinā haṭhaḥ |\\+}
\tl{na sidhyati tato yugmam āniṣpatteḥ samabhyaset\\+}
\tl{\textup{[middle hemistich not in mss. I, III, IV, VII, IX, X, XII, XIV--XVI]}\\+}
\tl{tasmāt pravartate yogī haṭhe sadgurumārgataḥ ||\\!}
\end{versinnote}
\end{sources}

\begin{testimonia}[hp02_077]
\emph{Haṭharatnāvalī} 1.19

\begin{versinnote}
\tl{haṭhaṃ vinā rājayogo rājayogaṃ vinā haṭhaḥ |\\+}
\tl{vyāptiḥ syād avinābhūtā śrīrājahaṭhayogayoḥ ||\\!}
\end{versinnote}

\emph{Yogacintāmaṇi} f.~21r (attr.~to the \emph{Haṭhapradīpika})

\begin{versinnote}
\tl{haṭhapradīpikāyām\\+}
\tl{haṭhaṃ vinā rājayogo rājayogaṃ vinā haṭhaḥ |\\+}
\tl{na sidhyati tato yugmaṃ manīṣy etau samabhyaset |\\+}
\tl{haṭhaṃ vinā rājayogaṃ rājayogaṃ vinā haṭham |\\+}
\tl{ye vai caranti tān manye prayāsaphalavarjitān iti ||\\!}
\end{versinnote}

\emph{Yuktabhavadeva} 7.127 (attr.~to Gorakṣanātha)

\begin{versinnote}
\tl{haṭhaṃ vinā rājayogo rājayogaṃ vinā haṭhaḥ |\\+}
\tl{na sidhyati tato yugmam āniṣpatteḥ samācaret ||\\!}
\end{versinnote}

%Śivayogadarpana
%\begin{versinnote}
%\tl{haṭhaṃ vinā rājayogo rājayogaṃ vinā haṭhaḥ |\\+}
%\tl{na sidhyati tato yugmam manīṣī tat samabhyaset || 6 ||\\!}
%\end{versinnote}
\end{testimonia}

\begin{philcomm}[hp02_077]
On the niṣpatti stage see 4.23–25.
\end{philcomm}

\begin{metre}[hp02_077]
Anuṣṭubh (a: ra-vipulā)
\end{metre}

%%%%%%%%%%
\subsection*{2.78}
\begin{translation}[hp02_078]
At the end of exhaling the retained breath, [the yogi] should make the mind supportless. By practising in this way he reaches the state of Rājayoga .
\end{translation}

\begin{testimonia}[hp02_078]
\emph{Yuktabhavadeva} 7.128 (attr.~to Gorakṣanātha)

\begin{versinnote}
\tl{kumbhitaḥ prāṇarecānte kuryyāc cittaṃ nirāmayam ||\\+}
\tl{evamabhyāsayogena rājayogapadaṃ vrajet || 128 ||\\!}
\end{versinnote}

\emph{Haṭhatattvakaumudī} 44.60 (attr.~to the \emph{Haṭhapradīpika})

\begin{versinnote}
\tl{kumbhitaḥ prāṇarecānte kuryyāc cittaṃ nirāśrayam |\\+}
\tl{evamabhyāsayogena rājayogaṃ labhet punaḥ || 60 ||\\+}
\tl{nirāśrayaṃ saṃkalparahitam ||\\!}
\end{versinnote}

\emph{Yogaprakāśikā} 4.67 (Ten-chapter \emph{Haṭhapradīpikā})

\begin{versinnote}
\tl{tad eva visadayati kumbhakam iti || kevalakumbhakābhyāsena cittaṃ dagdhaparṇavat nirvāsanaṃ bhavatītyarthaḥ || anyad vyākhyātam || 67 ||\\!}
\end{versinnote}
\end{testimonia}

\begin{philcomm}[hp02_078]
The reading \emph{kumbhitaḥ}, which is attested by \alphaOne\ and the testimonia, does not make sense here because the subject must be the yogi (\emph{prānaḥ} would be unmetrical). In other texts, the word \emph{kumbhita} is used to qualify the breath and means `retained' (e.g., \emph{Yogabīja} 94, \emph{Kumbhakapaddhati} 127, \emph{Yuktabhavadeva} 8.32, etc.).
\end{philcomm}

%%%%%%%%%%
\subsection*{2.79}
\begin{translation}[hp02_079]
As a result of retaining the breath, kuṇḍalinī awakens; as a result of the awakening of kuṇḍalinī, suṣumṇā becomes free of blockages and success in Haṭha arises.
\end{translation}

\begin{testimonia}[hp02_079]
\emph{Yogacintāmaṇi} f.~97a (attr.~to the \emph{Haṭhayoga})

\begin{versinnote}
\tl{kumbhakāt kuṇḍalībodhaḥ kuṇḍalībodhato bhavet |\\+}
\tl{anargalaḥ suṣumṇānto haṭhasiddhiś ca jāyate ||\\+}
\tl{anargalaḥ ] L, antargataḥ N\\!}
\end{versinnote}

\emph{Yuktabhavadeva} 7.129 (attr.~to Gorakṣanātha)

\begin{versinnote}
\tl{kumbhakāt kuṇḍalībodhaḥ kuṇḍalībodhato bhavet |\\+}
\tl{anargalā suṣumnā ca haṭhasiddhiśca jāyate ||\\!}
\end{versinnote}

\emph{Haṭhatattvakaumudī} 44.61

\begin{versinnote}
\tl{kumbhakāt kuṇḍalībodhaḥ kuṇḍalībodhato bhavet |\\+}
\tl{anargalā suṣumṇā ca haṭhasiddhiḥ prajāyate || iti ||\\+}
\tl{kumbhakaprāṇāyāmāt bodho jāgaraṇam | suṣumṇā anargalā bādhakarahitā bhavati | tato yogasiddhir bhavati iti ||\\!}
\end{versinnote}
\end{testimonia}

%\begin{philcomm}[hp02_079]
%\end{philcomm}

%%%%%%%%%%
\subsection*{2.80}
\begin{translation}[hp02_080]
Thinness of the body, clear complexion, clarity of the inner sound, very bright eyes, freedom from disease, mastery of semen, stimulation of the [body’s] fire [and] purification of the channels are the signs of success in Haṭha.
\end{translation}

\begin{testimonia}[hp02_080]
\emph{Haṭharatnāvalī} 1.59

\begin{versinnote}
\tl{vapuḥkṛśatvaṃ vadane prasannatā\\+} 
\tl{nādaspuṭatvaṃ nayane ca nirmale | \\+}
\tl{arogatā bindujayo 'gnidīpanaṃ\\+}  
\tl{nāḍīṣu śuddhir haṭhasiddhilakṣaṇam ||\\!}
\end{versinnote}

\emph{Yogacintāmaṇi} f.~111v (attr.~to the \emph{Haṭhapradīpikā})

\begin{versinnote}
\tl{vapuḥkṛśatvaṃ vadane prasannatā \\+}
\tl{nādasphuṭatvaṃ nayane sunirmale |\\+}
\tl{arogitā bindujayo 'gnidīpanaṃ \\+}
\tl{nāḍīviśuddhir haṭhasiddhilakṣaṇam ||\\!}
\end{versinnote}

\emph{Yuktabhavadeva} 7.129 (attr.~to Gorakṣanātha)

\begin{versinnote}
\tl{vapuḥkṛśatvaṃ vadane prasannatā \\+}
\tl{nādasphuṭatvaṃ nayane ca nirmale |\\+}
\tl{arogatā bindujayo'gnidīpanaṃ \\+}
\tl{nāḍīviśuddhir haṭhasiddhilakṣaṇam ||\\!}
\end{versinnote}
\end{testimonia}

%\begin{philcomm}[hp02_080]
%\end{philcomm} 

\begin{metre}[hp02_080]
Vaṃśamālā
\end{metre}

\end{ekdosis}
\end{document}
