\documentclass[10pt]{memoir}
\setstocksize{220mm}{155mm} 	        
\settrimmedsize{220mm}{155mm}{*}	
\settypeblocksize{170mm}{116mm}{*}	
\setlrmargins{18mm}{*}{*}
\setulmargins{*}{*}{1.2}
% \setlength{\headheight}{5pt}
\checkandfixthelayout[lines]
\linespread{1.16}

\setlength{\footmarkwidth}{1.3em}
\setlength{\footmarksep}{0em}
\setlength{\footparindent}{1.3em}
\footmarkstyle{\textsuperscript{#1} }
\usepackage{fnpos}
\makeFNbottom

\usepackage[teiexport=tidy,poetry=verse]{ekdosis}
\usepackage{sanskrit-poetry}

\usepackage[english]{babel}
\usepackage{babel-iast,xparse,xcolor}
\babelfont[iast]{rm}[Renderer=Harfbuzz, Scale=1.5]{AdishilaSan}
\babelfont[english]{rm}[Scale=0.9]{Adobe Text Pro}
\babeltags{dev = iast}
\babeltags{eng = english}

\SetHooks{
	lemmastyle=\bfseries,
	refnumstyle=\selectlanguage{english}\color{blue}\bfseries, 
	}
\newif\ifinapparatus
\DeclareApparatus{default}[
	lang=english,
	sep = {] },
	delim=\hskip 0.75em,
	rule=none,
	]
\DeclareApparatus{notes}[
	lang=english,
	sep = {},
	delim=\hskip 0.75em,
	rule=\rule{0.7in}{0.4pt},
	]

\DeclareShorthand{conj}{\texteng{\emph{conj.}}}{ego}
\DeclareShorthand{emend}{\texteng{\emph{em.}}}{ego}

\setlength{\vrightskip}{-10pt}
\setlength{\vgap}{3mm}
\verselinenumfont{\footnotesize\selectlanguage{english}\normalfont}




%%%%%%%%%%%%%%%%%%%% THE  MSS         %%%%%%%%%%%%%%%%%%%%%%%%%%%

%%% Versions
\DeclareWitness{Vu}{\selectlanguage{english}Vulg}{Vulgate, i.e. Brahmānanda's version}[]           
\DeclareWitness{X}{\selectlanguage{english}X}{TenChapter Version, Jodhpur 02228 and 02225 (ed. Lonavla)}[]
\DeclareWitness{Six}{\selectlanguage{english}Ṣ}{SixChapterVersion, ``6ChapterHPms'', fragment of enlarged text, Jodhpur}[]
% Mss. in Geographical Groups
%%%% Varanasi mss (Sampūrṇānanda mss). V1 is Important
\DeclareWitness{V1}{\selectlanguage{english}V\textsubscript{1}}{Sampurnananda Library Sarasvati Bhavan 30109}[]
        \DeclareHand{V1ac}{V1}{\selectlanguage{english}V\rlap{\textsubscript{1}}\textsuperscript{ac}}[] % added by MD
        \DeclareHand{V1pc}{V1}{\selectlanguage{english}V\rlap{\textsubscript{1}}\textsuperscript{pc}}[] % added by MD
\DeclareWitness{V2}{\selectlanguage{english}V\textsubscript{2}}{Sampurnananda Library Sarasvati Bhavan 29869}[]
\DeclareWitness{V3}{\selectlanguage{english}V\textsubscript{3}}{Sampurnananda Library Sarasvati Bhavan 29899}[]
\DeclareWitness{V4}{\selectlanguage{english}V\textsubscript{4}}{Sampurnananda Library Sarasvati Bhavan 29937}[]
\DeclareWitness{V5}{\selectlanguage{english}V\textsubscript{5}}{Sampurnananda Library Sarasvati Bhavan 29938}[]
\DeclareWitness{V6}{\selectlanguage{english}V\textsubscript{6}}{Sampurnananda Library Sarasvati Bhavan 29991}[]
\DeclareWitness{V8}{\selectlanguage{english}V\textsubscript{8}}{Sampurnananda Library Sarasvati Bhavan 30014}[]
\DeclareWitness{V11}{\selectlanguage{english}V\textsubscript{11}}{Sampurnananda Library Sarasvati Bhavan 30029}[]
\DeclareWitness{V12}{\selectlanguage{english}V\textsubscript{12}}{Sampurnananda Library Sarasvati Bhavan 30030}[]
\DeclareWitness{V13}{\selectlanguage{english}V\textsubscript{13}}{Sampurnananda Library Sarasvati Bhavan 30031}[]
\DeclareWitness{V14}{\selectlanguage{english}V\textsubscript{14}}{Sampurnananda Library Sarasvati Bhavan 30050}[]
\DeclareWitness{V15}{\selectlanguage{english}V\textsubscript{15}}{Sampurnananda Library Sarasvati Bhavan 30051}[]
\DeclareWitness{V15pc}{\selectlanguage{english}V\rlap{\textsubscript{15}}\textsuperscript{pc}\space}{}[]
\DeclareWitness{V16}{\selectlanguage{english}V\textsubscript{16}}{Sampurnananda Library Sarasvati Bhavan 30052}[]
\DeclareWitness{V17}{\selectlanguage{english}V\textsubscript{17}}{Sampurnananda Library Sarasvati Bhavan 30053}[] % added by MD
\DeclareWitness{V16pc}{\selectlanguage{english}V\rlap{\textsubscript{16}}\textsuperscript{pc}\space}{}[]
\DeclareWitness{V18}{\selectlanguage{english}V\textsubscript{18}}{Sampurnananda Library Sarasvati Bhavan 30064}[]
\DeclareWitness{V19}{\selectlanguage{english}V\textsubscript{19}}{Sampurnananda Library Sarasvati Bhavan 30069}[]
\DeclareWitness{V21}{\selectlanguage{english}V\textsubscript{21}}{Sampurnananda Library Sarasvati Bhavan 30104}[]
\DeclareWitness{V22}{\selectlanguage{english}V\textsubscript{22}}{Sampurnananda Library Sarasvati Bhavan 30110}[]
\DeclareWitness{V25}{\selectlanguage{english}V\textsubscript{25}}{Sampurnananda Library Sarasvati Bhavan 30122}[]
\DeclareWitness{V26}{\selectlanguage{english}V\textsubscript{26}}{Sampurnananda Library Sarasvati Bhavan 30123}[]
\DeclareWitness{V28}{\selectlanguage{english}V\textsubscript{28}}{Sampurnananda Library Sarasvati Bhavan 30136}[]
\DeclareWitness{W2}{\selectlanguage{english}W\textsubscript{2}}{Wai ??}[]
\DeclareWitness{W4}{\selectlanguage{english}W\textsubscript{4}}{Wai 399-6171}[]

%%%%%%%%%%%%%%%%%%%%%%%%%%%%%%%%%
%%% Jammu & Kaschmir
\DeclareWitness{K1}{\selectlanguage{english}K\textsubscript{1}}{Raghunātha Temple Library 4383}[settlement=Jammu]
        \DeclareWitness{K1ac}{\selectlanguage{english}K\rlap{\textsubscript{1}}\textsuperscript{ac}\space}{}[]
        \DeclareWitness{K1pc}{\selectlanguage{english}K\rlap{\textsubscript{1}}\textsuperscript{pc}\space}{}[]
\DeclareWitness{K3}{\selectlanguage{english}K\textsubscript{3}}{Privat collection}
\DeclareWitness{L1}{\selectlanguage{english}L\textsubscript{1}}{SOAS RE 43454}[settlement=Jammu]
% More details? Catalogue number? L1 And C1 very close (and come from same region)
%%%%%%%%%%%%%%%%%%%%%%%%%%%%%%%%
% Jodhpur
% J10 is important
\DeclareWitness{J10}{\selectlanguage{english}J\textsubscript{10}}{MSPP Jodhpur 2230}[]
        \DeclareHand{J10ac}{J10}{\selectlanguage{english}J\rlap{\textsubscript{10}}\textsuperscript{ac}}[] % modified by MD
        \DeclareHand{J10pc}{J10}{\selectlanguage{english}J\rlap{\textsubscript{10}}\textsuperscript{pc}}[] % modified by MD
\DeclareWitness{J1}{\selectlanguage{english}J\textsubscript{1}}{Jodhpur 02231}[]
\DeclareWitness{J2}{\selectlanguage{english}J\textsubscript{2}}{Jodhpur 02232}[]   
\DeclareWitness{J3}{\selectlanguage{english}J\textsubscript{3}}{Jodhpur 02233}[]
\DeclareWitness{J4}{\selectlanguage{english}J\textsubscript{4}}{Jodhpur 02234}[]
        \DeclareWitness{J4ac}{\selectlanguage{english}J\rlap{\textsubscript{4}}\textsuperscript{ac}\space}{MSPP Jodhpur 02234}[]
        \DeclareWitness{J4pc}{\selectlanguage{english}J\rlap{\textsubscript{4}}\textsuperscript{pc}\space}{MSPP Jodhpur 02234}[]
\DeclareWitness{J5}{\selectlanguage{english}J\textsubscript{5}}{Jodhpur 02235}[]  % 4 chapters, 34 jpgs,   long colophon, missing lines in the beginning.
\DeclareWitness{J6}{\selectlanguage{english}J\textsubscript{6}}{Jodhpur 02237}[]  % 4 chapters, 41 jpgs
%\DeclareWitness{J6ac}{\selectlanguage{english}J\rlap{\textsubscript{6}}\textsubscript{ac}}{Jodhpur 02237}[]  % 4 chapters, 49 jpgs,   1st folio: idaṃ gulābarāyasya
% tulasīrāmaśarmmaṇaḥ putrasya pustakaṃ ...        End: iti śrīsahajānandasantānacintāmaṇisvātmārāmaviracitāyāṃ ..
% saṃvat 1802   (more consistent text)
%\DeclareWitness{J6pc}{\selectlanguage{english}J\rlap{\textsubscript{6}}\textsubscript{pc}}{Jodhpur 02237}[] 
\DeclareWitness{J7}{\selectlanguage{english}J\textsubscript{7}}{Jodhpur 02241}[]  % 4 chapters, 41 jpgs
\DeclareWitness{J8}{\selectlanguage{english}J\textsubscript{8}}{Jodhpur 23709}[]  % 4 chapters,  87 jpgs.   saṃvat 1724
\DeclareHand{J8ac}{J8}{\selectlanguage{english}J\rlap{\textsubscript{8}}\textsuperscript{ac}}[]  % changed by MD
\DeclareHand{J8pc}{J8}{\selectlanguage{english}J\rlap{\textsubscript{8}}\textsuperscript{pc}}[]  % changed by MD
\DeclareWitness{J9}{\selectlanguage{english}J\textsubscript{9}}{Jodhpur 02224}[]  %  fragment, 20 jpgs.
\DeclareWitness{J11}{\selectlanguage{english}J\textsubscript{11}}{Jodhpur 23532}[]
        \DeclareHand{J11ac}{J11}{\selectlanguage{english}J\rlap{\textsubscript{11}}\textsuperscript{ac}}[] % added by MD
        \DeclareHand{J11pc}{J11}{\selectlanguage{english}J\rlap{\textsubscript{11}}\textsuperscript{pc}}[] % added by MD
\DeclareWitness{J12}{\selectlanguage{english}J\textsubscript{12}}{Jodhpur 18552}[] 
\DeclareWitness{J13}{\selectlanguage{english}J\textsubscript{13}}{Jodhpur 02229}[]  %  5 chapters, 93 jpgs.
\DeclareWitness{J14}{\selectlanguage{english}J\textsubscript{14}}{Jodhpur 02239}[]  %  4 chapters
\DeclareWitness{J15}{\selectlanguage{english}J\textsubscript{15}}{Jodhpur 9732A}[]
\DeclareWitness{J16}{\selectlanguage{english}J\textsubscript{16}}{Jodhpur 9732B}[]
\DeclareWitness{J17}{\selectlanguage{english}J\textsubscript{17}}{Jodhpur 3013}[]
% Haṭhapradīpikā with (non-Sanskrit) Bhāṣya RORI Jodhpur ACC.NO.18552
%  Haṭhapradīpikā with (non-Sanskrit) commentary, RORI Alwar 952, 4 chapters,  colophon of the comm:
% iti śrīlāhorīmiśravrajabhūṣanaviracitāyāṃ bhāvārthadīpikāyāṃ caturthodhyāya ..    
%  Haṭhapradīpikā (5 chapter) MSPP Jodhpur ACC.NO.02229/

%%%%%%%%%%        Bodleian, Oxford
\DeclareWitness{B1}{\selectlanguage{english}B\textsubscript{1}}{Bodleian Library No. d.457(8)}[settlement=Oxford]
\DeclareWitness{B2}{\selectlanguage{english}B\textsubscript{2}}{Bodleian Library No. d.458(1)}[settlement=Oxford]
\DeclareWitness{B3}{\selectlanguage{english}B\textsubscript{3}}{Bodleian Library No. d.458(9)}[settlement=Oxford]

%%%%%%%%%%%   Chandigarh
\DeclareWitness{C1}{\selectlanguage{english}C\textsubscript{1}}{Lalchand M-2080}[]%L1 And C1 very close (and come from same region)
\DeclareWitness{C2}{\selectlanguage{english}C\textsubscript{2}}{Lalchand M-6065}[]
\DeclareWitness{C3}{\selectlanguage{english}C\textsubscript{3}}{Lalchand M-1293}[]
\DeclareWitness{C4}{\selectlanguage{english}C\textsubscript{4}}{Lalchand M-2081}[]
\DeclareWitness{C4ac}{\selectlanguage{english}C\rlap{\textsubscript{4}}\textsuperscript{ac}\space}{}[]
\DeclareWitness{C4pc}{\selectlanguage{english}C\rlap{\textsubscript{4}}\textsuperscript{pc}\space}{}[]
\DeclareWitness{C5}{\selectlanguage{english}C\textsubscript{5}}{Lalchand M-2082}[]%doesn't have chapter 1
\DeclareWitness{C6}{\selectlanguage{english}C\textsubscript{6}}{Lalchand M-2089}[]
\DeclareWitness{C7}{\selectlanguage{english}C\textsubscript{7}}{Lalchand M-6494}[]
\DeclareWitness{C8}{\selectlanguage{english}C\textsubscript{8}}{Lalchand M-2091}[]
        \DeclareHand{C8ac}{C8}{\selectlanguage{english}C\rlap{\textsubscript{8}}\textsuperscript{ac}}[]
        \DeclareHand{C8pc}{C8}{\selectlanguage{english}C\rlap{\textsubscript{8}}\textsuperscript{pc}}[]
\DeclareWitness{C9}{\selectlanguage{english}C\textsubscript{9}}{Lalchand M-4530}[]


% %%%%%%%%%%        Nepalese
\DeclareWitness{N1}{\selectlanguage{english}N\textsubscript{1}}{NGMPP A1400-2}[]
\DeclareWitness{N2}{\selectlanguage{english}N\textsubscript{2}}{NGMPP B 39-19}[]
\DeclareWitness{N3}{\selectlanguage{english}N\textsubscript{3}}{NGMPP B 62-20}[]
\DeclareWitness{N5}{\selectlanguage{english}N\textsubscript{5}}{NGMPP A60-15 + A61-1}[]
\DeclareWitness{N4}{\selectlanguage{english}N\textsubscript{4}}{NGMPP A61-2}[]
\DeclareWitness{N6}{\selectlanguage{english}N\textsubscript{6}}{NGMPP A61-6}[]
\DeclareWitness{N9}{\selectlanguage{english}N\textsubscript{9}}{NGMPP A62-33}[]
\DeclareWitness{N10}{\selectlanguage{english}N\textsubscript{10}}{NGMPP A62-37}[]
\DeclareWitness{N11}{\selectlanguage{english}N\textsubscript{11}}{NGMPP A63-15}[]
\DeclareWitness{N12}{\selectlanguage{english}N\textsubscript{12}}{NGMPP A939-19}[]
\DeclareWitness{N13}{\selectlanguage{english}N\textsubscript{13}}{NGMPP A1378-18}[]
\DeclareWitness{N16}{\selectlanguage{english}N\textsubscript{16}}{NGMPP B39-20}[]
\DeclareWitness{N17}{\selectlanguage{english}N\textsubscript{17}}{NGMPP B 111-10}[]
\DeclareWitness{N18}{\selectlanguage{english}N\textsubscript{18}}{NGMPP E 929-3}[]
\DeclareWitness{N19}{\selectlanguage{english}N\textsubscript{19}}{NGMPP E-1528-1 / E-1527-7(4)}[]
\DeclareWitness{N20}{\selectlanguage{english}N\textsubscript{20}}{NGMPP E 2037-13 }[]
\DeclareWitness{N21}{\selectlanguage{english}N\textsubscript{21}}{NGMPP E 2097-31}[]
\DeclareWitness{N22}{\selectlanguage{english}N\textsubscript{22}}{NGMPP G 4-4}[]
\DeclareWitness{N23}{\selectlanguage{english}N\textsubscript{23}}{NGMPP G 25-2}[]
        \DeclareHand{N23ac}{N23}{\selectlanguage{english}N\rlap{\textsubscript{23}}\textsuperscript{ac}}[] % added by MD
        \DeclareHand{N23pc}{N23}{\selectlanguage{english}N\rlap{\textsubscript{23}}\textsuperscript{pc}}[] % added by MD
\DeclareWitness{N24}{\selectlanguage{english}N\textsubscript{24}}{NGMPP G 190-16}[]
\DeclareWitness{N24ac}{\selectlanguage{english}N\rlap{\textsubscript{24}}\textsuperscript{ac}\space}{}[]
\DeclareWitness{N24pc}{\selectlanguage{english}N\rlap{\textsubscript{24}}\textsuperscript{pc}\space}{}[]
\DeclareWitness{N26}{\selectlanguage{english}N\textsubscript{26}}{NGMPP T 24-3}[]

% %%%%%%%%%%        Pune

\DeclareWitness{P1}{\selectlanguage{english}P\textsubscript{1}}{Ānandāśrama S16-3-21}[]
\DeclareWitness{P2}{\selectlanguage{english}P\textsubscript{2}}{Ānandāśrama S16-2-20}[]
\DeclareWitness{P3}{\selectlanguage{english}P\textsubscript{3}}{BISM (79) 314}[]
\DeclareWitness{P4}{\selectlanguage{english}P\textsubscript{4}}{BISM (91) 191}[]
\DeclareWitness{P5}{\selectlanguage{english}P\textsubscript{5}}{BISM (29) 5790}[]
\DeclareWitness{P6}{\selectlanguage{english}P\textsubscript{6}}{BORI 263/1879-80}[]
\DeclareWitness{P7}{\selectlanguage{english}P\textsubscript{7}}{BORI 665/1883-84}[]
\DeclareWitness{P8}{\selectlanguage{english}P\textsubscript{8}}{BORI 316/1895-98}[]
\DeclareWitness{P9}{\selectlanguage{english}P\textsubscript{9}}{BORI 733-1891-95}[]
\DeclareWitness{P10}{\selectlanguage{english}P\textsubscript{10}}{BORI 222-1884-86}[]
\DeclareWitness{P11}{\selectlanguage{english}P\textsubscript{11}}{BORI 221-1882–83}[]
\DeclareWitness{P12}{\selectlanguage{english}P\textsubscript{12}}{Ānandāśrama S16-3-24}[]
\DeclareWitness{P13}{\selectlanguage{english}P\textsubscript{13}}{Ānandāśrama S16-2-22}[]
\DeclareWitness{P14}{\selectlanguage{english}P\textsubscript{14}}{Ānandāśrama S16-3-23}[]
\DeclareWitness{P15}{\selectlanguage{english}P\textsubscript{15}}{BISM (64) 919}[]
\DeclareWitness{P16}{\selectlanguage{english}P\textsubscript{16}}{BISM (64) 1115}[]
\DeclareWitness{P17}{\selectlanguage{english}P\textsubscript{17}}{BISM 620/1886-92}[]
\DeclareWitness{P18}{\selectlanguage{english}P\textsubscript{18}}{BORI 615/1887-91}[]
\DeclareWitness{P19}{\selectlanguage{english}P\textsubscript{19}}{BISM 46-39}[]
\DeclareWitness{P20}{\selectlanguage{english}P\textsubscript{20}}{BISM 39-273}[]
\DeclareWitness{P21}{\selectlanguage{english}P\textsubscript{21}}{BISM 37-743}[]
\DeclareWitness{P22}{\selectlanguage{english}P\textsubscript{22}}{BISM 37-729}[]
\DeclareWitness{P23}{\selectlanguage{english}P\textsubscript{23}}{BISM 33-60}[]
\DeclareWitness{P24}{\selectlanguage{english}P\textsubscript{24}}{BISM 29-5790}[]% =P5!
\DeclareWitness{P25}{\selectlanguage{english}P\textsubscript{25}}{BISM 29-3657}[]
\DeclareWitness{P26}{\selectlanguage{english}P\textsubscript{26}}{BISM 25-281}[]
\DeclareWitness{P27}{\selectlanguage{english}P\textsubscript{27}}{BISM 7-489}[]
\DeclareWitness{P28}{\selectlanguage{english}P\textsubscript{28}}{BORI 399-1895-1902}[]

%%%%%   Mysore
\DeclareWitness{M1}{\selectlanguage{english}M\textsubscript{1}}{P-5682/4}[]
%%%%%   Tübingen
\DeclareWitness{Tue}{\selectlanguage{english}Tü}{Ma I 339}[]
%%%%%%%%%%
\DeclareWitness{YC}{\selectlanguage{english}YC}{Yogacintāmaṇi}[]
\DeclareWitness{ceteri}{\selectlanguage{english}cett.}{ceteri}[]

%%%%%%%%%% Mss with Commentary
\DeclareWitness{A1}{\selectlanguage{english}A\textsubscript{1}}{Alwar 952}[]

\DeclareWitness{Jyo}{\selectlanguage{english}J\textsubscript{yo}}{Brahmānanda's version}[]

%%%%%%%%%%%%%%%%%%%%%%%%%%%%%%%%%%%%%%%%%%%
%List of all Sigla:
%A1,B1,B2,B3,C1,C2,C3,C4,C6,C7,C8,C9,J1,J2,J3,J4,J10,J13,J14,J15,J17,L1,M1,N3,N5,N6,N9,N10,N11,N12,N13,N16,N17,N19,N20,N21,N22,N23,N24,Tü,V1,V2,V3,V4,V5,V6,V8,V11,V19,V22,V26,Vu
%%%%%%%%%%%%%%%%%%%%%%%%%%%%%%%%%%%%%%%%%%%

\DeclareWitness{G4}{\selectlanguage{english}G\textsubscript{4}}{GOML D18885 (Bundle SD5051)}[]
\DeclareWitness{G5}{\selectlanguage{english}G\textsubscript{5}}{GOML R3841/ SR2190}[]
\DeclareWitness{G7}{\selectlanguage{english}G\textsubscript{7}}{GOML D4394}[]

\DeclareWitness{Ko}{\selectlanguage{english}K\textsubscript{o}}{Koba, Gujarat 55626}[]


%%%%%                   Abbreviation for the printed apparatus,        xml interface needed
%%%%%                   (synonyms in same line)

% Macro for Editing Abbrevs.
%\def\om{\textrm{\footnotesize \textit{omitted in}\ }} %prints om. for omitted in apparatus
%\def\korr{\textrm{\footnotesize \textit{em.}\ }} %prints em. for emended in apparatus
%\def\conj{\textrm{\footnotesize \textit{conj.}\ }} %prints conj. for conjectured in apparatus


\def\eyeskip{\textrm{{ab.\,oc. }}}   
\def\aberratio{\textrm{{ab.\,oc. }}}
\def\ad{\textrm{{ad}}}   
\def\add{\textrm{{add.\ }}}
\def\ann{\textrm{{ann.\ }}}
\def\ante{\textrm{{ante }}}
\def\post{\textrm{{post }}}
%\def\ceteri{cett.\,}             % for simplifying the apparatus in print                  
\def\codd{\textrm{{codd.\ }}}   %  the same
\def\conj{\textrm{{coni.\ }}}  
\def\coni{\textrm{{coni.\ }}}
\def\contin{\textrm{{contin.\ }}}
\def\corr{\textrm{{corr.\ }}}
\def\del{\textrm{{del.\ }}}
\def\dub{\textrm{{ dub.\ }}}
\def\emend{\textrm{{emend.\ }}}
\def\expl{\textrm{{explic.\ }}}   
\def\explicat{\textrm{{explic.\ }}}
\def\fol{\textrm{{fol.\ }}}         
\def\foll{\textrm{{foll.\ }}}
\def\gloss{\textrm{{glossa ad }}}
\def\ins{\textrm{{ins.\ }}}          \def\inseruit{\textrm{{ins.\ }}}
\def\im{{\kern-.7pt\lower-1ex\hbox{\textrm{\tiny{\emph{i.m.}}}\kern0pt}}}
\def\inmargine{{\kern-.7pt\lower-.7ex\hbox{\textrm{\tiny{\emph{i.m.}}}\kern0pt}}}
\def\intextu{{\kern-.7pt\lower-.95ex\hbox{\textrm{\tiny{\emph{i.t.}}}\kern0pt}}}%\textrm{\scriptsize{i.t.\ }}}               
\def\indist{\textrm{{indis.\ }}}          \def\indis{\textrm{{indis.\ }}}
\def\iteravit{\textrm{{iter.\ }}}          \def\iter{\textrm{{iter.\ }}}  
\def\lectio{\textrm{{lect.\ }}}             \def\lec{\textrm{{lect.\ }}}
\def\leginequit{\textrm{{l.n. }}}         \def\legn{\textrm{{l.n. }}}         \def\illeg{\textrm{{l.n. }}}
\def\om{\textrm{{om. }}}
\def\primman{\textrm{{pr.m.}}}
\def\prob{\textrm{{prob.}}}
\def\rep{\textrm{{repetitio }}}
% \def\secundamanu{\textrm{\scriptsize{s.m.}}}
% \def\secm{{\kern-.6pt\lower-.91ex\hbox{\textrm{\tiny{\emph{s.m.}}}\kern0pt}}}%   \textrm{\scriptsize{s.m.}}}
\def\sequentia{\textrm{{seq.\,inv.\ }}}         \def\seqinv{\textrm{{seq.\,inv.\ }}} \def\order{\textrm{{seq.\,inv.\ }}}
\def\supralineam{{\kern-.7pt\lower-.91ex\hbox{\textrm{\tiny{\emph{s.l.}}}\kern0pt}}} %\textrm{\scriptsize{s.l.}}}
\def\interlineam{{\kern-.7pt\lower-.91ex\hbox{\textrm{\tiny{\emph{s.l.}}}\kern0pt}}}   %\textrm{\scriptsize{s.l.}}}
\def\vl{\textrm{v.l.}}   \def\varlec{\textrm{v.l.}} \def\varialectio{\textrm{v.l.}}
\def\vide{\textrm{{cf.\ }}}           \def\cf{\textrm{{cf.\ }}}
\def\videtur{\textrm{{vid.\,ut}}}
\def\crux{\textup{[\ldots]} }
\def\cruxx{\textup{[\ldots]}}
\def\unm{\textit{unm.}}        % unmetrical
%%%%%%%%%%%%%%%%%%%%%%%%%%%%%%%%%%%%



%%% Local Variables:
%%% mode: latex
%%% TeX-master: t
%%% End:

% addition 2023-12-11 MD:
\TeXtoTEIPat{\begin {metre}[#1]}{<note type="metre" target="##1">}
\TeXtoTEIPat{\end {metre}}{</note>}
\TeXtoTEIPat{\texttheta}{θ}

% change 2023-12-05 mm
\TeXtoTEI{teimute}{} 

% changes/additions 2023-11-27 MM:
\TeXtoTEIPat{\medialink {#1}{#2}}{<ref target="resources/#2">#1</ref>}

% changes/additions 2023-10-25 MM:
% new Sigla
\TeXtoTEIPat{\textAlpha}{Α}
\TeXtoTEIPat{\textalpha}{α}
\TeXtoTEIPat{\textBeta}{Β}
\TeXtoTEIPat{\textbeta}{β}
\TeXtoTEIPat{\textGamma}{Γ}
\TeXtoTEIPat{\textgamma}{γ}
\TeXtoTEIPat{\textDelta}{Δ}
\TeXtoTEIPat{\textdelta}{δ}
\TeXtoTEIPat{\textEpsilon}{Ε}
\TeXtoTEIPat{\textepsilon}{ε}
\TeXtoTEIPat{\textEta}{Η}
\TeXtoTEIPat{\texteta}{η}
\TeXtoTEIPat{\textChi}{Χ}
\TeXtoTEIPat{\textchi}{χ}
\TeXtoTEIPat{\textOmega}{Ω}
\TeXtoTEIPat{\textomega}{ω}

%new environments
\TeXtoTEIPat{\begin {postmula}[#1]}{<note type="postmula" target="##1">}
  \TeXtoTEIPat{\end {postmula}}{</note>}
\TeXtoTEIPat{\begin {altava}[#1]}{<div type="altrec"><note type="avataranika" target="##1">} %%% changed 2023-12-05 mm
  \TeXtoTEIPat{\end {altava}}{</note></div>} %%% changed 2023-12-05 mm
\TeXtoTEIPat{\sgwit {#1}}{<note type="inlineref"><ref>#1</ref></note>}

% changes/additions 2023-10-12 MM:
\TeXtoTEIPat{\\.}{}

% changes/additions 2023-08-15 MD:
\TeXtoTEIPat{\lineom {#1}{#2}}{<note type="omission">#1 omitted in <ref>#2</ref></note>}
\TeXtoTEI{graus}{hi}[rend="grey"]
\TeXtoTEIPat{\startgray}{} %%% changed 2023-12-05 mm
\TeXtoTEIPat{\endgray}{} %%% changed 2023-12-05 mm



% additions/changes 2023-06-05 mm:
%\TeXtoTEIPat{\lineom {#1}}{<note type="omission">Line omitted in <ref>#1</ref></note>}
\TeXtoTEIPat{\NotIn {#1}}{<note type="omission">Stanza omitted in <ref>#1</ref></note>}

% additions 2023-04-16 MD:
\TeXtoTEIPat{\,}{ }

% additions 2023-04-13 mm:
\TeXtoTEIPat{\begin {versinnote}}{<lg>}
  \TeXtoTEIPat{\end {versinnote}}{</lg>}

% additions 2023-04-05 MD:
\TeXtoTEIPat{\begin {testimonia}[#1]}{<note type="testimonia" target="##1">}
  \TeXtoTEIPat{\end {testimonia}}{</note>}
\TeXtoTEI{devnote}{s}[xml:lang="sa-deva"]

% app in philcomm und testimonia %%% added MM 2023-12-02
\TeXtoTEI{var}{note}[type="appinnote"]


\TeXtoTEI{anm}{note}[type="memo"] %% change 2023-04-16 MD
\TeXtoTEI{Anm}{note}[type="memo"] %% change 2023-12-05 MM
\TeXtoTEIPat{\startverse}{} %%% marked for change 2023-04-13 mm
\TeXtoTEIPat{\endverse}{} %%% marked for change 2023-04-13 mm
\TeXtoTEIPat{\newpage}{}
\TeXtoTEIPat{\marma}{}
\TeXtoTEIPat{\marmas}{}
\TeXtoTEIPat{\vin}{} % added by MD 2023-11-14

%%% modify environments and commands
%%% TEI mapping
% additions/changes 2022-06-07 mm:
\TeXtoTEI{grau}{hi}[rend="grey"]
\TeXtoTEIPat{ \& }{ &amp; }

% additions/changes 2022-06-01 mm:
\TeXtoTEI{skp}{seg}[type="deva-ignore"]
\TeXtoTEI{skm}{seg}[type="ltn-ignore"]

\TeXtoTEIPat{\rlap {#1}}{#1}

% additions/changes 2022-04-06 mm:
%\TeXtoTEI{sgwit}{ref}
\TeXtoTEI{textdev}{s}[xml:lang="sa-deva"]
\TeXtoTEIPat{\begin {col}[#1]}{<div type="colophon" xml:id="#1"><p>}
  \TeXtoTEIPat{\end {col}}{</p></div>}
\TeXtoTEIPat{\begin {ava}[#1]}{<note type="avataranika" target="##1">}
  \TeXtoTEIPat{\end {ava}}{</note>}
												   
\TeXtoTEIPat{\outdent}{}
\TeXtoTEIPat{\startaltrecension}{} %%% changed 2023-12-05 mm
\TeXtoTEIPat{\endaltrecension}{} %%% changed 2023-12-05 mm
\TeXtoTEIPat{\startaltnormal}{} % added by MD 2023-11-14 %%% changed 2023-12-05 mm
\TeXtoTEIPat{\endaltnormal}{} % added by MD 2023-11-14 %%% changed 2023-12-05 mm
\TeXtoTEIPat{\begin {alttlg}[#1]}{<div type="altrec"><lg xml:id="#1">}
  \TeXtoTEIPat{\end {alttlg}}{</lg></div>}



% additions/changes 2022-03-12 mm:
\TeXtoTEIPat{\begin {tlg}[#1]}{<lg xml:id="#1">}
  \TeXtoTEIPat{\end {tlg}}{</lg>}

\TeXtoTEIPat{\begin {translation}[#1]}{<note type="translation" target="##1">}
  \TeXtoTEIPat{\end {translation}}{</note>}
\TeXtoTEIPat{\begin {philcomm}[#1]}{<note type="philcomm" target="##1">}
  \TeXtoTEIPat{\end {philcomm}}{</note>}
\TeXtoTEIPat{\begin {sources}[#1]}{<note type="sources" target="##1">}
  \TeXtoTEIPat{\end {sources}}{</note>}


\TeXtoTEIPat{\begin {marma}[#1]}{<note type="marma" target="##1">}
  \TeXtoTEIPat{\end {marma}}{</note>}

\TeXtoTEIPat{\begin {jyotsna}[#1]}{<note type="jyotsna" target="##1">}
  \TeXtoTEIPat{\end {jyotsna}}{</note>}

\EnvtoTEI{description}{list}
\EnvtoTEI{itemize}{list}
\TeXtoTEIPat{\item [#1]}{<label>#1</label>\item}

\TeXtoTEI{tl}{l}
\TeXtoTEI{myfn}{note}[type="myfn"]
\TeXtoTEIPat{\getsiglum {#1}}{<ref target="##1"/>}

\TeXtoTEI{SetLineation}{}
\TeXtoTEI{noindent}{}
\TeXtoTEI{subsection*}{}

\TeXtoTEI{rlap}{}

% end additions/changes
% \TeXtoTEIPat{\skp {#1}}{#1}
% \TeXtoTEIPat{\skm {#1}}{}

\TeXtoTEIPat{\begin {prose}}{<p>}
  \TeXtoTEIPat{\end {prose}}{</p>}

\TeXtoTEIPat{\begin {tlate}}{<p>}
  \TeXtoTEIPat{\end {tlate}}{</p>}

\TeXtoTEI{emph}{hi}
\TeXtoTEI{bigskip}{}
% \TeXtoTEI{/}{|}
\TeXtoTEI{tl}{l}
\TeXtoTEIPat{english}{}
%\TeXtoTEIPat{-}{ } %% change 2023-04-16 MD
%\TeXtoTEIPat{°}{} %% change 2023-04-16 MD
\TeXtoTEIPat{\textcolor {#1}{#2}}{<hi rend="#1">#2</hi>}

% \TeXtoTEIPat{\eyeskip}{}
% \TeXtoTEIPat{\aberratio}{}
% \TeXtoTEIPat{\ad}{}
\TeXtoTEIPat{\add}{<hi rend="italic">add.</hi>} %% change 2023-04-16 MD
% \TeXtoTEIPat{\ann}{}
\TeXtoTEIPat{\ante}{<hi rend="italic">ante</hi> } %% change 2023-04-16 MD
\TeXtoTEIPat{\post}{<hi rend="italic">post</hi> } %% change 2023-04-16 MD
% \TeXtoTEIPat{\codd}{}
% \TeXtoTEIPat{\conj }{}
% \TeXtoTEIPat{\contin}{}
% \TeXtoTEIPat{\corr}{}
% \TeXtoTEIPat{\del}{}
% \TeXtoTEIPat{\dub}{}
% \TeXtoTEIPat{\emend }{}
% \TeXtoTEIPat{\expl}{}
% \TeXtoTEIPat{\ȩxplicat}{}
% \TeXtoTEIPat{\fol}{}
% \TeXtoTEIPat{\gloss}{}
% \TeXtoTEIPat{\ins}{}
% \TeXtoTEIPat{\im}{}
% \TeXtoTEIPat{\inmargine}{}
% \TeXtoTEIPat{\intextu}{}
% \TeXtoTEIPat{\indist}{}
% \TeXtoTEIPat{\iteravit}{}
% \TeXtoTEIPat{\lectio}{}
% \TeXtoTEIPat{\leginequit}{}
% \TeXtoTEIPat{\legn}{}
% \TeXtoTEIPat{\illeg}{<hi rend="italic">illeg.</hi>}
\TeXtoTEIPat{\illeg}{<gap reason="illeg."/>} %%% change 2023-04-11 mm
% \TeXtoTEIPat{\om}{<hi rend="italic">om.</hi>}
\TeXtoTEIPat{\om}{<gap reason="om."/>} %%% change 2023-04-11 mm
% \TeXtoTEIPat{\primman}{}
% \TeXtoTEIPat{\prob}{}
% \TeXtoTEIPat{\rep}{}
% \TeXtoTEIPat{\sequentia}{}
% \TeXtoTEIPat{\supralineam}{}
% \TeXtoTEIPat{\interlineam}{}
\TeXtoTEIPat{\vl}{<hi rend="italic">v.l.</hi>}
% \TeXtoTEIPat{\vide}{}
% \TeXtoTEIPat{\videtur}{}
% \TeXtoTEIPat{\crux}{}
% \TeXtoTEIPat{\cruxxx}{}
\TeXtoTEIPat{\unm}{<hi rend="italic">unm.</hi>}


% List of Scholars
\DeclareScholar{nos}{nos}[
forename=HPP,
surname=Team]


% Nullify \selectlanguage in TEI as it has been used in
% \DeclareWitness but should be ignored in TEI.
\TeXtoTEI{selectlanguage}{}



\NewDocumentCommand{\skp}{m}{}
\NewDocumentCommand{\skm}{m}{\unless\ifinapparatus#1-\fi}

\SetTEIxmlExport{autopar=false}
\NewDocumentEnvironment{tlg}{O{}}{
	\begin{ekdverse}
	\indentpattern{0000}}{
	\end{ekdverse}
	\vskip 0.75\baselineskip}
\NewDocumentEnvironment{alttlg}{O{}}{}{}
\NewDocumentCommand{\tl}{m}{#1}

%%%%%%

\def\startaltrecension#1{
  \stopvline
  \begin{ekdverse}[type=altrecension]
    \indentpattern{0000} 
    \begin{patverse*}
      \color{gray}
      \setvnum{#1}}
\def\endaltrecension{
  \end{patverse*}
  \end{ekdverse}
  \vskip 0.75\baselineskip
  \startvline}

%%%%%%

\newcommand{\myfn}[1]{\footnote{\texteng{#1}}}
\renewcommand{\thefootnote}{\texteng{\arabic{footnote}}}
\newcommand{\devnote}[1]{\selectlanguage{iast}{\scriptsize #1}\selectlanguage{english}}
\newcommand{\outdent}{\hspace{-\vgap}}
\newcommand{\sgwit}[1]{{\small (\getsiglum{#1})}\selectlanguage{iast}}
\newcommand{\NotIn}[1]{\texteng{\footnotesize (om. \getsiglum{#1})}\selectlanguage{iast}}

\def\om{\emph{om.}} % \!}
\def\illeg{\emph{illeg.}} %\!}
\def\unm{\emph{unm.\:}}
\def\recte{\texteng{r.\:}}
\def\for{\texteng{for }}
\def\sic{\emph{sic}}

\makepagestyle{HPed}
\makeoddhead{HPed}{\small\texteng{HP1 OldMss}}{}{\small\texteng{\today}}
\makeevenhead{HPed}{\small\texteng{HP1 OldMss}}{}{\small\texteng{\today}}
\makeoddfoot{HPed}{}{\small\texteng{\thepage}}{}
\makeevenfoot{HPed}{}{\small\texteng{\thepage}}{}

\begin{document}
\pagestyle{HPed} %
\begin{otherlanguage}{iast}
\begin{ekdosis}


%1.1
\begin{tlg}[hp01_001]
\tl{
\pada{\app{\lem[wit={J8,J10,J17,N17,V19,W4}]{śrīādināthāya}
	\rdg[wit={V1}]{ādīśanāthāya}} namo'stu tasmai}
\pada{yenopadiṣṭā haṭhayogavidyā/}\\+}
\tl{
\pada{virājate pronnata%
	\app{\lem[wit={V1,V19},alt={rājasaudham}]{rājasaudha}
	\rdg[wit={J8,J10,J17,N17,W4}]{rājayogam}}}%  rājatogam? W4
\pada{\app{\lem[wit={J8,J10,N17,V1,V19,W4},alt={āroḍhum}]{māroḍhu}
	\rdg[wit={J17}]{ārohim}}micchoradhi%
	\app{\lem[wit={J10,J17,N17,V1,V19,W4}]{rohiṇīva}
	\rdg[wit={J8}]{roha eva}}//}\\!}
\end{tlg}

%1.2
\begin{tlg}[hp01_002]
\tl{
\pada{praṇamya śrīguruṃ nāthaṃ}
\pada{svātmārāmeṇa
	\app{\lem[wit={J8,J10,J17,N17,V1,W4}]{yoginā}
	\rdg[wit={V19}]{dhīmatā}}/}\\+}
\tl{
\pada{kevalaṃ rājayogāya}
\pada{haṭha\app{\lem[wit={J8pc,V1,V19}]{vidyopadiśyate}
	\rdg[wit={J8ac}]{vidyo diśyate}
	\rdg[wit={J10,N17,W4}]{yogopadiśyate}
	\rdg[wit={J17}]{yogopradiśyate}}//}\\!}
\end{tlg}

%1.3
\begin{tlg}[hp01_003]
\tl{
\pada{\app{\lem[wit={J8pc,J10,J17,N17,V1,W4}]{bhrāntyā}
	\rdg[wit={J8ac,V19}]{bhrāntvā}}
	bahu\app{\lem[wit={J8,J10,N17,V19,W4}]{matadhvānte}
	\rdg[wit={J17}]{matādhvānte}
	\rdg[wit={V1}]{mataṃ bhrāntaṃ}}}
\pada{rājayoga\app{\lem[wit={J8,J10,J17,N17,V1,W4},alt={ajānatām}]{majānatām}
	\rdg[wit={V19}]{ajānataḥ}}/}\\+}
\tl{
\pada{haṭhapradīpikāṃ dhatte}
\pada{svātmārāmaḥ % J8 om. Visarga
	\app{\lem[wit={V1}]{kṛpākaraḥ}
	\rdg[wit={J10pc,V19}]{kṛpāparaḥ}
	\rdg[wit={J8pc,J17,N17,W4}]{kṣamākaraḥ}
	\rdg[wit={J8ac,J10ac}]{prakāśyate}}//}\\!}
\end{tlg}

%1.4
\begin{tlg}[hp01_004]
\tl{
\pada{haṭha\app{\lem[wit={J8,J10,J17,V19,W4}]{vidyāṃ hi}
	\rdg[wit={N17,V1}]{vidyā hi}}
	matsyendra}\pada{gorakṣādyā vijānate/}\\+}
\tl{
\pada{\app{\lem[wit={J8,J10,J17,N17,V1,W4}]{svātmārāmo}
	\rdg[wit={V19}]{ātmārāmo}}}%
\pada{'thavā yogī
	\app{\lem[wit={J8pc,J10,N17,V1,V19,W4}]{jānīte}
	\rdg[wit={J17}]{jānite}
	\rdg[wit={J8ac}]{jānaṃte}} tatprasādataḥ//}\\!}
\end{tlg}


%1.5
\begin{tlg}[hp01_005]
\tl{
\pada{\app{\lem[wit={J8,V19}]{śrīādinātha}
	\rdg[wit={J10,J17,N17,V1,W4}]{ādināthādi}}%
	matsyendra}%
\pada{\app{\lem[wit={J10,J17,N17,W4}]{śābarā}
	\rdg[wit={V1}]{śabarā}
	\rdg[wit={J8}]{śāgarā}
	\rdg[wit={V19}]{śāradā}}nanda%
	\app{\lem[wit={J8,J10,J17,V1,V19,W4}]{bhairavāḥ}
	\rdg[wit={J17,N17}]{bhairavaḥ}}/}\\+}
\tl{
\pada{cauraṅgī%
	\app{\lem[wit={J8,J10,J17,N17,V1,W4}]{mīna}
	\rdg[wit={V19}]{ṣīna}}gorakṣa}%
\pada{\app{\lem[wit={J8,J10,J17,N17,V1,W4}]{virūpākṣa} % virupākṣa J17
	\rdg[wit={V19}]{virūpākṣaḥ}}%
	\app{\lem[wit={J8,J10,J17,N17,V1,W4}]{bileśayāḥ}
	\rdg[wit={V19}]{savālikaḥ}}//}\\!}
\end{tlg}

%1.6
\begin{tlg}[hp01_006]
\tl{
\pada{manthānabhairavo yogī}
\pada{\app{\lem[wit={J8,V19}]{siddha}
	\rdg[wit={J10,J17,N17,W4}]{śuddha}
	\rdg[wit={V1}]{suddha}}%
	\app{\lem[wit={J10,J17,N17,V1,V19,W4}]{buddhaś ca}
	\rdg[wit={J8}]{budhiśca}}
	\app{\lem[wit={V19}]{kanthaḍiḥ}
	\rdg[wit={V3,W4}]{kaṃthaḍī}
	\rdg[wit={J10pc,V1}]{kaṃthalī}
	\rdg[wit={J8}]{kaṃtharī}
	\rdg[wit={J10ac,J17,N17}]{kaṃdalī}}/}\\+}
\tl{
\pada{\app{\lem[wit={J8,J17,N17,V1,W4}]{pauraṇṭakaḥ}
	\rdg[wit={J10}]{pauraṇṭhakaḥ}}
	\app{\lem[wit={J10,J17,N17,V1,W4}]{surānandaḥ}
	\rdg[wit={J8}]{surānanda}}}
\pada{siddhapādaśca
	\app{\lem[wit={J8pc,J10,N17,W4},alt=carp(p)aṭiḥ]{carpaṭiḥ}
	\rdg[wit={J8ac}]{carppaṭi}
	\rdg[wit={J17,V1}]{carp(p)aṭī}}//}%
\myfn{6cd \emph{pauraṇṭakaḥ ... carpaṭiḥ} is omitted in \getsiglum{V19}.}\\!}
\end{tlg}

%1.7
\begin{tlg}[hp01_007]
\tl{
\pada{\app{\lem[wit={J8,J10,J17,W4}]{kānerī}
	\rdg[wit={N17}]{kaneri}
	\rdg[wit={V1}]{kaṇerī}
	\rdg[wit={V19}]{kariṇī}}
	\app{\lem[wit={J8,N17,V19}]{pūjya}
	\rdg[wit={V1}]{pūrya}
	\rdg[wit={J10,J17,W4}]{pūrva}}pādaśca}
\pada{\app{\lem[wit={J8,V1}]{nityanātho}
	\rdg[wit={J10,J17,N17,W4}]{dhvaninātho}
	\rdg[wit={V19}]{bilvanātho}}
	nirañjanaḥ/}\\+}  % °naṃ J8ac
\tl{
\pada{kapālī bindunāthaś ca}
\pada{kākacaṇḍīśvarā\app{\lem[wit={V1},alt={āhvayaḥ}]{hvayaḥ}
	\rdg[wit={J8,J10,J17,N17,V19,W4}]{ādayaḥ}}//}\\!}
\end{tlg}

\pagebreak

%1.8
\begin{tlg}[hp01_008]
\tl{
\pada{\app{\lem[resp=emend]{allāma}
	\rdg[wit={J8,J10,J17,V1,W4}]{allamaḥ}
	\rdg[wit={N17}]{akṣamaḥ}
	\rdg[wit={V19}]{sukṣamaḥ}}prabhudevaśca}
\pada{\app{\lem[wit={J8}]{ghoḍācolī}
	\rdg[wit={J10,J17,N17,V1,W4}]{ghorācolī}
	\rdg[wit={V19}]{ghoṭācolī}}
	\app{\lem[wit={J8,J10,N17,V1,W4}]{ca}
	\rdg[wit={V19}]{sa}}
	\app{\lem[wit={J8pc,J10,J17,V1}]{ṭiṇṭiṇī}
	\rdg[wit={N17,W4}]{ṭiṃṭiṇīḥ}
	\rdg[wit={V19}]{ṭiṃṭiniḥ}
	\rdg[wit={J8ac}]{ciṃciṇī}}/}\\+}
\tl{
\pada{\app{\lem[wit={J10,J17,N17,V1,W4}]{bhālukī}
	\rdg[wit={V19}]{vālukir}
	\rdg[wit={J8}]{vāsukir}}
	\app{\lem[wit={J8,J10,J17,N17,V1},alt={nāgabodhaś}]{nāgabodha}
	\rdg[wit={V19}]{nāgarodhaś}
	\rdg[wit={W4}]{nāma bodhaś}}śca}
\pada{\app{\lem[wit={J8,J10ac,J17,W4}]{khaṇḍa}
	\rdg[wit={J10pc}]{khaṃṭha}
	\rdg[wit={N17,V1}]{khaṃḍaṃ}
	\rdg[wit={V19}]{caṃḍī}}%
	\app{\lem[wit={J8,J10,N17,V1,V19,W4},alt={kāpālikas}]{kāpālika}
	\rdg[wit={J17}]{kapālikas}}stathā//}\\!}
\end{tlg}

%1.9
\begin{tlg}[hp01_009]
\tl{
\pada{ityādayo mahāsiddhā}
\pada{haṭhayoga\app{\lem[wit={J10,J17,N17,V1,V19,W4}]{prabhāvataḥ}
	\rdg[wit={J8}]{prasādataḥ}}/}\\+}
\tl{
\pada{\app{\lem[wit={J8,J10,J17,N17,V1,W4}]{khaṇḍayitvā}
	\rdg[wit={V19}]{khaṃḍa-i-tvā}} kāladaṇḍaṃ}
\pada{brahmāṇḍe
	\app{\lem[wit={J10,J17,N17,V1,W4}]{vicaranti}
	\rdg[wit={J8}]{tu caranti}
	\rdg[wit={V19}]{ṣu caranti}} te//}\\!}
\end{tlg}

%1.10
\begin{tlg}[hp01_010]
\tl{
\pada{\app{\lem[wit={V1,V19}]{saṃsāratāpa}
	\rdg[wit={J10,J17,N17,W4}]{saṃsāraśrama}
	\rdg[wit={J8}]{saṃsārāśrama}}taptānāṃ}
\pada{\app{\lem[wit={V1}]{samāśrayo}
	\rdg[wit={V19}]{samāśraya}
	\rdg[wit={J8,J10,J17,N17,W4}]{āśrayo yaṃ}}
	\app{\lem[wit={J8,J10,J17,N17}]{haṭho mataḥ}
	\rdg[wit={V1,W4}]{haṭho maṭhaḥ}
	\rdg[wit={V19}]{mato haṭhaḥ}}/}\\+}
\tl{
\pada{\app{\lem[wit={J8,J10,J17,V1,V19,W4}]{aśeṣa}
	\rdg[wit={N17}]{aśeṣo}}yoga%
	\app{\lem[wit={J8pc,J10,N17,V19,W4},alt={jagatām}]{jagatā}
	\rdg[wit={V1}]{jagatīm}
	\rdg[wit={J8ac}]{yuktānām}
	\rdg[wit={J17}]{juktānāṃm}}}%
\pada{\app{\lem[wit={V1,W4},alt={ādhāra}]{mādhāra}
	\rdg[wit={J8,J10,J17,N17,V19}]{ādhāraḥ}}%
	\app{\lem[wit={J8,J10,J17,V19,W4}]{kamaṭho haṭhaḥ}
	\rdg[wit={N17},post={\unm}]{kramahaṭhaḥ}
	\rdg[wit={V1}]{kahaṭho maṭhaḥ}}//}\\!}
\end{tlg}

%1.11
\begin{tlg}[hp01_011]
\tl{
\pada{haṭha\app{\lem[wit={J8,J10,N17,V1,V19,W4}]{vidyā}
	\rdg[wit={J17}]{vidyāṃ}} paraṃ
	\app{\lem[wit={J8pc,J10,J17,N17,V1,V19,W4}]{gopyā}
	\rdg[wit={J8ac}]{gopyaṃ}}
	\app{\lem[wit={J8pc,J10,J17,N17,V1,W4}]{yogināṃ}
	\rdg[wit={J8ac}]{yogīnāṃ}
	\rdg[wit={V19}]{yoginā}}}
\pada{siddhi\app{\lem[wit={J8,J10,J17,N17,V1,W4},alt={icchatām}]{micchatām}
	\rdg[wit={V19}]{icchitā}}/}\\+}
\tl{
\pada{bhavedvīryavatī guptā}
\pada{\app{\lem[wit={J8,J17,V1,W4}]{nirvīryā}
	\rdg[wit={V19}]{nirvījā}
	\rdg[wit={J10,N17}]{nirvāryā}}
	\app{\lem[wit={J8,J10,N17,V1,W4}]{tu}
	\rdg[wit={J17}]{nu}
	\rdg[wit={V19}]{ca}}
prakāśitā//}\\!}
\end{tlg}

%1.12
\begin{tlg}[hp01_012]
\tl{
\pada{surājye dhārmike deśe} \pada{subhikṣe nirupadrave/}\\+}
\tl{
\pada{\app{\lem[wit={J8,J10,J17,N17,V1,W4}]{ekānte}
	\rdg[wit={V19}]{ekāṃta}} maṭhikāmadhye}
\pada{sthātavyaṃ
	haṭha\app{\lem[wit={J8,J10,N17,V1,V19,W4}]{yoginā}
	\rdg[wit={J17}]{yogināṃ}}//}\\!}
\end{tlg}

%1.13
\begin{tlg}[hp01_013]
\tl{
\pada{alpadvāramarandhragarta%
	\app{\lem[wit={N17,V19}]{piṭharaṃ}
	\rdg[wit={J17}]{piṭhakaṃ}
	\rdg[wit={J10pc}]{viṭharaṃ} %
	\rdg[wit={J8,J10ac,W4}]{viṭapaṃ}
	\rdg[wit={V1}]{sahitaṃ}}
	\app{\lem[wit={J10,N17,V1,V19,W4}]{nātyuccanīcā}
	\rdg[wit={J17}]{nātyuccānīcā}
	\rdg[wit={J8}]{nātyuccanoccā}}%
	\app{\lem[wit={N17,V1}]{yutaṃ} % BKhP
	\rdg[wit={J8,V19}]{yataṃ}
	\rdg[wit={J10,W4}]{yitaṃ}
	\rdg[wit={J17}]{pitaṃ}}}\\+}
\tl{
\pada{samyaggomaya 			% samyagomaya J8
	\app{\lem[wit={V1,V19}]{sāndra}
	\rdg[wit={J8,J10,N17,W4}]{sārdra}
	\rdg[wit={J17}]{sāṃrdra}}%
	\app{\lem[wit={J10,N17,V19,W4}]{liptam amalaṃ}
	\rdg[wit={J17}]{līptam amalaṃ}
	\rdg[wit={J8}]{liptavimalaṃ}
	\rdg[wit={V1}]{liptam abilaṃ}} % YCM(P), 10chp
	\app{\lem[wit={J8,V1,V19}]{niḥśeṣa}
	\rdg[wit={J10,J17,N17,W4}]{nirdoṣa}}%
	\app{\lem[wit={J8,J10,J17,N17,V1,W4}]{bādhojjhitam}
	\rdg[wit={V19}]{jantūjjhitam}}/}\\+}
\tl{
\pada{bāhye
maṇḍapavedikūpa\app{\lem[wit={J10,J17,N17,V1,V19,W4}]{racitaṃ}
	\rdg[wit={J8}]{ruciraṃ}}
prākārasaṃveṣṭitaṃ}\\+}
\tl{
\pada{proktaṃ
	\app{\lem[wit={J8,J10,J17,V1,V19,W4}]{yogamaṭhasya}
	\rdg[wit={N17}]{yogahaṭhasya}} lakṣaṇamidaṃ siddhairhaṭhābhyāsibhiḥ//}\\!}
\end{tlg}

\pagebreak

%1.14
\begin{tlg}[hp01_014]
\tl{
\pada{\app{\lem[wit={J8,J10,J17,V1,V19,W4}]{evaṃvidhe}
	\rdg[wit={N17}]{evaṃvidha}} maṭhe sthitvā}
\pada{sarvacintāvivarjitaḥ/}\\+}
\tl{
\pada{\app{\lem[wit={J10,J17,N17,V1,V19,W4}]{gurūpa}
	\rdg[wit={J8}]{gurupa}}diṣṭamārgeṇa}
\pada{\app{\lem[wit={J10,J17,N17,V1,W4}]{yogam evaṃ}%
	\rdg[wit={V19}]{yogam eva}
	\rdg[wit={J8}]{yogamārgaṃ}}
	\app{\lem[wit={J8,J10,J17,N17,V1,W4}]{samabhyaset}
	\rdg[wit={V19}]{sadābhyaset}}//}\\!}
\end{tlg}

%1.15
\begin{tlg}[hp01_015]
\tl{
\pada{\app{\lem[wit={J8pc,J10,J17,N17,V1,W4}]{atyāhāraḥ}
	\rdg[wit={V19}]{alpāhāra}
	\rdg[wit={J8ac}]{ātmāhāraḥ}}
	\app{\lem[wit={J8,J10,J17,N17,V1,W4}, alt={prayāsaś ca}]{prayāsaśca}
	\rdg[wit={V19}]{prayāsaś cā}}}%
\pada{prajalpo%
	\app{\lem[wit={J8ac,J10ac,J17,N17,V1,V19},postwit={(\getsiglum{N17} with an explicit Avagraha)}]{'niyama}
	\rdg[wit={J10pc,W4}]{niyamā}
	% J10 corrects to niyamo, then deletes the inserted "o"
	\rdg[wit={J8pc}]{'niyamā}}grahaḥ/}\\+}
\tl{
\pada{janasaṅgaśca laulyaṃ ca}
\pada{ṣaḍbhi\app{\lem[wit={V1},alt={yogaḥ praṇaśyati}]{ryogaḥ praṇaśyati}
	\rdg[wit={J10ac,J17,N17}]{yogaś ca naśyati}
	\rdg[wit={J10pc}]{yogaḥ vinaśyati}
	\rdg[wit={J8,V19,W4}]{yogo vinaśyati}}// % vinasyati W4
	}\\!}
\end{tlg}
	

%1.16
\begin{tlg}[hp01_016]
\tl{
\pada{\app{\lem[wit={J8pc,J10,J17,V1,W4},alt={utsāhān}]{utsāhā}
	\rdg[wit={V19}]{utsāhāt}
	\rdg[wit={N17}]{utsāhā}
	\rdg[wit={J8ac}]{utsāha}}%
	\app{\lem[wit={J8pc,J17,W4},alt={niścayād dhairyāt}]{nniścayāddhairyā}
	\rdg[wit={J8ac}]{niścayād dhairyā}
	\rdg[wit={J10,V1}]{niścayād vairyāt}
	\rdg[wit={N17,V19}]{sāhasād dhairyāt}}}%
\pada{ttattva\app{\lem[wit={J8pc,J10,J17,N17,V1,W4}]{jñānāc ca darśanāt}
	\rdg[wit={J8ac}]{jñānā ca darśanāt}
	\rdg[wit={V19}]{jñānād viniścayāt}}/}\\+}
\tl{
\pada{janasaṅgaparityāgāt}
\pada{ṣaḍbhi\app{\lem[wit={J10,J17,V1,W4}, alt={yogas tu sidhyati}]{ryogastu sidhyati}
	\rdg[wit={N17}]{yogaś ca sidhyati}
	\rdg[wit={J8ac}]{yoga prasidhyati}
	\rdg[wit={J8pc}]{yogaḥ prasidhyati}
	\rdg[wit={V19}]{yogo prasidhyati}}//}\\!}
\end{tlg}
%\myfn{In \getsiglum{J8} the following verses are inserted after I.16:\\
%\textdev{ahiṃsā satyamasteyaṃ brahmacaryaṃ kṣamā dhṛtiḥ/
%dayārjavamitāhāraḥ śaucaṃ caiva yamā daśa//}\\
%\textdev{tapaḥ saṃtoṣamāstikyaṃ dānamīśvarapūjanam/
%siddhāntaṃ śravaṇaṃ caiva hrī matiśca japo hutam//}}\\!}


\startaltrecension{I.16+}
\begin{alttlg}[hp01_016_1]
\tl{
\pada{ahiṃsā satyamasteyaṃ}
\pada{brahmacaryaṃ kṣamā dhṛtiḥ/}\\+
}\tl{
\pada{dayārjavamitāhāraḥ}
\pada{śaucaṃ caiva yamā daśa//} \sgwit{J8}\\!
}\end{alttlg}

\begin{alttlg}[hp01_016_2]
\tl{
\pada{tapaḥ saṃtoṣamāstikyaṃ}
\pada{dānamīśvarapūjanam/}\\+
}\tl{
\pada{siddhāntaṃ śravaṇaṃ caiva}
\pada{hrī matiśca japo hutam//} \sgwit{J8}\\!
}\end{alttlg}
\endaltrecension

\startaltrecension{ }
\outdent\app{\lem[resp=emend]{atha}\rdg[wit={J17}]{athya}} āsanāni/ \sgwit{J17}
\endaltrecension


%1.17
\begin{tlg}[hp01_017]
\tl{
\pada{haṭhasya prathamāṅgatvād}\pada{āsanaṃ pūrvamucyate/}\\+}
\tl{
\pada{\app{\lem[wit={J8,V1,V19},alt={tat kuryād}]{tatkuryā}
	\rdg[wit={J10,J17,N17,W4}]{kuryāt tad}}%
	\app{\lem[wit={V1},alt={āsanaṃ sthairyam}]{dāsanaṃ sthairya}
	\rdg[wit={J8,V19}]{āsanasthairyam}
	\rdg[wit={J10,J17,N17,W4}]{āsanaṃ tasmād}}}%
\pada{mārogyaṃ cāṅga\app{\lem[wit={J8,J10,J17,V1,W4}]{pāṭavam}
	\rdg[wit={N17,V19}]{lāghavaṃ}}//}%
\myfn{I.17--18 are transposed in \getsiglum{V19}.}%
\\!}
\end{tlg}

%1.18
\begin{tlg}[hp01_018]
\tl{
\pada{vasiṣṭhādyaiśca munibhir}\pada{matsyendrādyaiśca yogibhiḥ/}\\+}
\tl{
\pada{aṅgīkṛtānyāsanāni} % aṃgi N17
\pada{\app{\lem[wit={J8,J10,J17,N17,W4}]{kathyante}
	\rdg[wit={V1}]{likhyante}
	\rdg[wit={V19}]{vakṣyante}}
kānicinmayā//}\linelabel{v18d}\\!}
\end{tlg}

\pagebreak

%1.19
\begin{tlg}[hp01_019]
\tl{
\pada{\app{\lem[wit={J8pc,J10,J17,V1,W4},alt={jānūrvor}]{jānūrvo}
	\rdg[wit={J8ac,N17}]{jānurvor}
	\rdg[wit={V19}]{jaṃtūrvo}}rantare
samya}\pada{kkṛtvā pādatale ubhe/}\\+}
\tl{
\pada{\app{\lem[wit={J8pc,J10,J17,N17,V19,W4}]{ṛjukāyaḥ} % ṛjū J17
	\rdg[wit={V1}]{ṛjukāya}
	\rdg[wit={J8ac}]{ṛjuḥ kāya}}
	\app{\lem[wit={J8pc,J10,J17,N17,V1,V19,W4}]{samāsīnaḥ}
	\rdg[wit={J8ac}]{samāsīnaṃ}}}
\pada{svastikaṃ tatpracakṣate//}\\!}
\end{tlg}

%1.20
\begin{tlg}[hp01_020]
\tl{
\pada{\app{\lem[wit={J8,J10,N17,V1,V19,W4}]{savye}
	\rdg[wit={J17}]{sarvye}}
	\app{\lem[wit={J10,J17,N17,V1,V19,W4}]{dakṣiṇa}
	\rdg[wit={J8}]{dakṣaṇa}}gulphaṃ tu}
\pada{pṛṣṭhapārśve niyojayet/}\\+}
\tl{
\pada{dakṣiṇe\app{\lem[wit={J8,J10,J17,N17,V1,W4}]{'pi}
	\rdg[wit={V19}]{tu}} tathā
	\app{\lem[wit={J8,J10,N17,V19,W4}]{savyaṃ} % BKhP
	\rdg[wit={J17}]{sarvyaṃ}
	\rdg[wit={V1}]{savye}}}
\pada{gomukhaṃ gomukhaṃ
	\app{\lem[wit={J10,J17,V1,V19}]{yathā}
	\rdg[wit={J8,N17,W4}]{tathā}}//}\\!}
\end{tlg}


%1.21
% testimonia
% VS 1.72
% 1.72ab ekaṃ pādam athaikasmin vinyasyorau ca saṃsthitam |
% 1.72cd itarasmiṃs tathaivoruṃ vīrāsanam itīritam ||
% YY 3.8
% 3.8ab ekaṃ pādam athaikasmin vinyasyoruṇi saṃsthitam
% 3.8cd itarasmiṃs tathā coruṃ vīrāsanam udāhṛtam

\begin{tlg}[hp01_021]
\tl{
\pada{\app{\lem[wit={J8,J10,J17,N17,V1,W4}]{ekaṃ}
	\rdg[wit={V19}]{eka}}
	\app{\lem[wit={J8,J10,J17,V1,V19,W4},alt={pādam}]{pāda}
	\rdg[wit={N17}]{pāda}}\app{\lem[wit={V1},alt=athaikasmin]{mathaikasmi}
	\rdg[wit={J8,J10,J17,N17,W4}]{tathaikasmin}      %A1
	\rdg[wit={V19}]{yathaikasmi}}}%
\pada{\app{\lem[wit={J8,N17,V1,W4},alt={vinyasyoruṇi}]{nvinyasyoruṇi}
	\rdg[wit={J10}]{vinyasyoruṃni}
	\rdg[wit={J17}]{vinyasyorunni}
	\rdg[wit={V19}]{vinyased ūru}}
saṃsthitam/}\\+}
\tl{
\pada{\app{\lem[wit={J8,J10,J17,N17,V1,W4}]{itarasmiṃ}
	\rdg[wit={V19}]{itarāsmiṃ}}stathā
%N17, J17 itarasmintathā
	\app{\lem[wit={J10,J17,V1,V3}]{coruṃ}
	\rdg[wit={W4}]{corau}
	\rdg[wit={J8pc}]{cānyaṃ}
	\rdg[wit={V19}]{ce ..}
	\rdg[wit={N17}]{caikaṃ}}}
\pada{vīrāsanamitīritam//}\\!}
\end{tlg}


%1.22
% Testimonia
%VS 1.80
% 1.80ab gudaṃ nirudhya gulphābhyāṃ vyutkrameṇa samāhitaḥ |
% 1.80cd kūrmāsanaṃ bhaved etad iti yogavido viduḥ ||

\begin{tlg}[hp01_022]
\tl{
\pada{gudaṃ
	\app{\lem[wit={V1}]{nibadhya}
	\rdg[wit={J8}]{nibaddhi}
	\rdg[wit={V19}]{niyamya}
	\rdg[wit={J10,J17,W4}]{niṣpīḍya}
	\rdg[wit={N17}]{niṣ.īḍya}} gulphābhyāṃ}
\pada{vyutkrameṇa
	\app{\lem[wit={J10,J17,N17,V1,V19,W4}]{samāhitaḥ}
	\rdg[wit={J8}]{samāhitaṃ}}/}\\+}
\tl{
\pada{kūrmāsanaṃ bhaved etad}
\pada{iti yoga\app{\lem[wit={J8,J10,J17,N17,V1,V19}]{vido}
	\rdg[wit={W4}]{vide}} viduḥ//}\\!}
\end{tlg}


%1.23
% Testimonia
%VS 1.78
% 1.78ab padmāsanaṃ samāsthāya jānūrvor antare karau |
% 1.78cd bhūmau niveśya saṃsthāpya vyomasthaṃ kukkuṭāsanam ||

\begin{tlg}[hp01_023]
\tl{
\pada{padmāsanaṃ
	\app{\lem[wit={J10,J17,N17,V1,W4}]{tu}
	\rdg[wit={J8,V19}]{su}}
	\app{\lem[wit={J8,J10,J17,V1,W4}]{saṃsthāpya}
	\rdg[wit={N17}]{saṃsthāpyaḥ}
	\rdg[wit={V19}]{saṃyojya}}}
\pada{jānūrvorantare
	\app{\lem[wit={J8,J10,J17,V1,V19,W4}]{karau}
	\rdg[wit={N17}]{karaiḥ}}/}\\+}
\tl{
\pada{\app{\lem[wit={J8,J10,J17,V1,V19,W4}]{niveśya}
	\rdg[wit={N17}]{vinyasyat}} bhūmau saṃsthāpya}
\pada{\app{\lem[wit={J8,J10,N17,V1,V19,W4}]{vyomasthaḥ}
	\rdg[wit={J17}]{vyomastha}}
	\app{\lem[wit={J10ac,N17,V1,V3},alt={kurk(k)uṭā}]{kurkuṭā}
	\rdg[wit={J8,J10pc,J17,V19}]{kukkuṭā}
	\rdg[wit={W4}]{kukuṭā}}sanam//}\\!}
\end{tlg}


%1.24
% test
% Haṭharatnāvalī 3.74
%kukkuṭāsanabandhastho dorbhyāṃ sambadhya kandharām||
%śete kūrmavad uttānam etad uttānakūrmakam ||74||
%
\begin{tlg}[hp01_024]
\tl{
\pada{\app{\lem[wit={J10ac,N17,V1,V3},alt={kurk(k)uṭā}]{kurkuṭā}
	\rdg[wit={J8,J10pc,J17,V19}]{kukkuṭā}
	\rdg[wit={W4}]{kukuṭā}}%
	\app{\lem[wit={J8pc,J10,J17,N17,V1,V19,W4},alt={āsana}]{sana}
	\rdg[wit={J8ac}]{āsanaṃ}}%
	\app{\lem[wit={J8}]{bandhastho}
	\rdg[wit={V19}]{bandhasthe}
	\rdg[wit={J10,J17,N17,W4}]{madhyastho}
	\rdg[wit={V1}]{vat kṛtvā}}}
\pada{\app{\lem[wit={J8,J10pc,J17,N17,V1,V19}]{dorbhyāṃ}
	\rdg[wit={J10ac}]{ddorbhyāṃ}
	\rdg[wit={W4}]{dvābhyāṃ}}
	\app{\lem[wit={J8,J10,J17,V1,V19,W4}]{saṃbadhya}
	\rdg[wit={N17}]{saṃdhya}}
	\app{\lem[wit={J8pc,J10,J17,N17,V1,V19,W4}]{kandharām}
	\rdg[wit={J8ac}]{kandharam}}/}\\+}
\tl{
\pada{\app{\lem[wit={J8pc,J10,J17,N17,V1,W4}]{śete}
	\rdg[wit={J8ac}]{śene}
	\rdg[wit={V19}]{bhavet}}
kūrmava\app{\lem[wit={V1,V19},alt={uttānam}]{duttāna}
	\rdg[wit={J8,J10,J17,N17,W4}]{uttāna}}m}%
\pada{etaduttānakūrmakam//}\\!}
\end{tlg}


%1.25
% Testimonia
% Haṭharatnāvalī 3.51
% pādāṅguṣṭhau tu pāṇibhyāṃ gṛhītvā śravaṇāvadhi||
% dhanurākarṣaṇaṃ kṛtvā dhanurāsanam ucyate||3.51||

\begin{tlg}[hp01_025]
\tl{
\pada{pādāṅguṣṭhau
	\app{\lem[wit={J8,J10,J17,N17,V1,W4}]{tu}
	\rdg[wit={V19}]{ca}}
	\app{\lem[wit={J8,V19}]{pāṇibhyāṃ}
	\rdg[wit={J10,J17,N17,V1,W4}]{bāhubhyāṃ}}}
\pada{gṛhītvā % gṛhitvā J8
śravaṇā\app{\lem[wit={J8,J10,J17,N17,V19,W4}]{vadhi}
	\rdg[wit={V1}]{vadhiṃ}}/}\\+}
\tl{
\pada{dhanurā\app{\lem[wit={J10,J17,N17,W4}]{karṣaṇaṃ}
	\rdg[wit={V1}]{karṣaṇaḥ}
	\rdg[wit={V19}]{karkhaṇa}
	\rdg[wit={J8}]{karṣaṇāt}}
	\app{\lem[wit={J10,J17,N17,W4},alt={kuryād}]{kuryā}
	\rdg[wit={V1}]{kṛṣṭaṃ}
	\rdg[wit={J8}]{kaṣṭaṃ}
	\rdg[wit={V19}]{kṛtvā}}d}%
\pada{dhanurāsana\app{\lem[wit={J8,J10,J17,N17,V1,W4}]{mucyate}
	\rdg[wit={V19}]{īritam}}//}\\!}
\end{tlg}

\pagebreak

%1.26
%Testimonia
% HR 3.57
% vāmorumūlārpitadakṣapādo jānvor bahir veṣṭitadakṣadoṣṇā | [P °dakṣapādaṃ]
% pragṛhya tiṣṭhet partivartitāṅgaḥ śrīmatsyanāthoditam āsanaṃ syāt |

\begin{tlg}[hp01_026]
\tl{
\pada{vāmorumūlārpitadakṣapādaṃ}
% [da -in dakṣa- is not clear, but I think it's there]. J10:  perhaps pādaṃ with ṃ deleted pc
\pada{\app{\lem[wit={J8pc,J10,J17,N17,V1,W4},alt={jānor}]{jāno}
	\rdg[wit={J8ac}]{jānaur}
	\rdg[wit={V19}]{jānvor}}rbahirveṣṭita%
	\app{\lem[wit={J8,J10,J17,N17,V1,W4}]{vāmapādam}
	\rdg[wit={V19}]{dakṣadoṣṇā}}/}\\+}
\tl{
\pada{pragṛhya tiṣṭhet pari\app{\lem[wit={J8,J10,J17,V1,W4}]{vartitāṅgaḥ} % °āṃgo W4
	\rdg[wit={N17}]{vartitāḥ gaḥ}
	\rdg[wit={V19}]{marditāṅgaḥ}}}
\pada{śrī\app{\lem[wit={J8pc,J10,J17,N17,V1,V19,W4}]{matsyanātho}
	\rdg[wit={J8ac}]{matsyadinātho}}ditamāsanaṃ syāt//}\\!}
\end{tlg}


%1.27
%Testimonia
% HR 3.58
% matsyendrapīṭhaṃ jaṭharapradīptaṃ pracaṇḍarugmaṇḍalakhaṇḍanāstram||
% abhyāsataḥ kuṇḍalinīprabodhaṃ daṇḍasthiratvaṃ ca dadāti puṃsām||3.58||%HP 1.27
%YCM
% matsyendrapīṭhaṃ jaṭharapravṛddhiṃ
% pracaṇḍaruṅmaṇḍalakhaṇḍanāstram/
% abhyāsataḥ kuṇḍalinīprabodhaṃ
% daṇḍe sthiratvaṃ pradadāti puṃsām//

\begin{tlg}[hp01_027]
\tl{
\pada{matsyendra\app{\lem[wit={J8,J10,J17,N17,V19,W4}]{pīṭhaṃ}
	\rdg[wit={V1}]{vīraṃ}}
	\app{\lem[wit={J10,J17,V1,V19}]{jaṭhara}
	\rdg[wit={J8,N17,W4}]{jvalana}}\app{\lem[resp=emend]{pradīptiṃ}
	\rdg[wit={J8,N17,W4}]{pradīptaṃ}
	\rdg[wit={V1}]{pracaṇḍaṃ}
	\rdg[wit={V19}]{prabuddhau}
	\rdg[wit={J10,J17}]{pravṛddha}}}
\pada{\app{\lem[wit={J8,J10,J17,N17,V19,W4}]{pracaṇḍa}
	\rdg[wit={V1}]{viccaṇḍa}}%
	\app{\lem[wit={J8,J10,J17,V19,W4},alt={ruṅ}]{ru}
	\rdg[wit={V1}]{rūr}
	\rdg[wit={N17}]{rug}}%
	\app{\lem[wit={J8,J10,J17,V1,V19,W4},alt={maṇḍala}]{ṅmaṇḍala}
	\rdg[wit={N17}]{maṇḍanaṃ}}%
	\app{\lem[wit={J8,J10,J17,N17,V1,W4}]{khaṇḍanāstram}
	\rdg[wit={V19}]{khaṇḍalāsyam}}/}\\+}
\tl{
\pada{abhyāsataḥ kuṇḍalinīprabodhaṃ}
\pada{\app{\lem[wit={J8,J10pc,N17,V1,V19}]{daṇḍa}
	\rdg[wit={J10ac,J17}]{candra}
	\rdg[wit={W4}]{candraḥ}}\app{\lem[wit={J8,J10,J17,N17,V1,W4}]{sthiratvaṃ}
	\rdg[wit={V19}]{sthitatvaṃ}}
	\app{\lem[wit={J8,J10,J17,N17,V1,W4}]{ca dadāti}
	\rdg[wit={V19}]{pradadāti}} puṃsām//}\\!}
\end{tlg}


%1.28
%Testimonia
% Cf. Śivasaṃhitā 2.108
%  prasārya caraṇadvandvaṃ parasparasusaṃyutam
% svapāṇibhyāṃ dṛḍhaṃ dhṛtvā jānūpari śiro nyaset 3.108
%
% HR
%prasārya pādau bhuvi daṇḍarūpau dorbhyāṃ padāgradvitayaṃ gṛhītvā||
% jānūpari nyastalalāṭadeśo vased idaṃ paścimatānam āhuḥ||3.66||%

\begin{tlg}[hp01_028]
\tl{
\pada{prasārya pādau bhuvi daṇḍarūpau} % J17 prasāryya
\pada{\app{\lem[wit={V1,V19}]{dorbhyāṃ}
	\rdg[wit={J8,J10,J17,N17,W4}]{dvābhyāṃ}}
	\app{\lem[wit={V1}]{padāgra}
	\rdg[wit={J8,J10,J17,N17,W4}]{karābhyāṃ}
	\rdg[wit={V19}]{ca pāda}}%
	\app{\lem[wit={J10,J17,V1,V19}]{dvitayaṃ}
	\rdg[wit={J8}]{dvitiyaṃ}
	\rdg[wit={N17,W4}]{dvitīyaṃ}}
gṛhītvā/}\\+} % gṛhitvā J8ac
\tl{
\pada{jānūpari nyastalalāṭa%
	\app{\lem[wit={J8pc,J10,N17,V1,V19,W4}]{deśo}
	\rdg[wit={J8ac}]{dese}
	\rdg[wit={J17}]{desā}}}
\pada{\app{\lem[wit={J8,J10pc,V1},alt={vased}]{vase}
	\rdg[wit={N17}]{vaṃśed}
	\rdg[wit={J10ac,J17},alt={d (vase \om)}]{d}
	\rdg[wit={V19,W4}]{'bhyased}}didaṃ paścima%
	\app{\lem[wit={J8,J10,J17,N17,V19,W4},alt={tānam}]{tāna}
	\rdg[wit={V1}]{tāṇa}}%
	\app{\lem[wit={J8,J10,J17,N17,V19,W4},alt={āhuḥ}]{māhuḥ}
	\rdg[wit={V1}]{bandhaḥ}}//}\\!}
\end{tlg}


%1.29
\begin{tlg}[hp01_029]
\tl{
\pada{iti paścima\app{\lem[wit={J8,J10,J17,N17,V19,W4}]{tāna}
	\rdg[wit={V1}]{tāṇa}}%
	\app{\lem[wit={J8,N17,V1,W4},alt={āsanāgryaṃ}]{māsanāgryaṃ}
	\rdg[wit={J10}]{āsanāśāgryaṃ}
	\rdg[wit={J17}]{āsanaṃ nāśāgryaṃ}
	\rdg[wit={V19}]{āsanākhyaṃ}}}
\pada{pavanaṃ
	\app{\lem[wit={J8,J10,J17,N17,V1,W4}]{paścima}
	\rdg[wit={V19}]{paścimā}}\app{\lem[wit={J10,J17,N17,V1,V19,W4}]{vāhinaṃ}
	\rdg[wit={J8}]{vāhanaṃ}} karoti/}\\+}
\tl{
\pada{\app{\lem[wit={J8,J10,J17,V1,V19,W4}]{udayaṃ}
	\rdg[wit={N17}]{udaraṃ}}  jaṭha\app{\lem[wit={J8,J10,J17,N17,V1,W4},alt={ānalasya}]{rānalasya}
	\rdg[wit={V19}]{ānilasya}} kuryād}%
\pada{udare
	\app{\lem[wit={J10,J17,N17,V1,V19,W4},alt={kārśyam}]{kārśya}
	\rdg[wit={J8}]{kṛśyam}}%
	\app{\lem[wit={J10ac,J17,V1ac,V19,W4},alt={arogitāṃ}]{marogitāṃ}  % arogitā is attested; e.g., Yogaśāstra 5.24, Yuktabhavadeva 7.76, etc.
	\rdg[wit={N17}]{arogitā}
	\rdg[wit={V1pc}]{arogatāṃ}
	\rdg[wit={J8}]{arogyatāṃ}
	\rdg[wit={J10pc}]{ārogyatāṃ}}
ca puṃsām//}\\!}
\end{tlg}


%1.30
% Testimonia
% VS 1.76-77
% 1.76ab avaṣṭabhya dharāṃ samyak talābhyāṃ ca karadvayam |
% 1.76cd hastayoḥ kūrparau cāpi sthāpayan nābhipārśvayoḥ ||% kūrparau em.;  kharpare ed
% 1.77ab samunnataśiraḥpādo daṇḍavad vyomni saṃsthitaḥ |
% 1.77cd mayūrāsanam etad dhi sarvapāpavināśanam ||

\begin{tlg}[hp01_030]
\tl{
\pada{dharāmavaṣṭabhya
	\app{\lem[wit={J8,J10,J17,N17,V1,W4}]{karadvayena}
	\rdg[wit={V19}]{punaḥ karābhyāṃ}}}
\pada{ta\app{\lem[wit={J10,J17,N17,V1,W4},alt={kūrpara}]{tkūrpara} % J17 kūrppara
	\rdg[wit={J8}]{kurpara}
	\rdg[wit={V19}]{kurpare}}sthāpitanābhi\app{\lem[wit={J10,V1,W4}]{pārśvaḥ} % pārśvo W4
	\rdg[wit={J8,J17,N17,V19}]{pārśve}}/}\\+}
\tl{
\pada{\app{\lem[wit={J8ac,J10,J17,N17,V1,W4}]{uccāsano}
	\rdg[wit={J8pc}]{uccāsane}
	\rdg[wit={V19}]{taccāsanaṃ}}
daṇḍava\app{\lem[wit={J10,V1,V19},alt={utthitaḥ khe}]{dutthitaḥ khe}
	\rdg[wit={J17}]{uthitaḥ kha}
	\rdg[wit={J8,N17}]{uthitasya}
	\rdg[wit={W4}]{athitasya}}}
\pada{\app{\lem[wit={J10,J17,N17,V19,W4}]{māyūra}
	\rdg[wit={J8,V1}]{mayūra}}m etat pravadanti
	\app{\lem[wit={J8,J10,J17,N17,V1,W4}]{pīṭham}
	\rdg[wit={V19}]{santaḥ}}//}\\!}
\end{tlg}


%1.31
% Testimonia
% HR 3.43
%  harati sakalarogān āśu gulmodarādīn
% abhibhavati ca doṣān āsanaṃ śrīmayūram||
% bahukadaśanabhuktaṃ bhasma kuryād vicitram
% janayati jaṭharāgniṃ jīryate kālakūṭam||3.43||
\begin{tlg}[hp01_031]
\tl{
\pada{harati sakala%
	\app{\lem[wit={J8pc,V1,V19},alt={rogān āśu}]{rogānāśu}
	\rdg[wit={J10,N17,W4}]{doṣān āśu}
	\rdg[wit={J8ac}]{doṣān asu}
	\rdg[wit={J17}]{doṣān āṃśu}}%
gulmoda\app{\lem[wit={J8pc,J10,N17,V1,V19,W4},alt={ādīn}]{rādī}
	\rdg[wit={J8ac,J17}]{ādi}}}-\\+}
\tl{
\pada{\app{\lem[wit={J10,J17,V1,V19,W4}]{n abhibhavati ca}
	\rdg[wit={J8}]{na bhavati}
	\rdg[wit={N17}]{naptibhavati ca}}
	\app{\lem[wit={J10,J17,N17,V1,V19}]{doṣānāsanaṃ}
	\rdg[wit={J17}]{doṣānāśana}
	\rdg[wit={W4}]{rogānāsanaṃ}} śrīmayūram/}\\+}
\tl{
\pada{bahu kadaśanabhuktaṃ
	\app{\lem[wit={J8,J10,J17,N17,V1,W4}]{bhasma}
	\rdg[wit={V19}]{tac ca}} kuryā\app{\lem[wit={J8,J10,J17,N17,V19,W4}, alt={aśeṣaṃ}]{daśeṣaṃ}
	\rdg[wit={V1},alt={aśeṣo}]{daśeṣo}}}\\+}
\tl{
\pada{janayati
	\app{\lem[wit={J8pc,J10,J17,N17,V1,V19,W4}]{jaṭharāgniṃ}
	\rdg[wit={V3}]{vaḍavāgniṃ}}
	\app{\lem[wit={J8,V1,V19},alt={jārayet}]{jāraye}
	\rdg[wit={J10,J17,N17,W4}]{jvālayet}}tkālakūṭam//}\\!}
\end{tlg}


%1.32
% Testimonia
% Cf. DYŚ 24cd
% uttānaśavavad bhūmau śayanaṃ coktam uttamam ||24||
% Cf. HR 3.76
% athāntimaṃ śavāsanam
% prasārya hastapādau ca viśrāntyā śayanaṃ tathā||
% sarvāsanaśramaharaṃ śayitaṃ tu śavāsanam||3.76||
%
\begin{tlg}[hp01_032]
\tl{
\pada{\app{\lem[wit={J8,V1,V19,W4}]{uttānaṃ}
	\rdg[wit={J10,J17,N17}]{uttāna}} śavavadbhūmau}
\pada{\app{\lem[wit={J8,V1}]{śayanaṃ tu śavāsanam}
	\rdg[wit={J10,J17,N17,W4}]{śayanaṃ ca śavāsanam}
	\rdg[wit={V19}]{śavāsanam idaṃ smṛtam}}/}\\+}
\tl{
\pada{\app{\lem[wit={J10,J17,V1}]{sarvāsana}% sarvva° J17
	\rdg[wit={V3}]{savāsana}
	\rdg[wit={J8pc,N17,V19,W4}]{śavāsanaṃ}}śrāntiharaṃ}
\pada{cittaviśrānti\app{\lem[wit={J10,J17,N17,V1,W4}]{kārakam}
	\rdg[wit={J8,V19}]{sādhanam}}//}\\!}
\end{tlg}


%1.33
%Testimonia
% HR
% caturaśīty āsanāni śivena kathitāni tu  |
% tebhyaś catuṣkam ādāya sārabhūtaṃ bravīmy aham ||3.23||
%
\begin{tlg}[hp01_033]
\tl{
\pada{\app{\lem[wit={J8,J10,J17,V1,V19,W4}]{caturā}
	\rdg[wit={N17}]{catura}}śītyāsanāni}
\pada{śivena
	\app{\lem[wit={J8pc,J10,J17,N17,V1,V19,W4}]{kathitāni}
	\rdg[wit={J8ac}]{kathitāna}}
	\app{\lem[wit={J8,J10,J17,N17,V1,W4}]{tu}
	\rdg[wit={V19}]{vai}} /}\\+}
\tl{
\pada{tebhyaś catuṣkamādāya} \pada{sārabhūtaṃ bravīmy aham//}\\!}
\end{tlg}


%1.34
%Testimonia
% HR 3.24
% siddhaṃ padmaṃ tathā siṃhaṃ bhadraṃ ceti catuṣṭayam  |
% śreṣṭhaṃ tatrāpi ca tathā tiṣṭhet siddhāsane sadā ||3.24||
\begin{tlg}[hp01_034]
\tl{
\pada{siddhaṃ
	\app{\lem[wit={J8,J10,J17,N17,V1,W4}]{padmaṃ}
	\rdg[wit={V19}]{bhadraṃ}}
	\app{\lem[wit={J10,J17,N17,V1,V19,W4}]{tathā}
	\rdg[wit={J8}]{yathā}}
	\app{\lem[wit={J8,J10,J17,N17,V1,W4}]{siṃhaṃ}
	\rdg[wit={V19}]{padmaṃ}}}
\pada{\app{\lem[wit={J8,J10,J17,N17,V1,W4}]{bhadraṃ}
	\rdg[wit={V19}]{siṃhaṃ}}
	\app{\lem[wit={J10,J17,N17,V19,W4}]{ceti}
	\rdg[wit={V1}]{caiva}
	\rdg[wit={J8}]{caitac}} catuṣṭayam/}\\+}
\tl{
\pada{śreṣṭhaṃ
	\app{\lem[wit={J10,J17,V1}]{tatrāpi ca sukhe}
	\rdg[wit={W4}]{tatrāpi ca sukhaṃ}
	\rdg[wit={J8}]{tatrāpi ca sakhe}
	\rdg[wit={V19}]{tathāpi bhadraṃ [ca]}
	\rdg[wit={N17}]{tatra dvayaṃ tiṣṭhet}}} % N17 śreṣṭhaṃ tatra dvayaṃ tiṣṭhet padmasiṃhāsane sadā |
\pada{\app{\lem[wit={J10,J17,V1,V19,W4},alt={tiṣṭhet}]{tiṣṭhe}
	\rdg[wit={J8}]{tiṣṭha}
	\rdg[wit={N17}]{padma}}%
	\app{\lem[wit={J8,J10,J17,V1,W4},alt={siddhāsane}]{tsiddhāsane}
	\rdg[wit={N17,V19}]{siṃhāsane}} sadā//}\\!}
\end{tlg}


\startaltrecension{}
%(not J8,V1,V19)
\outdent
tatra \app{\lem[wit={J10,J17,N17}]{siddhāsanam}
	\rdg[wit={W4}]{siddhāsanam āha}}/ \sgwit{J10,J17,N17,W4}
\endaltrecension


%1.35
% testimonia
%VM 7
% yonisthānakam aṅghrimūlaghaṭitaṃ kṛtvā dṛḍhaṃ vinyase[n] meḍhre pādam
% athaikam āsyahṛdaye dhṛtvā samaṃ vigraham |
% sthāṇuḥ saṃyamitendriyo 'caladṛśā paśyan bhruvor antaraṃ
% etan  mokṣakapāṭabhedajanakaṃ siddhāsanaṃ procyate ||
% HR
% tatra siddhāsanam
% yonisthānakam aṅghrimūlaghaṭitaṃ kṛtvā dṛḍhaṃ vinyasen
% meḍhre pādam athaikam eva niyataṃ kṛtvā samaṃ vigraham  |
% sthāṇuḥ saṃyamitendriyo 'caladṛśā paśyan bhruvor antaraṃ
% caitan mokṣakapāṭabhedajanakaṃ siddhāsanaṃ procyate||3.25||%
%YCM
%pavanayogasaṃgrahe--
% yonisthānakam aṅghrimūlaghaṭitaṃ kṛtvā dṛḍhaṃ vinyasen
% meḍhre pādam athaikam ekahṛdaye kṛtvā samaṃ vigraham/
% sthāṇuḥ saṃyamitendriyo 'caladṛśā paśyed bhruvor antaraṃ tv
% etan mokṣakapāṭabhedanakaraṃ siddhāsanaṃ procyate// = 1.35

\begin{tlg}[hp01_035]
\tl{
\pada{yoni\app{\lem[wit={J8,J10,J17,N17,V1,W4}]{sthānaka}
	\rdg[wit={V19}]{dvāraka}}%
	\app{\lem[wit={J8,J10pc,N17,V1}]{m aṅghrimūla}
	\rdg[wit={V19}]{m aṅghrimūlā}
	\rdg[wit={J17,W4}]{mūlam aṅghri}
	\rdg[wit={J10ac}]{mūlāṅghri}}ghaṭitaṃ kṛtvā dṛḍhaṃ vinyase}- \\+}
\tl{
\pada{\app{\lem[wit={J8,J10,J17,N17,V1,W4},alt={meḍhre}]{nmeḍhre} % J17 t meḍhre
	\rdg[wit={V19}]{madhye}}
	\app{\lem[wit={J8,J10,N17,V1,V19,W4},alt={pādam athaikam}]{pādam athaika}%
	\rdg[wit={J17}]{pādatathaikam}}%
	\app{\lem[wit={V1},alt={āsyahṛdaye}]{māsyahṛdaye}
	\rdg[wit={J10,J17,V19}]{eva hṛdaye}
	\rdg[wit={J8,N17,W4}]{eva niyataṃ}}
	\app{\lem[wit={J10,J17,V1,V19,W4}]{dhṛtvā}
	\rdg[wit={J8,N17}]{kṛtvā}} samaṃ vigraham/}\\+}
\tl{
\pada{\app{\lem[wit={J8,J10,J17,N17,V1,W4}]{sthāṇuḥ} % N17 unclear sthā+
	\rdg[wit={V19}]{sthāṇu}} saṃyamitendriyo'caladṛśā % J10 writes avagraha!
	\app{\lem[wit={J8,J10,J17,N17,V1}]{paśyan}
	\rdg[wit={W4},post={(two syllables om.)}]{n}
	\rdg[wit={V19}]{paśyed}} bhruvorantaraṃ} \\+}
\tl{
\pada{\app{\lem[wit={V1,V19},alt={etan}]{eta}
	\rdg[wit={J8,J10,J17,N17,W4}]{caitan}}%
nmokṣakapāṭabhedajanakaṃ siddhāsanaṃ procyate//}\\!}
\end{tlg}

\begin{ava}[hp01_036]
\outdent\app{\lem[wit={J10,N17,W4}]{matāntare tu }% N17 not very clear
	\rdg[wit={J17}]{mantāntare tu}
	\rdg[wit={J8,V1}]{matāntare}
	\rdg[wit={V19}]{matsyendraḥ| matāntaraṃ tu}}
\end{ava}

%1.36
%Testimonia
% VS 1.81
% meḍhrād upari nikṣipya gulphaṃ tathopari |
% gulphāntaraṃ vinikṣipya muktāsanam idaṃ smṛtam ||
% YY 3.15
% meḍhrād upari nikṣipya savyaṃ gulphaṃ tathopari
% gulphāntaraṃ ca nikṣipya muktāsanam idaṃ tu vā
% HR 3.26
% matāntare tu
% meḍhrād upari niḥkṣipya savyaṃ gulphaṃ tathopari  |
% gulphāntaraṃ ca niḥkṣipya siddhāḥ siddhāsanaṃ viduḥ||
%  śāṇḍilyopaniṣat 1.7
% yoniṃ vāmena saṃpīḍya meḍhrād upari dakṣiṇam |
% bhrūmadhye ca manolakṣyaṃ siddhāsanam idaṃ bhavet ||

\begin{tlg}[hp01_036]
\tl{
\pada{meḍhrādupari % meṃḍhrād W4
	\app{\lem[wit={V1}]{vinyasya}
	\rdg[wit={V19}]{vinyasyaṃ}
	\rdg[wit={J8}]{nikṣipya}
	\rdg[wit={J10,J17,N17,W4}]{niḥkṣipya}}}
\pada{\app{\lem[wit={J8,J10,W4}]{savya}
	\rdg[wit={J17}]{sarvya}
	\rdg[wit={N17}]{savyaṃ}
	\rdg[wit={V1},post={\unm}]{savyaṃ tu}
	\rdg[wit={V19}]{vāma}}gulphaṃ % N17 gulpha
	\app{\lem[wit={J8,J17,N17,V1,V19,W4}]{tathopari}
	\rdg[wit={J10}]{tathopariḥ}} /}\\+}
\tl{
\pada{gulphāntaraṃ % J17 °āntarañ, N17 °āntara
	\app{\lem[wit={V1}]{tu}
	\rdg[wit={J8,J10,J17,N17,V19,W4}]{ca}}
	\app{\lem[wit={J8,V1}]{nikṣipya}
	\rdg[wit={J10,J17,N17,W4}]{niḥkṣipya}
	\rdg[wit={V19}]{vinyasya}}}
\pada{siddhāsanam idaṃ bhavet//}\\!}
\end{tlg}

\pagebreak

%1.37
\begin{tlg}[hp01_037]
\tl{
\pada{\app{\lem[wit={J8,J10,J17,N17,V1,W4},alt={etat}]{eta}
	\rdg[wit={V19}]{kecit}}t siddhāsanaṃ prāhur}\pada{anye vajrāsanaṃ viduḥ/}\\+}
\tl{
\pada{\app{\lem[wit={J8,J10,J17,N17,V1,W4}]{muktā}
	\rdg[wit={V19}]{muktvā}}sanaṃ
vada\app{\lem[wit={J8,J10,J17,N17,V1,W4},alt={eke}]{ntyeke}
	\rdg[wit={V19}]{anye}}}
\pada{prāhur guptāsanaṃ pare//}\\!}   %N17 prāhur gguptā°
\end{tlg}


%1.38
%Testimonia
% Cf. YCM
% niyameṣu mitāharo yathāhịmsā yameṣv iva |
% mukhyaṃ sarvāsaneṣv evaṃ siddhāsanaṃ idaṃ viduḥ |
%
% 10 Chap.
% yameṣv iva mitāhāro 'hiṃsā ca niyameṣv iva।।
% tathā sarvāsane pūjyaṃ siddhāḥ siddhāsanaṃ viduḥ।। 24।।
%
% Cf. DYŚ 33
% laghvāhāras tu teṣv eko mukhyo bhavati nāpare |
% ahiṃsā niyameṣv eko mukhyo bhavati nāpare||33||

\begin{tlg}[hp01_038]
\tl{
\pada{yame\app{\lem[wit={J8,N17,V19,W4},alt={iva}]{ṣviva}
	\rdg[wit={J10,J17,V1}]{eva}}
	\app{\lem[wit={V19},alt={mitāhāram}]{mitāhāra}
	\rdg[wit={J8,V1}]{mitāhāra}
	\rdg[wit={J10,J17,N17,W4}]{mitāhāraḥ}}}%
\pada{\app{\lem[wit={V19},alt={ahiṃsāṃ}]{mahiṃsāṃ}
	\rdg[wit={J8,J10,J17,N17,V1,W4}]{ahiṃsā}}
	\app{\lem[wit={J8,J10,J17,N17,V19,W4}]{niyameṣv iva}
	\rdg[wit={V1}]{niyameṣu ca}}/}\\+}
\tl{
\pada{mukhyaṃ sarvāsane%J17 sarvvāsane
	\app{\lem[wit={J8,J10,J17,N17,V1,W4},alt={ekaṃ}]{ṣvekaṃ}
	\rdg[wit={V19}]{evaṃ}}}
\pada{\app{\lem[wit={J10,V1pc,W4}]{siddhāḥ siddhāsanaṃ}
	\rdg[wit={N17,V1ac,V3}]{siddhā siddhāsanaṃ}
	\rdg[wit={J8}]{siddhā siddhāsana}
	\rdg[wit={V19}]{siddhāsanam idaṃ}}
viduḥ//}\\!}  % J17 vidu
\end{tlg}

%1.39
% Testimonia
% Cf. YCm
% caturaśītipīṭheṣu siddhāsanaṃ samabhyaset/
% dvāsaptatisahasreṣu suṣumṇām iva nāḍiṣu//
%
% Yogacintāmaṇi-Ramya-18-6-2017.tex:2928:dvisaptatisahasrāṇāṃ nāḍ़īnāṃ % malaśodhanam||
%check more mss

\begin{tlg}[hp01_039]
\tl{
\pada{\app{\lem[wit={J8,J10,J17,V1,V19,W4}]{caturāśīti}
	\rdg[wit={N17}]{caturaśīti}}pīṭheṣu}
\pada{\app{\lem[wit={J8,J10,J17,N17,V1}]{siddham eva}
	\rdg[wit={W4}]{siddhemeva}
	\rdg[wit={V19}]{siddhāsanaṃ}}
	\app{\lem[wit={J8,J10,J17,N17,V1,V19}]{sadābhyaset}
	\rdg[wit={W4}]{samabhyaset}}/}\\+}
\tl{
\pada{\app{\lem[wit={J8,J10,J17,V1,V19,W4}]{dvāsaptati}
	\rdg[wit={N17}]{dvisaptati}}%
	\app{\lem[wit={J10,J17,N17,V1,V19,W4}]{sahasreṣu}
	\rdg[wit={J8ac}]{sahaśreṣu}
	\rdg[wit={J8pc}]{sahasrasya}}}
\pada{\app{\lem[wit={J10ac,J17,V19},alt={suṣumṇām}]{suṣumṇā}
	\rdg[wit={V1}]{suṣumṇā}
	\rdg[wit={J8,J10pc,N17,W4}]{nāḍīnāṃ}}
	\app{\lem[wit={V19},alt={iva nāḍiṣu}]{miva nāḍiṣu}
	\rdg[wit={V1}]{iva nāḍikā}
	\rdg[wit={J10ac,J17}]{eva nāḍiṣu}
	\rdg[wit={J8}]{malasodhanam}
	\rdg[wit={J10pc,N17,W4}]{malaśodhane}}//}\\!}
\end{tlg}


%1.40
%Testimonia
% Cf YCM
% ātmadhyāyī mitāhārī yāvad dvādaśavatsaram/
% sadā siddhāsanābhyāsād yogī niṣpattim āpnuyāt/
% śramadair bahubhiḥ pīṭhaiḥ kiṃ syāt siddhāsane sati// = 1.40
% prāṇānile sāvadhāne baddhe kevalakumbhake/

\begin{tlg}[hp01_040]
\tl{
\pada{\app{\lem[wit={J8,J10,J17,V1,V19,W4}]{ātmadhyāyī}
	\rdg[wit={N17}]{ātmadhyāyi}}
	\app{\lem[wit={J8,J10,J17,N17,V1,W4}]{mitāhārī}
	\rdg[wit={V19}]{mitāhāro}}} \pada{yāvad dvādaśavatsaram/}\\+}
\tl{
\pada{sadā siddhāsanā\app{\lem[wit={J8,J10,J17,N17,V1,W4},alt={bhyāsād}]{bhyāsā}
	\rdg[wit={V19}]{bhyānād}}d}
\pada{yogī
	\app{\lem[wit={J17,N17,V1,V19,W4},alt={niṣpattim āpnuyāt}]{niṣpattimāpnuyāt}%J17 nippatttim
	\rdg[wit={J10},post={\unm}]{niṣpattim avāpnuyāt}
	\rdg[wit={J8}]{siddhim avāpnuyāt}}/}\\+}
\tl{
\pada{\app{\lem[wit={YC}]{śramadair bahubhiḥ} % J13 śramadair bahubhiḥ
	\rdg [wit={V1}]{śramādau bahubhiḥ}
	\rdg[wit={J8,J10,J17,N17,W4}]{kim ādyair bahubhiḥ}% N17 bbahubhiḥ, J17 bbahubhiṣ
	\rdg[wit={V19}]{śramadairghyādibhiḥ}} pīṭhaiḥ}
\pada{\app{\lem[wit={V19},alt={kiṃ syāt}]{kiṃ syā}
	\rdg[wit={J8,J10,J17,N17,V1,W4}]{sadā}}tsiddhāsane sati//}\\!}
\end{tlg}


%1.41
\begin{tlg}[hp01_041]
\tl{
\pada{prāṇānile
	\app{\lem[wit={J10,J17,V1}]{sāvadhānaṃ}
	\rdg[wit={J8,N17,V19,W4,YC}]{sāvadhāne}}}
\pada{\app{\lem[wit={J10,J17,N17,V1,V19,W4}]{baddhe}
	\rdg[wit={J8ac}]{baṃdhe}
	\rdg[wit={J8pc}]{badhe}}
	kevala\app{\lem[wit={J10,J17,N17,V1,V19,W4}]{kumbhake}
	\rdg[wit={J8}]{kumbhakaḥ}}/}\\+}
\tl{
\pada{\app{\lem[wit={J8,J10,J17,V19,W4}]{utpadyate}
	\rdg[wit={N17,V1,V3}]{utpadyaṃte}} %N17 utpadyante
	\app{\lem[wit={J8,J10,J17,V1,V19,W4},alt={nirāyāsāt}]{nirāyāsā}
	\rdg[wit={N17}]{vinābhyāsāt}}t}
\pada{svayam evonmanī
	\app{\lem[wit={J8,J10,J17,N17,V19,W4}]{yathā}
	\rdg[wit={V1}]{tathā}}//}\\!}
\end{tlg}

%1.42
\begin{tlg}[hp01_042]
\tl{
\pada{\app{\lem[wit={J8,J10,J17,N17,V1,W4}]{tathaika}
	\rdg[wit={V19}]{athaika}}sminn eva
	\app{\lem[wit={J10,J17,N17,V1,W4}]{dṛḍhe}
	\rdg[wit={J8,V19}]{dṛḍhaṃ}}}
\pada{baddhe
	\app{\lem[wit={J8,J10,J17,N17,V1,W4}]{siddhāsane}
	\rdg[wit={V19}]{siṃhāsane}} sadā/} \\+}         %  V1 3r
\tl{
\pada{bandhatrayam anāyāsāt}\pada{svayame
	\app{\lem[wit={J8,J10,N17,V1,V19},alt={evopajāyate}]{vopajāyate}
	\rdg[wit={J17}]{evoprajāyate}
	\rdg[wit={W4}]{evopīyāyate}}//}\\!}  % after corr.
\end{tlg}

\pagebreak

%1.43
\begin{tlg}[hp01_043]
\tl{
\pada{\app{\lem[wit={J10,N17,V1,W4}]{na cāsanaṃ siddhasamaṃ}
	\rdg[wit={J17}]{na cāsanaṃ siddhāsanaṃ}
	\rdg[wit={V19}]{nāsanaṃ siddhasadṛśaṃ}
	\rdg[wit={J8},alt={\om}]{}}}
\pada{na
	\app{\lem[wit={J10,J17,N17,V1}]{kumbhasadṛśo'nilaḥ}
	\rdg[wit={W4}]{kumbhaṃ sadṛśo nalaḥ}
	\rdg[wit={V19}]{kumbhaḥ kevalopamaḥ}
	\rdg[wit={J8},alt={\om}]{}}/}\\+}
% J10 has vs no. 43 here, but counts atha padmāsanam as 45.
\tl{
\pada{na khecarīsamā mudrā}
\pada{na
	\app{\lem[wit={J10,J17,N17,V19,W4}]{nāda}
	\rdg[wit={V1}]{nādaḥ}
	\rdg[wit={J8},alt={\om}]{}}sadṛśo layaḥ//} \NotIn{J8}\\!}
\end{tlg}

\begin{ava}[hp01_044]
\outdent\app{\lem[wit={J10,J17,W4}]{atha}
	\rdg[wit={J8,V1}]{tathā}
	\rdg[wit={N17}]{.. .[ā]}
	\rdg[wit={V19},alt={\om}]{}}
	\app{\lem[wit={J8,J10,J17,V1,V19,W4}]{padmāsanam}
	\rdg[wit={N17}]{[san]. .. dam}% illegible scan: might be padmāsanaṃ tv idam
	\rdg[wit={V19},alt={\om}]{}} /
\end{ava}

%1.44
% Testimonia
% VM 8
% vāmorūpari dakṣiṇañ ca caraṇaṃ saṃsthāpya vāmaṃ tathā
% yāmyorūpari paścimena vidhinā dhṛtvā karābhyāṃ dṛḍham |
% aṅguṣṭhau hṛdaye nidhāya cibukaṃ nāsāgram ālokayed
% etad vyādhivikārahāri yamināṃ padmāsanaṃ procyate ||8||

\begin{tlg}[hp01_044]
\tl{
\pada{vāmorūpari
	\app{\lem[wit={J8pc,J10,J17,N17,V1,W4}]{dakṣiṇaṃ ca}% J17 dakṣiṇañ ca
	\rdg[wit={J8ac}]{dakṣaṇaṃ ca}
	\rdg[wit={V19}]{vidakṣiṇaṃ hi}}
	caraṇaṃ saṃsthāpya vāmaṃ % N17 saṃsthāṃpya, W4 vāma
	\app{\lem[wit={J10,J17,N17,V1,V19,W4}]{tathā}
	\rdg[wit={J8}]{tato}}}\\+}
\tl{
\pada{\app{\lem[wit={J8ac,V1}]{yāmyo}
	\rdg[wit={J8pc}]{vāmo}
	\rdg[wit={V19,W4}]{dakṣo}
	\rdg[wit={J10}]{jānvo}
	\rdg[wit={J17}]{jāmvau}
	\rdg[wit={N17}]{jānū}}rūpari
	\app{\lem[wit={J10,J17,N17,V1,V19,W4}]{paścimena vidhinā}
	\rdg[wit={J8}]{tasya bandhanavidhau}}
	\app{\lem[wit={J10,J17,N17,V1,V19,W4}]{dhṛtvā}
	\rdg[wit={J8}]{pṛṣṭe}} karābhyāṃ dṛḍham/}\\+}  % J17 draḍhaṃ
\tl{
\pada{aṅguṣṭhau hṛdaye nidhāya
	\app{\lem[wit={J8,J10,J17,V1,V19,W4}]{cibukaṃ}
	\rdg[wit={N17}]{.. .. kaṃ}} % N17 looks like two akṣaras have been crossed out
	nāsāgramālokaye-}\\+}
\tl{
\pada{detad vyādhi\app{\lem[wit={J10,J17,V1,W4}]{vināśakāri}
	\rdg[wit={J8}]{vināsanaṃ}
	\rdg[wit={N17}]{vināśanāya}
	\rdg[wit={V19}]{vikāranāśa}}
	\app{\lem[wit={J8,J10,J17,N17,V1,W4}]{yamināṃ}
	\rdg[wit={V19}]{nakaraṃ}}
	padmāsanaṃ procyate//}\\!}
\end{tlg}

\begin{ava}[hp01_045]
\outdent
	\app{\lem[wit={J10,J17,N17}]{matāntare tu}
	\rdg[wit={J8,V1,W4}]{matāntare}
	\rdg[wit={V19}]{matabhede}}/
\end{ava}

%1.45
% Testimony
% DYŚ 35
% uttānau caraṇau kṛtvā ūrusaṃsthau prayatnataḥ |
% ūrumadhye tathottānau pāṇī kṛtvā tato dṛśau ||35||

\begin{tlg}[hp01_045]
\tl{
\pada{uttānau caraṇau kṛtvā} \pada{\app{\lem[wit={J8pc,J10,N17,V1,V19,W4}]{ūru} % uru W4
	\rdg[wit={J8ac}]{kuru}}%
	\app{\lem[wit={J8,J10,J17,V1,V19,W4}]{saṃsthau}
	\rdg[wit={N17}]{saṃdhau}}
	\app{\lem[wit={J8,J10,J17,N17,V1,W4}]{prayatnataḥ}
	\rdg[wit={V19}]{vidhānataḥ}}/}\\+}
\tl{
\pada{ūrumadhye
	\app{\lem[wit={J10pc,J17,N17,V1,V3,W4}]{tathottānau}
	\rdg[wit={J8}]{tathāttānau}
	\rdg[wit={J10ac,V19}]{tathauttānau}}}
\pada{\app{\lem[wit={J8,J10,V1,V19}]{pāṇī}
	\rdg[wit={W4}]{pāṇīn}
	\rdg[wit={J17}]{pāṇiṃ}
	\rdg[wit={N17}]{pāṇau}} kṛtvā
	\app{\lem[wit={J8,J10,J17,N17,V1}]{tato}
	\rdg[wit={W4}]{tatā}
	\rdg[wit={V19}]{tu tā}}
	\app{\lem[wit={J10,J17,N17,V1,V19,W4}]{dṛśau}
	\rdg[wit={J8}]{dṛśai}}//}\\!}
\end{tlg}


%1.46
% Testimony
% DYŚ 36
% nāsāgre vinyased rājadantamūlaṃ ca jihvayā |
% uttabhya cibukaṃ vakṣasy āsthāpya pavanaṃ śanaiḥ ||36||
% [yathāśaktyā samākṛṣya pūrayed udaraṃ śanaiḥ |
% yathāśaktyaiva paścāt tu recayet pavanaṃ śanaiḥ ||37||]
%
% Note the HR 3.37-38 has the same problem with pavanaṃ śanaiḥ
% uttabhya cibukaṃ vakṣaḥ saṃsthāpya pavanaṃ śanaiḥ||3.37||%DYŚ 36
% idaṃ padmāsanaṃ proktaṃ sarvavyādhivināśanam |
% durlabhaṃ yena kenāpi dhīmatā labhyate bhuvi||3.38||

\begin{tlg}[hp01_046]
\tl{
\pada{\app{\lem[wit={J8,J10,J17,N17,V1,W4}]{nāsāgre}
	\rdg[wit={V19}]{nāsagre}}
	\app{\lem[wit={J10,J17,N17,V1,V19,W4},alt={vinyased}]{vinyase}
	\rdg[wit={J8}]{vinyasya}}%
	\app{\lem[wit={J10,J17,V19},alt={rāja}]{drāja}
	\rdg[wit={V1,W4},alt={dṛṣṭiṃ}]{ddṛṣṭiṃ}
	\rdg[wit={J8,N17}]{dṛṣṭī}}}\pada{danta\app{\lem[wit={J8,J10,J17,N17,V1,W4}]{mūlaṃ}
	\rdg[wit={V19}]{mūle}} ca jihvayā/}\\+}
\tl{
\pada{\app{\lem[wit={J10,J17,N17,W4}]{uttabhya}
	\rdg[wit={J8,V1,V19}]{uttambhya}} cibukaṃ
	\app{\lem[wit={V19},alt={vakṣasy}]{vakṣa}
	\rdg[wit={V1}]{vakṣaṃ}
	\rdg[wit={W4}]{vakṣaḥ}
	\rdg[wit={J8,J10,J17,N17}]{vakṣa}}}%
\pada{\app{\lem[wit={V19},alt={āsthāpya}]{syāsthāpya}
	\rdg[wit={J8,J10,J17,V1,W4}]{sthāpayet}
	\rdg[wit={N17}]{sthāpayat}} pavanaṃ śanaiḥ//}% śanai J17
\myfn{Description incomplete. Perhaps, the following verse is omitted by an eye-skip early in the transmission:\\
\textdev{yathāśaktyā samākṛṣya pūrayed udaraṃ śanaiḥ/
	yathāśaktyaiva paścāt tu recayet pavanaṃ śanaiḥ//} (DYŚ 37)}\\!}
\end{tlg}


%1.47
% Testimonia
% DYŚ 35
% idaṃ padmāsanaṃ proktaṃ sarvavyādhivināśanam |
% durlabhaṃ yena kenāpi dhimatā labhyate bhuvi || 35||
% Śivasaṃhitā 3.105
%  idaṃ padmāsanaṃ proktaṃ sarvavyādhivināśanam
% durlabhaṃ yena kenāpi dhīmatā labhyate param 3.105

\begin{tlg}[hp01_047]
\tl{
\pada{idaṃ padmāsanaṃ
	\app{\lem[wit={J8,J10,J17,N17,V1,W4}]{proktaṃ}
	\rdg[wit={V19}]{praktaṃ}}}
	\pada{sarvavyādhivināśanam/}\\+} % J17 sarvya; N17 sarvva° °vināsanaṃ
\tl{
\pada{\app{\lem[wit={J8,J10,J17,N17,V1,W4}]{durlabhaṃ} % N17, J17 durllabhaṃ
	\rdg[wit={V19}]{durlabha}} yena kenāpi}
\pada{\app{\lem[wit={J8,J10,N17,V1,V19,W4}]{dhīmatā} %
	\rdg[wit={J17}]{dhīmatāṃ}}
	\app{\lem[wit={J8,J10,N17,V1,V19,W4}]{labhyate}
	\rdg[wit={J17}]{labhate}}
	\app{\lem[wit={J8,J10,J17,V1,V19,W4}]{bhuvi}
	\rdg[wit={N17}]{bhuviḥ}}//}\\!}
\end{tlg}

\pagebreak

\begin{ava}[hp01_045]
\outdent
	\app{\lem[wit={J10,J17,N17}]{matāntare tu}
	\rdg[wit={J8,V1,W4}]{matāntare}
	\rdg[wit={V19}]{matabhede}}/
\end{ava}

%1.45
% Testimony
% DYŚ 35
% uttānau caraṇau kṛtvā ūrusaṃsthau prayatnataḥ |
% ūrumadhye tathottānau pāṇī kṛtvā tato dṛśau ||35||

\begin{tlg}[hp01_045]
\tl{
\pada{uttānau caraṇau kṛtvā} \pada{\app{\lem[wit={J8pc,J10,N17,V1,V19,W4}]{ūru} % uru W4
	\rdg[wit={J8ac}]{kuru}}%
	\app{\lem[wit={J8,J10,J17,V1,V19,W4}]{saṃsthau}
	\rdg[wit={N17}]{saṃdhau}}
	\app{\lem[wit={J8,J10,J17,N17,V1,W4}]{prayatnataḥ}
	\rdg[wit={V19}]{vidhānataḥ}}/}\\+}
\tl{
\pada{ūrumadhye
	\app{\lem[wit={J10pc,J17,N17,V1,V3,W4}]{tathottānau}
	\rdg[wit={J8}]{tathāttānau}
	\rdg[wit={J10ac,V19}]{tathauttānau}}}
\pada{\app{\lem[wit={J8,J10,V1,V19}]{pāṇī}
	\rdg[wit={W4}]{pāṇīn}
	\rdg[wit={J17}]{pāṇiṃ}
	\rdg[wit={N17}]{pāṇau}} kṛtvā
	\app{\lem[wit={J8,J10,J17,N17,V1}]{tato}
	\rdg[wit={W4}]{tatā}
	\rdg[wit={V19}]{tu tā}}
	\app{\lem[wit={J10,J17,N17,V1,V19,W4}]{dṛśau}
	\rdg[wit={J8}]{dṛśai}}//}\\!}
\end{tlg}


%1.46
% Testimony
% DYŚ 36
% nāsāgre vinyased rājadantamūlaṃ ca jihvayā |
% uttabhya cibukaṃ vakṣasy āsthāpya pavanaṃ śanaiḥ ||36||
% [yathāśaktyā samākṛṣya pūrayed udaraṃ śanaiḥ |
% yathāśaktyaiva paścāt tu recayet pavanaṃ śanaiḥ ||37||]
%
% Note the HR 3.37-38 has the same problem with pavanaṃ śanaiḥ
% uttabhya cibukaṃ vakṣaḥ saṃsthāpya pavanaṃ śanaiḥ||3.37||%DYŚ 36
% idaṃ padmāsanaṃ proktaṃ sarvavyādhivināśanam |
% durlabhaṃ yena kenāpi dhīmatā labhyate bhuvi||3.38||

\begin{tlg}[hp01_046]
\tl{
\pada{\app{\lem[wit={J8,J10,J17,N17,V1,W4}]{nāsāgre}
	\rdg[wit={V19}]{nāsagre}}
	\app{\lem[wit={J10,J17,N17,V1,V19,W4},alt={vinyased}]{vinyase}
	\rdg[wit={J8}]{vinyasya}}%
	\app{\lem[wit={J10,J17,V19},alt={rāja}]{drāja}
	\rdg[wit={V1,W4},alt={dṛṣṭiṃ}]{ddṛṣṭiṃ}
	\rdg[wit={J8,N17}]{dṛṣṭī}}}\pada{danta\app{\lem[wit={J8,J10,J17,N17,V1,W4}]{mūlaṃ}
	\rdg[wit={V19}]{mūle}} ca jihvayā/}\\+}
\tl{
\pada{\app{\lem[wit={J10,J17,N17,W4}]{uttabhya}
	\rdg[wit={J8,V1,V19}]{uttambhya}} cibukaṃ
	\app{\lem[wit={V19},alt={vakṣasy}]{vakṣa}
	\rdg[wit={V1}]{vakṣaṃ}
	\rdg[wit={W4}]{vakṣaḥ}
	\rdg[wit={J8,J10,J17,N17}]{vakṣa}}}%
\pada{\app{\lem[wit={V19},alt={āsthāpya}]{syāsthāpya}
	\rdg[wit={J8,J10,J17,V1,W4}]{sthāpayet}
	\rdg[wit={N17}]{sthāpayat}} pavanaṃ śanaiḥ//}% śanai J17
\myfn{Description incomplete. Perhaps, the following verse is omitted by an eye-skip early in the transmission:\\
\textdev{yathāśaktyā samākṛṣya pūrayed udaraṃ śanaiḥ/
	yathāśaktyaiva paścāt tu recayet pavanaṃ śanaiḥ//} (DYŚ 37)}\\!}
\end{tlg}


%1.47
% Testimonia
% DYŚ 35
% idaṃ padmāsanaṃ proktaṃ sarvavyādhivināśanam |
% durlabhaṃ yena kenāpi dhimatā labhyate bhuvi || 35||
% Śivasaṃhitā 3.105
%  idaṃ padmāsanaṃ proktaṃ sarvavyādhivināśanam
% durlabhaṃ yena kenāpi dhīmatā labhyate param 3.105

\begin{tlg}[hp01_047]
\tl{
\pada{idaṃ padmāsanaṃ
	\app{\lem[wit={J8,J10,J17,N17,V1,W4}]{proktaṃ}
	\rdg[wit={V19}]{praktaṃ}}}
	\pada{sarvavyādhivināśanam/}\\+} % J17 sarvya; N17 sarvva° °vināsanaṃ
\tl{
\pada{\app{\lem[wit={J8,J10,J17,N17,V1,W4}]{durlabhaṃ} % N17, J17 durllabhaṃ
	\rdg[wit={V19}]{durlabha}} yena kenāpi}
\pada{\app{\lem[wit={J8,J10,N17,V1,V19,W4}]{dhīmatā} %
	\rdg[wit={J17}]{dhīmatāṃ}}
	\app{\lem[wit={J8,J10,N17,V1,V19,W4}]{labhyate}
	\rdg[wit={J17}]{labhate}}
	\app{\lem[wit={J8,J10,J17,V1,V19,W4}]{bhuvi}
	\rdg[wit={N17}]{bhuviḥ}}//}\\!}
\end{tlg}

\pagebreak

\begin{ava}[hp01_048]
\outdent
paścāduktaṃ
	\app{\lem[wit={J8,J10,J17,V1}]{matsyamatam}
	\rdg[wit={W4}]{matsyamate}
	\rdg[wit={N17}]{matsyakṛtam}
	\rdg[wit={V19}]{matsyendramatam}}/
\end{ava}

%1.48
% Testimonia
% VM 36
%  kṛtvā saṃpuṭitau karau dṛḍhataraṃ baddhvātha padmāsanaṃ
% gāḍhaṃ vakṣasi sannidhāya cibukaṃ dhyānaṃś ca tac cetasi |
% vāraṃ vāram apānam ūrdhvam anilaṃ proccālayan pūritaṃ
% muṇcan prāṇam upaiti bodham atulaṃ śaktiprabhāvān naraḥ ||

\begin{ava}[hp01_048]
\outdent
paścāduktaṃ
	\app{\lem[wit={J8,J10,J17,V1}]{matsyamatam}
	\rdg[wit={W4}]{matsyamate}
	\rdg[wit={N17}]{matsyakṛtam}
	\rdg[wit={V19}]{matsyendramatam}}/
\end{ava}

%1.48
% Testimonia
% VM 36
%  kṛtvā saṃpuṭitau karau dṛḍhataraṃ baddhvātha padmāsanaṃ
% gāḍhaṃ vakṣasi sannidhāya cibukaṃ dhyānaṃś ca tac cetasi |
% vāraṃ vāram apānam ūrdhvam anilaṃ proccālayan pūritaṃ
% muṇcan prāṇam upaiti bodham atulaṃ śaktiprabhāvān naraḥ ||

\begin{tlg}[hp01_048]
\tl{
\pada{\app{\lem[wit={J8,J10,J17,N17,V1,W4}]{kṛtvā}
	\rdg[wit={V19}]{dhṛtvā}}
	sampuṭitau karau dṛḍhataraṃ baddhvā % baddhvāṃ J17
	\app{\lem[wit={J8,J10,J17,N17,V1,W4}]{tu}
	\rdg[wit={V19}]{ca}} padmāsanaṃ}\\+}
\tl{
\pada{gāḍhaṃ vakṣasi
	\app{\lem[wit={J8,N17,V19}]{sannidhāya}
	\rdg[wit={J10,J17,V1,W4}]{saṃvidhāya}} % Stronger evidence
	cibukaṃ dhyānaṃ
	\app{\lem[wit={J8,J10,J17,N17,V19,W4},alt={ca tac}]{ca ta}
	\rdg[wit={V1}]{tataś}}%
	\app{\lem[wit={J8,N17,V1,V19},alt={cetasi}]{ccetasi}
	\rdg[wit={J10,J17,W4}]{cepsitaṃ}}/}\\+}
\tl{
\pada{vāraṃ vāram apānamūrdhvam anilaṃ
	\app{\lem[resp=emend]{proccālayan}
	\rdg[wit={J8}]{procārayet}
	\rdg[wit={V1}]{protsālayan}
	\rdg[wit={V19}]{protsārayet}
	\rdg[wit={J10,J17,N17}]{prollāsayan}
	\rdg[wit={W4}]{prollāṣayan}}
	\app{\lem[wit={J8,J10,J17,N17}]{pūritaṃ}
	\rdg[wit={V1,V19,W4}]{pūrayan}}}\\+}
\tl{
\pada{\app{\lem[wit={J8pc,V1,W4}]{muñcan prāṇam upaiti}
	\rdg[wit={J8ac,J10,J17,N17,V3}]{muñcat prāṇam upaiti}
	\rdg[wit={V19}]{prāṇaṃ muñcati yāti}}
	bodham atulaṃ śakti%
	\app{\lem[wit={J8,V1}]{prabhāvān naraḥ}
	\rdg[wit={J10,W4}]{prabhāvād ataḥ}
	\rdg[wit={J17}]{prabhāvataḥ} % unmetrical
	\rdg[wit={V19}]{prabhāvodayāt}
	\rdg[wit={N17}]{prabodhodayāt}}//}\\!}
\end{tlg}


%1.49
% Testimonia
% HR 3.40
% padmāsane sthito yogī nāḍīdvāreṣu pūrayet  |
% pūritaṃ dhrīyate yas tu sa mukto nātra saṃśayaḥ||3.40||
% Dhyānabindu Up
% padmāsanasthito yogī nāḍīdvāreṣu pūrayan /
% mārutaṃ kumbhayan yas tu sa mukto nātra saṃśayaḥ // 70//

\begin{tlg}[hp01_049]
\tl{
\pada{\app{\lem[wit={V1,V3,V19,W4}]{padmāsana}
	\rdg[wit={J8,J10,J17,N17}]{padmāsane}}sthito yogī}
\pada{nāḍī\app{\lem[wit={J10,J17,V1,V19,W4}]{dvāreṣu}
	\rdg[wit={J8,N17}]{dvāreṇa}}
	\app{\lem[wit={J8,V1,V19}]{pūrayet}
	\rdg[wit={J17}]{pūraet}
	\rdg[wit={N17}]{pūrayat}
	\rdg[wit={J10,W4}]{pūrayan}}/}\\+}
\tl{
\pada{mārutaṃ
	\app{\lem[wit={V19}]{dhārayet yas tu}
	\rdg[wit={V1}]{pīyate yas tu} % better to conjecture pibate yas tu?
	\rdg[wit={J8ac}]{pīvyate yas tu}
	\rdg[wit={J8pc}]{pibati yastu}
	\rdg[wit={J10,J17,N17,W4}]{yas tu pibati}}} % J17 pibaṃti
\pada{sa mukto nātra saṃśayaḥ//}\\!}
\end{tlg}


\begin{ava}[hp01_050]
\outdent
	\app{\lem[wit={J8,J10,J17,N17,V1,W4}]{atha siṃhāsanam}
	\rdg[wit={V19}]{siṃhāsana yathā}}/
\end{ava}

%1.50
%Testimonia
% Vasiṣṭhasaṃhitā 1.73
% gulphau ca vṛṣaṇasyādhaḥ sīvanyāḥ pārśvayoḥ kṣipet |
% dakṣiṇaṃ savyagulphena dakṣiṇenetaretaram ||
%
% Yogayājñavalkya 3.9
% gulpau ca vṛṣaṇasyādhaḥ sīvanyāḥ pārśvayoḥ kṣipet
% dakṣiṇaṃ savyagulphena dakṣiṇena tathetaram
%
% Sūtasaṃhitā 15.7
% gulphau ca vṛṣaṇasyādhaḥ sīvanyāḥ pārśvayoḥ kṣipet।
% dakṣiṇaṃ savyagulphena vāmaṃ dakṣiṇagulphataḥ।। 7।।

\begin{tlg}[hp01_050]
\tl{
\pada{gulphau ca vṛṣaṇasyādhaḥ} % J17 gulpau, °ādha
\pada{\app{\lem[wit={J8,J10,N17,V1}]{sīvanyāḥ}
	\rdg[wit={J17,W4}]{sīvanyā}
	\rdg[wit={V19}]{sīmanyāḥ}} pārśvayoḥ kṣipet/}\\+} % J17 pārśvayo
\tl{
\pada{\app{\lem[wit={J10,J17,N17,V1,V19,W4}]{dakṣiṇe}
	\rdg[wit={J8}]{dakṣaṇe}} savyagulphaṃ % sarvyagulphaṃn tu J17
	\app{\lem[wit={J8,J10,J17,N17,V1,W4}]{tu}
	\rdg[wit={V19}]{ca}}}
\pada{dakṣagulphaṃ
	\app{\lem[wit={V1,V19}]{ca}
	\rdg[wit={J8,J10,J17,N17,W4}]{tu}}
	\app{\lem[wit={J8,J10,J17,N17,V1,W4}]{savyake}
	\rdg[wit={V19}]{guhyake}}//}\\!}
\end{tlg}


%1.51
% Testimonia
% Vasiṣṭhasaṃhitā 1.74
%  hastau jānau ca saṃsthāpya svāṅgulīś ca prasārya ca |
%  vyāttavaktro nirīkṣeta nāsāgraṃ susamāhitaḥ ||

% Yogayājñavalkya 3.10
%  hastau ca jānvoḥ saṃsthāpya svāṅgulīś ca prasārya ca
%  vyāttavaktro nirīkṣet nāsagraṃ susamāhitaḥ

% Sūtasaṃhitā 15.8
% hastau ca jānvoḥ saṃsthāpya svāṅgulīś ca prasārya ca।
% nāsāgraṃ ca nirīkṣeta bhavet siṃhāsanaṃ hi tat।। 8।।

\begin{tlg}[hp01_051]
\tl{
\pada{hastau
	\app{\lem[wit={J8}]{ca jānvoḥ}
	\rdg[wit={V1}]{tu jānvoḥ}
	\rdg[wit={V19}]{jānvoś ca}
	\rdg[wit={J10,J17,N17,W4}]{tu jānunoḥ}}
	\app{\lem[wit={J8,V1,V19}]{saṃsthāpya}
	\rdg[wit={J10,N17,W4}]{sthāpya}
	\rdg[wit={J17}]{sthāya}}}
\pada{\app{\lem[wit={J8,J10,J17,N17,V1,W4}]{svāṅgulīḥ}
	\rdg[wit={V19}]{aṅgulīḥ}} saṃprasārya ca/}\\+}
\tl{
\pada{\app{\lem[wit={J10,J17,N17,V1}]{vyāttavaktro}
	\rdg[wit={J8}]{vyātavakro}
	\rdg[wit={V19}]{vyālāvaktro}
	\rdg[wit={W4}]{vyāghravaktro}}
	\app{\lem[wit={V1,V19,W4}]{nirīkṣeta}
	\rdg[wit={J8}]{nirīkṣet}
	\rdg[wit={J10,J17,N17}]{nirīkṣyeta}}}
\pada{\app{\lem[wit={V1,N17}]{nāsāgre}
	\rdg[wit={J8,J10,J17,W4}]{nāsāgra}
	\rdg[wit={V19}]{nāsāgraṃ}}
	\app{\lem[wit={J10,J17,N17,V1}]{nyastalocanaḥ}
	\rdg[wit={J8}]{nyastalocanam}
	\rdg[wit={W4}]{nyastalocana}
	\rdg[wit={V19}]{susamāhitaḥ}}//}\\!}
\end{tlg}


%1.52
%Testimonia
% Yogavāsiṣṭha 1.75ab
% siṃhāsanaṃ bhaved etat pūjitaṃ yogibhiḥ sadā ||
% Yogayājñavalkya 3.11
% siṃhāsanaṃ bhaved etat pūjitaṃ yogibhiḥ sadā
% YCM attributes its citation of verses on siṃhāsana to the YY and does not include HP 1.52cd, which affirms that 1.52cd is not from the YY (or VS).
% Haṭharatnāvalī 3.33
% siṃhāsanaṃ bhaved etat sevitaṃ yogibhiḥ sadā ||
% bandhatritayasaṃsthānaṃ kurute cāsanottamam ||


\begin{tlg}[hp01_052]
\tl{
\pada{siṃhāsanaṃ bhaved eta}%
\pada{\app{\lem[wit={J8,J17,N17,V1,V19,W4},alt={pūjitaṃ}]{tpūjitaṃ}
	\rdg[wit={J10}]{pūjita}}
	\app{\lem[wit={J10,J17,N17,V1,W4}]{yogibhiḥ sadā}
	\rdg[wit={J8}]{yogabhiḥ sadā}
	\rdg[wit={V19}]{munipuṅgavaiḥ}}/}\\+}
\tl{
\pada{bandha\app{\lem[wit={J8,J10,J17,N17,W4}]{tritaya}
	\rdg[wit={V1}]{tritīya}
	\rdg[wit={V19}]{trayasya}}sandhānaṃ}
\pada{kurute
	\app{\lem[wit={J8,J10pc,N17}]{cāsanottamam}
	\rdg[wit={V1,V19}]{vāsanottamam}
	\rdg[wit={J10ac,J17}]{sādhanottamam}% J17 tamam
	\rdg[wit={W4}]{sādakottamaḥ}}//}\\!}
\end{tlg}

\pagebreak

\begin{ava}[hp01_053]
  \outdent atha \app{\lem[wit={J8,J10,J17,N17,V1,W4}]{bhadrāsanam} \rdg[wit={V19}]{bhadraṃ}}/
\end{ava}

%1.53
% Testimonia
% VS 1.79
% gulphau ca vṛṣaṇasyādhaḥ sīvanyāḥ pārśvayoḥ kṣipan | ~ Padmasaṃhitā 1.16
% pārśvapādau ca pāṇibhyāṃ dṛḍhaṃ baddhvā suniścalam |
%
% YY 3.11cd-3.12ab
% gulphau ca vṛṣaṇasyādhaḥ sīvanyāḥ pārśvayoḥ kṣipet
% 3.12ab pārśvapādau ca pāṇibhyāṃ dṛḍhaṃ baddhvā suniścalam

\begin{tlg}[hp01_053]
\tl{
\pada{gulphau
	\app{\lem[wit={J8,J10,J17,N17,V1,V19}]{ca}
	\rdg[wit={W4}]{vā}} vṛṣaṇasyādhaḥ} % J17 gulpau, °ādha
\pada{\app{\lem[wit={V1,N17}]{sīvanyāḥ}
	\rdg[wit={J10,W4}]{sīvanyā}
	\rdg[wit={J17}]{sīvanyāṃ}
	\rdg[wit={V19}]{sīmanyāḥ}} pārśvayoḥ kṣipet/}% J17 prārśvayo
\myfn{53ab \emph{gulphau ca ... kṣipet} is a repetition of 50ab; omitted in \getsiglum{J8}.}%
\\+}
\tl{
\pada{\app{\lem[wit={J8,J10,J17,N17,V1,W4}]{pārśva}
	\rdg[wit={V19}]{pārśve}}pādau ca
	\app{\lem[wit={J8,J10,J17,V1,V19,W4}]{pāṇibhyāṃ}
	\rdg[wit={N17}]{pābhyāṃ}}}
\pada{dṛḍhaṃ
	\app{\lem[wit={J8,J10,N17,V1,W4}]{baddhvā}
	\rdg[wit={J17}]{baddhvāṃ}
	\rdg[wit={V19}]{baddhaṃ}}
	\app{\lem[wit={J8,J10,J17,N17,V1,W4}]{suniścalam}
	\rdg[wit={V19}]{suniścitam}}//}\\!}
\end{tlg}


%1.54
% Testimonia
% VS 1.79ef
% bhadrāsanaṃ bhaved etat sarvavyādhiviṣāpaham ||
% YY 3.12cd
% bhadrāsanaṃ bhaved etat sarvavyādhiviṣāpaham

\begin{tlg}[hp01_054]
\tl{
\pada{bhadrāsanaṃ bhaved etat}
\pada{sarvavyādhi\app{\lem[wit={J8,J17,J10,V1pc,W4}]{viṣāpaham}
	\rdg[wit={V1ac,V19}]{vināśanam}
	\rdg[wit={N17}]{vināsanam}}/}\\+}    % V1(3v) correction by first hand!!
\tl{
\pada{gorakṣāsanam ityāhur}
\pada{idaṃ vai
	\app{\lem[wit={J8,J10,N17,V1,V19,W4}]{siddhayoginaḥ}
	\rdg[wit={J17}]{siddhiyogibhiḥ}}//}\\!}
\end{tlg}

%1.55
\begin{tlg}[hp01_055]
\tl{
\pada{eva\app{\lem[wit={J8,J10,J17,V1,V19,W4},alt={āsana}]{māsana}
	\rdg[wit={N17}]{āsane}}bandheṣu}
\pada{yogīndro % yogiṃdro J8,N17
	\app{\lem[wit={J10,J17,N17,V1,W4}]{vigataśramaḥ}
	\rdg[wit={J8pc}]{vijitaśramaḥ}
	\rdg[wit={J8ac}]{vijitaśramāṃ}
	\rdg[wit={V19}]{vijiteṃśramaḥ}}/}\\+}
\tl{
\pada{\app{\lem[wit={J8pc,J10,J17,N17,V19,W4},alt={athābhyasen}]{athābhyase}
	\rdg[wit={W4}]{athābhyāse}
	\rdg[wit={J8ac}]{athābhyāsaṃ}
	\rdg[wit={V1}]{athabhyā[g]e}}%
	\app{\lem[wit={J8,V19,W4},alt={nāḍiśuddhiṃ}]{nnāḍiśuddhiṃ} % ra vipulā (with caesura after the 4th)
	\rdg[wit={J10,J17}]{nāḍīśuddhiṃ} % ma vipulā (but no caesura after 5th)
	\rdg[wit={N17}]{nāḍiśuddhaṃ}
	\rdg[wit={V1}]{nāḍiśvaddhiṃ}}}
\pada{\app{\lem[wit={J8,J10,J17,N17,V1,W4}]{mudrādi}
	\rdg[wit={V19}]{subaddhvā}}pavanakriyām//}\\!} % J17 pavanaṃ kriyām
\end{tlg}


% These verses are in the J10, J17, N17 branch, with J8/V3 also. (but not in V1,V19)
% Testimonia
% DYŚ 42cd-43ab
% kriyāyuktasya siddhiḥ syād akriyasya kathaṃ bhavet |
% na śāstrapāṭhamātreṇa kā cit siddhiḥ prajāyate |

\startaltrecension{I.55+}
\begin{alttlg}[hp01_055_1]
\tl{
\pada{kriyāyuktasya siddhiḥ syād}% siddhi J8,J17
\pada{akriyasya kathaṃ bhavet/} \\+
}\tl{
\pada{na śāstrapāṭhamātreṇa} % J17 sāstra
\pada{yogasiddhiḥ prajāyate//} % siddhi J8
\sgwit{J8,J10,J17,N17,W4}\\!
}\end{alttlg}
%
% Testimonia
% DYŚ 46-47ab
% na veṣadhāraṇaṃ siddheḥ kāraṇaṃ na ca tatkathā|
% kriyaivakāraṇaṃ siddheḥ satyam eva tu sāṃkṛte||46||
% śiśnodarārthaṃ yogasya kathayā veṣadhāriṇaḥ|
% [anuṣṭhānavihīnās tu vañcayanti janān kila ||47||]
%
\begin{alttlg}[hp01_055_2]
\tl{
\pada{na
	\app{\lem[wit={J10,J17,N17}]{veṣadhāraṇaṃ}
	\rdg[wit={J8pc}]{veṣadhāriṇo}
	\rdg[wit={J8ac}]{veṣṭadhāriṇyo}
	\rdg[wit={W4}]{bheṣadhāraṇaṃ}}
	\app{\lem[wit={J10,N17,W4}]{siddheḥ}
	\rdg[wit={J17}]{siddhe}
	\rdg[wit={J8pc}]{siddhiḥ}
	\rdg[wit={J8ac}]{siddhi}}}
\pada{kāraṇaṃ
	\app{\lem[wit={J8pc,J10,J17,N17}]{na ca}
	\rdg[wit={J8ac}]{ca}
	\rdg[wit={W4}]{ca na}}
	\app{\lem[wit={J8,J17,W4}]{tatkathā}
	\rdg[wit={J10,N17}]{tatkathāḥ}}/}\\+
}\tl{
\pada{kriyaiva kāraṇaṃ % kāranaṃ N17
	\app{\lem[wit={J10,J17,N17,W4}]{siddheḥ}
	\rdg[wit={J8}]{siddhi}}}
\pada{satya\app{\lem[wit={J10,N17,W4},alt={etan}]{meta}
	\rdg[wit={J17}]{evan}
	\rdg[wit={N17}]{evaṃ}
	\rdg[wit={J8},post={\unm}]{eva tat}}nna saṃśayaḥ/}\\+
}\tl{
\pada{śiśnodara\app{\lem[wit={J8pc,J10,J17,N17,W4}]{ratāyeha}
	\rdg[wit={J8ac}]{ratāyena}}}
\pada{na
	\app{\lem[resp=emend]{deyā}
	\rdg[wit={J8,J17,W4}]{deyo}
	\rdg[wit={N17}]{deye}
	\rdg[wit={J10}]{dayo}} % read deyaṃ (or deyā -> kriyā)?
	\app{\lem[wit={J8pc,J10,J17,W4}]{veṣa}
	\rdg[wit={J8ac,N17}]{viṣa}}%
	\app{\lem[wit={J8,J10,N17}]{dhāriṇaḥ} % read veśadhāriṇe
	\rdg[wit={J17}]{dhāriṇa}
	\rdg[wit={W4}]{dhāriṇau}}//} \sgwit{J8,J10,J17,N17,W4}\\!
}\end{alttlg}


% Omitted in V1, J10, J17 (included in V19, N17)
% Testimonia
% Vārāhītantra
% mayi bodhī budho svasthe tucho yaṃ viśvabudbudaḥ |
% malīna udito vetti vikalpāvasaraḥ kutaḥ |

\begin{alttlg}[hp01_055_3]
\tl{
\pada{\app{\lem[wit={N17}]{mayi}
	\rdg[wit={V19}]{miyi}}
	\app{\lem[wit={N17}]{bodhāmbudhau}
	\rdg[wit={V19}]{bodhoṃbudhau}} svacche} % L1
\pada{tuccho'yaṃ viśvabudbudaḥ/}\\+
}\tl{
\pada{\app{\lem[wit={V19}]{pralīna}
	\rdg[wit={N17}]{malīna}} udito
	\app{\lem[wit={V19}]{veti}
	\rdg[wit={N17}]{vetti}}}
\pada{\app{\lem[wit={V19}]{vikalpapaṭalaḥ}
	\rdg[wit={N17}]{vikalpāvasaraḥ}} kutaḥ//} \sgwit{N17,V19}\\!
}\end{alttlg}

\pagebreak

%Testimonia
% YCM
% haṭhapradīpikāyām--
% śrutapratītiḥ svagurupratītiḥ, svātmapratītir manaso nirodhaḥ/
% etāni sarvāṇi samuccitāni, matāni dhīrair iha sādhanāni//"

\begin{alttlg}[hp01_055_4]
\tl{
\pada{śruti\app{\lem[resp=emend,postwit={(\getsiglum{L1})},type=alien]{pratītiḥ}\rdg[wit={V19}]{prītaḥ}}
svagurupratītiḥ}\\+
}\tl{
\pada{svātmapratītir manaso'pi bodhaḥ/}\\+ % L1/N5/V19; rodhaḥ J13
}\tl{
\pada{etāni sarvāṇi samuccitāni}\\+ % J13/V19; samuddhṛtāni L1/N5
}\tl{
\pada{matāni dhīrair iha sādhanāni//} \sgwit{V19}\\!
}\end{alttlg}
\endaltrecension


%1.56
% Testimonia
% YCM
% haṭhapradīpikāyām--
% āsanaṃ kumbhakaṃ citraṃ mudrākhyaṃ karaṇaṃ tathā/
% atha nādānusandhānam abhyāsānukrameṇa ca// = 1.56

\begin{tlg}[hp01_056]
\tl{
\pada{āsanaṃ
	\app{\lem[wit={J8,J10,N17,W4},alt={kumbhakaś}]{kumbhaka}
	\rdg[wit={V19}]{kumbhakaṃ}
	\rdg[wit={J17}]{kumbhakaṃś}}%
	\app{\lem[wit={J8ac,J10,J17,N17,V19,W4},alt={citraṃ}]{ścitraṃ}
	\rdg[wit={J8pc}]{citro}}}
\pada{\app{\lem[wit={V19}]{mudrākhyaṃ}
	\rdg[wit={J8,J10,J17,N17,W4}]{mudrādi}}\app{\lem[wit={V19}]{karaṇaṃ tathā}
	\rdg[wit={J8pc,J10,J17,N17,W4}]{karaṇāni ca}
	\rdg[wit={J8ac}]{pavanakriyā}}/}\\+}
\tl{
\pada{atha nādānusandhāna}%
\pada{\app{\lem[wit={J8pc,J10,J17},alt={°nam abhyāsā°}]{m abhyāsā}
	\rdg[wit={V19}]{nam abhyāsyā}
	\rdg[wit={J8ac}]{nasyābhyāsā}
	\rdg[wit={W4},post={\unm}]{naṃ syād abhyāsā}
	\rdg[wit={N17},post={\unm}]{na syād abhyāsā}}
	nu\app{\lem[wit={J8,J10,J17,N17,W4}]{kramo haṭhe}
	\rdg[wit={V19}]{krameṇa tu}}//} \NotIn{V1}\\!}
\end{tlg}


%1.57
% testimonia
% VM 37
% brahmacārī mitāhārī yogī yogaparāyaṇaḥ| %
% abdād ūrdhvaṃ bhavet siddho nātra kāryā vicāraṇā||

\begin{tlg}[hp01_057]
\tl{
\pada{brahmacārī
	\app{\lem[wit={J8,J10,J17,N17,V1,W4}]{mitāhārī}
	\rdg[wit={V19}]{mitāhāro}}}
\pada{\app{\lem[wit={V1,V19}]{yogī}
	\rdg[wit={J8,J10,J17,N17,W4}]{tyāgī}} yogaparāyaṇaḥ/}\\+}
\tl{
\pada{abdādūrdhvaṃ % J8ac,J10,N17 om. anusvāra
	bhave\app{\lem[wit={J8ac,V1,V19},alt={siddho}]{tsiddho}
	\rdg[wit={J8pc,N17,W4}]{siddhir}
	\rdg[wit={J10}]{siddhīn}
	\rdg[wit={J17}]{siddhān}}}
\pada{nātra
	\app{\lem[wit={J10,J17,N17,V1,V19,W4}]{kāryā}
	\rdg[wit={J8}]{kārya}}
	\app{\lem[wit={J8,N17,V1,V19,W4}]{vicāraṇā}
	\rdg[wit={J10,J17}]{vicāraṇāt}}//}\\!}
\end{tlg}


%1.58
% testimonia
% Gorakṣaśataka (original)
% susnigdhamadhurāhāraś caturthāṃśavivarjitaḥ |
% bhujyate śivasaṃprītyai mitāhāraḥ sa ucyate ||

\begin{tlg}[hp01_058]
\tl{
\pada{susnigdhamadhu\app{\lem[wit={J10,N17,V1,V19,W4},alt={āhāraś}]{rāhāra}
	\rdg[wit={J17}]{āhāraḥ}
	\rdg[wit={J8}]{āhāraṃ}}}%
\pada{\app{\lem[wit={J8,J10,V1,V19,W4},alt={caturthāṃśa}]{ścaturthāṃśa}%
	\rdg[wit={N17}]{caturthāśa}
	\rdg[wit={J17},post={\unm}]{caturthāsvāda}}%
	\app{\lem[wit={J10,J17,N17,V1,V19}]{vivarjitaḥ}
	\rdg[wit={J8}]{vivarjitam}
	\rdg[wit={W4}]{kavarjitaḥ}}/}\\+}
\tl{
\pada{\app{\lem[wit={J10,J17,N17,V1,V19,W4}]{bhujyate}
	\rdg[wit={J8}]{bhuṃjyate}}
	\app{\lem[wit={J8ac,J10,J17,V19,W4}]{śivasaṃprītyai}
	\rdg[wit={V1}]{śivasaṃpritya}
	\rdg[wit={J8pc}]{surasaṃprītyai}
	\rdg[wit={N17}]{surasaṃprī\,..air}}}
\pada{\app{\lem[wit={V1,J8,J10,V19,W4}]{mitāhāraḥ}
	\rdg[wit={J17}]{mityāhāraḥ}
	\rdg[wit={N17}]{mitāhārī}}
	\app{\lem[wit={J8,J10,J17,N17,V19,W4}]{sa ucyate}
	\rdg[wit={V1,V3}]{samucyate}}//}\\!}
\end{tlg}



%1.59
% Testimonia
% YCM
% haṭhapradīpikāyām—
% kaṭvamlatīkṣṇalavaṇoṣṇaharītaśāka-
% sauvīratailatilasarṣapamatsyamadyam/
% ajādimāṃsadadhitakrakulatthakola-
% piṇyākahiṅgulaśunādyam apathyam āhuḥ//

\begin{tlg}[hp01_059]
\tl{
\pada{\app{\lem[wit={J8pc,J10,V19,W4}]{kaṭvamla}% better; usually kaṭvamla in āyurvedic texts
	\rdg[wit={J8ac,J17,V1}]{kaṭvāmla}
	\rdg[wit={N17}]{ka .. mla}}%
	\app{\lem[wit={J8,J10,J17,N17,V1,W4}]{tīkṣṇa} % tīkṣaṇa J17
	\rdg[wit={V19}]{tikta}}lavaṇoṣṇa%
	\app{\lem[wit={J8pc,J10,N17,V1,V19,W4}]{harīta}
	\rdg[wit={J8ac,J17}]{harita}}%
	\app{\lem[wit={J8,V1}]{śākaṃ}
	\rdg[wit={N17,V19,W4}]{śāka}
	\rdg[wit={J10,J17}]{sāka}}}\\+}
\tl{
\pada{sauvīra\myfn{\emph{sauvīra} is glossed as \emph{kāṃjī} in \getsiglum{J8}. Cf. Brahmānanda's comm.: \emph{sauvīraṃ kāñjikam}.}%
	taila\app{\lem[wit={J8,J10,J17,N17,V19,W4}]{tila}
	\rdg[wit={V1},alt={\illeg}]{}}sarṣapa%
	\app{\lem[wit={J8,J10,J17,N17,V19,W4}]{matsyamadyam}
	\rdg[wit={V1}]{tsyamaghaṃ}}/}\\+}
\tl{
\pada{\app{\lem[wit={J10}]{ājāvi}
	\rdg[wit={J8,J17,N17,V1,V19}]{ajāvi}
	\rdg[wit={W4}]{ajādi}}māṃsadadhitakra%
	\app{\lem[wit={J10,J17,W4}]{kulattha}
	\rdg[wit={J8,V19}]{kulatha}
	\rdg[wit={V1}]{kulatthya}
	\rdg[wit={N17}]{kulaccha}}%
	\app{\lem[wit={J8,N17,V19,W4}]{kola}
	\rdg[wit={J10,J17}]{kodra}
	\rdg[wit={V1}]{koṣṇā}}% kolyā?
\myfn{\emph{kola} is glossed as \emph{bhaṭavāsa} in \getsiglum{J8}.}-}\\+}
\tl{
\pada{\app{\lem[wit={J17,N17,V1,V19,W4}]{piṇyāka}
	\rdg[wit={J10}]{piṃṇyāka}
	\rdg[wit={J8}]{pinnāka}}%
\myfn{\emph{piṇyāka} is glossed as \emph{ṣalī} in \getsiglum{J8}.}%
	hiṅgulaśunādyam% lasunā J8,J10,J17,N17,V1,V19,W4
	apathyamāhuḥ//}\\!}
\end{tlg}

\pagebreak

%1.60
% Testimonia
% YCM
% bhojanam ahitaṃ vidyāt punar <apy> uṣṇīkṛtaṃ tathā rukṣam/
% atilavaṇaṃ sapalaṃ vā prasitaṃ śākotkaṭaṃ varjyam// = 1.60
%
%HSC
% bhojanam ahitaṃ vidyāt punar uṣṇīkṛtaṃ rūkṣaṃ |
%atilavaṇādikayuktaṃ kadaśanaśākotkaṭaṃ duṣṭaṃ ||

\begin{tlg}[hp01_060]
\tl{
\pada{bhojanam ahitaṃ
	\app{\lem[wit={J8pc,J17,V19,W4},alt={vidyāt}]{vidyā}
	\rdg[wit={V1}]{vidyā}
	\rdg[wit={J8ac,J10,N17}]{viṃdyāt}}% =A1
	tpunara\app{\lem[wit={J8,J10,J17,N17,V1,W4},alt={apy}]{pyu} % prunar J17
	\rdg[wit={V19},alt={\om}]{}}%
	\app{\lem[wit={J10,J17,N17,V1,W4},alt={uṣṇīkṛtaṃ}]{ṣṇīkṛtaṃ}
	\rdg[wit={J8ac}]{uśnakrataṃ}
	\rdg[wit={J8pc}]{uśnaṃ kṛtaṃ}
	\rdg[wit={V19}]{uṣṇībhūtam}}
	\app{\lem[wit={J8,J10,J17,N17,W4}]{rūkṣam} % rukṣaṃ J17
	\rdg[wit={V1}]{rūkṣa}
	\rdg[wit={V19}]{apramitaṃ}}/}\\+}
\tl{
\pada{\app{\lem[wit={J8,J10,J17,N17,V1,W4}]{atilavaṇaṃ}
	\rdg[wit={V19}]{atilavaṇa}}
	\app{\lem[wit={J17}]{tilapiṇḍaṃ}
	\rdg[wit={J8pc,J10,N17,W4}]{tilapiṇḍa}
	\rdg[wit={J8ac}]{tilaṃ piṇḍa}
	\rdg[wit={V1}]{dyuṣṇataṃ} % unclear, check! ghuṣṭānnaṃ? dyumnataṃ?
	\rdg[wit={V19}]{savapalala}}
	\app{\lem[wit={V1,V19}]{kadaśana}
	\rdg[wit={J8ac,J10,J17,N17,W4}]{kadaśanaṃ} % unmetrical?
	\rdg[wit={J8pc}]{kamaśana}}%
	\app{\lem[wit={J10,J17,V19,W4}]{śākotkaṭaṃ}
	\rdg[wit={J8pc}]{śākotkaṭa}
	\rdg[wit={J8ac,V1}]{śokātkaṭa}
	\rdg[wit={N17},alt={\om}]{}}
	\app{\lem[wit={V3,V19}]{varjyam}
	\rdg[wit={J10,J17}]{varjjaṃ}
	\rdg[wit={J8,W4}]{varjitaṃ}
	\rdg[wit={V1},alt={\illeg}]{}
	\rdg[wit={N17},alt={\om}]{}}//}\\!}
\end{tlg}


\begin{ava}[hp01_061]
\outdent
	\app{\lem[wit={J10ac,J17,V1}]{tathā}
	\rdg[wit={J8}]{tathā hi}
	\rdg[wit={J10pc,W4}]{tathā ca}
	\rdg[wit={N17,V19},alt={\om}]{}}
	\app{\lem[wit={J8,J10,J17,V1,W4}]{gorakṣavacanam}
	\rdg[wit={N17,V19},alt={\om}]{}}/
\end{ava}

%1.61
% Testimonia
% Amaraugha (short recension)
% vahnistrīpathisevānām ādau varjanam ācaret ||35||
%
\begin{tlg}[hp01_061]
\tl{
\myfn{\getsiglum{V1} has \emph{tailāmlāloṇītīni kālikā bhā i}(?) before this verse.}\pada{\app{\lem[wit={J8,J10,J17,N17,V1,V19,W4},alt={varjayed}]{varjaye}
	\rdg[wit={J8,N17}]{varjaye}
	\rdg[wit={N17}]{.. [y]ed dūre}
}\app{\lem[wit={J8,J10,J17,N17,V19},alt={durjana}]{ddurjana}
	\rdg[wit={V1}]{tarjana}}%
	\app{\lem[wit={J10,J17,N17,V19,W4}]{prītiṃ}
	\rdg[wit={J8}]{prīti}
	\rdg[wit={V1}]{prātaṃ}}} % read prāntaṃ?
\pada{\app{\lem[wit={J8,J10,J17,N17,V1,W4}]{vahnistrī}
	\rdg[wit={V19}]{vastrī}}%
	\app{\lem[wit={J8pc,J10,J17,N17,V1,W4}]{patha}
	\rdg[wit={J8ac}]{pathya}
	\rdg[wit={V19}]{madhu}}sevanam/}\\+}
\tl{
\pada{\app{\lem[wit={N17,W4}]{prātaḥsnāno}
	\rdg[wit={J10,J17,V1,V19}]{prātasnāno}
	\rdg[wit={J8}]{prātaśnāno}}pavāsādi}
\pada{kāya\app{\lem[wit={J10,N17,V1,V19,W4}]{kleśādikaṃ}
	\rdg[wit={J17}]{kleśādikas}
	\rdg[wit={J8}]{kleśavidhiṃ}}
	\app{\lem[wit={J8,J10,J17,N17,V1,W4}]{tathā}
	\rdg[wit={V19}]{yathā}}//}\\!}
\end{tlg}

%1.62
\begin{tlg}[hp01_062]
\tl{
\pada{\app{\lem[wit={J8,J10,J17,N17,V1,W4}]{godhūma}
	\rdg[wit={V19}]{godhūmā}}śāli%
	\app{\lem[wit={J8,J10,J17,N17,V19,W4}]{yava}
	\rdg[wit={V1}]{java}}ṣaṣṭika%
	\app{\lem[wit={J8pc,J10,J17,N17,V19,W4}]{śobhanānnaṃ}
	\rdg[wit={J8ac}]{śobhanānna}
	\rdg[wit={V1}]{śobhanānnānī}}}\\+}
\tl{
\pada{kṣīrājya\app{\lem[wit={J10,J17,N17,W4}]{khaṇḍa}
	\rdg[wit={J8,V1}]{ṣaṃḍa}
	\rdg[wit={V19}]{maṃḍa}}nava\app{\lem[wit={J8,J10,J17,N17,V19,W4}]{nīta}
	\rdg[wit={V1}]{nīti}}\app{\lem[wit={J8,J10,J17,N17,V19,W4}]{sitā}
	\rdg[wit={V1}]{śītā}}\app{\lem[wit={J8,J10,J17,V1,V19,W4}]{madhūni}\rdg[wit={N17}]{madhūnī}}/}\\+}
\tl{
\pada{\app{\lem[wit={J8,J10,J17,N17,V1,W4}]{śuṇṭhī} % suṃṭhī J8,J17,N17,V1,W4
	\rdg[wit={V19}]{suṭhī}}\app{\lem[wit={J8,V19}]{paṭolaka}
	\rdg[wit={V1}]{paṭolika}
	\rdg[wit={J10,J17,N17,W4}]{paṭola}}%
\myfn{\emph{paṭolaka} is glossed as \emph{palavala} in \getsiglum{J8}. Cf. Brahmānanda's comm.: \emph{paṭolakaphalaṃ paravara iti bhāṣāyāṃ prasiddhaṃ}.}%
	\app{\lem[wit={J8}]{phalādi ca}
	\rdg[wit={V19}]{phalādika}
	\rdg[wit={J10,N17}]{phalakādi ca}
	\rdg[wit={J17}]{phalakādiś ca}
	\rdg[wit={W4}]{phalakādika}
	\rdg[wit={V1}]{phipalādika}}
	\app{\lem[wit={J8,V1,V19}]{pañcaśākaṃ}
	\rdg[wit={W4}]{pañcaśāka}
	\rdg[wit={J10,J17}]{śākabhuktaṃ}
	\rdg[wit={N17}]{śākamuktaṃ}}} \\+}
\tl{
\pada{\app{\lem[wit={J10,J17,N17,V1,V19,W4}]{mudgādi}
	\rdg[wit={J8}]{mudgā}}
	\app{\lem[wit={J8,J10,J17,N17,V1,W4},alt={divyam}]{divya}
	\rdg[wit={V19}]{cālpam}}mudakaṃ
	\app{\lem[wit={J10,N17,V1,V19,W4}]{ca}
	\rdg[wit={J17}]{ja}
	\rdg[wit={J8},alt={\om}]{}}
	\app{\lem[wit={J17,V1,V19}]{munīndra}
	\rdg[wit={J8,J10,N17,W4}]{yamīndra}}pathyam//}\\!}
\end{tlg}


\startaltrecension{I.62+}
\begin{alttlg}[hp01_062_1]
  \tl{
    \pada{\app{\lem[wit={W4}]{kṣīraparṇī}\rdg[wit={J8}]{kṣīravarṇī}} ca
      \app{\lem[wit={W4}]{jīvantī}\rdg[wit={J8}]{jaivantī}}}
    \pada{\app{\lem[resp=emend]{matsyākṣī}
	\rdg[wit={J8}]{matsāṣī}
	\rdg[wit={W4}]{matsyākī}} ca punarnavā}/\\+}
  \tl{
    \pada{\app{\lem[wit={W4}]{meghanādā ca pañcaitāḥ}
	\rdg[wit={J8}]{meghanādīti pañcaite}}}
    \pada{\app{\lem[wit={W4}]{śākasaṃjñāḥ prakīrtitāḥ}
	\rdg[wit={J8}]{śākanāma prakīrtitā}}//} \sgwit{J8,W4}\\!}
\end{alttlg}
\endaltrecension

\pagebreak

%1.63
\begin{tlg}[hp01_063]
\tl{
\pada{\app{\lem[wit={V1}]{mṛṣṭaṃ}
	\rdg[wit={J8}]{miṣṭaṃ}
	\rdg[wit={J10,J17,N17,V19,W4}]{iṣṭaṃ}}
	\app{\lem[wit={J10,V1}]{sumadhuraṃ}
	\rdg[wit={J8,J17,N17,V19,W4}]{samadhuraṃ}} % samadhūraṃ J17
snigdhaṃ gavyaṃ dhātuprapoṣaṇam/}\\+}
\tl{
\pada{mano\app{\lem[wit={J8,J10,J17,N17,V19,W4}]{'bhilaṣitaṃ}
	\rdg[wit={V1}]{bhilāṣitaṃ}}
	\app{\lem[wit={J10,J17,N17,V19,W4}]{yogyaṃ}
	\rdg[wit={J8}]{yonyaṃ}
	\rdg[wit={V1}]{bhojyaṃ}}
	yogī
	\app{\lem[wit={J8,J17,N17,V1,V19,W4},alt={bhojanam}]{bhojana}
	\rdg[wit={J10}]{bhojanasam}}mācaret//}\\!}
\end{tlg}


%1.64
%testimonia
%DYS 40
% yuvāvastho ’pi vṛddho vā vyādhito vā śanaiḥ śanaiḥ |
% abhyāsāt siddhim āpnoti yoge sarvo ’py atandritaḥ ||40||

\begin{tlg}[hp01_064]
\tl{
\pada{yuvā vṛddho'tivṛddho vā}
\pada{vyādhito
	\app{\lem[wit={J8,V1,V19}]{durbalo'pi vā}
	\rdg[wit={J10,J17,N17,W4}]{durbalas tathā}}/}\\+}
\tl{
\pada{abhyāsāt siddhimāpnoti}
\pada{\app{\lem[wit={J10,J17,N17,V19,W4}]{sarvayogeṣv atandritaḥ} % ḥ om. J17, yogaṣv W4
	\rdg[wit={J8}]{sarvayogeṣu tandritaḥ}
	\rdg[wit={V1}]{sarvaṃ yogī yatendriyaḥ}}//}\\!}
\end{tlg}


\begin{tlg}[hp01_065]
\tl{
\pada{\app{\lem[wit={J8,J10,J17,N17,V1,W4}]{pīṭhāni}
	\rdg[wit={V19}]{pīṭhādi}}
	\app{\lem[wit={J8,V1,V19},alt={kumbhakāś}]{kumbhakā}
	\rdg[wit={J10,J17,N17,W4}]{kumbhakaś}}%
	\app{\lem[wit={J8ac,V1,V19},alt={citrā}]{ścitrā}
	\rdg[wit={J8pc,J10,J17,N17,W4}]{citraṃ}}}
\pada{\app{\lem[wit={J8pc,J10,J17,N17,V1,W4}]{mudrādi}
	\rdg[wit={J8ac,V19}]{divyāni}}karaṇāni ca/}\\+}
\tl{
\pada{\app{\lem[wit={J8,J10,J17,N17,W4}]{sarvāṇy api}
	\rdg[wit={V1}]{sarvo'pi ca}
	\rdg[wit={V19}]{sarvo 'pi hi}}
	\app{\lem[wit={J8pc,J10,J17,N17,V1,W4}]{haṭhābhyāse}
	\rdg[wit={J8ac}]{haṭhābhyāso}
	\rdg[wit={V19}]{haṭhābhyāsād}}}
\pada{rājayoga\app{\lem[wit={J10,J17,N17,V1pc}]{phalāvadhi}
	\rdg[wit={J8,W4}]{phalāvadhiḥ}
	\rdg[wit={V1ac}]{yugāvadhi}
	\rdg[wit={V19}]{prasiddhaye}}//}\\!}
\end{tlg}


\begin{tp}
\outdent
	\app{\lem[wit={V1}]{iti svātmārāma}
	\rdg[wit={J8}]{iti śrīsvātmārāma}
	\rdg[wit={N17}]{iti śrīātmārāma}
	\rdg[wit={J10,W4}]{iti ātmārāma} % ātma° W4
	\rdg[wit={J17}]{ity ātmārāma}
	\rdg[wit={V19}]{iti}}%
	\app{\lem[wit={J10,J17,N17,V1,W4}]{yogīndra}
	\rdg[wit={J8}]{yogendra}
	\rdg[wit={V19},alt={\om}]{}}%
	\app{\lem[wit={J8,J10,J17,N17,V1,W4}]{viracitāyāṃ}
	\rdg[wit={V19},alt={\om}]{}}
	haṭhapradīpikāyāṃ
	\app{\lem[wit={V1}]{prathamo'dhyāyaḥ}
	\rdg[wit={J8,V19}]{prathamopadeśaḥ}
	\rdg[wit={J10,J17,N17,W4}]{prathama upadeśaḥ}}//
\end{tp}

\end{ekdosis}\end{otherlanguage}\end{document}
