\documentclass[10pt]{memoir}
\setstocksize{220mm}{155mm} 	        
\settrimmedsize{220mm}{155mm}{*}	
\settypeblocksize{170mm}{116mm}{*}	
\setlrmargins{18mm}{*}{*}
\setulmargins{*}{*}{1.2}
% \setlength{\headheight}{5pt}
\checkandfixthelayout[lines]
\linespread{1.16}

\setlength{\footmarkwidth}{1.3em}
\setlength{\footmarksep}{0em}
\setlength{\footparindent}{1.3em}
\footmarkstyle{\textsuperscript{#1} }
\usepackage{fnpos}
\makeFNbottom

\usepackage[teiexport=tidy,poetry=verse]{ekdosis}
\usepackage{sanskrit-poetry}

\usepackage[english]{babel}
\usepackage{babel-iast,xparse,xcolor}
\babelfont[iast]{rm}[Renderer=Harfbuzz, Scale=1.5]{AdishilaSan}
\babelfont[english]{rm}[Scale=0.9]{Adobe Text Pro}
\babeltags{dev = iast}
\babeltags{eng = english}

\SetHooks{
	lemmastyle=\bfseries,
	refnumstyle=\selectlanguage{english}\color{blue}\bfseries, 
	}
\newif\ifinapparatus
\DeclareApparatus{default}[
	lang=english,
	sep = {] },
	delim=\hskip 0.75em,
	rule=none,
	]
\DeclareApparatus{notes}[
	lang=english,
	sep = {},
	delim=\hskip 0.75em,
	rule=\rule{0.7in}{0.4pt},
	]

\DeclareShorthand{conj}{\texteng{\emph{conj.}}}{ego}
\DeclareShorthand{emend}{\texteng{\emph{em.}}}{ego}

\setlength{\vrightskip}{-10pt}
\setlength{\vgap}{3mm}
\verselinenumfont{\footnotesize\selectlanguage{english}\normalfont}




%%%%%%%%%%%%%%%%%%%% THE  MSS         %%%%%%%%%%%%%%%%%%%%%%%%%%%

%%% Versions
\DeclareWitness{Vu}{\selectlanguage{english}Vulg}{Vulgate, i.e. Brahmānanda's version}[]           
\DeclareWitness{X}{\selectlanguage{english}X}{TenChapter Version, Jodhpur 02228 and 02225 (ed. Lonavla)}[]
\DeclareWitness{Six}{\selectlanguage{english}Ṣ}{SixChapterVersion, ``6ChapterHPms'', fragment of enlarged text, Jodhpur}[]
% Mss. in Geographical Groups
%%%% Varanasi mss (Sampūrṇānanda mss). V1 is Important
\DeclareWitness{V1}{\selectlanguage{english}V\textsubscript{1}}{Sampurnananda Library Sarasvati Bhavan 30109}[]
        \DeclareHand{V1ac}{V1}{\selectlanguage{english}V\rlap{\textsubscript{1}}\textsuperscript{ac}}[] % added by MD
        \DeclareHand{V1pc}{V1}{\selectlanguage{english}V\rlap{\textsubscript{1}}\textsuperscript{pc}}[] % added by MD
\DeclareWitness{V2}{\selectlanguage{english}V\textsubscript{2}}{Sampurnananda Library Sarasvati Bhavan 29869}[]
\DeclareWitness{V3}{\selectlanguage{english}V\textsubscript{3}}{Sampurnananda Library Sarasvati Bhavan 29899}[]
\DeclareWitness{V4}{\selectlanguage{english}V\textsubscript{4}}{Sampurnananda Library Sarasvati Bhavan 29937}[]
\DeclareWitness{V5}{\selectlanguage{english}V\textsubscript{5}}{Sampurnananda Library Sarasvati Bhavan 29938}[]
\DeclareWitness{V6}{\selectlanguage{english}V\textsubscript{6}}{Sampurnananda Library Sarasvati Bhavan 29991}[]
\DeclareWitness{V8}{\selectlanguage{english}V\textsubscript{8}}{Sampurnananda Library Sarasvati Bhavan 30014}[]
\DeclareWitness{V11}{\selectlanguage{english}V\textsubscript{11}}{Sampurnananda Library Sarasvati Bhavan 30029}[]
\DeclareWitness{V12}{\selectlanguage{english}V\textsubscript{12}}{Sampurnananda Library Sarasvati Bhavan 30030}[]
\DeclareWitness{V13}{\selectlanguage{english}V\textsubscript{13}}{Sampurnananda Library Sarasvati Bhavan 30031}[]
\DeclareWitness{V14}{\selectlanguage{english}V\textsubscript{14}}{Sampurnananda Library Sarasvati Bhavan 30050}[]
\DeclareWitness{V15}{\selectlanguage{english}V\textsubscript{15}}{Sampurnananda Library Sarasvati Bhavan 30051}[]
\DeclareWitness{V15pc}{\selectlanguage{english}V\rlap{\textsubscript{15}}\textsuperscript{pc}\space}{}[]
\DeclareWitness{V16}{\selectlanguage{english}V\textsubscript{16}}{Sampurnananda Library Sarasvati Bhavan 30052}[]
\DeclareWitness{V17}{\selectlanguage{english}V\textsubscript{17}}{Sampurnananda Library Sarasvati Bhavan 30053}[] % added by MD
\DeclareWitness{V16pc}{\selectlanguage{english}V\rlap{\textsubscript{16}}\textsuperscript{pc}\space}{}[]
\DeclareWitness{V18}{\selectlanguage{english}V\textsubscript{18}}{Sampurnananda Library Sarasvati Bhavan 30064}[]
\DeclareWitness{V19}{\selectlanguage{english}V\textsubscript{19}}{Sampurnananda Library Sarasvati Bhavan 30069}[]
\DeclareWitness{V21}{\selectlanguage{english}V\textsubscript{21}}{Sampurnananda Library Sarasvati Bhavan 30104}[]
\DeclareWitness{V22}{\selectlanguage{english}V\textsubscript{22}}{Sampurnananda Library Sarasvati Bhavan 30110}[]
\DeclareWitness{V25}{\selectlanguage{english}V\textsubscript{25}}{Sampurnananda Library Sarasvati Bhavan 30122}[]
\DeclareWitness{V26}{\selectlanguage{english}V\textsubscript{26}}{Sampurnananda Library Sarasvati Bhavan 30123}[]
\DeclareWitness{V28}{\selectlanguage{english}V\textsubscript{28}}{Sampurnananda Library Sarasvati Bhavan 30136}[]
\DeclareWitness{W2}{\selectlanguage{english}W\textsubscript{2}}{Wai ??}[]
\DeclareWitness{W4}{\selectlanguage{english}W\textsubscript{4}}{Wai 399-6171}[]

%%%%%%%%%%%%%%%%%%%%%%%%%%%%%%%%%
%%% Jammu & Kaschmir
\DeclareWitness{K1}{\selectlanguage{english}K\textsubscript{1}}{Raghunātha Temple Library 4383}[settlement=Jammu]
        \DeclareWitness{K1ac}{\selectlanguage{english}K\rlap{\textsubscript{1}}\textsuperscript{ac}\space}{}[]
        \DeclareWitness{K1pc}{\selectlanguage{english}K\rlap{\textsubscript{1}}\textsuperscript{pc}\space}{}[]
\DeclareWitness{K3}{\selectlanguage{english}K\textsubscript{3}}{Privat collection}
\DeclareWitness{L1}{\selectlanguage{english}L\textsubscript{1}}{SOAS RE 43454}[settlement=Jammu]
% More details? Catalogue number? L1 And C1 very close (and come from same region)
%%%%%%%%%%%%%%%%%%%%%%%%%%%%%%%%
% Jodhpur
% J10 is important
\DeclareWitness{J10}{\selectlanguage{english}J\textsubscript{10}}{MSPP Jodhpur 2230}[]
        \DeclareHand{J10ac}{J10}{\selectlanguage{english}J\rlap{\textsubscript{10}}\textsuperscript{ac}}[] % modified by MD
        \DeclareHand{J10pc}{J10}{\selectlanguage{english}J\rlap{\textsubscript{10}}\textsuperscript{pc}}[] % modified by MD
\DeclareWitness{J1}{\selectlanguage{english}J\textsubscript{1}}{Jodhpur 02231}[]
\DeclareWitness{J2}{\selectlanguage{english}J\textsubscript{2}}{Jodhpur 02232}[]   
\DeclareWitness{J3}{\selectlanguage{english}J\textsubscript{3}}{Jodhpur 02233}[]
\DeclareWitness{J4}{\selectlanguage{english}J\textsubscript{4}}{Jodhpur 02234}[]
        \DeclareWitness{J4ac}{\selectlanguage{english}J\rlap{\textsubscript{4}}\textsuperscript{ac}\space}{MSPP Jodhpur 02234}[]
        \DeclareWitness{J4pc}{\selectlanguage{english}J\rlap{\textsubscript{4}}\textsuperscript{pc}\space}{MSPP Jodhpur 02234}[]
\DeclareWitness{J5}{\selectlanguage{english}J\textsubscript{5}}{Jodhpur 02235}[]  % 4 chapters, 34 jpgs,   long colophon, missing lines in the beginning.
\DeclareWitness{J6}{\selectlanguage{english}J\textsubscript{6}}{Jodhpur 02237}[]  % 4 chapters, 41 jpgs
%\DeclareWitness{J6ac}{\selectlanguage{english}J\rlap{\textsubscript{6}}\textsubscript{ac}}{Jodhpur 02237}[]  % 4 chapters, 49 jpgs,   1st folio: idaṃ gulābarāyasya
% tulasīrāmaśarmmaṇaḥ putrasya pustakaṃ ...        End: iti śrīsahajānandasantānacintāmaṇisvātmārāmaviracitāyāṃ ..
% saṃvat 1802   (more consistent text)
%\DeclareWitness{J6pc}{\selectlanguage{english}J\rlap{\textsubscript{6}}\textsubscript{pc}}{Jodhpur 02237}[] 
\DeclareWitness{J7}{\selectlanguage{english}J\textsubscript{7}}{Jodhpur 02241}[]  % 4 chapters, 41 jpgs
\DeclareWitness{J8}{\selectlanguage{english}J\textsubscript{8}}{Jodhpur 23709}[]  % 4 chapters,  87 jpgs.   saṃvat 1724
\DeclareHand{J8ac}{J8}{\selectlanguage{english}J\rlap{\textsubscript{8}}\textsuperscript{ac}}[]  % changed by MD
\DeclareHand{J8pc}{J8}{\selectlanguage{english}J\rlap{\textsubscript{8}}\textsuperscript{pc}}[]  % changed by MD
\DeclareWitness{J9}{\selectlanguage{english}J\textsubscript{9}}{Jodhpur 02224}[]  %  fragment, 20 jpgs.
\DeclareWitness{J11}{\selectlanguage{english}J\textsubscript{11}}{Jodhpur 23532}[]
        \DeclareHand{J11ac}{J11}{\selectlanguage{english}J\rlap{\textsubscript{11}}\textsuperscript{ac}}[] % added by MD
        \DeclareHand{J11pc}{J11}{\selectlanguage{english}J\rlap{\textsubscript{11}}\textsuperscript{pc}}[] % added by MD
\DeclareWitness{J12}{\selectlanguage{english}J\textsubscript{12}}{Jodhpur 18552}[] 
\DeclareWitness{J13}{\selectlanguage{english}J\textsubscript{13}}{Jodhpur 02229}[]  %  5 chapters, 93 jpgs.
\DeclareWitness{J14}{\selectlanguage{english}J\textsubscript{14}}{Jodhpur 02239}[]  %  4 chapters
\DeclareWitness{J15}{\selectlanguage{english}J\textsubscript{15}}{Jodhpur 9732A}[]
\DeclareWitness{J16}{\selectlanguage{english}J\textsubscript{16}}{Jodhpur 9732B}[]
\DeclareWitness{J17}{\selectlanguage{english}J\textsubscript{17}}{Jodhpur 3013}[]
% Haṭhapradīpikā with (non-Sanskrit) Bhāṣya RORI Jodhpur ACC.NO.18552
%  Haṭhapradīpikā with (non-Sanskrit) commentary, RORI Alwar 952, 4 chapters,  colophon of the comm:
% iti śrīlāhorīmiśravrajabhūṣanaviracitāyāṃ bhāvārthadīpikāyāṃ caturthodhyāya ..    
%  Haṭhapradīpikā (5 chapter) MSPP Jodhpur ACC.NO.02229/

%%%%%%%%%%        Bodleian, Oxford
\DeclareWitness{B1}{\selectlanguage{english}B\textsubscript{1}}{Bodleian Library No. d.457(8)}[settlement=Oxford]
\DeclareWitness{B2}{\selectlanguage{english}B\textsubscript{2}}{Bodleian Library No. d.458(1)}[settlement=Oxford]
\DeclareWitness{B3}{\selectlanguage{english}B\textsubscript{3}}{Bodleian Library No. d.458(9)}[settlement=Oxford]

%%%%%%%%%%%   Chandigarh
\DeclareWitness{C1}{\selectlanguage{english}C\textsubscript{1}}{Lalchand M-2080}[]%L1 And C1 very close (and come from same region)
\DeclareWitness{C2}{\selectlanguage{english}C\textsubscript{2}}{Lalchand M-6065}[]
\DeclareWitness{C3}{\selectlanguage{english}C\textsubscript{3}}{Lalchand M-1293}[]
\DeclareWitness{C4}{\selectlanguage{english}C\textsubscript{4}}{Lalchand M-2081}[]
\DeclareWitness{C4ac}{\selectlanguage{english}C\rlap{\textsubscript{4}}\textsuperscript{ac}\space}{}[]
\DeclareWitness{C4pc}{\selectlanguage{english}C\rlap{\textsubscript{4}}\textsuperscript{pc}\space}{}[]
\DeclareWitness{C5}{\selectlanguage{english}C\textsubscript{5}}{Lalchand M-2082}[]%doesn't have chapter 1
\DeclareWitness{C6}{\selectlanguage{english}C\textsubscript{6}}{Lalchand M-2089}[]
\DeclareWitness{C7}{\selectlanguage{english}C\textsubscript{7}}{Lalchand M-6494}[]
\DeclareWitness{C8}{\selectlanguage{english}C\textsubscript{8}}{Lalchand M-2091}[]
        \DeclareHand{C8ac}{C8}{\selectlanguage{english}C\rlap{\textsubscript{8}}\textsuperscript{ac}}[]
        \DeclareHand{C8pc}{C8}{\selectlanguage{english}C\rlap{\textsubscript{8}}\textsuperscript{pc}}[]
\DeclareWitness{C9}{\selectlanguage{english}C\textsubscript{9}}{Lalchand M-4530}[]


% %%%%%%%%%%        Nepalese
\DeclareWitness{N1}{\selectlanguage{english}N\textsubscript{1}}{NGMPP A1400-2}[]
\DeclareWitness{N2}{\selectlanguage{english}N\textsubscript{2}}{NGMPP B 39-19}[]
\DeclareWitness{N3}{\selectlanguage{english}N\textsubscript{3}}{NGMPP B 62-20}[]
\DeclareWitness{N5}{\selectlanguage{english}N\textsubscript{5}}{NGMPP A60-15 + A61-1}[]
\DeclareWitness{N4}{\selectlanguage{english}N\textsubscript{4}}{NGMPP A61-2}[]
\DeclareWitness{N6}{\selectlanguage{english}N\textsubscript{6}}{NGMPP A61-6}[]
\DeclareWitness{N9}{\selectlanguage{english}N\textsubscript{9}}{NGMPP A62-33}[]
\DeclareWitness{N10}{\selectlanguage{english}N\textsubscript{10}}{NGMPP A62-37}[]
\DeclareWitness{N11}{\selectlanguage{english}N\textsubscript{11}}{NGMPP A63-15}[]
\DeclareWitness{N12}{\selectlanguage{english}N\textsubscript{12}}{NGMPP A939-19}[]
\DeclareWitness{N13}{\selectlanguage{english}N\textsubscript{13}}{NGMPP A1378-18}[]
\DeclareWitness{N16}{\selectlanguage{english}N\textsubscript{16}}{NGMPP B39-20}[]
\DeclareWitness{N17}{\selectlanguage{english}N\textsubscript{17}}{NGMPP B 111-10}[]
\DeclareWitness{N18}{\selectlanguage{english}N\textsubscript{18}}{NGMPP E 929-3}[]
\DeclareWitness{N19}{\selectlanguage{english}N\textsubscript{19}}{NGMPP E-1528-1 / E-1527-7(4)}[]
\DeclareWitness{N20}{\selectlanguage{english}N\textsubscript{20}}{NGMPP E 2037-13 }[]
\DeclareWitness{N21}{\selectlanguage{english}N\textsubscript{21}}{NGMPP E 2097-31}[]
\DeclareWitness{N22}{\selectlanguage{english}N\textsubscript{22}}{NGMPP G 4-4}[]
\DeclareWitness{N23}{\selectlanguage{english}N\textsubscript{23}}{NGMPP G 25-2}[]
        \DeclareHand{N23ac}{N23}{\selectlanguage{english}N\rlap{\textsubscript{23}}\textsuperscript{ac}}[] % added by MD
        \DeclareHand{N23pc}{N23}{\selectlanguage{english}N\rlap{\textsubscript{23}}\textsuperscript{pc}}[] % added by MD
\DeclareWitness{N24}{\selectlanguage{english}N\textsubscript{24}}{NGMPP G 190-16}[]
\DeclareWitness{N24ac}{\selectlanguage{english}N\rlap{\textsubscript{24}}\textsuperscript{ac}\space}{}[]
\DeclareWitness{N24pc}{\selectlanguage{english}N\rlap{\textsubscript{24}}\textsuperscript{pc}\space}{}[]
\DeclareWitness{N26}{\selectlanguage{english}N\textsubscript{26}}{NGMPP T 24-3}[]

% %%%%%%%%%%        Pune

\DeclareWitness{P1}{\selectlanguage{english}P\textsubscript{1}}{Ānandāśrama S16-3-21}[]
\DeclareWitness{P2}{\selectlanguage{english}P\textsubscript{2}}{Ānandāśrama S16-2-20}[]
\DeclareWitness{P3}{\selectlanguage{english}P\textsubscript{3}}{BISM (79) 314}[]
\DeclareWitness{P4}{\selectlanguage{english}P\textsubscript{4}}{BISM (91) 191}[]
\DeclareWitness{P5}{\selectlanguage{english}P\textsubscript{5}}{BISM (29) 5790}[]
\DeclareWitness{P6}{\selectlanguage{english}P\textsubscript{6}}{BORI 263/1879-80}[]
\DeclareWitness{P7}{\selectlanguage{english}P\textsubscript{7}}{BORI 665/1883-84}[]
\DeclareWitness{P8}{\selectlanguage{english}P\textsubscript{8}}{BORI 316/1895-98}[]
\DeclareWitness{P9}{\selectlanguage{english}P\textsubscript{9}}{BORI 733-1891-95}[]
\DeclareWitness{P10}{\selectlanguage{english}P\textsubscript{10}}{BORI 222-1884-86}[]
\DeclareWitness{P11}{\selectlanguage{english}P\textsubscript{11}}{BORI 221-1882–83}[]
\DeclareWitness{P12}{\selectlanguage{english}P\textsubscript{12}}{Ānandāśrama S16-3-24}[]
\DeclareWitness{P13}{\selectlanguage{english}P\textsubscript{13}}{Ānandāśrama S16-2-22}[]
\DeclareWitness{P14}{\selectlanguage{english}P\textsubscript{14}}{Ānandāśrama S16-3-23}[]
\DeclareWitness{P15}{\selectlanguage{english}P\textsubscript{15}}{BISM (64) 919}[]
\DeclareWitness{P16}{\selectlanguage{english}P\textsubscript{16}}{BISM (64) 1115}[]
\DeclareWitness{P17}{\selectlanguage{english}P\textsubscript{17}}{BISM 620/1886-92}[]
\DeclareWitness{P18}{\selectlanguage{english}P\textsubscript{18}}{BORI 615/1887-91}[]
\DeclareWitness{P19}{\selectlanguage{english}P\textsubscript{19}}{BISM 46-39}[]
\DeclareWitness{P20}{\selectlanguage{english}P\textsubscript{20}}{BISM 39-273}[]
\DeclareWitness{P21}{\selectlanguage{english}P\textsubscript{21}}{BISM 37-743}[]
\DeclareWitness{P22}{\selectlanguage{english}P\textsubscript{22}}{BISM 37-729}[]
\DeclareWitness{P23}{\selectlanguage{english}P\textsubscript{23}}{BISM 33-60}[]
\DeclareWitness{P24}{\selectlanguage{english}P\textsubscript{24}}{BISM 29-5790}[]% =P5!
\DeclareWitness{P25}{\selectlanguage{english}P\textsubscript{25}}{BISM 29-3657}[]
\DeclareWitness{P26}{\selectlanguage{english}P\textsubscript{26}}{BISM 25-281}[]
\DeclareWitness{P27}{\selectlanguage{english}P\textsubscript{27}}{BISM 7-489}[]
\DeclareWitness{P28}{\selectlanguage{english}P\textsubscript{28}}{BORI 399-1895-1902}[]

%%%%%   Mysore
\DeclareWitness{M1}{\selectlanguage{english}M\textsubscript{1}}{P-5682/4}[]
%%%%%   Tübingen
\DeclareWitness{Tue}{\selectlanguage{english}Tü}{Ma I 339}[]
%%%%%%%%%%
\DeclareWitness{YC}{\selectlanguage{english}YC}{Yogacintāmaṇi}[]
\DeclareWitness{ceteri}{\selectlanguage{english}cett.}{ceteri}[]

%%%%%%%%%% Mss with Commentary
\DeclareWitness{A1}{\selectlanguage{english}A\textsubscript{1}}{Alwar 952}[]

\DeclareWitness{Jyo}{\selectlanguage{english}J\textsubscript{yo}}{Brahmānanda's version}[]

%%%%%%%%%%%%%%%%%%%%%%%%%%%%%%%%%%%%%%%%%%%
%List of all Sigla:
%A1,B1,B2,B3,C1,C2,C3,C4,C6,C7,C8,C9,J1,J2,J3,J4,J10,J13,J14,J15,J17,L1,M1,N3,N5,N6,N9,N10,N11,N12,N13,N16,N17,N19,N20,N21,N22,N23,N24,Tü,V1,V2,V3,V4,V5,V6,V8,V11,V19,V22,V26,Vu
%%%%%%%%%%%%%%%%%%%%%%%%%%%%%%%%%%%%%%%%%%%

\DeclareWitness{G4}{\selectlanguage{english}G\textsubscript{4}}{GOML D18885 (Bundle SD5051)}[]
\DeclareWitness{G5}{\selectlanguage{english}G\textsubscript{5}}{GOML R3841/ SR2190}[]
\DeclareWitness{G7}{\selectlanguage{english}G\textsubscript{7}}{GOML D4394}[]

\DeclareWitness{Ko}{\selectlanguage{english}K\textsubscript{o}}{Koba, Gujarat 55626}[]


%%%%%                   Abbreviation for the printed apparatus,        xml interface needed
%%%%%                   (synonyms in same line)

% Macro for Editing Abbrevs.
%\def\om{\textrm{\footnotesize \textit{omitted in}\ }} %prints om. for omitted in apparatus
%\def\korr{\textrm{\footnotesize \textit{em.}\ }} %prints em. for emended in apparatus
%\def\conj{\textrm{\footnotesize \textit{conj.}\ }} %prints conj. for conjectured in apparatus


\def\eyeskip{\textrm{{ab.\,oc. }}}   
\def\aberratio{\textrm{{ab.\,oc. }}}
\def\ad{\textrm{{ad}}}   
\def\add{\textrm{{add.\ }}}
\def\ann{\textrm{{ann.\ }}}
\def\ante{\textrm{{ante }}}
\def\post{\textrm{{post }}}
%\def\ceteri{cett.\,}             % for simplifying the apparatus in print                  
\def\codd{\textrm{{codd.\ }}}   %  the same
\def\conj{\textrm{{coni.\ }}}  
\def\coni{\textrm{{coni.\ }}}
\def\contin{\textrm{{contin.\ }}}
\def\corr{\textrm{{corr.\ }}}
\def\del{\textrm{{del.\ }}}
\def\dub{\textrm{{ dub.\ }}}
\def\emend{\textrm{{emend.\ }}}
\def\expl{\textrm{{explic.\ }}}   
\def\explicat{\textrm{{explic.\ }}}
\def\fol{\textrm{{fol.\ }}}         
\def\foll{\textrm{{foll.\ }}}
\def\gloss{\textrm{{glossa ad }}}
\def\ins{\textrm{{ins.\ }}}          \def\inseruit{\textrm{{ins.\ }}}
\def\im{{\kern-.7pt\lower-1ex\hbox{\textrm{\tiny{\emph{i.m.}}}\kern0pt}}}
\def\inmargine{{\kern-.7pt\lower-.7ex\hbox{\textrm{\tiny{\emph{i.m.}}}\kern0pt}}}
\def\intextu{{\kern-.7pt\lower-.95ex\hbox{\textrm{\tiny{\emph{i.t.}}}\kern0pt}}}%\textrm{\scriptsize{i.t.\ }}}               
\def\indist{\textrm{{indis.\ }}}          \def\indis{\textrm{{indis.\ }}}
\def\iteravit{\textrm{{iter.\ }}}          \def\iter{\textrm{{iter.\ }}}  
\def\lectio{\textrm{{lect.\ }}}             \def\lec{\textrm{{lect.\ }}}
\def\leginequit{\textrm{{l.n. }}}         \def\legn{\textrm{{l.n. }}}         \def\illeg{\textrm{{l.n. }}}
\def\om{\textrm{{om. }}}
\def\primman{\textrm{{pr.m.}}}
\def\prob{\textrm{{prob.}}}
\def\rep{\textrm{{repetitio }}}
% \def\secundamanu{\textrm{\scriptsize{s.m.}}}
% \def\secm{{\kern-.6pt\lower-.91ex\hbox{\textrm{\tiny{\emph{s.m.}}}\kern0pt}}}%   \textrm{\scriptsize{s.m.}}}
\def\sequentia{\textrm{{seq.\,inv.\ }}}         \def\seqinv{\textrm{{seq.\,inv.\ }}} \def\order{\textrm{{seq.\,inv.\ }}}
\def\supralineam{{\kern-.7pt\lower-.91ex\hbox{\textrm{\tiny{\emph{s.l.}}}\kern0pt}}} %\textrm{\scriptsize{s.l.}}}
\def\interlineam{{\kern-.7pt\lower-.91ex\hbox{\textrm{\tiny{\emph{s.l.}}}\kern0pt}}}   %\textrm{\scriptsize{s.l.}}}
\def\vl{\textrm{v.l.}}   \def\varlec{\textrm{v.l.}} \def\varialectio{\textrm{v.l.}}
\def\vide{\textrm{{cf.\ }}}           \def\cf{\textrm{{cf.\ }}}
\def\videtur{\textrm{{vid.\,ut}}}
\def\crux{\textup{[\ldots]} }
\def\cruxx{\textup{[\ldots]}}
\def\unm{\textit{unm.}}        % unmetrical
%%%%%%%%%%%%%%%%%%%%%%%%%%%%%%%%%%%%



%%% Local Variables:
%%% mode: latex
%%% TeX-master: t
%%% End:

% addition 2023-12-11 MD:
\TeXtoTEIPat{\begin {metre}[#1]}{<note type="metre" target="##1">}
\TeXtoTEIPat{\end {metre}}{</note>}
\TeXtoTEIPat{\texttheta}{θ}

% change 2023-12-05 mm
\TeXtoTEI{teimute}{} 

% changes/additions 2023-11-27 MM:
\TeXtoTEIPat{\medialink {#1}{#2}}{<ref target="resources/#2">#1</ref>}

% changes/additions 2023-10-25 MM:
% new Sigla
\TeXtoTEIPat{\textAlpha}{Α}
\TeXtoTEIPat{\textalpha}{α}
\TeXtoTEIPat{\textBeta}{Β}
\TeXtoTEIPat{\textbeta}{β}
\TeXtoTEIPat{\textGamma}{Γ}
\TeXtoTEIPat{\textgamma}{γ}
\TeXtoTEIPat{\textDelta}{Δ}
\TeXtoTEIPat{\textdelta}{δ}
\TeXtoTEIPat{\textEpsilon}{Ε}
\TeXtoTEIPat{\textepsilon}{ε}
\TeXtoTEIPat{\textEta}{Η}
\TeXtoTEIPat{\texteta}{η}
\TeXtoTEIPat{\textChi}{Χ}
\TeXtoTEIPat{\textchi}{χ}
\TeXtoTEIPat{\textOmega}{Ω}
\TeXtoTEIPat{\textomega}{ω}

%new environments
\TeXtoTEIPat{\begin {postmula}[#1]}{<note type="postmula" target="##1">}
  \TeXtoTEIPat{\end {postmula}}{</note>}
\TeXtoTEIPat{\begin {altava}[#1]}{<div type="altrec"><note type="avataranika" target="##1">} %%% changed 2023-12-05 mm
  \TeXtoTEIPat{\end {altava}}{</note></div>} %%% changed 2023-12-05 mm
\TeXtoTEIPat{\sgwit {#1}}{<note type="inlineref"><ref>#1</ref></note>}

% changes/additions 2023-10-12 MM:
\TeXtoTEIPat{\\.}{}

% changes/additions 2023-08-15 MD:
\TeXtoTEIPat{\lineom {#1}{#2}}{<note type="omission">#1 omitted in <ref>#2</ref></note>}
\TeXtoTEI{graus}{hi}[rend="grey"]
\TeXtoTEIPat{\startgray}{} %%% changed 2023-12-05 mm
\TeXtoTEIPat{\endgray}{} %%% changed 2023-12-05 mm



% additions/changes 2023-06-05 mm:
%\TeXtoTEIPat{\lineom {#1}}{<note type="omission">Line omitted in <ref>#1</ref></note>}
\TeXtoTEIPat{\NotIn {#1}}{<note type="omission">Stanza omitted in <ref>#1</ref></note>}

% additions 2023-04-16 MD:
\TeXtoTEIPat{\,}{ }

% additions 2023-04-13 mm:
\TeXtoTEIPat{\begin {versinnote}}{<lg>}
  \TeXtoTEIPat{\end {versinnote}}{</lg>}

% additions 2023-04-05 MD:
\TeXtoTEIPat{\begin {testimonia}[#1]}{<note type="testimonia" target="##1">}
  \TeXtoTEIPat{\end {testimonia}}{</note>}
\TeXtoTEI{devnote}{s}[xml:lang="sa-deva"]

% app in philcomm und testimonia %%% added MM 2023-12-02
\TeXtoTEI{var}{note}[type="appinnote"]


\TeXtoTEI{anm}{note}[type="memo"] %% change 2023-04-16 MD
\TeXtoTEI{Anm}{note}[type="memo"] %% change 2023-12-05 MM
\TeXtoTEIPat{\startverse}{} %%% marked for change 2023-04-13 mm
\TeXtoTEIPat{\endverse}{} %%% marked for change 2023-04-13 mm
\TeXtoTEIPat{\newpage}{}
\TeXtoTEIPat{\marma}{}
\TeXtoTEIPat{\marmas}{}
\TeXtoTEIPat{\vin}{} % added by MD 2023-11-14

%%% modify environments and commands
%%% TEI mapping
% additions/changes 2022-06-07 mm:
\TeXtoTEI{grau}{hi}[rend="grey"]
\TeXtoTEIPat{ \& }{ &amp; }

% additions/changes 2022-06-01 mm:
\TeXtoTEI{skp}{seg}[type="deva-ignore"]
\TeXtoTEI{skm}{seg}[type="ltn-ignore"]

\TeXtoTEIPat{\rlap {#1}}{#1}

% additions/changes 2022-04-06 mm:
%\TeXtoTEI{sgwit}{ref}
\TeXtoTEI{textdev}{s}[xml:lang="sa-deva"]
\TeXtoTEIPat{\begin {col}[#1]}{<div type="colophon" xml:id="#1"><p>}
  \TeXtoTEIPat{\end {col}}{</p></div>}
\TeXtoTEIPat{\begin {ava}[#1]}{<note type="avataranika" target="##1">}
  \TeXtoTEIPat{\end {ava}}{</note>}
												   
\TeXtoTEIPat{\outdent}{}
\TeXtoTEIPat{\startaltrecension}{} %%% changed 2023-12-05 mm
\TeXtoTEIPat{\endaltrecension}{} %%% changed 2023-12-05 mm
\TeXtoTEIPat{\startaltnormal}{} % added by MD 2023-11-14 %%% changed 2023-12-05 mm
\TeXtoTEIPat{\endaltnormal}{} % added by MD 2023-11-14 %%% changed 2023-12-05 mm
\TeXtoTEIPat{\begin {alttlg}[#1]}{<div type="altrec"><lg xml:id="#1">}
  \TeXtoTEIPat{\end {alttlg}}{</lg></div>}



% additions/changes 2022-03-12 mm:
\TeXtoTEIPat{\begin {tlg}[#1]}{<lg xml:id="#1">}
  \TeXtoTEIPat{\end {tlg}}{</lg>}

\TeXtoTEIPat{\begin {translation}[#1]}{<note type="translation" target="##1">}
  \TeXtoTEIPat{\end {translation}}{</note>}
\TeXtoTEIPat{\begin {philcomm}[#1]}{<note type="philcomm" target="##1">}
  \TeXtoTEIPat{\end {philcomm}}{</note>}
\TeXtoTEIPat{\begin {sources}[#1]}{<note type="sources" target="##1">}
  \TeXtoTEIPat{\end {sources}}{</note>}


\TeXtoTEIPat{\begin {marma}[#1]}{<note type="marma" target="##1">}
  \TeXtoTEIPat{\end {marma}}{</note>}

\TeXtoTEIPat{\begin {jyotsna}[#1]}{<note type="jyotsna" target="##1">}
  \TeXtoTEIPat{\end {jyotsna}}{</note>}

\EnvtoTEI{description}{list}
\EnvtoTEI{itemize}{list}
\TeXtoTEIPat{\item [#1]}{<label>#1</label>\item}

\TeXtoTEI{tl}{l}
\TeXtoTEI{myfn}{note}[type="myfn"]
\TeXtoTEIPat{\getsiglum {#1}}{<ref target="##1"/>}

\TeXtoTEI{SetLineation}{}
\TeXtoTEI{noindent}{}
\TeXtoTEI{subsection*}{}

\TeXtoTEI{rlap}{}

% end additions/changes
% \TeXtoTEIPat{\skp {#1}}{#1}
% \TeXtoTEIPat{\skm {#1}}{}

\TeXtoTEIPat{\begin {prose}}{<p>}
  \TeXtoTEIPat{\end {prose}}{</p>}

\TeXtoTEIPat{\begin {tlate}}{<p>}
  \TeXtoTEIPat{\end {tlate}}{</p>}

\TeXtoTEI{emph}{hi}
\TeXtoTEI{bigskip}{}
% \TeXtoTEI{/}{|}
\TeXtoTEI{tl}{l}
\TeXtoTEIPat{english}{}
%\TeXtoTEIPat{-}{ } %% change 2023-04-16 MD
%\TeXtoTEIPat{°}{} %% change 2023-04-16 MD
\TeXtoTEIPat{\textcolor {#1}{#2}}{<hi rend="#1">#2</hi>}

% \TeXtoTEIPat{\eyeskip}{}
% \TeXtoTEIPat{\aberratio}{}
% \TeXtoTEIPat{\ad}{}
\TeXtoTEIPat{\add}{<hi rend="italic">add.</hi>} %% change 2023-04-16 MD
% \TeXtoTEIPat{\ann}{}
\TeXtoTEIPat{\ante}{<hi rend="italic">ante</hi> } %% change 2023-04-16 MD
\TeXtoTEIPat{\post}{<hi rend="italic">post</hi> } %% change 2023-04-16 MD
% \TeXtoTEIPat{\codd}{}
% \TeXtoTEIPat{\conj }{}
% \TeXtoTEIPat{\contin}{}
% \TeXtoTEIPat{\corr}{}
% \TeXtoTEIPat{\del}{}
% \TeXtoTEIPat{\dub}{}
% \TeXtoTEIPat{\emend }{}
% \TeXtoTEIPat{\expl}{}
% \TeXtoTEIPat{\ȩxplicat}{}
% \TeXtoTEIPat{\fol}{}
% \TeXtoTEIPat{\gloss}{}
% \TeXtoTEIPat{\ins}{}
% \TeXtoTEIPat{\im}{}
% \TeXtoTEIPat{\inmargine}{}
% \TeXtoTEIPat{\intextu}{}
% \TeXtoTEIPat{\indist}{}
% \TeXtoTEIPat{\iteravit}{}
% \TeXtoTEIPat{\lectio}{}
% \TeXtoTEIPat{\leginequit}{}
% \TeXtoTEIPat{\legn}{}
% \TeXtoTEIPat{\illeg}{<hi rend="italic">illeg.</hi>}
\TeXtoTEIPat{\illeg}{<gap reason="illeg."/>} %%% change 2023-04-11 mm
% \TeXtoTEIPat{\om}{<hi rend="italic">om.</hi>}
\TeXtoTEIPat{\om}{<gap reason="om."/>} %%% change 2023-04-11 mm
% \TeXtoTEIPat{\primman}{}
% \TeXtoTEIPat{\prob}{}
% \TeXtoTEIPat{\rep}{}
% \TeXtoTEIPat{\sequentia}{}
% \TeXtoTEIPat{\supralineam}{}
% \TeXtoTEIPat{\interlineam}{}
\TeXtoTEIPat{\vl}{<hi rend="italic">v.l.</hi>}
% \TeXtoTEIPat{\vide}{}
% \TeXtoTEIPat{\videtur}{}
% \TeXtoTEIPat{\crux}{}
% \TeXtoTEIPat{\cruxxx}{}
\TeXtoTEIPat{\unm}{<hi rend="italic">unm.</hi>}


% List of Scholars
\DeclareScholar{nos}{nos}[
forename=HPP,
surname=Team]


% Nullify \selectlanguage in TEI as it has been used in
% \DeclareWitness but should be ignored in TEI.
\TeXtoTEI{selectlanguage}{}



\NewDocumentCommand{\skp}{m}{}
\NewDocumentCommand{\skm}{m}{\unless\ifinapparatus#1-\fi}

\SetTEIxmlExport{autopar=false}
\NewDocumentEnvironment{tlg}{O{}}{
	\begin{ekdverse}
	\indentpattern{0000}}{
	\end{ekdverse}
	\vskip 0.75\baselineskip}
\NewDocumentEnvironment{alttlg}{O{}}{}{}
\NewDocumentCommand{\tl}{m}{#1}

%%%%%%

\def\startaltrecension#1{
  \stopvline
  \begin{ekdverse}[type=altrecension]
    \indentpattern{0000} 
    \begin{patverse*}
      \color{gray}
      \setvnum{#1}}
\def\endaltrecension{
  \end{patverse*}
  \end{ekdverse}
  \vskip 0.75\baselineskip
  \startvline}

%%%%%%

\newcommand{\myfn}[1]{\footnote{\texteng{#1}}}
\renewcommand{\thefootnote}{\texteng{\arabic{footnote}}}
\newcommand{\devnote}[1]{\selectlanguage{iast}{\scriptsize #1}\selectlanguage{english}}
\newcommand{\outdent}{\hspace{-\vgap}}
\newcommand{\sgwit}[1]{{\small (\getsiglum{#1})}\selectlanguage{iast}}
\newcommand{\NotIn}[1]{\texteng{\footnotesize (om. \getsiglum{#1})}\selectlanguage{iast}}

\def\om{\emph{om.}} % \!}
\def\illeg{\emph{illeg.}} %\!}
\def\unm{\emph{unm.\:}}
\def\recte{\texteng{r.\:}}
\def\for{\texteng{for }}
\def\sic{\emph{sic}}

\makepagestyle{HPed}
\makeoddhead{HPed}{\small\texteng{HP Transl. \& Comm.}}{}{\small\texteng{\today}}
\makeevenhead{HPed}{\small\texteng{HP Transl. \& Comm.}}{}{\small\texteng{\today}}
\makeoddfoot{HPed}{}{\small\texteng{\thepage}}{}
\makeevenfoot{HPed}{}{\small\texteng{\thepage}}{}

\SetTEIxmlExport{autopar=false}
\NewDocumentEnvironment{translation}{O{}}{\textcolor{blue}{\textbf{Transl.:}}}{}
\NewDocumentEnvironment{philcomm}{O{}}{
	\textcolor{blue}{\textbf{Comm.:}}}{}
\NewDocumentEnvironment{sources}{O{}}{
	\textcolor{blue}{\textbf{Sources:}}\linebreak}{}
\NewDocumentEnvironment{testimonia}{O{}}{
	\textcolor{blue}{\textbf{Testimonia:}}\linebreak}{}
\NewDocumentEnvironment{versinnote}{O{}}{\begin{ekdverse}}{\end{ekdverse}}
%\newcommand{\var}[1]{\footnotesize\textup{#1}}
\newcommand{\medialink}[2]{\textcolor{green}{\underline{#1}}}
%%%\TeXtoTEIPat{\medialink {#1}{#2}}{<ref target="/images/#2">#1</ref>} %%%modified and moved to TeX2Tei-Commands.tex

\def\vl{\textit{v.l.}}
\def\var#1{{\footnotesize #1}}
\def\sl#1{\emph{#1}}

\linespread{1}
\setlength{\parskip}{0.3em}
\setlength\parindent{0pt}
\setlength{\vindent}{0pt}
\def\startverse{\begin{ekdverse}}
\def\endverse{\end{ekdverse}\normalsize}
\setvnum{}

\begin{document}
\pagestyle{HPed}
\begin{ekdosis}
\SetLineation{lineation = none,}

\chapter*{Translation \& philological commentary}
\subsection*{1.1}
\begin{translation}[hp01_001]
Homage to the glorious Ādinātha by whom the system of Haṭhayoga was taught. It shines forth like a ladder for one desirous of climbing to the lofty terrace of the royal palace.
\end{translation}

\begin{testimonia}[hp01_001]
Cf. Yogasārasaṃgraha, p. 54.

\begin{versinnote}
\tl{sadādināthāya namo'stu tubhyaṃ yenopadiṣṭā haṭhayogavidyā |\\+}
\tl{virājate pronnatarājayogam āroḍhum icchoradhirohiṇīva ||\\!}
\end{versinnote}

Gheraṇḍasaṃhitā 1.1
\startverse
ādīśvarāya praṇamāmi tasmai yenopadiṣṭā haṭhayogavidyā | \\
virājate pronnatarājayogam āroḍhum icchor adhirohiṇīva ||
\endverse
\end{testimonia}

\begin{philcomm}[hp01_001]
The reading \emph{rājasaudha} is preferable for poetical reasons, as explained in Hanneder 2020, p. 128–130. Also, the \emph{Jyotsnā} explains that the \emph{alaṃkāra} here is a comparison (\emph{upamā}), which consists of four elements: (1) a particle expressing the comparison
(\emph{iva}), (2) the object compared (\emph{upameya}), i.e., the \emph{haṭhayogavidyā} leading to Rājayoga (\emph{rājayogaprāpikā}), (3) the property compared, that is, effortlessness (\emph{anāyāsena}) and, most importantly, (4) an image to compare with, which in this case is \emph{rājasaudha}. This idea is confirmed by Brahmānanda, when he sums up and completes the poetical details as: \emph{yathā pronnatasaudham āroḍhum icchor adhirohiṇy anāyāsena saudhaprāpikā bhavati evaṃ haṭhadīpikāpi pronnatarājayogam āroḍhum icchor anāyāsena rājayogaprāpikā bhavatīti upamālaṅkāraḥ}.

The Sanskrit poeticians explain that when some elements of a comparison remains unexpressed we get an incomplete (\emph{lupta}) comparison. Often words like \emph{iva} or \emph{yathā} are missing or the common property, but if we read \emph{rājayoga} we lose the \emph{upamāna} (i.e., \emph{rājasaudha}). One part of the comparison should not be missing. But from the perspective of \emph{alaṃkāraśāstra} the verse has a problem that has surely caused the dilemma: it either leaves the \emph{upameya} or the \emph{upamāna} incomplete. Brahmānanda has mentioned both \emph{yathā adhirohiṇī saudhaprāpikā bhavati evaṃ haṭhadīpikā rājayogaprāpikā bhavati}, but had to supply \emph{rājasaudha}, while his text reads \emph{rājayoga}. But to have the \emph{upameya} in the text is odd. Even Brahmānanda could only know of the image (\emph{rājasaudha}) from the alternative reading he did not accept. However, it seems likely that the author would have included the \emph{upamāna} in the text, as in our critical text, so that the reader would know that the upper terrace of the palace is an image for \emph{rājayoga}. It is further likely that the substitution of \emph{rājayoga} for \emph{rājasaudha} was the result of the tendency to insert the word \emph{yoga} in the opening verses of the text wherever possible, even where it does not fit, as can be seen in 1.2d (\emph{haṭhayogopadiśyate}) and 1.3b (\emph{rājayogam ajānatām}). In the latter case the poetical image has also been lost.

The metre of 1.1 is Indravajrā
\end{philcomm}

\subsection*{1.2}
\begin{translation}[hp01_002]
Having bowed to the glorious guru, the Lord, the yogi Svātmārāma has taught the system of Haṭhayoga solely for [attaining] Rājayoga.
\end{translation}

\begin{philcomm}[hp01_002]
The reading \emph{yogopadiśyate} (J7,J10,N17,W4, etc.) is only possible if one accepts that double \emph{sandhi} is a feature of the style of composition, which it is not. Moreover, this reading appears to have resulted from an attempt to replace the word \emph{vidyā} with \emph{yoga} in the opening verses of the text.
\end{philcomm}

\subsection*{1.3}
\begin{translation}[hp01_003]
The compassionate Svātmārāma holds the Lamp on Haṭha for those who are ignorant of the royal path because of wandering in the darkness of many opinions.
\end{translation}

\begin{testimonia}[hp01_003]
Haṭharatnāvalī 1.4
\startverse
bhrāntyā bahumatadhvānte rājayogam ajānatām |\\
kevalaṃ rājayogāya haṭhavidyopadiśyate || 
\endverse
\end{testimonia}

\begin{philcomm}[hp01_003] 
Most witnesses have \emph{rājayogam ajānatām} (`for those ignorant of Rājayoga') in 1.3b. While this reading is well attested by the manuscripts transmission of the \emph{Haṭhapradīpikā} and some manuscripts of the \emph{Haṭharatnāvalī} (note that P,T,t1 of the critical edition [Gharote 2009: 5 n. 2] have \emph{rājamārgam ajānataḥ}), the reading of a royal path (\emph{rājamārgam}) is consistent with the metaphor of people wandering from a path in the darkness and, therefore, most probably authorial. 

In 1.1d both \emph{kṛpākaraḥ} and \emph{kṣamākaraḥ} are attested. The former is preserved by the V1 and group 2 whereas the latter by some manuscripts of the delta group. Since the context is the author helping yogis who have strayed from the royal path, \emph{kṛpākara} makes better sense. As Brahmānanda notes, this compound can be understood as one who is compassionate (\emph{kṛpā} + \emph{kara}) or one who is a mine (i.e., a rich source) of compassion (\emph{kṛpā} + \emph{ākara}). In the Devanagari transmission, the \emph{kṣa} of \emph{kṣamākaraḥ} may have arisen as a mistake for \emph{kṛ}.       
\end{philcomm}

\subsection*{1.4}
\begin{translation}[hp01_004]
In fact, Matsyendra, Gorakṣa and other [perfected yogis] knew the system of Haṭha, and the yogi Svātmārāma knows it owing to their favour.
\end{translation}

\begin{testimonia}[hp01_004]
Post-15th Century\\
Haṭharatnāvali 1.3
\startverse
haṭhavidyāṃ hi gorakṣamatsyendrādyā vijānate |\\
ātmārāmo 'pi jānīte śrīnivāsas tathā svayam || 
\endverse
\end{testimonia}

\begin{philcomm}[hp01_004]   
The word \emph{athavā} (‘or’) is difficult to construe here. Brahmānanda understands it as conjunction (\emph{athavāśabdaḥ samuccaye}), and this is how we have interpreted it. The variant \emph{mahāyogī} in group 4c (G5, J4, J11, Ko, P15) avoids this difficulty, but is probably an attempt to remove the difficulty of understanding \emph{athavā}. One could emend to \emph{tathā} but this would be a bold intervention given the weight of evidence supporting \emph{’thavā}.   
\end{philcomm}

\subsection*{1.5}
\begin{translation}[hp01_005]
The glorious Ādinātha, Matysendra, Śābara, Ānandabhairava, Cauraṅgī, Mīna, Gorakṣa, Virūpākṣa, Bileśaya,
\end{translation}

\begin{testimonia}[hp01_005]
Haṭharatnāvalī 1.80
\startverse
śrīādināthamatsyendraśābarānandabhairavāḥ |\\
śāraṅgīmīnagorakṣavirūpākṣabileśayāḥ || 
\endverse
\end{testimonia}

\begin{philcomm}[hp01_005]     
In Śaiva texts which predate the haṭha corpus, Mīnanātha and Matsyendra are one and the same, but they are differentiated in later Tibetan and Indian lists of siddhas (Mallinson 2019:273 n.35).   
\end{philcomm}

\subsection*{1.6}
\begin{translation}[hp01_006]
Manthānabhairava, Siddhabuddha, and Kanthaḍi, Koraṇṭaka, Surānanda, Siddhapāda, Carpaṭi.
\end{translation}

\begin{testimonia}[hp01_006]
Haṭharatnāvalī 1.81
\startverse
manthānabhairavo yogī siddhabuddhaś ca kandalī |\\
korandakaḥ surānandaḥ siddhipādaś ca carpaṭī ||
\endverse

Caturbhuja Misra's Mugdhāvabodhinī (1.7.8) on the Rasahṛdayatantra
\startverse
manthānabhairavo yogī siddhabuddhaś ca kanthaḍī |\\
koraṇṭakaḥ surānandaḥ siddhapādaś ca carpaṭī ||
\endverse
\end{testimonia}

\begin{philcomm}[hp01_006]        
Many manuscripts preserve the reading \emph{pauraṇṭaka}. We have accepted \emph{koraṇṭaka} because this name is associated with the \emph{Haṭhābhyāsapaddhati}. This reading is well attested by various manuscripts of the \emph{Haṭhapradīpikā}, including those of group 2, as well as those of the \emph{Haṭharatnāvalī} (Gharote 2009: 35 n. 3), which also attest \emph{gonandaka}. Also, \emph{Goraṇṭakuḍu} is the name of a disciple of Gorakṣanātha in the \emph{Navanāthacaritramu} (Jones 2017:194 n.3). The compound \emph{siddhapāda} could be a respectful affix. However, it seems unlikely here because it would cross the \emph{pāda} break.  
\end{philcomm}

\subsection*{1.7}
\begin{translation}[hp01_007]
Kānerī, Pūjyapāda, Nityanātha, Nirañjana, Kapālī, Bindunātha, and the one named Kākacaṇḍīśvara.
\end{translation}

\begin{testimonia}[hp01_007]
Haṭharatnāvalī 1.82
\startverse
karoṭiḥ pūjyapādaś ca nityanātho nirañjanaḥ |\\
kapālī bindunāthaś ca kākacaṇḍīśvarāhvayaḥ || 
\endverse

Caturbhuja Misra's Mugdhāvabodhinī on the Rasahṛdayatantra
​​\startverse
kaṇerī pūjyapādaśca nityanātho nirañjanaḥ |\\
kapālī bindunāthaśca kākacaṇḍīśvaro gajaḥ |
\endverse
\end{testimonia}

\begin{philcomm}[hp01_007]   
It is possible that \emph{pūjyapāda} could be a respectful affix to the name Kāṇerī. The variant \emph{dhvaninātha} may have resulted from a transposition of the first two syllables of \emph{nityanātha}.    
\end{philcomm}

\subsection*{1.8}
\begin{translation}[hp01_008]
Allamaprabhudeva, Ghoḍācolī, Ṭiṇṭiṇī, Bhālukī and Nāgabodha and Khaṇḍakāpālika.
\end{translation}

\begin{testimonia}[hp01_008]
Haṭharatnāvalī 1.83
\startverse
allamaḥ prabhudevaś ca naiṭacūṭiś ca ṭiṇṭiṇiḥ |\\
bhālukir nāgabodhaś ca khaṇḍakāpālikas tathā || 
\endverse

Caturbhuja Misra's Mugdhāvabodhinī on the Rasahṛdayatantra
\startverse
āllamaḥ prabhudevaś ca ghoḍācolī ca ṭhiṇṭhinī |\\
bhālukir nāgadevaś ca khaṇḍī kāpālikas tathā ||
\endverse
\end{testimonia}

\begin{philcomm}[hp01_008]  
The name Allamaprabhudeva (sometimes Allama Prabhu Deva or Allama Prabhudeva in secondary literaure) is largely transmitted as allamaḥ prabhudevaḥ, as though it were two names, although some manuscripts of the \emph{Haṭhapradīpikā} have \emph{allamaprabhudevaś ca} (i.e., Tue,V3,V8,V13,V16,V22,Vu,N24,N26) and so do some of the \emph{Haṭharatnāvalī} (i.e., P,T,t1 in Gharote 2009: 35 n. 8).

The names Nāgabodha, Nāgabodhi, Naradeva, Nāgadeva all seem possible in 1.8c. 

Many witnesses have khaṇḍa and kāpālika as separate names. However, khaṇḍakāpālika is well attested. Examples include \emph{Kathāsaritsāgara} 121.5 ff. (check), \emph{Bṛhatkathāmañjarī} 10.45 (check) and Vajrapāṇi’s \emph{Laghutantraṭīkā}, p.45 (\emph{vīrāḥ khaṇḍakāpālikādayaś caturviṃśatiḥ}). It may be a derogatory name for a Kāpālika, coined perhaps by an outsider and connoting something like a defective Kāpālika in the sense of a `part-time' Kāpālika. Alternatively, it could simply refer to one who used a broken skull as a bowl.
\end{philcomm}

\subsection*{1.9}
\begin{translation}[hp01_009]
Having destroyed the rod of death through the power of Haṭhayoga, these great perfected yogis and others wander in the world.
\end{translation}

\begin{testimonia}[hp01_009]
Haṭharatnāvalī 1.84
\startverse
ityādayo mahāsiddhāḥ haṭhayogaprasādataḥ |\\
khaṇḍayitvā kāladaṇḍaṃ brahmāṇḍe vicaranti te ||
\endverse

Caturbhuja Misra's Mugdhāvabodhinī on the Rasahṛdayatantra
\startverse
ityādayo mahāsiddhā rasabhogaprasādataḥ |\\
khaṇḍayitvā kāladaṇḍaṃ trilokyāṃ vicaranti te |
\endverse

Haṭhatattvakaumudī 17.24
\startverse
ūrdhvaṃretaḥprabhāvena sanakādyā maharṣayaḥ |\\
khaṇḍayitvā kāladaṇḍaṃ yathecchaṃ viharanti te || 24 ||
\endverse
\end{testimonia}

\begin{philcomm}[hp01_009]        
The reference to \emph{brahmāṇḍa} (‘the world’) implies liberation-in-life (\emph{jīvanmukti}) and physical immortality.  
\end{philcomm}

\subsection*{1.10}
\begin{translation}[hp01_010]
Haṭha is considered a refuge for those who are burnt by the scorching torment of transmigration. Haṭha is the tortoise that supports the worlds of all yogas.
\end{translation}

\begin{testimonia}[hp01_010]
Yogasārasaṃgraha (YSS), p.53.
\startverse
saṃsāratāpataptānāṃ samāśrayahaṭho haṭhaḥ |\\
aśeṣayogajagatām ādhārakamaṭho haṭhaḥ ||
\endverse
\end{testimonia}

\begin{philcomm}[hp01_010] 
The compound \emph{saṃsāratāpa°} is well attested and found elsewhere (e.g., \emph{Viṣṇupurāṇa} 6.7.62, \emph{Agnipurāṇa} 371.1, \emph{Haṭhatattvakaumudī} 38.92, emph{Haṭhābhyāsapaddhati} ms. 46/440, f. 1v). The reading of \emph{samāśrayo} in V1 is metrically faulty. The accepted reading \emph{āśrayo [']yam} is better metically and well attested. N23 preserves the reading \emph{samāśrayamaṭho haṭhaḥ} (similar to the \emph{Yogasārasaṃgraha}), which is similar to the fourth \emph{pāda}. Though this reading is possible, it seems more likely that it came about through dittography. The word \emph{°jagatāṃ} is probably authorial because it makes good sense with  \emph{ādhārakamaṭha} in light of the cosmological notion that the tortoise supports all the worlds. However, this reading may not have been understood by some and was changed in the vulgate and other witnesses to \emph{°yuktānāṃ} instead.      
\end{philcomm}

\subsection*{1.11}
\begin{translation}[hp01_011]
The doctrine of Haṭha should be kept completely secret by those yogis who are desiring success. When it is secret it becomes potent. However, when it has been revealed, it becomes impotent.
\end{translation}

\begin{sources}[hp01_011]
Śivasaṃhitā
\startverse
haṭhavidyā paraṃ gopyā yoginā siddhim icchatā |\\% (ab missing in some ŚS witnesses)
bhaved vīryavatī guptā nirvīryā ca prakāśitā || 5.254
\endverse
\end{sources}

\begin{testimonia}[hp01_011]
Yogacintāmaṇi
\startverse
tathā haṭhapradīpikāyām—\\
haṭhavidyā paraṃ gopyā yoginā siddhim icchatā |\\
bhaved vīryavatī guptā nirvīryā tu prakāśiteti ||
\endverse

BKhP 10v4
\end{testimonia}

\begin{philcomm}[hp01_011]        
Either the singular or plural of yogin could be read here. The singular is well attested among the testimonia. However, the weight of the manuscript evidence favours the plural.  
\end{philcomm}

\subsection*{1.12}
\begin{translation}[hp01_012]
In a well-ruled, righteous region, with plenty of food and free of disturbances, the Haṭhayogi should live remotely in a small hut.
\end{translation}

\begin{sources}[hp01_012]
References to maṭhikā here.
\end{sources}

\begin{testimonia}[hp01_012]
Haṭharatnāvalī 1.66
\startverse
surāṣṭre dhārmike deśe subhikṣe nirupadrave |\\
ekāntamaṭhikāmadhye sthātavyaṃ haṭhayoginā ||
\endverse

Yogacintāmaṇi
\startverse
haṭhapradīpikāyām—\\
surājye dhārmike deśe subhikṣe nirupadrave |\\
ekānte maṭhikāmadhye sthātavyaṃ haṭhayoginā ||
\endverse

BKhP 107v1
\end{testimonia}

\begin{philcomm}[hp01_012]        
In the \emph{Jyotsnā} and printed editions of the \emph{Haṭhapradīpikā}, including one by Digambara and Kokaje (1970: 6), this verse has the additional hemistich, \emph{dhanuḥpramāṇaparyantaṃ śilāgnijalavarjite}. This hemistich derives from the \emph{Gorakṣaśataka} (32cd), which has \emph{°paryante} instead of \emph{°paryantaṃ}. It stipulates that the hut should be built in a place measuring up to a bow length and free from rocks, fire and water. None of the early manuscripts have this hemistich which suggests that it was added at a later time. Nonetheless, it appears in over a dozen manuscripts that were consulted for this edition. These manuscripts are not close to an early hyparchetype of the text.
\end{philcomm}

\subsection*{1.13}
\begin{translation}[hp01_013]
It has a small door and is without cracks, holes and potsherds. It extends not too high or low, and is thickly smeared with cow dung in the proper way. It is clean, free from everything that annoys, furnished on the outside with a verandah, altar and well, surrounded by a wall: these are the characteristics of the yoga hut as taught by the adept practitioners of haṭha.
\end{translation}

\begin{sources}[hp01_013]
Cf. Dattātreyayogaśāstra
\startverse
suśobhanaṃ maṭhaṃ kuryāt sūkṣmadvāraṃ tu nirvraṇaṃ || 54 ||\\
suṣṭhu liptaṃ gomayena sudhayā vā prayatnataḥ |\\
matkuṇair maśakair bhūtair varjitaṃ ca prayatnataḥ || 55 ||\\
dine dine susammṛṣṭaṃ sammārjanyā hy atandritaḥ |\\
vāsitaṃ ca sugandhena dhūpitaṃ guggulādibhiḥ || 56 ||\\
malamūtrādibhir vargair aṣṭādaśabhir eva ca |\\
varjitaṃ dvārasampannaṃ vastrāvaraṇam eva vā || 57 ||
\endverse

\end{sources}

\begin{testimonia}[hp01_013]
Suśruta 6.17.67:
\emph{gṛhe nirābādhe}

Yogacintāmaṇi
\startverse
alpadvāram arandhragartaghaṭitaṃ nāpy uccanīcāyitam |\\
samyaggomayasāndraliptavimalaṃ niḥśeṣajantūjjhitam |\\
bāhye maṇḍapakūpavediracitaṃ prākārasaṃveṣṭitam |\\
proktaṃ yogamaṭhasya lakṣaṇam idaṃ siddhair haṭhābhyāsibhiḥ ||
\endverse

Haṭharatnāvalī 1.67
\startverse
alpadvāram arandhragartapiṭharaṃ nātyuccanīcāyataṃ\\
samyaggomayasāndraliptavimalaṃ niḥśeṣabādhojjhitaṃ |\\
bāhye maṇḍapavedikūparuciraṃ prākārasaṃveṣṭitam\\
proktaṃ yogamaṭhasya lakṣaṇam idaṃ siddhair haṭhābhyāsibhiḥ ||
\endverse

BKhP 107v3
\end{testimonia}

\begin{philcomm}[hp01_013]  
The syntax of this verse is rather problematic. One would expect the features of the hut, which are listed in the first three pādas of the verse, to be in the nominative case, with the words \emph{idaṃ lakṣaṇaṃ} in the fourth pāda referring back to them. However, the compounds in the first three pādas appear to qualify \emph{lakṣaṇa} as though they were adjectives, and this appears to have been the way the verse was composed.     

The manuscripts preserve many different readings at the end of the compound beginning with \emph{arandhragarta°}. The reading \emph{°piṭharaṃ} has been suggested by Dominic Goodall, who understands it as potsherds (ask ref from Dominic?). The sense is that the hut should be free of rubbish, such as potsherds. One would expect a word for a defect in a hut that is similar to, but not the same as, cracks (\emph{randhra}) and holes (\emph{garta}). For this reason, the reading \emph{°vivaraṃ} looks like a patch, as its meaning does not add anything to \emph{°randhragarta°}. The reading \emph{viṭapaṃ} is well attested among the witnesses of the \emph{Haṭhapradīpikā} and \emph{Haṭharatnāvalī}. However, it is difficult to construe in this context unless it was intended to refer to creepers or branches that might invade or encroach upon the hut.

Nearly all the early mss. have \emph{°bādhojjitaṃ}, whereas the Yogacintāmaṇi and V19 have the more easily understood reading of \emph{°jantūjjhitaṃ} (‘free from creatures’). However, \emph{°bādhojjitaṃ} may have been original because it is so well attested and a hut is described similarly in \emph{Suśrutasaṃhitā} 6.17.67 (\emph{gṛhe nirābādhe}).
\end{philcomm}

\subsection*{1.14}
\begin{translation}[hp01_014]
Locating oneself in a hut of such a kind, free from all worry, [the yogi] should practise only yoga, in the way taught by his guru.
\end{translation}

\begin{sources}[hp01_014]
Cf. Amanaska 2.15
\startverse
evaṃvidhaṃ guruṃ labdhvā sarvacintāvivarjitaḥ\\
sthitvā manohare deśe yogam eva samabhyaset
\endverse
\end{sources}

\begin{testimonia}[hp01_014]
Yogacintāmaṇi
\startverse
evaṃvidhe maṭhe sthitvā sarvacintāvivarjitaḥ |\\
gurūpadiṣṭamārgeṇa yogam eva sadābhyased iti ||
\endverse

Haṭharatnāvalī 1.68
\startverse
evaṃvidhe maṭhe sthitvā sarvacintāvivarjitaḥ |\\
gurūpadiṣṭamārgeṇa yogam eva sadābhyaset ||
\endverse
\end{testimonia}

%\begin{philcomm}[hp01_014]        
%\end{philcomm}

\subsection*{1.15}
\begin{translation}[hp01_015]
Overeating, exertion, idle chatter, not sticking to rules, socialising, sensuality: through [these] six, yoga is lost.
\end{translation}

\begin{testimonia}[hp01_015]
Yogacintāmaṇi
\startverse
atyāhāraḥ prayāsaś ca prajalpo niyamagrahaḥ |\\
janasaṅgaś ca laulyaṃ ca ṣaḍbhir yogaḥ praṇaśyati ||
\endverse

Haṭharatnāvalī 1.77
\startverse
atyāhāraḥ prayāsaś ca prajalpo niyamagrahaḥ |\\
janasaṅgaṃ ca laulyaṃ ca ṣaḍbhir yogo vinaśyati ||
\endverse

Yuktabhavadeva 4.25 (attributed to the śivayoga)
\startverse
atyāhāraḥ prayāsaśca prajalpo niyamāgrahaḥ |\\
janasaṃgaś ca laulyaṃ ca ṣaḍbhir yogo vinaśyati ||
\endverse

Jyotsnā
\startverse
śītodakena prātaḥsnānanaktabhojanaphalāhārādirūpaniyamasya grahaṇaṃ niyamagrahaḥ |
\endverse

Yogaprakāśikā
\startverse
niyamāgrahaḥ vakṣyamāṇaniyamāparipālanaṃ
\endverse
\end{testimonia}

\begin{philcomm}[hp01_015]        
It is impossible to be certain about the meaning of \emph{niyamagraha}, as the manuscripts do not indicate whether an \emph{avagraha} (i.e., \emph{prajalpo ’niyamagraha}) was intended. Although \emph{yama} and \emph{niyama} are not included in the \emph{Haṭhapradīpikā} as auxiliaries of Haṭhayoga, verse 2.14 implies that \emph{niyama} is necessary at least in the early stages of establishing a practice. Furthermore, verse 3.82 suggests that a yogi who does not practice \emph{niyama} might obtain success in yoga through the practice of \textit{vajroli}. Ambiguity over the role of \emph{yama} and \emph{niyama} in Haṭhayoga appears to have prompted some to insert verses on ten \emph{yama}s and ten \emph{niyama}s after the next verse (1.16). The additional verses may derive from the \emph{Śāradātilakatantra} (25.7–8) or the  \emph{Vasiṣṭhasaṃhitā} (1.38, 1.53). In the \emph{Jyotsnā}, Brahmānanda reads \emph{niyamāgraha} and takes it as though \emph{āgraha} was implied, which yields the meaning of ‘over-insistence on rules’, and he relates it to extreme ascetic practice.
\end{philcomm}

\subsection*{1.16}
\begin{translation}[hp01_016]
However, from zeal, conviction, resolve, contentment, understanding of the truth, and avoiding contact with people, yoga is successful because of [these] six.
\end{translation}

\begin{sources}[hp01_016]
Dharmaputrikā 137cd–38ab%Narharināth ed.?
\startverse
utsāho niścayo dhairyaṃ santoṣas tattvadarśanam | \\
kratūnāṃ copasaṃhāraḥ ṣaṭsādhanam iti smṛtam |
\endverse

Śivadharmottara 10 (W 122r):%JM: what ed or ms is this?
\startverse
utsāhān niścayād dhairyāt santoṣāt tattvadarśanāt |\\
muner janapadatyāgād ṣaḍbhir yogaḥ prasiddhyati |
\endverse

Jñānārṇava 20.1
\startverse
utsāhān niścayād dhairyāt saṃtoṣāt tattvaniścayāt |\\
muner janapadatyāgāt ṣaḍbhir yogaḥ prasidhyati || 1 ||
\endverse

Yogabindu by Haribhadra
\startverse
utsāhān niścayād dhairyāt saṃtopāt tattvadarśanāt |\\
muner janapadatyāgāt ṣaḍbhir yogaḥ prasidhyati || 411 ||
\endverse
\end{sources}

\begin{testimonia}[hp01_016]
Yogacintāmaṇi
\startverse
utsāhāt sāhasād dhairyāt tatvajñānād viniścayāt |\\
janasaṅgaparityāgāt ṣaḍbhir yogaḥ prasidhyati ||
\endverse

Haṭharatnāvalī 1.78:
\startverse
utsāhān niścayād dhairyāt tattvajñānārthadarśanāt
[utsāhān niścalād- P,T]\\
bindusthairyān mitāhārāj janasaṅgavivarjanāt |\\
nidrātyāgāj jitaśvāsāt pīṭhasthairyād anālasāt\\
gurvācāryaprasādāc ca ebhir yogas tu sidhyati || 
\endverse
\end{testimonia}

\begin{philcomm}[hp01_016]    
For the second \emph{pāda}, the north-Indian manuscripts have something along the lines of \emph{tattvajñānāc ca darśanāt}. The problem with this reading is the meaning of \emph{darśana} by itself (i.e., discernment of what?). The early sources of this verse, in particular the \emph{Śivadharmottara}, indicate that the second pāda read as \emph{santoṣāt tattvadarśanāt}, which makes much better sense of the word \emph{darśana} (i.e., ‘seeing the truth’). It seems that this reading may have been adopted by Svātmārāma because the word \emph{santoṣāt} is present in the south-Indian manuscripts. At some point in the transmission, \emph{tattvadarśanāt} became \emph{tattvajñānāt} and \emph{santoṣāt} was dropped. The word \emph{darśanāt} was changed to \emph{viniścayāt} in V19 and the \emph{Yogacintāmaṇi}, which indicates that we are not alone in questioning its meaning. The problem with \emph{viniścayāt} in \emph{pāda} two is that \emph{niścayāt} is widely attested in \emph{pāda} one. It should also be noted that the word \emph{tattva} could have a more specific meaning in the \emph{Haṭhapradīpikā} (4.3) as Svātmārāma states that it is a synonym of \emph{samādhi}. In other yoga texts, it can sometimes refer to the practices of yoga (e.g., \emph{tritattva} in \emph{Amṛtasiddhi} 13.12, 14.2--3). 
\end{philcomm}

\subsection*{1.17}
\begin{translation}[hp01_017]
Because it is the first auxiliary of haṭha, \emph{āsana} is taught first. This type of (\emph{tad}) \emph{āsana} brings about steadiness, good health and physical fitness.
\end{translation}

\begin{testimonia}[hp01_017]
Yogacintāmaṇi
\startverse
haṭhasya prathamāṅgatvād āsanaṃ pūrvam ucyate |\\
tat kuryād āsanasthairyam ārogyaṃ cāṅgalāghavaṃ ||
\endverse

Haṭharatnāvalī 3.5
\startverse
haṭhasya prathamāṅgatvād āsanaṃ dārśyate mayā |\\
tat kuryād āsanaṃ sthairyam ārogyaṃ cāṅgapāṭavam || 
\endverse
\end{testimonia}

\begin{philcomm}[hp01_017]        
The reading of \emph{aṅgapāṭavam} is attested among most of the early manuscripts. Although this compound rarely appears in other yoga texts, a similar term \emph{śarīrapāṭava} occurs in the \emph{Śivasaṃhitā} (2.35) as one of the benefits bestowed by digestive fire (\emph{vaiśvānarāgni}), which indicates that the word \emph{pāṭava} was used in relation to the body and the benefits of yoga. The compound \emph{aṅgapāṭava} seems to imply the optimal functioning of the body. However, the alternative reading, \emph{aṅgalāghava} (‘lightness of the limbs’ or ‘dexterity’) is more common in yoga texts, even in works known to Svātmārāma, such as the \emph{Dattātreyayogaśāstra} (\emph{śarīralaghutā}) and the \textit{Amanaska} ([...] \emph{laghutvaṃ ca śarīrasyopajāyate}). Therefore, it is likely that the less common term \emph{aṅgapāṭavam} was changed to the more widely used notion of \emph{aṅgalāghava}, perhaps early on in the transmission, as the latter is attested by group two (i.e., N23, V19, etc.) 
\end{philcomm}

\subsection*{1.18}
\begin{translation}[hp01_018]
I shall now teach some of the postures which have been accepted by both sages (\emph{muni}), such as Vasiṣṭha, and yogis, such as Matsyendra.
\end{translation}

\begin{testimonia}[hp01_018]
Yogacintāmaṇi
\startverse
haṭhapradīpikāyām—\\
vasiṣṭhādyaiś ca munibhir matsyendrādyaiś ca yogibhiḥ |\\
aṅgīkṛtāny āsanāni vakṣyante kānicin mayā ||
\endverse

Haṭharatnāvalī 3.6
\startverse
vasiṣṭhādyaiś ca munibhir matsyendrādyaiś ca yogibhiḥ ||\\
aṃgīkṛtāny āsanāni lakṣyante kāni cin mayā ||
\endverse
\end{testimonia}

\begin{philcomm}[hp01_018]        
On the historical implications of these two traditions of postural practice in early Haṭhayoga, see Mallinson 2016 (119–122) and Birch 2018 (45–46).
\end{philcomm}

\subsection*{1.19}
\begin{translation}[hp01_019]
Correctly placing the soles of both feet between the knees and thighs [and] sitting up with the body straight: they call that \emph{svastikāsana}.
\end{translation}

\begin{sources}[hp01_019]
Śāradātilaka 25.12
\startverse
jānūrvor antare  samyak  kṛtvā  pādatale ubhe |\\
ṛjukāyo viśed yogī svastikaṃ tat pracakṣate ||
\endverse

Vasiṣṭhasaṃhitā 1.68
\startverse
jānūrvor antaraṃ samyak kṛtvā pādatale ubhe |\\
ṛjukāyas tathāsīnaḥ svastikaṃ tat pracakṣate ||
\endverse

Yogayājñavalkya 3.3
\startverse
jānūrvor antare samyak kṛtvā pādatale ubhe\\
ṛjukāyaḥ sukhāsīnaḥ svastikaṃ tat pracakṣate
\endverse
\end{sources}

\begin{testimonia}[hp01_019]
Yogacintāmaṇi
\startverse
Yājñavalkyaḥ—\\
jānūrvor antare samyak kṛtvā pādatale ubhe |\\
ṛjukāyaḥ samāsīnaḥ svastikaṃ tat pracakṣate ||
\endverse

Haṭharatnāvalī 3.52
\startverse
atha svastikāsanam---\\
jānūrvor antaraṃ samyak kṛtvā padatale ubhe ||\\
ṛjukāyasamāsīnaḥ svastikaṃ tat pracakṣate || 
\endverse

\end{testimonia}

\begin{philcomm}[hp01_019]        
One might wonder how the soles of the feet could be placed between the knees and thighs. Brahmānanda explains that the region of the shank near the knee should be understood by the word ‘knee’ in this verse (\emph{atra jānuśabdena jānusaṃnihito jaṅghāpradeśo grāhyaḥ jānusaṃnihito jaṅghāpradeśaḥ}). This is consistent with the earliest known description of \emph{svastikāsana} in the \emph{Pātañjalayogaśāstravivaraṇa} (2.46), which states that the big toes of one foot are tucked in between the shank and thigh of the other so it is not seen (\emph{dakṣiṇaṃ pādāṅguṣṭhaṃ savyenorujaṅghena parigṛhyādṛśyaṃ kṛtvā tathā savyaṃ pādāṅguṣṭhuṃ dakṣiṇenorujaṅkgenādṛśyaṃ parihṛhya yathā ca pārṣṇibhyāṃ vṛṣaṇayor apīḍanaṃ tathā yenāste tat svastikam āsanam}). For a discussion of \emph{svastikāsana} in the Pātañjalayoga tradition, see Maas 2018: 68–69. The descriptions of \emph{svastikāsana} in early Śaiva Tantras do not mention the inserting of the toes between the knees and thighs (see Goodall 2004: 348–350, fn. 371).
\end{philcomm}

\subsection*{1.20}
\begin{translation}[hp01_020]
[The yogi] should place his right heel on the left side of the [lower] back, and the left [heel] on the right [side], in the same way. This is \emph{gomukhāsana}, which [looks] like a cow's face.
\end{translation}

\begin{sources}[hp01_020]
Cf. Ahirbudhnyasaṃhitā 31.45cd–46
\startverse
ubhayor gulphayoḥ kṛtvā pṛṣṭhapārśvāv ubhāv api ||\\
vyutkrameṇātha pāṇibhyāṃ vinyastābhyāṃ vigṛhya ca |\\
prṣṭhagābhyāṃ padāṅguṣṭhāv etad gomukham ucyate || 
\endverse

Vasiṣṭhasaṃhitā 1.70
\startverse
savye dakṣiṇagulphaṃ tu pṛṣṭhapārśve niveśayet |\\
dakṣiṇe 'pi tathā savyaṃ gomukhaṃ tat pracakṣate ||
\endverse

Yogayājñavalkya 3.5cd–3.6ab
\startverse
savye dakṣiṇagulphaṃ tu pṛṣṭhapārśve niveśayet\\
dakṣiṇe ‘pi tathā savyaṃ gomukhaṃ gomukhaṃ yathā
\endverse
\end{sources}

\begin{testimonia}[hp01_020]
Yogacintāmaṇi
\startverse
savye dakṣiṇagulphaṃ tu pṛṣthapārśve niveśayet |\\
dakṣiṇe'pi tathā savyaṃ gomukhaṃ gomukhaṃ yathā ||
\endverse

Haṭharatnāvalī 3.53
\startverse
atha gomukhāsanam---\\
savye dakṣiṇagulphaṃ tu pṛṣṭhapārśve niyojayet ||\\
dakṣiṇe 'pi tathā savyaṃ gomukhaṃ gomukhāsanam ||
\endverse
\end{testimonia}

\begin{philcomm}[hp01_020]        
This posture first appears in some Vaiṣṇava \emph{Saṃhitā}s that predate the \emph{Haṭhapradīpikā}, namely, the \emph{Ahirbudhnyasaṃhitā} and the \emph{Vasiṣṭhasaṃhitā}. The position of the ankles is the same in all the source texts. However, an interesting feature of the description in the \emph{Ahirbudhnyasaṃhitā} is the position of the hands, which are crossed behind the back and hold the big toes. The \emph{Haṭhapradīpikā}’s description, which derives from the \emph{Vasiṣṭhasaṃhitā}, does not mention the position of the hands. For illustrations of six possible positions of the arms and hands, see Gharote, Jha, Devnath, Sakhalkar 2006: 111–113.
\end{philcomm}

\subsection*{1.21}
\begin{translation}[hp01_021]
By fixing one foot on one thigh and placing the [other] thigh on the other foot, vīrāsana is taught [to arise].
\end{translation}

\begin{sources}[hp01_021]
Vasiṣṭhasaṃhitā 1.72
\startverse
ekaṃ pādam athaikasmin vinyasyorau ca saṃsthitam |\\
itarasmiṃs tathaivoruṃ vīrāsanam itīritam ||
\endverse

Cf. Śāradātilakatantra 25.15cd–16ab
\startverse
ekaṃ pādam adhaḥ kṛtvā vinyasyorau tathetaram ||\\
ṛjukāyo viśed yogī vīrāsanam itīritam |
\endverse

Yogayājñavalkya 3.8
\startverse
ekaṃ pādam athaikasmin vinyasyoruṇi saṃsthitam |\\
itarasmiṃs tathā coruṃ vīrāsanam udāhṛtam ||
\endverse
\end{sources}

\begin{testimonia}[hp01_021]
Yogacintāmaṇi
\startverse
ekaṃ pādam athaikasmin vinyasyoruṇi saṃsthitaḥ |\\
itarasmiṃs tathā coruṃ vīrāsanam udāhṛtam ||
\endverse

Haṭharatnāvalī 3.54
\startverse
atha vīrāsanam---\\
ekaṃ pādam athaikasmin vinyased ūruṇi sthiram ||\\
itarasmiṃs tathā coruṃ vīrāsanam īritam || 
\endverse
\end{testimonia}

\begin{philcomm}[hp01_021]   
Although most witnesses have \emph{tathā} in 1.21a, the word \emph{atha} has been accepted because it is attested by V1, the sources and the testimonia. It appears to be verse filler here rather than indicating a temporal sequence of actions. Svātmārāma borrowed the verse on \emph{vīrāsana} from the \emph{Vasiṣṭhasaṃhitā}, the redactor of which appears to have adapted its first hemistich from a description of this posture in the \emph{Śāradātilakatantra}. This would explain the rather strange syntax of the \emph{Vasiṣṭhasaṃhitā}’s version, in which \emph{adhaḥ kṛtvā} was changed to \emph{athaikasmin}, and \emph{tathetaram} became \emph{ca saṃsthitam}. It seems that \emph{saṃsthitaṃ} must be understood with \emph{ūruṃ} in the third \emph{pāda} in the sense of \emph{saṃsthāpya} (i.e., ‘having placed’).

Different versions of \emph{vīrāsana} are found in earlier Tantras, such as the Kiraṇatantra (58.9), Hemacandra’s \emph{Yogaśāstra} and commentaries on the \emph{Pātañjalayogaśāstra}. For a discussion of some of these sources, see Maas 2018: 66–68.
\end{philcomm}

\subsection*{1.22}
\begin{translation}[hp01_022]
Knowers of yoga know that \emph{kūrmāsana} arises by attentively blocking the anus with turned-out ankles.
\end{translation}

\begin{sources}[hp01_022]
Vasiṣṭhasaṃhitā 1.80
\startverse
gudaṃ nirudhya gulphābhyāṃ vyutkrameṇa samāhitaḥ |\\
kūrmāsanaṃ bhaved etad iti yogavido viduḥ ||
\endverse

Cf. Ahirbudhnyasaṃhitā 31.35
\startverse
gudaṃ nipīḍya gulphābhyāṃ vyutkrameṇa samāhitaḥ |\\
etat kūrmāsanaṃ proktaṃ yogasiddhikaraṃ param || 
\endverse
\end{sources}

\begin{testimonia}[hp01_022]
Yogacintāmaṇi
\startverse
gudaṃ niyamya gulphābhyaṃ vyutkrameṇa samāhitaḥ |\\
kūrmāsanaṃ bhaved etad iti yogavido viduḥ ||
\endverse

Yuktabhavadeva 6.15
\startverse
haṭhapradīpikāyām\\
gudaṃ niyamya gulphābhyāṃ vyutkrameṇa samāhitaḥ |\\
kūrmāsanaṃ bhavedetaditi yogavido viduḥ ||
\endverse

\end{testimonia}

\begin{philcomm}[hp01_022]   
In the first \emph{pāda}, the witnesses are split between \emph{nirudhya} (‘having blocked’), \emph{nibadhya} (‘having bound’), \emph{niyamya} (‘having restrained’) and \emph{niṣpīḍya} (‘having pressed’). The source, the \emph{Vasiṣṭhasaṃhitā}, supports \emph{nirudhya} whereas the testimonia supports \emph{niyamya}. In terms of closing the anus by sitting on the ankles, \emph{nirudhya} makes better sense and has been adopted because it is supported by N23 (an important witness of group 2) and the manuscripts reported in the critical edition of the \emph{Vasiṣṭhasaṃhitā}. The word \emph{vyutkrameṇa} appears to describe the position of the ankles. Its basic meaning is ‘against the normal direction’, which would suggest that the ankles are turned out or crossed rather than placed together naturally. If the yogi is in a kneeling-type position, turning the feet out would bring the ankles together, blocking the perineal area. See \emph{Yoga Mīmāṃsā}, vol 8, no. 2, pp. 29–30 for a discussion of \emph{vyutkramena} and the position of the ankles in \emph{kūrmāsana}, and vol 8, no.2, Figures 3–6 for photographs of a practitioner performing this \emph{āsana}. 
\end{philcomm}

\subsection*{1.23}
\begin{translation}[hp01_023]
[The yogi] adeptly assumes \emph{padmāsana}, inserts the hands between the knees and thighs, places [the hands] on the ground, and remains in the air. This is \emph{kurkuṭāsana}.
\end{translation}

\begin{sources}[hp01_023]
Vasiṣṭhasaṃhitā 1.78
\startverse
padmāsanaṃ samāsthāya jānūrvor antare karau |\\
bhūmau niveśya saṃsthāpya vyomasthaṃ kukkuṭāsanam ||\\
{[}niveśya bhūmau – mss. la, va, śa]
\endverse

Cf. Ahirbudhnyasaṃhitā 31.38
\startverse
kukkuṭāsanam\\
padmāsanam adhiṣṭhāya jānvantaraviniḥsṛtau |\\
karau bhūmau niveśyaitad vyomasthaṃ kukkuṭāsanam || 
\endverse
\end{sources}

\begin{testimonia}[hp01_023]
Yogacintāmaṇi
\startverse
padmāsanaṃ tu saṃyojya jānūrvor antare karau | \\
niveśya bhūmau saṃsthāpya vyomasthaṃ kukkuṭāsanam ||
\endverse

Haṭharatnāvalī 3.73
\startverse
atha kukkuṭāsanam---\\
padmāsanaṃ susaṃsthāpya jānūrvor antare karau ||\\
niveśya bhūmau saṃsthāpya vyomasthaḥ kukkuṭāsanam ||
\endverse
\end{testimonia}

\begin{philcomm}[hp01_023]
\emph{Kurkuṭa} and \emph{kurkkuṭa} in V1, J10ac, V3 are variant spellings attested in the \emph{Pañcatantra} (M-W).
\end{philcomm}

\subsection*{1.24}
\begin{translation}[hp01_024]
While maintaining kurkuṭāsana, [the yogi] binds the neck with the hands and lies like a tortoise on his back. This is uttānakūrmāsana.
\end{translation}

\begin{testimonia}[hp01_024]
Yogacintāmaṇi
\startverse
kukkuṭāsanabandhastho dorbhyāṃ saṃbadhya kandharam |\\
bhavet kūrmavad uttānam etad uttānakūrmakam ||
\endverse

Haṭharatnāvalī 3.74
\startverse
kukkuṭāsanabandhastho dorbhyāṃ sambadhya kandharām ||\\
śete kūrmavad uttānam etad uttānakūrmakam || 74 ||
\endverse
\end{testimonia}

\begin{philcomm}[hp01_024]
\emph{°bandhasthaḥ} or \emph{°vat kṛtvā}? Only V1 has the latter, which is simpler. Are the others trying to improve it? Stemmatically ambiguous as \emph{°bandha°} is on one branch (V3/J8, V19) and \emph{°madhya°} the other (J10, J17, N17). V1 looks like an outlier.

Adopt \emph{bandhastho} with note. \emph{°bandhastha} not found in any other texts. While \emph{°vat kṛtvā} is possible, it is a singular reading that appears to be unrelated to the other variants.
\end{philcomm}

\subsection*{1.25}
\begin{translation}[hp01_025]
Clasping the big toes with hands and performing the action of drawing a bow as far as the ear is called dhanurāsana.
\end{translation}

\begin{testimonia}[hp01_025]
Yogacintāmaṇi
\startverse
pādāṅguṣṭhau ca pāṇibhyāṃ gṛhītvā śravaṇāvadhi |\\
dhanurākarṣaṇaṃ kṛtvā dhanurāsanam īritam |
\endverse

Haṭharatnāvalī 3.51
\startverse
atha dhanurāsanam---\\
pādāṅguṣṭhau tu pāṇibhyāṃ gṛhītvā śravaṇāvadhi ||\\
dhanurākarṣaṇaṃ kṛtvā dhanurāsanam ucyate ||
\endverse

Cf. Haṭhayogasaṃhitā
\startverse
dhanurāsanam |\\
prasārya pādau bhuvi daṇḍarūpau \\
karau ca pṛṣṭhe dhṛtapādayugmau |\\
kṛtvā dhanustulyavivarttitāṅgaṃ \\
nigadyate vai dhanurāsanaṃ tat || 25 ||
\endverse
\end{testimonia}

\begin{philcomm}[hp01_025]
Comment on kṛtvā, which is the reading of V19, Yogacintāmaṇi and Haṭharatnāvalī.
\end{philcomm}

\subsection*{1.26}
\begin{translation}[hp01_026]
Having grasped the right foot, which is placed at the base of the left thigh, [the yogi’s] left foot is wrapped around the outside of the knee and he remains with his body twisted. This āsana was taught by Matsyendranātha.
\end{translation}

\begin{testimonia}[hp01_026]
Yogacintāmaṇi
\startverse
vāmorumūlārpitadakṣapādaṃ
jānvor bahir veṣṭitadakṣadoṣṇā |\\
pragṛhya tiṣṭhet parivartitāṅgaḥ
śrīmatsyanāthoditam āsanaṃ syāt ||
\endverse

Haṭharatnāvalī
\startverse
atha matsyendrāsanam---\\
vāmorumūlārpitā dakṣapādo jānvor bahir veṣṭitadakṣadoṣṇā ||\\
pragṛhya tiṣṭhet partivartitāṅgaḥ śrīmatsyanāthoditam āsanaṃ syāt || 3.57 ||
\endverse
\end{testimonia}

\subsection*{1.27}
\begin{translation}[hp01_027]
Matsyendra's seat, by which the digestive fire is kindled, is a destructive weapon for a range of terrible diseases; through practice it brings about in people the awakening of Kuṇḍalinī and steadiness of the spine.
\end{translation}

\begin{testimonia}[hp01_027]
Yogacintāmaṇi
\startverse
matsyendrapīṭhaṃ jaṭharapravṛddhiṃ\\
pracaṇḍaruṅmaṇḍalakhaṇḍanāstram |\\
abhyāsataḥ kuṇḍalinīprabodhaṃ\\
daṇḍe sthiratvaṃ pradadāti puṃsām ||
\endverse

Haṭharatnāvalī 3.58
\startverse
matsyendrapīṭhaṃ jaṭharapradīptaṃ \\
pracaṇḍarugmaṇḍalakhaṇḍanāstram ||\\
abhyāsataḥ kuṇḍalinīprabodhaṃ \\
daṇḍasthiratvaṃ ca dadāti puṃsām ||
\endverse

HTK
\startverse
matsyendrapīṭhaṃ jaṭharapracaṇḍa-\\
ruṅmaṇḍalakhaṇḍanakhaṇḍanāstram |\\
abhyāsataḥ kuṇḍalinīprabodhaṃ \\
daṇḍasthiratvaṃ ca dadāti puṃsām || 8 ||
\endverse

YBD
\startverse
matsyendrapīṭhaṃ jaṭharaprabuddhaṃ\\
pracaṇḍaruṅmaṇḍalakhaṇḍanāstram |\\
abhyasataṃ kuṇḍalinīprabodhaṃ\\
daṇḍasthiratvaṃ ca dadāti puṃsām ||
\endverse
\end{testimonia}

\begin{philcomm}[hp01_027]
Difficult to construe first half of verse:
We have not found an instance in another verse where \emph{jaṭhara} means \emph{jaṭharāgni}. So, the reading \emph{jvalana}° is better and most of the witnesses that have it attest \emph{jvalanapradīptam}. This compound has been accepted and understood as a reverse \emph{bahuvrīhi} (i.e., ‘the āsana by which the digestive fire is kindled). J10 and J17 have \emph{jaṭharapravṛddha}, which would qualify \emph{pracaṇḍaruṅmaṇḍala}°, but it seems rather to strange to say that a range of terrible diseases have increased in the stomach, rather than arisen in the stomach. KDham ed has variant \emph{jaṭharapradīpaṃ} but this is not attested among the oldest dated manuscripts.

Pāda d, J10ac and J17 have \emph{candra} for \emph{daṇḍa}; However,  also in Yogacintāmaṇi and 6-chapter HP.
\end{philcomm}

\subsection*{1.28}
\begin{translation}[hp01_028]
[The yogi] should stretch out both feet on the ground like staffs, hold the ends of both feet with the hands, place the forehead upon the knees and remain thus. They call this the back-stretch (\emph{paścimatānam}).
\end{translation}

\begin{sources}[hp01_028]
Cf. Śivasaṃhitā
\startverse
prasārya caraṇadvandvaṃ parasparasusaṃyutam |\\
svapāṇibhyāṃ dṛḍhaṃ dhṛtvā jānūpari śiro nyaset || 3.108 ||
\endverse
\end{sources}

\begin{testimonia}[hp01_028]
Yogacintāmaṇi
\startverse
prasārya pādau bhuvi daṇḍarūpau \\
dvābhyāṃ ca pādadvitayaṃ gṛhītvā |\\
jānūpari nyastalalāṭadeśo \\
'bhyased idaṃ paścimatānam āhuḥ ||
\endverse

Haṭharatnāvalī
\startverse
atha paścimatānāsanam---\\
prasārya pādau bhuvi daṇḍarūpau \\
dorbhyāṃ padāgradvitayaṃ gṛhītvā ||\\
jānūpari nyastalalāṭadeśo \\
vased idaṃ paścimatānam āhuḥ || 3.66 ||
\endverse
\end{testimonia}

\begin{philcomm}[hp01_028]
Only V1 has \emph{dorbhyāṃ padāgra}, others have variations on the much inferior \emph{dvābhyāṃ karābhyāṃ}. The karābhyaṃ reading was probably introduced to remove dorbhyāṃ, which usually means arms.The reading \emph{tāṇabandhaḥ} in V1 doesn’t work with \emph{idaṃ}.
\end{philcomm}

\subsection*{1.29}
\begin{translation}[hp01_029]
This back-stretch is the foremost among āsanas. It makes the breath flow in the back [i.e. Central channel], increases the digestive fire, makes the belly thin and prevents diseases in men.
\end{translation}

\begin{sources}[hp01_029]
Cf. Śivasaṃhitā
\startverse
Āsanāgryam idaṃ proktaṃ jaṭharānaladīpanam |\\
dehāvasādaharaṇaṃ paścimottānasaṃjñakam || 3.109 ||
\endverse
\end{sources}

\begin{testimonia}[hp01_029]
Yogacintāmaṇi
\startverse
iti paścimatānam āsanāgryaṃ
pavanaṃ paścimavāhinaṃ karoti |\\
udayaṃ jaṭharānalasya kuryād
udare kārśyam arogitāṃ ca puṃsām ||
\endverse

Haṭharatnāvalī
\startverse
iti paścimatānam āsanāgryaṃ pavanaṃ paścimavāhinaṃ karoti ||\\
udayaṃ jaṭharānalasya kuryād udare kārśyam arogatāṃ ca puṃsām || 3.67 ||
\endverse
\end{testimonia}

\begin{philcomm}[hp01_029]
Adopt \emph{arogitāṃ} (which is well attested), despite the fact that \emph{arogatāṃ} is much more common. The use of the word \emph{paścima} to mean the central channel is found in \emph{Yogabīja} 121. The  \emph{Yogabīja} (95) also refers to the path of the central channel (\emph{paścimamārga}), and this understanding of \emph{paścima} is found in the \emph{Jyotsnā 1.29}: \emph{paścimavāhinaṃ paścimena paścimamārgeṇa suṣumnāmārgeṇa vahatīti paścimavāhī}.
\end{philcomm}

\subsection*{1.30}
\begin{translation}[hp01_030]
Supporting oneself on the ground with both hands, their elbows placed on either side of the navel, with a raised position (? uccāsanaḥ) one is placed up into the air [as straight] as a stick. They call this posture the peacock.
\end{translation}

\begin{sources}[hp01_030]
Vimānārcanākalpa 96
\startverse
karatale bhūmau saṃsthāpya kūrparau nābhipārśvayor nyasya nataśirāḥ (unnataśirāḥ) pādau ḍaṇḍavad vyomni saṃsthito mayūrāsanam iti |
\endverse

Pādmasaṃhitā yogapāda 1.21c–22d:
\startverse
avaṣṭabhya dharāṃ samyak talābhyāṃ hastayor dvayoḥ ||\\
kūrparau nābhipārśve ca sthāpayitvā mayūravat |\\
samunnamya śiraḥpādau mayūrāsanam iṣyate ||
\endverse

Ahirbudhnyasaṃhitā 31.36–37
\startverse
mayūrāsanam\\
niveśya kūrparau samyaṅ nābhimaṇḍalapārśvayoḥ |\\
avaṣṭabhya bhuvaṃ pāṇitalābhyāṃ vyomni daṇḍavat || 31--36 ||
\endverse

Vasiṣṭhasaṃhitā 1.76–77
\startverse
avaṣṭabhya dharāṃ samyak talābhyāṃ ca karadvayam |\\
hastayoḥ kūrparau cāpi sthāpayan nābhipārśvayoḥ ||\\
samunnataśiraḥpādo daṇḍavad vyomni saṃsthitaḥ |\\
mayūrāsanam etad dhi sarvapāpavināśanam ||
\endverse

Yogayājñavalkya 3.15–16
\startverse
avaṣṭabhya dharāṃ samyak talābhyāṃ tu karadvayoḥ\\
hastayoḥ kūrparau cāpi sthāpayan nābhipārśvayoḥ\\
samunnataśiraḥpādo daṇḍavad vyomni saṃsthitaḥ |\\
mayūrāsanam etat tu sarvapāpapraṇāśanam ||
\endverse
\end{sources}

\begin{testimonia}[hp01_030]
Yogacintāmaṇi
\startverse
dharām avaṣṭabhya punaḥ karābhyāṃ
tat kūrpare sthāpitanābhipārśvaḥ | \\
tadāsane daṇḍavad utthitaḥ khe
mayūram etat pravadanti santaḥ ||
\endverse

Haṭharatnāvalī 3.42
\startverse
atha mayūram\\
dharām avaṣṭabhya karadvayena tatkūrpare sthāpitanābhipārśvaḥ ||\\
uccāsano daṇḍavad utthitaḥ khe mayūram etat pravadanti pīṭham ||
\endverse
\end{testimonia}

\begin{philcomm}[hp01_030]
The source of this verse is unknown, but it has the same elements as the two verses in the \emph{Vasiṣṭhasaṃhitā} (1.76–77). The compound \emph{uccāsano} seems to approximate in a somewhat vague way the \emph{Vasiṣṭhasaṃhitā}’s reading \emph{samunnataśiraḥpādaḥ}, which may be derived from earlier Vaiṣṇava sources, such as the \textit{Pādmasaṃhitā}. 
 

1.30b tat refers to \emph{karadvaya} (cf. Vasiṣṭhasaṃhitā).
\end{philcomm}

\subsection*{1.31}
\begin{translation}[hp01_031]
The glorious mayūra posture gets rid of all diseases of the abdomen such as bloating and overcomes humoral imbalances. It completely incinerates food which is bad or has been eaten to excess, it generates digestive fire and it digests strong poison.
\end{translation}

\begin{testimonia}[hp01_031]
Yogacintāmaṇi
\startverse
harati sakalarogān āśu gulmodarādīn\\
abhibhavati ca doṣān āsanaṃ śrīmayūram |\\
bahukadaśanabhuktaṃ bhasma kuryād aśeṣam\\
janayati jaṭharāgniṃ jārayet kālakūṭam ||
\endverse

Haṭharatnāvalī 3.43
\startverse
harati sakalarogān āśu gulmodarādīn\\
abhibhavati ca doṣān āsanaṃ śrīmayūram ||\\
bahukadaśanabhuktaṃ bhasma kuryād vicitram\\
janayati jaṭharāgniṃ jīryate kālakūṭam ||
\endverse
\end{testimonia}

\begin{philcomm}[hp01_031]
\emph{aśeṣaṃ} better
\end{philcomm}

\subsection*{1.32}
\begin{translation}[hp01_032]
Lying on one’s back on the ground like a corpse is the corpse posture. It removes the fatigue [caused by practising] any āsana and relaxes the mind.
\end{translation}

\begin{sources}[hp01_032]
Cf. Dattātreyayogaśāstra 24cd
\startverse
uttānaśavavad bhūmau śayanaṃ coktam uttamam ||
\endverse
\end{sources}

\begin{testimonia}[hp01_032]
Yogacintāmaṇi
\startverse
uttānaṃ śavavad bhūmau śavāsanam idaṃ smṛtam |\\
śavāsanaṃ śrāntiharaṃ cittaviśrāntisādhanam ||
\endverse

Haṭharatnāvalī 3.76
\startverse
athāntimaṃ śavāsanam\\
prasārya hastapādau ca viśrāntyā śayanaṃ tathā ||\\
sarvāsanaśramaharaṃ śayitaṃ tu śavāsanam || 3.76 ||
\endverse

Cf. HTK
\startverse
śavāsanaṃ hṛtkupitavātagranthivibhedakam |\\
sarvāsanaśrāntijit hṛtśramaghnaṃ yogi saukhyadam || 12 ||
\endverse

YBD
\startverse
uttānaṃ śavavad bhūmau śayanaṃ tu śavāsanam ||\\
śavāsanaṃ śrāntiharaṃ cittaviśrāntikārakam ||\\
iti śavāsanam || 21 ||
\endverse
\end{testimonia}

\subsection*{1.33}
\begin{translation}[hp01_033]
Śiva taught eighty-four āsanas. I shall take the four best and teach them.
\end{translation}

\begin{sources}[hp01_033]
Śivasaṃhitā 3.96
\startverse
caturaśīty āsanāni santi nānāvidhāni ca |\\
tebhyaś catuṣkam ādāya mayoktāni bravīmy aham ||
\endverse

Cf. Dattātreyayogaśāstra 5
\startverse
caturāśītilakṣānām ekaikaṃ samudāhṛtaṃ |\\
ataḥ śivena pīṭhānāṃ ṣoḍaśonaṃ śataṃ kṛtaṃ ||
\endverse

Cf. Vivekamārtaṇḍa 5
\startverse
caturāśītilakṣānām ekaikaṃ samudāhṛtaṃ |\\
ataḥ śivena pīṭhānāṃ ṣoḍaśonaṃ śataṃ kṛtaṃ ||
\endverse
\end{sources}

\begin{testimonia}[hp01_033]
Yogacintāmaṇi
\startverse
haṭhapradīpikāyām—\\
caturaśīty āsanāni śivena kathitāni vai |\\
tebhyaś catuṣkam ādāya sārabhūtaṃ bravīmy aham ||
\endverse

Haṭharatnāvalī
\startverse
caturaśīty āsanāni śivena kathitāni tu |\\
tebhyaś catuṣkam ādāya sārabhūtaṃ bravīmy aham || 3.23 ||
\endverse
\end{testimonia}

\begin{philcomm}[hp01_033]
In Pādas ac Svātmārāma may have rewritten Śivasaṃhitā 3.96 to include the information that it was Śiva (\emph{śivena}) who taught the 84 āsanas, whereas in the source Śiva is himself speaking. With this
Svātmārāma confuses the verse, since one would have to understand Pādas cd to be direct speech by Śiva in order to understand \emph{bravīmi} correctly (but cf. 4.2 which is a verse from the Gorakṣaśataka which includes pravakṣyāmi).

The Śivasaṃhitā follows 3.96 with teachings on siddha, padma, paścimottāna and svastika āsanas, but the HP teaches siddha, padma, siṃha and bhadra.

The word \emph{tu} is often used to introduce a new posture, but in this verse seems to be a verse filler.
\end{philcomm}

\subsection*{1.34}
\begin{translation}[hp01_034]
The adept, lotus, lion and auspicious pose are the best tetrad and, among those, always sit in the adept’s pose, my dear.
\end{translation}

\begin{sources}[hp01_034]
--
\end{sources}

\begin{testimonia}[hp01_034]
Yogacintāmaṇi
\startverse
siddhaṃ padmaṃ tathā bhadraṃ siṃhaṃ ceti catuṣṭayam |\\
śreṣṭhaṃ tatrāpi vai padmaṃ tiṣṭhet siddhāsane sadā ||
\endverse

Haṭharatnāvalī 3.24
\startverse
siddhaṃ padmaṃ tathā siṃhaṃ bhadraṃ ceti catuṣṭayam |\\
śreṣṭhaṃ tatrāpi ca tathā tiṣṭhet siddhāsane sadā ||
\endverse
\end{testimonia}

\begin{philcomm}[hp01_034]
It is likely that the original version of this verse contained the vocative with the imperative form of the verb. There is another instance where Svātmārāma included a verse with the vocative (4.58?) or the subject in the first person (4.2), as though the text were a dialogue. In this case, it seems that efforts have been made to write out the vocative and imperative verb. However, the \emph{sukhe} and \emph{sukham} is difficult to construe in this context, because the context suggests that the intended meaning was that one should always sit in Siddhāsana (as opposed to the other three), rather than the prescription to always sit in a comfortable Siddhāsana.   
\end{philcomm}

\subsection*{1.35}
\begin{translation}[hp01_035]
Having joined the place of the perineum with the heel of the foot, the yogi should firmly fix the [other] foot on the penis. Having held the face and chest together and the body erect, [the yogi] remains still, his senses restrained, gazing between the brows with his eyes unmoving. This breaks open the door to liberation and is called the adept’s pose.
\end{translation}

\begin{sources}[hp01_035]
Vivekamārtaṇḍa 7
\startverse
yonisthānakam aṅghrimūlaghaṭitaṃ kṛtvā dṛḍhaṃ vinyase[n] \\
meḍhre pādam athaikam āsyahṛdaye dhṛtvā samaṃ vigraham |\\
sthāṇuḥ saṃyamitendriyo 'caladṛśā paśyan bhruvor antaraṃ\\
etan  mokṣakapāṭabhedajanakaṃ siddhāsanaṃ procyate ||
\endverse
\end{sources}

\begin{testimonia}[hp01_035]
Yogacintāmaṇi
\startverse
pavanayogasaṃgrahe—\\
yonisthānakam aṅghrimūlaghaṭitaṃ kṛtvā dṛḍhaṃ vinyasen\\
meḍhre pādam athaikam ekahṛdaye kṛtvā samaṃ vigraham |\\
sthāṇuḥ saṃyamitendriyo 'caladṛśā paśyed bhruvor antaraṃ tv\\
etan mokṣakapāṭabhedanakaraṃ siddhāsanaṃ procyate ||
\endverse

Haṭharatnāvalī 3.25
\startverse
tatra siddhāsanam\\
yonisthānakam aṅghrimūlaghaṭitaṃ kṛtvā dṛḍhaṃ vinyasen\\
meḍhre pādam athaikam eva niyataṃ kṛtvā samaṃ vigraham |\\
sthāṇuḥ saṃyamitendriyo 'caladṛśā paśyan bhruvor antaraṃ\\
caitan mokṣakapāṭabhedajanakaṃ siddhāsanaṃ procyate ||
\endverse
\end{testimonia}

\begin{philcomm}[hp01_035]
KDh ed has \emph{āsya}, variant in mss kh and gha, both BORI.
Jürgen suggests we might take \emph{āsyahṛdaye} and \emph{vigraham} as the objects of \emph{dhṛtvā} and samam as an adverb (i.e., holding straight the face, chest and body). However, in a different doctrine.
\end{philcomm}

\subsection*{1.36}
\begin{translation}[hp01_036]
Having fixed the left heel on the penis, and put the other heel on that, this is siddhāsana.
\end{translation}

\begin{sources}[hp01_036]
Vasiṣṭhasaṃhitā 1.81
\startverse
meḍhrād upari nikṣipya gulphaṃ tathopari |\\
gulphāntaraṃ vinikṣipya muktāsanam idaṃ smṛtam ||
\endverse

Yogayājñavalkya 3.15
\startverse
meḍhrād upari nikṣipya savyaṃ gulphaṃ tathopari |\\
gulphāntaraṃ ca nikṣipya muktāsanam idaṃ tu vā ||
\endverse
\end{sources}

\begin{testimonia}[hp01_036]
Yogacintāmaṇi
\startverse
tathā |\\
meḍhrād upari vinyasya savyaṃ gulphaṃ tathopari |\\
gulphāntaraṃ tu vinyasya siddhāsanam idaṃ bhavet ||
\endverse

Haṭharatnāvalī 3.26
\startverse
matāntare tu\\
meḍhrād upari niḥkṣipya savyaṃ gulphaṃ tathopari |\\
gulphāntaraṃ ca niḥkṣipya siddhāḥ siddhāsanaṃ viduḥ ||
\endverse
\end{testimonia}

\subsection*{1.37}
\begin{translation}[hp01_037]
Some proclaim this is siddhāsana, others know it as vajrāsana, a few say it is muktāsana and some guptāsana.
\end{translation}

\begin{sources}[hp01_037]
--
\end{sources}

\begin{testimonia}[hp01_037]
Yogacintāmaṇi
\startverse
siddhaṃ padmaṃ tathā bhadraṃ siṃhaṃ ceti catuṣṭayam |\\
śreṣṭhaṃ tatrāpi vai padmaṃ tiṣṭhet siddhāsane sadā ||
\endverse

Haṭharatnāvalī
\startverse
siddhaṃ padmaṃ tathā siṃhaṃ bhadraṃ ceti catuṣṭayam |\\
śreṣṭhaṃ tatrāpi ca tathā tiṣṭhet siddhāsane sadā || 3.24 ||
\endverse

Cf. Śivayogasāram by Kolani Ganapatideva (date 14th c.)
\startverse
siddāsanambunu, gondaru vajrāsanambaniyu | \\
gondaru muktāsanambaniyu, gondadu  gulbāsanam ||
\endverse

Cf. A verse by the poet Pingali Surana (active 16th c.)
\startverse
kondaru siddāsanamani \\
kondaru vajrāsanamani koniyādudurī \\
pondaga dīnini mariyoka \\
kondaru guptāsamanu kondru mahātmā
\endverse
\end{testimonia}

\subsection*{1.38}
\begin{translation}[hp01_038]
The Siddhas know Siddhasana as the single most important amongst
all postures, in the same way as measured diet amongst rules and non-violence amongst observances.
\end{translation}

\begin{sources}[hp01_038]
Cf. Dattātreyayogaśāstra 33
\startverse
laghvāhāras tu teṣv eko mukhyo bhavati nāpare |\\
ahiṃsā niyameṣv eko mukhyo bhavati nāpare || 33 ||
\endverse
\end{sources}

\begin{testimonia}[hp01_038]
Yogacintāmaṇi
\startverse
niyameṣu mitāharo yathāhịmsā yameṣv iva |\\
mukhyaṃ sarvāsaneṣv evaṃ siddhāsanaṃ idaṃ viduḥ |
\endverse
\end{testimonia}

\begin{philcomm}[hp01_038]
\emph{iva} or \emph{eva}? \emph{iva} does work — like \emph{siddhāsana}, \emph{mitāhāra} and \emph{ahiṃsā} are the best, but for it to work properly \emph{mitāhāra} and \emph{ahiṃsā} should be accusative. V19 has acc + \emph{iva}, which seems best, especially with \emph{siddhāḥ viduḥ}, but this might be a correction as V19 often corrects. However, one old KDham BORI (?) ms (pha, 1695 CE) has it, as does Jyotsnā, so adopt.

In pāda d V19 has \emph{siddhāsanam idaṃ viduḥ}, but the reading of all other mss is preferable.

Clearly based on DYŚ 33, which includes \emph{ekaṃ} and \emph{mukhya}.
\end{philcomm}

\subsection*{1.39}
\begin{translation}[hp01_039]
Among the eighty-four postures, one should regularly practise just Siddha; in the same way one should practise Suṣumnā among the 72,000 channels.
\end{translation}

\begin{sources}[hp01_039]
\end{sources}

\begin{testimonia}[hp01_039]
Yogacintāmaṇi
\startverse
caturaśītipīṭheṣu siddhāsanaṃ samabhyaset | \\
dvāsaptatisahasreṣu suṣumṇām iva nāḍiṣu ||
\endverse

YSS
\startverse
caturāśītapīṭheṣu siddham eva samabhyaset |\\
dvisaptatisahasreṣu suṣumnām iva nāḍiṣu ||
\endverse

Yogacintāmaṇi (in a different passage)
​​\startverse
maṇḍalā dṛśyate siddhiḥ kuṇḍalyabhyāsayoginaḥ |\\
dvisaptatisahasrāṇāṃ nāḍīnāṃ malaśodhanam ||
\endverse

Cf. Kumbhakapaddhati (effects of practising kumbhaka)
\startverse
dvāsaptati sahāsrāṇāṃ nāḍīnāṃ malaśodhanam |\\
yatheṣṭaṃ dhāraṇaṃ vāyor vikārābhāva eva ca || 120 ||
\endverse
\end{testimonia}

\begin{philcomm}[hp01_039]
Odd to have \emph{suṣumnām} as object of \emph{abhyaset}. Yogacintāmaṇi has this reading though. Perhaps cd were added somewhat indiscriminately by Svātmārāma (with nominative \emph{suṣumnā}) and then others tried to make sense of it.

Some witnesses, including Jyotsnā, have \emph{nāḍīnāṃ malaśodhanam/e} for pāda d, which is probably a patch (no other texts say siddhāsana clears the channels), but cf. Amṛtasiddhi in which the practices are said to bring about cālana of the nāḍīs (e.g. 11.6).

Good example of early contamination.

[\emph{nāḍiṣu} is better supported (J10ac,V19,J17).]
\end{philcomm}

\subsection*{1.40}
\begin{translation}[hp01_040]
By meditating upon the self, restricting the diet and regularly practising Siddhāsana for twelve years, the yogi attains the Niṣpatti stage.
What’s the point of lots of exhausting postures when there is Siddhāsana?
\end{translation}

\begin{sources}[hp01_040]
--
\end{sources}

\begin{testimonia}[hp01_040]
Yogacintāmaṇi
\startverse
ātmadhyāyī mitāhārī yāvad dvādaśavatsaram |\\
sadā siddhāsanābhyāsād yogī niṣpattim āpnuyāt |\\
śramadair bahubhiḥ pīṭhaiḥ kiṃ syāt siddhāsane sati ||
\endverse

YSS
\startverse
ātmadhyāyo mitāhārī yāvad dvādaśavatsaram |\\
sadā siddhāsanābhyāsād yoganiṣpattim āpnuyāt ||\\
śramadair bahubhiḥ pīṭhair alaṃ siddhāsane sati |
\endverse
\end{testimonia}

\begin{philcomm}[hp01_040]
Only possible variant is \emph{mitāhāro} in V19.
J8 might be correction of J10’s unmetrical reading.

[\emph{sadāsiddhāsanābhyāsād}? Or maybe read \emph{sadā} with \emph{avāpnuyāt}]

V1 has
\startverse
śramādau bahubhiḥ pīṭhais sadā siddhāsane sati |\\
prāṇānile sāvadhānaṃ baddhe kevalakumbhake || 1.41 ||
\endverse

Mixing up both versions of the verse — contamination already?

V19 is found in Yogacintāmaṇi: \emph{śramadair bahubhịh pīṭhaiḥ kiṃ syāt siddhāsane sati}; JM this seems best to me.

Is this notion of āsanas causing \emph{śrama} already current in HY texts?
[Yes, it is mentioned above in the verse on śavāsana]
\end{philcomm}


\subsection*{1.41}
\begin{translation}[hp01_041]
Just as the Unmanī [state] arises automatically, without effort, when the prāṇa breath has been carefully stopped in kevala kumbhaka, [...]
\end{translation}

\begin{sources}[hp01_041]
\end{sources}

\begin{testimonia}[hp01_041]
Yogacintāmaṇi
\startverse
prāṇānile sāvadhāne baddhe kevalakumbhake |\\
utpatsyate nirāyāsāt svayam evonmanī yathā ||
\endverse
\end{testimonia}

\begin{philcomm}[hp01_041]
--
\end{philcomm}

\subsection*{1.42}
\begin{translation}[hp01_042]
So too the three bandhas arise automatically, without effort, when Siddhāsana alone is always firmly adopted,
\end{translation}

\begin{sources}[hp01_042]
--
\end{sources}

\begin{testimonia}[hp01_042]
Yogacintāmaṇi
\startverse
athaikasminn eva dṛḍhaṃ baddhe siddhāsane sadā |\\
bandhatrayam anāyāsāt svayam evopajāyate |
\endverse

YSS
\startverse
tathaikasminn eva baddhe siddhāsane sadā |\\
granthitrayam anāyāsāt svayamevopabhidyate |
\endverse
\end{testimonia}

\begin{philcomm}[hp01_042]
\emph{dṛḍhe} goes across stemma. It is likely that the adverb (\emph{dṛdhaṃ}) was intended because \emph{dṛdhataraṃ}, which is not ambiguous, is used in 1.48a to qualify how \emph{padmāsana} should be adopted.
\end{philcomm}

\subsection*{1.43}
\begin{translation}[hp01_043]
There is no posture like siddhāsana, no breath like the restrained breath, no mudrā like khecarī, no dissolution like nāda.
\end{translation}

\begin{sources}[hp01_043]
Śivasaṃhitā 5.47
\startverse
nāsanaṃ siddhasadṛśaṃ na kumbhasadṛśaṃ balam |\\
na khecarīsamā mudrā na nādasadṛśo layaḥ || 5.47 ||
\endverse
\end{sources}

\begin{testimonia}[hp01_043]
Yogacintāmaṇi
\startverse
nāsanaṃ siddhasadṛśaṃ na kumbhaḥ kevalopamaḥ |\\
na khecarīsamā mudrā na nādasadṛśo layaḥ ||
\endverse

Haṭharatnāvalī
\startverse
nāsanaṃ siddhasadṛśaṃ na kumbhaḥ kevalopamaḥ ||\\
na khecarīsamā mudrā na nādasadṛśo layaḥ || 3.29 ||
\endverse
\end{testimonia}

\begin{philcomm}[hp01_043]
\emph{na kumbhasadṛśo nilaḥ} is difficilior lectio and attested by all early witnesses except V19: (almost?) all testimonia have \emph{kumbhaḥ kevalopamaḥ}; Śivasaṃhitā has \emph{kumbhasadṛśaṃ balam}.
\end{philcomm}

\subsection*{1.44}
\begin{translation}[hp01_044]
Now lotus pose.
One should place the right foot on the left thigh, and the left on the right though, hold firmly the big toes with the hands behind the back, place the chin on the chest and gaze at the tip of the nose. This is called lotus pose, which destroys diseases for those who undertake the yamas.
\end{translation}

\begin{sources}[hp01_044]
Vivekamārtaṇḍa 8
\startverse
vāmorūpari dakṣiṇañ ca caraṇaṃ saṃsthāpya vāmaṃ tathā\\
yāmyorūpari paścimena vidhinā dhṛtvā karābhyāṃ dṛḍham |\\
aṅguṣṭhau hṛdaye nidhāya cibukaṃ nāsāgram ālokayed\\
etad vyādhivikārahāri yamināṃ padmāsanaṃ procyate || 8 ||
\endverse
\end{sources}

\begin{testimonia}[hp01_044]
Yogacintāmaṇi
\startverse
haṭhayoge—\\
vāmorūpari dakṣiṇaṃ hi caraṇaṃ saṃsthāpya vāmaṃ tathā\\
dakṣorūpari paścimena vidhinā dhṛtvā karābhyāṃ dṛḍham | \\
aṅguṣṭhau hṛdaye nidhāya civukaṃ nāsāgram ālokayet\\
ekad vyādhivikāranāśanakaraṃ padmāsanaṃ procyate ||
\endverse

Haṭharatnāvalī
\startverse
vāmorūpari dakṣiṇaṃ ca caraṇaṃ saṃsthāpya vāmaṃ tathā\\
yāmyorūpari paścimena vidhinā dhṛtvā karābhyāṃ dṛḍham |\\
aṅguṣṭhau hṛdaye nidhāya cibukaṃ nāsāgram ālokayed\\
etad vyādhivināśakāri yamināṃ padmāsanaṃ procyate || 3.34 ||
\endverse
\end{testimonia}

\subsection*{1.45--47}
\begin{translation}[hp01_045]
However, in another view,
Having put the upturned feet carefully on the thighs and the upturned hands in the middle of the thighs, one should fix the eyes on the tip of the nose. Having raised the root of the uvula with the tongue, one should place the chin on the chest and gradually [draw in\footnote{The verb \emph{ākṛśya} follows in the next verse in the \emph{Dattātreyayogaśāstra}.}] the breath [...].
\end{translation}

\begin{sources}[hp01_045]
Dattātreyayogaśāstra 35–37
\startverse
uttānau caraṇau kṛtvā ūrusaṃsthau prayatnataḥ |\\
ūrumadhye tathottānau pāṇī kṛtvā tato dṛśau || 35 ||\\
nāsāgre vinyased rājadantamūlaṃ ca jihvayā |\\
uttabhya cibukaṃ vakṣasy āsthāpya pavanaṃ śanaiḥ || 36 ||\\
yathāśaktyā samākṛṣya pūrayed udaraṃ śanaiḥ |\\
yathāśaktyaiva paścāt tu recayet pavanaṃ śanaiḥ || 37 ||
\endverse

Śivasaṃhitā 3.102–104
\startverse
uttānau caraṇau kṛtvā ūrusaṃsthau prayatnataḥ\\
ūrumadhye tathottānau pāṇī kṛtvā tu tādṛśau 3.102\\
nāsāgre vinyased dṛṣṭiṃ rājadantaṃ ca jihvayā\\
uttambhya cibukaṃ vakṣe saṃsthāpya pavanaṃ śanaiḥ 3.103\\
yathāśaktyā samākṛṣya pūrayed udaraṃ śanaiḥ\\
yathāśaktyaiva paścāt tu recayed anirodhataḥ 3.104
\endverse
\end{sources}

\begin{testimonia}[hp01_045]
Yogacintāmaṇi
\startverse
dattātreyaḥ—\\
uttānau caraṇau kṛtvā ūrusaṃsthau prayatnataḥ |\\
ūrumadhye tathottānau pāṇī kṛtvā tato dṛśau || = 1.45\\
nāsāgre vinyased rājadantamūlaṃ tu jihvayā |\\
uttabhya civukaṃ vakṣasy utthāpya pavanaṃ śanaiḥ || = 1.46\\
yathāśaktyā samākṛṣya pūrayed udaraṃ śanaiḥ |\\
yathāśaktyaiva paścāt tu recayet pavanaṃ śanaiḥ ||
\endverse

Haṭharatnāvalī
\startverse
dattātreyo 'pi\\
uttānau caraṇau kṛtvā ūrvoḥ saṃsthāpya yatnataḥ |\\
ūrumadhye tathottānau pāṇī kṛtvā tato dṛśau || 3.36 ||\\
nāsāgre vinyased rājadantamūlaṃ ca jihvayā |\\
uttabhya cibukaṃ vakṣaḥ saṃsthāpya pavanaṃ śanaiḥ || 3.37 ||
\endverse
\end{testimonia}

\begin{philcomm}[hp01_045]
\emph{uttabhya} vs \emph{uttambhya}.
The witnesses split predictably along the two main branches of the stemma. The evidence of the DYŚ is important here.
%\emph{vakṣasyāsthāpya} is a marmasthāna

The statement ending with \emph{pavanaṃ śanaiḥ} is left hanging, perhaps, because of an eyeskip that happened early in the transmission. The subsequent verse in the DYŚ ends with \emph{pavanaṃ śanaiḥ}.

\startverse
nāsāgre vinyased rājadantamūlaṃ ca jihvayā |\\
uttabhya cibukaṃ vakṣasy āsthāpya pavanaṃ śanaiḥ || 36 ||\\
yathāśaktyā samākṛṣya pūrayed udaraṃ śanaiḥ |\\
yathāśaktyaiva paścāt tu recayet pavanaṃ śanaiḥ || 37 ||\\
idaṃ padmāsanaṃ proktaṃ sarvavyādhivināśanam |
\endverse

The manuscript readings with \emph{vakṣa sthāpayet} (J7, V3, J8, J10, J17, N17) or something similar (V1, W4) do not offer a solution nor indicate how Svātmārāma may have redacted this to make the syntax complete. Instead, it seems that he quoted these two verses (1.45–46) from the Dattātreyayogaśāstra and simply omitted the next verse that made sense of \emph{pavanaṃ śanaiḥ} because it was not relevant to the posture itself.

% Jü:
[JH] The background of the passage \emph{rājadantamūlaṃ ca jihvayā uttambhya} in 1.46 is more complex
than it may appear. Here it is a literal quotation from the DYŚ, but many other Haṭhayogic texts
teach a particular position of the tongue, in which it is in one or the other way turned back in
the direction of the uvula, as we read explicitly in the \emph{Vivekamārtāṇḍa} (REF):
\emph{kapālakuhare jihvā praviṣṭā viparītagā}. Brief references to this practice can become
ambiguous for various reasons, and this has possibly confused Brahmānanda.

One reason is that there is a, probably older, rule for meditation postures according to which the
tongue rests near the teeth. One instance would be \emph{Svacchandatantra}, which teaches a
meditation pose called \emph{divyaṃ karaṇam} (4.365f.), in which the tongue is to rest at the tip
of the teeth (\emph{dantāgre jihvām ādāya}). Other Tantric texts have this or similar rules, in
which the tongue is supposed to rest either on the teeth or the palate,\footnote{This rule is
found in \emph{Iśānaśivapaddhati}: \emph{tāluke jihvāṃ saṃyojya kiñcidvivṛtavaktro dantair dantān
  asaṃspṛśan ṛjukāyaḥ}. REF} the earliest instance being perhaps \emph{Mrgendrāgama} ?.18.
Placing the tongue where it does not disturb the meditation seems quite appropriate for a
``normal'' meditative practice.

[Jason: There’s also a clear reference to the tip of the tongue being placed in the middle of the palate in 2.27 of the Yogapāda of the Mataṅgapārameśvaratantra (tālumadhyagatenaiva jihvāgreṇa mahāmune). In fact, in works that predate Haṭhayoga (i.e., 11th c.), the most common instruction is to put the tip of the tongue on the palate (tālu).]

We might try to interpret the passage in this manner, however, once Haṭhayogic physiology is at the
background, we must assume that the aim is to reach back to the uvula, to the source of the
"nectar".  For the background and for the crucial references see Mallinson's note on
\emph{Khecarīvidyā} 1.65ab.\footnote{p. 209.}  Confusingly Yogic terminology has used and possibly
invented new names for uvula, and among these especially the term \emph{rājadanta} may give rise to
confusion, since, as we have seen, the tongue might also in some Yoga systems be placed at the
(front) teeth.

Furthermore the details in these descriptions of the \emph{khecarīmudrā} are manifold. The 10th
century \emph{Mokṣopāya} says that the tongue rests at the ``source of the
palate''\footnote{\emph{tālumūlatalālagnajihvā} MU V.55.14c.} and the commentary
\emph{Saṃsārataraṇi} on the parallel passage in \emph{Laghuyogavāsiṣṭha} V.6.155, which reads
\emph{tālumūlāntarālagnajihva-}, explains that this means that the tongue is to be placed in the
middle of the two regions of the palate, and that this is the \emph{nabhomudrā}, alias
``\emph{khecarī}''.\footnote{\emph{tālumūlāntarālagnajihvamūlah. tālumūlayoh. kākudamūladeśayoh. āntare
  lagnam ālagnam jihvāmūlam yasyety anena nabhomudrā darśitā | yā hi khecarı̄ty ucyate.}}

A little later in the \emph{Mokṣopāya} it is made clear that one should reach the uvula, ``placed a
the root of the palate''.\footnote{\emph{tālumūlagatāṃ yatnāj jihvayākramya ghaṇṭikām}
(V.78.24ab)}

In view of this background we must conclude that the author of the \emph{Jyotsnā} was probably not
aware of the Yogic meaning of \emph{rājadanta} and has tried his best to make sense of the passage,
echoing the idea of the two roots of the palate (although his text is not talking about the
palate), but then wisely refers to the instruction of the teacher for practical details, probably
noticing that his literal interpretation is somewhat opaque (I omit the synonyms for clarity):

\begin{quote}
Pressing both roots of the front teeth on the left and right with the tongue [\ldots] — this fixation of the tongue has to be understood from the mouth of the teacher.

\emph{rājadantānāṃ daṃṣṭrāṇāṃ savyadakṣiṇabhāge sthitānāṃ mūle ubhe mūlasthāne jihvayā uttambhya ūrdhvaṃ stambhayitvā | guru-mukhād avagantavyo 'yaṃ jihvā-bandhaḥ |}
\end{quote}

[Jason: also, Brahmānanda’s comments on 3.22 indicate that he thought rājadanta refers to the front teeth. When commenting on rājadantasthajihvāyāṃ, he says kutaḥ ? yato dantānāṃ rājāno rājadantā rājadanteṣu tiṣṭhatīti rājadantasthāḥ, rājadantasthā cāsau jihvā ca tasyāṃ rājadantasthajihvāyāṃ bandhaḥ, taduparibhāgasya sambandhaḥ śastaḥ].
\end{philcomm}

\subsection*{1.47}
\begin{translation}[hp01_047]
This is taught as lotus pose, which cures all diseases. It is difficult for just anyone to accomplish; it is accomplished by a wise person [here] on earth.
\end{translation}

\begin{sources}[hp01_047]
DYŚ
\startverse
idaṃ padmāsanaṃ proktaṃ sarvavyādhivināśanam |\\
durlabhaṃ yena kenāpi dhīmatā labhyate bhuvi || 38 ||
\endverse

Śivasaṃhitā
\startverse
idaṃ padmāsanaṃ proktaṃ sarvavyādhivināśanam\\
durlabhaṃ yena kenāpi dhīmatā labhyate param 3.105
\endverse
\end{sources}

\begin{testimonia}[hp01_047]
Yogacintāmaṇi
\startverse
idaṃ padmāsanaṃ proktaṃ sarvavyādhivināśanam |\\
durlabhaṃ yena kenāpi dhīmatā labhyate bhuvi ||
\endverse

Haṭharatnāvalī
\startverse
idaṃ padmāsanaṃ proktaṃ sarvavyādhivināśanam |\\
durlabhaṃ yena kenāpi dhīmatā labhyate bhuvi || 3.38 ||
\endverse
\end{testimonia}

\begin{philcomm}[hp01_047]
In this context, the word \emph{durlabham} is ambiguous as to whether the posture is hard to perform or hard to acquire. The latter is the usual reading.
% Maybe add some remarks by commentators on how they understand it.
\end{philcomm}

\subsection*{1.48}
\begin{translation}[hp01_048]
Next is taught the doctrine of Matsya.

[The yogi] should put the hands together in a bowl shape, very firmly assume padmāsana, firmly place the chin on the chest and meditation in the mind, [and] raising the Apāna breath over and over again, releasing the held Prāṇa, a man attains unequalled knowledge through the power of the goddess [Kuṇḍalinī].
\end{translation}

\begin{sources}[hp01_048]
VM 36
\startverse
kṛtvā saṃpuṭitau karau dṛḍhataraṃ baddhvātha padmāsanaṃ\\
gāḍhaṃ vakṣasi sannidhāya cibukaṃ dhyānaṃś ca tac cetasi |\\
vāraṃ vāram apānam ūrdhvam anilaṃ proccālayan pūritaṃ\\
muṇcan prāṇam upaiti bodham atulaṃ śaktiprabhāvān naraḥ ||
\endverse
\end{sources}

\begin{testimonia}[hp01_048]
Yogacintāmaṇi
\startverse
tathāca granthāntare—\\
kṛtvā saṃpuṭitau karau dṛḍhataraṃ badhvā ca padmāsanam\\
gāḍhaṃ vakṣasi saṃnidhāya civukaṃ dhyānaṃ ca tac cetasi |\\
vāraṃ vāram apānam ūrdhvam anilaṃ protsārayet pūrayet\\
prāṇaṃ muñcati bodham eti niyataṃ śaktiprabodhodayāt ||
\endverse

Haṭharatnāvalī
\startverse
kṛtvā saṃpuṭitau karau dṛḍhataraṃ baddhvā tu padmāsanam\\
gāḍhaṃ vakṣasi sannidhāya cibukaṃ dhyānaṃ ca tac cetasi |\\
vāraṃ vāram apānam ūrdhvam anilaṃ proccārayet pūritam\\
muñcat prāṇam upaiti bodham atulaṃ śakteḥ prabhāvān naraḥ || 3.39 ||
\endverse
\end{testimonia}

\begin{philcomm}[hp01_048]
The reading of \emph{matsyamataṃ} in the subheading is unusual because elsewhere Matsyendra is not referred to as simply Matsya. Furthermore, this verse is from the \emph{Vivekamārtaṇḍa} (36), and the teachings of the \emph{Vivekamārtaṇḍa} are attributed to Gorakṣa, not Matsyendra. However, in the extended recension of it known as the  \emph{Gorakṣaśataka},  \emph{Gorakṣasaṃhitā}, etc. the second verse includes homage to Mīna, i.e. Matsyendra.

The end of pāda b is uncertain. Later witnesses including Brahmānanda have \emph{dhyāyaṃś ca} but none of the early ones has this reading. So, we have to take \emph{dhyānaṃ} with \emph{sannidhāya}. However, \emph{tat} is problematic because here it has no clear referent. In the \emph{Vivekamārtaṇḍa tat} appears to refer to the \emph{mokṣadvāra} broken by Kuṇḍalinī, which is mentioned in the previous verse.

The two participles \emph{proccālayan} and \emph{muñcan} imply that the two things are happening at the same time, which is surprising but perhaps possible. Or perhaps one can understand them as sequential.

In the VM this verse comes in the passage on raising Kuṇḍalinī, so tat could refer to visualising K opening the mokṣadvāra.
\end{philcomm}

\subsection*{1.49}
\begin{translation}[hp01_049]
While in padmāsana the yogi should inhale through the opening to the central channel. He who holds the breath is sure to be liberated.
\end{translation}

\begin{sources}[hp01_049]
--
\end{sources}

\begin{testimonia}[hp01_049]
Yogacintāmaṇi
\startverse
padmāsanasthito yogī nāḍīdvāreṣu pūrayan |\\
mārutaṃ dhārayed yas tu sa mukto nātra saṃśayaḥ ||
\endverse

Haṭharatnāvalī 3.40
\startverse
padmāsane sthito yogī nāḍīdvāreṣu pūrayet |\\
pūritaṃ dhrīyate yas tu sa mukto nātra saṃśayaḥ || 3.40 ||
\endverse

Dhyānabindu Up
\startverse
padmāsanasthito yogī nāḍīdvāreṣu pūrayan |\\
mārutaṃ kumbhayan yas tu sa mukto nātra saṃśayaḥ || 70 ||
\endverse
\end{testimonia}

%\begin{philcomm}[hp01_049]
% comment on \emph{dvāreṣu} and \emph{dvāreṇa} and explain how the reading with \emph{pibati} in the third pāda might have been understood (i.e., by supplying evam).
%\end{philcomm}

\subsection*{1.50--52}
\begin{translation}[hp01_050]
Now the lion’s pose
[The yogi] should put both ankles at the sides of the perineal seam below the scrotum. Having placed the left ankle on right, the right ankle on the left and both hands on the knees and spread his fingers, with his mouth open, he stares at the tip of the nose. This is the lion’s pose, which is always worshipped by yogis. It causes the three locks to arise together and is the best of [all] postures.
\end{translation}

\begin{sources}[hp01_050]
Vasiṣṭhasaṃhitā 1.73
\startverse
gulphau ca vṛṣaṇasyādhaḥ sīvanyāḥ pārśvayoḥ kṣipet |\\
dakṣiṇaṃ savyagulphena dakṣiṇenetaretaram ||\\
hastau jānau ca saṃsthāpya svāṅgulīś ca prasārya ca |\\
vyāttavaktro nirīkṣeta nāsāgraṃ susamāhitaḥ ||\\
siṃhāsanaṃ bhaved etat pūjitaṃ yogibhiḥ sadā ||
\endverse

Yogayājñavalkya 3.9
\startverse
gulpau ca vṛṣaṇasyādhaḥ sīvanyāḥ pārśvayoḥ kṣipet\\
dakṣiṇaṃ savyagulphena dakṣiṇena tathetaram ||\\
hastau ca jānvoḥ saṃsthāpya svāṅgulīś ca prasārya ca\\
vyāttavaktro nirīkṣet nāsagraṃ susamāhitaḥ ||\\
siṃhāsanaṃ bhaved etat pūjitaṃ yogibhiḥ sadā
\endverse

Sūtasaṃhitā 15.7
\startverse
gulphau ca vṛṣaṇasyādhaḥ sīvanyāḥ pārśvayoḥ kṣipet |\\
dakṣiṇaṃ savyagulphena vāmaṃ dakṣiṇagulphataḥ || 7 ||\\
hastau ca jānvoḥ saṃsthāpya svāṅgulīś ca prasārya ca |\\
nāsāgraṃ ca nirīkṣeta bhavet siṃhāsanaṃ hi tat || 8 ||
\endverse
\end{sources}

\begin{testimonia}[hp01_050]
Yogacintāmaṇi (citing Yogayājñavalkya)
\startverse
gulphau ca vṛṣaṇasyādhaḥ sīvanyāḥ pārśvayoḥ kṣipet |\\
dakṣiṇaṃ savyagulphena dakṣiṇena tathetaram || \\
hastau jānūpari sthāpya svāṅgulīḥ saṃprasārya ca |\\
vyāttavaktro nirīkṣeta nāsāgraṃ susamāhitaḥ | \\
siṃhāsanaṃ bhaved etat pūjitaṃ yogibhiḥ sadā |
\endverse

[Note: The Yogacintāmaṇi attributes its citation of verses on siṃhāsana to the Yogayājñavalkya and does not include HP 1.52cd, which affirms that 1.52cd is not from the Yogayājñavalkya (or Vasiṣṭhasaṃhitā)].

Haṭharatnāvalī
\startverse
atha siṃhāsanam\\
gulphau ca vṛṣaṇasyādhaḥ sīvanyāḥ pārśvayoḥ kṣipet |\\
dakṣiṇe savyagulphaṃ ca dakṣiṇe tu tathetaram || 3.31 ||\\
hastau tu jānvoḥ saṃsthāpya svāṅgulīḥ samprasārya ca ||\\
vyāttavaktro nirīkṣeta nāsāgraṃ tu samāhitaḥ || 3.32 ||\\
siṃhāsanaṃ bhaved etat sevitaṃ yogibhiḥ sadā ||\\
bandhatritayasaṃsthānaṃ kurute cāsanottamam || 3.33 ||
\endverse
\end{testimonia}

\begin{philcomm}[hp01_050--052]
% comment on nyastalocanaḥ and susamāhitaḥ. The latter is well attested by the sources and testimonia, and is in the V19 and N23 branch. However, most of the old mss. have nyastalocanaḥ. Maybe comment on the meaning of sandhāna, which seems to mean simply that the three locks arise together (cf. a similar comment on siddhāsana in 1.42). 
Spreading the fingers and keeping the mouth open seem to be imitating the lion, and this is depicted in some iconography of yoganarasiṃha (for example, Yoga Narasimha, Vishnu's Man-Lion Incarnation, Samuel Eilenberg Collection, Bequest of Samuel Eilenberg, 1998, Accession Number: 2000.284.4. https://www.metmuseum.org/art/collection/search/39251).  
\end{philcomm}

\subsection*{1.53--54}
\begin{translation}[hp01_053]
Now, the friendly pose.
[The yogi] should put both ankles at the sides of the perineal seam below the scrotum.  Having firmly and motionlessly held the sides of the feet with the hands, this is bhadrāsana, which cures all diseases and poisons. Yogis of the siddha tradition call it Gorakṣāsana.
\end{translation}

\begin{philcomm}[hp01_054]
We have understood \emph{pārśvapāda} (i.e., side of the foot) like the compound \emph{agrapāda} (i.e., front of the foot), which can be classified as a \emph{ekadeśitatpuruṣa}. See Sanskrit Vademecum, p.84 (ed.\ Maximilian Mehner and Jürgen Hanneder, Marburg 2019).
\end{philcomm}

\begin{sources}[hp01_053]
Vasiṣṭhasaṃhitā 1.79
\startverse
gulphau ca vṛṣaṇasyādhaḥ sīvanyāḥ pārśvayoḥ kṣipan | \\
pārśvapādau ca pāṇibhyāṃ dṛḍhaṃ baddhvā suniścalam |\\
bhadrāsanaṃ bhaved etat sarvavyādhiviṣāpaham ||
\endverse

Yogayājñavalkya 3.11cd--3.12ab
\startverse
gulphau ca vṛṣaṇasyādhaḥ sīvanyāḥ pārśvayoḥ kṣipet\\
pārśvapādau ca pāṇibhyāṃ dṛḍhaṃ baddhvā suniścalam\\
bhadrāsanaṃ bhaved etat sarvavyādhiviṣāpaham
\endverse
\end{sources}

\begin{testimonia}[hp01_053]
Yogacintāmaṇi (citing Yājñavalkya)
\startverse
gulphau ca vṛṣaṇasyādhaḥ sīvanyāḥ pārśvayoḥ kṣipet |\\
pārśvapādau ca pāṇibhyāṃ dṛḍhaṃ badhvā suniścalaḥ |\\
bhadrāsanaṃ bhaved etat sarvavyādhiviṣāpaham |
\endverse

Haṭharatnāvalī
\startverse
atha bhadrāsanam\\
gulphau ca vṛṣaṇasyādhaḥ sīvanyāḥ pārśvayoḥ kṣipet ||\\
pārśvapādau ca pāṇibhyāṃ dṛḍhaṃ baddhvā suniścalam ||\\
bhadrāsanaṃ bhaved etat sarvavyādhiviṣāpaham || 3.30 ||
\endverse

YSS
\startverse
gulphau ca vṛṣaṇasyāyasvīvinyāḥ pārśvayoḥ kṣipet |\\
pārśvapāde ca pāṇibhyāṃ dṛḍhaṃ badhvāsu niścalaḥ ||\\
bhadrāsanaṃ bhaved etat sarvavyādhivināśanam |\\
gorakṣāsanam ity āhur idaṃ vai siddhayoginaḥ ||
\endverse
\end{testimonia}

\subsection*{1.55}
\begin{translation}[hp01_055]
When the great yogi does not become tired from adopting the āsanas in this way, he should now practise the breath techniques with mudrās and so forth from which purification of the channels arise.
\end{translation}

\begin{sources}[hp01_055]
\end{sources}

\begin{testimonia}[hp01_055]
Yogacintāmaṇi
\startverse
evam āsanabandheṣu yogīndro vijitaśramaḥ |\\
abhyasen nāḍiśuddhiṃ ca mudrayā pavanakriyām || iti ||
\endverse

HSC
\startverse
evam āsanabandhastho yogīndro vigataśramaḥ |\\
athābhyasen nāḍiśuddhiṃ mudrādipavanakriyām || 73 ||
\endverse
\end{testimonia}

\begin{philcomm}[hp01_055]
55c is ra-vipulā with \emph{nāḍi}
The second hemistich can be interpreted in different ways. One possibility is to understand \emph{nāḍiśuddhiṃ} as a bahuvrīhi qualifying \emph{mudrādipavanakriyām} in the sense that one should practise a breathing technique with mudrās from which purification of the channels arise. Alternatively, one could separate \emph{mudrādi} from \emph{pavanakriyām} and understand three different techniques here, namely, the practice of purifying the channels (perhaps by the alternative nostril method mentioned at the beginning of the second chapter), the \emph{mudrā}s and the breathing techniques of prāṇāyāma.  
\end{philcomm}

\subsection*{1.55*1--2}
\begin{translation}[hp01_055_1]
Success arises for one engaged in practice. How can it arise for one who has no practice? Success in yoga does not arise by merely reading scriptures.

Wearing a robe does not bring about success, nor does talking [about yoga]. Practice alone is the cause of success. This is true, there is no doubt. In this tradition, it should not be given to one who wears robes and is devoted to his genitals and stomach.
\end{translation}

\begin{sources}[hp01_055_1]
Dattātreyayogaśāstra 42cd--47
\startverse
kriyāyuktasya siddhiḥ syād akriyasya kathaṃ bhavet |\\
na śāstrapāṭhamātreṇa kā cit siddhiḥ prajāyate |\\
na veṣadhāraṇaṃ siddheḥ kāraṇaṃ na ca tatkathā |\\
kriyaivakāraṇaṃ siddheḥ satyam eva tu sāṃkṛte || 46 ||\\
śiśnodarārthaṃ yogasya kathayā veṣadhāriṇaḥ |\\
anuṣṭhānavihīnās tu vañcayanti janān kila || 47 ||
\endverse
\end{sources}

\begin{testimonia}[hp01_055_1]
\end{testimonia}

\begin{philcomm}[hp01_055_1]
1.55*1–2 are omitted from V1 and V19, so may not have been included in the earliest version of the text. In fact, it is possible that both were added (perhaps initially as marginal notes) to elaborate on kriyā in 1.55d. Both verses are similar to verses from the Dattātreyayogaśāstra, and these verses (except 1.55.2ef) appear in the vulgate, but towards the end of chapter 1 (1.65–66).

And then just half of DYŚ 47 is given, resulting in a near-nonsensical hemistich.

The syntax of 1.55.2ef is corrupt. One has to emend to \emph{deyā} to make sense of it.
\end{philcomm}

\subsection*{1.55*3}
\begin{translation}[hp01_055_3]
In me, the pure ocean of awakening, is this empty bubble we call the universe dissolved or does it arise? Where does this veil of doubt about this come from?
\end{translation}

\begin{testimonia}[hp01_055_3]
Vārāhītantra
\startverse
mayi bodhībudho svasthe tucho yaṃ viśvabudbudaḥ |\\
malīna udito vetti vikalpāvasaraḥ kutaḥ |
\endverse

Siddhāntamuktāvalī
\startverse
śiśnodararatāya hi na deyaṃ veṣadhāriṇe ||\\
mayi bodhyaṃ buddhau svacche tad dheyaṃ viśvabudbudam || 7 ||
\endverse

Yogaprakāśikā
\startverse
“śiśnodararatāyaitan na deya" etat yogajñānam etena śiśnodararatas tyājyo nanv etanmate tyājyapadārtho 'prasiddha iti śaṃkāṃ nirasyati mayi iti svacche bodhasvarūpasamudre budbudatulyasya viśvasya heyatvād iti bhāvaḥ
\endverse
\end{testimonia}

\begin{philcomm}[hp01_055_3]
It is very difficult to find a reason why this verse should be inserted here. It is apparently a
\emph{muktaka} that would befit an accomplished spiritual poem more than our \emph{Haṭhapradīpikā},
even here, in what appears as a sort of miscellaneous section at the end of a chapter.  In this
verse, the lyrical subject wonders about why the mind is still able to doubt, despite its insight
into the nature of reality and the reader wonders, how the illusionist verse could be understood to
fit our Yoga text, perhaps the scribe of the archetype of V19 and N17, was fond of it.

The source is, as far as we can say, the \emph{Śāntiśataka} of the Kashmirian poet Sillana or
Silhaṇa,\footnote{The mss.\ of his \emph{Svātmopalabdhiśataka} give the name as Sillana, the mostly
Bengali mss. of the Śāntiśataka read Śilhaṇa, as does Aufrecht in his CC. See Hanneder:
forthcoming.} who cannot be dated with any certainty, but predates the HP by a few centuries. The
edition of this text – where hundred original verses had to be identified – places it into an
appendix of doubtful stanzas,\footnote{Karl Schönfeld: \emph{Das Śāntiśataka}. Leipzig:
Harrassowitz 1910, p.\,90 (A9).} but the editor saw no compelling reason not to regard it as
original except through the fact that it is not transmitted in all manuscripts.  What prevents
further investigation of the matter is the lack of Kashmirian mss.\ for the \emph{Śāntiśataka} and
its compilatory character: one quarter of the material is identical with Bhartṛhari's
\emph{Vairāgyaśataka}. A still superficial glance at Sillana's \emph{Svātmopalabdhiśataka} gives
the impression that our verse would fit there, but not so much in the \emph{Śāntiśataka}. Perhaps its first citation is in Advayavajra’s \emph{Tattva\-ratnāvalī} (ref??),  While
these are only preliminary observations the verse is probably not original in any stage of
development of the HYP, but limited to V19 (and N17).
\end{philcomm}

\subsection*{1.55*4}
\begin{translation}[hp01_055_4]
Realisation from scripture, one's own guru or oneself and the cessation of mind; all these methods have been combined and taught by the wise in this tradition.
\end{translation}

\begin{sources}[hp01_055_4]
\end{sources}

\begin{testimonia}[hp01_055_4]
Yogacintāmaṇi
\startverse
haṭhapradīpikāyām--\\
śrutapratītiḥ svagurupratītiḥ, svātmapratītir manaso nirodhaḥ |\\
etāni sarvāṇi samuccitāni, matāni dhīrair iha sādhanāni ||
\endverse
\end{testimonia}

\begin{philcomm}[hp01_055_4]
last greyscale verse (\emph{śrutipratītiḥ…}) is only in V19 of old mss (but some other later ones) and is quoted in Yogacintāmaṇi.

The reading in the Yogacintāmaṇi (\emph{manaso nirodhaḥ}) is better in a yogic context.
\end{philcomm}

\subsection*{1.56}
\begin{translation}[hp01_056]
The various āsana, breath retention, bodily technique (karaṇa) called seals (mudrā), and then the fusion of the mind with the internal resonance are the sequence of practice in Haṭha.
\end{translation}

\begin{sources}[hp01_056]
\end{sources}

\begin{testimonia}[hp01_056]
Yogacintāmaṇi
\startverse
haṭhapradīpikāyām--\\
āsanaṃ kumbhakaṃ citraṃ mudrākhyaṃ karaṇaṃ tathā |\\
atha nādānusandhānam abhyāsānukrameṇa ca ||
\endverse
\end{testimonia}

\begin{philcomm}[hp01_056]
Verse is omitted from V1, probably deliberately as it doesn’t have chapter 4, which teaches nādānusandhāna. The numbering in V1 suggests that its exemplar had this verse.

Kumbhaka is almost always masculine.

Marmasthāna: not clear whether to adopt \emph{citro} or \emph{citraṃ}, or V19’s \emph{mudrākhyaṃ karaṇaṃ tathā} or the others’ \emph{mudrādikaraṇāni ca}.

See also 1.65, which has \emph{mudrādikaraṇāni ca}, so perhaps it was through confusion with this that the same reading is found in some witnesses of 1.56.

Yes, V19 reading probably best.
\startverse
āsanaṃ kumbhakaś citraṃ mudrākhyaṃ karaṇaṃ tathā |\\
atha nādānusandhānam abhyāsānukramo haṭhe ||
\endverse

It seems that the four aṅgas of Haṭhayoga are being referred to in the singular (hence \emph{āsanaṃ}), whereas in 1.65 the plural is used (i.e., \emph{pīṭhāni}). Therefore, we should adopt \emph{citraṃ [...] karaṇaṃ tathā}.
\end{philcomm}

\subsection*{1.57}
\begin{translation}[hp01_057]
Celibate, restricted in diet and devoted to yoga, the renunciant succeeds in upwards of a year. No doubt about this should be entertained.
\end{translation}

\begin{sources}[hp01_057]
Vivekamārtaṇḍa 37
\startverse
brahmacārī mitāhārī yogī yogaparāyaṇaḥ |\\
abdād ūrdhvaṃ bhavet siddho nātra kāryā vicāraṇā ||
\endverse
\end{sources}

\begin{testimonia}[hp01_057]
Yogacintāmaṇi
\startverse
brahmacārī mitāhārī tyāgī yogaparāyaṇaḥ |\\
abdād ūrdhvaṃ bhavet siddho nātra kāryā vicāraṇā ||
\endverse

Haṭharatnāvalī
\startverse
brahmacārī mitāhārī tyāgī yogaparāyaṇaḥ ||\\
abdād ūrdhvaṃ bhavet siddho nātra kāryā vicāraṇā || 3.28 ||
\endverse
\end{testimonia}

\begin{philcomm}[hp01_057]
Comment on tyāgī and yogī, and the reading of this verse in the Vivekamārtaṇḍa. It seems likely that Svātmārāma was using a version of the VM with tyāgī (as it’s unlikely that yogī would be emended to tyāgī), but whether this was the original reading in the VM is more complicated because there tyāgī may be a dittographical-type mistake. 
\end{philcomm}

\subsection*{1.58}
\begin{translation}[hp01_058]
When very unctuous and sweet food that is without a quarter portion is eaten for love of śiva, it is called a restricted diet. (\emph{mitāhāra}).
\end{translation}

\begin{sources}[hp01_058]
Gorakṣaśataka (original)
\startverse
susnigdhamadhurāhāraś caturthāṃśavivarjitaḥ |\\
bhujyate śivasaṃprītyai mitāhāraḥ sa ucyate ||
\endverse
\end{sources}

\begin{testimonia}[hp01_058]
Yuktabhavadeva
\startverse
tad uktaṃ haṭhapradīpikāyām-\\
susnigdhamadhurāhārāś caturthāṃśavivarjitaḥ ||\\
bhujyate śivasamprītyai mitāhāraḥ sa ucyate || 16 ||
\endverse

Yogacūḍāmaṇyupaniṣat
\startverse
susnigdhamadhurāhāraścaturthāṃśavivarjitaḥ |\\
bhuñjate śivasaṃprītyā mitāhārī sa ucyate || 43 ||
\endverse
\end{testimonia}

\begin{philcomm}[hp01_058]
This verse probably derives from the ‘original’ GŚ (12c–13b). It is also found, but reworked to be about the \emph{mitāhārī}, in Nowotny’s GŚ (extended recension of VM) at 55. The idea of not eating the fourth portion of one’s food (\emph{caturthāṃśavivarjitaḥ}) can be found in older sources, such as Dharmaputrikā 1.51-52 (ṣaḍrasopetasuṣnigdhasvādusāndrasugandhinā |
udarasyārdhabhāgan tu bhojanena prapūrayet || pānīyena caturbhāgaṃ taccheṣaṃ śūnyam iṣyate | vāyos sañcāraṇānārtham āhāraniyamaḥ smṛtaḥ) and Aṣṭāṅgahṛdayasaṃhitā Sūtrasthāna 8.46c-47b ( annena kukṣer dvāv aṃśau pānenaikaṃ prapūrayet ||āśrayaṃ pavanādīnāṃ caturtham avaśeṣayet). As noted in the Jy

\end{philcomm}

\subsection*{1.59}
\begin{translation}[hp01_059]
[Adepts] say the [following] is unwholesome: pungent, sour, bitter, salty and hot foods, horseradish, sour gruel, [sesame] oil, sesame and mustard seeds, fish and intoxicating drink. Flesh of goats and so forth, curds, diluted buttermilk, poor man's pulse, Jujube fruit, the leftover paste of oily seeds, asafoetida, garlic and the like.
\end{translation}

\begin{sources}[hp01_059]
\end{sources}

\begin{testimonia}[hp01_059]
Yogacintāmaṇi
\startverse
haṭhapradīpikāyām--\\
kaṭvamlatīkṣṇalavaṇoṣṇaharītaśāka-\\
sauvīratailatilasarṣapamatsyamadyam |\\
ajādimāṃsadadhitakrakulatthakola-\\
piṇyākahiṅgulaśunādyam apathyam āhuḥ ||
\endverse

Haṭharatnāvalī
\startverse
kaṭvamlatīkṣṇalavaṇoṣṇaharītaśākaṃ \\
sauvīratailatilasarṣapamatsyamadyam |\\
ajādimāṃsadadhitakrakulatthakodra-\\
piṇyākahiṅgulaśunādyam apathyam āhuḥ || 1.72 ||
\endverse

HTK
\startverse
atha varjyāni –\\
kaṭvamlatīkṣṇalavaṇoṣṇa haritaśāka-\\
sauvīratailatilasarṣapamatsyamadyam ||\\
ajāvimāṃsadadhitakrakulatthakola-\\
piṇyākahiṃgulaśunādyam apathyam āhuḥ || 28 ||
\endverse
\end{testimonia}

\begin{philcomm}[hp01_059]
59a \emph{kaṭvamla°} is better than \emph{kaṭvāmla°} and well attested elsewhere in lists of tastes and types of foods.
On the meaning of \emph{uṣṇa} (in relation to food) see Meulenbeld 1974: 254 fn. 13: ‘Cakra mentions as a variant: \emph{katvamlalavaṇakṣāra} (pungent. acid, saline and caustic). Cakra remarks that the term `hot' (\emph{uṣṇa}) denotes hot on touch when it occurs the first time, and hot with regard to potency when it occurs for the second time.’

\emph{°hari}(\emph{ī} here for metre?)\emph{taśāka°} in some nighaṇṭus is horseradish, which makes better sense here than Brahmānanda’s gloss.
\startverse
śigrur haritaśākaś ca śākapattraḥ supattrakaḥ | \textup{Rājanighaṇṭu 7.26}\\
śigruko haritaśākaś ca mato vai mūlapatrakaḥ | \textup{Sauśrutanighaṇṭu 75ab}\\
\endverse

Brahmānanda’s understanding of \emph{harītaśāka} as \emph{patraśāka} is probably wrong if \emph{patraśāka} was intended as ‘leafy vegetables’. But perhaps \emph{patraśāka} can also mean \emph{śigru} (note \emph{śākapattraḥ} above).

Anusvāra at end of \emph{śāka}?

\emph{°sauvīra°} probably means sour gruel.
Brahmānanda: \emph{sauvīra} = \emph{kāñjika} (fermented rice water).
Meulenbeld, madhavanidāna pp. 516–517
\emph{sauvīra} is sour gruel made from barley and wheat. On the process see, Suśruta 1.44.35--40ab.
(PV Sharma’s translation of this passage:)

\begin{quote}
‘Roots of trivṛt etc., the first group (\emph{vidārigandhādi}), \emph{mahat pañcamūla}, \emph{mūrvā} and \emph{śārṅgaṣṭā}, and also of \emph{snuhī}, \emph{haimavatī}, \emph{triphalā}, \emph{ativiṣā} and \emph{vacā} -- these are taken and divided into two parts out of which one is decocted and the other is powdered; now, crushed barley grains are impregnated with the above decoction several times, dried and then slightly fried. Taking three parts of this and one part of the above powder are put in a jar and mixed with their (of \emph{trivṛt}, etc.) cold decoction and fermented properly. This is known as \emph{sauvīraka}.’
\end{quote}

But according to some nighaṇṭus, \emph{sauvīra} can also mean stibnite (an ingredient in some añjana’s and medicines):
\startverse
añjanaṃ yāmunaṃ kṛṣṇaṃ nādeyaṃ mecakaṃ tathā \\
srotojaṃ dṛkpradaṃ nīlaṃ sauvīraṃ ca suvīrajam ||  \textup{Rājanighaṇṭu 13.86}
\endverse

Note also that the Yogaprakāśikā takes \emph{sauvīra} with \emph{taila}, perhaps to solve the problem of \emph{taila} on its own (see below for more on this): \emph{sauvīrataila} -- oil produced in the place Suvīra (\emph{suvīradeśodbhavatailam}).

Suvīra , a country the people of which, also called Suvira (V.79), Sauvira (XVI.21) and Sauvīraka (IV.23) ... S.M. Ali identifies it with the Rohri - Khairpur region of Sind (Geography of the Purānas, Delhi, 1966, p. 144).

\emph{taila} could refer to \emph{tilataila}:
Śārṅgadharasaṃhitā:
\emph{anuktāvasthāyāṃ paribhāṣāvidhiḥ [...] taile ’nukte tilodbhavam} 48

Dominik Wujastyk supplied this reference and may be able to comment more on taila in this list.

On the translation of \emph{madya}, see Mchugh (An Unholy Brew) 2021: 8.

\emph{ājādimāṃsa}: āja° is required for the metre, so only ājādi makes sense, not ājāvi because there is no adjective āvi.

Diwakar Acharya: prohibition of goat flesh and fish is aimed at north/east India.

\emph{kulattha} is a kind of pulse, translated by Dominik Wujastyk as `poor man's pulse' (see Roots of Ayurveda).

\emph{kola}: Zizyphus Jujuba (Nadkarni 1926: pp. 919--920). Also known as \emph{badara}. This is how Brahmānanda takes it (\emph{kolaṃ kolyāḥ phalaṃ badaram}). According to Nadkarni, the fruit of the wild variety is very acid and astringent. It is believed to purify the blood and assist digestion. The bark is astringent and a simple remedy for diarrhoea. Root is useful as a decoction in fever and delirium.
kola can mean banana in some parts of India (Diwakar)

There are references to \emph{kola} being pungent, though this does not seem to indicate sufficiently why \emph{kola} is mentioned separately as an \emph{apathya} food.

\emph{piṇyāka}: Sharma (Ḍalhaṇa and his commentary on drugs: 1982: 69) says, ‘The remnant paste of oily seeds after pressing out the oil content is called \emph{piṇyāka}.’ Diwakar says it is an oil cake that has a strong flavour.

\emph{hiṅgu}: Asafoetida (Nadkarni 1926: pp 360–361): `If long continued even in moderate doses, it gives rise to alliaceous eructations, acrid irritation in the throat, flatulence, diarrhoea and burning in the urine.'

\emph{laśuna} = garlic (Nadkarni 1926: 45).
\end{philcomm}

\subsection*{1.60}
\begin{translation}[hp01_060]
One should know food that has been reheated, is dry, is too salty, the leftover paste of crushed sesame seeds (tilapiṇḍa), spoiled [or] contains an excess of leafy vegetables to be unwholesome. It is to be avoided.
\end{translation}

\begin{sources}[hp01_060]
\end{sources}

\begin{testimonia}[hp01_060]
Yogacintāmaṇi
\startverse
bhojanam ahitaṃ vidyāt punar <apy> uṣṇīkṛtaṃ tathā rukṣam |\\
atilavaṇaṃ sapalaṃ vā prasitaṃ śākotkaṭaṃ varjyam ||
\endverse

HSC
\startverse
bhojanam ahitaṃ vidyāt punar uṣṇīkṛtaṃ rūkṣaṃ | \\
atilavaṇādikayuktaṃ kadaśanaśākotkaṭaṃ duṣṭaṃ ||
\endverse
\end{testimonia}

\begin{philcomm}[hp01_060]
We have understood the compound \emph{kadaśanaśākotkaṭa} as a dvandva referring to spoiled food (\emph{kadaśana}) and an excess of leafy vegetables (\emph{śākotkaṭa}).
We have not found any conclusive evidence for the meaning of \emph{tilapiṇḍa}. Brahmānanda glosses it as \emph{piṇyāka} (on which see the notes for the previous verse).

Not sure how to take \emph{kadaśanaśākotkaṭaṃ}. Brahmānanda understands it as a dvandva (i.e., \emph{kadaśana}, \emph{śāka}, \emph{utkaṭa}), where \emph{śāka} is prohibited vegetables and \emph{utkaṭa} is pepper.

The meaning of \emph{utkaṭa} is not clear. The word \emph{utkaṭā} can mean pepper according to some nighaṇṭus (e.g., Rājanighaṇṭu 5.16 \emph{pārvatī śailajā tāmrā lambabījā tathotkaṭā}). But \emph{utkaṭa} can refer to Saccharum Sara and \emph{utkaṭā} also to Laurus Cassia (cinnamon).

Also, \emph{utkaṭa} can be an adjective that means ‘abounding in’ at the end of a compound. So could \emph{kadaśanaśākotkaṭaṃ} have been intended as an adjectival tatpuruṣa along the lines of ‘[food] full of spoiled vegetables’?
\end{philcomm}

\subsection*{1.61}
\begin{translation}[hp01_061]
Similarly a saying by Goraksa:
One should avoid a liking for bad people, frequenting fire, women and roads, and observances which harm the body such as early morning bathing and fasting.
\end{translation}

\begin{sources}[hp01_061]
Amṛtasiddhi 19.7
\startverse
agnisevābalāsevā pathasevā ca sarvadā |\\
prathamābhyāsakāle tu saṃtyājyā yoginā sadā || 19.7 ||
\endverse
\end{sources}

\begin{testimonia}[hp01_061]
Yogacintāmaṇi
\startverse
haṭhadīpikāyām—\\
varjayed durjanaprītiṃ vahnistrīpathasevanam |\\
prātaḥsnānopavāsādi kāyakleśādikaṃ tathā ||
\endverse

Haṭharatnāvalī
\startverse
tathā ca gorakṣavacanam---\\
varjayed durjanaprītivahnistrīpathasevanam |\\
prātaḥsnānopavāsādi kāyakleśādikaṃ tathā || 1.73 ||
\endverse
\end{testimonia}

\begin{philcomm}[hp01_061]
The vulgate has a parallel from the Amaraugha added. 
\end{philcomm}

\subsection*{1.62}
\begin{translation}[hp01_062]
Pure food with wheat, rice, śāli rice, barley, sixty-day śāli rice, milk, ghee, candied sugar, unclarified fermented butter, ground sugar and honey. Dried ginger, fruit of the snake gourd and so forth, the five vegetables, mung beans and so on, and rain water are wholesome for the best of sages.
\end{translation}

\begin{sources}[hp01_062]
\end{sources}

\begin{testimonia}[hp01_062]
Yogacintāmaṇi
\startverse
godhūmaśāliyavaṣāṣṭikaśobhanānnaṃ\\
kṣīrājyamaṇḍanavanītasitāmadhūni |\\
śuṇṭhīpaṭolakaphalādikapañcaśākaṃ\\
mudgādi cālpam udakaṃ ca munīndrapathyam ||
\endverse

Haṭharatnāvalī
\startverse
godhūmaśāliyavaṣaṣṭikaśobhanānnaṃ \\
kṣīrājyamaṇḍanavanītasitāmadhūni |\\
śuṇṭhīpaṭolaphalapatrajapañcaśākaṃ \\
mudgādidivyam udakaṃ ca yamīndrapathyam || 1.71 ||
\endverse

YBD
\startverse
tathā ca śivayoge-\\
godhūmaśāliyavaṣāṣṭikaśobhanānnaṃ \\
kṣīrājyakhaṇḍanavanītasitāmadhūni ||\\
śuṇṭhīpaṭolakaphalādi ca pañcaśāka-\\
mudgādidivyam udakaṃ ca munīndrapathyam || 21 ||
\endverse
\end{testimonia}

\begin{philcomm}[hp01_062]
\emph{khaṇḍa} -- candied sugar (Meul 507, different types of sugar).

\emph{navanīta} (MW fresh butter), Mchugh (2021) unclarified fermented butter.

\emph{sitā} -- ground sugar (Meul 507, different types of sugar) ``sitā is very white and looks like gravel"

\emph{madhu} -- honey.

\emph{paṭola} can refer to at least two different gourds. See Meul. p. 569 for a long list of possibilities, including TRICHOSANTHES DIOICA ROXB. (`pointed gourd'), T. CUCUMERINA LINN (snake gourd).

Nadkarni has two entries on \emph{paṭola}:
\begin{enumerate}
\item snake gourd (Nadkarni p. 863) is common in Bengal and cultivated in Northern India and Punjab. The unripe fruit of this climbing plant is generally used as a culinary vegetable and is very wholesome, specially suited for the convalescent.

\item smooth luffa (Nadkarni p. 518) is a hairy climbing herb extensively cultivated in several parts of India. The fruit is edible. Medicinally it is described as `cool, costive, demulcent, producive of loss of appetite and excitive of wind, bile and phlegm")
\end{enumerate}

Wikipedia : smooth luffa = Luffa aegyptiaca (sponge gourd)\\
Sharma (Syn. Kulaka. Well known (Trichosanthas dioica Roxb.)

Brahmānanda glosses it as \emph{kośātakī} (Meul p. 586 LUFFA ACUTANGULA ROXB), which suggests he thought it was some sort of luffa.

Brahmānanda also mentions the vernacular term \emph{paravara} for \emph{paṭola}, which the Lonavla ed. states is a kind of cucumber. However, Paras remarked that \emph{paravara} is more like a gourd (hard shell, etc.).

On \emph{pañcaśāka}, see GhS
\startverse
bālaśākaṃ kālaśākaṃ tathā paṭolapatrakam |\\
pañcaśākaṃ praśaṃsīyād vāstūkaṃ hilamocikāṃ || 5.20 ||
\endverse

HTK 4.26
\startverse
pañcaśākastu –\\
kṣīraparṇī ca jīvantī matsyākṣī ca punarnavā \\
meghanādaś ceti budhaiḥ pañcaśākaḥ prakīrtitaḥ || iti || 26
\endverse

Jyotsnā and Yuktabhavadeva 4.22
\startverse
sarvaśākam acākṣuṣyaṃ cākṣuṣyaṃ śākapañcakam |\\
jīvantī-vāstu-matsyākṣī-meghanāda- punarnavāḥ || iti ||
\endverse

It is not entirely clear how we should understand \emph{divya}. Brahmānanda glosses it with \emph{nirdoṣa} and takes it with \emph{udaka}. But could it refer more specifically to \emph{gaṅgāmbu} (as suggested by Paras) or rain water? MW has \emph{divyodaka} n. `divine water' i.e. rainwater L.

The term \emph{divyodaka} appears in Āyurvedic works (but we’re yet to find a gloss in a commentary). E.g.,
Aṣṭāṅgahṛdaya 8.42–43
\startverse
śīlayec chāligodhūmayavaṣaṣṭikajāṅgalam |\\
suniṣaṇṇakajīvantībālamūlakavāstukam || 42 ||\\
pathyāmalakamṛdvīkāpaṭolīmudgaśarkarāḥ |\\
ghṛtadivyodakakṣīrakṣaudradāḍimasaindhavam || 43 ||
\endverse

SriKanta Murty translates \emph{divyodaka} as ‘divyodaka (rain water or pure water)'.

The Rājanighaṇṭu says rainwater:
\startverse
divyodakaṃ kharāri syād ākāśasalilaṃ tathā |\\
vyomodakaṃ cāntarikṣajalaṃ ceṣvabhidhāhvayam || Rajni 14.4
\endverse

Kharāri? Maybe \emph{khavāri} was intended.

Vācaspatyam:
\textbf{divyodaka} na° karma°.
1 antarīkṣabhave jale divyaśabde bhāva° pra° vākye tadbhedādi dṛśyam.
ambuśabde vivṛtiḥ .

62*1 is quoted in the Jyotsna as from a medical work (“vaidyake”).
\end{philcomm}

\subsection*{1.63}
\begin{translation}[hp01_063]
The yogi should eat food that is sweet, delicious, unctuous, contains cow products, nourishes the bodily constituents, is desired by the mind and is appropriate.
\end{translation}

\begin{sources}[hp01_063]
\end{sources}

\begin{testimonia}[hp01_063]
Yogacintāmaṇi
\startverse
piṣṭaṃ sumadhuraṃ snigdhaṃ gavyaṃ dhātuprapoṣaṇam |\\
manobhilaṣitaṃ yogyaṃ yogī bhojanam ācared iti ||
\endverse

YBhD
\startverse
śreṣṭhaṃ sumadhuraṃ snigdhaṃ gavyaṃ dhātuprapoṣaṇam ||\\
mano'bhilaṣitaṃ yogyaṃ yogī bhojanamācaret || 23 ||
\endverse
\end{testimonia}

\begin{philcomm}[hp01_063]
The variants of 1.63a all seem possible: \emph{mṛṣṭaṃ}, \emph{miṣṭaṃ} and \emph{iṣṭaṃ}. Maybe the last is made redundant by \emph{mano 'bhilaṣitaṃ}.
\end{philcomm}

\subsection*{1.64}
\begin{translation}[hp01_064]
Whether young, old, very old, sick or even weak, the diligent yogi succeeds in all yogas through practice.
\end{translation}

\begin{sources}[hp01_064]
DYS 40
\startverse
yuvāvastho 'pi vṛddho vā vyādhito vā śanaiḥ śanaiḥ |\\
abhyāsāt siddhim āpnoti yoge sarvo 'py atandritaḥ || 40 ||
\endverse
\end{sources}

\begin{testimonia}[hp01_064]
Yogacintāmaṇi
\startverse
haṭhapradīpikāyām—\\
yuvā bālo 'tivṛddho vā vyādhito durbalo 'pi vā |\\
abhyāsāt siddhim āpnoti sarvayogeṣv atandritaḥ ||
\endverse

Haṭharatnāvalī
\startverse
yuvā bhavati vṛddho 'pi vyādhito durbalo 'pi vā |\\
abhyāsāt siddhim āpnoti sarvayogeṣv atandritaḥ || 1.23 ||
\endverse
\end{testimonia}

\begin{philcomm}[hp01_064]
Note the different reading in 164d for V1: \emph{sarvaṃ yogī yatendriyaḥ}. \emph{sarvaṃ} is not easy to construe, and the testimony of the DYŚ suggests that \emph{yoge sarvo ’py atandritaḥ} was original.
\end{philcomm}

\subsection*{1.65}
\begin{translation}[hp01_065]
The postures, various breath retentions and techniques, beginning with seals, are all [to be done] in the practice of Haṭha until the reward that is Rājayoga [is attained].
\end{translation}

\begin{sources}[hp01_065]
\end{sources}

\begin{testimonia}[hp01_065]
Haṭharatnāvalī
\startverse
pīthāni kumbhakāś citrā divyāni karaṇāni ca |\\
sāṅgo 'pi ca haṭhābhyāso rājayogaphalārthadaḥ || 1.17 ||
\endverse
\end{testimonia}

\begin{philcomm}[hp01_065]
\emph{sarvaṇy api} is better than \emph{sarvo pi} ca because it refers to all the practices mentioned in the first hemistich.

Reading of Brahmānanda is different for the third pāda: \emph{divyāni karaṇāni}.
\end{philcomm}

\end{ekdosis}
\end{document}
