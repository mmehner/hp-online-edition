\documentclass[10pt]{memoir}
\setstocksize{220mm}{155mm} 	        
\settrimmedsize{220mm}{155mm}{*}	
\settypeblocksize{170mm}{116mm}{*}	
\setlrmargins{18mm}{*}{*}
\setulmargins{*}{*}{1.2}
% \setlength{\headheight}{5pt}
\checkandfixthelayout[lines]
\linespread{1}
\setlength{\parskip}{0.3em}
\setlength\parindent{0pt}

\makepagestyle{HPed}
\makeoddhead{HPed}{\small{HP Transl. \& Comm.}}{}{\small{\today}}
\makeevenhead{HPed}{\small{HP Transl. \& Comm.}}{}{\small{\today}}
\makeoddfoot{HPed}{}{\small{\thepage}}{}
\makeevenfoot{HPed}{}{\small{\thepage}}{}

\usepackage[teiexport=tidy,poetry=verse]{ekdosis}
\usepackage{sanskrit-poetry,libertine,xcolor}
\usepackage[english]{babel}
\setlength{\vindent}{0pt}
\setvnum{}




%%%%%%%%%%%%%%%%%%%% THE  MSS         %%%%%%%%%%%%%%%%%%%%%%%%%%%

%%% Versions
\DeclareWitness{Vu}{\selectlanguage{english}Vulg}{Vulgate, i.e. Brahmānanda's version}[]           
\DeclareWitness{X}{\selectlanguage{english}X}{TenChapter Version, Jodhpur 02228 and 02225 (ed. Lonavla)}[]
\DeclareWitness{Six}{\selectlanguage{english}Ṣ}{SixChapterVersion, ``6ChapterHPms'', fragment of enlarged text, Jodhpur}[]
% Mss. in Geographical Groups
%%%% Varanasi mss (Sampūrṇānanda mss). V1 is Important
\DeclareWitness{V1}{\selectlanguage{english}V\textsubscript{1}}{Sampurnananda Library Sarasvati Bhavan 30109}[]
        \DeclareHand{V1ac}{V1}{\selectlanguage{english}V\rlap{\textsubscript{1}}\textsuperscript{ac}}[] % added by MD
        \DeclareHand{V1pc}{V1}{\selectlanguage{english}V\rlap{\textsubscript{1}}\textsuperscript{pc}}[] % added by MD
\DeclareWitness{V2}{\selectlanguage{english}V\textsubscript{2}}{Sampurnananda Library Sarasvati Bhavan 29869}[]
\DeclareWitness{V3}{\selectlanguage{english}V\textsubscript{3}}{Sampurnananda Library Sarasvati Bhavan 29899}[]
\DeclareWitness{V4}{\selectlanguage{english}V\textsubscript{4}}{Sampurnananda Library Sarasvati Bhavan 29937}[]
\DeclareWitness{V5}{\selectlanguage{english}V\textsubscript{5}}{Sampurnananda Library Sarasvati Bhavan 29938}[]
\DeclareWitness{V6}{\selectlanguage{english}V\textsubscript{6}}{Sampurnananda Library Sarasvati Bhavan 29991}[]
\DeclareWitness{V8}{\selectlanguage{english}V\textsubscript{8}}{Sampurnananda Library Sarasvati Bhavan 30014}[]
\DeclareWitness{V11}{\selectlanguage{english}V\textsubscript{11}}{Sampurnananda Library Sarasvati Bhavan 30029}[]
\DeclareWitness{V12}{\selectlanguage{english}V\textsubscript{12}}{Sampurnananda Library Sarasvati Bhavan 30030}[]
\DeclareWitness{V13}{\selectlanguage{english}V\textsubscript{13}}{Sampurnananda Library Sarasvati Bhavan 30031}[]
\DeclareWitness{V14}{\selectlanguage{english}V\textsubscript{14}}{Sampurnananda Library Sarasvati Bhavan 30050}[]
\DeclareWitness{V15}{\selectlanguage{english}V\textsubscript{15}}{Sampurnananda Library Sarasvati Bhavan 30051}[]
\DeclareWitness{V15pc}{\selectlanguage{english}V\rlap{\textsubscript{15}}\textsuperscript{pc}\space}{}[]
\DeclareWitness{V16}{\selectlanguage{english}V\textsubscript{16}}{Sampurnananda Library Sarasvati Bhavan 30052}[]
\DeclareWitness{V17}{\selectlanguage{english}V\textsubscript{17}}{Sampurnananda Library Sarasvati Bhavan 30053}[] % added by MD
\DeclareWitness{V16pc}{\selectlanguage{english}V\rlap{\textsubscript{16}}\textsuperscript{pc}\space}{}[]
\DeclareWitness{V18}{\selectlanguage{english}V\textsubscript{18}}{Sampurnananda Library Sarasvati Bhavan 30064}[]
\DeclareWitness{V19}{\selectlanguage{english}V\textsubscript{19}}{Sampurnananda Library Sarasvati Bhavan 30069}[]
\DeclareWitness{V21}{\selectlanguage{english}V\textsubscript{21}}{Sampurnananda Library Sarasvati Bhavan 30104}[]
\DeclareWitness{V22}{\selectlanguage{english}V\textsubscript{22}}{Sampurnananda Library Sarasvati Bhavan 30110}[]
\DeclareWitness{V25}{\selectlanguage{english}V\textsubscript{25}}{Sampurnananda Library Sarasvati Bhavan 30122}[]
\DeclareWitness{V26}{\selectlanguage{english}V\textsubscript{26}}{Sampurnananda Library Sarasvati Bhavan 30123}[]
\DeclareWitness{V28}{\selectlanguage{english}V\textsubscript{28}}{Sampurnananda Library Sarasvati Bhavan 30136}[]
\DeclareWitness{W4}{\selectlanguage{english}W\textsubscript{4}}{Wai 399-6171}[]

%%%%%%%%%%%%%%%%%%%%%%%%%%%%%%%%%
%%% Jammu & Kaschmir
\DeclareWitness{K1}{\selectlanguage{english}K\textsubscript{1}}{Raghunātha Temple Library 4383}[settlement=Jammu]
        \DeclareWitness{K1ac}{\selectlanguage{english}K\rlap{\textsubscript{1}}\textsuperscript{ac}\space}{}[]
        \DeclareWitness{K1pc}{\selectlanguage{english}K\rlap{\textsubscript{1}}\textsuperscript{pc}\space}{}[]
\DeclareWitness{L1}{\selectlanguage{english}L\textsubscript{1}}{SOAS RE 43454}[settlement=Jammu]
% More details? Catalogue number? L1 And C1 very close (and come from same region)
%%%%%%%%%%%%%%%%%%%%%%%%%%%%%%%%
% Jodhpur
% J10 is important
\DeclareWitness{J10}{\selectlanguage{english}J\textsubscript{10}}{MSPP Jodhpur 2230}[]
        \DeclareHand{J10ac}{J10}{\selectlanguage{english}J\rlap{\textsubscript{10}}\textsuperscript{ac}}[] % modified by MD
        \DeclareHand{J10pc}{J10}{\selectlanguage{english}J\rlap{\textsubscript{10}}\textsuperscript{pc}}[] % modified by MD
\DeclareWitness{J1}{\selectlanguage{english}J\textsubscript{1}}{Jodhpur 02231}[]
\DeclareWitness{J2}{\selectlanguage{english}J\textsubscript{2}}{Jodhpur 02232}[]   
\DeclareWitness{J3}{\selectlanguage{english}J\textsubscript{3}}{Jodhpur 02233}[]
\DeclareWitness{J4}{\selectlanguage{english}J\textsubscript{4}}{Jodhpur 02234}[]
        \DeclareWitness{J4ac}{\selectlanguage{english}J\rlap{\textsubscript{4}}\textsuperscript{ac}\space}{MSPP Jodhpur 02234}[]
        \DeclareWitness{J4pc}{\selectlanguage{english}J\rlap{\textsubscript{4}}\textsuperscript{pc}\space}{MSPP Jodhpur 02234}[]
\DeclareWitness{J5}{\selectlanguage{english}J\textsubscript{5}}{Jodhpur 02235}[]  % 4 chapters, 34 jpgs,   long colophon, missing lines in the beginning.
\DeclareWitness{J6ac}{\selectlanguage{english}J\rlap{\textsubscript{6}}\textsubscript{ac}}{Jodhpur 02237}[]  % 4 chapters, 49 jpgs,   1st folio: idaṃ gulābarāyasya
% tulasīrāmaśarmmaṇaḥ putrasya pustakaṃ ...        End: iti śrīsahajānandasantānacintāmaṇisvātmārāmaviracitāyāṃ ..
% saṃvat 1802   (more consistent text)
\DeclareWitness{J6pc}{\selectlanguage{english}J\rlap{\textsubscript{6}}\textsubscript{pc}}{Jodhpur 02237}[] 
\DeclareWitness{J7}{\selectlanguage{english}J\textsubscript{7}}{Jodhpur 02241}[]  % 4 chapters, 41 jpgs
\DeclareWitness{J8}{\selectlanguage{english}J\textsubscript{8}}{Jodhpur 23709}[]  % 4 chapters,  87 jpgs.   saṃvat 1724
\DeclareHand{J8ac}{J8}{\selectlanguage{english}J\rlap{\textsubscript{8}}\textsuperscript{ac}}[]  % changed by MD
\DeclareHand{J8pc}{J8}{\selectlanguage{english}J\rlap{\textsubscript{8}}\textsuperscript{pc}}[]  % changed by MD
\DeclareWitness{J9}{\selectlanguage{english}J\textsubscript{9}}{Jodhpur 02224}[]  %  fragment, 20 jpgs.
\DeclareWitness{J11}{\selectlanguage{english}J\textsubscript{11}}{Jodhpur 23532}[]
\DeclareWitness{J12}{\selectlanguage{english}J\textsubscript{12}}{Jodhpur 18552}[] 
\DeclareWitness{J13}{\selectlanguage{english}J\textsubscript{13}}{Jodhpur 02229}[]  %  5 chapters, 93 jpgs.
\DeclareWitness{J14}{\selectlanguage{english}J\textsubscript{14}}{Jodhpur 02239}[]  %  4 chapters
\DeclareWitness{J15}{\selectlanguage{english}J\textsubscript{15}}{Jodhpur 9732A}[]
\DeclareWitness{J17}{\selectlanguage{english}J\textsubscript{17}}{Jodhpur 3013}[]
% Haṭhapradīpikā with (non-Sanskrit) Bhāṣya RORI Jodhpur ACC.NO.18552
%  Haṭhapradīpikā with (non-Sanskrit) commentary, RORI Alwar 952, 4 chapters,  colophon of the comm:
% iti śrīlāhorīmiśravrajabhūṣanaviracitāyāṃ bhāvārthadīpikāyāṃ caturthodhyāya ..    
%  Haṭhapradīpikā (5 chapter) MSPP Jodhpur ACC.NO.02229/

%%%%%%%%%%        Bodleian, Oxford
\DeclareWitness{B1}{\selectlanguage{english}B\textsubscript{1}}{Bodleian Library No. d.457(8)}[settlement=Oxford]
\DeclareWitness{B2}{\selectlanguage{english}B\textsubscript{2}}{Bodleian Library No. d.458(1)}[settlement=Oxford]
\DeclareWitness{B3}{\selectlanguage{english}B\textsubscript{3}}{Bodleian Library No. d.458(9)}[settlement=Oxford]

%%%%%%%%%%%   Chandigarh
\DeclareWitness{C1}{\selectlanguage{english}C\textsubscript{1}}{Lalchand M-2080}[]%L1 And C1 very close (and come from same region)
\DeclareWitness{C2}{\selectlanguage{english}C\textsubscript{2}}{Lalchand M-6065}[]
\DeclareWitness{C3}{\selectlanguage{english}C\textsubscript{3}}{Lalchand M-1293}[]
\DeclareWitness{C4}{\selectlanguage{english}C\textsubscript{4}}{Lalchand M-2081}[]
\DeclareWitness{C4ac}{\selectlanguage{english}C\rlap{\textsubscript{4}}\textsuperscript{ac}\space}{}[]
\DeclareWitness{C4pc}{\selectlanguage{english}C\rlap{\textsubscript{4}}\textsuperscript{pc}\space}{}[]
\DeclareWitness{C5}{\selectlanguage{english}C\textsubscript{5}}{Lalchand M-2082}[]%doesn't have chapter 1
\DeclareWitness{C6}{\selectlanguage{english}C\textsubscript{6}}{Lalchand M-2089}[]
\DeclareWitness{C7}{\selectlanguage{english}C\textsubscript{7}}{Lalchand M-6494}[]
\DeclareWitness{C8}{\selectlanguage{english}C\textsubscript{8}}{Lalchand M-2091}[]
\DeclareWitness{C8pc}{\selectlanguage{english}C\rlap{\textsubscript{8}}\textsuperscript{pc}\space}{}[]
\DeclareWitness{C9}{\selectlanguage{english}C\textsubscript{9}}{Lalchand M-4530}[]

% %%%%%%%%%%        Nepalese
\DeclareWitness{N1}{\selectlanguage{english}N\textsubscript{1}}{NGMPP A1400-2}[]
\DeclareWitness{N2}{\selectlanguage{english}N\textsubscript{2}}{NGMPP B 39-19}[]
\DeclareWitness{N3}{\selectlanguage{english}N\textsubscript{3}}{NGMPP B 62-20}[]
\DeclareWitness{N5}{\selectlanguage{english}N\textsubscript{5}}{NGMPP A60-15 + A61-1}[]
\DeclareWitness{N6}{\selectlanguage{english}N\textsubscript{6}}{NGMPP A61-6}[]
\DeclareWitness{N9}{\selectlanguage{english}N\textsubscript{9}}{NGMPP A62-33}[]
\DeclareWitness{N10}{\selectlanguage{english}N\textsubscript{10}}{NGMPP A62-37}[]
\DeclareWitness{N11}{\selectlanguage{english}N\textsubscript{11}}{NGMPP A63-15}[]
\DeclareWitness{N12}{\selectlanguage{english}N\textsubscript{12}}{NGMPP A939-19}[]
\DeclareWitness{N13}{\selectlanguage{english}N\textsubscript{13}}{NGMPP A1378-18}[]
\DeclareWitness{N16}{\selectlanguage{english}N\textsubscript{16}}{NGMPP B39-20}[]
\DeclareWitness{N17}{\selectlanguage{english}N\textsubscript{17}}{NGMPP B 111-10}[]
\DeclareWitness{N18}{\selectlanguage{english}N\textsubscript{18}}{NGMPP E 929-3}[]
\DeclareWitness{N19}{\selectlanguage{english}N\textsubscript{19}}{NGMPP E-1528-1 / E-1527-7(4)}[]
\DeclareWitness{N20}{\selectlanguage{english}N\textsubscript{20}}{NGMPP E 2037-13 }[]
\DeclareWitness{N21}{\selectlanguage{english}N\textsubscript{21}}{NGMPP E 2097-31}[]
\DeclareWitness{N22}{\selectlanguage{english}N\textsubscript{22}}{NGMPP G 4-4}[]
\DeclareWitness{N23}{\selectlanguage{english}N\textsubscript{23}}{NGMPP G 25-2}[]
\DeclareWitness{N24}{\selectlanguage{english}N\textsubscript{24}}{NGMPP G 190-16}[]
\DeclareWitness{N24ac}{\selectlanguage{english}N\rlap{\textsubscript{24}}\textsuperscript{ac}\space}{}[]
\DeclareWitness{N24pc}{\selectlanguage{english}N\rlap{\textsubscript{24}}\textsuperscript{pc}\space}{}[]

\DeclareWitness{P28}{\selectlanguage{english}P\textsubscript{28}}{BORI 399-1895-1902}[]

%%%%%   Mysore
\DeclareWitness{M1}{\selectlanguage{english}M\textsubscript{1}}{P-5682/4}[]
%%%%%   Tübingen
\DeclareWitness{Tü}{\selectlanguage{english}Tü}{Ma I 339}[]
%%%%%%%%%%
\DeclareWitness{YC}{\selectlanguage{english}YC}{Yogacintāmaṇi}[]
\DeclareWitness{ceteri}{\selectlanguage{english}cett.}{ceteri}[]

%%%%%%%%%% Mss with Commentary
\DeclareWitness{A1}{\selectlanguage{english}A\textsubscript{1}}{Alwar 952}[]


%%%%%%%%%%%%%%%%%%%%%%%%%%%%%%%%%%%%%%%%%%%
%List of all Sigla:
%A1,B1,B2,B3,C1,C2,C3,C4,C6,C7,C8,C9,J1,J2,J3,J4,J10,J13,J14,J15,J17,L1,M1,N3,N5,N6,N9,N10,N11,N12,N13,N16,N17,N19,N20,N21,N22,N23,N24,Tü,V1,V2,V3,V4,V5,V6,V8,V11,V19,V22,V26,Vu
%%%%%%%%%%%%%%%%%%%%%%%%%%%%%%%%%%%%%%%%%%%

\DeclareShorthand{x}{\selectlanguage{english}δ}{J10,J17,N17,P28,W4}


%%% Local Variables:
%%% mode: latex
%%% TeX-master: t
%%% End:

%
%%%%%                   Abbreviation for the printed apparatus,        xml interface needed
%%%%%                   (synonyms in same line)

% Macro for Editing Abbrevs.
%\def\om{\textrm{\footnotesize \textit{omitted in}\ }} %prints om. for omitted in apparatus
%\def\korr{\textrm{\footnotesize \textit{em.}\ }} %prints em. for emended in apparatus
%\def\conj{\textrm{\footnotesize \textit{conj.}\ }} %prints conj. for conjectured in apparatus


\def\eyeskip{\textrm{{ab.\,oc. }}}   
\def\aberratio{\textrm{{ab.\,oc. }}}
\def\ad{\textrm{{ad}}}   
\def\add{\textrm{{add.\ }}}
\def\ann{\textrm{{ann.\ }}}
\def\ante{\textrm{{ante }}}
\def\post{\textrm{{post }}}
%\def\ceteri{cett.\,}             % for simplifying the apparatus in print                  
\def\codd{\textrm{{codd.\ }}}   %  the same
\def\conj{\textrm{{coni.\ }}}  
\def\coni{\textrm{{coni.\ }}}
\def\contin{\textrm{{contin.\ }}}
\def\corr{\textrm{{corr.\ }}}
\def\del{\textrm{{del.\ }}}
\def\dub{\textrm{{ dub.\ }}}
\def\emend{\textrm{{emend.\ }}}
\def\expl{\textrm{{explic.\ }}}   
\def\explicat{\textrm{{explic.\ }}}
\def\fol{\textrm{{fol.\ }}}         
\def\foll{\textrm{{foll.\ }}}
\def\gloss{\textrm{{glossa ad }}}
\def\ins{\textrm{{ins.\ }}}          \def\inseruit{\textrm{{ins.\ }}}
\def\im{{\kern-.7pt\lower-1ex\hbox{\textrm{\tiny{\emph{i.m.}}}\kern0pt}}}
\def\inmargine{{\kern-.7pt\lower-.7ex\hbox{\textrm{\tiny{\emph{i.m.}}}\kern0pt}}}
\def\intextu{{\kern-.7pt\lower-.95ex\hbox{\textrm{\tiny{\emph{i.t.}}}\kern0pt}}}%\textrm{\scriptsize{i.t.\ }}}               
\def\indist{\textrm{{indis.\ }}}          \def\indis{\textrm{{indis.\ }}}
\def\iteravit{\textrm{{iter.\ }}}          \def\iter{\textrm{{iter.\ }}}  
\def\lectio{\textrm{{lect.\ }}}             \def\lec{\textrm{{lect.\ }}}
\def\leginequit{\textrm{{l.n. }}}         \def\legn{\textrm{{l.n. }}}         \def\illeg{\textrm{{l.n. }}}
\def\om{\textrm{{om. }}}
\def\primman{\textrm{{pr.m.}}}
\def\prob{\textrm{{prob.}}}
\def\rep{\textrm{{repetitio }}}
% \def\secundamanu{\textrm{\scriptsize{s.m.}}}
% \def\secm{{\kern-.6pt\lower-.91ex\hbox{\textrm{\tiny{\emph{s.m.}}}\kern0pt}}}%   \textrm{\scriptsize{s.m.}}}
\def\sequentia{\textrm{{seq.\,inv.\ }}}         \def\seqinv{\textrm{{seq.\,inv.\ }}} \def\order{\textrm{{seq.\,inv.\ }}}
\def\supralineam{{\kern-.7pt\lower-.91ex\hbox{\textrm{\tiny{\emph{s.l.}}}\kern0pt}}} %\textrm{\scriptsize{s.l.}}}
\def\interlineam{{\kern-.7pt\lower-.91ex\hbox{\textrm{\tiny{\emph{s.l.}}}\kern0pt}}}   %\textrm{\scriptsize{s.l.}}}
\def\vl{\textrm{v.l.}}   \def\varlec{\textrm{v.l.}} \def\varialectio{\textrm{v.l.}}
\def\vide{\textrm{{cf.\ }}}           \def\cf{\textrm{{cf.\ }}}
\def\videtur{\textrm{{vid.\,ut}}}
\def\crux{\textup{[\ldots]} }
\def\cruxx{\textup{[\ldots]}}
\def\unm{\textit{unm.}}        % unmetrical
%%%%%%%%%%%%%%%%%%%%%%%%%%%%%%%%%%%%



%%% Local Variables:
%%% mode: latex
%%% TeX-master: t
%%% End:

% additions/changes 2024-07-04 mm:
\TeXtoTEIPat{\lb}{<lb/>}
\TeXtoTEIPat{\begin {quote}}{<q>}
  \TeXtoTEIPat{\end {quote}}{</q>}
\TeXtoTEIPat{\begin {enumerate}}{<list rend="numbered">}
  \TeXtoTEIPat{\end {enumerate}}{</list>}
\TeXtoTEI{item}{item}

% additions/changes 2024-07-01 mm:
\TeXtoTEIPat{\unavbl {#1}}{<note type="foliolost">Folio lost in <ref>#1</ref></note>}
\TeXtoTEIPat{\NotIn {#1}}{<note type="omission">Omitted in <ref>#1</ref></note>}
\TeXtoTEI{graus}{span}[type="altrec"]
\TeXtoTEI{grau}{span}[type="altrec"]

% addition 2024-03-15 MD
\TeXtoTEI{manuref}{}

\TeXtoTEIPat{\alphaOne}{α<hi rend="sub">1</hi>}% N3
\TeXtoTEIPat{\alphaTwo}{α<hi rend="sub">2</hi>}% J5
\TeXtoTEIPat{\alphaThree}{α<hi rend="sub">3</hi>}% G4
\TeXtoTEIPat{\betaOne}{β<hi rend="sub">1</hi>}% P11
\TeXtoTEIPat{\betaTwo}{β<hi rend="sub">2</hi>}% C6
\TeXtoTEIPat{\betaOmega}{β<hi rend="sub">ω</hi>}% V3
\TeXtoTEIPat{\gammaOne}{γ<hi rend="sub">1</hi>}% N23
\TeXtoTEIPat{\gammaTwo}{γ<hi rend="sub">2</hi>}% J7
\TeXtoTEIPat{\deltaOne}{δ<hi rend="sub">1</hi>}% V19
\TeXtoTEIPat{\deltaTwo}{δ<hi rend="sub">2</hi>}% K3
\TeXtoTEIPat{\deltaThree}{δ<hi rend="sub">3</hi>}% C7
\TeXtoTEIPat{\deltaOmega}{δ<hi rend="sub">ω</hi>}% J6
\TeXtoTEIPat{\epsilonOne}{ε<hi rend="sub">1</hi>}% P15
\TeXtoTEIPat{\epsilonTwo}{ε<hi rend="sub">2</hi>}% N19
\TeXtoTEIPat{\epsilonThree}{ε<hi rend="sub">3</hi>}% V15
\TeXtoTEIPat{\epsilonFour}{ε<hi rend="sub">4</hi>}% J11
\TeXtoTEIPat{\epsilonOmega}{ε<hi rend="sub">ω</hi>}% N26
\TeXtoTEIPat{\etaOne}{η<hi rend="sub">1</hi>}% V1
\TeXtoTEIPat{\etaTwo}{η<hi rend="sub">2</hi>}% J10
\TeXtoTEIPat{\etaOmega}{η<hi rend="sub">ω</hi>}% N9

% addition 2023-12-11 MD:
\TeXtoTEIPat{\begin {metre}[#1]}{<note type="metre" target="##1">}
\TeXtoTEIPat{\end {metre}}{</note>}
\TeXtoTEIPat{\texttheta}{θ}

% change 2023-12-05 mm
\TeXtoTEI{teimute}{} 

% changes/additions 2023-11-27 MM:
\TeXtoTEIPat{\medialink {#1}{#2}}{<ref target="resources/#2">#1</ref>}

% changes/additions 2023-10-25 MM:
% new Sigla
\TeXtoTEIPat{\textAlpha}{Α}
\TeXtoTEIPat{\textalpha}{α}
\TeXtoTEIPat{\textBeta}{Β}
\TeXtoTEIPat{\textbeta}{β}
\TeXtoTEIPat{\textGamma}{Γ}
\TeXtoTEIPat{\textgamma}{γ}
\TeXtoTEIPat{\textDelta}{Δ}
\TeXtoTEIPat{\textdelta}{δ}
\TeXtoTEIPat{\textEpsilon}{Ε}
\TeXtoTEIPat{\textepsilon}{ε}
\TeXtoTEIPat{\textEta}{Η}
\TeXtoTEIPat{\texteta}{η}
\TeXtoTEIPat{\textChi}{Χ}
\TeXtoTEIPat{\textchi}{χ}
\TeXtoTEIPat{\textOmega}{Ω}
\TeXtoTEIPat{\textomega}{ω}

%new environments
\TeXtoTEIPat{\begin {postmula}[#1]}{<div type="postmula" xml:id="#1">} %%% changed 2024-07-01 mm
  \TeXtoTEIPat{\end {postmula}}{</div>}  %%% changed 2024-07-01 mm
  
\TeXtoTEIPat{\begin {altpostmula}[#1]}{<div type="altrec"><div type="postmula" xml:id="#1">} %%% added 2024-07-03 md
  \TeXtoTEIPat{\end {altpostmula}}{</div></div>} %%% added 2024-07-03 md

\TeXtoTEIPat{\begin {altava}[#1]}{<div type="altrec"><div type="avataranika" xml:id="#1">} %%% changed 2024-07-01 mm
  \TeXtoTEIPat{\end {altava}}{</div></div>} %%% changed 2024-07-01 mm

\TeXtoTEIPat{\sgwit {#1}}{<note type="inlineref"><ref>#1</ref></note>}

% changes/additions 2023-10-12 MM:
\TeXtoTEIPat{\\.}{}

% changes/additions 2023-08-15 MD:
\TeXtoTEIPat{\lineom {#1}{#2}}{<note type="omission">#1 omitted in <ref>#2</ref></note>}
%\TeXtoTEIPat{\startgray}{} %%% changed 2023-12-05 mm; not used 2024-03-26 MD
%\TeXtoTEIPat{\endgray}{} %%% changed 2023-12-05 mm; not used 2024-03-26 MD

% additions/changes 2023-06-05 mm:
%\TeXtoTEIPat{\lineom {#1}}{<note type="omission">Line omitted in <ref>#1</ref></note>}

% additions 2023-04-16 MD:
\TeXtoTEIPat{\,}{ }

% additions 2023-04-13 mm:
\TeXtoTEIPat{\begin {versinnote}}{<lg>}
  \TeXtoTEIPat{\end {versinnote}}{</lg>}

% additions 2023-04-05 MD:
\TeXtoTEIPat{\begin {testimonia}[#1]}{<note type="testimonia" target="##1">}
  \TeXtoTEIPat{\end {testimonia}}{</note>}
\TeXtoTEI{devnote}{s}[xml:lang="sa-deva"]

% app in philcomm und testimonia %%% added MM 2023-12-02
\TeXtoTEI{var}{note}[type="appinnote"]


\TeXtoTEI{anm}{note}[type="memo"] %% change 2023-04-16 MD
\TeXtoTEI{Anm}{note}[type="memo"] %% change 2023-12-05 MM
\TeXtoTEIPat{\startverse}{} %%% marked for change 2023-04-13 mm
\TeXtoTEIPat{\endverse}{} %%% marked for change 2023-04-13 mm
\TeXtoTEIPat{\newpage}{}
\TeXtoTEIPat{\marmas}{ } % changed 2024-03-17 MD
\TeXtoTEIPat{\marma}{}
\TeXtoTEIPat{\vin}{} % added by MD 2023-11-14

%%% modify environments and commands
%%% TEI mapping
% additions/changes 2022-06-07 mm:
\TeXtoTEIPat{ \& }{ &amp; }

% additions/changes 2022-06-01 mm:
\TeXtoTEI{skp}{seg}[type="deva-ignore"]
\TeXtoTEI{skm}{seg}[type="ltn-ignore"]

\TeXtoTEIPat{\rlap {#1}}{#1}

% additions/changes 2022-04-06 mm:
%\TeXtoTEI{sgwit}{ref}
\TeXtoTEI{textdev}{s}[xml:lang="sa-deva"]
\TeXtoTEIPat{\begin {col}[#1]}{<div type="colophon" xml:id="#1">}
  \TeXtoTEIPat{\end {col}}{</div>}
\TeXtoTEIPat{\begin {ava}[#1]}{<div type="avataranika" xml:id="#1">} %%% changed 2024-07-01 mm
  \TeXtoTEIPat{\end {ava}}{</div>} %%% changed 2024-07-01 mm
												   
\TeXtoTEIPat{\outdent}{}
\TeXtoTEIPat{\startaltrecension}{} %%% changed 2023-12-05 mm
\TeXtoTEIPat{\endaltrecension}{} %%% changed 2023-12-05 mm
\TeXtoTEIPat{\startaltnormal}{} % added by MD 2023-11-14 %%% changed 2023-12-05 mm
\TeXtoTEIPat{\endaltnormal}{} % added by MD 2023-11-14 %%% changed 2023-12-05 mm
\TeXtoTEIPat{\begin {alttlg}[#1]}{<div type="altrec"><lg xml:id="#1">}
  \TeXtoTEIPat{\end {alttlg}}{</lg></div>}



% additions/changes 2022-03-12 mm:
\TeXtoTEIPat{\begin {tlg}[#1]}{<lg xml:id="#1">}
  \TeXtoTEIPat{\end {tlg}}{</lg>}

\TeXtoTEIPat{\begin {translation}[#1]}{<note type="translation" target="##1">}
  \TeXtoTEIPat{\end {translation}}{</note>}
\TeXtoTEIPat{\begin {philcomm}[#1]}{<note type="philcomm" target="##1">}
  \TeXtoTEIPat{\end {philcomm}}{</note>}
\TeXtoTEIPat{\begin {sources}[#1]}{<note type="sources" target="##1">}
  \TeXtoTEIPat{\end {sources}}{</note>}


\TeXtoTEIPat{\begin {marma}[#1]}{<note type="marma" target="##1">}
  \TeXtoTEIPat{\end {marma}}{</note>}

\TeXtoTEIPat{\begin {jyotsna}[#1]}{<note type="jyotsna" target="##1">}
  \TeXtoTEIPat{\end {jyotsna}}{</note>}

\EnvtoTEI{description}{list}
\EnvtoTEI{itemize}{list}
\TeXtoTEIPat{\item [#1]}{<label>#1</label>\item}

\TeXtoTEI{tl}{l}
\TeXtoTEI{myfn}{note}[type="myfn"]
\TeXtoTEIPat{\getsiglum {#1}}{<ref target="##1"/>}

\TeXtoTEI{SetLineation}{}
\TeXtoTEI{noindent}{}
\TeXtoTEI{subsection*}{}

\TeXtoTEI{rlap}{}

% end additions/changes
% \TeXtoTEIPat{\skp {#1}}{#1}
% \TeXtoTEIPat{\skm {#1}}{}

\TeXtoTEIPat{\begin {prose}}{<p>}
  \TeXtoTEIPat{\end {prose}}{</p>}

\TeXtoTEIPat{\begin {tlate}}{<p>}
  \TeXtoTEIPat{\end {tlate}}{</p>}

\TeXtoTEI{emph}{hi}
\TeXtoTEI{bigskip}{}
% \TeXtoTEI{/}{|}
\TeXtoTEI{tl}{l}
\TeXtoTEIPat{english}{}
%\TeXtoTEIPat{-}{ } %% change 2023-04-16 MD
%\TeXtoTEIPat{°}{} %% change 2023-04-16 MD
\TeXtoTEIPat{\textcolor {#1}{#2}}{<hi rend="#1">#2</hi>}

% \TeXtoTEIPat{\eyeskip}{}
% \TeXtoTEIPat{\aberratio}{}
% \TeXtoTEIPat{\ad}{}
\TeXtoTEIPat{\add}{<hi rend="italic">add.</hi>} %% change 2023-04-16 MD
% \TeXtoTEIPat{\ann}{}
\TeXtoTEIPat{\ante}{<hi rend="italic">ante</hi> } %% change 2023-04-16 MD
\TeXtoTEIPat{\post}{<hi rend="italic">post</hi> } %% change 2023-04-16 MD
% \TeXtoTEIPat{\codd}{}
% \TeXtoTEIPat{\conj }{}
% \TeXtoTEIPat{\contin}{}
% \TeXtoTEIPat{\corr}{}
% \TeXtoTEIPat{\del}{}
% \TeXtoTEIPat{\dub}{}
% \TeXtoTEIPat{\emend }{}
% \TeXtoTEIPat{\expl}{}
% \TeXtoTEIPat{\ȩxplicat}{}
% \TeXtoTEIPat{\fol}{}
% \TeXtoTEIPat{\gloss}{}
% \TeXtoTEIPat{\ins}{}
% \TeXtoTEIPat{\im}{}
% \TeXtoTEIPat{\inmargine}{}
% \TeXtoTEIPat{\intextu}{}
% \TeXtoTEIPat{\indist}{}
% \TeXtoTEIPat{\iteravit}{}
% \TeXtoTEIPat{\lectio}{}
% \TeXtoTEIPat{\leginequit}{}
% \TeXtoTEIPat{\legn}{}
% \TeXtoTEIPat{\illeg}{<hi rend="italic">illeg.</hi>}
\TeXtoTEIPat{\illeg}{<gap reason="illeg."/>} %%% change 2023-04-11 mm
% \TeXtoTEIPat{\om}{<hi rend="italic">om.</hi>}
\TeXtoTEIPat{\om}{<gap reason="om."/>} %%% change 2023-04-11 mm
% \TeXtoTEIPat{\primman}{}
% \TeXtoTEIPat{\prob}{}
% \TeXtoTEIPat{\rep}{}
% \TeXtoTEIPat{\sequentia}{}
% \TeXtoTEIPat{\supralineam}{}
% \TeXtoTEIPat{\interlineam}{}
\TeXtoTEIPat{\vl}{<hi rend="italic">v.l.</hi>}
% \TeXtoTEIPat{\vide}{}
% \TeXtoTEIPat{\videtur}{}
% \TeXtoTEIPat{\crux}{}
% \TeXtoTEIPat{\cruxxx}{}
\TeXtoTEIPat{\unm}{<hi rend="italic">unm.</hi>}
\TeXtoTEIPat{\lacuna}{<gap reason="lac."/>} % addition 2024-03-24 MD
\TeXtoTEIPat{\lost}{<gap reason="lost"/>} % addition 2024-06-24 MD

% List of Scholars
\DeclareScholar{nos}{nos}[
forename=HPP,
surname=Team]

% Nullify \selectlanguage in TEI as it has been used in
% \DeclareWitness but should be ignored in TEI.
\TeXtoTEI{selectlanguage}{}


\SetTEIxmlExport{autopar=false}

%%%%%%%%%%%

\SetTEIxmlExport{autopar=false}
\NewDocumentEnvironment{translation}{O{}}{\textcolor{blue}{\textbf{Translation:}}}{}
\NewDocumentEnvironment{philcomm}{O{}}{
	\textcolor{blue}{\textbf{Commentary:}}}{}
\NewDocumentEnvironment{metre}{O{}}{
	\textcolor{blue}{\textbf{Metre:}}}{} % added MD 2023-12-11
\NewDocumentEnvironment{sources}{O{}}{
	\textcolor{blue}{\textbf{Sources:}}\linebreak}{}
\NewDocumentEnvironment{testimonia}{O{}}{
	\textcolor{blue}{\textbf{Testimonia:}}\linebreak}{}
\NewDocumentEnvironment{versinnote}{O{}}{\begin{ekdverse}}{\end{ekdverse}}
%\newcommand{\var}[1]{\footnotesize\textup{#1}}
\newcommand{\medialink}[2]{\textcolor{green}{\underline{#1}}}
%\TeXtoTEIPat{\medialink {#1}{#2}}{<ref target="/images/#2">#1</ref>}

\NewDocumentCommand{\tl}{m}{#1}

\def\vl{\textit{v.l.}}
\def\var#1{{\footnotesize #1}}
\def\sl#1{\emph{#1}}

%%%%%%%%%%%%

\usepackage{textgreek}

\newcommand{\alphaOne}{\textalpha\textsubscript{1}}% N3
\newcommand{\alphaTwo}{\textalpha\textsubscript{2}}% J5
\newcommand{\alphaThree}{\textalpha\textsubscript{3}}% G4
\newcommand{\betaOne}{\textbeta\textsubscript{1}}% P11
\newcommand{\betaTwo}{\textbeta\textsubscript{2}}% C6
\newcommand{\betaOmega}{\textbeta\textsubscript{\textomega}}% V3
\newcommand{\gammaOne}{\textgamma\textsubscript{1}}% N23
\newcommand{\gammaTwo}{\textgamma\textsubscript{2}}% J7
\newcommand{\deltaOne}{\textdelta\textsubscript{1}}% V19
\newcommand{\deltaTwo}{\textdelta\textsubscript{2}}% K3
\newcommand{\deltaThree}{\textdelta\textsubscript{3}}% C7
\newcommand{\deltaOmega}{\textdelta\textsubscript{\textomega}}% J6
\newcommand{\epsilonOne}{\textepsilon\textsubscript{1}}% P15
\newcommand{\epsilonTwo}{\textepsilon\textsubscript{2}}% N19
\newcommand{\epsilonThree}{\textepsilon\textsubscript{3}}% V15
\newcommand{\epsilonFour}{\textepsilon\textsubscript{4}}% J11
\newcommand{\epsilonOmega}{\textepsilon\textsubscript{\textomega}}% N26
\newcommand{\etaOne}{\texteta\textsubscript{1}}% V1
\newcommand{\etaTwo}{\texteta\textsubscript{2}}% J10
\newcommand{\etaOmega}{\texteta\textsubscript{\textomega}}% N9

%%%%%%%%%%%%%%

\babelhyphenation{%
	Dattā-treya-yoga-śāstra
	Gorakṣa-śataka
	Haṭha-pra-dī-pikā
	Hātha-ratnā-valī
	Svātmā-rāma
	Śiva-saṃhitā
	Vasiṣṭha-saṃhitā
	Viveka-mārtaṇḍa
	Yukta-bhava-deva
	Yoga-cintā-maṇi
	Yoga-yājña-valkya}

\begin{document}
\pagestyle{HPed}
\begin{ekdosis}
\SetLineation{lineation = none,}

%\chapter*{Translation \& philological commentary}
%%%%%%%%%%
\subsection*{3.1}
\begin{translation}[hp03_001]
Just as the lord of snakes is the foundation of the regions of the earth along with their mountains and forests, so Kuṇḍalinī is the foundation of all systems of yoga.
\end{translation}

%\begin{sources}[hp03_001]
%\end{sources}

\begin{testimonia}[hp03_001]
\emph{Haṭharatnāvalī} 2.124
\begin{versinnote}
\tl{saśailavanadhātryās tu yathādhāro 'hināyakaḥ |\\+}
\tl{aśeṣayogatantrāṇāṃ tathādhāro hi kuṇḍalī ||\\!}
\end{versinnote}

\emph{Yogacintāmaṇi} f.~71v
\begin{versinnote}
\tl{haṭhapradīpikāyām—\\+}
\tl{saśailavanadhātrīṇāṃ yathādhāro 'hināyakaḥ |\\+}
\tl{sarveṣāṃ yogatantrāṇāṃ tathādhāro hi kuṇḍalī ||\\!}
\end{versinnote}

\emph{Yuktabhavadeva} 7.170
\begin{versinnote}
\tl{tatra haṭhapradīpikāyām--\\+}
\tl{saśailavanadhātrīṇāṃ yathādhāro 'hināyakaḥ |\\+}
\tl{sarveṣāṃ yogatantrāṇāṃ tathādhāro hi kuṇḍalī ||\\!}
\end{versinnote}
\end{testimonia}

\begin{philcomm}[hp03_001]
The plural \emph{°dhātrīṇām} is hard to construe. Only this world has mountains and forests, but we want a plural for the comparison with \emph{°tantrāṇām}. Brahmānanda (\emph{Jyotsnā} 3.1) understands \emph{°dhātrīṇām} to refer to the different regions of the earth, even though the world (\emph{dhātrī}) is a single entity (\emph{dhātryā ekatve 'pi deśabhedād bhedam ādāya bahuvacanam}). A similar comment occurs in the \emph{Yogaprakāśikā} 5.1 (\emph{yathā samastadvīpādisahitapṛthvī ādhāraḥ phaṇīndras tathā samastayogādhāraḥ kuṇḍalīty āha saśaileti}). The author of the \textit{Haṭharatnāvalī} circumvented this issue by adopting the reading °\textit{dhātryāḥ}.
\end{philcomm}

%%%%%%%%%%
\subsection*{3.2}
\begin{translation}[hp03_002]
When the sleeping Kuṇḍalinī awakens through the favour of the guru, then all the lotuses are pierced, and the knots too, [\dots] 
\end{translation}

\begin{sources}[hp03_002]
\emph{Śivasaṃhitā} 4.21
\begin{versinnote}
\tl{suptā guruprasādena yadā jāgarti kuṇḍalī |\\+}
\tl{tadā sarvāṇi padmāni bhidyante granthayo 'pi ca ||\\!}
\end{versinnote}
\end{sources}

\begin{testimonia}[hp03_002]
\emph{Yogacintāmaṇi} f.~71v (attr.~to the \emph{Haṭhapradīpikā})
\begin{versinnote}
\tl{suptā guruprasādena bodhitā sukhadā bhavet |\\+}
\tl{tathā sarvāṇi padmāni bhidyante granthayo 'pi ca ||\\!}
\end{versinnote}

\emph{Yuktabhavadeva} 171 (attr.~to the \emph{Haṭhapradīpikā})
\begin{versinnote}
\tl{suptā guruprasādena yadā jāgarti kuṇḍalī |\\+}
\tl{tadā sarvāṇi padmāni bhidyante granthayo 'pi ca ||\\!}
\end{versinnote}
\end{testimonia}

\begin{philcomm}[hp03_002]
The usual meaning of \emph{jāgarti} would be “is wakeful” rather than “awakens”, which explains the variant \emph{bodhitā}.
\end{philcomm}

%%%%%%%%%%
\subsection*{3.3}
\begin{translation}[hp03_003]
[\dots] the empty pathway becomes the royal highway for \emph{prāṇa}, the mind becomes free of support, and death is cheated.
\end{translation}

\begin{sources}[hp03_003]
\end{sources}

\begin{testimonia}[hp03_003]
\emph{Yogacintāmaṇi} f.~72r (attr.~to the \emph{Haṭhapradīpikā})

\begin{versinnote}
\tl{prāṇasya śūnyapadavī tathā rājapathāyate |\\+}
\tl{tathā cittaṃ nirālambaṃ tathā kālasya vañcanam ||\\!}
\end{versinnote}

\end{testimonia}

%\begin{philcomm}[hp03_003]
%JM: take prāṇasya with śūnyapadavī?
%JB: tadā in 3b indicates that this verse should be read with 3.2 (i.e., [when kuṇḍalinī has awakened...] then, the empty pathway becomes the royal pathway for prāṇa)
%\end{philcomm}

\begin{metre}[hp03_003]
Anuṣṭubh (a: na-vipulā)
\end{metre}

%%%%%%%%%%
\subsection*{3.4 heading}
\begin{translation}[hp03_004a]
What is ``the empty pathway"?
\end{translation}

% \begin{philcomm}[hp03_004a]
% \end{philcomm}

%%%%%%%%%%
\subsection*{3.4}
\begin{translation}[hp03_004]
Suṣumṇā, the empty pathway, the great path to the aperture of Brahmā, the cremation ground, Śāmbhavī, and the middle path are synonyms.
\end{translation}

\begin{sources}[hp03_004]
Cf.~\emph{Amṛtasiddhi} 2.6
\begin{versinnote}
\tl{avadhūtīpadaṃ ke cic chmaśānaṃ ca mahāpatham |\\+}
\tl{ke cid vadanti ādhārāṃ suṣumnāṃ ca sarasvatīm ||\\!}
\end{versinnote}

Cf.~\emph{Dattātreyayogaśāstra} 109c–110b
\begin{versinnote}
\tl{mahāpathaṃ śmaśānaṃ ca suṣumnāpy ekam eva hi ||\\+}
\tl{nāmnāṃ matāntare bhedaḥ phale bhedo na vidyate |\\!}
\end{versinnote}
\end{sources}

\begin{testimonia}[hp03_004]
\emph{Yogacintāmaṇi} f.~59r
\begin{versinnote}
\tl{haṭhapradīpikāyām—\\+}
\tl{suṣumṇā śūnyapadavī brahmarandhraṃ mahāpatham |\\+}
\tl{śmaśānī śāṃbhavī madhyamārgaś cety ekavācakā iti ||\\!}
\end{versinnote}

\emph{Yuktabhavadeva} 7.172 (attr.~to the \emph{Haṭhapradīpikā})
\begin{versinnote}
\tl{prāṇasya śūnyapadavī mahārandhraṃ mahāpatham |\\+}
\tl{śmaśānaṃ śāmbhavī madhyamārgaś cety ekavācakam ||\\!}
\end{versinnote}
\end{testimonia}

%\begin{philcomm}[hp03_004]
%In the context of \textit{suṣumṇā}, Svātmārāma seems to understand \emph{śāmbhavī} as the power (\emph{śakti}) of Śambhu (Cf.~\emph{Haṭhapradīpikā} 4.78c \emph{suṣumṇā śāmbhavī śaktiḥ}). JM: I'm not convinced by śāmbhavī = power of Śambhu; I think it's more feminine.
%\end{philcomm}

\begin{metre}[hp03_004]
Anuṣṭubh (a: na-vipulā)
\end{metre}

%%%%%%%%%%
\subsection*{3.5}
\begin{translation}[hp03_005]
Therefore, in order to do his utmost to awaken the goddess sleeping in front of the doorway of Brahman, [the yogi] should carry out the practice of \emph{mudrā}.
\end{translation}

\begin{sources}[hp03_005]
\emph{Śivasaṃhitā} 4.22
\begin{versinnote}
\tl{tasmāt sarvaprayatnena prabodhayitum īśvarīm |\\+}
\tl{brahmadvāramukhe suptāṃ mudrābhyāsaṃ samācaret ||\\!}
\end{versinnote}
\end{sources}

\begin{testimonia}[hp03_005]
\emph{Yogacintāmaṇi} f.~59r (attr.~to the \emph{Haṭhapradīpikā})
\begin{versinnote}
\tl{tasmāt sarvaprayatnena prabodhayitum īśvarīm |\\+}
\tl{brahmadvāramukhe suptāṃ mudrābhyāsaparo bhavet ||\\!}
\end{versinnote}

\emph{Yuktabhavadeva} 7.173 (attr.~to the \emph{Haṭhapradīpikā})
\begin{versinnote}
\tl{tasmāt sarvaprayatnena prabodhayitum īśvarīm |\\+}
\tl{brahmadvāramukhe suptāṃ mudrābhyāsaṃ samācaret ||\\!}
\end{versinnote}
\end{testimonia}

%\begin{philcomm}[hp03_005]
%Consider pathe for mukhe [JB: no support for pathe]
%\end{philcomm}

%%%%%%%%%%
\subsection*{3.6}
\begin{translation}[hp03_006]
The great seal, the great lock, the great piercing, the sky-roving [seal], the \emph{uḍḍiyāna} [lock], the root lock, then [the lock] called \emph{jālandhara}, [\dots] 
\end{translation}

\begin{sources}[hp03_006]
%Cf.~Amṛtasiddhi 13.12
%\begin{versinnote}
%\tl{mahāmudrā mahābandho mahāvedhas tṛtīyakaḥ |\\!}
%\end{versinnote}

\emph{Śivasaṃhitā} 4.23
\begin{versinnote}
\tl{mahāmudrā mahābandho mahāvedhaś ca khecarī |\\+}
\tl{jālandharo mūlabandho viparītakṛtis tathā ||\\!}
\end{versinnote}
\end{sources}

\begin{testimonia}[hp03_006]
\emph{Haṭharatnāvalī} 2.32

\begin{versinnote}
\tl{mahāmudrā mahābandho mahāvedhas tṛtīyakaḥ |\\+}
\tl{uḍḍiyānaṃ mūlabandho bandho jālandharābhidhaḥ ||\\!}
\end{versinnote}

\emph{Yogacintāmaṇi} f.~72r (attr.~to the \emph{Haṭhapradīpikā})
\begin{versinnote}
\tl{mahāmudrā mahābandho mahāvedhaś ca khecarī |\\+}
\tl{uḍḍiyānaṃ mūlabandho bandho jālandharābhidhaḥ ||\\!}
\end{versinnote}

\emph{Yuktabhavadeva} 7.174 (attr.~to the \emph{Haṭhapradīpikā})
\begin{versinnote}
\tl{mahāmudrā mahābandho mahāvedhaś ca khecarī |\\+}
\tl{uḍyānaṃ mūlabandhaś ca bandho jālandharas tathā ||\\!}
\end{versinnote}
\end{testimonia}


%\begin{philcomm}[hp03_006]
%JM: Do we translate mudrā or not?
%\end{philcomm}

\begin{metre}[hp03_006]
Anuṣṭubh (c: ra-vipulā)
\end{metre}

%%%%%%%%%%
\subsection*{3.7}
\begin{translation}[hp03_007]
[\ldots] the bodily position called inverted, \emph{vajrolī} [and] the stimulation of the goddess: this group of ten \emph{mudrā}s and other [practices] destroys old age and death. 
\end{translation}

\begin{sources}[hp03_007]
\emph{Śivasaṃhitā} 4.24
\begin{versinnote}
\tl{uḍyānaṃ caiva vajrolī daśamaṃ śakticālanam |\\+}
\tl{idaṃ hi mudrādaśakaṃ mudrāṇām uttamottamam ||\\!}
\end{versinnote}
\end{sources}

\begin{testimonia}[hp03_007]
\emph{Haṭharatnāvalī} 2.33
\begin{versinnote}
\tl{karaṇī viparītākhyā vajrolī śakticālanam |\\+}
\tl{sampradāyā khecarī sā daśa mudrāḥ prakīrtitāḥ ||\\!}
\end{versinnote}

\emph{Yogacintāmaṇi} f.~72r (attr.~to the \emph{Haṭhapradīpikā})
\begin{versinnote}
\tl{karaṇī viparītākhyā tathā vai śakticālanam |\\+}
\tl{etad dhi mudrānavakaṃ jarāmaraṇavarjitam ||\\!}
\end{versinnote}

\emph{Yuktabhavadeva} 7.175 (attr.~to the \emph{Haṭhapradīpikā})
\begin{versinnote}
\tl{viparītakṛtiś caiva vajrolī śakticālanam |\\+}
\tl{idaṃ hi mudrādaśakaṃ mudrāṇām uttamomam ||\\!}
\end{versinnote}

\end{testimonia}

%\begin{philcomm}[hp03_007]
%JB: hi (in 3.7c) is supported only by the \textit{Jyotsnā}. Mss. have tu and ca.
%\end{philcomm}

\begin{metre}[hp03_007]
Anuṣṭubh (c: na-vipulā)
\end{metre}

%%%%%%%%%%
\subsection*{3.8}
\begin{translation}[hp03_008]
It has been taught by Śiva, is divine, bestows the eight supramundane powers, is beloved of all the Siddhas, is difficult for even the gods to obtain, [\dots] 
\end{translation}

\begin{sources}[hp03_008]
\end{sources}

\begin{testimonia}[hp03_008]
\emph{Yogacintāmaṇi} f.~72r (attr.~to the \emph{Haṭhapradīpikā})
\begin{versinnote}
\tl{ādināthoditaṃ divyam aṣṭaiśvaryapradāyakam |\\+}
\tl{vallabhaṃ sarvasiddhānāṃ durlabhaṃ marutām api ||\\!}
\end{versinnote}

\emph{Yuktabhavadeva} 7.176 (attr.~to the \emph{Haṭhapradīpikā})
\begin{versinnote}
\tl{ādināthoditaṃ samyagaṣṭaiśvaryapradāyakam |\\+}
\tl{vallabhaṃ sarvasiddhendradurlabhaṃ marutām api ||\\!}
\end{versinnote}
\end{testimonia}

%\begin{philcomm}[hp03_008]
%\end{philcomm}

%%%%%%%%%%
\subsection*{3.9}
\begin{translation}[hp03_009]
[\ldots] should be carefully kept secret like a casket of gems [and] must not be spoken of to anyone, like sex with a respectable woman.
\end{translation}

\begin{sources}[hp03_009]
\end{sources}

\begin{testimonia}[hp03_009]
\emph{Yogacintāmaṇi} f.~72r (attr.~to the \emph{Haṭhapradīpikā})
\begin{versinnote}
\tl{gopanīyaṃ prayatnena yathā ratnakaraṇḍakam |\\+}
\tl{kasyacin naiva vaktavyaṃ kulastrīsurataṃ yathā ||\\!}
\end{versinnote}

\emph{Yuktabhavadeva} 7.177 (attr.~to the \emph{Haṭhapradīpikā})
\begin{versinnote}
\tl{gopanīyaṃ prayatnena jarāmaraṇanāśanam |\\+}
\tl{kasyacinnaiva vaktavyaṃ kulastrīsurataṃ yathā || \\!}
\end{versinnote}
\end{testimonia}

%\begin{philcomm}[hp03_009]
%\end{philcomm}

%%%%%%%%%%
\subsection*{3.9*1}
\begin{translation}[hp03_009_1]
[With] \emph{Amarolī} and \emph{Sahajolī}, Vajrolī is considered to be threefold. I shall teach their characteristics and the details of how they should be performed.
\end{translation}

\begin{sources}[hp03_009_1]
\emph{Dattātreyayogaśāstra} 31c–32b
\begin{versinnote}
\tl{vajrolī cāmarolī ca sahajolī tridhā matā |\\+}
\tl{eteṣāṃ lakṣaṇaṃ vakṣye kartavyaṃ ca viśeṣataḥ ||\\+}
\tl{\var{vajrolī ] vajrolir \vl • cāmarolī ] amaroliś \vl • sahajolī ] sahajolis \vl}\\!}
\end{versinnote}
\end{sources}


%\begin{philcomm}[hp03_009_1]
%\end{philcomm}

%%%%%%%%%%
\subsection*{3.10 heading}
\begin{translation}[hp03_010a]
Now the great seal (\emph{mahāmudrā}):
\end{translation}

% \begin{philcomm}[hp03_010a]
% \end{philcomm}

%%%%%%%%%%
\subsection*{3.10}
\begin{translation}[hp03_010]
{}[The yogi] should press the perineum with the heel of the left foot, hold [the foot of] the extended leg with the hands and breathe in through the mouth.
\end{translation}

\begin{sources}[hp03_010]
\emph{Amaraugha} 19
\begin{versinnote}
\tl{pādamūlena vāmena yoniṃ saṃpīḍya dakṣiṇam |\\+}
\tl{pādaṃ prasāritaṃ dhṛtvā karābhyāṃ pūrayen mukhe ||\\+}
\tl{\var{dhṛtvā ] kṛtvā \vl}\\!}
\end{versinnote}

Cf.~\emph{Amṛtasiddhi} 11.3
\begin{versinnote}
\tl{yoniṃ saṃpīḍya vāmena pādamūlena yatnataḥ |\\+}
\tl{savyaṃ prasāritaṃ pādaṃ karābhyāṃ dhārayed dṛḍham ||\\!}
\end{versinnote}
\end{sources}

\begin{testimonia}[hp03_010]
\emph{Haṭharatnāvalī} 2.37
\begin{versinnote}
\tl{pādamūlena vāmena yoniṃ sampīḍya dakṣiṇam |\\+}
\tl{pādaṃ prasāritaṃ kṛtvā karābhyāṃ pūrayen mukham ||\\+}
\tl{\var{mukham ] mukhe \vl}\\!}
\end{versinnote}

\emph{Yogacintāmaṇi} ff. 72v–73r (attr.~to the \emph{Haṭhapradīpikā})
\begin{versinnote}
\tl{pādamūlena vāmena yoniṃ saṃpīḍya dakṣiṇam |\\+}
\tl{pādaṃ prasāritaṃ dhṛtvā karābhyāṃ pūrayen mukham ||\\!}
\end{versinnote}

\emph{Yuktabhavadeva} 7.178 (attr.~to the \emph{Haṭhapradīpikā})
\begin{versinnote}
\tl{pādamūlena vāmena yoniṃ sampīḍya dakṣiṇam |\\+}
\tl{pādaṃ prasāritaṃ kṛtvā karābhyāṃ dhārayed dṛḍham ||\\!}
\end{versinnote}
\end{testimonia}

%\begin{philcomm}
%\end{philcomm}

%%%%%%%%%%
\subsection*{3.11}
\begin{translation}[hp03_011]
[The yogi] should apply a lock to the throat and hold the breath in the upper [part of the body]. Just as a snake hit with a staff assumes the form of a staff, [\dots] 
\end{translation}

\begin{sources}[hp03_011]
\emph{Amaraugha} 20
\begin{versinnote}
\tl{kaṇṭhe bandhaṃ samāropya dhārayed vāyum ūrdhvataḥ |\\+}
\tl{yathā daṇḍāhataḥ sarpo daṇḍākāraḥ prajāyate ||\\!}
\end{versinnote}
\end{sources}

\begin{testimonia}[hp03_011]
\emph{Haṭharatnāvalī} 2.37cd–38ab
\begin{versinnote}
\tl{kaṇṭhe bandhaṃ samāropya pūrayed vāyum ūrdhvataḥ ||\\+}
\tl{yathā daṇḍāhataḥ sarpo daṇḍākāraḥ prajāyate |\\!}
\end{versinnote}

\emph{Yogacintāmaṇi} ff.~73r (attr.~to the \emph{Haṭhapradīpikā})
\begin{versinnote}
\tl{kaṇṭhe bandhaṃ samāropya dhārayed vāyum ūrdhvataḥ |\\+}
\tl{yathā daṇḍāhataḥ sarpo daṇḍākāraḥ prajāyate ||\\!}
\end{versinnote}

\emph{Yuktabhavadeva} 7.179 (attr.~to the \emph{Haṭhapradīpikā})
\begin{versinnote}
\tl{kaṇṭhe bandhaṃ samāropya dhārayed vāyum ūrdhvataḥ |\\+}
\tl{yathā daṇḍāhataḥ sarpo daṇḍākāraḥ prajāyate ||\\!}
\end{versinnote}
\end{testimonia}

\begin{philcomm}[hp03_011]
The instruction to hold the breath upwards (\emph{ūrdhvataḥ}) is somewhat vague. In a commentarial passage on this verse in \emph{Yuktabhavadeva} 7.187, Bhavadevamiśra clarifies this by saying, `one should hold it higher than the heart' (\emph{hṛdayād ūrdhvato dhārayet}).
\end{philcomm}

%%%%%%%%%%
\subsection*{3.12}
\begin{translation}[hp03_012]
[\dots] so the goddess Kuṇḍalinī suddenly becomes straight. Then she becomes still in the vessel with two halves.
\end{translation}

\begin{sources}[hp03_012]
\emph{Amaraugha} 21
\begin{versinnote}
\tl{ṛjvībhūtā tathā śaktiḥ kuṇḍalī sahasā bhavet |\\+}
\tl{tadāsau maraṇāvasthā jāyate dvipuṭāśritā ||\\!}
\end{versinnote}
\end{sources}

\begin{testimonia}[hp03_012]
\emph{Haṭharatnāvalī} 2.38cd–39ab
\begin{versinnote}
\tl{ṛjvībhūtā tathā śaktiḥ kuṇḍalī sahajā bhavet ||\\+}
\tl{tathā sā maraṇāvasthā jāyate dvipuṭīsthitā |\\!}
\end{versinnote}

\emph{Yogacintāmaṇi} ff. 72v–73r (attr.~to the \emph{Haṭhapradīpikā}) % order of the hemistichs has been changed. I don't know if or where I should mention this.
\begin{versinnote}
\tl{ṛjvībhūtā tathā śaktiḥ kuṇḍalī sahajā bhavet |\\+}
\tl{tathāsau maraṇāvasthāṃ harate dvipuṭāśrayām ||\\+}
\tl{\var{dvipuṭā ] em., dvipaṭā° L, dvipadā° N.} \\!}
\end{versinnote}

\emph{Yuktabhavadeva} 7.180 (attr.~to the \emph{Haṭhapradīpikā})
\begin{versinnote}
\tl{ṛjvībhūtā tathā śaktiḥ kuṇḍalī sahasā bhavet |\\+}
\tl{tadāsau maraṇāvasthā jāyate dvipuṭāśritā ||\\!}
\end{versinnote}
\end{testimonia}

\begin{philcomm}[hp03_012]
This verse is taken from the \emph{Amaraugha}, which uses the alchemical imagery of the \emph{Amṛtasiddhi} to describe the stilling of Kuṇḍalinī in the central channel. Drawing on Hellwig 2009:238–240, Mallinson and Szanto (2021:21) note that “In alchemical texts \emph{māraṇa} (“killing”) involves heating a substance and thereby changing its state, usually through calcination or oxidation, so that it becomes inert. In the \emph{Amṛtasiddhi māraṇa} and other derivatives of the root \emph{mṛ}, “die”, are used to denote the stilling or stopping of either the breath or Bindu.” Thus when Kuṇḍalinī is said to be in the state of \emph{maraṇa} the meaning is that she is stilled. The \emph{dvipuṭa} or “vessel with two halves” in which this occurs is the same as the \emph{Amṛtasiddhi}’s \emph{saṃpuṭa}, which, drawing on Hellwig 2009:342, Mallinson and Szanto (2021:22) say “consists of two \emph{puṭa}s joined together to form a sealed crucible for heating reagents without evaporation”. In the yoga of the \emph{Amṛtasiddhi}, the bodily \emph{saṃpuṭa} is formed by applying locks at the top and bottom of the central channel, i.e.~constricting the perineal region and the throat. In the \emph{Haṭhapradīpikā} it is formed by pressing the perineum with the heel and constricting the throat.

As Birch (2019: 971) notes, it is unlikely that later non-Buddhist authors understood \textit{maraṇāvasthā} and \emph{dvipuṭa} according to the alchemical metaphors of the \textit{Amṛtasiddhi}. Later commentators take \emph{dvipuṭa} as the two nostrils (e.g., \textit{Yuktabhavadeva} 7.187, \textit{dvināsāpuṭa}); the \textit{iḍā} and \textit{piṅgalā} channels (e.g., \textit{Jyotsnā} 3.27, \textit{puṭayor dvayam iḍāpiṅgalayor yugmam}); or the in and out flows of the breath (e.g., \textit{Yogaprakāśikā} 5.16–17, \textit{vāyor bahirnirgamanam antaḥpraveśa iti yat puṭadvayaṃ tam}). How these commentators understood \textit{asau māraṇāvasthā} is less clear. Brahmānanda seems to take it as the death of \textit{prāṇa}, or in other words, the absence of the breath, in the two nostrils (\textit{maraṇāvasthā jāyate kuṇḍalībodhe sati suṣumnāyāṃ praviṣṭe prāṇe dvayoḥ puṭayoḥ prāṇaviyogāt}).  Bhavadeva thought that Kuṇḍalinī, along with \textit{prāṇa} and \textit{apāna}, remains in the two nostrils while the breath is being held (\textit{evaṃ vāyudhāraṇāyāṃ kriyamāṇāyāṃ vyākulā bhūtā Kuṇḍalinī apānaprāṇābhyāṃ saha nāsāpuṭadvayāśritā bhavati}). Others, such as Śivānanda and Bālakṛṣṇa, favour the reading \textit{tadā sā maraṇāvasthāṃ harate dvi\-puṭāśritām} (or \textit{dvipuṭāśrayām}), which is present in group 2 and 3 manuscripts of the \textit{Haṭhapradīpikā}. Bālakṛṣṇa understands this to mean that the great seal destroys death (\textit{maraṇāvasthāṃ harate mahāmudreti bhāvaḥ}) but it could also mean that the awakened Kuṇḍalinī destroys death, which is usually dependent on the in and out breaths.

%JB: I think our translation should reflect a likely Śaiva interpretation rather than the alchemical metaphors in the Amṛtasiddhi (e.g., `Then, that state of death arises in the two nostrils'). JM: this would imply that we think that Svātmārāma decided to use the verse with this understanding, which seems unlikely to me. JB. Unlikely? All the evidence in commentaries on the HP and later compendiums, like the YCM and Yuktabhavadeva, indicates that the alchemical metaphors of the AS were not understood here. It's not even clear that the alchemical metaphors work with raising moving or raising kuṇḍalinī. In contrast to this, there's not one commentarial explanation or rewrite of the verse that reveals an alchemical meaning. We also have to bear in mind that there's no evidence that Svātmārāma and later commentators knew the AS. In the one case where there is some evidence (i.e., Śivānanda quotes it in the YCM), it doesn't seem to have helped the author understand the alchemical metaphor in this instance (as Śivānanda rewrote the verse to say that kuṇḍalinī destroys the state of death).

%The meaning of "Śaiva" here and above is unclear to me. Doesn't it work to say  "later authors"? JB I've changed the note above to non-Buddhist.
\end{philcomm}

%%%%%%%%%%
\subsection*{3.13}
\begin{translation}[hp03_013]
{[}The yogi] should then exhale very slowly, not quickly. This is the great seal revealed by the great Siddhas.
\end{translation}

%\begin{sources}[hp03_013]
%\end{sources}

\begin{testimonia}[hp03_013]
\emph{Yuktabhavadeva} 181 (attr.~to the \emph{Haṭhapradīpikā})
\begin{versinnote}
\tl{tataḥ śanaiḥ śanair eva recayen na tu vegataḥ |\\+}
\tl{iyaṃ khalu mahāmudrā mahāsiddhaiḥ pradṛśyate ||\\!}
\end{versinnote}
\end{testimonia}

%\begin{philcomm}[hp03_013]
%\end{philcomm}

%%%%%%%%%%
\subsection*{3.14}
\begin{translation}[hp03_014]
Problems such as the great afflictions [and] death and so forth are removed [by it], and that is why the most wise call it the `great seal.'
\end{translation}

\begin{sources}[hp03_014]
\emph{Amaraugha} 22
\begin{versinnote}
\tl{mahākleśādayo doṣā bhidyante maraṇādayaḥ |\\+}
\tl{mahāmudrāṃ tu tenaiva vadanti vibudhottamāḥ ||\\+}
\tl{\var{mahākleśādayo doṣā ] mahārogā mahākleśā \vl}\\+}
\tl{\var{bhidyante ] jīryante \vl}\\!}
\end{versinnote}
\end{sources}

\begin{testimonia}[hp03_014]
\emph{Yogacintāmaṇi} f.~72v (attr.~to the \emph{Skandapurāṇa})
\begin{versinnote}
\tl{mahākleśā yato doṣā jīryante maraṇādayaḥ |\\+}
\tl{mahāmudrāṃ ca tenaiva vadanti vibudhottamā iti ||\\!}
\end{versinnote}

\emph{Yuktabhavadeva} 7.182 (attr.~to the \emph{Haṭhapradīpikā})
\begin{versinnote}
\tl{mahākleśā yato doṣā jīryante maraṇādayaḥ |\\+}
\tl{mahāmudrā ca tām eva vadanti vibudhottamāḥ ||\\!}
\end{versinnote}
\end{testimonia}

\begin{philcomm}[hp03_014]
%Discuss the group 2 reading \emph{mahākleśā yato}. Also the V19 reading \emph{mahākleśā mahādoṣā}, which is similar to the \emph{Amaraughaprabodha}'s reading (\textit{mahārogā mahākleśā}). [JB: can't see a need for a note as alpha reading is close to the Amaraugha.]
%JM: note on assonance with the sounds of first line supporting the name?
This verse might be explaining the name of \emph{mahāmudrā} through assonance with \emph{mahākleśādayo, doṣā} and \emph{maraṇādayaḥ} in the first line.
\end{philcomm}

%%%%%%%%%%
\subsection*{3.15}
\begin{translation}[hp03_015]
After practising on the lunar side of the body, the yogi should then practise on the solar side. [The yogi] should finish practising the seal when the count is even.
\end{translation}

\begin{sources}[hp03_015]
\emph{Vivekamārtaṇḍa} 60
\begin{versinnote}
\tl{candrāṅge tu samabhyasya sūryāṅge punar abhyaset |\\+}
\tl{yāvat tulyā bhavet saṅkhyā tato mudrāṃ visarjayet ||\\!}
%\tl{\var{candrāṅge tu ] candrāṅgena \vl • sūryāṅge punar abhyaset ] sūryāṅgenābhyaset punaḥ \vl}\\!}
\end{versinnote}
\end{sources}

\begin{testimonia}[hp03_015]
\emph{Yuktabhavadeva} 7.183 (attr.~to the \emph{Haṭhapradīpikā})
\begin{versinnote}
\tl{candrāṅgena samabhyasya sūryāṅgenābhyaset tataḥ |\\+}
\tl{yāvat tulyā bhavet saṃkhyā tato mudrāṃ visarjayet ||\\!}
\end{versinnote}
\end{testimonia}

\begin{philcomm}[hp03_015]
The terms \emph{candrāṅga} and \emph{sūryāṅga} are unusual and not used in other yoga texts outside the context of \emph{mahāmudrā}. In \emph{Jyotsnā} 3.15, Brahmānanda glosses \emph{candrāṅga} as \emph{vāmāṅga} (`the left side of the body') and \emph{sūryāṅga} as \emph{dakṣāṅga} (`the right side of the body') and goes on to explain the sequence of practice as follows:

\begin{quote}
This is the sequence in the [practice]. Joining the heel of the bent left leg with the region of the perineum and holding the big toe of the extended right leg with bent index fingers bent is the practice, that is, the practice on the left side of the body (\emph{vāmāṅga}). In this practice, the inhaled breath remains on the left side of the body. Joining the heel of the bent right leg with the region of the perineum and holding the big toe of the extended left leg with bent index fingers is the practice, that is, the practice on the right side of the body. In this practice, the inhaled breath remains on the right side of the body.
\emph{atrāyaṃ kramaḥ} | \emph{ākuñcitavāmapādapārṣṇiṃ yonisthāne saṃyojya prasāritadakṣiṇapādāṅguṣṭham ākuñcitatarjanībhyāṃ gṛhītvābhyāso vāmāṅge 'bhyāsaḥ} | \emph{asminn abhyāse pūrito vāyur vāmāṅge tiṣṭhati} | \emph{ākuñcita\-dakṣa\-pāda\-pārṣṇiṃ yonisthāne saṃyojya prasārita\-vāma\-pādā\-ṅguṣṭham ākuñcita\-tarjanībhyāṃ gṛhītvābhyāso dakṣāṅge 'bhyāsaḥ} | \emph{asminn abhyāse pūrito vāyur dakṣāṅge tiṣṭhati} |
\end{quote}

\end{philcomm}

%%%%%%%%%%
\subsection*{3.16}
\begin{translation}[hp03_016]
{}[For the yogi who practises thus] there is no wholesome or unwholesome [food], and all flavours without exception become flavourless. Even terrible poison, when consumed, is digested like nectar.
\end{translation}

\begin{sources}[hp03_016]
\emph{Vivekamārtaṇḍa} 61
\begin{versinnote}
\tl{na hi pathyam apathyaṃ vā rasāḥ sārve 'pi nīrasāḥ |\\+}
\tl{api bhuktaṃ viṣaṃ ghoraṃ pīyūṣam iva jīryati ||\\+}
\tl{\var{jīryati] jīryate \vl, jāyate \vl}\\!}
\end{versinnote}
\end{sources}

\begin{testimonia}[hp03_016]
\emph{Haṭharatnāvalī} 2.40
\begin{versinnote}
\tl{na hi pathyam apathyaṃ vā rasāḥ sarve 'pi nīrasāḥ |\\+}
\tl{api bhuktaṃ viṣaṃ ghoraṃ pīyūṣam iva jīryate ||\\!}
\end{versinnote}
 
\emph{Yogacintāmaṇi} f.~73r (attr.~to the \emph{Haṭhapradīpikā})
\begin{versinnote}
\tl{na hi pathyam apathyaṃ vā rasāḥ sarve 'pi nīrasāḥ |\\+}
\tl{api bhuktaṃ viṣaṃ ghoraṃ pīyūṣam iva jīryati ||\\!}
\end{versinnote}

\emph{Yuktabhavadeva} 7.184 (attr.~to the \emph{Haṭhapradīpikā})
\begin{versinnote}
\tl{iha pathyam apathyaṃ vā sarasaṃ nirasaṃ ca vā |\\+}
\tl{api bhuktaṃ viṣaṃ ghoraṃ pīyūṣam iva jīryati ||\\!}
\end{versinnote}
\end{testimonia}

%\begin{philcomm}[hp03_016]
%In \emph{Jyotsnā} 3.16, Brahmānanda understands the second verse quarter to mean all flavours –pungent (\emph{kaṭu}), acid (\emph{amlā}), etc.– and stale (\emph{nīrasa}) [food] and takes it with the verb \emph{jīryati}.
%[JB: This note doesnt seem relevant to our translation]
%\end{philcomm}

%%%%%%%%%%
\subsection*{3.17}
\begin{translation}[hp03_017]
Diseases such as consumption, skin afflictions, constipation, swelling and indigestion disappear for [the yogi] who practises the great seal.
\end{translation}

\begin{sources}[hp03_017]
\emph{Vivekamārtaṇḍa} 62
\begin{versinnote}
\tl{kṣayakuṣṭagudāvarttagulmājīrṇajvaravyathāḥ |\\+}
\tl{tasya doṣāḥ kṣayaṃ yānti mahāmudrāṃ tu yo 'bhyaset ||\\+}
\tl{\var{tasya doṣāḥ ] sarvarogāḥ \vl, rogās tasya \vl}\\!}
\end{versinnote}
\end{sources}

\begin{testimonia}[hp03_017]
\emph{Haṭharatnāvalī} 2.41
\begin{versinnote}
\tl{kṣayakuṣṭhagudāvartagulmājīrṇapurogamāḥ |\\+}
\tl{doṣāḥ sarve kṣayaṃ yānti mahāmudrāṃ tu yo 'bhyaset ||\\!}
\end{versinnote}

\emph{Yogacintāmaṇi} f.~73r (attr.~to the \emph{Haṭhapradīpikā})
\begin{versinnote}
\tl{kṣayakuṣṭhagudāvartagulmaplīhapurogamāḥ |\\+}
\tl{tasya doṣāḥ kṣayaṃ yānti mahāmudrāṃ ca yo 'bhyaset || \\!}
\end{versinnote}

\emph{Yuktabhavadeva} 7.185 (attr.~to the \emph{Haṭhapradīpikā})
\begin{versinnote}
\tl{kṣayakuṣṭhagudāvarttagulmājīrṇapurogamāḥ |\\+}
\tl{tasya rogāḥ kṣayaṃ yānti mahāmudrāṃ tu yo'bhyaset ||\\!}
\end{versinnote}
\end{testimonia}

%\begin{philcomm}[hp03_017]
%\end{philcomm}

%%%%%%%%%%
\subsection*{3.18}
\begin{translation}[hp03_018]
This great seal has been taught. It brings about the great \emph{siddhi} for men: it should be carefully kept secret and should not be given to all and sundry.
\end{translation}

\begin{sources}[hp03_018]
\emph{Vivekamārtaṇḍa} 63
\begin{versinnote}
\tl{kathiteyaṃ mahāmudrā mahāsiddhikarī nṛṇām |\\+}
\tl{gopanīyā prayatnena na deyā yasya kasya cit ||\\!}
\end{versinnote}
\end{sources}

\begin{testimonia}[hp03_018]
\emph{Haṭharatnāvalī} 2.42
\begin{versinnote}
\tl{kathiteyaṃ mahāmudrā jarāmṛtyuvināśinī |\\+}
\tl{gopanīyā prayatnena na deyā yasya kasya cit ||\\!}
\end{versinnote}

\emph{Yuktabhavadeva} 7.186 (attr.~to the \emph{Haṭhapradīpikā})
\begin{versinnote}
\tl{kathiteyaṃ mahāmudrā mahāsiddhikarī nṛṇām |\\+}
\tl{gopanīyā prayatnena na deyā yasya kasya cit ||\\!}
\end{versinnote}
\end{testimonia}

\begin{philcomm}[hp03_018]
Manuscripts of three groups (\textalpha, \textbeta\ and \textgamma), which are important stemmatically, have the reading \emph{jarāmṛtyuvināśinī} (and \getsiglum{J5} has \emph{nṛṇāṃ mṛtyuvināśinī}) for the second verse quarter. While this reading is possible, the play on \emph{mahāsiddhi} and \emph{mahāmudrā} seems more likely original, as seen in the source text, the \emph{Vivekamārtaṇḍa} (without significant variants), and the \texteta\ and \textepsilon\ groups. 

In \emph{Jyotsnā} 3.18, Brahmānanda understands \emph{mahāsiddhi} as referring to `great \emph{siddhi}s,' but in other works it can mean liberation (Mallinson 2012).
\end{philcomm}

%%%%%%%%%%
\subsection*{3.19 heading}
\begin{translation}[hp03_019a]
Now, the great lock (\emph{mahābandha}):
\end{translation}

% \begin{philcomm}[hp03_019a]
% \end{philcomm}

%%%%%%%%%%
\subsection*{3.19}
\begin{translation}[hp03_019]
{}[The yogi] should place the heel of the left foot on the perineal region. And he should put the right foot on the left thigh, [\dots]
\end{translation}

\begin{sources}[hp03_019]
19ab = \emph{Dattātreyayogaśāstra} 132cd (in the section on \emph{mahāmudrā})
\begin{versinnote}
\tl{pārṣṇiṃ vāmasya pādasya yonisthāne niyojayet ||\\!}
\end{versinnote}

19cd = Cf.~\emph{Vivekamārtaṇḍa} 8a (not \emph{anuṣṭubh})
\begin{versinnote}
\tl{vāmorūpari dakṣiṇañ ca caraṇaṃ saṃsthāpya\\!}
\end{versinnote}
\end{sources}

\begin{testimonia}[hp03_019]
\emph{Haṭharatnāvalī} 2.43
\begin{versinnote}
\tl{pārṣṇiṃ vāmasya pādasya yonisthāne niyojayet |\\+}
\tl{vāmorūpari saṃsthāpya dakṣiṇaṃ caraṇaṃ tathā ||\\!}
\end{versinnote}

\emph{Yogalakṣanāvalī} f.~31v
\begin{versinnote}
\tl{haṭhapradīpikāyām—\\+}
\tl{vāmāṅghripārṣṇibhāgena yonisthānaṃ nipīḍayet |\\+}
\tl{vāmorupari saṃsthāpya dakṣiṇaṃ caraṇaṃ tathā ||\\!}
\end{versinnote}

\emph{Yogacintāmaṇi} f.~73r
\begin{versinnote}
\tl{haṭhapradīpikāyām—\\+}
\tl{pārṣṇiṃ vāmasya pādasya yonisthāne niyojayet |\\+}
\tl{vāmorūpari saṃsthāpya dakṣiṇaṃ caraṇaṃ tathā ||\\!}
\end{versinnote}

\emph{Yuktabhavadeva} 7.190 (attr.~to the \emph{Haṭhapradīpikā})
\begin{versinnote}
\tl{pārṣṇivāmasya pādasya yonisthāne niyojayet |\\+}
\tl{vāmorupari saṃsthāpya dakṣiṇaṃ caraṇaṃ tathā ||\\!}
\end{versinnote}
\end{testimonia}

\begin{philcomm}[hp03_019]
The seated position for \emph{mahābandha} described in this verse is not in the \emph{Amṛtasiddhi} (chapter 12) or \emph{Amaraugha} (25cd–27). The \emph{Amṛtasiddhi} instructs the same position for \emph{mahāmudrā} and \emph{mahābandha}, and the \emph{Amaraugha} does not comment on the posture of \emph{mahābandha}, implying that its posture is the same as \emph{mahāmudrā}.
\end{philcomm}

%%%%%%%%%%
\subsection*{3.20}
\begin{translation}[hp03_020]
[\dots] inhale through the mouth, firmly put the chin on the chest, contract the perineum and fix the mind in the centre.
\end{translation}

\begin{sources}[hp03_020]
\emph{Amaraugha} 24
\begin{versinnote}
\tl{pūrayitvā mukhe vāyuṃ cibukaṃ hṛdaye dṛḍham |\\+}
\tl{nibhṛtya yonim ākuñcya mano madhye niyojayet ||\\+}
\tl{\var{nibhṛtya ] nibhṛtaṃ \vl}\\!}
\end{versinnote}
\end{sources}

\begin{testimonia}[hp03_020]
\emph{Haṭharatnāvalī} 2.44
\begin{versinnote}
\tl{pūrayen mukhato vāyuṃ hṛdaye cibukaṃ dṛḍham |\\+}
\tl{nibhṛtya yonim ākuñcya mano madhye niyojayet ||\\+}
\tl{\var{nibhṛtya ] niṣpīḍya \vl}\\!}
\end{versinnote}

\emph{Yogalakṣanāvalī} f.~31v (attr.~to the \emph{Haṭhapradīpikā})
\begin{versinnote}
\tl{pūrayitvā tato vāyuṃ hṛdaye cibukaṃ dṛḍham |\\+}
\tl{niḥpīḍya yonim ākuṃcya mano madhye niyojayet ||\\!}
\end{versinnote}

\emph{Yogacintāmaṇi} f.~73v (attr.~to the \emph{Haṭhapradīpikā})
\begin{versinnote}
\tl{pūrayitvā mukhe vāyuṃ hṛdaye cibukaṃ tathā |\\+}
\tl{niḥpīḍya yonim ākuñcya mano madhye niyojayet || \\!}
\end{versinnote}

\emph{Yuktabhavadeva} 7.190 (attr.~to the \emph{Haṭhapradīpikā})
\begin{versinnote}
\tl{pūrayitvā tato vāyuṃ cibukaṃ dṛḍham |\\+}
\tl{niḥkṣipya yonim ākuñcya mano madhye niyojayet ||\\!}
\end{versinnote}
\end{testimonia}

\begin{philcomm}[hp03_020]
The referent of \emph{madhye} is uncertain. The verse is derived from the \emph{Amaraugha}, and the \emph{Amṛtasiddhi} makes no mention of a place to focus the mind in its treatment of \emph{mahābandha} (it does however instruct the yogi to place the mind at the \emph{catuṣpatha} in its teachings on \emph{mahāmudrā}). Bhavadevamiśra (7.196), Brahmānanda (3.20) and Bālakṛṣṇa (5.24) take it to mean the central channel. It could also plausibly mean the region between the chest and perineum, or perhaps the place between the eyebrows. At 3.24 this practice is said to make the mind reach Kedāra, which is sometimes located between the eyebrows (see Mallinson 2007:214 n.285, Birch 2019: 967 n.57).
\end{philcomm}

%%%%%%%%%%
\subsection*{3.21}
\begin{translation}[hp03_021]
[The yogi] should hold the breath as long as possible and exhale slowly. And having practised it on the left side, he should practise it on the right side. 
\end{translation}

\begin{sources}[hp03_021]
\emph{Dattātreyayogaśāstra} 62cd (\emph{padmāsana}), 134cd (\emph{mahāmudrā})
\begin{versinnote}
\tl{dhārayitvā yathāśakti recayed iḍayā śanaiḥ ||\\+}
\tl{...\\+}
\tl{vāmāṅgena samabhyasya dakṣiṇāṅgena cābhyaset ||\\!}
\end{versinnote}
\end{sources}

\begin{testimonia}[hp03_021]
\emph{Haṭharatnāvalī} 2.44cd–2.45
\begin{versinnote}
\tl{recayec ca śanair evaṃ mahābandho 'yam ucyate ||\\+}
\tl{ayaṃ yogo mahābandhas sarvasiddhipradāyakaḥ |\\+}
\tl{savyāṅge ca samabhyasya dakṣiṇāṅge samabhyaset ||\\!}
\end{versinnote}

\emph{Yuktabhavadeva} 7.192ab (attr.~to the \emph{Haṭhapradīpikā})
\begin{versinnote}
\tl{dhārayitvā yathāśakti recayed anilaṃ sudhīḥ |\\!}
\end{versinnote}
\end{testimonia}

\begin{philcomm}[hp03_021]
This section in the description of \emph{mahābandha} has been subject to various revisions in the manuscript groups. The shortest version occurs in \alphaOne\ and \alphaTwo, as well as the \textbeta\ group and the \emph{Haṭharatnāvalī}, which omit verses 3.21 and 3.22. This omission results in the somewhat odd juxtaposition of two lines stating that this is \emph{mahābandha}. 

Verse 3.21 is in the \textgamma\ group and some manuscripts of the \textbeta\ and \texteta\ groups. It is a composite of hemistichs from the \emph{Dattātreyayogaśāstra} (62cd, 134cd). The first is taken from a description of \emph{padmāsana}, and may have been included by Svātmārāma in his description of \emph{mahābandha} to make it clear that the breath is held for as long as possible after the inhalation. The second hemistich is from a passage on \emph{mahāmudrā}, and it echoes a similar statement on \emph{mahāmudrā} in \emph{Haṭhapradīpikā} 3.15. It also occurs in the \textalpha\ group after 3.22.
\end{philcomm}

%%%%%%%%%%
%\subsection*{3.21*1}
%\begin{translation}[hp03_021*1]
%After holding the breath for as long as possible, [the yogi] should exhale gradually. And after practising on the left side, he should practise on the right.
%\end{translation}
%\begin{testimonia}[hp03_021*1]
%\end{testimonia}

%\begin{philcomm}[hp03_021*1]
%\end{philcomm}

%%%%%%%%%%
\subsection*{3.22}
\begin{translation}[hp03_022]
With regard to this [practice] some are of the opinion that [the yogi] should leave out the throat lock, saying that he should lift up the hollow at the uvula with the tongue instead.
\end{translation}

\begin{sources}[hp03_022]
Cf.~\emph{Vivekamārtaṇḍa} 126ab
\begin{versinnote}
\tl{saṃpīḍya rasanāgreṇa rājadantabilaṃ mahat |\\!}
\end{versinnote}
Cf.~\emph{Dattātreyayogaśāstra} 36
\begin{versinnote}
\tl{nāsāgre vinyased rājadantamūlaṃ ca jihvayā |\\+}
\tl{uttabhya cibukaṃ vakṣasy āsthāpya pavanaṃ śanaiḥ ||\\!}
\end{versinnote}
\end{sources}

\begin{testimonia}[hp03_022]
\emph{Yogacintāmaṇi} f.~73r (attr.~to Īśvara)% not in the \emph{Yogacintāmaṇi} spreadsheet, but in Muktabodha etext (in a passage attributed to īśvara)
\begin{versinnote}
\tl{matam atra tu keṣāṃ cit kaṇṭhabandhaṃ vivarjayet |\\+}
\tl{rājadantadvayaṃ tatra jihvayottambhayed iti ||\\!}
\end{versinnote}

\emph{Yuktabhavadeva} 7.192cd (attr.~to the \emph{Haṭhapradīpikā})
\begin{versinnote}
\tl{rājadantadvayaṃ tatra jihvayonnamayed iti ||\\!}
\end{versinnote}
\end{testimonia}

\begin{philcomm}[hp03_022]
This verse expresses an alternative to the application of the chin lock in \emph{mahā\-bandha} (mentioned in 3.20). It is found in all groups but is notably absent in \alphaOne\ and \alphaTwo, and the \textbeta\ group. At present we cannot be certain if this verse was written by Svātmārāma because it does not reflect textual teachings on \emph{mahābandha} in so far as no other text known to us advocates the use of the tongue rather than the chin lock in \emph{mahābandha}. If the verse was composed by Svātmārāma, it was omitted early in the transmission by someone who did not agree with the alternative teaching.

This verse occurs in the \emph{Yogacintāmaṇi} in the middle of a quotation attributed to Īśvara. The other verses of the quoted passage are found in the \emph{Śivasaṃhitā} (4.37–42), but the verse in question is not reported in the critical edition of the \emph{Śivasaṃhitā} (2018). The verse is absent in another passage on \emph{mahābandha} that the author of the \emph{Yogacintāmaṇi} cites and attributes to the \emph{Haṭhapradīpikā}.

We do not find the idea of lifting up the \emph{rājadantabila} with the tongue in other works, but \emph{Vivekamārtaṇḍa} 126ab instructs the yogi to press it with the tip of the tongue and \emph{Dattātreyayogaśāstra} 36 (found at \emph{Haṭhapradīpikā} 1.46) instructs the yogi in \emph{padmāsana} to lift up the `root of the uvula' (\emph{rājadantamūla}) with the tongue.
\end{philcomm}

%%%%%%%%%%
\subsection*{3.23}
\begin{translation}[hp03_023]
This is truly the great lock: it bestows the great \emph{siddhi} [and] is adept at loosening the great bond (\emph{°mahābandha°}) that is the noose of time.
\end{translation}

%\begin{sources}[hp03_023]
%\end{sources}

\begin{testimonia}[hp03_023]
\emph{Yogacintāmaṇi} f.~73v (attr.~to the \emph{Haṭhapradīpikā})
\begin{versinnote}
\tl{ayaṃ khalu mahābandho mahāsiddhipradāyakaḥ |\\+}
\tl{kālapāśamahābandhavimocanavicakṣaṇaḥ ||\\!}
\end{versinnote}

\emph{Yuktabhavadeva} 7.193 (attr.~to the \emph{Haṭhapradīpikā})
\begin{versinnote}
\tl{ayaṃ khalu mahābandho mahāsiddhipradāyakaḥ |\\+}
\tl{kālapāśamahābandhavimocanavicakṣaṇaḥ ||\\!}
\end{versinnote}
\end{testimonia}

%\begin{philcomm}[hp03_023]
%J5 reading for this passage is different from alpha1 so should be reported.
%MD: Read ayaṃ yogo mahābandho instead of ayaṃ khalu ...? Alpha has yoga (N3, acc.)/yogī (J5)/yogo (G4).
%JM: Alpha yogamahābandhaṃ works. Adopt alpha one alternative verse?
%(this was above) Adopt yogamahābandhaṃ in 3.23a (and report alpha readings). JM: khalu seems ok though. elsewhere it introduces a nirukti like this, and yoga is odd. I say go with khalu.
%\end{philcomm}

%%%%%%%%%%
\subsection*{3.24}
\begin{translation}[hp03_024]
And this [lock] initiates an upward flow in all the channels. It brings about a confluence at the Triveṇī [and] causes the mind to reach Kedāra.
\end{translation}

\begin{sources}[hp03_024]
\emph{Amaraugha} 25
\begin{versinnote}
\tl{ayaṃ ca sarvanāḍīnām ūrdhvaṃgativibodhakaḥ |\\+}
\tl{triveṇīsaṅgame dhatte kedāraṃ prāpayen manaḥ ||\\!}
\end{versinnote}
\end{sources}

\begin{testimonia}[hp03_024]
\emph{Haṭharatnāvalī} 2.46
\begin{versinnote}
\tl{ayaṃ ca sarvanāḍīnām ūrdhvagativibodhakaḥ |\\+}
\tl{triveṇīsaṅgamaṃ dhatte kedāraṃ prāpayen manaḥ ||\\!}
\end{versinnote}

\emph{Yogacintāmaṇi} f.~73v (attr.~to the \emph{Haṭhapradīpikā})
\begin{versinnote}
\tl{ayaṃ tu sarvanāḍīnām ūrdhvaṃgamanarodhakaḥ |\\+}
\tl{triveṇīsaṃgamaṃ dhatte kedāraṃ prāpayen manaḥ ||\\!}
\end{versinnote}

\emph{Yuktabhavadeva} 7.94 (attr.~to the \emph{Haṭhapradīpikā})
\begin{versinnote}
\tl{ayaṃ hi sarvanāḍīnām ūrdhvaṃ gativibodhakaḥ |\\+}
\tl{triveṇīsaṃgamaṃ dhatte kedāraṃ prāpayen manaḥ ||\\!}
\end{versinnote}
\end{testimonia}

\begin{philcomm}[hp03_024]
The reading \emph{ūrdhvaṃgatibodhakaḥ} (`initiates an upward flow') is found in the source (\emph{Amaraugha} 25) and all \emph{Haṭhapradīpikā} witnesses except the \emph{Jyotsnā} (where the line is found earlier). While the \emph{Amaraughaprabodha} has the reading \emph{ūrdhvaṃgativiśodhanaḥ} (`purification of the upward flow'), the \emph{Amṛtasiddhi} (12.14) states that the chin-lock prevents the upward flow (\emph{ūrdhvaṃgatinirodhakaḥ}) in all the channels. The \textit{Amaraugha} is referring to the idea (likely accepted by Svātmārāma) that the root lock creates an upward flow in all the channels that prevents the elements and essences of the body from escaping (cf.~\emph{Amṛtasiddhi} 12.8–10).

Triveṇī and Kedāra are pilgrimage sites, the former at Prayāga where the Gaṅgā and Yamunā meet, the latter in the Himālaya, near the source of the Gaṅgā. The bodily \emph{triveṇī} is located in the navel or heart by earlier Śaiva works (Birch 2019: 967). Here it may be the same as the \emph{trikūṭa} and located between the eyebrows (Mallinson 2007: 209 n.259). Brahmānanda, who does not identify a location for Triveṇī, understands Kedāra to be between the eyebrows. In the \emph{Khecarīvidyā} it is located on the back of the head above the nape of the neck (Mallinson 2007:214 n.285). For other references on the location of Kedāra, see Birch 2019:967 n.57.
\end{philcomm}

%%%%%%%%%%
\subsection*{3.25}
\begin{translation}[hp03_025]
Like a beautiful and charming woman without a man, the great seal and the great lock are barren without the great piercing.\\
\end{translation}

\begin{sources}[hp03_025]
\emph{Amaraugha} 26
\begin{versinnote}
\tl{rūpalāvaṇyasampannā yathā strī puruṣaṃ vinā |\\+}
\tl{mahāmudrāmahābandhau niṣphalau vedhavarjitau  ||\\+}
\tl{\var{°bandhau niṣphalau vedhavarjitau ] °bandho niṣphalo vedhavarjitaḥ \vl}\\!}
\end{versinnote}

Cf.~\emph{Śivasaṃhitā} 4.47
\begin{versinnote}
\tl{mahāmudrāmahābandhau niṣphalau vedhavarjitau |\\+}
\tl{tasmād yogī prayatnena karoti tritayaṃ kramāt ||\\!}
\end{versinnote}

Cf.~\emph{Amṛtasiddhi} 13.3
\begin{versinnote}
\tl{guṇarūpavatī nārī niṣphalā puruṣaṃ vinā |\\+}
\tl{mahāmudrāmahābandhau vinā vedhena niṣphalau ||\\!}
\end{versinnote}
\end{sources}

\begin{testimonia}[hp03_025]
\emph{Haṭharatnāvalī} 2.47
\begin{versinnote}
\tl{rūpalāvaṇyasampannā yathā strī puruṣaṃ vinā |\\+}
\tl{mahāmudrāmahābandhau niṣphalau vedhavarjitau ||\\+}
\tl{\var{°mahābandhau ] mahābandho \vl}\\!}
\end{versinnote}

\emph{Yogacintāmaṇi} f.~73v (attr.~to the \emph{Haṭhapradīpikā})
\begin{versinnote}
\tl{atha mahāvedhaḥ—\\+}
\tl{rūpalāvaṇyasaṃpannā yathā strī puruṣaṃ vinā |\\+}
\tl{mahāmudrāmahābandhau niṣphalau vedhavarjitau ||\\!}
\end{versinnote}

\emph{Yuktabhavadeva} 7.197 (attr.~to the \emph{Haṭhapradīpikā})
\begin{versinnote}
\tl{atha mahāvedhaḥ -\\+}
\tl{mahāmudrāmahābandhau niṣphalau vedavarjitau |\\+}
\tl{rūpalāvaṇyasampannā yathā strī puruṣaṃ vinā ||\\!}
\end{versinnote}
\end{testimonia}

\begin{philcomm}[hp03_025]
This verse, which is from the \emph{Amaraugha} and similar to verses in the \emph{Amṛtasiddhi} and \emph{Śivasaṃhitā}, is stating that the great seal, lock and piercing should be practised together. This can be done as a sequence as shown in \medialink{this video}{the_three_mudrās.mp4} (© Mark Robberds).
\end{philcomm}

%%%%%%%%%%
\subsection*{3.26 heading}
\begin{translation}[hp03_026a]
Now the great piercing (\emph{mahāvedha}):
\end{translation}

% \begin{philcomm}[hp03_026a]
% \end{philcomm}

%%%%%%%%%%
\subsection*{3.26}
\begin{translation}[hp03_026]
While in the great lock, the yogi should inhale, focus his mind and firmly block the flow of the bodily winds by means of the throat seal.
\end{translation}

\begin{sources}[hp03_026]
\emph{Amaraugha} 27
\begin{versinnote}
\tl{punar āsphālayet kaṭyāṃ susthiraṃ kaṇṭhamudrayā |\\+}
\tl{vāyūnāṃ gatim ārudhya kṛtvā pūrakakumbhakau ||\\+}
\tl{\var{ārudhya ] āvṛtya, āśritya \vl}\\!}
\end{versinnote}

Cf.~\emph{Śivasaṃhitā}  4.43
\begin{versinnote}
\tl{mahābandhasthito yogī kukṣim āpurya vāyunā |\\+}
\tl{sphicau saṃtāpayed dhīmān vedho 'yaṃ kīrtito mayā ||\\!}
\end{versinnote}
\end{sources}

\begin{testimonia}[hp03_026]
\emph{Haṭharatnāvalī} 2.48
\begin{versinnote}
\tl{mahābandhasthito yogī kṛtvā pūrakam ekadhīḥ |\\+}
\tl{vāyūnāṃ gatim ākṛṣya nibhṛtaṃ kaṇṭhamudrayā ||\\+}
\tl{\var{ekadhīḥ ] ekadhā \vl}\\!}
\end{versinnote}

\emph{Yogacintāmaṇi} f.~73v (attr.~to the \emph{Haṭhapradīpikā})
\begin{versinnote}
\tl{mahābandhe sthito yogī kṛtvā pūrakam ekadhā |\\+}
\tl{vāyūnāṃ gatim āvṛtya nibhṛtaṃ kaṇṭhamudrayā ||\\!}
\end{versinnote}

\emph{Yuktabhavadeva} 7.198 (attr.~to the \emph{Haṭhapradīpikā})
\begin{versinnote}
\tl{mahābandhasthito yogī kṛtvā pūrakam ekadhā |\\+}
\tl{vāyunāṃ gatim āvṛtya nibhṛtaṃ kaṇṭhamudrayā ||\\!}
\end{versinnote}
\end{testimonia}

\begin{philcomm}[hp03_026]
We are not certain of the meaning of \emph{nibhṛtam}, which is found in all of the collated witnesses and important testimonia. We have understood it as an adverb meaning `firmly.'
\end{philcomm}

%%%%%%%%%%
\subsection*{3.27}
\begin{translation}[hp03_027]
With hands even on the ground, he should gently tap the buttocks [on the ground]. The breath enters the vessel of two halves and quickly flashes forth.
\end{translation}

\begin{sources}[hp03_027]
\emph{Amaraugha} 28ab, 29cd
\begin{versinnote}
\tl{samahastayugo bhūmau samapādayugas tathā |\\+}
\tl{[vedhayet kramayogena catuṣpīṭhaṃ tu vāyunā ||\\+}
\tl{āsphālayen mahāmeruṃ vāyuvajrāgnikoṭibhiḥ | ]\\+}
\tl{puṭadvayaṃ samākṛṣya vāyuḥ sphurati satvaram ||\\+}
\tl{\var{samākṛṣya ] samākramya \vl }\\!}
\end{versinnote}

\emph{Dattātreyayogaśāstra} 136ab
\begin{versinnote}
\tl{mahābandhasthito bhūmau sphicau santāḍayec chanaiḥ |\\!}
\end{versinnote}
\end{sources}

\begin{testimonia}[hp03_027]
\emph{Haṭharatnāvalī} 2.51
\begin{versinnote}
\tl{samahastayugo bhūmau sphicau saṃtāḍayec chanaiḥ |\\+}
\tl{ayam eva mahāvedhaḥ siddhido 'bhyāsato bhavet \\!}
\end{versinnote}

\emph{Yogacintāmaṇi} f.~73v (attr.~to the \emph{Haṭhapradīpikā})
\begin{versinnote}
\tl{samahastayugo bhūmau sphijau saṃtāḍayec chanaiḥ |\\+}
\tl{puṭadvayaṃ samākramya vāyuḥ sphurati madhyagaḥ ||\\!}
\end{versinnote}

\emph{Yuktabhavedeva} 7.199 (attr.~to the \emph{Haṭhapradīpikā})
\begin{versinnote}
\tl{nyastahastayugo bhūmau sphicau santāḍayec chanaiḥ |\\+}
\tl{jaṃghādvayaṃ samākṛṣya vāyuḥ spharati satvaram ||\\!}
\end{versinnote}
\end{testimonia}

\begin{philcomm}[hp03_027]
The term \emph{puṭadvaya} is referring back to the \emph{dvipuṭa} mentioned in verse 3.12. On the alchemical meaning and other interpretations by later commentators, see the note to 3.12.

The greyscale hemistich added to this verse in the \textgamma\ group is taken from the \emph{Śivasaṃhitā} (4.42ab) in the section on \emph{mahābandha}.
\end{philcomm}

%%%%%%%%%%
\subsection*{3.28}
\begin{translation}[hp03_028]
And the union of the moon, sun and fire arises, which leads to immortality.
The state of a dead person has arisen, so where is the fear of death?
\end{translation}

\begin{sources}[hp03_028]
\emph{Amaraugha} 30
\begin{versinnote}
\tl{somasūryāgnisaṃbandhaṃ jānīyād amṛtāya vai |\\+}
\tl{mṛtāvasthā samutpannā tato mṛtyubhayaṃ kutaḥ ||\\+}
\tl{\var{°saṃbandhaṃ ] °saṃbandhāj, °saṃbandhā \vl}\\!}
\end{versinnote}
\end{sources}

\begin{testimonia}[hp03_028]
\emph{Yogacintāmaṇi} f.~73v (attr.~to the \emph{Haṭhapradīpikā})
\begin{versinnote}
\tl{somasūryāgnisandhānaṃ jāyate cāmṛtāyate |\\+}
\tl{mṛtāvasthāsamutpannaṃ tato mṛtyubhayaṃ kutaḥ ||\\+}
\tl{\var{°samutpannaṃ ] °samutpannā \vl}\\!}
\end{versinnote}

\emph{Yuktabhavadeva} 7.200 (attr.~to the \emph{Haṭhapradīpikā})
\begin{versinnote}
\tl{somasūryāgnisambandho jāyate cāmṛtāya ca |\\+}
\tl{samutpannā mṛtāvasthā tato vāyuṃ virecayet ||\\!}
\end{versinnote}
\end{testimonia}

\begin{philcomm}[hp03_028]
The \textalpha\ and \textgamma\ groups have °\emph{saṃbandhāj}, which is possible but somewhat awkward (i.e., `because of the union of the moon, sun and fire, the state of a dead person, which has arisen, leads to immortality'). The adopted reading °\emph{saṃbandho}, which is supported by \epsilonTwo, \emph{Yuktabhavadeva} 7.200 and \emph{Jyotsna} 3.28, makes better sense but its meaning is not as clear as the formulation in the source text (i.e., \emph{Amaraugha} 30).

The compound \emph{mṛtāvasthā} (`the state of death') likely refers to a dead person in the sense that the yogi appears as if dead when the moon, sun and fire have united. In the version found in \emph{Jyotsnā} 3.28 and \etaOne, \etaTwo\ and \epsilonThree, the final verse quarter has been rewritten to say that the yogi then exhales the breath (\emph{tato vāyuṃ virecayet}). This implies that \emph{mṛtāvasthā} is a breath retention (\emph{kumhaka}), which is apparent in Brahmānanda's explanation:
\begin{quote}
The state of death that has arisen is the state of one who has died, [that is,] of one who is devoid of the breath because of the absence of movement of \emph{prāṇa} in the \emph{iḍā} and \emph{piṅgalā} channels. Then, immediately after that [state], [the yogi] exhales the breath, [that is,] he gradually releases it through the nostrils.\\
\emph{mṛtasya prāṇaviyuktasyāvasthā mṛtāvasthā samutpannā bhavati, iḍāpiṅgalayoḥ prāṇasañcārābhāvāt} | \emph{tatas tadanantaraṃ vāyuṃ virecayen nāsikāpuṭābhyāṃ śanais tyajet} |
\end{quote}

\end{philcomm}

%%%%%%%%%%
\subsection*{3.29}
\begin{translation}[hp03_029]
Through practice, this great piercing bestows the great \emph{siddhi} [and] cures wrinkles, grey hair and trembling. It is used by the best practitioners.
\end{translation}

%\begin{sources}[hp03_029]
%\end{sources}

\begin{testimonia}[hp03_029]
\emph{Haṭharatnāvalī} 2.51cd
\begin{versinnote}
\tl{ayam eva mahāvedhaḥ siddhido 'bhyāsato bhavet ||\\!}
\end{versinnote}

\emph{Yogacintāmaṇi} f.~73v (attr.~to the \emph{Haṭhapradīpikā})
\begin{versinnote}
\tl{mahāvedho 'yam abhyāsān mahāsiddhipradāyakaḥ |\\+}
\tl{valīpalitavepaghnaḥ sevyate sādhakottamaiḥ ||\\!}
\end{versinnote}

\emph{Yuktabhavadeva} 7.201 (attr.~to the \emph{Haṭhapradīpikā})
\begin{versinnote}
\tl{mahāvedho 'yam abhyasto mahāsiddhipradāyakaḥ |\\+}
\tl{valīpalitavepaghnaḥ sevyate sādhakottamaiḥ ||\\!}
\end{versinnote}

\end{testimonia}

\begin{philcomm}[hp03_029]
In the third verse quarter, the reading \emph{°vedhaghnaḥ} (`cures wounds' ?) in \textalpha\ and the other groups, the exception being \textgamma\ (\emph{°vegaghnaḥ}), is strange as it does not seem related to the other two symptoms of old age (i.e., wrinkles and grey hair) that this \emph{mudrā} can cure. It is perhaps a misreading of \emph{°vepaghnaḥ}, which occurs in some manuscripts in the \textbeta\ (\getsiglum{J1}, \getsiglum{N12}), \textepsilon\ (\getsiglum{J14}) and \textdelta\ (\getsiglum{J3}, \getsiglum{N16}, \getsiglum{N18}) groups, as well as the \emph{Yogacintāmaṇi} and \emph{Jyotsnā}. The meaning of \emph{vepa}, `trembling,' is consistent with wrinkles and grey hair.

%The collocation \emph{valīpalitaroga} that we have adopted, found in \getsiglum{C8}, is quite common (e.g. \emph{Śāradātilaka} \emph{6.104, Ānandakanda} 1.7.13, .1.7.31).
%JB: I think we should adopt vepa on the strength of the marmasthāna entry (\item[valīpalitavepaghnaḥ] Ba5, J1, J3, J6, J14, N4a, N11, N12, N16, N18, ), and the fact that it's closer to vedha (alpha reading) than roga (which looks like an outlier in the marmasthāna data). 
% J1=6b, J3,J6,N16,N18=6a, J14=4c, N11 =Gr2, N12 =4b,
% MD: These mss are not very strong. Is it not possible to interpret °vedhaghnaḥ as "destroys by piercing"? This verse only has no source in the middle of the long citation from Amaraugha, is probably authorial. Svātmārāma might have wanted to use the word vedha of mahāvedha, similar to dhauti/dhāvanti.
\end{philcomm}

%%%%%%%%%%
\subsection*{3.30}
\begin{translation}[hp03_030]
This triad is a great secret that destroys old and death, increases [the body's] fire and bestows the powers beginning with minimisation.
\end{translation}

\begin{sources}[hp03_030]
\emph{Amaraugha} 31
\begin{versinnote}
\tl{etat trayaṃ mahāguhyaṃ jarāmṛtyuvināśanam |\\+}
\tl{vahnivṛddhikarañ caiva aṇimādiguṇapradam ||\\+}
\tl{\var{°guhyaṃ ] °guṇyaṃ \vl}\\!}
\end{versinnote}
\end{sources}

\begin{testimonia}[hp03_030]
\emph{Haṭharatnāvalī} 2.52
\begin{versinnote}
\tl{etat trayaṃ mahāguhyaṃ jarāmṛtyuvināśanam |\\+}
\tl{vahnivṛddhikaraṃ caiva hy aṇimādiguṇapradam ||\\!}
\end{versinnote}

\emph{Yogacintāmaṇi} f.~73v (attr.~to the \emph{Haṭhapradīpikā})
\begin{versinnote}
\tl{etat trayaṃ mahāguptaṃ jalāmṛtyuvināśanam |\\+}
\tl{vahnivṛddhikaraṃ caiva aṇimādiguṇapradam ||\\!}
\end{versinnote}

\emph{Yuktabhavadeva} 7.204 (attr.~to the \emph{Haṭhapradīpikā})
\begin{versinnote}
\tl{bandhatrayaṃ mahāguhyaṃ jarāmṛtyuvināśanam |\\+}
\tl{vahnivṛddhikaraṃ caiva aṇimādiguṇapradam ||\\!}
\end{versinnote}
\end{testimonia}

%\begin{philcomm}[hp03_030]
%\end{philcomm}

%%%%%%%%%%
\subsection*{3.31}
\begin{translation}[hp03_031]
It is practised eight times a day, every three hours. It always produces a wealth of merit and destroys an ocean of demerit.
\end{translation}

\begin{sources}[hp03_031]
\emph{Amaraugha} 32
\begin{versinnote}
\tl{aṣṭadhā kriyate caiva yāme yāme dine dine |\\+}
\tl{puṇyasañcayasambhāvi pāpaughabhiduraṃ sadā ||\\+}
\tl{\var{caiva ] caitad \vl}\\!}
\end{versinnote}
\end{sources}

\begin{testimonia}[hp03_031]
\emph{Haṭharatnāvalī} 2.49
\begin{versinnote}
\tl{aṣṭadhā kriyate caitat yāme yāme dine dine |\\+}
\tl{puṇyasaṅghātasandhāyī pāpaughabhiduraḥ sadā ||\\!}
\end{versinnote}

\emph{Yogacintāmaṇi} f.~73v (attr.~to the \emph{Haṭhapradīpikā})
\begin{versinnote}
\tl{aṣṭadhā kriyate caiva yāme yāme dine dine |\\+}
\tl{puṇyasaṃbhārasandhāyi pāpaughabhiduraṃ sadā ||\\!}
\end{versinnote}

\emph{Yuktabhavadeva} 7.205 (attr.~to the \emph{Haṭhapradīpikā})
\begin{versinnote}
\tl{aṣṭadhā kriyate caiva yāme yāme dine dine |\\+}
\tl{puṇyasambhārasandhāyi pāpaughabhiduraṃ sadā ||\\!}
\end{versinnote}
\end{testimonia}

%\begin{philcomm}[hp03_031]
%JB: It seems to make better sense to read this as saying that the yogi practises the three seals `eight times, every three hours, every day.' This would mean the triad is practised sixty-four times in total a day (commesurate with the eighty kumbhakas per day). not accepted at the workshop by Haru and Lubomir. (In a post reading session discussion, Haru thinks the interpretation of 64 times a day is possible)
%\end{philcomm}

%%%%%%%%%%
\subsection*{3.32}
\begin{translation}[hp03_032]
It is only for those who have received proper instruction. It is taught that they should [do it] a little in the first stage of the practice [and] that at the beginning [the yogi] should avoid frequenting fire, women and roads.
\end{translation}

\begin{sources}[hp03_032]
\emph{Amaraugha} 33
\begin{versinnote}
\tl{samyakśikṣāvatām eva svalpaṃ prathamasādhane |\\+}
\tl{vahnistrīpathasevānām ādau varjanam ācaret ||\\!}
\end{versinnote}
\end{sources}

\begin{testimonia}[hp03_032]
\emph{Yogacintāmaṇi} f.~73v (attr.~to the \emph{Haṭhapradīpikā})
\begin{versinnote}
\tl{samyakśikṣāvatām eva svalpaṃ prathamasādhane ||\\!}
\end{versinnote}

\emph{Yuktabhavadeva} 7.205 (attr.~to the \emph{Haṭhapradīpikā})
\begin{versinnote}
\tl{samyak śikṣāvatām evaṃ svalpaṃ prathamasādhanam |\\+}
\tl{vahnistrīpathisevānām ādau varjanam ācaret ||\\!}
\end{versinnote}

\end{testimonia}

%\begin{philcomm}[hp03_032]
%\end{philcomm}

%%%%%%%%%%
\subsection*{3.33 heading}
\begin{translation}[hp03_033a]
Now, the sky-roving [seal] (\emph{khecarī}):
\end{translation}

% \begin{philcomm}[hp03_033a]
% \end{philcomm}

%%%%%%%%%%
\subsection*{3.33}
\begin{translation}[hp03_033]
By cutting, moving, and milking, the yogi should gradually lengthen the tongue until it touches the middle of the brows. Then there is success in \emph{khecarī}.
\end{translation}

%\begin{sources}[hp03_033]
%\end{sources}

\begin{testimonia}[hp03_033]
\emph{Haṭharatnāvalī} 2.141
\begin{versinnote}
\tl{haṭhapradīpikāyām ||\\+}
\tl{chedanacālanadohaiḥ kalāṃ krameṇa vardhayet tāvat |\\+}
\tl{yāvad iyaṃ bhrūmadhye spṛśati tadānīṃ khecarīsiddhiḥ ||\\+}
\tl{\var{yāvad iyaṃ bhrūmadhye spṛśati ] sā yāti yāvad bhrūmadhyaṃ spṛśati hi \vl}\\!}
\end{versinnote}

\emph{Yogacintāmaṇi} f.~74r
\begin{versinnote}
\tl{haṭhapradīpikāyām—\\+}
\tl{chedanacālanadohair jihvāṃ saṃvardhayet tāvat |\\+}
\tl{yāvad iyaṃ bhrūmadhyaṃ spṛśati tadā khecarī siddhiḥ ||\\!}
\end{versinnote}

\emph{Haṭhatattvakaumudī} 14.18
\begin{versinnote}
\tl{haṭhapradīpikāmate tu –\\+}
\tl{chedanacālanadohaiḥ krameṇa jihvāṃ pravardhayet tāvat |\\+}
\tl{sā yāvad bhrūmadhyaṃ spṛśati tadā khecarīsiddhiḥ || \\!}
\end{versinnote}
\end{testimonia}

\begin{philcomm}[hp03_033]
%There are no \emph{āryā} or \emph{gīti} metres elsewhere in the text, but there are verses  in \emph{upagīti} metre (e.g. 1.60). We have therefore adopted variants that fit the \emph{upagīti} metre. Perhaps this is authorial, as we don't know a source for it, and we don't know a source for 1.60. % JB: this is a weak reason given the readings of alpha!

Various versions of this verse have been transmitted in \emph{āryā}, \emph{gīti} and \emph{anuṣṭubh} metres. We have adopted a version close to \alphaTwo, which has a slight metrical fault:
\begin{quote}
 chedanacālanadohaiḥ kalāṃ krameṇa pravardhayet tāvat |\\
 sā yāvad bhrūmadhyaṃ viśati tadānīṃ khecarīsiddhiḥ ||
\end{quote}
% alpha2
% chedanacālanaṃ dohaiḥ kalā kramaṇa pravardhaye tāvat
% sā yāvad bhrumadhyaṃ visati tadānī khecarīsiddhiḥ
The emendation of \emph{tadānīṃ} to \emph{tadā} renders the verse an \emph{āryā}. The word \emph{krameṇa} is well attested by manuscripts of the \textalpha, \textbeta, \texteta\ and  \textepsilon\ groups, which all have unmetrical or corrupted versions. 
% krameṇa is well attested
% close to alpha 2
% damaged in alpha 1: [damaged]ḥ kalāḥ krameṇa varddayet | sā yāva[d] bhṛmadhyaṃ viśa tadānīṃ khecarīsiddhiḥ || but it looks like someone has unsuccessfully tried to make it anuṣṭubh. Perhaps, suggests that a longer metre was original.

The meaning of \emph{kalā} as `tongue' is not attested by any Sanskrit dictionary but \emph{kalā} occurs in the sense of the tongue in a subsequent verse of this chapter (Cf.~3.35a) and it is glossed by Brahmānanda with \emph{jihvā} in \emph{Jyotsnā} 3.33 and 3.37.
%
\end{philcomm}

\begin{metre}[hp03_033]
Āryā
\end{metre}

%%%%%%%%%%
\subsection*{3.34}
\begin{translation}[hp03_034]
When the tongue is turned back and inserted into the cavity of the skull and the gaze is between the brows, \emph{khecarīmudrā} arises.
\end{translation}

\begin{sources}[hp03_034]
\emph{Vivekamārtaṇḍa} 47
\begin{versinnote}
\tl{kapālakuhare jihvā praviṣṭā viparītagā |\\+}
\tl{bhruvor antargatā dṛṣṭir mudrā bhavati khecarī ||\\!}
\end{versinnote}
\end{sources}

\begin{testimonia}[hp03_034]
\emph{Haṭharatnāvalī} 2.138
\begin{versinnote}
\tl{dattātreyas tu ||\\+}
\tl{kapālakuhare jihvā praviṣṭā viparītagā |\\+}
\tl{bhruvor antargatā dṛṣṭir mudrā bhavati khecarī ||\\!}
\end{versinnote}

\emph{Yogacintāmaṇi} f.~75r
\begin{versinnote}
\tl{skandapurāṇe—\\+}
\tl{kapālakuhare jihvā praviṣṭā viparītagā |\\+}
\tl{bhruvor antargatā dṛṣṭir mudrā bhavati khecarī ||\\!}
\end{versinnote}

\emph{Yuktabhavadeva} 7.207 (attr.~to the \emph{Haṭhapradīpikā})
\begin{versinnote}
\tl{atha khecarī -\\+}
\tl{kapālakuhare jihvā praviṣṭā viparītagā |\\+}
\tl{bhruvor antargatā dṛṣṭir mudrā bhavati khecarī ||\\!}
\end{versinnote}
\end{testimonia}

%\begin{philcomm}[hp03_034]
%\end{philcomm}

%%%%%%%%%%
\subsection*{3.34*1}
\begin{translation}[hp03_034_1]
If the yogi turns back the tongue and remains [like that] for half an instant, he is instantly freed from disease, death, old age and the like.
\end{translation}

\begin{sources}[hp03_034_1]
\emph{Śivasaṃhitā} 3.91
\begin{versinnote}
\tl{rasanām ūrdhvagāṃ kṛtvā kṣaṇārdhaṃ yadi tiṣṭhati |\\+}
\tl{kṣaṇena mucyate yogī vyādhimṛtyujarādibhiḥ ||\\!}
\end{versinnote}
\end{sources}

\begin{testimonia}[hp03_034_1]
\emph{Yogacintāmaṇi} f.~74r (attr.~to the \emph{Haṭhapradīpikā})
\begin{versinnote}
\tl{kalāṃ parāṅmukhīṃ kṛtvā tripathe parivartayet |\\+}
\tl{sā bhavet khecarī mudrā vyomacakraṃ tad ucyate ||\\+}
\tl{rasanām ūrdhvagāṃ kṛtvā kṣaṇārdhaṃ yadi tiṣṭhati |\\+}
\tl{viṣayair mucyate yogī vyādhimṛtyujarādibhiḥ ||\\!}
\end{versinnote}

\emph{Yuktabhavadeva} 7.209 (attr.~to the \emph{Haṭhapradīpikā})
\begin{versinnote}
\tl{jihvāṃ parāṅmukhīṃ kṛtvā kṣaṇārddhaṃ yadi tiṣṭhati |\\+}
\tl{kṣaṇena mucyate yogī vyādhimṛtyujarādibhiḥ ||\\!}
\end{versinnote}
\end{testimonia}

\begin{philcomm}[hp03_034_1]
This verse is absent in \alphaOne\ and was likely added to a block of verses that Svātmārāma borrowed from the \textit{Vivekamārtaṇḍa} (47–51). 3.34*1 is in \alphaTwo\ and \alphaThree\ (\alphaOne\ omits 2.35 and the greyscaled verses) and also the \textgamma\ and \texteta\ groups. Other manuscripts have an additional line that gives \emph{vyomacakra} as an alternative name for \emph{khecarīmudrā}. This alternative name does not occur in any of the source texts known to have been used by Svātmārāma, but it may have been inspired by the name \emph{nabhomudrā}, which is what the \emph{Vivekamārtaṇḍa} calls \emph{khecarīmudrā} (\emph{Vivekamārtaṇḍa} 40).   
\end{philcomm}

% (MD 2024-5-15): Should we perhaps take the absence of this verse in α1 seriously and regard it as inauthentic? HP 3.34-43 are all from VM, but not this 3.35. In α2 and α3 it may have been added independently by contamination. There are considerable differences in the readings between the two. JM: seems good to me.
% Pāda  α2 - α3
% e: nītvā - kṛtvā*
% f: api - yadi*
% g: viṣaye - kṣaṇena*
% The asterisk indicates that the reading has been adopted for our edition.

%%%%%%%%%%
\subsection*{3.36}
\begin{translation}[hp03_036]
For the yogi who knows \emph{khecarīmudrā} there is no disease, death, sleep, hunger, thirst or fainting.
\end{translation}

\begin{sources}[hp03_036]
\emph{Vivekamārtaṇḍa} 48
\begin{versinnote}
\tl{na rogo maraṇaṃ tandrā na nidrā na kṣudhā tṛṣā |\\+}
\tl{na ca mūrchā bhavet tasya yo mudrāṃ vetti khecarīm ||\\+}
\tl{\var{tandrā ] tasya \vl}\\!}
\end{versinnote}
\end{sources}

\begin{testimonia}[hp03_036]
\emph{Haṭharatnāvalī} 2.139 (attr.~to Dattātreya)
\begin{versinnote}
\tl{na rogo maraṇaṃ tasya na nidrā na kṣudhā tṛṣā |\\+}
\tl{na ca mūrcchā bhavet tasya yo mudrāṃ vetti khecarīm ||\\!}
\end{versinnote}

\emph{Yogacintāmaṇi} f.~75v (attr.~to Dattātreya)
\begin{versinnote}
\tl{na rogo maraṇaṃ tasya na nidrā na kṣudhā tṛṣā |\\+}
\tl{na ca mūrcchā bhavet tasya yo mudrāṃ vetti khecarīm ||\\!}
\end{versinnote}

\emph{Yuktabhavadeva} 7.210 (attr.~to the \emph{Haṭhapradīpikā})
\begin{versinnote}
\tl{na rogo maraṇaṃ tasya na nidrā na tṛṣā kṣudhā |\\+}
\tl{na ca mūrcchā bhavet tasya yo mudrāṃ vetti khecarīm ||\\!}
\end{versinnote}
\end{testimonia}

%\begin{philcomm}[hp03_036]
%\end{philcomm}

%%%%%%%%%%
\subsection*{3.37}
\begin{translation}[hp03_037]
[The yogi] who knows \emph{khecarīmudrā} is neither afflicted by disease, nor tainted by action, nor tormented by death.
\end{translation}

\begin{sources}[hp03_037]
\emph{Vivekamārtaṇḍa} 49
\begin{versinnote}
\tl{pīḍyate na sa rogeṇa lipyate na ca karmaṇā |\\+}
\tl{bādhyate na sa kālena yo mudrāṃ vetti khecarīṃ ||\\!}
%JB: Retain variants for lipyate na ca as alpha reads na ca lipyata/lipyati
%MD(2024-5-14): Perhaps we should read na ca lipyati karmaṇā as epic Skt.
\end{versinnote}
\end{sources}

\begin{testimonia}[hp03_037]
\emph{Haṭharatnāvalī} 2.140 (attr.~to Dattātreya)
\begin{versinnote}
\tl{pīḍyate na sa rogeṇa lipyate na ca karmaṇā|\\+}
\tl{bādhyate na ca kālena yo mudrāṃ vetti khecarīm||\\!}
\end{versinnote}

\emph{Yogacintāmaṇi} f.~75v (attr.~to the \emph{Skandapurāṇa})
\begin{versinnote}
\tl{piḍyate na sa rogeṇa na ca lipyeta karmaṇā |\\+}
\tl{bādhyate na sa kālena yo mudrāṃ vetti khecarīm ||\\!}
\end{versinnote}

\emph{Yuktabhavadeva} 7.211 (attr.~to the \emph{Haṭhapradīpikā})
\begin{versinnote}
\tl{pīḍyate na sa rogeṇa lipyate na sa karmaṇā |\\+}
\tl{bādhyate na sa kālena yasya mudrāsti khecarī ||\\!}
\end{versinnote}
\end{testimonia}

\begin{philcomm}[hp03_037]
The form of \emph{lipyati}, which is found in the \alpha \ manuscripts and has been adopted, is widely attested in epic Sanskrit.
\end{philcomm}

%%%%%%%%%%
\subsection*{3.38}
\begin{translation}[hp03_038]
Because the mind moves (\emph{carati}) in the ether (\emph{khe}) and the tongue moves (\emph{carati}) in the cavity (\emph{khe}), this mudrā is called \emph{khecarī} [and] is worshipped by the Siddhas.
\end{translation}

\begin{sources}[hp03_038]
\emph{Vivekamārtaṇḍa} 50
\begin{versinnote}
\tl{cittaṃ carati khe yasmāj  jihvā carati khe gatā |\\+}
\tl{tenaiṣā  khecarī nāma mudrā siddhair namaskṛtā ||\\+}
\tl{\var{37c tenaiṣā ] tenaiva, teneyaṃ \vl 37cd nāma mudrā ] mudrā sarva° \vl}\\!}
\end{versinnote}
\end{sources}

\begin{testimonia}[hp03_038]
\emph{Yogacintāmaṇi} (attributed to the Skandapurāṇa)
\begin{versinnote}
\tl{cittaṃ carati khe yasmāj jihvā carati khe gatā |\\+}
\tl{tenaiṣā khecarī nāma mudrā siddhair niṣevitā ||\\!}
\end{versinnote}

\emph{Yuktabhavadeva} 7.212 (attr.~to the \emph{Haṭhapradīpikā})
\begin{versinnote}
\tl{cittaṃ carati khe yasmāj jihvā carati khe yataḥ |\\+}
\tl{teneyaṃ khecarī mudrā sarvasiddhair namaskṛtā ||\\!}
\end{versinnote}
\end{testimonia}

%\begin{philcomm}[hp03_038]
%Judit: this mudrā is worshipped by the siddhas as the one called khecarī. (the idea being that this verse is explaining the name, khe + carī.
%\end{philcomm}

%%%%%%%%%%
\subsection*{3.39}
\begin{translation}[hp03_039]
The yogi who has sealed the cavity above the uvula with \emph{khecarī} does not lose his semen [even if] embraced by an amorous woman.
\end{translation}

\begin{sources}[hp03_039]
\emph{Vivekamārtaṇḍa} 51
\begin{versinnote}
\tl{khecaryā mudritaṃ yena vivaraṃ lambikordhvataḥ |\\+}
\tl{na tasya kṣarate binduḥ kāminyāliṅgitasya ca ||\\+}
\tl{\var{na tasya kṣarate binduḥ ] binduḥ kṣarati no tasya, tasya na kṣarate binduḥ \vl}\\+}
\tl{\var{°āliṃgitasya ] °āśleṣitasya \vl}\\!}
\end{versinnote}
\end{sources}

\begin{testimonia}[hp03_039]
\emph{Yogacintāmaṇi} f.~74v (attr.~to the \emph{Haṭhapradīpikā})
\begin{versinnote}
\tl{khecaryā mudritaṃ yena vivaraṃ lambikordhvataḥ |\\+}
\tl{na tasya kṣarate binduḥ kāminyāliṅgitasya ca ||\\!}
\end{versinnote}

\emph{Yuktabhavadeva} 7.213 (attr.~to the \emph{Haṭhapradīpikā})
\begin{versinnote}
\tl{khecaryā mudritaṃ yena vivaraṃ lambikordhvataḥ |\\+}
\tl{na tasya kṣarate binduḥ kāminyāliṃgitasya ca ||\\!}
\end{versinnote}
\end{testimonia}

%\begin{philcomm}[hp03_039]
%\end{philcomm}

%%%%%%%%%%
\subsection*{3.40}
\begin{translation}[hp03_040]
Even when semen has moved [down] and reached the region of the perineum, it moves upwards having been blocked by \emph{yonimudrā} and struck by the goddess [Kuṇḍalinī].
\end{translation}

\begin{sources}[hp03_040]
\emph{Vivekamārtaṇḍa} 53
\begin{versinnote}
\tl{calito 'pi yadā binduḥ saṃprāptaś ca hutāśanam |\\+}
\tl{gacchaty ūrdhvaṃ hataḥ śaktyā nibaddho yonimudrayā ||\\+}

\tl{\var{53c gacchaty ] vrajaty \vl 53d hataḥ ] tanaṃ, hṛtas, kṛte, kṛtaḥ, tadā, gatā \vl • nibaddho ] niruddho \vl}\\!}
\end{versinnote}

Cf.~\emph{Śivasaṃhitā} 4.82
\begin{versinnote}
\tl{svakaṃ binduṃ ca saṃbodhya liṅgacālanam ācaret |\\+}
\tl{daivāc calati ced ūrdhvaṃ nibaddho yonimudrayā ||\\!}
\end{versinnote}
\end{sources}

\begin{testimonia}[hp03_040]
\emph{Yogacintāmaṇi} f.~74v (attr.~to the \emph{Haṭhapradīpikā})
\begin{versinnote}
\tl{calito 'pi yadā binduḥ saṃprāptaś ca hutāśanam |\\+}
\tl{vrajaty ūrdhvaṃ hi tacchaktyā niruddho yonimudrayā ||\\!}
\end{versinnote}

\emph{Yuktabhavadeva} 7.278 (attr.~to the \emph{Haṭhapradīpikā})
\begin{versinnote}
\tl{calito'pi mahābinduḥ samprāpte'pi hutāśanam |\\+}
\tl{vrajaty ūrdhvaṃ haṭhaḥ śaktyā nibaddho yonimudrayā ||\\!}
\end{versinnote}
\end{testimonia}

\begin{philcomm}[hp03_040]
% JB Perhaps we should adopt \emph{yonimaṇḍalam} in 3.39b? It is attested by the alpha group (N3 and J5) as well as V1 and V15. It makes good sense as `region of perineum', as understood by Brahmānanda in \emph{Jyotsnā} 3.43 (\emph{yonimaṇḍalam yonisthānam}).

The third quarter of this verse has been subjected to much rewriting. Most of the collated manuscripts, including \alphaTwo\ (\alphaOne\ is illegible here and \alphaThree\ is missing this verse quarter), have the reading \emph{haṭhāt śaktyā}. This reading only makes sense if one infers that \emph{śaktyā} is referring to \emph{khecarīmudrā}, so that the second line means `blocked by \emph{yonimudrā}, semen goes up forcefully by the power [of \emph{khecarīmudrā}].' The reading \emph{hi tacchaktyā} of two delta manuscripts (\getsiglum{K3} and \getsiglum{C7}) and the \emph{Yogacintāmaṇi}), appears to be an attempt to render more clearly the meaning `by the power of \emph{khecarī}.' Such an interpretation suggests that \emph{yonimudrā} blocks \emph{bindu}'s downward course and \emph{khecarī} causes it to go upwards forcefully.

A manuscript of an early recension of the \emph{Vivekamārtaṇḍa} (ms.~no.~4110) has \emph{hataḥ śaktyā}, which is attested by three \emph{Haṭhapradīpikā} manuscripts on lower branches of the stemma (i.e.~\getsiglum{Ba} of \textdelta\ group and \getsiglum{C2} and \getsiglum{P4} of group 7a). The participle \emph{haṭaḥ} makes sense of the instrumental \emph{śaktyā}, rendering the meaning `struck by Kuṇḍalinī.' 
% MD: These groups don't have a Greek sigla yet.

Alternatively, the word \emph{haṭaḥ} (as well as the other variants \emph{kṛtaḥ, kṛte}, and even \emph{haṭhāt}) may derive from \emph{hṛtaḥ}, which is attested by \emph{Haṭhapradīpikā} manuscripts from group 4a (i.e., \getsiglum{Ba5}, \getsiglum{N4a} and \getsiglum{N13}). The reading \emph{hṛtaḥ śaktyā} renders the verse as saying that semen goes up, carried by Kuṇḍalinī. 





%Adopt hataḥ śaktyā, which is supported by C2,  and the good manuscripts of the \emph{Vivekamārtaṇḍa}.
%[MD: Text changed. C2 reads haṭ(h)aḥ, J5 haṭhāt saktyā, G4 damaged]
%
% JB haṭhāt is very well attested (J5, J7, Gr4c, V1, C6, V3,J10). Perhaps, we should reconsider this. Since the context is khecarīmudrā, it makes sense to infer it with śaktyā and perhaps haṭhāt fits with Svātmārāma's agenda to foreground Haṭhayoga?

%But Judit/Philipp think hṛtaḥ is better and explains all the variations (kṛtaḥ, kṛte, etc. and even haṭha). So, Kuṇḍalinī carries semen upwards.

%Adopt nibaddho (J5) "bound/stopped by yonimudrā". [MD: Text changed.]

% write a note on yonimudrā
In \emph{Jyotsnā} 3.43, Brahmānanda explains `\emph{yonimudrā}' as `taking the form of contracting the penis' (\emph{yonimudrayā meḍhrākuñcanarūpayā}). He may have had in mind the practice of contracting and drawing the urethra upwards, which is described below in the section on \emph{vajrolimudrā} (\emph{Haṭhapradīpikā} 3.82). The author of the \emph{Yogaprakāśikā} (5.66) says that \emph{yonimudrā} is well-known in treatises on \emph{mantra} (\emph{yonimudrayeti mantraśāstraprasiddhayety arthaḥ} | \emph{°prasidhyayety} ed.). This is consistent with the \emph{Śivasaṃhitā}'s teaching on \emph{yonimudrā}, where it is described as activating the perineum (\emph{yoni}) by contracting it (4.2) to bring about success in mantra repetition (5.12). Later compendiums on yoga reiterate the role of \emph{yonimudrā} in mantra practice (e.g., the \emph{Yogacintāmaṇi} f.~65r, citing the \emph{Pārameśvaratantra}, and \emph{Haṭhatattvakaumudī} 33.12). The \emph{Haṭhayogasaṃhitā} (43–48) teaches a different version of \emph{yonimudrā} in its reportoire of twenty-five \emph{mudrā}s. In this work, \emph{yonimudrā} is supposed to awaken Kuṇḍalinī and consists of sitting in \emph{siddhasana}, blocking the ears, eyes, nose and mouth with the thumbs, index, middle and ring fingers, uniting \emph{praṇa} and \emph{apāna}, meditating on the six \emph{cakra}s and repeating the mantra \emph{huṃ haṃsa}. 

Manuscripts of the main groups, including \textalpha, \textbeta\ and \textgamma, have an additional hemistich after 3.40, which seems to say that the tongue is in the cavity of the skull by means of a \emph{mudrā} for uniting the \emph{kalā}s (\emph{kapālakuhare jihvā kalāsandhānamudrayā}). This hemistich likely derives from a marginal note that was explaining \emph{ūrdhvajihvaḥ} in the next verse. The compound \emph{kalāsandhānamudrayā} may have been added as some form of dittography or as a gloss on \emph{yonimudrā}, which is not described elsewhere in the text. In a slightly modified form, this line appears in a verse in the six-chapter version of the \emph{Haṭhapradīpikā} (f. 112r–112v):

\begin{versinnote}
\tl{kapālakuhare jihvā kalāsaṃdhānavarjitā |\\+}
\tl{brahmarandhragatā nityāṃ tasya siddhi na dūrataḥ ||\\!}
\end{versinnote}


%RW: C2 includes greyscale 3.39 and adds another pāda; then skips 3.40 to get to 3.41 (MD: old numbering)
%The additional pāda is: tasmād idaṃ prakuvīrta nitya yuktuḥ samāhitaḥ
%It is difficult to understand the first hemistich without the second (and the second is not attested in any HP witnesses).
%Judit: kalāsaṃdhānamudrayā may have been added as some form of dittography
%JB: maybe this line was a marginal note attempting to explain yonimudrā, which is not described elsewhere in the text (?).
\end{philcomm}

%%%%%%%%%%
\subsection*{3.41}
\begin{translation}[hp03_041]
The knower of yoga who remains with the tongue upwards and drinks nectar certainly conquers death in half a month.
\end{translation}

\begin{sources}[hp03_041]
\emph{Vivekamārtaṇḍa} 125
\begin{versinnote}
\tl{ūrdhvajihvaḥ sthito bhūtvā somapānaṃ karoti yaḥ |\\+}
\tl{māsārdhena na sandeho mṛtyuṃ jayati yogavit ||\\+}
\tl{\var{ūrdhvajihvaḥ sthito bhūtvā ] ūrdhvajihvas tato bhūtvā, ūrdhvaṃ jihvāṃ sthiraṃ kṛtvā, ūrdhvāṃ jihvā sthirāṃ kṛtvā \vl}\\!}
\end{versinnote}
\end{sources}

\begin{testimonia}[hp03_041]
\emph{Yogacintāmaṇi} f.~75v (attr.~to the \emph{Skandapurāṇa})
\begin{versinnote}
\tl{ūrdhvajihvaḥ sthiro bhutvā somapānaṃ karoti yaḥ |\\+}
\tl{māsārdheṇa na saṃdeho mṛtyuṃ jayati yogavid ||\\!}
\end{versinnote}

\emph{Yuktabhavadeva} 7.215 (attr.~to the \emph{Haṭhapradīpikā})
\begin{versinnote}
\tl{ūrdhvajihvaḥ sa medhāvī somapānaṃ karoti yaḥ |\\+}
\tl{māsārddhena na sandeho mṛtyuṃ jayati yogavit ||\\!}
\end{versinnote}
\end{testimonia}

%\begin{philcomm}[hp03_041]
%Alpha and \emph{Vivekamārtaṇḍa} read sthito. Adopt
%MD: Done.
%\end{philcomm}

%%%%%%%%%%
\subsection*{3.42}
\begin{translation}[hp03_042]
Poison does not enter the yogi whose body is always filled by [nectar from] the digits of the moon, even if he is bitten by Takṣaka.
\end{translation}

\begin{sources}[hp03_042]
\emph{Vivekamārtaṇḍa} 130
\begin{versinnote}
\tl{nityaṃ somakalāpūrṇaṃ śarīraṃ yasya yoginaḥ |\\+}
\tl{takṣakenāpi daṣṭasya viṣaṃ tasya na pīḍayet ||\\+}
\tl{\var{pīḍayet ] pīḍyate, sarpati, bādhyate \vl}\\!}
\end{versinnote}
\end{sources}

\begin{testimonia}[hp03_042]
\emph{Yogacintāmaṇi} f.~75v (attr.~to the \emph{Skandapurāṇa})
\begin{versinnote}
\tl{nityaṃ somakalāpūrṇaṃ śarīraṃ yasya yoginaḥ |\\+}
\tl{takṣakenāpi daṣṭasya viṣaṃ taṃ na ca sarpati ||\\!}
\end{versinnote}

\emph{Yuktabhavadeva} 7.216 (attr.~to the \emph{Haṭhapradīpikā})
\begin{versinnote}
\tl{nityaṃ somakalāpūrṇaṃ śarīraṃ yasya yoginaḥ |\\+}
\tl{takṣakenāpi daṣṭasya viṣaṃ tasya na sarpati ||\\!}
\end{versinnote}
\end{testimonia}

\begin{philcomm}[hp03_042]
In the context of poison, \emph{takṣaka} refers to one of the three kings of the snakes (\emph{nāga}), the other two being Śeṣa and Vāsuki (Mani 1975: 782–783). % Purāṇic Encyclopaedia: A Comprehensive Dictionary with Special Reference to the Epic and Purāṇic Literature. Vettam Mani. Delhi: Motilal Banarsidass.

The verb \emph{sarpati} is well attested by the manuscripts of the \emph{Haṭhapradīpikā} and testimonia. It can take an object, which in this case is the yogi's body.
%At least two alpha mss have this verse. Therefore, keep it.
\end{philcomm}

%%%%%%%%%%
\subsection*{3.43}
\begin{translation}[hp03_043]
Just as fire does not leave its fuel nor light a wick in oil,
so the embodied person does not leave a body filled by the digits of the moon.
\end{translation}

\begin{sources}[hp03_043]
\emph{Vivekamārtaṇḍa} 131
\begin{versinnote}
\tl{indhanāni yathā vahnis tailavartiṃ ca dīpakaḥ |\\+}
\tl{tathā somakalāpūrṇaṃ dehī dehaṃ na muñcati ||\\+}
\tl{\var{°vartiṃ ca ] °vartīva, °vartti ca \vl}\\!}
\end{versinnote}
\end{sources}

\begin{testimonia}[hp03_043]
\emph{Yogacintāmaṇi} f.~74v (attr.~to the \emph{Haṭhapradīpikā})
\begin{versinnote}
\tl{indhanāni yathā vahnis tailavartī ca dīpakaḥ |\\+}
\tl{nityaṃ somakalāpūrṇaṃ dehī dehaṃ na muñcati ||\\!}
\end{versinnote}
\end{testimonia}

%\begin{philcomm}[hp03_043]
%3.42.1 is not found elsewhere (?). 3.42.2 is in the 6ch HP, but nowhere else.
%\end{philcomm}

%%%%%%%%%%
\subsection*{3.44}
\begin{translation}[hp03_044]
I consider he who regularly eats cow flesh and drinks God's liquor to be of good family. Others are destroyers of the family.
\end{translation}

%\begin{sources}[hp03_044]
%\end{sources}

\begin{testimonia}[hp03_044]
\emph{Haṭharatnāvalī} 2.158
\begin{versinnote}
\tl{gomāṃsaṃ bhakṣayen nityaṃ pibed amaravāruṇīṃ |\\+}
\tl{kulīnaṃ tam ahaṃ manye anye tu kulaghātakāḥ ||\\!}
\end{versinnote}

\emph{Yogacintāmaṇi} f.~74v (attr.~to the \emph{Haṭhapradīpikā})
\begin{versinnote}
\tl{gomāṃsaṃ bhakṣayen nityaṃ pibed amaravāruṇīm |\\+}
\tl{kulīnaṃ tam ahaṃ manye netarān kulaghātakān ||\\!}
\end{versinnote}


\end{testimonia}

%\begin{philcomm}[hp03_044]
%Brahmānanda identifies this (and next two verses?) as compositions of Svātmārāma, presumably because of \emph{manye}. [JB: I don't think this note is worth mentioning as it is not relevant to an editorial decision]
%\end{philcomm}

%%%%%%%%%%
\subsection*{3.45}
\begin{translation}[hp03_045]
By the word 'cow' is meant the tongue, for its insertion into the palate is the eating of cow's flesh. That [eating] is the destroyer of great sin.
\end{translation}

%\begin{sources}[hp03_045]
%\end{sources}

\begin{testimonia}[hp03_045]
\emph{Haṭharatnāvalī} 2.157
\begin{versinnote}
\tl{gośabdenoditā jihvā tatpraveśo hi tāluni |\\+}
\tl{gomāṃsabhakṣaṇaṃ tat tu mahāpātakanāśanaṃ ||\\!}
\end{versinnote}

\emph{Yogacintāmaṇi} f.~74v–75r (attr.~to the \emph{Haṭhapradīpikā})
\begin{versinnote}
\tl{gośabdenoditā jihvā tatpraveśo hi tāluni |\\+}
\tl{gomāṃsabhakṣaṇaṃ tat tu mahāpātakanāśanam ||\\!}
\end{versinnote}

\end{testimonia}

%\begin{philcomm}[hp03_045]
%44cd is in N3 and J5, so retain.
%\end{philcomm}

%%%%%%%%%%
\subsection*{3.46}
\begin{translation}[hp03_046]
The essence produced by the fire caused by the insertion of the tongue which flows from the moon is the liquor of the gods.
\end{translation}

%\begin{sources}[hp03_046]
%\end{sources}

\begin{testimonia}[hp03_046]
\emph{Haṭharatnāvalī} 2.159

\begin{versinnote}
\tl{jihvāpraveśasaṃbhūtavahninotthāpitā khalu |\\+}
\tl{candrāt sravati yaḥ sāraḥ sā syād amaravāruṇī ||\\!}
\end{versinnote}

\emph{Yogacintāmaṇi} f.~75r (attr.~to the \emph{Haṭhapradīpikā})
\begin{versinnote}
\tl{jihvāpraveśasaṃbhūtavahninotpāditaḥ khalu |\\+}
\tl{candrāt sravati yaḥ sāraḥ sā syād amaravāruṇī ||\\!}
\end{versinnote}

\end{testimonia}

%\begin{philcomm}[hp03_046]
%\end{philcomm}

%%%%%%%%%%
\subsection*{3.47}
\begin{translation}[hp03_047]
With his face upwards and his tongue fixed in the aperture [of the skull], the yogi should visualise as the supreme \emph{śakti} [the nectar] that is forcibly obtained from the breath as it drips from the head into the sixteen petals of the lotus. And he who drinks the gushing nectar, the pure fluid [surging] from the [moon's] digits in waves, is free of disease, has a body as soft as lotus fibre, and lives a long time.
\end{translation}

\begin{sources}[hp03_047]
\emph{Vivekamārtaṇḍa} 118
\begin{versinnote}
\tl{mūrdhnaḥ ṣoḍaśapatrapadmagalitaṃ prāṇād avāptaṃ haṭhād \\+}
\tl{ūrdhvāsyo rasanāṃ niyamya vivare śaktiṃ parāṃ cintayet |\\+}
\tl{utkallolakalājalaṃ ca vimalaṃ dhārāmayaṃ yaḥ piben \\+}
\tl{nirdoṣaḥ sa mṛṇālakomalatanur yogī ciraṃ jīvati ||\\+}
\tl{\var{118b niyamya ] nidhāya, vidhāya \vl • cintayet ] cālayet \vl 118c utkallolakalājalaṃ ] ca vimalaṃ dhārāmayaṃ ] (from HP); utkallolakalākalaṃ, utkallolajalākulaṃ, utkallolajalāmṛtaṃ, tat kallolakalājalaṃ, tat kallolajalākulaṃ \vl • ca vimalaṃ ] suvimalaṃ \vl • dhārāmayaṃ ] (from HP); jīvākulaṃ, jihvākulaṃ, dhārājalaṃ vl 118d tanur ] vapur \vl}\\!}
\end{versinnote}
\end{sources}

\begin{testimonia}[hp03_047]
Cf.~\emph{Haṭharatnāvalī} 2.150
\begin{versinnote}
\tl{utkallolakalāmṛtaṃ ca vimalaṃ dhārāmṛtaṃ yaḥ pibet |\\+}
\tl{nirdoṣaḥ sa mṛnālakomalatanur yogī ciraṃ jīvati ||\\+}
\tl{\var{utkallola° ] tatkallola° \vl, °tanur ] °vapur \vl}\\!}
\end{versinnote}

\emph{Yogacintāmaṇi} f.~75r (attr.~to the \emph{Haṭhapradīpikā})
\begin{versinnote}
\tl{mūrdhnaḥ ṣoḍaśapadmapatragalitaṃ prāṇād avāptaṃ haṭhād\\+}
\tl{ūrdhvāsyo rasanāṃ niyamya vivare śaktiṃ parāṃ cintayan |\\+}
\tl{tatkallolakalājalaṃ ca vimalaṃ jihvākulaṃ yaḥ piben\\+}
\tl{nirdoṣaḥ sa mṛṇālakomalatanur yogī ciraṃ jīvati ||\\!}
\end{versinnote}

\emph{Yuktabhavadeva} 7.217 (attr.~to Gorakṣanātha)
\begin{versinnote}
\tl{mūrdhnaḥ ṣoḍaśapatrapadmagalitaṃ prāṇād avāptaṃ haṭhāt \\+}
\tl{ūrdhvāsyo rasanāṃ niyamya vivare śaktiṃ parāṃ cintayet |\\+}
\tl{tatkallolakalājalaṃ suvimalaṃ dhārāmṛtaṃ yaḥ pibet\\+}
\tl{nirddoṣaḥ sa mṛṇālakomalatanur yogī paraṃ jīvati ||\\+}
\tl{\var{cintayet ] cintayan \vl}\\!}
\end{versinnote}

\emph{Haṭhatattvakaumudī} 14.24 (attr.~to the \emph{Haṭhapradīpikā})
\begin{versinnote}
\tl{ūrdhvaṃ ṣoḍaśapatrapadmagalitaṃ prāṇād avāptaṃ haṭhād \\+}
\tl{ūrdhvāsyo rasanāṃ niyamya kuhare śaktiṃ parāṃ cintayan |\\+}
\tl{utkallolakalājalaṃ suvimalaṃ dhārāmṛtaṃ yaḥ piben \\+}
\tl{nirddoṣaḥ sa mṛṇālakomalavapur yogī ciraṃ jīvati ||\\+}
\tl{\var{ūrdhvaṃ ] mūrdhvaṃ \vl}\\!}
\end{versinnote}
\end{testimonia}

\begin{philcomm}[hp03_047]
The meaning of \emph{prāṇāt} (`from the breath') in the first verse quarter is not easy to understand without the context of this verse in the source text, the \emph{Vivekamārtaṇḍa}. In the verse preceding this one in the \emph{Vivekamārtaṇḍa} (117), the breath, on reaching the “great lotus”, is said to turn into nectar (\emph{amṛta}). In \emph{Jyotsnā} 3.51, Brahmānanda says that there is a variant \emph{prāṇaiḥ} (`by means of the breaths'), which is easier to understand than \emph{prāṇāt}. He nonetheless accepts \emph{prāṇat} and understands it as being a means (\emph{prāṇāt sādhanabhūtād avāptam}). He also understands the sixteen-petalled lotus to be the lotus in the throat, into which the nectar drips.
%Adopt patrapadma. % MD: VM-T has also padmapattra...
\end{philcomm}

% MD(2024-5-14): adopt vapur (α1,β1,ε2 etc.) instead of tanur (α2,β2,βω etc.)?
\begin{metre}[hp03_047]
Śārdūlavikrīḍita 
\end{metre}

%%%%%%%%%%
\subsection*{3.48}
\begin{translation}[hp03_048]
If the tongue, while oozing nectar and constantly kissing the tip of the uvula, is salty, pungent, like  milk or the same as honey and ghee, diseases are eliminated for [the yogi], old age is stopped, he can recite treatises and scriptures, attains immortality together with the eightfold powers, and attracts Siddha women.
\end{translation}

\begin{sources}[hp03_048]
\emph{Vivekamārtaṇḍa} 128
\begin{versinnote}
\tl{cumbantī yadi lambikāgram aniśaṃ  jihvā rasasyandinī\\+}
\tl{sakṣārā kaṭukātha dugdhasadṛśā madhvājyatulyāthavā  |\\+}
\tl{vyādhīnāṃ haraṇaṃ  jaropaśamanaṃ śāstrāgamodīraṇaṃ\\+}
\tl{tasya syād amaratvaṃ aṣṭaguṇitaṃ siddhāṅgānākarṣaṇam ||\\+}
\tl{\var{128a rasasyandanī ] rasaspandanī VTG, rasāsvādinī A 128c jaropaśamanaṃ ] AGBGL; jarāpaharaṇaṃ V, jarāntakaranaṃ TGP • °odīraṇaṃ ] VA; °odgīraṇaṃ TGBGL, °occāraṇaṃ GP}\\!}
%
\end{versinnote}
\end{sources}

\begin{testimonia}[hp03_048]
\emph{Yogacintāmaṇi} f.~75r (attr.~to the \emph{Haṭhapradīpikā})
\begin{versinnote}
\tl{cumbantī yadi lambikāgram aniśaṃ jihvā rasasyandinī\\+}
\tl{sakṣārā kaṭukāmladugdhasadṛśaṃ madhvājyatulyaṃ yadā |\\+}
\tl{vyādhīnāṃ haraṇaṃ jarāntakaraṇaṃ śāstrāgamoddhāraṇaṃ\\+}
\tl{tasya syād iha siddhir aṣṭaguṇitā siddhāṅgaṇākarṣaṇam ||\\!}
\end{versinnote}

\emph{Yuktabhavadeva} 7.218 (attr.~to Gorakṣanātha)
\begin{versinnote}
\tl{cumbantī yadi lambikāgram aniśaṃ jihvā rasasyandinī \\+}
\tl{sakṣārā kaṭukāmladugdhasadṛśī madhvājyatulyāthavā |\\+}
\tl{vyādhīnāṃ haraṇaṃ jarāmbutaraṇaṃ śāstrāgamodgīraṇaṃ\\+}
\tl{tasya syād amaratvam aṣṭaguṇitaṃ siddhāṅganākarṣaṇam ||\\!}
\end{versinnote}

\emph{Haṭhatattvakaumudī} 14.25 (attr.~to the \emph{Haṭhapradīpikā})
\begin{versinnote}
\tl{cumbantī yadi lambikāgram anilaṃ jihvā rasasyandinī\\+}
\tl{sakṣārā kaṭukāmladugdhasadṛśāṃ madhvājyatulyā tathā |\\+}
\tl{vyādhīnāṃ haraṇaṃ jarāntakaraṇaṃ śāstrāgamodgīraṇaṃ\\+}
\tl{tasya syād amaratvam aṣṭaguṇavat siddhāṅganākarṣaṇam ||\\!}
\end{versinnote}
\end{testimonia}

%\begin{philcomm}[hp03_048]
%MD(2024-5-14): adopt sadṛśā? (alpha and VM)
%\end{philcomm}

\begin{metre}[hp03_048]
Śārdūlavikrīḍita 
\end{metre}

%%%%%%%%%%
\subsection*{3.49}
\begin{translation}[hp03_049]
There is one seed [syllable], which contains creation, one \emph{mudrā}, \emph{khecarī}, one god, the unsupported, and one state, beyond mind.
\end{translation}

\begin{sources}[hp03_049]
\emph{Timirodghāṭana} 5.14c–15b (NGMPP A35/3)
\begin{versinnote}
\tl{eka[ṃ] sṛṣṭimayaṃ bījaṃ ek[ā] mudrā tu khecarī |\\+}
\tl{dvāv etau jñāyate yena so pi śāntapade sthitam ||\\!}
\end{versinnote}

Quotation by Jayaratha \emph{ad} \emph{Tantrāloka} 32.63, introduced with \emph{yad āgamaḥ}
\begin{versinnote}
\tl{ekaṃ sṛṣṭimayaṃ bījam ekā mudrā ca khecarī |\\+}
\tl{dvāv ekaṃ yo vijānāti sa vai pūjyaḥ kulāgame ||\\!}
\end{versinnote}
\end{sources}

\begin{testimonia}[hp03_049]
\emph{Haṭharatnāvalī} 4.28
\begin{versinnote}
\tl{ekaṃ sṛṣṭimayaṃ bījam ekā mudrā ca khecarī |\\+}
\tl{eko devo nirālambaḥ ekāvasthā manonmanī  ||\\!}
\end{versinnote}

\emph{Yogacintāmaṇi} f.~75r (attr.~to the \emph{Haṭhapradīpikā})
\begin{versinnote}
\tl{ekaṃ sṛṣṭimayaṃ bījam ekā mudrā ca khecarī |\\+}
\tl{eko devo nirālamba ekāvasthā manonmanī ||\\!}
\end{versinnote}

\emph{Yuktabhavadeva} 7.219 (attr.~to Gorakṣanātha)
\begin{versinnote}
\tl{ekaṃ sṛṣṭimayaṃ bījaṃ ekā mudrā ca khecarī |\\+}
\tl{eko devo nirālamba ekāvasthā manonmanī ||\\!}
\end{versinnote}
\end{testimonia}

%\begin{philcomm}[hp03_049]
%\end{philcomm}

%%%%%%%%%%
\subsection*{3.49*1}


%%%%%%%%%%
\subsection*{3.49*2}
\begin{translation}[hp03_049_2]
That which enters the aperture into the underworld exists at the base of Meru. The wise [yogi] says that is the truth, the source of [all] rivers. The essence of the body flows from the moon. Because of that people die. [The yogi] should block it with the clay of the excellent [\emph{khecarī}] technique. Bodily perfection [arises] no other way.
\end{translation}% greyscale in ed.

%\begin{sources}[hp03_049_2]
%\end{sources}

\begin{testimonia}[hp03_049_2]
\emph{Haṭharatnāvalī} 4.30
\begin{versinnote}
\tl{pātāle yad viśati suṣiraṃ merūmūlaṃ tad asti\\+}
\tl{tattvaṃ caitad vadati sudhā tanmukhaṃ nimnagānām |\\+}
\tl{candrāt sāraṃ sravati vapuṣas tena mṛtyur narāṇāṃ \\+}
\tl{tad badhnīyāt sukharatimṛdur nānyathā kāryasiddhiḥ ||\\!}
\end{versinnote}
% after (ekaṃ sṛṣṭimayaṃ + mano yatra vilīyeta) and before (śarīraṃ tāvad eva)
%the context of this verse in the HR is rājayoga.

\emph{Yogacintāmaṇi} f.~75r (attr.~to the \emph{Haṭhapradīpikā})
\begin{versinnote}
\tl{tat pātālād viyati śikhare merumūle tad asti \\+}
\tl{tattvaṃ caitat pravadati sudhīḥ saṃmukhe nimnagānām |\\+}
\tl{candrāt sāraḥ sravati vapuṣas tena mṛtyur narāṇām \\+}
\tl{tad badhnīyāt svakaraṇamṛdā nānyathā kāyasiddhiḥ ||\\!}
\end{versinnote}
% after the verse (cittāyattaṃ nṛṇāṃ śukraṃ) and before (indhanāni yathā vahnis)
% context is khecarī (3rd chapter material)


\emph{Yuktabhavadeva} 7.220 (attr.~to Gorakṣanātha)
\begin{versinnote}
\tl{pātāle yad viśati suṣiraṃ merumūle yad asti \\+}
\tl{tadvac caitat pravadati sudhīs tanmukhaṃ nimnagānām |\\+}
\tl{candrāt sāraḥ sravati vapuṣas tena mṛtyur narāṇām \\+}
\tl{badhnīyāt tat sukaraṇam atho nānyathā kāyasiddhiḥ ||\\!}
\end{versinnote}
% context is khecarī, (3rd chapter material)
% after (ekaṃ sṛṣṭimayaṃ bījaṃ) and before (atha uḍḍiyānabandhaḥ)

\emph{Haṭhatattvakaumudī} 14.26 (attr.~to the \emph{Haṭhapradīpikā})
\begin{versinnote}
\tl{pātāle yad vitatasuṣiraṃ merumūle tad asmin \\+}
\tl{tadvac caitat pravadati sudhīs tanmukhaṃ nimnagānām |\\+}
\tl{candrāt sāraḥ sravati vapuṣas tena mṛtyur narāṇāṃ \\+}
\tl{taṃ badhnīyāt sukaraṇamṛdā nānyathā kāyasiddhiḥ ||\\!}
\end{versinnote}
% context is khecarī 3rd chapter material
% after (cumbantī yadi lambikā) and before (cittaṃ vicarati gagane)

\end{testimonia}

\begin{philcomm}[hp03_049_2]
%Workshop notes
% Sven: the first line could refer to Kuṇḍalinī
% J5 has this verse here.
% mṛdā = using clay to make a dam?
% Also see Netratantra 7.32
% Could this verse be about mūlabandha? (reading pātāle and mūle)
%J5: yat prāleyam pihitasukhire merumūle yad astī — this seems helpful.
% PN has a nice-ish reading: yat prāleyaṃ pihita sukhiraṃ merumūrdhnaṃ. mūrdhni etc. is much better.
% J10 has merumūrdhni sthitaṃ yat, which is good.
% If going with prāleyam, adopt yat prāleyam pihitasuṣiraṃ merumūrdhny asti tathyaṃ (~K1 Grp2)
% ... tasmin tattvaṃ pravadati
%Haru: Perhaps, praleyam was corrupted to pātāle and then the rest of the line changed.

This verse occurs in various places in the different recensions of the text. It is here in the \alphaTwo\ and \alphaThree\ manuscripts, in chapter four (4.8*3) in \alphaOne\ and the delta group, after 3.39 in \betaOne\ and \betaTwo, and in both the third and fourth chapters in the \textgamma\ manuscripts. It must have entered the transmission of the \emph{Haṭhapradīpikā} at an early stage, and has moved around, perhaps because the name of the technique alluded to is not stated and the meaning of the first half of the verse is somewhat vague without its original context (we are yet to identify a source).

The first verse quarter has many variants among the witnesses. \alphaTwo\ has:
\begin{versinnote}
\tl{yat prāleyaṃ pihitasukhire merumūle yad astī \\+}
\tl{tasmiṃs tattvaṃ pravadati sudhīs tan mukhaṃ nimnagānām\\!}
\end{versinnote}
This version of the verse is close to that of other important manuscripts in the \textbeta, \textgamma\ and \texteta\ groups. With the help of these manuscripts (in particular \getsiglum{K1} of the \textgamma\ group), the above can be emended and understood as follows:
\begin{versinnote}
\tl{yat prāleyaṃ pihitasuṣiraṃ merumūrdhny asti tathyaṃ \\+}
\tl{tasmiṃs tattvaṃ pravadati sudhīs tan mukhaṃ nimnagānām\\+}
\tl{\var{°suṣiraṃ ] \getsiglum{K1}, °sukhire \alphaTwo. °mūrdhny ] \getsiglum{P8}. °mūle \alphaTwo. asti tathyaṃ ] \getsiglum{K1}, yad astī \alphaTwo.}\\!}
“That cool liquid by which the aperture is filled at the top of Meru and exists as the truth, the wise [yogi] says that is the source of [all] rivers.” 
\end{versinnote}

%K1 in Mitsuyo's notes
%yat prāleya prahitasuṣiraṃ merumūrdhnāsti tathyaṃ
%tasmiṃs tatvaṃ pravadati sudhīs tan mukhaṃ nimnagānāṃ

% Other manuscripts have something more similar to:
% \begin{versinnote}
% \tl{pātāle yad viśati suṣiraṃ merumūlaṃ tad asti \\+}
% \tl{tattvaṃ caitat pravadati sudhīs tan mukhaṃ nimnagānām\\!}
% \end{versinnote}


\end{philcomm}

\begin{metre}[hp03_049_2]
Mandākrāntā 
\end{metre}

%%%%%%%%%%
\subsection*{3.50 heading}
\begin{translation}[hp03_050a]
The root lock (\emph{mūlabandha}):
\end{translation}

% \begin{philcomm}[hp03_050a]
% \end{philcomm}

%%%%%%%%%%
\subsection*{3.50}
\begin{translation}[hp03_050]
When [the yogi] presses the perineum with part of the heel, clenches the anus and draws up \emph{apāna}, it is called the root lock.
\end{translation}
% JB: the back part of the heel presses the perineum. It would be better to say, 'with part of the heel.'

\begin{sources}[hp03_050]
\emph{Vivekamārtaṇḍa} 42
\begin{versinnote}
\tl{pārṣṇibhāgena saṃpīḍya yonim ākuñcayed gudam |\\+}
\tl{apānam ūrdhvam ākṛṣya mūlabandho 'yam ucyate ||\\!}
\end{versinnote}
\end{sources}

\begin{testimonia}[hp03_050]
\emph{Haṭharatnāvalī} 2.58
\begin{versinnote}
\tl{pārṣṇibhāgena sampīḍya yonim ākuñcayed gudaṃ |\\+}
\tl{apānam ūrdhvam ākuñcya mūlabandho 'yam ucyate ||\\!}
\end{versinnote}

\emph{Yogacintāmaṇi} f.~76r (attr.~to the \emph{Haṭhapradīpikā})
\begin{versinnote}
\tl{pārṣṇibhāgena saṃpīḍya yonim ākuñcayed gudam |\\+}
\tl{apānam ūrdhvam ākṛṣya mūlabandho 'yam ucyate ||\\!}
\end{versinnote}

\end{testimonia}

%\begin{philcomm}[hp03_050]
%\end{philcomm}

%%%%%%%%%%
\subsection*{3.51}
\begin{translation}[hp03_051]
It forces the downward-moving \emph{apāna} breath to move upwards by contraction [of the anus]. Yogis call that \emph{mūlabandha}.
\end{translation}
% JB: Perhaps it is better to take MB as the subject of 51ab as MB is the subject of the previous verse? i.e., It forces the downward-moving \emph{apāna} breath to move upwards because of the contraction.

\begin{sources}[hp03_051]
\emph{Gorakṣaśataka} 53
\begin{versinnote}
\tl{adhogatim apānaṃ vai ūrdhvagaṃ kurute balāt |\\+}
\tl{ākuñcanena taṃ prāhur mūlabandhaṃ tu yoginaḥ ||\\+}
\tl{\var{53a °gatiṃ ] TU; °gataṃ G 53c ākuñcanena taṃ ] GU ākuñcane ca tat T}\\+}
\tl{\var{53d mūlabandhaṃ tu yoginaḥ ] T; mūlabandho yam ucyate GU}\\!}
\end{versinnote}
\end{sources}

\begin{testimonia}[hp03_051]
\emph{Haṭharatnāvalī} 2.59
\begin{versinnote}
\tl{adhogatim apānaṃ vai ūrdhvagaṃ kurute balāt |\\+}
\tl{ākuñcanena taṃ prāhur mūlabandhaṃ hi yoginaḥ ||\\!}
\end{versinnote}

\emph{Yogacintāmaṇi} f.~76r (attr.~to the \emph{Haṭhapradīpikā})
\begin{versinnote}
\tl{adhogatam apānaṃ ca tad ūrdhvaṃ kurute haṭhāt |\\+}
\tl{ākuñcanena taṃ prāhur mūlabandhaṃ tu yoginaḥ ||\\!}
\end{versinnote}
\end{testimonia}

%\begin{philcomm}[hp03_051]
%Adopt balāt (N3). haṭhāt seems to be an early alt. reading.
%MD: Text changed.
%\end{philcomm}

%%%%%%%%%%
\subsection*{3.52}
\begin{translation}[hp03_052]
[The yogi] should press his anus with his heel and forcefully contract the [\emph{apāna}] wind over and over again so that the breath goes upwards.
\end{translation}

\begin{sources}[hp03_052]
\emph{Dattātreyayogaśāstra} 144
\begin{versinnote}
\tl{gudaṃ pārṣṇyā tu saṃpīḍya yonim ākuñcayed balāt | \\+}
\tl{vāraṃ vāraṃ yathā cordhvaṃ samāyāti samīraṇaḥ ||\\+}
\tl{\var{yonim ] from \emph{Jyotsnā}, vāyum \emph{codd.}}\\!}
\end{versinnote}

Cf.~\emph{Śārṅgadharapaddhati} 4416
\begin{versinnote}
\tl{gudaṃ pārṣṇyā tu sampīḍya vāyum ākuñcayed balāt |\\+}
\tl{bāraṃ bāraṃ yathā cordhvaṃ samāyāti samīraṇaḥ ||\\!}
\end{versinnote}
\end{sources}

\begin{testimonia}[hp03_052]
\emph{Haṭharatnāvalī} 2.60
\begin{versinnote}
\tl{gudaṃ pārṣṇyā ca saṃpīḍya vāyum ākuñcayed balāt |\\+}
\tl{vāraṃ vāraṃ yathā cordhvaṃ samāyāti samīraṇaḥ ||\\!}
\end{versinnote} 

\emph{Yogacintāmaṇi} f.76r (attr.~to the \emph{Yogabīja})
\begin{versinnote}
\tl{gudaṃ pārṣṇyā tu saṃpīḍya vāyum ākuñcayed balāt |\\+}
\tl{vāraṃ vāraṃ tathā cordhvaṃ samāyāti samīraṇaḥ ||\\!}
\end{versinnote}

\emph{Yogabīja} 103 (south-Indian recension)
\begin{versinnote}
\tl{gudaṃ pārṣṇyā tu saṃpīḍya vāyum ākuñcayed balāt |\\+}
\tl{vāraṃ vāraṃ yathā cordhvaṃ samāyāti samīraṇaḥ ||\\!}
\end{versinnote}

\end{testimonia}

\begin{philcomm}[hp03_052]
The instruction to `contract the wind' (\emph{vāyum ākuñcayet}) is odd (especially with \emph{samīraṇaḥ} in the fourth \emph{pāda}) and not found in other texts. Mallin\-son has adopted \emph{yonim} for \emph{vāyum} (cf.~\emph{Haṭhapradīpikā} 3.50d) in his edition of this verse in its source text, the \emph{Dattātreyayogaśāstra}, which is not found in the manuscripts of that text but is in the \emph{Haṭhapradīpikā}’s \delta \ manuscripts. 

In the context of the root lock, \emph{vāyum ākuñcayet} can be understood as an instruction to contract \emph{apāna\-vāyu}, which is mentioned in the previous verse (3.51a). Instructions to contract \emph{apāna\-vāyu} are found in other yoga texts, such as \emph{Yoga\-tārā\-valī} 7b (\emph{ākuñcanaiḥ śaśvad apānavāyoḥ}), \emph{Śiva\-saṃhitā} 4.84cd (\emph{apāna\-vāyum ākuñcya balād...}), \emph{Śiva\-yoga\-pradīpikā} 2.53ab (\emph{athordhva\-madhya\-sthira\-bandha\-nābhyām ākuñcanād ūrdhvam apānavāyoḥ}) and \emph{Yukta\-bhava\-deva} 7.297 (\emph{ādhāra\-kamale suptāṃ cālayet kuṇḍalīṃ dṛḍhām} | \emph{apāna\-vāyum ākṛṣya balād ākuñcya buddhimān}). In \emph{Jyotsnā} 3.63, Brahmānanda understands \emph{vāyu} in this verse as \emph{apāna} when he says that `one should contract the wind, \emph{apāna}' (\emph{vāyum apānam ākuñcayed}), which he explains as, `one should pull it by contractions of the anus' (\emph{gudasyā\-kuñcanenā\-karṣayet}). In the same vein, Bhava\-deva\-miśra glosses `pulling \emph{apāna}' (\emph{apānā\-karṣaṇam}) as `contracting the anus' (\emph{gudā\-kuñcanam}), when commenting on `having pulled \emph{apāna\-vāyu} and forcefully contracted it ...' (\emph{apāna\-vāyum ākṛṣya balād ākuñcya...}) in \emph{Yukta\-bhava\-deva} 297 and 301.  
\end{philcomm} %

%%%%%%%%%%
\subsection*{3.53}
\begin{translation}[hp03_053]
When \emph{prāṇa} and \emph{apāna} [and] \emph{nāda} and \emph{bindu} become united by means of the root lock they are sure to bestow complete success in yoga.
\end{translation}

\begin{sources}[hp03_053]
\emph{Dattātreyayogaśāstra} 145
\begin{versinnote}
\tl{prāṇāpānau nādabindū mūlabandhena caikatām |\\+}
\tl{gatvā yogasya saṃsiddhiṃ yacchato nātra saṃśayaḥ ||\\+}
\tl{\var{145c gatvā yogasya saṃ° ] gacchato yogasaṃ° M1A}\\+}
\tl{\var{145d yacchato ] gacchato M1, kurute AM2, gachate \pi}\\!}
\end{versinnote}

%\emph{Śārṅgadharapaddhati} 4417
%\begin{versinnote}
%\tl{prāṇāpānau nādabindū mūlabandhena caikatām |\\+}
%\tl{gatvā yogasya saṃsiddhiṃ gacchatau nātra saṃśayaḥ ||\\!}
%\end{versinnote}
\end{sources}

\begin{testimonia}[hp03_053]
\emph{Haṭharatnāvalī} 2.61
\begin{versinnote}
\tl{prāṇāpānau nādabindū mūlabandhena caikatām |\\+}
\tl{gatau tadā yogasiddhiṃ prāpnoty eva na saṃśayaḥ ||\\!}
\end{versinnote}

\emph{Yogacintāmaṇi} f.76r (attr.~to the \emph{Yogabīja})
\begin{versinnote}
\tl{prāṇāpānau nādabindū mūlabandhena caikatām |\\+}
\tl{gatvā yogasya saṃsiddhiṃ gacchato nātra saṃśayaḥ ||\\!}
\end{versinnote}

\end{testimonia}

\begin{philcomm}[hp03_053]
Since the term \emph{nāda} usually means `internal sound' in Haṭha and Rājayoga texts, it is possible that \emph{bindu} here was understood by some to have the tantric connotations of sonic and visual foci (Mallinson 2007:219 n.325) or two levels of sonic emanation in \emph{mantroccāra}, where \emph{nāda} is an unvoiced sound and \emph{bindu} is the slightly coarser sound of inner murmuring (see \emph{Tāntrikābhidhānakośa} vol. 3, 2013: 278–279). However there is a passage in the \emph{Amaraugha} (10–12) where \emph{nāda} and \emph{bindu} are paired and it is clear that \emph{bindu} means generative fluid. 
\end{philcomm}


\begin{metre}[hp03_053]
Anuṣṭubh (a: ra-vipulā)
\end{metre}

%%%%%%%%%%
\subsection*{3.54}
\begin{translation}[hp03_054]
\textit{Prāṇa} and \textit{apāna} unite, urine and faeces diminish, [and] even an old man becomes young as a result of the continuous application of the root lock.
\end{translation}

\begin{sources}[hp03_054]
\emph{Vivekamārtaṇḍa} 41
\begin{versinnote}
\tl{apānaprāṇayor aikyaṃ kṣayo mūtrapurīṣayoḥ |\\+}
\tl{yuvā bhavati vṛddho ’pi satataṃ mūlabandhanāt ||\\!}
\end{versinnote}

%\emph{Śārṅgadharapaddhati} 4418
%\begin{versinnote}
%\tl{apānaprāṇayor aikyaṃ kṣayo mūtrapurīṣayoḥ |\\+}
%\tl{yuvā bhavati vṛddho 'pi satataṃ mūlabandhanāt ||\\!}
%\end{versinnote}
\end{sources}

\begin{testimonia}[hp03_054]
\emph{Haṭharatnāvalī} 2.62
\begin{versinnote}
\tl{apānaprāṇayor aikyaṃ kṣayo mūtrapurīṣayoḥ |\\+}
\tl{yuvā bhavati vṛddho 'pi satataṃ mūlabandhanāt ||\\!}
\end{versinnote}
\end{testimonia}

\begin{philcomm}[hp03_054]
The diminishing of urine and faeces as a result of success in yoga is mentioned in the \emph{Amanaska} (1.50c) and \emph{Dattātreyayogaśāstra} (80a).
%The term \emph{kṣaya} can mean either the end or diminishment of something. In the context of urine and faeces (\emph{mūtrapurīṣa}), their diminishment is the likely meaning intended as there are references in other texts to the successful practice of a yoga technique resulting in a reduced amount of urine and faeces, such as \emph{Amanaska} 1.50c (\emph{svalpamūtrapurīṣatvaṃ}) and \emph{Dattātreyayogaśāstra} 80a (\emph{alpamūtrapurīṣaḥ}), which were both known to Svātmārāma. In \emph{Jyotsnā} 3.65, Brahmānanda understands \emph{kṣaya} in this verse as `decline' (\emph{kṣayaḥ patanam}).%JM: I don't recall discussing this, but I think patanam here means excretion. Brahmānanda has  mūtrapurīṣayoḥ saṃcitayoḥ kṣayaḥ patanam. He's surely wrong in his understanding of the pāda. I have changed the note to just give the parallels.
\end{philcomm}

%%%%%%%%%%
\subsection*{3.55}
\begin{translation}[hp03_055]
When \emph{apāna} has turned upwards and reached the orb of fire, then the flame of the fire, fanned by the wind, grows tall.
\end{translation}

\begin{sources}[hp03_055]
\emph{Gorakṣaśataka} 54
\begin{versinnote}
\tl{apāne cordhvage jāte saṃprāpte vahnimaṇḍalam |\\+}
\tl{tato 'nalaśikhā dīrghā vardhate vāyunāhatā ||\\+}
\tl{\var{54a °maṇḍalam ] \emph{from Haṭhapradīpikā witnesses}, maṇḍale \emph{codd.}}\\!}
\end{versinnote}
\end{sources}

\begin{testimonia}[hp03_055]
\emph{Haṭharatnāvalī} 2.63
\begin{versinnote}
\tl{apāne cordhvage jāte prayāte vahnimaṇḍale |\\+}
\tl{tathānalaśikhādīptir vāyunā preritā yathā ||\\!}
\end{versinnote}

\emph{Yogacintāmaṇi} f.~76r (attr.~to the \emph{Yogabīja})
\begin{versinnote}
\tl{apāne cordhvage jāte saṃprāpte vahnimaṇḍale |\\+}
\tl{tathānalaśikhā dīrghā vardhate vāyunāhatā ||\\!}
\end{versinnote}

\end{testimonia}

%\begin{philcomm}[hp03_055]
%\end{philcomm}

%%%%%%%%%%
\subsection*{3.56}
\begin{translation}[hp03_056]
As a result, fire and \textit{apāna} reach \textit{prāṇa}, which is hot by nature, and the \emph{prāṇa} makes the fire in the body extremely hot.
\end{translation}

\begin{sources}[hp03_056]
\emph{Gorakṣaśataka} 55
\begin{versinnote}
\tl{tato yātau vahnyapānau prāṇam uṣṇasvarūpakam |\\+}
\tl{tenātyantapradīptena jvalano dehajas tathā ||\\+}
\tl{\var{dehajas ] T; dehagas GU}\\!}
\end{versinnote}
\end{sources}

\begin{testimonia}[hp03_056]
\emph{Haṭharatnāvalī} 2.64
\begin{versinnote}
\tl{yātāyātau vahnyapānau mūlarūpasvarūpakau |\\+}
\tl{tenābhyantaḥ pradīptas tu jvalano dehajas tathā || 2.64 ||\\!}
\end{versinnote}

\emph{Yogacintāmaṇi} f.~76r–76v (attr.~to the \emph{Yogabīja})
\begin{versinnote}
\tl{tato yātau vahnyapānau prāṇam uktasvarūpakau |\\+}
\tl{tenātyantapradīptas tu jvalano dehajas tathā ||\\!}
\end{versinnote}

\end{testimonia}

\begin{philcomm}[hp03_056]
The second verse quarter has been rewritten in the \textalpha\ manuscripts as \emph{prāṇamūlasvarūpakam}, which is similar to \etaOne\ (\emph{prāṇamūlasvarūpakau}) and the \emph{Haṭharatnāvalī} (\emph{mūlarūpasvarūpakau}). Manuscripts from the delta group, as well as the \emph{Yogacintāmaṇi}, have \emph{prāṇam uktasvarūpakam} or \emph{prāṇam uktasvarūpakau}. It appears that the intention behind these rewrites was to avoid the reading in the source text, `\textit{prāṇa} is hot by nature' (\emph{prāṇam uṣṇasvarūpakam}), which was likely accepted by Svātmārāma because it is in some witnesses of the \textbeta, \textgamma\ and \textepsilon\ groups. References to \emph{prāṇa} being hot by nature (and \emph{apāna} being cold) occur in other works, such as \emph{Mokṣopāya} 6.85.111-112 and \emph{Haṭhatattvakaumudī} 4.14, 41.2. The commentators, Bālakṛṣṇa (\emph{Yogaprakāśikā} 5.85) and Brahmānanda (\emph{Jyotsnā} 3.67) accept the idea that \emph{prāṇa} is hot by nature.
%

\end{philcomm}

\begin{metre}[hp03_056]
Anuṣṭubh (a: ra-vipulā)
\end{metre}

%%%%%%%%%%
\subsection*{3.57}
\begin{translation}[hp03_057]
Heated by that [blaze], the sleeping Kuṇḍalinī wakes up. Like a snake struck by a stick, she hisses and becomes straight.
\end{translation}

\begin{sources}[hp03_057]
\emph{Gorakṣaśataka} 56
\begin{versinnote}
\tl{tena kuṇḍalinī suptā saṃtaptā saṃprabudhyate |\\+}
\tl{daṇḍāhatā bhujaṃgīva niśvasya ṛjutāṃ vrajet ||\\!}
\end{versinnote}
\end{sources}

\begin{testimonia}[hp03_057]
\emph{Haṭharatnāvalī} 2.65ab
\begin{versinnote}
\tl{daṇḍāhatā bhujaṅgīva niścitaṃ ṛjutām iyāt |\\+}
\tl{\var{niścitaṃ ] niśvasya T,P,t1}\\!}
\end{versinnote}

\emph{Yogacintāmaṇi} f.~76v (attr.~to the \emph{Yogabīja})
\begin{versinnote}
\tl{tena kuṇḍalinī suptā satataṃ saṃprabodhyate |\\+}
\tl{daṇḍāhatabhujaṅgīva niścitam ṛjutāṃ vrajet ||\\!}
\end{versinnote}

\end{testimonia}

%\begin{philcomm}[hp03_057]
%\end{philcomm}

%%%%%%%%%%
\subsection*{3.58}
\begin{translation}[hp03_058]
Then, like [a snake] that has entered a hole, she goes into the channel of Brahman. Therefore, yogis should regularly practise the root lock.
\end{translation}

\begin{sources}[hp03_058]
\emph{Gorakṣaśataka} 57
\begin{versinnote}
\tl{bile praviṣṭe ca tato brahmanāḍyantaraṃ vrajet |\\+}
\tl{tasmān nityaṃ mūlabandhaḥ kartavyo yogibhiḥ sadā ||\\+}
\tl{\var{57a bile ] bil*e*T, bila° G2U, bilaṃ G1 • °praviṣṭe ca tato ] T; °praveśato yatra GU}\\!}
\end{versinnote}
\end{sources}

\begin{testimonia}[hp03_058]
\emph{Haṭharatnāvalī} 2.65c-f
\begin{versinnote}
\tl{bilaṃ praviṣṭeva tato brahmanāḍyantaraṃ vrajet |\\+}
\tl{tasmān nityaṃ mūlabandhaḥ kartavyo yogibhiḥ sadā ||\\!}
\end{versinnote}

\emph{Yogacintāmaṇi} f.~76v (attr.~to the \emph{Yogabīja})
\begin{versinnote}
\tl{bilaṃ praviṣṭeva tathā brahmanāḍyantaraṃ vrajet |\\+}
\tl{tasmān nityaṃ mūlabandhaḥ kartavyo yogipuṅgavaiḥ ||\\+}
\tl{\var{praviṣṭeva ] praviṣṭaiva \vl}\\!}
\end{versinnote}

\end{testimonia}

%\begin{philcomm}[hp03_058]
%\end{philcomm}

\begin{metre}[hp03_058]
Anuṣṭubh (a: bha-vipulā; c: ra-vipulā)
\end{metre}

%%%%%%%%%%
\subsection*{3.59 heading}
\begin{translation}[hp03_059a]
Now the Uḍḍīyaṇa lock:
\end{translation}

% \begin{philcomm}[hp03_059a]
% \end{philcomm}

%%%%%%%%%%
\subsection*{3.59}
\begin{translation}[hp03_059]
Because the breath flies up (\emph{uḍḍīyate}) into the Suṣumṇā when it is bound by it, yogis say that this [practice] is called Uḍḍīyaṇa.
\end{translation}

\begin{sources}[hp03_059]
\emph{Gorakṣaśataka} 58c–59b
\begin{versinnote}
\tl{baddho yena suṣumṇāyāṃ prāṇas tūḍḍīyate yataḥ |\\+}
\tl{tasmād uḍḍīyaṇākhyo 'yaṃ yogibhiḥ samudāhṛtaḥ ||\\+}
\tl{\var{58c baddho ] \emph{em.~from HP}; vajro G, bandho TU 58d yataḥ ] TU; tataḥ G}\\!}
\end{versinnote}
\end{sources}

\begin{testimonia}[hp03_059]
\emph{Haṭharatnāvalī} 2.53
\begin{versinnote}
\tl{baddho yena suṣumnāyāṃ prāṇas tūḍḍiyate yataḥ |\\+}
\tl{tasmād uḍḍiyānākhyo 'yaṃ yogibhiḥ samudāhṛtaḥ ||\\!}
\end{versinnote}

\emph{Yogacintāmaṇi} f.~76v (attr.~to the \emph{Yogabīja})
\begin{versinnote}
\tl{baddho yena suṣumṇāyāṃ prāṇas tūḍḍīyate yataḥ |\\+}
\tl{tasmād uḍḍiyānākhyo 'yaṃ yogibhiḥ samudāhṛtaḥ ||\\!}
\end{versinnote}

\emph{Yogabīja} 104cd–105ab (South-Indian recension)
\begin{versinnote}
\tl{baddho yena suṣumṇāyāṃ prāṇas tūḍḍīyate tataḥ|\\+}
\tl{tasmād uḍḍīyānākhyo 'yaṃ yogibhiḥ samudāhṛtaḥ||\\!}
\end{versinnote}

\end{testimonia}

%\begin{philcomm}[hp03_059]
%\end{philcomm}

%%%%%%%%%%
\subsection*{3.60}
\begin{translation}[hp03_060]
Because the great bird tirelessly flies up (\emph{uḍḍīnaṃ kurute}), it is [called] 'flying up'. In it, the [root] lock is applied.
\end{translation}

\begin{sources}[hp03_060]
\emph{Vivekamārtaṇḍa} 43
\begin{versinnote}
\tl{uḍḍīṇaṃ kurute yasmād aviśrāntaṃ mahākhagaḥ |\\+}
\tl{uḍḍiyānaṃ tad eva syāt tatra bandho vidhīyate ||\\+}
\tl{\var{43b aviśrāntaṃ ] \emph{em.~from HP}; aviśrānta VTU, aviśrānto G}\\!}
\end{versinnote}
\end{sources}

\begin{testimonia}[hp03_060]
\emph{Haṭharatnāvalī} 2.54
\begin{versinnote}
\tl{uḍḍīnaṃ kurute yasmād aviśrāntaṃ mahākhagaḥ |\\+}
\tl{uḍḍiyānaṃ tad eva syāt tatra bandho 'bhidhīyate ||\\!}
\end{versinnote}

\emph{Yogacintāmaṇi} f.~76v (attr.~to the \emph{Yogabīja})
\begin{versinnote}
\tl{uḍḍīnaṃ kurute yasmād aviśrāntaṃ mahākhagaḥ |\\+}
\tl{uḍḍiyānaṇ tad eva syān mūlabandho 'bhidhīyate ||\\!}
\end{versinnote}

%Skandapurāṇa 4.41.47
%\begin{versinnote}
%\tl{uḍḍīnaṃ kurute yasmād ahorātraṃ mahākhagaḥ |\\+}
%\tl{uḍḍīyānaṃ tataḥ proktaṃ tatra bandho vidhīyate ||\\!}
%\end{versinnote}
\end{testimonia}

\begin{philcomm}[hp03_060]
In the \emph{Vivekamārtaṇḍa} this verse is preceded by a passage on \emph{mūlabandha}, so the likely meaning of the fourth verse quarter is that (\emph{mūla})\emph{bandha} is to be performed in this practice. The \textbeta\ group and many other manuscripts of the \emph{Haṭhapradīpika} have `the root [lock] is applied' (\emph{mūlaṃ vidhīyate}), which appears to be an attempt to clarify the meaning of the original verse.
\end{philcomm}


%%%%%%%%%%
\subsection*{3.61}
\begin{translation}[hp03_061]
 [The yogi] should perform a rearward and upward stretching of the navel into the abdomen. That is the Uḍḍiyāna lock, a lion to the elephant of death.
\end{translation}

\begin{sources}[hp03_061]
\emph{Vivekamārtaṇḍa} 44
\begin{versinnote}
\tl{udare paścime tāṇaṃ nābher ūrdhvaṃ ca kārayet |\\+}
\tl{uḍḍīyāno hy asau bandho mṛtyumātaṅgakesarī ||\\!}
\end{versinnote}
% tāna is not in M-W so prob comes from Marathi. Molesworth:  ताण tāṇa (p. 374)
% ताण tāṇa m (तन S) The state of being stretched or strained; stretchedness or strain (as of a rope, cloth &c.) v दे, बस, भर. 2 fig. Intense anger, a rage, a passion. v ये. Ex. वर्माची गोष्ट काढतांच कसा ताण आला. 3...

\emph{Śivasaṃhitā} 4.73
\begin{versinnote}
\tl{udare paṣcimaṃ tānaṃ nābher ūrdhvaṃ tu kārayet |\\+}
\tl{uḍyānākhyo'tra bandho'yaṃ mṛtyumātaṅgakesarī ||\\!}
\end{versinnote}
\end{sources}

\begin{testimonia}[hp03_061]
\emph{Haṭharatnāvalī} 2.55
\begin{versinnote}
\tl{udare paścimaṃ tānaṃ nābher ūrdhvaṃ ca dhārayet | \\+}
\tl{uḍḍiyāno hy asau bandho mṛtyumātaṅgakesarī || \\!}
\end{versinnote}

\emph{Yogacintāmaṇi} f.~76v (attr.~to the \emph{Yogabīja})
\begin{versinnote}
\tl{udare paścimaṃ tānaṃ nābher ūrdhvaṃ samācaret |\\+}
\tl{uḍḍiyāno hy asau bandho mṛtyumātaṅgakesarī ||\\!}
\end{versinnote}

%Skandapurāṇa 4.41.48
%\begin{versinnote}
%\tl{jaṭhare paścimaṃ tānaṃ nābher ūrdhvaṃ ca dhārayet |\\+}
%\tl{uḍḍīyāno hy ayaṃ bandho mṛtyor api bhayaṃ tyajet || \\!}
%\end{versinnote}
\end{testimonia}

\begin{philcomm}[hp03_061]
The spelling \sl{tāṇa} (where many witnesses have \emph{tāna}) reflects vernacular pronunciation (see e.g.~Molesworth 1857 s.v.~\emph{tāṇa}). 
\end{philcomm}


%%%%%%%%%%
\subsection*{3.62}
\begin{translation}[hp03_062]
\emph{Uḍḍiyāṇa} is easy, but it is always taught by a guru. If he practises it tirelessly, even an old man becomes young.%
\end{translation}

\begin{sources}[hp03_062]
\emph{Dattātreyayogaśāstra} 141c–142b
\begin{versinnote}
\tl{uḍḍiyāṇaṃ tu sahajaṃ guruṇā kathitaṃ sadā |\\+}
\tl{abhyased astatandras tu vṛddho ’pi taruṇo bhavet ||\\+}
\tl{\var{141d guruṇā ] guṇaughāt \pi}\\!}
\end{versinnote}
\end{sources}

\begin{testimonia}[hp03_062]
\emph{Haṭharatnāvalī} 2.56
\begin{versinnote}
\tl{guruṇā sahajaṃ proktaṃ vṛddho 'pi taruṇo bhavet |\\+}
\tl{bāhyoḍyāṇaṃ ca kurute bāhyālaṅkāravardhanam ||\\!}
\end{versinnote}

\emph{Yogacintāmaṇi} f.~76v (attr.~to the \emph{Yogabīja})
\begin{versinnote}
\tl{uḍḍiyānaṃ tu sahajaṃ guruṇā kathitaṃ yathā |\\+}
\tl{abhyaset tad atandras tu vṛddho 'pi taruṇāyate ||\\!}
\end{versinnote}

\emph{Yogabīja} 106cd–107ab (South Indian recension)
\begin{versinnote}
\tl{uḍḍiyāṇaṃ tu sahajaṃ guruṇā kathitaṃ sadā  |\\+}
\tl{abhyased asvatantras tu vṛddho 'pi taruṇo bhavet ||\\!}
\end{versinnote}

\end{testimonia}

\begin{philcomm}[hp03_062]
At the end of the second \emph{pāda} \emph{sadā} is attested by witnesses of \textalpha, \texteta, \textepsilon\ and \delta \ groups, as well as the \emph{Dattātreyayogaśāstra}. We have understood the first line to mean that even though the basics of the practice of \emph{uḍḍiyāna} are easy, it still needs to be taught by the guru. Some witnesses read \emph{yathā} for \emph{sadā}, perhaps as a deliberate substitution of the more difficult \emph{sadā}, making the verse mean that \emph{uḍḍiyāna} is easy in the way that is taught by the guru. 
%The reading \emph{astatandraḥ} is probably the original one in the \emph{Dattātreyayogaśāstra} and is attested in some manuscripts of the \emph{Haṭhapradīpikā}. However, it is not strongly supported by the witnesses of the important groups and so may have changed to something simpler, such as \emph{tad atandras tu}.
%  JB: collate J5 reading astatandras tu, so that last note is not necessary
\end{philcomm}

\begin{metre}[hp03_062]
Anuṣṭubh (a: na-vipulā)
\end{metre}

%%%%%%%%%%
\subsection*{3.63}
\begin{translation}[hp03_063]
[The yogi] should carefully stretch [the region of the abdomen] above and below the navel. If he practises [like this] for six months, he is sure to conquer death.
\end{translation}

\begin{sources}[hp03_063]
\emph{Dattātreyayogaśāstra} 142c–143b
\begin{versinnote}
\tl{nābher ūrdhvam ataḥ paścāt tānaṃ kuryāt prayatnataḥ || 142 ||\\+}
\tl{ṣaṇmāsam abhyasen mṛtyuṃ jayaty eva na saṃśayaḥ |\\+}
\tl{\var{142c ataḥ paścāt ] PT, adhaś cāpi \emph{cett.}}\\!}
\end{versinnote}

Cf.~\emph{Śivasaṃhitā} 4.72
\begin{versinnote}
\tl{nābher ūrdhvam adhaś cāpi tānaṃ paścimam ācaret |\\+}
\tl{uḍyānabandha eṣaḥ syāt sarvaduḥkhaughanāśanaḥ  ||\\!}
\end{versinnote}
\end{sources}

\begin{testimonia}[hp03_063]
\emph{Haṭharatnāvalī} 2.57
\begin{versinnote}
\tl{nābher ūrdhvam adho vāpi tānaṃ kuryāt prayatnataḥ |\\+}
\tl{ṣaṇmāsam abhyasen mṛtyuṃ jayaty eva na saṃśayaḥ ||\\!}
\end{versinnote}

\emph{Yogacintāmaṇi} f.~76v (attr.~to the \emph{Yogabīja})
\begin{versinnote}
\tl{nābher ūrdhvam adho vāpi tānaṃ kuryāt prayatnataḥ |\\+}
\tl{ṣaṇmāsam abhyasen mṛtyuṃ jayaty eva na saṃśayaḥ ||\\!}
\end{versinnote}

Cf.~\emph{Yuktabhavadeva} f.~76v (attr.~to the \emph{Śivayoga})
\begin{versinnote}
\tl{nābher ūrdhvam adhaś cāpi tānaṃ nirbharam ācaret |\\+}
\tl{uḍḍiyāno hy ayaṃ bandhaḥ sarvaduḥkhaughanāśanaḥ ||\\!}
\end{versinnote}

\end{testimonia}

%\begin{philcomm}[hp03_063]
%\end{philcomm}


%%%%%%%%%%
\subsection*{3.64}
\begin{translation}[hp03_064]
%Sitting in \emph{vajrāsana}, [the yogi] should hold his feet firmly with his hands. And [now] the bulb is in the region of the ankles. He should press on it.
Sitting in \emph{vajrāsana}, [the yogi] should hold his feet firmly with his hands near the region of the ankles and press the bulb (\emph{kanda}) there.
\end{translation}

\begin{sources}[hp03_064]
\emph{Gorakṣaśataka} 59c–60b
\begin{versinnote}
\tl{sati vajrāsane jānū karābhyāṃ dhārayed dṛḍham ||\\+}
\tl{gulphadeśasamīpe ca kandaṃ tatra prapīḍayet |\\+}
\tl{\var{59c jānū ] em.; pādau GU, jānu T, prādau V}\\!}
\end{versinnote}
\end{sources}

\begin{testimonia}[hp03_064]
\emph{Yogacintāmaṇi} f.~76v (attr.~to the \emph{Haṭhapradīpikā})
\begin{versinnote}
\tl{sati vajrāsane pādau karābhyāṃ dhārayed dṛḍham |\\+}
\tl{gulphadeśasamīpe ca udaraṃ tat prapīḍayet ||\\!}
\end{versinnote}

Cf.~\emph{Yuktabhavadeva} 7.224 (commenting on \emph{uḍḍiyānabandha})
\begin{versinnote}
\tl{dṛḍham āsanaṃ baddhvā gulphadeśasamīpe karābhyāṃ pādau datvā nābhisamīpasthaṃ kandaṃ pīḍayann udare paścimatāṇaṃ tathā kuryād yathā vāyuḥ kukṣisandhiṃ na gacchaty evam uḍḍiyānabandho jarāmṛtyuvināśanaḥ sampadyate ||\\!}
\end{versinnote}

\emph{Haṭhasaṅketacandrikā} (ms.~no.~2244) f.~36r
\begin{versinnote}
\tl{tathā coktaṃ haṭhapradīpikāyām |\\+}
\tl{sati vajrāsane pādau karābhyāṃ dharayed dṛḍhaṃ |\\+}
\tl{gulphadeśasamīpe ca kandaṃ tatra nipīḍayet ||... \\+}
\tl{siddhāsane sthitvā hastābhyāṃ pādau gulphapradeśasamīpe dṛḍhaṃ dhṛtvā tunde nālotthāna[ṃ] sādhu vidhāya samāhitamanasā sudṛḍhamūlabandhajālandharabandhavatābhyāsinā sādhakena recakādau kuṃbhakānte udare paścimatāne kriyamāṇe nitarāṃ tadā tatra nābhikandanipīḍane paścimatānena sati nābhikandotthānāḍaya urdhvamukhā vikasitā viralā vimalā asaṃhatā vāyugrahasamarthā bhavanti tadā sakuṃbhitaḥ prāṇavāyuḥ śanaiḥ [||]\\!}
\end{versinnote}
\end{testimonia}

\begin{philcomm}[hp03_064]
In \emph{Haṭhapradīpikā} this verse seems to instruct the yogi to press the bulb (\emph{kanda}) with the feet while holding them with the hands. The adopted reading of the source text, the \emph{Gorakṣaśataka}, which is only found in witness T (and there in the singular \emph{jānu}, which has been emended to the dual \emph{jānū} in Mallinson's edition) says that it is the knees that are to be held, which would still allow for the feet to press the bulb. In the \emph{Yuktabhavadeva} (7.224), Bhavadevamiśra says that the bulb is near the navel and the legs are held near the ankles, suggesting that the bulb is pressed by using the hands to pull the feet into the lower abdomen. Holding the ankles with the hands and pressing the \emph{kanda} with the feet is also the view of Brahmānanda (\emph{Jyotsnā} 3.114), who follows the \emph{Yogayājñavalkya} (4.14, 4.16) in thinking that the place of the \emph{kanda} is nine fingerbreadths above the middle of the body, which is two fingerbreaths above the anus (\emph{Jyotsnā} 3.113). In the \emph{Haṭhasaṅketacandrikā} (f.~36r), Sundaradeva explains that the yogi presses the \emph{kanda} in the navel by performing \emph{uḍḍiyānabandha}, along with the root and chin locks, at the end of \emph{kumbhaka} and the beginning of exhalation. It is thus the backward stretch in the abdomen (\emph{udare paścimatāna}) that presses the \emph{kanda} in the navel.

%M3 and G5 (gr 4c)
\end{philcomm}


%%%%%%%%%%
\subsection*{3.65}
\begin{translation}[hp03_065]
[The yogi] should very gently stretch back his stomach, chest and †neck† in such a way that the breath does not come into contact with the stomach.
\end{translation}

\begin{sources}[hp03_065]
\emph{Gorakṣaśataka} 60c–61b
\begin{versinnote}
\tl{paścimaṃ tānam udare dhārayed dhṛdaye gate |\\+}
\tl{śanaiḥ śanair yathā prāṇas tundasaṃdhiṃ na gacchati ||\\+}
\tl{\var{60d dhārayedd°] GU; kārye*raṃ*T, kuryāñ ca V dhṛdaye gate] \emph{em.}~Sathyanarayanan; dhṛdaye gale GUT, civukaṃ hṛdi V}\\!}%gatam ? conj. em. hṛdayaṃ gataṃ? or hṛdayaṃ gate, with gate agreeing with udare? gate is found in several HP mss. One instance of hṛdayaṃ gataḥ.
\end{versinnote}
\end{sources}

\begin{testimonia}[hp03_065]
\emph{Yogacintāmaṇi} f.~76v (attr.~to the \emph{Haṭhapradīpikā})
\begin{versinnote}
\tl{paścimaṃ tānam udare kārayec cibukaṃ hṛdi |\\+}
\tl{śanaiḥ śanair yathā prāṇas tundasiddhiṃ na gacchati ||\\!}
\end{versinnote}

\emph{Haṭhsaṅketacandrikā}, f.~36r–36v (attr.~to the \emph{Haṭhapradīpikā})
\begin{versinnote}
\tl{paścime tānam udare ku[r]yac c[a] cibukaṃ hṛdi |\\+}
\tl{śanaiḥ śanair yathā prāṇaḥ kandasaṃdhi[ṃ] nigacchati ||\\+}
\tl{yathābhyāsānurūpaṃ kandasaṃdhiṃ kandagatanāḍisaṃghavimalavivaraprānteṣu nigachati nitarāṃ gacchati tathā tathā kuṃbhakavṛddhiḥ sukharūpodbhavati sādhakasya yenoḍḍīyānena bandhena vāyuḥ proḍḍīyāste brahmanāḍyāṃ yato 'sau uḍḍīyānākhyaḥ smṛto bandha ārthaḥ sevyas tasmād yogibhiḥ siddhasevyaḥ ||\\!}
\end{versinnote}

\emph{Yogaprakāśikā} 5.96
\begin{versinnote}
\tl{paścimaṃ tānam udare kārayec cibukaṃ hṛdi |\\+}
\tl{śanaiḥ śanairyathā prāṇaḥ skandhasaṅge na gacchati ||\\+}
\tl{uktalakṣaṇe vajrāsane baddhe satī gulphadeśasamīpena meḍhraṃ vā prapīḍayet cibukaṃ hṛdi kṛtvā prāṇasyordhvasañcalanaṃ kārayet tena prāṇ[a]ḥ skandhasandhiṃ gacchatīty arthaḥ || \\!}
\end{versinnote}
\end{testimonia}

\begin{philcomm}[hp03_065]
In 3.65b, the reading \emph{gale} (`in the neck') is very well attested by manuscripts of the source text, the \emph{Gorakṣaśataka}, and the \emph{Hathapradīpikā} (including all three \textalpha\ witnesses). Its meaning is not entirely clear to us as the `backward stretch' (\emph{paścimaṃ tānam}) usually occurs above and below the navel when the \emph{uḍḍiyāna} lock is applied, as stated above in verse 3.61. In 3.65, the mention of the stretch in the chest (\emph{hṛdaya}) may also be consistent with 3.61 in so far as `above the navel' might include the lower region of the chest. Drawing on x-ray experiments on \emph{uḍḍiyānabandha} conducted at the Kaivalyadhama Yoga Institute and published in the \emph{Yoga Mīmāṃsā} Journal (e.g.~vol.1, issues 1-2), Dr M.\,M.\,Gore (2005:144) mentions a sub-atmospheric (negative) pressure in visceral cavities, such as the oesophagus and stomach, as a physiological effect of applying \emph{uḍḍiyāna}. So, it may be possible that a `backward stretch' in the throat was intended in 3.65b. However, we have not seen the neck mentioned in this regard in any other premodern work and the absence of \emph{ca} suggests that \emph{gale} may be a corruption. The alternative reading \emph{cibukaṃ hṛdi} in manuscripts of the \emph{Hathapradīpikā} on lower branches of the stemma and in the testimonia is a reference to \emph{jālandharabandha} and appears to be a patch. In his edition of the \emph{Gorakṣaśataka}, Mallinson has adopted the emendation \emph{gate} suggested by Dr Sathyanaryanan, which he understands to mean that the rearward stretch reaches as far as the heart. 
\end{philcomm}
% `Anatomy and Physiology of Yogic Practices,' Dr. Makarand Madhukar Gore. New Delhi: Shri Jainendra Press. 2005.
% Remove crux marks in the edition as gale is not gibberish

\begin{metre}[hp03_065]
Anuṣṭubh (a: na-vipulā)
\end{metre}

%%%%%%%%%%
\subsection*{3.66}
\begin{translation}[hp03_066]
\emph{Uḍḍiyāna} is the best of all the locks. When the \emph{uḍḍiyāna} lock is firm, liberation becomes easy.
\end{translation}

%\begin{sources}[hp03_066]
%\end{sources}

\begin{testimonia}[hp03_066]
\emph{Yogacintāmaṇi} f.~76v (attr.~to the \emph{Haṭhapradīpikā})
\begin{versinnote}
\tl{sarveṣām eva bandhānāṃ hy uttamo hy uḍḍiyānakaḥ |\\+}
\tl{uḍḍiyāne dṛḍhe bandhe mūlaḥ svābhāviko bhavet ||\\!}
\end{versinnote}

\emph{Yogaprakāśikā} 5.97
\begin{versinnote}
\tl{sarveṣām eva bandhānām uttamo hy uḍḍiyāṇakaḥ |\\+}
\tl{uḍḍiyāṇe dṛḍhe bandhe mūlaṃ svābhāvikaṃ bhavet || \\+}
\tl{uḍḍīyānabandham upasamharati sarveṣām iti || mūlam iti mūlabandho 'nāyāsena sidhyatīty arthaḥ ||\\!}
\end{versinnote}
\end{testimonia}

%\begin{philcomm}[hp03_066]
%The reading \textit{mukti} is very well attested (alpha group, etc.) and often occurs with \textit{svābhāvikī}.
%The reading \textit{mūla} also makes sense as it elucidates on the relationship between these two locks.
% uḍḍiyāne kṛte bandhe mūlaṃ svābhāvikaṃ bhavet (Gr2,Gr3)
% uḍḍiyāṇe dṛḍhe bandhe muktiḥ svābhāvikī bhavet
%[JB: no issue now as alpha supports the readings we have adopted]
%\end{philcomm}


%%%%%%%%%%
\subsection*{3.67 heading}
\begin{translation}[hp03_067a]
Now the \sl{jālandhara} lock:
\end{translation}

% \begin{philcomm}[hp03_067a]
% \end{philcomm}

%%%%%%%%%%
\subsection*{3.67}
\begin{translation}[hp03_067]
{}[The yogi] should contract the throat and firmly place the chin on the chest. This is the lock called \sl{jālandhara}. It prevents loss of the nectar of immortality.
\end{translation}

\begin{sources}[hp03_067]
\emph{Dattātreyayogaśāstra} 138
\begin{versinnote}
\tl{kaṇṭham ākuñcya hṛdaye sthāpayec cibukaṃ dṛḍham |\\+}
\tl{jālandharo bandha eṣa amṛtāvyayakārakaḥ ||\\+}
\tl{\var{138b sthāpayec cibukaṃ dṛḍham] HRPTPT; sthāpayed dṛḍhayā dhiyā YTU, sthāpayed dṛḍham icchayā \emph{cett.}}\\!}
\end{versinnote}
\end{sources}

\begin{testimonia}[hp03_067]
\emph{Haṭharatnāvalī} 2.66
\begin{versinnote}
\tl{kaṇṭham ākuñcya hṛdaye sthāpayec cibukaṃ dṛḍham |\\+}
\tl{bandho jālandharākhyo 'yaṃ jarāmṛtyuvināśakaḥ ||\\!}
\end{versinnote}

\emph{Yogacintāmaṇi} f.~77r (attr.~to the \emph{Yogabīja})
\begin{versinnote}
\tl{kaṇṭham ākuñcya hṛdaye sthāpayed dṛḍham icchayā |\\+}
\tl{bandho jālandharākhyo 'yam amṛtāvyayakārakaḥ ||\\!}
\end{versinnote}

\emph{Yogabīja} 109 (south Indian recension)
\begin{versinnote}
\tl{kaṇṭham ākuñcya hṛdaye sthāpayed dṛḍham icchayā |\\+}
\tl{bandho jālandharākhyo 'yam amṛtāvyayakārakaḥ ||\\+}
\tl{\var{amṛtāvyaya°] amṛtavyaya° \vl}\\!}
\end{versinnote}

\end{testimonia}

\begin{philcomm}[hp03_067]
Manuscripts of the \textalpha, \textbeta, \texteta\ and delta groups have \emph{sthāpayed dṛḍham icchayā} (`one should place it firmly as desired') in the second verse quarter, which is also well-attested in the transmission of the source text, the \emph{Dattātreyayogaśāstra}. This reading seems secondary because, in a subsequent verse (3.69), contracting the throat is the main feature of \emph{jālandharabandha}, so it seems contradictory to say that it may be done `as one likes' in 3.67b. The word \emph{icchayā} may have crept in to this verse because someone wanted to make this practice optional in light of 3.22, or it might be a corruption of \emph{hṛdaye sthāpayed dṛḍhaṃ niścayam} which is found in some other manuscripts.
\end{philcomm}

\begin{metre}[hp03_067]
Anuṣṭubh (a: na-vipulā)
\end{metre}

%%%%%%%%%%
\subsection*{3.68}
\begin{translation}[hp03_068]
Because it binds all the channels in which the liquid from the void flows down it is [called] the \sl{jālandhara} lock. It gets rid of all problems in the throat.%
\end{translation}

\begin{sources}[hp03_068]
\emph{Vivekamārtaṇḍa} 45
\begin{versinnote}
\tl{badhnāti hi śirājālam adhogāminabhojalam |\\+}
\tl{tato jālandharo bandhaḥ kaṇṭhaduḥkhaughanāśanaḥ ||\\!}
\end{versinnote}
\end{sources}

\begin{testimonia}[hp03_068]
\emph{Haṭharatnāvalī} 2.66ef–2.67ab
\begin{versinnote}
\tl{badhnāti hi śirājālaṃ nādho yāti nabhojalam |\\+}
\tl{tato jālandharo bandhaḥ kaṇṭhasaṅkocane kṛte  ||\\!}
\end{versinnote}

\emph{Yogacintāmaṇi} f.~77r (attr.~to the \emph{Haṭhapradīpikā})
\begin{versinnote}
\tl{badhnātīha śirājālam adhogāminabhojalam |\\+}
\tl{tato jālandharaḥ proktaḥ kaṇṭhe duḥkhaughanāśanaḥ ||\\!}
\end{versinnote}

\emph{Yuktabhavadeva} 7.230 (attr.~to the \emph{Śivayoga})
\begin{versinnote}
\tl{badhnāti hi śirājālaṃ nādho yāti nabhojalam |\\+}
\tl{tato jālandharo bandhaḥ kaṇṭhaduḥkhaughanāśanaḥ ||\\!}
\end{versinnote}

%Skandapurāṇa 4.41.149
%\begin{versinnote}
%\tl{badhnāti hi śirājālam adhogāmi nabhojalam |\\+}
%\tl{eṣa jālandharo bandhaḥ kaṇṭhe duḥkhaughanāśanaḥ || \\!}
%\end{versinnote}
\end{testimonia}

%\begin{philcomm}[hp03_068]
%Judit: Read adhogāminabhojalam in which the fluid of the void flows down.
% JB: I've tweaked the translation to reflect Judit's suggestion.
%\end{philcomm}


%%%%%%%%%%
\subsection*{3.69}
\begin{translation}[hp03_069]
When the \sl{jālandhara} lock is performed, its defining feature being the contraction of the throat, nectar does not fall in the fire and the breath does not escape.
\end{translation}

\begin{sources}[hp03_069]
\emph{Vivekamārtaṇḍa} 46
\begin{versinnote}
\tl{jālandhare kṛte bandhe kaṇṭhasaṃkocalakṣaṇe |\\+}
\tl{na pīyūṣaṃ pataty agnau na ca vāyuḥ pradhāvati ||\\+}
\tl{\var{pradhāvati ] ; prakupyati AGT}\\!}
\end{versinnote}
\end{sources}

\begin{testimonia}[hp03_069]
\emph{Yogacintāmaṇi} f.~77v (attr.~to the \emph{Haṭhapradīpikā})
\begin{versinnote}
\tl{jālandhare kṛte bandhe kaṇṭhasaṃkocalakṣaṇe |\\+}
\tl{na pīyūṣaṃ pataty agnau na ca vāyuḥ prakupyati ||\\!}
\end{versinnote}

\emph{Yuktabhavadeva} 7.231 (attr.~to the \emph{Śivayoga})
\begin{versinnote}
\tl{jālandhare kṛte bandhe kaṇṭhasaṃkocalakṣaṇe |\\+}
\tl{na pīyūṣaṃ pataty agnau na ca vāyuḥ prakupyati ||\\!}
\end{versinnote}
\end{testimonia}

%\begin{philcomm}[hp03_069]
%\end{philcomm}


%%%%%%%%%%
\subsection*{3.70}
\begin{translation}[hp03_070]
By contracting the throat, the yogi firmly blocks the two channels. This should be known as the middle cakra, which binds [the mind to] the sixteen supports [in the body].
\end{translation}

%\begin{sources}[hp03_070]
%\end{sources}

\begin{testimonia}[hp03_070]
\emph{Yogaviṣaya} 19ab % probably not a source text as it seems to be a fragment
\begin{versinnote}
\tl{kaṇṭhasaṃkocanaṃ kṛtvā dve nāḍyau stambhayed dṛḍham |\\+}
\tl{rasanāpīḍyamānās tu ṣoḍaśaś cordhvagāminī ||\\!}
\end{versinnote}

\emph{Yogacintāmaṇi} f.~77v (attr.~to the \emph{Haṭhapradīpikā})
\begin{versinnote}
\tl{madhyacakram idaṃ jñeyaṃ ṣoḍaśādhārabandhanam |\\!}
\end{versinnote}

\emph{Yogakarṇikā} 85
\begin{versinnote}
\tl{kaṇṭhasaṅkocanenaiva dve nāḍye kumbhayed dṛḍham |\\!}
\end{versinnote}

\emph{Yogasarasaṅgraha} p.58
\begin{versinnote}
\tl{kaṇṭhasaṃkocanekaṇṭhasaṃkocane nanaiva dvināḍyo stambhayed dṛḍham~|\\+}
\tl{ayaṃ bandho mayā proktaḥ ṣoḍaśādhārabandhanam ||\\!}
\end{versinnote}

\emph{Haṭhayogasaṃhitā} p. 23
\begin{versinnote}
\tl{kaṇṭhasaṅkocanaṃ kṛtvā cibukaṃ hṛdaye nyaset |\\+}
\tl{jālandhare kṛte bandhe ṣoḍaśādhārabandhanam ||\\!}
\end{versinnote}
\end{testimonia}

\begin{philcomm}[hp03_070]
The import of the second line of this verse is obscure to us. In \emph{Jyotsnā} 3.73, Brahmānanda thinks that the middle cakra (\emph{madhyacakra}) is \emph{viśuddha} cakra. The main reason for this appears to be that this cakra is located in the throat and the salient feature of the \sl{jālandhara} lock is contracting the throat. However, he also seems to connect the \emph{viśuddha} cakra to the sixteen supports (\emph{ṣoḍaśādhāra}) at the end of this verse, perhaps because this cakra has sixteen petals (as mentioned in 3.47). On the meaning of \emph{ādhāra} in yogic contexts, see entry no.~3 in the \emph{Tāntrikābhidhānakośa} vol.~1 2000: 191.
%Jyotsnā: kaṇṭhasaṃkocaneneti. dṛḍham gāḍhaṃ kaṇṭhasaṃkocanenaiva kaṇṭhasaṃkocanamātreṇa dve nāḍyau iḍāpiṅgale stambhayed bandhayet. ayaṃ jālandhara iti kartṛpadādhyāhāraḥ. idaṃ kaṇṭhasthāne sthitaṃ viśuddhākhyaṃ cakram madhyacakram madhyamaṃ cakram jñeyam | kīdṛśam ? ṣoḍaśādhārabandhanam ṣoḍaśasaṅkhyākā ye ādhārā aṅguṣṭhādhārādibrahmarandhrāntās teṣāṃ bandhanaṃ bandhanaprakāram |
%Yogaprakāśikā: uktalakṣaṇaṃ nāḍīstambhanaṃ madhyacakrasaṃjñakaṃ bhavati || siddhasiddhāntapaddhatau pratipāditānāṃ ṣoḍaśādhārāṇāṃ nirodhane sādhanaṃ ca ||
\end{philcomm}


%%%%%%%%%%
\subsection*{3.71}
\begin{translation}[hp03_071]
This triad of locks is the best [and] has been practised by the great Siddhas. Yogis know it to be a method of all systems of Haṭha.
\end{translation}

%\begin{sources}[hp03_071]
%\end{sources}

\begin{testimonia}[hp03_071]
\emph{Haṭharatnāvalī} 2.68
\begin{versinnote}
\tl{bandhatrayam idaṃ śreṣṭhaṃ mahāsiddhaiś ca sevitam |\\+}
\tl{sarveṣāṃ yogatantrāṇāṃ sādhanaṃ yogino viduḥ ||\\!}
\end{versinnote}

\emph{Yogacintāmaṇi} f.~77v (attr.~to the \emph{Haṭhapradīpikā})
\begin{versinnote}
\tl{bandhatrayam idaṃ śreṣṭhaṃ mahāsiddhaniṣevitam |\\+}
\tl{sarveṣāṃ haṭhatantrāṇāṃ sādhane yoginām iti ||\\!}
\end{versinnote}

\emph{Haṭhatattvakaumudī} 15.24
\begin{versinnote}
\tl{idaṃ bandhatrayaṃ śreṣṭhaṃ marujjayasusiddhadam |\\+}
\tl{sarveṣāṃ yogatantrāṇāṃ sādhanaṃ yogino viduḥ ||\\!}
\end{versinnote}

%Yogakarṇikā 85
%\begin{versinnote}
%\tl{bandhatrayam idaṃ guhyaṃ mahāsiddhiniṣevitam || \\!}
%\end{versinnote}
\end{testimonia}

%\begin{philcomm}[hp03_071]
%\end{philcomm}


%%%%%%%%%%
\subsection*{3.71*1}
\begin{translation}[hp03_071_1]
By immediately contracting the lower [part of the body] (i.e.~by the root lock) when the neck has been contracted (i.e.~by the \emph{jālandhara} lock) and by stretching the abdomen backwards in the middle [of the body] (i.e.~by the \emph{uḍḍiyāna} lock), the breath enters the channel of Brahman.
\end{translation}
% greyscaled as alpha

\begin{sources}[hp03_071_1]
\emph{Gorakṣaśataka} 63 (see 2.46)
\end{sources}

\begin{testimonia}[hp03_071_1]
\emph{Haṭharatnāvalī} 2.8, \emph{Yogacintāmaṇī} f.80r, \emph{Yuktabhavadeva} 7.95 and \emph{Haṭhatattvakaumudī} 15.25–27 (see 2.46).

\end{testimonia}

%\begin{philcomm}[hp03_071_1]
%In the second chapter, Group 2 has the reading \emph{kuñcanenāśu kaṇṭhasaṃkocane kṛte}.
%\end{philcomm}


%%%%%%%%%%
\subsection*{3.72}
\begin{translation}[hp03_072]
[The yogi] should contract the place of the root  and do the \emph{uḍḍiyāna} [lock]. He should [then] block the Iḍā and Piṅgalā [channels] and make [the breath] flow in the rear pathway.
\end{translation}

%\begin{sources}[hp03_072]
%\end{sources}

\begin{testimonia}[hp03_072]
\emph{Haṭharatnāvalī} 2.70
\begin{versinnote}
\tl{mūlasthānaṃ samākuñcya uḍḍiyānaṃ tu kārayet |\\+}
\tl{iḍāṃ ca piṅgalāṃ baddhvā vāhayet paścimaṃ pathaṃ ||\\!}
\end{versinnote}

\emph{Yogacintāmaṇi} f.~79v (attr.~to the \emph{Haṭhapradīpikā})
\begin{versinnote}
\tl{mūlasthānaṃ samākṛṣya uḍḍiyānaṃ tu kārayet |\\+}
\tl{iḍāṃ ca piṅgalāṃ baddhvā vāhayet paścime pathi ||\\!}
\end{versinnote}

\emph{Haṭhatattvakaumudī} 15.25–27
\begin{versinnote}
\tl{mūlasthānaṃ samākuñcya uḍyānaṃ tu kārayet |\\+}
\tl{iḍāṃ ca piṃgalāṃ baddhvā vāhayet paścimāpathaṃ ||\\+}
\tl{mūlasthānaṃ mārgasaṃkocanaṃ vidhāya uḍyānam udarasaṃkocanaṃ tataḥ | iḍāṃ piṃgalāṃ baddhvā kaṇṭhasaṃkocanena paścimapathaṃ pṛṣṭhavaṃśamārge pavanaṃ vāhayet kuryāt ||\\!}
\end{versinnote}
\end{testimonia}


%\begin{philcomm}[hp03_072]
%No known source?
%\end{philcomm}


%%%%%%%%%%
\subsection*{3.73}
\begin{translation}[hp03_073]
By this method alone, the breath attains dissolution. Then death does not arise nor old age, disease and the like.
\end{translation}

\begin{sources}[hp03_073]
\end{sources}

\begin{testimonia}[hp03_073]
\emph{Haṭharatnāvalī} 2.71
\begin{versinnote}
\tl{anenaiva vidhānena prayāti pavano layam |\\+}
\tl{tato na jāyate mṛtyur jarārogādikaṃ tathā || \\!}
\end{versinnote}

\emph{Yogacintāmaṇi} f.~79v (attr.~to the \emph{Haṭhapradīpikā})
\begin{versinnote}
\tl{anenaiva vidhānena sevayet pavanālayam |\\+}
\tl{tato na jāyate mṛtyur jarārogādikaṃ tathā ||\\!}
\end{versinnote}

\end{testimonia}

%\begin{philcomm}[hp03_073]
%The \textalpha, \textbeta\ and delta groups have the reading \emph{sevayet pavanālayam} as the second verse quarter of this verse. It renders the meaning, `by this method alone, one should honour the abode of the breath.' As far as we know, the compound \emph{pavanālaya} does not occur in other yoga texts. The similar compound \emph{prāṇālaya}) is mentioned in other yoga texts, such as the \emph{Yogayājñavalkya} (4.52–53), but it refers to the locations in the body where \emph{prāṇa} resides, as opposed to the other bodiy winds.
% JB: reconsider the alpha reading, which is well attested in other groups? Maybe \emph{sevayet pavanālayam} is just some general laudatory statement?
%Judit: ādikam is okay.
%\end{philcomm}



%%%%%%%%%%
\subsection*{3.74 heading}
\begin{translation}[hp03_074a]
Now the inverted bodily position:
\end{translation}

% \begin{philcomm}[hp03_074a]
% \end{philcomm}

%%%%%%%%%%
\subsection*{3.73*1}
\begin{translation}[hp03_073_1]
The sun devours whatever nectar flows from the divine moon. As a result, the body is afflicted by old age.
\end{translation}
%3.75-76 gray-scaled.

%\begin{sources}[hp03_073_1]
%\end{sources}

\begin{testimonia}[hp03_073_1]
\emph{Haṭharatnāvalī} 2.72 
\begin{versinnote}
\tl{atha viparītakaraṇī–\\+}
\tl{yat kiñ cit sravate candrād amṛtaṃ divyarūpi ca |\\+}
\tl{tatsarvaṃ grasate sūryas tena piṇḍaṃ vināśi ca ||\\!}
\end{versinnote}
 
\emph{Yogacintāmaṇi} f.~77v
\begin{versinnote}
\tl{haṭhapradīpikāyām—\\+}
\tl{yat kiṃ cin sravate candrād amṛtaṃ divyarūpi ca |\\+}
\tl{tat sarvaṃ grasate sūryas tena piṇḍaṃ vināśi ca ||\\!}
\end{versinnote}

Cf.~\emph{Haṭhayogasaṃhitā} 38 (p. 26)
\begin{versinnote}
\tl{nābhimūle vaset sūryas tālumūle ca candramāḥ |\\+}
\tl{amṛtaṃ grasate sūryas tato mṛtyuvaśo naraḥ ||\\!}
\end{versinnote}
\end{testimonia}

\begin{philcomm}[hp03_073_1]
%No source?
%The term \emph{piṇḍa} is rarely neuter, hence the change to the masculine in the \emph{Jyotsnā}.
%Both ca-s are infelicitous J5 has jīryate for vināśi ca
The \textalpha\ group do not have 3.73*1 and 3.73*2 in the third chapter (but rather in the fourth) and other manuscripts omit them as well (notably the \texteta\ group). It appears that they have been inserted at the beginning of the section on \emph{viparītakaraṇī} as a kind of preamble, which is unusual as the other techniques in this chapter do not have such introductions.

%are found in alpha in ch.4, but should they be there? They are very physical and not suited to ch.4. But here they are unusual because they are preamble and the practices already described do not have preamble. Perhaps drop them altogether. But then there is no explanation of how VK works.

%
\end{philcomm}



%%%%%%%%%%
\subsection*{3.73*2}
\begin{translation}[hp03_073_2]
There is a divine bodily position for this, which blocks the mouth of the sun. It is to be known from the teaching of a guru and not through the countless interpretations of scriptures.
\end{translation}
% greyscaled

\begin{sources}[hp03_073_2]
\end{sources}

\begin{testimonia}[hp03_073_2]
Haṭharatnāvali 2.73 (on \emph{viparītakaraṇī})
\begin{versinnote}
\tl{tatrāsti divyaṃ karaṇaṃ sūryasya mukhabandhanam |\\+}
\tl{gurūpadeśato jñeyaṃ na tu śāstrārthakoṭibhiḥ || \\!}
\end{versinnote}

\emph{Yogacintāmaṇi} f.~77v (attr.~to the \emph{Haṭhapradīpikā})
\begin{versinnote}
\tl{tatrāsti karaṇaṃ divyaṃ sūryasya mukhabandhanam |\\+}
\tl{gurūpadeśato jñeyaṃ na tu śāstrārthakoṭibhiḥ ||\\!}
\end{versinnote}

%Yogakarṇikā 147c–148b
%\begin{versinnote}
%\tl{tat sarvaṃ grasate sūryas tena piṇḍo jarāyutaḥ || 147 ||\\+}
%\tl{tatrāsti karaṇaṃ divyaṃ sūryasya mukhavañcanam |\\!}
%\end{versinnote}
\end{testimonia}

%\begin{philcomm}[hp03_073_2]
%No source? Only Brahmānanda reads mukhavañcanam, which makes better sense. Judit; bandhanaṃ makes sense with grasate sūryas
%\end{philcomm}



%%%%%%%%%%
\subsection*{3.74}
\begin{translation}[hp03_074]
The navel is up, the palate down; the sun up, the moon down: the bodily position called “inverted” is obtained through the teaching of a guru.
\end{translation}
% 

\begin{sources}[hp03_074]
\emph{Vivekamārtaṇḍa} 115
\begin{versinnote}
\tl{ūrdhvaṃ nābhir adhas tālur ūrdhvaṃ bhānur adhaḥ śaśī |\\+}
\tl{karaṇī viparītākhyā guruvākyena labhyate ||\\!}
\end{versinnote}
\end{sources}

\begin{testimonia}[hp03_074]
\emph{Haṭharatnāvalī} 2.74
\begin{versinnote}
\tl{ūrdhvaṃ nābhir adhas tālur ūrdhvaṃ bhānur adhaḥ śaśī |\\+}
\tl{karaṇī viparītākhyā guruvākyena labhyate ||\\!}
\end{versinnote}

\emph{Yogacintāmaṇi} f.~73r (attr.~to the \emph{Haṭhapradīpikā})
\begin{versinnote}
\tl{ūrdhvanābhir adhastālur ūrdhvabhānur adhaḥśaśī |\\+}
\tl{karaṇī viparītākhyā sarvavyādhivināśinī ||\\!}
\end{versinnote}

\emph{Yuktabhavadeva} 7.236 (attr.~to Gorakṣanātha)
\begin{versinnote}
\tl{ūrdhvaṃ nābhir adhas tālur ūrdhvaṃ bhānur adhaḥ śaśī |\\+}
\tl{karaṇī viparītākhyā guruvaktreṇa gamyate  ||\\!}
\end{versinnote}
\end{testimonia}

%\begin{philcomm}[hp03_074]
%\end{philcomm}


%%%%%%%%%%
\subsection*{3.75}
\begin{translation}[hp03_075]
The bodily position called “inverted" destroys all diseases. For [the yogi] who regularly engages in [its] practice, it increases the digestive fire.
\end{translation}

\begin{sources}[hp03_075]
\emph{Dattātreyayogaśāstra} 146
\begin{versinnote}
\tl{karaṇaṃ viparītākhyaṃ sarvavyādhivināśanam |\\+}
\tl{nityam abhyāsayuktasya jaṭharāgnir vivardhate ||\\!}
\end{versinnote}
\end{sources}

\begin{testimonia}[hp03_075]
\emph{Haṭharatnāvalī} 2.75
\begin{versinnote}
\tl{karaṇī viparītākhyā sarvavyādhivināśinī |\\+}
\tl{nityam abhyāsayuktasya jaṭharāgnivivardinī  ||\\!}
\end{versinnote}

\emph{Yogacintāmaṇi} f.~78r (attr.~to Dattātreya)
\begin{versinnote}
\tl{nityam abhyāsayuktasya jaṭharāgnivivardhanam ||\\!}
\end{versinnote}
\end{testimonia}

%\begin{philcomm}[hp03_075]
%\end{philcomm}


%%%%%%%%%%
\subsection*{3.76}
\begin{translation}[hp03_076]
A lot of food should be provided for the practitioner. If the practitioner eats little, the fire will quickly consume his body.
\end{translation}

\begin{sources}[hp03_076]
\emph{Dattātreyayogaśāstra} 147
\begin{versinnote}
\tl{āhāro bahulas tasya saṃpādyaḥ sāṃkṛte dhruvam |\\+}
\tl{alpāhāro yadi bhaved agnir dehaṃ dahet kṣaṇāt ||\\!}
\end{versinnote}
\end{sources}

\begin{testimonia}[hp03_076]

\emph{Haṭharatnāvalī} 2.76
\begin{versinnote}
\tl{āhāro bahulas tasya sampādyaḥ sādhakena vai |\\+}
\tl{alpāhāro yadi bhaved deham agnir dahet kramāt ||\\!}
\end{versinnote}

\emph{Yogacintāmaṇi} f.~78r (attr.~to Dattātreya)
\begin{versinnote}
\tl{āhāro bahulas tasya saṃpādyaḥ sāṃkṛte dhruvam |\\+}
\tl{alpāhāro yadi bhaved agnir dāhaṃ karoti vai ||\\!}
\end{versinnote}

Cf.~\emph{Yuktabhavadeva} 7.238
\begin{versinnote}
\tl{asyāṃ kriyamāṇāyāṃ sādhakasya bhakṣyaṃ bahulaṃ sampādyam anyathā pravṛddho jāṭharānalo dhātuṃ dahatīti ||\\!}
\end{versinnote}
\end{testimonia}

\begin{philcomm}[hp03_076]
Svātmārāma has removed the vocative from the \emph{Dattātreyayogaśāstra}, changing \emph{sāṃkṛte dhruvam} to \emph{sādhakasya tu}.
\end{philcomm}

\begin{metre}[hp03_076]
Anuṣṭubh (c: na-vipulā)
\end{metre}

%%%%%%%%%%
\subsection*{3.77}
\begin{translation}[hp03_077]
On the first day [the yogi] should keep his head down and his feet up for a short while, and he should [then] practise for a little longer every day.
\end{translation}

\begin{sources}[hp03_077]
\emph{Dattātreyayogaśāstra} 148c–149b
\begin{versinnote}
\tl{adhaḥśirāś cordhvapādaḥ kṣaṇaṃ syāt prathame dine ||\\+}
\tl{kṣaṇāc ca kiṃ cid adhikam abhyasec ca dine dine |\\!}
\end{versinnote}
\end{sources}

\begin{testimonia}[hp03_077]
\emph{Haṭhratnāvalī} 2.77
\begin{versinnote}
\tl{adhaḥ śiraś cordhvapādau kṣaṇaṃ syāt prathame dine |\\+}
\tl{kṣaṇāc ca kiñ cid adhikam abhyasec ca dine dine ||\\+}
\tl{\var{cordhvapādau ] cordhvapādaḥ \vl}\\!}
\end{versinnote}

\emph{Yogacintāmaṇi} f.~78r (attr.~to Dattātreya)
\begin{versinnote}
\tl{adhaḥśirāś cordhvapādaḥ kṣaṇaṃ syāt prathame dine |\\+}
\tl{kṣaṇāc ca kiñ cid adhikam abhyasec ca dine dine ||\\!}
\end{versinnote}

Cf.~\emph{Yuktabhavadeva} 7.237
\begin{versinnote}
\tl{sa ca prathamadine kṣaṇamātraṃ vidheyā dvitīyadine | kiñcidadhikaṃ kālam evaṃ yāmaparyantaṃ vidheyā |\\!}
\end{versinnote}
\end{testimonia}

%\begin{philcomm}[hp03_077]
%\end{philcomm}

\begin{metre}[hp03_077]
Anuṣṭubh (a: ra-vipulā; c: na-vipulā)
\end{metre}

%%%%%%%%%%
\subsection*{3.78}
\begin{translation}[hp03_078]
After six months grey hair and wrinkles disappear. [The yogi] who regularly practises for three hours conquers death.
\end{translation}

\begin{sources}[hp03_078]
\emph{Dattātreyayogaśāstra} 149c–150b
\begin{versinnote}
\tl{valiś ca palitaṃ caiva ṣaṇmāsordhvaṃ na dṛśyate ||\\+}
\tl{yāmamātraṃ hi yo nityam abhyaset sa tu kālajit |\\+}
\tl{\var{°māsordhvaṃ na ] °māsāṃ hi na M1, °māsāc ca na AM2, °māsārdhān na YTU, °māsān na tu HR , °māsārddhena \emph{Yogacintāmaṇi} 150b  kālajit ] yogavit πDYŚPT}\\!}
\end{versinnote}
\end{sources}

\begin{testimonia}[hp03_078]
\emph{Haṭharatnāvalī} 2.78
\begin{versinnote}
\tl{valitaṃ palitaṃ caiva ṣaṇmāsān na tu dṛśyate |\\+}
\tl{yāmamātraṃ tu yo nityam abhyaset sa tu kālajit || \\!}
\end{versinnote}

\emph{Yogacintāmaṇi} f.~78r (attr.~to Dattātreya)
\begin{versinnote}
\tl{valiś ca palitaṃ caiva ṣaṇmāsārdhe na dṛśyate |\\+}
\tl{yāmamātraṃ tu yo nityam abhyaset sa tu kālajit ||\\!}
\end{versinnote}

\emph{Yuktabhavadeva} 7.238 (attr.~to Gorakṣanātha)
\begin{versinnote}
\tl{valitaṃ palitaṃ caiva ṣaṇmāsārdhān na dṛśyate |\\+}
\tl{yāmamātraṃ tu yo nityam abhyaset sa tu kālajit ||\\!}
\end{versinnote}

Cf.~\emph{Haṭhatattvakaumudī} 14.3
 \begin{versinnote}
\tl{ūrdhvapādo hy adhomastakaḥ syāt kṣaṇaṃ\\+}
\tl{vāsare 'thādime 'bhyāsaṃ vṛddhyā dhayet |\\+}
\tl{evam abhyāsato yāmamātraṃ sadā \\+}
\tl{mṛtyujit syāj jarājic ca ṣaṇmāsataḥ ||\\!}
\end{versinnote}
\end{testimonia}

\begin{philcomm}[hp03_078]
We have adopted the reading \emph{ṣaṇmāsordhvaṃ} in the second verse quarter. It is attested by manuscripts of the \emph{Dattātreyayogaśāstra} (the source text) and the \emph{Jyotsnā} (3.82). It makes good sense and explains the rather odd readings in \textalpha\ and other manuscripts, \emph{ṣaṇmāsārdhān}, \emph{ṣaṇmāsārdhaṃ} and \emph{ṣaṇmāsārdhe}. The \textgamma\ and delta groups have a different verb as well, \emph{ṣaṇmāsārdhena naśyati}. The original reading was likely \emph{ṣaṇmāsordhvaṃ na dṛśyate} because the compound \emph{ṣaṇmāsārdha} (`half of six months') is very strange and \emph{dṛśyate} is better attested.

\end{philcomm}


%%%%%%%%%%
\subsection*{3.79 heading}
\begin{translation}[hp03_079a]
Now \emph{vajrolī}:
\end{translation}

% \begin{philcomm}[hp03_079a]
% \end{philcomm}

%%%%%%%%%%
\subsection*{3.79}
\begin{translation}[hp03_079]
Even if he behaves as he wishes without [following] the observances (\emph{niyama}) taught in yoga, the [yogi] who knows \emph{vajrolī} is worthy of success.
\end{translation}

\begin{sources}[hp03_079]
\emph{Dattātreyayogaśāstra} 152
\begin{versinnote}
\tl{svecchayā varttamāno ’pi yogoktaniyamair vinā |\\+}
\tl{vajroliṃ yo vijānāti sa yogī siddhibhājanaḥ ||\\+}
\tl{\var{152d °bhājanaḥ ] °mān bhavet M1AM2, °bhājanam YTU}\\!}
\end{versinnote}

Cf.~\emph{Śivasaṃhitā} 4.79
\begin{versinnote}
\tl{svecchayā vartamāno 'pi yogoktaniyamair vinā |\\+}
\tl{mukto bhaved gṛhastho'pi vajrolyabhyāsayogataḥ ||\\!}
\end{versinnote}
\end{sources}

\begin{testimonia}[hp03_079]
\emph{Haṭharatnāvalī} 2.79 (on \emph{viparītakaraṇī})
\begin{versinnote}
\tl{svasthaṃ yo vartamāno 'pi yogoktair niyamair vinā |\\+}
\tl{karaṇī viparītākhyā śrīnivāsena lakṣitā ||\\!}
\end{versinnote}

\emph{Yogalakṣaṇāvalī} f.~31r
\begin{versinnote}
\tl{svecchayā vartamāno 'pi yogoktaniyamair vinā |\\+}
\tl{vajrolyabhyāsayogena yogī siddhim avāpnuyāt ||\\!}
\end{versinnote}

\emph{Haṭhayogasaṃhitā} p. 38
\begin{versinnote}
\tl{svecchayā varttamāno 'pi yogoktair niyamair vinā |\\+}
\tl{vajrolīṃ yo vijānāti sa yogī siddhibhājanam || \\!}
\end{versinnote}

Cf.~\emph{Yuktabhavadeva} 7.240 (attr.~to Gorakṣanātha)
\begin{versinnote}
\tl{vajrolīṃ kathayiṣyāmi gopitāṃ sarvayogibhiḥ|\\+}
\tl{tyaktayogoktaniyamā yayā sidhyanti yoginaḥ || \\!}
\end{versinnote}


Cf.~\emph{Haṭhatattvakaumudī} 16.3
\begin{versinnote}
\tl{svecchayā varttamāno 'pi yogoditaiḥ\\+}
\tl{sadvidhānair vinā sādhakaḥ sābalaḥ |\\+}
\tl{mucyate 'sau suvajrolikābhyāsataḥ\\+}
\tl{sarvasiddhyāspadaṃ yāti bhūmaṇḍale ||\\!}
\end{versinnote}

%Vajroliyoga 1
%\begin{versinnote}
%\tl{svecchayā vartamāno ’pi yogoktair niyamair vinā |\\+}
%\tl{vajrolī yo vijānāti sa yogī siddhibhājanaṃ ||\\!}
%\end{versinnote}

\end{testimonia}

\begin{philcomm}[hp03_079]
In manuscripts of the delta group, the \emph{vajrolī} section is placed at the end of the work and the following comment is inserted at this place in the third chapter:
\begin{versinnote}
\emph{Vajrolī}, which is [usually] here, has been copied at the end of the text. Even though it is found in the sequence here, it has been left out because it is be practised by special individuals.\\
\emph{atratyā vajrolī granthānte likhitā} | \emph{kramaprāptāpy atra tyaktā} | \emph{asādhāraṇaprāṇyanuṣṭheyatvāt tasyāḥ} |
\end{versinnote}

\end{philcomm}


%%%%%%%%%%
\subsection*{3.80}
\begin{translation}[hp03_080]
I shall teach you two substances [needed] for it which are hard for just anyone to obtain. One is milk and the second is an obedient woman.
\end{translation}

\begin{sources}[hp03_080]
\emph{Dattātreyayogaśāstra} 153ab-154ab
\begin{versinnote}
\tl{tatra vastudvayaṃ vakṣye durlabhaṃ yena kena cit |\\+}
\tl{kṣīraṃ caikaṃ dvitīyaṃ ca nārī ca vaśavartinī |\\!}
\end{versinnote}
\end{sources}

\begin{testimonia}[hp03_080]
\emph{Yuktabhavadeva} 7.241 (attr.~to Gorakṣanātha)
\begin{versinnote}
\tl{atra vastudvayaṃ manye durlabhaṃ yasya kasyacit |\\+}
\tl{kṣīram ekaṃ dvitīyaṃ tu nārī svavaśavarttinī || \\!}
\end{versinnote}

\emph{Haṭhayogasaṃhitā} p. 39
\begin{versinnote}
\tl{tatra vastudvayaṃ vakṣye durlabhaṃ yasya kasya cit |\\+}
\tl{kṣīraṃ caikaṃ dvitīyaṃ tu nārī ca vaśavarttinī || \\!}
\end{versinnote}

%Vajroliyoga 2
%\begin{versinnote}
%\tl{tatra vastudvayaṃ vakṣye durlabhaṃ yasya kasya cit |\\+}
%\tl{kṣīraṃ caikaṃ dvitīyaṃ tu nārī ca vaśavartinī ||\\!}
%\end{versinnote}
\end{testimonia}

\begin{philcomm}[hp03_080]
On the possible referents of \emph{kṣīra}, see Mallinson 2024 on \emph{Dattātreyayogaśāstra} 154. According to Brahmānanda (\emph{Jyotsnā} 3.84), the compound \emph{vaśavartinī}, which we have translated as `an obedient woman,' could be a wife (\emph{vaśavartinī svādhīnā nārī vanitā}). In 3.83, the reading \emph{bhāryābhage} in \getsiglum{V1}, \getsiglum{V3} and \getsiglum{J10} supports Brahmānanda's view that the woman is the yogi's wife. 
%Is kṣīra semen? Or a sap that is hard to find? Difficult to say since nothing more is said about kṣīra.
%In DYŚ it's kṣīra and aṅgirasa, so maybe this is addressed to male and female practitioners, who can get one or the other. But what about the celibate yogi?
%LO: in Bengali texts vastu can mean sexual fluid.
% JB should we mention that 3.80–83ab are missing from Alpha Three. Also, Alpha Three is missing 3.84ab, 3.86cd
\end{philcomm}


%%%%%%%%%%
\subsection*{3.81}
\begin{translation}[hp03_081]
[The yogi] should gently practise a full upward contraction through the urethra. Either a man or a woman may obtain success in \emph{vajrolī}.
\end{translation}

%\begin{sources}[hp03_081]
%\end{sources}

\begin{testimonia}[hp03_081]
\emph{Haṭhatattvakaumudī} 16.4
\begin{versinnote}
\tl{apānamārgataḥ samyag ūrdhvakuñcanam abhyaset |\\+}
\tl{puruṣo vāpi nārī vā vajrolīsiddhibhājanam ||\\+}
\tl{apānamārgato gudadeśena ūrdhvam upari kuñcanaṃ saṃkocanam ūrdhvam ākarṣaṇaṃ vā abhyaset || iti ||\\!}
\end{versinnote}

\emph{Haṭhayogasaṃhitā} 53 (p. 39)
\begin{versinnote}
\tl{mehanena śanaiḥ samyag ūrdhvākuñcanam abhyaset |\\+}
\tl{puruṣo 'py athavā nārī vajrolīsiddhim āpnuyāt ||\\!}
\end{versinnote}

%Vajroliyoga 3
%\begin{versinnote}
%\tl{mehanena śanaiḥ samyag ūrdhvākuñcanam abhyaset |\\+}
%\tl{puruṣo vāpi nārī vā vajrolīsiddhim āpnuyāt ||\\!}
%\end{versinnote}

\emph{Yogaprakāśikā} 118ab
\begin{versinnote}
\tl{mehanena śanaiḥ samyag ūrdhvaṃ kuñcanam abhyaset |\\+}
\tl{ūrdhvaṃ yathā syāt tathā bindor ākarṣaṇaṃ meḍhreṇābhyased ity arthaḥ ||\\!}
\end{versinnote}
\end{testimonia}

\begin{philcomm}[hp03_081]
 The \sl{Haṭhatattvakaumudī} says that this upward contraction of the urethra, which is the method by which fluids are drawn up it, is done in the region of \emph{apānavāyu} and the anus (\emph{gudadeśa}). Brahmānanda states that this practice is done immediately after sex (\emph{strīsaṅgānantaram}).
\end{philcomm}

%%%%%%%%%%
\subsection*{3.82}
\begin{translation}[hp03_082]
Carefully using a hollow stalk of bamboo grass, [the yogi] should very gently blow into the opening of the penis in order to make air move [into the urethra].
\end{translation}

\begin{sources}[hp03_082]
\emph{Dattātreyayogaśāstra} 165
\begin{versinnote}
\tl{tatas tu śaranālena phūtkāraṃ vajrakandare |\\+}
\tl{śanaiḥ śanaiḥ prakurvīta vāyusaṃcārakāraṇāt ||\\!}
\end{versinnote}
\end{sources}

\begin{testimonia}[hp03_082]
\emph{Haṭharatnāvalī} 2.86–2.87
\begin{versinnote}
\tl{haṭhapradīpikākāras tu\\+}
\tl{yatnataḥ śaranālena phūtkāraṃ vajrakandare |\\+}
\tl{śanaiḥ śanaiḥ prakurvīta vāyusaṃcārakāraṇāt ||\\!}
\end{versinnote}

\emph{Haṭhasaṅketacandrikā} (f. 39r)
\begin{versinnote}
\tl{taduktaṃ haṭhapradīpikāyāṃ \\+}
\tl{yantritaḥ śaranālena phūtkāraṃ vajrakandare |\\+}
\tl{śanaiḥ śanaiḥ prakurv[ī]ta vāyusaṃcārakāraṇād iti ||\\+}
\tl{asyārthaḥ ||\\+}
\tl{ṣoḍaśāṃgulamānāṃ 16 tu prakuryād vaṃśanālikāṃ sūkṣmā'g[r]amūlāntāṃ li[ṃ]gachi[dra]mukhe datvā svāsye 'nu tanmukhaṃ dhṛtvā phūtkāram ante syāḥ k[u]ryād bāḍhaṃ muhur muhuḥ pratyahaṃ tena vivṛtaṃ liṅgadvāraṃ kramād bhavet [|] \\+}
\tl{tato †nālyānayāto† yam alpaṃ phūtkārato 'ntare [|]\\+}%nālpānn apāno
\tl{liṅgara[n]dhreṇa gṛhṇīyāt kramavṛddhyā susādhakaḥ [|]\\+}
\tl{liṃgachidre 'tha vivṛte kṣīrākṛṣṭiṃ tato bhajed iti [|] \\+}
\tl{vajrakandare liṅgachidre [||]\\!}
\end{versinnote}

\emph{Yogaprakāśikā} 118cd–ef
\begin{versinnote}
\tl{yatnataḥ śaranālena phūtkāraṃ vajrakandare |\\+}
\tl{śanaiḥ śanaiḥ prakurvīta vāyusañcārakāraṇāt ||\\+}
\tl{śareti meḍhranālenety arthaḥ || vāyusañcārakāraṇam iti bindor ākarṣaṇaṃ kāraṇam ity arthaḥ \\!}
\end{versinnote}

Cf.~\emph{Yuktabhavadeva} 7.248cd--249ab
\begin{versinnote}
\tl{rasanālena phūtkāraṃ vāyoḥ sañcārakāraṇāt || \\+}
\tl{kuryāt śanaiḥ śanair yogī yāvac chaktiḥ prajāyate |\\!}
\end{versinnote}
\end{testimonia}

%\begin{philcomm}[hp03_082]
%\end{philcomm}


%%%%%%%%%%
\subsection*{3.83}
\begin{translation}[hp03_083]
With practice, [the yogi] may draw up semen which is falling into a woman’s vagina. And [even] if his own semen has moved [down], he may draw it upwards and retain it.
\end{translation}

\begin{sources}[hp03_083]
\emph{Dattātreyayogaśāstra} 166
\begin{versinnote}
\tl{tadbhage patitaṃ bindum abhyāsenordhvam āharet | \\+}
\tl{calitaṃ ca tathā bindum ūrdhvam ākṛṣya rakṣayet ||\\!}
\end{versinnote}
% there are differences bw the DYS and HP in the 1st and 3rd pādas, but the HP text is solid
\end{sources}

\begin{testimonia}[hp03_083]
\emph{Haṭharatnāvalī} 2.96cd--2.97ab
\begin{versinnote}
\tl{nāryā bhagāt patadbindum abhyāsenordhvam āharet ||\\+}
\tl{calitaṃ ca nijaṃ bindum ūrdhvam ākṛṣya rakṣayet |\\!}
\end{versinnote}

\emph{Haṭhayogasaṃhitā} p. 39
\begin{versinnote}
\tl{nārībhage pated bindum abhyāsenordhvam āharet |\\+}
\tl{calitaṃ ca nijaṃ bindum ūrdhvam ākṛṣya rakṣayet ||\\!}
\end{versinnote}

\emph{Yogaprakāśikā} 5.120
\begin{versinnote}
\tl{nāryā bhage patadbindum abhyāsenordhvam āharet |\\+}
\tl{calitaṃ ca svayaṃ bindum ūrdhvam ākṛṣya rakṣayet ||\\+}
\tl{nārīsaṃyoge bindupatanaṃ syād ity āśaṅkya nirasyati nāryā iti || patato bindor ūrdhvam āhared āhīyamāṇaṃ svayaṃ calitaṃ bindum ākṛṣyety anvayaḥ || \\!}
\end{versinnote}

Cf.~\emph{Haṭhasaṅketacandrikā} f.~39r
\begin{versinnote}
\tl{apānam ākuñcya tato 'balenordhvaṃ dugdham ākṛṣṭividhikrameṇa |\\+}
\tl{samabhyasen niścalam alpam alpaṃ bhage patadbindum athārdhvam āharet ||\\!}
\end{versinnote}

Cf.~\emph{Yuktabhavadeva} 7.249cd, 259
\begin{versinnote}
\tl{tato maithunakāle tu patadbinduṃ samunnayet ||\\+}
\tl{{[}...] patadbindum apānena huṃ huṃkārasahitena balād ūrdhvam ākṛṣya kiñcit kālaṃ vilambya ramet punaḥ || yadā tu na dhārayituṃ śakyate tadā bahiḥskhalitena bindunā saha prasvedenāṅgaṃ marddayet ||\\!}
\end{versinnote}

\end{testimonia}

%\begin{philcomm}[hp03_083]
%The reading \emph{bhāryābhage} in V1, V3 and J10 affirms Brahmānanda's view that the women is the yogi's wife. [JB: I've moved this comment to our commentary on 3.83.
%\end{philcomm}

%%%%%%%%%%
\subsection*{3.84}
\begin{translation}[hp03_084]
[If] the knower of yoga preserves his semen thus, he conquers death. Death arises through the loss of semen and life from retaining semen.
\end{translation}

\begin{sources}[hp03_084]
\emph{Dattātreyayogaśāstra} 167
\begin{versinnote}
\tl{evaṃ ca rakṣito bindur mṛtyuṃ jayati tattvataḥ | \\+}
\tl{maraṇaṃ bindupātena jīvanaṃ bindudhāraṇāt ||\\!}
\end{versinnote}

Cf.~\emph{Amṛtasiddhi} 3.87cd
\begin{versinnote}
\tl{maraṇaṃ bindupātena jīvanaṃ bindudhāraṇāt ||\\+}
\end{versinnote}

\end{sources}

\begin{testimonia}[hp03_084]
\emph{Haṭhratnāvalī} 2.97cd-2.98ab
\begin{versinnote}
\tl{evaṃ saṃrakṣayed binduṃ mṛtyuṃ jayati yogavit ||\\+}
\tl{maraṇaṃ bindupātena jīvitaṃ bindudhāraṇāt |\\!}
\end{versinnote}

\emph{Yuktabhavadeva} 252cd-253ab
\begin{versinnote}
\tl{evaṃ bindau sthire jāte mṛtyuṃ jayati sarvathā ||\\+}
\tl{maraṇaṃ bindupātena jīvanaṃ bindudhāraṇāt |\\!}
\end{versinnote}

\emph{Haṭhayogasaṃhitā} p. 39
\begin{versinnote}
\tl{evaṃ saṃrakṣayed binduṃ mṛtyuṃ jayati yogavit |\\+}
\tl{maraṇaṃ bindupātena jīvanaṃ bindudhāraṇāt || \\!}
\end{versinnote}

%Vajroliyoga 22
%\begin{versinnote}
%\tl{evaṃ bindau sthire yāte mṛtyuṃ jayati sarvathā |\\+}
%\tl{maraṇaṃ bindupātena jīvanaṃ bindudhāraṇāt ||\\!}
%\end{versinnote}
\end{testimonia}

%\begin{philcomm}[hp03_084]
%\end{philcomm}



%%%%%%%%%%
\subsection*{3.85}
\begin{translation}[hp03_085]
As a result of the retention of semen, the yogi's body becomes fragrant. As long as semen is steady in the body then why fear death?%
\end{translation}

\begin{sources}[hp03_085]
\emph{Dattātreyayogaśāstra} 86cd:
\begin{versinnote}
\tl{yogino ’ṅge sugandhaḥ syāt satataṃ bindudhāraṇāt ||\\!}
\end{versinnote}

\emph{Vivekamārtaṇḍa} 52ad
\begin{versinnote}
\tl{yāvad binduḥ sthito dehe tāvad mṛtyubhayaṃ kutaḥ |\\!}
\end{versinnote}
\end{sources}

\begin{testimonia}[hp03_085]
\emph{Haṭharatnāvalī} 2.112ab
\begin{versinnote}
\tl{sugandhir yogino dehe jāyate bindudhāraṇāt ||\\!}
\end{versinnote}

\emph{Haṭhayogasaṃhitā} p. 39
\begin{versinnote}
\tl{sugandho yogino dehe jāyate bindudhāraṇāt |\\+}
\tl{yāvad binduḥ sthiro dehe tāvat kālabhayaṃ kutaḥ || \\!}
\end{versinnote}

\emph{Haṭhatattvakaumudī} 16.10
\begin{versinnote}
\tl{tathā coktaṃ granthāntare –\\+}
\tl{calitaṃ tu svakaṃ bindum ūrdhvam ākuñcya rakṣayet |\\+}
\tl{sugandho yogināṃ dehe jāyate bindudhāraṇād || iti ||\\!}
\end{versinnote}
\end{testimonia}

\begin{philcomm}[hp03_085]
The omission of 3.85ab in the \texteta\ group and \deltaThree\ is likely to be the result of haplography (\emph{bindudhāraṇāt} is repeated).
% MD: I have deleted "in \alphaThree", because the damaged part has enough space for 3.84cd and 3.85ab.

%sthito makes better sense (adopt?), but J5 has sthiro (missing in N3) nad G4 has kṣīro  (closer to sthiro)
%MD: Does 4.88 speak in favor of sthiro? Or is this bindu something different?
% JB: it looks like \alphaTwo and other mss are preserving a version of 3.85 where deha is neuter (I've come across this in the early recension of the Yogabīja.) maybe we should consider it: J5 sugandhi yogino dehaṃ jāyate bindudhāraṇāt. The J5 reading means that we dont have to emend to °gandhir

The readings \emph{mṛtyubhayaṃ} (\alphaThree, \textbeta, \texteta) and \emph{kālabhayaṃ} (\alphaTwo, \textgamma) are well attested by the main manuscript groups, but \emph{mṛtyubhayaṃ} is in the important witnesses of the source text, the \emph{Vivekamārtaṇḍa}. % MD: Which reading does ms T have?
%  JB: record that alphaThree has mṛtyubhayaṃ in 85d? alphaTwo has kālabhayaṃ, so the stemma (as a whole) seems split. [MD: done]
\end{philcomm}

%%%%%%%%%%
\subsection*{3.86}
\begin{translation}[hp03_086]
In men semen is dependent on the mind and life is dependent on semen, so semen and the mind should be carefully guarded.
\end{translation}

%\begin{sources}[hp03_086]
%\begin{versinnote}
%\end{versinnote}
%\end{sources}

\begin{testimonia}[hp03_086]
\emph{Haṭharatnāvalī} 2.98
\begin{versinnote}
\tl{cittāyattaṃ nṛṇāṃ śukraṃ śukrāyattaṃ ca jīvitam |\\+}
\tl{tasmāc chukraṃ manaś caiva rakṣanīyaṃ prayatnataḥ ||\\!}
\end{versinnote}

\emph{Yogacintāmaṇi} f.~74v (attr.~to the \emph{Haṭhapradīpikā})
\begin{versinnote}
\tl{cittāyattaṃ nṛṇāṃ śukraṃ śukrāyattaṃ ca jīvitam |\\+}
\tl{tasmāc cittaṃ ca śukraṃ ca rakṣaṇīyaṃ prayatnataḥ ||\\!}
\end{versinnote}

%Vajroliyoga 4
%\begin{versinnote}
%\tl{cittāyattaṃ nṛṇāṃ śukraṃ śukrāyattaṃ tu jīvitam |\\+}
%\tl{tasmāc chukraṃ manaś caiva rakṣaṇīyaṃ prayatnataḥ ||\\!}
%\end{versinnote}
\end{testimonia}

\begin{philcomm}[hp03_086]
Both \alphaTwo\ and \alphaThree\ indicate that \textit{manas} instead of \textit{citta} was the reading of the initial compound. Therefore, we have conjectured \textit{manāyattaṃ}, assuming double \textit{sandhi} from \emph{manas-āyattam}. 
\end{philcomm}
%MD (2024-06-14): The original reading may have been mana-āyattaṃ as supported by α3 (anā°). manomayaṃ (α2), mano'dhīnaṃ (B) and cittāyattaṃ (the others) seem to have  resulted from an attempt to avoid the middle Indic form mana or the double Saṃdhi. And we have manas in Pāda c, not citta! Cf. also 4.8*22b.

%%%%%%%%%%
\subsection*{3.87}
\begin{translation}[hp03_087]
In this way a [the yogi] may also hold on to [both] the menses of a menstruating woman and his semen. By practising correctly he may draw up [both] through the urethra by the proper practice.
\end{translation}

%\begin{sources}[hp03_087]
%\end{sources}

\begin{testimonia}[hp03_087]
\emph{Haṭhratnāvalī} 2.100cd
\begin{versinnote}
\tl{ṛtumatyā rajo 'py evaṃ rajo binduṃ ca rakṣayet ||\\!}
\end{versinnote}

\emph{Haṭhayogasaṃhitā} p.39
\begin{versinnote}
\tl{ṛtumatyā rajo 'py evaṃ bijaṃ binduṃ ca rakṣayet |\\+}
\tl{meḍhreṇākarṣayed ūrdhvaṃ samyagabhyāsayogavit || 59 || \\!}
\end{versinnote}
\end{testimonia}

\begin{philcomm}[hp03_087]
%No known source. 
We have understood the reading of \emph{ṛtumatyā} as qualifying \emph{striyāḥ} (i.e.,`a menstruating woman'). Alternatively, the term \emph{ṛtumati} could mean a post-pubescent woman.
\end{philcomm}



%%%%%%%%%%
\subsection*{3.87*1}
\begin{translation}[hp03_087_1]
This yoga succeeds for those who have merit, are fortunate, abide in truth, and are without jealousy, not for those who are jealous.
\end{translation}
%greyscaled

\begin{sources}[hp03_087_1]
\emph{Dattātreyayogaśāstra} 176
\begin{versinnote}
\tl{ayaṃ yogaḥ puṇyavatāṃ dhanyānāṃ tattvaśalinām |\\+}
\tl{nirmatsarāṇāṃ sidhyeta na tu mātsaryaśālinām ||\\!}
\end{versinnote}
\end{sources}

\begin{testimonia}[hp03_087_1]
\emph{Haṭharatnāvalī} 2.110
\begin{versinnote}
\tl{ayaṃ yogaḥ puṇyavatāṃ dhanyānāṃ tattvaśālinām |\\+}
\tl{nirmatsarāṇāṃ sidhyeta na tu matsaraśālinām ||\\!}
\end{versinnote}

\emph{Haṭhayogasaṃhitā} pp. 40-41
\begin{versinnote}
\tl{ayaṃ yāgaḥ puṇyavatāṃ dhīrāṇāṃ tattvadarśinām | \\+}
\tl{nirmatsarāṇāṃ sidhyeta na tu mātsaryaśālinām || \\!}
\end{versinnote}

%Vajroliyoga 27
%\begin{versinnote}
%\tl{ayaṃ yogaḥ puṇyavatāṃ siddhe saṃsāriṇāṃ na hi |\\+}
%\tl{amunāṃ siddhim āpnoti yogād yogaḥ pravartate ||\\!}
%\end{versinnote}
\end{testimonia}

\begin{philcomm}[hp03_087_1]
This verse is omitted in \alphaTwo\ and \alphaThree\ (and the folio on which it would be found is missing in \alphaOne).
\end{philcomm}

\begin{metre}[hp03_087_1]
Anuṣṭubh (a: bha-vipulā; c: ma-vipulā)
\end{metre}

%%%%%%%%%%
\subsection*{3.88 heading}
\begin{translation}[hp03_088a]
Now \emph{sahajolī}:
\end{translation}
% atha sahajolī/
% \begin{philcomm}[hp03_088a]
% \end{philcomm}

%%%%%%%%%%
\subsection*{3.88}
\begin{translation}[hp03_088]
\emph{Sahajolī} and \emph{amarolī} are varieties of \emph{vajrolī}. 
\end{translation}

\begin{sources}[hp03_088]
Cf. \emph{Dattātreyayogaśāstra} 31cd
\begin{versinnote}
\tl{vajrolir amaroliś ca sahajolis tridhā matā |\\!}
\end{versinnote}

\emph{Śivasaṃhitā} 4.95ab
\begin{versinnote}
\tl{sahajolyamarolī ca vajrolyā bhedato bhavet |\\!}
\end{versinnote}
\end{sources}

\begin{testimonia}[hp03_088]
\emph{Haṭharatnāvalī} 2.113cd
\begin{versinnote}
\tl{atha sahajoliḥ -\\+}
\tl{sahajolī cāmarolī vajrolyā eva bhedataḥ ||\\!}
\end{versinnote}

\emph{Haṭhayogasamhitā} p.40
\begin{versinnote}
\tl{sahajoliś cāmarolir vajrolyā bheda eva te |\\!}
\end{versinnote}
\end{testimonia}

\begin{philcomm}[hp03_088]
These two \emph{pāda}s introduce the practices of \emph{sahajolī} and \emph{amarolī}, which are described in the verses that follow it. The \textalpha\ and \textbeta\ groups omit the headings for \emph{sahajolī} and \emph{amarolī}. Since 3.88 introduces these practices, the headings are probably not original.

\end{philcomm}

\begin{metre}[hp03_088]
Anuṣṭubh (a: ra-vipulā)
\end{metre}

%%%%%%%%%%
\subsection*{3.89}
\begin{translation}[hp03_089]
After intercourse using \emph{vajrolī}, the woman and man should put ash made from burnt cow dung in water [and] smear their bodies [with it...]
\end{translation}
% greyscaled 

\begin{sources}[hp03_089]
\emph{Dattātreyayogaśāstra} 182
\begin{versinnote}
\tl{tajjale bhasma saṃkṣipya dagdhagomayasaṃbhavam |\\+}
\tl{vajrolīmaithunād ūrdhvaṃ strīpuṃsor aṅgalepanam ||\\+}
\tl{\var{182a tajjale bhasma saṃkṣipya ] M2; tajjale bhasmasāt kṣipya M1, tajjale bhasma saddravyaṃ A}\\!}
\end{versinnote}

\end{sources}

\begin{testimonia}[hp03_089]
\emph{Haṭharatnāvalī} 2.114
\begin{versinnote}
\tl{jale subhasma nikṣipya dagdhagomayasaṃbhavam |\\+}
\tl{vajrolīmaithunād ūrdhvaṃ strīpuṃsoś cāṅgalepanam ||\\!}
\end{versinnote}

\emph{Haṭhayogasamhitā} p.40
\begin{versinnote}
\tl{jale subhasma nikṣipya dagdhagomayasambhavam ||\\+}
\tl{vajrolī maithunād ūrdhvaṃ strīpuṃsoḥ svāṅgalepanam | \\!}
\end{versinnote}
\end{testimonia}

\begin{philcomm}[hp03_089]
Some manuscripts, including \alphaTwo\ and \alphaThree\ (missing in \alphaOne), omit 3.89ab. We have included it because in the \emph{Dattātreyayogaśāstra}, the source of this verse, 3.89ab specifies the substance mentioned in 3.89cd that the man and woman are supposed to rub into their bodies after sexual intercourse.

In the \emph{Dattātreyayogaśāstra}’s teaching on \emph{sahajolī} (163 and 181–183) a rag is used to wipe up the residue of a mixture of semen and sweat that has been rubbed into the body, and then soaked in a paste of water and ash before being rubbed over the body.

Although the plural in 3.89a is awkward, \emph{jaleṣu} was probably the result of Svātmārāma removing the pronoun from the compound \emph{tajjale} in the \emph{Dattātreyayogaśāstra}'s verse because it has no referent in the \emph{Haṭhapradīpikā}'s compilation.
\end{philcomm}

%%%%%%%%%%
\subsection*{3.90}
\begin{translation}[hp03_090]
[...] while sitting at complete ease, having just finished intercourse. This is called \emph{sahajolī}. It is always to be trusted by yogis. 
\end{translation}


\begin{sources}[hp03_090]
\emph{Dattātreyayogaśāstra} 183
\begin{versinnote}
\tl{āsīnayoḥ sukhenaiva muktavyāparayoḥ kṣaṇam |\\+}
\tl{sahajolī ca saṃproktā śraddheyā yogibhiḥ sadā ||\\!}
\end{versinnote}
\end{sources}

\begin{testimonia}[hp03_090]
\emph{Haṭharatnāvalī} 2.115
\begin{versinnote}
\tl{āsīnayoḥ sukhenaiva muktavyāpārayoḥ kṣaṇam |\\+}
\tl{sahajolir iyaṃ proktā kartavyā yogibhiḥ sadā ||\\!}
\end{versinnote}

\emph{Haṭhasaṃhitā} p. 40
\begin{versinnote}
\tl{āsīnayoḥ sukhenaiva muktavyāpārayoḥ kṣaṇāt ||\\+}
\tl{sahajolir iyaṃ proktā śraddheyā yogibhiḥ sadā \\!}
\end{versinnote}
\end{testimonia}

\begin{philcomm}[hp03_090]
We have understood the \emph{repha} in \emph{sahajolīr iyam} as a hiatus bridge. Elsewhere the nominative of this name is found only as \emph{sahajolī} or \emph{sahajoliḥ}.
\end{philcomm}


%%%%%%%%%%
\subsection*{3.90*1}
\begin{translation}[hp03_090_1]
This auspicious yoga bestows liberation even when pleasure has been enjoyed.
\end{translation}
%greyscaled

%\begin{sources}[hp03_090_1]
%\end{sources}

\begin{testimonia}[hp03_090_1]
\emph{Haṭhayogasaṃhitā} p. 40
\begin{versinnote}
\tl{ayaṃ śubhakaro yogī bhogayukto'pi muktidaḥ ||\\!}
\end{versinnote}
\end{testimonia}

\begin{philcomm}[hp03_090_1]
This line is absent in \alphaTwo,\ \alphaThree\ and \gammaOne\ (missing in \alphaOne). It may have been adapted from \emph{Dattātreyayogaśāstra} 179cd (\emph{tasmād ayaṃ vakṣyamāṇo bhoge bhukte ’pi muktidaḥ}). 
\end{philcomm}


%%%%%%%%%%
\subsection*{3.91 heading}
\begin{translation}[hp03_091a]
Now \emph{amarolī}:
\end{translation}

% \begin{philcomm}[hp03_091a]
% \end{philcomm}

%%%%%%%%%%
\subsection*{3.91}
\begin{translation}[hp03_091]
Leaving out the first flow of urine because of its excessive heat and the last flow because it is worthless, the cool middle flow of urine is used by Kāpālikas of the Khaṇḍa school.
\end{translation}

\begin{sources}[hp03_091]
\end{sources}

\begin{testimonia}[hp03_091]
\emph{Haṭharatnāvalī} 2.116
\begin{versinnote}
\tl{athāmarolī\\+}
\tl{vihāya nityāṃ prathamāṃ ca dhārāṃ \\+}
\tl{vihāya niḥsāratayāntyadhārām |\\+}
\tl{niṣevyate śītalamadhyadhārāṃ \\+}
\tl{kāpālikaiḥ khaṇḍamatair anarghyām ||\\+}
\tl{\var{vihāya nityāṃ ] pittolbaṇatvāt \vl}\\+}
\tl{\var{anarghyām ] anarghyā}\\!}
\end{versinnote}

\emph{Haṭhatattvakaumudī} 16.17
\begin{versinnote}
\tl{athāmarolī –\\+}
\tl{pittolbaṇatvāt prathamāṃ ca dhārāṃ \\+}
\tl{vihāya niḥsāratayāntyadhārām |\\+}
\tl{niṣevyate śītamadhyadhārā \\+}
\tl{kāpālikaiḥ khaṇḍamate 'marolī ||\\!}
\end{versinnote}

\emph{Haṭhayogasaṃhitā} p. 41
\begin{versinnote}
\tl{pittolvaṇatvāt prathamāmbudhārāṃ \\+}
\tl{niṣevyate śītalamadhyadhārā |\\+}
\tl{vihāya niḥsāratayāntyadhārāṃ \\+}
\tl{kāpālike khaṇḍamate 'marolī ||\\!}
\end{versinnote}
\end{testimonia}

\begin{philcomm}[hp03_091]
We understand `Kāpālikas of the Khaṇḍa school' (\emph{kāpālikair khaṇḍamataiḥ}) to be referring to followers of the Khaṇḍakāpālika who is mentioned in the list of siddhas given at 1.5–9, \emph{pace} Marcinkowska-Rosół and Sellmer (2021: 105–108) who understand \emph{khaṇḍamataiḥ} to mean `whose doctrine is defective'. 

%Maria Marcinkowska-Rosół & Sven Sellmer. “Notes on some difficult passages of the Haṭhapradīpikā”. Zeitschrift der Deutschen Morgenländischen Gesellschaft 171 (2021), pp. 101–21
\end{philcomm}

\begin{metre}[hp03_091]
Upajāti
\end{metre}

%%%%%%%%%%
\subsection*{3.92}
\begin{translation}[hp03_092]
[The yogi] who regularly imbibes urine, taking it by the nose every day, practises \emph{vajrolī} thus. This is called \emph{amarolī}.
\end{translation}

\begin{sources}[hp03_092]
\emph{Dattātreyayogaśāstra} 180c–181b
\begin{versinnote}
\tl{amarīṃ yaḥ piben nityaṃ nasyaṃ kurvan dine dine ||\\+}
\tl{vajrolīm abhyasec ceyam amarolīti kathyate |\\+}
\tl{\var{181a abhyaset ceyam] \emph{em.}; abhyasec chrayam M1, abhyaset yeyam A, abhyasec caivam M2}}
\end{versinnote}
\end{sources}

\begin{testimonia}[hp03_092]
\emph{Haṭharatnāvalī} 2.117
\begin{versinnote}
\tl{amarīṃ yaḥ piben nityaṃ nasyaṃ kuryād dine dine  |\\+}
\tl{vajrolīm abhyasen nityam amarolīti kathyate ||\\!}
\end{versinnote}

\emph{Haṭhayogasaṃhitā} 65 (p.41)
\begin{versinnote}
\tl{amarīṃ yaḥ piben nityaṃ nasyaṃ kurvan dine dine |\\+}
\tl{vajrolīm abhyaset samyag amarolīti kathyate ||\\!}
\end{versinnote}
\end{testimonia}

%\begin{philcomm}[hp03_092]
%\end{philcomm}


%%%%%%%%%%
\subsection*{3.92*1}
\begin{translation}[hp03_092_1]
If a woman draws up the semen of a man through skillfulness in the correct practice and retains her menses by means of \emph{vajrolī}, it is she who is a [true] yoginī.
\end{translation}
%greyscaled

\begin{sources}[hp03_092_1]
\emph{Dattātreyayogaśāstra} 169cd
\begin{versinnote}
\tl{yadi nārī rajo rakṣed vajrolyā sā hi yoginī ||\\!}
\end{versinnote}
\end{sources}

\begin{testimonia}[hp03_092_1]
\emph{Haṭhayogasaṃhitā} p. 41
\begin{versinnote}
\tl{puṃso binduṃ samākuñcya samyagabhyāsapāṭavāt |\\+}
\tl{yadi nārī rajo rakṣed vajrolyā sā'pi yoginī ||\\!}
\end{versinnote}
\end{testimonia}

\begin{philcomm}[hp03_092_1]
In the first verse quarter, the gerund \emph{samākṛṣya} (\textgamma) has been adopted, instead of \emph{samākuñcya}, as it yields a better meaning and is used similarly to \emph{ākṛṣya} in 3.83.

Verses 3.92*1–3 have been greyscaled because they are absent in \alphaThree \ (and missing in  \alphaOne), and appear to have been added to the text from the \emph{Dattātreyayogaśāstra}'s section on \emph{vajrolī} to provide further details of how a woman practises \emph{vajrolī} to those found in 3.93–94. The verses are present in \alphaTwo\ after verse 3.87ab where the verse quarter \emph{vajrolyā saha yoginī} occurs twice (also at 3.93b), which suggests that the version of \emph{vajrolī} in \alphaTwo\ has been subject to further revision. The fact that 3.92*1–3 are in groups \textbeta, \textgamma\ and \texteta\ indicates that they were added early in the transmission of the \emph{Haṭhapradīpikā}.     
% J5
% ṛtumatyā rajo 'py evaṃ vīryaṃ binduṃ ca rakṣayet |
% yadi nārī rajo rakṣet vajrolyā saha yoginī || (note the difference with our current 3.90c)
% tasyāḥ kiñ cid rajo nāśaṃ na gacchati na saṃśayaḥ |
% tasyāḥ śarīre nādas tu bindu[t]ām eva gacchati |
% sa bindus tad rajaś caiva ekībhūtaṃ svadehagau | 
% vajrolyābhyāsayogena sarvasiddhiḥ prakurvati [prajāyate] ||
% sahajolī cāmarolī [ca?] vajrolyā eva bhedataḥ
% 3.93
\end{philcomm}

%%%%%%%%%%
\subsection*{3.92*2}
\begin{translation}[hp03_092_2]
Assuredly none of her menses is lost. The \emph{nāda} in her body turns into \emph{bindu}.
\end{translation}

\begin{sources}[hp03_092_2]
\emph{Dattātreyayogaśāstra} 174
\begin{versinnote}
\tl{tasyās tadā rajo nāśaṃ na gacchati na saṃśayaḥ | \\+}
\tl{tasyāḥ śarīre nādas tu bindutām eva gacchati || 174 ||\\!}
\end{versinnote}
\end{sources}

\begin{testimonia}[hp03_092_2]
\emph{Haṭharatnāvalī} 2.108ab
\begin{versinnote}
\tl{tasyāḥ śarīre nādas tu bindutām eva gacchati |\\!}
\end{versinnote}

\emph{Haṭhayogasaṃhitā} pp. 41--42
\begin{versinnote}
\tl{tasyāḥ kiñ cid rajo nāśaṃ na gacchati na saṃśayaḥ |\\+}
\tl{tasyāḥ śarīre nādaś ca bindutām eva gacchati ||\\!}
\end{versinnote}
\end{testimonia}

\begin{philcomm}[hp03_092_2]
On why this verse is in greyscale, see the note to 3.92*1.

On \emph{nāda} and \emph{bindu} see the note to 3.53. 
%Brahmānanda identifies nāda with rajas. LO: bindu is source of everything, nāda is going back to its source.
%
\end{philcomm}

\begin{metre}[hp03_092_2]
Anuṣṭubh (c: ma-vipulā)
\end{metre}

%%%%%%%%%%
\subsection*{3.92*3}
\begin{translation}[hp03_092_3]
The \emph{bindu} and \emph{rajas}, which are produced in her own body, become one through \emph{vajrolī} and bring about complete perfection by means of practice.%
\end{translation}

\begin{sources}[hp03_092_3]
\emph{Dattātreyayogaśāstra} 175
\begin{versinnote}
\tl{sa bindus tad rajaś caiva ekībhūya svadehagau | \\+}
\tl{vajrolyābhyāsayogena sarvasiddhiḥ prajāyate || 175 ||\\!}
\end{versinnote}
\end{sources}

\begin{testimonia}[hp03_092_3]
\emph{Haṭharatnāvalī} 2.108cd--109ab
\begin{versinnote}
\tl{sa bindus tad rajaś caiva ekīkṛtya svadehajau ||\\+}
\tl{vajrolyabhyāsayogena yogasiddhiḥ kare sthitā |\\!}
\end{versinnote}

\emph{Haṭhayogasaṃhitā} p. 42
\begin{versinnote}
\tl{sa bindus tad rajaś caiva ekībhūya svadehagau |\\+}
\tl{vajrolyabhyāsayogena sarvasiddhiṃ prayacchataḥ || \\!}
\end{versinnote}
\end{testimonia}

\begin{philcomm}[hp03_092_3]
On why this verse is in greyscale, see the note to 3.92*1.
\end{philcomm}

%%%%%%%%%%
\subsection*{3.93}
\begin{translation}[hp03_093]
It is she who preserves her menses by means of the upward contraction who is the [true] yoginī.
She knows the past and the future, and is sure to become a sky-rover (\emph{khecarī}).
\end{translation}

\begin{sources}[hp03_093]
\emph{Dattātreyayogaśāstra} 170ab
\begin{versinnote}
\tl{atītānāgataṃ vetti khecarī vā bhaved dhruvam |\\!}
\end{versinnote}
\end{sources}

\begin{testimonia}[hp03_093]
\emph{Haṭhayogasaṃhitā} p. 42
\begin{versinnote}
\tl{rakṣed ākuñcanād ūrdhvaṃ yā rajaḥ sā hi yoginī | \\+}
\tl{atītānāgatajñānaṃ khecarī ca bhaved dhruvam ||\\!}
\end{versinnote}

%Haṭhapradīpikā 5.140--141ab
%\begin{versinnote}
%\tl{mehanākuñcanād ūrdhvaṃ rajasāpi ca yoginī |\\+}
%\tl{atītānāgataṃ vetti khecarī ca bhaved dhruvam ||\\+}
%\tl{dehasiddhiṃ ca labhate vajrolyabhyāsayogataḥ |\\!}
%\end{versinnote}
\end{testimonia}

%\begin{philcomm}[hp03_093]
%\end{philcomm}

\begin{metre}[hp03_093]
Anuṣṭubh (c: na-vipulā)
\end{metre}

%%%%%%%%%%
\subsection*{3.94}
\begin{translation}[hp03_094]
And she attains perfection of the body as a result of the practice of \emph{vajrolī}. This auspicious yoga bestows liberation even when pleasure has been enjoyed.
\end{translation}
%Therefore this yoga succeeds only for those who have merit.

\begin{sources}[hp03_094]
\emph{Dattātreyayogaśāstra} 179
\begin{versinnote}
\tl{dehasiddhiṃ ca labhate vajrolyabhyāsayogataḥ |\\+}
\tl{tasmād ayaṃ vakṣyamāṇo bhoge bhukte ’pi muktidaḥ \\+}
\tl{\var{179d bhoge bhukte ’pi muktidaḥ ] conj.; bhoge bhukte tv abhuktidaḥ M1, bhogo yogaś ca muktidaḥ AM2}\\!}
\end{versinnote}
\end{sources}

\begin{testimonia}[hp03_094]
\emph{Haṭharatnāvalī} 2.111
\begin{versinnote}
\tl{sarveṣām eva yogānām ayaṃ yogaḥ śubhaṅkaraḥ |\\+}
\tl{tasmād ayaṃ variṣṭho 'sau bhuktimuktiphalapradaḥ ||\\!}
\end{versinnote}

\emph{Haṭhayogasaṃhitā} p. 42
\begin{versinnote}
\tl{dehasiddhiṃ ca labhate vajrolyabhyāsayogataḥ |\\+}
\tl{ayaṃ puṇyakaro yogo bhoge bhukte 'pi muktidaḥ ||\\!}
\end{versinnote}

\end{testimonia}

%\begin{philcomm}[hp03_094]
%tasmād ayaṃ sādhakāya hemistich is not in Gr2 and seems like an unnecessary repetition (that may have occurred to change a 6 pāda verse into two verses).
%
% MD: 3.101*1ab and 3.101cd by the old numbering are now merged as 3.94*cd.
%\end{philcomm}

\begin{metre}[hp03_094]
Anuṣṭubh (a: ra-vipulā)
\end{metre}

%%%%%%%%%%
\subsection*{3.94*1}
\begin{philcomm}[hp03_094_1]
This verse is not in \textalpha\ and \textgamma, and seems like an unnecessary repetition of 3.101.
\end{philcomm}

%%%%%%%%%%
\subsection*{3.94*2 heading}
\begin{translation}[hp03_094_2a]
Now the Stimulation of the Goddess (\emph{śakticālanam}):
\end{translation}

% \begin{philcomm}[hp03_094_2a]
% \end{philcomm}

%%%%%%%%%%
\subsection*{3.94*2}
\begin{translation}[hp03_094_2]
She whose body is bent (\emph{kuṭilāṅgī}), she who is coiled (\emph{kuṇḍalinī}), the female snake (\emph{bhujaṅgī}), the power (\emph{śakti}), the goddess (\emph{īśvarī}), she who is coiled (\emph{kuṇḍalī}) and Arundhatī: these words are synonyms.
\end{translation}
% 
%\begin{sources}[hp03_094_2]
%\end{sources}

\begin{testimonia}[hp03_094_2]

\emph{Haṭharatnāvalī} 2.125–127
\begin{versinnote}
\tl{phaṇī kuṇḍalinī nāgī cakrī vakrī sarasvatī |\\+}
\tl{lalanā rasanā kṣatrī lalāṭī śaktiḥ śaṃkhinī ||\\+}
\tl{rajvī bhujaṅgī śeṣā ca kuṇḍalī sarpiṇī maṇiḥ |\\+}
\tl{ādhāraśaktiḥ kuṭilā karālī prāṇavāhinī ||\\+}
\tl{aṣṭavakrā ṣaḍādhārā vyāpinī kalanādharā ||\\+}
\tl{kurīty evaṃ ca vikhyātāḥ śabdāḥ paryāyavācakāḥ ||\\!}
\end{versinnote}

\emph{Yogacintāmaṇi} f.~78v (attr.~to the \emph{Haṭhayoga})
\begin{versinnote}
\tl{kuṇḍalāṅgī kuṇḍalinī bhujaṅgī śaktir īśvarī |\\+}
\tl{kuṭilārundhatī devī śabdāḥ paryāyavācakāḥ ||\\!}
\end{versinnote}

\emph{Yuktabhavadeva} 7.300 (attr.~to the \emph{Śivayoga})
\begin{versinnote}
\tl{kuṭilāṃgī kuṇḍalinī bhujaṅgī śaktir īśvarī |\\+}
\tl{kuṇḍaly arundhatī devī śabdāḥ paryāyavācakāḥ ||\\!}
\end{versinnote}

%Haṃsavilāsa (p. 46)
%\begin{versinnote}
%\tl{āgame nāmāntarāṇi ca |\\+}
%\tl{kuṭilāṅgī kuṇḍalinī bhujaṅgī śaktir īśvarī |\\+}
%\tl{kuṇḍalyā ceti ruddhanti śabdāḥ paryāyavācakāḥ ||\\!}
%\end{versinnote}

%Prāṇatoṣiṇī Part 6 (possiby attr.~to the Dattātreyasaṃhitā) %line 3723
%\begin{versinnote}
%\tl{kuṇḍalāṅgī kuṇḍalinī bhujaṅgī śaktirīśvarī | \\+}
%\tl{kuṇḍalyarundhatī devī śabdāḥ paryāyavācakāḥ |\\!}
%\end{versinnote}
\end{testimonia}

\begin{philcomm}[hp03_094_2]
%Authorial?
%3.4 and 4.4 have cety ekavācakāḥ, but devī paryāyavācakāḥ is well attested among the HP manuscripts (Groups 2 and 3) and the testimonia.
The \alphaThree\ manuscript has a significantly shorter and more coherent version of \emph{śakticālana}. It omits five introductory verses, of which three are from the \emph{Vivekamārtaṇḍa} or one of its longer recensions and three have no known source, including one that contains a list of synonyms for \emph{kuṇḍalinī}. This section is missing in \alphaOne\ (3.83–3.98) and \alphaTwo\ adds these verses (except 3.94*6) after 3.97, which suggests that they have been inserted from elsewhere. Generally speaking, it appears that some redactors have taken the section on \emph{śakticālana} in the \emph{Haṭhapradīpikā} as an opportunity to add material on \emph{kuṇḍalinī}, in particular her location, shape, and soteriological importance. 

\end{philcomm}

\begin{metre}[hp03_094_2]
Anuṣṭubh (a: bha-vipulā)
\end{metre}

%%%%%%%%%%
\subsection*{3.94*3}
\begin{translation}[hp03_094_3]
Just as one might use a key to force open a double door, so the yogi breaks open the door to liberation with Kuṇḍalinī.
\end{translation}

\begin{sources}[hp03_094_3]
\emph{Vivekamārtaṇḍa} 35
\begin{versinnote}
\tl{udghāṭayet kapāṭaṃ tu yathā kuñcikayā haṭhāt |\\+}
\tl{kuṇḍalinyā tathā yogī mokṣadvāraṃ vibhedayet ||\\!}
\end{versinnote}
\end{sources}

\begin{testimonia}[hp03_094_3]
\emph{Yogacintāmaṇi}
\begin{versinnote}
\tl{haṭhayoge\\+}
\tl{udghāṭayet kapāṭaṃ tu yathā kuñcikayā haṭhāt |\\+}
\tl{kuṇḍalinyā tathā yogī mokṣadvāraṃ vibhedayet ||\\!}
\end{versinnote}

Haṭhasaṅketacandrikā f.~110r
\begin{versinnote}
\tl{tathā coktaṃ haṭhapradīpikāyāṃ |\\+}
\tl{udghāṭayet kapāṭaṃ tu yathā kuñcikayā haṭhāt |\\+}
\tl{kuṇḍalinyā tathā yogī mokṣadvāraṃ vibhedayet ||\\!}
\end{versinnote}
\end{testimonia}

%\begin{philcomm}[hp03_094_3]
%SS: von Hinuber 1992 booklet has something relevant to this. Can provide.
%\end{philcomm}


%%%%%%%%%%
\subsection*{3.94*4}
\begin{translation}[hp03_094_4]
The supreme goddess sleeps with her mouth covering the opening of the pathway by which the wholesome place of Brahman is reached.
\end{translation}
%

\begin{sources}[hp03_094_4]
\emph{Vivekamārtaṇḍa} 33
\begin{versinnote}
\tl{yena mārgeṇa gantavyaṃ brahmasthānaṃ nirāmayam |\\+}
\tl{mukhenācchādya taddvāraṃ prasuptā parameśvarī ||\\!}
\end{versinnote}
\end{sources}

\begin{testimonia}[hp03_094_4]
\emph{Yogacintāmaṇi} f.~78v (attr.~to the \emph{Haṭhayoga})
\begin{versinnote}
\tl{yena dvāreṇa gantavyaṃ brahmasthānaṃ nirāmayam |\\+}
\tl{mukhenācchādya taddvāraṃ prasuptā parameśvarī ||\\!}
\end{versinnote}

\emph{Haṭhasaṅketacandrikā} f.~110r (attr.~to the \emph{Haṭhapradīpikā})
\begin{versinnote}
\tl{yena mārgeṇa gaṃtavyaṃ brahmasthānaṃ nirāmayaṃ |\\+}
\tl{mukhenācchādya taddvāraṃ prasuptā parameśvarī ||\\!}
\end{versinnote}
\end{testimonia}

%\begin{philcomm}[hp03_094_4]
%\end{philcomm}


%%%%%%%%%%
\subsection*{3.94*5}
\begin{translation}[hp03_094_5]
The coiled goddess, who sleeps above the bulb [in the abdomen], leads to liberation for yogis and bondage for the deluded. He who knows her knows yoga.
\end{translation}

\begin{sources}[hp03_094_5]
\emph{Vivekamārtaṇḍa} 39
\begin{versinnote}
\tl{kandordhvaṃ kuṇḍalī śaktir suptā mokṣāya yoginām |\\+}
\tl{bandhanāya ca mūḍhānāṃ yas tāṃ vetti sa yogavit ||\\+}
\tl{\var{suptā mokṣāya yoginām ] VTvlH; aṣṭadhā kuṃḍalīkṛtā A, aṣṭadhā kuṇḍalākṛtiḥ GLGPk, śubhamokṣāpradāyinī GB, śubhā mokṣapradāyinī GP, aṣṭadhā kuṭilīkṛtā T}\\!}
\end{versinnote}
\end{sources}

\begin{testimonia}[hp03_094_5]
\emph{Yogacintāmaṇi} f.~78v (attr.~to the \emph{Haṭhayoga})
\begin{versinnote}
\tl{kandordhve kuṇḍalī śaktir buddhā mokṣāya yoginām |\\+}
\tl{bandhanāya ca mūḍhānāṃ yas tām vetti sa yogavit ||\\!}
\end{versinnote}

\emph{Haṭhasaṅketacandrikā} f.~110r (attr.~to the \emph{Haṭhapradīpikā})
\begin{versinnote}
\tl{kandordhvaṃ kuḍalī śakti suptā mokṣāya yogināṃ |\\+}
\tl{bandhanāya ca mūḍhānāṃ yas taṃ vetti sa yogavit ||\\!}
\end{versinnote}
\end{testimonia}

%\begin{philcomm}[hp03_094_5]
%\end{philcomm}


%%%%%%%%%%
\subsection*{3.94*6}
\begin{translation}[hp03_094_6]
[Just as] the coiled serpent Ananta (\emph{śeṣa\-kuṇḍalī}) is the foundation of the oceans, mountains and islands, so Kuṇḍalinī is the foundation of all systems of yoga.
\end{translation}

%\begin{sources}[hp03_094_6]
%\end{sources}

\begin{testimonia}[hp03_094_6]
%\emph{Haṭharatnāvalī} 2.124
%\begin{versinnote}
%\tl{saśailavanadhātryās tu yathādhāro 'hināyakaḥ |\\+}
%\tl{aśeṣayogatantrāṇāṃ tathādhāro hi kuṇḍalī ||\\!}
%\end{versinnote}

\emph{Yogacintāmaṇi} f.~78v (attr.~to the \emph{Haṭhayoga})
\begin{versinnote}
\tl{ambhodhiśailadvīpānām ādhāraḥ śeṣakuṇḍalī |\\+}
\tl{aśeṣayogatantrāṇām ādhāraḥ kuṇḍalī tathā ||\\!}
\end{versinnote}
\end{testimonia}

\begin{philcomm}[hp03_094_6]
%No known source. Authorial? 
This verse is similar to 3.1.
\end{philcomm}

\begin{metre}[hp03_094_6]
Anuṣṭubh (a: ma-vipulā)
\end{metre}

%%%%%%%%%%
\subsection*{3.94*7}
\begin{translation}[hp03_094_7]
Kuṇḍalinī is said to have a curved shape like a snake. The person who makes that goddess move is sure to be liberated.
\end{translation}

%\begin{sources}[hp03_094_7]
%\end{sources}

\begin{testimonia}[hp03_094_7]
\emph{Yogacintāmaṇi} f.~78v–79r (attr.~to the \emph{Haṭhayoga})
\begin{versinnote}
\tl{kuṇḍalī kuṭilākārā sarpavat parikīrtitā |\\+}
\tl{sā śaktiś cālitā yena sa mukto nātra saṃśayaḥ ||\\!}
\end{versinnote}
\end{testimonia}

%\begin{philcomm}[hp03_094_7]
%No known source.
%\end{philcomm}


%%%%%%%%%%
\subsection*{3.95}
\begin{translation}[hp03_095]
Between the Gaṅgā and Yamuna is the wretched young widow. [The yogi] should forcefully take [her]. That is the supreme state of Viṣṇu.
\end{translation}

\begin{sources}[hp03_095]
Cf.~\emph{Śivasaṃhitā} 5.169
\begin{versinnote}
\tl{gaṅgāyamunayor madhye vahaty eṣā sarasvatī |\\+}
\tl{tāsāṃ tu saṃgame snātvā dhanyo yāti parāṃ gatim\\!}
\end{versinnote}
\end{sources}

\begin{testimonia}[hp03_095]
\emph{Yogacintāmaṇi} f.~79r (attr.~to the \emph{Haṭhayoga})
\begin{versinnote}
\tl{gaṅgāyamunayor madhye bālaraṇḍā tapasvinī |\\+}
\tl{balātkāreṇa gṛhṇīyāt tad viṣṇoḥ paramaṃ padam ||\\!}
\end{versinnote}

%Haṃsavilāsa (p. 46)
%\begin{versinnote}
%\tl{gaṅgāyamunayor madhye bālaraṇḍāṃ tapasvinīm |\\+}
%\tl{balātkāreṇa gṛhṇīyāt tad viṣṇoḥ paramaṃ padam ||\\!}
%\end{versinnote}

Prāṇatoṣiṇī Part 6 (attr.~to the \emph{Dattātreyasaṃhitā})
\begin{versinnote}
\tl{gaṅgāyamunayor madhye bālāraṇḍā tapasvinī | \\+}
\tl{balād ākṛṣya gṛhnīyāt tad viṣṇoḥ paramaṃ padam ||\\!}
\end{versinnote}

%Bodhasāra 39.34, 12.3 (check, has commentary too):

\end{testimonia}

\begin{philcomm}[hp03_095]
%No known source. 
The referent of \emph{bālaraṇḍā tapasvinī}, here is unclear. In some manuscripts this verse is followed by one in which \emph{bālaraṇḍā} is identified as \emph{sarasvatī}, which in the context of \emph{śakticālana} could refer to the tongue. She could also be Kuṇḍalinī, who in 3.94*5 is located at the navel, which is said to be the location of Viṣṇu (e.g.~\emph{Dhyāna\-bindū\-paniṣat} 28-30). In his commentary on this verse in the \emph{Bodhasāra} (1906: 137), Divākara says that the seizing of Kuṇḍalinī itself is the highest state of Viṣṇu (... \emph{bālaraṇḍāṃ ... gṛhṇīyād vaśīkuryāt tat tasyā vaśīkaraṇam eva viṣṇor vyāpana\-lakṣa\-nasya paramā\-tmanaḥ paramaṃ kevalaṃ ... padaṃ svarūpaṃ jñeyam}).
% Narahari N Divākara Divākara Dayānanda Sarasvatī. Bodhasāraḥ = Bodhsār : A Treatise on Vedanta. Arthadīptyā Sahitaḥ. Benares: Vidya Vilas Pr; 1906.

\end{philcomm}

%%%%%%%%%%
\subsection*{3.95*1}
% \begin{translation}[hp03_095_1]
% \end{translation}

\begin{philcomm}[hp03_095_1]
Verse 3.95*1, which has no known source, simply identifies the technical terms in 3.95, namely, \emph{gaṅgā}, \emph{yamunā} and \emph{bālaraṇḍā} as \emph{iḍā, piṅgalā} and \emph{sarasvatī}. It is absent in the \textalpha\ manuscripts and probably crept into the text as a marginal note early in the transmission.     
\end{philcomm}
% \alphaTwo\ has this verse in a correct position. Its second half is different from that of the other recensions:
% iḍā bhagavatī gaṅgā piṅgalā yamunā nadī |
% tayor madhye prayāgaṃ tu yas taṃ veda sa vedavit ||
% Cf. Mahābhārata Bhīṣmaparvan
%06,040.078d@003A_0041 iḍā bhagavatī gaṅgā piṅgalā yamunā nadī |
%06,040.078d@003A_0042 tayor madhye tṛtīyā tu tat prayāgam anusmaret ||

%%%%%%%%%%
\subsection*{3.96}
\begin{translation}[hp03_096]
Seizing her tail, the fearless [yogi] wakes the sleeping serpent.  She shakes off sleep and is forced to stand up straight.
\end{translation}

%\begin{sources}[hp03_096]
%\end{sources}

\begin{testimonia}[hp03_096]
\emph{Haṭharatnāvalī} 2.118
\begin{versinnote}
\tl{pucche pragṛhya bhujagīṃ suptām udbodhayed abhīḥ  |\\+}
\tl{nidrāṃ vihāya sā ṛjvī ūrdhvam uttiṣṭhate haṭhāt ||\\!}
\end{versinnote}

\emph{Yogacintāmaṇi} f.~79r (attr.~to the \emph{Haṭhayoga})
\begin{versinnote}
\tl{pucchaṃ pragṛhya bhujagīṃ suptām udbodhayed abhi |\\+}
\tl{nidrāṃ vihāya sā ṛjvī ūrdhvam uttiṣṭhate haṭhāt ||\\!}
\end{versinnote}

\end{testimonia}

%\begin{philcomm}[hp03_096]
%No known source.
%\end{philcomm}

\begin{metre}[hp03_096]
Anuṣṭubh (a: na-vipulā)
\end{metre}

%%%%%%%%%%
\subsection*{3.96*1}
\begin{translation}[hp03_096_1]
The yogi should regularly move the coiled (\emph{paristhitā}), hooded [serpent] by breathing in through the sun channel and holding her using a cloth for an hour and a half in the morning and evening.
\end{translation}
%greyscaled

%\begin{sources}[hp03_096_1]
%\end{sources}

\begin{testimonia}[hp03_096_1]
\emph{Yogacintāmaṇi} f.~79r (attr.~to the \emph{Haṭhayoga})
\begin{versinnote}
\tl{paristhitā caiva phaṇāvatī sā\\+}
\tl{prātaś ca sāyaṃ praharārdhamātram |\\+}
\tl{prapūrya sauryā paridhānamuktā\\+}
\tl{pragṛhya niryāti vicālitā sā ||\\!}
\end{versinnote}

\emph{Haṭhasaṅketacandrikā} f.~110v-111r (attr.~to the \emph{Haṭhapradīpikā})
\begin{versinnote}
\tl{tadvidhim āha |\\+}
\tl{paristhitā caiva phaṇāvatī sā\\+}
\tl{prātaś ca sāyaṃ praharārdhamātraṃ |\\+}
\tl{prapūrya sūryāt paridhānayuktā\\+}
\tl{pragṛhya tīrthāt paricālanīyā ||\\+}
\tl{paridhān[a]yukteti dvādaśāṅgulapramitasitasūkṣmacaturaṅgulavisṛtaśuddhavastrakhaṇḍena dṛḍhaṃ veṣṭatā sā prasiddhā [ph]aṇāvatī suṣumṇātmakā arundhatī jihvaiva kuṇḍalinī || uktaṃ ca ||\\+}
\tl{arundhatī bhavej jihvā dhruvo nāsāgramaṇḍalam iti || \\+}
\tl{tāṃ jihvāṃ laṃbikāyogenordhvaṃ tālvantarbhrūmadhyadeśe vihitāṃ tatas tīrthād bhrūmadhyāt pragṛhya adhaḥ kṛtvā tasyā gurūpadiṣṭavartmanā cālanaṃ vidheyam iti saṃketaḥ [|] cālanaṃ tu khecarī mudrā sādhanavad vidheyaṃ [|] tīrthaṃ bhrūmadhyaḥ [|] \\!}
\end{versinnote}

\emph{Haṭhatattvakaumudī} 44.5
\begin{versinnote}
\tl{paristhitasyeha phaṇāvatī sā\\+}
\tl{prātastu sāyaṃ praharārdhamātram |\\+}
\tl{prapūrya sūryāt paridhānayuktā\\+} 
\tl{pragṛhya niryāt paricālanīyā ||\\!}
\end{versinnote}
\end{testimonia}

\begin{philcomm}[hp03_096_1]
Verses 3.96*1–2 are absent in \alphaOne\ and \alphaTwo. They introduce the idea of awakening \emph{kụṇḍalinī} by moving the tongue with a cloth, which is a practice called \emph{sarasvatīcālana} in the \emph{Gorakṣaśataka} (16–25). These verses do not have a known source and are somewhat obscure unless one is one aware of the more coherent explanation of this practice in the \emph{Gorakṣaśataka}. In his \emph{Haṭhasaṅketacandrikā} (see testimonia), Sundaradeva makes sense of this verse by equating the tongue with Kuṇḍalinī. This enables him to understand the reference to the cloth (\emph{paridhāna}) as the technique of wrapping the tongue in a cloth and milking it, which is a practice called \emph{sarasvatīcālana} in the \emph{Gorakṣaśataka} (16–25). This interpretation also makes sense of the next verse in the \emph{Haṭhapradīpikā} (3.11), which describes the cloth.

%

%C2 - nityaṃ
%adopt yuktyā, nityam paricālanīyā (C6 - 4b group) [MD: done]
%GŚ 22 says to move the tongue for two muhūrtas
%adopt?:
%paristhitā caiva phaṇāvatī sā
%prātaś ca sāyaṃ praharārdhamātram
%prapūrya sūryāt paridhānayuktyā
%pragṛhya nityaṃ paricālanīyā
\end{philcomm}

\begin{metre}[hp03_096_1]
Upajāti
\end{metre}

%%%%%%%%%%
\subsection*{3.96*2}
\begin{translation}[hp03_096_2]
It is said that the characteristics of the cloth for wrapping around [the tongue] are a length measuring a handspan and a width of four fingerbreadths, and it is soft and white.
\end{translation}
%greyscaled

\begin{sources}[hp03_096_2]
Cf.~\emph{Gorakṣaśataka} 20cd
\begin{versinnote}
\tl{dvādaśāṅguladairghyaṃ cāmbaraṃ caturaṅgulam\\!}
\end{versinnote}
\end{sources}

\begin{testimonia}[hp03_096_2]
\emph{Yogabīja} 81 (South Indian recension)
\begin{versinnote}
\tl{vitastipramitaṃ dairghyaṃ vistāre caturaṅgulam |\\+}
\tl{mṛdulaṃ dhavalaṃ proktaṃ veṣṭanāmbaralakṣaṇam  ||\\!}
\end{versinnote}

\emph{Yogacintāmaṇi} f.~74r (attr.~to the \emph{Yogabīja} in the context of \emph{khecarīmudrā})
\begin{versinnote}
\tl{yogabīje—\\+}
\tl{vitastipramitaṃ dīrgha[ṃ] vistāraṃ caturaṅgulam |\\+}
\tl{mṛdulaṃ dhavalaṃ proktaṃ veṣṭanādhāralakṣaṇam ||\\!}
\end{versinnote}

\emph{Haṭhayogasaṃhitā} p. 44
\begin{versinnote}
\tl{vitastipramitaṃ dīrghaṃ vistāre caturaṅgulam |\\+}
\tl{mṛdulaṃ dhavalaṃ sūkṣmaṃ veṣṭanāmbaralakṣaṇam ||\\!}
\end{versinnote}
\end{testimonia}

\begin{philcomm}[hp03_096_2]
This verse was likely added to explain `by the method of the cloth' (\emph{paridhānayuktyā}) in the previous verse. One would expect to read \emph{caturaṅgulavistāram} in the second verse quarter, and the current reading is probably a result of the metre.
\end{philcomm}


%%%%%%%%%%
\subsection*{3.97}
\begin{translation}[hp03_097]
Sitting in \emph{vajrāsana}, the yogī should make Kuṇḍalinī move and immediately afterwards perform \emph{bhastrī}. He quickly awakens Kuṇḍalinī.
\end{translation}

%\begin{sources}[hp03_097]
%\end{sources}

\begin{testimonia}[hp03_097]
\emph{Yogabīja} 111 (South Indian recension)
\begin{versinnote}
\tl{vajrāsanasthito yogī cālayitvā tu kuṇḍalīm |\\+}
\tl{kuryād anantaraṃ bhastrāṃ kuṇḍalīṃ āśu bodhayet ||\\!}
\end{versinnote}

\emph{Yogacintāmaṇi} f.~79r (attr.~to the \emph{Haṭhayoga})
\begin{versinnote}
\tl{vajrāsanasthito yogī cālayitvā tu kuṇḍalīm |\\+}
\tl{sūryād anantaraṃ bhastrā kuṇḍalīm āśu bodhayet ||\\!}
\end{versinnote}

\emph{Haṭhasaṅketacandrikā} f.~111r
\begin{versinnote}
\tl{vajrāsanasthito yogī cālayitvā tu kuṃḍalīṃ |\\+}
\tl{sūryād anantaraṃ bhastrī kuṃḍalīm āśu bodhayet ||\\!}
\end{versinnote}
\end{testimonia}

\begin{philcomm}[hp03_097]
\emph{Bhastrī} or \emph{bhastrikā kumbhaka} is taught at 2.60–68.The reference to \emph{vajrāsana} may be pointing to the practice of \emph{uḍḍiyāna}, which was described earlier in the chapter and is supposed to awaken Kuṇḍalinī. The contraction of the sun mentioned in the next verse supports this.
\end{philcomm}


%%%%%%%%%%
\subsection*{3.98}
\begin{translation}[hp03_098]
[The yogi] should contract the sun and then make Kuṇḍalinī move. Even if he were in the jaws of death, why would he fear death?%
\end{translation}

%\begin{sources}[hp03_098]
%\end{sources}

\begin{testimonia}[hp03_098]
\emph{Yogabīja} 83 (South Indian recension)
\begin{versinnote}
\tl{bhānor ākuñcanaṃ kuryāt kuṇḍalīṃ cālayet tataḥ | \\+}
\tl{mṛtyuvartmagatasyāpi tasya mṛtyubhayaṃ kutaḥ ||\\+}
\tl{\var{mṛtyuvartma° ] mṛtyuvaktra°,  mṛtyuvajra°, mṛtyupadma° \vl}\\!}
% mṛtyuvaktra° is supported by the North Indian recension and printed editions
\end{versinnote}

\emph{Yogacintāmaṇi} f.~79r (attr.~to the \emph{Haṭhayoga})
\begin{versinnote}
\tl{bhānor ākuñcanaṃ kuryāt kuṇḍalīṃ cālayet tataḥ |\\+}
\tl{mṛtyuvaktragatasyāpi tasya mṛtyubhayaṃ kutaḥ ||\\!}
\end{versinnote}

\emph{Haṭhasaṅketacandrikā} (of Sundaradeva) f.~111r
\begin{versinnote}
\tl{bhānor ākuñcanaṃ kuryāt kuṃḍalīṃ cālayet svataḥ |\\+}
\tl{mṛtyuvakragatasyāpi tasya mṛtyu[bha]yaṃ kutaḥ ||\\+}
\tl{asyārthaḥ sūryanāḍyākarṣaṇena vahneḥ prācuryaṃ tasmāj jvalanatejasā apānavāyvākarṣaṇena vā kuṇḍalyābodho bhavati [|] tasya śakticālanakṛtābhyāsasya mṛtyubhayaṃ kuta iti [||]\\!}
\end{versinnote}
\end{testimonia}

\begin{philcomm}[hp03_098]
In \emph{Jyotsnā} 3.116, Brahmānanda understands \emph{bhānor ākuñcanaṃ} as a contraction of the navel, whereas Sundaradeva in his \emph{Haṭhasaṃketacandrikā} (see testimonia) interprets it as drawing \emph{prāṇa} through the sun channel (\emph{sūryanāḍyākarṣaṇa}), thereby intensifiying the bodily fire.
\end{philcomm}


%%%%%%%%%%
\subsection*{3.98*1}
\begin{translation}[hp03_098_1]
When \emph{prāṇa} has been greatly extended, [the yogi's] breath flows through the path of the right nostril and [his] body is immortal, filled with the nectar of the moon from the uvula for the first time. Sprinkling [with nectar] the network of channels at the aperture at [the centre of the] brow that is subjugated by the mighty fire of time, [the yogi] then makes his body completely new again like [the regeneration of] the trunk of an ancient tree.
\end{translation}
%greyscaled

\begin{sources}[hp03_098_1]
\emph{Amaraughaśāsana} 6.1-2 (sic; a single śārdūlavikrīḍita verse is numbered thus)
\begin{versinnote}
\tl{nāsāpaścimamārgavāhapavanāt prāṇe 'tidīrghīkṛte\\+}
\tl{candrāmbupratisāraṇāṃ sukṛtinaḥ prāg ghaṇṭikāyāḥ pathaḥ |\\+}
\tl{siñcan kālaviśālavahnivaśagaṃ bhūtvā sa nāḍīśataṃ \\+}
\tl{tat kāryaṃ kurute punar navatanuṃ jīrṇadrumaskandhavat ||\\+}
\tl{pratisāraṇānantaraṃ śaṅkhasāraṇā kathyate\\+}
\tl{\var{nāḍīśataṃ ] nāḍīgataṃ, nāḍīgaṇaṃ \vl}\\!}
\end{versinnote}
\end{sources}

\begin{testimonia}[hp03_098_1]


\emph{Yogacintāmaṇi} f.~79r (attr.~to the \emph{Haṭhayoga})
\begin{versinnote}
\tl{nāsādakṣiṇamārgavāhipavano ghrāṇe 'tidīrghīkṛtaḥ\\+}
\tl{candrābhaḥparipūritāmṛtatanuḥ prāg ghaṇṭikāyās tataḥ |\\+}
\tl{bhindan kālaviśālavahnivaśagān bhrūrandhranāḍīgaṇān \\+}
\tl{taṃ kāyaṃ kurute punar navataraṃ jīrṇadrumaskandhavat ||\\!}
\end{versinnote}

\emph{Haṭhasaṅketacandrikā} (f. 111r–111v)
\begin{versinnote}
\tl{nāsādakṣiṇamārgavāhipavano ghrāṇe tidī[r]ghīkṛtaś\\+}
\tl{caṃdrāṃ'bhaḥparipūritā'mṛtatanuḥ prāgh ghaṃṭikāyās tataḥ [|]\\+}
\tl{bhindan kālaviśālavahnivaśagān bh[r]ū[ran]dhranāḍīgaṇāṃs \\+}
\tl{taṃ kāyaṃ kurute punar navataraṃ jīrṇadrumaskandhavat [||]\\+}
\tl{dakṣiṇe ghrāṇe nāsikāyām atidī[r]ghīkṛtaś ciraṃ kuṃbhakīkṛtaḥ prāk prathamaṃ caṃdrāṃ'bhaḥparipūritā'mṛtatanuḥ sādhakaḥ kartā pūrvam iḍayā dhṛtakuṃbhakenātisukhakarasudhopamena saṃtṛptiṃ samupagataḥ satatas tadanantaraṃ sūryanāḍyā cirāyā kalitaṃ kuṃbhauṣṇyātīkṣṇatarakuṃ tenaivauṣṇasvabhāvād ghaṇṭikāyāṃ jatruṇaḥ pañcamy arthaḥ tatsaṃbadhikān ity arthaḥ [|] evaṃvidhān kālabījān bh[r]ūrandhragataśirāpuñjān bhindan saṃchedayan svaṃ nijaśarīraṃ punar navataraṃ kuruta iti asyārthaḥ {|}\\!}
\end{versinnote}

%Yogasārasaṃgraha has this too, check
\end{testimonia}

\begin{philcomm}[hp03_098_1]
Verse 3.98*1 is absent in the \textalpha\ group. It is very close to a verse in the \textit{Amaraughaśāsana}, which is likely to be its source, although the date of the \textit{Amaraughaśāsana} is yet to be firmly established. This verse's import of rejuvenating the body by flooding it with nectar is not directly connected with those proceeding it. 

%Adopt °pavano prāṇe 'tidīrghīkṛte
%Adopt siñcan from AŚ, or (JT) understand bhindan to mean cutting off channels that are under the influence of time
%Adopt tat pāda 4 and understand as tasmāt
%If yogi is subject taṃ doesn't work, adopt tat
%Simile of tree coming back to life after being burned or withering, through nectar passing through the channels
\end{philcomm}

\begin{metre}[hp03_098_1]
Śārdūlavikrīḍita
\end{metre}

%%%%%%%%%%
\subsection*{3.98*2}
\begin{translation}[hp03_098_2]
After making Kuṇḍalinị move, the yogi should perform \emph{bhastrī} in particular. The god of death is afraid of the ascetic who regularly practises in this way.
\end{translation}
%greyscaled

%\begin{sources}[hp03_098_2]
%\end{sources}

\begin{testimonia}[hp03_098_2]
\emph{Yogacintāmaṇi} f.~79r (attr.~to the \emph{Haṭhayoga})
\begin{versinnote}
\tl{kuṇḍalīṃ cālayitvā tu kuryād bhastrāṃ viśeṣataḥ |\\+}
\tl{evamabhyāsato nityaṃ yaminaḥ śaṅkate yamaḥ ||\\!}
\end{versinnote}

\emph{Haṭhasaṅketacandrikā} (f. 111v–112r)
\begin{versinnote}
\tl{kuṃḍalīṃ cālayitvā tu kuryād bhastrīṃ viśeṣataḥ |\\+}
\tl{evamabhyāsato nityaṃ yaminaḥ śaṃphate manaḥ ||\\!}
\end{versinnote}
\end{testimonia}

\begin{philcomm}[hp03_098_2]
Verses 3.98*2–3 are absent in the \textalpha\ group and have no known source. They elaborate further on the instruction to practise \emph{bhastrī kumbhaka} in 3.97. The practice of other \emph{kumbhaka}s for moving Kuṇḍalinī is mentioned in 3.98*3. 

The reading \emph{abhyāsato} in 3.98*2c, which is attested by manuscripts of the \textgamma\ group and the testimonia, is possible but seems to be a later corruption of the participle, \emph{abhyasyato}.
\end{philcomm}


%%%%%%%%%%
\subsection*{3.98*3}
\begin{translation}[hp03_098_3]
Then [the yogi] should practise \emph{sūryabheda}, \emph{ujjāyī} and also \emph{śītalī}. Where is the god of death for the ascetic engaged in the practice in this way?
\end{translation}
%greyscaled

%\begin{sources}[hp03_098_3]
%\end{sources}

\begin{testimonia}[hp03_098_3]
\emph{Yogacintāmaṇi} f.~79r (attr.~to the \emph{Haṭhayoga})
\begin{versinnote}
\tl{tadābhyaset sūryabhedam ujjāyīṃ cāpi śītalīm |\\+}
\tl{evam abhyāsayuktasya yamas tu yaminaḥ kutaḥ ||\\!}
\end{versinnote}

\emph{Haṭhasaṅketacandrikā} f.~112r
\begin{versinnote}
\tl{tadābhyaset sūryabhedam ujjāyīṃ vāpi śītalīm |\\+}
\tl{evamabhyāsayuktasya yamas tu yaminaḥ kutaḥ ||\\!}
\end{versinnote}
\end{testimonia}

\begin{philcomm}[hp03_098_3]
On why this verse is in greyscale, see the note to 3.98*2.

%No known source. the readings \emph{cāpi} and \emph{vāpi} are both possible, but the former is better attested.

%MD: adopt śamano in Pāda d? śamana = Yama.
\end{philcomm}

\begin{metre}[hp03_098_3]
Anuṣṭubh (a: ra-vipulā)
\end{metre}

%%%%%%%%%%
\subsection*{3.99}
\begin{translation}[hp03_099]
That fearless [yogi] should move [Kuṇḍalinī] for up to one hour 36 minutes (two \emph{muhūrta}s). Suṣumṇā at Kuṇḍalinī is drawn up slightly.
\end{translation}

\begin{sources}[hp03_099]
\emph{Gorakṣaśataka} 22c–23b
\begin{versinnote}
\tl{muhūrtadvayaparyantaṃ nirbhayaś cālayed imām |\\+}
\tl{ūrdhvam ākṛṣyate kiṃcit suṣumnā kuṇḍalīgatā ||\\+}
\tl{\var{22c nirbhayaś ] YL ; nirbharaś T, nirbhayāc GU}\\+}
\tl{\var{23a ākṛṣyate ] YL ; ākarṣayet TGU}\\+}
\tl{\var{23b suṣumnā kuṇḍalīgatā ] TG; suṣumnāṃ kuṇḍalīgatāṃ U,  suṣumnā kuṇḍalīyutā YL}\\!}
\end{versinnote}

Cf.~\emph{Śivasamhitā} 4.109
\begin{versinnote}
\tl{gurūpadeśavidhinā tasya mṛtyubhayaṃ kutaḥ |\\+}
\tl{muhūrtadvayaparyantaṃ vidhinā śakticālanam ||\\!}
\end{versinnote}

\end{sources}

\begin{testimonia}[hp03_099]
\emph{Haṭharatnāvalī} 2.121
\begin{versinnote}
\tl{muhūrtadvayaparyantaṃ nirbhītaś cālayed asau |\\+}
\tl{ūrdhvam ākṛṣyate kiṃ cit suṣumṇāṃ kuṇḍalīgatām |\\+}
\tl{ṣaṇmāsāc cālanenaiva śaktis tasyordhvagā bhavet ||\\+}
\tl{\var{ākṛṣyate] ākṛṣya tau P, T, t1. kuṇḍalīgatām ] kuṇḍalī gatā P,T,t1}\\!}
\end{versinnote}

\emph{Yogalakṣaṇāvalī} (f.~31r) (attr.~to the \emph{Gorakṣaśata})
\begin{versinnote}
\tl{muhūrtadvayaparyantaṃ nirbhayaś cālayed imām |\\+}
\tl{ūrdhvam ākṛṣyate kiṃcit suṣumnā kuṇḍalīyutā ||\\!}
\end{versinnote}

\emph{Yogacintāmaṇi} f.~79r (attr.~to the \emph{Haṭhayoga})
\begin{versinnote}
\tl{muhūrtadvayaparyantaṃ nirbharaṃ cālanād dhi vai |\\!}
\end{versinnote}

\emph{Haṭhasaṅketacandrikā} (f. 112r)
\begin{versinnote}
\tl{muhūrtadvayaparyantaṃ nirbharaṃ cālanād asau |\\+}
\tl{ūrdhvam ākṛṣyate kiṃcit suṣumnākuṇḍalīgatam ||\\+}
\tl{ku[ṇ]ḍalyās tadānīṃ gatam upari yātaṃ kiṃ cit svalpamātra abhyāsasadṛśam ity arthaḥ [|] akṛṣyate uccaiḥ karoti [|] \\!}
\end{versinnote}
\end{testimonia}

\begin{philcomm}[hp03_099]
%?? Add reference to introduction
As noted in our introduction, Svātmārāma appears not to have understood the practice of \emph{śakticālana} in the same way as his primary source text for its description, the \emph{Gorakṣaśataka}, in which a cloth is wrapped around the tongue so that it can be repeatedly pulled, thereby lifting up the base of the central channel. He does not include the \emph{Gorakṣaśataka} verses which mention the tongue or the cloth (but some later recensions of the \emph{Haṭhapradīpikā} do introduce them). Verses 98 and 99 suggest that he understood the practice to involve repeated contraction of the region of the sun at the lower end of the central channel. The result is the same, namely that Kuṇḍalinī is awakened and uncoils herself, thereby allowing Prāṇa to enter the central channel.
\end{philcomm}


%%%%%%%%%%
\subsection*{3.100}
\begin{translation}[hp03_100]
Extracted from Suṣumṇā by this [practice], Kuṇḍalinī leaves it. As a result of this, \emph{prāṇa} automatically enters Suṣumṇā.
\end{translation}

\begin{sources}[hp03_100]
\emph{Gorakṣaśataka} 23c-24b
\begin{versinnote}
\tl{tena kuṇḍalinī tasyāḥ suṣumnāyā mukhaṃ dhruvam |\\+}
\tl{jahāti tasmāt prāno’yaṃ suṣumnāṃ vrajati svataḥ ||\\!}
\end{versinnote}
\end{sources}

\begin{testimonia}[hp03_100]
\emph{Haṭhatattvakaumudī} 44.25
\begin{versinnote}
\tl{tadā kuṇḍalinī tasyāḥ suṣumṇāyā mukhaṃ dhruvam |\\+}
\tl{jahāti tasmāt prāṇo 'yaṃ suṣumṇāṃ vrajati svataḥ || \\!}
\end{versinnote}

\emph{Haṭhasaṅketacandrikā} f.~112r
\begin{versinnote}
\tl{tena proktaśakticālanena vidhinā kuṇḍalinī tasyāḥ suṣumṇāyāḥ samuddhṛtā jahāti tasmāt prāṇo [']yaṃ suṣumṇāṃ vrajati svataḥ [|] suṣumṇāyā antaḥ kiṃ cit tatka[r]tṛkordhvākarṣaṇena samyag ūrdhvavihitā yadā kuṇḍalī bhūry antaḥ praviṣṭety arthaḥ [|]   tadāyaṃ va[hny]āpānamanobhiḥ sārdhaṃ vijitaḥ kuṃḍalīpadaṃ prāptaḥ prāṇavāyuḥ svataḥ svasmāt pārthivarājasavikāraśoṣam iti śeṣaḥ [|] jahāti kuṃḍalībodhe suṣumṇāṃtaḥ pātaprabhāvād vigatāśeṣabāhyavāhaprasaṃ[ga] iti bhāvaḥ || tasmād dhetoḥ suṣumṇaṃ gacchatīti kevalakuṃbhako bhavatīty arthaḥ ||\\!}
\end{versinnote}
\end{testimonia}

\begin{philcomm}[hp03_100]
We have supplied `mouth' (\emph{mukha}) in our translation on the basis of the reading found in the source text (the \emph{Gorakṣaśataka}).
\end{philcomm}

\begin{metre}[hp03_100]
Anuṣṭubh (c: ma-vipulā)
\end{metre}

%%%%%%%%%%
\subsection*{3.101}
\begin{translation}[hp03_101]
Therefore [the yogi] should regularly make \emph{arundhatī} move, she who contains speech. By making her move the yogi is freed from diseases.
\end{translation}
%

\begin{sources}[hp03_101]
\emph{Gorakṣaśataka} 26cd–27ab
\begin{versinnote}
\tl{tasmāt saṃcālayen nityaṃ śabdagarbhāṃ sarasvatīm |\\+}
\tl{yasyāḥ saṃcālanenaiva yogī rogaiḥ pramucyate ||\\!}
\end{versinnote}
\end{sources}

\begin{testimonia}[hp03_101]
Cf.~\emph{Haṭharatnāvalī} 2.122
\begin{versinnote}
\tl{sūryeṇa pūrayed vāyuṃ sarasvatyās tu cālayet |\\+}
\tl{śabdagarbhācālanena yogī rogaiḥ pramucyate ||\\!}
\end{versinnote}

\emph{Yogalakṣaṇāvalī} f.~31r  (attr.~to the \emph{Gorakṣaśata})
\begin{versinnote}
\tl{tasmāt saṃcālayen nityaṃ śabdagarbhāṃ sarasvatī |\\+}
\tl{asyāḥ saṃcālanenaivaṃ rogā naśyaṃti niścitaṃ ||\\!}
\end{versinnote}

\emph{Haṭhasaṅketacandrikā} f.~112r (attr.~to the \emph{Haṭhapradīpikā})
\begin{versinnote}
\tl{tasmāt saṃcālayen nityaṃ śabdagarbhām arundhatīm ||\\+}
\tl{yasyāḥ saṃcālanenāśu yogī rogaiḥ pramucyate ||\\!}
\end{versinnote}

%Yogasārasaṅgraha (p.59)
%\begin{versinnote}
%\tl{tasmāt saṃcālayen nityaṃ śaṃbhugarbhām aruṃdhatīm |\\+}
%\tl{yasyāḥ sañcālanenāśu yogī rogaiḥ pramucyate ||\\!}
%\end{versinnote}
\end{testimonia}

\begin{philcomm}[hp03_101]
%comment on śabdagarbhām aruṃdhatīṃ.
In the \emph{Gorakṣaśataka} (26cd–27ab), the source text, this verse occurs in a passage on \emph{sarasvatīcālana}, which is the practice of moving the tongue (i.e.~\emph{sarasvatī}) by wrapping a cloth around it and tugging it in order to raise Kuṇḍalinī. In the \emph{Gorakṣaśataka}, \emph{sarasvatī} is also called \emph{arundhatī} and, since the tongue is instrumental for speech and \emph{sarasvatī} is the name of a Goddess identified with speech (\emph{vāc}), the \emph{Gorakṣaśataka}'s reading of \emph{śabdagarbhāṃ sarasvatīm} makes good sense. However, it seems that Svātmārāma has changed 3.101b to read \emph{śabdagarbhām arundhatīm} and has understood \emph{arundhatī} as Kuṇḍalinī. This is affirmed in longer versions of the \emph{Haṭhapradīpikā} (3.94*2) that contain a verse on synonyms of Kuṇḍalinī, which includes \emph{arundhatī}. We are yet to find \emph{arundhatī} equated with Kuṇḍalinī in a text composed before the \emph{Haṭhapradīpikā} but this identification is found in subsequent compendiums and commentaries (e.g.~\emph{Yogacintāmaṇi} f.~78v, \emph{Yuktabhavadeva} 7.300, \emph{Jyotsnā} 104, 119, \emph{Yogaprakāśikā} 5.166).
\end{philcomm}


%%%%%%%%%%
\subsection*{3.102}
\begin{translation}[hp03_102]
The yogi who has made Kuṇḍalinī move is worthy of success. There is no point in speaking at length about this. He easily conquers death.
\end{translation}

%\begin{sources}[hp03_102]
%\end{sources}

\begin{testimonia}[hp03_102]
\emph{Haṭharatnāvalī} 2.123
\begin{versinnote}
\tl{yena saṃcālitā śaktiḥ sa yogī siddhibhājanaḥ |\\+}
\tl{kim atra bahunoktena mṛtyuṃ jayati līlayā || 2.123 ||\\!}
\end{versinnote}

\emph{Yogalakṣaṇāvalī} (f. 31r)  (attrib. Gorakṣaśata)
\begin{versinnote}
\tl{yena saṃcālitā śaktiḥ śabdagarbhā tv aruṃdhatī |\\+}
\tl{kim atra bahunoktena tasya kālabhayaṃ na hi ||\\!}
\end{versinnote}

\emph{Haṭhasaṅketacandrikā} f.~112r (attr.~to the \emph{Haṭhapradīpikā})
\begin{versinnote}
\tl{yena saṃcālitā śaktiḥ sa yogī siddhibhājanaṃ |\\+}
\tl{kim atra bahunoktena kālaṃ jayati līlayā ||\\!}
\end{versinnote}
\end{testimonia}

%\begin{philcomm}[hp03_102]
%siddhibhājana is also possible.
%\end{philcomm}


%%%%%%%%%%
\subsection*{3.102*1}
\begin{translation}[hp03_102_1]
For [the yogi] devoted to celibacy and always eating a healthy and measured diet, success appears after forty days (\emph{maṇḍalāt}) as a result of engaging in the practice of Kuṇḍalinī.
\end{translation}
%greyscaled

%\begin{sources}[hp03_102_1]
%\end{sources}

\begin{testimonia}[hp03_102_1]
\emph{Yogacintāmaṇi} f.~79r (attr.~to the \emph{Haṭhayoga})
\begin{versinnote}
\tl{brahmacaryaratasyaiva nityaṃ hitamitāśinaḥ |\\+}
\tl{maṇḍalād dṛśyate siddhiḥ kuṇḍalyabhyāsayoginaḥ ||\\!}
\end{versinnote}

\emph{Yogalakṣaṇāvalī} (f. 31r)  (attrib. to Gorakṣaśataka)
\begin{versinnote}
\tl{brahmacaryavratasyaiva kuṃḍalyabhyāsayoginaḥ ||\\+}
\tl{maṇḍalād dṛśyate siddhir iti yogavido viduḥ ||\\!}
\end{versinnote}

\emph{Haṭhasaṅketacandrikā}  f.~112r--112v (attr.~to the \emph{Haṭhapradīpikā})
\begin{versinnote}
\tl{brahmacaryajatasyaiva nityaṃ hitamitāṃ śanaiḥ |\\+}
\tl{maṇḍalād dṛśyate siddhiḥ kuṇḍalyabhyāsayoginaḥ ||\\!}
\end{versinnote}
\end{testimonia}

\begin{philcomm}[hp03_102_1]
Verse 3.102*1 has been omitted by \alphaOne\ and \alphaThree. It is in \alphaTwo\ at the end of a block of verses (3.94*2–5, 94*7) that is excluded by \alphaThree. This block appears after 3.97 and appears to have been inserted from elsewhere. 3.102*1 has no known source and appears to have been added as a general laudatory statement on the benefits of practising with Kuṇḍalinī.

The meaning of \emph{maṇḍalād} in 3.102*1c is not clear. Brahmānanda understands it as a period of time (i.e., forty days) but we are yet to find this attested elsewhere.
\end{philcomm}


%%%%%%%%%%
\subsection*{3.103}
\begin{translation}[hp03_103]
The yogi should mix with ash the fluid of the moon emitted as a result of the practice. Wearing that [mixture] on the head bestows divine sight.
\end{translation}

%\begin{sources}[hp03_103]
%\end{sources}

\begin{testimonia}[hp03_103]
\emph{Yogalakṣaṇāvalī} f.~31r  (attr.~to \emph{Gorakṣaśataka})
\begin{versinnote}
\tl{abhyāsaniḥsṛtāṃ cāndrīṃ vibhūtyā saha miśrayet |\\+}
\tl{taddhāraṇaṃ cottamāṃge divyadṛṣṭipradāyakaṃ ||\\!}
\end{versinnote}

\emph{Haṭhasaṅketacandrikā} f.~112v (attr.~to the \emph{Haṭhapradīpikā})
\begin{versinnote}
\tl{abhyāsaniḥsṛtāṃ cāndrīṃ vibhūtyā saha miśrayet [|]\\+}
\tl{taddhāraṇaṃ tattamāṃge divyadṛṣṭipradāyakaṃ [||] 19\\+}
\tl{cāndrīṃ lalāṭacandrān niḥsṛtāṃ abhyāse śramajātāṃ gharmadhārāṃ tāṃ vibhūtyā vimiśrayet | tām uttamāṃge śirasi dhārayed asau sādhakasya divyadṛṣṭipradā bhravatīty arthaḥ [|]\\!}
\end{versinnote}

\emph{Haṭhayogasaṃhitā} p. 41 (on \emph{amarolī})
\begin{versinnote}
\tl{abhyāsān niḥsṛtāṃ cāndrīṃ vibhūtyā saha miśrayet |\\+}
\tl{dhārayed uttamāṅgeṣu divyadṛṣṭiḥ prajāyate || \\!}
\end{versinnote}
\end{testimonia}

\begin{philcomm}[hp03_103]
In the important manuscripts of the \emph{Haṭhapradīpikā}, including those of \textalpha, this verse occurs in the section on \emph{śakticālana}. This is also the case in the \emph{Yogalakṣaṇāvalī}, \emph{Haṭhasaṅketacandrikā} and the longer recensions of the \emph{Haṭhapradīpikā} with six and ten chapters. However, in the context of \emph{śakticālana}, the referent of \emph{cāndrī} is unclear. It appears to be understood as some sort of lunar fluid. Sundaradeva (see the \emph{Haṭhasaṅketacandrikā} in the testimonia) defines it as a flow of perspiration (\emph{gharmadhārā}) that arises from exertion in the practice and is emitted from moon in forehead (\emph{lalāṭacandra}). In \emph{Yogaprakāśikā}, Bālakṛṣṇa glosses it simply as nectar (\emph{sudhā}) (5.182). In the \emph{Haṭhayogasaṃhitā} (p. 41) and \emph{Jyotsnā} (3.98), this verse is in the section on \emph{amarolī}, which provides a clear referent of \emph{cāndrī} as the cool middle flow of urine (see 3.95).  

%Verse 120 is more appropriate for finishing the śakticālana section.
%Note Brahmānanda's interpretation of uttamāṅgeṣu, but usually uttamāṅga refers to the head.
\end{philcomm}

\begin{metre}[hp03_103]
Anuṣṭubh (c: ra-vipulā)
\end{metre}

%%%%%%%%%%
\subsection*{3.103*1}
\begin{translation}[hp03_103_1]
For purifiying the seventy-two thousand channels, there is no method of cleansing without the practice of Kuṇḍalinī.
\end{translation}
%greyscaled
%\begin{sources}[hp03_103_1]
%\end{sources}

\begin{testimonia}[hp03_103_1]
\emph{Yogacintāmaṇi} f.~79v (attr.~to the \emph{Haṭhayoga})
\begin{versinnote}
\tl{dvisaptatisahasrāṇāṃ nāḍīnāṃ malaśodhanam |\\+}
\tl{kutaḥ prakṣālanopāyaḥ kuṇḍalyabhyāsato vinā ||\\!}
\end{versinnote}

\emph{Yogalakṣaṇāvalī} f.~31r  (attrib. to Gorakṣaśataka)
\begin{versinnote}
\tl{dvisaptatisahasrāṇāṃ nāḍīnām api śodhanaṃ |\\+}
\tl{asatkalpaṃ smṛtaṃ siddhaiḥ kuṃḍalobhyasānād ṛte ||\\!}
\end{versinnote}

%\emph{Upāsanāsārasaṅgraha} (p. 36) % IFP Transcript T1095
%\begin{versinnote}
%\tl{dvāsaptatisahasrāṇāṃ nāḍīnāṃ malaśodhane | \\+}
%\tl{kutaḥ prakṣālanopāyaḥ kuṇḍalyabhyasanād ṛte |\\!}
%\end{versinnote}

\emph{Haṭhasaṅketacandrikā} f.~112v (attr.~to the \emph{Haṭhapradīpikā})
\begin{versinnote}
\tl{dvisaptatisahastrāṇāṃ nāḍīnāṃ malaśodhanaṃ |\\+}
\tl{kutaḥ prakṣālanopāyaḥ kuṃḍalyābhyāsanād ṛte ||\\!}
\end{versinnote}
\end{testimonia}

\begin{philcomm}[hp03_103_1]
This verse is omitted by the \textalpha\ group, and was probably added to the original text as a further laudatory statement on the practice of Kuṇḍalinī.
\end{philcomm}


%%%%%%%%%%
\subsection*{3.103*1 ending}
% iti śakticālanaṃ/
\begin{translation}[hp03_103_1p]
\end{translation}

% \begin{philcomm}[hp03_103_1p]
% \end{philcomm}

%%%%%%%%%%
\subsection*{3.104}
\begin{translation}[hp03_104]
Thus have the ten \emph{mudrā}s been taught by Śiva Ādinātha. Each one among them can bestow liberation for ascetics.
\end{translation}

\begin{sources}[hp03_104]
\end{sources}

\begin{testimonia}[hp03_104]

\emph{Haṭharatnāvalī} 2.35
\begin{versinnote}
\tl{iti mudrā daśa proktā ādināthena śambhunā |\\+}
\tl{ekaikā tāsu mukhyā syān mahāsiddhipradāyinī ||\\!}
\end{versinnote}

\emph{Yogacintāmaṇi} f.~79v
\begin{versinnote}
\tl{iti mudrā nava proktā ādināthena śambhunā |\\+}
\tl{ekaikā tāsu yamināṃ mahāsiddhipradāyinī ||\\!}
\end{versinnote}

\end{testimonia}

\begin{philcomm}[hp03_104]
Manuscripts of the \textbeta, \texteta, and \textepsilon\ groups have a different reading for the second line; `each mudrā is capable of being a cause of all powers' (\emph{kāraṇaṃ sarvasiddhīnāṃ ekaikāpi kṣamaiva sā}).
\end{philcomm}


%%%%%%%%%%
\subsection*{3.105}
\begin{translation}[hp03_105]
Without a king the earth is not resplendent (\emph{rājate}), without the moon the night does not sparkle (\emph{rājate}), without Rājayoga even the wonderful [practice of] \emph{mudrā} does not shine(\emph{rājate}).
\end{translation}

%\begin{sources}[hp03_105]

%\end{sources}

\begin{testimonia}[hp03_105]
\emph{Haṭharatnāvalī} 1.16
\begin{versinnote}
\tl{rājayogaṃ vinā pṛthvī rājayogaṃ vinā niśā |\\+}
\tl{rājayogaṃ vinā mudrā vicitrāpi na rājate ||\\!}
\end{versinnote}

%Yogasārasaṅgraha (p. 59)
%\begin{versinnote}
%\tl{rājayogaṃ vinā pṛthvī rājayogaṃ vinā niśi |\\+}
%\tl{rājayogaṃ vinā mudrā vicitrāpi na rājate ||\\!}
%\end{versinnote}
\end{testimonia}

\begin{philcomm}[hp03_105]
The \emph{Yogaprakāśikā} (5.186) interprets the similes in this verse as we have translated them, `Just as without a king [and] moon, the earth and night do not shine...' (\emph{yathā mahīpālaṃ candramasaṃ vinā pṛthvīniśe na rājete}...). However, in \emph{Jyotsnā} 3.126, Brahmānanda interprets the earth (\emph{pṛthvī}) as \emph{āsana} because both are connected by the quality of steadiness (\emph{sthairyaguṇayogāt}), and the night (\emph{niśā}) as breath retention (\emph{kumbhaka}) because both are characterised by the absence of movement of people and wind (\emph{prāṇasañcārābhāvalakṣaṇaḥ}). Brahmānanda's interpretation seems somewhat far-fetched. 
\end{philcomm}

\begin{metre}[hp03_105]
Anuṣṭubh (c: na-vipulā)
\end{metre}

%%%%%%%%%%
\subsection*{3.106}
\begin{translation}[hp03_106]
[The yogi] should carry out all breath practice with his mind engaged. The wise man must not let his attention wander.
\end{translation}

%\begin{sources}[hp03_106]
%\end{sources}

\begin{testimonia}[hp03_106]
\emph{Haṭhasaṅketacandrikā} f.~92v (attr.~to the \emph{Haṭhapradīpikā})
\begin{versinnote}
\tl{mārutābhyasanaṃ kiṃ cin manoyuktaṃ samācaret |\\+}
\tl{itaratra na kartavyā manovṛttir manīṣiṇā ||\\!}
\end{versinnote}
\end{testimonia}

%\begin{philcomm}[hp03_106]
%\end{philcomm}


%%%%%%%%%%
\subsection*{3.107}
\begin{translation}[hp03_107]
By means of postures, breath retentions and \emph{mudrā}s, the central channel, even though untraversed, becomes straight through yogis' firm practice.
\end{translation}
%

%\begin{sources}[hp03_107]
%\end{sources}

\begin{testimonia}[hp03_107]
\emph{Upāsanāsārasaṅgraha} p. 36 % IFP Transcript T1095
\begin{versinnote}
\tl{iyaṃ tu madhyamā nāḍī dṛḍhābhyāsena yoginām  |\\+}
\tl{āsanaprāṇasaṃyāmamudrābhiḥ saralā bhavet  ||\\!}
\end{versinnote}
\end{testimonia}

\begin{philcomm}[hp03_107]
The reading of \emph{khilāpi} in the first verse quarter is unusual but well attested by the witnesses including the \textalpha\ group of manuscripts. In the \emph{Abhidhāna\-cintāmaṇi} (940), \emph{khila} is defined as something uncultivated such as field (\emph{kṣetrādyaprahataṃ khilam}) or, as the \emph{Amarapadavivṛti} (2.1.5) puts it, `not marked by a plough' (\emph{lāṅgalena na likhitam iti khilam}). In the context of \emph{Haṭhapradīpikā} 3.107, qualifying \emph{suṣumnā} with \emph{khila} implies that the central channel has yet to be cultivated (i.e., traversed).


%khila seems odd (could it mean defective or unperfected here?), but the alternative readings are not helpful (except iyaṃ tu, which seems like a patch).
%Philipp: Apte: khila can mean blocked or a wedge
%Haru: khila - untraversed (aprahata), common in kośas.
%Group 6 has kuṭilā for khilāpi. Unlikely to be original but makes sense.

\end{philcomm}


%%%%%%%%%%
\subsection*{3.108}
\begin{translation}[hp03_108]
For those who are tireless in their dedication, Rājayoga has a \emph{mudrā}. That is the supreme \emph{rudrāṇi} \emph{mudrā}, which bestows beneficial success.
\end{translation}

%\begin{sources}[hp03_108]
%\end{sources}

%\begin{testimonia}[hp03_108]
%\end{testimonia}

\begin{philcomm}[hp03_108]
%Consider J5 emendation rājayogasamudbhavā ('produced from Rājayoga')
%samudraka : rājayoga has a mudrā, is sealed, has the seal of approval,
\emph{Rudrāṇī} may be the \emph{mudrā} usually called \emph{śāmbhavī}.

\end{philcomm}


%%%%%%%%%%
\subsection*{3.108*1}
\begin{translation}[hp03_108_1]
May [the yogi] who offers the traditional teaching of the \emph{mudrā}s be the guru, the master. He is none but the Lord himself.
\end{translation}
%greyscaled

%\begin{sources}[hp03_108_1]
%\end{sources}

\begin{testimonia}[hp03_108_1]
\emph{Upāsanāsārasaṅgraha} p. 40 % IFP Transcript T1095
\begin{versinnote}
\tl{upadeśaṃ hi mudrāṇāṃ yo datte sāṃpradāyikam |\\+}
\tl{sa eva śrīguruḥ svāmī sākṣād īśvara eva saḥ ||\\!}
\end{versinnote}
\end{testimonia}

\begin{philcomm}[hp03_108_1]
Verses 3.108*1–2 have no known source and are absent in the \textalpha\ group. It is likely both were added to the original text as further praise of those practising the haṭhayogic \emph{mudrā}s.
\end{philcomm}


%%%%%%%%%%
\subsection*{3.108*2}
\begin{translation}[hp03_108_2]
The yogi who has become intent on that [guru's] teaching and practises with a focused mind obtains mastery of the powers beginning with minimisation and the cheating of death.
\end{translation}
%greyscaled

%\begin{sources}[hp03_108_2]
%\end{sources}

%\begin{testimonia}[hp03_108_2]
%\end{testimonia}

\begin{philcomm}[hp03_108_2]
No version of this verse is entirely satisfactory. See the note on 3.108*1 for why it is in greyscale.
\end{philcomm}

\end{ekdosis}
\end{document}
