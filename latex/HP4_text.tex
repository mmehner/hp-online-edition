\documentclass[10pt]{memoir}
\setstocksize{220mm}{155mm} 	        
\settrimmedsize{220mm}{155mm}{*}	
\settypeblocksize{170mm}{116mm}{*}	
\setlrmargins{18mm}{*}{*}
\setulmargins{*}{*}{1.2}
% \setlength{\headheight}{5pt}
\checkandfixthelayout[lines]
\linespread{1.2}

\setlength{\footmarkwidth}{1.3em}
\setlength{\footmarksep}{0em}
\setlength{\footparindent}{1.3em}
\footmarkstyle{\textsuperscript{#1} }
\usepackage{fnpos}
\makeFNbottom

\usepackage[teiexport=tidy,poetry=verse]{ekdosis}
\usepackage{libertine}
\usepackage{sanskrit-poetry}
\usepackage{xcolor}

\usepackage[english]{babel}
\usepackage{babel-iast,xparse,xcolor}
\babelfont[iast]{rm}[Renderer=Harfbuzz, Scale=1.5]{AdishilaSan}
%\babelfont[english]{rm}[Scale=0.9]{Adobe Text Pro}
\babeltags{dev = iast}
\babeltags{eng = english}

\SetHooks{
	lemmastyle=\bfseries,
	refnumstyle=\selectlanguage{english}\color{blue}\bfseries, 
	}
\newif\ifinapparatus
\DeclareApparatus{default}[
	lang=english,
	sep = {] },
	delim=\hskip 0.75em,
	rule=none,
	]
\DeclareApparatus{notes}[
	lang=english,
	sep = {},
	delim=\hskip 0.75em,
	rule=\rule{0.7in}{0.4pt},
	]

%\DeclareShorthand{conj}{\texteng{\emph{conj.}}}{ego}
\DeclareShorthand{emend}{\texteng{\emph{em.}}}{ego}

\setlength{\vrightskip}{-10pt}
%\setlength{\vgap}{3mm} % default 1.5em
\verselinenumfont{\footnotesize\selectlanguage{english}\normalfont}
\setlength{\stanzaskip}{0.6\baselineskip}

%Define two commands: \skp ("sanskrit plus"), to be ignored by TeX in
%the edition text, but processed in the TEI output. Conversely, \skm
%("sanskrit minus") is to be processed in the edition text, but
%ignored if found in the apparatus criticus and in the TEI output:

\NewDocumentCommand{\skp}{m}{}
%\NewDocumentCommand{\skm}{m}{\unless\ifinapparatus#1-\fi}
\NewDocumentCommand{\skm}{m}{\unless\ifinapparatus#1\fi} % modified by MD 2022-05-31

%


%%%%%%%%%%%%%%%%%%%% THE  MSS         %%%%%%%%%%%%%%%%%%%%%%%%%%%

%%% Versions
\DeclareWitness{Vu}{\selectlanguage{english}Vulg}{Vulgate, i.e. Brahmānanda's version}[]           
\DeclareWitness{X}{\selectlanguage{english}X}{TenChapter Version, Jodhpur 02228 and 02225 (ed. Lonavla)}[]
\DeclareWitness{Six}{\selectlanguage{english}Ṣ}{SixChapterVersion, ``6ChapterHPms'', fragment of enlarged text, Jodhpur}[]
% Mss. in Geographical Groups
%%%% Varanasi mss (Sampūrṇānanda mss). V1 is Important
\DeclareWitness{V1}{\selectlanguage{english}V\textsubscript{1}}{Sampurnananda Library Sarasvati Bhavan 30109}[]
        \DeclareHand{V1ac}{V1}{\selectlanguage{english}V\rlap{\textsubscript{1}}\textsuperscript{ac}}[] % added by MD
        \DeclareHand{V1pc}{V1}{\selectlanguage{english}V\rlap{\textsubscript{1}}\textsuperscript{pc}}[] % added by MD
\DeclareWitness{V2}{\selectlanguage{english}V\textsubscript{2}}{Sampurnananda Library Sarasvati Bhavan 29869}[]
\DeclareWitness{V3}{\selectlanguage{english}V\textsubscript{3}}{Sampurnananda Library Sarasvati Bhavan 29899}[]
\DeclareWitness{V4}{\selectlanguage{english}V\textsubscript{4}}{Sampurnananda Library Sarasvati Bhavan 29937}[]
\DeclareWitness{V5}{\selectlanguage{english}V\textsubscript{5}}{Sampurnananda Library Sarasvati Bhavan 29938}[]
\DeclareWitness{V6}{\selectlanguage{english}V\textsubscript{6}}{Sampurnananda Library Sarasvati Bhavan 29991}[]
\DeclareWitness{V8}{\selectlanguage{english}V\textsubscript{8}}{Sampurnananda Library Sarasvati Bhavan 30014}[]
\DeclareWitness{V11}{\selectlanguage{english}V\textsubscript{11}}{Sampurnananda Library Sarasvati Bhavan 30029}[]
\DeclareWitness{V12}{\selectlanguage{english}V\textsubscript{12}}{Sampurnananda Library Sarasvati Bhavan 30030}[]
\DeclareWitness{V13}{\selectlanguage{english}V\textsubscript{13}}{Sampurnananda Library Sarasvati Bhavan 30031}[]
\DeclareWitness{V14}{\selectlanguage{english}V\textsubscript{14}}{Sampurnananda Library Sarasvati Bhavan 30050}[]
\DeclareWitness{V15}{\selectlanguage{english}V\textsubscript{15}}{Sampurnananda Library Sarasvati Bhavan 30051}[]
\DeclareWitness{V15pc}{\selectlanguage{english}V\rlap{\textsubscript{15}}\textsuperscript{pc}\space}{}[]
\DeclareWitness{V16}{\selectlanguage{english}V\textsubscript{16}}{Sampurnananda Library Sarasvati Bhavan 30052}[]
\DeclareWitness{V17}{\selectlanguage{english}V\textsubscript{17}}{Sampurnananda Library Sarasvati Bhavan 30053}[] % added by MD
\DeclareWitness{V16pc}{\selectlanguage{english}V\rlap{\textsubscript{16}}\textsuperscript{pc}\space}{}[]
\DeclareWitness{V18}{\selectlanguage{english}V\textsubscript{18}}{Sampurnananda Library Sarasvati Bhavan 30064}[]
\DeclareWitness{V19}{\selectlanguage{english}V\textsubscript{19}}{Sampurnananda Library Sarasvati Bhavan 30069}[]
\DeclareWitness{V21}{\selectlanguage{english}V\textsubscript{21}}{Sampurnananda Library Sarasvati Bhavan 30104}[]
\DeclareWitness{V22}{\selectlanguage{english}V\textsubscript{22}}{Sampurnananda Library Sarasvati Bhavan 30110}[]
\DeclareWitness{V25}{\selectlanguage{english}V\textsubscript{25}}{Sampurnananda Library Sarasvati Bhavan 30122}[]
\DeclareWitness{V26}{\selectlanguage{english}V\textsubscript{26}}{Sampurnananda Library Sarasvati Bhavan 30123}[]
\DeclareWitness{V28}{\selectlanguage{english}V\textsubscript{28}}{Sampurnananda Library Sarasvati Bhavan 30136}[]
\DeclareWitness{W2}{\selectlanguage{english}W\textsubscript{2}}{Wai ??}[]
\DeclareWitness{W4}{\selectlanguage{english}W\textsubscript{4}}{Wai 399-6171}[]

%%%%%%%%%%%%%%%%%%%%%%%%%%%%%%%%%
%%% Jammu & Kaschmir
\DeclareWitness{K1}{\selectlanguage{english}K\textsubscript{1}}{Raghunātha Temple Library 4383}[settlement=Jammu]
        \DeclareWitness{K1ac}{\selectlanguage{english}K\rlap{\textsubscript{1}}\textsuperscript{ac}\space}{}[]
        \DeclareWitness{K1pc}{\selectlanguage{english}K\rlap{\textsubscript{1}}\textsuperscript{pc}\space}{}[]
\DeclareWitness{K3}{\selectlanguage{english}K\textsubscript{3}}{Privat collection}
\DeclareWitness{L1}{\selectlanguage{english}L\textsubscript{1}}{SOAS RE 43454}[settlement=Jammu]
% More details? Catalogue number? L1 And C1 very close (and come from same region)
%%%%%%%%%%%%%%%%%%%%%%%%%%%%%%%%
% Jodhpur
% J10 is important
\DeclareWitness{J10}{\selectlanguage{english}J\textsubscript{10}}{MSPP Jodhpur 2230}[]
        \DeclareHand{J10ac}{J10}{\selectlanguage{english}J\rlap{\textsubscript{10}}\textsuperscript{ac}}[] % modified by MD
        \DeclareHand{J10pc}{J10}{\selectlanguage{english}J\rlap{\textsubscript{10}}\textsuperscript{pc}}[] % modified by MD
\DeclareWitness{J1}{\selectlanguage{english}J\textsubscript{1}}{Jodhpur 02231}[]
\DeclareWitness{J2}{\selectlanguage{english}J\textsubscript{2}}{Jodhpur 02232}[]   
\DeclareWitness{J3}{\selectlanguage{english}J\textsubscript{3}}{Jodhpur 02233}[]
\DeclareWitness{J4}{\selectlanguage{english}J\textsubscript{4}}{Jodhpur 02234}[]
        \DeclareWitness{J4ac}{\selectlanguage{english}J\rlap{\textsubscript{4}}\textsuperscript{ac}\space}{MSPP Jodhpur 02234}[]
        \DeclareWitness{J4pc}{\selectlanguage{english}J\rlap{\textsubscript{4}}\textsuperscript{pc}\space}{MSPP Jodhpur 02234}[]
\DeclareWitness{J5}{\selectlanguage{english}J\textsubscript{5}}{Jodhpur 02235}[]  % 4 chapters, 34 jpgs,   long colophon, missing lines in the beginning.
\DeclareWitness{J6}{\selectlanguage{english}J\textsubscript{6}}{Jodhpur 02237}[]  % 4 chapters, 41 jpgs
%\DeclareWitness{J6ac}{\selectlanguage{english}J\rlap{\textsubscript{6}}\textsubscript{ac}}{Jodhpur 02237}[]  % 4 chapters, 49 jpgs,   1st folio: idaṃ gulābarāyasya
% tulasīrāmaśarmmaṇaḥ putrasya pustakaṃ ...        End: iti śrīsahajānandasantānacintāmaṇisvātmārāmaviracitāyāṃ ..
% saṃvat 1802   (more consistent text)
%\DeclareWitness{J6pc}{\selectlanguage{english}J\rlap{\textsubscript{6}}\textsubscript{pc}}{Jodhpur 02237}[] 
\DeclareWitness{J7}{\selectlanguage{english}J\textsubscript{7}}{Jodhpur 02241}[]  % 4 chapters, 41 jpgs
\DeclareWitness{J8}{\selectlanguage{english}J\textsubscript{8}}{Jodhpur 23709}[]  % 4 chapters,  87 jpgs.   saṃvat 1724
\DeclareHand{J8ac}{J8}{\selectlanguage{english}J\rlap{\textsubscript{8}}\textsuperscript{ac}}[]  % changed by MD
\DeclareHand{J8pc}{J8}{\selectlanguage{english}J\rlap{\textsubscript{8}}\textsuperscript{pc}}[]  % changed by MD
\DeclareWitness{J9}{\selectlanguage{english}J\textsubscript{9}}{Jodhpur 02224}[]  %  fragment, 20 jpgs.
\DeclareWitness{J11}{\selectlanguage{english}J\textsubscript{11}}{Jodhpur 23532}[]
        \DeclareHand{J11ac}{J11}{\selectlanguage{english}J\rlap{\textsubscript{11}}\textsuperscript{ac}}[] % added by MD
        \DeclareHand{J11pc}{J11}{\selectlanguage{english}J\rlap{\textsubscript{11}}\textsuperscript{pc}}[] % added by MD
\DeclareWitness{J12}{\selectlanguage{english}J\textsubscript{12}}{Jodhpur 18552}[] 
\DeclareWitness{J13}{\selectlanguage{english}J\textsubscript{13}}{Jodhpur 02229}[]  %  5 chapters, 93 jpgs.
\DeclareWitness{J14}{\selectlanguage{english}J\textsubscript{14}}{Jodhpur 02239}[]  %  4 chapters
\DeclareWitness{J15}{\selectlanguage{english}J\textsubscript{15}}{Jodhpur 9732A}[]
\DeclareWitness{J16}{\selectlanguage{english}J\textsubscript{16}}{Jodhpur 9732B}[]
\DeclareWitness{J17}{\selectlanguage{english}J\textsubscript{17}}{Jodhpur 3013}[]
% Haṭhapradīpikā with (non-Sanskrit) Bhāṣya RORI Jodhpur ACC.NO.18552
%  Haṭhapradīpikā with (non-Sanskrit) commentary, RORI Alwar 952, 4 chapters,  colophon of the comm:
% iti śrīlāhorīmiśravrajabhūṣanaviracitāyāṃ bhāvārthadīpikāyāṃ caturthodhyāya ..    
%  Haṭhapradīpikā (5 chapter) MSPP Jodhpur ACC.NO.02229/

%%%%%%%%%%        Bodleian, Oxford
\DeclareWitness{B1}{\selectlanguage{english}B\textsubscript{1}}{Bodleian Library No. d.457(8)}[settlement=Oxford]
\DeclareWitness{B2}{\selectlanguage{english}B\textsubscript{2}}{Bodleian Library No. d.458(1)}[settlement=Oxford]
\DeclareWitness{B3}{\selectlanguage{english}B\textsubscript{3}}{Bodleian Library No. d.458(9)}[settlement=Oxford]

%%%%%%%%%%%   Chandigarh
\DeclareWitness{C1}{\selectlanguage{english}C\textsubscript{1}}{Lalchand M-2080}[]%L1 And C1 very close (and come from same region)
\DeclareWitness{C2}{\selectlanguage{english}C\textsubscript{2}}{Lalchand M-6065}[]
\DeclareWitness{C3}{\selectlanguage{english}C\textsubscript{3}}{Lalchand M-1293}[]
\DeclareWitness{C4}{\selectlanguage{english}C\textsubscript{4}}{Lalchand M-2081}[]
\DeclareWitness{C4ac}{\selectlanguage{english}C\rlap{\textsubscript{4}}\textsuperscript{ac}\space}{}[]
\DeclareWitness{C4pc}{\selectlanguage{english}C\rlap{\textsubscript{4}}\textsuperscript{pc}\space}{}[]
\DeclareWitness{C5}{\selectlanguage{english}C\textsubscript{5}}{Lalchand M-2082}[]%doesn't have chapter 1
\DeclareWitness{C6}{\selectlanguage{english}C\textsubscript{6}}{Lalchand M-2089}[]
\DeclareWitness{C7}{\selectlanguage{english}C\textsubscript{7}}{Lalchand M-6494}[]
\DeclareWitness{C8}{\selectlanguage{english}C\textsubscript{8}}{Lalchand M-2091}[]
        \DeclareHand{C8ac}{C8}{\selectlanguage{english}C\rlap{\textsubscript{8}}\textsuperscript{ac}}[]
        \DeclareHand{C8pc}{C8}{\selectlanguage{english}C\rlap{\textsubscript{8}}\textsuperscript{pc}}[]
\DeclareWitness{C9}{\selectlanguage{english}C\textsubscript{9}}{Lalchand M-4530}[]


% %%%%%%%%%%        Nepalese
\DeclareWitness{N1}{\selectlanguage{english}N\textsubscript{1}}{NGMPP A1400-2}[]
\DeclareWitness{N2}{\selectlanguage{english}N\textsubscript{2}}{NGMPP B 39-19}[]
\DeclareWitness{N3}{\selectlanguage{english}N\textsubscript{3}}{NGMPP B 62-20}[]
\DeclareWitness{N5}{\selectlanguage{english}N\textsubscript{5}}{NGMPP A60-15 + A61-1}[]
\DeclareWitness{N4}{\selectlanguage{english}N\textsubscript{4}}{NGMPP A61-2}[]
\DeclareWitness{N6}{\selectlanguage{english}N\textsubscript{6}}{NGMPP A61-6}[]
\DeclareWitness{N9}{\selectlanguage{english}N\textsubscript{9}}{NGMPP A62-33}[]
\DeclareWitness{N10}{\selectlanguage{english}N\textsubscript{10}}{NGMPP A62-37}[]
\DeclareWitness{N11}{\selectlanguage{english}N\textsubscript{11}}{NGMPP A63-15}[]
\DeclareWitness{N12}{\selectlanguage{english}N\textsubscript{12}}{NGMPP A939-19}[]
\DeclareWitness{N13}{\selectlanguage{english}N\textsubscript{13}}{NGMPP A1378-18}[]
\DeclareWitness{N16}{\selectlanguage{english}N\textsubscript{16}}{NGMPP B39-20}[]
\DeclareWitness{N17}{\selectlanguage{english}N\textsubscript{17}}{NGMPP B 111-10}[]
\DeclareWitness{N18}{\selectlanguage{english}N\textsubscript{18}}{NGMPP E 929-3}[]
\DeclareWitness{N19}{\selectlanguage{english}N\textsubscript{19}}{NGMPP E-1528-1 / E-1527-7(4)}[]
\DeclareWitness{N20}{\selectlanguage{english}N\textsubscript{20}}{NGMPP E 2037-13 }[]
\DeclareWitness{N21}{\selectlanguage{english}N\textsubscript{21}}{NGMPP E 2097-31}[]
\DeclareWitness{N22}{\selectlanguage{english}N\textsubscript{22}}{NGMPP G 4-4}[]
\DeclareWitness{N23}{\selectlanguage{english}N\textsubscript{23}}{NGMPP G 25-2}[]
        \DeclareHand{N23ac}{N23}{\selectlanguage{english}N\rlap{\textsubscript{23}}\textsuperscript{ac}}[] % added by MD
        \DeclareHand{N23pc}{N23}{\selectlanguage{english}N\rlap{\textsubscript{23}}\textsuperscript{pc}}[] % added by MD
\DeclareWitness{N24}{\selectlanguage{english}N\textsubscript{24}}{NGMPP G 190-16}[]
\DeclareWitness{N24ac}{\selectlanguage{english}N\rlap{\textsubscript{24}}\textsuperscript{ac}\space}{}[]
\DeclareWitness{N24pc}{\selectlanguage{english}N\rlap{\textsubscript{24}}\textsuperscript{pc}\space}{}[]
\DeclareWitness{N26}{\selectlanguage{english}N\textsubscript{26}}{NGMPP T 24-3}[]

% %%%%%%%%%%        Pune

\DeclareWitness{P1}{\selectlanguage{english}P\textsubscript{1}}{Ānandāśrama S16-3-21}[]
\DeclareWitness{P2}{\selectlanguage{english}P\textsubscript{2}}{Ānandāśrama S16-2-20}[]
\DeclareWitness{P3}{\selectlanguage{english}P\textsubscript{3}}{BISM (79) 314}[]
\DeclareWitness{P4}{\selectlanguage{english}P\textsubscript{4}}{BISM (91) 191}[]
\DeclareWitness{P5}{\selectlanguage{english}P\textsubscript{5}}{BISM (29) 5790}[]
\DeclareWitness{P6}{\selectlanguage{english}P\textsubscript{6}}{BORI 263/1879-80}[]
\DeclareWitness{P7}{\selectlanguage{english}P\textsubscript{7}}{BORI 665/1883-84}[]
\DeclareWitness{P8}{\selectlanguage{english}P\textsubscript{8}}{BORI 316/1895-98}[]
\DeclareWitness{P9}{\selectlanguage{english}P\textsubscript{9}}{BORI 733-1891-95}[]
\DeclareWitness{P10}{\selectlanguage{english}P\textsubscript{10}}{BORI 222-1884-86}[]
\DeclareWitness{P11}{\selectlanguage{english}P\textsubscript{11}}{BORI 221-1882–83}[]
\DeclareWitness{P12}{\selectlanguage{english}P\textsubscript{12}}{Ānandāśrama S16-3-24}[]
\DeclareWitness{P13}{\selectlanguage{english}P\textsubscript{13}}{Ānandāśrama S16-2-22}[]
\DeclareWitness{P14}{\selectlanguage{english}P\textsubscript{14}}{Ānandāśrama S16-3-23}[]
\DeclareWitness{P15}{\selectlanguage{english}P\textsubscript{15}}{BISM (64) 919}[]
\DeclareWitness{P16}{\selectlanguage{english}P\textsubscript{16}}{BISM (64) 1115}[]
\DeclareWitness{P17}{\selectlanguage{english}P\textsubscript{17}}{BISM 620/1886-92}[]
\DeclareWitness{P18}{\selectlanguage{english}P\textsubscript{18}}{BORI 615/1887-91}[]
\DeclareWitness{P19}{\selectlanguage{english}P\textsubscript{19}}{BISM 46-39}[]
\DeclareWitness{P20}{\selectlanguage{english}P\textsubscript{20}}{BISM 39-273}[]
\DeclareWitness{P21}{\selectlanguage{english}P\textsubscript{21}}{BISM 37-743}[]
\DeclareWitness{P22}{\selectlanguage{english}P\textsubscript{22}}{BISM 37-729}[]
\DeclareWitness{P23}{\selectlanguage{english}P\textsubscript{23}}{BISM 33-60}[]
\DeclareWitness{P24}{\selectlanguage{english}P\textsubscript{24}}{BISM 29-5790}[]% =P5!
\DeclareWitness{P25}{\selectlanguage{english}P\textsubscript{25}}{BISM 29-3657}[]
\DeclareWitness{P26}{\selectlanguage{english}P\textsubscript{26}}{BISM 25-281}[]
\DeclareWitness{P27}{\selectlanguage{english}P\textsubscript{27}}{BISM 7-489}[]
\DeclareWitness{P28}{\selectlanguage{english}P\textsubscript{28}}{BORI 399-1895-1902}[]

%%%%%   Mysore
\DeclareWitness{M1}{\selectlanguage{english}M\textsubscript{1}}{P-5682/4}[]
%%%%%   Tübingen
\DeclareWitness{Tue}{\selectlanguage{english}Tü}{Ma I 339}[]
%%%%%%%%%%
\DeclareWitness{YC}{\selectlanguage{english}YC}{Yogacintāmaṇi}[]
\DeclareWitness{ceteri}{\selectlanguage{english}cett.}{ceteri}[]

%%%%%%%%%% Mss with Commentary
\DeclareWitness{A1}{\selectlanguage{english}A\textsubscript{1}}{Alwar 952}[]

\DeclareWitness{Jyo}{\selectlanguage{english}J\textsubscript{yo}}{Brahmānanda's version}[]

%%%%%%%%%%%%%%%%%%%%%%%%%%%%%%%%%%%%%%%%%%%
%List of all Sigla:
%A1,B1,B2,B3,C1,C2,C3,C4,C6,C7,C8,C9,J1,J2,J3,J4,J10,J13,J14,J15,J17,L1,M1,N3,N5,N6,N9,N10,N11,N12,N13,N16,N17,N19,N20,N21,N22,N23,N24,Tü,V1,V2,V3,V4,V5,V6,V8,V11,V19,V22,V26,Vu
%%%%%%%%%%%%%%%%%%%%%%%%%%%%%%%%%%%%%%%%%%%

\DeclareWitness{G4}{\selectlanguage{english}G\textsubscript{4}}{GOML D18885 (Bundle SD5051)}[]
\DeclareWitness{G5}{\selectlanguage{english}G\textsubscript{5}}{GOML R3841/ SR2190}[]
\DeclareWitness{G7}{\selectlanguage{english}G\textsubscript{7}}{GOML D4394}[]

\DeclareWitness{Ko}{\selectlanguage{english}K\textsubscript{o}}{Koba, Gujarat 55626}[]

%
%%%%%                   Abbreviation for the printed apparatus,        xml interface needed
%%%%%                   (synonyms in same line)

% Macro for Editing Abbrevs.
%\def\om{\textrm{\footnotesize \textit{omitted in}\ }} %prints om. for omitted in apparatus
%\def\korr{\textrm{\footnotesize \textit{em.}\ }} %prints em. for emended in apparatus
%\def\conj{\textrm{\footnotesize \textit{conj.}\ }} %prints conj. for conjectured in apparatus


\def\eyeskip{\textrm{{ab.\,oc. }}}   
\def\aberratio{\textrm{{ab.\,oc. }}}
\def\ad{\textrm{{ad}}}   
\def\add{\textrm{{add.\ }}}
\def\ann{\textrm{{ann.\ }}}
\def\ante{\textrm{{ante }}}
\def\post{\textrm{{post }}}
%\def\ceteri{cett.\,}             % for simplifying the apparatus in print                  
\def\codd{\textrm{{codd.\ }}}   %  the same
\def\conj{\textrm{{coni.\ }}}  
\def\coni{\textrm{{coni.\ }}}
\def\contin{\textrm{{contin.\ }}}
\def\corr{\textrm{{corr.\ }}}
\def\del{\textrm{{del.\ }}}
\def\dub{\textrm{{ dub.\ }}}
\def\emend{\textrm{{emend.\ }}}
\def\expl{\textrm{{explic.\ }}}   
\def\explicat{\textrm{{explic.\ }}}
\def\fol{\textrm{{fol.\ }}}         
\def\foll{\textrm{{foll.\ }}}
\def\gloss{\textrm{{glossa ad }}}
\def\ins{\textrm{{ins.\ }}}          \def\inseruit{\textrm{{ins.\ }}}
\def\im{{\kern-.7pt\lower-1ex\hbox{\textrm{\tiny{\emph{i.m.}}}\kern0pt}}}
\def\inmargine{{\kern-.7pt\lower-.7ex\hbox{\textrm{\tiny{\emph{i.m.}}}\kern0pt}}}
\def\intextu{{\kern-.7pt\lower-.95ex\hbox{\textrm{\tiny{\emph{i.t.}}}\kern0pt}}}%\textrm{\scriptsize{i.t.\ }}}               
\def\indist{\textrm{{indis.\ }}}          \def\indis{\textrm{{indis.\ }}}
\def\iteravit{\textrm{{iter.\ }}}          \def\iter{\textrm{{iter.\ }}}  
\def\lectio{\textrm{{lect.\ }}}             \def\lec{\textrm{{lect.\ }}}
\def\leginequit{\textrm{{l.n. }}}         \def\legn{\textrm{{l.n. }}}         \def\illeg{\textrm{{l.n. }}}
\def\om{\textrm{{om. }}}
\def\primman{\textrm{{pr.m.}}}
\def\prob{\textrm{{prob.}}}
\def\rep{\textrm{{repetitio }}}
% \def\secundamanu{\textrm{\scriptsize{s.m.}}}
% \def\secm{{\kern-.6pt\lower-.91ex\hbox{\textrm{\tiny{\emph{s.m.}}}\kern0pt}}}%   \textrm{\scriptsize{s.m.}}}
\def\sequentia{\textrm{{seq.\,inv.\ }}}         \def\seqinv{\textrm{{seq.\,inv.\ }}} \def\order{\textrm{{seq.\,inv.\ }}}
\def\supralineam{{\kern-.7pt\lower-.91ex\hbox{\textrm{\tiny{\emph{s.l.}}}\kern0pt}}} %\textrm{\scriptsize{s.l.}}}
\def\interlineam{{\kern-.7pt\lower-.91ex\hbox{\textrm{\tiny{\emph{s.l.}}}\kern0pt}}}   %\textrm{\scriptsize{s.l.}}}
\def\vl{\textrm{v.l.}}   \def\varlec{\textrm{v.l.}} \def\varialectio{\textrm{v.l.}}
\def\vide{\textrm{{cf.\ }}}           \def\cf{\textrm{{cf.\ }}}
\def\videtur{\textrm{{vid.\,ut}}}
\def\crux{\textup{[\ldots]} }
\def\cruxx{\textup{[\ldots]}}
\def\unm{\textit{unm.}}        % unmetrical
%%%%%%%%%%%%%%%%%%%%%%%%%%%%%%%%%%%%



%%% Local Variables:
%%% mode: latex
%%% TeX-master: t
%%% End:

% addition 2023-12-11 MD:
\TeXtoTEIPat{\begin {metre}[#1]}{<note type="metre" target="##1">}
\TeXtoTEIPat{\end {metre}}{</note>}
\TeXtoTEIPat{\texttheta}{θ}

% change 2023-12-05 mm
\TeXtoTEI{teimute}{} 

% changes/additions 2023-11-27 MM:
\TeXtoTEIPat{\medialink {#1}{#2}}{<ref target="resources/#2">#1</ref>}

% changes/additions 2023-10-25 MM:
% new Sigla
\TeXtoTEIPat{\textAlpha}{Α}
\TeXtoTEIPat{\textalpha}{α}
\TeXtoTEIPat{\textBeta}{Β}
\TeXtoTEIPat{\textbeta}{β}
\TeXtoTEIPat{\textGamma}{Γ}
\TeXtoTEIPat{\textgamma}{γ}
\TeXtoTEIPat{\textDelta}{Δ}
\TeXtoTEIPat{\textdelta}{δ}
\TeXtoTEIPat{\textEpsilon}{Ε}
\TeXtoTEIPat{\textepsilon}{ε}
\TeXtoTEIPat{\textEta}{Η}
\TeXtoTEIPat{\texteta}{η}
\TeXtoTEIPat{\textChi}{Χ}
\TeXtoTEIPat{\textchi}{χ}
\TeXtoTEIPat{\textOmega}{Ω}
\TeXtoTEIPat{\textomega}{ω}

%new environments
\TeXtoTEIPat{\begin {postmula}[#1]}{<note type="postmula" target="##1">}
  \TeXtoTEIPat{\end {postmula}}{</note>}
\TeXtoTEIPat{\begin {altava}[#1]}{<div type="altrec"><note type="avataranika" target="##1">} %%% changed 2023-12-05 mm
  \TeXtoTEIPat{\end {altava}}{</note></div>} %%% changed 2023-12-05 mm
\TeXtoTEIPat{\sgwit {#1}}{<note type="inlineref"><ref>#1</ref></note>}

% changes/additions 2023-10-12 MM:
\TeXtoTEIPat{\\.}{}

% changes/additions 2023-08-15 MD:
\TeXtoTEIPat{\lineom {#1}{#2}}{<note type="omission">#1 omitted in <ref>#2</ref></note>}
\TeXtoTEI{graus}{hi}[rend="grey"]
\TeXtoTEIPat{\startgray}{} %%% changed 2023-12-05 mm
\TeXtoTEIPat{\endgray}{} %%% changed 2023-12-05 mm



% additions/changes 2023-06-05 mm:
%\TeXtoTEIPat{\lineom {#1}}{<note type="omission">Line omitted in <ref>#1</ref></note>}
\TeXtoTEIPat{\NotIn {#1}}{<note type="omission">Stanza omitted in <ref>#1</ref></note>}

% additions 2023-04-16 MD:
\TeXtoTEIPat{\,}{ }

% additions 2023-04-13 mm:
\TeXtoTEIPat{\begin {versinnote}}{<lg>}
  \TeXtoTEIPat{\end {versinnote}}{</lg>}

% additions 2023-04-05 MD:
\TeXtoTEIPat{\begin {testimonia}[#1]}{<note type="testimonia" target="##1">}
  \TeXtoTEIPat{\end {testimonia}}{</note>}
\TeXtoTEI{devnote}{s}[xml:lang="sa-deva"]

% app in philcomm und testimonia %%% added MM 2023-12-02
\TeXtoTEI{var}{note}[type="appinnote"]


\TeXtoTEI{anm}{note}[type="memo"] %% change 2023-04-16 MD
\TeXtoTEI{Anm}{note}[type="memo"] %% change 2023-12-05 MM
\TeXtoTEIPat{\startverse}{} %%% marked for change 2023-04-13 mm
\TeXtoTEIPat{\endverse}{} %%% marked for change 2023-04-13 mm
\TeXtoTEIPat{\newpage}{}
\TeXtoTEIPat{\marma}{}
\TeXtoTEIPat{\marmas}{}
\TeXtoTEIPat{\vin}{} % added by MD 2023-11-14

%%% modify environments and commands
%%% TEI mapping
% additions/changes 2022-06-07 mm:
\TeXtoTEI{grau}{hi}[rend="grey"]
\TeXtoTEIPat{ \& }{ &amp; }

% additions/changes 2022-06-01 mm:
\TeXtoTEI{skp}{seg}[type="deva-ignore"]
\TeXtoTEI{skm}{seg}[type="ltn-ignore"]

\TeXtoTEIPat{\rlap {#1}}{#1}

% additions/changes 2022-04-06 mm:
%\TeXtoTEI{sgwit}{ref}
\TeXtoTEI{textdev}{s}[xml:lang="sa-deva"]
\TeXtoTEIPat{\begin {col}[#1]}{<div type="colophon" xml:id="#1"><p>}
  \TeXtoTEIPat{\end {col}}{</p></div>}
\TeXtoTEIPat{\begin {ava}[#1]}{<note type="avataranika" target="##1">}
  \TeXtoTEIPat{\end {ava}}{</note>}
												   
\TeXtoTEIPat{\outdent}{}
\TeXtoTEIPat{\startaltrecension}{} %%% changed 2023-12-05 mm
\TeXtoTEIPat{\endaltrecension}{} %%% changed 2023-12-05 mm
\TeXtoTEIPat{\startaltnormal}{} % added by MD 2023-11-14 %%% changed 2023-12-05 mm
\TeXtoTEIPat{\endaltnormal}{} % added by MD 2023-11-14 %%% changed 2023-12-05 mm
\TeXtoTEIPat{\begin {alttlg}[#1]}{<div type="altrec"><lg xml:id="#1">}
  \TeXtoTEIPat{\end {alttlg}}{</lg></div>}



% additions/changes 2022-03-12 mm:
\TeXtoTEIPat{\begin {tlg}[#1]}{<lg xml:id="#1">}
  \TeXtoTEIPat{\end {tlg}}{</lg>}

\TeXtoTEIPat{\begin {translation}[#1]}{<note type="translation" target="##1">}
  \TeXtoTEIPat{\end {translation}}{</note>}
\TeXtoTEIPat{\begin {philcomm}[#1]}{<note type="philcomm" target="##1">}
  \TeXtoTEIPat{\end {philcomm}}{</note>}
\TeXtoTEIPat{\begin {sources}[#1]}{<note type="sources" target="##1">}
  \TeXtoTEIPat{\end {sources}}{</note>}


\TeXtoTEIPat{\begin {marma}[#1]}{<note type="marma" target="##1">}
  \TeXtoTEIPat{\end {marma}}{</note>}

\TeXtoTEIPat{\begin {jyotsna}[#1]}{<note type="jyotsna" target="##1">}
  \TeXtoTEIPat{\end {jyotsna}}{</note>}

\EnvtoTEI{description}{list}
\EnvtoTEI{itemize}{list}
\TeXtoTEIPat{\item [#1]}{<label>#1</label>\item}

\TeXtoTEI{tl}{l}
\TeXtoTEI{myfn}{note}[type="myfn"]
\TeXtoTEIPat{\getsiglum {#1}}{<ref target="##1"/>}

\TeXtoTEI{SetLineation}{}
\TeXtoTEI{noindent}{}
\TeXtoTEI{subsection*}{}

\TeXtoTEI{rlap}{}

% end additions/changes
% \TeXtoTEIPat{\skp {#1}}{#1}
% \TeXtoTEIPat{\skm {#1}}{}

\TeXtoTEIPat{\begin {prose}}{<p>}
  \TeXtoTEIPat{\end {prose}}{</p>}

\TeXtoTEIPat{\begin {tlate}}{<p>}
  \TeXtoTEIPat{\end {tlate}}{</p>}

\TeXtoTEI{emph}{hi}
\TeXtoTEI{bigskip}{}
% \TeXtoTEI{/}{|}
\TeXtoTEI{tl}{l}
\TeXtoTEIPat{english}{}
%\TeXtoTEIPat{-}{ } %% change 2023-04-16 MD
%\TeXtoTEIPat{°}{} %% change 2023-04-16 MD
\TeXtoTEIPat{\textcolor {#1}{#2}}{<hi rend="#1">#2</hi>}

% \TeXtoTEIPat{\eyeskip}{}
% \TeXtoTEIPat{\aberratio}{}
% \TeXtoTEIPat{\ad}{}
\TeXtoTEIPat{\add}{<hi rend="italic">add.</hi>} %% change 2023-04-16 MD
% \TeXtoTEIPat{\ann}{}
\TeXtoTEIPat{\ante}{<hi rend="italic">ante</hi> } %% change 2023-04-16 MD
\TeXtoTEIPat{\post}{<hi rend="italic">post</hi> } %% change 2023-04-16 MD
% \TeXtoTEIPat{\codd}{}
% \TeXtoTEIPat{\conj }{}
% \TeXtoTEIPat{\contin}{}
% \TeXtoTEIPat{\corr}{}
% \TeXtoTEIPat{\del}{}
% \TeXtoTEIPat{\dub}{}
% \TeXtoTEIPat{\emend }{}
% \TeXtoTEIPat{\expl}{}
% \TeXtoTEIPat{\ȩxplicat}{}
% \TeXtoTEIPat{\fol}{}
% \TeXtoTEIPat{\gloss}{}
% \TeXtoTEIPat{\ins}{}
% \TeXtoTEIPat{\im}{}
% \TeXtoTEIPat{\inmargine}{}
% \TeXtoTEIPat{\intextu}{}
% \TeXtoTEIPat{\indist}{}
% \TeXtoTEIPat{\iteravit}{}
% \TeXtoTEIPat{\lectio}{}
% \TeXtoTEIPat{\leginequit}{}
% \TeXtoTEIPat{\legn}{}
% \TeXtoTEIPat{\illeg}{<hi rend="italic">illeg.</hi>}
\TeXtoTEIPat{\illeg}{<gap reason="illeg."/>} %%% change 2023-04-11 mm
% \TeXtoTEIPat{\om}{<hi rend="italic">om.</hi>}
\TeXtoTEIPat{\om}{<gap reason="om."/>} %%% change 2023-04-11 mm
% \TeXtoTEIPat{\primman}{}
% \TeXtoTEIPat{\prob}{}
% \TeXtoTEIPat{\rep}{}
% \TeXtoTEIPat{\sequentia}{}
% \TeXtoTEIPat{\supralineam}{}
% \TeXtoTEIPat{\interlineam}{}
\TeXtoTEIPat{\vl}{<hi rend="italic">v.l.</hi>}
% \TeXtoTEIPat{\vide}{}
% \TeXtoTEIPat{\videtur}{}
% \TeXtoTEIPat{\crux}{}
% \TeXtoTEIPat{\cruxxx}{}
\TeXtoTEIPat{\unm}{<hi rend="italic">unm.</hi>}


% List of Scholars
\DeclareScholar{nos}{nos}[
forename=HPP,
surname=Team]


% Nullify \selectlanguage in TEI as it has been used in
% \DeclareWitness but should be ignored in TEI.
\TeXtoTEI{selectlanguage}{}



% additions/changes 2022-04-06 mm:
%\NewDocumentEnvironment{ava}{O{}}{\begin{ekdpar}\SetLineation{lineation=none}}{\end{ekdpar}}
%\NewDocumentEnvironment{col}{O{}}{\begin{ekdpar}\SetLineation{lineation=none}}{\end{ekdpar}}

% end additions
% added by MM 2022-10-25:
\NewDocumentEnvironment{postmula}{O{}}{
  \begin{ekdverse}
    \hspace{-\vgap}}{
  \end{ekdverse}
  \vskip 0.6\baselineskip
}
% modified by MD 2022-05-8:
\NewDocumentEnvironment{ava}{O{}}{
  \begin{ekdverse}
    \hspace{-\vgap}}{
  \end{ekdverse}
  \vskip 0.6\baselineskip
}
\NewDocumentEnvironment{col}{O{}}{
  \medskip
  \setvnum{col}
%  \selectlanguage{iast}
  \begin{ekdverse}
    \hspace{-\vgap}}{
  \end{ekdverse}
}

        
% modifications/additions by MM 2022-06-07:
\NewDocumentEnvironment{altava}{O{}}{
  \begin{ekdverse}\color{gray}
    \hspace{-\vgap}}{
  \end{ekdverse}
  \vskip 0.6\baselineskip
}   

% end additions

\SetTEIxmlExport{autopar=false}

\NewDocumentEnvironment{tlg}{O{}}{
  \begin{ekdverse}}{
  \end{ekdverse}
  \vskip 0.6\baselineskip}

% additions/changes 2022-08-22 mm:
\NewDocumentEnvironment{alttlg}{O{}}{
%  \stopvline
%  \addtocounter{saved@poemline}{-1}
%  \setvnum{\hindsection.\arabic{saved@poemline}*\arabic{poemline}}
%  \selectlanguage{iast}
  \begin{ekdverse}[type=altrecension]
    \color{gray}
  }{
  \end{ekdverse}
  \vskip 0.6\baselineskip
%  \addtocounter{saved@poemline}{1}
%  \startvline
%  \setvnum{\hindsection.\arabic{poemline}}
%  \selectlanguage{iast}
}

% additions/changes 2022-08-22 mm:
\def\startaltrecension{
  \stopvline
  \addtocounter{saved@poemline}{-1}
  \setvnum{\hindsection.\arabic{saved@poemline}*\arabic{poemline}}
	%\selectlanguage{iast}
	%\begin{ekdverse}[type=altrecension]
	%\color{gray}
	\small  % added 2023-10-12 MD
	}
\def\endaltrecension{
	%\end{ekdverse}
	%\vskip 0.75\baselineskip
  \addtocounter{saved@poemline}{1}
  \startvline
  \setvnum{\hindsection.\arabic{poemline}}
%  \selectlanguage{iast}
	\normalsize  % added 2023-10-12 MD
	}

\def\startaltnormal{
	\stopvline
	\addtocounter{saved@poemline}{-1}
	\setvnum{\hindsection.\arabic{saved@poemline}*\arabic{poemline}}}
\def\endaltnormal{\endaltrecension}



\NewDocumentCommand{\tl}{m}{#1}

%%%%%%

\def\startverse{\begin{ekdverse}} % übergangsweise
\def\endverse{\end{ekdverse}\vskip 0.6\baselineskip} % übergangsweise
\def\startgray{\color{gray}} % NEW! 2023-06-16
\def\endgray{\color{black}} % NEW! 2023-06-16


%%%%%%

\newcommand{\myfn}[1]{\footnote{\texteng{#1}}}
\renewcommand{\thefootnote}{\texteng{\arabic{footnote}}}
\newcommand{\devnote}[1]{\textdev{\scriptsize #1}}
%\newcommand{\outdent}{\hspace{-\vgap}}
\newcommand{\sgwit}[1]{{\footnotesize (\getsiglum{#1})}}
\newcommand{\NotIn}[1]{\texteng{\footnotesize (om. \getsiglum{#1})}}
\newcommand{\lineom}[2]{\texteng{\footnotesize (#1 om. \getsiglum{#2})}}
\newcommand{\grau}[1]{\textcolor{gray}{#1}} % partial altrecension
\newcommand{\graus}[1]{\small\textcolor{gray}{#1}\normalsize} % partial altrecension
\newcommand{\Anm}[1]{\begin{ekdverse}
	\texteng{\footnotesize (#1)}
	\end{ekdverse}
	\vskip 0.6\baselineskip}
\newcommand{\anm}[1]{\texteng{\footnotesize [#1]}}

\def\om{\texteng{\emph{om.\kern-0.8ex}}}
\def\illeg{\texteng{\emph{illeg.\kern-0.8ex}}} 
\def\damaged{\texteng{\emph{damaged}}} 
\def\unm{\texteng{\emph{unm.\ }}}
\def\gap{\texteng{\emph{gap}}}
%\def\recte{\texteng{r.\:}}
%\def\for{\texteng{for\ }}
%\def\sic{\texteng{\emph{sic}}}
%\def\oder{\texteng{\emph{or\ }}}
\def\ante{\texteng{\normalfont\emph{ante\ }}}
\def\add{\texteng{\normalfont\emph{add.}}}
\def\post{\texteng{\normalfont\emph{post\ }}}
\def\antecorr{\texteng{\textsubscript{ac}}}
\def\postcorr{\texteng{\textsubscript{pc}}}
\def\marma{\texteng{\textsuperscript{\#}}}
\def\marmas{\texteng{\textsuperscript{\#}} }
\def\crux{\texteng{\textsuperscript{\textdagger}}}

\newcommand{\teimute}[1]{#1}

\usepackage{textgreek}

%%% Gr1,4b,6
\DeclareWitness{N3}{\texteng{\textalpha\textsubscript{1}}}{NGMPP B 62-20}[]
        \DeclareHand{N3ac}{N3}{\texteng{\textalpha\rlap{\textsubscript{1}}\textsuperscript{ac}}}[]
        \DeclareHand{N3pc}{N3}{\texteng{\textalpha\rlap{\textsubscript{1}}\textsuperscript{pc}}}[]
\DeclareWitness{J5}{\texteng{\textalpha\textsubscript{2}}}{Jodhpur 02235}[]
\DeclareWitness{G4}{\texteng{\textalpha\textsubscript{3}}}{GOML 18885}[]% Telugu script
\DeclareWitness{N24}{\texteng{\textalpha\textsubscript{4}}}{NGMPP G 190-16}[]
\DeclareWitness{Gr1r}{\texteng{\textAlpha *}}{Gr1 reconstructed}[]

\DeclareWitness{P11}{\texteng{\textbeta\textsubscript{1}}}{}[]
\DeclareWitness{C6}{\texteng{\textbeta\textsubscript{2}}}{Lalchand M-2089}[]

\DeclareWitness{V3}{\texteng{\textbeta\textsubscript{\textomega}}}{Sampurnananda Library Sarasvati Bhavan 29899}[]

%%% Gr2

\DeclareWitness{N23}{\texteng{\textgamma\textsubscript{1}}}{NGMPP G 25-2}[]
        \DeclareHand{N23ac}{N23}{\texteng{\textgamma\rlap{\textsubscript{1}}\textsuperscript{ac}}}[]
        \DeclareHand{N23pc}{N23}{\texteng{\textgamma\rlap{\textsubscript{1}}\textsuperscript{pc}}}[]
\DeclareWitness{J7}{\texteng{\textgamma\textsubscript{2}}}{Jodhpur 02241}[]
%\DeclareWitness{V6}{\texteng{V\textsubscript{6}}}{Sampurnananda Library Sarasvati Bhavan 29991}[]
\DeclareWitness{K1}{\texteng{K\textsubscript{1}}}{Raghunātha Temple Library 4383}[settlement=Jammu]
        \DeclareWitness{K1ac}{\texteng{K\rlap{\textsubscript{1}}\textsuperscript{ac}\space}}{}[]
        \DeclareWitness{K1pc}{\texteng{K\rlap{\textsubscript{1}}\textsuperscript{pc}\space}}{}[]


%%% Gr3

\DeclareWitness{V19}{\texteng{\textdelta\textsubscript{1}}}{Sampurnananda Library Sarasvati Bhavan 30069}[]
\DeclareWitness{K3}{\texteng{\textdelta\textsubscript{2}}}{Privat collection}
\DeclareWitness{C7}{\texteng{\textdelta\textsubscript{3}}}{Lalchand M-6494}[]
%\DeclareWitness{C1}{\texteng{C\textsubscript{1}}}{Lalchand M-2080}[]%L1 And C1 very close (and come from same region)
%\DeclareWitness{P23}{\texteng{P\textsubscript{23}}}{}[]
%\DeclareWitness{L1}{\texteng{L\textsubscript{1}}}{SOAS RE 43454}[settlement=Jammu]

\DeclareWitness{J6}{\texteng{\textdelta\textsubscript{\textomega}}}{Jodhpur 02237}[]
        \DeclareHand{J6ac}{J6}{\texteng{\textdelta\rlap{\textomega}\textsuperscript{ac}}}[]
        \DeclareHand{J6pc}{J6}{\texteng{\textdelta\rlap{\textomega}\textsuperscript{pc}}}[]

%%% Gr4c

\DeclareWitness{P15}{\texteng{\textepsilon\textsubscript{1}}}{}[]
\DeclareWitness{N19}{\texteng{\textepsilon\textsubscript{2}}}{NGMPP E-1528-1 / E-1527-7(4)}[]
\DeclareWitness{V15}{\texteng{\textepsilon\textsubscript{3}}}{Sampurnananda Library Sarasvati Bhavan 30051}[]
        \DeclareHand{V15ac}{V15}{\texteng{\textepsilon\rlap{\textsubscript{3}}\textsuperscript{ac}}}[]
        \DeclareHand{V15pc}{V15}{\texteng{\textepsilon\rlap{\textsubscript{3}}\textsuperscript{pc}}}[]
\DeclareWitness{J11}{\texteng{\textepsilon\textsubscript{4}}}{Jodhpur 23532}[]
        \DeclareHand{J11ac}{J11}{\texteng{\textepsilon\rlap{\textsubscript{4}}\textsuperscript{i.t.}}}[]
        \DeclareHand{J11pc}{J11}{\texteng{\textepsilon\rlap{\textsubscript{4}}\textsuperscript{mg.}}}[alternative reading written by the first hand in margin or interlinearly (J11)]
%\DeclareWitness{J14}{\texteng{\textepsilon\textsubscript{5}}}{Jodhpur 02239}[]

%\DeclareWitness{L2}{\texteng{L\textsubscript{2}}}{Wellcome Collection O.36]}
\DeclareWitness{M1}{\texteng{M\textsubscript{1}}}{P-5682/4}[]

\DeclareWitness{N26}{\texteng{\textepsilon\textsubscript{\textomega}}}{NGMPP}[]
%\DeclareWitness{V17}{\texteng{\textepsilon\textsubscript{\textomega 3}}}{Sampurnananda Library Sarasvati Bhavan 30053}[]

\DeclareWitness{V1}{\texteng{\texteta\textsubscript{1}}}{Sampurnananda Library Sarasvati Bhavan 30109}[]
        \DeclareHand{V1ac}{V1}{\texteng{\texteta\rlap{\textsubscript{1}}\textsuperscript{ac}}}[]
        \DeclareHand{V1pc}{V1}{\texteng{\texteta\rlap{\textsubscript{1}}\textsuperscript{pc}}}[]

%%% Gr4d

\DeclareWitness{J10}{\texteng{\texteta\textsubscript{2}}}{MSPP Jodhpur 2230}[]
        \DeclareHand{J10ac}{J10}{\texteng{\texteta\rlap{\textsubscript{2}}\textsuperscript{ac}}}[]
        \DeclareHand{J10pc}{J10}{\texteng{\texteta\rlap{\textsubscript{2}}\textsuperscript{pc}}}[]

\DeclareWitness{N9}{\texteng{\texteta\textsubscript{\textomega}}}{NGMPP A62-33}[]
%\DeclareWitness{J15}{\texteng{\textepsilon\textsubscript{\textomega 4}}}{Jodhpur 9732A}[]

%%%

\DeclareWitness{Jyo}{\texteng{\textchi}}{Brahmānanda's version}[]
%\DeclareWitness{Tue}{\texteng{Tü}}{Ma I 339}[]

\DeclareWitness{ceteri}{\texteng{cett.}}{ceteri}[]

%%% Group Sigla

\DeclareWitness{Gr1}{\texteng{\textAlpha}}{N3,J5,G4}

\DeclareWitness{Gr2}{\texteng{\textGamma}}{N23,J7}
%\DeclareWitness{Gr2}{\texteng{%
%	\textbeta\textsubscript{1}%
%	\textbeta\textsubscript{2}%
%	}}{N23,J7}
\DeclareWitness{Gr3a}{\texteng{\textDelta}}{V19,K3,C7}
\DeclareWitness{Gr4b}{\texteng{%
	\textbeta\textsubscript{1}%
	\textbeta\textsubscript{2}%
	}}{C6,P11}
\DeclareWitness{GrB}{\texteng{%
	\textbeta\textsubscript{1}%
	\textbeta\textsubscript{2}%
	\textbeta\textsubscript{\textomega}%
	}}{C6,P11,V3}
\DeclareWitness{Gr4c}{\texteng{\textEpsilon}}{P15,N19,V15}

% \DeclareWitness{Gr4d}{\texteng{%
	% \texteta\textsubscript{1}%
	% \texteta\textsubscript{2}%
	% }}{V1,J10}
\DeclareWitness{Gr6}{\texteng{\textOmega}}{V3,J6,N9,N26}

\makepagestyle{HPed}
\makeoddhead{HPed}{\small\texteng{HP4}}{}{\small\texteng{\today}}
\makeevenhead{HPed}{\small\texteng{HP4}}{}{\small\texteng{\today}}
\makeoddfoot{HPed}{}{\small\texteng{\thepage}}{}
\makeevenfoot{HPed}{}{\small\texteng{\thepage}}{}
\def\hindsection{4}

% Chp. 4 - N23,J7; V19,K3,C7; N19,V15; N3,C6(P11),V3; J10,Jyo

\begin{document}
\pagestyle{HPed}
\begin{otherlanguage}{iast}
\begin{ekdosis}


\teimute{\setcounter{saved@poemline}{1}}
\startaltnormal
\begin{altava}[hp04_000]
atha samādhiḥ/ \sgwit{N3,J5,Gr4b}% G4 broken
\end{altava}

\begin{alttlg}[hp04_000_1]
\tl{
\pada{\app{\lem[wit={ceteri}]{namaḥ}
	\rdg[wit={V3,N23,K3,C7}]{oṃ namaḥ}} śivāya gurave}
\pada{nādabindu%
	\app{\lem[wit={C6,Gr2,Gr3a,J10,Jyo}]{kalātmane}
		\rdg[wit={P11,V3,N19,V15}]{layātmane}% ##?
		}/}\\+}
\tl{
\pada{\app{\lem[wit={ceteri}]{nirañjanapadaṃ}
		\rdg[wit={V3}]{nirañjanaṃ padaṃ}
		\rdg[wit={N23},alt={\om}]{\skp{\om}}}
	\app{\lem[wit={ceteri}]{yāti}
		\rdg[wit={C6,V3,N19}]{yānti}}}
\pada{\app{\lem[wit={C6,V3,J7,V19,N19,V15,Jyo}]{nityaṃ}
		\rdg[wit={N23}]{aharniśaṃ}
		\rdg[wit={J10}]{yato}
		\rdg[wit={K3,C7}]{yatra}}
	\app{\lem[wit={P11,V3,V19,V15}]{yatra}
		\rdg[wit={Gr2,N19}]{yatna}% yatta? N23
		\rdg[wit={Jyo}]{tatra}
		\rdg[wit={C6}]{ca yat}
		\rdg[wit={J10}]{yogī}
		\rdg[wit={K3,C7}]{nityaṃ}}%
	\app{\lem[wit={ceteri}]{parāyaṇaḥ}
		\rdg[wit={C6,N19}]{parāyaṇāḥ}}//}\\!}
\end{alttlg}


\begin{alttlg}[hp04_000_2]
\tl{
\pada{\app{\lem[wit={ceteri}]{athedānīṃ} % °dāniṃ V15
		\rdg[wit={V3}]{athodānī}
		\rdg[wit={N23}]{athekṣanīṃ}}
	pravakṣyāmi} % vakṣāmi N19,N23,V19,V3
\pada{samādhikrama%
	\app{\lem[wit={C6,V3,N19,V15,J10,Jyo},alt={°m uttamam}]{\skp{°}m uttamam}
		\rdg[wit={Gr2,Gr3a}]{lakṣaṇam}}/}\\+}
\tl{
\pada{mṛtyughnaṃ % ghaṃ N23
	\app{\lem[wit={C6,V3,Gr2,K3,C7}]{tu}
		\rdg[wit={N19,V15,J10,Jyo}]{ca}
		\rdg[wit={V19}]{su}} sukhopāyaṃ} % muṣo° N23, mukho° N19
\pada{brahmānandakaraṃ param//}\\!}
\end{alttlg}


%\startaltrecension
\teimute{\small}
\begin{alttlg}[hp04_000_3]
\tl{\vin
\pada{\app{\lem[wit={V15,Jyo}]{rājayogaḥ}
		\rdg[wit={C6,N19,J10}]{rājayoga}}
	\app{\lem[wit={N19,V15},alt={samādhiḥ syād}]{samādhiḥ syā\skp{d}} % samādhi V15; +N22,P6
		\rdg[wit={C6,J10,Jyo}]{samādhiś ca}}d} % +G7
\pada{unmanī ca manonmanī/}\\+}
\tl{\vin
\pada{\app{\lem[wit={V15,J10}]{amaraugho}
		\rdg[wit={C6}]{amaraughi}
		\rdg[wit={N19}]{avaraubhū}
		\rdg[wit={Jyo}]{amaratvaṃ}}
	\app{\lem[wit={C6,N19,J10,Jyo},alt={layas}]{laya\skp{s}}
		\rdg[wit={V15}]{layes}}%
	\app{\lem[wit={C6,N19,V15,Jyo},alt={tattvaṃ}]{\skm{s }tattvaṃ}
		\rdg[wit={J10}]{tatra}}}
\pada{\app{\lem[wit={N19,V15,J10,Jyo}]{śūnyāśūnyaṃ} % °śūnya N19
		\rdg[wit={C6}]{śūnyāc chūnyaṃ}}
		paraṃ padaṃ//}\label{synonym3}
	\sgwit{C6,N19,V15,J10,Jyo} \anm{cf. \ref{A1}}\\!}
\end{alttlg}

\begin{alttlg}[hp04_000_4]
\tl{\vin
\pada{amanaskaṃ tathādvaitaṃ}
\pada{nirālambaṃ nirañjanam/}\\+}
\tl{\vin
\pada{jīvanmuktiś ca % jīvamuktiś N19, jīvanamu° V17
	\app{\lem[wit={C6,N19,J10}]{sahajaṃ}
		\rdg[wit={Jyo}]{sahajā}
		\rdg[wit={V15},alt={\om}]{\skp{\om}}}}
\pada{\app{\lem[wit={C6,V15}]{turyaṃ} % +J11
		\rdg[wit={N19}]{turyai}
		\rdg[wit={Jyo}]{turyā}
		\rdg[wit={J10}]{muktiś}}
	\app{\lem[wit={J10,Jyo},postwit=\texteng{(°kaḥ \getsiglum{J10ac})}]{cety ekavācakāḥ}
		\rdg[wit={C6}]{caityekavācakam}
		\rdg[wit={V15}]{cittaikavācakam}% +N22,P6
		\rdg[wit={N19}]{ciṃtaikavācakam}}//}\label{synonym4}
\sgwit{C6,N19,V15,J10,Jyo} \anm{cf. \ref{A2}}%
\myfn{\getsiglum{C6} has these verses on synonyms both here and at \ref{A1}/\ref{A2}, but \getsiglum{P11} has them at the latter place only.}\\!}
\end{alttlg}
%\endaltrecension
\teimute{\normalsize}

\begin{alttlg}[hp04_000_5]
\tl{
\pada{salile saindhavaṃ
	\app{\lem[wit={ceteri},alt={yadvat}]{yadva\skp{t}}
		\rdg[wit={N19}]{tadvat}}t}
\pada{sāmyaṃ \app{\lem[wit={C6,Gr2,Gr3a,J10,Jyo}]{bhajati}% ttadgati P11
		\rdg[wit={V3}]{bhajata}
		\rdg[wit={N19,V15}]{bhavati}} yogataḥ/}\\+} % °ta N19
\tl{
\pada{\app{\lem[wit={ceteri}]{tathā}
		\rdg[wit={V3}]{athā}
		\rdg[wit={J10}]{yathā}}%
	\app{\lem[wit={ceteri},alt={°tmamanasor}]{\skp{°}tmamanaso\skp{r}} % °saur N23
		\rdg[wit={J10}]{tmānamanor}}r aikyaṃ} % ma om. P7
\pada{samādhiḥ % °dhir J10,Jyo
	\app{\lem[wit={ceteri}]{so}
		\rdg[wit={J10,Jyo}]{a°}}%
	\app{\lem[wit={ceteri}]{'bhidhīyate}
		\rdg[wit={N19}]{'bhidhīte}
		\rdg[wit={N23}]{vidhīyate}}//}\label{salile}\\!}
\end{alttlg}

\Anm{\getsiglum{Jyo} has \ref{visvarupa} \textit{yadā saṃkṣīyate prāṇo} here%
\myfn{In the following, not all of the differences in the verse order of \getsiglum{GrB} and \getsiglum{Jyo} are noted. \getsiglum{GrB} follow the order of \getsiglum{Gr2} (or of \getsiglum{Gr3a}?) in the beginning and the end (after 4.72). The middle part is a kind of mix of \getsiglum{Gr2} and \getsiglum{N19,V15}. The verse order of \getsiglum{Jyo} is similar to that of \getsiglum{N19,V15}, but with many small differences.}}%

\newpage
%\startaltrecension
\teimute{\small}
\begin{alttlg}[hp04_000_6]
\tl{\vin
\pada{\app{\lem[wit={N19,V15}]{yat samatvaṃ dvayor eva}
	\rdg[wit={J10,Jyo}]{tat samaṃ ca dvayor aikyaṃ}}} % jat J10pc
\pada{jīvātmaparamātmanoḥ/}\\+}
\tl{\vin
\pada{\app{\lem[wit={N19,V15,J10}]{samastanaṣṭa}
	\rdg[wit={Jyo}]{pranaṣṭasarva}}%
\app{\lem[wit={V15,Jyo}]{saṃkalpaḥ}
	\rdg[wit={N19,J10}]{saṃkalpa}}}
\pada{samādhiḥ so'bhidhīyate//}
\sgwit{N19,V15,J10,Jyo}\label{yatsamatvam}%
\myfn{\getsiglum{J10} inserts another similar verse here:
\textit{karpūraṃ salile yadvat saindhavaṃ salile yathā |
tathātmamanasor aikyaṃ samādhiḥ so'bhidhīyate ||} (cf. \ref{karpura}ab and 4.3cd)}\\!}
\end{alttlg}
%\endaltrecension
\teimute{\normalsize}

%\newpage
\begin{alttlg}[hp04_000_7]
\tl{
\pada{rājayogasya
	\app{\lem[wit={ceteri}]{māhātmyaṃ}
		\rdg[wit={J7}]{māhatmyaṃ}
		\rdg[wit={V15}]{mahā}}}
\pada{ko vā jānāti tattvataḥ/}\\+} % ki N19; jānaṃti N19
\tl{
\pada{\crux\app{\lem[wit={ceteri},alt={jñānān}]{jñānā\skp{n}}
		\rdg[wit={V15,J10}]{jñāna}
		\rdg[wit={Jyo}]{jñānaṃ}
		\rdg[wit={V19}]{jñān}}%
	\app{\lem[resp=emend,alt={muktiḥ sthitā}]{\skm{n }muktiḥ sthitā}% muktisthitā P6
		\rdg[wit={C6,Gr2,K3,C7,Jyo}]{muktiḥ sthitiḥ}
		\rdg[wit={V19,J10},post=\texteng{(°sthiti<<ḥ>> \getsiglum{V19})}]{muktisthitiḥ}
		\rdg[wit={V3,N19}]{muktisthite} % muktisthitai P11, muktiḥ sthite K1
		\rdg[wit={V15}]{muktis tato}}
	\app{\lem[wit={C6,N19,V15,Jyo},alt={siddhir}]{siddhi\crux\skp{r}}% +K1
		\rdg[wit={V3,J10}]{siddhi}
		\rdg[wit={Gr2,Gr3a}]{siddhā}}r}
\pada{guru\app{\lem[wit={ceteri}]{vākyena}
		\rdg[wit={N23}]{vākyāt pra°}}
	\app{\lem[wit={ceteri}]{labhyate}
		\rdg[wit={J10}]{sidhyati}}//}\\!}
\end{alttlg}


\begin{alttlg}[hp04_000_8]
\tl{
\pada{durlabho viṣayatyāgo} % durlabha C6; viṣayāt* yogo N23
\pada{durlabhaṃ tattvadarśanam/}\\+} % labha V3
\tl{
\pada{durlabhā sahajāvasthā}
\pada{sadguroḥ karuṇāṃ vinā//}\\!}
\end{alttlg}

%\newpage

\Anm{\getsiglum{N19,V15,J10} have \ref{yavan} \textit{yāvan naiva praviśati} here}

\begin{alttlg}[hp04_000_9]
\tl{
\pada{vividhai\app{\lem[wit={ceteri},alt={āsanaiḥ}]{\skm{r }āsanaiḥ}
		\rdg[wit={V15}]{āsanaḥ}} kumbhai}% °bhai N23,C6
\pada{\app{\lem[wit={ceteri},alt={vicitra}]{\skm{r }vicitra}% +P11
		\rdg[wit={C6,K3,Jyo}]{vicitraiḥ}}%
	\app{\lem[wit={C6,V3,Gr3a,J10,Jyo}]{karaṇair api} % karaṇe
		\rdg[wit={J7}]{karuṇair api}
		\rdg[wit={N23}]{kalaṇair api}
		\rdg[wit={N19,V15}]{karaṇair atha}}/}\\+}
\tl{
\pada{\app{\lem[wit={ceteri},alt={prabuddhāyām}]{prabuddhāyā\skp{m}}% pravudhā° V19
		\rdg[wit={N19}]{pradhadhāyām}}%
	\app{\lem[wit={ceteri},alt={ādi}]{\skm{m }ādi}
		\rdg[wit={V15}]{idaṃ}
		\rdg[wit={Jyo}]{mahā}}%
	\app{\lem[wit={ceteri}]{śaktau} % proktau J10ac
		\rdg[wit={N23}]{śaktiḥ}}}
\pada{prāṇaḥ śūnye % prāṇaṃ C6; prāṇa V15
	\app{\lem[wit={C6,N23,Gr3a,J10}]{vilīyate}
		\rdg[wit={J7}]{vidhīyate}
		\rdg[wit={P11,V3,N19,V15,Jyo}]{pralīyate}% +K1,M1,G7,N22,P6 ##
		}//}\\!}
\end{alttlg}


\begin{alttlg}[hp04_000_10]
\tl{
\pada{\app{\lem[wit={ceteri}]{utpanna}
		\rdg[wit={V19}]{utpannā}
		\rdg[wit={N23}]{ut<<pā>>na}}%
	\app{\lem[wit={ceteri}]{śaktibodhasya}
		\rdg[wit={N23}]{śaktibodhaḥ syāt}
		\rdg[wit={V15}]{śaktibodhaś ca}}}
\pada{\app{\lem[wit={ceteri}]{tyakta}
		\rdg[wit={N23}]{prakṣa}}%
	niḥśeṣakarmaṇaḥ/} \lineom{ab}{C6}\\+}
\tl{
\pada{\app{\lem[wit={ceteri}]{yoginaḥ}
	\rdg[wit={C6}]{yogināṃ}} sahajāvasthā}
\pada{svaya%m
	\app{\lem[wit={C6,Gr2,V19,Jyo},alt={eva prajāyate}]{\skm{m }eva prajāyate}
		\rdg[wit={K3}]{evopajāyate}
		\rdg[wit={P11,V3,C7,V15,J10}]{eva prakāśate}% +K1; prakāśyate N22,P6 ##
		\rdg[wit={N19}]{eva prakāśayet}}//}\\!}
\end{alttlg}

\begin{alttlg}[hp04_000_11]
\tl{
\pada{suṣumṇā\app{\lem[wit={ceteri}]{vāhini}
		\rdg[wit={V3,N23,N19}]{vāhinī}
		\rdg[wit={V19}]{vāhi}}
	\app{\lem[wit={ceteri}]{prāṇe}
		\rdg[wit={V3}]{prāṇa}}}
\pada{\app{\lem[wit={P11,V15}]{śūnyaṃ}
		\rdg[wit={J10}]{śūnya}% śūnyaṃ J10pc
		\rdg[wit={C6,Gr2,Gr3a,Jyo}]{śūnye}% +K1,N22,P6
		\rdg[wit={V3}]{śūne}
		\rdg[wit={N19}]{śūnyā}} viśati
	\app{\lem[wit={P11,V3,Jyo}]{mānase} % Marmasthāna +M1,M3
		\rdg[wit={J10}]{mārutaḥ}
		\rdg[wit={C6,Gr2,Gr3a,N19,V15}]{mārute}}\marma/}\\+} % +K1,N22,P6,G7
\tl{
\pada{\app{\lem[wit={Gr2,Gr3a}]{tathā}
		\rdg[wit={C6,V3,N19,V15,J10,Jyo}]{tadā}} % ##
	\app{\lem[wit={ceteri}]{samasta}
		\rdg[wit={J10,Jyo}]{sarvāṇi}}karmāṇi}
\pada{\app{\lem[wit={ceteri}]{nirmūlayati}
		\rdg[wit={V19,V15}]{nimūlayati}
		\rdg[wit={N23}]{nirmūlaṃ yāti}} % °ḷayati V15
	\app{\lem[wit={Gr3a,Jyo}]{yogavit}
		\rdg[wit={N23,V15}]{karmavit}%## P6,P22,M3
		\rdg[wit={J7}]{karmakṛt}
		\rdg[wit={C6,V3,N19,J10}]{marmavit}% K1,P11,M1,G7
		}//}\\!}
\end{alttlg}

\newpage
\begin{alttlg}[hp04_000_12]
\tl{
\pada{\app{\lem[wit={J10,Jyo}]{amarāya}% +C2
		\rdg[wit={V3,V15}]{amaraugha}% ##
		\rdg[wit={N19}]{amarogha}% +P11,M3?
		\rdg[wit={C6}]{amaraughi}
		\rdg[wit={Gr2}]{amano nir°}
		\rdg[wit={Gr3a}]{amalo nir°}} % verse om. K1,P6
	\app{\lem[wit={ceteri}]{namas tubhyaṃ}
	\rdg[wit={Gr2}]{°manāḥ śūnyaṃ}
	\rdg[wit={Gr3a}]{°malaḥ śūnyaṃ}}}
\pada{so'pi \app{\lem[wit={C6,V3,N19,V15,Jyo}]{kālas tvayā} % kāla V15; trayā V3
	\rdg[wit={J10}]{kālantayā}
	\rdg[wit={Gr2,Gr3a},alt={\om}]{\skp{\om}}}
\app{\lem[wit={C6,V3,N19,V15,J10}]{hataḥ}
	\rdg[wit={N19}]{hata}
	\rdg[wit={Jyo}]{jitaḥ}
	\rdg[wit={Gr2,Gr3a},alt={\om}]{\skp{\om}}}/}\\+}
\tl{
\pada{patitaṃ \app{\lem[wit={C6,V3,N19,V15,Jyo}]{vadane}
	\rdg[wit={J10}]{pavane}
	\rdg[wit={Gr2,Gr3a},alt={\om}]{\skp{\om}}} yasya}
\pada{jagad etac carācaram//} % yagad V19; carāccaraṃ J7
	\lineom{bc}{Gr2,Gr3a}\\!}
\end{alttlg}

\begin{alttlg}[hp04_000_13]
\tl{
\pada{citte
	\app{\lem[wit={C6,V3,J7,Gr3a,J10,Jyo},alt={samatvam}]{samatva\skp{m}}
		\rdg[wit={N19,V15}]{śamatvam}
		\rdg[wit={N23}]{samatyam}}m āpanne}
\pada{\app{\lem[wit={J7,Gr3a,N19,Jyo}]{vāyau}
		\rdg[wit={V15}]{vāyo}
		\rdg[wit={V3,N23}]{vāyor}
		\rdg[wit={C6,J10}]{vāyur}}
	\app{\lem[wit={ceteri}]{vrajati}
	\rdg[wit={N23}]{javati}} madhyame/}\\+}
\tl{
\pada{\app{\lem[wit={N19}]{tadāmaraugha}
		\rdg[wit={P11,V3}]{eṣāmaraugha}
		\rdg[wit={V15}]{tadāmaroḷi}
		\rdg[wit={Jyo}]{tadāmarolī}
		\rdg[wit={J10}]{tathāmarolī}
		\rdg[wit={C6}]{saivāmarolī}
		\rdg[wit={V19}]{eṣā naulīti}
		\rdg[wit={C7}]{eṣā naulī ca}
		\rdg[wit={K3}]{eṣā naulīva}}%
	\app{\lem[wit={C6,V3,Gr3a,N19,J10,Jyo}]{vajrolī}
		\rdg[wit={V15}]{vajrolis}}}
\pada{\app{\lem[wit={N19,V15}]{tadāśājīvite'pi ca}% +M3; tādhāśrājiṃtasya ca G7
		\rdg[wit={C6,V3}]{sadā me bhimateti ca} % bhimate cita V3
		\rdg[wit={Gr3a}]{sadā cābhimateti ca}
		\rdg[wit={J10}]{sahajolī mato pi ca}
		\rdg[wit={Jyo}]{sahajolī prajāyate}}//}
		\lineom{cd}{Gr2}\\!}
\end{alttlg}
% G7 tadha manā vajroci tādhāśrājiṃtasya ca (*12 om.)
% M3 tato maśā .. vajraulī tadāśājīvitepi ca
% K1 eṣāmarolī vajrolī sadā abhinayāti ca
% P6 eṣāmaroli vajroli sadā abhinayaṃti ca (*12 om.)
% N22 eṣāmaroli vajroli sadā savitayiti ca (*12 om.)
% P11 eṣāmaraughavajrolī sadā me timateti vaḥ


%\newpage
\begin{alttlg}[hp04_000_14]
\tl{
\pada{jñānaṃ kuto manasi
	\app{\lem[wit={ceteri}]{jīvati devi yāvat} % +P11, jāvat V19, yā<<va>>t N23
		\rdg[wit={C7,N19}]{jīvati devi tāvat}
		\rdg[wit={Jyo}]{saṃbhavatīha tāvat}
		\rdg[wit={V15}]{jīvati durvikalpe}}}\\+}
\tl{
\pada{\app{\lem[wit={ceteri}]{prāṇo}
	\rdg[wit={C6,V15}]{prāṇe}}'pi jīvati mano
	\app{\lem[wit={ceteri}]{mriyate} % sriyate N23, mrīyate P7
		\rdg[wit={J7,V19}]{mṛyate}
		\rdg[wit={V15}]{miyata}}
	\app{\lem[wit={ceteri}]{na}
		\rdg[wit={N19}]{ca}}
	\app{\lem[wit={ceteri}]{yāvat}
		\rdg[wit={P11,C6,V3}]{tāvat}}/}\\+}
\tl{
\pada{\app{\lem[wit={ceteri}]{prāṇo}% K3 viell. ṇaṃ
		\rdg[wit={V19,C7}]{prāṇaṃ}}
	\app{\lem[wit={ceteri}]{mano}
		\rdg[wit={N19}]{'pi ca}} dvayam idaṃ % dvayām N23
	\app{\lem[wit={ceteri}]{vilayaṃ}
		\rdg[wit={V15}]{na vilī°}}
	\app{\lem[wit={V19,C7,Jyo}]{nayed yo}
		\rdg[wit={J7}]{naved yo}
		\rdg[wit={N23}]{jayed  yo}
		\rdg[wit={N19}]{na yāvat}
		\rdg[wit={K3,J10}]{na yāti}% +G7; nayete M3; gate cen M1
		\rdg[wit={P11,C6}]{prayāti}% +K1,P6 ##
		\rdg[wit={V3}]{prajāti}
		\rdg[wit={V15}]{°yate tra}}}\\+}
\tl{
\pada{mokṣaṃ
	\app{\lem[wit={ceteri}]{sa}
		\rdg[wit={V15}]{na}
		\rdg[wit={C6}]{ca}} gacchati % gacchatiti V19
	\app{\lem[wit={ceteri}]{naro na}% narā N23
		\rdg[wit={K3}]{naro pi}}
	kathaṃci%d
	\app{\lem[wit={ceteri},alt={anyaḥ}]{\skm{d }anyaḥ}
		\rdg[wit={J10}]{anyat}
		\rdg[wit={V3}]{anya}}//}\\!}
\end{alttlg}


\Anm{\getsiglum{N19,V15,J10,Jyo} have \ref{jnatva}--\ref{tatraika} \textit{jñātvā suṣumṇāsadbhedaṃ} here}

\newpage
\begin{alttlg}[hp04_000_15]
\tl{
\pada{\app{\lem[wit={ceteri}]{rasasya}% +P6,M3
		\rdg[wit={J7,N19,V15}]{rasaś ca}} % +G7
	\app{\lem[wit={ceteri}]{manasaś caiva} % caivaṃ J10
		\rdg[wit={V3}]{manaś caiva}% $$
		\rdg[wit={N23}]{manasaiva caṃ°}}}
\pada{\app{\lem[wit={ceteri}]{cañcalatvaṃ}
		\rdg[wit={N23}]{°calatvaṃ ca}
		\rdg[wit={N19}]{vaṃcatvaṃ ca}} svabhāvataḥ/}\\+}
\tl{
\pada{\app{\lem[wit={C6,V3,J7,Gr3a,J10,Jyo}]{raso}
		\rdg[wit={N23,N19}]{rase}
		\rdg[wit={V15}]{rasa}}
	\app{\lem[wit={ceteri}]{baddho}
		\rdg[wit={N19,V15}]{baṃdhe}} mano
	\app{\lem[wit={ceteri}]{baddhaṃ} % varddhaṃ N23
		\rdg[wit={C6}]{baddho}
		\rdg[wit={V15}]{baṃdhe}}}
\pada{\app{\lem[wit={ceteri}]{kiṃ}
		\rdg[wit={N19}]{tan}}
	na sidhyati bhūtale//}\\!} % siddhyaṃti V3
\end{alttlg}

\begin{alttlg}[hp04_000_16]
\tl{
\pada{mūrcchito % mūcchito
	\app{\lem[wit={ceteri}]{harate}
		\rdg[wit={V3,J10}]{harati}}
	\app{\lem[wit={ceteri}]{vyādhiṃ}
		\rdg[wit={V3,J10}]{vyādhi}
		\rdg[wit={Jyo}]{vyādhīn}}}
\pada{mṛto
	\app{\lem[wit={ceteri}]{jīvayati}
		\rdg[wit={V15}]{jīvayate}}
	\app{\lem[wit={ceteri}]{svayaṃ}
		\rdg[wit={K3}]{dhruvam}}/}\\+}
\tl{
\pada{baddhaḥ
	\app{\lem[wit={ceteri}]{khecaratāṃ}
		\rdg[wit={V19}]{khacatāṃ}}
	\app{\lem[wit={ceteri}]{dhatte}
		\rdg[wit={N23,N19}]{dhartte}
		\rdg[wit={V3}]{yāti}}}
\pada{\app{\lem[wit={ceteri}]{raso vāyuś ca}
		\rdg[wit={V3}]{vāyuś ca}
		\rdg[wit={J10}]{sa jīveśvara}}
	\app{\lem[wit={C6,Gr3a}]{bhairavi}
		\rdg[wit={Gr2,N19,V15}]{bhairavī}
		\rdg[wit={V3},post=\texteng{(tathā for missing raso)}]{bhairavī tathā}
		\rdg[wit={Jyo}]{pārvati}
		\rdg[wit={J10}]{seśvaraḥ}}//}\\!}
\end{alttlg}

\Anm{\getsiglum{N19,V15,J10} have \ref{vayumargena} \textit{vāyumārgeṇa saṃcārī} here}

\Anm{\getsiglum{N19,V15,J10,Jyo} have \ref{manahsthairye} \textit{manaḥsthairye} here}
\endaltnormal

\newpage

\begin{tlg}[hp04_001]
\tl{
\pada{\app{\lem[wit={ceteri}]{indriyāṇāṃ} % °nāṃ V19, °ṇā N3
		\rdg[wit={N19}]{indriyāṇi}} mano nātho}
\pada{\app{\lem[wit={N3,C6,V3,Jyo}]{manonāthas tu}
		\rdg[wit={N19}]{manonāthaḥ su}
		\rdg[wit={N23,Gr3a,V15,J10}]{manonāthaś ca}
		\rdg[wit={J7}]{manaso nātha}} mārutaḥ/}\\+}
\tl{
\pada{mārutasya layo
	\app{\lem[wit={ceteri},alt={nāthas/nāthaḥ/nātho}]{nātha\skp{s/nāthaḥ/nātho}}
		\rdg[wit={J7}]{nāthāḥ}}}%
\pada{\app{\lem[wit={N3,V3,N19,V15,J10},alt={taṃ nāthaṃ layam āśrayet}]{\skm{s }taṃ nāthaṃ layam āśrayet}% K1A; laya nātha niraṃjanāṃ P11
		\rdg[wit={C6,Gr2,K3,C7,Jyo}]{sa layo nādam āśritaḥ}% K1B
		\rdg[wit={V19},post={\unm}]{layo dasamāśrayaḥ}}//}\\!}
\end{tlg}
		% laya nātha niraṃjanāṃ P11!

\startaltnormal
\begin{alttlg}[hp04_001_1]
\tl{
\pada{\app{\lem[wit={C6,V3,V15,Jyo}]{so'yam evāstu}% +P11
		\rdg[wit={N19}]{soyamo vāstu}
		\rdg[wit={J10}]{svayam evāstu}% +K1,P6
		\rdg[wit={Gr2,Gr3a},post=\texteng{(evaṃ \getsiglum{N23})}]{ayam eva tu}}
	\app{\lem[wit={ceteri}]{mokṣākhyo} % °kṣyo N19
		\rdg[wit={J10}]{vā mokṣaḥ}}}
\pada{\app{\lem[wit={C6,V3,V15,Jyo}]{māstu vāpi}% +P11; astu K1,P6
		\rdg[wit={N19}]{māstu kapi}
		\rdg[wit={J10}]{sosti vāpi}
		\rdg[wit={J7}]{'stu vāpi sa}
		\rdg[wit={Gr3a}]{yas tu vāpi}
		\rdg[wit={N23}]{aya vāpi}}
	matāntare/}\\+} % matātare J7
\tl{
\pada{manaḥprāṇa%
	\app{\lem[wit={C6,Gr2,V15}]{layānando} % layāṃnado N23; layo? V15
		\rdg[wit={N19}]{layānanda}
		\rdg[wit={V3}]{layāna} % 1 syllable too short
		\rdg[wit={V19,C7}]{layo nādo}
		\rdg[wit={K3}]{layenaiva}
		\rdg[wit={Jyo}]{laye kaścid}
		\rdg[wit={J10}]{°m apānaṃ vā}}}
\pada{\app{\lem[wit={Gr2,Gr3a}]{nāpi}
		\rdg[wit={C6,V15}]{mayi}% +K1,P11,M1,M3,P6 ##
		\rdg[wit={N19}]{mapi}
		\rdg[wit={V3}]{māpi}
		\rdg[wit={J10}]{layaḥ}
		\rdg[wit={Jyo}]{āna°}}
	\app{\lem[wit={ceteri},alt={kaścid/°cit}]{kaści\skp{d/°cit}}
		\rdg[wit={V19}]{kviṃcid}
		\rdg[wit={Jyo}]{°ndaḥ saṃ°}}%
	\app{\lem[wit={J7,Gr3a},alt={vibhidyate}]{\skm{d }vibhidyate}
		\rdg[wit={N23}]{vibhedyate}
		\rdg[wit={C6,N19,V15,J10,Jyo}]{pravartate}% +K1,P11,M1,M3,P6 ##
		\rdg[wit={V3}]{pravartate na}}//} \NotIn{Gr1}\\!}
\end{alttlg}
\endaltnormal


%\newpage
\begin{tlg}[hp04_002]
\tl{
\pada{\app{\lem[wit={V3,J7,Gr3a,V15,J10},alt={praṇaṣṭo-/pranaṣṭocchvāsa}]{praṇaṣṭocchvāsa}% pranaṣṭ° V3,J7,V15,J10
		\rdg[wit={N19}]{pranaṣṭauśvāsa}
		\rdg[wit={N23}]{prabhṛṣṭo\,\_\,sa}
		\rdg[wit={N3,Jyo}]{praṇaṣṭaśvāsa}
		\rdg[wit={C6}]{pranaṣṭaḥ svāsa}}%
	\app{\lem[wit={N3,V15,Jyo}]{niśvāsaḥ}
		\rdg[wit={N19,J10}]{niḥśvāsaḥ} % niḥñcāsaḥ N19
		\rdg[wit={V3}]{niśvāsa}
		\rdg[wit={C6,V19,C7}]{niḥśvāsa}% niḥsvāsa V19
		\rdg[wit={J7}]{niśvāsāḥ}
		\rdg[wit={K3}]{niḥśvāsāḥ}
		\rdg[wit={N23}]{niśvāsā}
		}}
\pada{\app{\lem[wit={ceteri}]{pradhvasta}% praddhasta N19
		\rdg[wit={J10}]{pranaṣṭa}% °naṣṭaḥ V17, prā° N26.
		\rdg[wit={K3}]{.r.\,ṇ.\,.[ṭ].}}%
	\app{\lem[wit={ceteri}]{viṣaya}
		\rdg[wit={N19}]{viṣaga}}%
	\app{\lem[wit={N3,C6,V3,V19,J10,Jyo}]{grahaḥ}
		\rdg[wit={Gr2,K3,C7}]{grahāḥ}
		\rdg[wit={V15}]{jvaraḥ}
		\rdg[wit={N19}]{hvaraḥ}}/}\\+}
\tl{
\pada{\app{\lem[wit={N3,C6,V3,Jyo}]{niśceṣṭo}
		\rdg[wit={Gr2,Gr3a,V15}]{niśceṣṭā}
		\rdg[wit={N19}]{nidyeṣṭo}
		\rdg[wit={J10}]{niścalo}}
	\app{\lem[wit={C6,V3,N23,N19,V15,J10,Jyo}]{nirvikāraś ca} % nirvikā<<ra>>ś N23, nivikāraś V15
		\rdg[wit={N3}]{nirvikāras tu} % nivikalpas tu J5
		\rdg[wit={J7,Gr3a}]{nirvikārāś ca}}}
\pada{\app{\lem[wit={N3,C6,V3,N19,V15,J10,Jyo}]{layo}
		\rdg[wit={V19}]{laye}
		\rdg[wit={Gr2,K3,C7}]{layaṃ}}
	\app{\lem[wit={N3,C6,V3,N19,V15,J10,Jyo}]{jayati}
		\rdg[wit={Gr2,Gr3a}]{yānti ca}}
	\app{\lem[wit={N3,C6,V3,N19,V15,Jyo}]{yoginām}
		\rdg[wit={Gr2,Gr3a,J10}]{yoginaḥ}}//}\\!}
\end{tlg}

\begin{tlg}[hp04_003]
\tl{
\pada{\app{\lem[wit={ceteri}]{ucchinna}
		\rdg[wit={N3,V15}]{ucchinnaḥ}
		\rdg[wit={Gr3a}]{ucchūna}}%
	sarva\app{\lem[wit={ceteri}]{saṃkalpo}
		\rdg[wit={V19}]{saṃkalpe}
		\rdg[wit={K3}]{saṅkalpa}}}
\pada{\app{\lem[wit={ceteri}]{niḥśeṣāśeṣa}
		\rdg[wit={Gr2}]{niḥśeṣagata}
		\rdg[wit={V3}]{niḥśeṣoṣeṣa}}%
	\app{\lem[wit={ceteri}]{ceṣṭitaḥ}
		\rdg[wit={C6}]{ceṣṭitam}
		\rdg[wit={K3,C7}]{veṣṭitaḥ}
		\rdg[wit={V15}]{varjitaḥ}}/}\\+}
\tl{
\pada{\app{\lem[wit={N3,V3,V19,C7,J10,Jyo}]{svāvagamyo}
		\rdg[wit={C6}]{sovagamyo}
		\rdg[wit={N19}]{svāgamyo}
		\rdg[wit={V15}]{svānugamyo}
		\rdg[wit={Gr2}]{svāgate cā}
		\rdg[wit={K3}]{svāvegasya}}
		layaḥ ko'pi} % pi added in margin C7
\pada{\app{\lem[wit={N3,C6},alt={jayatāṃ vāg}]{jayatāṃ vā\skp{g}}
		\rdg[wit={V15}]{jāyatāṃ vāg}
		\rdg[wit={N19}]{japatāṃ vāg}
		\rdg[wit={V3,J10,Jyo}]{jāyate vāg}
		\rdg[wit={Gr2,Gr3a}]{manovācām}}gagocaraḥ//}\\!} % agocara V3,N19
\end{tlg}

%\newpage
\begin{tlg}[hp04_004]
\tl{
\pada{yatra % yabha N23, yanna N19
	\app{\lem[wit={ceteri},alt={dṛṣṭir}]{dṛṣṭi\skp{r}}
		\rdg[wit={N3,V15,J10}]{dṛṣṭi}
		\rdg[wit={K3}]{sṛṣṭi}}r layas tatra}
\pada{bhūtendriya% bhute° N19
	\app{\lem[wit={N3,V3,V15}]{sanātanaḥ}
		\rdg[wit={N19}]{sanātanaṃ}
		\rdg[wit={C6,Gr2,Gr3a,J10,Jyo}]{sanātanī}}/}\\+}
\tl{
\pada{\app{\lem[wit={N3,Gr2,Gr3a},alt={syāc chaktir/°tiḥ}]{syāc chakti\skp{r}}
		\rdg[wit={J5}]{syāt saktir}
		\rdg[wit={C6,V3,N19,V15,J10,Jyo},post=\texteng{(sa \getsiglum{V15})}]{sā śaktir}}% sa V15
	\app{\lem[wit={N3,C6,V3,J10,Jyo},alt={jīva}]{\skm{r }jīva}
		\rdg[wit={Gr2,Gr3a}]{sarva}
		\rdg[wit={N19,V15}]{bhāva}}%
	\app{\lem[wit={ceteri}]{bhūtānāṃ}
		\rdg[wit={N23}]{bhūtānī}
		\rdg[wit={N19}]{bhūnāṃ}}}
\pada{\app{\lem[wit={N3,C6,V3,Gr2,J10},alt={dṛṣṭir}]{dṛṣṭi\skp{r}}
		\rdg[wit={Gr3a}]{dṛṣṭi}
		\rdg[wit={N19,V15}]{dṛṣṭe}
		\rdg[wit={Jyo}]{dve a°}}%
	\app{\lem[wit={N3,P11,V3,N19},alt={lakṣ(y)e layaṃ gatā}]{\skm{r }lakṣye layaṃ gatā} % gatāḥ N3
		\rdg[wit={J10,Jyo}]{lakṣye layaṃ gate}
		\rdg[wit={V15}]{lakṣaṃ layaṃ gatau}
		\rdg[wit={J7}]{lakṣe na saṃgatā}
		\rdg[wit={N23}]{lakṣana saṃgatā}
		\rdg[wit={Gr3a},alt={lakṣ(y)eṇa saṃgatā}]{lakṣyeṇa saṃgatā}%lakṣyeṇa K3,C7
		\rdg[wit={C6}]{gacchel layaṃ gate}}//}\label{yatradrsti}\\!}
\end{tlg}

\Anm{\getsiglum{Jyo} has \ref{layo} \textit{layo laya iti} here}

\newpage
\begin{tlg}[hp04_005]
\tl{
\pada{vedaśāstra\app{\lem[wit={N3,C6,N19,V15,J10,Jyo}]{purāṇāni}
		\rdg[wit={N23}]{purāṇādyāḥ}
		\rdg[wit={J7}]{puraṇādyāḥ}
		\rdg[wit={K3,C7}]{purāṇaughāḥ}
		\rdg[wit={V19}]{purāṇaiś ca}}}
\pada{sāmānya% samāni C6
	\app{\lem[wit={ceteri}]{gaṇikā} % gatikā N26
		\rdg[wit={V19}]{gaṇivā}} iva/}\\+}
\tl{
\pada{ekaiva śāṃbhavī % ekeva? K3
	\app{\lem[wit={N3,C6,Gr2,Gr3a,N19,Jyo}]{mudrā}
		\rdg[wit={V15}]{māyā}
		\rdg[wit={J10}]{vidyā}}}
\pada{\app{\lem[wit={N3,C6,Gr2,Jyo}]{guptā kulavadhūr iva}% vadhū iva P7
		\rdg[wit={J10}]{gopyā kulavadhūr iva}
		\rdg[wit={N19,V15},post=\texteng{(cf.\,\ref{gopita}d)}]{sarvatantreṣu gopitā}% ##
		\rdg[wit={Gr3a}]{sarvatantreṣu gopitā rakṣaṇīyā prayatnena guptā kulavadhūr iva}}//}\label{vedasastra} \NotIn{V3}\\!}
\end{tlg}


%\newpage
\begin{tlg}[hp04_006]
\tl{
\pada{anta\app{\lem[wit={V3,Gr2,J10,Jyo},alt={lakṣ(y)aṃ}]{\skm{r }lakṣyaṃ} % antalakṣaṃ P7; °lakṣyaṃ Jyo,C6ac
		\rdg[wit={V19,K3}]{lakṣ(y)ā} % y K3
		\rdg[wit={N3,C6,C7}]{lakṣ(y)a}} % y C7,C6
	\app{\lem[wit={N3,V3,Gr2,Gr3a,Jyo},alt={bahir}]{bahi\skp{r}}
		\rdg[wit={J10}]{mano}}%
	\app{\lem[wit={ceteri},alt={dṛṣṭir}]{\skm{r }dṛṣṭi\skp{r}} % Gr2, alle Hss checken! ##
		\rdg[wit={V3,V19,J10}]{dṛṣṭi}}}%
\pada{\app{\lem[wit={N3,C6,V3,J7,Gr3a,J10,Jyo},alt={nimeṣonmeṣa}]{\skm{r }nimeṣonmeṣa}
		\rdg[wit={N23}]{nirmiṣonmeṣya}}varjitā/}\\+}
\tl{
\pada{\app{\lem[wit={N3,C6,Jyo}]{eṣā sā}
		\rdg[wit={V3}]{eṣā hi}
		\rdg[wit={J10}]{eṣā tu}
		\rdg[wit={Gr2,Gr3a}]{saiṣā tu}}
		śāṃbhavī mudrā}
\pada{\app{\lem[wit={N3,C6,V3,Gr2,J10}]{sarvatantreṣu}% sarve J10
		\rdg[wit={K3,C7}]{sarvaśāstreṣu}
		\rdg[wit={V19}]{sarvatantreṣu śastreṣu}
		\rdg[wit={Jyo}]{vedaśāstreṣu}}
		gopitā//}\label{gopita}
		\NotIn{N19,V15} \anm{eye-skip?}\\!}
\end{tlg}
		%\myfn{The omission was probably caused by haplography. \getsiglum{L2} inserts the following passage after 4.20a: \devnote{eṣā tu śāṃbhavī vidyā gopā ca kulavadhūr iva/ aṃtarlakṣamanodṛṣṭi nimeṣonmeṣonmeṣa(!)varjito/}}\\!

\begin{tlg}[hp04_007]
\tl{
\pada{anta\app{\lem[wit={N3,C6,V19,C7,J10,Jyo},alt={lakṣya}]{\skm{r}lakṣya}
		\rdg[wit={V3,Gr2,K3,N19,V15}]{lakṣa}}%
		vilīnacittapavano yogī
	\app{\lem[wit={ceteri}]{yadā}
		\rdg[wit={J10}]{yathā}
		\rdg[wit={N3,N19}]{sadā}} vartate}\\+}
\tl{
\pada{\app{\lem[wit={ceteri}]{dṛṣṭyā}
		\rdg[wit={J10}]{dṛṣṭvā}
		\rdg[wit={V3}]{dṛśyā}}
	\app{\lem[wit={ceteri}]{niścalatārayā}
		\rdg[wit={N23}]{niścalatāra}}
	\app{\lem[wit={ceteri},alt={bahir}]{bahi\skp{r}}
		\rdg[wit={N23}]{hir}}%
	\app{\lem[wit={N3,C6,V3,V15,J10,Jyo},alt={adhaḥ}]{\skm{r }adhaḥ}
		\rdg[wit={N19}]{adhraḥ}
		\rdg[wit={Gr2,Gr3a}]{asau}}
	\app{\lem[wit={J5,Gr3a,N19,V15,Jyo}]{paśyann apaśyann api}
		\rdg[wit={Gr2}]{paśyan na paśyaty api}
		\rdg[wit={J10}]{paśyann api}
		\rdg[wit={N3}]{paśyann apaśyann ivā}% illeg. G4
		\rdg[wit={P11}]{paśyann apaśyan tataḥ}
		\rdg[wit={C6}]{paśyen na paśyet tataḥ}
		\rdg[wit={V3}]{paśyan na paśyet tata}}/}\\+}
\tl{
\pada{\app{\lem[wit={ceteri}]{mudreyaṃ}
		\rdg[wit={V15}]{mudre}} khalu % khaluṃ N23
	\app{\lem[wit={N3,P11,V3,J10}]{khecarī} % = source; +K1,P6
		\rdg[wit={C6,Gr2,Gr3a,N19,V15,Jyo}]{śāṃbhavī}}
		bhavati sā
	\app{\lem[wit={N3,V3,Gr3a,N19,V15},alt={yuṣmat}]{yuṣma\skp{t}}% +J10pc
		\rdg[wit={J7}]{<<yu>>ṣmat}
		\rdg[wit={J10}]{yuṣmān}
		\rdg[wit={N23}]{puṣpat}
		\rdg[wit={C6}]{yasya}
		\rdg[wit={Jyo}]{labdhā}}tprasādā%d
	\app{\lem[wit={V3,Gr2,Gr3a,V15,J10},alt={guro}]{\skm{d }guro}
		\rdg[wit={C6,N19,Jyo}]{guroḥ}
		\rdg[wit={N3}]{gurau}
		\rdg[wit={J5}]{gure}}}\\+}
\tl{
\pada{śūnyāśūnya% °nyaṃ N19, śūnyācchūnya C6
	\app{\lem[wit={ceteri}]{vivarjitaṃ}
		\rdg[wit={N23}]{vivarjite}
		\rdg[wit={V19}]{vivarjiti}
		\rdg[wit={Jyo}]{vilakṣaṇaṃ}}
	\app{\lem[wit={ceteri}]{sphurati}
		\rdg[wit={V19}]{spharati}}
	\app{\lem[wit={ceteri},alt={yat}]{ya\skp{t}}% pat N23
		\rdg[wit={V3}]{ya}
		\rdg[wit={V19}]{[pta]t}
		\rdg[wit={N3,Jyo}]{tat}}t
	tattvaṃ \app{\lem[wit={ceteri}]{padaṃ}
		\rdg[wit={N19},alt={\om}]{\skp{\om}}} śāṃbhavam//}\label{antarlaksya}\\!}
\end{tlg}

%\newpage
\startaltnormal
\begin{alttlg}[hp04_007_1]
\tl{
\pada{\app{\lem[wit={ceteri}]{ardhodghāṭita}
		\rdg[wit={K3}]{ardhodghātavi}
		\rdg[wit={N23}]{arddhocchādita}
		\rdg[wit={Jyo}]{ardhonmīlita}}%
	\app{\lem[wit={Gr3a,V15,Jyo}]{locanaḥ}
		\rdg[wit={C6,V3,Gr2,N19,J10}]{locana}}
	\app{\lem[wit={ceteri}]{sthira}
		\rdg[wit={N23}]{sthila}}manā
	nāsāgradatte% nāśā° N19, nāśādagra° V17
	\app{\lem[wit={ceteri},alt={kṣaṇaḥ/-aś}]{kṣaṇaḥ}
		\rdg[wit={V3,N23}]{kṣaṇāś}
		\rdg[wit={N19}]{kṣaṇaṃ}}}\\+}
\tl{
\pada{\app{\lem[wit={ceteri},alt={candrārkāv}]{candrārkā\skp{v}}
		\rdg[wit={J7}]{candrārkoc}
		\rdg[wit={J10}]{candrārkau}}%
	\app{\lem[wit={C6,V3,N23,Gr3a,V15,Jyo},alt={api}]{\skm{v }api}
		\rdg[wit={J7}]{avi}
		\rdg[wit={N19}]{aca}
		\rdg[wit={J10}]{ca vi°}} līnatā%m
	\app{\lem[wit={Gr2,Gr3a,N19,V15},alt={upanayen}]{\skm{m }upanaye\skp{n}} % upanaye V19,N19,P6; eva nayet G7; +K1
		\rdg[wit={Jyo}]{upanayan}% M1,M3
		\rdg[wit={C6,V3}]{upagatau}% = source, +P11
		\rdg[wit={J10}]{gatau}}%
	\app{\lem[wit={ceteri},alt={niṣpanda}]{\skm{n }niṣpanda} % nispanda K3,C7,Jyo, niṣyaṃda? N19
		\rdg[wit={J10}]{nikṣipya}}%
	\app{\lem[wit={N23,V19}]{bhāvo'ntare}% bhāvottaraḥ P6
		\rdg[wit={J7}]{bhāvotare}
		\rdg[wit={K3,C7}]{bhāvāntare}
		\rdg[wit={J10}]{bhāsoṃtare}
		\rdg[wit={V15}]{bāṣpaṃ tataḥ}
		\rdg[wit={N19}]{vāpyaṃ tataḥ}
		\rdg[wit={C6}]{rūpaṃ tataḥ}
		\rdg[wit={P11}]{rūpaṃ tanu}% ##
		\rdg[wit={V3}]{rūpatanu}
		\rdg[wit={Jyo}]{bhāvena yaḥ}}/}\\+}
\tl{
\pada{jyotī% jyoṃtī J7, yotī N19
	\app{\lem[wit={ceteri},alt={rūpam}]{rūpa\skp{m}}
		\rdg[wit={N19,V15}]{rūpa}
		\rdg[wit={J7}]{yatsyam}}%
	\app{\lem[wit={ceteri},alt={aśeṣa}]{\skm{m }aśeṣa}%
		\rdg[wit={N19,V15}]{viśeṣa}}%
	\app{\lem[wit={ceteri}]{bāhyarahitaṃ}
		\rdg[wit={Jyo}]{bījam akhilaṃ}}
	\app{\lem[wit={ceteri}]{dedīpya}
		\rdg[wit={N23}]{devadīpya}}mānaṃ paraṃ}\\+}
\tl{
\pada{\app{\lem[wit={ceteri}]{tattvaṃ} % tatva V3
		\rdg[wit={K3}]{tattve}}
	\app{\lem[wit={ceteri},alt={tat}]{ta\skp{t}}
		\rdg[wit={J10}]{yac}}%
	\app{\lem[wit={Gr2,Gr3a,Jyo},alt={padam eti}]{\skm{t }padam eti}% +K1,P6
		\rdg[wit={C6,V3}]{param eti}% +P11
		\rdg[wit={N19,V15}]{param asti}% +HTK
		\rdg[wit={J10}]{carama}}
	\app{\lem[wit={ceteri}]{vastu}% +M1,M3,G7,K1,P6
		\rdg[wit={N23}]{vasta}
		\rdg[wit={V3}]{yastu}% +P11
		\rdg[wit={C6}]{yat tu}} paramaṃ
	\app{\lem[wit={ceteri}]{vācyaṃ} % vācya V19
		\rdg[wit={N23}]{vāpyaṃ}} ki%m
	\app{\lem[wit={ceteri},alt={atrādhikam}]{\skm{m }atrādhikam}
		\rdg[wit={N23}]{andrādhikaṃ}
		\rdg[wit={V19}]{atrāsanaṃ}}//}\label{ardhogha} \NotIn{Gr1}\\+}
\tl{\anm{\getsiglum{C6,V3,N19,V15,J10,Jyo} have this after \ref{kecid}}\\!}
\end{alttlg}
\endaltnormal

\newpage
\begin{tlg}[hp04_008]
\tl{
\pada{śrīśāṃbha\app{\lem[wit={N3,J7,Gr3a,Jyo},alt={°vyāś ca khecaryā}]{\skp{°}vyāś ca khecaryā} % khecarayā J7, °caryyāḥ V19
		\rdg[wit={N23}]{°vyāḥ khecaryā\,\_}
		\rdg[wit={GrB}]{°vyāḥ khecaryāś ca}}} % vyā V3,P6
\pada{\app{\lem[wit={P11}]{avasthāyām abhedatā}% +K1 (ana°)
		\rdg[wit={V3}]{avasthāyāṃ ca bhedatā}
		\rdg[wit={C6}]{hy avasthāyām abhedataḥ}
		\rdg[wit={N3}]{avasthāyāṃ na bhedataḥ}
		\rdg[wit={Jyo}]{avasthādhāmabhedataḥ}
		\rdg[wit={Gr2},post=\texteng{(bhedanaḥ \getsiglum{N23})}]{avasthā ca na bhedataḥ}% °naḥ N23
		\rdg[wit={Gr3a}]{avasthā balabhedataḥ}}\marma/}
		\NotIn{N19,V15,J10}\\+}
% J5 śobhavyā khecaryā avaschāyasya bhedataḥ
% G4 śrīśāṃbhavavyā khecaryā avasthāyā {{tavalabdhi}} na bhedataḥ
% K1a! śrīśāṃbhavyāś ca khecarya anasthāyām abhedatā
% K1b! śrīśāṃbhavyāḥ khecaryāś ca avasthā tu na bhedataḥ
% M1 śrīśāṃbhavyāś ca khecar.ā avasthāyām abh[e]dataḥ
% M3 śriśāṃbhavyāś ca khecaryā avastāyān na bhedatā
% G7 śāṃbaryāś ca khecarya atastā naiva bhedataḥ
% P6 śrīśaṃbhavyā khecaryāś ca avasthā tu na bhedataḥ
% N22 śrīśaṃbhavyā khecaryāś ca avasthā <!> na bhedataḥ
\tl{
\graus{\pada{bhavec cittalayānandaḥ}
\pada{śūnye citsukharūpiṇi//} 
\sgwit{Jyo}}\\!} % +M1,G8
\end{tlg}


\startaltnormal
\begin{alttlg}[hp04_008_1]
\tl{
\pada{\app{\lem[wit={P11,J7,V15,Jyo}]{tāre}
		\rdg[wit={N23}]{vāre}
		\rdg[wit={V3,V19}]{tāra}
		\rdg[wit={K3,C7}]{tāraṃ}
		\rdg[wit={C6}]{tārāṃ}
		\rdg[wit={J10}]{tārā}
		\rdg[wit={N19}]{tāva}}
	\app{\lem[wit={C6,Gr2,K3,C7,V15,Jyo}]{jyotiṣi}% jyotiṣīṃ P11
		\rdg[wit={N19}]{dyotiṣi}
		\rdg[wit={J10}]{jyotiṣu}
		\rdg[wit={V3}]{jyotīṣa}
		\rdg[wit={V19}]{jyotiso}}
	\app{\lem[wit={ceteri}]{saṃyojya}
		\rdg[wit={J10}]{saṃyojyā}
		\rdg[wit={N23}]{samojyaṃ}
		\rdg[wit={V19}]{jojya}}}
\pada{kiṃci%d
	\app{\lem[wit={N23,K3,C7},alt={uccālayed}]{\skm{d }uccālaye\skp{d}}
		\rdg[wit={J7}]{uccalayed}
		\rdg[wit={J10}]{uccārayed}% +K1b
		\rdg[wit={V19}]{uccācayed}
		\rdg[wit={C6,V3,V15,Jyo}]{unnamayed} % °yet* V3 ##? +P11,M3,P6,N22
		\rdg[wit={N19}]{uṣṭānnama}}%
	\app{\lem[wit={ceteri},alt={bhruvau}]{\skm{d }bhruvau}% bhuvau? J10
		\rdg[wit={N23}]{bhūvo<<ḥ>>}}/}
	\lineom{ab}{Gr1}\\+}
\tl{
\pada{\app{\lem[wit={V3,K3,C7,N19,V15},alt={pūrvayogasya mārgo'yam}]{pūrvayogasya mārgo'ya\skp{m}}% +P11
		\rdg[wit={C6}]{pūrvayogasya mārgeṇa}
		\rdg[wit={J10}]{sūryayogasya mārge ca}
		\rdg[wit={V19}]{pūrvayogasya māhātmyam}
		\rdg[wit={Jyo}]{pūrvayogaṃ mano yuñjann}
		\rdg[wit={Gr2},alt={\om}]{\skp{\om}}}}%
\pada{\app{\lem[wit={V3,Gr3a,N19,V15,Jyo},alt={unmanī}]{\skm{m }unmanī} % yaṃmunmanī N19
		\rdg[wit={C6}]{hy unmanī}
		\rdg[wit={J10}]{yunmanī}
		\rdg[wit={Gr2},alt={\om}]{\skp{\om}}}%
	\app{\lem[wit={Gr3a,V15}]{karaṇaṃ kṣaṇāt}
		\rdg[wit={V3}]{kāraṇaḥ kṣaṇāt}
		\rdg[wit={P11,Jyo}]{kārakaḥ kṣaṇāt} % ##
		\rdg[wit={N19}]{kārakaṃ kṣaṇāt}
		\rdg[wit={C6}]{kārakakṣaṇāt}
		\rdg[wit={J10}]{kāralakṣaṇam}% +K1
		\rdg[wit={Gr2},alt={\om}]{\skp{\om}}}//}
	\lineom{cd}{Gr1,Gr2}\\!} % om. +P6,N22
%\sgwit{Gr3a,N19,V15,C8,V3,N9,V17,J10}
\end{alttlg}

%\newpage
\begin{alttlg}[hp04_008_2]
\tl{
\pada{kecid
	āgama\app{\lem[wit={C6,V3,Gr3a,N19,V15,Jyo}]{jālena}
		\rdg[wit={J10}]{yogena}
		\rdg[wit={Gr2},alt={\om}]{\skp{\om}}}}
\pada{keci%n % cecin N19
	\app{\lem[wit={V3,Gr3a},alt={niyama}]{\skm{n }niyama}
		\rdg[wit={P11,C6,N19,J10,Jyo}]{nigama}% =source
		\rdg[wit={V15}]{nima}
		\rdg[wit={Gr2},alt={\om}]{\skp{\om}}}%
	\app{\lem[wit={P11,C6,N19,J10,Jyo}]{saṃkulaiḥ}% saṃkulai P11
		\rdg[wit={V3,V15}]{saṃkule}
		\rdg[wit={K3,C7}]{saṃkulāḥ}
		\rdg[wit={V19}]{saṃkulā}
		\rdg[wit={Gr2},alt={\om}]{\skp{\om}}}/}\\+}
\tl{
\pada{kecit tarkeṇa muhyanti} % ke<ci>nnarkkeṇa N19
\pada{naiva jānanti tārakam//}
%\myfn{Pādas ab and cd are transposed in \getsiglum{V19} and the correct order is indicated by a small number 1 and 2 above the hemistiches.}
\NotIn{Gr1,Gr2}\label{kecid}\\!} % C6?
\end{alttlg}
%\sgwit{C6,V3,Gr3a,N19,V15,N9,V17,J10}

\begin{alttlg}[hp04_008_3]
\tl{
\pada{
	\app{\lem[wit={N3,J5}]{pātāle yadvitaya}% yadvita+ G4
		\rdg[wit={Gr2}]{pātālād yad viśati}
		\rdg[wit={K3}]{pātālād yad viyati}% pātālaṃ yadviyati P6
		\rdg[wit={V19,C7}]{pātālād vā viyati}}% 
	\app{\lem[wit={J5}]{suṣiraṃ}% sukhiraṃ K1,P6
		\rdg[wit={N3}]{suśiraṃ}
		\rdg[wit={N23}]{śikhiraṃ}
		\rdg[wit={J7,K3}]{śikharaṃ}
		\rdg[wit={V19,C7}]{śikhare}} merumūle
	\app{\lem[wit={N3}]{tad asmin}
		\rdg[wit={J7}]{tad asti}% +K1,P6
		\rdg[wit={N23}]{tasti}
		\rdg[wit={K3,C7}]{tad āste}
		\rdg[wit={V19}]{tadāstā}
		\rdg[wit={J5}]{yadismi}}}\\+}
\tl{
\pada{tattvaṃ caitat pravadati % pravahati C7
	\app{\lem[wit={N3,Gr2}]{sudhīs tan mukhaṃ}
		\rdg[wit={K3,C7}]{sudhīḥ saṃmukhaṃ}
		\rdg[wit={V19}]{susaṃmukhaṃ}} nimnagānām/}\\+} % ninmaśanāṃ V19
\tl{
\pada{candrā%t % caṃdrā N3
	\app{\lem[wit={Gr2},alt={sāraḥ}]{\skm{t }sāraḥ}% sāra K1, sāraṃ P6
		\rdg[wit={Gr3a}]{srāvaḥ}
		\rdg[wit={N3,J5}]{sāro}}
	\app{\lem[wit={Gr2,K3,C7}]{sravati}% +K1,P6; savati J7
		\rdg[wit={V19}]{śravati}
		\rdg[wit={N3}]{grasati}
		\rdg[wit={J5},alt={\om}]{\skm{\om}}}
	\app{\lem[wit={N3,J5,N23,Gr3a},alt={vapuṣas}]{vapuṣa\skp{s}}
		\rdg[wit={J7}]{puruṣas}}s
	tena mṛtyur narāṇāṃ}\\+} % mūtyur N23
\tl{
\pada{\app{\lem[wit={N3,J5,J7,Gr3a},alt={taṃ badhnīyāt}]{taṃ badhnīyā\skp{t}}
		\rdg[wit={N23}]{tadvahyaṃpāt}}%
	\app{\lem[wit={N3,J5},alt={sukaraṇamṛdā}]{\skm{t }sukaraṇamṛdā}% mudā G4
		\rdg[wit={J7,C7}]{svakaraṇamṛdā}% sakaraṇamṛtā K1,P6
		\rdg[wit={V19}]{svakaraṇamṛjā}
		\rdg[wit={N23}]{svakaraṇai mṛdā}
		\rdg[wit={K3}]{svakara[ṇamṛ]\,..}}
		nānyathā
	\app{\lem[wit={N3,J7,K3,C7}]{kāyasiddhiḥ}% +K1
		\rdg[wit={N23}]{kāyaḥ siddhiḥ}
		\rdg[wit={J5,G4,V19}]{kāryasiddhi(ḥ)}% V19 om. h; +P6
		}//}
	\sgwit{Gr1,Gr2,Gr3a} \anm{cf. 3.49}\label{patala}\\!}
\end{alttlg}
	%\anm{\getsiglum{Gr2,Gr3a}; the other mss have \ref{ardhogha} instead of this verse}
	%\NotIn{C6,V3,N19,V15,J10,Jyo}


\begin{alttlg}[hp04_008_4]
\tl{
\pada{yat kiṃcit sravate candrād}
\pada{amṛtaṃ divyarūpiṇaḥ/}\\+} % N24 has °rūpiṇaḥ too.
\tl{
\pada{tat sarvaṃ grasate sūryas}
\pada{tena piṇḍaṃ jarāyutaṃ//} % piṇḍo jarāyutaḥ N24
\sgwit{Gr1} \anm{cf. 3.75}\\!} % almost the same J5; G4 damaged
\end{alttlg}


\begin{alttlg}[hp04_008_5]
\tl{
\pada{tatrāsti karaṇaṃ divyaṃ}
\pada{sūryasya paribandhanaṃ/}\\+} % paripaṃthi ca N24
\tl{
\pada{gurūpadeśato jñeyaṃ}
\pada{na tu śāstrārthakoṭibhiḥ//}
\sgwit{Gr1} \anm{cf. 3.76}\\!} % almost the same J5; G4 damaged
\end{alttlg}

\newpage

%\Anm{\getsiglum{V3,N19,V15,J10} have Vulg 4.42--65 about Khecarīsamādhi here}
\Anm{The following verses are not found in \getsiglum{Gr1,Gr2,Gr3a}, but in \getsiglum{GrB,N19,V15,J10,Jyo}}

%\startaltrecension
\teimute{\small}
\begin{alttlg}[hp04_008_6]
\tl{
\pada{\app{\lem[wit={GrB,N19,V15,Jyo}]{divā na}
		\rdg[wit={J10}]{vāsare}} pūjayel liṅgaṃ}
\pada{\app{\lem[wit={P11,N19}]{rātrau naiva ca pūjayet}
		\rdg[wit={J10,Jyo}]{rātrau caiva na pūjayet} % pūyet J10
		\rdg[wit={C6,V3}]{rātrau naiva prapūjayet} % ##?
		\rdg[wit={V15}]{rātrau liṃgaṃ na pūjayet}}/}\\+}
\tl{
\pada{\app{\lem[wit={GrB,N19,V15,J10}]{satataṃ}
		\rdg[wit={Jyo}]{sarvadā}} pūjayel liṅgaṃ} % pūyel J10
\pada{\app{\lem[wit={Jyo}]{divārātrinirodhataḥ}
		\rdg[wit={N19,V15,J10}]{divārātrau na pūjayet}
		\rdg[wit={P11,V3}]{divārātraṃ na pūjayet}
		\rdg[wit={C6}]{divārātrau ca pūjayet}
		}//}\\!}
\end{alttlg}


\begin{altava}[hp04_008_7]
atha \app{\lem[wit={C6,Jyo}]{khecarī}
	\rdg[wit={P11}]{khecarīsamādhiḥ}}/ \sgwit{Gr4b,Jyo}
\end{altava}

\begin{alttlg}[hp04_008_7]
\tl{
\pada{\app{\lem[resp=emend]{suṣiraṃ}
		\rdg[wit={V3,J10}]{sukhiraṃ} % cf. 3.50 suciraṃ
		\rdg[wit={N19}]{suṣiro}
		\rdg[wit={Gr4b}]{sukhiro} % susthiro P11
		\rdg[wit={V15}]{dṛṅmukhaṃ}}
	jñāna\app{\lem[wit={V3,J10}]{janakaṃ}
		\rdg[wit={V15}]{jaṃnakaṃ}
		\rdg[wit={Gr4b,N19}]{janakaḥ}}}
\pada{pañca\app{\lem[wit={Gr4b,V15}]{srotaḥ}
	\rdg[wit={V3,N19,J10}]{śrotaḥ}}%
	\app{\lem[wit={V3,V15}]{samanvitam}% J10pc
		\rdg[wit={Gr4b,N19}]{samanvitaḥ}
		\rdg[wit={J10}]{samanvite}}/}\\+} % °vita V3
\tl{
\pada{tiṣṭhate khecarī mudrā}
\pada{\app{\lem[wit={J10}]{tasmin śūnye}
		\rdg[wit={Gr4b,V15}]{tasmāc chūnye}% P6
		\rdg[wit={N19}]{satyaṃ tatra}
		\rdg[wit={V3},alt={\om},post=\texteng{(eye-skip?)}]{}} % cf. V3 resumes with tasmin!
	\app{\lem[wit={Gr4b,V15,J10}]{nirañjane}
		\rdg[wit={N19},post=\texteng{(cf. Pāda d of the next verse)}]{na saṃśayaḥ}
		\rdg[wit={V3},alt={\om}]{\skp{\om}}}//}
	\NotIn{Jyo} \anm{= 3.48*1}\\!}
\end{alttlg}
%\myfn{\getsiglum{L2} omits the 2nd half of this verse and the 1st half of the next verse (prob. by haplography) and adds them after the 1st half of 4.25*3.}

\begin{alttlg}[hp04_008_8]
\tl{
\pada{\app{\lem[wit={C6,N19,V15,J10,Jyo},post=\texteng{(nāḍi \getsiglum{N19,J10})}]{savyadakṣiṇanāḍīstho}
		\rdg[wit={P11}]{savyadakṣanāḍistho}
		\rdg[wit={V3},alt={\om}]{\skp{\om}}}} % J6 has this omission too!!
\pada{\app{\lem[wit={N19}]{madhye calati mārutaḥ}
		\rdg[wit={Gr4b,Jyo}]{madhye carati mārutaḥ}% madhyaṃ P11
		\rdg[wit={J10}]{madhye vahati mārutaḥ}
		\rdg[wit={V15}]{madhyacaritamāruta}
		\rdg[wit={V3},alt={\om}]{\skp{\om}}}/}\\+}
\tl{
\pada{\app{\lem[wit={Gr4b,N19,V15,J10,Jyo}]{tiṣṭhate khecarī mudrā} % tiṣṭhajña? N19
		\rdg[wit={V3},alt={\om}]{\skp{\om}}}}
\pada{\app{\lem[wit={GrB,V15,Jyo}]{tasmin sthāne}
		\rdg[wit={N19}]{satyaṃ tatra}
		\rdg[wit={J10}]{tatra satyaṃ}}
		na saṃśayaḥ//}\\!}
\end{alttlg}


\begin{alttlg}[hp04_008_9]
\tl{
\pada{cittaṃ carati khe yasmāj} % citraṃ N19; cārati V15; yasmā V15,V3
\pada{jihvā carati
	\app{\lem[wit={GrB,N19}]{khe gatā}
		\rdg[wit={V15}]{vegataḥ}}/}\\+}
\tl{
\pada{\app{\lem[wit={P11,V3,V15}]{tenaiṣā}% P6
		\rdg[wit={C6}]{teneyaṃ}
		\rdg[wit={N19}]{tenaiva}} khecarī  % khecarīṃ N19
	\app{\lem[wit={P11,V3,N19}]{nāma}
		\rdg[wit={C6,V15}]{mudrā}}}
\pada{\app{\lem[wit={P11,V3,N19}]{mudrā}
		\rdg[wit={V15}]{satyaṃ}
		\rdg[wit={C6}]{sarvā}} siddhai%r  % siddhai N19
	\app{\lem[wit={GrB,N19},alt={namaskṛtā}]{\skm{r }namaskṛtā}
		\rdg[wit={V15}]{nigadyate}}//}
	\NotIn{J10,Jyo} \anm{= 3.37}\\!}
\end{alttlg}

\begin{alttlg}[hp04_008_10]
\tl{
\pada{iḍāpiṅgalayo%r  % idā N19
	\app{\lem[wit={GrB,N19},alt={yoge}]{\skm{r }yoge}
		\rdg[wit={Jyo}]{madhye}}}
\pada{\app{\lem[wit={C6,Jyo}]{śūnyaṃ}
		\rdg[wit={P11,N19}]{śūnye}% P6
		\rdg[wit={V3}]{śūne}} % +J11ac
	\app{\lem[wit={V3,N19,Jyo}]{caivānilaṃ}
		\rdg[wit={Gr4b}]{caiva bilaṃ}}
	\app{\lem[wit={P11,V3,N19,Jyo}]{graset}% grasit P11
		\rdg[wit={C6}]{viśet}}/}\\+}
\tl{
\pada{\app{\lem[wit={C6,V3,N19,Jyo}]{tiṣṭhate}
		\rdg[wit={P11}]{tiṣṭhati}} khecarī mudrā}
\pada{\app{\lem[wit={C6,V3,Jyo}]{tatra satyaṃ punaḥ punaḥ}
		\rdg[wit={P11}]{tatra satyaṃ na saṃśayaḥ}% P6
		\rdg[wit={N19}]{satyaṃ tatra na saṃśayaḥ}}//}
\NotIn{V15,J10}\\!}
\end{alttlg}

\begin{alttlg}[hp04_008_11]
\tl{
\pada{\app{\lem[wit={GrB,Jyo},alt={sūryācandramasor}]{sūryācandramaso\skp{r}}% sūrya C6, sūryāc P11, °masaur V3
		\rdg[wit={N19,J10}]{somasūryadvayor} % P6; sūryā J10
		\rdg[wit={V15}]{candrasūryadvayor}}r madhye}
\pada{\app{\lem[wit={GrB,N19,V15}]{nirālambe tale}
		\rdg[wit={J10}]{nirālambo'ntarā}%
		\rdg[wit={Jyo}]{nirālambāntare}} punaḥ/}\\+}
\tl{
\pada{saṃsthitā vyomacakre yā}
\pada{sā mudrā nāma khecarī//}\\!}
\end{alttlg}

\newpage
\begin{alttlg}[hp04_008_12]
\tl{
\pada{\app{\lem[wit={P11,V3}]{sā mayodbheditā vāmā}
		\rdg[wit={N19}]{sā mayodve\,\_\,tā vāmā}
		\rdg[wit={V15}]{sā mayodve\,.itā vāmā}
		\rdg[wit={J10}]{somayodbheditā dhāma}% °rbhidādhāmāmā P6
		\rdg[wit={Jyo}]{somād yatroditā dhārā}}}
\pada{\app{\lem[wit={P11,V3,N19,V15}]{sākṣāc ca}
		\rdg[wit={J10}]{sākṣād vai}
		\rdg[wit={Jyo}]{sākṣāt sā}} śivavallabhā/}\\+}
\tl{
\pada{\app{\lem[wit={P11,V3,N19,V15},alt={pūrayen}]{pūraye\skp{n}}
		\rdg[wit={Jyo}]{pūrayed}
		\rdg[wit={J10}]{pūjayed}}%
	\app{\lem[wit={P11,V3,N19,V15},alt={mārutaṃ divyaṃ}]{\skm{n }mārutaṃ divyaṃ}
		\rdg[wit={J10,Jyo}]{atulāṃ divyāṃ}}}
\pada{\app{\lem[wit={P11,V3,N19,V15,J10}]{suṣumṇā}
		\rdg[wit={Jyo}]{suṣumṇāṃ}}
	\app{\lem[wit={P11,N19,V15,J10,Jyo}]{paścime}%
		\rdg[wit={V3}]{paścimā}}
	mukhe//} \NotIn{C6}\\!}
\end{alttlg}

\begin{alttlg}[hp04_008_13]
\tl{
\pada{purastāc caiva pūryeta} % t* caiva N19, °tā V3
\pada{\app{\lem[wit={GrB,N19,V15,Jyo}]{niścitā}
		\rdg[wit={J10}]{niśritā}} khecarī bhavet/}\\+}
\tl{
\pada{\app{\lem[wit={Gr4b,N19},alt={abhyaset}]{abhyase\skp{t}}
		\rdg[wit={V3}]{abhyase}
		\rdg[wit={J10,Jyo}]{abhyastā}
		\rdg[wit={V15},alt={\om},post=\texteng{(eye-skip?)}]{}}%
	\app{\lem[wit={C6,V3,N19},alt={khecarīmudrām}]{\skm{t }khecarīmudrā\skp{m}}
		\rdg[wit={P11}]{khecarīṃ mudrām}
		\rdg[wit={J10,Jyo}]{khecarīmudrā}
		\rdg[wit={V15},alt={\om}]{\skp{\om}}}}%
\pada{\app{\lem[wit={GrB,N19,J10},alt={unmanī}]{\skm{m }unmanī}
		\rdg[wit={Jyo}]{py unmanī}
		\rdg[wit={V15},alt={\om}]{\skp{\om}}}
	\app{\lem[wit={C6,V3}]{sā prajāyate}
		\rdg[wit={N19,J10,Jyo}]{saṃprajāyate}
		\rdg[wit={P11}]{sāṃdrajāyate}
		\rdg[wit={V15},alt={\om}]{\skp{\om}}}//}\\!}
\end{alttlg}

\begin{alttlg}[hp04_008_14]
\tl{
\pada{\app{\lem[wit={GrB,N19,Jyo},alt={abhyaset}]{abhyase\skp{t}}
		\rdg[wit={V15}]{abhyasat}
		\rdg[wit={J10}]{abhyaste}}%
	\app{\lem[wit={Jyo},alt={khecarīṃ}]{\skm{t }khecarīṃ}
		\rdg[wit={GrB,N19,V15,J10}]{khecarī}} % +J11
	\app{\lem[wit={GrB,Jyo},alt={tāvad}]{tāva\skp{d}}
		\rdg[wit={V15,J10}]{mudrāṃ}
		\rdg[wit={N19}]{mudrā}}}%
\pada{\app{\lem[wit={GrB,Jyo},alt={yāvat}]{\skm{d }yāva\skp{t}}
		\rdg[wit={N19,V15,J10}]{tāvat}}t syād
	yoga\app{\lem[wit={C6,N19,V15,Jyo}]{nidritaḥ}
		\rdg[wit={P11}]{nidritāḥ}
		\rdg[wit={J10}]{nidratāḥ}
		\rdg[wit={V3}]{niṃdrataḥ}}/}\\+}
\tl{
\pada{saṃprāptayoga\app{\lem[wit={Gr4b,N19,V15,J10,Jyo}]{nidrasya}
		\rdg[wit={V3}]{niṃdrasya}}}
\pada{kālo nāsti kadācana//}% canaṃ V3
\myfn{This verse and the next one are transposed in \getsiglum{Jyo}.}\\!}
\end{alttlg}

\begin{alttlg}[hp04_008_15]
\tl{
\pada{bhruvor madhye  % bhrūvaur N19
	\app{\lem[wit={V3,N19,V15,J10,Jyo}]{śiva}
	\rdg[wit={P11}]{bhavet}}sthānaṃ}
\pada{manas tatra vilīyate/}\\+}
\tl{
\pada{jñātavyaṃ tat padaṃ turyaṃ}
\pada{\app{\lem[wit={GrB,N19,J10,Jyo}]{tatra}
		\rdg[wit={V15}]{yatra}}
	\app{\lem[wit={GrB,V15,J10,Jyo}]{kālo}
		\rdg[wit={N19}]{kopi}} na vidyate//}\\!}
\end{alttlg}

\begin{alttlg}[hp04_008_16]
\tl{
\pada{candrasūryadvayor madhye}
\pada{\app{\lem[wit={GrB,V15,J10}]{mudrāṃ}
		\rdg[wit={N19}]{mudrā}}
	\app{\lem[wit={GrB,V15,J10}]{dadyāc ca} % dadyā V3
		\rdg[wit={N19}]{divyā ca}}
	\app{\lem[wit={C6,V15,J10}]{khecarīm}
		\rdg[wit={V3,N19}]{khecarī}
		\rdg[wit={P11}]{khecare}}/}\\+}
\tl{
\pada{\app{\lem[wit={C6,J10}]{nirālambe}
		\rdg[wit={V3,N19,V15}]{nirālambaṃ}
		\rdg[wit={P11}]{nirālambas}}
	\app{\lem[wit={C6,J10}]{mahāśūnye}
		\rdg[wit={N19,V15}]{mahāśūnyaṃ}
		\rdg[wit={V3}]{mahāśūnya}
		\rdg[wit={P11}]{tadā śūnya}}}
\pada{vyoma\app{\lem[wit={GrB,N19,J10}]{cakre}
		\rdg[wit={V15}]{cakraṃ}}
	\app{\lem[wit={C6,V3,J10}]{vyavasthitām}
		\rdg[wit={N19,V15}]{vyavasthitaṃ}
		\rdg[wit={P11}]{vyavasthitā}% P6
		}//}
\NotIn{Jyo}\\!} % cf. 4.25*6
\end{alttlg}

\begin{alttlg}[hp04_008_17]
\tl{
\pada{nirālambaṃ manaḥ kṛtvā}
\pada{na kiṃcid api cintayet/}\\+}
\tl{
\pada{sabāhyā\app{\lem[wit={GrB,N19,V15,Jyo}]{bhyantare}
		\rdg[wit={J10}]{bhyantaraṃ}} vyomni}
\pada{\app{\lem[wit={GrB,V15,J10,Jyo}]{ghaṭa}
		\rdg[wit={N19}]{paṭa}}va%t
	\app{\lem[wit={GrB,J10,Jyo},alt={tiṣṭhati}]{\skm{t }tiṣṭhati}
		\rdg[wit={N19,V15}]{tiṣṭhate}} dhruvam//}\\!}
\end{alttlg}

\newpage
\begin{alttlg}[hp04_008_18]
\tl{
\pada{bāhyavāyu%r
	\app{\lem[wit={N19,V15},alt={yadā}]{\skm{r }yadā}
		\rdg[wit={GrB,J10,Jyo}]{yathā}}% P6 ##
	\app{\lem[wit={Gr4b,V15}]{līnaḥ}
		\rdg[wit={N19}]{līna}
		\rdg[wit={V3}]{līnaṃ}
		\rdg[wit={J10,Jyo}]{līnas}}}
\pada{\app{\lem[wit={P11,V3}]{khasya madhye}
		\rdg[wit={C6}]{khamadhye tu}
		\rdg[wit={V15}]{khamadhye ca}
		\rdg[wit={N19}]{khamadhya\,\_}
		\rdg[wit={J10}]{tathā madhye}
		\rdg[wit={Jyo}]{tathā madhyo}}
	 \app{\lem[wit={GrB,V15,J10,Jyo}]{na saṃśayaḥ}
		\rdg[wit={N19}]{\_\,\_\,sayaḥ}}/}\\+}
\tl{
\pada{\app{\lem[wit={GrB,N19,V15,J10}]{svasthānaṃ gacchati prāṇaḥ} % prāṇa N19,V3
	\rdg[wit={Jyo}]{svasthāne sthiratām eti}}}
\pada{\crux\app{\lem[wit={C6,V3,N19,V15}]{sūryāṅge manasā tathā}% sūryāgnir M1; manatā P6
		\rdg[wit={P11,J10}]{sūryāṅge pavane tathā}% +P17
		\rdg[wit={Jyo}]{pavano manasā saha}}\crux//}\\!}
\end{alttlg}

%\newpage
\begin{alttlg}[hp04_008_19]
\tl{
\pada{eva\app{\lem[wit={GrB,N19,V15,J10},alt={abhyasyamānasya}]{\skm{m }abhyasyamānasya}
		\rdg[wit={Jyo}]{abhyasyatas tasya}}}
\pada{\app{\lem[wit={GrB,J10,Jyo}]{vāyumārge}
		\rdg[wit={N19,V15}]{vāyor mārge}} % vāyomārgre N19, mārgaṃ J10
	\app{\lem[wit={C6,Jyo}]{divāniśam}% +M1
		\rdg[wit={P11}]{divā niśi}
		\rdg[wit={V3}]{divādisam}
		\rdg[wit={J10}]{sadāniśaṃ}
		\rdg[wit={N19,V15}]{sadānilaṃ}}/}\\+}
\tl{
\pada{\app{\lem[wit={GrB,N19,J10,Jyo}]{abhyāsāj jīryate} % abhyāsā V3
		\rdg[wit={V15}]{abhyāsāl līyate}}
		vāyur} % vāyu V3
\pada{mana\app{\lem[wit={N19,V15,J10},alt={tatra vilīyate}]{\skm{s }tatra vilīyate}% P6
		\rdg[wit={GrB,Jyo}]{tatraiva līyate}}//}\\!}% ##?
\end{alttlg}


\begin{alttlg}[hp04_008_20]
\tl{
\pada{\app{\lem[wit={P11,V3,N19},alt={amṛtaṃ plāvayed deham}]{amṛtaṃ plāvayed deha\skp{m}}
		\rdg[wit={V15}]{amṛte plāvayed deham}
		\rdg[wit={C6}]{amṛtaṃ plavate \_\,\_}
		\rdg[wit={Jyo}]{amṛtaiḥ plāvayed deham}
		\rdg[wit={J10}]{ajaratvaṃ bhaved dehe}}m}
\pada{ā pādatala\app{\lem[wit={GrB,V15,Jyo}]{mastakam} % dehaṃāpādattala°  apādapala J10
		\rdg[wit={J10}]{mastake}
		\rdg[wit={N19}]{mastakān}}/}\\+}
\tl{
\pada{\app{\lem[wit={V3,Jyo}]{sidhyaty eva}
		\rdg[wit={N19}]{siddhaty eva}
		\rdg[wit={V15}]{siddhyaty evaṃ}
		\rdg[wit={J10}]{sidhyate ca}
		\rdg[wit={C6}]{siddhadeho}
		\rdg[wit={P11}]{siddhideho}}
	\app{\lem[wit={C6,Jyo}]{mahākāyo}
		\rdg[wit={P11}]{mahākāryo}
		\rdg[wit={J10}]{mahāyogo}
		\rdg[wit={V3,N19}]{sadā kāyo}% sadā kāryaṃ ... °kramāt P6
		\rdg[wit={V15}]{tadā kāyo}}}
\pada{mahābalaparākramaḥ//}\\!}
\end{alttlg}

\begin{postmula}[hp04_008_20]
\grau{iti khecarī/ \sgwit{Jyo}}
\end{postmula}

\begin{altava}[hp04_008_21]
\app{\lem[wit={N19}]{atha}\rdg[wit={P11},alt={\om}]{\skp{\om}}}
\app{\lem[wit={P11}]{śāmbhavī}
	\rdg[wit={N19}]{śāmbhavī śaktiḥ}}/ \sgwit{P11,N19}
\end{altava}

\begin{alttlg}[hp04_008_21]
\tl{
\pada{śaktimadhye manaḥ kṛtvā}
\pada{\app{\lem[wit={N19}]{śaktiṃ ca manamadhyagāṃ}% P6
		\rdg[wit={V15}]{śaktiṃ ca svāṃtamadhyagāṃ}
		\rdg[wit={Jyo}]{śaktiṃ mānasamadhyagām}
		\rdg[wit={J10}]{śaktiṃ manasi madhyataḥ}
		\rdg[wit={P11}]{sumadhyagaṃ}
		\rdg[wit={C6,V3}]{manaḥ śaktes tu madhyagam}}/}\\+}
\tl{
\pada{manasā
	\app{\lem[wit={Gr4b,N19,V15,J10,Jyo},post=\texteng{(ārokya \getsiglum{N19})}]{mana ālokya}
		\rdg[wit={V3}]{manam ālokya}}}
\pada{\app{\lem[wit={C6,N19,V15},alt={tad dhyāyet}]{tad dhyāye\skp{t}}
		\rdg[wit={P11}]{taṃ dhātaṃ}
		\rdg[wit={V3}]{vaddhyāyait}
		\rdg[wit={J10,Jyo}]{dhārayet}% dhāryate P6
		}t paramaṃ padam//}\\!}
\end{alttlg}

\begin{alttlg}[hp04_008_22]
\tl{
\pada{\app{\lem[wit={C6,V3,N19,V15,J10,Jyo}]{khamadhye}
		\rdg[wit={P11}]{khaṃmadhye}} kuru cātmānam}
\pada{ātmamadhye ca khaṃ kuru/}\\+} % ātmā° N19,J10ac
\tl{
\pada{\app{\lem[wit={N19,V15,J10,Jyo}]{sarvaṃ ca}% P6
		\rdg[wit={C6,V3}]{ātmānaṃ}
		\rdg[wit={P11}]{evaṃ kṛ°}}
	\app{\lem[wit={V3,N19,V15,Jyo}]{khamayaṃ kṛtvā}
		\rdg[wit={C6,J10}]{khaṃmayaṃ kṛtvā}
		\rdg[wit={P11}]{°tvā tayoś cāpi}}}
\pada{na kiṃcid api cintayet//}\\!}
\end{alttlg}

\begin{alttlg}[hp04_008_23]
\tl{
\pada{antaḥśūnyo bahiḥśūnyaḥ} % aṃtaśūnye °śūnyo J10; bahiśūnyaṃ P11
\pada{\app{\lem[wit={Gr4b,J10}]{śūnya}
	\rdg[wit={Jyo}]{śūnyaḥ}}kumbha ivāmbare/}\\+} % °baro P11
\tl{
\pada{antaḥpūrṇo bahiḥpūrṇaḥ}
\pada{\app{\lem[wit={Gr4b,J10}]{pūrṇa}
	\rdg[wit={Jyo}]{pūrṇaḥ}}kumbha
	\app{\lem[wit={J10,Jyo}]{ivārṇave}% ivādhare P6
	\rdg[wit={P11}]{ivāṃbare}% +N22
	\rdg[wit={C6}]{ivāmbudhau}}//}
	\NotIn{V3,N19,V15}\\!}
\end{alttlg}


\newpage
\begin{alttlg}[hp04_008_24]
\tl{
\pada{bāhyacintā na kartavyā} % °tavyo J10
\pada{tathaivāntara%
	\app{\lem[wit={J10,Jyo}]{cintanam}% P6/N22
		\rdg[wit={C6,V3}]{cintanā}
		\rdg[wit={P11}]{ciṃtamān}}/}\\+}
\tl{
\pada{\app{\lem[wit={GrB,Jyo}]{sarvacintāṃ parityajya} % cintā P11,V3
	\rdg[wit={J10}]{sarvacintā parityājyā}}}
\pada{na kiṃcid api cintayet//}
\NotIn{N19,V15}\\!}
\end{alttlg}


%\newpage
\begin{alttlg}[hp04_008_25]
\tl{
\pada{saṃkalpamātra%
	\app{\lem[wit={N19,V15,J10,Jyo}]{kalanaiva}
		\rdg[wit={V3}]{kalanaṃ ca}} jaga%t
	\app{\lem[wit={V3,N19,V15,Jyo},alt={samagraṃ}]{\skm{t }samagraṃ}
		\rdg[wit={J10}]{samastaṃ}}} \lineom{a}{P11,C6}\\+}
\tl{
\pada{saṃkalpamātra%
	\app{\lem[wit={V3,N19,V15}]{kalanā hi}% karahāṇi P6,N22b
		\rdg[wit={J10,Jyo}]{kalanaiva}} % phalanaiva N22a
	mano\app{\lem[wit={J10,Jyo}]{vilāsaḥ}% P6
		\rdg[wit={V3}]{vilāsā}
		\rdg[wit={N19}]{vilīnā}
		\rdg[wit={V15}]{valīnā}}/} \lineom{b}{P11,C6}\\+}
\tl{
\pada{saṃkalpamātra%
	\app{\lem[wit={V15}]{matam utsṛja}% P6,N22
		\rdg[wit={N19}]{matatsṛja}
		\rdg[wit={Jyo}]{matim utsṛjya}
		\rdg[wit={P11}]{mim utsṛja}
		\rdg[wit={V3}]{m idam utsṛja}
		\rdg[wit={J10}]{kalanaiva vikṛtis tu}}
	\app{\lem[wit={P11,V3,N19,V15,Jyo}]{nirvikalpaṃ}
		\rdg[wit={J10}]{nityaṃ}}} \lineom{c}{C6}\\+}
\tl{
\pada{\app{\lem[wit={P11,V3,N19,Jyo}]{āśritya} % āśrītya V3
		\rdg[wit={V15}]{āśrita}
		\rdg[wit={J10}]{saṃkalpa}}
	\app{\lem[wit={J10,Jyo},alt={niścayam}]{niścaya\skp{m}} % P6; = South Ind. 4c
		\rdg[wit={P11}]{niścalam}
		\rdg[wit={V3}]{niścalayam}
		\rdg[wit={N19,V15}]{niścitam}}%
	\app{\lem[wit={V3,N19,V15,Jyo},alt={avāpnuhi}]{\skm{m }avāpnuhi}
		\rdg[wit={J10}]{avāpnudhi}
		\rdg[wit={P11}]{anāpnuhi}}
	\app{\lem[wit={P11,V3,J10,Jyo}]{rāma}% śāmya P6, ?? N22
		\rdg[wit={V15}]{rāga}
		\rdg[wit={N19}]{roga}} śāntim//} \lineom{d}{C6}\\!} % śānti P11,V3
\end{alttlg}

%\newpage
\begin{alttlg}[hp04_008_26]
\tl{
\pada{karpūra%m
	\app{\lem[wit={P11,V3,N19,V15,Jyo},alt={anale}]{\skm{m }anale}
		\rdg[wit={C6}]{anile}} yadvat}
\pada{saindhavaṃ salile yathā/}\\+}
\tl{
\pada{\app{\lem[wit={GrB,V15,Jyo}]{tathā}
		\rdg[wit={N19}]{yathā}}
	\app{\lem[wit={GrB,Jyo}]{saṃdhīyamānaṃ ca}
		\rdg[wit={N19,V15}]{saṃdīpamānaṃ ca}}}
\pada{mana\app{\lem[wit={C6,V15,Jyo},alt={tattve}]{\skm{s }tattve}
		\rdg[wit={P11}]{tātva}
		\rdg[wit={V3}]{tatva}
		\rdg[wit={N19}]{tatra}}
	\app{\lem[wit={GrB,N19,Jyo}]{vilīyate}
		\rdg[wit={V15}]{valīyate}}//}\label{karpura}
	\NotIn{J10}\\!}
\end{alttlg}

\begin{alttlg}[hp04_008_27]
\tl{
\pada{jñeyaṃ % jñayaṃ N19
	\app{\lem[wit={Gr4b,Jyo}]{sarvaṃ pratītaṃ}
		\rdg[wit={V3,N19,V15}]{sarvapratītaṃ}
		\rdg[wit={J10}]{sarvam atītaṃ}} ca}
\pada{\app{\lem[wit={N19,V15}]{tajjñānaṃ}
		\rdg[wit={J10,Jyo}]{jñānaṃ ca} % jñāna J10
		\rdg[wit={GrB}]{jñānaṃ tu} % P6
		} mana ucyate/}\\+} % ucyata P11
\tl{
\pada{jñānaṃ \app{\lem[wit={Gr4b,N19,V15,J10,Jyo}]{jñeyaṃ} % jñoyaṃ N19
		\rdg[wit={V3}]{jñeya}}
	\app{\lem[wit={C6,V3,N19,V15,Jyo}]{samaṃ naṣṭaṃ}
		\rdg[wit={P11}]{manaṃ naṣṭaṃ}
		\rdg[wit={J10}]{manaś caiva}}}
\pada{nānyaḥ
	\app{\lem[wit={C6,N19,J10,Jyo}]{panthā}
		\rdg[wit={V15}]{paṃtha}
		\rdg[wit={P11}]{paṃthyā}
		\rdg[wit={V3}]{pathā}}
	\app{\lem[wit={C6,V15,J10,Jyo}]{dvitīyakaḥ}
		\rdg[wit={P11,N19}]{dvitīyakaṃ}
		\rdg[wit={V3}]{dvitiyaka}}//}\\!}
\end{alttlg}

\begin{alttlg}[hp04_008_28]
\tl{
\pada{manodṛśyam idaṃ sarvaṃ}
\pada{yat kiṃcit sacarācaraṃ/}\\+}
\tl{
\pada{\app{\lem[wit={GrB,V15}]{manaso'py unmanī}
		\rdg[wit={N19}]{manosopy unmanī}
		\rdg[wit={J10,Jyo}]{manaso hy unmanī}}%
	\app{\lem[wit={V3,V15,J10}]{bhāve}
		\rdg[wit={P11}]{bhāvai}
		\rdg[wit={C6}]{bhāvo}
		\rdg[wit={Jyo}]{bhāvād}
		\rdg[wit={N19},alt={\om},post=\texteng{(eye-skip?)}]{}}}
\pada{\app{\lem[wit={Gr4b,V15}]{dvaitābhāvaṃ pracakṣate}% °cakṣyate P11
		\rdg[wit={N19}]{bhāvaṃ pracakṣyate}
		\rdg[wit={V3}]{dvaitābhāva pracakṣate}
		\rdg[wit={J10,Jyo}]{dvaitaṃ naivopalabhyate}}//}\\!}
\end{alttlg}

\begin{alttlg}[hp04_008_29]
\tl{
\pada{jñeyavastuparityāgād} % jñeyaṃ?  parī° N19; jñeyaṃ yas tu P11
\pada{vilayaṃ yāti % yāṃti P11
	\app{\lem[wit={GrB,V15,J10,Jyo}]{mānasam}
		\rdg[wit={N19}]{mārutaṃ}}/}\\+}
\tl{
\pada{\app{\lem[wit={GrB,N19,V15}]{mānase}
		\rdg[wit={J10,Jyo}]{manaso}}
	\app{\lem[wit={P11,V3,N19,V15,J10}]{vilayaṃ}
		\rdg[wit={C6,Jyo}]{vilaye}}
	\app{\lem[wit={P11,N19,V15}]{yāte}
		\rdg[wit={C6,V3,J10,Jyo}]{jāte}}}
\pada{kaivalya%m
	\app{\lem[wit={GrB,V15,Jyo},alt={avaśiṣyate}]{\skm{m }avaśiṣyate}
		\rdg[wit={N19}]{anasīṣyate}
		\rdg[wit={J10}]{api kalpate}}//}\\!}
\end{alttlg}

\newpage
\begin{alttlg}[hp04_008_30]
\tl{
\pada{layo laya iti prāhuḥ} % prāhur N19,V15
\pada{\app{\lem[wit={GrB,J10,Jyo}]{kīdṛśaṃ}
		\rdg[wit={N19,V15}]{īdṛśaṃ}} layalakṣaṇam/}\\+}
\tl{
\pada{\app{\lem[wit={GrB,V15,J10,Jyo}]{apunarvāsano}
		\rdg[wit={N19}]{apurvāsano}}%
	\app{\lem[wit={Gr4b,N19,Jyo},alt={°tthānāl}]{\skp{°}tthānā\skp{l}}% °tthānā P7, °tthanāl P11
		\rdg[wit={J10}]{tthānād}
		\rdg[wit={V3,V15}]{tthānā}}}%
\pada{\app{\lem[wit={GrB,N19,V15,Jyo},alt={layo viṣaya}]{\skm{l }layo viṣaya}
		\rdg[wit={J10}]{vṛttyayā viśva}}vismṛtiḥ//}% °smṛti N19,V3; °smati P11
	\label{layo}\myfn{\getsiglum{Jyo} has this verse between \ref{yatradrsti} and \ref{vedasastra}.}\\!}
\end{alttlg}

\begin{alttlg}[hp04_008_31]
\tl{
\pada{evaṃ nānāvidhopāyāḥ} % eva N19; ḥ om. P11,V3
\pada{samya%k
	\app{\lem[wit={GrB,N19,J10,Jyo},alt={svānubhavānvitāḥ}]{\skm{k }svānubhavānvitāḥ} % ḥ om. V3; sānubhavanvitā P11
		\rdg[wit={V15}]{svānubhavātmikāḥ}}/}\\+}
\tl{
\pada{samādhi\app{\lem[wit={Gr4b,N19,V15,Jyo}]{mārgāḥ}
		\rdg[wit={J10}]{mārge}
		\rdg[wit={V3},alt={\illeg}]{\skp{\illeg}}} kathitāḥ} % °tā V15
\pada{pūrvācāryair mahātmabhiḥ//}\\!} % °ryai V3
\end{alttlg}

%\newpage
\begin{altava}[hp04_008_32]
atha viśrāntiḥ/ \sgwit{N19,V15} \texteng{or:}
iti viśrāntiḥ/ \sgwit{Gr4b} \anm{?}
\end{altava}

\begin{alttlg}[hp04_008_32]
\tl{
\pada{\app{\lem[wit={GrB,V15,Jyo}]{suṣumṇāyai}
		\rdg[wit={N19}]{sukhayaiḥ}} kuṇḍalinyai}
\pada{sudhāyai candra% ceṃdra V3
	\app{\lem[wit={GrB,Jyo}]{janmane} % M1
		\rdg[wit={N19,V15}]{maṇḍalāt}}/}\\+} % G7,G11?
\tl{
\pada{manonmanyai namas tubhyaṃ}
\pada{mahā\app{\lem[wit={Gr4b,N19,V15}]{śakti}
		\rdg[wit={V3}]{śakte}
		\rdg[wit={Jyo}]{śaktyai}}cidātmane//}
	\NotIn{J10}\\!}
\end{alttlg}


\begin{alttlg}[hp04_008_33]
\tl{
\pada{\app{\lem[wit={P11,N19,V15,Jyo}]{aśakya}
		\rdg[wit={J10}]{aśakyaṃ}
		\rdg[wit={C6,V3}]{aśakta}}tattvabodhānāṃ}
\pada{\app{\lem[wit={C6,V3,N19,V15,J10,Jyo},alt={mūḍhānām}]{mūḍhānā\skp{m}}
		\rdg[wit={P11}]{gūḍhānām}}%
	\app{\lem[wit={GrB,J10,Jyo},alt={api saṃmatam}]{\skm{m }api saṃmatam}
		\rdg[wit={V15}]{api saṃtataṃ}
		\rdg[wit={N19}]{atisaṃtataṃ}}/} \anm{cf. \ref{saukhya}ab}\\+}
\tl{
\pada{proktaṃ gorakṣanāthena}
\pada{nādopāsana%m
	\app{\lem[wit={V3,N19,V15,J10,Jyo},alt={ucyate}]{\skm{m }ucyate}
		\rdg[wit={Gr4b}]{uttamam}}//}\\!}
\end{alttlg}

\endaltrecension


\begin{tlg}[hp04_009]% = 4c 4.64
\tl{
\pada{śrīādināthena sapādakoṭi}-\\+}% nāthona N23
\tl{
\pada{\app{\lem[wit={ceteri}]{laya}
		\rdg[wit={N3,Gr2,N19}]{layaḥ}}prakārāḥ % prakār<<āḥ>> J10
		kathitā  % kathaṃ J10
	\app{\lem[wit={C6,V3,Gr2,C7,V15,J10,Jyo}]{jayanti} % jayati V17
		\rdg[wit={N3,N19}]{jayante}
		\rdg[wit={K3}]{jaganti}
		\rdg[wit={V19}]{yayaṃti}}/}\\+}
\tl{
\pada{nādānusaṃdhānaka%m % nodānasaṃdhānasaṃdhānakam N19
	\app{\lem[wit={N3,C6,Jyo},alt={ekam eva}]{\skm{m }ekam eva}
		\rdg[wit={V3}]{eva}
		\rdg[wit={N19,J10}]{eva nānyaṃ}
		\rdg[wit={V15}]{eva mānyaṃ}
		\rdg[wit={Gr2,Gr3a}]{eva kāryaṃ}}}\\+} % °karmeva kārya N23
\tl{
\pada{\app{\lem[wit={ceteri}]{manyāmahe} % manyāṃmahe N19
	\rdg[wit={C6}]{gaṇyāmahe}}
	\app{\lem[wit={N3,P11,V3,N19,V15}]{mānyatamaṃ}
		\rdg[wit={Gr2,Gr3a}]{nānyatamaṃ}
		\rdg[wit={C6}]{nānyamataṃ}
		\rdg[wit={J10}]{tātarasaṃ}
		\rdg[wit={Jyo}]{mukhyatamaṃ}} layānām//}\label{sapadakoti}\\!} % layāṇāṃ J10
\end{tlg}


\Anm{\getsiglum{GrB,N19,V15,J10,Jyo} have \ref{sravanaputa} \textit{śravaṇamukhanayana} here}


\begin{tlg}[hp04_010]% (cf. 4.47)
\tl{
\pada{\app{\lem[wit={N3,C6,K3,C7}]{muktāsana}
	\rdg[wit={V19,Jyo}]{muktāsane}}sthito yogī}
\pada{mudrāṃ saṃdhāya śāṃbhavīm/} % sadhāya? V19
	\sgwit{Gr1,C6,Gr3a,Jyo}\\+}
\tl{
\pada{śṛṇuyād dakṣiṇe karṇe} % śṛṇuyā J7, śuṇu° V19
\pada{nāda\app{\lem[wit={N3,P11,Gr2,Jyo},alt={anta(ḥ)stham ekadhīḥ}]{\skm{m }antaḥstham ekadhīḥ}
	\rdg[wit={V19}]{ekāntake sudhīḥ}
	\rdg[wit={K3,C7}]{ekāntike sudhīḥ}
	\rdg[wit={C6}]{ataṃ sadā}}//}\label{muktasana2}
\sgwit{Gr1,Gr4b,Gr2,Gr3a,Jyo} \anm{cf. \ref{muktasana}}\\!}
\end{tlg}
%\NotIn{V3,N19,V15,J10}


\newpage

\Anm{\getsiglum{N19,V15,J10} have the following 5 verses after \ref{yatrakutrapi}, and \getsiglum{GrB} after \ref{muktasana}}

\begin{tlg}[hp04_011]% = V3_4.65 = C8_68
\tl{
\pada{\app{\lem[wit={N3,P11,V3,N19,V15,J10,Jyo}]{kāṣṭhe}
		\rdg[wit={C6,J7,Gr3a}]{kāṣṭhaiḥ}
		\rdg[wit={N23}]{kaṣṭaiḥ}}
	\app{\lem[wit={N3,C6,V3,Gr2,Gr3a,N19,Jyo}]{pravartito}
		\rdg[wit={V15,J10}]{pravartate}} vahniḥ} % vahni V3
\pada{\app{\lem[wit={ceteri}]{kāṣṭhena}
		\rdg[wit={N23}]{kaṣṭena}}
	\app{\lem[wit={ceteri}]{saha}
		\rdg[wit={V15}]{sa}}
	\app{\lem[wit={C6,Gr2,K3,C7,N19,J10,Jyo}]{śāmyati}
		\rdg[wit={N3,V3,V19}]{sāmyati}
		\rdg[wit={V15}]{līyate}}/}\\+}
\tl{
\pada{\app{\lem[wit={ceteri}]{nāde}
		\rdg[wit={N23}]{nā}}
	\app{\lem[wit={ceteri}]{pravartitaṃ} %°taṃś N19
		\rdg[wit={V15}]{pravartite}
		\rdg[wit={J10}]{pravartate}}
	\app{\lem[wit={ceteri}]{cittaṃ}
		\rdg[wit={N23},alt={\om}]{\skp{\om}}}}
\pada{nādena saha līyate//}\label{kasthe}\\!}
\end{tlg}

\begin{tlg}[hp04_012]% = V3_4.66
\tl{
\pada{\app{\lem[wit={N3,C6,V3,Gr2,V19,K3,N19,V15}]{vismṛtya}
		\rdg[wit={C7}]{nismṛtya}} sakalaṃ bāhyaṃ} % bāhya N19
\pada{\app{\lem[wit={N3,C6,V3,J7,Gr3a,V15}]{nāde}
		\rdg[wit={N19}]{nāda}
		\rdg[wit={N23}]{na\_}}
	\app{\lem[wit={N3,C6,V3,J7,Gr3a,N19,V15}]{dugdhāmbu}
		\rdg[wit={N23}]{gugyāṃbu}}va%n
	\app{\lem[wit={N3,C6,J7,N19,V15},alt={manaḥ}]{\skm{n }manaḥ}
		\rdg[wit={V3}]{mana}
		\rdg[wit={N23,Gr3a}]{naraḥ}}/}\\+}
\tl{
\pada{\app{\lem[wit={C6,Gr2,K3,C7,N19,V15}]{ekībhūyātha}
		\rdg[wit={V19}]{ekībhūyāya}
		\rdg[wit={V3}]{ekībhūyā}
		\rdg[wit={N3}]{ekībhūtvātha}}
	\app{\lem[wit={N3,C6,Gr2,Gr3a,N19,V15}]{sahasā}
		\rdg[wit={V3}]{sahasā ca}}}
	% ekībhūyād atha saha cidā(page break)ekībhūyātha sahasā J7
\pada{\app{\lem[wit={N3,C6,V3,Gr3a,N19,V15}]{cidākāśe}
		\rdg[wit={N23}]{vidāktośe}
		\rdg[wit={J7}]{cidākaro}} vilīyate//}
		\NotIn{J10,Jyo}\\!}
\end{tlg}

\begin{tlg}[hp04_013]% = V3_4.67
\tl{
\pada{\app{\lem[wit={V19,J10}]{audāsīnya}
		\rdg[wit={V15}]{audāsinya}
		\rdg[wit={C6,C7}]{audāsīna}
		\rdg[wit={K3}]{audāsīnye}
		\rdg[wit={N23}]{odāsīnya}
		\rdg[wit={V3,J7}]{udāsīnya}
		\rdg[wit={N3}]{udāsonya}
		\rdg[wit={N19}]{ṛdāsīnya}}paro bhūtvā}
\pada{sadābhyāsena saṃyamī/}\\+} % °bhyosena V3
\tl{
\pada{unmanī\app{\lem[wit={N3,C6,Gr2,Gr3a}]{karaṇaṃ}
		\rdg[wit={V3}]{karaṇa}
		\rdg[wit={N19,V15,J10}]{kārakaṃ}} sadyo} % sadyā N3
\pada{\app{\lem[wit={ceteri},alt={nādam}]{nāda\skp{m}}
		\rdg[wit={N19}]{bhāda}}%
	\app{\lem[wit={ceteri},alt={evāvadhārayet}]{\skm{m }evāvadhārayet}
		\rdg[wit={V15}]{eva sadābhyaset}}//}
	\NotIn{Jyo}\\!}
\end{tlg}

\begin{ava}[hp04_014]
\app{\lem[wit={N3,N23},alt={kīdṛśam}]{kīdṛśa\skp{m}} % ki° N3
		\rdg[wit={C7}]{kīdṛṣam}
		\rdg[wit={J7}]{kīdṛśīm}
		\rdg[wit={C6,V3,K3}]{kīdṛśyam}
		\rdg[wit={N19,J10}]{idṛśam}
		\rdg[wit={V19}]{kim}
		\rdg[wit={V15},alt={\om}]{\skp{\om}}}%
	\app{\lem[wit={ceteri},alt={audāsīnyam}]{\skm{m }audāsīnyam} % °śīnyaṃ V3
		\rdg[wit={N19,V15}]{audāsinyaṃ}}/ \NotIn{Jyo}
\end{ava}

\begin{tlg}[hp04_014]% = V3_4.68
\tl{
\pada{\app{\lem[wit={ceteri}]{śīte} % sīte N19,V3
		\rdg[wit={V15}]{śīti}
		\rdg[wit={J10}]{jñāte}}
	\app{\lem[wit={C6,V3,N23,Gr3a,N19,V15}]{kāle}
		\rdg[wit={J7}]{kāla}
		\rdg[wit={J10}]{kā}
		\rdg[wit={N3},alt={\om}]{\skp{\om}}}
	\app{\lem[wit={N3}]{caupaṭī vā paṭī vā}
		\rdg[wit={N19}]{copaṭī vā paṭī vā}
		\rdg[wit={J7}]{cāpaṭī vā paṭī vā}
		\rdg[wit={N23,C7}]{cāpaṭī cāpaṭī vā}
		\rdg[wit={V19}]{cāpaṭī vā paṭīkā} % vā paṭī? V19
		\rdg[wit={K3}]{cāpaṭe cāpaṭī}
		\rdg[wit={V3,J10}]{caupaṭī vākuṭī vā}% =J6
		\rdg[wit={C6}]{cāpaṭī cākuṭī vā}
		\rdg[wit={V15}]{paṭī vā}}}\\+}
\tl{
\pada{\app{\lem[wit={N3,V3,N19}]{pathyāhāre}
		\rdg[wit={C6,J7,K3,C7,V15,J10}]{pathyāhāro}
		\rdg[wit={N23}]{yathāhārā}
		\rdg[wit={V19}]{<<mi>>thyāhāro}}
	\app{\lem[wit={N3,C6,V3,Gr2,K3,N19,V15,J10}]{gopayo}
		\rdg[wit={V19}]{gopatho}
		\rdg[wit={C7}]{gomayo}}
	\app{\lem[wit={ceteri}]{vā}
		\rdg[wit={J10}]{co}}
	\app{\lem[wit={N3,V3,J7,K3,C7,N19,V15,J10}]{payo vā}
		\rdg[wit={N23}]{<<payo>>}
		\rdg[wit={V19}]{patho vā}
		\rdg[wit={C6}]{°tha pānaṃ}}/}\\+}
\tl{
\pada{\app{\lem[wit={N3,P11,V3}]{bhojye}
		\rdg[wit={V15,J10}]{bhojyaṃ}
		\rdg[wit={N19}]{bhojya}
		\rdg[wit={C6,V19}]{bhakṣyaṃ}
		\rdg[wit={C7}]{bhakṣye}
		\rdg[wit={Gr2}]{bhakṣe}
		\rdg[wit={K3}]{bh.\,kṣy.}}
	\app{\lem[wit={ceteri}]{bhikṣā} % bhīkṣā V15, bhikṣyā N3
		\rdg[wit={J10}]{bhuktaṃ}}%
	\app{\lem[wit={N3,C6,V3,Gr2,Gr3a,N19,V15},alt={vṛndam}]{vṛnda\skp{m}}
		\rdg[wit={J10}]{cānnam}}%
	\app{\lem[wit={N3,V3,J7,Gr3a,V15},alt={āraṇyakandaṃ}]{\skm{m }āraṇyakandaṃ}% +Gr1
		\rdg[wit={N23}]{āramyakaṃdaṃ}
		\rdg[wit={V3,N19,J10}]{āraṇyakaṃda} % bhikṣā kandam āraṇyakaṃ vā N22
		\rdg[wit={P11}]{āraṇyakaṃdā}
		\rdg[wit={C6}]{āpaṇyakaṃ vā}}}\\+} % āraṇyakaṃ vā N12
\tl{
\pada{\app{\lem[wit={N3,P11,J7,Gr3a}]{pāṇī droṇī}
		\rdg[wit={V15,J10}]{pāṇi droṇī}
		\rdg[wit={N19}]{pāṇī drāṇi}
		\rdg[wit={N23}]{pāṇīndrāṇī}
		\rdg[wit={C6}]{pāṇiṃ droṇe}
		\rdg[wit={V3}]{pāṇi}}
	\app{\lem[wit={N3,P11,N19,V15}]{kāpi vā}
		\rdg[wit={V3}]{kāpivāṃ}
		\rdg[wit={J10}]{kāthivā}
		\rdg[wit={K3,C7}]{karparā}
		\rdg[wit={C6}]{karpaṭaṃ}
		\rdg[wit={J7}]{kāpaṭo}
		\rdg[wit={N23}]{khapaḍā}
		\rdg[wit={V19}]{kharparo}}
	\app{\lem[wit={J5,N19}]{bhojyapātre}% +G4
		\rdg[wit={N3,C6,V3,Gr3a,V15,J10}]{bhojyapātraṃ}% patram C6
		\rdg[wit={N23}]{bhājapatraṃ}
		\rdg[wit={J7}]{bhūrjapātram}}//}
	\NotIn{Jyo}\\!}
\end{tlg}

\newpage
\begin{tlg}[hp04_015]% = V3_4.69 = C8_72
\tl{
\pada{\app{\lem[wit={J7,Gr3a,N19}]{sarvacintāṃ}
		\rdg[wit={N3,C6,V3,V15,J10}]{sarvacintā}
		\rdg[wit={N23},alt={\om}]{\skp{\om}}}
	\app{\lem[wit={P11,V3,N19,V15,J10}]{samutsṛjya}
		\rdg[wit={N3}]{samutyajya}
		\rdg[wit={C6,J7,Gr3a}]{parityajya}
		\rdg[wit={N23},alt={\om}]{\skp{\om}}}}
\pada{sarva\app{\lem[wit={N3,C6,V3,V15}]{ceṣṭāṃ}
		\rdg[wit={J10}]{ceṣṭāś}
		\rdg[wit={N19}]{ceṣṭī}
		\rdg[wit={Gr2,Gr3a}]{kāle}} ca sarvadā/}\\+}
\tl{
\pada{\app{\lem[resp=emend]{nāda}
	\rdg[wit={ceteri}]{nādam}}
	evānu\app{\lem[wit={N3,C6},alt={saṃdhānān}]{saṃdhānā\skp{n}}
		\rdg[wit={V3}]{saṃdhānā}
		\rdg[wit={J5,N19,V15,J10}]{saṃdadhyān} % °dhyā N19, saṃ<<da>>dhyān J10
		\rdg[wit={Gr2,Gr3a}]{saṃdhatte}}}n % dhartte N23
\pada{nāde cittaṃ vilīyate//}\label{sarvacinta} % saṃdhānād devi C6!
%	\lineom{J11}\myfn{\getsiglum{J11} haplography. Jumped to the next verse.}
	\NotIn{Jyo}\\!}
\end{tlg}

%\Anm{\getsiglum{N19,V15,J10} have \ref{sarvacinta2} and \ref{makaranda1}--64 after these verses.}

%\newpage
\begin{tlg}[hp04_016]% = V3_4.52 = C8_55 = V17_70
\tl{
\pada{ārambha%ś  % araṃbha° P7
	\app{\lem[wit={ceteri},alt={ca}]{\skm{ś }ca}
		\rdg[wit={V19}]{ca\,\_}}
	\app{\lem[wit={ceteri},alt={ghaṭaś}]{ghaṭa\skp{ś}}
		\rdg[wit={N23}]{gha\,\_\,ś}}%
	\app{\lem[wit={ceteri},alt={caiva}]{\skm{ś }caiva}
		\rdg[wit={J10}]{caivas}
		\rdg[wit={V19}]{ca}}}
\pada{tathā \app{\lem[wit={N3,C6,V3,N19,J10},alt={paricayas}]{paricaya\skp{s}}
		\rdg[wit={V15}]{paricas}
		\rdg[wit={N23,Gr3a,Jyo}]{paricayo}
		\rdg[wit={J7}]{pariyo}}%
	\app{\lem[wit={N3,V3,V15},alt={tathā}]{\skm{s }tathā}% trayaṃ N24
		\rdg[wit={P11,C6,N19,J10}]{tataḥ}% +P11,G4,G11
		\rdg[wit={V19}]{pi vā}
		\rdg[wit={Gr2,K3,C7,Jyo}]{'pi ca}}/}\\+} % +J5
\tl{
\pada{\app{\lem[wit={ceteri}]{niṣpattiḥ sarva} % niṣpaṃti om. N23
		\rdg[wit={C6,V3}]{niṣpattiś ceti}}yogeṣu} % yogiṣu P7
\pada{\app{\lem[wit={N3}]{yogāvasthā bhavanti tāḥ}% =Gr1
		\rdg[wit={Gr2,Gr3a}]{yogāvasthā prakīrtitā}
		\rdg[wit={C6,V3,N19,V15,J10,Jyo}]{syād avasthācatuṣṭayaṃ}}\marma//}\\!}
\end{tlg}

%\newpage
\begin{ava}[hp04_017]
\app{\lem[wit={N23,Jyo}]{athārambhāvasthā}
		\rdg[wit={J7}]{ārambhāvasthātha}
		\rdg[wit={Gr3a}]{athārambharakṣā}
		\rdg[wit={G4,N19,V15}]{tatra ārambhaḥ}
		\rdg[wit={J10}]{tatra cārambhaḥ}
		\rdg[wit={N3,J5,C6,V3},alt={\om}]{\skp{\om}}}/
	\NotIn{N3,C6,V3}
\end{ava}

\begin{tlg}[hp04_017]% = V3_4.53 = C8_56
\tl{
\pada{brahma\app{\lem[wit={N3,Jyo},alt={granther}]{granthe\skp{r}}% +J11pc
		\rdg[wit={P11}]{granthe}
		\rdg[wit={V3,Gr2,Gr3a,V15}]{granthir}  % grathir J7
		\rdg[wit={C6,N23}]{granthi}
		\rdg[wit={J10}]{granthiṃ}
		\rdg[wit={N19}]{raṃdhre}}r bhave%d % bhavod C6
	\app{\lem[wit={N3,C6,V3},alt={bhedād}]{\skm{d }bhedā\skp{d}}
		\rdg[wit={Gr2,Gr3a}]{bhinna}
		\rdg[wit={J10}]{bhinnā}
		\rdg[wit={V15}]{bhinnād}
		\rdg[wit={Jyo}]{bhedo hy}
		\rdg[wit={N19}]{bhed}}}%
\pada{\app{\lem[wit={ceteri},alt={ānandaḥ}]{\skm{d }ānandaḥ}
		\rdg[wit={C6,N23}]{ānaṃda}
		\rdg[wit={J10}]{nādaḥ}}
	śūnya\app{\lem[wit={ceteri}]{saṃbhavaḥ} % °bhava N19, °bhavaṃ P7
		\rdg[wit={J10}]{samaṃbhavaḥ}}/}\\+}
\tl{
\pada{\app{\lem[wit={N3}]{vicitrakvaṇako}% kvanako N3ac
		\rdg[wit={V15}]{vicitrakvaṇiko}
		\rdg[wit={V3,N19}]{vicitrakaṇako}
		\rdg[wit={J10}]{vicitrakuṇako}
		\rdg[wit={C6}]{vicitrakuṇape}
		\rdg[wit={Jyo}]{vicitraḥ kvaṇako}
		\rdg[wit={K3,C7}]{vicitrakṣaṇike}
		\rdg[wit={V19}]{vicitrakṣike}
		\rdg[wit={Gr2}]{vicitras tatkṣaṇād}} % ta<<t>> N23
	\app{\lem[wit={ceteri}]{dehe}
		\rdg[wit={C6}]{caivā}}}% caiva C6ac
\pada{\app{\lem[wit={N3,C6,V3,N19,V15,J10,Jyo}]{'nāhataḥ śrūyate}
		\rdg[wit={Gr2}]{sarvataḥ śrūyate}
		\rdg[wit={Gr3a},post=\texteng{(°hato \getsiglum{K3})}]{śrūyate (')nāhata}}
		dhvaniḥ//}\\!} % dhvani J7,V3
\end{tlg}

\begin{tlg}[hp04_018]% = V3_4.54ab = C8_57
\tl{
\pada{\app{\lem[wit={N3,C6,Gr2,Jyo}]{divyadehaś ca tejasvī} % tejasvā N3
		\rdg[wit={N19},post={\unm}]{ādityatejaś ca tejasvī}
		\rdg[wit={V15}]{tejasvī divyagandhaś ca}
		\rdg[wit={J10}]{divyagandho divyacakṣuś ca}
		\rdg[wit={V3,Gr3a},alt={\om}]{\skp{\om}}}}
\pada{\app{\lem[wit={N3,C6,Gr2,Jyo}]{divyagandhas tv arogavān} % gaṃdhās N3
		\rdg[wit={N19}]{divyagandho parogavān}
		\rdg[wit={V15}]{divyadeho py arogavān}
		\rdg[wit={J10}]{tejasvī ārogavān}
		\rdg[wit={V3,Gr3a},alt={\om}]{\skp{\om}}}/}
		\lineom{ab}{Gr3a,V3}\\+}
\tl{
\pada{\app{\lem[wit={ceteri}]{saṃpūrṇa}
		\rdg[wit={V15}]{saṃpūrṇe}}%
	\app{\lem[wit={N3,P11,N19,Jyo}]{hṛdayaḥ}
		\rdg[wit={J7}]{hṛdaya}
		\rdg[wit={C6,V3,N23,V19,K3,V15,J10}]{hṛdaye}% +J11pc
		\rdg[wit={C7}]{nilaye}}
	\app{\lem[wit={N3,N19,V15},alt={śūnye tv}]{śūnye\skp{ tv}}
		\rdg[wit={C6,Gr2,Gr3a,J10}]{śūnye}
		\rdg[wit={V3,Jyo}]{śūnya}}}%
\pada{\app{\lem[wit={ceteri},alt={ārambhe}]{\skm{tv }ārambhe}
		\rdg[wit={V3}]{ārambha}
		\rdg[wit={J10}]{āraṃbho}}
	\app{\lem[wit={ceteri},alt={yogavān}]{yogavā\skp{n}}
		\rdg[wit={N23}]{bhogavān}}n bhavet//}\\!}
\end{tlg}

\begin{ava}[hp04_019]
atha
\app{\lem[wit={ceteri}]{ghaṭāvasthā}% ghaṭavaṃsthā N19, ghaṃṭā° V17
		\rdg[wit={Gr3a}]{ghaṭarakṣā}}/
\end{ava}

\begin{tlg}[hp04_019]% = V3_4.54/55
\tl{
\pada{\app{\lem[wit={N3,C6,V3,Gr2,K3,C7,V15pc,N19,Jyo}]{dvitīyāyāṃ} % dviti° V3
		\rdg[wit={V19,V15ac}]{dvitīyā}
		\rdg[wit={J10}]{dvitīye}}
	\app{\lem[wit={ceteri}]{ghaṭī}
		\rdg[wit={V15}]{ghaṃṭi}
		\rdg[wit={N19}]{ghaṭāṃ}
		\rdg[wit={J10}]{bheda}}%
	\app{\lem[wit={N3,C6,V3,N23,Gr3a,N19,Jyo}]{kṛtya}
		\rdg[wit={J7,V15}]{kṛtvā}
		\rdg[wit={J10}]{mukte tu}}}
\pada{vāyur bhavati  % vāḥsur P7
	\app{\lem[wit={ceteri}]{madhyagaḥ}
		\rdg[wit={K3,C7}]{madhyamaḥ}}/}\\+}
\tl{
\pada{\app{\lem[wit={ceteri}]{dṛḍhāsano}
		\rdg[wit={K3}]{dṛḍhāsane}
		\rdg[wit={J10}]{haṭhāsano}} bhaved yogī}
\pada{jñānī
	\app{\lem[wit={ceteri}]{deva}
		\rdg[wit={V3}]{devaḥ}
		\rdg[wit={C6,J10}]{deha}}sama%s
	\app{\lem[wit={N3,C6,V3,Jyo},alt={tadā}]{\skm{s }tadā}
		\rdg[wit={ceteri}]{tathā}}//}\\!}
\end{tlg}

\newpage
\begin{tlg}[hp04_020]% = V3_4.55/56
\tl{
\pada{viṣṇu\app{\lem[wit={N3,P11}]{granthes tadā}% +G4,N24
		\rdg[wit={V3}]{granthis tadā}
		\rdg[wit={N19}]{granthe sadā}
		\rdg[wit={J10}]{granthes tathā}% +J5
		\rdg[wit={C6}]{granther yadā}
		\rdg[wit={Gr2,Gr3a,V15}]{granthir yadā}% <<graṃthi>> C7
		\rdg[wit={Jyo}]{granthes tato}}
	\app{\lem[wit={N3,C6,V3,N19,J10,Jyo}]{bhedāt}
		\rdg[wit={Gr2,V19,K3}]{bhinnaḥ} % vimugraṃthipadābhinna N23
		\rdg[wit={C7,V15}]{bhinnā}}}
\pada{\app{\lem[wit={ceteri}]{paramānanda}
		\rdg[wit={N19}]{sadānandasya}}%
	\app{\lem[wit={ceteri}]{sūcakaḥ} % °caka V3
		\rdg[wit={V15}]{sūcakā}
		\rdg[wit={C6}]{kārakaḥ}}/}\\+}% °kāḥ V15pc
\tl{
\pada{\app{\lem[wit={N3,P11,V3,Jyo}]{atiśūnye}
		\rdg[wit={Gr2,Gr3a,V15,J10}]{atiśūnya}
		\rdg[wit={C6}]{aṃtyaśūnye}
		\rdg[wit={N19}]{api śūnyo}}
	\app{\lem[wit={N3,C6,V3,Jyo}]{vimardaś ca}
		\rdg[wit={N19}]{'saṃmardā}
		\rdg[wit={J10}]{visaṃmardo}
		\rdg[wit={Gr2,Gr3a,V15}]{vibhedaś ca}}}
\pada{bherīśabdas % bheṭī V17, bhīrī K3
	\app{\lem[wit={N3,C6,V3,V15,Jyo}]{tadā}
		\rdg[wit={Gr2,Gr3a,N19,J10}]{tathā}% +N24, tadhā G4, tatho J5
		} bhavet//}\\!}
\end{tlg}

%\newpage


\begin{ava}[hp04_021]
\app{\lem[wit={ceteri}]{atha}
		\rdg[wit={C6}]{tathā}
		\rdg[wit={Jyo},alt={\om}]{\skp{\om}}}
\app{\lem[wit={ceteri}]{paricayāvasthā}% <<yā>> N23
		\rdg[wit={N19,V15}]{paricayaḥ} % °caya N19
		\rdg[wit={Jyo},alt={\om}]{\skp{\om}}}/ \NotIn{Jyo}
\end{ava}

\begin{tlg}[hp04_021]% = V3_4.57 = C8_60
\tl{
\pada{\app{\lem[wit={N3,C6,V3,V19,C7,V15}]{tṛtīyāyāṃ tato bhittvā} % tṛtiyāyāṃ V3
		\rdg[wit={K3}]{dvitīyāyāṃ tato bhittvā}
		\rdg[wit={Gr2}]{karṇikāṃ tu tato bhittvā}
		\rdg[wit={N19}]{karttikāyāṃ tato bhittvā}
		\rdg[wit={J10}]{atha granthitrayaṃ bhittvā}
		\rdg[wit={Jyo}]{tṛtīyāyāṃ tu vijñeyo}}}
\pada{\app{\lem[wit={J5,N19,Jyo}]{vihāyo}% +J5
		\rdg[wit={Gr2,V15}]{vihāya}
		\rdg[wit={V19,C7}]{vimalo}
		\rdg[wit={K3}]{mimalo}
		\rdg[wit={V3}]{vimāyo}
		\rdg[wit={C6}]{visphāro}
		\rdg[wit={J10}]{jāyate}}%
	\app{\lem[wit={J5,C6,V3,Gr2,N19,J10,Jyo}]{mardala}% +J5
		\rdg[wit={Gr3a}]{mandala} % maṃdala V19
		\rdg[wit={V15}]{mṛḍula}}%
	\app{\lem[wit={J5,C6,N23,Gr3a,N19,V15,J10,Jyo}]{dhvaniḥ}% +J5
		\rdg[wit={J7}]{dhvaniṃ}
		\rdg[wit={V3}]{dhvani}}/}\marma\\+}
\tl{
\pada{\app{\lem[wit={ceteri}]{mahāśūnyaṃ}% +J5
		\rdg[wit={V15}]{mahāśūnya}}
	\app{\lem[wit={Gr2,N19}]{tathā}
		\rdg[wit={Gr3a}]{tato}
		\rdg[wit={J5,C6,V3,Jyo}]{tadā}% +J5,N24 ##
		\rdg[wit={V15}]{tayā}
		\rdg[wit={J10}]{samā}}
	\app{\lem[wit={ceteri}]{yāti}
		\rdg[wit={N19}]{jātiḥ}}} % jāti J5
\pada{\app{\lem[wit={ceteri}]{sarvasiddhi} % +J5,P11
		\rdg[wit={V3}]{mahāsiddhi}
		\rdg[wit={C6}]{siddhisādha°}
		\rdg[wit={N19}]{sarva}}% sarva<siddhi> N19
	\app{\lem[wit={ceteri}]{samāśrayam}% matāśrayāt J5, samāśriyaṃ P11
		\rdg[wit={C6}]{kam āśrayaṃ}}//}\label{cittananda}
		\anm{Pāda b--\ref{nadanu}d lost \getsiglum{N3}}\\!}
\end{tlg}

\begin{tlg}[hp04_022]% = V3_4.58
\tl{
\pada{\app{\lem[wit={G4,C6,Gr2,Gr3a,Jyo}]{cittānandaṃ}% <<da>> C7; +G4,N24
		\rdg[wit={J5,V3,V15}]{cidānanda(ṃ)}% cidānaṃdaṃ J5,HR
		\rdg[wit={J10}]{ciṃtāmanas}
		\rdg[wit={N19}]{virāmānaṃ}} % vivarttānaṃdaṃ P11
	\app{\lem[wit={ceteri}]{tato}
		\rdg[wit={Jyo}]{tadā}}
	\app{\lem[wit={C6,V3,N19,V15,J10,Jyo}]{jitvā}
		\rdg[wit={Gr2,Gr3a}]{bhittvā}}}
\pada{sahajānanda%
	\app{\lem[wit={ceteri}]{saṃbhavaḥ}
		\rdg[wit={N19}]{saṃbhava}}/}\\+}
\tl{
\pada{\app{\lem[wit={ceteri}]{doṣaduḥkha}
		\rdg[wit={N23}]{dokhaduḥkhe}}%
	\app{\lem[wit={C6,V3,V15,J10}]{jarāmṛtyu}
		\rdg[wit={J5,N19}]{jarāmṛtyuḥ}% +J5
		\rdg[wit={Jyo}]{jarāvyādhi}
		\rdg[wit={Gr2,Gr3a}]{kṣudhānidrā}}}%
\pada{\app{\lem[wit={J5,C6,N19,V15,J10,Jyo}]{kṣudhānidrā}% +J5
		\rdg[wit={V3}]{kṣudhātṛṣā}
		\rdg[wit={Gr2,Gr3a}]{jarāmṛtyu}}% mṛrtyu N23
	\app{\lem[wit={ceteri}]{vivarjitaḥ}
		\rdg[wit={C6},alt={°tāḥ}]{vivarjitāḥ}
		\rdg[wit={V3},alt={°taṃ}]{vivarjitaṃ}
		\rdg[wit={J10}]{tṛṣā tathā}}//}\\!}
\end{tlg}

\begin{ava}[hp04_023]
atha \app{\lem[wit={C6,V3,Gr2}]{niṣpattyavasthā} % niḥpatya° N23, niṣpatya° J7, niḥpatti-avasthā J5
		\rdg[wit={Gr3a}]{niṣṭhāvasthā}
		\rdg[wit={N19,V15,J10}]{niṣpattiḥ}% niṣpati N19, °tiḥ J10
		}/\myfn{%
		In \getsiglum{J5,C6,V3,J7,Gr3a} the header is found after the first line of \ref{rudra}.}
		\NotIn{Jyo}
\end{ava}

\begin{tlg}[hp04_023]% = V3_4.59/60
\tl{
\pada{rudragranthiṃ % rudraṃ graṃ{{ga}}thi N23; grathiṃ J7, graṃthi C7,N19,V3
	\app{\lem[wit={ceteri}]{tato}
		\rdg[wit={Jyo}]{yadā}}
	\app{\lem[wit={ceteri}]{bhittvā} % bhītvā P7
		\rdg[wit={N19}]{bhūtvā}}}
\pada{\app{\lem[wit={ceteri}]{sarva}
		\rdg[wit={Jyo}]{śarva}}pīṭha%
	\app{\lem[wit={ceteri}]{gato'nilaḥ}
		\rdg[wit={J7}]{gatonalaḥ}
		\rdg[wit={V3}]{gatānila}}/}\\+}
\tl{
\pada{\app{\lem[wit={C6,V3,J7,Jyo}]{niṣpattau}
		\rdg[wit={N19,V15}]{niṣpannau} % niḥṣpannau? V15
		\rdg[wit={J10}]{niṣpanno} % niḥpanno J10
		\rdg[wit={N23}]{niṣpatto}
		\rdg[wit={Gr3a}]{niṣṭhāto}}
	\app{\lem[wit={ceteri}]{vaiṇavaḥ śabdaḥ}% veṇavaḥ V17
		\rdg[wit={J7}]{vaiṇavaśabdaḥ}
		\rdg[wit={N23}]{veṇacaśabdaṃ}}}
\pada{\app{\lem[wit={V15,Jyo}]{kvaṇadvīṇākvaṇo}% viṇā V15
		\rdg[wit={N19}]{kaṇatvīnakvaṇo}% kaṇatvīna sic!
		\rdg[wit={J7}]{kvaṇadvīṇotvaṇo}% < °vīṇolbaṇo?
		\rdg[wit={V3}]{kvaṇatuvītakvaṇo}% kvaṇanvītaḥ kvaṇo P11
		\rdg[wit={C6}]{kvacid vīṇākvaṇo}
		\rdg[wit={J10}]{kvaṇantenākvuṇo}
		\rdg[wit={Gr3a}]{kvaṇadvīṇāsamo}% vīnā V19; kvaṇadvīvakvaṇo K1
		\rdg[wit={N23}]{karṇavīṇādgato}}\marmas
		bhavet//}\label{rudra}\\!}
\end{tlg}

% MD: marking für die andere Recensions hier abgebrochen.
\newpage
\begin{tlg}[hp04_024]% = V3_4.60/61 = C8_63 %% P11 has Pada a only
\tl{
\pada{ekībhūtaṃ % eka° C6
	\app{\lem[wit={J5,C6,V3,Jyo}]{tadā} % P11
		\rdg[wit={Gr2,Gr3a,J10}]{tathā}} cittaṃ}
\pada{\app{\lem[wit={ceteri}]{rājayogā}
		\rdg[wit={J10}]{rājayoga}
		\rdg[wit={V3}]{rājayogo}}%
	\app{\lem[wit={V3,J7}]{bhidhāyakam}% lost P11
		\rdg[wit={J5}]{vidhāyakaḥ}% °yogāvi° J5
		\rdg[wit={N23}]{bhidhāyanaṃ}
		\rdg[wit={G4,C6,Gr3a,J10,Jyo}]{bhidhānakaṃ}}\marma/}\\+} % °nakaṃ G4,N24
\tl{
\pada{sṛṣṭisaṃhāra%
	\app{\lem[wit={ceteri}]{kartāsau}
		\rdg[wit={N23}]{karttasau}
		\rdg[wit={V3}]{karttāso}}}
\pada{yogīśvarasamo bhavet//}
	\NotIn{N19,V15} \label{ekibhutam}
	\anm{\getsiglum{C7} in mg. sec. m.}\\!}
\end{tlg}

%\newpage
\startaltrecension
\begin{alttlg}[hp04_024_1]% = V3_4.62 = C8_64
\tl{
\pada{rājayoga\app{\lem[wit={V3,V15,J10,Jyo}]{padaṃ}
		\rdg[wit={Gr4b,N19}]{pada}}
	\app{\lem[wit={J10,Jyo}]{prāptuṃ}
		\rdg[wit={V3}]{prāptaṃ}
		\rdg[wit={N19}]{prāptaḥ}
		\rdg[wit={V15}]{prāpti}
		\rdg[wit={Gr4b}]{prāptau}% +C8,P7,M3 ##
		}}
\pada{\app{\lem[wit={Gr4b,N19,V15,J10,Jyo}]{sukhopāyo'lpa}
		\rdg[wit={V3}]{sukhopāyogya}}cetasām/}\\+}
\tl{
\pada{sadyaḥ
	pratyaya\app{\lem[wit={C6,V3,N19,J10,Jyo}]{saṃdhāyī}
		\rdg[wit={P11,V15}]{saṃdhāyi}}}
\pada{\app{\lem[wit={GrB,N19,V15,Jyo}]{jāyate}
		\rdg[wit={J10}]{sevyate}}
	\app{\lem[wit={C6,V3,N19,Jyo}]{nādajo layaḥ} % laya V3
		\rdg[wit={P11,J10}]{nādayo layaḥ}
		\rdg[wit={V15}]{nātra saṃśayaḥ}}//}
	\sgwit{GrB,N19,V15,J10,Jyo} \anm{cf. \ref{saukhya}}\\!}
\end{alttlg}
\endaltrecension



\Anm{Verses \ref{astuva}--\ref{svastho} are found after \ref{sravanaputa} in \getsiglum{N19,V15,J10}}

\begin{tlg}[hp04_025]% = V3_4.61/62 = 4c_106
\tl{
\pada{astu vā
	\app{\lem[wit={C6,Gr2,V19,C7,N19,J10,Jyo}]{māstu} % māsta N19
		\rdg[wit={V3,V15}]{mastu}
		\rdg[wit={K3}]{nāstu}} vā
	\app{\lem[wit={C6,K3,C7,N19,J10,Jyo},alt={muktir}]{mukti\skp{r}}% +J5
		\rdg[wit={V15}]{muktis}
		\rdg[wit={V3}]{muktiṃ}
		\rdg[wit={Gr2}]{śaktir}
		\rdg[wit={V19}]{kiṃcid}}}%
\pada{\app{\lem[wit={C6,Gr3a,Jyo},post=\texteng{(°te \getsiglum{K3})},alt={atraivākhaṇḍitaṃ}]{\skm{r }atraivākhaṇḍitaṃ}
		\rdg[wit={J7}]{atraiva khaṇḍitaṃ}% +J5
		\rdg[wit={J10}]{atra vākhaṇḍitaṃ}
		\rdg[wit={N23}]{ātrevikhaṇḍitaṃ}
		\rdg[wit={N19}]{atraivāṣaṃḍitaṃ}
		\rdg[wit={V3,V15}]{tatraivākhaṇḍitaṃ}}
	\app{\lem[wit={ceteri}]{mahat}
		\rdg[wit={N23}]{marut}
		\rdg[wit={C6}]{manaḥ}
		\rdg[wit={V19}]{bhavet}
		\rdg[wit={Jyo}]{sukham}}/}\\+}
\tl{
\pada{\app{\lem[wit={C6,N19,V15}]{layāmṛtamayaṃ}% leyā° J5, staye° N24
		\rdg[wit={V3}]{layāmṛtalayaṃ}
		\rdg[wit={J7,Gr3a}]{layāmṛtam idaṃ}
		\rdg[wit={N23}]{layāmṛdaṃmitaṃ}
		\rdg[wit={J10}]{layāmṛtakaraṃ}
		\rdg[wit={Jyo}]{layodbhavam idaṃ}}
	\app{\lem[wit={ceteri}]{saukhyaṃ}
		\rdg[wit={N23}]{sokhyaṃ}
		\rdg[wit={J7,J10}]{sauṣyaṃ}
		\rdg[wit={N19}]{saukṣaṃ}}}
\pada{\app{\lem[wit={ceteri}]{rājayogād avāpyate}
		\rdg[wit={J10}]{rājayogam avāpyate}
		\rdg[wit={V19},alt={\om}]{\skp{\om}}}//}\label{astuva}\\!}
\end{tlg}
%	\anm{\getsiglum{N19,V15,J10} after 4.71?}

%\Anm{\getsiglum{V3} has 4.48*1--\ref{muktasana}, \ref{kasthe}--\ref{sarvacinta} etc. here}

%\newpage
\begin{tlg}[hp04_026]%
\tl{
\pada{haṭhaṃ vinā rājayogo} % yogaḥ J11
\pada{rājayogaṃ vinā haṭhaḥ/}\\+} % yoga J11
\tl{
\pada{na sidhyati tato yugmam}
\pada{ā niṣpatteḥ samabhyaset//}% +M1,J11
\myfn{The verse is abbreviated with \textit{haṭhaṃ vinā rājayoga iti} in \getsiglum{N19,V15}, probably because it is same as 2.77.}\label{hathamvina}
\NotIn{V3,Gr2,Gr3a,J10,Jyo} 
\anm{= 2.77}\\!}

%\sgwit{Gr1r,Gr4b,N19,V15} 
\end{tlg}

% \myfn{\getsiglum{N19} has then \devnote{460 saṃ ka} before its last verse \ref{ajananta}.}

\begin{tlg}[hp04_027]%
\teimute{\normalsize}\color{black}
\tl{
\pada{rājayogam ajānantaḥ} % ajānaṃteḥ N19
\pada{kevalaṃ
	haṭha\app{\lem[wit={P11,V15}]{karmaṭhāḥ}
		\rdg[wit={N19}]{karmacā}
		\rdg[wit={C6,V3}]{karmaṇā}
		\rdg[wit={J10}]{karmaṇaḥ}
		\rdg[wit={Jyo}]{karmiṇaḥ}}/}\\+}
\tl{
\pada{\app{\lem[wit={Gr4b}]{ye tu tān karṣakān manye}% kāmukān J5 ##
		\rdg[wit={N19,V15}]{ye tu tān karkaśān manye}
		\rdg[wit={J10}]{ye tuṃgān karmavasān manye}
		\rdg[wit={Jyo}]{etān abhyāsino manye}
		\rdg[wit={V3},alt={\gap}]{\skp{\gap}}}} % V3 gap
\pada{\app{\lem[wit={N19,V15,J10,Jyo},post=\texteng{(°varjitāḥ \getsiglum{J10})}]{prayāsaphalavarjitān}
		\rdg[wit={P11}]{prāyaśaphalavarjitān}
		\rdg[wit={C6}]{prāyaśaḥ phalavarjitān}
		\rdg[wit={V3},alt={\gap}]{\skp{\gap}}}//}
	\NotIn{Gr2,Gr3a}\label{ajananta}
	\anm{\getsiglum{N19} ends with this}\\!}
%		\sgwit{Gr1r,GrB,N19,V15,J10,Jyo}
\end{tlg}


\startaltrecension
\begin{alttlg}[hp04_027_1]%
\tl{\texteng{[Alt] }\pada{\app{\lem[wit={Gr2,K3,C7}]{haṭhaṃ vinā}
		 \rdg[wit={V19},alt={\om}]{\skp{\om}}}
	 \app{\lem[wit={J7,K3,C7}]{rājayogaṃ}
		 \rdg[wit={N23}]{rājayogo}
		 \rdg[wit={V19},alt={\om}]{\skp{\om}}}}
\pada{rājayogaṃ vinā
	 \app{\lem[wit={J7,Gr3a}]{haṭhaṃ}
		 \rdg[wit={N23}]{haṭhaḥ}}/}\\+}
\tl{
\pada{ye \app{\lem[wit={N23,Gr3a}]{vai}
		 \rdg[wit={J7}]{cai}}
	 \app{\lem[wit={Gr3a}]{caranti}
		 \rdg[wit={Gr2}]{varaṃti}} tā%n
	 \app{\lem[wit={N23,Gr3a},alt={\skm{n }manye}]{manye}
		 \rdg[wit={J7}]{madhye}}}
\pada{prayāsa\app{\lem[wit={J7,Gr3a}]{phala} % varjitāḥ J10
		 \rdg[wit={N23}]{pralevi}}varjitān//}
		 \sgwit{Gr2,Gr3a}%
\myfn{\getsiglum{Gr2,Gr3a} have this verse in place of \ref{hathamvina}--\ref{ajananta}.}\\!}
\end{alttlg}
\endaltrecension

%\Anm{\getsiglum{V3} has \ref{ajananta}--\ref{saukhya} after the Kālavañcana section}

\newpage
\begin{tlg}[hp04_028]% = V3_4.161
\tl{
\pada{tattvaṃ\marmas bījaṃ % tattva N23,V3
	\app{\lem[wit={V19,Jyo}]{haṭhaḥ}
		\rdg[wit={P11,Gr2,J11,V15}]{haṭha}% +J5
		\rdg[wit={C6,V3,K3,C7,J10}]{haṭhaṃ}}
	\app{\lem[wit={ceteri},alt={kṣetram}]{kṣetra\skp{m}}
	\rdg[wit={J11}]{kṣetre}}}%m 
\pada{\app{\lem[wit={GrB,Gr2,K3,C7,J11,J10,Jyo},alt={audāsīnyaṃ}]{\skm{m }audāsīnyaṃ}
		\rdg[wit={V15}]{audāsinyaṃ}
		\rdg[wit={V19}]{<<sau>>dāmanyaṃ}}
	\app{\lem[wit={J5,P11,V3,V15,J10,Jyo}]{jalaṃ tribhiḥ}% tribhi V3
		\rdg[wit={C6,Gr2,V19,C7}]{jalaṃ smṛtam}
		\rdg[wit={K3}]{jalaṃ matam}
		\rdg[wit={J11}]{jalaplavaṃ}}/}\\+}
\tl{
\pada{unmanīkalpalatikā}
\pada{sadya
	\app{\lem[wit={J5,GrB,Gr3a,J11,V15,J10}]{evodbhaviṣyati}% yevo C6, evā P11
		\rdg[wit={Gr2}]{eva bhaviṣyati}
		\rdg[wit={Jyo}]{eva pravartate}}//}\\!}
\end{tlg}


%\newpage
\begin{tlg}[hp04_029]% = V3_4.162
\tl{
\pada{\app{\lem[wit={V3,J7,Gr3a}]{rājayogaḥ}
		\rdg[wit={J5,Gr4b,N23}]{rājayoga}} % +J5
	samādhi\app{\lem[wit={P11,V3,Gr2,Gr3a},alt={ca}]{\skm{ś }ca}% cā P11
		\rdg[wit={C6}]{ca hy}}}
\pada{unmanī ca manonmanī/}\\+}
\tl{
\pada{\app{\lem[wit={V3},postwit=\texteng{(amaro°)}]{amaraugho'pi cādvaitaṃ}% ~J5
		\rdg[wit={P11}]{amarogho pi vādvaitaṃ}
		\rdg[wit={C6}]{amaraughāpi cādvaitaṃ}
		\rdg[wit={J7}]{amaraudhyaighacāṃdrī ca}
		\rdg[wit={N23}]{araughaughatvīṃdrī ca}
		\rdg[wit={Gr3a}]{amaroly abhicāndrī ca}}}
\pada{\app{\lem[wit={GrB,Gr2}]{nirālambaṃ}
		\rdg[wit={Gr3a}]{nirālambo}} nirañjanam//}\label{A1}
	\sgwit{Gr1r,GrB,Gr2,Gr3a} \anm{cf. \ref{synonym3}}\\!}
\end{tlg}


\begin{tlg}[hp04_030]% = V3_4.163
\tl{
\pada{\app{\lem[wit={GrB,J7,V19}]{amanasko} % °skoṃ? V19
		\rdg[wit={N23}]{amanaskau}
		\rdg[wit={K3,C7}]{amanaskaṃ}}
	\app{\lem[wit={GrB}]{layas tattvaṃ}
		\rdg[wit={J5}]{layas tatra}
		\rdg[wit={J7,Gr3a}]{layaś caiva}
		\rdg[wit={N23}]{lyayāś caiva}}}
\pada{\app{\lem[wit={P11,J7,Gr3a}]{śūnyāśūnyaṃ}
		\rdg[wit={V3,N23}]{śūnyāśūnya}
		\rdg[wit={C6}]{śūnyāc chūnyaṃ}}
	\app{\lem[wit={J5,G4,GrB}]{paraṃ padam}
		\rdg[wit={Gr3a}]{parāparaṃ}
		\rdg[wit={N23}]{parāparāṃ}
		\rdg[wit={J7}]{parāvaraṃ}}/}\\+}
\tl{
\pada{\app{\lem[wit={GrB,J7,Gr3a}]{jīvanmuktiś ca}
		\rdg[wit={N23}]{jīvanmuktiḥ}} sahajaṃ} % ṃ om. N23
\pada{\app{\lem[wit={Gr4b,Gr2,K3,C7}]{turyaṃ}
		\rdg[wit={V19}]{turjaṃ}
		\rdg[wit={V3}]{tuṣkaṃ}}
	\app{\lem[wit={P11,J7,V19,K3}]{cety eka}
		\rdg[wit={C7}]{cety eva}
		\rdg[wit={N23}]{vatyaka}
		\rdg[wit={V3}]{caiyeka}
		\rdg[wit={C6}]{cityeka}}%
	\app{\lem[wit={N23}]{vācakāḥ}
		\rdg[wit={J7}]{vācakīṃ}
		\rdg[wit={GrB,Gr3a}]{vācakaṃ}}//}\label{A2}
	\sgwit{Gr1r,GrB,Gr2,Gr3a} \anm{cf. \ref{synonym4}}\\!}
\end{tlg}


%\newpage
\begin{tlg}[hp04_031]% = V3_4.164
\tl{
\pada{\app{\lem[wit={J5,GrB,Jyo}]{unmanyavāptaye}
		\rdg[wit={V19}]{unmanyavāsayet}
		\rdg[wit={K3,C7}]{unmanyā vāsayec}} śīghraṃ}
\pada{\app{\lem[wit={Gr3a}]{dvau mārgau}
		\rdg[wit={J5,Gr4b}]{mārgau dvau}% Gr1*
		\rdg[wit={V3}]{mārgo dvau}
		\rdg[wit={Jyo}]{bhrūdhyānaṃ}}
	\app{\lem[wit={J5,C6,V3}]{mama saṃmatau}
		\rdg[wit={P11,Gr3a}]{samasaṃmatau}
		\rdg[wit={Jyo}]{mama saṃmatam}}/}
	\sgwit{Gr1r,GrB,Gr3a,Jyo}\\+}
\tl{
	%\lineom{ab}{Gr2,N19,V15,J10}
\pada{tattvaṃ parama\app{\lem[wit={C6,Gr2}]{saukhyaṃ} % 2 x ṃ om. N23
		\rdg[wit={J5}]{sākhyaṃ}
		\rdg[wit={V3}]{sāṃkhyaṃ}
		\rdg[wit={P11}]{vāgraṃ}} vā}
\pada{nādopāsanam eva  % nadipā°? J7ac
	\app{\lem[wit={J5,V3,Gr2}]{ca}% +J5
	\rdg[wit={Gr4b}]{vā}}//} % cā N24
	\sgwit{Gr1r,GrB,Gr2}\\!}% G4 only supposedly
\end{tlg}
%	\lineom{cd}{Gr3a,N19,V15,J10,Jyo}

\begin{tlg}[hp04_032]% = V3_4.165 %%% perhaps 6pāda-verse
\tl{
\pada{\app{\lem[wit={C6,N23}]{saukhya}
		\rdg[wit={J7}]{saukhyā}
		\rdg[wit={P11,V3}]{sāṃkhya}}%
	\app{\lem[wit={C6,V3,J7}]{praviṣṭa}
		\rdg[wit={P11,N23}]{pratiṣṭha}}cittānāṃ} % cittānyu N23
\pada{mūḍhānām api saṃmatam/}
	\sgwit{Gr1r,GrB,Gr2}\\+} % G4 only supposedly
	%\lineom{ab}{Gr3a,N19,V15,J10,Jyo}
\tl{
\pada{\app{\lem[wit={Gr4b,Gr3a}]{sadya}
		\rdg[wit={V3}]{sadyaṃ}
		\rdg[wit={Gr2}]{satyam}}%
	\app{\lem[wit={GrB,Gr2,K3,C7}]{ānanda}
		\rdg[wit={V19}]{ādāya}}%
	\app{\lem[wit={J7,V19,C7}]{saṃdhāyī}
		\rdg[wit={N23}]{saṃdhyāyī}
		\rdg[wit={P11,K3}]{saṃdāyī}
		\rdg[wit={V3}]{sadāyī}
		\rdg[wit={C6}]{saṃdāyi}}}
\pada{\app{\lem[wit={ceteri}]{jāyate}
		\rdg[wit={V19}]{jāvate}}
	\app{\lem[wit={C6,V3,Gr2,Gr3a}]{nādajo}
		\rdg[wit={P11}]{nādato}} layaḥ//} % laya V3
	\sgwit{Gr1r,GrB,Gr2,Gr3a}\label{saukhya}\\!}
\end{tlg}
	%\NotIn{N19,V15,J10,Jyo}

%\newpage
\startaltrecension
\begin{alttlg}[hp04_032_1]%
\tl{
\pada{ekaṃ sṛṣṭimayaṃ bījaṃ}
\pada{ekā mudrā \app{\lem[wit={V15,J11}]{tu}
	\rdg[wit={J10}]{ca}} khecarī/}\\+}
\tl{
\pada{eko devo nirālamba}
\pada{ekāvasthā manonmanī//} % °mani V15
\sgwit{V15,J11,J10} \anm{= 3.48}\\!} % Not in V3,G7, but in G11
\end{alttlg}

%\newpage
\begin{alttlg}[hp04_032_2]% = V3_4.167
\tl{
\pada{śaṅkhadundubhi\app{\lem[wit={P11,V15,J11,J10,Jyo}]{nādaṃ ca}
		\rdg[wit={V3}]{nādaś ca}
		\rdg[wit={C6}]{nādāṃś ca}}}
\pada{na śṛṇoti kadācana/}\\+} % sṛṇoti J10; canaḥ J11
\tl{
\pada{\app{\lem[wit={V15,J11,J10,Jyo}]{kāṣṭhavaj jāyate}% =G11 jñāyate
		\rdg[wit={C6}]{sthāṇuvad vartate}% =M1
		\rdg[wit={P11}]{sthāṇu varddhattayed}% °yej jogī P11
		\rdg[wit={V3}]{sthāṇu vardhate}}
	\app{\lem[wit={J10,Jyo}]{deha}
		\rdg[wit={V15}]{dehe}
		\rdg[wit={J11}]{dehī}
		\rdg[wit={GrB}]{yogī hy}}} % =M1
\pada{unmanyā\app{\lem[wit={GrB,V15,J11,Jyo},alt={°vasthayā}]{\skp{°}vasthayā} % unmanyava° would be unmetrical. V15 has avagraha: unmanyā'va°
		\rdg[wit={J10}]{vasthāyāṃ}} dhruvam//} \sgwit{GrB,V15,J11,J10,Jyo}\\!}
\end{alttlg}

\newpage
\begin{alttlg}[hp04_032_3]% = V3_4.168
\tl{
\pada{sarvāvasthāvinirmuktaḥ}
\pada{sarvacintā\app{\lem[wit={Gr4b,V15,J11,J10,Jyo}]{vivarjitaḥ} % jita P11
		\rdg[wit={V3}]{vivarjitaṃ}}/}\\+}
\tl{
\pada{\app{\lem[wit={V15,J11,J10,Jyo},alt={mṛtavat}]{mṛtava\skp{t}} % =G11
		\rdg[wit={GrB}]{kāṣṭhavat}}% =M1
	\app{\lem[wit={Gr4b,V15,J11,J10,Jyo},alt={tiṣṭhate}]{\skm{t }tiṣṭhate}
		\rdg[wit={V3}]{tiṣṭhayed}} yogī}
\pada{sa mukto nātra saṃśayaḥ//} \sgwit{GrB,V15,J11,J10,Jyo}\\!}
\end{alttlg}

\Anm{\getsiglum{Jyo} has Vulg 4.108 \textit{khādyate na ca kālena}... here}

%\newpage
\begin{alttlg}[hp04_032_4]% = V3_4.169
\tl{
\pada{na \app{\lem[wit={V15,J11,Jyo}]{vijānāti}
		\rdg[wit={P11}]{hi jānāti}
		\rdg[wit={V3}]{hi jānaṃti}} śītoṣṇaṃ}
\pada{\app{\lem[wit={P11,V15,J11,Jyo}]{na duḥkhaṃ na sukhaṃ}
		\rdg[wit={V3}]{na ca duḥkhaṃ sukhaṃ}} tathā/}\\+}
\tl{
\pada{\app{\lem[wit={V15,J11,Jyo}]{na mānaṃ nāpamānaṃ}
		\rdg[wit={P11}]{na mānaṃ cāpamānaṃ}
		\rdg[wit={V3}]{na ca mānāpamānaṃ}} ca}
\pada{yogī \app{\lem[wit={Gr4b,Jyo}]{yuktaḥ}
		\rdg[wit={V15,J11}]{muktaḥ}
		\rdg[wit={V3}]{yukti}} samādhinā//}
		\sgwit{GrB,V15,J11,Jyo}\\!}
\end{alttlg}

%\newpage
\begin{alttlg}[hp04_032_5]%
\tl{
\pada{\app{\lem[resp=emend,postwit=\texteng{(cf.\,VM)}]{avedhyaḥ}
	\rdg[wit={V15,J10,Jyo}]{avadhyaḥ}
	\rdg[wit={J11}]{avadhya}} sarvaśastrāṇā}%m
\pada{\app{\lem[wit={V15,J11,J10},alt={avadhyaḥ}]{\skm{m }avadhyaḥ}
	\rdg[wit={Jyo}]{aśakyaḥ}} sarvadehinām/}\\+}
\tl{
\pada{agrāhyo mantra\app{\lem[wit={V15,J11,J10}]{tantrāṇāṃ}
	\rdg[wit={Jyo}]{yantrāṇāṃ}}} % tatrāṇāṃ V15
\pada{yogī \app{\lem[wit={J10,Jyo}]{yuktaḥ}
	\rdg[wit={V15,J11}]{muktaḥ}} samādhinā//}
	\sgwit{V15,J11,J10,Jyo}\\!} % NotIn{V3}
\end{alttlg}


\begin{alttlg}[hp04_032_6]% = V3_4.170
\tl{
\pada{na gandhaṃ na rasaṃ rūpaṃ}
\pada{\app{\lem[resp=emend]{na sparśaṃ na ca nisvanam}
	\rdg[wit={V3}]{sparśaṃ na ca na śrutaṃ}% the 1st na omitted
	\rdg[wit={Jyo}]{na ca sparśaṃ na niḥsvanam}}/}\\+}
\tl{
\pada{nātmānaṃ
	\app{\lem[wit={Jyo}]{na paraṃ vetti}
	\rdg[wit={V3}]{paramaṃ vetti}}}
\pada{yogī
	\app{\lem[wit={Jyo}]{yuktaḥ samādhinā}
	\rdg[wit={V3}]{yuktisamādhinā}}//} \sgwit{V3,Jyo}\\!} %\NotIn{N19,V15,J11,J10}
\end{alttlg}

\Anm{\getsiglum{V15,J11,J10} have \ref{pravese} \textit{praveśe nirgame vāme} here}

\begin{alttlg}[hp04_032_7]% = V3_4.171
\tl{
\pada{cittaṃ na suptaṃ no jāgrat} % jāgrati J10ac
\pada{\app{\lem[resp=emend,postwit=\texteng{(=\,G11)}]{smṛtiman na ca}
		\rdg[wit={C6}]{smṛtyamanna}
		\rdg[wit={V3}]{sṛtinannaṃ ca}
		\rdg[wit={V15,J11}]{smṛtivarṇaṃ ca}
		\rdg[wit={P11}]{na smṛtir na ca}
		\rdg[wit={Jyo}]{smṛtivismṛti}
		\rdg[wit={J10}]{spṛśati vastu ca}}
	\app{\lem[wit={GrB,V15,J11,J10}]{nānyathā}
		\rdg[wit={Jyo}]{varjitam}}/}\\+}
\tl{
\pada{\app{\lem[wit={GrB,V15,J11}]{nāstam eti}
		\rdg[wit={J10}]{na vāstum eti}
		\rdg[wit={Jyo}]{na cāstam eti}}
	\app{\lem[wit={Gr4b,V15,J11,J10}]{na codeti}
		\rdg[wit={V3}]{na cādeti}
		\rdg[wit={Jyo}]{nodeti}}}
\pada{\app{\lem[wit={Gr4b,V15,J11,Jyo}]{yasyāsau}
		\rdg[wit={J10}]{yathāsau} % mukti J10ac
		\rdg[wit={V3},alt={\illeg}]{\skp{\illeg}}}
		mukta eva saḥ//} 
		\sgwit{GrB,V15,J11,J10,Jyo}\label{nasuptam}\\!} % yavaprasa in J8
\end{alttlg}

%\newpage
\begin{alttlg}[hp04_032_8]% = V3_4.172
\tl{
\pada{\app{\lem[wit={V3,J11,Jyo}]{svastho}
		\rdg[wit={P11}]{svapno}
		\rdg[wit={C6}]{supto}
		\rdg[wit={V15}]{svecchā}} jāgradavasthāyāṃ} % ṃ om. V3
\pada{\app{\lem[wit={GrB,Jyo}]{suptavad yo}
		\rdg[wit={V15,J11}]{suptaḥ sadyo}}%
	\app{\lem[wit={V3,V15,J11,Jyo}]{'vatiṣṭhate}
		\rdg[wit={Gr4b}]{vatiṣṭhati}}/}\\+} % °tiṣṭhati V15pc?
\tl{
\pada{\app{\lem[wit={V15,J11,Jyo}]{niḥsvāsocchvāsa}
		\rdg[wit={V3}]{niśvāsośvāsa}
		\rdg[wit={P11}]{nisvāsośvaḥsa}
		\rdg[wit={C6}]{niḥśvāsaśvāsa}}%
	\app{\lem[wit={V3,V15,J11,Jyo}]{hīnaś ca}
		\rdg[wit={Gr4b}]{hīnas tu}}}
\pada{\app{\lem[wit={V15,J11,Jyo}]{niścitaṃ}
		\rdg[wit={V3}]{niścito}
		\rdg[wit={P11}]{niścitto}
		\rdg[wit={C6}]{niśceṣṭo}} mukta eva saḥ//}
		\sgwit{GrB,V15,J11,Jyo}\label{svastho}\\!} % sa V3
\end{alttlg}
\endaltrecension


%\Anm{The following verses appear immediately after \ref{ekibhutam} in \getsiglum{N19,V15,J10} and after 4.42 in \getsiglum{GrB}}


%\newpage
%==================================
\begin{tlg}[hp04_033]% = V3_4.63
\tl{
\pada{nādānusaṃdhānasamādhibhājāṃ}\\+} % dānā° V15
\tl{
\pada{\app{\lem[wit={ceteri}]{yogīśvarāṇāṃ}
		\rdg[wit={J7}]{yogeśvarāṇāṃ}} % <<yo>>gośvarā° J7
	\app{\lem[wit={C6,V3,J7,Gr3a,V15}]{hṛdaye prarūḍham}% prarūḍha V3
		\rdg[wit={N23}]{hṛdayapra[rū]ḍhaṃ}
		\rdg[wit={N19,J10,Jyo}]{hṛdi vardhamānaṃ}}/}\\+}  % vaddha° N19, vaddhra°? N26
\newpage
\tl{
\pada{ānandam ekaṃ vacasā%m
	\app{\lem[wit={ceteri},alt={avācyaṃ}]{\skm{m }avācyaṃ} % yacasām V3
		\rdg[wit={N19}]{avākyaṃ}
		\rdg[wit={C6,Jyo}]{agamyaṃ}}}\\+}
\tl{
\pada{\app{\lem[wit={ceteri}]{jānāti}
		\rdg[wit={C6}]{jānāty a°}
		\rdg[wit={N19}]{jānaṃti}}
	\app{\lem[wit={V3,J7,N19,V15,J10,Jyo}]{taṃ śrī}
		\rdg[wit={C6}]{°taḥ śrī}
		\rdg[wit={N23}]{tatvaṃ śrī}
		\rdg[wit={Gr3a}]{tattvaṃ}}%
	\app{\lem[wit={ceteri}]{gurunātha}
		\rdg[wit={Gr3a}]{guṇanātha}}
	\app{\lem[wit={N3,C6,V3,J7,Gr3a,V15}]{eva}
		\rdg[wit={N23}]{evaṃ}
		\rdg[wit={N19,Jyo}]{ekaḥ}
		\rdg[wit={J10}]{ekaṃ}}//}
	\label{nadanu}\\!} % N3 resumes with nātha eva
\end{tlg}

\startaltrecension
\begin{alttlg}[hp04_033_1]% = V3_4.64
\tl{
\pada{\app{\lem[wit={ceteri}]{muktāsanasthito}
		\rdg[wit={N23}]{mudrāsanasthite}} yogī}
\pada{mudrāṃ saṃdhāya śāṃbhavīm/}\\+} % mudrā N23,V3,J10; śāṃbhavī Gr2,V3,J10
\tl{
\pada{śṛṇuyād dakṣiṇe karṇe} % śṛṇuyā V3, śruṇuyād V19,V15
\pada{nāda%m % nadam P11
	\app{\lem[wit={Gr2,Gr3a,N19,V15},alt={antargataṃ sadā}]{\skm{m }antargataṃ sadā}
		\rdg[wit={V3,J10}]{antargataṃ mahat}}//}\label{muktasana}
		 \sgwit{V3,Gr2,Gr3a,N19,V15,J10} \anm{cf. \ref{muktasana2}}\\!}
\end{alttlg}
		 % \NotIn{Gr1r,Jyo}
\endaltrecension


\begin{tlg}[hp04_034]% = V3_4.71
\tl{
\pada{sarva\app{\lem[wit={ceteri}]{cintāṃ} % cintā V3
	\rdg[wit={J10}]{ciṃtāḥ}} parityajya}
\pada{\app{\lem[wit={ceteri}]{sāvadhānena}
	\rdg[wit={N19,J10}]{sarvadānena}}
	cetasā/}\\+}
\tl{
\pada{\app{\lem[wit={C6,V3,Gr2,C7,V15,Jyo}]{nāda evānusaṃdheyo}% nāda yecānusaṃdhyeyo J5
	\rdg[wit={N19}]{nādam evānusaṃdh(y)e}% yo om. by haplogr.
	\rdg[wit={V19,K3,J10}]{nādam evānusaṃdhatte}
	}}
\pada{yoga\app{\lem[wit={ceteri},alt={sāmrājyam}]{sāmrājya\skp{m}}
	\rdg[wit={K3}]{sāmājyam}
	\rdg[wit={V19}]{samrājyam}}%
	\app{\lem[wit={C6,V3,J7,Gr3a,V15,Jyo},alt={icchatā}]{\skm{m }icchatā} % iṣṭatā J5
		\rdg[wit={N19}]{icchatāṃ}
		\rdg[wit={N23,J10}]{icchati}}//}%
\myfn{This verse is transposed with the next one in \getsiglum{V3}.}
\label{sarvacinta2} 
\anm{after \ref{sarvacinta} \getsiglum{N19,V15,J10}}\\!}
\end{tlg}

%\newpage
\begin{tlg}[hp04_035]% = V3_4.70
\tl{
\pada{\app{\lem[wit={ceteri}]{karṇau}
		\rdg[wit={N3,N23}]{karṇo}}
	\app{\lem[wit={N3,C6,V3,Gr2,K3,C7,N19,V15,Jyo}]{pidhāya}
		\rdg[wit={V19}]{pi}}
	\app{\lem[wit={N19}]{tūlena}
		\rdg[wit={P11}]{tulyena}
		\rdg[wit={N3,V3}]{mūlena}
		\rdg[wit={Gr2}]{hastena}
		\rdg[wit={C6,K3,C7,Jyo}]{hastābhyāṃ}
		\rdg[wit={V19}]{hastābhya[ṃ]}
		\rdg[wit={V15}]{śū\,\_\,na}}}
\pada{\app{\lem[wit={N3,N19,V15,Jyo}]{yaṃ}
		\rdg[wit={C6,Gr2,Gr3a}]{yaḥ}
		\rdg[wit={P11}]{saṃ}
		\rdg[wit={V3}]{sa}} śṛṇoti % śruṇoti V3
	\app{\lem[wit={N3,C6,V3,V19,K3,N19,V15,Jyo}]{dhvaniṃ muniḥ} % muni V3
		\rdg[wit={N23}]{dhvaniṃ muniṃ}
		\rdg[wit={J7}]{munir dhvanim}
		\rdg[wit={C7}]{dhvaniṃ dhvaniḥ}}/}\\+}
\tl{
\pada{tatra cittaṃ % citta N23
	\app{\lem[wit={N3,C6,V3,Jyo}]{sthirī}
		\rdg[wit={Gr2,Gr3a,N19,V15}]{sthiraṃ}} kuryād}
\pada{yāva%t  % yāva Gr2,V3
	\app{\lem[wit={N3,C6,Gr2,Gr3a,N19,V15,Jyo},alt={sthirapadaṃ}]{\skm{t }sthirapadaṃ}
		\rdg[wit={V3}]{sthiparamaṃ}}
	\app{\lem[wit={N3,C6,V3,Gr2,Gr3a,Jyo}]{vrajet}
		\rdg[wit={N19,V15}]{bhavet}}//}
		\NotIn{J10}\\!}
\end{tlg}

%\Anm{\getsiglum{V3} has \ref{makaranda1}--\ref{makaranda6} here}

%\newpage
%========= common verses ==================
\begin{tlg}[hp04_036]% = V3_4.079 = 4c_75
\tl{
\pada{abhyasyamāno %  abhyāsya° N23, °mano N3
	\app{\lem[wit={ceteri}]{nādo}
		\rdg[wit={N23}]{nātho}}%
	\app{\lem[wit={ceteri}]{'yaṃ}% +P11
		\rdg[wit={C6}]{yo}}}
\pada{\app{\lem[wit={C6,J7,Jyo}]{bāhyam āvṛṇute} % bāhyanāvṛṇute P11
		\rdg[wit={N23}]{bāhyanā\,\_\,ṇute}
		\rdg[wit={N3}]{bāhyam āśṛṇu} % bāhyaṃ ca śṛṇute J5
		\rdg[wit={V3}]{bāhyam āsṛṇate}
		\rdg[wit={N19}]{bāhyamānaśṛṇvate}
		\rdg[wit={J10}]{cānyam āśṛṇute}
		\rdg[wit={Gr3a,V15}]{bāhyam āvartayed}\marma} % °ye V15, avarttaye V19
	\app{\lem[wit={N3,J7,Gr3a,V15,J10,Jyo}]{dhvanim}
		\rdg[wit={N23}]{dhvani}
		\rdg[wit={C6,V3,N19}]{dhvaniḥ}}/}\\+}
\tl{
\pada{\app{\lem[wit={N3,C6,V3,Gr2,K3,C7,N19,V15,Jyo},alt={pakṣād/pakṣāt}]{pakṣā\skp{d}}
%		\rdg[wit={K3}]{pakṣāt}
		\rdg[wit={V19,J10}]{paścād}}%
	\app{\lem[wit={N3,V3,J7,C7,J10,Jyo},alt={vikṣepam akhilaṃ}]{\skm{d }vikṣepam akhilaṃ}
		\rdg[wit={N23}]{vikṣeyam akhilaṃ}
		\rdg[wit={V19}]{vikṣepam atulaṃ}
		\rdg[wit={P11}]{vikṣyemanilaṃ}
		\rdg[wit={N19,V15}]{vipakṣam akhilaṃ}
		\rdg[wit={K3}]{prakṣepam akṣilaṃ}
		\rdg[wit={C6}]{vipakṣayed enaṃ}}}
\pada{\app{\lem[wit={ceteri}]{jitvā} % jītvā P7
		\rdg[wit={J10}]{jīvo}} yogī sukhī bhavet//}\\!}
\end{tlg}


%\newpage
\begin{tlg}[hp04_037]% = V3_4.80
\tl{
\pada{\app{\lem[wit={ceteri}]{śrūyate}
		\rdg[wit={C7}]{jāyate}}
	\app{\lem[wit={ceteri}]{prathamābhyāse}
		\rdg[wit={V19}]{prathame bhyāse}
		\rdg[wit={N3}]{prathamābhyāso}}}
\pada{nādo nānāvidho
	\app{\lem[wit={ceteri}]{mahān}
		\rdg[wit={N19}]{mahāt}}/}\\+} % nādā N23; vidhā N19
\tl{
\pada{\app{\lem[wit={ceteri}]{vardhamāne tato'bhyāse}
		\rdg[wit={V15,Jyo}]{tato'bhyāse vardhamāne}}}
\pada{śrūyate % srūyate N3ac
	\app{\lem[wit={N3,C6,V3,Gr3a,J10}]{sūkṣmasūkṣmataḥ}
		\rdg[wit={Gr2,V15,Jyo}]{sūkṣmasūkṣmakaḥ} % sūjyasūjyakaḥ N23
		\rdg[wit={N19}]{sūkṣmata}}//}\\!} % haplogr.
\end{tlg}

\newpage
\begin{tlg}[hp04_038]% = V3_4.81
\tl{
\pada{ādau jaladhi% adau P7
	\app{\lem[wit={N3,C6,J7,Gr3a,N19,V15,J10,Jyo}]{jīmūta}
		\rdg[wit={P11,V3,N23}]{jīmūte}}}% °te J10pc
\pada{bherī% bhīrī K3
	\app{\lem[wit={P11,N19,V15,J10}]{nirjhara}
		\rdg[wit={V19}]{nirjara}
		\rdg[wit={C6,V3}]{nirbhara} % nigama C6, but nirbhara P7
		\rdg[wit={C7}]{bhūrbhūra}
		\rdg[wit={K3}]{durdura}
		\rdg[wit={N23}]{sarāva}
		\rdg[wit={J7}]{śabdatu}
		\rdg[wit={N3}]{rsara} % unm.
		\rdg[wit={Jyo}]{jharjhara}}%
	\app{\lem[wit={N3,C6,N19,Jyo}]{saṃbhavāḥ}
		\rdg[wit={Gr2,Gr3a,V15}]{saṃbhavaḥ}
		\rdg[wit={V3,J10}]{nisvanaḥ}}/}\\+} % niśvanaḥ
\tl{
\pada{madhye
	\app{\lem[wit={ceteri}]{mardala}
		\rdg[wit={K3,C7}]{mandala}}%
	\app{\lem[wit={N3,N19,V15,Jyo}]{śaṃkhotthā}
		\rdg[wit={C6,V3,Gr2,V19,C7,J10}]{śaṃkhottha}% °otha Gr2, saṃkho V3
		\rdg[wit={K3}]{śaṅkhottho}}\marma}
\pada{ghaṇṭā\app{\lem[wit={C6,V3,J7,N19,V15,Jyo}]{kāhala}
		\rdg[wit={N3,P11}]{kāhāla}
		\rdg[wit={N23}]{kāhla}
		\rdg[wit={Gr3a}]{kalaha}
		\rdg[wit={J10}]{kolāha}}%
	\app{\lem[wit={N3,C6,V3,Jyo},alt={°jās}]{\skp{°}jā\skp{s}}
		\rdg[wit={Gr2,Gr3a}]{jas}
		\rdg[wit={N19,V15}]{kās}
		\rdg[wit={J10}]{las}}s tathā//}\\!} % tatha J10; tataḥ C6 (tathā P11)
\end{tlg}

%\newpage
\begin{tlg}[hp04_039]% = V3_4.82
\tl{
\pada{\app{\lem[wit={N3,C6,V3,J7,Gr3a,V15,Jyo}]{ante}
		\rdg[wit={N19,J10}]{anye}
		\rdg[wit={N23}]{avai}}
	\app{\lem[wit={ceteri}]{tu}
		\rdg[wit={K3}]{ca}}
	kiṅkiṇī% °nī V19
	\app{\lem[wit={N3,N19,V15,J10,Jyo}]{vaṃśa}% vaṃśaṃ
		\rdg[wit={G4,C6,V3,Gr2,Gr3a}]{vṛnda}% vṛṃda -> <<śa>>bdaṃva?! J7, vaṃda P11
		\rdg[wit={J5}]{śabda}}}%
\pada{\app{\lem[wit={N3,C6,V3,Gr2,Gr3a,J10,Jyo}]{vīṇā}
		\rdg[wit={N19,V15}]{nādā}}bhramara% bhumara N23, bhrasara N19
	\app{\lem[wit={N3,C6,N19}]{nisvanāḥ}
		\rdg[wit={V3,J10}]{nisvanā} % nisvānā J10
		\rdg[wit={V15,Jyo}]{niḥsvanāḥ}
		\rdg[wit={J7,V19,C7}]{nisvanaḥ} % niśvanaḥ V19
		\rdg[wit={N23,K3}]{niḥsvanaḥ}}/}\\+} % nissvanaḥ K3
\tl{
\pada{iti \app{\lem[wit={N3,C6,N19,V15,J10,Jyo}]{nānāvidhā}
		\rdg[wit={V3,Gr2,Gr3a}]{nānāvidho}}
	\app{\lem[wit={N3,C6,J10,Jyo}]{nādāḥ}
		\rdg[wit={V3,V15}]{nādā}
		\rdg[wit={J7,Gr3a}]{nādaḥ}
		\rdg[wit={N23}]{nādaṃ}
		\rdg[wit={N19}]{vādāḥ}}}
\pada{\app{\lem[wit={C6,V15,J10,Jyo}]{śrūyante}
		\rdg[wit={ceteri}]{śrūyate}} % śṛyate N3
	\app{\lem[wit={N3,C6,V3,Gr2,Gr3a,Jyo}]{deha}
		\rdg[wit={N19,J10}]{yatra}
		\rdg[wit={V15}]{tatra}}%
	\app{\lem[wit={N3,P11,V3,N19,V15,J10}]{madhyataḥ}
		\rdg[wit={C6,Jyo}]{madhyagāḥ}
		\rdg[wit={Gr2,Gr3a}]{madhyagaḥ}}//}\\!}
\end{tlg}

%\newpage
\begin{tlg}[hp04_040]% = V3_4.83
\tl{
\pada{\app{\lem[wit={ceteri}]{mahati} % so P11; śrūyamāne'pi nāde vai C6,P7
		\rdg[wit={V15}]{mahatī}}
	\app{\lem[wit={ceteri},alt={śrūyamāṇe/-māne}]{śrūyamāṇe}
		\rdg[wit={N23}]{{[ṇya]}yatamāne}}%
	\app{\lem[wit={ceteri}]{'pi}
		\rdg[wit={Gr2}]{ti}}}
\pada{\app{\lem[wit={ceteri}]{megha}
		\rdg[wit={K3}]{bhīka}}bhe%ry% bhīry K3
	\app{\lem[wit={Gr2,N19,J10},alt={ādikadhvanau}]{\skm{ry}ādikadhvanau}
		\rdg[wit={C6,V3,Gr3a,Jyo}]{ādike dhvanau}
		\rdg[wit={V15}]{ādike svane}
		\rdg[wit={N3}]{ādidaṃ dhvanau}}/}\\+}
\tl{
\pada{\app{\lem[wit={N3,C6,V3,N19,V15,J10,Jyo}]{tatra}
		\rdg[wit={Gr2,Gr3a}]{tataḥ}}
	\app{\lem[wit={ceteri},alt={sūkṣmāt}]{sūkṣmā\skp{t}}% sūkṣmā<<t>> J7
		\rdg[wit={N19}]{sūkṣmā}
		\rdg[wit={J10}]{sūkṣmaṃ}}%
	\app{\lem[wit={ceteri},alt={sūkṣmataraṃ}]{\skm{t }sūkṣmataraṃ}% °tara N23
		\rdg[wit={C7}]{sūkṣmatamaṃ}
		\rdg[wit={J10}]{nādam eva}}}
\pada{\app{\lem[wit={ceteri}]{nādam eva}
		\rdg[wit={J7}]{nādam evaṃ}
		\rdg[wit={J10}]{paritopi}}
	\app{\lem[wit={ceteri}]{parāmṛśet} % pasamṛśet N23
		\rdg[wit={V19}]{parāmṛṣet} % pasamṛṣet V19ac
		\rdg[wit={J7}]{samabhyaset}}//}\\!}
\end{tlg}

%\newpage
\begin{tlg}[hp04_041]% = V3_4.84
\tl{
\pada{\app{\lem[wit={ceteri},alt={ghanam}]{ghana\skp{m}}
		\rdg[wit={J10}]{dhvanam}}m utsṛjya
	\app{\lem[wit={N3,C6,V3,N19,V15,J10,Jyo}]{vā sūkṣme}
		\rdg[wit={Gr2,V19,K3}]{vā sūkṣmaṃ}
		\rdg[wit={C7}]{sūkṣmaṃ vā}}} % sūkṣmo P7
\pada{sūkṣmam utsṛjya vā % sūkṣmasūtsṛjya N23
	\app{\lem[wit={N3,C6,N19,V15,Jyo}]{ghane}
		\rdg[wit={V3}]{ghanen}
		\rdg[wit={Gr2,Gr3a}]{ghanam}
		\rdg[wit={J10}]{dhune}}\marma/}\\+}
\tl{
\pada{\app{\lem[resp=emend]{tau tyaktvā madhyame syād vā} % madhyame syātaṃstā J5! G4 broken
		\rdg[wit={N3,P11,V3},alt={madhyama \emph{pro} madhyame}]{tau tyaktvā madhyama syād vā}
		\rdg[wit={C6},alt={madhyamaḥ \emph{pro} madhyame}]{tau tyaktvā madhyamaḥ syād vā}
		\rdg[wit={N19,V15}]{ramamāṇam api kṣipraṃ}
		\rdg[wit={J10,Jyo}]{ramamāṇam api kṣiptaṃ}
		\rdg[wit={Gr2,Gr3a}]{paraṃ tatraiva nikṣipya}}}
\pada{mano % manā? N23, manau V19
	\app{\lem[wit={ceteri}]{nānyatra}
		\rdg[wit={N19,V15,J10}]{nātra pra°}}
	\app{\lem[wit={ceteri}]{cālayet} % c<<ā>>layet J7
		\rdg[wit={J10}]{cālet}
		\rdg[wit={N23}]{vālayet}
		\rdg[wit={V3}]{cālayan}}//}\\!}
\end{tlg}

%\newpage
\begin{tlg}[hp04_042]% = V3_4.085
\tl{
\pada{yatra kutrāpi vā nāde}
\pada{\app{\lem[wit={ceteri}]{lagati}
		\rdg[wit={N23}]{lagavi}
		\rdg[wit={J10}]{galati}}
	\app{\lem[wit={ceteri}]{prathamaṃ}
		\rdg[wit={V19}]{prathame}}
	\app{\lem[wit={ceteri}]{manaḥ}
		\rdg[wit={N23},postwit=\texteng{\getsiglum{C7}\textsubscript{ac}}]{mataḥ}}/}\\+} %
\tl{
\pada{\app{\lem[wit={N3,C6,V15},alt={tatraiva tat}]{tatraiva ta\skp{t}}% ta(l.br.)tat V15
		\rdg[wit={V3,N19}]{tatraivata}
		\rdg[wit={J7,Gr3a,Jyo}]{tatraiva su°}
		\rdg[wit={N23}]{tatraivastu}
		\rdg[wit={J10}]{tatraiva niś°}}%
	\app{\lem[wit={ceteri},alt={sthirī}]{\skm{t }sthirī}
		\rdg[wit={N19}]{śarī}
		\rdg[wit={J10}]{°calo}}%
	\app{\lem[wit={N3,C6,V3,N19,V15,J10}]{bhūtvā}
		\rdg[wit={Jyo}]{bhūya}
		\rdg[wit={K3}]{[bhū]yāt}
		\rdg[wit={Gr2,V19,C7}]{kuryāt}}} % kuryā J7
\pada{tena sārdhaṃ vilīyate//}\label{yatrakutrapi}\\!}
\end{tlg}

\Anm{\getsiglum{N19,V15,J10} have \ref{kasthe}--\ref{sarvacinta} and \ref{sarvacinta2} here, and \getsiglum{V3} \ref{anahata}}

\newpage
%======== passage 4 (makaranda) ======

\begin{tlg}[hp04_043]% = V3_4.72
\tl{
\pada{makarandaṃ % ṃ om. V15,J10
	\app{\lem[wit={N3,C6,V3,Gr2,V19,C7,V15,J10,Jyo},alt={piban}]{piba\skp{n}}
		\rdg[wit={K3}]{pived}
		\rdg[wit={N19}]{piven}}%
	\app{\lem[wit={N3,C6,V3,C7,V15,J10,Jyo},alt={bhṛṅgo}]{\skm{n }bhṛṅgo}
		\rdg[wit={Gr2,V19,K3}]{bhṛṅgī}
		\rdg[wit={N19}]{śṛṃgo}}}
\pada{\app{\lem[wit={N3,V3,V19},alt={gandhān}]{gandhā\skp{n}}
		\rdg[wit={K3,C7}]{gandhā}
		\rdg[wit={C6,J7,N19,V15,J10,Jyo}]{gandhaṃ}
		\rdg[wit={N23}]{gandha}}%
	\app{\lem[wit={N3,C6,V3,J7,Gr3a,V15,Jyo},alt={nāpekṣate}]{\skm{n }nāpekṣate}
		\rdg[wit={N23}]{napekṣate}
		\rdg[wit={N19,J10}]{nopekṣate}}
	\app{\lem[wit={ceteri}]{yathā}
		\rdg[wit={N19}]{'nyathā}}/}\\+}
\tl{
\pada{\app{\lem[wit={N3,C6,V3,V19,K3,N19,V15,J10,Jyo}]{nādāsaktaṃ}
		\rdg[wit={Gr2,C7}]{nādasaktaṃ}} % śaktaṃ N19
		tathā cittaṃ} % yathā C6, but tathā P7
\pada{viṣayā%n % viṣayā J7, °yāṃ N19
	\app{\lem[wit={ceteri},alt={na hi}]{\skm{n }na hi}
		\rdg[wit={V15}]{naiva}
		\rdg[wit={C7}]{api}}
	\app{\lem[wit={N3,C6,V3,N19,Jyo}]{kāṅkṣate}
		\rdg[wit={Gr2,Gr3a,V15,J10}]{kāṅkṣati}}//}
	\label{makaranda1}\\!}
\end{tlg}


\Anm{\getsiglum{Gr2,Gr3a} have \ref{nadakoti} \textit{nādakoṭisahasrāṇi} here}
 

%\newpage
\begin{tlg}[hp04_044]% = V3_4.73
\tl{
\pada{\app{\lem[wit={C6,V3,N19,V15,Jyo}]{baddhaṃ}
		\rdg[wit={J10}]{buddhaṃ}
		\rdg[wit={N3}]{baṃdhaṃ}}
	\app{\lem[wit={N3,C6,Jyo}]{vimukta}
		\rdg[wit={N19}]{vimuktaṃ}
		\rdg[wit={V15,J10}]{viyuktaṃ}
		\rdg[wit={V3}]{timukta}}cāñcalyaṃ}
\pada{nāda\app{\lem[wit={N3,V3,N19,V15,J10,Jyo}]{gandhaka}
		\rdg[wit={C6}]{gandhena}
		\rdg[wit={P11}]{gandhāya}}%
	\app{\lem[wit={N3,C6,V3,V15,Jyo}]{jāraṇāt}
		\rdg[wit={P11,N19,J10}]{jīraṇāt}}/} \lineom{ab}{Gr2,Gr3a}\\+}
\tl{
		%\sgwit{C6,V3,N19,V15,J10,Jyo}
\pada{\app{\lem[wit={N3,C6,J7,Gr3a,N19,V15,J10,Jyo}]{manaḥ}
		\rdg[wit={V3}]{mana}
		\rdg[wit={N23}]{vona}}%
	\app{\lem[wit={C6,N19,J10,Jyo}]{pāradam āpnoti}
		\rdg[wit={V15}]{pārada āpnoti}
		\rdg[wit={V3}]{pāradham āpnoti}
		\rdg[wit={N3}]{pārajam āpnoti}
		\rdg[wit={J7,Gr3a}]{pākam avāpnoti}
		\rdg[wit={N23}]{cāvam avāpnoti}}}
\pada{\app{\lem[wit={ceteri}]{nirālambākhya}
		\rdg[wit={C7}]{nirālambākṣa}}%
	\app{\lem[wit={P11,V3}]{khoṭatāṃ}
		\rdg[wit={N19}]{khoṭatī}
		\rdg[wit={V15}]{khoṭakaṃ}
		\rdg[wit={Jyo}]{khe'ṭanaṃ} % +J11pc
		\rdg[wit={J10}]{khegataṃ}
		\rdg[wit={N3,C6}]{ghoṭatāṃ}
		\rdg[wit={Gr2}]{ghoṭanam}
		\rdg[wit={V19}]{codanaṃ}
		\rdg[wit={C7}]{yodanaṃ}
		\rdg[wit={K3}]{yogadam}}//}\\!}
\end{tlg}

%\newpage
\startaltrecension
\begin{alttlg}[hp04_044_1]% = V3_4.74
\tl{
\pada{\app{\lem[wit={C6},alt={baddhas}]{baddha\skp{s}}
		\rdg[wit={V3,N19,V15}]{baddhaḥ}
		\rdg[wit={Jyo}]{baddhaṃ}
		\rdg[wit={J10}]{baddha}
		\rdg[wit={P11}]{baṃdhaḥ}}%
	\app{\lem[wit={C6},alt={tu nādagandhena}]{\skm{s }tu nādagandhena}
		\rdg[wit={Jyo}]{tu nādabandhena}
		\rdg[wit={V3}]{sunādagandhena}
		\rdg[wit={P11}]{sunādavānpana}
		\rdg[wit={N19}]{sunāde gandhena}
		\rdg[wit={J10}]{sven nādagandhena}
		\rdg[wit={V15}]{suṃdhanādena}}}
\pada{\app{\lem[wit={GrB,N19,V15,J10}]{sadyaḥ}
		\rdg[wit={Jyo}]{manaḥ}}
	\app{\lem[wit={Gr4b,N19,V15,J10,Jyo}]{saṃtyakta}
		\rdg[wit={V3}]{sa tyakta}}%
	\app{\lem[wit={GrB,N19,V15,J10}]{cāpalaḥ}
		\rdg[wit={Jyo}]{cāpalam}}/}\\+}
\tl{
\pada{prayāti
	\app{\lem[resp=emend]{cetaḥsūtendraḥ}
		\rdg[wit={V3}]{cetaḥsuteṃdra}
		\rdg[wit={C6}]{cetaḥsūtrendre}
		\rdg[wit={P11}]{cet sthūlendraḥ}
		\rdg[wit={V15}]{sūtacittendraḥ}
		\rdg[wit={N19}]{sūtaś citteṃdra}
		\rdg[wit={J10}]{svataś caikyaṃ iṃdra}
		\rdg[wit={Jyo}]{sutarāṃ sthairyaṃ}}}
\pada{\app{\lem[wit={P11,C6,N19,V15}]{pakṣachinna}
		\rdg[wit={J10}]{pacchacchinna}
		\rdg[wit={Jyo}]{chinnapakṣaḥ}
		\rdg[wit={V3},alt={\gap}]{\skp{\gap}}}
	\app{\lem[resp=emend,postwit=\texteng{(=\,\getsiglum{M1})}]{iti prathām}
		\rdg[wit={P11}]{dṛti pṛthāṃ}
		\rdg[wit={C6}]{\_\,va patham}
		\rdg[wit={N19}]{iva prabhāṃ}
		\rdg[wit={V15}]{ivāprabhuḥ}
		\rdg[wit={J10}]{iva parvataḥ drumāḥ}
		\rdg[wit={Jyo}]{khago yathā}
		\rdg[wit={V3},alt={\gap}]{\skp{\gap}}}//}
	\sgwit{GrB,N19,V15,J10,Jyo}\\!}
\end{alttlg}
% ithi pradhā M3, iti prathāṃ M1, iva prathāṃ G11, iva dyāmaḥ G7
% cf. Rasendracūḍāmaṇi 16.52-54
%{pañcamo grāsaḥ}
%evaṃ ca pañcamo grāsaḥ pradātavyo'ṣṭamāṃśataḥ /
%sa pātrastho'gnisaṃtapto na gacchati kathañcana // Rcūm_16.52 //
%sa pakṣacchinna ity uktaḥ sa mukto'khiladurguṇaiḥ /
%so'yaṃ niṣevitaḥ sūtastrimāsaṃ rājikāmitaḥ // Rcūm_16.53 //
%viḍaṅgatriphalākṣaudraiḥ khe devaiḥ saha saṅgamam /
%ghrāṇamātreṇa sūtendraḥ sarvaroganikṛntanaḥ // Rcūm_16.54 //
%guṇā ete vinirdiṣṭā rasasya rasavādibhiḥ /
%sakalāste guṇāḥ satyā bhairaveṇa prakīrtitāḥ // Rcūm_16.55 //
% cf. also NWS pakṣaccheda
\endaltrecension

%\newpage
\begin{tlg}[hp04_045]% = V3_4.75
\tl{
\pada{\app{\lem[wit={N3,C6,J7,Gr3a,V15},alt={nādaśravaṇataś cittam}]{nādaśravaṇataś citta\skp{m}} % cittaṃm V19
		\rdg[wit={V3,N19},post=\texteng{(°taḥścitam \getsiglum{V3})}]{nādaḥ śravaṇataś cittam} % °taḥś V3
		\rdg[wit={N23}]{nādaśravaṇaś cittaṃ matam}
		\rdg[wit={J10}]{nādena praṇataṃ cittam}
		\rdg[wit={Jyo}]{nādaśravaṇataḥ kṣipram}}}%
\pada{\app{\lem[wit={N3,C6,V3,Gr2,C7,Jyo},alt={antaraṅga}]{\skm{m }antaraṅga}
		\rdg[wit={J10}]{aṃtaraṃgā}
		\rdg[wit={N19,V15}]{aṃtaraṃgaṃ}
		\rdg[wit={V19}]{aṃtaraṃ sa}
		\rdg[wit={K3}]{sarveṣām}}%
	\app{\lem[wit={N3,C6,V3,V19,N19,V15,J10,Jyo}]{bhujaṅgamaḥ} % ṃ om. V3
		\rdg[wit={J7,C7}]{turaṅgamaḥ}
		\rdg[wit={N23}]{turaṃgavaḥ}
		\rdg[wit={K3}]{antaraṅgamam}}/}\\+}
\tl{
\pada{\app{\lem[wit={P11,V3,Gr2,N19,V15,J10,Jyo}]{vismṛtya}
		\rdg[wit={N3,C6}]{saṃsmṛtya} % +J5
		\rdg[wit={Gr3a}]{viśūnyaṃ}}
	\app{\lem[wit={N3,C6,V3,Gr2,Gr3a,Jyo},alt={sarvam}]{sarva\skp{m}}
		\rdg[wit={N19,V15,J10}]{viśvam}}%
	\app{\lem[wit={N3,Jyo},alt={ekāgraḥ}]{\skm{m }ekāgraḥ}
		\rdg[wit={C6,V3,N23,Gr3a,J10}]{ekāgraṃ}% °gra J5, lost G4
		\rdg[wit={J7}]{ekāgryaṃ}
		\rdg[wit={V15}]{evāgraḥ}
		\rdg[wit={N19}]{evāgra}}}
\pada{kutracin na hi dhāvati//}\\!}
\end{tlg}

\newpage
\begin{tlg}[hp04_046]% = V3_4.76
\tl{
\pada{\app{\lem[wit={N3,C6,J7,Gr3a,N19,V15,J10,Jyo}]{manomatta}
		\rdg[wit={N23}]{manomantra}
		\rdg[wit={V3}]{manonmatta}}gajendrasya} % maje° N19; °āsya V15
\pada{\app{\lem[wit={ceteri}]{viṣayodyāna}
		\rdg[wit={V3},alt={°dhāma}]{viṣayodhāma}}%
	\app{\lem[wit={ceteri}]{cāriṇaḥ}
		\rdg[wit={N23}]{vāriṇaṃ}}/}\\+}
\tl{
\pada{\app{\lem[wit={N3,V3,Gr3a}]{niyāmana}
		\rdg[wit={V15}]{niyāmane}
		\rdg[wit={J10}]{nīyamānaḥ}
		\rdg[wit={C6}]{niyamena}
		\rdg[wit={J7}]{niryāmana}
		\rdg[wit={N19}]{niryāsane}
		\rdg[wit={N23}]{niyamitra}
		\rdg[wit={Jyo}]{samartho'yaṃ}}%
	\app{\lem[wit={ceteri}]{samartho'yaṃ}
		\rdg[wit={Jyo}]{niyamane}}}
\pada{\app{\lem[wit={N3,C6,V3,Gr2,Gr3a}]{ninādo}
		\rdg[wit={N19,V15,J10,Jyo}]{nināda}}
	\app{\lem[wit={C6,V3,Gr2,V15,J10,Jyo}]{niśitāṅkuśaḥ} % °kuśa V3, °kuśaṃ J10
		\rdg[wit={N19}]{niśatāṅkuḥ}
		\rdg[wit={Gr3a}]{niścayāṅkuśaḥ}
		\rdg[wit={N3}]{niyatāṃkuśaḥ}}//}\\!}
\end{tlg}

%\newpage
\begin{tlg}[hp04_047]% = V3_4.78
\tl{
\pada{\app{\lem[wit={ceteri}]{antaraṅga}
		\rdg[wit={V19,K3,N19}]{aṃtaraṃgaṃ}
		\rdg[wit={J10}]{nādoṃtaraṃ}}%
	\app{\lem[wit={C6,V3}]{\skp{°}sya javino}
		\rdg[wit={N3}]{°sya javinaḥ}
		\rdg[wit={Jyo}]{°sya yamino}
		\rdg[wit={P11}]{ca mano}
		\rdg[wit={Gr2,Gr3a,N19,V15}]{turaṅgasya} % truraṃga N19
		\rdg[wit={J10}]{tu saṃgamya}}}
\pada{\app{\lem[wit={GrB,N19,V15,J10,Jyo}]{vājinaḥ}
		\rdg[wit={N3}]{kariṇaḥ}
		\rdg[wit={Gr2,Gr3a}]{vijñānaṃ}} % °na N23
	\app{\lem[wit={N3,Jyo}]{parighāyate}% +C8,P7
		\rdg[wit={P11}]{parighātayaḥ}
		\rdg[wit={C6}]{pariṣāyate}
		\rdg[wit={V3,Gr2,N19,J10}]{paridhāyate}
		\rdg[wit={V15}]{paridhāvataḥ}
		\rdg[wit={K3,C7}]{parimīyate}
		\rdg[wit={V19}]{parimeyate}}/}\\+}
\tl{
\pada{\app{\lem[wit={N3,GrB,K3,N19,V15,Jyo}]{nādopāstir ato}%
		\rdg[wit={Gr2}]{nādopāstivato}
		\rdg[wit={C7}]{nādopāstimato}
		\rdg[wit={V19}]{nādopāstiratir}
		\rdg[wit={J10},alt={\om}]{\skp{\om}}} nitya}%m}
\pada{\app{\lem[wit={N3,P11,V3,V19,C7},alt={avadhāryāpi}]{\skm{m }avadhāryāpi}
		\rdg[wit={J7}]{avadhāyāpi}
		\rdg[wit={N23}]{anadhāyāpi}
		\rdg[wit={C6}]{avadhāryo pi}
		\rdg[wit={V15,Jyo}]{avadhāryā hi}
		\rdg[wit={K3}]{avidhāryaṃ hi}
		\rdg[wit={N19}]{avagamyaṃ hi}
		\rdg[wit={J10},alt={\om}]{\skp{\om}}}
	\app{\lem[wit={GrB,Jyo}]{yoginā}
		\rdg[wit={N3,N19,V15}]{yogināṃ}
		\rdg[wit={Gr2,Gr3a}]{yoginaḥ}
		\rdg[wit={J10},alt={\om}]{\skp{\om}}}//}\myfn{In \getsiglum{Gr2,Gr3a} the second hemistich only is written here and the whole verse and the next one (\ref{makaranda6}) are found after \ref{anahata}. The text of the hemistich is not the same in the two instances. In the apparatus the readings of the first instance only are reported. The last Pāda of the second instance reads \textit{avagamyā hi yogibhiḥ}.}
		\lineom{cd}{J10}\label{IV95Vu}\\!}
\end{tlg}
		% K3 omits the 2nd hemistich in the second instance!



%\newpage
\startaltrecension
\begin{alttlg}[hp04_047_1]% = V3_4.77
\tl{
\pada{\app{\lem[wit={P11,Gr2,K3,C7,V15,Jyo}]{nādo'ntaraṅga}% °ttaraṅga P11
		\rdg[wit={C6,V3}]{nādotaraṅga}
		\rdg[wit={N19}]{nādāṃtaraṅga}
		\rdg[wit={V19}]{nādaturaṃga}
		\rdg[wit={J10},alt={\om}]{\skp{\om}}}%
	\app{\lem[wit={ceteri}]{sāraṅga}
		\rdg[wit={C7}]{mātaṃga}
		\rdg[wit={J10},alt={\om}]{\skp{\om}}}}% sāraṅgaṃ V15
\pada{\app{\lem[wit={ceteri}]{bandhane}
		\rdg[wit={N23}]{baṃdhāna}
		\rdg[wit={V3}]{baṃdhana}
		\rdg[wit={J10},alt={\om}]{\skp{\om}}}
	\app{\lem[wit={ceteri}]{vāgurāyate}
		\rdg[wit={N23}]{yāgurāyate}
		\rdg[wit={J10},alt={\om}]{\skp{\om}}}/}\\+}
\tl{
\pada{antaraṅga% °aṃ? V15
	\app{\lem[wit={V15,Jyo}]{kuraṅgasya}
		\rdg[wit={GrB,Gr2,V19,N19,J10}]{turaṅgasya}
		\rdg[wit={K3,C7}]{turaṅgasyā°}}}
\pada{\marma\app{\lem[wit={Jyo}]{vadhe vyādhāyate}
		\rdg[wit={V15}]{nādo vyādhāyate}
		\rdg[wit={V3}]{rodhe vādhāyate} %  parighāyate P7, rodhye pi parivāryate C8
		\rdg[wit={P11}]{rodhe vādyāyate}
		\rdg[wit={C6}]{rodhe pi pariṣāyate}
		\rdg[wit={N19}]{rodhe vā gāyate}
		\rdg[wit={J10}]{rogo vā gīyate}
		\rdg[wit={N23}]{bāhye pi līyate}
		\rdg[wit={J7}]{bodho pi līyate}
		\rdg[wit={K3}]{°varodhe līyate}
		\rdg[wit={C7}]{°vabodhe līyate}
		\rdg[wit={V19},alt={\gap}]{\skp{\gap}}}%
	\app{\lem[wit={ceteri}]{'pi ca}
		\rdg[wit={P11}]{ti ca}
		\rdg[wit={V19},alt={\gap}]{\skp{\gap}}}//}%
		\myfn{Transposed with the previous verse in \getsiglum{GrB};
		\getsiglum{J10} merges the two into one:
		\textit{nādo'ntaraṃ tu saṃgamya vājinaḥ paridhāyate |
		antaraṅgaturaṃgasya rogo vā gīyate pi ca ||}}
		\label{makaranda6}\NotIn{Gr1}\\!}
\end{alttlg}
\endaltrecension

%\newpage
%
%% 4.064_1 (cf. 4.65cd)
%\pada{nādopāsti\app{\lem[wit={K3}]{r ato}
%		\rdg[wit={V19}]{ratir}
%		\rdg[wit={Gr2}]{vato}
%		\rdg[wit={C7}]{mato}} nityam}
%\pada{\app{\lem[wit={V19,C7}]{avadhāryāpi}
%		\rdg[wit={J7}]{avadhāyāpi}
%		\rdg[wit={N23}]{anadhāyāpi}
%		\rdg[wit={K3}]{avidhāryaṃ hi}} yoginaḥ/}\label{nadopasti}
%		\sgwit{Gr2,Gr3a} \anm{≈ \ref{IV95Vu}cd}\\!}

%\newpage
\begin{tlg}[hp04_048]% = V3_4.166; Upagīti
\tl{
\pada{\app{\lem[wit={N3,P11,V3,Jyo},post=\texteng{(°ādī° \getsiglum{N3})}]{ghaṇṭādināda}
%	\rdg[wit={N3}]{ghaṃṭādīnāda}
	\rdg[wit={C6,Gr2,Gr3a}]{ghaṇṭānināda}}% ghaṃṭāki? V19
\app{\lem[resp=emend,postwit=\texteng{(śakti \getsiglum{J5})}]{sakti}
	\rdg[wit={P11,V3,Jyo}]{sakta} % sadaṃkatā P11
	\rdg[wit={N3}]{śaktaś ca }
	\rdg[wit={Gr2,Gr3a}]{saktasya}% sukta N23
	\rdg[wit={C6}]{kuliśa}}%
\app{\lem[wit={P11,Jyo}]{stabdhāntaḥ}% stabdhyaṃtaḥ P11
	\rdg[wit={N3}]{stavyāṃtaḥ}
	\rdg[wit={V3}]{statravadhātaḥ}
	\rdg[wit={N23}]{śabdāntaḥ}
	\rdg[wit={J7}]{śabdataḥ}
	\rdg[wit={Gr3a}]{śuddhāntaḥ}%  °syaṃ? V19
	\rdg[wit={C6}]{pradhvānta}}%
\app{\lem[wit={P11,V3,Jyo}]{karaṇahariṇasya}% karāṇa P11
	\rdg[wit={N3}]{karaṇaṃ hariṇasya}
	\rdg[wit={J7,Gr3a}]{karaṇasya ca}
	\rdg[wit={N23}]{karaṇasya na}}/} \sgwit{Gr1,GrB,Gr2,Gr3a,Jyo}\\+}
\tl{
\pada{praharaṇa%m
\app{\lem[wit={GrB,Jyo},alt={atisukaraṃ}]{\skm{m }atisukaraṃ}
	\rdg[wit={N3}]{atisukasteraṃ}}
\app{\lem[wit={N3,Gr4b,Jyo}]{syāc chara}% so P7; syāt sadṛśaṃdhātā C6
	\rdg[wit={V3}]{syāra}}%
\app{\lem[wit={N3,GrB}]{saṃdhātā}
	\rdg[wit={Jyo}]{saṃdhāna}} pravīṇaś cet//}
	\sgwit{Gr1,GrB,Jyo}%
	\myfn{In \getsiglum{V3} this verse is found after 4.48.}\\!} % REF
\end{tlg}
% P11
% ghaṃṭādinādasadaṃkatāstabdhāṃtaḥkaraṇahariṇasya/
% praharaṇam atisukaraṃ syāc charasaṃdhātā pravīṇaś cet//
% M1:
% ghaṃṭāninādasaktastabdhāṃtaḥkaraṇahariṇasya/
% praharaṇe sukaraṃ syā(c) [ch](ara)[sa]ṃdhānapravīṇaś cet// 4.103//
% P7:
% ghaṃṭādinādakuliśapradhvāṃtaharisasya ca/
% praharaṇam atisūkaraṃ syā<c> charasaṃghātā praviṇaś cet// 106/107 //
% C6:
% ghaṇṭāninādakuliśapradhvāṃtahariṇasya ca/
% praharaṇam atisukaraṃ syāt sadṛśaṃdhātā pravīnaś cet//
% N12:
% ghaṇṭādinākalādhvāṃtaḥkaraṇahariṇyaḥ syāt/
% praharaṇam atisukaraṃ syāc charasaṃdhānā pravīṇaś ca/

\newpage
\begin{tlg}[hp04_049]% = V3_4.86
\tl{\texteng{[Alt1]}
\pada{anāhatadhvaner antar} % anāhataḥ V17
\pada{jñeyaṃ yat sūkṣmasūkṣmakam/}\\+} % sūkṣmya° N19
\tl{
\pada{manas tatra layaṃ yāti}
\pada{tad viṣṇoḥ paramaṃ padam//} % viṣṇo
\label{anahata2}\sgwit{N19,V15,J10}\\!}
\end{tlg}

%\newpage
\startaltnormal
\begin{alttlg}[hp04_049_1]%
\tl{\texteng{[Alt2]}
\pada{\app{\lem[wit={N3,P11,V3,Gr2,Gr3a,Jyo},post=\texteng{(sabdasya \getsiglum{V3,N23})}]{anāhatasya śabdasya}
%		\rdg[wit={V3,N23}]{anāhatasya sabdasya}
		\rdg[wit={C6}]{anāhatas tu yaḥ śabdas}}}
\pada{\app{\lem[wit={J5,C6,Gr2,Gr3a}]{tasya śabdasya yo dhvaniḥ}
		\rdg[wit={N3}]{tasya śabdasya ca dhvaniḥ}
		\rdg[wit={V3}]{śabdasyāṃtargato dhvaniḥ}
		\rdg[wit={P11}]{śabdasyāṃganabho dhvaniḥ}
		\rdg[wit={Jyo}]{dhvanir ya upalabhyate}}
		/}\\+}
\tl{
\pada{\app{\lem[wit={N3,Gr4b,Gr3a,Jyo},postwit=\texteng{\getsiglum{N23}\textsubscript{pc}},alt={dhvaner}]{dhvane\skp{r}}
		\rdg[wit={J5,G4,V3,Gr2}]{dhvanir}}r
		antargataṃ % °gata N23
	\app{\lem[wit={N3,Jyo}]{jñeyaṃ}
		\rdg[wit={P11,V3}]{geyaṃ}
		\rdg[wit={G4,N23,K3}]{jyotir}
		\rdg[wit={C6,J7,V19,C7}]{jyoti} % om. C6, jyotī P7
		\rdg[wit={J5},alt={\om}]{\skp{\om}}}} % =P11
\pada{\app{\lem[wit={Jyo},alt={jñeyasyāntar}]{jñeyasyānta\skp{r}}
		\rdg[wit={N3}]{yasyāṃtvaṃtar}
		\rdg[wit={P11,V3}]{geyasyāntar}
		\rdg[wit={J5,Gr2,K3}]{jyotirantar}% yotir° J5
		\rdg[wit={C6,V19,C7}]{jyoterantar}}rgataṃ manaḥ/}\\+} % mana V3
\tl{
\pada{\app{\lem[wit={N3,J5,P11,V3,J7}]{tan mano vilayaṃ}% lost G4
		\rdg[wit={C6,N23,V19,C7}]{yan mano vilayaṃ}
		\rdg[wit={K3}]{yan mano gomayaṃ}
		\rdg[wit={Jyo}]{manas tatra layaṃ}}
	\app{\lem[wit={GrB,N23,Gr3a}]{yāti}
	\rdg[wit={N3,J7}]{yāṃti}}}
\pada{tad viṣṇoḥ paramaṃ padam//} % viṣṇo C6,N3
\sgwit{Gr1,GrB,Gr2,Gr3a,Jyo}\label{anahata}\\!}
\end{alttlg}
\endaltnormal


%\newpage
\begin{tlg}[hp04_050]% = V3_4.87
\tl{
\pada{\app{\lem[wit={ceteri},alt={tāvad ā°}]{tāvad ā}% tavad N23
		\rdg[wit={J10}]{bhāvanā°}}kāśasaṃkalpo} % kāsa J10; saḥkalpo J7
\pada{\app{\lem[wit={N3,GrB,Gr2,V15,J10,Jyo}]{yāvac chabdaḥ}% ḥ om. P7
		\rdg[wit={V19,C7}]{yāvad bandhaḥ}
		\rdg[wit={K3}]{yāvad baddhaḥ}
		\rdg[wit={N19}]{yāvad vādhaḥ}} pravartate/}\\+}
\tl{
\pada{niḥśabdaṃ  % niśabdaṃ V19,N3; °bdāṃ V17
	\app{\lem[wit={ceteri}]{tat paraṃ}
		\rdg[wit={N23}]{paramaṃ}} brahma}
\pada{\app{\lem[wit={ceteri}]{paramātmā}
		\rdg[wit={Jyo}]{paramātme°}} % paramatmā V19
	\app{\lem[wit={N3,C6,V3,J7}]{samīryate} % so P11, yam īryate C6; for workshop!
		\rdg[wit={P11,N23,Gr3a}]{samīyate} % to be treated equally, be placed on a level with (Apte)
		\rdg[wit={N19,V15,J10}]{°numīyate}
		\rdg[wit={Jyo}]{°ti gīyate}}//}\\!}
\end{tlg}

%\newpage
\begin{tlg}[hp04_051]% = V3_4.88
\tl{
\pada{\app{\lem[wit={N3,C6,Gr2,Gr3a,Jyo},alt={yat}]{ya\skp{t}}
		\rdg[wit={V3},alt={\om}]{\skp{\om}}}t kiṃci%n
	\app{\lem[wit={N3,C6,V3,Jyo},alt={nāda}]{\skm{n }nāda}
		\rdg[wit={Gr2,Gr3a}]{nāma}}rūpeṇa}
\pada{śrūyate śaktir eva sā/}\\+}
\tl{
\pada{\app{\lem[wit={N3,C6,Gr2,K3,C7}]{yas tacchrotā}
		\rdg[wit={V19}]{yat ta[cch]roto}
		\rdg[wit={V3}]{yac chrotā ca}
		\rdg[wit={Jyo}]{yas tattvānto}} nirākāraḥ} % °kāra V3
\pada{sa eva parameśvaraḥ//} \NotIn{N19,V15,J10}\\!} % ḥ om. V3; e<<va>>? N23
\end{tlg}

%\newpage
\begin{tlg}[hp04_052]% = V3_4.51 % Upagīti
\tl{
\pada{śravaṇa\app{\lem[wit={N3,C6,V3,N19,V15}]{mukha}
		\rdg[wit={Gr2,Gr3a,J10,Jyo}]{puṭa}}%
	\app{\lem[wit={N3,C6,V3,Gr2,Gr3a,N19,V15}]{nayana} % cayana N23
		\rdg[wit={J10,Jyo}]{nayanayugala}}%
	\app{\lem[wit={ceteri}]{nāsā} % nāśā V19,N19,V3
		\rdg[wit={Jyo}]{ghrāṇa}}%
	\app{\lem[resp=emend,postwit=\texteng{(cf.\,K1,P6,M3)}]{nirodhanaṃ caiva kartavyam}
		\rdg[wit={C6,N19,V15}]{nirodhanaṃ naiva kartavyaṃ} % nirodhamaṃ N19
		\rdg[wit={N3}]{nirodhaṃ naiva kartavyaṃ}
		\rdg[wit={V3}]{nirodhanenaiva kartavyaṃ}
		\rdg[wit={Gr2,K3,C7}]{mukhapuṭasaṃrodhanaṃ kāryam}
		\rdg[wit={V19}]{mukhapuṭarodhane kāryaṃ}
		\rdg[wit={J10}]{mukharodhanam eva kartavyaṃ}
		\rdg[wit={Jyo}]{mukhānāṃ nirodhanaṃ kāryam}}/}\\+}
\tl{
\pada{\app{\lem[wit={N3,C6,Gr3a,N19,V15,J10,Jyo}]{śuddha} % suddha
		\rdg[wit={Gr2}]{śrīśuddha}
		\rdg[wit={V3},alt={\om}]{\skp{\om}}}%
	\app{\lem[wit={N3,C6,V3,J7,Gr3a,N19,V15,J10,Jyo}]{suṣumṇā}
		\rdg[wit={N23}]{suṣumū}}% suṣumūṇau N23 by haplography?
	\app{\lem[wit={J7,Gr3a,Jyo}]{saraṇau}% śaraṇau V19,J10pc
		\rdg[wit={N19,V15,J10}]{śaraṇe}
		\rdg[wit={N3}]{tsaraṇaḥ}
		\rdg[wit={C6}]{tmaśaraṇaiḥ}
		\rdg[wit={V3}]{maraṇai}
		\rdg[wit={N23}]{ṇau}}
	\app{\lem[wit={N3,Gr2,Gr3a,J10,Jyo}]{sphuṭam amalaḥ śrūyate}
		\rdg[wit={V3}]{sphuṭam amalaṃ śrūyate}
		\rdg[wit={C6}]{sphurad amalaḥ śrūyate}
		\rdg[wit={V15}]{vimalaḥ saṃśrūyate}
		\rdg[wit={N19}]{vimalaḥ śrūyate}}
	nādaḥ//}\label{sravanaputa}\\+}  % nāda V3
\tl{
	\anm{\getsiglum{Gr1,Gr2,Gr3a} have this verse here, while the other mss immediately after \ref{sapadakoti}}\\!}
\end{tlg}

\newpage
\startaltrecension
\begin{alttlg}[hp04_052_1]% = V3_4.89
\tl{
\pada{\app{\lem[wit={C6,V3,V15,J10}]{nādaḥ}
		\rdg[wit={P11,N19}]{nāda}} śaktir iti
	\app{\lem[wit={V15,J10}]{khyāto} % khyato J10ac
		\rdg[wit={N19}]{kṣāto}
		\rdg[wit={C6}]{jñeyā}
		\rdg[wit={P11}]{jñeyaṃ}
		\rdg[wit={V3}]{jñeya}}}
\pada{\app{\lem[wit={P11,V3,N19,V15}]{nādajñānaṃ}
		\rdg[wit={C6,J10}]{nādo jñānaṃ}} sadāśivaḥ/}\\+} % śiva V3
\tl{
\pada{\app{\lem[wit={N19},post=\texteng{neṣṭe tat},alt={nādajñāne ca naṣṭe tad}]{nādajñāne ca naṣṭe ta\skp{d}}
		\rdg[wit={V15}]{nādajñāne vinaṣṭe ca tad}
		\rdg[wit={J10}]{nādajñānena naṣṭena}
		\rdg[wit={P11}]{jñeye jñāne vilīnāṃta}
		\rdg[wit={V3}]{jñeye jñāne vilineṃta}
		\rdg[wit={C6}]{jñeyo jñāne vilīne tu}}}% jñeyo jñāne vilīyeta P7
\pada{\app{\lem[wit={V15},alt={unmany}]{\skm{d }unma\skp{ny}}% +J11
		\rdg[wit={N19}]{unmadhy}
		\rdg[wit={J10}]{hy unmany}
		\rdg[wit={GrB}]{sonmany}}%
	\app{\lem[wit={C6,J10},alt={evāvaśiṣyate}]{\skm{ny }evāvaśiṣyate}
		\rdg[wit={N19}]{edhāvaśiṣyate}
		\rdg[wit={V3}]{avāvaśiṣyate}
		\rdg[wit={P11}]{enāvaśiṣyati}
		\rdg[wit={V15}]{eva śiṣyate}}//}
\sgwit{GrB,N19,V15,J10}\\!}
\end{alttlg}

\begin{alttlg}[hp04_052_2]% = V3_4.90
\tl{
\pada{nādo yāvan manas tāva}%n}
\pada{\app{\lem[wit={P11,V3,N19,J10},alt={nādānte tu}]{\skm{n }nādānte tu}
	\rdg[wit={V15}]{nādānte ca}
	\rdg[wit={C6}]{tādātīte}} manonmanī/}\\+}
\tl{
\pada{saśabdaṃ kathitaṃ vyoma}
\pada{niḥśabdaṃ brahma kathyate//}
\sgwit{GrB,N19,V15,J10}\\!}
\end{alttlg}

%\newpage
\begin{alttlg}[hp04_052_3]% = V3_4.91
\tl{
\pada{sadā nādānusaṃdhānāt}
\pada{\app{\lem[wit={GrB,N19,V15,J10}]{saṃkṣīṇe}
		\rdg[wit={Jyo}]{kṣīyante}}
	\app{\lem[wit={Gr4b}]{vāsanācaye}% ## +P7
		\rdg[wit={J10}]{vāsanodaye}
		\rdg[wit={V3}]{vāsanāvayo}
		\rdg[wit={N19}]{vāsanākṣaye}
		\rdg[wit={V15}]{vāsanākṣaṇe}
		\rdg[wit={Jyo}]{pāpasaṃcayāḥ}}/}\\+}
\tl{
\pada{nirañjane
	\app{\lem[wit={V15,J10}]{ca līyete}
		\rdg[wit={N19}]{ca līyeta}
		\rdg[wit={C6}]{vilīyeta}
		\rdg[wit={P11,V3}]{vilīyaṃte}
		\rdg[wit={Jyo}]{vilīyete}}}
\pada{\app{\lem[wit={V15,Jyo}]{niścitaṃ cittamārutau}
		\rdg[wit={N19}]{niścitta manamārutau}
		\rdg[wit={J10}]{niścitau manamārutau}
		\rdg[wit={P11,V3}]{niścitaṃ māruto manaḥ} % niścita māruto mana V3
		\rdg[wit={C6}]{marutā niścitaṃ manaḥ}}//}
\sgwit{GrB,N19,V15,J10,Jyo}\label{sadanada}\\!}
\end{alttlg}


%\newpage
\begin{alttlg}[hp04_052_4]% = C8_84/85
\tl{
\pada{nādakoṭisahasrāṇi} % sahasrāṇī J7, sahasrāni N23
\pada{bindukoṭiśatāni ca/}\\+} % veda C6; saṭāni N23ac, satāni N23pc
\tl{
\pada{\app{\lem[wit={ceteri}]{sarve}
		\rdg[wit={N23}]{sarvaṃ}} tatra layaṃ
	\app{\lem[wit={ceteri}]{yānti}
		\rdg[wit={C6,V19}]{yāti}}}
\pada{yatra
	\app{\lem[wit={ceteri}]{devo}
		\rdg[wit={V3,N19}]{deva}}
	\app{\lem[wit={ceteri}]{nirañjanaḥ}
		\rdg[wit={V3}]{nirañjanam}}//}
	\label{nadakoti} \sgwit{GrB,N19,V15,J10}\\+}
\tl{
	\anm{\getsiglum{Gr2,Gr3a} have this verse immediately after \ref{makaranda1}}\\!}
\end{alttlg}

%\newpage


\begin{postmula}[hp04_052_4]
\grau{\app{\lem[wit={P11,J10,Jyo}]{iti nādānusaṃdhānam}
\rdg[wit={N19}]{iti nādānusaṃdhānāṃ yathā vṛddho veti}
\rdg[wit={V15},post=\texteng{(metrical!)}]{iti nādānusaṃdhānaṃ yathā vṛddhaiḥ prabhāṣitaṃ}
\rdg[wit={C6,V3}]{iti nādānusaṃdhānavidhiḥ}}// \sgwit{GrB,N19,V15,J10,Jyo}}
\end{postmula}

\Anm{\getsiglum{V3} has Kālajñāna, Videhamuktikathana, and Kālavañcana sections here}

%%%%%%%%%%%%%%%%%%%%%%%
%\newpage
\begin{alttlg}[hp04_052_5]% = V3_4.159, cf. 4.089
\tl{
\pada{sarve \app{\lem[wit={C6,V3}]{haṭhalayopāyā}
	\rdg[wit={P11}]{haṭhalayā bhāvyā}}}
\pada{rājayoga\app{\lem[wit={P11}]{padāvadhi}
	\rdg[wit={C6}]{padāvadhiḥ}
	\rdg[wit={V3}]{padāvadhiṃ}}/}\\+}
\tl{
\pada{rājayogapadaṃ prāpya}
\pada{jāyate\app{\lem[wit={Gr4b}]{'sau}
	\rdg[wit={V3}]{so}} nirañjanaḥ//} \sgwit{GrB} \anm{cf. \ref{kalavancaka}}\\!}
\end{alttlg}
\endaltrecension



\Anm{\getsiglum{N19,V15,J10} have \ref{kalavancaka} \textit{sarve layahaṭhābhyāsāḥ} and \ref{astuva}ff. \textit{astu vā māstu vā} here}

\newpage

\begin{tlg}[hp04_053]% = V3_4.173
\tl{
\pada{\app{\lem[wit={N3,GrB,Gr2},alt={kāṣṭa/kāṣṭha}]{kāṣṭha}
		\rdg[wit={Gr3a}]{koṣṭha}}% śāṣtadoṣi G4; ab lost J5; none N24
	\app{\lem[wit={Gr3a}]{goṣṭhī}
		\rdg[wit={N3,J7}]{goṣṭhi}
		\rdg[wit={V3,N23}]{goṣṭha}
		\rdg[wit={P11}]{mathnī}
		\rdg[wit={C6}]{mathnā}}%
	\app{\lem[wit={V3}]{prapañcena}
		\rdg[wit={N3}]{prapaṃce}
		\rdg[wit={Gr2,Gr3a}]{prasaṅgena}% +G4; none J5,N24
		\rdg[wit={P11}]{pravacane}
		\rdg[wit={C6}]{pravartaṃ}}\marma}
\pada{\app{\lem[wit={N3,GrB}]{kiṃ sakhe śrūyatām idam}
		\rdg[wit={J7,Gr3a}]{nādam antargataṃ śṛṇu} % śruṇu V19
		\rdg[wit={N23}]{nāgadaṃtaṃmatargataṃ sṛṇu}}/}\\+}
\tl{
\pada{purā matsyendra\app{\lem[wit={N3,GrB},alt={bodhārtham}]{bodhārtha\skp{m}}
		\rdg[wit={Gr2,Gr3a}]{bodhāya}}}%
\pada{\app{\lem[wit={N3,Gr4b,J7,Gr3a},alt={ādināthoditaṃ}]{\skm{m }ādināthoditaṃ}
		\rdg[wit={N23}]{ādināthotigaditaṃ}
		\rdg[wit={V3}]{ānināthodinaṃ}}
		vacaḥ//} \NotIn{N19,V15,J10,Jyo}\\!} % vaca N3,V3
\end{tlg}

% C6 4.110: kaṣṭamathnā pravartaṃ (1 syll. ami.) kiṃ sakhe śrūyatām idam/
% purā matsyendrabodhārtham ādināthoditaṃ vacaḥ//
% P7: kāṣṭamathnvāṃ pravartante kiṃ sakhe śrūyatām idam/
% purā matsyendrabodhārtham ādināthoditaṃ vacaḥ//
% C8: kāṣṭamanāpravartaṃte ... bodhāya
% P11: kāṣṭhamathnī pravacane ... bodhārtham



\begin{tlg}[hp04_054]% = V3_4.174
\tl{
\pada{yāvan naiva \app{\lem[wit={ceteri}]{praviśati}
		\rdg[wit={N23}]{\_viśati}}
	\app{\lem[wit={ceteri},alt={caran}]{cara\skp{n}}
		\rdg[wit={J7}]{calan}
		\rdg[wit={N23}]{palan}
		\rdg[wit={N3}]{care}
		\rdg[wit={V3},alt={\om}]{\skp{\om}}}n māruto % mādgato N23, mārutaṃ N3
	\app{\lem[wit={ceteri}]{madhya}
		\rdg[wit={V15}]{mādhya}}%
	\app{\lem[wit={N3,C6,J7,V19,K3,N19,J10,Jyo}]{mārge}
		\rdg[wit={P11,N23}]{mārgo}
		\rdg[wit={C7,V15}]{mārgaṃ}
		\rdg[wit={V3}]{mārgā}}}\\+}
\tl{
\pada{yāva%d
	\app{\lem[wit={ceteri},alt={bindur}]{\skm{d }bindu\skp{r}} % vāyuḥ V19ac
		\rdg[wit={V15}]{bandho}
		\rdg[wit={N19}]{bandhaṃ}}r na bhavati
	\app{\lem[wit={ceteri}]{dṛḍhaḥ} % dṛḍha V19
		\rdg[wit={N3,P11}]{dṛḍhaṃ}} % sthiraḥ J5
	prāṇa\app{\lem[wit={N3,GrB,J7,J10,Jyo}]{vāta}
		\rdg[wit={N23,Gr3a,V15}]{vātaḥ}
		\rdg[wit={N19}]{vātaṃ}}%
	\app{\lem[wit={C6,Gr2}]{prabaddhaḥ}% prabandhāt M1
		\rdg[wit={P11,V15}]{prabandhaḥ}% ḥ om. P11
		\rdg[wit={Gr3a,J10}]{prabuddhaḥ}
		\rdg[wit={N3}]{prabodhaḥ}
		\rdg[wit={V3}]{prabodhakaḥ}
		\rdg[wit={N19}]{na bandhanaḥ}
		\rdg[wit={Jyo}]{prabandhāt}}/}\\+}
\tl{
\pada{yāva%
	\app{\lem[wit={N3,Gr4b,N19,V15},alt={vyomnā}]{\skm{d }vyomnā}
		\rdg[wit={J7,Gr3a,J10}]{vyomnaḥ}% dyo° J7; rather °nmaḥ J10
		\rdg[wit={N23}]{\_mnaḥ}
		\rdg[wit={V3}]{byomna}
		\rdg[wit={Jyo}]{dhyāne}}
	sahaja\app{\lem[wit={ceteri}]{sadṛśaṃ}
		\rdg[wit={N23}]{saṃśaṃ}} jāyate naiva
	\app{\lem[wit={ceteri}]{tattvaṃ}
		\rdg[wit={V3,V15,J10}]{cittaṃ}}}\\+} % ?Jyo
\tl{
\pada{tāva\app{\lem[wit={ceteri},alt={sarvaṃ}]{\skm{t }sarvaṃ}
		\rdg[wit={V3,J10,Jyo}]{jñānaṃ}} vadati % vadaṃti V3
	\app{\lem[wit={N3,C6,J7,C7,N19,V15,J10}]{yad idaṃ}
		\rdg[wit={V19,K3,Jyo}]{tad idaṃ}% tada M1
		\rdg[wit={P11,N23}]{yadi}% post=\texteng{(daṃ om. by haplography)}
		\rdg[wit={V3}]{satataṃ}}
	\app{\lem[wit={ceteri}]{dambha}
		\rdg[wit={N19}]{ḍaṃbha}}mithyāpralāpaḥ//} % °lāpa N19, prallāpaḥ V3, pralābhaḥ C6
	\anm{after 4.5 \getsiglum{N19,V15,J10}}\label{yavan}\\!}
\end{tlg}

%========== Passage B ==========
%\newpage

\Anm{The following verses \ref{jnatva}--\ref{tatraika} are found immediately after 4.10 in \getsiglum{N19,V15,J10,Jyo}}

\begin{tlg}[hp04_055]% = V3_4.175
\tl{
\pada{\app{\lem[wit={ceteri}]{jñātvā}
		\rdg[wit={V15}]{suṣu°}}
	\app{\lem[wit={N3,J10,Jyo}]{suṣumṇāsadbhedaṃ}
		\rdg[wit={GrB}]{suṣumṇāsaṃbhedaṃ}
		\rdg[wit={N19}]{suṣumṇāṃ saśvedaṃ}
		\rdg[wit={J7,V19,C7}]{suṣumṇābhedaṃ hi}% sukhu° V19
		\rdg[wit={N23}]{suṣu<<m>>ṇāṃmedehi}
		\rdg[wit={K3}]{suṣumṇābhedaṃ ca}
		\rdg[wit={V15}]{°mnāṃtagataṃ mārgaṃ}}}
\pada{\app{\lem[wit={ceteri}]{kṛtvā vāyuṃ}% kṛtvān V3; vāyu V19, pāyu N19
		\rdg[wit={V15}]{vāyuṃ kṛtvā}
		\rdg[wit={K3}]{jñātvā vāyuṃ}} ca
	\app{\lem[wit={ceteri}]{madhyagam}
		\rdg[wit={P11}]{madhyamaḥ}}/}\\+}
\tl{
\pada{\app{\lem[resp=emend]{nītvā tam aindave sthāne}
		\rdg[wit={Gr3a}]{nītvā tām anavasthāne}
		\rdg[wit={N23}]{nītvā tāv iṃdavasthāne}% kṛtvā tāv iṃdavasthāne K1
		\rdg[wit={J7}]{nītvā tāvad avasthāne}% kṛtvā tāvad vindusthairyaṃ B2
		\rdg[wit={N3,V3}]{kṛtvāsāv aindave sthāne} % aidava V3; kṛtvā tām J5; damaged G4
		\rdg[wit={P11}]{kṛtvāsav aidavai sthānair}
		\rdg[wit={C6}]{hṛtvā mamedaṃ ca sthānaṃ}
		\rdg[wit={N19}]{sthitvāsāṃcaiṃdave sthāne}
		\rdg[wit={J10}]{sthitvā sadaiṃdave sthāne}
		\rdg[wit={Jyo}]{sthitvā sadaiva susthāne}
		\rdg[wit={V15}]{samāvasthā sthito yogī}}}
\pada{\app{\lem[wit={Gr2,V19,C7,V15,J10}]{prāṇa}% +N24
		\rdg[wit={N3,GrB,N19}]{ghrāṇa}% +J5, illeg. G4
		\rdg[wit={K3}]{payo}
		\rdg[wit={Jyo}]{brahma}}%
	\app{\lem[wit={N3,C6,V3,J7,K3,J10,Jyo}]{randhre}
		\rdg[wit={N23,V19,C7,N19,V15}]{randhraṃ}
		\rdg[wit={P11}]{randhra}}
	\app{\lem[wit={N3,GrB,C7,N19,V15,J10,Jyo}]{nirodhayet}
		\rdg[wit={Gr2,V19,K3}]{nirundhayet}}//}\label{jnatva}\\!}
\end{tlg}
% V15 totally different: suṣumnāṃtagataṃ mārgaṃ vāyuṃ kṛtvā ca madhyagaṃ/ samāvasthā sthito yogī prāṇaraṃdhraṃ nirodhayet//
% K1 kṛtvā tāv iṃdavasthāne prāṇa°
% B2 kṛtvā tāvad bindusthairyaṃ prāṇa°
% J11 sthitvā sā caiṃdavasthāne ghrāṇa°


% between V3_4.175-176
\begin{ava}[hp04_056]
\app{\lem[wit={N3,C6}]{tathā ca vasiṣṭhaḥ} % vasiṣṭāḥ P7, ca vasiṣṭaḥ C6
		\rdg[wit={J5}]{tathā vaśiṣṭhavacanaṃ}
		\rdg[wit={V3}]{tatvāva || ☼ ||}/} \sgwit{N3,C6,V3} % om. P11; broken G4
\end{ava}

\begin{tlg}[hp04_056]% = V3_4.176
\tl{
\pada{iḍāyāṃ \app{\lem[wit={N3,Gr4b,Gr2,Gr3a}]{piṅgalāyāṃ ca}
		\rdg[wit={V3}]{piṅgalāyāṃśca}}}
\pada{carataś candrabhāskarau/}\\+}
\tl{
\pada{candras tāmasa ity uktaḥ}
\pada{sūryo
	\app{\lem[wit={N3,GrB,J7,Gr3a}]{rājasa}
		\rdg[wit={N23},post=\texteng{(end of the last available folio)}]{rā}} ucyate//}\myfn{\getsiglum{N23} breaks at \textit{sūryo rā} pāda d.}
		\NotIn{N19,V15,J10,Jyo}\\!}
\end{tlg}
		%\sgwit{N3,GrB,Gr2,Gr3a}

\newpage
\begin{tlg}[hp04_057]% = V3_4.177
\tl{
\pada{\app{\lem[wit={N3,P11}]{tāv eva dhattaḥ sakalaṃ} % dhataḥ N3
		\rdg[wit={J7,K3,C7}]{tāv eva dattaḥ sakalaṃ}
		\rdg[wit={V19}]{tā eva dhattaḥ sakalaṃ}
		\rdg[wit={V3}]{tāṃve dhattaḥ sakala}
		\rdg[wit={C6}]{tau eva vahataḥ sarvaṃ}
		\rdg[wit={V15,Jyo}]{sūryācandramasau dhattaḥ}
		\rdg[wit={N19}]{sūryacandrau sadā dhatte}
		\rdg[wit={J10}]{sūryācandramasau kṛtvā}}}
\pada{\app{\lem[wit={P11,J7,Gr3a,V15,Jyo}]{kālaṃ}
		\rdg[wit={N3,C6}]{kāla}
		\rdg[wit={N19}]{kālāṃ} % kālāṃśatri N19
		\rdg[wit={V3,J10},alt={\om}]{\skp{\om}}}
	\app{\lem[wit={Jyo}]{rātriṃdivātmakam}
		\rdg[wit={N3,Gr4b,J7,V15}]{rātridivātmakam}% +V6
		\rdg[wit={Gr3a},post=\texteng{(rātridi° \getsiglum{K3})}]{rātrindinātmakaṃ}
		\rdg[wit={V3}]{rātridivātmakaṃ yogavit}
		\rdg[wit={N19}]{°śa tridivātmakaṃ}
		\rdg[wit={J10},alt={\om}]{\skp{\om}}}/}\\+}
\tl{
\pada{\app{\lem[wit={N3,P11,J7,Gr3a,V15,Jyo}]{bhoktrī} % bho<<ktrī>> N3
		\rdg[wit={N19}]{bhoktī}
		\rdg[wit={V3}]{bhoktā}
		\rdg[wit={C6}]{bhoktṛ}
		\rdg[wit={J10},alt={\om}]{\skp{\om}}}
	suṣumṇā kālasya}
\pada{\app{\lem[wit={N3,GrB,J7,N19,V15,Jyo},alt={guhyam etad}]{guhyam eta\skp{d}}
		\rdg[wit={V19}]{guptam etad}
		\rdg[wit={C7}]{sattvam etad}
		\rdg[wit={K3}]{supyate tad}
		\rdg[wit={J10},alt={\om}]{\skp{\om}}}d udāhṛtam//}
		\lineom{bcd}{J10}\\!}
\end{tlg}

%\newpage
\begin{ava}[hp04_058]
\app{\lem[wit={N3,C6,V3,Gr3a}]{tathā hi}
	\rdg[wit={P11}]{tathāpi hi}
	\rdg[wit={J7}]{tathā}}
\app{\lem[wit={N3,Gr3a}]{saubhadraṃ nāma}
	\rdg[wit={J7}]{saubhadranāmā}
	\rdg[wit={GrB}]{saubhadreyaṃ nāma}} % saubhadreya C6, saubhadreryān P11
	ślokacatuṣṭayam\app{\lem[nolem]{}
	\rdg[wit={J7},alt={\post °catuṣṭayam \add\ āha}]{\skm{\post °catuṣṭayam \add\ }āha}}/
%	\sgwit{N3,GrB,J7,Gr3a}
	\NotIn{N19,V15,J10,Jyo}
\end{ava}

\begin{tlg}[hp04_058]% = V3_4.178/9
\tl{
\pada{ṣaṭcakraṃ ṣoḍaśādhāraṃ} % cakra K3, raktaṃ N3
\pada{\app{\lem[wit={V3,J7,V19,K3},alt={tridhā lakṣ(y)aṃ}]{tridhā lakṣyaṃ}% lakṣaṃ V3,J7
		\rdg[wit={N3}]{tridhā bhajyaṃ}
		\rdg[wit={C7}]{tridhā yuktaṃ}
		\rdg[wit={P11}]{tridhākṣa ca}
		\rdg[wit={C6}]{trilakṣyaṃ ca}} guṇatrayam/}\\+}
\tl{
\pada{\app{\lem[wit={N3,GrB}]{śeṣaṃ tu} % m vs n! ##
		\rdg[wit={J7,Gr3a}]{śeṣas tu}}
	\app{\lem[wit={N3,GrB,J7,V19,K3}]{grantha}
		\rdg[wit={C7}]{granthi}}%
	\app{\lem[wit={N3,GrB}]{vistāraṃ}
		\rdg[wit={J7,Gr3a}]{vistāras}}}
\pada{\app{\lem[wit={N3,C6,V3,J7,V19}]{trikūṭaṃ}
		\rdg[wit={C6}]{trikoṭi}
		\rdg[wit={K3,C7}]{trirūpaṃ}} paramaṃ padam//}
%	\sgwit{N3,GrB,J7,Gr3a}
	\NotIn{N19,V15,J10,Jyo}\\!}
\end{tlg}


\begin{tlg}[hp04_059]% = V3_4.179/180
\tl{
\pada{kuṇḍalī kuṭilākārā}
\pada{sarpavat parikīrtitā/}\\+}
\tl{
\pada{sā śaktiḥ
	\app{\lem[wit={N3,GrB}]{cālitā}
	\rdg[wit={V19,K3}]{kīlitā}
	\rdg[wit={C7}]{kelitā}} yena}
\pada{sa \app{\lem[wit={Gr3a}]{mukto}% +G4,G7 (eye-skip to the next verse?)
	\rdg[wit={N3,GrB}]{yogī}} % +J5
	nātra saṃśayaḥ//}
	\sgwit{Gr1,GrB,Gr3a} \anm{= 3.107}\\!} % +B2
\end{tlg}
% V3 has a gap indicated before the following verse.

%\newpage
\begin{tlg}[hp04_060]% = V3_4.181
\tl{
\pada{yadā kūṭaṃ trikūṭasthaṃ}
\pada{cittaṃ  % citta V3
	\app{\lem[wit={N3}]{citraṃ}% cittaṃ J5
		\rdg[wit={GrB}]{tatra}}
	\app{\lem[wit={N3,P11,V3}]{nirantaram}
		\rdg[wit={C6}]{nirañjanaṃ}}/}\\+}
\tl{
\pada{kuṇḍalyās tu
	\app{\lem[wit={N3,P11,V3}]{prayogeṇa}
		\rdg[wit={C6}]{prabodhena}}}
\pada{sa mukto nātra saṃśayaḥ//} \sgwit{Gr1,GrB}\\!} % not in G4,B2
\end{tlg}


%\newpage
\begin{tlg}[hp04_061]% = V3_4.182
\tl{
\pada{\app{\lem[wit={N3,GrB,J7,Gr3a,Jyo}]{dvāsaptati}
		\rdg[wit={N19,V15}]{dvisaptati}
		\rdg[wit={J10},alt={\om}]{\skp{\om}}}sahasrāṇi} % sahaśrāṇi V19
\pada{\app{\lem[wit={N3,GrB,J7,V15,Jyo}]{nāḍīdvārāṇi}
		\rdg[wit={N19}]{nāḍīdvāre ca}
		\rdg[wit={K3,C7}]{nāḍīnāṃ deha}
		\rdg[wit={V19}]{nāḍīnāṃdeda}
		\rdg[wit={J10}]{datvā kārāpi}}\marmas
	\app{\lem[wit={ceteri}]{pañjare}
	\rdg[wit={N3}]{paṃkaje}}/}\\+}
\tl{
\pada{suṣumṇā śāṃbhavī śaktiḥ} % sāṃbhavī J10; śakti V19,V3
\pada{\app{\lem[wit={N3,GrB,K3,C7,N19,Jyo}]{śeṣās tv eva}
		\rdg[wit={J10}]{śeṣās tv evaṃ}
		\rdg[wit={J7,V19,V15}]{śeṣāś caiva}}
	\app{\lem[wit={ceteri}]{nirarthakāḥ} % nina° J10; °kā V3
		\rdg[wit={N19}]{nivarttakāḥ}
		\rdg[wit={K3},post=\texteng{(end of the last existing folio)}]{nira}}//}%
	\myfn{\getsiglum{K3} breaks at \textit{nira} in pāda d.}\\!}
\end{tlg}

%\Anm{\getsiglum{K3} discontinues}

\newpage
\begin{tlg}[hp04_062]% = V3_4.183
\tl{
\pada{vāyuḥ % vāyu V19,P11
	\app{\lem[wit={N3,C6,N19,V15,J10,Jyo}]{paricito}
		\rdg[wit={V3}]{paricipta}
		\rdg[wit={J7}]{sa parito}
		\rdg[wit={V19,C7}]{saṃparito}
		\rdg[wit={P11}]{parivṛtto}}
	\app{\lem[wit={N3,Gr4b,J7,V19,N19,V15},alt={yatnād}]{yatnā\skp{d}}
		\rdg[wit={C7}]{yadvad}
		\rdg[wit={J10,Jyo}]{yasmād}
		\rdg[wit={V3}]{nādād}}}%
\pada{\app{\lem[wit={GrB,V19,C7,N19,V15,J10,Jyo},alt={agninā}]{\skm{d }agninā} % agnidā N19
		\rdg[wit={J7}]{ṛgvinā}
		\rdg[wit={N3}]{yaṣṭinā}} saha
	\app{\lem[wit={C7,Jyo}]{kuṇḍalīm}
		\rdg[wit={N3,GrB,J7,V19,N19,V15,J10}]{kuṇḍalī}}/}\\+}
\tl{
\pada{bodhayitvā suṣumṇāyāṃ} % °yā N3
\pada{\app{\lem[wit={N3,Gr4b,J7,V19,C7,N19,V15,Jyo},alt={praviśed}]{praviśe\skp{d}}
		\rdg[wit={V3}]{praveśad}
		\rdg[wit={J10},alt={\om}]{\skp{\om}}}%
	\app{\lem[wit={N3,J7,V19,C7},alt={avirodhataḥ}]{\skm{d }avirodhataḥ} % +J5; aviśeṣataḥ V19ac
		\rdg[wit={GrB,V15,Jyo}]{anirodhataḥ} % +G4; °ta V3
		\rdg[wit={N19}]{atirodhataḥ}
		\rdg[wit={J10},alt={\om}]{\skp{\om}}}//}\label{bodhayitva} \lineom{cd}{J10}\\!}
\end{tlg}

%\newpage
\begin{tlg}[hp04_063]% = V3_4.184
\tl{
\pada{suṣumṇā\app{\lem[wit={C6,V3,J7,C7,Jyo}]{vāhini}
		\rdg[wit={N3,P11,N19,V15}]{vāhinī}
		\rdg[wit={V19}]{hini}
		\rdg[wit={J10},alt={\om}]{\skp{\om}}} prāṇe}
\pada{\app{\lem[wit={GrB,J7,V19,N19,V15,Jyo}]{sidhyaty eva} % siddhaty N19
		\rdg[wit={N3}]{siddhyety eva}
		\rdg[wit={C7}]{siddhyatīva}
		\rdg[wit={J10},alt={\om}]{\skp{\om}}} manonmanī/} \lineom{ab}{J10}\\+}
\tl{
\pada{\app{\lem[wit={N3,GrB,J7}]{anyathā vividhā}
		\rdg[wit={C7}]{anye ca vividhā}
		\rdg[wit={V19}]{anye ye vividhā}
		\rdg[wit={N19,V15}]{anyathā tv itare}
		\rdg[wit={Jyo}]{anyathā tv itarā}
		\rdg[wit={J10}]{atha cittāntare}}%
	\app{\lem[wit={N3,C6,C7,Jyo},post=\texteng{(°sā<<ḥ>> \getsiglum{C7})}]{bhyāsāḥ}% bhyāsā<<ḥ>> C7
		\rdg[wit={V3,J7,V19}]{bhyāsā}
		\rdg[wit={P11,N19}]{bhyāsāt}
		\rdg[wit={V15,J10}]{bhyāsa}}}
\pada{\app{\lem[wit={N3,GrB,J7,C7,Jyo}]{prayāsāyaiva}
		\rdg[wit={V19}]{prāyāsāś caiva}
		\rdg[wit={V15}]{prayāsā eva}
		\rdg[wit={N19}]{prayāsā eka}
		\rdg[wit={J10}]{pratyāśā jīva}}
	\app{\lem[wit={N3,Gr4b,J7,V19,C7,V15,Jyo}]{yoginām}
		\rdg[wit={V3,J10}]{yoginā}
		\rdg[wit={N19}]{yoginī}}//}\\!}
\end{tlg}

%\newpage
\begin{tlg}[hp04_064]% = V3_4.185
\tl{
\pada{pavano badhyate yena}
\pada{\app{\lem[wit={N3,GrB,J7,V19,C7,N19,V15,Jyo}]{manas tenaiva badhyate} % <va>dhyate N19
		\rdg[wit={J10}]{tenaiva badhyate manaḥ}}/}\\+}
\tl{
\pada{\app{\lem[wit={N3,P11,V3,N19,V15,Jyo}]{manaś ca}
		\rdg[wit={V19,C7}]{manas tu}
		\rdg[wit={C6}]{manas tad}} badhyate yena}
\pada{\app{\lem[wit={N3,Gr4b,V19,C7,N19,V15,Jyo}]{pavanas tena} % <pa>vanas V19
		\rdg[wit={V3}]{pavanamana}} badhyate//}
		\lineom{cd}{J7,J10}\\!}
\end{tlg}

\begin{tlg}[hp04_065]% = V3_4.186
\tl{
\pada{\app{\lem[wit={N3,GrB,J7,V19,N19,V15,J10,Jyo}]{hetu}
		\rdg[wit={C7}]{deha}}%
	\app{\lem[wit={N3,C7,J10,Jyo}]{dvayaṃ tu}
		\rdg[wit={P11,V3,J7}]{dvayaṃ hi}
		\rdg[wit={C6,V19}]{dvayaṃ ca}
		\rdg[wit={N19,V15}]{dvayasya}}
	\app{\lem[wit={N3,GrB,N19,V15,J10,Jyo}]{cittasya}
		\rdg[wit={J7,V19,C7}]{manaso}}}
\pada{vāsanā ca samīraṇaḥ/}\\+} % ḥ om. V3
\tl{
\pada{tayor vinaṣṭa ekasmi}%n % tayo J7; ekasmi V19; vinaṣṭas tv ekaś ca hy C6
\pada{\app{\lem[resp=emend,alt={drutaṃ dvāv api naśyataḥ}]{\skm{n }drutaṃ dvāv api naśyataḥ}
		\rdg[wit={N3}]{dguttaṃ dvāv api naśyataḥ}
		\rdg[wit={J5}]{dṛtaṃ vāvatinasyataḥ}% dhṛtaṃ? G4
		\rdg[wit={P11,V3,N19,V15,Jyo}]{tau dvāv api vinaśyataḥ} % °ta V3, °tiḥ N19
		\rdg[wit={C6,J7,C7,J10}]{ubhāv api vinaśyataḥ}
		\rdg[wit={V19}]{svabhāvo pi vinaśyataḥ}}//}%
	\myfn{\getsiglum{V19} has this verse and the next one after \ref{dugdha}.}\\!}
\end{tlg}

%\newpage
\begin{tlg}[hp04_066]% = V3_4.187
\tl{
\pada{mano yatra \app{\lem[wit={N3,Gr4b,J7,V19,C7,N19,V15,Jyo}]{vilīyeta} % °līyena N19
		\rdg[wit={V3}]{vilīyate}}}
\pada{\app{\lem[wit={N3,GrB,J7,V19,C7,Jyo},alt={pavanas}]{pavana\skp{s}}
		\rdg[wit={N19,V15}]{mārutas}}s tatra līyate/}\\+}
\tl{
\pada{\app{\lem[wit={N3,C6,J7,Jyo}]{pavano līyate yatra}
		\rdg[wit={V19,C7}]{pavano yatra līyeta}
		\rdg[wit={P11,V3}]{pavano yatra līyate}% līyaṃte P11
		\rdg[wit={N19,V15},alt={\om}]{\skp{\om}}}}
\pada{mana\app{\lem[wit={N3,GrB,V19,C7},alt={tatraiva līyate}]{\skm{s }tatraiva līyate}% +N2
		\rdg[wit={J7,Jyo}]{tatra vilīyate}% +V6
		\rdg[wit={N19,V15},alt={\om}]{\skp{\om}}}//}\myfn{%
		\getsiglum{J10} have an abridged version:
		\textit{yatraiva līyate vāyur manas tatraiva līyate};\\
		\getsiglum{V15} has an incomplete passage \textit{ekatra[m]iśritau} after this verse.}
	\lineom{cd}{N19,V15}\\!}
\end{tlg}


\newpage
\begin{tlg}[hp04_067]% = V3_4.188
\tl{
\pada{dugdhāmbuvat saṃmilitau % °tāv J10, °to N19
	\app{\lem[wit={N3,GrB,N19,V15}]{sadaiva}
		\rdg[wit={J7,V19,C7}]{tathaiva}
		\rdg[wit={J10,Jyo}]{ubhau tau}}}\\+}
\tl{
\pada{tulyakriyau % kriyo N19
	\app{\lem[wit={N3,V3,J7,V19,C7,N19,V15,J10,Jyo}]{mānasamārutau} % māruto N19; 2 x mānasa J10
		\rdg[wit={Gr4b}]{mārutamānasau}}
	\app{\lem[wit={N3,P11,N19,V15,J10,Jyo}]{hi}
		\rdg[wit={C6,V3,J7,V19,C7}]{ca}% + all testimonia
		}/}\\+}
\tl{
\pada{\app{\lem[wit={N3,GrB,J7,V19,C7,N19,V15},alt={yāvan manas}]{yāvan mana\skp{s}}
		\rdg[wit={J10,Jyo}]{yato marut}}s tatra
	\app{\lem[wit={N3,GrB,J7,V19,C7,N19,V15},alt={marut}]{maru\skp{t}} % marat V15
		\rdg[wit={J10,Jyo}]{manaḥ}}%
	\app{\lem[wit={N3,GrB,J7,V19,C7,V15,J10,Jyo},alt={pravṛttir}]{\skm{t}pravṛtti\skp{r}} % vṛtti J7, vṛtta C6, varti P7
		\rdg[wit={N19}]{pravṛddhitti}}-}\\+}
\tl{
\pada{\app{\lem[wit={N3,GrB,J7,V19,C7},alt={yāvan}]{\skm{r }yāva\skp{n}}
		\rdg[wit={J10,Jyo}]{yato}
		\rdg[wit={N19,V15},alt={\om},post=\texteng{(pāda d om.)}]{}}%
	\app{\lem[wit={N3,J7,V19,C7Gr4b,},alt={maruc cāpi}]{\skm{n }maruc cāpi}
		\rdg[wit={V3}]{marut tatra}
		\rdg[wit={J10,Jyo}]{manas tatra}
		\rdg[wit={N19,V15},alt={\om}]{\skp{\om}}}
	\app{\lem[wit={N3,GrB,J7,V19,C7}]{manaḥ} % mana V19
		\rdg[wit={J10,Jyo}]{marut}% or: marun
		\rdg[wit={N19,V15},alt={\om}]{\skp{\om}}}%
	\app{\lem[wit={N3,GrB,J7,V19,C7,Jyo}]{pravṛttiḥ} % °vṛttaḥ C6, °vartiḥ P7
		\rdg[wit={J10}]{nivṛttiḥ}
		\rdg[wit={N19,V15},alt={\om}]{\skp{\om}}}//}\label{dugdha}\\!}
\end{tlg}

%\newpage
\begin{tlg}[hp04_068]% = V3_4.189
\tl{
\pada{\app{\lem[wit={N3,GrB,J7,V19,C7,Jyo}]{tatraika}% tatr<<aika>> N3
		\rdg[wit={N19,V15}]{atraika}
		\rdg[wit={J10}]{ekasya}}nāśād aparasya % nā<śā>d N19
	\app{\lem[wit={N3,C6,J7,C7,Jyo}]{nāśa}
		\rdg[wit={V19}]{nāśam}
		\rdg[wit={N19,V15}]{nāśaḥ}
		\rdg[wit={J10}]{nāśas}
		\rdg[wit={P11}]{nāśe}
		\rdg[wit={V3}]{nāśo}}}\\+}
\tl{
\pada{\app{\lem[wit={N3,GrB,J7,N19,Jyo},alt={ekapravṛtter}]{ekapravṛtte\skp{r}}% vṛtte ca C6
		\rdg[wit={V19,C7,V15}]{ekapravṛttāv}
		\rdg[wit={J10}]{tatraikavṛtter}}%
	\app{\lem[wit={N3,GrB,J7,V19,C7,N19,V15,Jyo},alt={aparapravṛttiḥ}]{\skm{r }aparapravṛttiḥ} % °ttir N19, °tt<<i>>ḥ N3
		\rdg[wit={J10}]{aparasya vṛttiḥ}}/}%
	\myfn{In \getsiglum{V19} Pādas ab and cd are transposed;
	\getsiglum{V15} inserts here a variant reading for Pāda a\,: \textit{ekasya nā<śā>d aparasya nāśaḥ}.}\\+}
\tl{
\pada{\app{\lem[wit={N3,P11,Jyo},alt={adhvastayoś}]{adhvastayo\skp{ś}}
		\rdg[wit={V15}]{adhvastayor}
		\rdg[wit={J7}]{adhyastayor}
		\rdg[wit={V19,C7}]{adhastayor}
		\rdg[wit={N19}]{addhastayoś}
		\rdg[wit={C6,J10}]{adhastayoś}
		\rdg[wit={V3}]{atastayoś}}%
	\app{\lem[wit={N3,GrB,N19,J10,Jyo},alt={cendriya}]{\skm{ś }cendriya}
		\rdg[wit={J7,V19,C7,V15}]{indriya}}varga% caidriya N19
	\app{\lem[wit={N3,G4},alt={buddhir}]{buddhi\skp{r}}% +G4?
		\rdg[wit={V3}]{vudhir}
		\rdg[wit={J7,C7}]{vṛddhir}
		\rdg[wit={V19,N19,V15,J10,Jyo}]{vṛttiḥ}% vṛttir V19, vṛtti V15
		\rdg[wit={P11}]{baṃdhir}
		\rdg[wit={J5,C6}]{śuddhir}
		}\marma-}\\+}
\tl{
\pada{\app{\lem[wit={N3,GrB,V19,C7,V15},alt={vidhvastayor}]{\skm{r }vidhvastayo\skp{r}}
		\rdg[wit={J7}]{vivṛddhayor}% °yo J7
		\rdg[wit={J10}]{vijñātayor}
		\rdg[wit={N19}]{addhvastayor}
		% vṛttiḥ|(gap for about one hemistich)raddhvasta° N19
		\rdg[wit={Jyo}]{pradhvastayor}}%
	\app{\lem[wit={N3,GrB,V19,N19,V15,J10,Jyo},alt={mokṣapadasya siddhiḥ}]{\skm{r }mokṣapadasya siddhiḥ} % siddhi V19
		\rdg[wit={C7}]{mokṣapathasya siddhiḥ}
		\rdg[wit={J7}]{mokṣapradasya siddhiḥ}}//}\label{tatraika}\\!}
\end{tlg}

%\newpage
%====== Passage B and yāvanna-stanza collated with more mss (V19,N19,J10)

\begin{tlg}[hp04_069]% = V3_4.190
%\Anm{This verse appears after 4.12 in \getsiglum{N19,V15,J10}}\\+}
\tl{
\pada{\app{\lem[wit={N3,GrB,J7,C7,N19,J10}]{vāyu}
		\rdg[wit={V19,V15}]{vāyur}}%
	\app{\lem[wit={V19,C7}]{mārgeṇa saṃcārī}
		\rdg[wit={N3,GrB,J7}]{mārgeṇa saṃcāre}% +J5,G4
		\rdg[wit={N19}]{mārge tha saṃcāre}
		\rdg[wit={J10}]{mārge ca saṃcāre}
		\rdg[wit={V15}]{mārge py asaṃcāre}}}
\pada{\app{\lem[wit={N3,V3,J7,V19,C7}]{sakalāṃ}
		\rdg[wit={C6,N19,V15}]{sakalaṃ}
		\rdg[wit={J10}]{sa phalaṃ}
		\rdg[wit={P11}]{sakalyāt}}
	\app{\lem[wit={N3,P11,V15,J10}]{labhate}% +J5,G4
		\rdg[wit={C6,N19}]{labhyate}
		\rdg[wit={J7,V19,C7}]{bhramate}
		\rdg[wit={V3}]{carate}}\marmas
	\app{\lem[wit={N3,P11,J7,V19,C7}]{mahīm} % mahī V3
		\rdg[wit={C6,V3}]{mahī}
		\rdg[wit={N19,V15}]{mahaḥ}
		\rdg[wit={J10}]{mahān}}/}\\+}
\tl{
\pada{\app{\lem[wit={N3,V19,C7}]{tathāṣṭa}
		\rdg[wit={J7}]{na tathā}
		\rdg[wit={N19,V15,J10}]{tato'ṣṭa}
		\rdg[wit={C6,V3}]{athāṣṭa}
		\rdg[wit={P11}]{aṣṭadhā}}guṇam aiśvaryaṃ} % ṃ om. J7
\pada{satyaṃ \app{\lem[wit={N3,GrB,J7,V19,C7}]{satyaṃ varānane}
		\rdg[wit={N19,V15,J10}]{ity āha śaṃkaraḥ}}//}
	\NotIn{Jyo}\label{vayumargena}
	\anm{after 4.12 \getsiglum{N19,V15,J10}}\\!}
\end{tlg}

\newpage
\begin{ava}[hp04_070]
\app{\lem[wit={N3,Gr4b}]{tathā}
	\rdg[wit={J5}]{tathā ca}
	\rdg[wit={G4}]{tathāha}
	\rdg[wit={J7,V19,C7},alt={\om}]{\skp{\om}}} 
	viśvarūpācāryaḥ/
	\sgwit{Gr1,Gr4b,J7,V19,C7}% °yāḥ J7
\end{ava}

\begin{tlg}[hp04_070]%
\tl{
\pada{yadā \app{\lem[wit={N3,C6,V19,C7,Jyo}]{saṃkṣīyate}
		\rdg[wit={P11,J7}]{sa kṣīyate}} prāṇo} % =P11,P7; prāṇe C6
\pada{mānasaṃ \app{\lem[wit={N3,Gr4b,C7}]{ca vilīyate}
		\rdg[wit={J7,Jyo}]{ca pralīyate}
		\rdg[wit={V19}]{pravilīyate}}/}\\+}
\tl{
\pada{tadā samarasatvaṃ
	\app{\lem[wit={N3,C6,J7},alt={yat}]{ya\skp{t}}
		\rdg[wit={V19}]{yaḥ}
		\rdg[wit={C7}]{hi}
		\rdg[wit={P11,Jyo}]{ca}}}%
\pada{\app{\lem[wit={N3,C6,J7,V19,C7},alt={samādhiḥ so'bhidhīyate}]{\skm{t }samādhiḥ so'bhidhīyate} % samādhi V19
		\rdg[wit={P11}]{samādhī sau bhidhīyate}
		\rdg[wit={Jyo}]{samādhir abhidhīyate}}//}\label{visvarupa}
\NotIn{V3,N19,V15,J10} \anm{after \ref{salile} \getsiglum{Jyo}}\\!}
\end{tlg}

%\newpage
\begin{tlg}[hp04_071]%
%\Anm{This verse appears before 4.13 in \getsiglum{N19,V15,J10,Jyo}}
\tl{
\pada{\app{\lem[wit={N3,Gr4b,J7,V19,C7,Jyo}]{manaḥ}
		\rdg[wit={N19,V15,J10}]{mana}}%
\app{\lem[wit={N3,Gr4b,J7,N19,J10,Jyo}]{sthairye}
		\rdg[wit={V19}]{sthairya}
		\rdg[wit={C7}]{sthairyaṃ}
		\rdg[wit={V15}]{sthairyaḥ}}
	\app{\lem[wit={ceteri}]{sthiro}
		\rdg[wit={V15,Jyo}]{sthito}} vāyus}
\pada{tato \app{\lem[wit={N3,J7,C7,V15,Jyo}]{binduḥ}
		\rdg[wit={Gr4b,V19,N19,J10}]{bindu}}
	\app{\lem[wit={ceteri}]{sthiro}
		\rdg[wit={C7}]{sthito}} bhavet/}\\+}
\tl{
\pada{\app{\lem[wit={ceteri}]{bindu}
		\rdg[wit={J7}]{binduḥ}}%
	\app{\lem[wit={N3,Gr4b,C7},alt={sthairyodayāt}]{sthairyodayā\skp{t}}
		\rdg[wit={V19}]{sthairyād yathā}
		\rdg[wit={J10}]{sthairyād dayā}
		\rdg[wit={J7}]{sthairyād athā}
		\rdg[wit={N19}]{sthairyodayā}
		\rdg[wit={V15}]{sthairye dayā}
		\rdg[wit={Jyo}]{sthairyāt sadā}}%
	\app{\lem[wit={N3,P11},alt={putra}]{\skm{t }putra} % tatra G4
		\rdg[wit={J7}]{panna}% pannaṃ V6,YCM; panna J7,N2
		\rdg[wit={C6}]{mūtra}
		\rdg[wit={C7,N19,V15}]{satyaṃ}% +Ten
		\rdg[wit={J10,Jyo}]{satvaṃ}
		\rdg[wit={V19},alt={\gap}]{\_\,\_\,\_}}}
\pada{piṇḍasthairyaṃ prajāyate//}
	\NotIn{V3}\label{manahsthairye}
	\anm{after 4.12 \getsiglum{N19,V15,J10,Jyo}}\\!}
\end{tlg}

%\newpage
%%%%%%%%%%%%%%%%%%%%%%%
\begin{tlg}[hp04_072]% = V3_4.191 (N23,K3,J15 lost)
\tl{
\pada{dṛṣṭiḥ sthirā yasya % dṛṣṭi V19,V3,J10
	\app{\lem[wit={N3,GrB,V15,J10}]{vinaiva}
		\rdg[wit={C7}]{vinā ca}
		\rdg[wit={J7,V19}]{vināpi}}
	\app{\lem[wit={N3,GrB,V15},alt={dṛśyād}]{dṛśyā\skp{d}}
		\rdg[wit={J7,V19,C7,J10}]{dṛśyaṃ}}-}\\+}
\tl{
\pada{d vāyuḥ sthiro yasya % vāyu J7,V19,V15,J10
	\app{\lem[wit={ceteri}]{vinā prayatnāt}
		\rdg[wit={J7}]{vināpi yatnaṃ}}/}\\+} % prayatnataḥ
\tl{
\pada{cittaṃ sthiraṃ yasya
	\app{\lem[wit={N3,C6,V3,V15}]{vināvalambāt}
		\rdg[wit={J10}]{vināvalaṃnaṃ}
		\rdg[wit={V19}]{vināvalaṃbanaṃ}% +N2
		\rdg[wit={P11}]{vinā vilambāt}
		\rdg[wit={C7}]{vinā balaṃ ca}
		\rdg[wit={J7}]{vinā prayatnāt}}}\\+}
\tl{
\pada{sa eva yogī
	\app{\lem[wit={ceteri}]{sa guruḥ}
		\rdg[wit={J10}]{sadguruḥ}}
	\app{\lem[wit={ceteri}]{sa sevyaḥ}
		\rdg[wit={J7,V19}]{sa śiṣyaḥ}}//}
		\NotIn{N19,Jyo}\\!}
\end{tlg}

%%%%%%%%%%

\begin{tlg}[hp04_073]% = V3_4.192
\tl{
\pada{praveśe nirgame
	\app{\lem[wit={N3,C6,V3,J7,V19,C7,J10}]{vāme}
		\rdg[wit={P11}]{vāpi}
		\rdg[wit={V15}]{cāpi}}}
\pada{dakṣiṇe \app{\lem[wit={N3,P11}]{cordhvam apy adhaḥ}
		\rdg[wit={C6}]{cordhvage'py adhaḥ}
		\rdg[wit={C7}]{cordhvamadhyamaḥ}
		\rdg[wit={J7,V19}]{cordhvamadhyagaḥ} % cordha V19
		\rdg[wit={V15,J10}]{cordhvamadhyataḥ}
		\rdg[wit={V3}]{tanirodhataḥ}}/}\\+}
\tl{
\pada{\app{\lem[wit={N3,P11,V3,J7,V19,C7,V15,J10}]{na yasya}
		\rdg[wit={C6}]{layasya}}
	\app{\lem[wit={N3,Gr4b,J7,V19,C7,V15,J10}]{vāyur vahati}
		\rdg[wit={V3}]{vahate vāyu}}}
\pada{sa mukto nātra saṃśayaḥ//} % yukto? V19
	\NotIn{N19,Jyo}\label{pravese}
	\anm{before \ref{nasuptam} \getsiglum{V15,J10}}\\!}
\end{tlg}


%\newpage
\begin{tlg}[hp04_074]% = V3_4.193
%\myfn{From here on, \getsiglum{V6} is used for collation as a replacement of \getsiglum{J7} which breaks at \textit{rājayo} in Pāda c.}%
\tl{
\pada{sarve \app{\lem[wit={N3,GrB,V15,J10,Jyo}]{haṭhalayopāyā}
		\rdg[wit={N19}]{haṭhalayoyāgā}
		\rdg[wit={V19}]{haṭhā layābhyāsā}
		\rdg[wit={C7}]{layahaṭhābhyāsā}}} % +V6,°haṭha°J7
\pada{\app{\lem[wit={N3,Gr4b,V19,C7,Jyo}]{rājayogasya siddhaye} % +J7
		\rdg[wit={N19,V15,J10}]{rājayogāya kevalaṃ} % yogaya N19
		\rdg[wit={V3}]{rājayogaphalāvadhi}}/}\\+}
\tl{
\pada{\app{\lem[wit={N3,GrB,V19,N19,V15,J10,Jyo}]{rājayoga}
		\rdg[wit={C7}]{rājayoge}}%
	samā\app{\lem[wit={N3,Gr4b,V19,C7,N19,V15,J10,Jyo}]{rūḍhaḥ} % rūḍha V15
		\rdg[wit={V3}]{rūḍhā}}}
\pada{puruṣaḥ kālavañcakaḥ//}\label{kalavancaka}
\anm{after \ref{sravanaputa} \getsiglum{N19,V15,J10}}\\!} % ḥ om. V3
\end{tlg}

\newpage
\startaltrecension
\begin{alttlg}[hp04_074_1]%
\tl{
\pada{iḍā bhagavatī gaṅgā}
\pada{piṅgalā
	\app{\lem[wit={C7}]{yamunā}
	\rdg[wit={V19}]{jamunā}} nadī/}\\+}
\tl{
\pada{\app{\lem[wit={C7}]{vijñeyā}
	\rdg[wit={V19}]{vidheyā}} taddvayor madhye} % tadvayor V19
\pada{suṣumṇā \app{\lem[wit={C7}]{tu}
	\rdg[wit={V19}]{ca}} sarasvatī//} \sgwit{V19,C7}\\!}
\end{alttlg}

\begin{alttlg}[hp04_074_2]%
\tl{
\pada{triveṇīsaṃgamo yatra}
\pada{tīrtharājaḥ sa ucyate/}\\+}
\tl{
\pada{\app{\lem[wit={V19}]{tatra snānaṃ prakurvīta}
	\rdg[wit={C7}]{tasmiṃs tīrthavare snātvā}}}
\pada{sarvapāpaiḥ pramucyate//} \sgwit{V19,C7}\\!}
\end{alttlg}
\endaltrecension

%\newpage
\begin{tlg}[hp04_075]% = V3_4.194
\tl{
\pada{iti tu sakalayogaśāstra%
	\app{\lem[wit={N3pc,C6,C7}]{sindhoḥ}
		\rdg[wit={V19}]{sindhau}
		\rdg[wit={N3ac}]{siddhāḥ}
		\rdg[wit={P11}]{siddheḥ}
		\rdg[wit={V3}]{siddhyaiḥ}}}\\+}
\tl{
\pada{\app{\lem[wit={N3,Gr4b,V19,C7},alt={parimathitād}]{parimathitā\skp{d}}
		\rdg[wit={V3}]{paripaṭhitā}}%
	\app{\lem[wit={N3ac,V19},alt={avakṛṣṭa}]{\skm{d }avakṛṣṭa}
		\rdg[wit={N3pc,C6}]{avakṛṣya}
		\rdg[wit={P11}]{avakṛṣṇa}
		\rdg[wit={C7}]{apakṛṣṭa}
		\rdg[wit={V3}]{kṛṣṭa}}% \marma
	\app{\lem[wit={N3,GrB,C7}]{sāra}
		\rdg[wit={V19}]{sarva}}bhūtam/}\\+} % bhūtaḥ P11
\tl{
\pada{\app{\lem[wit={N3,V3,V19,C7}]{anubhavata}
		\rdg[wit={C6}]{anubhavatu}
		\rdg[wit={P11}]{anubhava}}
		haṭhāmṛtaṃ % <<ha>>ṭhā° V19
	\app{\lem[wit={N3,V3,C7}]{yamīndrā}
		\rdg[wit={C6}]{yamīndro} % < P7, mayedaṃ C6
		\rdg[wit={P11,V19}]{yatīndrā}}}\\+}
\tl{
\pada{yadi bhavatā%m
	\app{\lem[wit={N3,P11,V19,C7},alt={ajarāmaratvavāñchā}]{\skm{m }ajarāmaratvavāñchā}
		\rdg[wit={C6},alt={°vāñchāḥ}]{ajarāmaratvavāñchāḥ}
		\rdg[wit={V3}]{ajarājaraṃ tvaṃ vā}}//}
		\NotIn{N19,V15,J10,Jyo}\\!}
\end{tlg}
		% \sgwit{N3,V3,V19,C7}

%\newpage
\startaltrecension
\begin{alttlg}[hp04_075_1]%
\tl{\pada{vidyātīrthe \app{\lem[resp=emend]{jagati}\rdg[wit={J10}]{yagati}} vibudhāḥ sādhavaḥ satyatīrthe}\\+}
\tl{\pada{gaṅgātīrthe malinamanaso yogino jñānatīrthe/}\\+}
\tl{\pada{dhārātīrthe dharaṇipatayo dānatīrthe dhanāḍhyāḥ}\\+}
\tl{\pada{lajjātīrthe kulayuvatayaḥ pātakaṃ kṣālayanti//}
\sgwit{J10}\\!}
\end{alttlg}
\endaltrecension

\begin{col}[hp04_col]
iti \app{\lem[wit={V3,C7,J10}]{śrī}
	\rdg[wit={N3}]{śrīsadguru}
	\rdg[wit={V15}]{śrīsahajānaṃdasaṃtānaciṃtāmaṇinā}
	\rdg[wit={P11,C6,V19},alt={\om}]{\skp{\om}}}%
\app{\lem[wit={C6,V3}]{svātmārāmayogīndra} % yogiṃdra V3
	\rdg[wit={N3}]{svātmārāmayogendra}
	\rdg[wit={V15}]{svātmārāmayogīṃdreṇa}
	\rdg[wit={J10}]{ātmārāmayogīṃdra}
	\rdg[wit={P11},post=\texteng{(sic!)}]{°yo°}
	\rdg[wit={V19,C7},alt={\om}]{\skp{\om}}}%
\app{\lem[wit={ceteri}]{viracitāyāṃ}
	\rdg[wit={N3ac}]{pravaracitāyāṃ}
	\rdg[wit={N3pc}]{praviracitāyāṃ}} haṭhapradīpikāyāṃ
\app{\lem[alt={\ante caturtho° \add},nosep]{\skp{\ante caturtho° \add}}
	\rdg[wit={V15}]{nādopāsanaṃ nāma}
	\rdg[wit={V3}]{siddhāntamuktāvalī nāma}}%
\app{\lem[wit={N3,GrB,V15}]{caturthopadeśaḥ}
	\rdg[wit={V19}]{caturtha upadeśaḥ}
	\rdg[wit={C7}]{caturtho\{\{dhyā\}\}yam upadeśaḥ}
	\rdg[wit={J10}]{caturthodhyāyaḥ}}//%\sgwit{N3,GrB,V19,C7,V15,J10}
\myfn{The colophon is found only in \getsiglum{N3,GrB,V19,C7,V15,J10}. \getsiglum{N19} has no colophon. \getsiglum{N23,J7,K3} have lost their last folios. \getsiglum{Jyo} reads: \devnote{iti śrīsvātmārāmayogīṃdraviracitāyāṃ haṭhayogapradīpikāyāṃ nāma caturtho'dhyāyaḥ} (Wai) or \devnote{iti śrīsajahānandasantānacintāmaṇisvātmārāmayogīṃdraviracitāyāṃ haṭhayogapradīpikāyāṃ samādhilakṣaṇaṃ nāma caturthopadeśaḥ samāptaḥ} (Tue)}
\end{col}

\end{ekdosis}
\end{otherlanguage}
\newpage
%\vspace*{2cm}
%\bigskip
\teimute{\small}
\begin{tabular}{l l l l}
\multicolumn{3}{l}{\textbf{List of Sigla}} \\
\\
\getsiglum{N3} & N3 & Gr1 & one folio missing in Ch. 4 (\ref{cittananda}b--\ref{nadanu}d)\\
\getsiglum{J5} & J5 & Gr1 & consulted sporadically\\
\getsiglum{G4} & G4 & Gr1 & consulted sporadically \\
\getsiglum{P11} & P11 & Gr4b & partially collated\\
\getsiglum{C6} & C6 & Gr4b \\
\getsiglum{V3} & V3 & Gr6\\
\getsiglum{N23} & N23 & Gr2 & incomplete; breaks at 4.75d\\
\getsiglum{J7} & J7 & Gr2 & incomplete; breaks at 4.91b\\
%\getsiglum{V6} & Gr2 & collated for 4.91--92 only\\
\getsiglum{V19} & V19 & Gr3\\
\getsiglum{K3} & K3 & Gr3 & incomplete; breaks at 4.78d\\
\getsiglum{C7} & C7 & Gr3\\
\getsiglum{N19} & N19 & Gr4c\\
\getsiglum{V15} & V15 & Gr4c\\
\getsiglum{J11} & J11 & Gr4c & collated for 4.44 and 4.48*1--8 only\\
\getsiglum{J10} & J10 & Gr4d\\
\getsiglum{Jyo} & Jyo & Gr4a & Brahmānanda's version, based on the edition 1972 \\
\end{tabular}

\end{document}
