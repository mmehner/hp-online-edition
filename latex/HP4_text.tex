\documentclass[10pt]{memoir}
\setstocksize{220mm}{155mm} 	        
\settrimmedsize{220mm}{155mm}{*}	
\settypeblocksize{170mm}{116mm}{*}	
\setlrmargins{18mm}{*}{*}
\setulmargins{*}{*}{1.2}
% \setlength{\headheight}{5pt}
\checkandfixthelayout[lines]
\linespread{1.2}

\setlength{\footmarkwidth}{1.3em}
\setlength{\footmarksep}{0em}
\setlength{\footparindent}{1.3em}
\footmarkstyle{\textsuperscript{#1} }
\usepackage{fnpos}
\makeFNbottom

\usepackage[teiexport=tidy,poetry=verse]{ekdosis}
\usepackage{libertine}
\usepackage{sanskrit-poetry}
\usepackage{xcolor}

\usepackage[english]{babel}
\usepackage{babel-iast,xparse,xcolor}
\babelfont[iast]{rm}[Renderer=Harfbuzz, Scale=1.5]{AdishilaSan}
%\babelfont[english]{rm}[Scale=0.9]{Adobe Text Pro}
\babeltags{dev = iast}
\babeltags{eng = english}

\SetHooks{
	lemmastyle=\bfseries,
	refnumstyle=\selectlanguage{english}\color{blue}\bfseries, 
	}
\newif\ifinapparatus
\DeclareApparatus{default}[
	lang=english,
	sep = {] },
	delim=\hskip 0.75em,
	rule=none,
	]
\DeclareApparatus{notes}[
	lang=english,
	sep = {},
	delim=\hskip 0.75em,
	rule=\rule{0.7in}{0.4pt},
	]

%\DeclareShorthand{conj}{\texteng{\emph{conj.}}}{ego}
\DeclareShorthand{emend}{\texteng{\emph{em.}}}{ego}

\setlength{\vrightskip}{-10pt}
%\setlength{\vgap}{3mm} % default 1.5em
\verselinenumfont{\footnotesize\selectlanguage{english}\normalfont}
\setlength{\stanzaskip}{0.6\baselineskip}

%Define two commands: \skp ("sanskrit plus"), to be ignored by TeX in
%the edition text, but processed in the TEI output. Conversely, \skm
%("sanskrit minus") is to be processed in the edition text, but
%ignored if found in the apparatus criticus and in the TEI output:

\NewDocumentCommand{\skp}{m}{}
%\NewDocumentCommand{\skm}{m}{\unless\ifinapparatus#1-\fi}
\NewDocumentCommand{\skm}{m}{\unless\ifinapparatus#1\fi} % modified by MD 2022-05-31

%


%%%%%%%%%%%%%%%%%%%% THE  MSS         %%%%%%%%%%%%%%%%%%%%%%%%%%%

%%% Versions
\DeclareWitness{Vu}{\selectlanguage{english}Vulg}{Vulgate, i.e. Brahmānanda's version}[]           
\DeclareWitness{X}{\selectlanguage{english}X}{TenChapter Version, Jodhpur 02228 and 02225 (ed. Lonavla)}[]
\DeclareWitness{Six}{\selectlanguage{english}Ṣ}{SixChapterVersion, ``6ChapterHPms'', fragment of enlarged text, Jodhpur}[]
% Mss. in Geographical Groups
%%%% Varanasi mss (Sampūrṇānanda mss). V1 is Important
\DeclareWitness{V1}{\selectlanguage{english}V\textsubscript{1}}{Sampurnananda Library Sarasvati Bhavan 30109}[]
        \DeclareHand{V1ac}{V1}{\selectlanguage{english}V\rlap{\textsubscript{1}}\textsuperscript{ac}}[] % added by MD
        \DeclareHand{V1pc}{V1}{\selectlanguage{english}V\rlap{\textsubscript{1}}\textsuperscript{pc}}[] % added by MD
\DeclareWitness{V2}{\selectlanguage{english}V\textsubscript{2}}{Sampurnananda Library Sarasvati Bhavan 29869}[]
\DeclareWitness{V3}{\selectlanguage{english}V\textsubscript{3}}{Sampurnananda Library Sarasvati Bhavan 29899}[]
\DeclareWitness{V4}{\selectlanguage{english}V\textsubscript{4}}{Sampurnananda Library Sarasvati Bhavan 29937}[]
\DeclareWitness{V5}{\selectlanguage{english}V\textsubscript{5}}{Sampurnananda Library Sarasvati Bhavan 29938}[]
\DeclareWitness{V6}{\selectlanguage{english}V\textsubscript{6}}{Sampurnananda Library Sarasvati Bhavan 29991}[]
\DeclareWitness{V8}{\selectlanguage{english}V\textsubscript{8}}{Sampurnananda Library Sarasvati Bhavan 30014}[]
\DeclareWitness{V11}{\selectlanguage{english}V\textsubscript{11}}{Sampurnananda Library Sarasvati Bhavan 30029}[]
\DeclareWitness{V12}{\selectlanguage{english}V\textsubscript{12}}{Sampurnananda Library Sarasvati Bhavan 30030}[]
\DeclareWitness{V13}{\selectlanguage{english}V\textsubscript{13}}{Sampurnananda Library Sarasvati Bhavan 30031}[]
\DeclareWitness{V14}{\selectlanguage{english}V\textsubscript{14}}{Sampurnananda Library Sarasvati Bhavan 30050}[]
\DeclareWitness{V15}{\selectlanguage{english}V\textsubscript{15}}{Sampurnananda Library Sarasvati Bhavan 30051}[]
\DeclareWitness{V15pc}{\selectlanguage{english}V\rlap{\textsubscript{15}}\textsuperscript{pc}\space}{}[]
\DeclareWitness{V16}{\selectlanguage{english}V\textsubscript{16}}{Sampurnananda Library Sarasvati Bhavan 30052}[]
\DeclareWitness{V17}{\selectlanguage{english}V\textsubscript{17}}{Sampurnananda Library Sarasvati Bhavan 30053}[] % added by MD
\DeclareWitness{V16pc}{\selectlanguage{english}V\rlap{\textsubscript{16}}\textsuperscript{pc}\space}{}[]
\DeclareWitness{V18}{\selectlanguage{english}V\textsubscript{18}}{Sampurnananda Library Sarasvati Bhavan 30064}[]
\DeclareWitness{V19}{\selectlanguage{english}V\textsubscript{19}}{Sampurnananda Library Sarasvati Bhavan 30069}[]
\DeclareWitness{V21}{\selectlanguage{english}V\textsubscript{21}}{Sampurnananda Library Sarasvati Bhavan 30104}[]
\DeclareWitness{V22}{\selectlanguage{english}V\textsubscript{22}}{Sampurnananda Library Sarasvati Bhavan 30110}[]
\DeclareWitness{V25}{\selectlanguage{english}V\textsubscript{25}}{Sampurnananda Library Sarasvati Bhavan 30122}[]
\DeclareWitness{V26}{\selectlanguage{english}V\textsubscript{26}}{Sampurnananda Library Sarasvati Bhavan 30123}[]
\DeclareWitness{V28}{\selectlanguage{english}V\textsubscript{28}}{Sampurnananda Library Sarasvati Bhavan 30136}[]
\DeclareWitness{W4}{\selectlanguage{english}W\textsubscript{4}}{Wai 399-6171}[]

%%%%%%%%%%%%%%%%%%%%%%%%%%%%%%%%%
%%% Jammu & Kaschmir
\DeclareWitness{K1}{\selectlanguage{english}K\textsubscript{1}}{Raghunātha Temple Library 4383}[settlement=Jammu]
        \DeclareWitness{K1ac}{\selectlanguage{english}K\rlap{\textsubscript{1}}\textsuperscript{ac}\space}{}[]
        \DeclareWitness{K1pc}{\selectlanguage{english}K\rlap{\textsubscript{1}}\textsuperscript{pc}\space}{}[]
\DeclareWitness{L1}{\selectlanguage{english}L\textsubscript{1}}{SOAS RE 43454}[settlement=Jammu]
% More details? Catalogue number? L1 And C1 very close (and come from same region)
%%%%%%%%%%%%%%%%%%%%%%%%%%%%%%%%
% Jodhpur
% J10 is important
\DeclareWitness{J10}{\selectlanguage{english}J\textsubscript{10}}{MSPP Jodhpur 2230}[]
        \DeclareHand{J10ac}{J10}{\selectlanguage{english}J\rlap{\textsubscript{10}}\textsuperscript{ac}}[] % modified by MD
        \DeclareHand{J10pc}{J10}{\selectlanguage{english}J\rlap{\textsubscript{10}}\textsuperscript{pc}}[] % modified by MD
\DeclareWitness{J1}{\selectlanguage{english}J\textsubscript{1}}{Jodhpur 02231}[]
\DeclareWitness{J2}{\selectlanguage{english}J\textsubscript{2}}{Jodhpur 02232}[]   
\DeclareWitness{J3}{\selectlanguage{english}J\textsubscript{3}}{Jodhpur 02233}[]
\DeclareWitness{J4}{\selectlanguage{english}J\textsubscript{4}}{Jodhpur 02234}[]
        \DeclareWitness{J4ac}{\selectlanguage{english}J\rlap{\textsubscript{4}}\textsuperscript{ac}\space}{MSPP Jodhpur 02234}[]
        \DeclareWitness{J4pc}{\selectlanguage{english}J\rlap{\textsubscript{4}}\textsuperscript{pc}\space}{MSPP Jodhpur 02234}[]
\DeclareWitness{J5}{\selectlanguage{english}J\textsubscript{5}}{Jodhpur 02235}[]  % 4 chapters, 34 jpgs,   long colophon, missing lines in the beginning.
\DeclareWitness{J6ac}{\selectlanguage{english}J\rlap{\textsubscript{6}}\textsubscript{ac}}{Jodhpur 02237}[]  % 4 chapters, 49 jpgs,   1st folio: idaṃ gulābarāyasya
% tulasīrāmaśarmmaṇaḥ putrasya pustakaṃ ...        End: iti śrīsahajānandasantānacintāmaṇisvātmārāmaviracitāyāṃ ..
% saṃvat 1802   (more consistent text)
\DeclareWitness{J6pc}{\selectlanguage{english}J\rlap{\textsubscript{6}}\textsubscript{pc}}{Jodhpur 02237}[] 
\DeclareWitness{J7}{\selectlanguage{english}J\textsubscript{7}}{Jodhpur 02241}[]  % 4 chapters, 41 jpgs
\DeclareWitness{J8}{\selectlanguage{english}J\textsubscript{8}}{Jodhpur 23709}[]  % 4 chapters,  87 jpgs.   saṃvat 1724
\DeclareHand{J8ac}{J8}{\selectlanguage{english}J\rlap{\textsubscript{8}}\textsuperscript{ac}}[]  % changed by MD
\DeclareHand{J8pc}{J8}{\selectlanguage{english}J\rlap{\textsubscript{8}}\textsuperscript{pc}}[]  % changed by MD
\DeclareWitness{J9}{\selectlanguage{english}J\textsubscript{9}}{Jodhpur 02224}[]  %  fragment, 20 jpgs.
\DeclareWitness{J11}{\selectlanguage{english}J\textsubscript{11}}{Jodhpur 23532}[]
\DeclareWitness{J12}{\selectlanguage{english}J\textsubscript{12}}{Jodhpur 18552}[] 
\DeclareWitness{J13}{\selectlanguage{english}J\textsubscript{13}}{Jodhpur 02229}[]  %  5 chapters, 93 jpgs.
\DeclareWitness{J14}{\selectlanguage{english}J\textsubscript{14}}{Jodhpur 02239}[]  %  4 chapters
\DeclareWitness{J15}{\selectlanguage{english}J\textsubscript{15}}{Jodhpur 9732A}[]
\DeclareWitness{J17}{\selectlanguage{english}J\textsubscript{17}}{Jodhpur 3013}[]
% Haṭhapradīpikā with (non-Sanskrit) Bhāṣya RORI Jodhpur ACC.NO.18552
%  Haṭhapradīpikā with (non-Sanskrit) commentary, RORI Alwar 952, 4 chapters,  colophon of the comm:
% iti śrīlāhorīmiśravrajabhūṣanaviracitāyāṃ bhāvārthadīpikāyāṃ caturthodhyāya ..    
%  Haṭhapradīpikā (5 chapter) MSPP Jodhpur ACC.NO.02229/

%%%%%%%%%%        Bodleian, Oxford
\DeclareWitness{B1}{\selectlanguage{english}B\textsubscript{1}}{Bodleian Library No. d.457(8)}[settlement=Oxford]
\DeclareWitness{B2}{\selectlanguage{english}B\textsubscript{2}}{Bodleian Library No. d.458(1)}[settlement=Oxford]
\DeclareWitness{B3}{\selectlanguage{english}B\textsubscript{3}}{Bodleian Library No. d.458(9)}[settlement=Oxford]

%%%%%%%%%%%   Chandigarh
\DeclareWitness{C1}{\selectlanguage{english}C\textsubscript{1}}{Lalchand M-2080}[]%L1 And C1 very close (and come from same region)
\DeclareWitness{C2}{\selectlanguage{english}C\textsubscript{2}}{Lalchand M-6065}[]
\DeclareWitness{C3}{\selectlanguage{english}C\textsubscript{3}}{Lalchand M-1293}[]
\DeclareWitness{C4}{\selectlanguage{english}C\textsubscript{4}}{Lalchand M-2081}[]
\DeclareWitness{C4ac}{\selectlanguage{english}C\rlap{\textsubscript{4}}\textsuperscript{ac}\space}{}[]
\DeclareWitness{C4pc}{\selectlanguage{english}C\rlap{\textsubscript{4}}\textsuperscript{pc}\space}{}[]
\DeclareWitness{C5}{\selectlanguage{english}C\textsubscript{5}}{Lalchand M-2082}[]%doesn't have chapter 1
\DeclareWitness{C6}{\selectlanguage{english}C\textsubscript{6}}{Lalchand M-2089}[]
\DeclareWitness{C7}{\selectlanguage{english}C\textsubscript{7}}{Lalchand M-6494}[]
\DeclareWitness{C8}{\selectlanguage{english}C\textsubscript{8}}{Lalchand M-2091}[]
\DeclareWitness{C8pc}{\selectlanguage{english}C\rlap{\textsubscript{8}}\textsuperscript{pc}\space}{}[]
\DeclareWitness{C9}{\selectlanguage{english}C\textsubscript{9}}{Lalchand M-4530}[]

% %%%%%%%%%%        Nepalese
\DeclareWitness{N1}{\selectlanguage{english}N\textsubscript{1}}{NGMPP A1400-2}[]
\DeclareWitness{N2}{\selectlanguage{english}N\textsubscript{2}}{NGMPP B 39-19}[]
\DeclareWitness{N3}{\selectlanguage{english}N\textsubscript{3}}{NGMPP B 62-20}[]
\DeclareWitness{N5}{\selectlanguage{english}N\textsubscript{5}}{NGMPP A60-15 + A61-1}[]
\DeclareWitness{N6}{\selectlanguage{english}N\textsubscript{6}}{NGMPP A61-6}[]
\DeclareWitness{N9}{\selectlanguage{english}N\textsubscript{9}}{NGMPP A62-33}[]
\DeclareWitness{N10}{\selectlanguage{english}N\textsubscript{10}}{NGMPP A62-37}[]
\DeclareWitness{N11}{\selectlanguage{english}N\textsubscript{11}}{NGMPP A63-15}[]
\DeclareWitness{N12}{\selectlanguage{english}N\textsubscript{12}}{NGMPP A939-19}[]
\DeclareWitness{N13}{\selectlanguage{english}N\textsubscript{13}}{NGMPP A1378-18}[]
\DeclareWitness{N16}{\selectlanguage{english}N\textsubscript{16}}{NGMPP B39-20}[]
\DeclareWitness{N17}{\selectlanguage{english}N\textsubscript{17}}{NGMPP B 111-10}[]
\DeclareWitness{N18}{\selectlanguage{english}N\textsubscript{18}}{NGMPP E 929-3}[]
\DeclareWitness{N19}{\selectlanguage{english}N\textsubscript{19}}{NGMPP E-1528-1 / E-1527-7(4)}[]
\DeclareWitness{N20}{\selectlanguage{english}N\textsubscript{20}}{NGMPP E 2037-13 }[]
\DeclareWitness{N21}{\selectlanguage{english}N\textsubscript{21}}{NGMPP E 2097-31}[]
\DeclareWitness{N22}{\selectlanguage{english}N\textsubscript{22}}{NGMPP G 4-4}[]
\DeclareWitness{N23}{\selectlanguage{english}N\textsubscript{23}}{NGMPP G 25-2}[]
\DeclareWitness{N24}{\selectlanguage{english}N\textsubscript{24}}{NGMPP G 190-16}[]
\DeclareWitness{N24ac}{\selectlanguage{english}N\rlap{\textsubscript{24}}\textsuperscript{ac}\space}{}[]
\DeclareWitness{N24pc}{\selectlanguage{english}N\rlap{\textsubscript{24}}\textsuperscript{pc}\space}{}[]

\DeclareWitness{P28}{\selectlanguage{english}P\textsubscript{28}}{BORI 399-1895-1902}[]

%%%%%   Mysore
\DeclareWitness{M1}{\selectlanguage{english}M\textsubscript{1}}{P-5682/4}[]
%%%%%   Tübingen
\DeclareWitness{Tü}{\selectlanguage{english}Tü}{Ma I 339}[]
%%%%%%%%%%
\DeclareWitness{YC}{\selectlanguage{english}YC}{Yogacintāmaṇi}[]
\DeclareWitness{ceteri}{\selectlanguage{english}cett.}{ceteri}[]

%%%%%%%%%% Mss with Commentary
\DeclareWitness{A1}{\selectlanguage{english}A\textsubscript{1}}{Alwar 952}[]


%%%%%%%%%%%%%%%%%%%%%%%%%%%%%%%%%%%%%%%%%%%
%List of all Sigla:
%A1,B1,B2,B3,C1,C2,C3,C4,C6,C7,C8,C9,J1,J2,J3,J4,J10,J13,J14,J15,J17,L1,M1,N3,N5,N6,N9,N10,N11,N12,N13,N16,N17,N19,N20,N21,N22,N23,N24,Tü,V1,V2,V3,V4,V5,V6,V8,V11,V19,V22,V26,Vu
%%%%%%%%%%%%%%%%%%%%%%%%%%%%%%%%%%%%%%%%%%%

\DeclareShorthand{x}{\selectlanguage{english}δ}{J10,J17,N17,P28,W4}


%%% Local Variables:
%%% mode: latex
%%% TeX-master: t
%%% End:

%
%%%%%                   Abbreviation for the printed apparatus,        xml interface needed
%%%%%                   (synonyms in same line)

% Macro for Editing Abbrevs.
%\def\om{\textrm{\footnotesize \textit{omitted in}\ }} %prints om. for omitted in apparatus
%\def\korr{\textrm{\footnotesize \textit{em.}\ }} %prints em. for emended in apparatus
%\def\conj{\textrm{\footnotesize \textit{conj.}\ }} %prints conj. for conjectured in apparatus


\def\eyeskip{\textrm{{ab.\,oc. }}}   
\def\aberratio{\textrm{{ab.\,oc. }}}
\def\ad{\textrm{{ad}}}   
\def\add{\textrm{{add.\ }}}
\def\ann{\textrm{{ann.\ }}}
\def\ante{\textrm{{ante }}}
\def\post{\textrm{{post }}}
%\def\ceteri{cett.\,}             % for simplifying the apparatus in print                  
\def\codd{\textrm{{codd.\ }}}   %  the same
\def\conj{\textrm{{coni.\ }}}  
\def\coni{\textrm{{coni.\ }}}
\def\contin{\textrm{{contin.\ }}}
\def\corr{\textrm{{corr.\ }}}
\def\del{\textrm{{del.\ }}}
\def\dub{\textrm{{ dub.\ }}}
\def\emend{\textrm{{emend.\ }}}
\def\expl{\textrm{{explic.\ }}}   
\def\explicat{\textrm{{explic.\ }}}
\def\fol{\textrm{{fol.\ }}}         
\def\foll{\textrm{{foll.\ }}}
\def\gloss{\textrm{{glossa ad }}}
\def\ins{\textrm{{ins.\ }}}          \def\inseruit{\textrm{{ins.\ }}}
\def\im{{\kern-.7pt\lower-1ex\hbox{\textrm{\tiny{\emph{i.m.}}}\kern0pt}}}
\def\inmargine{{\kern-.7pt\lower-.7ex\hbox{\textrm{\tiny{\emph{i.m.}}}\kern0pt}}}
\def\intextu{{\kern-.7pt\lower-.95ex\hbox{\textrm{\tiny{\emph{i.t.}}}\kern0pt}}}%\textrm{\scriptsize{i.t.\ }}}               
\def\indist{\textrm{{indis.\ }}}          \def\indis{\textrm{{indis.\ }}}
\def\iteravit{\textrm{{iter.\ }}}          \def\iter{\textrm{{iter.\ }}}  
\def\lectio{\textrm{{lect.\ }}}             \def\lec{\textrm{{lect.\ }}}
\def\leginequit{\textrm{{l.n. }}}         \def\legn{\textrm{{l.n. }}}         \def\illeg{\textrm{{l.n. }}}
\def\om{\textrm{{om. }}}
\def\primman{\textrm{{pr.m.}}}
\def\prob{\textrm{{prob.}}}
\def\rep{\textrm{{repetitio }}}
% \def\secundamanu{\textrm{\scriptsize{s.m.}}}
% \def\secm{{\kern-.6pt\lower-.91ex\hbox{\textrm{\tiny{\emph{s.m.}}}\kern0pt}}}%   \textrm{\scriptsize{s.m.}}}
\def\sequentia{\textrm{{seq.\,inv.\ }}}         \def\seqinv{\textrm{{seq.\,inv.\ }}} \def\order{\textrm{{seq.\,inv.\ }}}
\def\supralineam{{\kern-.7pt\lower-.91ex\hbox{\textrm{\tiny{\emph{s.l.}}}\kern0pt}}} %\textrm{\scriptsize{s.l.}}}
\def\interlineam{{\kern-.7pt\lower-.91ex\hbox{\textrm{\tiny{\emph{s.l.}}}\kern0pt}}}   %\textrm{\scriptsize{s.l.}}}
\def\vl{\textrm{v.l.}}   \def\varlec{\textrm{v.l.}} \def\varialectio{\textrm{v.l.}}
\def\vide{\textrm{{cf.\ }}}           \def\cf{\textrm{{cf.\ }}}
\def\videtur{\textrm{{vid.\,ut}}}
\def\crux{\textup{[\ldots]} }
\def\cruxx{\textup{[\ldots]}}
\def\unm{\textit{unm.}}        % unmetrical
%%%%%%%%%%%%%%%%%%%%%%%%%%%%%%%%%%%%



%%% Local Variables:
%%% mode: latex
%%% TeX-master: t
%%% End:

% additions/changes 2024-07-04 mm:
\TeXtoTEIPat{\lb}{<lb/>}
\TeXtoTEIPat{\begin {quote}}{<q>}
  \TeXtoTEIPat{\end {quote}}{</q>}
\TeXtoTEIPat{\begin {enumerate}}{<list rend="numbered">}
  \TeXtoTEIPat{\end {enumerate}}{</list>}
\TeXtoTEI{item}{item}

% additions/changes 2024-07-01 mm:
\TeXtoTEIPat{\unavbl {#1}}{<note type="foliolost">Folio lost in <ref>#1</ref></note>}
\TeXtoTEIPat{\NotIn {#1}}{<note type="omission">Omitted in <ref>#1</ref></note>}
\TeXtoTEI{graus}{span}[type="altrec"]
\TeXtoTEI{grau}{span}[type="altrec"]

% addition 2024-03-15 MD
\TeXtoTEI{manuref}{}

\TeXtoTEIPat{\alphaOne}{α<hi rend="sub">1</hi>}% N3
\TeXtoTEIPat{\alphaTwo}{α<hi rend="sub">2</hi>}% J5
\TeXtoTEIPat{\alphaThree}{α<hi rend="sub">3</hi>}% G4
\TeXtoTEIPat{\betaOne}{β<hi rend="sub">1</hi>}% P11
\TeXtoTEIPat{\betaTwo}{β<hi rend="sub">2</hi>}% C6
\TeXtoTEIPat{\betaOmega}{β<hi rend="sub">ω</hi>}% V3
\TeXtoTEIPat{\gammaOne}{γ<hi rend="sub">1</hi>}% N23
\TeXtoTEIPat{\gammaTwo}{γ<hi rend="sub">2</hi>}% J7
\TeXtoTEIPat{\deltaOne}{δ<hi rend="sub">1</hi>}% V19
\TeXtoTEIPat{\deltaTwo}{δ<hi rend="sub">2</hi>}% K3
\TeXtoTEIPat{\deltaThree}{δ<hi rend="sub">3</hi>}% C7
\TeXtoTEIPat{\deltaOmega}{δ<hi rend="sub">ω</hi>}% J6
\TeXtoTEIPat{\epsilonOne}{ε<hi rend="sub">1</hi>}% P15
\TeXtoTEIPat{\epsilonTwo}{ε<hi rend="sub">2</hi>}% N19
\TeXtoTEIPat{\epsilonThree}{ε<hi rend="sub">3</hi>}% V15
\TeXtoTEIPat{\epsilonFour}{ε<hi rend="sub">4</hi>}% J11
\TeXtoTEIPat{\epsilonOmega}{ε<hi rend="sub">ω</hi>}% N26
\TeXtoTEIPat{\etaOne}{η<hi rend="sub">1</hi>}% V1
\TeXtoTEIPat{\etaTwo}{η<hi rend="sub">2</hi>}% J10
\TeXtoTEIPat{\etaOmega}{η<hi rend="sub">ω</hi>}% N9

% addition 2023-12-11 MD:
\TeXtoTEIPat{\begin {metre}[#1]}{<note type="metre" target="##1">}
\TeXtoTEIPat{\end {metre}}{</note>}
\TeXtoTEIPat{\texttheta}{θ}

% change 2023-12-05 mm
\TeXtoTEI{teimute}{} 

% changes/additions 2023-11-27 MM:
\TeXtoTEIPat{\medialink {#1}{#2}}{<ref target="resources/#2">#1</ref>}

% changes/additions 2023-10-25 MM:
% new Sigla
\TeXtoTEIPat{\textAlpha}{Α}
\TeXtoTEIPat{\textalpha}{α}
\TeXtoTEIPat{\textBeta}{Β}
\TeXtoTEIPat{\textbeta}{β}
\TeXtoTEIPat{\textGamma}{Γ}
\TeXtoTEIPat{\textgamma}{γ}
\TeXtoTEIPat{\textDelta}{Δ}
\TeXtoTEIPat{\textdelta}{δ}
\TeXtoTEIPat{\textEpsilon}{Ε}
\TeXtoTEIPat{\textepsilon}{ε}
\TeXtoTEIPat{\textEta}{Η}
\TeXtoTEIPat{\texteta}{η}
\TeXtoTEIPat{\textChi}{Χ}
\TeXtoTEIPat{\textchi}{χ}
\TeXtoTEIPat{\textOmega}{Ω}
\TeXtoTEIPat{\textomega}{ω}

%new environments
\TeXtoTEIPat{\begin {postmula}[#1]}{<div type="postmula" xml:id="#1">} %%% changed 2024-07-01 mm
  \TeXtoTEIPat{\end {postmula}}{</div>}  %%% changed 2024-07-01 mm
  
\TeXtoTEIPat{\begin {altpostmula}[#1]}{<div type="altrec"><div type="postmula" xml:id="#1">} %%% added 2024-07-03 md
  \TeXtoTEIPat{\end {altpostmula}}{</div></div>} %%% added 2024-07-03 md

\TeXtoTEIPat{\begin {altava}[#1]}{<div type="altrec"><div type="avataranika" xml:id="#1">} %%% changed 2024-07-01 mm
  \TeXtoTEIPat{\end {altava}}{</div></div>} %%% changed 2024-07-01 mm

\TeXtoTEIPat{\sgwit {#1}}{<note type="inlineref"><ref>#1</ref></note>}

% changes/additions 2023-10-12 MM:
\TeXtoTEIPat{\\.}{}

% changes/additions 2023-08-15 MD:
\TeXtoTEIPat{\lineom {#1}{#2}}{<note type="omission">#1 omitted in <ref>#2</ref></note>}
%\TeXtoTEIPat{\startgray}{} %%% changed 2023-12-05 mm; not used 2024-03-26 MD
%\TeXtoTEIPat{\endgray}{} %%% changed 2023-12-05 mm; not used 2024-03-26 MD

% additions/changes 2023-06-05 mm:
%\TeXtoTEIPat{\lineom {#1}}{<note type="omission">Line omitted in <ref>#1</ref></note>}

% additions 2023-04-16 MD:
\TeXtoTEIPat{\,}{ }

% additions 2023-04-13 mm:
\TeXtoTEIPat{\begin {versinnote}}{<lg>}
  \TeXtoTEIPat{\end {versinnote}}{</lg>}

% additions 2023-04-05 MD:
\TeXtoTEIPat{\begin {testimonia}[#1]}{<note type="testimonia" target="##1">}
  \TeXtoTEIPat{\end {testimonia}}{</note>}
\TeXtoTEI{devnote}{s}[xml:lang="sa-deva"]

% app in philcomm und testimonia %%% added MM 2023-12-02
\TeXtoTEI{var}{note}[type="appinnote"]


\TeXtoTEI{anm}{note}[type="memo"] %% change 2023-04-16 MD
\TeXtoTEI{Anm}{note}[type="memo"] %% change 2023-12-05 MM
\TeXtoTEIPat{\startverse}{} %%% marked for change 2023-04-13 mm
\TeXtoTEIPat{\endverse}{} %%% marked for change 2023-04-13 mm
\TeXtoTEIPat{\newpage}{}
\TeXtoTEIPat{\marmas}{ } % changed 2024-03-17 MD
\TeXtoTEIPat{\marma}{}
\TeXtoTEIPat{\vin}{} % added by MD 2023-11-14

%%% modify environments and commands
%%% TEI mapping
% additions/changes 2022-06-07 mm:
\TeXtoTEIPat{ \& }{ &amp; }

% additions/changes 2022-06-01 mm:
\TeXtoTEI{skp}{seg}[type="deva-ignore"]
\TeXtoTEI{skm}{seg}[type="ltn-ignore"]

\TeXtoTEIPat{\rlap {#1}}{#1}

% additions/changes 2022-04-06 mm:
%\TeXtoTEI{sgwit}{ref}
\TeXtoTEI{textdev}{s}[xml:lang="sa-deva"]
\TeXtoTEIPat{\begin {col}[#1]}{<div type="colophon" xml:id="#1">}
  \TeXtoTEIPat{\end {col}}{</div>}
\TeXtoTEIPat{\begin {ava}[#1]}{<div type="avataranika" xml:id="#1">} %%% changed 2024-07-01 mm
  \TeXtoTEIPat{\end {ava}}{</div>} %%% changed 2024-07-01 mm
												   
\TeXtoTEIPat{\outdent}{}
\TeXtoTEIPat{\startaltrecension}{} %%% changed 2023-12-05 mm
\TeXtoTEIPat{\endaltrecension}{} %%% changed 2023-12-05 mm
\TeXtoTEIPat{\startaltnormal}{} % added by MD 2023-11-14 %%% changed 2023-12-05 mm
\TeXtoTEIPat{\endaltnormal}{} % added by MD 2023-11-14 %%% changed 2023-12-05 mm
\TeXtoTEIPat{\begin {alttlg}[#1]}{<div type="altrec"><lg xml:id="#1">}
  \TeXtoTEIPat{\end {alttlg}}{</lg></div>}



% additions/changes 2022-03-12 mm:
\TeXtoTEIPat{\begin {tlg}[#1]}{<lg xml:id="#1">}
  \TeXtoTEIPat{\end {tlg}}{</lg>}

\TeXtoTEIPat{\begin {translation}[#1]}{<note type="translation" target="##1">}
  \TeXtoTEIPat{\end {translation}}{</note>}
\TeXtoTEIPat{\begin {philcomm}[#1]}{<note type="philcomm" target="##1">}
  \TeXtoTEIPat{\end {philcomm}}{</note>}
\TeXtoTEIPat{\begin {sources}[#1]}{<note type="sources" target="##1">}
  \TeXtoTEIPat{\end {sources}}{</note>}


\TeXtoTEIPat{\begin {marma}[#1]}{<note type="marma" target="##1">}
  \TeXtoTEIPat{\end {marma}}{</note>}

\TeXtoTEIPat{\begin {jyotsna}[#1]}{<note type="jyotsna" target="##1">}
  \TeXtoTEIPat{\end {jyotsna}}{</note>}

\EnvtoTEI{description}{list}
\EnvtoTEI{itemize}{list}
\TeXtoTEIPat{\item [#1]}{<label>#1</label>\item}

\TeXtoTEI{tl}{l}
\TeXtoTEI{myfn}{note}[type="myfn"]
\TeXtoTEIPat{\getsiglum {#1}}{<ref target="##1"/>}

\TeXtoTEI{SetLineation}{}
\TeXtoTEI{noindent}{}
\TeXtoTEI{subsection*}{}

\TeXtoTEI{rlap}{}

% end additions/changes
% \TeXtoTEIPat{\skp {#1}}{#1}
% \TeXtoTEIPat{\skm {#1}}{}

\TeXtoTEIPat{\begin {prose}}{<p>}
  \TeXtoTEIPat{\end {prose}}{</p>}

\TeXtoTEIPat{\begin {tlate}}{<p>}
  \TeXtoTEIPat{\end {tlate}}{</p>}

\TeXtoTEI{emph}{hi}
\TeXtoTEI{bigskip}{}
% \TeXtoTEI{/}{|}
\TeXtoTEI{tl}{l}
\TeXtoTEIPat{english}{}
%\TeXtoTEIPat{-}{ } %% change 2023-04-16 MD
%\TeXtoTEIPat{°}{} %% change 2023-04-16 MD
\TeXtoTEIPat{\textcolor {#1}{#2}}{<hi rend="#1">#2</hi>}

% \TeXtoTEIPat{\eyeskip}{}
% \TeXtoTEIPat{\aberratio}{}
% \TeXtoTEIPat{\ad}{}
\TeXtoTEIPat{\add}{<hi rend="italic">add.</hi>} %% change 2023-04-16 MD
% \TeXtoTEIPat{\ann}{}
\TeXtoTEIPat{\ante}{<hi rend="italic">ante</hi> } %% change 2023-04-16 MD
\TeXtoTEIPat{\post}{<hi rend="italic">post</hi> } %% change 2023-04-16 MD
% \TeXtoTEIPat{\codd}{}
% \TeXtoTEIPat{\conj }{}
% \TeXtoTEIPat{\contin}{}
% \TeXtoTEIPat{\corr}{}
% \TeXtoTEIPat{\del}{}
% \TeXtoTEIPat{\dub}{}
% \TeXtoTEIPat{\emend }{}
% \TeXtoTEIPat{\expl}{}
% \TeXtoTEIPat{\ȩxplicat}{}
% \TeXtoTEIPat{\fol}{}
% \TeXtoTEIPat{\gloss}{}
% \TeXtoTEIPat{\ins}{}
% \TeXtoTEIPat{\im}{}
% \TeXtoTEIPat{\inmargine}{}
% \TeXtoTEIPat{\intextu}{}
% \TeXtoTEIPat{\indist}{}
% \TeXtoTEIPat{\iteravit}{}
% \TeXtoTEIPat{\lectio}{}
% \TeXtoTEIPat{\leginequit}{}
% \TeXtoTEIPat{\legn}{}
% \TeXtoTEIPat{\illeg}{<hi rend="italic">illeg.</hi>}
\TeXtoTEIPat{\illeg}{<gap reason="illeg."/>} %%% change 2023-04-11 mm
% \TeXtoTEIPat{\om}{<hi rend="italic">om.</hi>}
\TeXtoTEIPat{\om}{<gap reason="om."/>} %%% change 2023-04-11 mm
% \TeXtoTEIPat{\primman}{}
% \TeXtoTEIPat{\prob}{}
% \TeXtoTEIPat{\rep}{}
% \TeXtoTEIPat{\sequentia}{}
% \TeXtoTEIPat{\supralineam}{}
% \TeXtoTEIPat{\interlineam}{}
\TeXtoTEIPat{\vl}{<hi rend="italic">v.l.</hi>}
% \TeXtoTEIPat{\vide}{}
% \TeXtoTEIPat{\videtur}{}
% \TeXtoTEIPat{\crux}{}
% \TeXtoTEIPat{\cruxxx}{}
\TeXtoTEIPat{\unm}{<hi rend="italic">unm.</hi>}
\TeXtoTEIPat{\lacuna}{<gap reason="lac."/>} % addition 2024-03-24 MD
\TeXtoTEIPat{\lost}{<gap reason="lost"/>} % addition 2024-06-24 MD

% List of Scholars
\DeclareScholar{nos}{nos}[
forename=HPP,
surname=Team]

% Nullify \selectlanguage in TEI as it has been used in
% \DeclareWitness but should be ignored in TEI.
\TeXtoTEI{selectlanguage}{}



% additions/changes 2022-04-06 mm:
%\NewDocumentEnvironment{ava}{O{}}{\begin{ekdpar}\SetLineation{lineation=none}}{\end{ekdpar}}
%\NewDocumentEnvironment{col}{O{}}{\begin{ekdpar}\SetLineation{lineation=none}}{\end{ekdpar}}

% end additions
% added by MM 2022-10-25:
\NewDocumentEnvironment{postmula}{O{}}{
  \begin{ekdverse}
    \hspace{-\vgap}}{
  \end{ekdverse}
  \vskip 0.6\baselineskip
}
% modified by MD 2022-05-8:
\NewDocumentEnvironment{ava}{O{}}{
  \begin{ekdverse}
    \hspace{-\vgap}}{
  \end{ekdverse}
  \vskip 0.6\baselineskip
}
\NewDocumentEnvironment{col}{O{}}{
  \medskip
  \setvnum{col}
%  \selectlanguage{iast}
  \begin{ekdverse}
    \hspace{-\vgap}}{
  \end{ekdverse}
}

        
% modifications/additions by MM 2022-06-07:
\NewDocumentEnvironment{altava}{O{}}{
  \begin{ekdverse}\color{gray}
    \hspace{-\vgap}}{
  \end{ekdverse}
  \vskip 0.6\baselineskip
}   

% end additions

\SetTEIxmlExport{autopar=false}

\NewDocumentEnvironment{tlg}{O{}}{
  \begin{ekdverse}}{
  \end{ekdverse}
  \vskip 0.6\baselineskip}

% additions/changes 2022-08-22 mm:
\NewDocumentEnvironment{alttlg}{O{}}{
%  \stopvline
%  \addtocounter{saved@poemline}{-1}
%  \setvnum{\hindsection.\arabic{saved@poemline}*\arabic{poemline}}
%  \selectlanguage{iast}
  \begin{ekdverse}[type=altrecension]
    \color{gray}
  }{
  \end{ekdverse}
  \vskip 0.6\baselineskip
%  \addtocounter{saved@poemline}{1}
%  \startvline
%  \setvnum{\hindsection.\arabic{poemline}}
%  \selectlanguage{iast}
}

% additions/changes 2022-08-22 mm:
\def\startaltrecension{
  \stopvline
  \addtocounter{saved@poemline}{-1}
  \setvnum{\hindsection.\arabic{saved@poemline}*\arabic{poemline}}
	%\selectlanguage{iast}
	%\begin{ekdverse}[type=altrecension]
	%\color{gray}
	\small  % added 2023-10-12 MD
	}
\def\endaltrecension{
	%\end{ekdverse}
	%\vskip 0.75\baselineskip
  \addtocounter{saved@poemline}{1}
  \startvline
  \setvnum{\hindsection.\arabic{poemline}}
%  \selectlanguage{iast}
	\normalsize  % added 2023-10-12 MD
	}

\def\startaltnormal{
	\stopvline
	\addtocounter{saved@poemline}{-1}
	\setvnum{\hindsection.\arabic{saved@poemline}*\arabic{poemline}}}
\def\endaltnormal{\endaltrecension}



\NewDocumentCommand{\tl}{m}{#1}

%%%%%%

\def\startverse{\begin{ekdverse}} % übergangsweise
\def\endverse{\end{ekdverse}\vskip 0.6\baselineskip} % übergangsweise
\def\startgray{\color{gray}} % NEW! 2023-06-16
\def\endgray{\color{black}} % NEW! 2023-06-16


%%%%%%

\newcommand{\myfn}[1]{\footnote{\texteng{#1}}}
\renewcommand{\thefootnote}{\texteng{\arabic{footnote}}}
\newcommand{\devnote}[1]{\textdev{\scriptsize #1}}
%\newcommand{\outdent}{\hspace{-\vgap}}
\newcommand{\sgwit}[1]{{\footnotesize (\getsiglum{#1})}}
\newcommand{\NotIn}[1]{\texteng{\footnotesize (om. \getsiglum{#1})}}
\newcommand{\lineom}[2]{\texteng{\footnotesize (#1 om. \getsiglum{#2})}}
\newcommand{\grau}[1]{\textcolor{gray}{#1}} % partial altrecension
\newcommand{\graus}[1]{\small\textcolor{gray}{#1}\normalsize} % partial altrecension
\newcommand{\Anm}[1]{\begin{ekdverse}
	\texteng{\footnotesize (#1)}
	\end{ekdverse}
	\vskip 0.6\baselineskip}
\newcommand{\anm}[1]{\texteng{\footnotesize [#1]}}

\def\om{\texteng{\emph{om.\kern-0.8ex}}}
\def\illeg{\texteng{\emph{illeg.\kern-0.8ex}}} 
\def\damaged{\texteng{\emph{damaged}}} 
\def\unm{\texteng{\emph{unm.\ }}}
\def\gap{\texteng{\emph{gap}}}
%\def\recte{\texteng{r.\:}}
%\def\for{\texteng{for\ }}
%\def\sic{\texteng{\emph{sic}}}
%\def\oder{\texteng{\emph{or\ }}}
\def\ante{\texteng{\normalfont\emph{ante\ }}}
\def\add{\texteng{\normalfont\emph{add.}}}
\def\post{\texteng{\normalfont\emph{post\ }}}
\def\antecorr{\texteng{\textsubscript{ac}}}
\def\postcorr{\texteng{\textsubscript{pc}}}
\def\marma{\texteng{\textsuperscript{\#}}}
\def\marmas{\texteng{\textsuperscript{\#}} }
\def\crux{\texteng{\textsuperscript{\textdagger}}}


\usepackage{textgreek}

%%% Gr1,4b,6
\DeclareWitness{N3}{\texteng{\textalpha\textsubscript{1}}}{NGMPP B 62-20}[]
        \DeclareHand{N3ac}{N3}{\texteng{\textalpha\rlap{\textsubscript{1}}\textsuperscript{ac}}}[]
        \DeclareHand{N3pc}{N3}{\texteng{\textalpha\rlap{\textsubscript{1}}\textsuperscript{pc}}}[]
\DeclareWitness{J5}{\texteng{\textalpha\textsubscript{2}}}{Jodhpur 02235}[]
\DeclareWitness{G4}{\texteng{\textalpha\textsubscript{3}}}{GOML 18885}[]% Telugu script
\DeclareWitness{N24}{\texteng{\textalpha\textsubscript{4}}}{NGMPP G 190-16}[]
\DeclareWitness{Gr1r}{\texteng{\textAlpha *}}{Gr1 reconstructed}[]

\DeclareWitness{P11}{\texteng{\textbeta\textsubscript{1}}}{}[]
\DeclareWitness{C6}{\texteng{\textbeta\textsubscript{2}}}{Lalchand M-2089}[]

\DeclareWitness{V3}{\texteng{\textbeta\textsubscript{\textomega}}}{Sampurnananda Library Sarasvati Bhavan 29899}[]

%%% Gr2

\DeclareWitness{N23}{\texteng{\textgamma\textsubscript{1}}}{NGMPP G 25-2}[]
        \DeclareHand{N23ac}{N23}{\texteng{\textgamma\rlap{\textsubscript{1}}\textsuperscript{ac}}}[]
        \DeclareHand{N23pc}{N23}{\texteng{\textgamma\rlap{\textsubscript{1}}\textsuperscript{pc}}}[]
\DeclareWitness{J7}{\texteng{\textgamma\textsubscript{2}}}{Jodhpur 02241}[]
%\DeclareWitness{V6}{\texteng{V\textsubscript{6}}}{Sampurnananda Library Sarasvati Bhavan 29991}[]
\DeclareWitness{K1}{\texteng{K\textsubscript{1}}}{Raghunātha Temple Library 4383}[settlement=Jammu]
        \DeclareWitness{K1ac}{\texteng{K\rlap{\textsubscript{1}}\textsuperscript{ac}\space}}{}[]
        \DeclareWitness{K1pc}{\texteng{K\rlap{\textsubscript{1}}\textsuperscript{pc}\space}}{}[]


%%% Gr3

\DeclareWitness{V19}{\texteng{\textdelta\textsubscript{1}}}{Sampurnananda Library Sarasvati Bhavan 30069}[]
\DeclareWitness{K3}{\texteng{\textdelta\textsubscript{2}}}{Privat collection}
\DeclareWitness{C7}{\texteng{\textdelta\textsubscript{3}}}{Lalchand M-6494}[]
%\DeclareWitness{C1}{\texteng{C\textsubscript{1}}}{Lalchand M-2080}[]%L1 And C1 very close (and come from same region)
%\DeclareWitness{P23}{\texteng{P\textsubscript{23}}}{}[]
%\DeclareWitness{L1}{\texteng{L\textsubscript{1}}}{SOAS RE 43454}[settlement=Jammu]

\DeclareWitness{J6}{\texteng{\textdelta\textsubscript{\textomega}}}{Jodhpur 02237}[]
        \DeclareHand{J6ac}{J6}{\texteng{\textdelta\rlap{\textomega}\textsuperscript{ac}}}[]
        \DeclareHand{J6pc}{J6}{\texteng{\textdelta\rlap{\textomega}\textsuperscript{pc}}}[]

%%% Gr4c

\DeclareWitness{P15}{\texteng{\textepsilon\textsubscript{1}}}{}[]
\DeclareWitness{N19}{\texteng{\textepsilon\textsubscript{2}}}{NGMPP E-1528-1 / E-1527-7(4)}[]
\DeclareWitness{V15}{\texteng{\textepsilon\textsubscript{3}}}{Sampurnananda Library Sarasvati Bhavan 30051}[]
        \DeclareHand{V15ac}{V15}{\texteng{\textepsilon\rlap{\textsubscript{3}}\textsuperscript{ac}}}[]
        \DeclareHand{V15pc}{V15}{\texteng{\textepsilon\rlap{\textsubscript{3}}\textsuperscript{pc}}}[]
\DeclareWitness{J11}{\texteng{\textepsilon\textsubscript{4}}}{Jodhpur 23532}[]
        \DeclareHand{J11ac}{J11}{\texteng{\textepsilon\rlap{\textsubscript{4}}\textsuperscript{i.t.}}}[]
        \DeclareHand{J11pc}{J11}{\texteng{\textepsilon\rlap{\textsubscript{4}}\textsuperscript{mg.}}}[alternative reading written by the first hand in margin or interlinearly (J11)]
%\DeclareWitness{J14}{\texteng{\textepsilon\textsubscript{5}}}{Jodhpur 02239}[]

%\DeclareWitness{L2}{\texteng{L\textsubscript{2}}}{Wellcome Collection O.36]}
\DeclareWitness{M1}{\texteng{M\textsubscript{1}}}{P-5682/4}[]

\DeclareWitness{N26}{\texteng{\textepsilon\textsubscript{\textomega}}}{NGMPP}[]
%\DeclareWitness{V17}{\texteng{\textepsilon\textsubscript{\textomega 3}}}{Sampurnananda Library Sarasvati Bhavan 30053}[]

\DeclareWitness{V1}{\texteng{\texteta\textsubscript{1}}}{Sampurnananda Library Sarasvati Bhavan 30109}[]
        \DeclareHand{V1ac}{V1}{\texteng{\texteta\rlap{\textsubscript{1}}\textsuperscript{ac}}}[]
        \DeclareHand{V1pc}{V1}{\texteng{\texteta\rlap{\textsubscript{1}}\textsuperscript{pc}}}[]

%%% Gr4d

\DeclareWitness{J10}{\texteng{\texteta\textsubscript{2}}}{MSPP Jodhpur 2230}[]
        \DeclareHand{J10ac}{J10}{\texteng{\texteta\rlap{\textsubscript{2}}\textsuperscript{ac}}}[]
        \DeclareHand{J10pc}{J10}{\texteng{\texteta\rlap{\textsubscript{2}}\textsuperscript{pc}}}[]

\DeclareWitness{N9}{\texteng{\texteta\textsubscript{\textomega}}}{NGMPP A62-33}[]
%\DeclareWitness{J15}{\texteng{\textepsilon\textsubscript{\textomega 4}}}{Jodhpur 9732A}[]

%%%

\DeclareWitness{Jyo}{\texteng{\textchi}}{Brahmānanda's version}[]
%\DeclareWitness{Tue}{\texteng{Tü}}{Ma I 339}[]

\DeclareWitness{ceteri}{\texteng{cett.}}{ceteri}[]

%%% Group Sigla

\DeclareWitness{Gr1}{\texteng{\textAlpha}}{N3,J5,G4}

\DeclareWitness{Gr2}{\texteng{\textGamma}}{N23,J7}
%\DeclareWitness{Gr2}{\texteng{%
%	\textbeta\textsubscript{1}%
%	\textbeta\textsubscript{2}%
%	}}{N23,J7}
\DeclareWitness{Gr3a}{\texteng{\textDelta}}{V19,K3,C7}
\DeclareWitness{Gr4b}{\texteng{%
	\textbeta\textsubscript{1}%
	\textbeta\textsubscript{2}%
	}}{C6,P11}
\DeclareWitness{GrB}{\texteng{%
	\textbeta\textsubscript{1}%
	\textbeta\textsubscript{2}%
	\textbeta\textsubscript{\textomega}%
	}}{C6,P11,V3}
\DeclareWitness{Gr4c}{\texteng{\textEpsilon}}{P15,N19,V15}

% \DeclareWitness{Gr4d}{\texteng{%
	% \texteta\textsubscript{1}%
	% \texteta\textsubscript{2}%
	% }}{V1,J10}
\DeclareWitness{Gr6}{\texteng{\textOmega}}{V3,J6,N9,N26}

\makepagestyle{HPed}
\makeoddhead{HPed}{\small\texteng{HP4}}{}{\small\texteng{\today}}
\makeevenhead{HPed}{\small\texteng{HP4}}{}{\small\texteng{\today}}
\makeoddfoot{HPed}{}{\small\texteng{\thepage}}{}
\makeevenfoot{HPed}{}{\small\texteng{\thepage}}{}
\def\hindsection{4}

% Chp. 4 - N3,J5,G4; P11,C6,V3; N23,J7; V19,E2,C7(partly); G11,N19,V15; J10,Jyo

\begin{document}
\pagestyle{HPed}
\begin{otherlanguage}{iast}
\begin{ekdosis}


\teimute{\setcounter{saved@poemline}{1}}
\begin{ava}[hp04_001a]
atha samādhiḥ/ \sgwit{N3,J5,P11,C6}% G4 broken
\end{ava}

\startaltnormal
\begin{alttlg}[hp04_000_1]
\tl{
\pada{\app{\lem[wit={ceteri}]{namaḥ}
	\rdg[wit={V3,N23,E2}]{oṃ namaḥ}% +K3,C7
	} śivāya gurave}
\pada{nādabindu%
	\app{\lem[wit={C6,Gr2,Gr3a,G5,J10,Jyo}]{kalātmane}
		\rdg[wit={P11,V3,G11,N19,V15}]{layātmane}% +M3,F; °tmano? V3
		}/}\\+}
\tl{
\pada{\app{\lem[wit={ceteri}]{nirañjanapadaṃ}
		\rdg[wit={V3}]{nirañjanaṃ padaṃ}
		\rdg[wit={N23},alt={\om}]{\skp{\om}}}
	\app{\lem[wit={ceteri}]{yāti}
		\rdg[wit={GrB,G5,N19}]{yānti}}}
\pada{\app{\lem[wit={GrB,J7,V19,G11,G5,N19,V15,Jyo}]{nityaṃ}
		\rdg[wit={N23}]{aharniśaṃ}
		\rdg[wit={J10}]{yato}
		\rdg[wit={E2}]{yatra}} % +K3,C7
	\app{\lem[wit={P11,V3,V19,G11,G5,V15}]{yatra}
		\rdg[wit={Gr2,N19}]{yatna}% yatta? N23; +F
		\rdg[wit={Jyo}]{tatra}
		\rdg[wit={C6}]{ca yat}
		\rdg[wit={J10}]{yogī}
		\rdg[wit={E2}]{nityaṃ}% +K3,C7
		}%
	\app{\lem[wit={V3,Gr2,Gr3a,V15,J10,Jyo}]{parāyaṇaḥ}
		\rdg[wit={P11,C6,G11,G5,N19}]{parāyaṇāḥ}%##
		}//}\\!}
\end{alttlg}


\begin{alttlg}[hp04_000_2]
\tl{
\pada{\app{\lem[wit={ceteri}]{athedānīṃ} % °dāniṃ V15
		\rdg[wit={V3}]{athodānī}
		\rdg[wit={N23}]{athekṣanīṃ}}
	pravakṣyāmi} % vakṣāmi N19,N23,V19,V3
\pada{samādhikrama%
	\app{\lem[wit={GrB,G11,G5,N19,V15,J10,Jyo},alt={°m uttamam}]{\skp{°}m uttamam}
		\rdg[wit={Gr2,Gr3a}]{lakṣaṇam}}/}\\+}
\tl{
\pada{mṛtyughnaṃ % ghaṃ N23
	\app{\lem[wit={GrB,Gr2,E2,G11,G5}]{tu}
		\rdg[wit={N19,V15,J10,Jyo}]{ca}
		\rdg[wit={V19}]{su}} sukhopāyaṃ} % muṣopāpaṃ N23, mukho° N19
\pada{brahmānandakaraṃ 
	\app{\lem[wit={ceteri}]{param}
		\rdg[wit={G5}]{sadā}}//}\\!}% brahmanāṃda J7ac; para J10ac
\end{alttlg}


%\startaltrecension
%\teimute{\small}
\begin{alttlg}[hp04_000_3]
\tl{%\vin
\pada{\app{\lem[wit={G11,V15,Jyo}]{rājayogaḥ}
		\rdg[wit={C6,G5,N19,J10}]{rājayoga}% +F
		}
	\app{\lem[wit={C6,G11,G5,J10,Jyo}]{samādhiś ca}% +G7
		\rdg[wit={N19,V15}]{samādhiḥ syād}% samādhi V15; +F,N22,P6
		}}
\pada{\app{\lem[wit={ceteri}]{unmanī}
	\rdg[wit={G11}]{py unmanī}} ca manonmanī/}\\+}
\tl{%\vin
\pada{\app{\lem[wit={V15,J10}]{amaraugho}
		\rdg[wit={G11}]{amaraughā}
		\rdg[wit={C6}]{amaraughi}
		\rdg[wit={N19}]{avaraubhū}
		\rdg[wit={Jyo}]{amaratvaṃ}
		\rdg[wit={G5}]{aromaro}}
	\app{\lem[wit={C6,G11,N19,J10,Jyo},alt={layas}]{laya\skp{s}}
		\rdg[wit={V15}]{layes}
		\rdg[wit={G5}]{yas tat}}%
	\app{\lem[wit={C6,G11,N19,V15,Jyo},alt={tattvaṃ}]{\skm{s }tattvaṃ}
		\rdg[wit={J10}]{tatra}
		\rdg[wit={G5}]{tulyaḥ}}}
\pada{\app{\lem[wit={G11,N19,V15,J10,Jyo}]{śūnyāśūnyaṃ} % °śūnya N19
		\rdg[wit={C6}]{śūnyāc chūnyaṃ}
		\rdg[wit={G5}]{śūnyāt śūnya}}
		paraṃ padam//}\label{synonym3}
	\sgwit{C6,G11,G5,N19,V15,J10,Jyo} \anm{\textleftarrow\ \ref{A1}}\\!}
\end{alttlg}

\begin{alttlg}[hp04_000_4]
\tl{%\vin
\pada{amanaskaṃ tathādvaitaṃ}
\pada{nirālambaṃ nirañjanam/}\\+}
\tl{%\vin
\pada{jīvanmuktiś ca % jīvamuktiś N19, °muktaś G5
	\app{\lem[wit={C6,G11,G5,N19,J10}]{sahajaṃ}
		\rdg[wit={Jyo}]{sahajā}
		\rdg[wit={V15},alt={\om}]{\skp{\om}}}}
\pada{\app{\lem[wit={C6,V15}]{turyaṃ} % +J11
		\rdg[wit={G5}]{tulyaṃ}
		\rdg[wit={N19}]{turyai}
		\rdg[wit={Jyo}]{turyā}
		\rdg[wit={G11}]{turīyaṃ}
		\rdg[wit={J10}]{muktiś}}
	\app{\lem[wit={J10pc,Jyo}]{cety ekavācakāḥ}
		\rdg[wit={J10ac},alt={°kaḥ}]{cety ekavācakaḥ}
		\rdg[wit={C6,G5}]{cety ekavācakam}% +F; caitye° C6
		\rdg[wit={V15}]{cittaikavācakam}% +N22,P6
		\rdg[wit={N19}]{ciṃtaikavācakam}
		\rdg[wit={G11}]{caikavācakaṃ}
		}//}\label{synonym4}
\sgwit{C6,G11,G5,N19,V15,J10,Jyo} \anm{\textleftarrow\ \ref{A2}}%
\myfn{\getsiglum{C6} has these verses on synonyms both here and at \ref{A1}/\ref{A2}, but \getsiglum{P11} has them at the latter place only.}\\!}
\end{alttlg}
%\endaltrecension
%\teimute{\normalsize}

\begin{alttlg}[hp04_000_5]
\tl{
\pada{salile saindhavaṃ
	\app{\lem[wit={ceteri},alt={yadvat}]{yadva\skp{t}}
		\rdg[wit={N19}]{tadvat}}t}
\pada{sāmyaṃ \app{\lem[wit={C6,Gr2,Gr3a,J10,Jyo}]{bhajati}
		\rdg[wit={V3}]{bhajata}
		\rdg[wit={G11,G5,N19,V15}]{bhavati}% +F
		\rdg[wit={P11}]{ttadgati}} yogataḥ/}\\+} % °ta N19
\tl{
\pada{\app{\lem[wit={ceteri}]{tathā}
		\rdg[wit={V3}]{athā}
		\rdg[wit={J10}]{yathā}}%
	\app{\lem[wit={ceteri},alt={°tmamanasor}]{\skp{°}tmamanaso\skp{r}} % °saur N23
		\rdg[wit={J10}]{tmānamanor}}r aikyaṃ} % ma om. P7
\pada{samādhiḥ % °dhi P11, °dhir J10,Jyo
	\app{\lem[wit={ceteri}]{so}
		\rdg[wit={P11}]{sā}
		\rdg[wit={G11,G5,J10,Jyo}]{a°}}%
	\app{\lem[wit={ceteri}]{'bhidhīyate}
		\rdg[wit={N19}]{'bhidhīte}
		\rdg[wit={N23}]{vidhīyate}}//}\label{salile}\\!}
\end{alttlg}

\Anm{\getsiglum{G11plus,Jyo} have \ref{visvarupa} \textit{yadā saṃkṣīyate prāṇo} here%
\myfn{In the following, not all of the differences in the verse order of \getsiglum{GrB} and \getsiglum{Jyo} are noted. \getsiglum{GrB} follow the order of \getsiglum{Gr2} (or of \getsiglum{Gr3a}?) in the beginning and the end (after 4.72). The middle part is a kind of mix of \getsiglum{Gr2} and \getsiglum{N19,V15}. The verse order of \getsiglum{Jyo} is similar to that of \getsiglum{N19,V15}, but with many small differences.}}%

\newpage
%\startaltrecension
%\teimute{\small}
\begin{alttlg}[hp04_000_6]
\tl{%\vin
\pada{\app{\lem[wit={N19,V15}]{yat samatvaṃ dvayor eva}
	\rdg[wit={G11,G5}]{tat samatvaṃ dvayor atra}
	\rdg[wit={J10,Jyo}]{tat samaṃ ca dvayor aikyaṃ}}} % jat J10pc
\pada{jīvātmaparamātmanoḥ/}\\+} % ātmā G11
\tl{%\vin
\pada{\app{\lem[wit={G5,N19,V15,J10}]{samastanaṣṭa}
	\rdg[wit={G11}]{samastaṃ naṣṭa}
	\rdg[wit={Jyo}]{pranaṣṭasarva}}%
\app{\lem[wit={G11,G5,V15,Jyo}]{saṃkalpaḥ}
	\rdg[wit={N19,J10}]{saṃkalpa}}}
\pada{samādhiḥ so'bhidhīyate//} % sā G5
\sgwit{G11,G5,N19,V15,J10,Jyo}\label{yatsamatvam}%
\myfn{\getsiglum{J10} inserts another similar verse here:
\textit{karpūraṃ salile yadvat saindhavaṃ salile yathā |
tathātmamanasor aikyaṃ samādhiḥ so'bhidhīyate ||} (cf. \ref{karpura}ab and 4.3cd)}\\!}
\end{alttlg}
%\endaltrecension
%\teimute{\normalsize}

%\newpage
\begin{alttlg}[hp04_000_7]
\tl{
\pada{rājayogasya
	\app{\lem[wit={ceteri}]{māhātmyaṃ}
		\rdg[wit={J7}]{māhatmyaṃ}
		\rdg[wit={V15}]{mahā}}}
\pada{ko vā jānāti tattvataḥ/}\\+} % ki N19; cā P11; jānaṃti N19
\tl{
\pada{\app{\lem[wit={ceteri},alt={jñānān}]{jñānā\skp{n}}
		\rdg[wit={V15,J10}]{jñāna}
		\rdg[wit={Jyo}]{jñānaṃ}
		\rdg[wit={V19}]{jñān}}%
	\app{\lem[wit={C6,Gr2,E2,Jyo},alt={muktiḥ}]{\skm{n }muktiḥ}% +K1
		\rdg[wit={P11,V3,V19,G11,G5,N19,V15,J10}]{mukti}}
	\app{\lem[wit={G11,G5}]{sthirā}% +M3
		\rdg[wit={V3,N19}]{sthite}% +F,K1
		\rdg[wit={P11}]{sthitai}% sthitā P6
		\rdg[wit={C6,Gr2,E2,J10,Jyo}]{sthitiḥ}
		\rdg[wit={V19}]{sthiti<<ḥ>>}
		\rdg[wit={V15}]{°s tato}}
	\app{\lem[wit={P11,C6,N19,V15,Jyo},alt={siddhir}]{siddhi\skp{r}}% +F,K1
		\rdg[wit={V3,J10}]{siddhi}
		\rdg[wit={Gr2,Gr3a,G11,G5}]{siddhā}}r}
\pada{guru\app{\lem[wit={ceteri}]{vākyena} % gurur P11
		\rdg[wit={N23}]{vākyāt <<pra>>}}
	\app{\lem[wit={ceteri}]{labhyate}
		\rdg[wit={J10}]{sidhyati}}//}\\!}
\end{alttlg}


\begin{alttlg}[hp04_000_8]
\tl{
\pada{durlabho viṣayatyāgo} % durlabha C6; viṣayāt* yogo N23
\pada{durlabhaṃ tattvadarśanam/}\\+} % labha V3
\tl{
\pada{durlabhā sahajāvasthā}% dullā G11, tu sahāvasthā G5
\pada{sadguroḥ karuṇāṃ vinā//}% guro P11,V3; karuṇā P11,N19
\label{durlabho}\\!}
\end{alttlg}

%\newpage
\Anm{\getsiglum{G11} has \ref{kastha} \textit{kāṣṭhagoṣṭhīprapañcena} here}

\Anm{\getsiglum{G11,G5,N19,V15,J10} have \ref{yavan} \textit{yāvan naiva praviśati} here}

\begin{alttlg}[hp04_000_9]
\tl{
\pada{vividhair % vicitrair G5
	āsanaiḥ % āsanai P11, āsanaḥ V15
	kumbhai}% °bhai N23,C6
\pada{\app{\lem[wit={C6,G5,Jyo},alt={vicitraiḥ}]{\skm{r }vicitraiḥ}% +K3,M3,F
		\rdg[wit={P11,V3,Gr2,Gr3a,N19,V15,J10}]{vicitra}
		\rdg[wit={G11}]{citraiś ca}}
	\app{\lem[wit={GrB,Gr3a,G11,G5,J10,Jyo}]{karaṇair api}
		\rdg[wit={J7}]{karuṇair api}
		\rdg[wit={N23}]{kalaṇair api}
		\rdg[wit={N19,V15}]{karaṇair atha}}/}\\+}
\tl{
\pada{\app{\lem[wit={ceteri},alt={prabuddhāyām}]{prabuddhāyā\skp{m}}% pravudhā° V19
		\rdg[wit={N19}]{pradhadhāyām}}%
	\app{\lem[wit={ceteri},alt={ādi}]{\skm{m }ādi}
		\rdg[wit={V15}]{idaṃ}
		\rdg[wit={Jyo}]{mahā}}%
	\app{\lem[wit={ceteri}]{śaktau} % proktau J10ac
		\rdg[wit={N23}]{śaktiḥ}}}
\pada{prāṇaḥ śūnye % prāṇaṃ C6, prāṇa V15; sampra° G5
	\app{\lem[wit={C6,N23,Gr3a,G11,G5,J10}]{vilīyate}
		\rdg[wit={J7}]{vidhīyate}
		\rdg[wit={P11,V3,N19,V15,Jyo}]{pralīyate}% +F
		}//}\\!}
\end{alttlg}


\begin{alttlg}[hp04_000_10]
\tl{
\pada{\app{\lem[wit={ceteri}]{utpanna}
		\rdg[wit={V19}]{utpannā}
		\rdg[wit={N23}]{ut<<pan>>na}}%
	\app{\lem[wit={ceteri}]{śaktibodhasya}
		\rdg[wit={N23}]{śaktibodhaḥ syāt}
		\rdg[wit={V15}]{śaktibodhaś ca}}}
\pada{\app{\lem[wit={ceteri}]{tyakta}
		\rdg[wit={N23}]{prakṣa}}%
	niḥśeṣakarmaṇaḥ/} \lineom{ab}{C6}\\+}
\tl{
\pada{\app{\lem[wit={ceteri}]{yoginaḥ}
	\rdg[wit={C6}]{yogināṃ}} sahajāvasthā}
\pada{svaya%m
	\app{\lem[wit={P11,V3,G11,G5,V15,J10},alt={eva prakāśate}]{\skm{m }eva prakāśate}% +K1; prakāśyate N22,P6
		\rdg[wit={N19}]{eva prakāśayet}
		\rdg[wit={C6,Gr2,Gr3a,Jyo}]{eva prajāyate}
		}//}\\!}
\end{alttlg}

\newpage
\begin{alttlg}[hp04_000_11]
\tl{
\pada{suṣumṇā\app{\lem[wit={ceteri}]{vāhini}% °<<m>>ṇā N23
		\rdg[wit={V3,N23,G5,N19}]{vāhinī}
		\rdg[wit={V19}]{vāhi}}
	\app{\lem[wit={ceteri}]{prāṇe}
		\rdg[wit={V3}]{prāṇa}}}
\pada{\app{\lem[wit={P11,G11,G5,V15}]{śūnyaṃ}
		\rdg[wit={J10}]{śūnya}% śūnyaṃ J10pc
		\rdg[wit={C6,Gr2,Gr3a,Jyo}]{śūnye}% +K1,N22,P6
		\rdg[wit={V3}]{śūne}
		\rdg[wit={N19}]{śūnyā}} 
	\app{\lem[wit={ceteri}]{viśati}
		\rdg[wit={P11}]{vasati}}
	\app{\lem[wit={P11,V3,G11,Jyo}]{mānase} % +M1,M3
		\rdg[wit={J10}]{mārutaḥ}
		\rdg[wit={C6,Gr2,Gr3a,G5,N19,V15}]{mārute}}\marma/}\\+} % mārutai N23; +K1,N22,P6,G7
\tl{
\pada{\app{\lem[wit={Gr2,Gr3a,G11}]{tathā}
		\rdg[wit={GrB,G5,N19,V15,J10,Jyo}]{tadā}}
	\app{\lem[wit={ceteri}]{samasta}
		\rdg[wit={J10,Jyo}]{sarvāṇi}}karmāṇi}
\pada{\app{\lem[wit={ceteri}]{nirmūlayati}
		\rdg[wit={V19,V15}]{nimūlayati}
		\rdg[wit={N23}]{nirmūlaṃ yāti}
		\rdg[wit={G5}]{nirmalaṃ yāti}} % °ḷayati V15
	\app{\lem[wit={GrB,G11,N19,J10}]{marmavit}% K1,M1,G7
		\rdg[wit={N23,G5,V15}]{karmavit}% P6,P22,G5,M3,F
		\rdg[wit={J7}]{karmakṛt}
		\rdg[wit={Gr3a,Jyo}]{yogavit}
		}//}\\!}
\end{alttlg}

%\newpage
\begin{alttlg}[hp04_000_12]
\tl{
\pada{\app{\lem[wit={G11}]{amareśa}
		\rdg[wit={V3,G5,V15}]{amaraugha}% +F
		\rdg[wit={P11,N19}]{amarogha}% +M3?
		\rdg[wit={C6}]{amaraughi}
		\rdg[wit={J10,Jyo}]{amarāya}% +C2
		\rdg[wit={Gr2}]{amano nir°}
		\rdg[wit={Gr3a}]{amalo nir°}} % verse om. K1,P6
	\app{\lem[wit={GrB,G11,G5,N19,V15,J10,Jyo}]{namas tubhyaṃ}
	\rdg[wit={Gr2}]{°manāḥ śūnyaṃ}
	\rdg[wit={Gr3a}]{°malaḥ śūnyaṃ}}}
\pada{so'pi \app{\lem[wit={C6,V3,G11,N19,Jyo}]{kālas tvayā}% trayā V3
	\rdg[wit={P11}]{kālaṃ tvayā}
	\rdg[wit={V15}]{kāla tvayā}
	\rdg[wit={J10}]{kālantayā}
	\rdg[wit={G5}]{kālasya vā°}
	\rdg[wit={Gr2,Gr3a},alt={\om}]{\skp{\om}}}
\app{\lem[wit={GrB,G11,N19,V15,J10}]{hataḥ}% hata N19
	\rdg[wit={G5}]{°hakaḥ}
	\rdg[wit={Jyo}]{jitaḥ}
	\rdg[wit={Gr2,Gr3a},alt={\om}]{\skp{\om}}}/}\\+}
\tl{
\pada{patitaṃ \app{\lem[wit={GrB,G11,G5,N19,V15,Jyo}]{vadane}
	\rdg[wit={J10}]{pavane}
	\rdg[wit={Gr2,Gr3a},alt={\om}]{\skp{\om}}
	} yasya} % yasyā P11
\pada{jagad etac carācaram//} % yagad V19; carāccaraṃ J7, °care P11
	\lineom{bc}{Gr2,Gr3a}\\!}
\end{alttlg}

\begin{alttlg}[hp04_000_13]
\tl{
\pada{citte
	\app{\lem[wit={GrB,J7,Gr3a,G11,G5,J10,Jyo},alt={samatvam}]{samatva\skp{m}}
		\rdg[wit={N19,V15}]{śamatvam}
		\rdg[wit={N23}]{samatyam}}m āpanne}
\pada{\app{\lem[wit={J7,Gr3a,G11,G5,N19,Jyo}]{vāyau}
		\rdg[wit={V15}]{vāyo}
		\rdg[wit={V3,N23}]{vāyor}
		\rdg[wit={C6,J10}]{vāyur}
		\rdg[wit={P11}]{vāyu}}
	\app{\lem[wit={ceteri}]{vrajati}
	\rdg[wit={N23}]{javati}} madhyame/}\\+}
\tl{
\pada{\app{\lem[wit={G11,G5,N19}]{tadāmaraugha}
		\rdg[wit={P11,V3}]{eṣāmaraugha}
		\rdg[wit={V15}]{tadāmaroḷi}
		\rdg[wit={Jyo}]{tadāmarolī}
		\rdg[wit={J10}]{tathāmarolī}
		\rdg[wit={C6}]{saivāmarolī}
		\rdg[wit={V19}]{eṣā naulīti}
		\rdg[wit={E2}]{eṣā naulī ca}}% +C7
	\app{\lem[wit={GrB,V19,G5,N19,J10,Jyo}]{vajrolī}% +G5,M3,K3,C7
		\rdg[wit={G11}]{vajrolīs}
		\rdg[wit={V15}]{vajrolis}
		\rdg[wit={E2}]{vajrī ca}}}
\pada{\crux\app{\lem[wit={G11,G5,N19,V15}]{tadāśājīvite'pi ca}% +M3
		\rdg[wit={GrB}]{sadā me bhimateti ca} % bhimate cita V3; timateti vaḥ P11
		\rdg[wit={Gr3a}]{sadā cābhimateti ca}
		\rdg[wit={J10}]{sahajolī mato pi ca}
		\rdg[wit={Jyo}]{sahajolī prajāyate}}\crux//}
		\lineom{cd}{Gr2}\\!}
\end{alttlg}
% G7 tadha manā vajroci tādhāśrājiṃtasya ca (*12 om.)
% M3 tato maśā .. vajraulī tadāśājīvitepi ca
% K1 eṣāmarolī vajrolī sadā abhinayāti ca
% P6 eṣāmaroli vajroli sadā abhinayaṃti ca (*12 om.)
% N22 eṣāmaroli vajroli sadā savitayiti ca (*12 om.)
% P11 eṣāmaraughavajrolī sadā me timateti vaḥ


%\newpage
\begin{alttlg}[hp04_000_14]
\tl{
\pada{jñānaṃ 
	\app{\lem[wit={ceteri}]{kuto}
		\rdg[wit={G11}]{tato}} manasi
	\app{\lem[wit={GrB,Gr2,Gr3a,J10}]{jīvati devi yāvat} % jāvat V19, yā<<va>>t N23
		\rdg[wit={G11,N19}]{jīvati devi tāvat}
		\rdg[wit={G5}]{jīvati tepi tāvat}
		\rdg[wit={Jyo}]{saṃbhavatīha tāvat}
		\rdg[wit={V15}]{jīvati durvikalpe}}}\\+}
\tl{
\pada{\app{\lem[wit={ceteri}]{prāṇo'pi}
		\rdg[wit={C6,G11,V15}]{prāṇe pi}
		\rdg[wit={G5}]{prāṇeha}} jīvati mano
	\app{\lem[wit={ceteri}]{mriyate} % sriyate N23, mrīyate P7
		\rdg[wit={J7,V19}]{mṛyate}
		\rdg[wit={V15}]{miyata}
		\rdg[wit={G5}]{priyate}}
	\app{\lem[wit={ceteri}]{na}
		\rdg[wit={N19}]{ca}}
	\app{\lem[wit={GrB}]{tāvat}
		\rdg[wit={ceteri}]{yāvat}% +F,G5
		}/}\\+}
\tl{
\pada{\app{\lem[wit={ceteri}]{prāṇo}
		\rdg[wit={Gr3a}]{prāṇaṃ}}
	\app{\lem[wit={ceteri}]{mano}
		\rdg[wit={G11,G5,N19}]{'pi ca}
		} dvayam idaṃ % dvayām N23
	\app{\lem[wit={ceteri}]{vilayaṃ}
		\rdg[wit={V15}]{na vilī°}}
	\app{\lem[wit={P11,C6}]{prayāti}% +K1,P6,F,source
		\rdg[wit={V3}]{prajāti}
		\rdg[wit={J10}]{na yāti}% +G7,K3
		\rdg[wit={N19}]{na yāvat}
		\rdg[wit={G5}]{na yattat}
		\rdg[wit={Gr3a,Jyo}]{nayed yo}
		\rdg[wit={J7}]{naved yo}
		\rdg[wit={N23}]{jayed  yo}
		\rdg[wit={G11}]{nayet taṃ}% nayete M3; gate cen M1
		\rdg[wit={V15}]{°yate tra}}}\\+}
\tl{
\pada{mokṣaṃ
	\app{\lem[wit={ceteri}]{sa}
		\rdg[wit={V15}]{na}
		\rdg[wit={C6}]{ca}} gacchati % gacchatiti V19
	naro na kathaṃci%d % narā N23
	\app{\lem[wit={ceteri},alt={anyaḥ}]{\skm{d }anyaḥ}
		\rdg[wit={G5}]{anyam}
		\rdg[wit={J10}]{anyat}
		\rdg[wit={V3}]{anya}}//}
	\label{jnanam}\\!}
\end{alttlg}


\Anm{\getsiglum{G11plus,N19,V15,J10,Jyo} have \ref{jnatva}--\ref{tatraika} \textit{jñātvā suṣumṇāsadbhedaṃ} here}

\newpage
\begin{alttlg}[hp04_000_15]
\tl{
\pada{\app{\lem[wit={ceteri}]{rasasya}% +P6,M3
		\rdg[wit={J7,N19,V15}]{rasaś ca}} % +G7
	\app{\lem[wit={ceteri}]{manasaś caiva} % caivaṃ G11,J10
		\rdg[wit={V3}]{manaś caiva}
		\rdg[wit={N23}]{manasaiva caṃ°}}}
\pada{\app{\lem[wit={ceteri}]{cañcalatvaṃ}
		\rdg[wit={N23}]{°calatvaṃ ca}
		\rdg[wit={N19}]{vaṃcatvaṃ ca}} svabhāvataḥ/}\\+} % °ta P11
\tl{
\pada{\app{\lem[wit={G11,G5,V15}]{rasa}% +F
		\rdg[wit={N23,N19}]{rase}
		\rdg[wit={GrB,J7,Gr3a,J10,Jyo}]{raso}
		}%
	\app{\lem[wit={G5,N19,V15}]{bandhe}
		\rdg[wit={G11}]{baddhe}% +G7, rasabandhe manobandhe M3,V15
		\rdg[wit={ceteri}]{baddho}% +F
		} 
	mano\app{\lem[wit={V15}]{bandhe}
		\rdg[wit={G11}]{baddhe}
		\rdg[wit={C6}]{baddho}
		\rdg[wit={ceteri}]{baddhaṃ} % varddhaṃ N23
		\rdg[wit={P11}]{baddhaḥ}% +F
		\rdg[wit={G5}]{dhatte}}}
\pada{\app{\lem[wit={ceteri}]{kiṃ}
		\rdg[wit={N19}]{tan}}
	na sidhyati bhūtale//}\\!} % siddhyaṃti V3
\end{alttlg}

\begin{alttlg}[hp04_000_16]
\tl{
\pada{mūrchito % mūrchato E2
	\app{\lem[wit={Gr2,Gr3a,C6,N19,V15,Jyo}]{harate}
		\rdg[wit={P11,V3,G11,G5,J10}]{harati}% +F
		}
	\app{\lem[wit={ceteri}]{vyādhiṃ}
		\rdg[wit={V3,J10}]{vyādhi}
		\rdg[wit={P11}]{vyādhin}
		\rdg[wit={G5,Jyo}]{vyādhīn}}}
\pada{mṛto
	\app{\lem[wit={ceteri}]{jīvayati}
		\rdg[wit={V15}]{jīvayate}}
		svayam/}\\+}
\tl{
\pada{baddhaḥ % badho P11, bandhaḥ G5
	\app{\lem[wit={ceteri}]{khecaratāṃ}
		\rdg[wit={V19}]{khacatāṃ}}
	\app{\lem[wit={ceteri}]{dhatte}
		\rdg[wit={N23,N19}]{dhartte}
		\rdg[wit={V3}]{yāti}}}
\pada{\app{\lem[wit={ceteri}]{raso vāyuś ca}
		\rdg[wit={V3}]{vāyuś ca}
		\rdg[wit={J10}]{sa jīveśvara}}
	\app{\lem[wit={C6,Gr3a}]{bhairavi}
		\rdg[wit={Gr2,G11,G5,N19,V15}]{bhairavī}
		\rdg[wit={V3},post=\texteng{(tathā for missing raso)}]{bhairavī tathā}
		\rdg[wit={P11}]{tad dvayaṃ}
		\rdg[wit={Jyo}]{pārvati}
		\rdg[wit={J10}]{seśvaraḥ}}//}\\!}
\end{alttlg}

\Anm{\getsiglum{G11,N19,V15,J10} have \ref{vayumargena} \textit{vāyumārge tv asaṃcāre} here}

\Anm{\getsiglum{G11,N19,V15,J10,Jyo} have \ref{manahsthairye} \textit{manaḥsthairye} here}
\endaltnormal

\newpage

\begin{tlg}[hp04_001]
\tl{
\pada{\app{\lem[wit={ceteri}]{indriyāṇāṃ} % °nāṃ V19, °ṇā N3
		\rdg[wit={N19}]{indriyāṇi}} mano nātho} % nāthe G11
\pada{\app{\lem[wit={N3,J5,GrB,G11,Jyo}]{manonāthas tu}
		\rdg[wit={G4}]{manonāthasu}
		\rdg[wit={N19}]{manonāthaḥ su}
		\rdg[wit={N23,Gr3a,V15,J10}]{manonāthaś ca}
		\rdg[wit={J7}]{manaso nātha}} mārutaḥ/}\\+} % mārute P11
\tl{
\pada{mārutasya layo
	\app{\lem[wit={ceteri},alt={nāthas/nāthaḥ/nātho}]{nātha\skp{s/nāthaḥ/nātho}}
		\rdg[wit={J7}]{nāthāḥ}}}%
\pada{\app{\lem[wit={N3,J5,V3,G11,N19,V15,J10},alt={taṃ nāthaṃ layam āśrayet}]{\skm{s }taṃ nāthaṃ layam āśrayet}% K1A; nātha N3
		\rdg[wit={G4}]{tan nātho laya\,+\,+\,+}
		\rdg[wit={C6,Gr2,E2,Jyo}]{sa layo nādam āśritaḥ}% K1B
		\rdg[wit={P11}]{laya nātha niraṃjanāṃ}% tasya nātho nirañjanaḥ F
		\rdg[wit={V19},post={\unm}]{layo dasamāśrayaḥ}}//}\\!}
\end{tlg}


\startaltnormal
\begin{alttlg}[hp04_001_1]
\tl{
\pada{\app{\lem[wit={GrB,G11,G5,V15,Jyo}]{so'yam evāstu}
		\rdg[wit={N19}]{soyamo vāstu}
		\rdg[wit={J10}]{svayam evāstu}% +K1,P6
		\rdg[wit={Gr2,Gr3a}]{ayam eva tu}% +F
		} % evaṃ N23
	\app{\lem[wit={ceteri}]{mokṣākhyo} % °kṣyo N19, °syo P11, °khyā G5
		\rdg[wit={J10}]{vā mokṣaḥ}
		}}
\pada{\app{\lem[wit={GrB,G11,G5,V15,Jyo}]{māstu vāpi}% +P11; astu K1,P6
		\rdg[wit={N19}]{māstu kapi}
		\rdg[wit={J10}]{sosti vāpi}
		\rdg[wit={J7}]{'stu vāpi sa}
		\rdg[wit={V19}]{yas tu vāpi}% +K3,C7
		\rdg[wit={E2}]{yas tu vyāpi}
		\rdg[wit={N23}]{aya vāpi}}
	matāntare/}\\+} % matātare J7, matāṃbare P11, mātā° J10ac
\tl{
\pada{manaḥprāṇa% mana P11
	\app{\lem[wit={P11,C6,Gr2,G11,G5,V15}]{layānando} % layāṃnado N23; layo? V15
		\rdg[wit={N19}]{layānanda}
		\rdg[wit={V3}]{layāna} % 1 syllable too short
		\rdg[wit={Gr3a}]{layo nādo}
		\rdg[wit={Jyo}]{laye kaścid}
		\rdg[wit={J10}]{°m apānaṃ vā}}}
\pada{\app{\lem[wit={P11,C6,G11,N19,V15}]{mayi}% +K1,M1,M3,P6
		\rdg[wit={V3}]{māpi}
		\rdg[wit={Gr2,Gr3a}]{nāpi}
		\rdg[wit={G5}]{bhuvi}
		\rdg[wit={J10}]{layaḥ}
		\rdg[wit={Jyo}]{āna°}}
	\app{\lem[wit={ceteri},alt={kaścit/°cid}]{kaści\skp{t/°cid}}
		\rdg[wit={V19}]{kviṃcid}
		\rdg[wit={Jyo}]{°ndaḥ saṃ°}}%
	\app{\lem[wit={P11,C6,G11,N19,V15,J10,Jyo},alt={pravartate}]{\skm{t }pravartate}% °vattate G11
		\rdg[wit={V3}]{pravartate na}
		\rdg[wit={G5}]{pravartatām}
		\rdg[wit={N23}]{vibhedyate}
		\rdg[wit={J7,Gr3a}]{vibhidyate}
		}//} \NotIn{Gr1}\\!}
\end{alttlg}
\endaltnormal


%\newpage
\begin{tlg}[hp04_002]
\tl{
\pada{\app{\lem[wit={Gr3a,G11}]{praṇaṣṭocchvāsa}% +F
		\rdg[wit={V3,J7,V15,J10}]{pranaṣṭocchvāsa}
		\rdg[wit={P11}]{pranaṣṭosvāsa}
		\rdg[wit={N19}]{pranaṣṭauśvāsa}
		\rdg[wit={N23}]{prabhṛṣṭo\,\_\,sa}
		\rdg[wit={N3,Jyo}]{praṇaṣṭaśvāsa}% na Jyo ##?
		\rdg[wit={J5}]{praṇaṣṭabhyāsa}
		\rdg[wit={C6}]{pranaṣṭaḥ svā<<sa>>}}%
	\app{\lem[wit={N3,G11,V15,Jyo}]{niśvāsaḥ}
		\rdg[wit={J5,V3}]{niśvāsa}
		\rdg[wit={P11,C6pc,N19,J10}]{niḥśvāsaḥ} % niḥñcāsaḥ N19, °svāsaḥ P11
		\rdg[wit={C6ac,Gr3a}]{niḥśvāsa}% niḥsvāsa V19
		\rdg[wit={J7}]{niśvāsāḥ}
		\rdg[wit={N23}]{niśvāsā}
		}}
\pada{\app{\lem[wit={ceteri}]{pradhvasta}% praddhasta N19; +G5,M3
		\rdg[wit={G11}]{prabhṛṣṭa}
		\rdg[wit={J10}]{pranaṣṭa}% °naṣṭaḥ V17, prā° N26.
		}%
	\app{\lem[wit={ceteri}]{viṣaya}
		\rdg[wit={G11}]{viṣayā}
		\rdg[wit={N19}]{viṣaga}}%
	\app{\lem[wit={N3,J5,C6,V3,V19,G11,J10,Jyo}]{grahaḥ}
		\rdg[wit={Gr2,E2}]{grahāḥ}
		\rdg[wit={P11}]{grataḥ}
		\rdg[wit={V15}]{jvaraḥ}
		\rdg[wit={N19}]{hvaraḥ}}/}\\+}
\tl{
\pada{\app{\lem[wit={N3,J5,C6,V3,G11,Jyo},post=\texteng{(niḥś° \getsiglum{N3})}]{niśceṣṭo}% niścaiṣṭo J5
		\rdg[wit={Gr2,Gr3a,V15}]{niśceṣṭā}
		\rdg[wit={P11}]{niḥśreṣṭo}
		\rdg[wit={N19}]{nidyeṣṭo}
		\rdg[wit={J10}]{niścalo}}
	\app{\lem[wit={GrB,N23,G11,N19,V15,J10,Jyo}]{nirvikāraś ca} % nirvikā<<ra>>ś N23, nivikāraś V15
		\rdg[wit={J7,Gr3a}]{nirvikārāś ca}
		\rdg[wit={N3}]{nirvikāras tu}
		\rdg[wit={J5}]{nivikalpas tu}
		}}
\pada{\app{\lem[wit={N3,J5,GrB,G11,N19,V15,J10,Jyo}]{layo}
		\rdg[wit={V19}]{laye}
		\rdg[wit={Gr2,E2}]{layaṃ}}
	\app{\lem[wit={N3,J5,GrB,G11,N19,V15,J10,Jyo}]{jayati}
		\rdg[wit={Gr2,Gr3a}]{yānti ca}}
	\app{\lem[wit={N3,J5,GrB,G11,N19,V15,Jyo}]{yoginām}% °naṃ N3
		\rdg[wit={Gr2,Gr3a,J10}]{yoginaḥ}}//}\\!}
\end{tlg}

\begin{tlg}[hp04_003]
\tl{
\pada{\app{\lem[wit={ceteri}]{ucchinna}
		\rdg[wit={N3,G11,V15}]{ucchinnaḥ}
		\rdg[wit={V19}]{ucchūna}% +K3,C7
		}%
	sarva\app{\lem[wit={ceteri}]{saṃkalpo}
		\rdg[wit={V19}]{saṃkalpe}}}
\pada{\app{\lem[wit={ceteri}]{niḥśeṣāśeṣa}% ni<<ḥ>>śeṣā° J10
		\rdg[wit={Gr2}]{niḥśeṣagata}
		\rdg[wit={J5,V3}]{niḥśeṣoṣeṣa}}%
	\app{\lem[wit={ceteri}]{ceṣṭitaḥ}
		\rdg[wit={C6}]{ceṣṭitam}
		%\rdg[wit={K3,C7}]{veṣṭitaḥ}
		\rdg[wit={V15}]{varjitaḥ}}/}\\+}
\tl{
\pada{\app{\lem[wit={N3,J5,V3,V19,J10,Jyo}]{svāvagamyo}
		\rdg[wit={G4,P11,G11}]{svāvagamya}
		\rdg[wit={C6}]{sovagamyo}
		\rdg[wit={N19}]{svāgamyo}
		\rdg[wit={V15}]{svānugamyo}
		\rdg[wit={Gr2}]{svāgate cā}
		}
		layaḥ ko'pi} % pi added in margin C7; laya kaupi P11
\pada{\app{\lem[wit={Gr1,C6,G11}]{jayatāṃ vāgagocaraḥ}% °ra J5
		\rdg[wit={N19}]{japatāṃ vāgagocara}
		\rdg[wit={V15}]{jāyatāṃ vāgagocaraḥ}
		\rdg[wit={P11}]{jāyatāṃ cāpi gaucaraḥ}
		\rdg[wit={V3,J10,Jyo}]{jāyate vāgagocaraḥ} % agocara V3
		\rdg[wit={Gr2,V19}]{manovācām agocaraḥ}% +K3,C7
		}//}
		\NotIn{E2}\\!}
\end{tlg}

%\newpage
\begin{tlg}[hp04_004]
\tl{
\pada{yatra % yabha N23, yanna N19
	\app{\lem[wit={ceteri},alt={dṛṣṭir}]{dṛṣṭi\skp{r}}
		\rdg[wit={N3,V15,J10}]{dṛṣṭi}
		\rdg[wit={C6}]{vṛṣṭir}
		}r layas tatra} % yayas? G11
\pada{bhūtendriya% bhute° N19
	\app{\lem[wit={N3,J5,V3,G11,V15}]{sanātanaḥ}
		\rdg[wit={P11}]{sanātana}
		\rdg[wit={N19}]{sanātanaṃ}
		\rdg[wit={C6,Gr2,V19,J10,Jyo}]{sanātanī}% +K3,C7
		}/}\\+}
\tl{
\pada{\app{\lem[wit={N3,Gr2,V19},alt={syāc chaktir/°tiḥ}]{syāc chakti\skp{r}}% °kti<<ḥ>> N23; +K3,C7
		\rdg[wit={J5}]{syāt saktir}
		\rdg[wit={GrB,G11,N19,J10,Jyo}]{sā śaktir}% śakti P11,G11; +F
		\rdg[wit={V15}]{sa śaktir}}%
	\app{\lem[wit={N3,J5,GrB,G11,J10,Jyo},alt={jīva}]{\skm{r }jīva}
		\rdg[wit={Gr2,V19}]{sarva}% +K3,C7
		\rdg[wit={N19,V15}]{bhāva}}%
	\app{\lem[wit={ceteri}]{bhūtānāṃ}
		\rdg[wit={N23}]{bhūtānī}
		\rdg[wit={N19}]{bhūnāṃ}}}
\pada{\app{\lem[wit={N3,G4,C6,V3,Gr2,J10},alt={dṛṣṭir}]{dṛṣṭi\skp{r}}
		\rdg[wit={J5,P11,V19,G11}]{dṛṣṭi}% +K3,C7
		\rdg[wit={N19,V15}]{dṛṣṭe}
		\rdg[wit={Jyo}]{dve a°}}%
	\app{\lem[wit={P11,V3,G11,N19},alt={lakṣ(y)e layaṃ gatā}]{\skm{r }lakṣye layaṃ gatā}% lakṣe P11,V3,N19
		\rdg[wit={J5}]{lakṣe la(!) gatā}
		\rdg[wit={N3}]{lakṣe layaṃ gatāḥ}
		\rdg[wit={G4}]{lakṣy[e] layaṃ gataḥ}
		\rdg[wit={J10,Jyo}]{lakṣye layaṃ gate}
		\rdg[wit={V15}]{lakṣaṃ layaṃ gatau}
		\rdg[wit={J7}]{lakṣe na saṃgatā}
		\rdg[wit={N23}]{lakṣana saṃgatā}
		\rdg[wit={V19}]{lakṣeṇa saṃgatā}%lakṣyeṇa K3,C7
		\rdg[wit={C6}]{gacchel layaṃ gate}}//}
		\label{yatradrsti}
		\NotIn{E2}\\!}
\end{tlg}

%\Anm{\getsiglum{Jyo} has \ref{layo} \textit{layo laya iti} here}

\newpage
\begin{tlg}[hp04_005]
\tl{
\pada{vedaśāstra\app{\lem[wit={N3,G4,P11,C6,G11,N19,V15,J10,Jyo}]{purāṇāni}
		\rdg[wit={N23}]{purāṇādyāḥ}
		\rdg[wit={J7}]{puraṇādyāḥ}
		\rdg[wit={E2}]{purāṇaughāḥ}% +K3,C7
		\rdg[wit={V19}]{purāṇaiś ca}}}
\pada{\app{\lem[wit={ceteri}]{sāmānya}% °yaṃ G11, °nyā E2
		\rdg[wit={C6}]{samāni}}% 
	\app{\lem[wit={ceteri}]{gaṇikā} % gatikā N26
		\rdg[wit={V19}]{gaṇivā}} iva/}
		\lineom{ab}{J5,V3}\\+}
\tl{
\pada{\app{\lem[wit={ceteri}]{ekaiva}
		\rdg[wit={E2}]{idaṃ tu}
		} śāṃbhavī % sāṃ° J10
	\app{\lem[wit={Gr1,P11,C6,Gr2,Gr3a,G11,N19,Jyo}]{mudrā}
		\rdg[wit={V15}]{māyā}
		\rdg[wit={J10}]{vidyā}}}
\pada{\app{\lem[wit={N3,J5,P11,C6,Gr2,Jyo}]{guptā kulavadhūr iva}% vadhū iva P7
		\rdg[wit={J10}]{gopyā kulavadhūr iva}
		\rdg[wit={G11,N19,V15},post=\texteng{(cf.\,\ref{gopita}d)}]{sarvatantreṣu gopitā}
		\rdg[wit={Gr3a}]{sarvatantreṣu gopitā rakṣaṇīyā prayatnena guptā kulavadhūr iva}}//}\label{vedasastra} \lineom{cd}{V3}\\!}
\end{tlg}

%\newpage
\begin{tlg}[hp04_006]
\tl{
\pada{anta\app{\lem[wit={J5,C6ac,V3,Gr2,J10,Jyo},alt={lakṣ(y)aṃ}]{\skm{r }lakṣyaṃ} % antalakṣaṃ P7; °lakṣaṃ V3,N23,J10; °lakṣyaṃ Jyo,C6ac
		\rdg[wit={V19}]{lakṣā} % lakṣyā K3
		\rdg[wit={E2}]{lakṣyo}
		\rdg[wit={N3,P11,C6pc,G11}]{lakṣ(y)a}} % y C7,P11,C6
	\app{\lem[wit={N3,J5,GrB,Gr2,Gr3a,G11,Jyo},alt={bahir}]{bahi\skp{r}}
		\rdg[wit={J10}]{mano}}%
	\app{\lem[wit={ceteri},alt={dṛṣṭir}]{\skm{r }dṛṣṭi\skp{r}}
		\rdg[wit={J5,V3,V19,G11,J10}]{dṛṣṭi}}}%
\pada{\app{\lem[wit={N3,J5,C6,V3,J7,Gr3a,G11,J10,Jyo},alt={nimeṣonmeṣa}]{\skm{r }nimeṣonmeṣa}
		\rdg[wit={P11,N23}]{nirmiṣonmeṣa}% °ṣya N23
		}\app{\lem[wit={ceteri}]{varjitā}
		\rdg[wit={P11,E2}]{varjjitaḥ}}/}\\+}
\tl{
\pada{\app{\lem[wit={N3,P11,C6,G11,Jyo}]{eṣā sā}% sāṃ P11
		\rdg[wit={J5}]{eṣāsau}
		\rdg[wit={V3}]{eṣā hi}
		\rdg[wit={J10}]{eṣā tu}
		\rdg[wit={E2}]{eṣā vai}
		\rdg[wit={Gr2,V19}]{saiṣā tu}% +K3,C7
		}
		śāṃbhavī mudrā} % sāṃbhavī J10
\pada{\app{\lem[wit={N3,J5,GrB,Gr2,E2,G11,J10}]{sarvatantreṣu}% sarve J10
		%\rdg[wit={K3,C7}]{sarvaśāstreṣu}
		\rdg[wit={V19}]{sarvatantreṣu śastreṣu}
		\rdg[wit={Jyo}]{vedaśāstreṣu}}
		gopitā//}\label{gopita}
		\NotIn{N19,V15} \anm{eye-skip?}\\!}
\end{tlg}
		%\myfn{The omission was probably caused by haplography. \getsiglum{L2} inserts the following passage after 4.20a: \devnote{eṣā tu śāṃbhavī vidyā gopā ca kulavadhūr iva/ aṃtarlakṣamanodṛṣṭi nimeṣonmeṣonmeṣa(!)varjito/}}\\!

\begin{tlg}[hp04_007]
\tl{
\pada{anta\app{\lem[wit={N3,P11,C6,Gr3a,G11,J10,Jyo},alt={lakṣya}]{\skm{r}lakṣya}
		\rdg[wit={J5,V3,Gr2,N19,V15}]{lakṣa}}%
		vilīnacittapavano yogī
	\app{\lem[wit={ceteri}]{yadā}
		\rdg[wit={J10}]{yathā}
		\rdg[wit={Gr1,N19}]{sadā}% +F
		} vartate}\\+} % ##?
\tl{
\pada{\app{\lem[wit={ceteri}]{dṛṣṭyā}
		\rdg[wit={J10}]{dṛṣṭvā}
		\rdg[wit={P11}]{dṛṣyā}
		\rdg[wit={V3}]{dṛśyā}}
	niścala\app{\lem[wit={ceteri}]{tārayā}
		\rdg[wit={P11}]{tālayā}
		\rdg[wit={N23}]{tāra}}
	\app{\lem[wit={ceteri},alt={bahir}]{bahi\skp{r}}% bahidharaḥ! G11
		\rdg[wit={N23}]{hir}}%
	\app{\lem[wit={Gr1,GrB,G11,V15,J10,Jyo},alt={adhaḥ}]{\skm{r }adhaḥ}
		\rdg[wit={N19}]{adhraḥ}
		\rdg[wit={Gr2,Gr3a}]{asau}}
	\app{\lem[wit={J5,Gr3a,G11,N19,V15,Jyo}]{paśyann apaśyann api}
		\rdg[wit={N3}]{paśyann apaśyann ivā}% illeg. G4 ##?
		\rdg[wit={Gr2}]{paśyan na paśyaty api}
		\rdg[wit={J10}]{paśyann api}
		\rdg[wit={P11,V3}]{paśyan na paśyet tataḥ}% tata V3; paśyet tadā F
		\rdg[wit={C6}]{paśyen na paśyet tataḥ}}/}\\+}
\tl{
\pada{\app{\lem[wit={ceteri}]{mudreyaṃ}
		\rdg[wit={V15}]{mudre}} khalu % khaluṃ N23
	\app{\lem[wit={N3,J5,P11,V3,J10}]{khecarī} % = source; +K1,P6
		\rdg[wit={C6,Gr2,Gr3a,G11,N19,V15,Jyo}]{śāṃbhavī}% +F
		}
	\app{\lem[wit={ceteri}]{bhavati sā}% +G5,M3
		\rdg[wit={V3}]{bhavati}
		\rdg[wit={G11}]{°ti kathitā}}
	\app{\lem[wit={N3,J5,V3,Gr3a,N19,V15},alt={yuṣmat}]{yuṣma\skp{t}}% +J10pc
		\rdg[wit={J7}]{<<yu>>ṣmat}
		\rdg[wit={J10}]{yuṣmān}
		\rdg[wit={N23}]{puṣpat}
		\rdg[wit={C6,G11}]{yasya}
		\rdg[wit={P11}]{yāsya}
		\rdg[wit={Jyo}]{labdhā}}tprasādā%d
	\app{\lem[wit={P11,V3,Gr2,V19,V15,J10ac},alt={guro}]{\skm{d }guro}% +K3,C7
		\rdg[wit={C6,E2,G11,N19,J10pc,Jyo}]{guroḥ}
		\rdg[wit={N3}]{gurau}
		\rdg[wit={J5}]{gure}}}\\+}
\tl{
\pada{\app{\lem[wit={ceteri}]{śūnyāśūnya}% °nyaṃ N19
	\rdg[wit={C6}]{śūnyāc chūnya}}%
	\app{\lem[wit={ceteri}]{vivarjitaṃ}
		\rdg[wit={N23}]{vivarjite}
		\rdg[wit={V19}]{vivarjiti}
		\rdg[wit={J5}]{vivarjito}
		\rdg[wit={Jyo}]{vilakṣaṇaṃ}}
	\app{\lem[wit={ceteri}]{sphurati}
		\rdg[wit={V19}]{spharati}}
	\app{\lem[wit={ceteri},alt={yat}]{ya\skp{t}}% pat N23
		\rdg[wit={V3}]{ya}
		\rdg[wit={V19}]{[pta]t}
		\rdg[wit={N3,Jyo}]{tat}% ##?
		\rdg[wit={J5}]{ttat}}t
	tattvaṃ \app{\lem[wit={ceteri}]{padaṃ}
		\rdg[wit={G11,N19},alt={\om}]{\skp{\om}}} śāṃbhavam//}\label{antarlaksya}\\!}
\end{tlg}

%\newpage
\begin{tlg}[hp04_008]
\tl{
\pada{śrīśāṃbha\app{\lem[wit={N3,J7,Gr3a,Jyo},alt={°vyāś ca khecaryā}]{\skp{°}vyāś ca khecaryā} % khecarayā J7, °caryyāḥ V19
		\rdg[wit={G11}]{°vāś ca khecaryā}
		\rdg[wit={N23}]{°vyāḥ khecaryā\,\_}
		\rdg[wit={GrB}]{°vyā(ḥ) khecaryāś ca} % vyā V3,P6
		\rdg[wit={J5}]{°vyā khecaryā}
		\rdg[wit={G4}]{°vavyā khecaryā}}}
\pada{\app{\lem[wit={P11}]{avasthāyām abhedatā}% +K1 (ana°)
		\rdg[wit={C6}]{hy avasthāyām abhedataḥ}
		\rdg[wit={N3,G11}]{avasthāyāṃ na bhedataḥ}% bhedatā M3
		\rdg[wit={G4}]{avasthāyā na bhedataḥ}% +G5?
		\rdg[wit={J5}]{avasthāyasya bhedataḥ}
		\rdg[wit={Jyo}]{avasthādhāmabhedataḥ}
		\rdg[wit={V3}]{avasthāyāṃ ca bhedatā}
		\rdg[wit={Gr2},post=\texteng{(bhedanaḥ \getsiglum{N23})}]{avasthā ca na bhedataḥ}% °naḥ N23
		\rdg[wit={Gr3a}]{avasthā balabhedataḥ}}\marma/}%
		\myfn{In \getsiglum{Jyo} this half verse is followed by another half verse: \devnote{bhavec cittalayānandaḥ śūnye citsukharūpiṇi}.} % +M1,G8
		\NotIn{N19,V15,J10}\\!}
\end{tlg}

% J5 śobhavyā khecaryā avaschāyasya bhedataḥ
% G4 śrīśāṃbhavavyā khecaryā avasthāyā {{tavalabdhi}} na bhedataḥ
% K1a! śrīśāṃbhavyāś ca khecarya anasthāyām abhedatā
% K1b! śrīśāṃbhavyāḥ khecaryāś ca avasthā tu na bhedataḥ
% M1 śrīśāṃbhavyāś ca khecar.ā avasthāyām abh[e]dataḥ
% M3 śriśāṃbhavyāś ca khecaryā avastāyān na bhedatā
% G5 śriśāṃbhavyāś ca khecaryā avastāyā na bhedataḥ
% G7 śāṃbaryāś ca khecarya atastā naiva bhedataḥ
% P6 śrīśaṃbhavyā khecaryāś ca avasthā tu na bhedataḥ
% N22 śrīśaṃbhavyā khecaryāś ca avasthā <!> na bhedataḥ
% IFP: śriśāṃbhavyāś ca khecaryā avasthā sthānabhedataḥ

\newpage
\begin{tlg}[hp04_009]
\tl{
\pada{
	\app{\lem[wit={N3,J5}]{pātāle yadvitaya}% °tayas G11
		\rdg[wit={G4}]{pātāḷe yadvita\,..}
		\rdg[wit={Gr2}]{pātālād yad viśati}
		\rdg[wit={V19,C7}]{pātālād vā viyati}
		}% 
	\app{\lem[wit={J5}]{suṣiraṃ}% sukhiraṃ K1,P6
		\rdg[wit={N3}]{suśiraṃ}
		\rdg[wit={N23}]{śikhiraṃ}
		\rdg[wit={J7}]{śikharaṃ}% +K3
		\rdg[wit={V19,C7}]{śikhare}
		} merumūle
	\app{\lem[wit={N3}]{tad asmin}% +G11
		\rdg[wit={J5}]{yadismi}
		\rdg[wit={N23}]{tasti}
		\rdg[wit={J7}]{tad asti}% +K1,P6
		\rdg[wit={V19}]{tadāstā}
		\rdg[wit={C7}]{tad āste}% +K3
		}}\\+}
\tl{
\pada{tattvaṃ caitat pravadati % pravahati C7
	\app{\lem[wit={N3,Gr2}]{sudhīs tan mukhaṃ}
		\rdg[wit={C7}]{sudhīḥ saṃmukhaṃ}% +K3
		\rdg[wit={J5}]{sudhī sanmukhaṃ}
		\rdg[wit={V19}]{susaṃmukhaṃ}} 
		nimnagānām/}\\+} % ninmaganāṃ V19
\tl{
\pada{candrā%t % caṃdrā N3
	\app{\lem[wit={Gr2},alt={sāraḥ}]{\skm{t }sāraḥ}% sāra K1, sāraṃ P6
		\rdg[wit={V19,C7}]{srāvaḥ}
		\rdg[wit={N3,J5}]{sāro}}
	\app{\lem[wit={N23,C7}]{sravati}% +K1,P6,G11,K3
		\rdg[wit={V19}]{śravati}
		\rdg[wit={J7}]{savati}
		\rdg[wit={N3}]{grasati}
		\rdg[wit={J5},alt={\om}]{\skp{\om}}}
	\app{\lem[wit={N3,J5,N23,V19,C7},alt={vapuṣas}]{vapuṣa\skp{s}}
		\rdg[wit={J7}]{puruṣas}}s
	tena mṛtyur narāṇāṃ}\\+} % mūtyur N23
\tl{
\pada{\app{\lem[wit={Gr1,J7,V19,C7},alt={taṃ badhnīyāt}]{taṃ badhnīyā\skp{t}}
		\rdg[wit={N23}]{tadvahmaṃpāt}}%
	\app{\lem[wit={N3,J5},alt={sukaraṇamṛdā}]{\skm{t }sukaraṇamṛdā}
		\rdg[wit={G4}]{sukaraṇāmudā}
		\rdg[wit={J7,C7}]{svakaraṇamṛdā}% sakaraṇamṛtā K1,P6
		\rdg[wit={N23}]{svakaraṇaimṛdā}
		\rdg[wit={V19}]{svakaraṇamṛjā}
		} % sukaraṇam amṛtaṃ G11 (unm.)
		nānyathā
	\app{\lem[wit={N3,J7,C7}]{kāyasiddhiḥ}% +K1,G11
		\rdg[wit={N23}]{kāyaḥ siddhiḥ}
		\rdg[wit={J5,G4,V19}]{kāryasiddhi(ḥ)}% V19 om. h; +P6
		}//}
	\sgwit{Gr1,Gr2,V19,C7}\myfn{This verse is found after \ref{kecid} in \getsiglum{Gr2,V19}.} % Not in E2!
	\anm{\textrightarrow\ \manuref{3.49*2}}\label{patala}\\!}
\end{tlg}
	%\anm{\getsiglum{Gr2,V19,K3,C7}; the other mss have \ref{ardhogha} instead of this verse}
	%\NotIn{GrB,G11,N19,V15,J10,Jyo}


\begin{tlg}[hp04_010]
\tl{
\pada{yat kiṃcit sravate candrād}
\pada{amṛtaṃ divyarūpiṇaḥ/}\\+} % N24 has °rūpiṇaḥ too.
\tl{
\pada{tat sarvaṃ grasate sūryas}
\pada{tena piṇḍaṃ jarāyutam//} % piṇḍo jarāyutaḥ N24
\sgwit{Gr1} \anm{\textrightarrow\ \manuref{3.73*1}}\\!} % almost the same J5; G4 damaged
\end{tlg}


\begin{tlg}[hp04_011]
\tl{
\pada{tatrāsti karaṇaṃ divyaṃ}
\pada{sūryasya \app{\lem[resp=emend,post=\texteng{(cf. \manuref{3.73*2})}]{mukhabandhanam}
	\rdg[wit={N3,J5}]{paribandhanaṃ}
	\rdg[wit={G4},alt={\illeg}]{\skp{\illeg}}}/}\\+} % paripaṃthi ca N24
\tl{
\pada{gurūpadeśato jñeyaṃ}
\pada{na tu śāstrārthakoṭibhiḥ//}
\sgwit{Gr1} \anm{\textrightarrow\ \manuref{3.73*2}}\\!} % almost the same J5; G4 damaged
\end{tlg}


%\Anm{The numbering of the last three verses will be changed to 4.9--11 later.}

\newpage
\startaltrecension
\begin{alttlg}[hp04_011_1]
\tl{
\pada{\app{\lem[wit={P11,J7,E2,V15,Jyo}]{tāre}
		\rdg[wit={V3,V19}]{tāra}% E4
		%\rdg[wit={K3,C7}]{tāraṃ}
		\rdg[wit={C6}]{tārāṃ}
		\rdg[wit={J10}]{tārā}% +F
		\rdg[wit={N19}]{tāvad}
		\rdg[wit={N23}]{vāre}
		\rdg[wit={G11}]{kalāṃ}
		\rdg[wit={G5}]{kalā}
		}
	\app{\lem[wit={C6,Gr2,E2,G11,G5,V15,Jyo}]{jyotiṣi}% jotiṣi G5
		\rdg[wit={P11}]{jyotiṣīṃ}
		\rdg[wit={V3}]{jyotīṣa}
		\rdg[wit={V19}]{jyotiso}
		\rdg[wit={N19}]{yotiṣi}
		\rdg[wit={J10}]{jyotiṣu}
		}
	\app{\lem[wit={ceteri}]{saṃyojya}
		\rdg[wit={J10}]{saṃyojyā}
		\rdg[wit={N23}]{samojyaṃ}
		\rdg[wit={V19}]{jojya}}}
\pada{kiṃci%d
	\app{\lem[wit={GrB,G11,G5,V15,Jyo},alt={unnamayed}]{\skm{d }unnamaye\skp{d}} % °yet* V3; +M3,P6,N22
		\rdg[wit={N23,E2}]{uccālayed}
		\rdg[wit={J7}]{uccalayed}
		\rdg[wit={J10}]{uccārayed}% +K1b
		\rdg[wit={V19}]{uccācayed}
		\rdg[wit={N19}]{uṣṭānnama}}%
	\app{\lem[wit={ceteri},alt={bhruvau}]{\skm{d }bhruvau}% bhuvau? P11,J10
		\rdg[wit={N23}]{bhūvo<<ḥ>>}}/}
	\lineom{ab}{Gr1}\\+}
\tl{
\pada{\app{\lem[wit={P11,V3,E2,G11,G5,N19,V15},alt={pūrvayogasya mārgo'yam}]{pūrvayogasya mārgo'ya\skp{m}}% +K3,C7
		\rdg[wit={C6}]{pūrvayogasya mārgeṇa}
		\rdg[wit={J10}]{sūryayogasya mārge ca}
		\rdg[wit={V19}]{pūrvayogasya māhātmyam}
		\rdg[wit={Jyo}]{pūrvayogaṃ mano yuñjann}
		\rdg[wit={Gr2},alt={\om}]{\skp{\om}}}}%
\pada{\app{\lem[wit={P11,V3,Gr3a,G11,G5,N19,V15,Jyo},alt={unmanī}]{\skm{m }unmanī} % yaṃmunmanī N19
		\rdg[wit={C6}]{hy unmanī}
		\rdg[wit={J10}]{yunmanī}
		\rdg[wit={G11}]{kiṃcid un°}
		\rdg[wit={Gr2},alt={\om}]{\skp{\om}}}%
	\app{\lem[wit={P11,G11,Jyo}]{kārakaḥ kṣaṇāt}
		\rdg[wit={C6}]{kārakakṣaṇāt}
		\rdg[wit={N19}]{kārakaṃ kṣaṇāt}
		\rdg[wit={V3}]{kāraṇaḥ kṣaṇāt}
		\rdg[wit={G5}]{kāraṇaṃ kṣaṇāt}
		\rdg[wit={Gr3a,V15}]{karaṇaṃ kṣaṇāt}% +M3
		\rdg[wit={J10}]{kāralakṣaṇam}% +K1
%		\rdg[wit={G11}]{°namayet kṣaṇāt | unmanīkārakaḥ}
		\rdg[wit={Gr2},alt={\om}]{\skp{\om}}}//}
	\lineom{cd}{Gr1,Gr2}\\!} % om. +P6,N22
%\sgwit{Gr3a,N19,V15,C8,V3,G11,G5,N9,V17,J10}
\end{alttlg}

%\newpage
\begin{alttlg}[hp04_011_2]
\tl{
\pada{keci%d
	\app{\lem[wit={ceteri},alt={āgama}]{\skm{d }āgama}
		\rdg[wit={G11,G5}]{nigama}}%
	\app{\lem[wit={ceteri}]{jālena}
		\rdg[wit={J10}]{yogena}
		\rdg[wit={Gr2},alt={\om}]{\skp{\om}}}}
\pada{keci%n % cecin N19, keccin V19
	\app{\lem[wit={P11,C6,N19,J10,Jyo},alt={nigama}]{\skm{n }nigama}% =source
		\rdg[wit={V3,Gr3a}]{niyama}
		\rdg[wit={V15}]{nima}
		\rdg[wit={G11,G5}]{āgama}
		\rdg[wit={Gr2},alt={\om}]{\skp{\om}}}%
	\app{\lem[wit={P11,C6,G11,G5,N19,J10,Jyo}]{saṃkulaiḥ}% saṃkulai P11
		\rdg[wit={V3,V15}]{saṃkule}
		\rdg[wit={E2}]{saṃkulāḥ}% +K3,C7
		\rdg[wit={V19}]{saṃkulā}
		\rdg[wit={Gr2},alt={\om}]{\skp{\om}}}/}\\+}
\tl{
\pada{kecit tarkeṇa muhyanti} % ke<ci>nnarkkeṇa N19, tarkena E2
\pada{naiva jānanti tārakam//} % kārakam G5
%\myfn{Pādas ab and cd are transposed in \getsiglum{V19} and the correct order is indicated by a small number 1 and 2 above the hemistiches.}
\NotIn{Gr1,Gr2}\label{kecid}\\!} % C6?
\end{alttlg}
%\sgwit{GrB,Gr3a,G11,G5,N19,V15,N9,V17,J10}

%\newpage
%\startaltnormal
\begin{alttlg}[hp04_011_3]
\tl{
\pada{\app{\lem[wit={ceteri}]{ardhodghāṭita}
		\rdg[wit={P11}]{arddhoghāṭita}
		\rdg[wit={N23}]{arddhocchā[d]ita}
		\rdg[wit={Jyo}]{ardhonmīlita}}%
	\app{\lem[wit={V19,V15,Jyo}]{locanaḥ}
		\rdg[wit={GrB,Gr2,G11,G5,N19,J10}]{locana}}
	\app{\lem[wit={ceteri}]{sthira}
		\rdg[wit={N23}]{sthila}}manā
	nāsāgradatte% nāśā° N19, nāśādagra° V17
	\app{\lem[wit={ceteri},alt={°kṣaṇaḥ/-aś}]{\skp{°}kṣaṇaḥ\skp{/-aś}}% ś P11,C6,V19,J10
		\rdg[wit={V3,N23}]{°kṣaṇāś}% °āḥś V3
		\rdg[wit={N19}]{°kṣaṇaṃ}}}\\+}
\tl{
\pada{\app{\lem[wit={ceteri},alt={candrārkāv}]{candrārkā\skp{v}}
		\rdg[wit={V3}]{cāndrārkāv}
		\rdg[wit={J10}]{candrārkau}}%
	\app{\lem[wit={GrB,N23,V19,G11,G5,V15,Jyo},alt={api}]{\skm{v }api}
		\rdg[wit={J7}]{avi}
		\rdg[wit={N19}]{aca}
		\rdg[wit={J10}]{ca vi°}} līnatā%m
	\app{\lem[wit={G11,Jyo},alt={upanayan}]{\skm{m }upanaya\skp{n}}% M1,M3,F
		\rdg[wit={G5}]{apanayan}
		\rdg[wit={Gr2,V19,N19,V15}]{upanayen}% upanaye V19,N19,P6; eva nayet G7; +K1
		\rdg[wit={GrB}]{upagatau}% = source
		\rdg[wit={J10}]{gatau}}%
	\app{\lem[wit={ceteri},alt={niṣpanda}]{\skm{n }niṣpanda} % nispanda K3,C7,Jyo, niṣyaṃda? N19
		\rdg[wit={P11}]{nirvyaṃda}
		\rdg[wit={G5}]{diṣyanda}
		\rdg[wit={J10}]{nikṣipya}}%
	\app{\lem[wit={G11,G5}]{bhāvāntare}% +C7
		\rdg[wit={N23,V19}]{bhāvo'ntare}% bhāvottaraḥ P6,F
		\rdg[wit={J7}]{bhāvotare}
		\rdg[wit={J10}]{bhāsoṃtare}
		\rdg[wit={V15}]{bāṣpaṃ tataḥ}
		\rdg[wit={N19}]{vāpyaṃ tataḥ}
		\rdg[wit={C6}]{rūpaṃ tataḥ}
		\rdg[wit={P11}]{rūpaṃ tanu}% ##
		\rdg[wit={V3}]{rūpatanu}
		\rdg[wit={Jyo}]{bhāvena yaḥ}}/}\\+}
\tl{
\pada{jyotī% jyoti P11, jyoṃtī J7, yotī N19, jyotī{{kha}} J10
	\app{\lem[wit={ceteri},alt={rūpam}]{rūpa\skp{m}}
		\rdg[wit={N19,V15}]{rūpa}
		\rdg[wit={J7}]{yatsyam}}%
	\app{\lem[wit={ceteri},alt={aśeṣa}]{\skm{m }aśeṣa}% aśaiṣa P11
		\rdg[wit={N19,V15}]{viśeṣa}}%
	\app{\lem[wit={ceteri}]{bāhyarahitaṃ}
		\rdg[wit={Jyo}]{bījam akhilaṃ}}
	\app{\lem[wit={ceteri}]{dedīpya}
		\rdg[wit={N23}]{devadīpya}}mānaṃ paraṃ}\\+} % puraṃ G5
\tl{
\pada{tattvaṃ % tatva V3
	\app{\lem[wit={ceteri},alt={tat}]{ta\skp{t}}
		\rdg[wit={J10}]{yac}}%
	\app{\lem[wit={Gr2,V19,Jyo},alt={padam eti}]{\skm{t }padam eti}% +K1,P6
		\rdg[wit={GrB,G11,G5}]{param eti}% +F
		\rdg[wit={N19,V15}]{param asti}% +HTK
		\rdg[wit={J10}]{carama}}
	\app{\lem[wit={ceteri}]{vastu}% +M1,M3,G7,K1,P6
		\rdg[wit={N23}]{vasta}
		\rdg[wit={P11,V3}]{yastu}
		\rdg[wit={C6}]{yat tu}} 
		paramaṃ % param avācyaṃ J7ac
	\app{\lem[wit={ceteri}]{vācyaṃ} % vācya V19
		\rdg[wit={N23}]{vāpyaṃ}} ki%m
	\app{\lem[wit={ceteri},alt={atrādhikam}]{\skm{m }atrādhikam}% °dikam G5
		\rdg[wit={N23}]{andrādhikaṃ}
		\rdg[wit={V19}]{atrāsanaṃ}}//}\label{ardhogha} \NotIn{Gr1,E2}
	\anm{after \ref{antarlaksya} 
	\getsiglum{Gr2,V19}}\\!} % Not in E2!
\end{alttlg}
%\endaltnormal

%\newpage

%\Anm{\getsiglum{V3,N19,V15,J10} have Vulg 4.42--65 about Khecarīsamādhi here}
\Anm{The following verses are not found in \getsiglum{Gr1,Gr2,Gr3a}, but in \getsiglum{GrB,G11,N19,V15,J10,Jyo}}

%\startaltrecension
%\teimute{\small}
\begin{alttlg}[hp04_011_4]
\tl{
\pada{\app{\lem[wit={GrB,G11,G5,N19,V15,Jyo}]{divā na}
		\rdg[wit={J10}]{vāsare}} pūjayel liṅgaṃ}
\pada{\app{\lem[wit={P11,N19}]{rātrau naiva ca pūjayet}
		\rdg[wit={C6,V3,G11}]{rātrau naiva prapūjayet}
		\rdg[wit={G5,J10,Jyo}]{rātrau caiva na pūjayet} % pūyet J10; vāpi na M3
		\rdg[wit={V15}]{rātrau liṃgaṃ na pūjayet}}/}\\+}
\tl{
\pada{\app{\lem[wit={GrB,G11,G5,N19,V15,J10}]{satataṃ}
		\rdg[wit={Jyo}]{sarvadā}} pūjayel liṅgaṃ} % pū<<ja>>yel J10
\pada{\app{\lem[wit={Jyo}]{divārātrinirodhataḥ}
		\rdg[wit={P11,V3,G11,G5}]{divārātraṃ na pūjayet}% +M1,M3,F
		\rdg[wit={N19,V15,J10}]{divārātrau na pūjayet}
		\rdg[wit={C6}]{divārātrau ca pūjayet}
		}//}\\!}
\end{alttlg}

\newpage

\begin{altava}[hp04_011_5a]
atha \app{\lem[wit={C6,Jyo}]{khecarī}
	\rdg[wit={P11}]{khecarīsamādhiḥ}% +F
	}/ \sgwit{P11,C6,Jyo}
\end{altava}

\begin{alttlg}[hp04_011_5]
\tl{
\pada{\app{\lem[wit={G11,G5,N19}]{suṣiro}% +F
		\rdg[wit={C6}]{sukhiro}
		\rdg[wit={P11}]{susthiro}
		\rdg[wit={V3,J10}]{sukhiraṃ}
		\rdg[wit={V15}]{dṛṅmukhaṃ}}
	jñāna\app{\lem[wit={P11,C6,G11,G5,N19}]{janakaḥ}% +F
		\rdg[wit={V3,J10}]{janakaṃ}
		\rdg[wit={V15}]{jaṃnakaṃ}
		}}
\pada{pañcasrotaḥ% srota G5
%\app{\lem[wit={P11,C6,G11,G5,V15}]{srotaḥ}
%	\rdg[wit={V3,N19,J10}]{śrotaḥ}}%
	\app{\lem[wit={P11,C6,G11,G5,N19}]{samanvitaḥ}% +F
		\rdg[wit={V3}]{samanvita}
		\rdg[wit={V15}]{samanvitam}% J10pc
		\rdg[wit={J10}]{samanvite}}/}\\+}
\tl{
\pada{tiṣṭhate khecarī mudrā} % khacarī G11
\pada{\app{\lem[wit={J10}]{tasmin śūnye}% +F
		\rdg[wit={P11,C6,G11,G5,V15}]{tasmāc chūnye}
		\rdg[wit={N19}]{satyaṃ tatra}
		\rdg[wit={V3},alt={\om},post=\texteng{(eye-skip?)}]{}} % cf. V3 resumes with tasmin!
	\app{\lem[wit={P11,C6,G11,G5,V15,J10}]{nirañjane}
		\rdg[wit={N19},post=\texteng{(cf. Pāda d of the next verse)}]{na saṃśayaḥ}
		\rdg[wit={V3},alt={\om}]{\skp{\om}}}//}
	\NotIn{Jyo} \anm{=\,\manuref{3.49*1}}\\!}
\end{alttlg}
%\myfn{\getsiglum{L2} omits the 2nd half of this verse and the 1st half of the next verse (prob. by haplography) and adds them after the 1st half of 4.25*3.}

\begin{alttlg}[hp04_011_6]
\tl{
\pada{\app{\lem[wit={P11,C6,G11,G5,N19,V15,J10,Jyo},post=\texteng{(nāḍi \getsiglum{G11,G5,N19,J10})}]{savyadakṣiṇanāḍīstho}
		\rdg[wit={V3},alt={\om}]{\skp{\om}}}} % J6 has this omission too!!
\pada{\app{\lem[wit={G11,N19}]{madhye calati mārutaḥ}% ḥ om. N19
		\rdg[wit={P11,C6,G5,Jyo},post=\texteng{(madhyaṃ \getsiglum{P11})}]{madhye carati mārutaḥ}% madhyaṃ P11; <<carati>> C6
		\rdg[wit={J10}]{madhye vahati mārutaḥ}
		\rdg[wit={V15}]{madhyacaritamārutaḥ}
		\rdg[wit={V3},alt={\om}]{\skp{\om}}}/}\\+}
\tl{
\pada{\app{\lem[wit={P11,C6,G11,G5,N19,V15,J10,Jyo}]{tiṣṭhate khecarī mudrā} % tiṣṭhajña? N19
		\rdg[wit={V3},alt={\om}]{\skp{\om}}}}
\pada{\app{\lem[wit={GrB,G11,G5,V15,Jyo}]{tasmin sthāne}
		\rdg[wit={N19}]{satyaṃ tatra}
		\rdg[wit={J10}]{tatra satyaṃ}}
		na saṃśayaḥ//}\\!}
\end{alttlg}


\begin{alttlg}[hp04_011_7]
\tl{
\pada{cittaṃ carati khe yasmāj} % citraṃ N19; cārati V15; yasmā V15,V3
\pada{jihvā carati
	\app{\lem[wit={GrB,G5,N19}]{khe gatā}% ga<tā> P11
		\rdg[wit={G11}]{khe yadā}
		\rdg[wit={V15}]{vegataḥ}}/}\\+}
\tl{
\pada{\app{\lem[wit={P11,V3,G11,V15}]{tenaiṣā}% P6
		\rdg[wit={C6}]{teneyaṃ}
		\rdg[wit={N19}]{tenaiva}
		\rdg[wit={G5}]{iyaṃ ca}} khecarī  % khecarīṃ N19
	\app{\lem[wit={P11,V3,G11,G5,N19}]{nāma}
		\rdg[wit={C6,V15}]{mudrā}}}
\pada{\app{\lem[wit={P11,V3,G11,G5,N19}]{mudrā}% mudra P11
		\rdg[wit={V15}]{satyaṃ}
		\rdg[wit={C6}]{sarva}} 
		siddhai%r  % siddhai N19
	\app{\lem[wit={GrB,G11,G5,N19},alt={namaskṛtā}]{\skm{r }namaskṛtā}% °tāḥ P11
		\rdg[wit={V15}]{nigadyate}}//}
	\NotIn{J10,Jyo} \anm{=\,\manuref{3.38}}\\!}
\end{alttlg}

\begin{alttlg}[hp04_011_8]
\tl{
\pada{iḍāpiṅgalayo%r  % idā N19
	\app{\lem[wit={GrB,G11,G5,N19},alt={yoge}]{\skm{r }yoge}
		\rdg[wit={Jyo}]{madhye}}}
\pada{\app{\lem[wit={C6,G11,G5,Jyo}]{śūnyaṃ}
		\rdg[wit={P11,N19}]{śūnye}% +F,P6
		\rdg[wit={V3}]{śūne}} % +J11ac
	\app{\lem[wit={V3,G11,G5,N19,Jyo}]{caivānilaṃ}
		\rdg[wit={P11,C6}]{caiva bilaṃ}}
	\app{\lem[wit={P11,V3,G11,G5,N19,Jyo}]{graset}% grasit P11
		\rdg[wit={C6}]{viśet}}/}\\+}
\tl{
\pada{\app{\lem[wit={C6,V3,G11,G5,N19,Jyo}]{tiṣṭhate}
		\rdg[wit={P11}]{tiṣṭhati}} khecarī mudrā} % khacarī G11
\pada{\app{\lem[wit={P11,G11,G5}]{tatra satyaṃ na saṃśayaḥ}% P6,F
		\rdg[wit={N19}]{satyaṃ tatra na saṃśayaḥ}
		\rdg[wit={C6,V3,Jyo}]{tatra satyaṃ punaḥ punaḥ}
		}//}
\NotIn{V15,J10}\\!}
\end{alttlg}

\begin{alttlg}[hp04_011_9]
\tl{
\pada{\app{\lem[wit={G11,G5,N19,J10}]{somasūryadvayo\skp{r}} % +G7,P6; sūryā J10
		\rdg[wit={V15}]{candrasūryadvayor}
		\rdg[wit={GrB,Jyo},alt={sūryācandramasor}]{sūryācandramasor}% sūrya C6, sūryāc P11, °masaur V3
		}r madhye}
\pada{\app{\lem[wit={C6,V3,N19,V15}]{nirālambe tale}
		\rdg[wit={P11,G5}]{nirālambatale}% °laṃbatare M3,F
		\rdg[wit={G11}]{nirālambe kale}
		\rdg[wit={J10}]{nirālambo'ntarā}%
		\rdg[wit={Jyo}]{nirālambāntare}}
		punaḥ/}\\+}
\tl{
\pada{saṃsthitā vyomacakre yā} % cakreṇa G11; °ścaikacakre yā G5; 
\pada{sā mudrā nāma khecarī//}\\!}
\end{alttlg}

\begin{alttlg}[hp04_011_10]
\tl{
\pada{\app{\lem[wit={P11,V3}]{sā mayodbheditā vāmā}
		\rdg[wit={G11}]{sā māyodbhedikā vāmā}%  sā mayābhedatā M3, sā madhyodbheditā F
		\rdg[wit={G5}]{sā māyābhedito vāmā}
		\rdg[wit={N19}]{sā mayodve\,\_\,tā vāmā}
		\rdg[wit={V15}]{sā mayodve[dh]itā vāmā}
		\rdg[wit={J10}]{somayodbheditā dhāma}% °rbhidādhāmāmā P6
		\rdg[wit={Jyo}]{somād yatroditā dhārā}}}
\pada{\app{\lem[wit={P11,V3,N19,V15}]{sākṣāc ca}% +G7
		\rdg[wit={J10}]{sākṣād vai}
		\rdg[wit={G11,G5}]{sā sākṣāt}% +M3
		\rdg[wit={Jyo}]{sākṣāt sā}} śivavallabhā/}\\+}
\tl{
\pada{\app{\lem[wit={P11,V3,G11,G5,N19,V15},alt={pūrayen}]{pūraye\skp{n}}
		\rdg[wit={Jyo}]{pūrayed}
		\rdg[wit={J10}]{pūjayed}}%
	\app{\lem[wit={P11,V3,N19,V15},alt={mārutaṃ divyaṃ}]{\skm{n }mārutaṃ divyaṃ}% +G7
		\rdg[wit={G11}]{na tu tad divyaṃ}% + na tu tan divya M3
		\rdg[wit={G5}]{satataṃ divyaṃ}
		\rdg[wit={J10,Jyo}]{atulāṃ divyāṃ}}}
\pada{\app{\lem[wit={P11,V3,G11,G5,N19,V15,J10}]{suṣumṇā}
		\rdg[wit={Jyo}]{suṣumṇāṃ}}
	\app{\lem[wit={P11,G11,G5,N19,V15,J10,Jyo}]{paścime}%
		\rdg[wit={V3}]{paścimā}}
	mukhe//} \NotIn{C6}\\!}
\end{alttlg}

\newpage
\begin{alttlg}[hp04_011_11]
\tl{
\pada{purastāc caiva pūryeta} % °stāvaiva P11, °stācaiva V3, t* caiva N19; °yetā V3
\pada{\app{\lem[wit={GrB,G11,G5,N19,V15,Jyo}]{niścitā}% °taṃ F, °cittā G5
		\rdg[wit={J10}]{niśritā}
		} khecarī bhavet/}\\+}
\tl{
\pada{\app{\lem[wit={P11,C6,G11,G5,N19},alt={abhyaset}]{abhyase\skp{t}}
		\rdg[wit={V3}]{abhyase}
		\rdg[wit={J10,Jyo}]{abhyastā}
		\rdg[wit={V15},alt={\om},post=\texteng{(eye-skip?)}]{}}%
	\app{\lem[wit={C6,V3,G11,G5,N19},alt={khecarīmudrām}]{\skm{t }khecarīmudrā\skp{m}}
		\rdg[wit={P11}]{khecarīṃ mudrām}
		\rdg[wit={J10,Jyo}]{khecarīmudrā}
		\rdg[wit={V15},alt={\om}]{\skp{\om}}}}%
\pada{\app{\lem[wit={GrB,G11,G5,N19,J10},alt={unmanī}]{\skm{m }unmanī}
		\rdg[wit={Jyo}]{py unmanī}
		\rdg[wit={V15},alt={\om}]{\skp{\om}}}
	\app{\lem[wit={G11,G5,N19,J10,Jyo}]{saṃprajāyate}
		\rdg[wit={P11}]{sāṃdrajāyate}
		\rdg[wit={C6,V3}]{sā prajāyate}
		\rdg[wit={V15},alt={\om}]{\skp{\om}}}//}\\!}
\end{alttlg}

\begin{alttlg}[hp04_011_12]
\tl{
\pada{\app{\lem[wit={GrB,G11,N19,Jyo},alt={abhyaset}]{abhyase\skp{t}}
		\rdg[wit={V15}]{abhyasat}
		\rdg[wit={G5}]{abhyasya}
		\rdg[wit={J10}]{abhyaste}}%
	\app{\lem[wit={Jyo},alt={khecarīṃ}]{\skm{t }khecarīṃ}
		\rdg[wit={GrB,G11,G5,N19,V15,J10}]{khecarī}} % +J11
	\app{\lem[wit={GrB,Jyo},alt={tāvad}]{tāva\skp{d}}
		\rdg[wit={G11,V15,J10}]{mudrāṃ}
		\rdg[wit={G5,N19}]{mudrā}}}%
\pada{\app{\lem[wit={GrB,Jyo},alt={yāvat}]{\skm{d }yāva\skp{t}}
		\rdg[wit={G11,G5,N19,V15,J10}]{tāvat}}t syā%d
	\app{\lem[wit={C6,G11,N19,V15,Jyo},alt={yoganidritaḥ}]{\skm{d }yoganidritaḥ}
		\rdg[wit={P11}]{yoganidritāḥ}
		\rdg[wit={J10}]{yoganidratāḥ}
		\rdg[wit={V3}]{yoganiṃdrataḥ}
		\rdg[wit={G5}]{coramudritā}}/}\\+}
\tl{
\pada{saṃprāptayoganidrasya} % niṃdrasya V3
\pada{kālo nāsti kadācana//}% canaṃ V3, canā G11
\myfn{This verse and the next one are transposed in \getsiglum{Jyo}.}\\!}
\end{alttlg}

\begin{alttlg}[hp04_011_13]
\tl{
\pada{bhruvor madhye  % °vo P11, bhrūvaur N19
	\app{\lem[wit={C6,V3,N19,V15,J10,Jyo}]{śiva}
	\rdg[wit={P11}]{bhavet}}sthānaṃ}
\pada{manas tatra vilīyate/}  % vilīyati P11
	\lineom{ab}{G11,G5}\\+} % M3 omits too.
\tl{
\pada{jñātavyaṃ tat padaṃ turyaṃ} % talpadaṃ G11; tūryaṃ P11,V15
\pada{\app{\lem[wit={GrB,G11,G5,N19,J10,Jyo}]{tatra}
		\rdg[wit={V15}]{yatra}}
	\app{\lem[wit={GrB,G11,G5,V15,J10,Jyo}]{kālo}
		\rdg[wit={N19}]{kopi}} 
		na vidyate//}\\!}
\end{alttlg}

\begin{alttlg}[hp04_011_14]
\tl{
\pada{candrasūryadvayor madhye}
\pada{\app{\lem[wit={GrB,G11,G5,V15,J10}]{mudrāṃ}
		\rdg[wit={N19}]{mudrā}}
	\app{\lem[wit={GrB,G5,V15,J10}]{dadyāc ca} % +M3; dadyā V3
		\rdg[wit={G11}]{dadyāt tu}
		\rdg[wit={N19}]{divyā ca}}
	\app{\lem[wit={C6,G11,G5,V15,J10}]{khecarīm}
		\rdg[wit={V3,N19}]{khecarī}
		\rdg[wit={P11}]{khecare}}/}\\+}
\tl{
\pada{\app{\lem[wit={C6,G11,J10}]{nirālambe}
		\rdg[wit={V3,N19,V15}]{nirālambaṃ}% +F
		\rdg[wit={P11}]{nirālambas}
		\rdg[wit={G5}]{nirālamba}}
	\app{\lem[wit={C6,J10}]{mahāśūnye}
		\rdg[wit={N19,V15}]{mahāśūnyaṃ}% +F
		\rdg[wit={G11}]{mahacchūnye}
		\rdg[wit={V3,G5}]{mahāśūnya}
		\rdg[wit={P11}]{tadā śūnya}}}
\pada{vyoma\app{\lem[wit={GrB,G11,G5,N19,J10}]{cakre}
		\rdg[wit={V15}]{cakraṃ}}
	\app{\lem[wit={C6,V3,G11,J10}]{vyavasthitām}
		\rdg[wit={N19,V15}]{vyavasthitaṃ}% +G7,F
		\rdg[wit={P11,G5}]{vyavasthitā}% P6; °taḥ M3?
		}//}
\NotIn{Jyo}\\!} % cf. 4.25*6
\end{alttlg}

\begin{alttlg}[hp04_011_15]
\tl{
\pada{nirālambaṃ manaḥ kṛtvā} % manaṃ P11
\pada{na kiṃcid api cintayet/}\\+}
\tl{
\pada{sabāhyā\app{\lem[wit={GrB,G11,N19,V15,Jyo},alt={°bhyantare}]{\skp{°}bhyantare}
		\rdg[wit={G5,J10}]{bhyantaraṃ}} vyomni} % vyomani G11, vyoma G5
\pada{\app{\lem[wit={GrB,V15,J10,Jyo},alt={ghaṭavat}]{ghaṭava\skp{t}}
		\rdg[wit={N19}]{paṭavat}
		\rdg[wit={G11}]{aṭavat}
		\rdg[wit={G5}]{maghaṭat}}%
	\app{\lem[wit={G11,G5,N19,V15},alt={tiṣṭhate}]{\skm{t }tiṣṭhate}
		\rdg[wit={GrB,J10,Jyo}]{tiṣṭhati}} 
		dhruvam//}\\!}
\end{alttlg}

\begin{alttlg}[hp04_011_16]
\tl{
\pada{bāhyavāyu%r
	\app{\lem[wit={GrB,J10,Jyo},alt={yathā}]{\skm{r }yathā}% P6
		\rdg[wit={G11,G5}]{tathā}% +M3
		\rdg[wit={N19,V15}]{yadā}}
	\app{\lem[wit={C6,G11,V15}]{līnaḥ}
		\rdg[wit={P11,N19}]{līna}
		\rdg[wit={V3}]{līnaṃ}
		\rdg[wit={G5}]{līnā}
		\rdg[wit={J10,Jyo}]{līnas}}}
\pada{\app{\lem[wit={P11,V3,G11,G5}]{khasya madhye}
		\rdg[wit={C6}]{khamadhye tu}
		\rdg[wit={V15}]{khamadhye ca}
		\rdg[wit={N19}]{khamadhya\,\_}
		\rdg[wit={J10}]{tathā madhye}
		\rdg[wit={Jyo}]{tathā madhyo}}
	 \app{\lem[wit={GrB,G11,G5,V15,J10,Jyo}]{na saṃśayaḥ}
		\rdg[wit={N19}]{\_\,\_\,sayaḥ}}/}\\+}
\tl{
\pada{\app{\lem[wit={GrB,G11,N19,V15,J10}]{svasthānaṃ gacchati prāṇaḥ} % gacchatiṃ G11; prāṇa N19,V3,F
	\rdg[wit={G5}]{saṃsthānaṃ gacchati prāṇaḥ}
	\rdg[wit={Jyo}]{svasthāne sthiratām eti}}}
\pada{\crux\app{\lem[wit={GrB,G11,G5,N19,V15}]{sūryāṅge manasā tathā}% sūryāgnir M1; manatā P6, mānasaṃ G7
		\rdg[wit={J10}]{sūryāṅge pavane tathā}% +P17
		\rdg[wit={Jyo}]{pavano manasā saha}}\crux//}\\!} % sveyāṅge F
\end{alttlg}

\newpage
\begin{alttlg}[hp04_011_17]
\tl{
\pada{eva\app{\lem[wit={GrB,G11,G5,N19,V15,J10},alt={abhyasyamānasya}]{\skm{m }abhyasyamānasya}
		\rdg[wit={Jyo}]{abhyasyatas tasya}}}
\pada{\app{\lem[wit={GrB,G11,G5,J10,Jyo}]{vāyumārge}% vāyurmāge C6
		\rdg[wit={N19,V15}]{vāyor mārge}} % vāyomārgre N19, mārgaṃ J10
	\app{\lem[wit={C6,Jyo}]{divāniśam}% +M1,F
		\rdg[wit={P11}]{divā niśi}
		\rdg[wit={V3}]{divādisam}
		\rdg[wit={G11,G5,J10}]{sadāniśaṃ}
		\rdg[wit={N19,V15}]{sadānilaṃ}}/}\\+}
\tl{
\pada{\app{\lem[wit={GrB,G11,G5,N19,J10,Jyo}]{abhyāsāj jīryate} % abhyāsā V3; jāyate J10ac?
		\rdg[wit={V15}]{abhyāsāl līyate}}
		vāyur} % vāyu V3
\pada{mana\app{\lem[wit={G11,G5,N19,V15,J10},alt={tatra vilīyate}]{\skm{s }tatra vilīyate}% P6
		\rdg[wit={GrB,Jyo}]{tatraiva līyate}}//}\\!}% +F
\end{alttlg}


\begin{alttlg}[hp04_011_18]
\tl{
\pada{\app{\lem[wit={P11,V3,N19},alt={amṛtaṃ plāvayed deham}]{amṛtaṃ plāvayed deha\skp{m}}
		\rdg[wit={G11,G5},postwit=\texteng{(amṛtā \getsiglum{G11})}]{amṛtāt plāvayed deham}% amṛtā G11; +F
		\rdg[wit={V15}]{amṛte plāvayed deham}
		\rdg[wit={C6}]{amṛtaṃ plavate \_\,\_}
		\rdg[wit={Jyo}]{amṛtaiḥ plāvayed deham}
		\rdg[wit={J10}]{ajaratvaṃ bhaved dehe}}}%m}
\pada{\app{\lem[wit={ceteri},alt={ā pādatala}]{\skm{m }ā pādatala}
		\rdg[wit={J10}]{apādapala}
		\rdg[wit={C6},alt={\lacuna}]{\skp{\lacuna}}}%
	\app{\lem[wit={GrB,G11,G5,V15,Jyo}]{mastakam} % dehaṃāpādattala°  apādapala J10
		\rdg[wit={J10}]{mastake}
		\rdg[wit={N19}]{mastakān}
		\rdg[wit={C6},alt={\lacuna}]{\skp{\lacuna}}}/}\\+}
\tl{
\pada{\app{\lem[wit={V3,G11,G5,Jyo}]{sidhyaty eva}% °ateva G11
		\rdg[wit={N19}]{siddhaty eva}
		\rdg[wit={V15}]{siddhyaty evaṃ}
		\rdg[wit={J10}]{sidhyate ca}
		\rdg[wit={C6}]{siddhadeho}
		\rdg[wit={P11}]{siddhideho}}
	\app{\lem[wit={V3,G11,G5,N19}]{sadā kāyo}
		\rdg[wit={C6,Jyo}]{mahākāyo}
		\rdg[wit={P11}]{mahākāryo}
		\rdg[wit={J10}]{mahāyogo}
		\rdg[wit={V15}]{tadā kāyo}% +F
		}}
\pada{mahābalaparākramaḥ//}\\!} % °kramī P11
\end{alttlg}

\begin{altpostmula}[hp04_011_18p]
iti khecarī/ \sgwit{Jyo}
\end{altpostmula}

\begin{altava}[hp04_011_19a]
\app{\lem[wit={N19}]{atha}\rdg[wit={P11},alt={\om}]{\skp{\om}}}
\app{\lem[wit={P11}]{śāṃbhavī}
	\rdg[wit={N19}]{śāṃbhavī śaktiḥ}}/ \sgwit{P11,N19}% +F
\end{altava}

\begin{alttlg}[hp04_011_19]
\tl{
\pada{śaktimadhye manaḥ kṛtvā} % mana P11
\pada{\app{\lem[wit={G11,G5,N19}]{śaktiṃ ca manamadhyagām}% P6
		\rdg[wit={V15}]{śaktiṃ ca svāṃtamadhyagām}
		\rdg[wit={Jyo}]{śaktiṃ mānasamadhyagām}
		\rdg[wit={J10}]{śaktiṃ manasi madhyataḥ}
		\rdg[wit={P11}]{sumadhyagaṃ}
		\rdg[wit={C6,V3}]{manaḥ śaktes tu madhyagam}% +F
		}/}\\+}
\tl{
\pada{manasā
	\app{\lem[wit={P11,C6,G11,G5,V15,J10,Jyo}]{mana ālokya}
		\rdg[wit={N19}]{mana ārokya}
		\rdg[wit={V3}]{manam ālokya}}}
\pada{\app{\lem[wit={C6,G11,N19,V15},alt={tad dhyāyet}]{tad dhyāye\skp{t}}
		\rdg[wit={G5}]{taṃ dhyāyet}
		\rdg[wit={P11}]{taṃ dhātaṃ}
		\rdg[wit={V3}]{vaddhyāyait}
		\rdg[wit={J10,Jyo}]{dhārayet}% dhāryate P6
		}t paramaṃ padam//}\\!}
\end{alttlg}

\begin{alttlg}[hp04_011_20]
\tl{
\pada{\app{\lem[wit={C6,V3,G11,G5,N19,V15,J10,Jyo}]{khamadhye}
		\rdg[wit={P11}]{khaṃmadhye}% +F
		} kuru cātmāna}%m}
\pada{\app{\lem[wit={V3,G11,G5,V15,Jyo},alt={ātmamadhye}]{\skm{m }ātmamadhye} % J10pc
		\rdg[wit={P11,C6,N19,J10}]{ātmāmadhye}} % J10ac
		ca khaṃ kuru/}\\+}
\tl{
\pada{\app{\lem[wit={C6,V3,G11}]{ātmānaṃ}% +G7,??
		\rdg[wit={G5,N19,V15,J10,Jyo}]{sarvaṃ ca}% +M1,P6,G10,F
		\rdg[wit={P11}]{evaṃ kṛ°}}
	\app{\lem[wit={V3,N19,V15,Jyo}]{khamayaṃ kṛtvā}
		\rdg[wit={C6,G11,G5,J10}]{khaṃmayaṃ kṛtvā}% +F
		\rdg[wit={P11}]{°tvā tayoś cāpi}}}
\pada{na kiṃcid api cintayet//}\\!}
\end{alttlg}

\begin{alttlg}[hp04_011_21]
\tl{
\pada{antaḥśūnyo bahiḥśūnyaḥ} % aṃtaśūnye °śūnyo J10; bahiśūnyaṃ P11
\pada{\app{\lem[wit={P11,C6,G11,G5,J10}]{śūnya}
	\rdg[wit={Jyo}]{śūnyaḥ}}kumbha ivāmbare/}\\+} % °baro P11
\tl{
\pada{antaḥpūrṇo bahiḥpūrṇaḥ} % bahipūrṇa P11, pūrṇa<<ḥ>> J10
\pada{\app{\lem[wit={P11,C6,G5,J10}]{pūrṇa}
	\rdg[wit={Jyo}]{pūrṇaḥ}}kumbha
	\app{\lem[wit={G5,J10,Jyo}]{ivārṇave}% ivādhare P6
	\rdg[wit={P11}]{ivāṃbare}% +N22
	\rdg[wit={C6}]{ivāmbudhau}}//}
	\lineom{cd}{G11} \NotIn{V3,N19,V15}\\!} % G5 vollständig, M3 om.
\end{alttlg}


\newpage
\begin{alttlg}[hp04_011_22]
\tl{
\pada{bāhyacintā na kartavyā} % °tavyo J10
\pada{tathaivāntara%
	\app{\lem[wit={G11,G5,J10,Jyo}]{cintanam}% P6/N22
		\rdg[wit={C6,V3}]{cintanā}% +F
		\rdg[wit={P11}]{ciṃtamān}}/}\\+}
\tl{
\pada{\app{\lem[wit={C6,G11,G5,Jyo}]{sarvacintāṃ parityajya}
	\rdg[wit={P11,V3}]{sarvacintā parityajya}
	\rdg[wit={J10}]{sarvacintā parityājyā}}}
\pada{na kiṃcid api cintayet//}
\NotIn{N19,V15}\\!} % M3 omits. too, but G11,G5 have it.
\end{alttlg}


%\newpage
\begin{alttlg}[hp04_011_23]
\tl{
\pada{saṃkalpamātra%
	\app{\lem[wit={G11,G5,N19,V15,J10,Jyo}]{kalanaiva}% ka<la>naiva G11
		\rdg[wit={V3}]{kalanaṃ ca}} jaga%t
	\app{\lem[wit={V3,G11,G5,N19,V15,Jyo},alt={samagraṃ}]{\skm{t }samagraṃ}
		\rdg[wit={J10}]{samastaṃ}}} \lineom{a}{P11,C6}\\+}
\tl{
\pada{saṃkalpamātra% kal<p>a G11
	\app{\lem[wit={V3,G11,G5,N19,V15}]{kalanā hi}% karahāṇi P6,N22b
		\rdg[wit={J10,Jyo}]{kalanaiva}} % phalanaiva N22a
	mano\app{\lem[wit={G5,J10,Jyo}]{vilāsaḥ}% +P6,M3
		\rdg[wit={V3}]{vilāsā}
		\rdg[wit={G11}]{vivāsaḥ}
		\rdg[wit={N19}]{vilīnā}
		\rdg[wit={V15}]{valīnā}}/} \lineom{b}{P11,C6}\\+}
\tl{
\pada{\app{\lem[wit={G11}]{saṃkalpam etam ata}
		\rdg[wit={V15}]{saṃkalpamātramatam}% P6,N22
		\rdg[wit={N19},alt={°mātramata}]{saṃkalpamātramata}
		\rdg[wit={G5},alt={°mātramanam}]{saṃkalpamātramanam}
		\rdg[wit={Jyo},alt={°mātramatim}]{saṃkalpamātramatim}
		\rdg[wit={P11},alt={°mātrami[m]}]{saṃkalpamātrami[m]}
		\rdg[wit={V3},alt={°mātram idam}]{saṃkalpamātram idam}
		\rdg[wit={J10},alt={°mātrakalanaiva}]{saṃkalpamātrakalanaiva}
		}
	\app{\lem[wit={P11,V3,G11,G5,V15}]{utsṛja}
		\rdg[wit={Jyo}]{utsṛjya}
		\rdg[wit={N19}]{tsṛja}
		\rdg[wit={J10}]{vikṛtis tu}}
	\app{\lem[wit={P11,V3,G11,G5,N19,V15,Jyo}]{nirvikalpaṃ}
		\rdg[wit={J10}]{nityaṃ}}} \lineom{c}{C6}\\+}
\tl{
\pada{\app{\lem[wit={P11,V3,G11,G5,N19,Jyo}]{āśritya} % āśrītya V3
		\rdg[wit={V15}]{āśrita}
		\rdg[wit={J10}]{saṃkalpa}}
	\app{\lem[wit={G11,J10,Jyo},alt={niścayam}]{niścaya\skp{m}} % P6; = South Ind. 4c
		\rdg[wit={P11,G5}]{niścalam}
		\rdg[wit={V3}]{niścalayam}
		\rdg[wit={N19,V15}]{niścitam}}%
	\app{\lem[wit={V3,G11,G5,N19,V15,Jyo},alt={avāpnuhi}]{\skm{m }avāpnuhi}
		\rdg[wit={J10}]{avāpnudhi}
		\rdg[wit={P11}]{anāpnuhi}}
	\app{\lem[wit={P11,V3,G11,J10,Jyo}]{rāma}% śāmya P6, ?? N22
		\rdg[wit={G5}]{kāma}
		\rdg[wit={V15}]{rāga}
		\rdg[wit={N19}]{roga}} 
		śāntim//} \lineom{d}{C6}\\!} % śānti P11,V3
\end{alttlg}

%\newpage
\begin{alttlg}[hp04_011_24]
\tl{
\pada{karpūra%m
	\app{\lem[wit={P11,V3,G11,G5,N19,V15,Jyo},alt={anale}]{\skm{m }anale}
		\rdg[wit={C6}]{anile}} yadvat}
\pada{saindhavaṃ salile yathā/}\\+}
\tl{
\pada{\app{\lem[wit={GrB,G11,G5,V15,Jyo}]{tathā}
		\rdg[wit={N19}]{yathā}}
	\app{\lem[wit={GrB,G11,G5,Jyo}]{saṃdhīyamānaṃ ca}
		\rdg[wit={N19,V15}]{saṃdīpamānaṃ ca}}}
\pada{mana\app{\lem[wit={C6,G11,V15,Jyo},alt={tattve}]{\skm{s }tattve}
		\rdg[wit={P11}]{tātva}
		\rdg[wit={V3}]{tatva}
		\rdg[wit={G5,N19}]{tatra}}
	\app{\lem[wit={GrB,G11,G5,N19,Jyo}]{vilīyate}
		\rdg[wit={V15}]{valīyate}}//}\label{karpura}
	\NotIn{J10}\\!}
\end{alttlg}

\begin{alttlg}[hp04_011_25]
\tl{
\pada{jñeyaṃ % jñoyaṃ N19
	\app{\lem[wit={P11,C6,G11,G5,Jyo}]{sarvaṃ pratītaṃ}
		\rdg[wit={V3,N19,V15}]{sarvapratītaṃ}
		\rdg[wit={J10}]{sarvam atītaṃ}} ca}
\pada{\app{\lem[wit={G11,N19,V15}]{tajjñānaṃ}
		\rdg[wit={G5}]{tat jñātaṃ}
		\rdg[wit={J10,Jyo}]{jñānaṃ ca}% jñāna J10; +F
		\rdg[wit={GrB}]{jñānaṃ tu} % P6
		} mana ucyate/}\\+} % ucyata P11
\tl{
\pada{jñānaṃ % + yaṃ! G11; but jñānaṃ G5,M3
	  jñeyaṃ % m. om. V3, jñoyaṃ N19
	\app{\lem[wit={GrB,G11,G5,N19,V15,Jyo}]{samaṃ naṣṭaṃ}
		\rdg[wit={J10}]{manaś caiva}}}
\pada{\app{\lem[wit={ceteri}]{nānyaḥ}
		\rdg[wit={P11}]{mānyaḥ}}
	\app{\lem[wit={C6,G11,G5,N19,J10,Jyo}]{panthā}
		\rdg[wit={V15}]{paṃtha}
		\rdg[wit={P11}]{paṃthyā}
		\rdg[wit={V3}]{pathā}}
	\app{\lem[wit={C6,G11,G5,V15,J10,Jyo}]{dvitīyakaḥ}
		\rdg[wit={P11,N19}]{dvitīyakaṃ}
		\rdg[wit={V3}]{dvitiyaka}}//}\\!}
\end{alttlg}

\begin{alttlg}[hp04_011_26]
\tl{
\pada{manodṛśyam idaṃ sarvaṃ} % the last 3 akṣaras damaged G11
\pada{yat kiṃcit sacarācaraṃ/}\\+}
\tl{
\pada{\app{\lem[wit={J10,Jyo}]{manaso hy unmanī}
		\rdg[wit={G11}]{manaso hy amanī}
		\rdg[wit={GrB,G5,V15}]{manasopy unmanī}% +M3
		\rdg[wit={N19}]{mano sopy unmanī}
		}%
	\app{\lem[wit={V3,V15,J10pc},alt={°bhāve}]{\skp{°}bhāve}% +G7,F
		\rdg[wit={P11}]{bhāvai}
		\rdg[wit={C6,G11,G5}]{bhāvo}% +M3
		\rdg[wit={J10ac}]{bhāvavo}
		\rdg[wit={Jyo}]{bhāvād}
		\rdg[wit={N19},alt={\om},post=\texteng{(eye-skip?)}]{}}}
\pada{\app{\lem[wit={P11,C6,V15}]{dvaitābhāvaṃ}% +M3,G7; bhāvaḥ G5
		\rdg[wit={G11}]{dvaitābhā\,+}
		\rdg[wit={G5}]{dvaitābhāvaḥ}
		\rdg[wit={V3}]{dvaitābhāva}
		\rdg[wit={N19}]{bhāvaṃ}
		\rdg[wit={J10,Jyo}]{dvaitaṃ naivo°}}
	\app{\lem[wit={C6,V3,G11,G5,V15}]{pracakṣate}% +Loc-Nom
		\rdg[wit={P11,N19}]{pracakṣyate}% +Nom-Nom
		\rdg[wit={J10,Jyo}]{°palabhyate}}//}\\!}
\end{alttlg}

%\newpage
\begin{alttlg}[hp04_011_27]
\tl{
\pada{jñeyavastuparityāgād} % jñeyaṃ?  parī° N19; jñeyaṃ yas tu P11
\pada{vilayaṃ yāti % yāṃti P11, damaged G11
	\app{\lem[wit={GrB,G11,G5,V15,J10,Jyo}]{mānasam}
		\rdg[wit={N19}]{mārutaṃ}}/}\\+}
\tl{
\pada{\app{\lem[wit={GrB,G11,N19,V15}]{mānase}
		\rdg[wit={G5,J10,Jyo}]{manaso}}
	\app{\lem[wit={P11,V3,G11,G5,N19,V15,J10}]{vilayaṃ}
		\rdg[wit={C6,Jyo}]{vilaye}}
	\app{\lem[wit={P11,G11,N19,V15}]{yāte}
		\rdg[wit={G5}]{yāti}
		\rdg[wit={C6,V3,J10,Jyo}]{jāte}}}
\pada{kaivalya%m
	\app{\lem[wit={GrB,G11,G5,V15,Jyo},alt={avaśiṣyate}]{\skm{m }avaśiṣyate}
		\rdg[wit={N19}]{anasīṣyate}
		\rdg[wit={J10}]{api kalpate}}//}\\!}
\end{alttlg}

\newpage
\begin{alttlg}[hp04_011_28]
\tl{
\pada{layo laya iti prāhuḥ} % first 2 akṣaras damaged G11; layaṃ? C6; prāhur N19,V15
\pada{\app{\lem[wit={GrB,G11,G5,J10,Jyo}]{kīdṛśaṃ}
		\rdg[wit={N19,V15}]{īdṛśaṃ}} layalakṣaṇam/}\\+}
\tl{
\pada{apunarvāsano% apu<na>r° N19
	\app{\lem[wit={P11,C6,N19,J10,Jyo},alt={°tthānāt}]{\skp{°}tthānāt}% °nāl P11,Jyo, °tthā .. l C6, nāt* N19, °nād J10
		\rdg[wit={V3,G11,V15}]{°tthānā}
		\rdg[wit={G5}]{°tthāna}}} % +P7
\pada{\app{\lem[wit={GrB,G5,N19,V15,Jyo}]{layo viṣaya}
		\rdg[wit={G11}]{yalo viṣaya}%
		\rdg[wit={J10}]{vṛttyayā viśva}
		}vismṛtiḥ//}% °smṛti N19,V3; °smati P11, niśrutiḥ G5
	\label{layo}\myfn{\getsiglum{Jyo} has this verse between \ref{yatradrsti} and \ref{vedasastra}.}\\!}
\end{alttlg}

\begin{alttlg}[hp04_011_29]
\tl{
\pada{evaṃ nānāvidhopāyāḥ} % eva N19; ḥ om. P11,V3
\pada{samyaksvānu%
	\app{\lem[wit={GrB,G11,N19,J10,Jyo}]{bhavānvitāḥ} % ḥ om. V3; sānubhavanvitā P11
		\rdg[wit={G5}]{bhavānyuta}
		\rdg[wit={V15}]{bhavātmikāḥ}}/}\\+}
\tl{
\pada{samādhi\app{\lem[wit={P11,C6,G11,G5,N19,V15,Jyo}]{mārgāḥ}% ḥ om. P11
		\rdg[wit={J10}]{mārge}
		\rdg[wit={V3},alt={\illeg}]{\skp{\illeg}}} 
		kathitāḥ} % °tā V15
\pada{pūrvācāryair mahātmabhiḥ//}\\!}% °ryai P11,V3
%\myfn{After this verse, \getsiglum{P11,C6} have \devnote{iti viśrāntiḥ} and \getsiglum{N19,V15} \devnote{atha viśrāntiḥ}.}
\end{alttlg}

%\newpage
\begin{altava}[hp04_011_30a]
\app{\lem[wit={G11,G5,N19,V15}]{atha}
	\rdg[wit={P11,C6}]{iti}% °ti P11; +F
	} viśrāntiḥ/ 
	\NotIn{V3,Jyo}
%	\sgwit{P11,C6,G11,G5,N19,V15}
\end{altava}

\begin{alttlg}[hp04_011_30]
\tl{
\pada{\app{\lem[wit={GrB,G11,G5,V15,Jyo}]{suṣumṇāyai}
		\rdg[wit={N19}]{sukhayaiḥ}} kuṇḍalinyai}
\pada{sudhāyai candra% ceṃdra V3
	\app{\lem[wit={G11,G5}]{maṇḍale}% +M3
		\rdg[wit={N19,V15}]{maṇḍalāt}
		\rdg[wit={GrB,Jyo}]{janmane} % M1
		}/}\\+} % +G7
\tl{
\pada{manonmanyai namas tubhyaṃ} % °nye P11,G5
\pada{mahā\app{\lem[wit={P11,C6,G11,G5,N19,V15}]{śakti}
		\rdg[wit={V3}]{śakte}
		\rdg[wit={Jyo}]{śaktyai}% +F
		}%
	\app{\lem[wit={ceteri}]{cidātmane}
		\rdg[wit={P11}]{cidātmani}% +G7 
		\rdg[wit={G11}]{cidātmike}% °ātmake M3
		\rdg[wit={G5}]{cidātmine}}//}
	\NotIn{J10}\\!}
\end{alttlg}


\begin{alttlg}[hp04_011_31]
\tl{
\pada{\app{\lem[wit={P11,G5,N19,V15,Jyo}]{aśakya}% +G5
		\rdg[wit={G11,J10}]{aśakyaṃ}
		\rdg[wit={C6,V3}]{aśakta}}tattvabodhānāṃ} % ṃ om. G11
\pada{\app{\lem[wit={C6,V3,G11,G5,N19,V15,J10,Jyo},alt={mūḍhānām}]{mūḍhānā\skp{m}}
		\rdg[wit={P11}]{gūḍhānām}}%
	\app{\lem[wit={GrB,G11,G5,J10,Jyo},alt={api saṃmatam}]{\skm{m }api saṃmatam}
		\rdg[wit={V15}]{api saṃtataṃ}
		\rdg[wit={N19}]{atisaṃtataṃ}}/} \anm{cf. \ref{saukhya}ab}\\+}
\tl{
\pada{proktaṃ 
	\app{\lem[wit={ceteri}]{gorakṣa}
		\rdg[wit={G11,G5}]{śrīśaṃbhu}}nāthena}
\pada{nādopāsana%m % nāptepā° P11
	\app{\lem[wit={P11,V3,G11,G5,N19,V15,J10,Jyo},alt={ucyate}]{\skm{m }ucyate}
		\rdg[wit={C6}]{uttamam}}//}\\!}
\end{alttlg}

\endaltrecension

%\newpage

\begin{tlg}[hp04_012]% = 4c 4.64
\tl{
\pada{\app{\lem[wit={ceteri}]{śrīādināthena}% nāthona N23
	\rdg[wit={G11,G5}]{śrīśaṃbhunāthena}} 
	sapādakoṭi}-\\+}% koṭī G11
\tl{
\pada{\app{\lem[wit={ceteri}]{laya}
		\rdg[wit={N3,Gr2,N19}]{layaḥ}
		\rdg[wit={J5}]{laṣa}
		}prakārāḥ % prakār<<āḥ>> J10
		kathitā  % kathaṃ J10
	\app{\lem[wit={N3,J5,G11,G5,N19}]{jayante}% +F
		\rdg[wit={GrB,Gr2,E2,V15,J10,Jyo}]{jayanti} % jayati V17
		\rdg[wit={V19}]{yayaṃti}}/}\\+}
\tl{
\pada{nādānusaṃdhānaka%m % nodānasaṃdhānasaṃdhānakam N19
	\app{\lem[wit={N3,P11,C6,G11,G5,Jyo},alt={ekam eva}]{\skm{m }ekam eva}% +F
		\rdg[wit={J5,V3}]{eva}
		\rdg[wit={N19,J10}]{eva nānyaṃ}
		\rdg[wit={V15}]{eva mānyaṃ}
		\rdg[wit={Gr2,Gr3a}]{eva kāryaṃ}}}\\+} % °karmeva kārya N23
\tl{
\pada{\app{\lem[wit={ceteri}]{manyāmahe} % manyāṃmahe N19
	\rdg[wit={C6}]{gaṇyāmahe}}
	\app{\lem[wit={N3,P11,V3,N19,V15}]{mānyatamaṃ}% myanya° P11
		\rdg[wit={J5,Gr2,Gr3a,G11,G5}]{nānyatamaṃ}% +F
		\rdg[wit={C6}]{nānyamataṃ}
		\rdg[wit={J10}]{tātarasaṃ}
		\rdg[wit={Jyo}]{mukhyatamaṃ}} 
		layānām//}\label{sapadakoti}\\!} % layāṇāṃ J10
\end{tlg}


\Anm{\getsiglum{GrB,G11plus,N19,V15,J10,Jyo} have \ref{sravanaputa} \textit{śravaṇamukhanayana} here}


\begin{tlg}[hp04_013]% (cf. 4.47)
\tl{
\pada{\app{\lem[wit={N3,J5,C6,E2}]{muktāsana}% +K3,C7
	\rdg[wit={V19,Jyo}]{muktāsane}}sthito yogī} % or: °sāna? V19
\pada{mudrāṃ saṃdhāya śāṃbhavīm/} % sadhāya? V19
	\lineom{ab}{P11,V3,Gr2,N19,V15,J10}
	%\sgwit{Gr1,C6,V19,K3,C7,Jyo}
	\\+}
\tl{
\pada{śṛṇuyād dakṣiṇe karṇe} % śṛṇuyā J7, śuṇu° V19
\pada{nāda\app{\lem[resp=emend]{\skm{m }antaḥstham ekadhīḥ}% 
	\rdg[wit={N3,G4,P11,Gr2,Jyo}]{antastham ekadhīḥ}% aṃtta G4
	\rdg[wit={J5}]{atastham ekadhā}
	\rdg[wit={V19}]{ekāntake sudhīḥ}
	\rdg[wit={E2}]{ekāntike sudhīḥ}% +K3,C7
	\rdg[wit={C6}]{ataṃ sadā}}//}\label{muktasana2}
	\lineom{cd}{P11,V3,N19,V15,J10}
	%\sgwit{Gr1,P11,C6,Gr2,Gr3a,Jyo} 
\anm{cf. \ref{muktasana}}\\!}
\end{tlg}
%\NotIn{V3,G11,N19,V15,J10}


\newpage

\Anm{\getsiglum{G11,N19,V15,J10} have the following 5 verses after \ref{yatrakutrapi}, and \getsiglum{GrB} after \ref{muktasana}}


\begin{tlg}[hp04_014]% = V3_4.65 = C8_68
\tl{
\pada{\app{\lem[wit={ceteri}]{kāṣṭhe}
		\rdg[wit={C6,J7,Gr3a}]{kāṣṭhaiḥ}
		\rdg[wit={N23}]{kaṣṭaiḥ}}
	\app{\lem[wit={ceteri}]{pravartito}% pravartte J5
		\rdg[wit={V15,J10}]{pravartate}} vahniḥ} % vahni J5,P11,V3
\pada{\app{\lem[wit={ceteri}]{kāṣṭhena}
		\rdg[wit={N23}]{kaṣṭena}}
	\app{\lem[wit={ceteri}]{saha}
		\rdg[wit={V15}]{sa}}
	\app{\lem[wit={ceteri}]{śāmyati}
		\rdg[wit={N3,J5,V3,V19}]{sāmyati}
		\rdg[wit={V15}]{līyate}}/}\\+}
\tl{
\pada{\app{\lem[wit={ceteri}]{nāde}% nādena J5
		\rdg[wit={N23}]{nā}}
	\app{\lem[wit={ceteri}]{pravartitaṃ} %°taṃś N19
		\rdg[wit={V15}]{pravartite}
		\rdg[wit={J10}]{pravartate}}
	\app{\lem[wit={ceteri}]{cittaṃ}
		\rdg[wit={N23},alt={\om}]{\skp{\om}}}}
\pada{nādena saha līyate//}\label{kasthe}\\!}
\end{tlg}

\begin{tlg}[hp04_015]% = V3_4.66
\tl{
\pada{\app{\lem[wit={ceteri}]{vismṛtya}
%		\rdg[wit={C7}]{nismṛtya}
		\rdg[wit={E2}]{niḥsṛtya}
		} sakalaṃ bāhyaṃ} % bāhya N19
\pada{\app{\lem[wit={N3,J5,GrB,J7,Gr3a,V15}]{nāde}% +G5
		\rdg[wit={G11}]{nādo}
		\rdg[wit={N19}]{nāda}
		\rdg[wit={N23}]{na\_}}
	\app{\lem[wit={ceteri}]{dugdhāmbu}
		\rdg[wit={N23}]{gugyāṃbu}}va%n
	\app{\lem[wit={ceteri},alt={manaḥ}]{\skm{n }manaḥ}
		\rdg[wit={V3}]{mana}
		\rdg[wit={N23,Gr3a}]{naraḥ}}/}\\+}
\tl{
\pada{\app{\lem[wit={G4,C6,Gr2,E2,G11,N19,V15}]{ekībhūyātha}
		\rdg[wit={J5}]{ekībhūyotha}
		\rdg[wit={P11}]{ekībhūyādya}
		\rdg[wit={V19}]{ekībhūyāya}
		\rdg[wit={V3}]{ekībhūyā}
		\rdg[wit={N3}]{ekībhūtvātha}}
	\app{\lem[wit={ceteri}]{sahasā}
		\rdg[wit={V3}]{sahasā ca}
		\rdg[wit={J5}]{manasā}}}
	% ekībhūyād atha saha cidā(page break)ekībhūyātha sahasā J7
\pada{\app{\lem[wit={cetwG4}]{cidākāśe}
		\rdg[wit={J5}]{cidāśe}
		\rdg[wit={N23}]{vidāktośe}
		\rdg[wit={J7}]{cidākaro}} 
\app{\lem[wit={ceteri}]{vilīyate}
	\rdg[wit={N3}]{valīyate}
	\rdg[wit={G4}]{na lipyate}}//}
		\NotIn{J10,Jyo}\\!}
\end{tlg}

\begin{tlg}[hp04_016]% = V3_4.67
\tl{
\pada{\app{\lem[wit={P11,Gr3a,G11,J10}]{audāsīnya}
		\rdg[wit={V15}]{audāsinya}
		\rdg[wit={G4}]{audāśinya}
		\rdg[wit={C6}]{audāsīna}
		\rdg[wit={N23}]{odāsīnya}
		\rdg[wit={V3,J7}]{udāsīnya}
		\rdg[wit={J5}]{udāsinya}
		\rdg[wit={N3}]{udāsonya}
		\rdg[wit={N19}]{ṛdāsīnya}}paro bhūtvā}
\pada{sadābhyāsena saṃyamī/}\\+} % °bhyosena V3
\tl{
\pada{unmanī\app{\lem[wit={N3,P11,C6,Gr2,Gr3a}]{karaṇaṃ}
		\rdg[wit={V3}]{karaṇa}
		\rdg[wit={J5}]{karaṇe}
		\rdg[wit={G11,N19,V15,J10}]{kārakaṃ}} sadyo} % sadyā N3
\pada{\app{\lem[wit={ceteri},alt={nādam}]{nāda\skp{m}}
		\rdg[wit={N19}]{bhāda}}%
	\app{\lem[wit={ceteri},alt={evāvadhārayet}]{\skm{m }evāvadhārayet}
		\rdg[wit={J5}]{evāvadhārayan}
		\rdg[wit={V15}]{eva sadābhyaset}}//}
	\NotIn{Jyo}\\!}
\end{tlg}

\begin{ava}[hp04_017a]
\app{\lem[wit={N3,P11,N23,G11},alt={kīdṛśam},post=\texteng{(ki° \getsiglum{N3})}]{kīdṛśa\skp{m}} % ki° N3
%		\rdg[wit={C7}]{kīdṛṣam}
		\rdg[wit={J5,J7}]{kīdṛśīm}
		\rdg[wit={C6,V3}]{kīdṛśyam}% +K3
		\rdg[wit={N19,J10}]{idṛśam}% +F
		\rdg[wit={V19}]{kim}
		\rdg[wit={E2,V15},alt={\om}]{\skp{\om}}}%
	\app{\lem[wit={ceteri},alt={audāsīnyam}]{\skm{m }audāsīnyam} % °śīnyaṃ V3
		\rdg[wit={N19,V15}]{audāsinyaṃ}
		\rdg[wit={N3}]{audasīnyaṃ}
		\rdg[wit={J5}]{audāsinyā}
		\rdg[wit={E2}]{athaudāsīnyam}}/ \NotIn{Jyo}
\end{ava}

\begin{tlg}[hp04_017]% = V3_4.68
\tl{
\pada{\app{\lem[wit={ceteri}]{śīte} % sīte N19,V3
		\rdg[wit={V15}]{śīti}
		\rdg[wit={J5}]{śīta}
		\rdg[wit={J10}]{jñāte}}
	\app{\lem[wit={ceteri}]{kāle}
		\rdg[wit={J7}]{kāla}
		\rdg[wit={J10}]{kā}
		\rdg[wit={J5}]{rakṣa°}
		\rdg[wit={N3},alt={\om}]{\skp{\om}}}
	\app{\lem[wit={N3}]{caupaṭī vā paṭī vā}
		\rdg[wit={N19}]{copaṭī vā paṭī vā}
		\rdg[wit={J7,E2}]{cāpaṭī vā paṭī vā}% cāpaṭe E2ac
		\rdg[wit={V19}]{cāpaṭī vā paṭīkā}
		\rdg[wit={N23}]{cāpaṭī cāpaṭī vā}% +C7
		\rdg[wit={V3,J10}]{caupaṭī vā kuṭī vā}% =J6
		\rdg[wit={P11}]{copaṭī vā kuṭī vā}
		\rdg[wit={C6}]{cāpaṭī vā kuṭī vā}
		\rdg[wit={G11}]{dvaupaṭī vā kuṭī vā}% caupaṭī vā kuṭī vā? G5,G7,M3; dvau paṭau vā kuṭī vā F
		\rdg[wit={V15}]{paṭī vā}
		\rdg[wit={J5}]{°ṇe kathā vā paṭī vā}}}\\+}
\tl{
\pada{\app{\lem[wit={N3,J5,P11,V3,E2,G11,N19}]{pathyāhāre}
		\rdg[wit={C6,J7,V15,J10}]{pathyāhāro}% +K3,C7
		\rdg[wit={N23}]{yathāhārā}
		\rdg[wit={V19}]{<<mi>>thyāhāro}}
	\app{\lem[wit={ceteri}]{gopayo}% +K3
		\rdg[wit={V19}]{gopatho}
%		\rdg[wit={E2,C7}]{gomayo}% +F
		}
	\app{\lem[wit={ceteri}]{vā}
		\rdg[wit={J10}]{co}
		\rdg[wit={N23},alt={\om}]{\skp{\om}}}
	\app{\lem[wit={ceteri}]{payo vā}
		\rdg[wit={N23}]{<<payo>> vā}
		\rdg[wit={V19}]{patho vā}
		\rdg[wit={C6}]{°tha pānaṃ}}/}\\+}
\tl{
\pada{\app{\lem[wit={Gr1,P11,V3,G11}]{bhojye}% G11pc,G5
		\rdg[wit={V15,J10}]{bhojyaṃ}% +M3
		\rdg[wit={N19}]{bhojya}
		\rdg[wit={Gr2}]{bhakṣe}
		\rdg[wit={C6,V19}]{bhakṣyaṃ}
		\rdg[wit={E2}]{bhikṣye}}
	\app{\lem[wit={ceteri}]{bhikṣā} % bhīkṣā V15, bhikṣyā N3,P11
		\rdg[wit={J10}]{bhuktaṃ}}%
	\app{\lem[wit={ceteri},alt={vṛndam}]{vṛnda\skp{m}}% vṛndapā° E2
		\rdg[wit={P11}]{mṛdam}
		\rdg[wit={G11plus}]{kandam}
		\rdg[wit={J10}]{cānnam}}%
	\app{\lem[wit={Gr1,J7,Gr3a,V15},alt={āraṇyakandaṃ}]{\skm{m }āraṇyakandaṃ}% +F
		\rdg[wit={V3,N19,J10},alt={°kaṃda}]{āraṇyakaṃda}
		\rdg[wit={P11},alt={°kaṃdā}]{āraṇyakaṃdā}
		\rdg[wit={N23}]{āramyakaṃdaṃ}
		\rdg[wit={G11plus}]{āraṇyakaṃ vā}% +N22,N12
		\rdg[wit={C6}]{āpaṇyakaṃ vā}}}\\+}
\tl{
\pada{\app{\lem[wit={N3,P11,J7,Gr3a,G11}]{pāṇī droṇī}
		\rdg[wit={J5,V15,J10}]{pāṇi droṇī}
		\rdg[wit={G4}]{pāṇi droṇi}
		\rdg[wit={N19}]{pāṇī drāṇi}
		\rdg[wit={N23}]{pāṇīndrāṇī}
		\rdg[wit={C6}]{pāṇiṃ droṇe}
		\rdg[wit={V3}]{pāṇi}}
	\app{\lem[wit={N3,G4,P11,G11,N19,V15}]{kāpi vā}
		\rdg[wit={V3}]{kāpivāṃ}
		\rdg[wit={J10}]{kāthivā}
		\rdg[wit={J5}]{vā kapī}
		\rdg[wit={E2}]{karparā}% +K3,C7
		\rdg[wit={C6}]{karpaṭaṃ}
		\rdg[wit={J7}]{kāpaṭo}
		\rdg[wit={N23}]{khapaḍā}
		\rdg[wit={V19}]{kharparo}}
	\app{\lem[wit={J5,G4,P11,G11,N19}]{bhojyapātre}% patre P11
		\rdg[wit={N3,V3,Gr3a,V15,J10}]{bhojyapātraṃ}% +F
		\rdg[wit={C6}]{bhojapatraṃ}
		\rdg[wit={N23}]{bhājapatraṃ}
		\rdg[wit={J7}]{bhūrjapatraṃ}}//}
	\NotIn{Jyo}\\!}
\end{tlg}

\newpage
\begin{tlg}[hp04_018]% = V3_4.69 = C8_72
\tl{
\pada{\app{\lem[wit={J7,Gr3a,G11,N19}]{sarvacintāṃ}
		\rdg[wit={N3,J5,GrB,V15,J10}]{sarvacintā}
		\rdg[wit={N23},alt={\om}]{\skp{\om}}}
	\app{\lem[wit={J5,P11,V3,N19,V15,J10}]{samutsṛjya}% +M3
		\rdg[wit={G11}]{samṛtsṛjya}
		\rdg[wit={N3}]{samutyajya}
		\rdg[wit={C6,J7,Gr3a}]{parityajya}% +G5
		\rdg[wit={N23},alt={\om}]{\skp{\om}}}}
\pada{sarva\app{\lem[wit={N3,GrB,G11,V15}]{ceṣṭāṃ}
		\rdg[wit={J5}]{ceṣṭā}
		\rdg[wit={J10}]{ceṣṭāś}
		\rdg[wit={N19}]{ceṣṭī}
		\rdg[wit={Gr2,Gr3a}]{kāle}} ca sarvadā/}\\+}
\tl{
\pada{nādam % -m- is a hiatus bridge.
	evānu\app{\lem[wit={N3,P11,C6},alt={saṃdhānān}]{saṃdhānā\skp{n}}
		\rdg[wit={V3}]{saṃdhānā}
		\rdg[wit={J5,G11,N19,V15,J10}]{saṃdadhyān} % °dhyā N19, saṃ<<da>>dhyān J10
		\rdg[wit={Gr2,Gr3a}]{saṃdhatte}}}%n % dhartte N23
\pada{\app{\lem[wit={ceteri},alt={nāde}]{\skm{n }nāde}
		\rdg[wit={C6}]{devi}} cittaṃ vilīyate//} % vittaṃ vileyate P11
		\label{sarvacinta}
%	\lineom{J11}\myfn{\getsiglum{J11} haplography. Jumped to the next verse.}
	\NotIn{Jyo}\\!}
\end{tlg}

%\Anm{\getsiglum{G11,N19,V15,J10} have \ref{sarvacinta2} and \ref{makaranda1}--64 after these verses.}

%\newpage
\begin{tlg}[hp04_019]% = V3_4.52 = C8_55 = V17_70
\tl{
\pada{ārambha%ś  % araṃbha° P7
	\app{\lem[wit={ceteri},alt={ca}]{\skm{ś }ca}
		\rdg[wit={V19}]{ca\,\_}}
	\app{\lem[wit={ceteri},alt={ghaṭaś}]{ghaṭa\skp{ś}}
		\rdg[wit={N23}]{gha\,\_\,ś}}%
	\app{\lem[wit={ceteri},alt={caiva}]{\skm{ś }caiva}
		\rdg[wit={J10}]{caivas}
		\rdg[wit={V19}]{ca}}}
\pada{tathā
	\app{\lem[wit={N3,G4,GrB,G11,N19,J10},alt={paricayas}]{paricaya\skp{s}}
		\rdg[wit={V15}]{paricas}
		\rdg[wit={J5,N23,Gr3a,Jyo}]{paricayo}
		\rdg[wit={J7}]{pariyo}}%
	\app{\lem[wit={N3,V3,V15},alt={tathā}]{\skm{s }tathā}% =ŚS; trayaṃ N24; +F
		\rdg[wit={G4,P11,C6,G11,N19,J10}]{tataḥ}% +M3, smṛtaḥ G5
		\rdg[wit={V19}]{pi vā}
		\rdg[wit={J5,Gr2,E2,Jyo}]{'pi ca}}/}\\+}
\tl{
\pada{niṣpattiḥ % niṣpaṃti N23
	\app{\lem[wit={ceteri}]{sarvayogeṣu}
		\rdg[wit={E2}]{sarvayoge ca}
		\rdg[wit={GrB}]{ceti yogeṣu}
		}}
\pada{\app{\lem[wit={N3,G4}]{yogāvasthā bhavanti tāḥ}
		\rdg[wit={J5}]{yogāvasthā bhavanti te}
		\rdg[wit={Gr2,Gr3a}]{yogāvasthā prakīrtitā}
		\rdg[wit={GrB,G11,N19,V15,J10,Jyo}]{syād avasthācatuṣṭayaṃ}% +F
		}\marma//}\\!}
\end{tlg}

%\newpage
\begin{ava}[hp04_020a]
\app{\lem[resp=emend]{tatrārambhāvasthā}
		\rdg[wit={G4,N19,V15}]{tatra ārambhaḥ}
		\rdg[wit={G11}]{tatrārambhaḥ}
		\rdg[wit={J10}]{tatra cārambhaḥ}
		\rdg[wit={N23,Jyo}]{athārambhāvasthā}
		\rdg[wit={V19}]{athārambharakṣā}% +K3,C7
		\rdg[wit={E2}]{athārambhadīkṣā}
		\rdg[wit={J7}]{ārambhāvasthātha}
		\rdg[wit={N3,J5,GrB},alt={\om}]{\skp{\om}}}/
	\NotIn{N3,J5,GrB}
\end{ava}

\begin{tlg}[hp04_020]% = V3_4.53 = C8_56
\tl{
\pada{brahma\app{\lem[wit={N3,Jyo},alt={granther}]{granthe\skp{r}}% +J11pc,G5,M3
		\rdg[wit={P11}]{granthe}
		\rdg[wit={E2}]{granthau}
		\rdg[wit={V3,J7,V19,V15}]{granthir}% grathir J7; +K3,C7
		\rdg[wit={C6,N23}]{granthi}
		\rdg[wit={J10}]{granthiṃ}
		\rdg[wit={J5}]{granthid}
		\rdg[wit={G11}]{gra\,+}
		\rdg[wit={N19}]{raṃdhre}}r bhave%d % bhavod C6
	\app{\lem[wit={N3,C6,G11,V3},alt={bhedād}]{\skm{d }bhedā\skp{d}}
		\rdg[wit={J5,P11}]{bhedā}
		\rdg[wit={Gr2,V19}]{bhinna}
		\rdg[wit={E2}]{bhinne}
		\rdg[wit={J10}]{bhinnā}
		\rdg[wit={V15}]{bhinnād}
		\rdg[wit={Jyo}]{bhedo hy}
		\rdg[wit={N19}]{bhed}}}%
\pada{\app{\lem[wit={ceteri},alt={ānandaḥ}]{\skm{d }ānandaḥ}
		\rdg[wit={J5,C6,N23}]{ānaṃda}
		\rdg[wit={P11}]{nanādaḥ}
		\rdg[wit={J10}]{nādaḥ}}
	śūnya\app{\lem[wit={ceteri}]{saṃbhavaḥ} % śūnyaṃ J10; °bhava N19, °bhavaṃ P7, °bhavā J5
		\rdg[wit={J10}]{samaṃbhavaḥ}}/}\\+}
\tl{
\pada{vicitra\app{\lem[wit={E2,G11}]{kvaṇako}
		\rdg[wit={N3}]{kvana˟ko}
		\rdg[wit={V15}]{kvaṇiko}
		\rdg[wit={V3,N19}]{kaṇako}
		\rdg[wit={J5}]{kanako}
		\rdg[wit={J10}]{kuṇako}
		\rdg[wit={C6}]{kuṇape}
		\rdg[wit={Jyo}]{°ḥ kvaṇako}
		\rdg[wit={P11}]{°ṣkāṇako}
		%\rdg[wit={K3,C7}]{kṣaṇike}
		\rdg[wit={V19}]{kṣike}
		\rdg[wit={Gr2}]{°s tatkṣaṇād}} % ta<<t>> N23
	\app{\lem[wit={ceteri}]{dehe}
		\rdg[wit={J5}]{deho}
		\rdg[wit={C6}]{caivā}}}% caiva C6ac
\pada{\app{\lem[wit={N3,J5,GrB,G11,N19,V15,J10,Jyo}]{'nāhataḥ śrūyate}% °hata P11
		\rdg[wit={Gr2}]{sarvataḥ śrūyate}
		\rdg[wit={Gr3a}]{śrūyate (')nāhata}}
		dhvaniḥ//}\\!} % dhvani J5,P11,V3,J7, ddhavaniḥ N19
\end{tlg}

\begin{tlg}[hp04_021]% = V3_4.54ab = C8_57
\tl{
\pada{\app{\lem[wit={N3,J5,P11,C6,Gr2,Jyo}]{divyadehaś ca tejasvī} % tejasvā N3; +G5,M3
		\rdg[wit={G11}]{divyadehasya tejasvī}
		\rdg[wit={N19},post={\unm}]{ādityatejaś ca tejasvī}
		\rdg[wit={V15}]{tejasvī divyagandhaś ca}
		\rdg[wit={J10}]{divyagandho divyacakṣuś ca}
		\rdg[wit={V3,Gr3a},alt={\om}]{\skp{\om}}}}
\pada{\app{\lem[wit={N3,G4,P11,C6,Gr2,Jyo}]{divyagandhas tv arogavān} % gaṃdhās N3
		\rdg[wit={G11,N19}]{divyagandho py arogavān}% paro° N19
		\rdg[wit={V15}]{divyadeho py arogavān}
		\rdg[wit={J5}]{divyadeham arogavān}
		\rdg[wit={J10}]{tejasvī ārogavān}
		\rdg[wit={V3,Gr3a},alt={\om}]{\skp{\om}}}/}
		\lineom{ab}{Gr3a,V3}\\+}
\tl{
\pada{\app{\lem[wit={ceteri}]{saṃpūrṇa}
		\rdg[wit={V15}]{saṃpūrṇe}}%
	\app{\lem[wit={Gr1,P11,N19,Jyo}]{hṛdayaḥ}
		\rdg[wit={J7,G11}]{hṛdaya}
		\rdg[wit={C6,V3,N23,V19,V15,J10}]{hṛdaye}% +J11pc
		%\rdg[wit={E2,C7}]{nilaye}
		}
	\app{\lem[wit={Gr1,N19,V15},alt={śūnye tv}]{śūnye\skp{ tv}}
		\rdg[wit={C6,Gr2,Gr3a,G11,J10}]{śūnye}
		\rdg[wit={V3,Jyo}]{śūnya}% +G5,M3,F
		\rdg[wit={P11}]{śūra}}}
\pada{\app{\lem[wit={ceteri},alt={ārambhe}]{\skm{tv }ārambhe}
		\rdg[wit={V3}]{ārambha}
		\rdg[wit={J10}]{āraṃbho}}
	\app{\lem[wit={ceteri},alt={yogavān}]{yogavā\skp{n}}
		\rdg[wit={N23}]{bhogavān}}n bhavet//}\\!}
\end{tlg}

\newpage
\begin{ava}[hp04_022a]
atha
\app{\lem[wit={ceteri}]{ghaṭāvasthā}% ghaṭavaṃsthā N19, ghaṃṭā° V17
		\rdg[wit={G4}]{khaṭavasthā}
		\rdg[wit={J5}]{ghaṭā arthaḥ}
		\rdg[wit={Gr3a}]{ghaṭarakṣā}
		\rdg[wit={P11}]{ghaṭaḥ}}/
\end{ava}

\begin{tlg}[hp04_022]% = V3_4.54/55
\tl{
\pada{\app{\lem[wit={N3,GrB,Gr2,E2,G11,N19,V15pc,Jyo}]{dvitīyāyāṃ} % dviti° V3
		\rdg[wit={V19,V15ac}]{dvitīyā}
		\rdg[wit={J10}]{dvitīye}
		\rdg[wit={J5}]{dvitī}}
	\app{\lem[wit={ceteri}]{ghaṭī}
		\rdg[wit={V15}]{ghaṃṭi}
		\rdg[wit={N19}]{ghaṭāṃ}
		\rdg[wit={J5}]{ghaṭikā}
		\rdg[wit={G11plus}]{sphuṭī}
		\rdg[wit={J10}]{bheda}}%
	\app{\lem[wit={N3,J5,GrB,Gr2,Gr3a,G11,N19,Jyo}]{kṛtya}
		\rdg[wit={V15}]{kṛtvā}
		\rdg[wit={J10}]{mukte tu}}}
\pada{vāyur bhavati madhyagaḥ/}\\+} % vāḥsur P7
%	\app{\lem[wit={ceteri}]{madhyagaḥ}
%		\rdg[wit={K3,C7}]{madhyamaḥ}}/}\\+}
\tl{
\pada{\app{\lem[wit={ceteri}]{dṛḍhāsano}
		\rdg[wit={J10}]{haṭhāsano}} bhaved yogī}
\pada{jñānī
	\app{\lem[wit={ceteri}]{deva}
		\rdg[wit={V3}]{devaḥ}
		\rdg[wit={P11,C6,E2,J10}]{deha}}sama%s  % samās N3
	\app{\lem[wit={N3,J5,GrB,Jyo},alt={tadā}]{\skm{s }tadā}
		\rdg[wit={Gr2,Gr3a,G11,N19,V15,J10}]{tathā}}//}\\!}
\end{tlg}

%\newpage
\begin{tlg}[hp04_023]% = V3_4.55/56
\tl{
\pada{viṣṇu\app{\lem[wit={N3,P11}]{granthes tadā}% +N24,F
		\rdg[wit={V3}]{granthis tadā}
		\rdg[wit={N19}]{granthe sadā}
		\rdg[wit={J5,J10}]{granthes tathā}% +G5
		\rdg[wit={G11}]{granthe tathā}
		\rdg[wit={C6}]{granther yadā}
		\rdg[wit={Gr2,Gr3a,V15}]{granthir yadā}% <<graṃthi>> C7
		\rdg[wit={Jyo}]{granthes tato}}
	\app{\lem[wit={N3,GrB,G11,N19,J10,Jyo},alt={bhedāt}]{bhedā\skp{t}}
		\rdg[wit={J5}]{bhidā}
		\rdg[wit={Gr2,Gr3a}]{bhinnaḥ} % vimugraṃthipadābhinna N23
		\rdg[wit={V15}]{bhinnā}}}%
\pada{\app{\lem[wit={ceteri},alt={paramānanda}]{\skm{t }paramānanda}
		\rdg[wit={N19}]{sadānandasya}}%
	\app{\lem[wit={ceteri}]{sūcakaḥ} % śū° N23, °caka V3
		\rdg[wit={V15}]{sūcakā<<ḥ>>}
		\rdg[wit={C6}]{kārakaḥ}}/}\\+}% °kāḥ V15pc
\tl{
\pada{\app{\lem[wit={Gr1,P11,V3,G11,Jyo}]{atiśūnye}% śunye N3
		\rdg[wit={Gr2,Gr3a,V15,J10}]{atiśūnya}
		\rdg[wit={C6}]{aṃtyaśūnye}
		\rdg[wit={N19}]{api śūnyo}}
	\app{\lem[wit={N3,G4,GrB,Jyo}]{vimardaś ca}
		\rdg[wit={J5}]{vimardasya}
		\rdg[wit={N19}]{'saṃmardā}
		\rdg[wit={G11}]{visanmarde}% 'pi saṃmardo G5
		\rdg[wit={J10}]{visaṃmardo}
		\rdg[wit={Gr2,Gr3a,V15}]{vibhedaś ca}}}
\pada{bherīśabda%s % sabdas V3
	\app{\lem[wit={N3,GrB,V15,Jyo},alt={tadā}]{\skm{s }tadā}
		\rdg[wit={G4,Gr2,Gr3a,G11,N19,J10}]{tathā}% +N24
		\rdg[wit={J5}]{tatho}
		} bhavet//}\\!}
\end{tlg}

%\newpage


\begin{ava}[hp04_024a]
\app{\lem[wit={ceteri}]{atha}
		\rdg[wit={C6}]{tathā}
		\rdg[wit={E2,Jyo},alt={\om}]{\skp{\om}}}
\app{\lem[wit={ceteri}]{paricayāvasthā}% <<yā>> N23
		\rdg[wit={P11,N19,V15}]{paricayaḥ} % °caya P11,N19
		\rdg[wit={Jyo},alt={\om}]{\skp{\om}}}/ \NotIn{Jyo}
\end{ava}

\begin{tlg}[hp04_024]% = V3_4.57 = C8_60
\tl{
\pada{\app{\lem[wit={N3,GrB,Gr3a,G11,V15}]{tṛtīyāyāṃ tato bhittvā} % tṛtiyāyāṃ V3
		\rdg[wit={J5}]{tṛtīyāyāṃ tathā bhitvā}
		\rdg[wit={Gr2}]{karṇikāṃ tu tato bhittvā}
		\rdg[wit={N19}]{karttikāyāṃ tato bhittvā}
		\rdg[wit={J10}]{atha granthitrayaṃ bhittvā}
		\rdg[wit={Jyo}]{tṛtīyāyāṃ tu vijñeyo}}}
\pada{\app{\lem[wit={J5,G11,N19,Jyo}]{vihāyo}
		\rdg[wit={Gr2,V15}]{vihāya}
		\rdg[wit={P11}]{vikāryo}
		\rdg[wit={Gr3a}]{vimalo}
		\rdg[wit={V3}]{vimāyo}
		\rdg[wit={C6}]{visphāro}
		\rdg[wit={J10}]{jāyate}}%
	\app{\lem[wit={J5,GrB,Gr2,N19,J10,Jyo}]{mardala}
		\rdg[wit={G11}]{maddala}
		\rdg[wit={Gr3a}]{mandala} % maṃdala V19
		\rdg[wit={V15}]{mṛḍula}}%
	\app{\lem[wit={ceteri}]{dhvaniḥ}
		\rdg[wit={J7}]{dhvaniṃ}
		\rdg[wit={P11,V3}]{dhvani}}/}\marma\\+}
\tl{
\pada{\app{\lem[wit={ceteri}]{mahāśūnyaṃ}
		\rdg[wit={P11,V15}]{mahāśūnya}
		\rdg[wit={G11}]{mahāśūnyas}
		}
	\app{\lem[wit={J5,GrB,G11,Jyo}]{tadā}% +N24
		\rdg[wit={Gr2,N19}]{tathā}
		\rdg[wit={Gr3a}]{tato}
		\rdg[wit={V15}]{tayā}
		\rdg[wit={J10}]{samā}}
	\app{\lem[wit={ceteri}]{yāti}
		\rdg[wit={J5}]{jāti}
		\rdg[wit={N19}]{jātiḥ}}}
\pada{\app{\lem[wit={ceteri}]{sarvasiddhi} % +P11
		\rdg[wit={V3}]{mahāsiddhi}
		\rdg[wit={C6}]{siddhisādha°}
		\rdg[wit={N19}]{sarva}}% sarva<siddhi> N19
	\app{\lem[wit={ceteri}]{samāśrayam}
		\rdg[wit={P11}]{samāśriyaṃ}
		\rdg[wit={J5}]{matāśrayāt}
		\rdg[wit={C6}]{kam āśrayaṃ}}//}\label{cittananda}
	\anm{Pāda b--\ref{nadanu}d lost \getsiglum{N3}}
	\unavbl{N3}\\!}
\end{tlg}

\begin{tlg}[hp04_025]% = V3_4.58
\tl{
\pada{\app{\lem[wit={G4,C6,Gr2,Gr3a,G11,Jyo}]{cittānandaṃ}% <<da>> C7; +N24
		\rdg[wit={J5,V3,V15}]{cidānaṃda}% cidānaṃdaṃ HR,F, cidātmānaṃ M3, cittamānaṃ G5 ##
		\rdg[wit={P11}]{vivarttānaṃdaṃ}
		\rdg[wit={J10}]{ciṃtāmanas}
		\rdg[wit={N19}]{virāmānaṃ}}
	\app{\lem[wit={ceteri}]{tato}
		\rdg[wit={Jyo}]{tadā}}
	\app{\lem[wit={J5,G4,GrB,G11,N19,V15,J10,Jyo}]{jitvā}
		\rdg[wit={Gr2,Gr3a}]{bhittvā}}}
\pada{sahajānanda%
	\app{\lem[wit={ceteri}]{saṃbhavaḥ}
		\rdg[wit={P11,N19}]{saṃbhava}}/}\\+}
\tl{
\pada{\app{\lem[wit={ceteri}]{doṣaduḥkha}
		\rdg[wit={P11}]{doṣaduḥkhaṃ}
		\rdg[wit={N23}]{dokhaduḥkhe}}%
	\app{\lem[wit={G4,GrB,G11,V15,J10}]{jarāmṛtyu}
		\rdg[wit={J5,N19}]{jarāmṛtyuḥ}
		\rdg[wit={Jyo}]{jarāvyādhi}
		\rdg[wit={Gr2,Gr3a}]{kṣudhānidrā}}}%
\pada{\app{\lem[wit={J5,G4,P11,C6,G11,N19,V15,J10,Jyo}]{kṣudhānidrā}
		\rdg[wit={V3}]{kṣudhātṛṣā}
		\rdg[wit={Gr2,Gr3a}]{jarāmṛtyu}}% mṛrtyu N23
	\app{\lem[wit={ceteri}]{vivarjitaḥ}
		\rdg[wit={C6},alt={°tāḥ}]{vivarjitāḥ}
		\rdg[wit={V3},alt={°taṃ}]{vivarjitaṃ}
		\rdg[wit={J10}]{tṛṣā tathā}}//}
	\unavbl{N3}\\!}
\end{tlg}

\newpage
\begin{ava}[hp04_026a]
atha \app{\lem[wit={C6,V3,Gr2}]{niṣpattyavasthā} % niḥpatya° N23, niṣpatya° J7
		\rdg[wit={J5}]{niḥpatti-avasthā}
		\rdg[wit={Gr3a}]{niṣṭhāvasthā}
		\rdg[wit={P11,G11,N19,V15,J10}]{niṣpattiḥ}% niṣpatti P11, niṣpati N19, °tiḥ J10; +F ##
		}/\myfn{%
		In \getsiglum{J5,GrB,J7,Gr3a} the heading is found after the first line of \ref{rudra}.}
		\NotIn{Jyo}
\end{ava}

\begin{tlg}[hp04_026]% = V3_4.59/60
\tl{
\pada{rudragranthiṃ % urddha J5; rudraṃ graṃ{{ga}}thi N23; grathiṃ J7, graṃthi C7,N19,P11,V3, grandhiṃ G11
	\app{\lem[wit={ceteri}]{tato}
		\rdg[wit={Jyo}]{yadā}}
	\app{\lem[wit={ceteri}]{bhittvā} % bhītvā J5,P7
		\rdg[wit={N19}]{bhūtvā}}}
\pada{\app{\lem[wit={ceteri}]{sarva}
		\rdg[wit={Jyo}]{śarva}
		\rdg[wit={P11}]{satva}}pīṭha% pīca P11, pīḍa G11
	\app{\lem[wit={ceteri}]{gato'nilaḥ}% +M3
		\rdg[wit={J7}]{gatonalaḥ}
		\rdg[wit={G11}]{gatānilaḥ}% °laṃ G5
		\rdg[wit={J5,V3}]{gatānila}}/}\\+}
\tl{
\pada{\app{\lem[wit={J5,GrB,J7,Jyo}]{niṣpattau}
		\rdg[wit={N19,V15}]{niṣpannau} % niḥṣpannau? V15
		\rdg[wit={G11,J10}]{niṣpanno} % niḥpanno J10
		\rdg[wit={N23}]{niṣpatto}
		\rdg[wit={Gr3a}]{niṣṭhāto}}
	\app{\lem[wit={ceteri}]{vaiṇavaḥ śabdaḥ}% veṇavaḥ V17
		\rdg[wit={J7}]{vaiṇavaśabdaḥ}
		\rdg[wit={J5}]{vauṇāvat sado}
		\rdg[wit={N23}]{veṇacaśabdaṃ}}}
\pada{\app{\lem[wit={V15,Jyo}]{kvaṇadvīṇākvaṇo}% viṇā V15; +F
		\rdg[wit={G11}]{kvaṇan vīṇakvaṇo}% kvaṇan* gīta° G5
		\rdg[wit={N19}]{kaṇatvītakvaṇo}% sic!
		\rdg[wit={J7}]{kvaṇadvīṇotvaṇo}% < °vīṇolbaṇo?
		\rdg[wit={P11}]{kvaṇan vītaḥ kvaṇo} 
		\rdg[wit={C6}]{kvacid vīṇākvaṇo}
		\rdg[wit={V3}]{kvaṇatuvītakvaṇo}
		\rdg[wit={J10}]{kvaṇantenākvuṇo}
		\rdg[wit={Gr3a}]{kvaṇadvīṇāsamo}% vīnā V19; kvaṇadvīvakvaṇo K1
		\rdg[wit={N23}]{karṇavīṇādgato}
		\rdg[wit={J5}]{kṛṇanityakṛṇo}}\marmas
	\app{\lem[wit={ceteri}]{bhavet}
	\rdg[wit={C6}]{°dayaḥ}}//}\label{rudra}
	\unavbl{N3}\\!}
\end{tlg}

% MD: marking für die andere Recensions hier abgebrochen.
%\newpage
\begin{tlg}[hp04_027]% = V3_4.60/61 = C8_63 %% P11 has Pada a only
\tl{
\pada{ekībhūtaṃ % ekāṃ P11, eka° C6
	\app{\lem[wit={J5,GrB,G11,Jyo}]{tadā}
		\rdg[wit={G4,Gr2,Gr3a,J10}]{tathā}} cittaṃ}
\pada{\app{\lem[wit={ceteri}]{rājayogā}
		\rdg[wit={J10}]{rājayoga}
		\rdg[wit={V3}]{rājayogo}}%
	\app{\lem[wit={V3,J7,G11}]{bhidhāyakam}
		\rdg[wit={J5}]{vidhāyakaḥ}
		\rdg[wit={N23}]{bhidhāyanaṃ}
		\rdg[wit={G4,C6,Gr3a,J10,Jyo}]{bhidhānakaṃ}% +F
		}\marma/}\\+} % +N24
\tl{
\pada{sṛṣṭisaṃhāra%
	\app{\lem[wit={ceteri}]{kartāsau}
		\rdg[wit={N23}]{karttasau}
		\rdg[wit={V3}]{karttāso}}}
\pada{yogīśvarasamo bhavet//}
	\lineom{bcd}{P11} \NotIn{N19,V15} \label{ekibhutam}
	\unavbl{N3}\\!}
%	\anm{\getsiglum{C7} in mg. sec. m.}
\end{tlg}


\begin{altava}[hp04_027_1a]
atha nādānusaṃdhānam/ \sgwit{G11plus}
\end{altava}


%\newpage
\startaltrecension % B1 has this verse too.
\begin{alttlg}[hp04_027_1]% = V3_4.62 = C8_64
\tl{
\pada{rājayoga\app{\lem[wit={P11,C6,G5}]{padaprāptau}% +C8,P7,M3,F
		\rdg[wit={G11}]{padaprāptā}
		\rdg[wit={N19}]{padaprāptaḥ}
		\rdg[wit={V3}]{padaṃ prāptaṃ}
		\rdg[wit={J10,Jyo}]{padaṃ prāptuṃ}
		\rdg[wit={V15}]{padaṃ prāpti}
		}}
\pada{\app{\lem[wit={P11,C6,G11,G5,N19,V15,J10,Jyo}]{sukhopāyo'lpa}
		\rdg[wit={V3}]{sukhopāyogya}}cetasām/}\\+} % cetasam P11
\tl{
\pada{sadyaḥpratyaya%
	\app{\lem[wit={C6,V3,N19,J10,Jyo}]{saṃdhāyī}
		\rdg[wit={P11,V15}]{saṃdhāyi}
		\rdg[wit={G11,G5}]{saṃdāyī}}}
\pada{\app{\lem[wit={GrB,G5,N19,V15,Jyo}]{jāyate}% +M3
		\rdg[wit={G11}]{līyate}
		\rdg[wit={J10}]{sevyate}}
	\app{\lem[wit={C6,V3,N19,Jyo}]{nādajo layaḥ} % laya V3
		\rdg[wit={P11,J10}]{nādayo layaḥ}
		\rdg[wit={G5}]{nādamūlayā}
		\rdg[wit={V15}]{nātra saṃśayaḥ}}//}
	\sgwit{GrB,G11,G5,N19,V15,J10,Jyo} \anm{after \ref{astuva} \getsiglum{GrB}}
	\anm{cf. \ref{saukhya}}\\!}
\end{alttlg}
\endaltrecension



\Anm{Verses \ref{astuva}--\ref{svastho} are found before \ref{drsti} in \getsiglum{G11,N19,V15,J10}}

\begin{tlg}[hp04_028]% = V3_4.61/62 = 4c_106
\tl{
\pada{astu vā
	\app{\lem[wit={J5,C6,Gr2,Gr3a,G11,N19,J10,Jyo}]{māstu} % māsta N19
		\rdg[wit={V3,V15}]{mastu}
		} vā
	\app{\lem[wit={J5,C6,E2,G11,N19,J10,Jyo},alt={muktir}]{mukti\skp{r}}% +K3,C7
		\rdg[wit={V15}]{muktis}
		\rdg[wit={V3}]{muktiṃ}
		\rdg[wit={Gr2}]{śaktir}
		\rdg[wit={V19}]{kiṃcid}}}%
\pada{\app{\lem[wit={C6,Gr3a,G11,N19,Jyo},alt={atraivākhaṇḍitaṃ}]{\skm{r }atraivākhaṇḍitaṃ}% °ṣaṃḍitaṃ N19
		\rdg[wit={J5,J7}]{atraiva khaṇḍitaṃ}% ṣaṃḍitaṃ
%		\rdg[wit={J5}]{atraiva ṣaṃḍitaṃ}
		\rdg[wit={J10}]{atra vākhaṇḍitaṃ}
		\rdg[wit={N23}]{ātrevikhaṇḍitaṃ}
%		\rdg[wit={N19}]{atraivāṣaṃḍitaṃ}
		\rdg[wit={V3,V15}]{tatraivākhaṇḍitaṃ}% +F
		}
	\app{\lem[wit={ceteri}]{mahat}
		\rdg[wit={N23}]{marut}
		\rdg[wit={C6}]{manaḥ}
		\rdg[wit={V19}]{bhavet}
		\rdg[wit={Jyo}]{sukham}}/}\\+}
\tl{
\pada{\app{\lem[wit={J5,C6,G11,N19,V15}]{layāmṛtamayaṃ}% leyā° J5, staye° N24
		\rdg[wit={V3}]{layāmṛtalayaṃ}
		\rdg[wit={J7,Gr3a}]{layāmṛtam idaṃ}% +F
		\rdg[wit={N23}]{layāmṛdaṃmitaṃ}
		\rdg[wit={J10}]{layāmṛtakaraṃ}
		\rdg[wit={Jyo}]{layodbhavam idaṃ}}
	\app{\lem[wit={ceteri}]{saukhyaṃ}
		\rdg[wit={N23}]{sokhyaṃ}
		\rdg[wit={J5,J7,J10}]{sauṣyaṃ}
		\rdg[wit={N19}]{saukṣaṃ}}}
\pada{\app{\lem[wit={ceteri}]{rājayogād avāpyate}
		\rdg[wit={J10}]{rājayogam avāpyate}
		\rdg[wit={V19},alt={\om}]{\skp{\om}}}//}%
\myfn{In \getsiglum{N19}, this verse is followed by the letters \devnote{460 saṃ ka} and the verses \ref{hathamvina}--\ref{ajananta}. With this, the 346th folio of the ms has just been filled, the text of Haṭhapradīpikā ends without a colophon and another text begins in the next folio.}
		\label{astuva}
		\NotIn{P11}
	\unavbl{N3}\\!}
\end{tlg}
%	\anm{\getsiglum{N19,V15,J10} after 4.71?}

%\Anm{\getsiglum{V3} has 4.48*1--\ref{muktasana}, \ref{kasthe}--\ref{sarvacinta} etc. here}

\newpage
\begin{tlg}[hp04_029]%
\tl{
\pada{haṭhaṃ vinā rājayogo} % yogaḥ J11, yogaṃ G4
\pada{rājayogaṃ vinā haṭhaḥ/}\\+} % yoga J5,J11; haṭhaṃ P11
\tl{
\pada{na sidhyati tato yugmam}
\pada{ā niṣpatteḥ samabhyaset//}% +M1,J11; °ttiḥ P11, °tte C6
\myfn{The verse is abbreviated with \textit{haṭhaṃ vinā rājayoga iti} in \getsiglum{N19,V15}, probably because it is same as \manuref{2.77}.}
\NotIn{V3,Gr2,Gr3a,G11plus,J10,Jyo} 
\anm{= \manuref{2.77}}
	\unavbl{N3}\label{hathamvina}\\!}

%\sgwit{J5,G4,P11,C6,N19,V15} 
\end{tlg}

% \myfn{\getsiglum{N19} has then \devnote{460 saṃ ka} before its last verse \ref{ajananta}.}

\begin{tlg}[hp04_030]%
\teimute{\normalsize}\color{black}
\tl{
\pada{rājayogam ajānantaḥ} % °nataḥ V3, °naṃteḥ N19
\pada{kevalaṃ
	haṭha\app{\lem[wit={P11,G11,V15}]{karmaṭhāḥ}
		\rdg[wit={J5}]{karmaṭhaḥ}
		\rdg[wit={N19}]{karmacā}
		\rdg[wit={C6,V3}]{karmaṇā}
		\rdg[wit={J10}]{karmaṇaḥ}
		\rdg[wit={Jyo}]{karmiṇaḥ}}/}\\+}
\tl{
\pada{\app{\lem[wit={P11,C6,G11}]{ye tu tān karṣakān manye}
		\rdg[wit={N19,V15}]{ye tu tān karkaśān manye}
		\rdg[wit={J5}]{ye ca te kāmukān manne}
		\rdg[wit={J10}]{ye tuṃgān karmavasān manye}
		\rdg[wit={Jyo}]{etān abhyāsino manye}
		\rdg[wit={V3},alt={\lacuna}]{\skp{\lacuna}}}} % V3 gap
\pada{\app{\lem[wit={P11,G11,N19,V15,J10,Jyo},post=\texteng{(°tāḥ \getsiglum{J10})}]{prayāsaphalavarjitān}
		\rdg[wit={C6}]{prāyaśaḥ phalavarjitāḥ}
		\rdg[wit={J5}]{prayāsakalavarjitaḥ}
		\rdg[wit={V3},alt={\lacuna}]{\skp{\lacuna}}}//}
	\NotIn{Gr2,Gr3a}\label{ajananta}
	\anm{\getsiglum{N19} ends with this}
	\unavbl{N3}\\!}
%		\sgwit{J5,G4,GrB,G11,N19,V15,J10,Jyo}
\end{tlg}


\startaltrecension
\begin{alttlg}[hp04_030_1]%
\tl{\texteng{[Alt] }\pada{\app{\lem[wit={Gr2,E2}]{haṭhaṃ vinā}
		 \rdg[wit={V19},alt={\om}]{\skp{\om}}}
	 \app{\lem[wit={J7,E2}]{rājayogaṃ}
		 \rdg[wit={N23}]{rājayogo}
		 \rdg[wit={V19},alt={\om}]{\skp{\om}}}}
\pada{rājayogaṃ vinā
	 \app{\lem[wit={J7,Gr3a}]{haṭham}
		 \rdg[wit={N23}]{haṭhaḥ}}/}\\+} % hathaḥ N23ac
\tl{
\pada{ye \app{\lem[wit={N23,Gr3a}]{vai}
		 \rdg[wit={J7}]{cai}}
	 \app{\lem[wit={Gr3a}]{caranti}
		 \rdg[wit={Gr2}]{varaṃti}} tā%n
	 \app{\lem[wit={N23,Gr3a},alt={manye}]{\skm{n }manye}
		 \rdg[wit={J7}]{madhye}}}
\pada{prayāsa\app{\lem[wit={J7,Gr3a}]{phala} % varjitāḥ J10
		 \rdg[wit={N23}]{ptalevi}}varjitān//}
		 \sgwit{Gr2,Gr3a}%
\myfn{\getsiglum{Gr2,Gr3a} have this verse in place of \ref{hathamvina}--\ref{ajananta}.}\\!}
\end{alttlg}
\endaltrecension

%\Anm{\getsiglum{V3} has \ref{ajananta}--\ref{saukhya} after the Kālavañcana section}

%\newpage
\begin{tlg}[hp04_031]% = V3_4.161
\tl{
\pada{\app{\lem[wit={ceteri}]{tattvaṃ}% +M3
		\rdg[wit={N23,V3,G11}]{tattva}}\marmas 
		bījaṃ
	\app{\lem[wit={V19,Jyo}]{haṭhaḥ}% +M3
		\rdg[wit={J5,P11,Gr2,G11,V15}]{haṭha}% +J11
		\rdg[wit={G4,C6,V3,J10}]{haṭhaṃ}% +F,K3,C7
		} kṣetra}%m 
\pada{\app{\lem[wit={C6,V3,Gr2,J10,Jyo},alt={audāsīnyaṃ}]{\skm{m }audāsīnyaṃ}% +K3,C7
		\rdg[wit={J5,P11,V15}]{audāsinyaṃ}
		\rdg[wit={G4}]{audāśinyaṃ}
		\rdg[wit={G11}]{audāsīnya}
		\rdg[wit={V19}]{<<sau>>dāmanyaṃ}}
	\app{\lem[wit={J5,P11,V3,V15,J10,Jyo}]{jalaṃ tribhiḥ}% tribhi V3
		\rdg[wit={G11}]{layaṃ tribhiḥ}
		\rdg[wit={G4,C6,Gr2,V19}]{jalaṃ smṛtam}
		}/}\\+}
\tl{
\pada{unmanīkalpalatikā}
\pada{sadya
	\app{\lem[wit={J5,C6,V3,V19,G11,V15,J10}]{evodbhaviṣyati}% yevo C6; K3,C7
		\rdg[wit={P11}]{evādbhaviṣyati}
		\rdg[wit={G4,Gr2}]{eva bhaviṣyati}% +F
		\rdg[wit={Jyo}]{eva pravartate}}//}
	\unavbl{N3}\NotIn{E2}\\!}
\end{tlg}


%\newpage
\begin{tlg}[hp04_032]% = V3_4.162
\tl{
\pada{\app{\lem[wit={V3,J7,Gr3a}]{rājayogaḥ}
		\rdg[wit={J5,P11,C6,N23}]{rājayoga}}
	samādhi\app{\lem[wit={V3,Gr2,Gr3a},alt={ca}]{\skm{ś }ca} % samādhīś P11
		\rdg[wit={P11},post=\texteng{(cānmatī!)}]{cā}
		\rdg[wit={C6}]{ca hy}
		\rdg[wit={J5}]{ca py}}}
\pada{unmanī ca manonmanī/}\\+} % va P11
\tl{
\pada{\app{\lem[wit={V3},postwit=\texteng{(amaro°)}]{amaraugho'pi cādvaitaṃ}
		\rdg[wit={J5,P11}]{amarodyo pi cādvaitaṃ}% vādvaitaṃ P11
		\rdg[wit={C6}]{amaraughāpi cādvaitaṃ}
		\rdg[wit={J7}]{amaraudhyaighacāṃdrī ca}
		\rdg[wit={N23}]{araughaughatvīṃdrī ca}
		\rdg[wit={Gr3a}]{amaroly abhicāndrī ca}}}
\pada{\app{\lem[wit={J5,GrB,Gr2}]{nirālambaṃ}
		\rdg[wit={Gr3a}]{nirālambo}} 
	\app{\lem[wit={ceteri}]{nirañjanam}
	\rdg[wit={J5}]{nirāmayaṃ}}//}\label{A1}
%	\sgwit{J5,G4,GrB,Gr2,Gr3a} 
	\anm{as \ref{synonym3} in \getsiglum{G11,N19,V15,J10,Jyo}; 
	twice \getsiglum{C6}}
	\unavbl{N3}\\!}
\end{tlg}


\begin{tlg}[hp04_033]% = V3_4.163
\tl{
\pada{\app{\lem[wit={GrB,J7,V19}]{amanasko} % °skoṃ? V19
		\rdg[wit={J5}]{amarasko}
		\rdg[wit={N23}]{amanaskau}
		\rdg[wit={E2}]{amanaskaṃ}} % +K3,C7
	\app{\lem[wit={GrB}]{layas tattvaṃ}
		\rdg[wit={J5}]{layas tatra}
		\rdg[wit={J7,Gr3a}]{layaś caiva}
		\rdg[wit={N23}]{lyayāś caiva}}}
\pada{\app{\lem[wit={J5,P11,J7,Gr3a}]{śūnyāśūnyaṃ}% °sūnyaṃ J5
		\rdg[wit={V3,N23}]{śūnyāśūnya}
		\rdg[wit={C6}]{śūnyāc chūnyaṃ}}
	\app{\lem[wit={J5,G4,GrB}]{paraṃ padam}
		\rdg[wit={N23,Gr3a}]{parāparaṃ}
		\rdg[wit={J7}]{parāvaraṃ}}/}\\+}
\tl{
\pada{\app{\lem[wit={J5,GrB,J7,Gr3a}]{jīvanmuktiś ca}
		\rdg[wit={G4}]{jīvanmuktaś ca}
		\rdg[wit={N23}]{jīvanmuktiḥ}} 
		sahajaṃ} % ṃ om. N23
\pada{\app{\lem[wit={J5,G4,P11,Gr2,E2}]{turyaṃ} % turya J5
		\rdg[wit={C6}]{turyāṃ}
		\rdg[wit={V19}]{turjaṃ}
		\rdg[wit={V3}]{tuṣkaṃ}}
	\app{\lem[wit={J5,P11,C6,J7,Gr3a}]{cety eka}
		\rdg[wit={G4}]{..\,ty eka}
%		\rdg[wit={C7}]{cety eva}
		\rdg[wit={N23}]{vatyaka}
		\rdg[wit={V3}]{caiyeka}}%
	\app{\lem[wit={N23}]{vācakāḥ}
		\rdg[wit={J5}]{vācakaḥ}
		\rdg[wit={J7}]{vācakīṃ}
		\rdg[wit={G4,GrB,Gr3a}]{vācakaṃ}}//}\label{A2}
%	\sgwit{J5,G4,GrB,Gr2,Gr3a} 
	\anm{as \ref{synonym4} in \getsiglum{G11,N19,V15,J10,Jyo}; 
	twice \getsiglum{C6}}
	\unavbl{N3}\\!}
\end{tlg}


\newpage
\begin{tlg}[hp04_034]% = V3_4.164
\tl{
\pada{\app{\lem[wit={J5,P11,V3,G11,Jyo}]{unmanyavāptaye}
		\rdg[wit={C6}]{unmanyā\,\_\,\_\,ye}
		\rdg[wit={V19}]{unmanyavāsayet}
		%\rdg[wit={K3,C7}]{unmanyā vāsayec}
		\rdg[wit={G4}]{unmanyaye}} śīghraṃ} % chīghraṃ V19,K3,C7
\pada{\app{\lem[wit={J5,P11,C6,G11}]{mārgau dvau}
		\rdg[wit={V3}]{mārgo dvau}
		\rdg[wit={G4}]{mārgā\,..}
		\rdg[wit={V19}]{dvau mārgau}% +K3,C7
		\rdg[wit={Jyo}]{bhrūdhyānaṃ}}
	\app{\lem[wit={J5,V3,G11}]{mama saṃmatau}
		\rdg[wit={G4}]{myama saṃ[m].\,+}
		\rdg[wit={C6}]{mamatau}
		\rdg[wit={P11,V19}]{samasaṃmatau}% +K3,C7
		\rdg[wit={Jyo}]{mama saṃmatam}}/}
	\lineom{ab}{Gr2,E2,N19,V15,J10}\\+}
	%\sgwit{J5,G4,GrB,V19,K3,C7,G11,Jyo}\\+}
\tl{
\pada{tattvaṃ parama% padaṃ J5
	\app{\lem[wit={C6,Gr2,G5}]{saukhyaṃ} % +G5; 2 x ṃ om. N23
		\rdg[wit={J5}]{sākhyaṃ}
		\rdg[wit={V3,G11}]{sāṃkhyaṃ}% saṃkhyaṃ F
		\rdg[wit={P11}]{vāgraṃ}
		} vā} % ca J5
\pada{nādopāsanam eva  % nadipā°? J7ac
	\app{\lem[wit={J5,V3,Gr2}]{ca}
		\rdg[wit={P11,C6,G11,G5}]{vā}% +F
		}//} % cā N24
	\lineom{cd}{Gr3a,N19,V15,J10,Jyo}
	\unavbl{N3}\\!}
%	\sgwit{J5,G4,GrB,G11,Gr2}\\!}% G4 only supposedly
\end{tlg}

\begin{tlg}[hp04_035]% = V3_4.165 %%% perhaps 6pāda-verse
\tl{
\pada{\app{\lem[wit={C6,N23,G5}]{saukhya}
		\rdg[wit={J7}]{saukhyā}
		\rdg[wit={P11,V3}]{sāṃkhya}
		\rdg[wit={G11}]{sāṃkhyaṃ}
		\rdg[wit={J5}]{sākṣaṃ}}%
	\app{\lem[wit={C6,V3,J7,G11}]{praviṣṭa}
		\rdg[wit={J5}]{pravṛṣṭa}
		\rdg[wit={P11,N23}]{pratiṣṭha}}cittānāṃ} % cittānyu N23
\pada{mūḍhānām api saṃmatam/} % °taḥ P11, dṛḍhānām! G11
	\lineom{ab}{Gr3a,N19,V15,J10,Jyo}\\+}
%	\sgwit{J5,G4,GrB,Gr2,G11}\\+} % G4 only supposedly
\tl{
\pada{\app{\lem[wit={J5,GrB,V19,C7,G11}]{sadya}% +K3
		\rdg[wit={Gr2}]{satyam}}%
	\app{\lem[wit={J5,GrB,Gr2,C7,G11}]{ānanda}% +K3
		\rdg[wit={V19}]{ādāya}}%
	\app{\lem[wit={G4,J7,V19,C7}]{saṃdhāyī}
		\rdg[wit={N23}]{saṃdhyāyī}
		\rdg[wit={P11,G11}]{saṃdāyī}% +K3
		\rdg[wit={V3}]{sadāyī}
		\rdg[wit={C6}]{saṃdāyi}
		\rdg[wit={J5}]{saṃdāï}}}
\pada{\app{\lem[wit={ceteri}]{jāyate}
		\rdg[wit={V19}]{jāvate}}
	\app{\lem[wit={G4,C6,V3,Gr2,V19,C7,G11}]{nādajo}% °jau N23; +K3
		\rdg[wit={P11}]{nādato}
		\rdg[wit={J5}]{natato}} layaḥ//} % laya V3
	\lineom{cd}{E2,N19,V15,J10,Jyo}\label{saukhya}
	\unavbl{N3}\\!}
%	\sgwit{J5,G4,GrB,Gr2,V19,K3,C7,G11}
\end{tlg}

%\newpage
\startaltrecension
\begin{alttlg}[hp04_035_1]%
\tl{
\pada{ekaṃ sṛṣṭimayaṃ bījaṃ} % svasti° G11
\pada{ekā mudrā 
\app{\lem[wit={G11,G5,J10}]{ca}
	\rdg[wit={V15}]{tu}% +J11
	} khecarī/}\\+}
\tl{
\pada{eko devo \app{\lem[wit={V15,J10}]{nirālamba}% +J11
	\rdg[wit={G11,G5}]{nirālambo hy}}}
\pada{ekāvasthā manonmanī//} % °mani V15
\sgwit{G11,G5,V15,J10} \anm{=\,\manuref{3.49}}\\!} % Not in V3,G7, but in G11
\end{alttlg}

%\newpage
\begin{alttlg}[hp04_035_2]% = V3_4.167
\tl{
\pada{śaṅkhadundubhi%
	\app{\lem[wit={P11,G5,V15,J10,Jyo}]{nādaṃ ca}
		\rdg[wit={V3,G11}]{nādaś ca}
		\rdg[wit={C6}]{nādāṃś ca}}}
\pada{na śṛṇoti kadācana/}\\+} % sṛṇoti V3,J10; canaḥ J11, canā G11
\tl{
\pada{\app{\lem[wit={G5,V15,J10,Jyo}]{kāṣṭhavaj jāyate}
		\rdg[wit={G11}]{kāṣṭhavaj jñāyate}
		\rdg[wit={C6}]{sthāṇuvad vartate}% =M1,F
		\rdg[wit={P11}]{sthāṇuvarddhattayed}% °yej jogī P11
		\rdg[wit={V3}]{sthāṇu vardhate}}
	\app{\lem[wit={J10,Jyo}]{deha}
		\rdg[wit={V15}]{dehe}
		\rdg[wit={G11,G5}]{nādam}
		\rdg[wit={GrB}]{yogī hy}}} % =M1
\pada{unmanyā\app{\lem[wit={GrB,G11,G5,V15,Jyo},alt={°vasthayā}]{\skp{°}vasthayā} % unmanyava° would be unmetrical. V15 has avagraha: unmanyā'va°
		\rdg[wit={J10}]{vasthāyāṃ}} dhruvam//} 
		\sgwit{GrB,G11,G5,V15,J10,Jyo}\\!}
\end{alttlg}

%\newpage
\begin{alttlg}[hp04_035_3]% = V3_4.168
\tl{
\pada{sarvāvasthāvinirmuktaḥ}
\pada{sarvacintā%
	\app{\lem[wit={P11,C6,G11,G5,V15,J10,Jyo}]{vivarjitaḥ} % jita P11
		\rdg[wit={V3}]{vivarjitaṃ}}/}\\+}
\tl{
\pada{\app{\lem[wit={G11,G5,V15,J10,Jyo},alt={mṛtavat}]{mṛtava\skp{t}}
		\rdg[wit={GrB}]{kāṣṭhavat}}% =M1,F
	\app{\lem[wit={P11,C6,G5,V15,J10,Jyo},alt={tiṣṭhate}]{\skm{t }tiṣṭhate}
		\rdg[wit={V3}]{tiṣṭhayed}
		\rdg[wit={G11}]{vartate}} yogī}
\pada{sa mukto nātra saṃśayaḥ//} \sgwit{GrB,G11,G5,V15,J10,Jyo}\\!}
\end{alttlg}

\Anm{\getsiglum{Jyo} has Vulg 4.108 \textit{khādyate na ca kālena}... here}

%\newpage
\begin{alttlg}[hp04_035_4]% = V3_4.169
\tl{
\pada{na \app{\lem[wit={P11,G11,G5}]{hi jānāti}% +F
		\rdg[wit={V15,Jyo}]{vijānāti}
		\rdg[wit={V3}]{hi jānaṃti}} śītoṣṇaṃ}
\pada{\app{\lem[wit={P11,G11,V15,Jyo}]{na duḥkhaṃ na sukhaṃ}
		\rdg[wit={G5}]{na duḥkhaṃ sukham eva vā}
		\rdg[wit={V3}]{na ca duḥkhaṃ sukhaṃ}} tathā/}\\+}
\tl{
\pada{\app{\lem[wit={V15,Jyo}]{na mānaṃ nāpamānaṃ}
		\rdg[wit={G11,G5}]{na mānaṃ nāvamānaṃ}% +F
		\rdg[wit={P11}]{na mānaṃ cāpamānaṃ}
		\rdg[wit={V3}]{na ca mānāpamānaṃ}} ca}
\pada{yogī \app{\lem[wit={P11,C6,G11,Jyo}]{yuktaḥ}
		\rdg[wit={G5,V15}]{muktaḥ}
		\rdg[wit={V3}]{yukti}} samādhinā//}
		\sgwit{GrB,G11,G5,V15,Jyo}\\!}
\end{alttlg}

\newpage
\begin{alttlg}[hp04_035_5]%
\tl{
\pada{\app{\lem[resp=emend,postwit=\texteng{(cf.\,VM)}]{avedhyaḥ}
	\rdg[wit={V15,J10,Jyo}]{avadhyaḥ}
	\rdg[wit={G11}]{adhyāpyāḥ}% avāpyaḥ M3
	\rdg[wit={G5}]{adhyāpaḥ}} 
	sarva\app{\lem[wit={ceteri},alt={śastrāṇām}]{śastrāṇā\skp{m}}
	\rdg[wit={G11}]{śāstrāṇām}}}%m
\pada{\app{\lem[wit={G11,V15,J10},alt={avadhyaḥ}]{\skm{m }avadhyaḥ}
	\rdg[wit={Jyo}]{aśakyaḥ}} sarvadehinām/}\\+}
\tl{
\pada{\app{\lem[wit={ceteri}]{agrāhyo}
	\rdg[wit={G11},alt={\om}]{\skp{\om}}} 
\app{\lem[wit={V15,J10}]{mantratantrāṇāṃ} % tatrāṇāṃ V15
	\rdg[wit={G5,Jyo}]{mantrayantrāṇāṃ}% +M3
	\rdg[wit={G11},alt={\om}]{\skp{\om}}}}
\pada{yogī \app{\lem[wit={J10,Jyo}]{yuktaḥ}% M3
	\rdg[wit={V15}]{muktaḥ}% G5
	\rdg[wit={G11},alt={\om}]{\skp{\om}}} samādhinā//}
	\sgwit{G11,V15,J10,Jyo}\myfn{In \getsiglum{G11,G5} this verse is transposed with the next one.}\\!} % NotIn{V3}
\end{alttlg}


\begin{alttlg}[hp04_035_6]% = V3_4.170
\tl{
\pada{na gandhaṃ na rasaṃ rūpaṃ} % nāgandhaṃ G11, raso G5
\pada{\app{\lem[resp=emend,post=\texteng{(cf.\,VM)}]{na sparśaṃ na ca nisvanam}
	\rdg[wit={V3,G11,G5},post=\texteng{(the first na om. \getsiglum{V3})}]{na sparśaṃ na ca na śrutaṃ}% niśrutaṃ F
	\rdg[wit={Jyo}]{na ca sparśaṃ na niḥsvanam}
	}/}\\+}
\tl{
\pada{nātmānaṃ
	\app{\lem[wit={G11,G5,Jyo}]{na paraṃ}
	\rdg[wit={V3}]{paramaṃ}}
	vetti}
\pada{yogī
	\app{\lem[wit={G11,Jyo}]{yuktaḥ}
	\rdg[wit={G5}]{muktaḥ}
	\rdg[wit={V3}]{yukti}
	} samādhinā//} 
	\sgwit{V3,G11,G5,Jyo}\\!} %\NotIn{N19,V15,J10}
\end{alttlg}

\Anm{\getsiglum{G11,V15,J10} have \ref{pravese} \textit{praveśe nirgame vāme} here}

%\newpage
\begin{alttlg}[hp04_035_7]% = V3_4.171
\tl{
\pada{cittaṃ na suptaṃ no jāgrat} % nna P11; jāgrati J10ac
\pada{\app{\lem[wit={G11}]{smṛtiman na ca}% +F
		\rdg[wit={C6}]{smṛtyamanna}
		\rdg[wit={V3}]{sṛtinannaṃ ca}
		\rdg[wit={V15}]{smṛtivarṇaṃ ca}
		\rdg[wit={P11}]{na smṛtir na ca}
		\rdg[wit={Jyo}]{smṛtivismṛti}
		\rdg[wit={J10}]{spṛśati vastu ca}}
	\app{\lem[wit={GrB,G11,V15,J10}]{nānyathā}
		\rdg[wit={Jyo}]{varjitam}}/}
		\lineom{ab}{G5}\\+}
\tl{
\pada{\app{\lem[wit={GrB,G11,G5,V15}]{nāstam eti}
		\rdg[wit={J10}]{na vāstum eti}
		\rdg[wit={Jyo}]{na cāstam eti}}
	\app{\lem[wit={P11,C6,G11,G5,V15,J10}]{na codeti}
		\rdg[wit={V3}]{na cādeti}
		\rdg[wit={Jyo}]{nodeti}}}
\pada{\app{\lem[wit={P11,C6,G11,G5,V15,Jyo}]{yasyāsau}
		\rdg[wit={J10}]{yathāsau} % mukti J10ac
		\rdg[wit={V3},alt={\illeg}]{\skp{\illeg}}}
	\app{\lem[wit={ceteri}]{mukta eva saḥ} % mukti J10ac
		\rdg[wit={V3},alt={\illeg}]{\skp{\illeg}}}//} 
		\sgwit{GrB,G11,G5,V15,J10,Jyo}\label{nasuptam}\\!} % yavaprasa in J8
\end{alttlg}

%\newpage
\begin{alttlg}[hp04_035_8]% = V3_4.172
\tl{
\pada{\app{\lem[wit={V3,G11,Jyo}]{svastho}% +M3?
		\rdg[wit={P11}]{svapno}
		\rdg[wit={G5}]{svapne}
		\rdg[wit={C6}]{supto}
		\rdg[wit={V15}]{svecchā}
		} jāgradavasthāyāṃ} % °sthīyāṃ P11; ṃ om. V3
\pada{\app{\lem[wit={P11,V3,G11,Jyo}]{suptavad yo}
		\rdg[wit={C6}]{suptavadhyo}
		\rdg[wit={V15}]{suptaḥ sadyo}
		\rdg[wit={G5}]{pūrvavad yo}}%
	\app{\lem[wit={V3,G11,G5,V15,Jyo}]{'vatiṣṭhate}
		\rdg[wit={P11,C6}]{vatiṣṭhati}}/}\\+} % °tiṣṭhati V15pc?
\tl{
\pada{\app{\lem[wit={V15,Jyo}]{niḥsvāsocchvāsa}
		\rdg[wit={G11,G5}]{niśvāsocchvāsa}% +F
		\rdg[wit={V3}]{niśvāsośvāsa}
		\rdg[wit={P11}]{nisvāsośvaḥsa}
		\rdg[wit={C6}]{niḥśvāsaśvāsa}}%
	\app{\lem[wit={V3,V15,Jyo}]{hīnaś ca}% source & testimonia
		\rdg[wit={P11,C6,G11}]{hīnas tu}% +F
		\rdg[wit={G5}]{hīnasya}
		}} 
\pada{\app{\lem[wit={G11,G5,V15,Jyo}]{niścitaṃ}
		\rdg[wit={V3}]{niścito}
		\rdg[wit={P11}]{niścitto}
		\rdg[wit={C6}]{niśceṣṭo}} mukta eva saḥ//} % sa V3
		\sgwit{GrB,G11,G5,V15,Jyo}\label{svastho}\\!}
\end{alttlg}
\endaltrecension


%\Anm{The following verses appear immediately after \ref{ekibhutam} in \getsiglum{N19,V15,J10} and after 4.42 in \getsiglum{GrB}}


\newpage
%==================================
\begin{tlg}[hp04_036]% = V3_4.63
\tl{
\pada{nādānusaṃdhānasamādhibhājāṃ}\\+} % dānā° V15, saṃdhāta P11, samādhinā J5
\tl{
\pada{yogīśvarāṇāṃ % <<yo>>gośvarāṇāṃ J7, °svarā° N23
	\app{\lem[wit={J5,C6,V3,J7,Gr3a,G11,V15}]{hṛdaye prarūḍham}% prarūḍha V3
		\rdg[wit={P11,N23}]{hṛdayaprarūḍhaṃ}% [rū]ḍhaṃ N23, rūṇāṃ P11; +F
		\rdg[wit={N19,J10,Jyo}]{hṛdi vardhamānaṃ}}/}\\+}  % vaddha° N19, vaddhra°? N26
\tl{
\pada{ānandam ekaṃ vacasā%m
	\app{\lem[wit={ceteri},alt={avācyaṃ}]{\skm{m }avācyaṃ} % yacasām V3
%		\rdg[wit={G11}]{avācaṃ}
		\rdg[wit={N19}]{avākyaṃ}
		\rdg[wit={C6,Jyo}]{agamyaṃ}}}\\+}
\tl{
\pada{\app{\lem[wit={ceteri}]{jānāti}
		\rdg[wit={P11,C6}]{jānāty a°}
		\rdg[wit={N19}]{jānaṃti}}
	\app{\lem[wit={V3,J7,G11,N19,V15,J10,Jyo}]{taṃ śrī}
		\rdg[wit={C6}]{°taḥ śrī}
		\rdg[wit={P11}]{°tītaṃ}
		\rdg[wit={N23}]{tatvaṃ śrī}
		\rdg[wit={Gr3a}]{tattvaṃ}}%
	\app{\lem[wit={ceteri}]{gurunātha}
		\rdg[wit={Gr3a}]{guṇanātha}}
	\app{\lem[wit={N3,GrB,J7,Gr3a,G11,V15}]{eva}
		\rdg[wit={N23}]{evaṃ}
		\rdg[wit={N19,Jyo}]{ekaḥ}% +F
		\rdg[wit={J10}]{ekaṃ}}//}
	\label{nadanu}
	\lineom{cd}{J5}
	\unavbl{N3}\\!} % N3 resumes with nātha eva
\end{tlg}

\startaltrecension
\begin{alttlg}[hp04_036_1]% = V3_4.64
\tl{
\pada{\app{\lem[wit={ceteri}]{muktāsanasthito}% °sthite? V19
		\rdg[wit={N23}]{mudrāsanasthite}} yogī}
\pada{mudrāṃ saṃdhāya śāṃbhavīm/}\\+} 
% mudrā P11,N23,V3,J10; śāṃbhavī Gr2,P11,V3, sāṃbhavī J10
\tl{
\pada{śṛṇuyād dakṣiṇe karṇe} % śṛṇuyā V3, śruṇuyād V19,V15
\pada{nāda%m 
	\app{\lem[wit={Gr2,Gr3a,G11,G5,N19,V15},alt={antargataṃ sadā}]{\skm{m }antargataṃ sadā}
		\rdg[wit={V3,J10}]{antargataṃ mahat}
		\rdg[wit={P11}]{antastham ekadhī}}//}\label{muktasana}
		 \sgwit{P11,V3,Gr2,Gr3a,G11plus,N19,V15,J10} \anm{cf. \ref{muktasana2}}\\!}
\end{alttlg}
		 % \NotIn{J5,G4,Jyo}
\endaltrecension


\begin{tlg}[hp04_037]% = V3_4.71
\tl{
\pada{sarvacintāṃ parityajya} % ciya J5, ciṃtā C6,V3, ciṃtāḥ J10
\pada{\app{\lem[wit={ceteri}]{sāvadhānena}
	\rdg[wit={N19,J10}]{sarvadānena}}
	cetasā/}\\+}
\tl{
\pada{\app{\lem[wit={Gr1,GrB,Gr2,E2,G11,V15,Jyo}]{nāda evānusaṃdheyo}% nāda yecānusaṃdhyeyo J5
	\rdg[wit={N19},post=\texteng{(yo om. by haplogr.)}]{nādam evānusaṃdhe}
	\rdg[wit={V19,J10}]{nādam evānusaṃdhatte}% +K3
	}}
\pada{yoga\app{\lem[wit={ceteri},alt={sāmrājyam}]{sāmrājya\skp{m}}
	\rdg[wit={V19}]{samrājyam}
	\rdg[wit={C6}]{samrājam}
	}%
	\app{\lem[wit={N3,GrB,J7,Gr3a,G11,V15,Jyo},alt={icchatā}]{\skm{m }icchatā}
		\rdg[wit={G4,N19}]{icchatāṃ}
		\rdg[wit={N23,J10}]{icchati}
		\rdg[wit={J5}]{iṣṭatā}
		}//}%
\myfn{This verse is transposed with the next one in \getsiglum{GrB}.}
\label{sarvacinta2} 
\anm{after \ref{sarvacinta} \getsiglum{G11,N19,V15,J10}}\\!}
\end{tlg}

%\newpage
\begin{tlg}[hp04_038]% = V3_4.70
\tl{
\pada{\app{\lem[wit={ceteri}]{karṇau}
		\rdg[wit={N3,N23}]{karṇo}
		\rdg[wit={G4}]{karṇā}
		\rdg[wit={P11}]{karṇa}}
	\app{\lem[wit={N3,GrB,Gr2,E2,G11,N19,V15,Jyo}]{pidhāya}
		\rdg[wit={G4}]{pidhāna}
		\rdg[wit={V19}]{pi}
		\rdg[wit={J5}]{nidhāya}}
	\app{\lem[wit={G4,G5,N19}]{tūlena}% +G5
		\rdg[wit={P11}]{tulyena}
		\rdg[wit={N3,J5,V3,G11}]{mūlena}% +F
		\rdg[wit={Gr2}]{hastena}
		\rdg[wit={C6,E2,Jyo}]{hastābhyāṃ}
		\rdg[wit={V19}]{hastābhya[ṃ]}
		\rdg[wit={V15}]{śū\,\_\,na}}}
\pada{\app{\lem[wit={N3,J5,G11,N19,V15,Jyo}]{yaṃ}
		\rdg[wit={G4,C6,Gr2,Gr3a}]{yaḥ}% +F
		\rdg[wit={P11}]{saṃ}
		\rdg[wit={V3}]{sa}} śṛṇoti % °tiṃ P11, śruṇoti V3
	\app{\lem[wit={N3,J5,GrB,Gr3a,G11,N19,V15,Jyo}]{dhvaniṃ muniḥ} % dhvani V3, ddhvaniṃ N19; muni P11,V3
		\rdg[wit={N23}]{dhvaniṃ muniṃ}
		\rdg[wit={J7}]{munir dhvanim}
%		\rdg[wit={C7}]{dhvaniṃ dhvaniḥ}
		}/}\\+}
\tl{
\pada{\app{\lem[wit={ceteri}]{tatra cittaṃ} % citta N23
		\rdg[wit={J5,P11}]{tatra ciṃtāṃ}} % ṃ oṃ J5
	\app{\lem[wit={N3,J5,GrB,G5,Jyo}]{sthirī}
		\rdg[wit={Gr2,Gr3a,N19,V15}]{sthiraṃ}
		\rdg[wit={G11}]{sthitaṃ}}kuryād}
\pada{yāva%t  % yāva Gr2,J5,V3
	\app{\lem[wit={ceteri},alt={sthirapadaṃ}]{\skm{t }sthirapadaṃ}% sira° P11
		\rdg[wit={V3}]{sthiparamaṃ}}
	\app{\lem[wit={ceteri}]{vrajet}% vraje N23
		\rdg[wit={N19,V15}]{bhavet}}//}
		\NotIn{J10}\\!}
\end{tlg}

%\Anm{\getsiglum{V3} has \ref{makaranda1}--\ref{makaranda6} here}

\newpage
%========= common verses ==================
\begin{tlg}[hp04_039]% = V3_4.079 = 4c_75
\tl{
\pada{abhyasyamāno %  abhyāsya° N23, °mano N3
	\app{\lem[wit={ceteri}]{nādo}% nādau J5
		\rdg[wit={N23}]{nātho}}%
	\app{\lem[wit={ceteri}]{'yaṃ}% +G5, ya G11
		\rdg[wit={C6}]{yo}}}
\pada{\app{\lem[wit={C6,J7,G11,Jyo}]{bāhyam āvṛṇute}
		\rdg[wit={P11}]{bāhyanāvṛṇute}
		\rdg[wit={N23}]{bāhyanā\,\_\,ṇute}
		\rdg[wit={N3}]{bāhyam āśṛṇu}
		\rdg[wit={V3}]{bāhyam āsṛṇate}
		\rdg[wit={J5}]{bāhyaṃ ca śṛṇute}
		\rdg[wit={N19}]{bāhyamānaśṛṇvate}
		\rdg[wit={J10}]{cānyam āśṛṇute}
		\rdg[wit={V19,V15}]{bāhyam āvartaye}% °yed +K3,C7
		\rdg[wit={E2}]{bāhyād āvartayed}\marma} 
	\app{\lem[wit={N3,J7,Gr3a,V15,J10,Jyo}]{dhvanim}
		\rdg[wit={N23}]{dhvani}
		\rdg[wit={GrB,G11,G5,N19}]{dhvaniḥ}% ddhaniḥ N19; +F
		\rdg[wit={J5}]{dhvaniṃḥ}}/}\\+}
\tl{
\pada{\app{\lem[wit={N3,J5,GrB,Gr2,E2,G5,N19,V15,Jyo},alt={pakṣād/pakṣāt}]{pakṣā\skp{d}}
%		\rdg[wit={K3}]{pakṣāt}
		\rdg[wit={G4,V19,G11,J10}]{paścād}}%
	\app{\lem[wit={N3,J5,V3,J7,E2,J10,Jyo},alt={vikṣepam akhilaṃ}]{\skm{d }vikṣepam akhilaṃ}% +C7
		\rdg[wit={N23}]{vikṣeyam akhilaṃ}
		\rdg[wit={V19}]{vikṣepam atulaṃ}
		\rdg[wit={G4}]{vikṣiptam a[nila]ṃ}
		\rdg[wit={G11,G5}]{vikṣiptam akhilaṃ}
		\rdg[wit={P11}]{vikṣyemanilaṃ}
		\rdg[wit={N19,V15}]{vipakṣam akhilaṃ}
		\rdg[wit={C6}]{vipakṣayed enaṃ}}}
\pada{\app{\lem[wit={ceteri}]{jitvā} % jītvā P7
		\rdg[wit={J10}]{jīvo}} yogī sukhī bhavet//}\\!}
\end{tlg}


%\newpage
\begin{tlg}[hp04_040]% = V3_4.80
\tl{
\pada{\app{\lem[wit={ceteri}]{śrūyate}% śūyate N3
		\rdg[wit={E2}]{jāyate}% +C7
		}
	\app{\lem[wit={ceteri}]{prathamābhyāse}
		\rdg[wit={V19}]{prathame bhyāse}
		\rdg[wit={N3}]{prathamābhyāso}}}
\pada{nādo nānāvidho mahān/}\\+} % nādau J5, nādā N23; vidhā N19; mahāt N19 
\tl{
\pada{\app{\lem[wit={ceteri}]{vardhamāne tato'bhyāse}
		\rdg[wit={V15,Jyo}]{tato'bhyāse vardhamāne}}}
\pada{śrūyate % srū˟yate N3, śrūyete P11
	\app{\lem[wit={N3,J5,C6,V3,Gr3a,G11,J10}]{sūkṣmasūkṣmataḥ}
		\rdg[wit={J7,V15,Jyo}]{sūkṣmasūkṣmakaḥ} % 
		\rdg[wit={N23}]{sūjyasūjyakaḥ}
		\rdg[wit={P11,N19}]{sūkṣmataḥ}}//}\\!} % śū° P11; ḥ om. N19; haplogr.
\end{tlg}

%\newpage
\begin{tlg}[hp04_041]% = V3_4.81
\tl{
\pada{ādau jaladhi% adau P7
	\app{\lem[wit={N3,J5,C6,J7,Gr3a,G11,N19,V15,J10,Jyo}]{jīmūta}% jāmūta J5
		\rdg[wit={P11,V3,N23}]{jīmūte}}}% °te J10pc
\pada{bherī% bhīrī K3
	\app{\lem[wit={P11,G11,N19,V15,J10}]{nirjhara}
		\rdg[wit={V19}]{nirjara}
		\rdg[wit={V3}]{nirbhara} % +P7
		\rdg[wit={C6}]{nigama}
		\rdg[wit={J5}]{nisara}
		\rdg[wit={N3}]{rsara} % unm.
		\rdg[wit={N23}]{sarāva}
		\rdg[wit={J7}]{śabdatu}
		\rdg[wit={E2}]{bhūrbhūra}% +C7
		\rdg[wit={Jyo}]{jharjhara}}%
	\app{\lem[wit={C6,N19,Jyo}]{saṃbhavāḥ}
		\rdg[wit={N3,J5,P11}]{saṃbhavā}
		\rdg[wit={Gr2,Gr3a,G11,V15}]{saṃbhavaḥ}
		\rdg[wit={V3,J10}]{nisvanaḥ}}/}\\+} % niśvanaḥ
\tl{
\pada{madhye
	\app{\lem[wit={ceteri}]{mardala}% mardvala N23
		\rdg[wit={G11}]{maddala}
		\rdg[wit={E2}]{mandala}% +K3,C7
		}%
	\app{\lem[wit={N3,J5,G11,N19,V15,Jyo}]{śaṃkhotthā}% śaṃdho J5; °otchā N3
		\rdg[wit={P11,C6pc,V3,Gr3a,G5,J10}]{śaṃkhottha}% saṃkho V3, śaṃ<<kho>>tha C6; °occha P11
		\rdg[wit={Gr2}]{śaṃkhotha}
		\rdg[wit={C6ac}]{śaṅkhottho}% +K3
		\rdg[wit={G4}]{śaṃkhoddhāḥ}}\marma}
\pada{ghaṇṭā\app{\lem[wit={J5,G4,C6,V3,J7,G11,N19,V15,Jyo}]{kāhala}
		\rdg[wit={N3,P11}]{kāhāla}
		\rdg[wit={N23}]{kāhla}
		\rdg[wit={Gr3a}]{kalaha}
		\rdg[wit={J10}]{kolāha}}%
	\app{\lem[wit={N3,J5,GrB,Jyo},alt={°jās}]{\skp{°}jā\skp{s}}
		\rdg[wit={Gr2,Gr3a,G11}]{jas}
		\rdg[wit={G4,N19,V15}]{kās}
		\rdg[wit={J10}]{las}}%
	\app{\lem[wit={ceteri}]{\skm{s }tathā} % tatha J10
		\rdg[wit={C6}]{tataḥ}}//}\\!}
\end{tlg}

%\newpage
\begin{tlg}[hp04_042]% = V3_4.82
\tl{
\pada{\app{\lem[wit={Gr1,GrB,J7,Gr3a,G11,V15,Jyo}]{ante}
		\rdg[wit={N19,J10}]{anye}
		\rdg[wit={N23}]{avai}
		} tu kiṅkiṇī% kikiṇī N3, °nī V19, kiṃkaṇī J5
	\app{\lem[wit={N3,G11,N19,V15,J10,Jyo}]{vaṃśa}% vaṃśaṃ
		\rdg[wit={C6,V3,Gr2,Gr3a}]{vṛnda}% vṛṃda -> <<śa>>bdaṃva?! J7
		\rdg[wit={G4}]{bṛṃdā}
		\rdg[wit={P11}]{vaṃda}
		\rdg[wit={J5}]{śabda}}}%
\pada{\app{\lem[wit={Gr1,GrB,Gr2,Gr3a,G11,J10,Jyo}]{vīṇā}
		\rdg[wit={N19,V15}]{nādā}
		}bhramara% bhrumara N23, bhrasara N19
	\app{\lem[wit={N3,G4,C6,G11,N19}]{nisvanāḥ}
		\rdg[wit={V3,J10}]{nisvanā} % nisvānā J10
		\rdg[wit={J7,V19}]{nisvanaḥ} % niśvanaḥ V19
		\rdg[wit={V15,Jyo}]{niḥsvanāḥ}
		\rdg[wit={J5}]{niḥśvanā}
		\rdg[wit={P11,N23,E2}]{niḥsvanaḥ}% nissvanaḥ K3
		}/}\\+}
\tl{
\pada{iti \app{\lem[wit={N3,J5,P11,C6,G11,N19,V15,J10,Jyo}]{nānāvidhā}
		\rdg[wit={V3,Gr2,Gr3a}]{nānāvidho}}
	\app{\lem[wit={N3,C6,J10,Jyo}]{nādāḥ}
		\rdg[wit={J5,P11,V3,G11,V15}]{nādā}
		\rdg[wit={J7,Gr3a}]{nādaḥ}
		\rdg[wit={N23}]{nādaṃ}
		\rdg[wit={N19}]{vādāḥ}}}
\pada{\app{\lem[wit={J5,P11,C6,G11,V15,J10,Jyo}]{śrūyante}
		\rdg[wit={N3,V3,Gr2,Gr3a,N19}]{śrūyate}} % śṛyate N3
	\app{\lem[wit={N3,J5,GrB,Gr2,Gr3a,G11,Jyo}]{deha}
		\rdg[wit={N19,J10}]{yatra}
		\rdg[wit={V15}]{tatra}}%
	\app{\lem[wit={N3,J5,P11,V3,G11,N19,V15,J10}]{madhyataḥ}
		\rdg[wit={C6,Jyo}]{madhyagāḥ}
		\rdg[wit={Gr2,Gr3a}]{madhyagaḥ}}//}\\!}
\end{tlg}

%\newpage
\begin{tlg}[hp04_043]% = V3_4.83
\tl{
\pada{\app{\lem[wit={ceteri}]{mahati}
		\rdg[wit={J5}]{mahatiḥ}
		\rdg[wit={V15}]{mahatī}
		\rdg[wit={C6},alt={\om}]{\skp{\om}}}
	\app{\lem[wit={ceteri},alt={śrūyamāṇe/-māne}]{śrūyamāṇe}% °ṇo J5, °ne J10
		\rdg[wit={N23}]{{[ṇya]}yatamāne}}%
	\app{\lem[wit={ceteri}]{'pi}
		\rdg[wit={Gr2}]{ti}
		\rdg[wit={C6}]{pi nāde vai}}} % +P7
\pada{meghabhe%ry
	\app{\lem[wit={J5,Gr2,G5,N19,J10},alt={ādikadhvanau}]{\skm{ry}ādikadhvanau}% ddhūnau N19
		\rdg[wit={G11}]{ākadhvanau}
		\rdg[wit={C6,V3,Gr3a,Jyo}]{ādike dhvanau}
		\rdg[wit={P11}]{ādike dhṛti}
		\rdg[wit={V15}]{ādike svane}
		\rdg[wit={N3}]{ādidaṃ dhvanau}}/}\\+}
\tl{
\pada{\app{\lem[wit={N3,J5,GrB,G11,N19,V15,J10,Jyo}]{tatra}
		\rdg[wit={Gr2,Gr3a}]{tataḥ}}
	\app{\lem[wit={ceteri},alt={sūkṣmāt}]{sūkṣmā\skp{t}}% sūkṣmā<<t>> J7
		\rdg[wit={J5,N19}]{sūkṣmā}
		\rdg[wit={P11}]{sūkṣmāṃ°}
		\rdg[wit={J10}]{sūkṣmaṃ}
		\rdg[wit={P11},alt={\om}]{\skp{\om}}}%
	\app{\lem[wit={ceteri},alt={sūkṣmataraṃ}]{\skm{t }sūkṣmataraṃ}% °tara N23
%		\rdg[wit={C7}]{sūkṣmatamaṃ}
		\rdg[wit={P11}]{°taraṃ nādaṃ}
		\rdg[wit={J10}]{nādam eva}}}
\pada{\app{\lem[wit={ceteri}]{nādam eva}
		\rdg[wit={J7}]{nādam evaṃ}
		\rdg[wit={J10}]{paritopi}}
	\app{\lem[wit={ceteri}]{parāmṛśet} % pasamṛśet N23
		\rdg[wit={V19}]{parāmṛṣet} % pasamṛṣet V19ac
		\rdg[wit={J5}]{parāmṛśaṃ}
		\rdg[wit={J7}]{samabhyaset}}//}\\!}
\end{tlg}

\newpage
\begin{tlg}[hp04_044]% = V3_4.84
\tl{
\pada{\app{\lem[wit={ceteri},alt={ghanam}]{ghana\skp{m}}
		\rdg[wit={J10}]{dhvanam}}m utsṛjya
	\app{\lem[wit={N3,GrB,G11,N19,V15,J10,Jyo}]{vā sūkṣme}
		\rdg[wit={J5,G4,Gr2,V19}]{vā sūkṣmaṃ}
		}} % sūkṣmo P7
\pada{sūkṣmam utsṛjya vā % sūkṣmasūtsṛjya N23; śū° P11
	\app{\lem[wit={Gr1,P11,C6,G11,N19,V15,Jyo}]{ghane}
		\rdg[wit={V3}]{ghanen}
		\rdg[wit={Gr2,V19}]{ghanam}% +K3,C7
		\rdg[wit={J10}]{dhune}}\marma/}\\+}
\tl{
\pada{\app{\lem[wit={N3,J5,GrB,G11},alt={tau tyaktvā ... syād vā}]{%	dau F; tyaktrā J5, tyaktā P11, tyakvā V3; tu tyaktvā madh. + + + G4
			tau tyaktvā
			\app{\lem[wit={J5},alt={madhyame}]{madhyame}% +F
				\rdg[wit={N3,P11,V3,G11},alt={madhyama}]{madhyama}
				\rdg[wit={C6},alt={madhyama<<ḥ>>}]{madhyama<<ḥ>>}} 
			\app{\lem[wit={N3,GrB},alt={syād vā}]{syād vā}% madhyame vāpi F
				\rdg[wit={G11,G5},alt={syādau}]{syādau}
				\rdg[wit={J5},alt={syātaṃstā}]{syātaṃstā}
				}}% nested!
		\rdg[wit={N19,V15}]{ramamāṇam api kṣipraṃ}
		\rdg[wit={J10,Jyo}]{ramamāṇam api kṣiptaṃ}
		\rdg[wit={Gr2,V19}]{paraṃ tatraiva nikṣipya}% +K3,C7
		}}
\pada{mano % manā? N23, manau V19
	\app{\lem[wit={ceteri}]{nānyatra}
		\rdg[wit={N19,V15,J10}]{nātra pra°}}
	\app{\lem[wit={ceteri}]{cālayet} % c<<ā>>layet J7
		\rdg[wit={J10}]{cālet}
		\rdg[wit={N23}]{vālayet}
		\rdg[wit={V3}]{cālayan}}//}
		\NotIn{E2}\\!}
\end{tlg}

%\newpage
\begin{tlg}[hp04_045]% = V3_4.085
\tl{
\pada{yatra kutrāpi vā nāde}
\pada{\app{\lem[wit={ceteri}]{lagati}
		\rdg[wit={N23}]{lagavi}
		\rdg[wit={P11}]{lagnaṃti}
		\rdg[wit={J10}]{galati}}
	\app{\lem[wit={ceteri}]{prathamaṃ}% °ma J5
		\rdg[wit={V19}]{prathame}}
	\app{\lem[wit={ceteri}]{manaḥ}
		\rdg[wit={N23}]{mataḥ}}/}\\+} %
\tl{
\pada{tatraiva
	\app{\lem[wit={N3,P11,C6,G11,V15},alt={tat}]{ta\skp{t}}% ta(l.br.)tat V15
		\rdg[wit={V3,N19}]{ta}
		\rdg[wit={J5}]{tā}
		\rdg[wit={J7,Gr3a,Jyo}]{su°}% +F
		\rdg[wit={N23}]{stu}
		\rdg[wit={J10}]{niś°}}%
	\app{\lem[wit={ceteri},alt={sthirī}]{\skm{t }sthirī}% sthirā J5; +G5
		\rdg[wit={G11}]{sthiro}
		\rdg[wit={N19}]{śarī}
		\rdg[wit={J10}]{°calo}}%
	\app{\lem[wit={Gr1,GrB,G11,N19,V15,J10}]{bhūtvā}
		\rdg[wit={Jyo}]{bhūya}
		\rdg[wit={Gr2,Gr3a}]{kuryāt}}} % kuryā J7
\pada{tena sārdhaṃ vilīyate//}\label{yatrakutrapi}\\!} % vinīyate N3
\end{tlg}

\Anm{\getsiglum{G11,N19,V15,J10} have \ref{kasthe}--\ref{sarvacinta} and \ref{sarvacinta2} here}% , and \getsiglum{V3} \ref{anahata}

%\newpage
%======== passage 4 (makaranda) ======

\begin{tlg}[hp04_046]% = V3_4.72
\tl{
\pada{makarandaṃ % ṃ om. J5,V15,J10
	\app{\lem[wit={ceteri},alt={piban}]{piba\skp{n}}
		\rdg[wit={J5}]{pived}
		\rdg[wit={N19}]{piven}}%
	\app{\lem[wit={Gr1,GrB,G11,V15,J10,Jyo},alt={bhṛṅgo}]{\skm{n }bhṛṅgo}% bhraṃgo P11
		\rdg[wit={Gr2,Gr3a}]{bhṛṅgī}
		\rdg[wit={N19}]{śṛṃgo}}}
\pada{\app{\lem[wit={N3,G4,V3,Gr3a,G11},alt={gandhān}]{gandhā\skp{n}}
		%\rdg[wit={K3,C7}]{gandhā}
		\rdg[wit={J7,N19,V15,J10,Jyo}]{gandhaṃ}
		\rdg[wit={J5,C6,N23}]{gandha}
		\rdg[wit={P11}]{gandho}}%
	\app{\lem[wit={ceteri},alt={nāpekṣate}]{\skm{n }nāpekṣate}% nā[p]i° G4; °kṣata J5
		\rdg[wit={N23}]{napekṣate}
		\rdg[wit={N19,J10}]{nopekṣate}}
	\app{\lem[wit={ceteri}]{yathā}
		\rdg[wit={N19}]{'nyathā}
		\rdg[wit={E2}]{yadā}}/}\\+}
\tl{
\pada{\app{\lem[wit={ceteri}]{nādāsaktaṃ}
		\rdg[wit={Gr2}]{nādasaktaṃ}} % śaktaṃ N19
	\app{\lem[wit={ceteri}]{tathā} % +P7
		\rdg[wit={C6}]{yathā}} cittaṃ}
\pada{viṣayā%n % viṣayā J5,J7, viṣayā<<n>> N23, °yāṃ V3,N19
	\app{\lem[wit={ceteri},alt={na hi}]{\skm{n }na hi}% nāhi P11
		\rdg[wit={V15}]{naiva}
%		\rdg[wit={C7}]{api}
		}
	\app{\lem[wit={N3,GrB,G11,N19,Jyo}]{kāṅkṣate}
		\rdg[wit={J5,Gr2,Gr3a,V15,J10}]{kāṅkṣati}}//}
	\label{makaranda1}\\!}
\end{tlg}


\Anm{\getsiglum{Gr2,Gr3a} have \ref{nadakoti} \textit{nādakoṭisahasrāṇi} here}
 

%\newpage
\begin{tlg}[hp04_047]% = V3_4.73
\tl{
\pada{\app{\lem[wit={J5,GrB,G11,N19,V15,Jyo}]{baddhaṃ}
		\rdg[wit={J10}]{buddhaṃ}
		\rdg[wit={N3}]{baṃdhaṃ}}
	\app{\lem[wit={N3,J5,P11,C6,G11,Jyo}]{vimukta}
		\rdg[wit={N19}]{vimuktaṃ}
		\rdg[wit={V15,J10}]{viyuktaṃ}
		\rdg[wit={V3}]{timukta}
		}cāñcalyaṃ}
\pada{nāda\app{\lem[wit={N3,J5,V3,G11,N19,V15,J10,Jyo}]{gandhaka}
		\rdg[wit={C6}]{gandhena}
		\rdg[wit={P11}]{gandhāya}}%
	\app{\lem[wit={N3,J5,C6,V3,G11,V15,Jyo}]{jāraṇāt}
		\rdg[wit={P11,N19,J10}]{jīraṇāt}}/} 
		\lineom{ab}{Gr2,Gr3a}\\+}
\tl{
		%\sgwit{C6,V3,G11,N19,V15,J10,Jyo}
\pada{\app{\lem[wit={N3,J5,C6,J7,V19,G11,N19,V15,J10,Jyo}]{manaḥ}% +K3,C7
		\rdg[wit={P11,V3}]{mana}
		\rdg[wit={N23}]{vona}
		\rdg[wit={E2},alt={\om}]{\skp{\om}}
		}%
	\app{\lem[wit={J5,P11,C6,G11,N19,J10,Jyo}]{pāradam āpnoti}
		\rdg[wit={V15}]{pārada āpnoti}
		\rdg[wit={V3}]{pāradham āpnoti}
		\rdg[wit={N3}]{pārajam āpnoti}
		\rdg[wit={J7,V19}]{pākam avāpnoti}% +K3,C7
		\rdg[wit={N23}]{cāvam avāpnoti}
		\rdg[wit={E2},alt={\om}]{\skp{\om}}
		}}
\pada{\app{\lem[wit={ceteri}]{nirālambākhya}
%		\rdg[wit={C7},alt={°ākṣa}]{nirālambākṣa}
		\rdg[wit={P11},alt={°āsthya}]{nirālambāsthya}
		\rdg[wit={J5},alt={°aratha}]{nirālaṃbaratha}
		\rdg[wit={E2},alt={\om}]{\skp{\om}}
		}%
	\app{\lem[wit={P11,V3}]{khoṭatāṃ}
		\rdg[wit={N19}]{khoṭatī}
		\rdg[wit={V15}]{khoṭakaṃ}
		\rdg[wit={Jyo}]{khe'ṭanaṃ} % +J11pc
		\rdg[wit={J10}]{khegataṃ}
		\rdg[wit={N3,C6,G11,G5}]{ghoṭatāṃ}
		\rdg[wit={J5}]{ghoṭatā}
		\rdg[wit={Gr2}]{ghoṭanam}
		\rdg[wit={G4}]{gopitāṃ}
		\rdg[wit={V19}]{codanaṃ}
%		\rdg[wit={C7}]{yodanaṃ}
		\rdg[wit={E2},alt={\om}]{\skp{\om}}
		}//}\\!} %\lineom{cd}{E2}
\end{tlg}

\newpage
\startaltrecension
\begin{alttlg}[hp04_047_1]% = V3_4.74
\tl{
\pada{\app{\lem[wit={V3,G11,N19,V15}]{baddhaḥ}
		\rdg[wit={C6}]{baddhas}
		\rdg[wit={J10}]{baddha}
		\rdg[wit={G5,Jyo}]{baddhaṃ}
		\rdg[wit={P11}]{baṃdhaḥ}}
	\app{\lem[wit={V3,G11,G5}]{sunādagandhena}
		\rdg[wit={N19}]{sunāde gandhena}
		\rdg[wit={P11}]{sunādavānpana}
		\rdg[wit={J10}]{sven nādagandhena}
		\rdg[wit={C6}]{tu nādagandhena}
		\rdg[wit={Jyo}]{tu nādabandhena}
		\rdg[wit={V15}]{suṃdhanādena}}}
\pada{\app{\lem[wit={GrB,G11,G5,N19,V15,J10}]{sadyaḥ}
		\rdg[wit={Jyo}]{manaḥ}}%
	\app{\lem[wit={P11,C6,G11,G5,N19,V15,J10,Jyo}]{saṃtyakta}
		\rdg[wit={V3}]{sa tyakta}}%
	\app{\lem[wit={GrB,G11,G5,N19,V15,J10}]{cāpalaḥ}
		\rdg[wit={Jyo}]{cāpalam}}/}\\+}
\tl{
\pada{prayāti
	\app{\lem[wit={G11}]{cetaḥsūtendraḥ}
		\rdg[wit={V3}]{cetaḥsuteṃdra}
		\rdg[wit={C6}]{cetaḥsūtrendre}
		\rdg[wit={G5}]{cetaḥśailendra}
		\rdg[wit={P11}]{cet sthūlendraḥ}
		\rdg[wit={V15}]{sūtacittendraḥ}
		\rdg[wit={N19}]{sūtaś citteṃdra}
		\rdg[wit={J10}]{svataś caikyaṃ iṃdra}
		\rdg[wit={Jyo}]{sutarāṃ sthairyaṃ}}}
\pada{\app{\lem[wit={P11,C6,G11,G5,N19,V15}]{pakṣachinna}
		\rdg[wit={J10}]{pacchacchinna}
		\rdg[wit={Jyo}]{chinnapakṣaḥ}
		\rdg[wit={V3},alt={\lacuna}]{\skp{\lacuna}}}
	\app{\lem[resp=emend,postwit=\texteng{(=\,\getsiglum{M1})}]{iti prathām}% +F
		\rdg[wit={G5}]{iti prathā}% +M3?
		\rdg[wit={P11}]{dṛti pṛthāṃ}
		\rdg[wit={C6}]{\_\,va patham}
		\rdg[wit={G11}]{iva prathāṃ}
		\rdg[wit={N19}]{iva prabhāṃ}
		\rdg[wit={V15}]{ivāprabhuḥ}
		\rdg[wit={J10}]{iva parvataḥ drumāḥ}
		\rdg[wit={Jyo}]{khago yathā}
		\rdg[wit={V3},alt={\lacuna}]{\skp{\lacuna}}}//}
	\sgwit{GrB,G11,N19,V15,J10,Jyo}\\!}
\end{alttlg}
% ithi pradhā M3, iti prathāṃ M1, iva prathāṃ G11, iva dyāmaḥ G7
% cf. Rasendracūḍāmaṇi 16.52-54
%{pañcamo grāsaḥ}
%evaṃ ca pañcamo grāsaḥ pradātavyo'ṣṭamāṃśataḥ /
%sa pātrastho'gnisaṃtapto na gacchati kathañcana // Rcūm_16.52 //
%sa pakṣacchinna ity uktaḥ sa mukto'khiladurguṇaiḥ /
%so'yaṃ niṣevitaḥ sūtastrimāsaṃ rājikāmitaḥ // Rcūm_16.53 //
%viḍaṅgatriphalākṣaudraiḥ khe devaiḥ saha saṅgamam /
%ghrāṇamātreṇa sūtendraḥ sarvaroganikṛntanaḥ // Rcūm_16.54 //
%guṇā ete vinirdiṣṭā rasasya rasavādibhiḥ /
%sakalāste guṇāḥ satyā bhairaveṇa prakīrtitāḥ // Rcūm_16.55 //
% cf. also NWS pakṣaccheda
\endaltrecension

%\newpage
\begin{tlg}[hp04_048]% = V3_4.75
\tl{
\pada{\app{\lem[wit={N3,P11,C6,J7,Gr3a,V15},alt={nādaśravaṇataś cittam}]{nādaśravaṇataś citta\skp{m}} % cittaṃm V19
		\rdg[wit={N19}]{nādaḥ śravaṇataś cittam}
		\rdg[wit={V3}]{nādaḥ śravaṇataḥś citam}
		\rdg[wit={J5}]{nādaḥ śravaṇañ vittaṃm}
		\rdg[wit={G11}]{nadaśravaṇakṛc cittaṃ}
		\rdg[wit={N23}]{nādaśravaṇaś cittaṃ matam}
		\rdg[wit={J10}]{nādena praṇataṃ cittam}
		\rdg[wit={Jyo}]{nādaśravaṇataḥ kṣipram}
		}}%
\pada{\app{\lem[wit={N3,GrB,Gr2,E2,G11,Jyo},alt={antaraṅga}]{\skm{m }antaraṅga}
		\rdg[wit={J5}]{anataraṃga}
		\rdg[wit={N19,V15}]{aṃtaraṃgaṃ}
		\rdg[wit={J10}]{aṃtaraṃgā}
		\rdg[wit={V19}]{aṃtaraṃ sa}
		}%
	\app{\lem[wit={ceteri}]{bhujaṅgamaḥ} % ṃ om. V3
		\rdg[wit={J7,E2}]{turaṅgamaḥ}% +C7
		\rdg[wit={N23}]{turaṃgavaḥ}
		}/}\\+}
\tl{
\pada{\app{\lem[wit={P11,V3,Gr2,N19,V15,J10,Jyo}]{vismṛtya}
		\rdg[wit={N3,J5,C6,G11}]{saṃsmṛtya}% ??
		\rdg[wit={Gr3a}]{viśūnyaṃ}}
	\app{\lem[wit={ceteri},alt={sarvam}]{sarva\skp{m}}
		\rdg[wit={N19,V15,J10}]{viśvam}}%
	\app{\lem[wit={N3,Jyo},alt={ekāgraḥ}]{\skm{m }ekāgraḥ}
		\rdg[wit={GrB,N23,Gr3a,G11,J10}]{ekāgraṃ}% aikā° G11
		\rdg[wit={J5}]{(e)kāgra}% me om. J5
		\rdg[wit={J7}]{ekāgryaṃ}
		\rdg[wit={V15}]{evāgraḥ}
		\rdg[wit={N19}]{evāgra}}}
\pada{kutracin na hi dhāvati//} % naṃ hi P11
	\NotIn{G4} \anm{after \ref{manomatta} \getsiglum{G11}}\\!} % om. G5,M3
\end{tlg}

%\newpage
\begin{tlg}[hp04_049]% = V3_4.76
\tl{
\pada{\app{\lem[wit={ceteri}]{manomatta}
		\rdg[wit={N23}]{manomantra}
		\rdg[wit={V3,J10}]{manonmatta}}gajendrasya} % maje° N19; °āsya V15
\pada{\app{\lem[wit={ceteri}]{viṣayodyāna}% °naṃ P11
		\rdg[wit={C6}]{viṣayoḍyā}
		\rdg[wit={J5}]{viṣayodhanu}
		\rdg[wit={V3}]{viṣayodhāma}
		\rdg[wit={G4}]{viṣayeṣudra}}%
	\app{\lem[wit={ceteri}]{cāriṇaḥ}
		\rdg[wit={P11}]{cāriṇaṃ}
		\rdg[wit={G4}]{cāraṇā[ḥ]}
		\rdg[wit={J5}]{vāriṇaḥ}
		\rdg[wit={N23}]{vāriṇaṃ}}/}\\+}
\tl{
\pada{\app{\lem[wit={N3,G4,V3,Gr3a}]{niyāmana}
		\rdg[wit={G11,V15}]{niyāmane}
		\rdg[wit={J10}]{nīyamānaḥ}
		\rdg[wit={J5,P11,C6}]{niyamena}
		\rdg[wit={J7}]{niryāmana}
		\rdg[wit={N19}]{niryāsane}
		\rdg[wit={N23}]{niyamitra}
		\rdg[wit={Jyo}]{samartho'yaṃ}}%
	\app{\lem[wit={ceteri}]{samartho'yaṃ}% samatho C6f
		\rdg[wit={G11}]{samartheyaṃ}
		\rdg[wit={Jyo}]{niyamane}}}
\pada{\app{\lem[wit={N3,J5,GrB,Gr2,Gr3a,G11}]{ninādo}
		\rdg[wit={N19,V15,J10,Jyo}]{nināda}}
	\app{\lem[wit={J5,GrB,Gr2,G11,V15,J10,Jyo}]{niśitāṅkuśaḥ} % niśinā° N23; °kuśa V3, °kuśaṃ J10, ṅ om. G11
		\rdg[wit={N19}]{niśatāṅkuḥ}
		\rdg[wit={Gr3a}]{niścayāṅkuśaḥ}
		\rdg[wit={N3}]{niyatāṃkuśaḥ}}//}\label{manomatta}\\!}
\end{tlg}

%\newpage
\begin{tlg}[hp04_050]% = V3_4.78
\tl{
\pada{\app{\lem[wit={ceteri}]{antaraṅga}
		\rdg[wit={V19}]{aṃtaraṃgaṃ}
		\rdg[wit={J10}]{nādoṃtaraṃ}}%
	\app{\lem[wit={C6,V3,G11},alt={°sya javino}]{\skp{°}sya javino}
		\rdg[wit={N3,J5}]{°sya javinaḥ}
		\rdg[wit={Jyo}]{°sya yamino}
		\rdg[wit={P11}]{°sya ca mano}
		\rdg[wit={Gr2,Gr3a,N19,V15}]{turaṅgasya} % truraṃga N19
		\rdg[wit={J10}]{tu saṃgamya}}}
\pada{\app{\lem[wit={GrB,N19,V15,J10,Jyo}]{vājinaḥ}
		\rdg[wit={N3,J5}]{kariṇaḥ}
		\rdg[wit={G11}]{<<ga>>jasya}% G5 omits this hemistich.
		\rdg[wit={Gr2,Gr3a}]{vijñānaṃ}} % °na N23
	\app{\lem[wit={N3,G11,Jyo}]{parighāyate}% +C8,P7
		\rdg[wit={P11}]{parighātayaḥ}
		\rdg[wit={C6}]{pariṣāyate}
		\rdg[wit={J5,V3,Gr2,N19,J10}]{paridhāyate}
		\rdg[wit={V15}]{paridhāvataḥ}
		%\rdg[wit={K3,C7}]{parimīyate}
		\rdg[wit={V19}]{parimeyate}
		\rdg[wit={E2}]{parameyate}}/}\\+}
\tl{
\pada{\app{\lem[wit={N3,J5,GrB,Gr3a,G11,N19,V15,Jyo}]{nādopāstir ato}%
		\rdg[wit={Gr2}]{nādopāstivato}
%		\rdg[wit={C7}]{nādopāstimato}
%		\rdg[wit={V19}]{nādopāstiratir}% 1st occurrence
		\rdg[wit={J10},alt={\om}]{\skp{\om}}} 
		nitya}%m % nityaṃm N3
\pada{\app{\lem[wit={N3,J5,P11,V3, V19},postwit=\texteng{(1)},alt={avadhāryāpi}]{\skm{m }avadhāryāpi}% +V19(1),C7(1)
		\rdg[wit={J7},postwit=\texteng{(1)}]{avadhāyāpi}
		\rdg[wit={N23},postwit=\texteng{(1)}]{anadhāyāpi}% 1st occurrence
		\rdg[wit={C6}]{avadhāryo pi}
		\rdg[wit={V15,Jyo}]{avadhāryā hi}
		\rdg[wit={E2,G11, Gr2},postwit=\texteng{(2)}]{avagamyā hi}% avagamyo G5
		\rdg[wit={V19},postwit=\texteng{(2)}]{avagamya hi}
		\rdg[wit={N19}]{avagamyaṃ hi}
		\rdg[wit={J10},alt={\om}]{\skp{\om}}}
	\app{\lem[wit={J5,GrB,Jyo}]{yoginā}
		\rdg[wit={N3,G11,N19,V15}]{yogināṃ}
		\rdg[wit={Gr2,V19},postwit=\texteng{(1)}]{yoginaḥ}% +G5,K3,C7   1st
		\rdg[wit={E2, Gr2,V19},postwit=\texteng{(2)}]{yogibhiḥ}
		\rdg[wit={J10},alt={\om}]{\skp{\om}}}//}%
\myfn{In \getsiglum{Gr2,Gr3a}, this verse and the following verse (\ref{makaranda6}) are located after \ref{anahata}. However, the second half of the verse, without the first half, is also written here (except in \getsiglum{E2}). The text of the hemistich differs between the two instances. The last Pāda reads more or less \textit{avadhāryāpi yoginaḥ} in the first occurrence (i.e., here), while it reads \textit{avagamyā hi yogibhiḥ} in the second occurrence (i.e., after \ref{anahata}).}
		\lineom{cd}{J10} \label{IV95Vu}\\!}
\end{tlg}


\newpage
\startaltrecension
\begin{alttlg}[hp04_050_1]% = V3_4.77
\tl{
\pada{\app{\lem[wit={P11,Gr2,E2,G11,V15,Jyo}]{nādo'ntaraṅga}
		\rdg[wit={C6,V3}]{nādotaraṅga}
		\rdg[wit={N19}]{nādāṃtaraṅga}
		\rdg[wit={V19}]{nādaturaṃga}
		\rdg[wit={J10},alt={\om}]{\skp{\om}}}%
	\app{\lem[wit={ceteri}]{sāraṅga}% sātaṅga? E2
%		\rdg[wit={C7}]{mātaṃga}
		\rdg[wit={J10},alt={\om}]{\skp{\om}}}}% sāraṅgaṃ V15
\pada{\app{\lem[wit={ceteri}]{bandhane}
		\rdg[wit={N23}]{baṃdhāna}
		\rdg[wit={V3}]{baṃdhana}
		\rdg[wit={J10},alt={\om}]{\skp{\om}}}
	\app{\lem[wit={ceteri}]{vāgurāyate}
		\rdg[wit={N23}]{yāgurāyate}
		\rdg[wit={J10},alt={\om}]{\skp{\om}}}/}\\+}
\tl{
\pada{\app{\lem[wit={ceteri}]{antaraṅga}
		\rdg[wit={N19,V15}]{antaraṅgaṃ}}%
	\app{\lem[wit={V15,Jyo}]{kuraṅgasya}
		\rdg[wit={GrB,Gr2,Gr3a,N19,J10}]{turaṅgasya}
		%\rdg[wit={K3,C7}]{turaṅgasyā°}
		}}
\pada{\app{\lem[wit={GrB,N19}]{rodhe}
		\rdg[wit={J10}]{rogo}
		\rdg[wit={V15}]{nādo}
		\rdg[wit={Jyo}]{vadhe}
		\rdg[wit={N23}]{bāhye}
		\rdg[wit={J7}]{bodho}
		\rdg[wit={E2}]{°vabodhe}% +C7; °varodhe K3
		\rdg[wit={V19},alt={\lacuna}]{\skp{\lacuna}}}\marmas
	\app{\lem[wit={V15,Jyo}]{vyādhāyate}
		\rdg[wit={V3}]{vādhāyate}
		\rdg[wit={P11}]{vādyāyate}
		\rdg[wit={C6}]{pi pariṣā°}%  parighāyate P7
		\rdg[wit={N19}]{vā gāyate}
		\rdg[wit={J10}]{vā gīyate}
		\rdg[wit={Gr2}]{pi līyate}
		\rdg[wit={E2}]{līyate}% +K3,C7
		\rdg[wit={V19},alt={\lacuna}]{\skp{\lacuna}}}%
	\app{\lem[wit={ceteri}]{'pi ca}
		\rdg[wit={P11}]{ti ca}
		\rdg[wit={C6}]{°yate}
		\rdg[wit={V19},alt={\lacuna}]{\skp{\lacuna}}}//}%
		\myfn{In \getsiglum{G11}, the first hemistich is found between \ref{IV95Vu}ab and cd, and the second hemistich is omitted;
		In \getsiglum{GrB,Jyo}, the whole verse is found before \ref{IV95Vu};
		\getsiglum{J10} merges the two verses into one:
		\textit{nādo'ntaraṃ tu saṃgamya vājinaḥ paridhāyate |
		antaraṅgaturaṃgasya rogo vā gīyate pi ca ||}}
		\label{makaranda6}\NotIn{Gr1} \lineom{cd}{G11plus}\\!}
\end{alttlg}
\endaltrecension

%\newpage
%
%% 4.064_1 (cf. 4.65cd)
%\pada{nādopāsti\app{\lem[wit={K3}]{r ato}
%		\rdg[wit={V19}]{ratir}
%		\rdg[wit={Gr2}]{vato}
%		\rdg[wit={C7}]{mato}} nityam}
%\pada{\app{\lem[wit={V19,C7}]{avadhāryāpi}
%		\rdg[wit={J7}]{avadhāyāpi}
%		\rdg[wit={N23}]{anadhāyāpi}
%		\rdg[wit={K3}]{avidhāryaṃ hi}} yoginaḥ/}\label{nadopasti}
%		\sgwit{Gr2,V19,K3,C7} \anm{≈ \ref{IV95Vu}cd}\\!}

%\newpage
\begin{tlg}[hp04_051]% = V3_4.166; Upagīti
\tl{
\pada{\app{\lem[wit={N3,J5,P11,V3,Jyo},post=\texteng{(ādī \getsiglum{N3})}]{ghaṇṭādināda}
%	\rdg[wit={N3}]{ghaṃṭādīnāda}
	\rdg[wit={C6,Gr2,V19,G11}]{ghaṇṭānināda}}% ghaṃṭāki? V19
\app{\lem[wit={V3,Jyo}]{sakta}
	\rdg[wit={J5}]{śakti}
	\rdg[wit={N3}]{śaktaś ca}
	\rdg[wit={Gr2,V19,G11}]{saktasya}% sukta N23, °syaṃ? V19
	\rdg[wit={P11}]{sadaṃkatā}
	\rdg[wit={C6}]{kuliśa}}%
\app{\lem[wit={Jyo}]{stabdhāntaḥ}
	\rdg[wit={P11}]{stabdhyaṃtaḥ}
	\rdg[wit={J5}]{stadhvāṃta}
	\rdg[wit={N3}]{stavyāṃtaḥ}
	\rdg[wit={V3}]{statravadhātaḥ}
	\rdg[wit={G11}]{stabdhasyāntaḥ}
	\rdg[wit={N23}]{sabdāntaḥ}
	\rdg[wit={J7}]{śabdataḥ}
	\rdg[wit={V19}]{śuddhāntaḥ}%  °syaṃ? V19; +K3,C7
	\rdg[wit={C6}]{pradhvānta}}%
\app{\lem[wit={P11,V3,G11,Jyo}]{karaṇahariṇasya}
	\rdg[wit={N3}]{karaṇaṃ hariṇasya}
	\rdg[wit={J5}]{karaṇaṃ mṛgasya}
	\rdg[wit={C6}]{hariṇasya ca}
	\rdg[wit={J7,V19}]{karaṇasya ca}
	\rdg[wit={N23}]{karaṇasya na}}/} 
	\lineom{a}{E2,N19,V15,J10}\\+}
%	\sgwit{Gr1,GrB,Gr2,V19,K3,C7,G11,Jyo}
\tl{
\pada{praharaṇa%m % karṇam! J5
\app{\lem[wit={GrB,G11},alt={atisukaraṃ}]{\skm{m }atisukaraṃ}
	\rdg[wit={N3}]{atisukasteraṃ}
	\rdg[wit={J5}]{avisukaraṇaṃ}
	\rdg[wit={Jyo}]{api sukaraṃ}}
\app{\lem[wit={N3,P11,G11,Jyo}]{syāc chara}% so P7
	\rdg[wit={C6}]{syāt sadṛ°}
	\rdg[wit={V3}]{syāra}
	\rdg[wit={J5}]{chara}}%
\app{\lem[wit={N3,P11,V3,G11}]{saṃdhātā}
	\rdg[wit={C6}]{°śaṃ dhātā}
	\rdg[wit={J5}]{saṃdhā}
	\rdg[wit={Jyo}]{saṃdhāna}} 
	pravīṇaś cet//} % vra° N3; °ṇāś G11
	\lineom{b}{Gr2,Gr3a,N19,V15,J10}%
%	\sgwit{Gr1,GrB,G11,Jyo}%
	\myfn{In \getsiglum{V3,G11} this verse is found after \ref{saukhya}.}
	\\!}
\end{tlg}
% P11
% ghaṃṭādinādasadaṃkatāstabdhāṃtaḥkaraṇahariṇasya/
% praharaṇam atisukaraṃ syāc charasaṃdhātā pravīṇaś cet//
% M1:
% ghaṃṭāninādasaktastabdhāṃtaḥkaraṇahariṇasya/
% praharaṇe sukaraṃ syā(c) [ch](ara)[sa]ṃdhānapravīṇaś cet// 4.103//
% P7:
% ghaṃṭādinādakuliśapradhvāṃtaharisasya ca/
% praharaṇam atisūkaraṃ syā<c> charasaṃghātā praviṇaś cet// 106/107 //
% C6:
% ghaṇṭāninādakuliśapradhvāṃtahariṇasya ca/
% praharaṇam atisukaraṃ syāt sadṛśaṃdhātā pravīnaś cet//
% N12:
% ghaṇṭādinākalādhvāṃtaḥkaraṇahariṇyaḥ syāt/
% praharaṇam atisukaraṃ syāc charasaṃdhānā pravīṇaś ca/

%\newpage
\begin{tlg}[hp04_052]%
\tl{\texteng{[Alt1]}
\pada{\app{\lem[wit={Gr1,P11,V3,Gr2,Gr3a,G11,Jyo}]{anāhatasya śabdasya}% sabda° V3,N23
%		\rdg[wit={V3,N23}]{anāhatasya sabdasya}
		\rdg[wit={C6}]{anāhatas tu yaḥ śabdas}}}
\pada{\app{\lem[wit={J5,C6,Gr2,Gr3a}]{tasya śabdasya yo dhvaniḥ}
		\rdg[wit={G11}]{tasya śabdasya yā dhvaniḥ}
		\rdg[wit={N3}]{tasya śabdasya ca dhvaniḥ}% +F
		\rdg[wit={G4}]{tasya yo dhvaniḥ}
		\rdg[wit={V3}]{śabdasyāṃtargato dhvaniḥ}
		\rdg[wit={P11}]{śabdasyāṃganabho dhvaniḥ}
		\rdg[wit={Jyo}]{dhvanir ya upalabhyate}}
		/}\\+}
\tl{
\pada{\app{\lem[wit={N3,P11,C6,Gr3a,G11,Jyo},postwit=\texteng{\getsiglum{N23}\textsubscript{pc}},alt={dhvaner}]{dhvane\skp{r}}
		\rdg[wit={J5,G4,V3,Gr2}]{dhvanir}}r
		antargataṃ % °gata N23
	\app{\lem[wit={G4,N23,E2,G11},alt={jyotir}]{jyoti\skp{r}}
		\rdg[wit={J7,V19}]{jyoti} % jyotī P7
		\rdg[wit={N3,Jyo}]{jñeyaṃ}% +F ##
		\rdg[wit={P11,V3}]{geyaṃ}% geya P11
		\rdg[wit={J5,C6},alt={\om}]{\skp{\om}}}}%
\pada{\app{\lem[wit={Gr2,G11},alt={jyotirantar}]{\skm{r }jyotiranta\skp{r}}% +K3
		\rdg[wit={C6,Gr3a}]{jyoterantar}
		\rdg[wit={J5}]{yotiraṃtar}
		\rdg[wit={G4}]{jyoti\,..\,.. }
		\rdg[wit={Jyo}]{jñeyasyāntar}% +F
		\rdg[wit={P11,V3}]{geyasyāntar}
		\rdg[wit={N3}]{yasyāṃtvaṃtar}
		}rgataṃ manaḥ/}\\+} % mana V3
\tl{
\pada{\app{\lem[wit={N3,P11,V3,J7}]{tan mano vilayaṃ}% lost G4
		\rdg[wit={G11}]{tan mano nilayaṃ}
		\rdg[wit={J5}]{tan maṃnaṃ vilayaṃ}
		\rdg[wit={C6,N23,Gr3a}]{yan mano vilayaṃ}
		\rdg[wit={Jyo}]{manas tatra layaṃ}% +F
		}
	\app{\lem[wit={J5,C6,V3,N23,Gr3a,G11,Jyo}]{yāti}
		\rdg[wit={N3,P11,J7}]{yāṃti}}}
\pada{tad viṣṇoḥ paramaṃ padam//} % viṣṇo N3,J5,C6
	\sgwit{Gr1,GrB,Gr2,Gr3a,G11,Jyo}%
	\label{anahata}\\!}
\end{tlg}


\startaltnormal
\begin{alttlg}[hp04_052_1]% = V3_4.86
\tl{\texteng{[Alt2]}
\pada{anāhatadhvaner anta}%r anāhataḥ V17; anta<<r>> J10
\pada{\app{\lem[wit={N19,V15,J10},alt={°r jñeyaṃ yat}]{\skp{°}r jñeyaṃ ya\skp{t}}% gataṃ M3
	\rdg[wit={G11}]{°r geyaṃ yat}
	\rdg[wit={G5}]{°m āpnuyāt}}t
	sūkṣma\app{\lem[wit={N19,V15,J10}]{sūkṣmakam} % +M3; sūkṣmya° N19
	\rdg[wit={G11,G5}]{sūkṣmataḥ}}/}\\+}
\tl{
\pada{manas tatra layaṃ yāti}
\pada{tad viṣṇoḥ paramaṃ padam//} % viṣṇo
\label{anahata2}\sgwit{G11,G5,N19,V15,J10}%
	\myfn{\getsiglum{G11} has both versions -- Alt\,2 here and Alt\,1 after \ref{sadanada} (preceded by three additional lines: 
	\devnote{vindur bhidyati nādena sa nādaḥ khena bhidyate/
	oṃkāradhvaninādena vāyus saṃharaṇāntikaṃ/
	nirālaṃbaṃ samuddiśya yatra nādo layaṃ gataḥ//}) --, while \getsiglum{G5,M3} have Alt\,2 only.}
\\!}
\end{alttlg}
\endaltnormal

%\newpage

\newpage
\begin{tlg}[hp04_053]% = V3_4.87
\tl{
\pada{\app{\lem[wit={ceteri},alt={tāvad ā°}]{tāvad ā}% tavad N23
		\rdg[wit={J10}]{bhāvanā°}}kāśasaṃkalpo} % kāsa J10; saḥkalpo J7
\pada{\app{\lem[wit={N3,J5,GrB,Gr2,G11,V15,J10,Jyo}]{yāvac chabdaḥ}% ḥ om. J5,P11,P7
		\rdg[wit={V19}]{yāvad bandhaḥ}% +C7
		\rdg[wit={N19}]{yāvad vādhaḥ}} pravartate/}\\+}
\tl{
\pada{niḥśabdaṃ  % niśabdaṃ V19,N3,J5; °bdāṃ V17
	\app{\lem[wit={ceteri}]{tat paraṃ}
		\rdg[wit={N23}]{paramaṃ}} brahma}
\pada{\app{\lem[wit={ceteri}]{paramātmā}
		\rdg[wit={Jyo}]{paramātme°}} % paramatmā V19
	\app{\lem[wit={N3,V3,J7}]{samīryate} % for workshop!
		\rdg[wit={J5,P11,N23,V19}]{samīyate} % to be treated equally, be placed on a level with (Apte)
		\rdg[wit={C6}]{°yam īryate}
		\rdg[wit={G4}]{samīkṣate}
		\rdg[wit={N19,V15,J10}]{°numīyate}
		\rdg[wit={G11,Jyo}]{°ti gīyate}}//}
	\NotIn{E2}\\!}
\end{tlg}

%\newpage
\begin{tlg}[hp04_054]% = V3_4.88
\tl{
\pada{\app{\lem[wit={Gr1,P11,C6,Gr2,V19,G11,Jyo},alt={yat}]{ya\skp{t}}
		\rdg[wit={V3},alt={\om}]{\skp{\om}}}t kiṃci%n
	\app{\lem[wit={Gr1,GrB,G11,Jyo},alt={nāda}]{\skm{n }nāda}
		\rdg[wit={Gr2,V19}]{nāma}}rūpeṇa}
\pada{śrūyate śaktir eva sā/}\\+} % śū° N3
\tl{
\pada{\app{\lem[wit={N3,P11,Gr2,G11}]{yas tacchrotā}% +K3,C7
		\rdg[wit={C6}]{yas tatsrotā}
		\rdg[wit={V19}]{yat ta[cch]roto}
		\rdg[wit={V3}]{yac chrotā ca}
		\rdg[wit={J5}]{yasmin śrato}
		\rdg[wit={Jyo}]{yas tattvānto}} 
		nirākāraḥ} % °kāra P11,V3; °kārā J5
\pada{sa eva parameśvaraḥ//} \NotIn{E2,N19,V15,J10}\\!} % ḥ om. J5,V3; e<<va>> N23
\end{tlg}

%\newpage
\begin{tlg}[hp04_055]% = V3_4.51 % Upagīti
\tl{
\pada{śravaṇa\app{\lem[wit={N3,J5,GrB,G11,N19,V15}]{mukha}
		\rdg[wit={Gr2,Gr3a,J10,Jyo}]{puṭa}}%
	\app{\lem[wit={ceteri}]{nayana} % cayana N23, naya<<na>> N3
		\rdg[wit={J10,Jyo}]{nayanayugala}}%
	\app{\lem[wit={ceteri}]{nāsā} % nāśā V19,N19,J5,V3,J10
		\rdg[wit={Jyo}]{ghrāṇa}}%
	\app{\lem[wit={J5,P11,C6,G11,N19,V15}]{nirodhanaṃ naiva kartavyam}% nirodhamaṃ N19
		\rdg[wit={N3}]{nirodhaṃ naiva kartavyaṃ}
		\rdg[wit={V3}]{nirodhanenaiva kartavyaṃ}
		\rdg[wit={Gr2,E2}]{mukhapuṭasaṃrodhanaṃ kāryam}
		\rdg[wit={V19}]{mukhapuṭarodhane kāryaṃ}
		\rdg[wit={J10}]{mukharodhanam eva kartavyaṃ}
		\rdg[wit={Jyo}]{mukhānāṃ nirodhanaṃ kāryam}}/}\\+}
\tl{
\pada{\app{\lem[wit={ceteri}]{śuddha}
		\rdg[wit={Gr2}]{śrīśuddha} % suddha N23
		\rdg[wit={V3},alt={\om}]{\skp{\om}}}%
	\app{\lem[wit={ceteri}]{suṣumṇā}% śu° P11,V15
		\rdg[wit={N23}]{suṣumūṇau}}% by haplography?
	\app{\lem[wit={J7,Gr3a,G11,Jyo}]{saraṇau}% śaraṇau V19,J10pc
		\rdg[wit={N19,V15,J10}]{śaraṇe}
		\rdg[wit={N3}]{tsaraṇaḥ}
		\rdg[wit={J5}]{śarada}
		\rdg[wit={G4}]{saraṇaiḥ}
		\rdg[wit={C6}]{tmaśaraṇaiḥ}
		\rdg[wit={P11}]{tmakārausaṃ}
		\rdg[wit={V3}]{maraṇai}
		\rdg[wit={N23},alt={\om}]{\skp{\om}}}
	\app{\lem[wit={Gr1,P11,V3,Gr2,Gr3a,G11,J10,Jyo}]{sphuṭam amalaḥ}% amala J5,G4, amalaṃ V3
%		\rdg[wit={V3}]{sphuṭam amalaṃ}
		\rdg[wit={C6}]{saṃsphurad amalaḥ}
		\rdg[wit={V15}]{vimalaḥ saṃ°}
		\rdg[wit={N19}]{vimalaḥ}}
	śrūyate nādaḥ//} % śṛyate N3, śraya .. + + G11; nāda J5,P11,V3
	\label{sravanaputa}
	\anm{after \ref{sapadakoti} in \getsiglum{GrB,G11,N19,V15,J10,Jyo}}\\!}
\end{tlg}

%\newpage
\startaltrecension
\begin{alttlg}[hp04_055_1]% = V3_4.89
\tl{
\pada{\app{\lem[wit={C6,V3,V15,J10}]{nādaḥ}% +M3
		\rdg[wit={P11,G11,G5,N19}]{nāda}} % +F
		śaktir iti
	\app{\lem[wit={V15,J10}]{khyāto} % khyato J10ac
		\rdg[wit={G11,G5}]{khyātā}% +M3
		\rdg[wit={N19}]{kṣāto}
		\rdg[wit={P11}]{jñeyaṃ}
		\rdg[wit={C6}]{jñeyā}% +F
		\rdg[wit={V3}]{jñeya}}}
\pada{\app{\lem[wit={P11,V3,G11,G5,N19,V15}]{nādajñānaṃ}
		\rdg[wit={C6,J10}]{nādo jñānaṃ}} sadāśivaḥ/}\\+} % śiva P11,V3
\tl{
\pada{\app{\lem[wit={G11}]{jñeye jñāne ca naṣṭe tu}
		\rdg[wit={G5}]{jñeyajñāne ca naṣṭe ca}
		\rdg[wit={V3}]{jñeye jñāne vilineṃta}
		\rdg[wit={P11}]{jñeye jñāne vilīnīṃta}% jñeyajñāne hi līyete
		\rdg[wit={C6}]{jñeyo jñāne vilīne tu}% vilīyeta P7
		\rdg[wit={N19}]{nādajñāne ca neṣṭe tad}% +J11ac
		\rdg[wit={V15}]{nādajñāne vinaṣṭe ca tad}
		\rdg[wit={J10}]{nādajñānena naṣṭena}% +J11pc
		}}
\pada{\app{\lem[wit={G11,G5,V15},alt={unmany}]{unma\skp{ny}}% +J11
		\rdg[wit={N19}]{unmadhy}
		\rdg[wit={J10}]{hy unmany}
		\rdg[wit={GrB}]{sonmany}% +F
		}%
	\app{\lem[wit={C6,G11,G5,J10},alt={evāvaśiṣyate}]{\skm{ny }evāvaśiṣyate}
		\rdg[wit={N19}]{edhāvaśiṣyate}
		\rdg[wit={V3}]{avāvaśiṣyate}
		\rdg[wit={P11}]{enāvaśiṣyati}
		\rdg[wit={V15}]{eva śiṣyate}}//}
\sgwit{GrB,G11,G5,N19,V15,J10}\\!}
\end{alttlg}

\begin{alttlg}[hp04_055_2]% = V3_4.90
\tl{
\pada{nādo yāvan manas tāva}%n
\pada{\app{\lem[wit={P11,V3,G11,G5,N19,J10},alt={nādānte tu}]{\skm{n }nādānte tu}
	\rdg[wit={V15}]{nādānte ca}
	\rdg[wit={C6}]{nādātīte}} manonmanī/}\\+}
\tl{
\pada{saśabdaṃ kathitaṃ vyoma} % na śabdaṃ G5
\pada{niḥśabdaṃ brahma 
	\app{\lem[wit={GrB,N19,V15,J10}]{kathyate}
	\rdg[wit={G11,G5}]{ucyate}% +F
	}//}
\sgwit{GrB,G11,G5,N19,V15,J10}\\!}
\end{alttlg}

%\newpage
\begin{alttlg}[hp04_055_3]% = V3_4.91
\tl{
\pada{sadā nādānusaṃdhānāt} % °bandhānāt G5
\pada{\app{\lem[wit={GrB,G11,G5,N19,V15,J10}]{saṃkṣīṇe}% sa P11
		\rdg[wit={Jyo}]{kṣīyante}}
	\app{\lem[wit={P11,C6,G11,G5}]{vāsanācaye}% +P7
		\rdg[wit={J10}]{vāsanodaye}% °dvaye F
		\rdg[wit={V3}]{vāsanāvayo}
		\rdg[wit={N19}]{vāsanākṣaye}
		\rdg[wit={V15}]{vāsanākṣaṇe}
		\rdg[wit={Jyo}]{pāpasaṃcayāḥ}}/}\\+}
\tl{
\pada{nirañjane
	\app{\lem[wit={G11,Jyo}]{vilīyete}% +F
		\rdg[wit={C6}]{vilīyeta}
		\rdg[wit={P11,V3}]{vilīyaṃte}
		\rdg[wit={V15,J10}]{ca līyete}
		\rdg[wit={G5,N19}]{ca līyeta}
		}}
\pada{\app{\lem[wit={G11}]{niścitaṃ manamārutau}% ##
		\rdg[wit={G5}]{niścitaṃ manamārute}
		\rdg[wit={N19}]{niścitta manamārutau}
		\rdg[wit={J10}]{niścitau manamārutau}% +F
		\rdg[wit={P11}]{niścitaṃ māruto manaḥ}
		\rdg[wit={V3}]{niścita māruto mana}
		\rdg[wit={V15,Jyo}]{niścitaṃ cittamārutau}
		\rdg[wit={C6}]{marutā niścitaṃ manaḥ}
		}//}
\sgwit{GrB,G11,N19,V15,J10,Jyo}\label{sadanada}\\!}
\end{alttlg}


\newpage
\begin{alttlg}[hp04_055_4]% = C8_84/85
\tl{
\pada{nādakoṭisahasrāṇi} % sahasrāṇī J7, sahasrāni N23, °śrāṇi V3,V19
\pada{\app{\lem[wit={ceteri}]{bindu}
	\rdg[wit={C6}]{veda}}koṭiśatāni ca/}\\+} % saṭāni N23ac, satāni N23pc
\tl{
\pada{\app{\lem[wit={ceteri}]{sarve}
		\rdg[wit={N23}]{sarvaṃ}} tatra layaṃ
	\app{\lem[wit={ceteri}]{yānti}
		\rdg[wit={C6,V19}]{yāti}}}
\pada{yatra % ya ca P11
	\app{\lem[wit={ceteri}]{devo}% +G5,M3
		\rdg[wit={G11}]{deve}
		\rdg[wit={V3,N19}]{deva}}
	\app{\lem[wit={ceteri}]{nirañjanaḥ}% +G5,M3; °jana P11, janaṃ V3
		\rdg[wit={G11}]{nirañjane}
		}//}
	\label{nadakoti} \NotIn{Gr1} %\sgwit{GrB,Gr2,Gr3a,G11,N19,V15,J10}
	\anm{after \ref{makaranda1} \getsiglum{Gr2,Gr3a}}\\!}
\end{alttlg}

%\newpage


\begin{altpostmula}[hp04_055_4p]
\app{\lem[wit={P11,G11,J10,Jyo}]{iti nādānusaṃdhānam}
\rdg[wit={C6,V3,G5},postwit=\texteng{(found between Pāda ab and cd of the next verse \getsiglum{C6})}]{iti nādānusaṃdhānavidhiḥ} % vidhi V3; +F
\rdg[wit={N19}]{iti nādānusaṃdhānāṃ yathā vṛddho veti}
\rdg[wit={V15},post=\texteng{(metrical!)}]{iti nādānusaṃdhānaṃ yathā vṛddhaiḥ prabhāṣitaṃ}}//
\sgwit{GrB,G11,G5,N19,V15,J10,Jyo}
\end{altpostmula}

\Anm{\getsiglum{V3} has Kālajñāna, Videhamuktikathana, and Kālavañcana sections here}

%%%%%%%%%%%%%%%%%%%%%%%
%\newpage

\begin{altava}[hp04_055_5a]
atha rājayogaḥ/ \sgwit{G11,G5}
\end{altava}

\Anm{\getsiglum{G11,G5} has 1.64 \textit{yuvā vṛddho 'tivṛddho vā} here.%
\myfn{\getsiglum{N19,V15} have the remnant of this verse. See the apparatus to the ending ``iti nādānusaṃdhānam" above.}}

\Anm{\getsiglum{G11,N19,V15,J10} have \ref{kalavancaka} \textit{sarve layahaṭhābhyāsāḥ} and \ref{astuva}ff. \textit{astu vā māstu vā} here}

\begin{alttlg}[hp04_055_5]% = V3_4.159, cf. 4.089
\tl{
\pada{sarve \app{\lem[wit={C6,V3}]{haṭhalayopāyā}
	\rdg[wit={P11}]{haṭhalayā bhāvyā}}}
\pada{rājayoga\app{\lem[wit={P11}]{padāvadhi}
	\rdg[wit={C6}]{padāvadhiḥ}
	\rdg[wit={V3}]{padāvadhiṃ}}/}\\+}
\tl{
\pada{rājayogapadaṃ prāpya}
\pada{jāyate\app{\lem[wit={P11,C6}]{'sau}
	\rdg[wit={V3}]{so}} nirañjanaḥ//} % °jana V3
	\sgwit{GrB}
	\anm{cf.\,\ref{kalavancaka}}\\!}
\end{alttlg}
\endaltrecension



%\newpage

\begin{tlg}[hp04_056]% = V3_4.173
\tl{
\pada{\app{\lem[wit={N3,G4,GrB,Gr2,G11}]{kāṣṭha}
		\rdg[wit={Gr3a}]{koṣṭha}}% ab lost J5; none N24
	\app{\lem[wit={Gr3a,G11}]{goṣṭhī}
		\rdg[wit={N3,G4,J7}]{goṣṭhi}% +F
		\rdg[wit={V3,N23}]{goṣṭha}
		\rdg[wit={P11}]{mathnī}
		\rdg[wit={C6}]{mathnā}}%
	\app{\lem[wit={V3,G11}]{prapañcena}
		\rdg[wit={N3}]{prapaṃce}
		\rdg[wit={G4,Gr2,Gr3a}]{prasaṅgena}% none J5,N24
		\rdg[wit={P11}]{pravacane}
		\rdg[wit={C6}]{pravartaṃ}}\marma}
\pada{\app{\lem[wit={N3,G4,GrB,G11}]{kiṃ sakhe śrūyatām idam}% śṛ° N3; °tam P11
		\rdg[wit={J7,Gr3a}]{nādam antargataṃ śṛṇu} % śruṇu V19
		\rdg[wit={N23}]{nāgadaṃtaṃmatargataṃ sṛṇu}}/}
	\lineom{ab}{J5}\\+}
\tl{
\pada{purā matsyendra\app{\lem[wit={N3,J5,GrB,G11},alt={bodhārtham}]{bodhārtha\skp{m}}
		\rdg[wit={Gr2,Gr3a}]{bodhāya}}}%
\pada{\app{\lem[wit={N3,J5,P11,C6,J7,Gr3a,G11},alt={ādināthoditaṃ}]{\skm{m }ādināthoditaṃ}
		\rdg[wit={N23}]{ādināthotigaditaṃ}
		\rdg[wit={V3}]{ānināthodinaṃ}}
		vacaḥ//}\label{kastha}
	\NotIn{N19,V15,J10,Jyo} \anm{after \manuref{4.0*8} \getsiglum{G11}}\\!} % vaca N3,V3
\end{tlg}

% C6 4.110: kaṣṭamathnā pravartaṃ (1 syll. ami.) kiṃ sakhe śrūyatām idam/
% purā matsyendrabodhārtham ādināthoditaṃ vacaḥ//
% P7: kāṣṭamathnvāṃ pravartante kiṃ sakhe śrūyatām idam/
% purā matsyendrabodhārtham ādināthoditaṃ vacaḥ//
% C8: kāṣṭamanāpravartaṃte ... bodhāya
% P11: kāṣṭhamathnī pravacane ... bodhārtham


\newpage
\begin{tlg}[hp04_057]% = V3_4.174
\tl{
\pada{yāvan naiva 
	\app{\lem[wit={ceteri}]{praviśati}
		\rdg[wit={N23}]{\_viśati}}
	\app{\lem[wit={ceteri},alt={caran}]{cara\skp{n}}
		\rdg[wit={J7}]{calan}
		\rdg[wit={N23}]{palan}
		\rdg[wit={N3}]{care}
		\rdg[wit={V3},alt={\om}]{\skp{\om}}}n 
	\app{\lem[wit={ceteri}]{māruto}
		\rdg[wit={N3}]{mārutaṃ}} % mādgato N23
	\app{\lem[wit={ceteri}]{madhya}
		\rdg[wit={V15}]{mādhya}}%
	\app{\lem[wit={N3,J5,C6,J7,V19,G11,N19,J10,Jyo}]{mārge}
		\rdg[wit={P11,N23}]{mārgo}
		\rdg[wit={E2,V15}]{mārgaṃ}% +C7
		\rdg[wit={V3}]{mārgā}}}\\+}
\tl{
\pada{yāva%d
	\app{\lem[wit={ceteri},alt={bindur}]{\skm{d }bindu\skp{r}} % bindu P11, vāyuḥ V19ac
		\rdg[wit={V15}]{bandho}
		\rdg[wit={N19}]{bandhaṃ}}r 
		na bhavati
	\app{\lem[wit={ceteri}]{dṛḍhaḥ} % dṛḍha V19
		\rdg[wit={N3,P11,G11}]{dṛḍhaṃ}% +F
		\rdg[wit={J5}]{sthiraḥ}}
	prāṇa\app{\lem[wit={Gr1,GrB,J7,G11,J10,Jyo}]{vāta}% prāṇa om. J5, prā<<ṇa>> V3
		\rdg[wit={N23,Gr3a,V15}]{vātaḥ}
		\rdg[wit={N19}]{vātaṃ}}%
	\app{\lem[wit={C6,Gr2}]{prabaddhaḥ}
		\rdg[wit={G4}]{prabaddhaṃ}
		\rdg[wit={P11,G11,V15}]{prabandhaḥ}% ḥ om. P11
		\rdg[wit={Gr3a,J10}]{prabuddhaḥ}
		\rdg[wit={N3}]{prabodhaḥ}
		\rdg[wit={V3}]{prabodhakaḥ}
		\rdg[wit={J5}]{prakṛddhaḥ}
		\rdg[wit={N19}]{na bandhanaḥ}
		\rdg[wit={Jyo}]{prabandhāt}% +F(twice),M1
		}/}\\+}
\tl{
\pada{\app{\lem[wit={P11,C6,N19,V15}]{yāvad vyomnā}
		\rdg[wit={N3,G4,G11}]{yāvad yomnā}
		\rdg[wit={J5}]{yād vyemnā}
		\rdg[wit={J7,Gr3a,J10}]{yāvad vyomnaḥ}% dyo° J7; rather °nmaḥ J10
		\rdg[wit={N23}]{yāva\,\_\,mnaḥ}% yāva _ mnaḥ N23
		\rdg[wit={V3}]{yāvad byomna}
		\rdg[wit={Jyo}]{yāvad dhyāne}}
	\app{\lem[wit={ceteri}]{sahajasadṛśaṃ}% °sadṛsāṃ J5
		\rdg[wit={N23}]{sahajasaṃśaṃ}
		\rdg[wit={G11}]{sadṛśasahajā}% °sahajaṃ F
		} jāyate naiva
	\app{\lem[wit={ceteri}]{tattvaṃ}
		\rdg[wit={V3,V15,J10}]{cittaṃ}}}\\+} % ?Jyo
\tl{
\pada{tāva\app{\lem[wit={ceteri},alt={sarvaṃ}]{\skm{t }sarvaṃ}
		\rdg[wit={G11}]{satvaṃ}
		\rdg[wit={V3,J10,Jyo}]{jñānaṃ}} vadati % vadaṃti V3
	\app{\lem[wit={N3,J5,C6,J7,E2,N19,V15,J10}]{yad idaṃ}% +C7
		\rdg[wit={V19,Jyo}]{tad idaṃ}% +K3
		\rdg[wit={P11,N23}]{yadi}% post=\texteng{(daṃ om. by haplography)}
		\rdg[wit={G11}]{yadi tat}% yadi taṃ F
		\rdg[wit={V3}]{satataṃ}}
	\app{\lem[wit={ceteri}]{dambha}
		\rdg[wit={G11,N19}]{ḍaṃbha}
		}mithyā%
	\app{\lem[wit={ceteri}]{pralāpaḥ}% °lāpa J5,N19, prallāpaḥ V3
		\rdg[wit={C6}]{pralābhaḥ}}//} 
	\anm{after \ref{durlabho} \getsiglum{G11,N19,V15,J10}}\label{yavan}\\!}
\end{tlg}

%========== Passage B ==========
%\newpage

\Anm{The following verses \ref{jnatva}--\ref{tatraika} are found immediately after \ref{jnanam} in \getsiglum{G11,N19,V15,J10,Jyo}}

\begin{tlg}[hp04_058]% = V3_4.175
\tl{
\pada{\app{\lem[wit={ceteri}]{jñātvā}
		\rdg[wit={C6}]{jitvā}
		\rdg[wit={V15}]{suṣu°}}
	\app{\lem[wit={N3,J5,J10,Jyo}]{suṣumṇāsadbhedaṃ}
		\rdg[wit={GrB,G11}]{suṣumṇāsaṃbhedaṃ}% +F
		\rdg[wit={N19}]{suṣumṇāṃ saśvedaṃ}
		\rdg[wit={J7,Gr3a}]{suṣumṇābhedaṃ hi}% sukhu° V19
		\rdg[wit={N23}]{suṣu<<m>>ṇāṃmedehi}
		\rdg[wit={V15}]{°mnāṃtagataṃ mārgaṃ}}}
\pada{\app{\lem[wit={ceteri}]{kṛtvā vāyuṃ}% kṛtvān V3; vāyu V19, pāyu N19
		\rdg[wit={V15}]{vāyuṃ kṛtvā}
		\rdg[wit={J5}]{tvāpa vāyuṃ}} ca
	\app{\lem[wit={ceteri}]{madhyagam}
		\rdg[wit={P11}]{madhyamaḥ}}/}\\+}
\tl{
\pada{\app{\lem[wit={N3,V3}]{kṛtvāsāv aindave sthāne} % aidava V3
		\rdg[wit={P11}]{kṛtvāsav aidavai sthānair}
		\rdg[wit={J5}]{kṛtvā tām aidave tthāne}
		\rdg[wit={N23}]{nītvā tāv iṃdavasthāne}% kṛtvā tāv iṃdavasthāne K1
		\rdg[wit={J7}]{nītvā tāvad avasthāne}% kṛtvā tāvad vindusthairyaṃ B2
		\rdg[wit={Gr3a}]{nītvā tām anavasthāne}
		\rdg[wit={G4}]{[dh]ṛ\,..\,[sāv a]ṃdra\,..\,[sthā]ne}% dhṛtvā taṃ caindavasthāne F
		\rdg[wit={C6}]{hṛtvā mamedaṃ ca sthānaṃ}
		\rdg[wit={G11}]{sthitvā sa vaindave sthāne}
		\rdg[wit={N19}]{sthitvā sāṃcaiṃdave sthāne}
		\rdg[wit={J10}]{sthitvā sadaiṃdave sthāne}
		\rdg[wit={Jyo}]{sthitvā sadaiva susthāne}
		\rdg[wit={V15}]{samāvasthā sthito yogī}}}
\pada{\app{\lem[wit={Gr1,GrB,G11,N19}]{ghrāṇa}
		\rdg[wit={Gr2,Gr3a,V15,J10}]{prāṇa}% +N24
		\rdg[wit={Jyo}]{brahma}}%
	\app{\lem[wit={Gr1,C6,V3,J7,J10,Jyo}]{randhre}
		\rdg[wit={N23,Gr3a,G11,N19,V15}]{randhraṃ}
		\rdg[wit={P11}]{randhra}}
	\app{\lem[wit={N3,G4,GrB,G11,N19,V15,J10,Jyo}]{nirodhayet}% +C7
		\rdg[wit={Gr2,Gr3a}]{nirundhayet}
		\rdg[wit={J5}]{niyojayet}% +F
		}//}\label{jnatva}\\!}
\end{tlg}
% V15 totally different: suṣumnāṃtagataṃ mārgaṃ vāyuṃ kṛtvā ca madhyagaṃ/ samāvasthā sthito yogī prāṇaraṃdhraṃ nirodhayet//
% K1 kṛtvā tāv iṃdavasthāne prāṇa°
% B2 kṛtvā tāvad bindusthairyaṃ prāṇa°
% J11 sthitvā sā caiṃdavasthāne ghrāṇa°


% between V3_4.175-176
\begin{ava}[hp04_059a]
\app{\lem[wit={N3,G4,C6}]{tathā ca vasiṣṭhaḥ} % <<ḥ>> N3, vasiṣṭāḥ P7, ca vasiṣṭaḥ C6
		\rdg[wit={J5}]{tathā vaśiṣṭhavacanaṃ}
		\rdg[wit={V3}]{tatvāva || ☼ ||}/} \sgwit{Gr1,C6,V3} % om. P11
\end{ava}

\begin{tlg}[hp04_059]% = V3_4.176
\tl{
\pada{iḍāyāṃ % idā° N23
	\app{\lem[wit={N3,J5,P11,C6,Gr2,Gr3a}]{piṅgalāyāṃ ca}
		\rdg[wit={V3}]{piṅgalāyāṃśca}}}
\pada{carataś candrabhāskarau/}\\+}
\tl{
\pada{candras tāmasa ity uktaḥ}
\pada{sūryo
	\app{\lem[wit={N3,J5,GrB,J7,Gr3a}]{rājasa}
		\rdg[wit={N23},post=\texteng{(end of the last available folio)}]{rā}} ucyate//}\myfn{\getsiglum{N23} is lost after \textit{sūryo rā} in pāda d.}
		\NotIn{G11,N19,V15,J10,Jyo}\\!}
\end{tlg}
		%\sgwit{N3,J5,GrB,Gr2,Gr3a}

\newpage
\begin{tlg}[hp04_060]% = V3_4.177
\tl{
\pada{\app{\lem[wit={N3,J5,GrB,J7,Gr3a},alt={tāv eva ... sakalaṃ}]{%
	\app{\lem[wit={N3,J5,P11,C6,J7,E2},alt={tāv eva}]{tāv eva}% tau eva C6
		\rdg[wit={V19},alt={tā eva}]{tā eva}
		\rdg[wit={V3},alt={tāṃve}]{tāṃve}} 
	\app{\lem[wit={N3,P11,V3,Gr3a},alt={dhattaḥ}]{dhattaḥ}% dhataḥ N3, dharttaḥ P11
		\rdg[wit={J7},alt={dattaḥ}]{dattaḥ}% +K3,C7
		\rdg[wit={J5},alt={dhanva}]{dhanva}
		\rdg[wit={C6},alt={vahataḥ}]{vahataḥ}}
	\app{\lem[wit={N3,J5,P11,V3,J7,Gr3a},alt={sakalaṃ}]{sakalaṃ}% ṃ om. V3
%		\rdg[wit={J5},alt={śakalaṃ}]{śakalaṃ}
		\rdg[wit={C6},alt={sarvaṃ}]{sarvaṃ}
		}} % nested!!
	\rdg[wit={G11}]{sūryaś candraḥ sadā dhatte}% om. G5; sūryaṃ candraṃ M3
	\rdg[wit={N19}]{sūryacandrau sadā dhatte}% dhattaḥ J11
	\rdg[wit={V15,Jyo}]{sūryācandramasau dhattaḥ}% +M1; sūryacandramasā yuktaṃ F
	\rdg[wit={J10}]{sūryācandramasau kṛtvā}}}
\pada{\app{\lem[wit={P11,J7,Gr3a,G11,V15,Jyo}]{kālaṃ}
		\rdg[wit={N3,J5,C6}]{kāla}
		\rdg[wit={N19}]{kālāṃ} % kālāṃśatri N19
		\rdg[wit={V3,J10},alt={\om}]{\skp{\om}}}
	\app{\lem[wit={G11,Jyo}]{rātriṃdivātmakam}
		\rdg[wit={N3,J5,P11,C6,J7,V15}]{rātridivātmakam}% °ka J5; +V6,F
		\rdg[wit={Gr3a}]{rātrindinātmakaṃ}
		\rdg[wit={G4}]{rātriṃ divākaraṃ}
		\rdg[wit={V3}]{rātridivātmakaṃ yogavit}
		\rdg[wit={N19}]{°śa tridivātmakaṃ}
		\rdg[wit={J10},alt={\om}]{\skp{\om}}}/}\\+}
\tl{
\pada{\app{\lem[wit={N3,P11,J7,Gr3a,G11,V15,Jyo}]{bhoktrī} % bho<<ktrī>> N3
		\rdg[wit={N19}]{bhoktī}
		\rdg[wit={V3}]{bhoktā}
		\rdg[wit={C6}]{bhoktṛ}
		\rdg[wit={J5}]{bhoktu}
		\rdg[wit={G4}]{[bho]gī}
		\rdg[wit={J10},alt={\om}]{\skp{\om}}}
	suṣumṇā kālasya}
\pada{\app{\lem[wit={Gr1,GrB,J7,G11,N19,V15,Jyo},alt={guhyam etad}]{guhyam eta\skp{d}}
		\rdg[wit={V19}]{guptam etad}
		\rdg[wit={E2}]{sattvam etad}% +C7
		\rdg[wit={J10},alt={\om}]{\skp{\om}}}%
	\app{\lem[wit={ceteri},alt={udāhṛtam}]{\skm{d }udāhṛtam}
	\rdg[wit={J5}]{udīritaṃ}}//}
		\lineom{bcd}{J10}
	\unavbl{N23}\\!}
\end{tlg}

%\newpage
\begin{ava}[hp04_061a]
\app{\lem[wit={Gr1,C6,V3,V19,G5}]{tathā hi}% +K3,C7
	\rdg[wit={P11}]{tathāpi hi}
	\rdg[wit={J7,E2}]{tathā}
	\rdg[wit={G11}]{athā hi}}
\app{\lem[wit={N3,J5,Gr3a}]{saubhadraṃ nāma}% rāma J5
	\rdg[wit={G11}]{sobhadrā nāma}
	\rdg[wit={J7}]{saubhadranāmā}
	\rdg[wit={G5}]{saubhadranāmaś ca}
	\rdg[wit={V3}]{saubhadreyaṃ nāma}
	\rdg[wit={C6}]{saubhadreyanāma}
	\rdg[wit={P11}]{saubhadreryān nāma}}
\app{\lem[wit={N3,GrB,Gr3a,G11,G5}]{ślokacatuṣṭayam}% śloṣa P11
	\rdg[wit={J5}]{ślokam eva catuṣṭayaṃ}
	\rdg[wit={J7}]{ślokacatuṣṭayam āha}}/%
\myfn{\getsiglum{G11,G5} have this set of verses as \manuref{3.94*7}ff. in a different order.}
%	\sgwit{N3,J5,GrB,J7,Gr3a}
	\NotIn{N19,V15,J10,Jyo}
\end{ava}

\begin{tlg}[hp04_061]% = V3_4.178/9
\tl{
\pada{\app{\lem[wit={J5,GrB,J7,Gr3a,G11,G5}]{ṣaṭcakraṃ}
	\rdg[wit={N3}]{ṣaḍraktaṃ}} ṣoḍaśādhāraṃ}
\pada{\app{\lem[wit={V3,J7,Gr3a,G11,G5},alt={tridhā lakṣ(y)aṃ}]{tridhā lakṣyaṃ}% lakṣaṃ V3,J7,G11
		\rdg[wit={N3,J5}]{tridhā bhajyaṃ}% m. om. J5
%		\rdg[wit={C7}]{tridhā yuktaṃ}
		\rdg[wit={P11}]{tridhākṣa ca}
		\rdg[wit={C6}]{trilakṣyaṃ ca}} guṇatrayam/}\\+}
\tl{
\pada{\app{\lem[wit={N3,J5,GrB,G5}]{śeṣaṃ tu} % m vs n!
		\rdg[wit={J7,Gr3a}]{śeṣas tu}% +F
		\rdg[wit={G11}]{śeṣaṃ tat}}
	\app{\lem[wit={ceteri}]{grantha}% ceteri = N3,J5,GrB,J7,V19,K3; gratha N3
		\rdg[wit={C6,G11}]{granthi}}% +C7
	\app{\lem[wit={N3,GrB,G11,G5}]{vistāraṃ}
		\rdg[wit={J5}]{vistāra}
		\rdg[wit={J7,Gr3a}]{vistāras}% +F
		}}
\pada{\app{\lem[wit={N3,J5,P11,V3,J7,V19,G11,G5}]{trikūṭaṃ}% °kuṭaṃ G11
		\rdg[wit={C6}]{trikoṭi}
		\rdg[wit={E2}]{trirūpaṃ}% +K3,C7
		} paramaṃ padam//}
%	\sgwit{N3,J5,GrB,J7,Gr3a}
	\NotIn{N19,V15,J10,Jyo}% \anm{as \manuref{3.94*??} \getsiglum{G11}}
	\unavbl{N23}\\!}
\end{tlg}


\begin{tlg}[hp04_062]% = V3_4.179/180
\tl{
\pada{kuṇḍalī kuṭilākārā} % kuṭulākāraṃ G4, kuḍalākārā J5
\pada{sarpavat parikīrtitā/}\\+} % sarvasthat G4, sarvavat G11
\tl{
\pada{sā śakti\app{\lem[wit={N3,J5,V3,G11,G5},alt={cālitā}]{\skm{ś }cālitā}
	\rdg[wit={P11}]{calitā}
	\rdg[wit={G4}]{cāri\,..}
	\rdg[wit={V19}]{kīlitā}% +K3
	\rdg[wit={E2}]{kelitā}% +C7
	} yena}
\pada{sa \app{\lem[wit={Gr3a,G11,G5}]{mukto}% +G7 (eye-skip to the next verse?)
	\rdg[wit={N3,J5,P11,V3}]{yogī}} % +F ##
	nātra saṃśayaḥ//} % śaṃsayaḥ V19
	\NotIn{C6,J7,N19,V15,J10,Jyo}
%	\sgwit{Gr1,P11,V3,Gr3a} 
%	\anm{as \manuref{3.94*7} \getsiglum{G11}}
	\unavbl{N23}\\!} % +B2
\end{tlg}
% V3 has a gap indicated before the following verse.

%\newpage
\begin{tlg}[hp04_063]% = V3_4.181
\tl{
\pada{\app{\lem[wit={ceteri}]{yadā}\rdg[wit={G5}]{yathā}} 
	\app{\lem[wit={ceteri},alt={kūṭaṃ tri°}]{kūṭaṃ tri\skp{°}}
		\rdg[wit={C6}]{kūṭasti}}kūṭasthaṃ} % 
\pada{cittaṃ  % citta V3
	\app{\lem[wit={N3}]{citraṃ}
		\rdg[wit={J5}]{cittaṃ}
		\rdg[wit={GrB}]{tatra}% +F
		\rdg[wit={G11,G5}]{yatra}
		} 
	\app{\lem[wit={ceteri}]{nirantaram}
		\rdg[wit={G11,G5}]{nirajñanaṃ}}/}\\+}
\tl{
\pada{\app{\lem[wit={ceteri}]{kuṇḍalyās tu} % kuṃḍilyās P11
		\rdg[wit={G11}]{kuṇḍalyāpta}
		\rdg[wit={G5}]{kuṇḍalinyāḥ}}
	\app{\lem[wit={N3,J5,P11,V3,G11,G5},post=\texteng{(°na˟ \getsiglum{N3})}]{prayogeṇa}%  + + geṇa G4, °na G11
		\rdg[wit={C6}]{prabodhena}}}
\pada{sa mukto nātra saṃśayaḥ//} 
	\NotIn{J7,Gr3a,N19,V15,J10,Jyo}
%	\anm{as \manuref{3.94*??} \getsiglum{G11}}
	\unavbl{N23}\\!}
%	\sgwit{Gr1,GrB}\\!} % not in B2
\end{tlg}


%\newpage
\begin{tlg}[hp04_064]% = V3_4.182
\tl{
\pada{\app{\lem[wit={N3,J5,GrB,J7,Gr3a,Jyo}]{dvāsaptatisahasrāṇi}
		\rdg[wit={G4,G11,G5,N19,V15},alt={dvisaptati°}]{dvisaptatisahasrāṇi}% +F
		\rdg[wit={J10},alt={\om}]{\skp{\om}}}} % °śrāṇi V3,V19
\pada{\app{\lem[wit={Gr1,GrB,J7,G11,G5,V15,Jyo},post=\texteng{(nāḍi° \getsiglum{J5,P11})}]{nāḍīdvārāṇi}% nāḍi J5,P11
		\rdg[wit={N19}]{nāḍīdvāre ca}
		\rdg[wit={E2}]{nāḍīnāṃ deha}% +K3,C7
		\rdg[wit={V19}]{nāḍīnāṃdeda}
		\rdg[wit={J10}]{datvā kārāpi}}\marmas
	\app{\lem[wit={ceteri}]{pañjare}
		\rdg[wit={N3}]{paṃkaje}
		\rdg[wit={G4}]{maṃjarī}}/}\\+}
\tl{
\pada{suṣumṇā śāṃbhavī śaktiḥ} % sāṃbhavī J10; śakti V19,V3
\pada{\app{\lem[wit={N3,GrB,E2,G11,G5,N19,Jyo}]{śeṣās tv eva}% aiva P11; +C7
		\rdg[wit={J10}]{śeṣās tv evaṃ}
		\rdg[wit={J5}]{śeṣāsvevaṃ}
		\rdg[wit={J7,V19,V15}]{śeṣāś caiva}}
	\app{\lem[wit={ceteri}]{nirarthakāḥ} % nina° J10; °kā J5,V3
		\rdg[wit={N19}]{nivarttakāḥ}
		}//}%
	\myfn{\getsiglum{G11} has this verse in both Ch. 3 and 4.}
%	\anm{as \manuref{3.94*??} \getsiglum{G11}}
	\unavbl{N23}\\!}
\end{tlg}


\newpage
\begin{tlg}[hp04_065]% = V3_4.183
\tl{
\pada{vāyuḥ % vāyu P11,V19
	\app{\lem[wit={N3,J5,C6,G11,N19,V15,J10,Jyo}]{paricito}
		\rdg[wit={V3}]{paricipta}
		\rdg[wit={J7}]{sa parito}
		\rdg[wit={V19}]{saṃparito}% +C7
		\rdg[wit={P11}]{parivṛtto}}
	\app{\lem[wit={N3,P11,C6,J7,V19,G11,N19,V15},alt={yatnād}]{yatnā\skp{d}}% yannād P11
%		\rdg[wit={C7}]{yadvad}
		\rdg[wit={J5,J10,Jyo}]{yasmād}% +F
		\rdg[wit={V3}]{nādād}}}%
\pada{\app{\lem[wit={GrB,V19,G11,N19,V15,J10,Jyo},alt={agninā}]{\skm{d }agninā} % agnidā N19
		\rdg[wit={J7}]{ṛgvinā}
		\rdg[wit={N3}]{yaṣṭinā}
		\rdg[wit={J5}]{yadasthā}} saha
	\app{\lem[wit={G11,Jyo}]{kuṇḍalīm}% +C7
		\rdg[wit={N3,J5,GrB,J7,V19,N19,V15,J10}]{kuṇḍalī}}/}\\+}
\tl{
\pada{bodhayitvā suṣumṇāyāṃ} % °yā N3
\pada{\app{\lem[wit={ceteri},alt={praviśed}]{praviśe\skp{d}}% °viśad J5
		\rdg[wit={V3}]{praveśad}
		\rdg[wit={J10},alt={\om}]{\skp{\om}}}%
	\app{\lem[wit={G4,GrB,G11,V15,Jyo},alt={anirodhataḥ}]{\skm{d }anirodhataḥ} % °ta P11,V3; +M1
		\rdg[wit={N3,J5,J7,V19}]{avirodhataḥ} % aviśeṣataḥ V19ac; +C7
		\rdg[wit={N19}]{atirodhataḥ}
		\rdg[wit={J10},alt={\om}]{\skp{\om}}}//}
	\label{bodhayitva} \lineom{cd}{J10}
	\unavbl{N23} \NotIn{E2}\\!}
\end{tlg}

%\newpage
\begin{tlg}[hp04_066]% = V3_4.184
\tl{
\pada{suṣumṇā\app{\lem[wit={G4,C6,V3,J7,E2,G11,Jyo}]{vāhini}
		\rdg[wit={N3,J5,P11,N19,V15}]{vāhinī}
		\rdg[wit={V19}]{hini}
		\rdg[wit={J10},alt={\om}]{\skp{\om}}} 
		prāṇe} % prāṇo J5
\pada{\app{\lem[wit={G4,C6,V3,J7,Gr3a,G11,V15,Jyo}]{sidhyaty eva}% °teva G11
		\rdg[wit={N3}]{siddhyety eva}
%		\rdg[wit={C7}]{siddhyatīva}
		\rdg[wit={P11,N19}]{siddhaty eva}
		\rdg[wit={J5}]{siddhity eva}
		\rdg[wit={J10},alt={\om}]{\skp{\om}}} 
	manonmanī/} \lineom{ab}{J10}\\+}
\tl{
\pada{\app{\lem[wit={Gr1,GrB,J7}]{anyathā vividhā}% anātha J5
%		\rdg[wit={C7}]{anye ca vividhā}
		\rdg[wit={Gr3a}]{anye ye vividhā}
		\rdg[wit={N19,V15}]{anyathā tv itare}% +G5
		\rdg[wit={Jyo}]{anyathā tv itarā}
		\rdg[wit={J10}]{atha cittāntare}
		\rdg[wit={G11}]{prāṇe suṣumnāṃ saṃ°}}%
	\app{\lem[wit={N3,C6,E2,Jyo}]{bhyāsāḥ}% bhyāsā<<ḥ>> C7
		\rdg[wit={G4,V3,J7,V19}]{bhyāsā}
		\rdg[wit={J5,P11,N19}]{bhyāsāt}% +G5
		\rdg[wit={V15,J10}]{bhyāsa}
		\rdg[wit={G11}]{°prāpte}}}
\pada{\app{\lem[wit={N3,J5,GrB,J7,G11,Jyo}]{prayāsāyaiva}% +C7
		\rdg[wit={E2}]{prayāsāyai}
		\rdg[wit={V19}]{prāyāsāś caiva}
		\rdg[wit={V15}]{prayāsā eva}
		\rdg[wit={N19}]{prayāsā eka}
		\rdg[wit={J10}]{pratyāśā jīva}}
	\app{\lem[wit={ceteri}]{yoginām}
		\rdg[wit={J5,V3,J10}]{yoginā}
		\rdg[wit={N19}]{yoginī}}//}
	\unavbl{N23}\\!}
\end{tlg}

%\newpage
\begin{tlg}[hp04_067]% = V3_4.185
\tl{
\pada{pavano badhyate 
	\app{\lem[wit={ceteri}]{yena}\rdg[wit={J5}]{deva}}}
\pada{\app{\lem[wit={ceteri}]{manas tenaiva badhyate} % <va>dhyate N19
		\rdg[wit={J10}]{tenaiva badhyate manaḥ}}/}\\+}
\tl{
\pada{\app{\lem[wit={N3,P11,V3,G11,N19,V15,Jyo}]{manaś ca}
		\rdg[wit={Gr3a}]{manas tu}% +F
		\rdg[wit={C6}]{manas tad}
		\rdg[wit={J5,J7,J10},alt={\om}]{\skp{\om}}} 
		badhyate yena}
\pada{\app{\lem[wit={ceteri}]{pavanas tena} % <pa>vanas V19
		\rdg[wit={V3}]{pavanamana}
		\rdg[wit={J5,J7,J10},alt={\om}]{\skp{\om}}} 
		badhyate//}
		\lineom{cd}{J5,J7,J10}
	\unavbl{N23}\\!}
\end{tlg}

\begin{tlg}[hp04_068]% = V3_4.186
\tl{
\pada{\app{\lem[wit={ceteri}]{hetu}
%		\rdg[wit={C7}]{deha}
		\rdg[wit={J5}]{heta}
		\rdg[wit={G4}]{eta}}%
	\app{\lem[wit={N3,G4,E2,J10,Jyo}]{dvayaṃ tu}% +C7
		\rdg[wit={P11,V3,J7,G11}]{dvayaṃ hi}
		\rdg[wit={C6,V19}]{dvayaṃ ca}
		\rdg[wit={N19,V15}]{dvayasya}
		\rdg[wit={J5}]{dvāv api}}
	\app{\lem[wit={ceteri}]{cittasya}
		\rdg[wit={J7,Gr3a}]{manaso}}}
\pada{vāsanā ca samīraṇaḥ/}\\+} % ḥ om. V3; vāsanācca samīraṇa<<ḥ>> N3, vāsanaṃ ca samīraṇaṃ J5
\tl{
\pada{tayo\app{\lem[wit={ceteri},alt={vinaṣṭa ekasmin}]{\skm{r }vinaṣṭa ekasmi}% tayo J5,J7; ekasminn J7; ekasmi V19
		\rdg[wit={G11}]{vinaṣṭa etasmin}
		\rdg[wit={C6}]{vinaṣṭas tv ekaś ca hy}}}%n 
\pada{\app{\lem[wit={N3},alt={drutaṃ dvāv api naśyataḥ},post=\texteng{(druttaṃ)}]{\skm{n }drutaṃ dvāv api naśyataḥ}% +M3
%		\rdg[wit={N3}]{druttaṃ dvāv api naśyataḥ}
		\rdg[wit={G4}]{dhṛtaṃ dvāv api naśyataḥ}
		\rdg[wit={J5}]{dṛtaṃ vāvati nasyataḥ}
		\rdg[wit={G11}]{nṛtaṃ dvāv api naśyati}
		\rdg[wit={P11,V3,N19,V15,Jyo}]{tau dvāv api vinaśyataḥ} % °ta P11,V3, °tiḥ N19
		\rdg[wit={C6,J7,E2,J10}]{ubhāv api vinaśyataḥ}
		\rdg[wit={V19}]{svabhāvo pi vinaśyataḥ}}//}%
	\myfn{\getsiglum{V19} has this verse and the next one after \ref{dugdha}.}
	\unavbl{N23}\\!}
\end{tlg}

%\newpage
\begin{tlg}[hp04_069]% = V3_4.187
\tl{
\pada{mano yatra \app{\lem[wit={ceteri}]{vilīyeta} % °līyena N19
		\rdg[wit={V3}]{vilīyate}
		\rdg[wit={J10},alt={\om}]{\skp{\om}}}}
\pada{\app{\lem[wit={ceteri},alt={pavanas}]{pavana\skp{s}}
		\rdg[wit={G11,N19,V15}]{mārutas}
		\rdg[wit={J10},alt={\om}]{\skp{\om}}}s 
		tatra līyate/}
		\lineom{ab}{J10}\\+}
\tl{
\pada{\app{\lem[wit={N3,C6,J7,Jyo}]{pavano līyate yatra}
		\rdg[wit={Gr3a}]{pavano yatra līyeta}
		\rdg[wit={P11,V3}]{pavano yatra līyate}% līyaṃte P11
		\rdg[wit={G11}]{māruto yatra līyeta}
		\rdg[wit={J10}]{yatraiva līyate vāyur}
		\rdg[wit={J5,N19,V15},alt={\om}]{\skp{\om}}}}
\pada{mana\app{\lem[wit={N3,GrB,Gr3a,G11,J10},alt={tatraiva līyate}]{\skm{s }tatraiva līyate}% +N2
		\rdg[wit={J7,Jyo}]{tatra vilīyate}% +V6
		\rdg[wit={J5,N19,V15},alt={\om}]{\skp{\om}}}//}\myfn{%
		\getsiglum{V15} has an incomplete passage \textit{ekatra[m]iśritau} after this verse.}
	\lineom{cd}{J5,N19,V15}
	\unavbl{N23}\\!}
\end{tlg}


\newpage
\begin{tlg}[hp04_070]% = V3_4.188
\tl{
\pada{dugdhāmbuvat saṃmilitau % dugdhābu° N3; saṃmilaṃ J5, <<li>> C6, °tāv J10,Jyo, °to N19
	\app{\lem[wit={N3,J5,GrB,G11,N19,V15}]{sadaiva}
		\rdg[wit={G4}]{sadeva}
		\rdg[wit={J7,Gr3a}]{tathaiva}
		\rdg[wit={J10,Jyo}]{ubhau tau}}}\\+}
\tl{
\pada{tulyakriyau % kriyo N19, tulyasya kriyā J5, tvalya C6
	\app{\lem[wit={ceteri}]{mānasamārutau} % māruto N19; 2 x mānasa J10
		\rdg[wit={P11,C6,G11}]{mārutamānasau}
		\rdg[wit={V3},alt={\illeg}]{\skp{\illeg}}}
	\app{\lem[wit={N3,G4,P11,G11,N19,V15,J10,Jyo}]{hi}
		\rdg[wit={J5,C6,J7,Gr3a}]{ca}
		\rdg[wit={V3},alt={\illeg}]{\skp{\illeg}}
		}/}\\+}
\tl{
\pada{\app{\lem[wit={ceteri},alt={yāvan manas}]{yāvan mana\skp{s}}
		\rdg[wit={J10,Jyo}]{yato marut}}%
	\app{\lem[wit={ceteri},alt={tatra}]{\skm{s }tatra}
		\rdg[wit={J5}]{caiva}}
	\app{\lem[wit={ceteri},alt={marut}]{maru\skp{t}} % murut G11, marat V15
		\rdg[wit={J10,Jyo}]{manaḥ}
		\rdg[wit={C6}]{\_\,sat}}%
	\app{\lem[wit={ceteri},alt={pravṛttir}]{\skm{t}pravṛtti\skp{r}} % vṛtti J7
		\rdg[wit={C6}]{pravṛtta}% varti P7
		\rdg[wit={N19}]{pravṛddhitti}}-}\\+}
\tl{
\pada{\app{\lem[wit={Gr1,GrB,J7,Gr3a,G11},alt={yāvan}]{\skm{r }yāva\skp{n}}
		\rdg[wit={J10,Jyo}]{yato}
		\rdg[wit={N19,V15},alt={\om},post=\texteng{(pāda d om.)}]{}}%
	\app{\lem[wit={N3,J5,P11,C6,J7,Gr3a,G11},alt={maruc cāpi}]{\skm{n }maruc cāpi}
		\rdg[wit={V3}]{marut tatra}
		\rdg[wit={J10,Jyo}]{manas tatra}
		\rdg[wit={N19,V15},alt={\om}]{\skp{\om}}}
	\app{\lem[wit={N3pc,C6,V3,J7,E2,G11}]{manaḥ}
		\rdg[wit={N3ac,J5,P11,V19}]{mana}
		\rdg[wit={J10,Jyo}]{marut}% or: marun
		\rdg[wit={N19,V15},alt={\om}]{\skp{\om}}}%
	\app{\lem[wit={N3,P11,V3,J7,Gr3a,G11,Jyo}]{pravṛttiḥ} % °vartiḥ P7
		\rdg[wit={C6}]{pravṛttaḥ}
		\rdg[wit={J5}]{pravittato}
		\rdg[wit={J10}]{nivṛttiḥ}
		\rdg[wit={N19,V15},alt={\om}]{\skp{\om}}}//}\label{dugdha}
	\unavbl{N23}\\!} % \lineom{d}{N19,V15}
\end{tlg}

%\newpage
\begin{tlg}[hp04_071]% = V3_4.189
\tl{
\pada{\app{\lem[wit={ceteri}]{tatraika}% tatr<<aika>> N3
		\rdg[wit={N3ac}]{tatra}
		\rdg[wit={N19,V15}]{atraika}
		\rdg[wit={J10}]{ekasya}}nāśād aparasya % nā<śā>d J5,N19, nāsād J10
	\app{\lem[wit={N3,J5,C6,J7,E2,N19,V15,J10,Jyo},alt={nāśa(ḥ)}]{nāśa\skp{(ḥ)}}
		\rdg[wit={V3}]{nāśo}
		\rdg[wit={P11}]{nāśe}
		\rdg[wit={G11}]{nāśā}
		\rdg[wit={V19}]{nāśam}
%		\rdg[wit={J5,N19,V15}]{nāśaḥ}
%		\rdg[wit={J10}]{nāśas}
		}}\\+}
\tl{
\pada{\app{\lem[wit={N3,J5,P11,J7,N19,Jyo},alt={ekapravṛtter}]{ekapravṛtte\skp{r}}
		\rdg[wit={C6}]{ekapravṛtte}
		\rdg[wit={Gr3a,G11,V15}]{ekapravṛttāv}
		\rdg[wit={V3}]{e\,..\,..\,..\,..}
		\rdg[wit={J10}]{tatraikavṛtter}}%
	\app{\lem[wit={ceteri},alt={aparapravṛttiḥ}]{\skm{r }aparapravṛttiḥ} % °ttir N19, °tt<<i>>ḥ N3; +G5
		\rdg[wit={C6}]{ca parapravṛttiḥ}
		\rdg[wit={J10}]{aparasya vṛttiḥ}
		\rdg[wit={G11}]{itarapravṛttiḥ}
		\rdg[wit={V3}]{..\,..\,..\,..\,..\,ttiḥ}}/}%
	\myfn{In \getsiglum{V19} Pādas ab and cd are transposed;
	\getsiglum{V15} inserts here a variant reading for Pāda a ``\textit{ekasya nā<śā>d aparasya nāśaḥ}" here.}\\+}
\tl{
\pada{\app{\lem[wit={N3,P11,Jyo},alt={adhvastayoś}]{adhvastayo\skp{ś}}
		\rdg[wit={C6,J10}]{adhastayoś}
		\rdg[wit={E2,V15}]{adhvastayor}
		\rdg[wit={J7}]{adhyastayor}
		\rdg[wit={V19}]{adhastayor}
		\rdg[wit={G11}]{adhvaścayoś}
		\rdg[wit={N19}]{addhastayoś}
		\rdg[wit={V3}]{atastayoś}
		\rdg[wit={J5}]{adhastasar}
		}%
	\app{\lem[wit={Gr1,GrB,G11,N19,J10,Jyo},alt={cendriya}]{\skm{ś }cendriya}% veddriya J5, caiṃdriya P11, cenniya G11, caidriya N19
		\rdg[wit={J7,Gr3a,V15}]{indriya}}varga% 
	\app{\lem[wit={N3,G4},alt={buddhir}]{buddhi\skp{r}}
		\rdg[wit={V3}]{vudhir}
		\rdg[wit={J7,E2}]{vṛddhir}% +C7
		\rdg[wit={V19,G11,N19,V15,J10,Jyo}]{vṛttiḥ}% vṛttir V19, vṛtti V15
		\rdg[wit={P11}]{baṃdhir}
		\rdg[wit={J5,C6}]{śuddhir}
		}\marma-}\\+}
\tl{
\pada{\app{\lem[wit={N3,G4,GrB,Gr3a,V15},alt={vidhvastayor}]{\skm{r }vidhvastayo\skp{r}}
		\rdg[wit={J5}]{adhastayor}
		\rdg[wit={J7}]{vivṛddhayor}% °yo J7
		\rdg[wit={G11}]{nidhvastayo}
		\rdg[wit={N19}]{addhvastayor}% +G5
		\rdg[wit={J10}]{vijñātayor}
		% vṛttiḥ|(gap for about one hemistich)raddhvasta° N19
		\rdg[wit={Jyo}]{pradhvastayor}}%
	\app{\lem[wit={ceteri},alt={mokṣapadasya}]{\skm{r }mokṣapadasya}
		\rdg[wit={J7},alt={°pradasya}]{mokṣapradasya}
%		\rdg[wit={C7},alt={°pathasya}]{mokṣapathasya}
		}
		siddhiḥ//}\label{tatraika}
	\unavbl{N23}\\!} % siddhi J5,P11,V19
\end{tlg}

%\newpage
%====== Passage B and yāvanna-stanza collated with more mss (V19,N19,J10)

\begin{tlg}[hp04_072]% = V3_4.190
%\Anm{This verse appears after 4.0*16 in \getsiglum{N19,V15,J10}}\\+}
\tl{
\pada{\app{\lem[wit={ceteri}]{vāyu}
		\rdg[wit={V19,V15}]{vāyur}}%
	\app{\lem[wit={G11}]{mārge tv asaṃcāre}
		\rdg[wit={V15}]{mārge py asaṃcāre}
		\rdg[wit={N19}]{mārge tha saṃcāre}
		\rdg[wit={J10}]{mārge ca saṃcāre}
		\rdg[wit={Gr1,GrB,J7}]{mārgeṇa saṃcāre}
		\rdg[wit={Gr3a}]{mārgeṇa saṃcārī}
		}}
\pada{\app{\lem[wit={N3,V3,J7,Gr3a}]{sakalāṃ}
		\rdg[wit={G4}]{sakalā}
		\rdg[wit={J5,C6,G11,N19,V15}]{sakalaṃ}
		\rdg[wit={J10}]{sa phalaṃ}
		\rdg[wit={P11}]{saṃkalpāt}}
	\app{\lem[wit={Gr1,P11,G11,V15,J10}]{labhate}
		\rdg[wit={C6,N19}]{labhyate}
		\rdg[wit={J7,Gr3a}]{bhramate}
		\rdg[wit={V3}]{carate}}\marmas
	\app{\lem[wit={N3,G4,P11,J7,Gr3a,G11}]{mahīm}
		\rdg[wit={C6,V3}]{mahī}
		\rdg[wit={J5}]{mahiḥ}
		\rdg[wit={N19,V15}]{mahaḥ}
		\rdg[wit={J10}]{mahān}}/}\\+}
\tl{
\pada{\app{\lem[wit={Gr1,Gr3a,G11},post=\texteng{(tathā<<ṣṭa>> \getsiglum{N3})}]{tathāṣṭa}
		\rdg[wit={P11}]{aṣṭadhā}
		\rdg[wit={C6,V3}]{athāṣṭa}
		\rdg[wit={N19,V15,J10}]{tato'ṣṭa}
		\rdg[wit={J7}]{na tathā}
		}guṇam aiśvaryaṃ} % ṃ om. J7
\pada{\app{\lem[wit={N3,G4,GrB,J7,Gr3a}]{satyaṃ satyaṃ varānane}
		\rdg[wit={G11,N19,V15,J10}]{satyam ity āha śaṃkaraḥ}
		\rdg[wit={J5}]{labhate sakalān varān}}//}
	\NotIn{Jyo}\label{vayumargena}
	\anm{after \manuref{4.0*16} \getsiglum{G11,N19,V15,J10}}
	\unavbl{N23}\\!}
\end{tlg}

\newpage
\begin{ava}[hp04_073a]
\app{\lem[wit={N3,P11,C6}]{tathā}
	\rdg[wit={J5}]{tathā ca}
	\rdg[wit={G4}]{tathāha}
	\rdg[wit={J7,Gr3a},alt={\om}]{\skp{\om}}} 
	viśvarūpācāryaḥ/ % °yāḥ J7
	\NotIn{V3,G11,N19,V15,J10,Jyo}
%	\sgwit{Gr1,P11,C6,J7,Gr3a}
\end{ava}

\begin{tlg}[hp04_073]%
\tl{
\pada{\app{\lem[wit={Gr1,C6,Gr3a,Jyo}]{yadā saṃkṣīyate}
		\rdg[wit={P11,J7}]{yadā sa kṣīyate}
		\rdg[wit={G11},alt={\om}]{\skp{\om}}} 
		prāṇo} % =P7; prāṇaṃ J5, prāṇe C6
\pada{mānasaṃ \app{\lem[wit={Gr1,P11,C6,G11}]{ca vilīyate}% vīlī° J5; +C7
		\rdg[wit={J7,Jyo}]{ca pralīyate}
		\rdg[wit={V19}]{pravilīyate}
		\rdg[wit={E2}]{saṃpralīyate}}/}\\+}
\tl{
\pada{\app{\lem[wit={ceteri}]{tadā}
		\rdg[wit={G11}]{tayoḥ}} % tadvā N3ac
	\app{\lem[wit={ceteri}]{samarasatvaṃ}
		\rdg[wit={J5},post={\unm}]{samarasaikatvaṃ}} % = VM6! J5 is hypermetrical, since it has yat too.
	\app{\lem[wit={N3,J5,C6,J7,E2,G11},alt={yat}]{ya\skp{t}}% +F
		\rdg[wit={G4,V19}]{yaḥ}
%		\rdg[wit={C7}]{hi}
		\rdg[wit={P11,Jyo}]{ca}}}%
\pada{\app{\lem[wit={N3,G4,C6,J7,V19,G11},alt={samādhiḥ so'bhidhīyate}]{\skm{t }samādhiḥ so'bhidhīyate} % samādhi G4,V19
		\rdg[wit={P11}]{samādhī sau bhidhīyate}
		\rdg[wit={E2}]{samādhiḥ sābhidhīyate}
		\rdg[wit={Jyo}]{samādhir abhidhīyate}
		\rdg[wit={J5}]{samādhiś ca vilīyate}}//}
		\label{visvarupa}\NotIn{V3,N19,V15,J10} 
		\anm{after \ref{salile} \getsiglum{Jyo}}
	\unavbl{N23}\\!}
\end{tlg}

%\newpage
\begin{tlg}[hp04_074]%
%\Anm{This verse appears before 4.13 in \getsiglum{G11,N19,V15,J10,Jyo}}
\tl{
\pada{\app{\lem[wit={N3pc,C6,J7,Gr3a,Jyo}]{manaḥ}
		\rdg[wit={N3ac,J5,G4,P11,G11,N19,V15,J10}]{mana}}%
\app{\lem[wit={N3,J5,P11,C6,J7,G11,N19,J10,Jyo}]{sthairye}
		\rdg[wit={G4,V19}]{sthairya}
		\rdg[wit={E2}]{sthairyāt}
		\rdg[wit={V15}]{sthairyaḥ}}
	\app{\lem[wit={ceteri}]{sthiro}
		\rdg[wit={G4,G11,V15}]{sthito}} vāyus} % vāyaḥ G11
\pada{tato \app{\lem[wit={N3pc,G4,J7,E2,V15,Jyo}]{binduḥ}
		\rdg[wit={N3ac,J5,P11,C6,V19,G11,N19,J10}]{bindu}}
	\app{\lem[wit={ceteri}]{sthiro bhavet}
%		\rdg[wit={C7}]{sthito bhavet}
		\rdg[wit={G4}]{tato layaḥ}}/}\\+}
\tl{
\pada{\app{\lem[wit={ceteri}]{bindu}
		\rdg[wit={J7}]{binduḥ}}%
	\app{\lem[wit={N3,P11,C6,E2},alt={sthairyodayāt}]{sthairyodayā\skp{t}}% +C7
		\rdg[wit={G11}]{sthairyoyadāt}
		\rdg[wit={G4,N19}]{sthairyodayā}
		\rdg[wit={V15}]{sthairye dayā}
		\rdg[wit={J10}]{sthairyād dayā}
		\rdg[wit={J7}]{sthairyād athā}
		\rdg[wit={V19}]{sthairyād yathā}
		\rdg[wit={Jyo}]{sthairyāt sadā}
		\rdg[wit={J5}]{sthairyo sthiro}}%
	\app{\lem[wit={N3,P11},alt={putra}]{\skm{t }putra}
		\rdg[wit={C6}]{mūtra}
		\rdg[wit={G4}]{tatra}
		\rdg[wit={J7}]{panna}% pannaṃ V6,YCM; panna J7,N2
		\rdg[wit={G11plus}]{samyak}% +G5,M3
		\rdg[wit={E2,N19,V15}]{satyaṃ}% +Ten,C7
		\rdg[wit={J10,Jyo}]{satvaṃ}
		\rdg[wit={J5}]{vāyu}
		\rdg[wit={V19},alt={\lacuna}]{\_\,\_\,\_}}}
\pada{piṇḍasthairyaṃ prajāyate//} % ddhaḍa J5; sthairya P11
	\NotIn{V3}\label{manahsthairye}
	\anm{after \manuref{4.0*16} \getsiglum{G11,N19,V15,J10,Jyo}}
	\unavbl{N23}\\!}
\end{tlg}

%\newpage
%%%%%%%%%%%%%%%%%%%%%%%
\begin{tlg}[hp04_075]% = V3_4.191 (N23,K3,J15 lost)
\tl{
\pada{dṛṣṭiḥ sthirā yasya % dṛṣṭi N3ac,J5,G4,P11,V19,V3,J10
	\app{\lem[wit={Gr1,GrB,G11,V15,J10}]{vinaiva}
%		\rdg[wit={C7}]{vinā ca}
		\rdg[wit={J7,Gr3a}]{vināpi}}
	\app{\lem[wit={N3,G4,GrB,V15},alt={dṛśyād}]{dṛśyā\skp{d}}% +G5
		\rdg[wit={J7,Gr3a,G11,J10}]{dṛśyaṃ}
		\rdg[wit={J5}]{dṛśyavān}}-}\\+}
\tl{
\pada{d vāyuḥ sthiro yasya % vāyu N3ac,J5,G4,P11,J7,V19,G11,V15,J10
	\app{\lem[wit={ceteri}]{vinā prayatnāt} % prayatnataḥ J10ac
		\rdg[wit={J7}]{vināpi yatnaṃ}}/}\\+}
\tl{
\pada{cittaṃ sthiraṃ yasya
	\app{\lem[wit={N3pc,G4,C6,V3,G11,V15}]{vināvalambāt}% °laṃbā C6
		\rdg[wit={N3ac}]{vināvalambanāt}
		\rdg[wit={J5,Gr3a}]{vināvalaṃbanaṃ}% +N2
		\rdg[wit={J10}]{vināvalaṃnaṃ}
		\rdg[wit={P11}]{vinā vilambāt}
%		\rdg[wit={C7}]{vinā balaṃ ca}
		\rdg[wit={J7}]{vinā prayatnāt}}}\\+}
\tl{
\pada{sa eva yogī % yeva J5,P11
	\app{\lem[wit={ceteri}]{sa guruḥ}
		\rdg[wit={J10}]{sadguruḥ}}
	\app{\lem[wit={ceteri}]{sa sevyaḥ} % sevya P11
		\rdg[wit={J7,V19}]{sa śiṣyaḥ}}//}
	\NotIn{N19,Jyo}\label{drsti}
	\unavbl{N23}\\!}
\end{tlg}

%%%%%%%%%%

\begin{tlg}[hp04_076]% = V3_4.192
\tl{
\pada{praveśe nirgame % praveśa G4; prathame niyame J5
	\app{\lem[wit={ceteri}]{vāme}
		\rdg[wit={G4}]{vāma}
		\rdg[wit={P11}]{vāpi}
		\rdg[wit={V15}]{cāpi}}}
\pada{dakṣiṇe \app{\lem[wit={Gr1,P11,G11}]{cordhvam apy adhaḥ}% adha G4
		\rdg[wit={C6}]{cordhvage'py adhaḥ}
%		\rdg[wit={C7}]{cordhvamadhyamaḥ}
		\rdg[wit={J7,Gr3a}]{cordhvamadhyagaḥ} % cordha V19
		\rdg[wit={V15,J10}]{cordhvamadhyataḥ}
		\rdg[wit={V3}]{tanirodhataḥ}}/}\\+}
\tl{
\pada{\app{\lem[wit={ceteri}]{na yasya}
		\rdg[wit={C6}]{layasya}}
	\app{\lem[wit={ceteri}]{vāyur vahati}% vahari G4
		\rdg[wit={V3}]{vahate vāyu}}}
\pada{sa mukto nātra saṃśayaḥ//} % yukto? V19
	\NotIn{N19,Jyo}\label{pravese}
	\anm{before \ref{nasuptam} \getsiglum{G11,V15,J10}}
	\unavbl{N23}\\!}
\end{tlg}


%\newpage
\begin{tlg}[hp04_077]% = V3_4.193
%\myfn{From here on, \getsiglum{V6} is used for collation as a replacement of \getsiglum{J7} which breaks at \textit{rājayo} in Pāda c.}%
\tl{
\pada{sarve \app{\lem[wit={ceteri}]{haṭhalayopāyā}% °pāya J5
		\rdg[wit={G11}]{layahaṭhopāyā}
		\rdg[wit={N19}]{haṭhalayoyāgā}
		\rdg[wit={V19}]{haṭhālayābhyāsā}
		\rdg[wit={J7,E2}]{layahaṭhābhyāsā}}} % +V6,°haṭha°J7
\pada{\app{\lem[wit={ceteri}]{rājayogasya siddhaye} % +J7
		\rdg[wit={G11,N19,V15,J10}]{rājayogāya kevalaṃ} % yogaya N19
		\rdg[wit={V3}]{rājayogaphalāvadhi}}/}\\+}
\tl{
\pada{\app{\lem[wit={ceteri}]{rājayoga}
		\rdg[wit={G4}]{rajayogaṃ}
		\rdg[wit={E2}]{rājayoge}% +C7
		\rdg[wit={J7},post=\texteng{(then lost)}]{rājayo\skp{ (then lost)}}}%
	\app{\lem[wit={ceteri}]{samārūḍhaḥ}% rūḍha V15, rūḍhā V3
	\rdg[wit={J5}]{padaprāptaḥ}}}
\pada{puruṣaḥ kālavañcakaḥ//}\label{kalavancaka} % °caka V3
\anm{after \ref{nadakoti} \getsiglum{G11,N19,V15,J10}}
	\unavbl{N23}\\!} % ḥ om. V3
\end{tlg}

\newpage
\startaltrecension
\begin{alttlg}[hp04_077_1]%
\tl{
\pada{iḍā bhagavatī gaṅgā}
\pada{piṅgalā
	\app{\lem[wit={C7}]{yamunā}
	\rdg[wit={V19}]{jamunā}} nadī/}\\+}
\tl{
\pada{\app{\lem[wit={C7}]{vijñeyā}
	\rdg[wit={V19}]{vidheyā}} taddvayor madhye} % tadvayor V19
\pada{suṣumṇā \app{\lem[wit={C7}]{tu} % so auch in L1
	\rdg[wit={V19}]{ca}} sarasvatī//} 
	\sgwit{V19,C7} \NotIn{E2} % E5 hat dies auch nicht
	\anm{cf. 3.95*1}\\!}
\end{alttlg}

\begin{alttlg}[hp04_077_2]%
\tl{
\pada{triveṇīsaṃgamo yatra}
\pada{tīrtharājaḥ sa ucyate/}\\+}
\tl{
\pada{\app{\lem[wit={C7}]{tasmiṃs tīrthavare snātvā} % +L1,K2
	\rdg[wit={V19}]{tatra snānaṃ prakurvīta}
	}}
\pada{sarvapāpaiḥ pramucyate//} 
	\sgwit{V19,C7} \NotIn{E2}\\!}
\end{alttlg}
\endaltrecension

%\newpage
\begin{tlg}[hp04_078]% = V3_4.194
\tl{
\pada{iti 
	\app{\lem[wit={GrB,Gr3a}]{tu}
		\rdg[wit={N3}]{<<tu>>}
		\rdg[wit={J5,G11},alt={\om}]{\skp{\om}}% +F
		\rdg[wit={G4}]{śrī}}
	\app{\lem[wit={ceteri}]{sakalayoga}
		\rdg[wit={G11}]{sakalasuyoga}}śāstra% 
	\app{\lem[wit={N3pc,C6,E2}]{sindhoḥ}% +C7
		\rdg[wit={J5}]{sindhauḥ}
		\rdg[wit={V19}]{sindhau}
		\rdg[wit={G11}]{siddhoḥ}
		\rdg[wit={P11}]{siddheḥ}
		\rdg[wit={N3ac}]{siddhāḥ}
		\rdg[wit={V3}]{siddhyaiḥ}
		\rdg[wit={G4},alt={\om}]{\skp{\om}}}}\\+}
\tl{
\pada{\app{\lem[wit={N3,J5,P11,C6,Gr3a,G11},alt={parimathitād}]{parimathitā\skp{d}}
		\rdg[wit={V3}]{paripaṭhitā}
		\rdg[wit={G4}]{mathitā pari}}%
	\app{\lem[wit={N3ac,J5,Gr3a,G11},alt={avakṛṣṭa}]{\skm{d }avakṛṣṭa}
		\rdg[wit={N3pc,C6}]{avakṛṣya}
		\rdg[wit={P11}]{avakṛṣṇa}
%		\rdg[wit={C7}]{apakṛṣṭa}
		\rdg[wit={V3}]{kṛṣṭa}
		\rdg[wit={G4}]{sāra}}% \marma
	\app{\lem[wit={Gr1,C6,V3,E2,G11}]{sāra}% +C7
		\rdg[wit={P11}]{sārā}
		\rdg[wit={V19}]{sarva}}bhūtam/}\\+} % bhūtaḥ P11
\tl{
\pada{\app{\lem[wit={N3,G4,V3,Gr3a}]{anubhavata}
		\rdg[wit={C6}]{anubhavatu}
		\rdg[wit={J5}]{anubhavān}
		\rdg[wit={P11,G11}]{anubhava}}
		haṭhāmṛtaṃ % <<ha>>ṭhā° V19, ayamṛtaṃ J5
	\app{\lem[wit={N3,G4,V3,C7}]{yamīndrā}% °rāḥ N3pc, yamiṃdrā N3ac, yamīndro F
		\rdg[wit={P11,V19,G11}]{yatīndrā}
		\rdg[wit={J5}]{yogīdrā}
		\rdg[wit={C6}]{mayedaṃ}
		\rdg[wit={E2}]{ya\,(text stopps here)}
		}}\\+}
\tl{
\pada{yadi bhavatā%m
	\app{\lem[wit={N3,J5,P11,V19,C7},alt={ajarāmaratvavāñchā}]{\skm{m }ajarāmaratvavāñchā}% ajjarā° P11
		\rdg[wit={C6},alt={°vāṃchāḥ}]{ajarāmaratvavāṃchāḥ}
		\rdg[wit={G4},alt={°vāṃcchāṃ}]{ajarāmaratvavāṃcchāṃ}
		\rdg[wit={G11}]{ajarāmṛtatvavāṃcha}
		\rdg[wit={V3}]{ajarājaraṃ tvaṃ vā}}//}
		\NotIn{N19,V15,J10,Jyo}
	\unavbl{N23,J7}\\!}
\end{tlg}
		% \sgwit{Gr1,V3,V19,C7,G11}

%\newpage
\startaltrecension
\begin{alttlg}[hp04_078_1]%
\tl{\pada{vidyātīrthe \app{\lem[resp=emend]{jagati}\rdg[wit={J10}]{yagati}} vibudhāḥ sādhavaḥ satyatīrthe}\\+}
\tl{\pada{gaṅgātīrthe malinamanaso yogino jñānatīrthe/}\\+}
\tl{\pada{dhārātīrthe dharaṇipatayo dānatīrthe dhanāḍhyāḥ}\\+}
\tl{\pada{lajjātīrthe kulayuvatayaḥ pātakaṃ kṣālayanti//}
\sgwit{J10}\\!}
\end{alttlg}
\endaltrecension

%\newpage
\begin{col}[hp04_col]
iti \app{\lem[wit={N3,J5,V3,C7,V15,J10}]{śrī}
	\rdg[wit={G4,P11,C6,V19,G11},alt={\om}]{\skp{\om}}}%
\app{\lem[alt={\ante śrī \add},nosep]{\skp{\ante śrī \add}}
	\rdg[wit={N3}]{sadguru}
	\rdg[wit={J5}]{madguru}
	\rdg[wit={G11,V15}]{sahajānandasaṃtānacintāmaṇinā}}% saṃtāra G11
\app{\lem[wit={J5,C6,V3}]{svātmārāmayogīndra} % yogiṃdra V3
	\rdg[wit={N3}]{svātmārāmayogendra}
	\rdg[wit={V15}]{svātmārāmayogīṃdreṇa}
	\rdg[wit={G4,J10}]{ātmārāmayogīṃdra}
	\rdg[wit={P11},post=\texteng{(sic!)}]{°yo°}
	\rdg[wit={V19,C7,G11},alt={\om}]{\skp{\om}}}%
\app{\lem[wit={ceteri}]{viracitāyāṃ}
	\rdg[wit={N3}]{pravaracitāyāṃ}
	\rdg[wit={P11,V19},alt={\om}]{\skp{\om}}} haṭhapradīpikāyāṃ
\app{\lem[alt={\ante caturtho° \add},nosep]{\skp{\ante caturtho° \add}}
	\rdg[wit={V15}]{nādopāsanaṃ nāma}
	\rdg[wit={V3}]{siddhāntamuktāvalī nāma}}%
\app{\lem[wit={Gr1,GrB,G11,V15}]{caturthopadeśaḥ}
	\rdg[wit={V19}]{caturtha upadeśaḥ}
	\rdg[wit={C7}]{caturtho\{\{dhyā\}\}yam upadeśaḥ}
	\rdg[wit={J10}]{caturthodhyāyaḥ}}// 4//%\sgwit{N3,GrB,V19,C7,G11,V15,J10}
\myfn{%
%The colophon is found only in \getsiglum{Gr1,GrB,V19,C7,G11,V15,J10}. 
%\getsiglum{N23,J7,K3} have lost their last folios.
%\getsiglum{N19} has no colophon.
\getsiglum{N19} has no colophon. Its last verse is \ref{ajananta}, which just fills fol. no. 346, and from the next folio another text begins. The colophon of \getsiglum{Jyo} reads: \devnote{iti śrīsvātmārāmayogīṃdraviracitāyāṃ haṭhayogapradīpikāyāṃ nāma caturtho'dhyāyaḥ} (Wai) or \devnote{iti śrīsajahānandasantānacintāmaṇisvātmārāmayogīṃdraviracitāyāṃ haṭhayogapradīpikāyāṃ samādhilakṣaṇaṃ nāma caturthopadeśaḥ samāptaḥ} (Tue)}
\unavbl{N23,J7,E2}%\NotIn{N19}
\end{col}

% N3 śrīsahajasadguruātmārāmārpaṇa saṃpūrṇaṃ |
% P11 (immeadiately followed by another text)
% C6 oṃ || pustakam idaṃ jājīrāmavyāsasya || śrīrāmaḥ
% V3 subhaṃ būyāt @ idaṃ graṃthasaṃṣyā || 502 || @ @ samvat || 16 || 93 || varṣe || || || liṣitaṃ jamadagnigirisanyāsī svātmapaṭhanārthaṃ || oṃ namaḥ śivāya ||
% V19
% C7
% G11 śrī śivāya namaḥ || śrī śaṃkarācāryagurucaraṇāravindābhyāṃ namaḥ || gaṇapati svahasta likhitaṃ | śrī gurubhyo namaḥ || oṃ | śubhaṃ |
% J10 saṃvat || 16 || 83 ||


\end{ekdosis}
\end{otherlanguage}
\newpage
%\vspace*{2cm}
%\vfill
\teimute{\small}
\begin{tabular}{l l l}
\multicolumn{3}{l}{\textbf{List of Sigla}} \\
\\
\getsiglum{N3} & N3 & one folio missing in Ch. 4 (\ref{cittananda}b--\ref{nadanu}d)\\
\getsiglum{J5} & J5 \\
\getsiglum{G4} & G4 & damaged; collated only when available\\
\getsiglum{P11} & P11 & \\
\getsiglum{C6} & C6 \\
\getsiglum{V3} & V3 \\
\getsiglum{N23} & N23 & incomplete; breaks at \manuref{4.56d}\\
\getsiglum{J7} & J7 & incomplete; breaks at \manuref{4.74b}\\
%\getsiglum{V6} & collated for 4.91--92 only\\
\getsiglum{V19} & V19 \\
\getsiglum{E2} & E2 \\
\getsiglum{C7} & C7 & partially collated, when \getsiglum{E2} is not available\\
\getsiglum{G11} & G11 \\
\getsiglum{G5} & G5 & collated for gray-scaled verses only \\
\getsiglum{N19} & N19 \\
\getsiglum{V15} & V15 \\
%\getsiglum{J11} & J11 & collated for \manuref{4.28} and \manuref{4.32*1--8} only\\
\getsiglum{J10} & J10 \\
\getsiglum{Jyo} & Jyo & Brahmānanda's version, based on the edition 1972 \\
\end{tabular}

\end{document}
