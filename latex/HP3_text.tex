\documentclass[10pt]{memoir}
\setstocksize{220mm}{155mm} 	        
\settrimmedsize{220mm}{155mm}{*}	
\settypeblocksize{170mm}{116mm}{*}	
\setlrmargins{18mm}{*}{*}
\setulmargins{*}{*}{1.2}
% \setlength{\headheight}{5pt}
\checkandfixthelayout[lines]
\linespread{1.2}

\setlength{\footmarkwidth}{1.3em}
\setlength{\footmarksep}{0em}
\setlength{\footparindent}{1.3em}
\footmarkstyle{\textsuperscript{#1} }
\usepackage{fnpos}
\makeFNbottom

\usepackage[teiexport=tidy,poetry=verse]{ekdosis}
\usepackage{libertine}
\usepackage{sanskrit-poetry}
\usepackage{xcolor}

\usepackage[english]{babel}
\usepackage{babel-iast,xparse,xcolor}
\babelfont[iast]{rm}[Renderer=Harfbuzz, Scale=1.5]{AdishilaSan}
%\babelfont[english]{rm}[Scale=0.9]{Adobe Text Pro}
\babeltags{dev = iast}
\babeltags{eng = english}

\SetHooks{
	lemmastyle=\bfseries,
	refnumstyle=\selectlanguage{english}\color{blue}\bfseries, 
	}
\newif\ifinapparatus
\DeclareApparatus{default}[
	lang=english,
	sep = {] },
	delim=\hskip 0.75em,
	rule=none,
	]
\DeclareApparatus{notes}[
	lang=english,
	sep = {},
	delim=\hskip 0.75em,
	rule=\rule{0.7in}{0.4pt},
	]

%\DeclareShorthand{conj}{\texteng{\emph{conj.}}}{ego}
\DeclareShorthand{emend}{\texteng{\emph{em.}}}{ego}

\setlength{\vrightskip}{-10pt}
%\setlength{\vgap}{3mm} % default 1.5em
\verselinenumfont{\footnotesize\selectlanguage{english}\normalfont}
\setlength{\stanzaskip}{0.6\baselineskip}

%Define two commands: \skp ("sanskrit plus"), to be ignored by TeX in
%the edition text, but processed in the TEI output. Conversely, \skm
%("sanskrit minus") is to be processed in the edition text, but
%ignored if found in the apparatus criticus and in the TEI output:

\NewDocumentCommand{\skp}{m}{}
%\NewDocumentCommand{\skm}{m}{\unless\ifinapparatus#1-\fi}
\NewDocumentCommand{\skm}{m}{\unless\ifinapparatus#1\fi} % modified by MD 2022-05-31

%


%%%%%%%%%%%%%%%%%%%% THE  MSS         %%%%%%%%%%%%%%%%%%%%%%%%%%%

%%% Versions
\DeclareWitness{Vu}{\selectlanguage{english}Vulg}{Vulgate, i.e. Brahmānanda's version}[]           
\DeclareWitness{X}{\selectlanguage{english}X}{TenChapter Version, Jodhpur 02228 and 02225 (ed. Lonavla)}[]
\DeclareWitness{Six}{\selectlanguage{english}Ṣ}{SixChapterVersion, ``6ChapterHPms'', fragment of enlarged text, Jodhpur}[]
% Mss. in Geographical Groups
%%%% Varanasi mss (Sampūrṇānanda mss). V1 is Important
\DeclareWitness{V1}{\selectlanguage{english}V\textsubscript{1}}{Sampurnananda Library Sarasvati Bhavan 30109}[]
        \DeclareHand{V1ac}{V1}{\selectlanguage{english}V\rlap{\textsubscript{1}}\textsuperscript{ac}}[] % added by MD
        \DeclareHand{V1pc}{V1}{\selectlanguage{english}V\rlap{\textsubscript{1}}\textsuperscript{pc}}[] % added by MD
\DeclareWitness{V2}{\selectlanguage{english}V\textsubscript{2}}{Sampurnananda Library Sarasvati Bhavan 29869}[]
\DeclareWitness{V3}{\selectlanguage{english}V\textsubscript{3}}{Sampurnananda Library Sarasvati Bhavan 29899}[]
\DeclareWitness{V4}{\selectlanguage{english}V\textsubscript{4}}{Sampurnananda Library Sarasvati Bhavan 29937}[]
\DeclareWitness{V5}{\selectlanguage{english}V\textsubscript{5}}{Sampurnananda Library Sarasvati Bhavan 29938}[]
\DeclareWitness{V6}{\selectlanguage{english}V\textsubscript{6}}{Sampurnananda Library Sarasvati Bhavan 29991}[]
\DeclareWitness{V8}{\selectlanguage{english}V\textsubscript{8}}{Sampurnananda Library Sarasvati Bhavan 30014}[]
\DeclareWitness{V11}{\selectlanguage{english}V\textsubscript{11}}{Sampurnananda Library Sarasvati Bhavan 30029}[]
\DeclareWitness{V12}{\selectlanguage{english}V\textsubscript{12}}{Sampurnananda Library Sarasvati Bhavan 30030}[]
\DeclareWitness{V13}{\selectlanguage{english}V\textsubscript{13}}{Sampurnananda Library Sarasvati Bhavan 30031}[]
\DeclareWitness{V14}{\selectlanguage{english}V\textsubscript{14}}{Sampurnananda Library Sarasvati Bhavan 30050}[]
\DeclareWitness{V15}{\selectlanguage{english}V\textsubscript{15}}{Sampurnananda Library Sarasvati Bhavan 30051}[]
\DeclareWitness{V15pc}{\selectlanguage{english}V\rlap{\textsubscript{15}}\textsuperscript{pc}\space}{}[]
\DeclareWitness{V16}{\selectlanguage{english}V\textsubscript{16}}{Sampurnananda Library Sarasvati Bhavan 30052}[]
\DeclareWitness{V17}{\selectlanguage{english}V\textsubscript{17}}{Sampurnananda Library Sarasvati Bhavan 30053}[] % added by MD
\DeclareWitness{V16pc}{\selectlanguage{english}V\rlap{\textsubscript{16}}\textsuperscript{pc}\space}{}[]
\DeclareWitness{V18}{\selectlanguage{english}V\textsubscript{18}}{Sampurnananda Library Sarasvati Bhavan 30064}[]
\DeclareWitness{V19}{\selectlanguage{english}V\textsubscript{19}}{Sampurnananda Library Sarasvati Bhavan 30069}[]
\DeclareWitness{V21}{\selectlanguage{english}V\textsubscript{21}}{Sampurnananda Library Sarasvati Bhavan 30104}[]
\DeclareWitness{V22}{\selectlanguage{english}V\textsubscript{22}}{Sampurnananda Library Sarasvati Bhavan 30110}[]
\DeclareWitness{V25}{\selectlanguage{english}V\textsubscript{25}}{Sampurnananda Library Sarasvati Bhavan 30122}[]
\DeclareWitness{V26}{\selectlanguage{english}V\textsubscript{26}}{Sampurnananda Library Sarasvati Bhavan 30123}[]
\DeclareWitness{V28}{\selectlanguage{english}V\textsubscript{28}}{Sampurnananda Library Sarasvati Bhavan 30136}[]
\DeclareWitness{W2}{\selectlanguage{english}W\textsubscript{2}}{Wai ??}[]
\DeclareWitness{W4}{\selectlanguage{english}W\textsubscript{4}}{Wai 399-6171}[]

%%%%%%%%%%%%%%%%%%%%%%%%%%%%%%%%%
%%% Jammu & Kaschmir
\DeclareWitness{K1}{\selectlanguage{english}K\textsubscript{1}}{Raghunātha Temple Library 4383}[settlement=Jammu]
        \DeclareWitness{K1ac}{\selectlanguage{english}K\rlap{\textsubscript{1}}\textsuperscript{ac}\space}{}[]
        \DeclareWitness{K1pc}{\selectlanguage{english}K\rlap{\textsubscript{1}}\textsuperscript{pc}\space}{}[]
\DeclareWitness{K3}{\selectlanguage{english}K\textsubscript{3}}{Privat collection}
\DeclareWitness{L1}{\selectlanguage{english}L\textsubscript{1}}{SOAS RE 43454}[settlement=Jammu]
% More details? Catalogue number? L1 And C1 very close (and come from same region)
%%%%%%%%%%%%%%%%%%%%%%%%%%%%%%%%
% Jodhpur
% J10 is important
\DeclareWitness{J10}{\selectlanguage{english}J\textsubscript{10}}{MSPP Jodhpur 2230}[]
        \DeclareHand{J10ac}{J10}{\selectlanguage{english}J\rlap{\textsubscript{10}}\textsuperscript{ac}}[] % modified by MD
        \DeclareHand{J10pc}{J10}{\selectlanguage{english}J\rlap{\textsubscript{10}}\textsuperscript{pc}}[] % modified by MD
\DeclareWitness{J1}{\selectlanguage{english}J\textsubscript{1}}{Jodhpur 02231}[]
\DeclareWitness{J2}{\selectlanguage{english}J\textsubscript{2}}{Jodhpur 02232}[]   
\DeclareWitness{J3}{\selectlanguage{english}J\textsubscript{3}}{Jodhpur 02233}[]
\DeclareWitness{J4}{\selectlanguage{english}J\textsubscript{4}}{Jodhpur 02234}[]
        \DeclareWitness{J4ac}{\selectlanguage{english}J\rlap{\textsubscript{4}}\textsuperscript{ac}\space}{MSPP Jodhpur 02234}[]
        \DeclareWitness{J4pc}{\selectlanguage{english}J\rlap{\textsubscript{4}}\textsuperscript{pc}\space}{MSPP Jodhpur 02234}[]
\DeclareWitness{J5}{\selectlanguage{english}J\textsubscript{5}}{Jodhpur 02235}[]  % 4 chapters, 34 jpgs,   long colophon, missing lines in the beginning.
\DeclareWitness{J6}{\selectlanguage{english}J\textsubscript{6}}{Jodhpur 02237}[]  % 4 chapters, 41 jpgs
%\DeclareWitness{J6ac}{\selectlanguage{english}J\rlap{\textsubscript{6}}\textsubscript{ac}}{Jodhpur 02237}[]  % 4 chapters, 49 jpgs,   1st folio: idaṃ gulābarāyasya
% tulasīrāmaśarmmaṇaḥ putrasya pustakaṃ ...        End: iti śrīsahajānandasantānacintāmaṇisvātmārāmaviracitāyāṃ ..
% saṃvat 1802   (more consistent text)
%\DeclareWitness{J6pc}{\selectlanguage{english}J\rlap{\textsubscript{6}}\textsubscript{pc}}{Jodhpur 02237}[] 
\DeclareWitness{J7}{\selectlanguage{english}J\textsubscript{7}}{Jodhpur 02241}[]  % 4 chapters, 41 jpgs
\DeclareWitness{J8}{\selectlanguage{english}J\textsubscript{8}}{Jodhpur 23709}[]  % 4 chapters,  87 jpgs.   saṃvat 1724
\DeclareHand{J8ac}{J8}{\selectlanguage{english}J\rlap{\textsubscript{8}}\textsuperscript{ac}}[]  % changed by MD
\DeclareHand{J8pc}{J8}{\selectlanguage{english}J\rlap{\textsubscript{8}}\textsuperscript{pc}}[]  % changed by MD
\DeclareWitness{J9}{\selectlanguage{english}J\textsubscript{9}}{Jodhpur 02224}[]  %  fragment, 20 jpgs.
\DeclareWitness{J11}{\selectlanguage{english}J\textsubscript{11}}{Jodhpur 23532}[]
        \DeclareHand{J11ac}{J11}{\selectlanguage{english}J\rlap{\textsubscript{11}}\textsuperscript{ac}}[] % added by MD
        \DeclareHand{J11pc}{J11}{\selectlanguage{english}J\rlap{\textsubscript{11}}\textsuperscript{pc}}[] % added by MD
\DeclareWitness{J12}{\selectlanguage{english}J\textsubscript{12}}{Jodhpur 18552}[] 
\DeclareWitness{J13}{\selectlanguage{english}J\textsubscript{13}}{Jodhpur 02229}[]  %  5 chapters, 93 jpgs.
\DeclareWitness{J14}{\selectlanguage{english}J\textsubscript{14}}{Jodhpur 02239}[]  %  4 chapters
\DeclareWitness{J15}{\selectlanguage{english}J\textsubscript{15}}{Jodhpur 9732A}[]
\DeclareWitness{J16}{\selectlanguage{english}J\textsubscript{16}}{Jodhpur 9732B}[]
\DeclareWitness{J17}{\selectlanguage{english}J\textsubscript{17}}{Jodhpur 3013}[]
% Haṭhapradīpikā with (non-Sanskrit) Bhāṣya RORI Jodhpur ACC.NO.18552
%  Haṭhapradīpikā with (non-Sanskrit) commentary, RORI Alwar 952, 4 chapters,  colophon of the comm:
% iti śrīlāhorīmiśravrajabhūṣanaviracitāyāṃ bhāvārthadīpikāyāṃ caturthodhyāya ..    
%  Haṭhapradīpikā (5 chapter) MSPP Jodhpur ACC.NO.02229/

%%%%%%%%%%        Bodleian, Oxford
\DeclareWitness{B1}{\selectlanguage{english}B\textsubscript{1}}{Bodleian Library No. d.457(8)}[settlement=Oxford]
\DeclareWitness{B2}{\selectlanguage{english}B\textsubscript{2}}{Bodleian Library No. d.458(1)}[settlement=Oxford]
\DeclareWitness{B3}{\selectlanguage{english}B\textsubscript{3}}{Bodleian Library No. d.458(9)}[settlement=Oxford]

%%%%%%%%%%%   Chandigarh
\DeclareWitness{C1}{\selectlanguage{english}C\textsubscript{1}}{Lalchand M-2080}[]%L1 And C1 very close (and come from same region)
\DeclareWitness{C2}{\selectlanguage{english}C\textsubscript{2}}{Lalchand M-6065}[]
\DeclareWitness{C3}{\selectlanguage{english}C\textsubscript{3}}{Lalchand M-1293}[]
\DeclareWitness{C4}{\selectlanguage{english}C\textsubscript{4}}{Lalchand M-2081}[]
\DeclareWitness{C4ac}{\selectlanguage{english}C\rlap{\textsubscript{4}}\textsuperscript{ac}\space}{}[]
\DeclareWitness{C4pc}{\selectlanguage{english}C\rlap{\textsubscript{4}}\textsuperscript{pc}\space}{}[]
\DeclareWitness{C5}{\selectlanguage{english}C\textsubscript{5}}{Lalchand M-2082}[]%doesn't have chapter 1
\DeclareWitness{C6}{\selectlanguage{english}C\textsubscript{6}}{Lalchand M-2089}[]
\DeclareWitness{C7}{\selectlanguage{english}C\textsubscript{7}}{Lalchand M-6494}[]
\DeclareWitness{C8}{\selectlanguage{english}C\textsubscript{8}}{Lalchand M-2091}[]
        \DeclareHand{C8ac}{C8}{\selectlanguage{english}C\rlap{\textsubscript{8}}\textsuperscript{ac}}[]
        \DeclareHand{C8pc}{C8}{\selectlanguage{english}C\rlap{\textsubscript{8}}\textsuperscript{pc}}[]
\DeclareWitness{C9}{\selectlanguage{english}C\textsubscript{9}}{Lalchand M-4530}[]


% %%%%%%%%%%        Nepalese
\DeclareWitness{N1}{\selectlanguage{english}N\textsubscript{1}}{NGMPP A1400-2}[]
\DeclareWitness{N2}{\selectlanguage{english}N\textsubscript{2}}{NGMPP B 39-19}[]
\DeclareWitness{N3}{\selectlanguage{english}N\textsubscript{3}}{NGMPP B 62-20}[]
\DeclareWitness{N5}{\selectlanguage{english}N\textsubscript{5}}{NGMPP A60-15 + A61-1}[]
\DeclareWitness{N4}{\selectlanguage{english}N\textsubscript{4}}{NGMPP A61-2}[]
\DeclareWitness{N6}{\selectlanguage{english}N\textsubscript{6}}{NGMPP A61-6}[]
\DeclareWitness{N9}{\selectlanguage{english}N\textsubscript{9}}{NGMPP A62-33}[]
\DeclareWitness{N10}{\selectlanguage{english}N\textsubscript{10}}{NGMPP A62-37}[]
\DeclareWitness{N11}{\selectlanguage{english}N\textsubscript{11}}{NGMPP A63-15}[]
\DeclareWitness{N12}{\selectlanguage{english}N\textsubscript{12}}{NGMPP A939-19}[]
\DeclareWitness{N13}{\selectlanguage{english}N\textsubscript{13}}{NGMPP A1378-18}[]
\DeclareWitness{N16}{\selectlanguage{english}N\textsubscript{16}}{NGMPP B39-20}[]
\DeclareWitness{N17}{\selectlanguage{english}N\textsubscript{17}}{NGMPP B 111-10}[]
\DeclareWitness{N18}{\selectlanguage{english}N\textsubscript{18}}{NGMPP E 929-3}[]
\DeclareWitness{N19}{\selectlanguage{english}N\textsubscript{19}}{NGMPP E-1528-1 / E-1527-7(4)}[]
\DeclareWitness{N20}{\selectlanguage{english}N\textsubscript{20}}{NGMPP E 2037-13 }[]
\DeclareWitness{N21}{\selectlanguage{english}N\textsubscript{21}}{NGMPP E 2097-31}[]
\DeclareWitness{N22}{\selectlanguage{english}N\textsubscript{22}}{NGMPP G 4-4}[]
\DeclareWitness{N23}{\selectlanguage{english}N\textsubscript{23}}{NGMPP G 25-2}[]
        \DeclareHand{N23ac}{N23}{\selectlanguage{english}N\rlap{\textsubscript{23}}\textsuperscript{ac}}[] % added by MD
        \DeclareHand{N23pc}{N23}{\selectlanguage{english}N\rlap{\textsubscript{23}}\textsuperscript{pc}}[] % added by MD
\DeclareWitness{N24}{\selectlanguage{english}N\textsubscript{24}}{NGMPP G 190-16}[]
\DeclareWitness{N24ac}{\selectlanguage{english}N\rlap{\textsubscript{24}}\textsuperscript{ac}\space}{}[]
\DeclareWitness{N24pc}{\selectlanguage{english}N\rlap{\textsubscript{24}}\textsuperscript{pc}\space}{}[]
\DeclareWitness{N26}{\selectlanguage{english}N\textsubscript{26}}{NGMPP T 24-3}[]

% %%%%%%%%%%        Pune

\DeclareWitness{P1}{\selectlanguage{english}P\textsubscript{1}}{Ānandāśrama S16-3-21}[]
\DeclareWitness{P2}{\selectlanguage{english}P\textsubscript{2}}{Ānandāśrama S16-2-20}[]
\DeclareWitness{P3}{\selectlanguage{english}P\textsubscript{3}}{BISM (79) 314}[]
\DeclareWitness{P4}{\selectlanguage{english}P\textsubscript{4}}{BISM (91) 191}[]
\DeclareWitness{P5}{\selectlanguage{english}P\textsubscript{5}}{BISM (29) 5790}[]
\DeclareWitness{P6}{\selectlanguage{english}P\textsubscript{6}}{BORI 263/1879-80}[]
\DeclareWitness{P7}{\selectlanguage{english}P\textsubscript{7}}{BORI 665/1883-84}[]
\DeclareWitness{P8}{\selectlanguage{english}P\textsubscript{8}}{BORI 316/1895-98}[]
\DeclareWitness{P9}{\selectlanguage{english}P\textsubscript{9}}{BORI 733-1891-95}[]
\DeclareWitness{P10}{\selectlanguage{english}P\textsubscript{10}}{BORI 222-1884-86}[]
\DeclareWitness{P11}{\selectlanguage{english}P\textsubscript{11}}{BORI 221-1882–83}[]
\DeclareWitness{P12}{\selectlanguage{english}P\textsubscript{12}}{Ānandāśrama S16-3-24}[]
\DeclareWitness{P13}{\selectlanguage{english}P\textsubscript{13}}{Ānandāśrama S16-2-22}[]
\DeclareWitness{P14}{\selectlanguage{english}P\textsubscript{14}}{Ānandāśrama S16-3-23}[]
\DeclareWitness{P15}{\selectlanguage{english}P\textsubscript{15}}{BISM (64) 919}[]
\DeclareWitness{P16}{\selectlanguage{english}P\textsubscript{16}}{BISM (64) 1115}[]
\DeclareWitness{P17}{\selectlanguage{english}P\textsubscript{17}}{BISM 620/1886-92}[]
\DeclareWitness{P18}{\selectlanguage{english}P\textsubscript{18}}{BORI 615/1887-91}[]
\DeclareWitness{P19}{\selectlanguage{english}P\textsubscript{19}}{BISM 46-39}[]
\DeclareWitness{P20}{\selectlanguage{english}P\textsubscript{20}}{BISM 39-273}[]
\DeclareWitness{P21}{\selectlanguage{english}P\textsubscript{21}}{BISM 37-743}[]
\DeclareWitness{P22}{\selectlanguage{english}P\textsubscript{22}}{BISM 37-729}[]
\DeclareWitness{P23}{\selectlanguage{english}P\textsubscript{23}}{BISM 33-60}[]
\DeclareWitness{P24}{\selectlanguage{english}P\textsubscript{24}}{BISM 29-5790}[]% =P5!
\DeclareWitness{P25}{\selectlanguage{english}P\textsubscript{25}}{BISM 29-3657}[]
\DeclareWitness{P26}{\selectlanguage{english}P\textsubscript{26}}{BISM 25-281}[]
\DeclareWitness{P27}{\selectlanguage{english}P\textsubscript{27}}{BISM 7-489}[]
\DeclareWitness{P28}{\selectlanguage{english}P\textsubscript{28}}{BORI 399-1895-1902}[]

%%%%%   Mysore
\DeclareWitness{M1}{\selectlanguage{english}M\textsubscript{1}}{P-5682/4}[]
%%%%%   Tübingen
\DeclareWitness{Tue}{\selectlanguage{english}Tü}{Ma I 339}[]
%%%%%%%%%%
\DeclareWitness{YC}{\selectlanguage{english}YC}{Yogacintāmaṇi}[]
\DeclareWitness{ceteri}{\selectlanguage{english}cett.}{ceteri}[]

%%%%%%%%%% Mss with Commentary
\DeclareWitness{A1}{\selectlanguage{english}A\textsubscript{1}}{Alwar 952}[]

\DeclareWitness{Jyo}{\selectlanguage{english}J\textsubscript{yo}}{Brahmānanda's version}[]

%%%%%%%%%%%%%%%%%%%%%%%%%%%%%%%%%%%%%%%%%%%
%List of all Sigla:
%A1,B1,B2,B3,C1,C2,C3,C4,C6,C7,C8,C9,J1,J2,J3,J4,J10,J13,J14,J15,J17,L1,M1,N3,N5,N6,N9,N10,N11,N12,N13,N16,N17,N19,N20,N21,N22,N23,N24,Tü,V1,V2,V3,V4,V5,V6,V8,V11,V19,V22,V26,Vu
%%%%%%%%%%%%%%%%%%%%%%%%%%%%%%%%%%%%%%%%%%%

\DeclareWitness{G4}{\selectlanguage{english}G\textsubscript{4}}{GOML D18885 (Bundle SD5051)}[]
\DeclareWitness{G5}{\selectlanguage{english}G\textsubscript{5}}{GOML R3841/ SR2190}[]
\DeclareWitness{G7}{\selectlanguage{english}G\textsubscript{7}}{GOML D4394}[]

\DeclareWitness{Ko}{\selectlanguage{english}K\textsubscript{o}}{Koba, Gujarat 55626}[]

%
%%%%%                   Abbreviation for the printed apparatus,        xml interface needed
%%%%%                   (synonyms in same line)

% Macro for Editing Abbrevs.
%\def\om{\textrm{\footnotesize \textit{omitted in}\ }} %prints om. for omitted in apparatus
%\def\korr{\textrm{\footnotesize \textit{em.}\ }} %prints em. for emended in apparatus
%\def\conj{\textrm{\footnotesize \textit{conj.}\ }} %prints conj. for conjectured in apparatus


\def\eyeskip{\textrm{{ab.\,oc. }}}   
\def\aberratio{\textrm{{ab.\,oc. }}}
\def\ad{\textrm{{ad}}}   
\def\add{\textrm{{add.\ }}}
\def\ann{\textrm{{ann.\ }}}
\def\ante{\textrm{{ante }}}
\def\post{\textrm{{post }}}
%\def\ceteri{cett.\,}             % for simplifying the apparatus in print                  
\def\codd{\textrm{{codd.\ }}}   %  the same
\def\conj{\textrm{{coni.\ }}}  
\def\coni{\textrm{{coni.\ }}}
\def\contin{\textrm{{contin.\ }}}
\def\corr{\textrm{{corr.\ }}}
\def\del{\textrm{{del.\ }}}
\def\dub{\textrm{{ dub.\ }}}
\def\emend{\textrm{{emend.\ }}}
\def\expl{\textrm{{explic.\ }}}   
\def\explicat{\textrm{{explic.\ }}}
\def\fol{\textrm{{fol.\ }}}         
\def\foll{\textrm{{foll.\ }}}
\def\gloss{\textrm{{glossa ad }}}
\def\ins{\textrm{{ins.\ }}}          \def\inseruit{\textrm{{ins.\ }}}
\def\im{{\kern-.7pt\lower-1ex\hbox{\textrm{\tiny{\emph{i.m.}}}\kern0pt}}}
\def\inmargine{{\kern-.7pt\lower-.7ex\hbox{\textrm{\tiny{\emph{i.m.}}}\kern0pt}}}
\def\intextu{{\kern-.7pt\lower-.95ex\hbox{\textrm{\tiny{\emph{i.t.}}}\kern0pt}}}%\textrm{\scriptsize{i.t.\ }}}               
\def\indist{\textrm{{indis.\ }}}          \def\indis{\textrm{{indis.\ }}}
\def\iteravit{\textrm{{iter.\ }}}          \def\iter{\textrm{{iter.\ }}}  
\def\lectio{\textrm{{lect.\ }}}             \def\lec{\textrm{{lect.\ }}}
\def\leginequit{\textrm{{l.n. }}}         \def\legn{\textrm{{l.n. }}}         \def\illeg{\textrm{{l.n. }}}
\def\om{\textrm{{om. }}}
\def\primman{\textrm{{pr.m.}}}
\def\prob{\textrm{{prob.}}}
\def\rep{\textrm{{repetitio }}}
% \def\secundamanu{\textrm{\scriptsize{s.m.}}}
% \def\secm{{\kern-.6pt\lower-.91ex\hbox{\textrm{\tiny{\emph{s.m.}}}\kern0pt}}}%   \textrm{\scriptsize{s.m.}}}
\def\sequentia{\textrm{{seq.\,inv.\ }}}         \def\seqinv{\textrm{{seq.\,inv.\ }}} \def\order{\textrm{{seq.\,inv.\ }}}
\def\supralineam{{\kern-.7pt\lower-.91ex\hbox{\textrm{\tiny{\emph{s.l.}}}\kern0pt}}} %\textrm{\scriptsize{s.l.}}}
\def\interlineam{{\kern-.7pt\lower-.91ex\hbox{\textrm{\tiny{\emph{s.l.}}}\kern0pt}}}   %\textrm{\scriptsize{s.l.}}}
\def\vl{\textrm{v.l.}}   \def\varlec{\textrm{v.l.}} \def\varialectio{\textrm{v.l.}}
\def\vide{\textrm{{cf.\ }}}           \def\cf{\textrm{{cf.\ }}}
\def\videtur{\textrm{{vid.\,ut}}}
\def\crux{\textup{[\ldots]} }
\def\cruxx{\textup{[\ldots]}}
\def\unm{\textit{unm.}}        % unmetrical
%%%%%%%%%%%%%%%%%%%%%%%%%%%%%%%%%%%%



%%% Local Variables:
%%% mode: latex
%%% TeX-master: t
%%% End:

% addition 2023-12-11 MD:
\TeXtoTEIPat{\begin {metre}[#1]}{<note type="metre" target="##1">}
\TeXtoTEIPat{\end {metre}}{</note>}
\TeXtoTEIPat{\texttheta}{θ}

% change 2023-12-05 mm
\TeXtoTEI{teimute}{} 

% changes/additions 2023-11-27 MM:
\TeXtoTEIPat{\medialink {#1}{#2}}{<ref target="resources/#2">#1</ref>}

% changes/additions 2023-10-25 MM:
% new Sigla
\TeXtoTEIPat{\textAlpha}{Α}
\TeXtoTEIPat{\textalpha}{α}
\TeXtoTEIPat{\textBeta}{Β}
\TeXtoTEIPat{\textbeta}{β}
\TeXtoTEIPat{\textGamma}{Γ}
\TeXtoTEIPat{\textgamma}{γ}
\TeXtoTEIPat{\textDelta}{Δ}
\TeXtoTEIPat{\textdelta}{δ}
\TeXtoTEIPat{\textEpsilon}{Ε}
\TeXtoTEIPat{\textepsilon}{ε}
\TeXtoTEIPat{\textEta}{Η}
\TeXtoTEIPat{\texteta}{η}
\TeXtoTEIPat{\textChi}{Χ}
\TeXtoTEIPat{\textchi}{χ}
\TeXtoTEIPat{\textOmega}{Ω}
\TeXtoTEIPat{\textomega}{ω}

%new environments
\TeXtoTEIPat{\begin {postmula}[#1]}{<note type="postmula" target="##1">}
  \TeXtoTEIPat{\end {postmula}}{</note>}
\TeXtoTEIPat{\begin {altava}[#1]}{<div type="altrec"><note type="avataranika" target="##1">} %%% changed 2023-12-05 mm
  \TeXtoTEIPat{\end {altava}}{</note></div>} %%% changed 2023-12-05 mm
\TeXtoTEIPat{\sgwit {#1}}{<note type="inlineref"><ref>#1</ref></note>}

% changes/additions 2023-10-12 MM:
\TeXtoTEIPat{\\.}{}

% changes/additions 2023-08-15 MD:
\TeXtoTEIPat{\lineom {#1}{#2}}{<note type="omission">#1 omitted in <ref>#2</ref></note>}
\TeXtoTEI{graus}{hi}[rend="grey"]
\TeXtoTEIPat{\startgray}{} %%% changed 2023-12-05 mm
\TeXtoTEIPat{\endgray}{} %%% changed 2023-12-05 mm



% additions/changes 2023-06-05 mm:
%\TeXtoTEIPat{\lineom {#1}}{<note type="omission">Line omitted in <ref>#1</ref></note>}
\TeXtoTEIPat{\NotIn {#1}}{<note type="omission">Stanza omitted in <ref>#1</ref></note>}

% additions 2023-04-16 MD:
\TeXtoTEIPat{\,}{ }

% additions 2023-04-13 mm:
\TeXtoTEIPat{\begin {versinnote}}{<lg>}
  \TeXtoTEIPat{\end {versinnote}}{</lg>}

% additions 2023-04-05 MD:
\TeXtoTEIPat{\begin {testimonia}[#1]}{<note type="testimonia" target="##1">}
  \TeXtoTEIPat{\end {testimonia}}{</note>}
\TeXtoTEI{devnote}{s}[xml:lang="sa-deva"]

% app in philcomm und testimonia %%% added MM 2023-12-02
\TeXtoTEI{var}{note}[type="appinnote"]


\TeXtoTEI{anm}{note}[type="memo"] %% change 2023-04-16 MD
\TeXtoTEI{Anm}{note}[type="memo"] %% change 2023-12-05 MM
\TeXtoTEIPat{\startverse}{} %%% marked for change 2023-04-13 mm
\TeXtoTEIPat{\endverse}{} %%% marked for change 2023-04-13 mm
\TeXtoTEIPat{\newpage}{}
\TeXtoTEIPat{\marma}{}
\TeXtoTEIPat{\marmas}{}
\TeXtoTEIPat{\vin}{} % added by MD 2023-11-14

%%% modify environments and commands
%%% TEI mapping
% additions/changes 2022-06-07 mm:
\TeXtoTEI{grau}{hi}[rend="grey"]
\TeXtoTEIPat{ \& }{ &amp; }

% additions/changes 2022-06-01 mm:
\TeXtoTEI{skp}{seg}[type="deva-ignore"]
\TeXtoTEI{skm}{seg}[type="ltn-ignore"]

\TeXtoTEIPat{\rlap {#1}}{#1}

% additions/changes 2022-04-06 mm:
%\TeXtoTEI{sgwit}{ref}
\TeXtoTEI{textdev}{s}[xml:lang="sa-deva"]
\TeXtoTEIPat{\begin {col}[#1]}{<div type="colophon" xml:id="#1"><p>}
  \TeXtoTEIPat{\end {col}}{</p></div>}
\TeXtoTEIPat{\begin {ava}[#1]}{<note type="avataranika" target="##1">}
  \TeXtoTEIPat{\end {ava}}{</note>}
												   
\TeXtoTEIPat{\outdent}{}
\TeXtoTEIPat{\startaltrecension}{} %%% changed 2023-12-05 mm
\TeXtoTEIPat{\endaltrecension}{} %%% changed 2023-12-05 mm
\TeXtoTEIPat{\startaltnormal}{} % added by MD 2023-11-14 %%% changed 2023-12-05 mm
\TeXtoTEIPat{\endaltnormal}{} % added by MD 2023-11-14 %%% changed 2023-12-05 mm
\TeXtoTEIPat{\begin {alttlg}[#1]}{<div type="altrec"><lg xml:id="#1">}
  \TeXtoTEIPat{\end {alttlg}}{</lg></div>}



% additions/changes 2022-03-12 mm:
\TeXtoTEIPat{\begin {tlg}[#1]}{<lg xml:id="#1">}
  \TeXtoTEIPat{\end {tlg}}{</lg>}

\TeXtoTEIPat{\begin {translation}[#1]}{<note type="translation" target="##1">}
  \TeXtoTEIPat{\end {translation}}{</note>}
\TeXtoTEIPat{\begin {philcomm}[#1]}{<note type="philcomm" target="##1">}
  \TeXtoTEIPat{\end {philcomm}}{</note>}
\TeXtoTEIPat{\begin {sources}[#1]}{<note type="sources" target="##1">}
  \TeXtoTEIPat{\end {sources}}{</note>}


\TeXtoTEIPat{\begin {marma}[#1]}{<note type="marma" target="##1">}
  \TeXtoTEIPat{\end {marma}}{</note>}

\TeXtoTEIPat{\begin {jyotsna}[#1]}{<note type="jyotsna" target="##1">}
  \TeXtoTEIPat{\end {jyotsna}}{</note>}

\EnvtoTEI{description}{list}
\EnvtoTEI{itemize}{list}
\TeXtoTEIPat{\item [#1]}{<label>#1</label>\item}

\TeXtoTEI{tl}{l}
\TeXtoTEI{myfn}{note}[type="myfn"]
\TeXtoTEIPat{\getsiglum {#1}}{<ref target="##1"/>}

\TeXtoTEI{SetLineation}{}
\TeXtoTEI{noindent}{}
\TeXtoTEI{subsection*}{}

\TeXtoTEI{rlap}{}

% end additions/changes
% \TeXtoTEIPat{\skp {#1}}{#1}
% \TeXtoTEIPat{\skm {#1}}{}

\TeXtoTEIPat{\begin {prose}}{<p>}
  \TeXtoTEIPat{\end {prose}}{</p>}

\TeXtoTEIPat{\begin {tlate}}{<p>}
  \TeXtoTEIPat{\end {tlate}}{</p>}

\TeXtoTEI{emph}{hi}
\TeXtoTEI{bigskip}{}
% \TeXtoTEI{/}{|}
\TeXtoTEI{tl}{l}
\TeXtoTEIPat{english}{}
%\TeXtoTEIPat{-}{ } %% change 2023-04-16 MD
%\TeXtoTEIPat{°}{} %% change 2023-04-16 MD
\TeXtoTEIPat{\textcolor {#1}{#2}}{<hi rend="#1">#2</hi>}

% \TeXtoTEIPat{\eyeskip}{}
% \TeXtoTEIPat{\aberratio}{}
% \TeXtoTEIPat{\ad}{}
\TeXtoTEIPat{\add}{<hi rend="italic">add.</hi>} %% change 2023-04-16 MD
% \TeXtoTEIPat{\ann}{}
\TeXtoTEIPat{\ante}{<hi rend="italic">ante</hi> } %% change 2023-04-16 MD
\TeXtoTEIPat{\post}{<hi rend="italic">post</hi> } %% change 2023-04-16 MD
% \TeXtoTEIPat{\codd}{}
% \TeXtoTEIPat{\conj }{}
% \TeXtoTEIPat{\contin}{}
% \TeXtoTEIPat{\corr}{}
% \TeXtoTEIPat{\del}{}
% \TeXtoTEIPat{\dub}{}
% \TeXtoTEIPat{\emend }{}
% \TeXtoTEIPat{\expl}{}
% \TeXtoTEIPat{\ȩxplicat}{}
% \TeXtoTEIPat{\fol}{}
% \TeXtoTEIPat{\gloss}{}
% \TeXtoTEIPat{\ins}{}
% \TeXtoTEIPat{\im}{}
% \TeXtoTEIPat{\inmargine}{}
% \TeXtoTEIPat{\intextu}{}
% \TeXtoTEIPat{\indist}{}
% \TeXtoTEIPat{\iteravit}{}
% \TeXtoTEIPat{\lectio}{}
% \TeXtoTEIPat{\leginequit}{}
% \TeXtoTEIPat{\legn}{}
% \TeXtoTEIPat{\illeg}{<hi rend="italic">illeg.</hi>}
\TeXtoTEIPat{\illeg}{<gap reason="illeg."/>} %%% change 2023-04-11 mm
% \TeXtoTEIPat{\om}{<hi rend="italic">om.</hi>}
\TeXtoTEIPat{\om}{<gap reason="om."/>} %%% change 2023-04-11 mm
% \TeXtoTEIPat{\primman}{}
% \TeXtoTEIPat{\prob}{}
% \TeXtoTEIPat{\rep}{}
% \TeXtoTEIPat{\sequentia}{}
% \TeXtoTEIPat{\supralineam}{}
% \TeXtoTEIPat{\interlineam}{}
\TeXtoTEIPat{\vl}{<hi rend="italic">v.l.</hi>}
% \TeXtoTEIPat{\vide}{}
% \TeXtoTEIPat{\videtur}{}
% \TeXtoTEIPat{\crux}{}
% \TeXtoTEIPat{\cruxxx}{}
\TeXtoTEIPat{\unm}{<hi rend="italic">unm.</hi>}


% List of Scholars
\DeclareScholar{nos}{nos}[
forename=HPP,
surname=Team]


% Nullify \selectlanguage in TEI as it has been used in
% \DeclareWitness but should be ignored in TEI.
\TeXtoTEI{selectlanguage}{}



% additions/changes 2022-04-06 mm:
%\NewDocumentEnvironment{ava}{O{}}{\begin{ekdpar}\SetLineation{lineation=none}}{\end{ekdpar}}
%\NewDocumentEnvironment{col}{O{}}{\begin{ekdpar}\SetLineation{lineation=none}}{\end{ekdpar}}

% end additions
% added by MM 2022-10-25:
\NewDocumentEnvironment{postmula}{O{}}{
  \begin{ekdverse}
    \hspace{-\vgap}}{
  \end{ekdverse}
  \vskip 0.6\baselineskip
}
% modified by MD 2022-05-8:
\NewDocumentEnvironment{ava}{O{}}{
  \begin{ekdverse}
    \hspace{-\vgap}}{
  \end{ekdverse}
  \vskip 0.6\baselineskip
}
\NewDocumentEnvironment{col}{O{}}{
  \medskip
  \setvnum{col}
%  \selectlanguage{iast}
  \begin{ekdverse}
    \hspace{-\vgap}}{
  \end{ekdverse}
}

        
% modifications/additions by MM 2022-06-07:
\NewDocumentEnvironment{altava}{O{}}{
  \begin{ekdverse}\color{gray}
    \hspace{-\vgap}}{
  \end{ekdverse}
  \vskip 0.6\baselineskip
}   

% end additions

\SetTEIxmlExport{autopar=false}

\NewDocumentEnvironment{tlg}{O{}}{
  \begin{ekdverse}}{
  \end{ekdverse}
  \vskip 0.6\baselineskip}

% additions/changes 2022-08-22 mm:
\NewDocumentEnvironment{alttlg}{O{}}{
%  \stopvline
%  \addtocounter{saved@poemline}{-1}
%  \setvnum{\hindsection.\arabic{saved@poemline}*\arabic{poemline}}
%  \selectlanguage{iast}
  \begin{ekdverse}[type=altrecension]
    \color{gray}
  }{
  \end{ekdverse}
  \vskip 0.6\baselineskip
%  \addtocounter{saved@poemline}{1}
%  \startvline
%  \setvnum{\hindsection.\arabic{poemline}}
%  \selectlanguage{iast}
}

% additions/changes 2022-08-22 mm:
\def\startaltrecension{
  \stopvline
  \addtocounter{saved@poemline}{-1}
  \setvnum{\hindsection.\arabic{saved@poemline}*\arabic{poemline}}
	%\selectlanguage{iast}
	%\begin{ekdverse}[type=altrecension]
	%\color{gray}
	\small  % added 2023-10-12 MD
	}
\def\endaltrecension{
	%\end{ekdverse}
	%\vskip 0.75\baselineskip
  \addtocounter{saved@poemline}{1}
  \startvline
  \setvnum{\hindsection.\arabic{poemline}}
%  \selectlanguage{iast}
	\normalsize  % added 2023-10-12 MD
	}

\def\startaltnormal{
	\stopvline
	\addtocounter{saved@poemline}{-1}
	\setvnum{\hindsection.\arabic{saved@poemline}*\arabic{poemline}}}
\def\endaltnormal{\endaltrecension}



\NewDocumentCommand{\tl}{m}{#1}

%%%%%%

\def\startverse{\begin{ekdverse}} % übergangsweise
\def\endverse{\end{ekdverse}\vskip 0.6\baselineskip} % übergangsweise
\def\startgray{\color{gray}} % NEW! 2023-06-16
\def\endgray{\color{black}} % NEW! 2023-06-16


%%%%%%

\newcommand{\myfn}[1]{\footnote{\texteng{#1}}}
\renewcommand{\thefootnote}{\texteng{\arabic{footnote}}}
\newcommand{\devnote}[1]{\textdev{\scriptsize #1}}
%\newcommand{\outdent}{\hspace{-\vgap}}
\newcommand{\sgwit}[1]{{\footnotesize (\getsiglum{#1})}}
\newcommand{\NotIn}[1]{\texteng{\footnotesize (om. \getsiglum{#1})}}
\newcommand{\lineom}[2]{\texteng{\footnotesize (#1 om. \getsiglum{#2})}}
\newcommand{\grau}[1]{\textcolor{gray}{#1}} % partial altrecension
\newcommand{\graus}[1]{\small\textcolor{gray}{#1}\normalsize} % partial altrecension
\newcommand{\Anm}[1]{\begin{ekdverse}
	\texteng{\footnotesize (#1)}
	\end{ekdverse}
	\vskip 0.6\baselineskip}
\newcommand{\anm}[1]{\texteng{\footnotesize [#1]}}

\def\om{\texteng{\emph{om.\kern-0.8ex}}}
\def\illeg{\texteng{\emph{illeg.\kern-0.8ex}}} 
\def\damaged{\texteng{\emph{damaged}}} 
\def\unm{\texteng{\emph{unm.\ }}}
\def\gap{\texteng{\emph{gap}}}
%\def\recte{\texteng{r.\:}}
%\def\for{\texteng{for\ }}
%\def\sic{\texteng{\emph{sic}}}
%\def\oder{\texteng{\emph{or\ }}}
\def\ante{\texteng{\normalfont\emph{ante\ }}}
\def\add{\texteng{\normalfont\emph{add.}}}
\def\post{\texteng{\normalfont\emph{post\ }}}
\def\antecorr{\texteng{\textsubscript{ac}}}
\def\postcorr{\texteng{\textsubscript{pc}}}
\def\marma{\texteng{\textsuperscript{\#}}}
\def\marmas{\texteng{\textsuperscript{\#}} }
\def\crux{\texteng{\textsuperscript{\textdagger}}}

\newcommand{\teimute}[1]{#1}

\usepackage{textgreek}

%%% Gr1,4b,6
\DeclareWitness{N3}{\texteng{\textalpha\textsubscript{1}}}{NGMPP B 62-20}[]
        \DeclareHand{N3ac}{N3}{\texteng{\textalpha\rlap{\textsubscript{1}}\textsuperscript{ac}}}[]
        \DeclareHand{N3pc}{N3}{\texteng{\textalpha\rlap{\textsubscript{1}}\textsuperscript{pc}}}[]
\DeclareWitness{J5}{\texteng{\textalpha\textsubscript{2}}}{Jodhpur 02235}[]
\DeclareWitness{G4}{\texteng{\textalpha\textsubscript{3}}}{GOML 18885}[]% Telugu script
\DeclareWitness{N24}{\texteng{\textalpha\textsubscript{4}}}{NGMPP G 190-16}[]
\DeclareWitness{Gr1r}{\texteng{\textAlpha *}}{Gr1 reconstructed}[]

\DeclareWitness{P11}{\texteng{\textbeta\textsubscript{1}}}{}[]
\DeclareWitness{C6}{\texteng{\textbeta\textsubscript{2}}}{Lalchand M-2089}[]

\DeclareWitness{V3}{\texteng{\textbeta\textsubscript{\textomega}}}{Sampurnananda Library Sarasvati Bhavan 29899}[]

%%% Gr2

\DeclareWitness{N23}{\texteng{\textgamma\textsubscript{1}}}{NGMPP G 25-2}[]
        \DeclareHand{N23ac}{N23}{\texteng{\textgamma\rlap{\textsubscript{1}}\textsuperscript{ac}}}[]
        \DeclareHand{N23pc}{N23}{\texteng{\textgamma\rlap{\textsubscript{1}}\textsuperscript{pc}}}[]
\DeclareWitness{J7}{\texteng{\textgamma\textsubscript{2}}}{Jodhpur 02241}[]
%\DeclareWitness{V6}{\texteng{V\textsubscript{6}}}{Sampurnananda Library Sarasvati Bhavan 29991}[]
\DeclareWitness{K1}{\texteng{K\textsubscript{1}}}{Raghunātha Temple Library 4383}[settlement=Jammu]
        \DeclareWitness{K1ac}{\texteng{K\rlap{\textsubscript{1}}\textsuperscript{ac}\space}}{}[]
        \DeclareWitness{K1pc}{\texteng{K\rlap{\textsubscript{1}}\textsuperscript{pc}\space}}{}[]


%%% Gr3

\DeclareWitness{V19}{\texteng{\textdelta\textsubscript{1}}}{Sampurnananda Library Sarasvati Bhavan 30069}[]
\DeclareWitness{K3}{\texteng{\textdelta\textsubscript{2}}}{Privat collection}
\DeclareWitness{C7}{\texteng{\textdelta\textsubscript{3}}}{Lalchand M-6494}[]
%\DeclareWitness{C1}{\texteng{C\textsubscript{1}}}{Lalchand M-2080}[]%L1 And C1 very close (and come from same region)
%\DeclareWitness{P23}{\texteng{P\textsubscript{23}}}{}[]
%\DeclareWitness{L1}{\texteng{L\textsubscript{1}}}{SOAS RE 43454}[settlement=Jammu]

\DeclareWitness{J6}{\texteng{\textdelta\textsubscript{\textomega}}}{Jodhpur 02237}[]
        \DeclareHand{J6ac}{J6}{\texteng{\textdelta\rlap{\textomega}\textsuperscript{ac}}}[]
        \DeclareHand{J6pc}{J6}{\texteng{\textdelta\rlap{\textomega}\textsuperscript{pc}}}[]

%%% Gr4c

\DeclareWitness{P15}{\texteng{\textepsilon\textsubscript{1}}}{}[]
\DeclareWitness{N19}{\texteng{\textepsilon\textsubscript{2}}}{NGMPP E-1528-1 / E-1527-7(4)}[]
\DeclareWitness{V15}{\texteng{\textepsilon\textsubscript{3}}}{Sampurnananda Library Sarasvati Bhavan 30051}[]
        \DeclareHand{V15ac}{V15}{\texteng{\textepsilon\rlap{\textsubscript{3}}\textsuperscript{ac}}}[]
        \DeclareHand{V15pc}{V15}{\texteng{\textepsilon\rlap{\textsubscript{3}}\textsuperscript{pc}}}[]
\DeclareWitness{J11}{\texteng{\textepsilon\textsubscript{4}}}{Jodhpur 23532}[]
        \DeclareHand{J11ac}{J11}{\texteng{\textepsilon\rlap{\textsubscript{4}}\textsuperscript{i.t.}}}[]
        \DeclareHand{J11pc}{J11}{\texteng{\textepsilon\rlap{\textsubscript{4}}\textsuperscript{mg.}}}[alternative reading written by the first hand in margin or interlinearly (J11)]
%\DeclareWitness{J14}{\texteng{\textepsilon\textsubscript{5}}}{Jodhpur 02239}[]

%\DeclareWitness{L2}{\texteng{L\textsubscript{2}}}{Wellcome Collection O.36]}
\DeclareWitness{M1}{\texteng{M\textsubscript{1}}}{P-5682/4}[]

\DeclareWitness{N26}{\texteng{\textepsilon\textsubscript{\textomega}}}{NGMPP}[]
%\DeclareWitness{V17}{\texteng{\textepsilon\textsubscript{\textomega 3}}}{Sampurnananda Library Sarasvati Bhavan 30053}[]

\DeclareWitness{V1}{\texteng{\texteta\textsubscript{1}}}{Sampurnananda Library Sarasvati Bhavan 30109}[]
        \DeclareHand{V1ac}{V1}{\texteng{\texteta\rlap{\textsubscript{1}}\textsuperscript{ac}}}[]
        \DeclareHand{V1pc}{V1}{\texteng{\texteta\rlap{\textsubscript{1}}\textsuperscript{pc}}}[]

%%% Gr4d

\DeclareWitness{J10}{\texteng{\texteta\textsubscript{2}}}{MSPP Jodhpur 2230}[]
        \DeclareHand{J10ac}{J10}{\texteng{\texteta\rlap{\textsubscript{2}}\textsuperscript{ac}}}[]
        \DeclareHand{J10pc}{J10}{\texteng{\texteta\rlap{\textsubscript{2}}\textsuperscript{pc}}}[]

\DeclareWitness{N9}{\texteng{\texteta\textsubscript{\textomega}}}{NGMPP A62-33}[]
%\DeclareWitness{J15}{\texteng{\textepsilon\textsubscript{\textomega 4}}}{Jodhpur 9732A}[]

%%%

\DeclareWitness{Jyo}{\texteng{\textchi}}{Brahmānanda's version}[]
%\DeclareWitness{Tue}{\texteng{Tü}}{Ma I 339}[]

\DeclareWitness{ceteri}{\texteng{cett.}}{ceteri}[]

%%% Group Sigla

\DeclareWitness{Gr1}{\texteng{\textAlpha}}{N3,J5,G4}

\DeclareWitness{Gr2}{\texteng{\textGamma}}{N23,J7}
%\DeclareWitness{Gr2}{\texteng{%
%	\textbeta\textsubscript{1}%
%	\textbeta\textsubscript{2}%
%	}}{N23,J7}
\DeclareWitness{Gr3a}{\texteng{\textDelta}}{V19,K3,C7}
\DeclareWitness{Gr4b}{\texteng{%
	\textbeta\textsubscript{1}%
	\textbeta\textsubscript{2}%
	}}{C6,P11}
\DeclareWitness{GrB}{\texteng{%
	\textbeta\textsubscript{1}%
	\textbeta\textsubscript{2}%
	\textbeta\textsubscript{\textomega}%
	}}{C6,P11,V3}
\DeclareWitness{Gr4c}{\texteng{\textEpsilon}}{P15,N19,V15}

% \DeclareWitness{Gr4d}{\texteng{%
	% \texteta\textsubscript{1}%
	% \texteta\textsubscript{2}%
	% }}{V1,J10}
\DeclareWitness{Gr6}{\texteng{\textOmega}}{V3,J6,N9,N26}

\makepagestyle{HPed}
\makeoddhead{HPed}{\small\texteng{HP3 edition}}{}{\small\texteng{\today}}
\makeevenhead{HPed}{\small\texteng{HP3 edition}}{}{\small\texteng{\today}}
\makeoddfoot{HPed}{}{\small\texteng{\thepage}}{}
\makeevenfoot{HPed}{}{\small\texteng{\thepage}}{}
\def\hindsection{3}

% N3,(P11,)C6,V3, N23,J7, V19,K3(except Vajrolī),C7, P15(up to 13a),N19,V15,J11(3.49-66 only) V1,J10,Jyo
% N26,J6,N9 (for the Khecaryabhyāsakrama only)
% not included: C1(lost),P23,P11(errorneus),N26
% discarded: C6,C8,N17,J11,J15


\begin{document}
\pagestyle{HPed}
\begin{otherlanguage}{iast}
\begin{ekdosis}

\begin{tlg}[hp03_001]
\tl{
\pada{\app{\lem[wit={ceteri}]{saśaila}
		\rdg[wit={V3}]{saśaile}
		\rdg[wit={K3},alt={\om}]{\skp{\om}}}% lost K3
	\app{\lem[wit={ceteri}]{vana}
		\rdg[wit={N23}]{vane}
		\rdg[wit={K3},alt={\om}]{\skp{\om}}}% lost K3
	\app{\lem[wit={ceteri}]{dhātrīṇāṃ}
		\rdg[wit={C6}]{dhātṝṇāṃ}
		\rdg[wit={K3},alt={\om}]{\skp{\om}}}} % lost K3
\pada{\app{\lem[wit={ceteri}]{yathādhāro}
		\rdg[wit={K3},alt={\om}]{\skp{\om}}} % lost K3
	\app{\lem[wit={ceteri}]{'hināyakaḥ}
		\rdg[wit={J7}]{himālayaḥ}
		\rdg[wit={K3},alt={\om}]{\skp{\om}}}/} \\+}
\tl{
\pada{sarveṣāṃ  \app{\lem[wit={ceteri}]{yoga}
		\rdg[wit={C6,V3}]{haṭha}
		\rdg[wit={K3},alt={\om}]{\skp{\om}}}%
	\app{\lem[wit={ceteri},postwit=\texteng{(\getsiglum{K3}\,\textsubscript{s.l.})}]{tantrāṇāṃ}
		\rdg[wit={K3},postwit=\texteng{\textsubscript{i.t.}}]{śāstrāṇā}}}
\pada{tathādhāro hi kuṇḍalī//}\\!} %3.1
\end{tlg}

\begin{tlg}[hp03_002]
\tl{
\pada{suptā guruprasādena} % prasānena N23
\pada{\app{\lem[wit={N3,P11,V3,V15,J10,Jyo}]{yadā jāgarti kuṇḍalī}
		\rdg[wit={C6,P15,N19,V1}]{yathā jāgarti kuṇḍalī}
		\rdg[wit={Gr2,Gr3a}]{bodhitā sukhadā bhavet}}/}\\+}
\tl{
\pada{\app{\lem[wit={N3,C6,V3,Gr2,V15,Jyo}]{tadā}% +J17
		\rdg[wit={Gr3a,P15,N19,V1,J10}]{tathā}}
	\app{\lem[wit={ceteri}]{sarvāṇi padmāni}
		\rdg[wit={J10}]{padmāni sarvāṇi}
		\rdg[wit={V19}]{pi sarvapadmāni}}}
\pada{bhidyante granthayo'pi ca//}\\!}  %3.2 % bidyante K3; grathīyā N23, gaṃthiyo N3
\end{tlg}

\begin{tlg}[hp03_003]
\tl{
\pada{\app{\lem[wit={ceteri}]{prāṇasya}
		\rdg[wit={K3,N19}]{praṇamya}
		\rdg[wit={C6}]{prāṇa}}
	śūnya% śūnyā J7, śunya N3
	\app{\lem[wit={ceteri}]{padavī}
		\rdg[wit={K3,N19,V15,V1}]{padavīṃ}}}\myfn{\getsiglum{P15,N19,V15} jump to \devnote{śūnyapadavī} in the next verse.} % padavīṃ V1?
\pada{\app{\lem[wit={N3,V3,Gr2,J10,Jyo}]{tadā}
		\rdg[wit={C6,Gr3a}]{tathā}
		\rdg[wit={V1}]{yathā}}
rāja\app{\lem[wit={ceteri}]{pathāyate}% raja° C6
		\rdg[wit={V1}]{padāyate}}/}\\+}
\tl{
\pada{\app{\lem[wit={N3,C6,V3,J7,J10,Jyo}]{tadā}% +C6pc
		\rdg[wit={V19,C7,V1}]{tathā} % +C6ac; tathac V19
		\rdg[wit={K3}]{yathā}
		\rdg[wit={N23}]{yadā}}
	cittaṃ nirālambaṃ} % °lambhaṃ K3
\pada{\app{\lem[wit={N3,C6,V3,Gr2,J10,Jyo}]{tadā}
		\rdg[wit={Gr3a,V1}]{tathā}}
		kālasya vañcanam//}
		\lineom{bcd}{P15,N19,V15} \anm{eye-skip}\\!}  %3.3
\end{tlg}

\begin{ava}[hp03_004]
\app{\lem[wit={N3,C6,Gr2,K3,C7}]{śūnyapadavīti kim} % +J17
		\rdg[wit={J10}]{atha śūnyapadavīm iti kim ucyate}}/
	\NotIn{V3,V19,P15,N19,V15,V1,Jyo}
%	\sgwit{N3,C6,Gr2,K3,C7,J10}
\end{ava}

\begin{tlg}[hp03_004]
\tl{
\pada{suṣumṇā śūnyapadavī} % pavī K3, padavīva N3; sukhu° N23
\pada{brahma\app{\lem[wit={N23,V1}]{randhra}
		\rdg[wit={ceteri}]{randhraṃ}}%
	mahā\app{\lem[wit={P15,V1,J10,Jyo}]{pathaḥ}
		\rdg[wit={J5,C6,V3,Gr2,Gr3a,N19}]{pathaṃ}
		\rdg[wit={N3,V15}]{pathāḥ}}/}\\+}
\tl{
\pada{\app{\lem[wit={N3,C6,V3,P15,N19,V15,V1,J10,Jyo}]{śmaśānaṃ}
		\rdg[wit={V19}]{śmaśāne} % C8ac? °naṃ pc?
		\rdg[wit={J7,K3,C7}]{śmaśānī}% #
		\rdg[wit={N23}]{aiśānī}} śāmbhavī
	\app{\lem[wit={N3,Gr2,V19,C7,P15,V15,Jyo}]{madhya}
		\rdg[wit={C6,V3,V1,J10}]{madhyaṃ}
		\rdg[wit={N19}]{madhye}
		\rdg[wit={K3}]{mudrā}}}%
\pada{\app{\lem[wit={N3,C6,V3,Gr2,V19,P15,V15,Jyo}]{mārgaś cety eka} % māgraśceteka P15, māga° V15; catye°? V19,
		\rdg[wit={V1}]{mārgeś cety eka}
		\rdg[wit={N19}]{mārgapratyeka}
		\rdg[wit={K3,C7}]{mārgaḥ śūnyeva}}%
	\app{\lem[wit={J7,J10,Jyo}]{vācakāḥ}
		\rdg[wit={N23}]{vācakā}
		\rdg[wit={N3,C6}]{vācakaḥ}
		\rdg[wit={V3}]{vācaka}
		\rdg[wit={J5,Gr3a,P15,N19,V15,V1}]{vācakam}}//}
		\lineom{a}{P15,N19,V15}\\!}  %3.4
\end{tlg}

\begin{tlg}[hp03_005]
\tl{
\pada{tasmāt sarvaprayatnena}
\pada{\app{\lem[wit={N3,Gr2,P15,V15,J10,Jyo},alt={prabodhayitum}]{prabodhayitu\skp{m}} % °yītum N3
		\rdg[wit={C6,V3,N19,V1}]{prabodhayatum}
%		\rdg[wit={J8}]{prabodhayatim}
		\rdg[wit={Gr3a}]{tāṃ bodhayituṃ}}%
	\app{\lem[wit={ceteri},alt={īśvarīṃ}]{\skm{m }īśvarīm}
		\rdg[wit={V3,N23,N19}]{īśvarī}% °rīṃ J8
		\rdg[wit={V19}]{īśvaraṃ}}/}\\+}
\tl{
\pada{brahma\app{\lem[wit={ceteri}]{dvāra}
		\rdg[wit={P15,N19}]{dvāre}}%
	\app{\lem[wit={ceteri}]{mukhe}
		\rdg[wit={N23}]{mukha}
		\rdg[wit={P15}]{sukhe}}
	\app{\lem[wit={ceteri}]{suptāṃ}
		\rdg[wit={V3}]{supto}}}
\pada{mudrā\app{\lem[wit={N3,P11,V3,P15,N19,V15,V1,J10,Jyo}]{bhyāsaṃ samācaret}% bhyāsa V3; samabhyaset J14
		\rdg[wit={C6,Gr2,Gr3a}]{bhyāsena bodhayet}}//}\\!}  %3.5
\end{tlg}


\newpage
\begin{tlg}[hp03_006]
\tl{
\pada{mahāmudrā mahābandho} % mudrāṃ N3
\pada{mahāvedhaś ca khecarī/}\\+} % °vedhoś N19
\tl{
\pada{\app{\lem[wit={N23,K3,C7,J10}]{uḍḍiyānaṃ}
		\rdg[wit={C6,V3}]{uḍiyānaṃ}% +J17
		\rdg[wit={N3pc,J7,N19}]{uḍḍīyāṇaṃ}
		\rdg[wit={N3ac}]{uḍḍīyānaṃ}
		\rdg[wit={V1}]{uḍḍīyāṇo}
		\rdg[wit={Jyo}]{uḍyānaṃ}
		\rdg[wit={P15,V15}]{uḍyāṇa}
		\rdg[wit={V19},alt={\om}]{\skp{\om}}}
  mūla\app{\lem[wit={Gr2,K3,C7},alt={mūlabandho}]{bandho}% muḷa° V15 % ra-vipulā
		\rdg[wit={C6}]{°bandhas}
		\rdg[wit={J10}]{°bandhaḥ}
		\rdg[wit={N3,J5}]{°bandhaṃ}
		\rdg[wit={V3}]{°bandha}
		\rdg[wit={N19,V15,Jyo}]{°bandhaś ca}
		\rdg[wit={V1}]{°bandhāś ca}
		\rdg[wit={V19},alt={\om}]{\skp{\om}}}}
\pada{\app{\lem[wit={J7,Gr3a,N19,V1,Jyo}]{bandho}% + 3 testimonies
		\rdg[wit={N23}]{bandhā}
		\rdg[wit={N3,C6,V3,P15,V15,J10}]{tato}} % ##
	\app{\lem[wit={ceteri}]{jālandharā}
		\rdg[wit={N23}]{jāladharā}
		\rdg[wit={V1}]{jālaṃjarā}}bhidhaḥ//}
		\lineom{bc}{V19}\\!}  %3.6
\end{tlg}


\begin{tlg}[hp03_007]
\tl{
\pada{karaṇī  % karanī N23
	\app{\lem[wit={ceteri}]{viparītākhyā}
		\rdg[wit={N19}]{viparītākhyaṃ}
		\rdg[wit={P15}]{viparītāni}}}
\pada{\app{\lem[wit={ceteri}]{vajrolī} % varjrolī N23, vajroli V15, varjālī P15, vajrālī N3, vajolī N19
		\rdg[wit={V19}]{vajro}} śakticālanam/} \\+}
\tl{
\pada{\app{\lem[wit={N3,J5}]{idaṃ mudrādi}
		\rdg[wit={P15,N19}]{idaṃ tu mudrā}
		\rdg[wit={V3,V15,V1,J10}]{idaṃ ca mudrā}
		\rdg[wit={Jyo}]{idaṃ hi mudrā}
		\rdg[wit={C6,Gr2,Gr3a}]{etad dhi mudrā}% etadvi? N23
		}daśakaṃ}
\pada{jarā\app{\lem[wit={ceteri}]{maraṇa}
		\rdg[wit={V3}]{marṇavi}
		\rdg[wit={N23}]{maṇa}}%
	\app{\lem[wit={ceteri}]{nāśanam}
		\rdg[wit={Gr3a}]{varjitaṃ}}//}\\!}  %3.7 % \lineom{cd}{P11}!!
\end{tlg}


\begin{tlg}[hp03_008]
\tl{
\pada{\app{\lem[wit={ceteri}]{ādinātho}
		\rdg[wit={V19,C7}]{ādīśvaro}
		\rdg[wit={K3}]{ādyeśvaro}}ditaṃ
	\app{\lem[wit={ceteri},alt={divyam}]{divya\skp{m}}
		\rdg[wit={J10}]{sarvaṃ}}m}
\pada{aṣṭaiśvarya\app{\lem[wit={ceteri}]{pradāyakam}% +P11
	\rdg[wit={C6}]{phalapradaṃ}}/}% aṣṭaiḥ N3
	\myfn{In \getsiglum{V15} this hemistich is found after pādas ab of the next verse.} \\+}
\tl{
\pada{vallabhaṃ sarva% sarve J17
	\app{\lem[wit={ceteri}]{siddhānāṃ}% +J5
		\rdg[wit={N3,V3}]{siddhīnāṃ}
		\rdg[wit={V15}]{vidyānāṃ}}}
\pada{durlabhaṃ marutām api//}\\!}  %3.8
\end{tlg}

\begin{tlg}[hp03_009]
\tl{
\pada{gopanīyaṃ prayatnena} \pada{yathā ratnakaraṇḍakam/} \\+}
\tl{
\pada{kasyacin naiva % kasyacci V3
	\app{\lem[wit={ceteri}]{vaktavyaṃ}
		\rdg[wit={V1}]{vaktavyā}
		\rdg[wit={V3,P15}]{kartavyaṃ}}}
\pada{\app{\lem[wit={ceteri}]{kulastrīsurataṃ}
		\rdg[wit={V1}]{kulastrīṣu rataṃ}
		\rdg[wit={V3},post={\unm}]{kulastrīsukharataṃ}}
	\app{\lem[wit={ceteri}]{yathā}
		\rdg[wit={J10}]{tathā}}//}\\!}  %3.9
\end{tlg}


\startaltrecension
\begin{alttlg}[hp03_009_1]
\tl{
\pada{vajrolī
	\app{\lem[wit={C6}]{tv amarolī} % §§
		\rdg[wit={V3}]{amarolīś}
		\rdg[wit={J10}]{°r amaroliś}} ca}
\pada{\app{\lem[wit={C6,V3}]{sahajolī}
		\rdg[wit={J10}]{sahajolis}} tridhā
	\app{\lem[wit={C6}]{matāḥ}
		\rdg[wit={J10}]{mataḥ}
		\rdg[wit={V3}]{magaḥ}}/} \\+}
\tl{
\pada{\app{\lem[wit={V3}]{eteṣāṃ}% +J17
		\rdg[wit={C6}]{etāsāṃ} % ##?
		\rdg[wit={J10}]{eteṣā}} lakṣaṇaṃ
	\app{\lem[wit={J10}]{vakṣye}
		\rdg[wit={V3}]{vakṣe}}}
\pada{kartavyaṃ ca viśeṣataḥ//} \sgwit{C6,V3,J10}\\!}  % 3.9*1  NOT IN P11!
\end{alttlg}
\endaltrecension



%%%%%%%%%%%%%%
%\newpage
\begin{ava}[hp03_010]
	\app{\lem[wit={V3,N23,N19,Jyo}]{atha mahāmudrā}
		\rdg[wit={C6}]{tatha mahāmudrā}
		\rdg[wit={P11,V1,J10}]{tatra mahāmudrā}% ##
		\rdg[wit={P15}]{tatra mahāmudrā yathā}
		\rdg[wit={V15}]{atha tatra mahāmudrā}
		\rdg[wit={N3,J7,Gr3a},alt={\om}]{\skp{\om}}}/
\end{ava}

\begin{tlg}[hp03_010]
\tl{
\pada{pādamūlena vāmena} % °mūle<<na>> C8
\pada{\app{\lem[wit={ceteri}]{yoniṃ} %%% CHECK! MD
		\rdg[wit={N3,V3}]{yoni}% yoniṃ J8
		\rdg[wit={N19}]{yoniḥ}}
	\app{\lem[wit={ceteri}]{saṃpīḍya}
		\rdg[wit={P15,N19}]{pīḍya}} dakṣiṇam/}\\+} % °ṇāṃ N3
\tl{
\pada{\app{\lem[wit={ceteri}]{pādaṃ}
		\rdg[wit={J10}]{pāda}
		\rdg[wit={V3}]{padaṃ}
		\rdg[wit={Jyo}]{prasā°}}
	\app{\lem[wit={ceteri}]{prasāritaṃ}
		\rdg[wit={V3}]{prasaritaṃ}
		\rdg[wit={V1}]{prasāditaṃ}
		\rdg[wit={Jyo}]{°ritaṃ padaṃ}}
	\app{\lem[wit={J7,K3,C7,V15,V1,J10}]{dhṛtvā}
		\rdg[wit={N3,C6,V3,N23,V19,P15,N19,Jyo}]{kṛtvā}}}
\pada{karābhyāṃ
	\app{\lem[wit={ceteri},alt={pūrayen}]{pūraye\skp{n}}% so in Amaraugha
		\rdg[wit={K3}]{pūrayet}
		\rdg[wit={J10}]{dhārayen}
		\rdg[wit={Jyo}]{dhārayed}}%
	\app{\lem[wit={N3,P11,V3,N19},alt={mukhe}]{\skm{n }mukhe}% so in Amaraugha
		\rdg[wit={C6,Gr2,V19,C7,P15,V15,V1,J10}]{mukham}
		\rdg[wit={K3}]{sukham}
		\rdg[wit={Jyo}]{dṛḍhaṃ}}\marma//}\\!}  %3.10
\end{tlg}

\newpage
\begin{tlg}[hp03_011]
\tl{
\pada{\app{\lem[wit={ceteri}]{kaṇṭhe}% +J8
		\rdg[wit={V3,Gr3a,N19}]{kaṇṭha}}
	\app{\lem[wit={J5,J7,V19,P15,V15,J10,Jyo}]{bandhaṃ}
		\rdg[wit={C6,V3,N23,V1}]{bandha}
		\rdg[wit={N19}]{bandhaḥ}
		\rdg[wit={K3}]{bandhe}
		\rdg[wit={N3}]{budha}
		\rdg[wit={C7}]{madhye}} samāropya}
\pada{\app{\lem[wit={ceteri},alt={dhārayed}]{dhāraye\skp{d}}
		\rdg[wit={V19}]{dhānayed}}d vāyum ūrdhvataḥ/}\\+} % °ta N3
\tl{
\pada{\app{\lem[wit={ceteri}]{yathā}
		\rdg[wit={V1}]{pathi}} % prefer yathā ? And then read with the next line?
	\app{\lem[wit={N3,P11,V3,P15,V15,J10}]{daṇḍāhataḥ} % + J7pc; daṃḍohataḥ C8, °hata V3
		\rdg[wit={C6,Gr2,Gr3a,N19,V1,Jyo}]{daṇḍahataḥ}} sarpo}
\pada{\app{\lem[wit={ceteri}]{daṇḍākāraḥ} % kāraṃ J7ac, kāraḥ J7pc
		\rdg[wit={N19}]{daṇḍakāraḥ}}
	\app{\lem[wit={ceteri}]{prajāyate}
		\rdg[wit={V1}]{prayujyate}
		\rdg[wit={C7},postwit=\texteng{(lost up to 3.19c saṃsthāpya; one folio missing)},alt={\om}]{\skp{\om}}}//}\\!}  %3.11
\end{tlg}


\begin{tlg}[hp03_012]
\tl{
\pada{\app{\lem[wit={P11,J7,V19,K3,V15,V1,J10,Jyo}]{ṛjvībhūtā}
		\rdg[wit={C6}]{ṛjvībhūtvā}
		\rdg[wit={N3,V3}]{rujvībhūtvā}% °bhūtatathā śaktiḥ J5
		\rdg[wit={N19}]{rajvībhūtā}
		\rdg[wit={P15}]{vajrībhūtā}
		\rdg[wit={N23}]{ṛ\,\_\,bhūtrā}}
	\app{\lem[wit={ceteri}]{tathā}
		\rdg[wit={N19}]{yathā}}
	\app{\lem[wit={ceteri}]{śaktiḥ}
		\rdg[wit={V3,V19,N19,V1}]{śakti}}}
\pada{kuṇḍalī sahasā bhavet/}\\+} % kuṃḍali N23
\tl{
\pada{\app{\lem[wit={N3,C6,V3,Gr2,P15,N19,V1}]{tadāsau}
		\rdg[wit={V19,K3}]{tathāsau}
		\rdg[wit={V15,J10,Jyo}]{tadā sā}} % tada N17
	\app{\lem[wit={ceteri}]{maraṇā}% +J17
		\rdg[wit={P15}]{maraṇa}
		\rdg[wit={V1}]{maraṇī}
		\rdg[wit={V3}]{ramaṇā}
		\rdg[wit={J10}]{maṇā}}%
	\app{\lem[wit={ceteri}]{vasthā}
		\rdg[wit={J7,V19,K3,V1}]{vasthāṃ}
		\rdg[wit={P15}]{sthā}}}
\pada{\app{\lem[wit={ceteri}]{jāyate}
		\rdg[wit={P15}]{yāyate}
		\rdg[wit={Gr2,V19,K3}]{harate}}
	\app{\lem[wit={N3,C6,V3,V19,K3,V1,J10,Jyo}]{dvipuṭā}
		\rdg[wit={N23}]{dvipūtā}
		\rdg[wit={P15,N19}]{nṛpuṭā}
		\rdg[wit={V15}]{tripuṭā}
		\rdg[wit={J7}]{vapurā}}%
	\app{\lem[wit={N3,P11,V3,N19,J10}]{śritā}
		\rdg[wit={J7}]{śrayāṃ}
		\rdg[wit={V19,K3,Jyo}]{śrayā}
		\rdg[wit={N23}]{śrayī}
		\rdg[wit={V1}]{ā[śr]i\,..}
		\rdg[wit={P15}]{smṛtā}
		\rdg[wit={V15}]{sanāṃ}
		\rdg[wit={C6}]{hi sā}}\marma//}\\!}  %3.12
\end{tlg}

%\newpage
\begin{tlg}[hp03_013]
\tl{
\myfn{\getsiglum{V19,K3} have a different order for the following 4 verses: 16 \rightarrow\ 15 \rightarrow\ 13 \rightarrow\ 14.\\
\getsiglum{P15} is lost after \devnote{tataḥ śanaiḥ śanai}.}%
\pada{tataḥ
	\app{\lem[wit={ceteri}]{śanaiḥ śanair eva} % śanai(1) V3
		\rdg[wit={N23}]{śanaiḥ śanair yeca}}}
\pada{\app{\lem[wit={ceteri},alt={recayen}]{recaye\skp{n}}
		\rdg[wit={N19}]{recayan}}% °yana N19
	\app{\lem[wit={ceteri},alt={na tu}]{\skm{n }na tu} % recayet tanu! K3
		\rdg[wit={V3}]{na ca}
		\rdg[wit={Jyo}]{naiva}} vegataḥ/}\\+}
\tl{
\pada{\app{\lem[wit={ceteri}]{iyaṃ}
		\rdg[wit={V3}]{idaṃ}} khalu mahāmudrā}
\pada{mahā\app{\lem[wit={ceteri}]{siddhaiḥ}% siddhai V19
		\rdg[wit={N19,V15}]{siddhiḥ}}
	\app{\lem[wit={J5,Jyo}]{pradarśitā}
		\rdg[wit={N3}]{pradarśanā}
		\rdg[wit={ceteri}]{praśasyate}% prasaśyate V3
		\rdg[wit={N19}]{prajāyate}}//}\\!}  %3.13
\end{tlg}

\begin{tlg}[hp03_014]
\tl{
\pada{\app{\lem[wit={ceteri}]{mahā}
		\rdg[wit={J10}]{mahān}}%
	\app{\lem[wit={N3,C6,V3,V1,J10,Jyo}]{kleśādayo}
		\rdg[wit={J7}]{kleśā yato}
		\rdg[wit={N23}]{kleśa yato}
		\rdg[wit={P11}]{kleśāyatā}
		\rdg[wit={N19}]{kleśā yathā}
		\rdg[wit={V15}]{kleśa yathā}
		\rdg[wit={V19,K3}]{kleśā mahā}}\marmas
	\app{\lem[wit={ceteri}]{doṣā}
		\rdg[wit={J10,Jyo}]{doṣāḥ}
		\rdg[wit={J7}]{doṣa}}}
\pada{\app{\lem[wit={N3}]{hīyante}
		\rdg[wit={J5}]{hrīyaṃte}
		\rdg[wit={V3,J10,Jyo}]{kṣīyante}% +J14
		\rdg[wit={P11,C6,Gr2,V19,K3,V15,V1}]{jīryante} % +SouthIndMss; jīryate K3, jīyaṃte V15; jaryante G4
		\rdg[wit={N19}]{jāyante}}
		maraṇādayaḥ/}\\+} % °ādaya N19,V3
\tl{
\pada{mahā\app{\lem[wit={C6,V3,V15,J10,Jyo}]{mudrāṃ}
		\rdg[wit={V1}]{mudrā[ś]}
		\rdg[wit={N3,Gr2,N19}]{mudrā}
		\rdg[wit={V19,K3},postwit=\texteng{(Pādas c--d omitted)},alt={\om}]{\skp{\om}}}
	\app{\lem[wit={ceteri}]{ca} % +J5
		\rdg[wit={N3}]{tu}} % +K3
	\app{\lem[wit={ceteri}]{tenaiva}
		\rdg[wit={N23}]{tenai} % haplo
		\rdg[wit={V15}]{tenetāṃ}
		\rdg[wit={V19,K3},alt={\om}]{\skp{\om}}}}
\pada{vadanti \app{\lem[wit={ceteri}]{vibudho}
		\rdg[wit={J7}]{vividho}
		\rdg[wit={V19,K3},alt={\om}]{\skp{\om}}}ttamāḥ//} \lineom{cd}{V19,K3}\\!}  %3.14  % ttamaḥ N3, ttamā V3
\end{tlg}

%\newpage
\begin{tlg}[hp03_015]
\tl{
\pada{\app{\lem[wit={ceteri}]{candrāṅge} % °rṃge
		\rdg[wit={V1}]{cāndrāṅge}
		\rdg[wit={N19}]{candrāṃgaṃ}
		\rdg[wit={V19,K3}]{candrāṃśaṃ}}
	\app{\lem[wit={ceteri}]{tu}
		\rdg[wit={V3,J10}]{ca}} samabhyasya}
\pada{\app{\lem[wit={ceteri}]{sūryāṅge}
		\rdg[wit={V1}]{sūryāṅge°}
		\rdg[wit={N19}]{sūryāṃgaṃ}
		\rdg[wit={V19,K3}]{sūryāṃśaṃ}}
	\app{\lem[wit={N3,C6,V3,Gr2,J10,Jyo}]{punar abhyaset}
		\rdg[wit={V19,K3,N19,V15}]{tu samabhyaset}
		\rdg[wit={V1}]{°ṣu samabhyaset}}/}\\+}
\tl{
\pada{yāva\app{\lem[wit={N3,C6,V3,Gr2,N19,V1,Jyo},alt={tulyā}]{\skm{t }tulyā} % tulyāṃ V1?
		\rdg[wit={J10}]{saṃkhyā}
		\rdg[wit={V19,K3}]{tayor}
		\rdg[wit={V15},alt={\om}]{\skp{\om}}}
	\app{\lem[wit={ceteri},alt={bhavet}]{bhave\skp{t}}
		\rdg[wit={J7,V1}]{bhavat}
		\rdg[wit={V15},alt={\om}]{\skp{\om}}}%
	\app{\lem[wit={N3,C6,V3,Gr2,V1,Jyo},alt={saṃkhyā}]{\skm{t }saṃkhyā}
		\rdg[wit={N19}]{saṃkṣā}
		\rdg[wit={J10}]{tulyā}
		\rdg[wit={V19,K3}]{sāmyaṃ}
		\rdg[wit={V15},alt={\om}]{\skp{\om}}}}
\pada{tato mudrāṃ % mudrā C8
	\app{\lem[wit={ceteri}]{visarjayet}
		\rdg[wit={V19}]{visaryayet}
		\rdg[wit={V3}]{vivarjayet}
		\rdg[wit={V15},alt={\om}]{\skp{\om}}}//} \lineom{cd}{V15}\\!}  %3.15
\end{tlg}



\newpage
\begin{tlg}[hp03_016]
\tl{
\pada{\app{\lem[wit={ceteri}]{na hi pathyam apathyaṃ vā}
		\rdg[wit={J10}]{nāpathyaṃ na hi pathyaṃ ca}
		\rdg[wit={N19},post=\texteng{(3 akṣaras missing)}]{na hi madhyaṃ vā}}}
\pada{rasāḥ sarve'pi nīrasāḥ/}\\+} % rasā C6; nīrasā V3
\tl{
\pada{\app{\lem[wit={ceteri}]{api bhuktaṃ} % muktaṃ N23?
		\rdg[wit={N19,V15}]{ahimuktaṃ}} viṣaṃ
	\app{\lem[wit={ceteri}]{ghoraṃ}
		\rdg[wit={V1}]{khāraṃ}}}
\pada{\app{\lem[wit={ceteri},alt={pīyūṣam}]{pīyūṣa\skp{m}}
		\rdg[wit={V3}]{piyuṣam}}%
	\app{\lem[wit={ceteri},alt={iva}]{\skm{m }iva}
		\rdg[wit={K3}]{api}}
	\app{\lem[wit={ceteri}]{jīryate}
		\rdg[wit={Gr2,Jyo}]{jīryati}
		\rdg[wit={V19}]{jīrjyate}}//}\\!}
\end{tlg}

\begin{tlg}[hp03_017] %\NotIn{P11}
\tl{
\pada{kṣaya\app{\lem[wit={ceteri}]{kuṣṭha} % kṣayaṃ N19
		\rdg[wit={V1}]{kuṣṭhaṃ}}%
	\app{\lem[wit={ceteri}]{gudā}
		\rdg[wit={V19,N19,V15}]{mudā}}varta}%
\pada{\app{\lem[wit={ceteri}]{gulmājīrṇa}
		\rdg[wit={C6}]{gulmajīrṇa}
		\rdg[wit={Gr2}]{gulmaplīha}}%
	\app{\lem[wit={ceteri}]{purogamāḥ}
		\rdg[wit={V3}]{purogamā}
		\rdg[wit={V19,K3}]{jvarās tathā}}/}\\+}
\tl{
\pada{\app{\lem[wit={ceteri}]{tasya doṣāḥ} % doṣā V19,V3
		\rdg[wit={V1,J10}]{doṣāḥ sarve}}
		kṣayaṃ yānti} % jāṃti V19, yā[ṃ]ti N3
\pada{mahāmudrāṃ \app{\lem[wit={ceteri}]{tu yo'bhyaset}
		\rdg[wit={K3}]{yo bhyaset}
		\rdg[wit={C6,V15}]{ca yo bhyaset}
		\rdg[wit={V3}]{yomabhyaset}}//}\\!}  %3.17
\end{tlg}


\begin{tlg}[hp03_018] %\NotIn{P11}
\tl{
\pada{\app{\lem[wit={ceteri}]{kathiteyaṃ} % °ya V15
		\rdg[wit={J5,V3,N19}]{kathitoyaṃ}} mahāmudrā}
\pada{\app{\lem[wit={G4,V3,N19,V15,V1,J10,Jyo},post=\texteng{(nṛṇā \getsiglum{V15,V1})}]{mahāsiddhikarī nṛṇām}% +G4,N24,M3; keep!
		\rdg[wit={N3,C6,Gr2,V19,K3}]{jarāmṛtyuvināśinī} % +M1; °vināśanaṃ G7
		\rdg[wit={J5}]{nṛṇāṃ mṛtyuvināśinī}}\marma/}\\+}
\tl{
\pada{\app{\lem[wit={ceteri}]{gopanīyā}
		\rdg[wit={V3,N19}]{gopanīyaṃ}
		\rdg[wit={J10}]{gopanīyāṃ}} prayatnena}
\pada{na
	\app{\lem[wit={ceteri}]{deyā}
		\rdg[wit={V3}]{deyaṃ}}
	yasya kasyacit//}\\!}   % yasyā J7ac %3.18
\end{tlg}

%%%%%%%%%%%%%%
%\newpage
\begin{ava}[hp03_019]
\app{\lem[wit={ceteri}]{atha}
		\rdg[wit={Gr2,K3},alt={\om}]{\skp{\om}}}
	mahābandhaḥ/ % baṃdha N3,V3,N19
\end{ava}

\begin{tlg}[hp03_019]
\tl{
\pada{\app{\lem[wit={N3,V19,K3,V15,V1,Jyo}]{pārṣṇiṃ}
%		\rdg[wit={J8}]{pārṣmiṃ}
		\rdg[wit={C6,V3,J7,N19,J10}]{pārṣṇi}
		\rdg[wit={N23}]{yāṣi}}
	\app{\lem[wit={ceteri}]{vāmasya}
		\rdg[wit={J10}]{bhāgena}} pādasya}
\pada{yonisthāne % yonī V15, yoniḥ N19
	\app{\lem[wit={ceteri}]{niyojayet}
		\rdg[wit={N19}]{yojayet}}/}\\+}
\tl{
\pada{vāmorūpari saṃsthāpya}
\pada{\app{\lem[wit={ceteri}]{dakṣiṇaṃ}
		\rdg[wit={V3}]{dakṣaṇaṃ}} caraṇaṃ
	\app{\lem[wit={ceteri}]{tathā}%
		\rdg[wit={C7}]{tataḥ}}//}\myfn{%
		In \getsiglum{N19,V15} this and the following two hemistiches are found after \ref{III22}ab. Probably they were omitted by eye-skip due to \devnote{niyojayet} and inserted at a wrong place.}\\!} % V15 dhārayitvā; N19,J11 cālayitvā
\end{tlg}
		%3.19

%\newpage
\begin{tlg}[hp03_020]
\tl{
\pada{pūrayitvā
	\app{\lem[wit={N3,C6,Gr2,Gr3a}]{mukhe}
		\rdg[wit={V3,V1,J10,Jyo}]{tato}
		\rdg[wit={N19,V15}]{tathā}}
	\app{\lem[wit={ceteri}]{vāyuṃ}
		\rdg[wit={V3,Gr2}]{vāyu}}}
\pada{hṛdaye \app{\lem[wit={ceteri}]{cibukaṃ}% V19 corr. from cibuke tathā dṛḍhaṃ
		\rdg[wit={V15}]{sasvanaṃ}
		\rdg[wit={N19}]{svasanaṃ}}
	\app{\lem[wit={ceteri}]{dṛḍham}
		\rdg[wit={C6}]{tathā}}/}\\+} % V19 had both tathā (cancelled?) and dṛḍhaṃ
\tl{
\pada{\app{\lem[wit={N3,V3}]{nibhṛtya} % = Amaraugha
		\rdg[wit={C6}]{nibhṛtaṃ}
		\rdg[wit={N19,V15}]{nivṛtya}
		\rdg[wit={Gr2,V19,C7,V1,Jyo}]{niṣpīḍya} % niḥ° Gr2,j14
		\rdg[wit={K3}]{nipīḍya}
		\rdg[wit={J10}]{nikṣipya}} yonim ākuñcya}
\pada{mano madhye niyojayet//}\\!}  % nijojayet N23
\end{tlg}

\newpage
\begin{tlg}[hp03_021]
\tl{
\pada{\app{\lem[wit={C6,Gr2,V1,Jyo}]{dhārayitvā yathāśakti}
	\rdg[wit={P11,V3,V15,J10}]{dhārayitvā yathāśaktyā}
	\rdg[wit={N19}]{cālayitvā yathāśaktyā}
	\rdg[wit={N3,J5,Gr3a},post=\texteng{(tu for ca \getsiglum{K3})}]{recayec ca śanair eva}}}
\pada{\app{\lem[wit={P11,C6,V3,Gr2,N19,V15,V1,J10,Jyo}]{recayed anilaṃ śanaiḥ} % śanai V3
	\rdg[wit={N3,J5,Gr3a}]{mahābandho'yam ucyate}}//}\\+}
\tl{
\pada{savyāṅge \app{\lem[wit={N3,J5,J7,N19,V15,V1,J10}]{ca samabhyasya}% °sye N3
		\rdg[wit={Jyo}]{tu samabhyasya}
		\rdg[wit={P11,C6,V3}]{pūrvam abhyasya}
		\rdg[wit={N23},alt={\om}]{\skp{\om}}}}
\pada{\app{\lem[wit={C6,V15,V1},alt={dakṣiṇāṅge sam°}]{dakṣiṇāṅge sama}
		\rdg[wit={N3,J5,N19}]{dakṣāṅge ca sam°}% +G7,
		\rdg[wit={N23}]{sam°}
		\rdg[wit={V3,J7,Jyo}]{dakṣāṅge punar}% +M3,G4?
		\rdg[wit={P11}]{dakṣiṇāṅge punar}
		\rdg[wit={J10}]{dakṣiṇe punar}}bhyaset//}\label{III21}
		\lineom{cd}{Gr3a}\myfn{\getsiglum{N3,J5} have this hemistich after \ref{III23}.}\\!}
%		\sgwit{C6,V3,Gr2,N19,V15,J10,V1,Jyo}\\!
\end{tlg}

%\newpage
\begin{tlg}[hp03_022]
\tl{
\pada{\app{\lem[wit={ceteri}]{matam atra}
		\rdg[wit={V1}]{matam etat}
		\rdg[wit={V3}]{matāntare}
		\rdg[wit={J10}]{matārettamaṃtra}}
	\app{\lem[wit={ceteri}]{tu}
		\rdg[wit={Gr2}]{ca}} keṣāṃcit} % V3 ci
\pada{\app{\lem[wit={ceteri}]{kaṇṭha}
		\rdg[wit={J10}]{kaṇṭhe}}%
	\app{\lem[wit={ceteri}]{bandhaṃ}
		\rdg[wit={V3}]{bandha}
		\rdg[wit={N23}]{yaṃ}}
	\app{\lem[wit={P11,V3,N19,V15}]{visarjayet}
		\rdg[wit={V1,J10,Jyo}]{vivarjayet}
		\rdg[wit={C6,Gr2}]{tu varjayet}}/}
		\label{III22}\\+}
\tl{
\pada{\app{\lem[wit={C6}]{rājadantabilaṃ tatra}
		\rdg[wit={V3}]{rājadantabilaṃ jatra}
		\rdg[wit={N19,V15}]{rājadantabalaṃ haṃti}% rājaddaṃntabaḷaṃ V15
		\rdg[wit={Gr2}]{rājadantadvayaṃ tatra}
		\rdg[wit={V1,Jyo}]{rājadantasthajihvāyā(ṃ)}% ṃ om. V1
		\rdg[wit={J10}]{rājadantasya jihvāyāṃ}}\marma}
\pada{\app{\lem[wit={C6,N19,V15},alt={jihvayottambhayed}]{jihvayottambhaye\skp{d}}
		\rdg[wit={V3,Gr2}]{jihvayottaṃbhaved}
		\rdg[wit={V1}]{bandhaś ca staṃbhayed}
		\rdg[wit={J10,Jyo}]{bandhaḥ śasto bhaved}}%
	\app{\lem[wit={C6,V3,Gr2,N19,V15,Jyo},alt={iti}]{\skm{d }iti}
		\rdg[wit={J10}]{dhitaḥ}
		\rdg[wit={V1}]{dhi tat}}//}\myfn{In \getsiglum{N19,V15} the 2nd hemistich is found betweem 3.28 and 3.29.} % [MD: reference!]
		%\sgwit{P11,C6,V3,Gr2,N19,V15,V1,J10,Jyo}
		\NotIn{N3,J5,Gr3a}\\!} %3.22
\end{tlg}


\begin{tlg}[hp03_023]
\tl{
\pada{\app{\lem[wit={ceteri}]{ayaṃ}
		\rdg[wit={J5}]{asaṃ}
		\rdg[wit={N3}]{amuṃ}}
	\app{\lem[wit={ceteri}]{khalu}% +J8
		\rdg[wit={V3}]{ṣalu}
		\rdg[wit={V1,J10}]{kila}
		\rdg[wit={J5}]{yogī}
		\rdg[wit={N3}]{yoga}
		}
	mahā\app{\lem[wit={ceteri}]{bandho}
		\rdg[wit={J10}]{bandhaḥ}
		\rdg[wit={N3}]{bandhaṃ}}}
\pada{\app{\lem[wit={ceteri}]{mahā}
		\rdg[wit={N23}]{sahā}
		\rdg[wit={J10}]{sarva}}siddhi%
	\app{\lem[wit={ceteri}]{pradāyakaḥ} % +J5,G4,N24; dā  om. in V1
	\rdg[wit={N3}]{pradāyakaṃ}}/}\\+}
\tl{
\pada{kāla\app{\lem[wit={ceteri}]{pāśa} % pāsa V3
		\rdg[wit={N23}]{pāśaṃ}}%
		mahā\app{\lem[wit={ceteri}]{bandha}
		\rdg[wit={J5,N23}]{bandho}
		\rdg[wit={N19}]{baddho}}}%
\pada{\app{\lem[wit={ceteri}]{vimocana}
		\rdg[wit={V3}]{mocayec ca}}% J8ac moccaye
	\app{\lem[wit={ceteri}]{vicakṣaṇaḥ}
		\rdg[wit={V3}]{vicakṣaṇam}}//}\label{III23}\myfn{\getsiglum{Jyo} has a different verse order: \ref{III24}ab \rightarrow\ \ref{III23}abcd \rightarrow\ \ref{III24}cd.}
			\lineom{cd}{N3}\\!}  %3.23
\end{tlg}
%		\sgwit{C6,V3,Gr2,Gr3a,N19,V15,V1,J10,Jyo}



\begin{tlg}[hp03_024]
\tl{
\pada{ayaṃ \app{\lem[wit={ceteri}]{ca}
		\rdg[wit={J7,K3,C7,Jyo}]{tu}}
	sarvanāḍīnā}% nāḍīṣu J17
\pada{\app{\lem[wit={ceteri},alt={ūrdhvaṃ}]{\skm{m }ūrdhvaṃ}
		\rdg[wit={N3,V1,N23,Jyo}]{ūrdhva}}% -ṃ ū- V1
	\app{\lem[wit={N3,P11,V3,V15,V1,J10}]{gativibodhakaḥ}
		\rdg[wit={N19}]{gatinibodhakaḥ} % C8ac/pc difficult to read; °kāḥ N19
		\rdg[wit={Jyo}]{gatinirodhakaḥ}
		\rdg[wit={C6,Gr2,Gr3a}]{gamanabodhakaḥ}}/}\\+} % V3 bodhaka
\tl{
\pada{triveṇīsaṅgamaṃ dhatte} % °veṇīṃ, dhartte N23; °veṇi V15
\pada{kedāraṃ
	\app{\lem[wit={ceteri}]{prāpayen manaḥ} % prāpyate J5; mana V3
		\rdg[wit={V1}]{prāpaye naraḥ} % prāpnuyān naraḥ J14
		\rdg[wit={N19}]{prāpaye naraṃ}}\marma//}\label{III24}\\!}  %3.24
\end{tlg}


%%%%%%%%%%%%%%
\newpage

\begin{ava}[hp03_025]
\app{\lem[wit={C6,C7,N19}]{atha mahāvedhaḥ} % °vedha N19,V3
		\rdg[wit={Gr2,K3}]{mahāvedhaḥ}}/
		\sgwit{Gr2,K3,C7,N19} %\NotIn{V19,N3}
\end{ava}

\begin{tlg}[hp03_025]
\tl{
\pada{rūpalāvaṇya\app{\lem[wit={ceteri}]{saṃpannā}
		\rdg[wit={J10}]{saṃpannaṃ}
		\rdg[wit={N23}]{saṃpattī}
		\rdg[wit={V19}]{saṃyuktā}}}
\pada{yathā \app{\lem[wit={ceteri}]{strī puruṣaṃ}
		\rdg[wit={V19}]{nārī patiṃ}} vinā/}\\+}
\tl{
\pada{mahāmudrā%
	mahā\app{\lem[wit={C6,J7,Gr3a,Jyo}]{bandhau}
		\rdg[wit={N3,J5,P11,V3,N23,N19,J10}]{bandho}
		\rdg[wit={V1}]{bandha}
		\rdg[wit={V15},alt={\om}]{\skp{\om}}}}
\pada{\app{\lem[wit={J7,Gr3a,J10,Jyo}]{niṣphalau}
		\rdg[wit={C6,N23}]{niṣphalo}% niḥphalo N23
		\rdg[wit={J5}]{niṣkalaḥ}
		\rdg[wit={N3}]{niṣkalā}
		\rdg[wit={N19}]{mahābaṃdhaṃ} % °baṃdha N19
		\rdg[wit={V3,V1}]{mahāvedha(ṃ)}% +J14
		\rdg[wit={V15},alt={\om}]{\skp{\om}}} % °vedhaṃ J8
	\app{\lem[wit={C6,Gr2,Gr3a,Jyo}]{vedhavarjitau} % vetha N23
		\rdg[wit={J5,P11}]{vedhavarjitaḥ}
		\rdg[wit={N3}]{vedhavarttina}
		\rdg[wit={J10}]{vedhavarttitau}
		\rdg[wit={N19,V1}]{vinā tathā}% +J14
		\rdg[wit={V3}]{vinānyathā}
		\rdg[wit={V15},alt={\om}]{\skp{\om}}}//} \lineom{cd}{V15}\\!}  %3.25
\end{tlg}


\begin{postmula}[hp03_026a]
iti mahābandhaḥ/ \sgwit{V1}
\end{postmula}

\begin{altava}[hp03_026b]
atha mahāvedhaḥ/ \sgwit{V3,V15,J10,Jyo}
\end{altava}


%\newpage
\begin{tlg}[hp03_026]
\tl{
\pada{\app{\lem[wit={N3,N19,V15,V1,Jyo}]{mahābandha}
		\rdg[wit={J7},post=\texteng{(followed by a double daṇḍa and corrected to °vedhaḥ)}]{mahābandhaḥ}
		\rdg[wit={P11,N23}]{mahābandho}
		\rdg[wit={C6,Gr3a}]{mahāvedhe}
		\rdg[wit={V3,J10}]{mahāvedha}}%
	\app{\lem[wit={ceteri}]{sthito}
		\rdg[wit={N23}]{sthite}
		\rdg[wit={J10}]{sthitau}} yogī}
\pada{kṛtvā pūraka%
	\app{\lem[wit={C6,J7,V15,V1,Jyo},alt={ekadhīḥ}]{\skm{m }ekadhīḥ}
		\rdg[wit={N3}]{ekadhī}
		\rdg[wit={V19,N19}]{ekadhā}
		\rdg[wit={K3,C7}]{ekayā}
		\rdg[wit={N23}]{eva dhīḥ}
		\rdg[wit={V3}]{eva dhī}
		\rdg[wit={J5}]{eva ca dhā}}/}\\+}
\tl{
\pada{\app{\lem[wit={J7,Gr3a,V15,Jyo}]{vāyūnāṃ}
		\rdg[wit={V1}]{vāyunāṃ}
		\rdg[wit={N3,J5,C6,V3,N23,N19,J10}]{vāyunā}}
	\app{\lem[wit={ceteri}]{gatim āvṛtya}
		\rdg[wit={N3}]{gam āvṛtya}
		\rdg[wit={J5,V15}]{gatim ākṛṣya}}} % gatin? V15
\pada{nibhṛtaṃ\marmas kaṇṭha%
	\app{\lem[wit={ceteri}]{mudrayā}
		\rdg[wit={J10}]{mudrāyā}
		\rdg[wit={N23},postwit=\texteng{(jumped to pāda c after gatim)},alt={\om}]{\skp{\om}}}//}\\!}  %3.26
\end{tlg}

\begin{tlg}[hp03_027]
\tl{
\pada{\app{\lem[wit={ceteri}]{samahasta}
		\rdg[wit={N3}]{samahāsta}% nyastahasta J5,J14
		\rdg[wit={J10}]{samahastā}
		\rdg[wit={N23}]{samahaste}
		\rdg[wit={C6}]{samau hasta}}%
	\app{\lem[wit={V3,N23,V19,J10,Jyo}]{yugo}% yuga J5
		\rdg[wit={C6,J7,K3,C7,V15,V1}]{yugau}
		\rdg[wit={N3,N19}]{yuge}} bhūmau}
\pada{\app{\lem[wit={N3,P11,V3,J7,Gr3a,J10}]{sphijau}
		\rdg[wit={N23,Jyo}]{sphicau}
		\rdg[wit={V1}]{sphidau} % rather sphiṭṭau?
		\rdg[wit={C6}]{sphītau}
		\rdg[wit={N19}]{dvijāt}
		\rdg[wit={V15}]{dvijā}}
	\app{\lem[wit={ceteri},alt={saṃtāḍayec}]{saṃtāḍaye\skp{c}} % saṃtāḍanec N23; °yechanaiḥ J10
		\rdg[wit={V1}]{saṃ[c]ālayec}
		\rdg[wit={V15}]{nutāḍayec}}c chanaiḥ/}\\+} % V3 om. ḥ; chūnaiḥ N19
\tl{
\pada{\app{\lem[wit={ceteri}]{puṭadvayaṃ}
		\rdg[wit={N23}]{jaṃghāyuṭadvayam}}
	\app{\lem[wit={ceteri}]{samākramya}% +K3
		\rdg[wit={J5,J7,C7}]{samākṛṣya}% +J5,N24
		\rdg[wit={N23}]{ākṛṣya}
		\rdg[wit={Jyo}]{atikramya}}}
\pada{\app{\lem[wit={C6,J7,Gr3a,V1,J10,Jyo}]{vāyuḥ}
		\rdg[wit={N3,P11,V3,N23,N19,V15}]{vāyu}}
		sphurati % V3 sphuraṃti
	\app{\lem[wit={N3,J5,N19,J10}]{satvaraṃ}% +M1,M3,G7 ## ratvaraṃ P11
		\rdg[wit={V3}]{tatvaraṃ}
		\rdg[wit={V1}]{tatparaṃ}
		\rdg[wit={C6}]{tatparaḥ}
		\rdg[wit={Gr2,V19,V15,Jyo}]{madhyagaḥ}
		\rdg[wit={K3,C7}]{madhyamaḥ}
		}/}\\+}
\tl{
\graus{\pada{bandhenānena
	\app{\lem[wit={J7}]{yogīndraḥ}\rdg[wit={N23}]{yogīndra}}}
	\pada{sādhayet sarvam īpsitam//} \sgwit{Gr2}}\\!}  %3.27
\end{tlg}


\begin{tlg}[hp03_028]
\tl{
\pada{somasūryāgni\app{\lem[wit={N19,Jyo}]{saṃbandho}% M1
		\rdg[wit={V3,V1}]{sambandhā}
		\rdg[wit={N3,J5,C6,Gr2,J10}]{sambandhāj}
		\rdg[wit={Gr3a,V15}]{saṃdhānaṃ}}}\marmas
\pada{jāyate % jāyata? V15
	\app{\lem[wit={N3,P11,Jyo}]{cāmṛtāya vai}
		\rdg[wit={C6,Gr2,N19,V15}]{cāmṛtāyate}
		\rdg[wit={Gr3a}]{vāmṛtāyate}
		\rdg[wit={V1}]{cāmṛtāye vaiḥ}
		\rdg[wit={V3}]{ca mṛtāya vai}
		\rdg[wit={J10}]{ca mṛturjayaḥ}}\marma/}\\+}
\tl{
\pada{\app{\lem[wit={ceteri}]{mṛtāvasthā}
		\rdg[wit={N23}]{mṛtāmasthā}}
	\app{\lem[wit={ceteri}]{samutpannā}
		\rdg[wit={N23},alt={\om}]{\skp{\om}}}}
\pada{tato \app{\lem[wit={N3,C6,V3,J7,Gr3a,N19}]{mṛtyubhayaṃ kutaḥ}
		\rdg[wit={V15,V1,J10,Jyo}]{vāyuṃ virecayet} % °cayat V15
		\rdg[wit={N23}]{vāyuṃ nirundhayet kumbhakena}}\marma//}\\!}  %3.28
\end{tlg}

\newpage
\begin{tlg}[hp03_029]
\tl{
\pada{\app{\lem[wit={ceteri}]{mahāvedho}
		\rdg[wit={V15}]{mahābaṃdho}}%
		'ya\app{\lem[wit={N3,C6,V3,J7,K3,C7,J10,Jyo},alt={abhyāsān}]{\skm{m }abhyāsā\skp{n}} % V19 °vedhopamanabhyā°?
		\rdg[wit={N23}]{abhyāsāt}
		\rdg[wit={V19}]{anabhyāsān}
		\rdg[wit={V1}]{abhyāso}
		\rdg[wit={N19,V15}]{abhyasto}}}%
\pada{\app{\lem[wit={ceteri},alt={mahā}]{\skm{n }mahā} % K3 unclear
		\rdg[wit={N23}]{sarva}}siddhipradāyakaḥ/}\\+} % siddhiḥ J10pc; dāyaka V3
\tl{
\pada{\app{\lem[wit={ceteri}]{valī}% vaḷī V15
		\rdg[wit={J10},post={\unm}]{valīta}
		\rdg[wit={N23,V1}]{vali}}%
	\app{\lem[wit={ceteri}]{palita}
		\rdg[wit={J7}]{palīta}}%
	\app{\lem[resp=emend,postwit=\texteng{(cf.\,C8)}]{roga}
		\rdg[wit={Gr3a}]{vega} % corrupt from roga? C8
		\rdg[wit={N3,C6,V3,Gr2,N19,V15,V1}]{vedha} % J7ac
		\rdg[wit={Jyo}]{vepa}
		\rdg[wit={J10}]{bandha} % J7pc
		}\marma%
	\app{\lem[wit={ceteri}]{ghnaḥ}
		\rdg[wit={N3,V3}]{ghnaṃ}
		\rdg[wit={N23}]{ghna}}}
\pada{sevyate % savyate N19, sevyato K3
	\app{\lem[wit={ceteri}]{sādhakottamaiḥ}
		\rdg[wit={V3}]{sādhakottamaṃ}}//}\\!}  %3.29
\end{tlg}


%\newpage
\begin{tlg}[hp03_030]
\tl{
\pada{\app{\lem[wit={ceteri}]{etat trayaṃ mahā}% °traya N23; V3,V19 etatrayaṃ; J10 sahā
		\rdg[wit={V15}]{mahāmudrātrayaṃ}
		\rdg[wit={V1},post={\unm}]{mahāmudrātrayatraṃ}}%
	\app{\lem[wit={N3,C6,V3,Gr2,N19,V15,V1,Jyo}]{guhyaṃ}
		\rdg[wit={Gr3a}]{guptaṃ}
		\rdg[wit={J10}]{mudrā}}}
\pada{jarāmṛtyu\app{\lem[wit={ceteri}]{vināśanam}
		\rdg[wit={J10}]{vināśinī}}/}\\+}
\tl{
\pada{vahnivṛddhikaraṃ % buddhivṛddhi C6
	\app{\lem[wit={N3,C6,Gr2}]{caiva}
		\rdg[wit={V3,N19,V15,J10}]{caivam}
		\rdg[wit={Jyo}]{caiva hy}
		\rdg[wit={Gr3a}]{caitad}
		\rdg[wit={V1}]{viśvam}}}
\pada{aṇimādi\app{\lem[wit={ceteri}]{guṇa}
		\rdg[wit={N19}]{gaṇa}}%
	\app{\lem[wit={ceteri}]{pradam}
		\rdg[wit={N23}]{pradī}}//}\\!}  %3.30
\end{tlg}

%\newpage
\begin{tlg}[hp03_031]
\tl{
\pada{aṣṭadhā kriyate % aṣṭādi C6
	\app{\lem[wit={N3,C6},alt={caitad}]{caita\skp{d}}% ## caihad G7?
		\rdg[wit={Gr3a,N19,Jyo}]{caiva}% +N24, taitva J5
		\rdg[wit={P11,Gr2}]{caivaṃ}% M3
		\rdg[wit={V3,V1,J10}]{caikaṃ}
		\rdg[wit={V15}]{caika}}}%
\pada{\app{\lem[wit={ceteri},alt={yāme yāme}]{\skm{d }yāme yāme}
		\rdg[wit={V15}]{yāmayāme}
		\rdg[wit={V1}]{yāmaṃ yamāṃ}} % +J14
		dine dine/}\\+}
\tl{
\pada{\app{\lem[wit={ceteri}]{puṇya}
		\rdg[wit={V15}]{puṇyaṃ}
		\rdg[wit={J10}]{sarva}}%
	\app{\lem[wit={N3,J7,Gr3a,N19,Jyo}]{saṃbhāra}
		\rdg[wit={V3}]{sahāra}
		\rdg[wit={V1,J10}]{saṃcāra}
		\rdg[wit={V15}]{saṃsāra}
		\rdg[wit={C6}]{saṃdhāna}
		\rdg[wit={N23},alt={\om}]{\skp{\om}}}%
	\app{\lem[wit={V3,Gr2,N19}]{sambhāvi}
		\rdg[wit={N3,J5}]{saṃbhāvī}
		\rdg[wit={V1}]{sabhāvī}
		\rdg[wit={J10}]{saṃdhāyī}
		\rdg[wit={C6,V15,Jyo}]{saṃdhāyi}
		\rdg[wit={Gr3a}]{saṃpādi}}}
\pada{\app{\lem[wit={ceteri}]{pāpaugha}
		\rdg[wit={J7}]{pāprogha}
		\rdg[wit={N23}]{padhau\,\_\,dhava}}%
		bhiduraṃ sadā//}\\!}  %3.31
\end{tlg}


\begin{tlg}[hp03_032]
\tl{
\pada{samyak\app{\lem[wit={ceteri},alt={śikṣāvatām}]{śikṣāvatā\skp{m}} % sikṣī V19; °catām N23
		\rdg[wit={C6}]{śikṣāvatā}
		\rdg[wit={J5,N19}]{śiṣyāvatām}
		\rdg[wit={J10}]{jijñāsatām}}%
	\app{\lem[wit={J5,P11,Gr2,Gr3a,V15},alt={eva}]{\skm{m }eva} % M3?
		\rdg[wit={N3,V3,N19,V1,J10,Jyo}]{evaṃ}
		\rdg[wit={C6}]{bhavyaṃ}}}
\pada{svalpaṃ % V1 svaplaṃ
	prathama\app{\lem[wit={J5,C6,N23,Gr3a,N19,V15,V1}]{sādhane}
		\rdg[wit={N3}]{sādhanaiḥ}
		\rdg[wit={V3,J7,J10,Jyo}]{sādhanaṃ}}/}\\+} % pratyama N23; J10ac sādhana?
\tl{
\pada{vahnistrīpatha% stri N19; paṭha N3
	\app{\lem[wit={ceteri},alt={sevānām}]{sevānā\skp{m}}
		\rdg[wit={N19}]{sevācanām}
		\rdg[wit={J10}]{sevanām}
		\rdg[wit={V1}]{sevanam}
		\rdg[wit={N23}]{sevenam}
		\rdg[wit={Jyo},alt={\om}]{\skp{\om}}}}% °nāṃmādau J7
\pada{m ādau varjana%
	\app{\lem[wit={N3,J5,C6,N19,V15,V1},alt={ādiśet}]{\skm{m }ādiśet}
		\rdg[wit={V3}]{ādṛśyet}
		\rdg[wit={Gr2,Gr3a,J10}]{ācaret}% +M3,G7
		\rdg[wit={Jyo},alt={\om}]{\skp{\om}}}//}\myfn{%
	\getsiglum{Gr2} adds here:
	\devnote{mahāmudrā mahābandho mahāvedhaś ca nityaśaḥ/ % °bandhā N23
	etat trayaṃ prayatnena caturvāraṃ karoti yaḥ/
	ṣaṇmāsābhyantare mṛtyuṃ jayaty eva na saṃśayaḥ//} % mṛtyu N23
	 (= Śivasaṃhitā xx)}
	\lineom{cd}{Jyo}\myfn{\getsiglum{Jyo} has this line just before 1.61.}\\!}  %3.32
\end{tlg}

%%%%%%%%%%%%%%%%%%%%%%%%
%\newpage
\begin{ava}[hp03_033]
\app{\lem[wit={ceteri}]{atha}
\rdg[wit={Gr2,K3},alt={\om}]{\skp{\om}}} khecarī/
\end{ava}

\startaltrecension
\begin{alttlg}[hp03_033_0] % MD Numbering problem
\tl{
\pada{nāsanaṃ siddhasadṛśaṃ} % proof that J8 is a copy of V3? Halanta of k in the above line was mistaken for e?(V3=9v4, J8=16r6)
\pada{na \app{\lem[resp=emend]{kumbhaṃ}
		\rdg[wit={V3,N26,N9}]{kumbha}
		\rdg[wit={J6}]{kuṃbhaka}}
	\app{\lem[resp=emend]{kevalopamam} % =J15
		\rdg[wit={V3,N9}]{kevalokanam}
		\rdg[wit={N26}]{sadṛśo nilaḥ}
		\rdg[wit={J6}]{samonilaṃ}}/}\\+}
\tl{
\pada{na khecarīsamā mudrā} \pada{na nādasadṛśo layaḥ//}
\sgwit{Gr6} \anm{= 1.43}\\!}
\end{alttlg}
\endaltrecension

\newpage
\begin{tlg}[hp03_033]
\tl{
\pada{\app{\lem[wit={V3,N23,V19,C7,J6,V15,V1,Jyo}]{chedana}% +J17
		\rdg[wit={J10,N26,N9}]{chedanaṃ}% +J8
		\rdg[wit={C6}]{chedanaiś}
		\rdg[wit={K3}]{khedana}
		\rdg[wit={N19}]{vedana}
		\rdg[wit={J7}]{rasanā}
		\rdg[wit={N3},alt={\damaged}]{\skp{\damaged}}}%
	\app{\lem[wit={J7,Gr3a,J6,N19,V15,V1,Jyo}]{cālanadohaiḥ}
		\rdg[wit={N23}]{cālajadohaiḥ}
		\rdg[wit={N26}]{cālanaṃ dohaṃ}
		\rdg[wit={V3,J10,N9}]{cālanaṃ dohau}
		\rdg[wit={C6}]{cālanair dāsyai}
		\rdg[wit={N3},alt={\damaged}]{\skp{\damaged}}}
\app{\lem[wit={C6,J7,N19,V1,Jyo}]{kalāṃ}
		\rdg[wit={N23}]{kalaṃ}
		\rdg[wit={N3,N26}]{kalāḥ}
		\rdg[wit={V3,N9}]{kalā}
		\rdg[wit={J10}]{kāla}
		\rdg[wit={Gr3a,J6}]{jihvāṃ} % jihvā C7, jihva V19
		\rdg[wit={V15}]{krameṇa}}
	\app{\lem[wit={Gr2}]{tu}
		\rdg[wit={Gr3a}]{vai}
		\rdg[wit={V15}]{jihvāṃ}
		\rdg[wit={N3,C6,V3,N19,V1,J10,N9,N26}]{krameṇa}
		\rdg[wit={Jyo}]{krameṇātha}
		\rdg[wit={J6},alt={\om}]{\skp{\om}}}
	\app{\lem[wit={Gr2,J6},alt={saṃvardhayet}]{saṃvardhaye\skp{t}}
		\rdg[wit={V15,V1,J10}]{pravardhayet}
		\rdg[wit={N3,C6,V3,Gr3a,N19,N26,N9,Jyo}]{vardhayet}}%
	\app{\lem[wit={C6,J7,Gr3a,J6,N19,V15,Jyo},alt={tāvat}]{\skm{t }tāvat}
		\rdg[wit={J10}]{tāt}
		\rdg[wit={N3,V3,N23,V1,N26,N9},alt={\om}]{\skp{\om}}}/}\\+}
\tl{
\pada{\app{\lem[wit={Gr2,Gr3a,J6}]{yāvad iyaṃ}% yāvaṃd iya
		\rdg[wit={N3,C6,V15,V1,J10,Jyo}]{sā yāvad}% yāva N3, pāvad J10
		\rdg[wit={N19}]{sā}
		\rdg[wit={V3,N26,N9}]{yāvad}}
	\app{\lem[wit={ceteri}]{bhrūmadhyaṃ} % bhū N23, bhṛ N3
		\rdg[wit={V19,V1}]{bhrūmadhya}
		\rdg[wit={N26}]{spṛśati}}
	\app{\lem[wit={ceteri}]{spṛśati}
		\rdg[wit={N23}]{sparśati}
		\rdg[wit={N3}]{viśa}
		\rdg[wit={N26}]{bhrūmadhyaṃ}}
\app{\lem[wit={Gr2,C7,J6,V15,J10,Jyo}]{tadā khecarīsiddhiḥ}
		\rdg[wit={N26}]{tadā sidhyati khecarī}
		\rdg[wit={N3,C6,V3,N9}]{tadānīṃ khecarīsiddhiḥ}% °nī siddhi C6
		\rdg[wit={N19}]{tadānīṃ hi khecarīsiddhiḥ}
		\rdg[wit={V1}]{tadānī siddhiḥ}
		\rdg[wit={V19}]{tadā khecarī bhavati}}//}\myfn{Based on \getsiglum{Gr2,Gr3a}. The metre is Upagīti.\\
	\getsiglum{N19,V15} (Gīti?):
		\devnote{chedanacālanadohaiḥ kalāṃ krameṇa (pra)vardhayet tāvat/
		sā yāvad bhrūmadhyaṃ spṛśati tadānīṃ hi khecarīsiddhiḥ//}\\
	\getsiglum{J10} (Āryā):
		\devnote{chedanacālanadohaiḥ kalāṃ krameṇa pravardhayet tāvat/
		sā yāvad bhrūmadhyaṃ spṛśati tadā khecarīsiddhiḥ//}\\
	\getsiglum{V3} (Anuṣṭubh):
		\devnote{chedanaṃ cālanaṃ dohau, kalākrameṇa vardhayet/
		yāvad bhrūmadhyaṃ spṛśati tadānīṃ khecarīsiddhiḥ(!)//}\\
	Perhaps to read:
	\devnote{chedanacālanadohaiḥ, kalāṃ kramād vardhayet tāvat/
	sā yāvad bhrūmadhyaṃ, spṛśati tadā khecarīsiddhiḥ//}
	}\\!}  %3.33
\end{tlg}

%\newpage
% Khecaryabhyāsakrama

\startaltrecension

\begin{alttlg}[hp03_033_01] % KhV 1.46
\tl{
\pada{\app{\lem[wit={J6,Jyo}]{snuhī}
		\rdg[wit={V15,N26,N9}]{snuhi}
		\rdg[wit={V3}]{śnuhi}}pattranibhaṃ śastraṃ} % putra J6ac
\pada{sutīkṣṇaṃ snigdhanirmalam/}\\+}
\tl{
\pada{samādāya tatas tena}
\pada{romamātraṃ \app{\lem[wit={V3,N26}]{samucchidet} % cci J15
		\rdg[wit={V15,N9,Jyo}]{samucchinet}
		\rdg[wit={J6}]{samucchiṃdyāt}}//} \sgwit{V15,Gr6,Jyo}\\!}
\end{alttlg}

\begin{alttlg}[hp03_033_02] % KhV 1.47
\tl{
\pada{\app{\lem[wit={V15,Gr6}]{kṛtvā}
		\rdg[wit={Jyo}]{tataḥ}}
	\app{\lem[wit={V3,N26,N9,J6}]{saindhavapathyādi}
		\rdg[wit={Jyo}]{saindhavapathyābhyāṃ}
		\rdg[wit={V15}]{saindhavapakṣyādi}}}%
\pada{cūrṇitābhyāṃ pragharṣayet/}\\+} % cūṇi° V15
\tl{
\pada{\app{\lem[wit={V15,V3,N9,J6,Jyo}]{punaḥ}
	\rdg[wit={N26}]{tataḥ}} saptadine prāpte}
\pada{romamātraṃ \app{\lem[wit={V3,N26,N9}]{samucchidet}
		\rdg[wit={Jyo}]{samucchinet}
		\rdg[wit={V15}]{punaḥ chidet}
		\rdg[wit={J6}]{samutthiyāt}}//} \sgwit{V15,Gr6,Jyo}\\!}
\end{alttlg}

\begin{alttlg}[hp03_033_03] % KhV 1.48
\tl{
\pada{evaṃ krameṇa \app{\lem[wit={V3,N26,J6,Jyo}]{ṣaṇmāsaṃ}
		\rdg[wit={N9}]{ṣaṇmāse}
		\rdg[wit={V15}]{ṣaṇmāsān}}}
\pada{\app{\lem[wit={V3,J6,V15,N26}]{nitya}
		\rdg[wit={Jyo}]{nityaṃ} % +J15
		\rdg[wit={N9}]{netya}}%
	\app{\lem[wit={V3,V15,N26,N9}]{yuktaṃ}
		\rdg[wit={Jyo}]{yuktaḥ}
		\rdg[wit={J6}]{muktaṃ}} samācaret/}\\+}
\tl{
\pada{\app{\lem[wit={Gr6,Jyo},alt={ṣaṇmāsād}]{ṣaṇmāsā\skp{d}}
		\rdg[wit={V15}]{ṣaṇmāse}}d
	rasanā\app{\lem[wit={V3,N26,N9,J6,Jyo}]{mūla}
		\rdg[wit={V15}]{mūlaṃ}}}%
\pada{\app{\lem[wit={V3}]{śarābandhaṃ}
		\rdg[wit={J6}]{śarabaṃdhaṃ}
		\rdg[wit={N9}]{śarābadho}
		\rdg[wit={V15,Jyo}]{śirābandhaḥ}
		\rdg[wit={N26}]{sirābandho}}
	\app{\lem[wit={Gr6}]{vinaśyati}
		\rdg[wit={V15,Jyo}]{praṇaśyati}}//} \sgwit{V15,Gr6,Jyo}\\!}
\end{alttlg}

%\newpage
\begin{alttlg}[hp03_033_04] % KhV 1.49
\tl{
\pada{atha vāgīśvarīdhāma}
\pada{śiro vastreṇa veṣṭayet/}\\+}
\tl{
\pada{śanair utkarṣayed yogī}
\pada{kālavelā% kālā N9
	\app{\lem[wit={V3,J6}]{vidhānavit}
		\rdg[wit={N26,N9}]{vidhānataḥ}}//} \sgwit{Gr6}\\!}
\end{alttlg}

\newpage
\begin{alttlg}[hp03_033_05] % from long recension of the Yogabīja
\tl{
\pada{vitasti\app{\lem[wit={V3,N26,J6}]{pramitaṃ}
		\rdg[wit={N9}]{pratima}}
	\app{\lem[wit={V3,N9}]{dairghyaṃ}% dairghaṃ N9
		\rdg[wit={N26,J6pc}]{dairghye}
		\rdg[wit={J6ac}]{dairghya}}}
\pada{\app{\lem[wit={V3,N9,J6}]{vistāraṃ}
	\rdg[wit={N26}]{vistāre}} caturaṅgulam/}\\+}
\tl{
\pada{mṛdulaṃ dhavalaṃ proktaṃ}
\pada{\app{\lem[wit={V3,N9,J6}]{veṣṭitāmbara}
	\rdg[wit={N26}]{veṣṭitādhāra}}lakṣaṇam//} \sgwit{Gr6} \anm{=\,3.111}\\!}
\end{alttlg}

%\newpage
\begin{alttlg}[hp03_033_06] % KhV 1.50
\tl{
\pada{punaḥ ṣaṇmāsamātreṇa}
\pada{punaḥ saṃkarṣaṇāt priye/}\\+} % °karṣaṇa N9
\tl{
\pada{bhrūmadhyāvadhi vardheta}
\pada{tiryakkarṇa% tiryakarṇa V3,N9
	bilāvadhi//} \sgwit{Gr6}\\!}
\end{alttlg}

%\newpage
\begin{alttlg}[hp03_033_07] % KhV 1.51ab, 1.52ab
\tl{
\pada{adhastā\app{\lem[wit={N26,N9,J6},alt={cibukaṃ mūlaṃ}]{\skm{c }cibukaṃ mūlaṃ} % cf. J3
		\rdg[wit={V3}]{cibukamūla}}}
\pada{prayāti \app{\lem[wit={V3,J6}]{kramakāritā}
		\rdg[wit={N26}]{kramakāritaḥ}
		\rdg[wit={N9}]{tramakārikā}}/}\\+}
\tl{
\pada{krośād ūrdhvaṃ % krośāharddhaṃ J8
	\app{\lem[wit={V3,N9,J6}]{ca}
		\rdg[wit={N26},alt={\om}]{\skp{\om}}}
	\app{\lem[wit={V3,N9}]{kramati}
		\rdg[wit={J6}]{krāmati}
		\rdg[wit={N26}]{kramatī}}}
\pada{tirya\app{\lem[wit={V3,J6},alt={saṃkhyā}]{\skm{k}saṃkhyā}
		\rdg[wit={N26,N9}]{saṃsthā}}vadhi priye//} \sgwit{Gr6}\\!}
\end{alttlg}

\begin{alttlg}[hp03_033_08] % = HP6 and HP10
\tl{
\pada{punaḥ saṃvatsarād devi}
\pada{dvitīyā caiva līlayā/}\\+} % dvitiyā V3,J15; lilayā N9,J15
\tl{
\pada{brahma\app{\lem[wit={V3,N26,J6},alt={randhrāntam}]{randhrānta\skp{m}}
		\rdg[wit={N9}]{raṃdhraṃ tam}}m āvṛtya}
\pada{\app{\lem[wit={V3,N26,J6},alt={tiṣṭhet}]{tiṣṭhe\skp{t}}
		\rdg[wit={N9}]{viṣṭaitet}}%
	\app{\lem[wit={V3,N26,J6},alt={paramavandite}]{\skm{t }paramavandite}
		\rdg[wit={N9}]{paramavidite}}//} \sgwit{Gr6}\\!}
\end{alttlg}

%\newpage
\begin{alttlg}[hp03_033_09]% = HP6 and HP10
\tl{
\pada{svatālumūlaṃ saṃghṛṣya}
\pada{saptavāsaram ātmani/}\\+} % atmani N9
\tl{
\pada{svagurūktaprakāreṇa}
\pada{malaṃ sarvaṃ viśoṣayet//} % viśeṣayet J8
\sgwit{Gr6}\\!}
\end{alttlg}

%\newpage
\begin{alttlg}[hp03_033_10]% = HP10 (33.10ab = KhV 1.56cd)
\tl{
\pada{aṅgulyagreṇa saṃghṛṣya}
\pada{jihvāṃ tatra niveśayet/}\\+}
\tl{
\pada{śanaiḥ śanair mastakāc ca} % one śanair om. N9
\pada{mahāvajrakapāṭabhit//} \sgwit{Gr6}\\!} % bhīt J15
\end{alttlg}

%\newpage
\begin{alttlg}[hp03_033_11]% = KhV 1.34
\tl{
\pada{pūrva\app{\lem[wit={N26,N9}]{bīja}
		\rdg[wit={V3}]{vīya}
		\rdg[wit={J6}]{vīrya}}yutāṃ vidyāṃ}
\pada{\app{\lem[wit={V3,N9,J6}]{vyākhyātām ati}
		\rdg[wit={N26}]{vikhyātām api}}durlabhām/}\\+}
\tl{
\pada{asyāḥ \app{\lem[wit={V3,N26,J6}]{ṣaḍaṅgaṃ}
		\rdg[wit={N9}]{ṣaḍaṃhva}} kurvīta}
\pada{tayā ṣaṭcakrabhinnayā//} \sgwit{Gr6}\\!}
\end{alttlg}

%\newpage
% metre: Rathoddhatā
\begin{alttlg}[hp03_033_12]% = HP6 and HP10
\tl{
\pada{khe
	\app{\lem[wit={V3,J6}]{nirasta}
	\rdg[wit={N9}]{cirasta}
	\rdg[wit={N26}]{lirasta}}sakalakriyākrame}\\+} % sakalā N9
\tl{
\pada{\app{\lem[resp=emend,postwit=\texteng{(cf.\,Yoginīhṛdaya)}]{yā citiś carati} % J15 yojitaś, many other mss °ṇa cittaś (unmetrical)
		\rdg[wit={N9}]{yācitaś carati}
		\rdg[wit={V3}]{yā cittaś carati}
		\rdg[wit={J6}]{°ṇa cittaś carati}
		\rdg[wit={N26}]{cittam ācarati}} śāśvatodaye/}\\+}
\tl{
\pada{sā śivatva\app{\lem[wit={V3,N26,N9}]{samavāya} % śivatta N9
		\rdg[wit={J6}]{samavāyi}}%
	\app{\lem[wit={N26,N9,J6}]{kāriṇī}
		\rdg[wit={V3}]{kariṇī}}}\\+}
\tl{
\pada{khecarī
	\app{\lem[wit={V3,N9}]{ca bhava}
		\rdg[wit={J6}]{\{\{ca\}\} bhavati}
		\rdg[wit={N26}]{bhavati}}khedahāriṇī//} \sgwit{Gr6}\\!}
\end{alttlg}

%\newpage
\begin{alttlg}[hp03_033_13]% = HP6 and HP10, ~ KhV 1.54
\tl{
\pada{krameṇaiva
	\app{\lem[wit={V3,N26,N9}]{prakartavyā}
		\rdg[wit={J6}]{pravartavyā}}}%
\pada{bhyāsena
	vara\app{\lem[wit={V3,N26,N9}]{varṇini}
		\rdg[wit={J6}]{varṇinī}}/}\\+}
\tl{
\pada{yugapa%
	\app{\lem[wit={V3,N9,J6pc},alt={yatate}]{\skm{d }yatate}
		\rdg[wit={J6ac}]{yatete}
		\rdg[wit={N26}]{utpadyate}}
	\app{\lem[wit={V3,N9,J6}]{tasya}
	\rdg[wit={N26}]{samyak}}}
\pada{śarīraṃ vilayaṃ vrajet//} \sgwit{Gr6}\\!}
\end{alttlg}

\newpage
\begin{alttlg}[hp03_033_14]% = HP6 and HP10, KhV 1.55ab
\tl{
\pada{tasmāc chanaiḥ śanaiḥ kāryo}% one śanaiḥ om. N9
\pada{\app{\lem[wit={V3,J6}]{'bhyāso na}
		\rdg[wit={N26,N9}]{bhyāsena}} 
	yugapa%
	\app{\lem[wit={V3,N9,J6},alt={priye}]{\skm{t }priye}
	\rdg[wit={N26}]{kriye}}/}\\+}
\tl{
\pada{\app{\lem[wit={N26,N9,J6}]{evaṃ}
		\rdg[wit={V3}]{eva}} varṣatrayaṃ kṛtvā} % kṛtvā illeg. N26
\pada{brahmadvāraṃ
	\app{\lem[wit={N26,J6},alt={viśed}]{viśe\skp{d}}
		\rdg[wit={N9}]{vaśed}
		\rdg[wit={V3}]{biśe}}d dhruvam//} \sgwit{Gr6}\\!}
\end{alttlg}

%\newpage
% metre: Śārdūlavikrīḍita
\begin{alttlg}[hp03_033_15]% = HP6 and HP10
\tl{
\pada{saṭcakrāṇi
	\app{\lem[wit={V3,N9,J6}]{vibhidya}
	\rdg[wit={N26}]{vibhedya}}
śakti\app{\lem[wit={N26,J6}]{bhujagīṃ}
		\rdg[wit={V3}]{bhujaṃgī}
		\rdg[wit={N9}]{bhujaṃgīṃ}}
	\app{\lem[wit={V3,N26,N9}]{protthāpya}
		\rdg[wit={J6}]{protthāya}} mūlasthitāṃ}\\+} % stitāṃ N9
\tl{
\pada{bhittvā granthitrayaṃ ca % tra of trayaṃ treated as a single consonant?
paścima\app{\lem[wit={V3,N9,J6}]{śirā}
	\rdg[wit={N26}]{śirāṃ}}prākāra%
	\app{\lem[wit={V3,N9,J6}]{rūpaṃ}
	\rdg[wit={N26}]{rūpāṃ}} mahat/}\\+}
\tl{
\pada{nītvā prāṇam ataḥ śirobilam alaṃ % anaḥ? J6; ala N9
	nirmathya cittena
	\app{\lem[wit={V3,N26}]{tat}
		\rdg[wit={N9,J6}]{tal}}}\\+}
\tl{
\pada{liṅgaṃ yaḥ % yāḥ N9
\app{\lem[wit={N26,J6}]{pibatī}
		\rdg[wit={V3,N9}]{pibate}}ndumaṇḍalagala% galat V3,J15
	\app{\lem[wit={V3,N26},alt={muktaḥ sa sākṣācchivaḥ}]{\skm{n }muktaḥ sa sākṣācchivaḥ}
		\rdg[wit={J6}]{muktaś ca sākṣācchivaḥ}
		\rdg[wit={N9}]{muktaḥ kṣamāddhivaḥ}}//} \sgwit{Gr6}\\!}
\end{alttlg}

%\newpage
% metre: Sragdharā
\begin{alttlg}[hp03_033_16]% = HP6 and HP10
\tl{
\pada{nityaṃ
	\app{\lem[wit={V3,N26,N9}]{yas tūrdhva}
		\rdg[wit={J6}]{yasphūrja}}%
	\app{\lem[wit={N26}]{jihvo yadi}
		\rdg[wit={V3}]{jihvogradi}
		\rdg[wit={J6}]{jihvāgrayā}
		\rdg[wit={N9}]{jihvā yadi}}
pibati pumān
saptadhārāmṛ\app{\lem[wit={V3,N26}]{taughaṃ}
	\rdg[wit={N9}]{tauṣaṃ}
	\rdg[wit={J6}]{tauccaṃ}}}\\+}
\tl{
\pada{\app{\lem[wit={V3,N26,N9}]{susvādaṃ}
	\rdg[wit={J6}]{su[kha]daṃ}} śītalāṅgaṃ
	duritabhayaharaṃ % rahaṃ N9
	kṣutpipāsānivāri/}\\+} % nivāritaṃ J6ac
\tl{
\pada{piṇḍasthairyaṃ hi tasmād bhavati
	\app{\lem[wit={N26,N9}]{mṛtapathā} % mṛtta N9
	\rdg[wit={V3}]{mṛtayathā}
	\rdg[wit={J6}]{mṛtaṃ yathā}}
	mṛtyu\app{\lem[wit={V3,J6},alt={rogād}]{rogā\skp{d}}
		\rdg[wit={N26}]{rogod}
		\rdg[wit={N9}]{śeṣād}}%
	\app{\lem[wit={V3,N9},alt={bhavanti}]{\skm{d }bhavanti}
		\rdg[wit={J6}]{bhavati}
		\rdg[wit={N26}]{bhavaṃtu}}}\\+}
\tl{
\pada{\app{\lem[wit={N26,J6}]{daurbhāgyaṃ}
	\rdg[wit={V3,N9}]{daurbhyāgyaṃ}}
yāti nāśaṃ prasarati sakalaṃ yāti % yātī N26ac?
	\app{\lem[wit={V3,N9}]{kālaṃ}
	\rdg[wit={N26,J6}]{kālo}} bhramitvā//} \sgwit{Gr6}\\!}
\end{alttlg}


\begin{alttlg}[hp03_033_17]% = HP6 and HP10
\tl{
\pada{\app{\lem[wit={N26,J6}]{tīkṣṇakaṃ}
	\rdg[wit={V3}]{tīkṣṇake}
	\rdg[wit={N9}]{tīkṣṇako}}
	\app{\lem[wit={V3,N9,J6}]{harate}
	\rdg[wit={N26}]{harati}} vyādhiṃ} % vyādhiḥ N9
\pada{\app{\lem[wit={V3,J6}]{kaṭukaṃ kuṣṭhanāśanam}
	\rdg[wit={N26}]{kaṭutvaṃ kuṣṭhanāśanam}
	\rdg[wit={N9}]{kaṭukuṭivināśanaṃ}}/}\\+} % u of kaṭuka resembles to a virāma in V3,N9
\tl{
\pada{\app{\lem[wit={V3,N26,J6}]{ghṛta}
		\rdg[wit={N9}]{dhṛtvā}}svādūpamaṃ caiva}   % caivāmaratvaṃ J6
\pada{amaratvaṃ \app{\lem[wit={V3,N26},alt={labhed}]{labhe\skp{d}}
		\rdg[wit={N9,J6}]{labhate}}d dhruvam//} \sgwit{Gr6}\\!} % labhe V3
\end{alttlg}

%\newpage
\begin{alttlg}[hp03_033_18]% = HP6
\tl{
\pada{madhusvādūpamaṃ caiva}
\pada{śāstra%m
	\app{\lem[wit={V3,N26,N9},alt={udgirate}]{\skm{m }udgirate}
	\rdg[wit={J6}]{udgirati}}
	\app{\lem[wit={N26,N9,J6}]{bahu}
	\rdg[wit={V3}]{bahuḥ}}/}\\+}
\tl{
\pada{\app{\lem[wit={N26}]{laḍḍu}
	\rdg[wit={V3,N9,J6}]{laḍu}}%
	\app{\lem[wit={V3,N9}]{ṣaṇḍakapādyāni}
	\rdg[wit={J6}]{khaṃḍakapādyāni}
	\rdg[wit={N26}]{piṇḍakakhādyāni}}}
\pada{\app{\lem[wit={V3,N9,J6}]{pakvānnāni}
	\rdg[wit={N26}]{pakvānyanmāny}} anekaśaḥ//} \sgwit{Gr6}\\!}
\end{alttlg}

\begin{alttlg}[hp03_033_19]% = HP6
\tl{
\pada{divya\app{\lem[wit={V3,N9,J6}]{kalpaṃ}
	\rdg[wit={N26}]{kalpai}}
	\app{\lem[wit={ceteri},alt={ramen}]{rame\skp{n}}
	\rdg[wit={N26}]{racen}
	\rdg[wit={J6}]{krīḍen}}n nityaṃ} % rame J8
\pada{utkṛṣṭo jāyate dhruvam/}\\+}
\tl{
\pada{tanmayatvam avāpnoti}
\pada{\app{\lem[wit={J6}]{kośakārīva}
	\rdg[wit={V3}]{kauśakārīva}
	\rdg[wit={N26}]{koṣakārī ca}
	\rdg[wit={N9}]{kauṣṭakārīva}} kīṭakaḥ//} \sgwit{Gr6}\\!}
\end{alttlg}

\endaltrecension

\newpage
\begin{tlg}[hp03_034]
\tl{
\pada{kapālakuhare jihvā} % kalāpa C6
\pada{\app{\lem[wit={ceteri}]{praviṣṭā viparītagā}
	\rdg[wit={N3}]{pra\,+\,+\,+\,+\,+\,+}}/}\\+}
\tl{
\pada{bhruvo%r  % bhṛ° N3, bhrū° V3, bhrūvaur N23
	\app{\lem[wit={ceteri},alt={antargatā}]{\skm{r }antargatā}
	\rdg[wit={N3}]{aṃtagatā}
	\rdg[wit={C6}]{madhye gatā}}
	\app{\lem[wit={ceteri},alt={dṛṣṭir}]{dṛṣṭi\skp{r}}
		\rdg[wit={N3,N23,V15}]{dṛṣṭi}}r}
\pada{mudrā bhavati
	\app{\lem[wit={ceteri}]{khecarī}
		\rdg[wit={V1}]{carī}}//}\myfn{\getsiglum{Jyo} has this verse at the very beginning of the Khecarī-section.}\\!}  %3.34
\end{tlg}

%\newpage
\begin{tlg}[hp03_035]
\tl{
\pada{\app{\lem[wit={ceteri}]{kalāṃ}
		\rdg[wit={N23}]{kalā}
		\rdg[wit={J10}]{kālaṃ}} % +J6 sonst wie lemmata
	\app{\lem[wit={P11,V19,K3,N19,V15,V1,Jyo}]{parāṅmukhīṃ}
		\rdg[wit={V3,Gr2,C7,J10}]{parāṅmukhī}
		\rdg[wit={C6}]{avāṅmukhī}}
	\app{\lem[wit={C6,V3,Gr3a,N19,V15,V1,Jyo}]{kṛtvā}
		\rdg[wit={J10}]{kṛtya}
		\rdg[wit={Gr2}]{nītvā}}}
\pada{\graus{\app{\lem[wit={V3,Gr3a,V15,Jyo}]{tripathe}% tṛpathe V19
		\rdg[wit={N19}]{tripathaṃ}
		\rdg[wit={C6}]{tripatha}}
	\app{\lem[wit={V15}]{parivartayet}
		\rdg[wit={C6,V3,V19,N19}]{parivarjayet} % dv<<ā>>ravarjayet C6
		\rdg[wit={K3,C7}]{parivardhayet}
		\rdg[wit={Jyo}]{pariyojayet}}/}}\\+}  % om. in V1,J10
\tl{
\pada{\graus{\app{\lem[wit={C6,Gr3a,N19,V15,Jyo}]{sā} % sāṃ N19
		\rdg[wit={V3}]{sa}}
	\app{\lem[wit={C6,V3,K3,C7,N19,V15,Jyo}]{bhavet khecarī}
		\rdg[wit={V19}]{bhat ṣecarī}} mudrā}}
\pada{\graus{vyomacakraṃ tad ucyate/}}\\+}
\tl{
\pada{\graus{rasanām ūrdhvagāṃ kṛtvā}}\myfn{The Pādas in grey scale are not found in \getsiglum{Gr2,V1,J10}, but in \getsiglum{C6,V3,Gr3a,N19,V15,Jyo}.
\getsiglum{N3} omits the whole verse. \getsiglum{J5,G4} have this verse without the grey-scaled part.}
\pada{kṣaṇārdhaṃ
	\app{\lem[wit={ceteri}]{yadi}
		\rdg[wit={J10,Jyo}]{api}} tiṣṭhati/}\\+}
\tl{
\pada{\app{\lem[wit={V3,J7,N19,V15,V1,J10}]{kṣaṇena}
		\rdg[wit={N23}]{kṣaṇe [ca]}
		\rdg[wit={Gr3a}]{viṣayair} % = YCM; viṣayai V19
		\rdg[wit={P11,Jyo}]{viṣair vi°}
		\rdg[wit={C6}]{duḥkhair vi°}}
		mucyate
	\app{\lem[wit={ceteri}]{yogī}\rdg[wit={J7},alt={\om}]{\skp{\om}}}}
\pada{\app{\lem[wit={ceteri}]{vyādhi}
		\rdg[wit={J7}]{vyādhijanma}}mṛtyujarādibhiḥ//} \NotIn{N3}\\!}
\end{tlg}
		%3.35

%\newpage
\begin{tlg}[hp03_036]
\tl{
\pada{na \app{\lem[wit={ceteri}]{rogo}
		\rdg[wit={V1}]{roga}
		\rdg[wit={J10}]{rogān}}
	maraṇaṃ
	\app{\lem[wit={ceteri}]{tasya}
		\rdg[wit={Jyo}]{tandrā}}}
\pada{na nidrā na \app{\lem[wit={ceteri}]{kṣudhā tṛṣā}% +P11
		\rdg[wit={C7}]{kṣudhā nandaṭ}
		\rdg[wit={C6,V19},post=\texteng{(tṛkhā \getsiglum{V19})}]{tṛṣā kṣudhā}}/}\\+}
\tl{
\pada{na
	\app{\lem[wit={ceteri}]{ca}
		\rdg[wit={V3}]{bhra}
		\rdg[wit={C7},alt={\om}]{\skp{\om}}} mūrchā
	\app{\lem[wit={ceteri},alt={bhavet}]{bhave\skp{t}}
		\rdg[wit={J10}]{bhave}
		\rdg[wit={C7}]{tu bhavet}}%
	\app{\lem[wit={ceteri},alt={tasya}]{\skm{t }tasya}
	\rdg[wit={N3}]{ta\,+}}}
\pada{\app{\lem[wit={ceteri}]{yo mudrāṃ vetti}
	\rdg[wit={N3},alt={\damaged}]{\skp{\damaged}}}
	\app{\lem[wit={J7,V19,K3,N19,V15,V1,J10,Jyo}]{khecarīm}
		\rdg[wit={N3,C6,V3,N23,C7}]{khecarī}}//}\myfn{In \getsiglum{J7} this verse is found after \ref{III38}.}\\!}  %3.36
\end{tlg}


%\newpage
\startaltrecension\normalsize\color{black}
\begin{tlg}[hp03_036_1]
\tl{
\pada{\app{\lem[wit={N3,P11,Jyo}]{pīḍyate}
		\rdg[wit={C6,V3,J10}]{bādhyate}
		\rdg[wit={V1}]{chādyate}}
	na \app{\lem[wit={V3,V1,J10,Jyo}]{sa}
	\rdg[wit={N3,J5,P11,C6}]{ca}} rogeṇa} % rogena V3
\pada{\app{\lem[wit={C6,V3,V1,J10}]{lipyate na sa}
		\rdg[wit={Jyo}]{lipyate na ca}
		\rdg[wit={J5,P11}]{na ca lipyati}
		\rdg[wit={N3}]{na ca lipyata}} karmaṇā/}\\+}
\tl{
\pada{\app{\lem[wit={N3,P11,V3,V1,J10,Jyo}]{bādhyate}% +P11
	\rdg[wit={C6}]{khādyate}}
	na \app{\lem[wit={C6,V3,V1,J10,Jyo}]{sa}
	\rdg[wit={N3,J5,P11}]{ca}} kālena}
\pada{\app{\lem[wit={N3,P11,C6,V3,J10,Jyo}]{yo mudrāṃ vetti}% mudrā N3
		\rdg[wit={V1}]{yasya mudrāsti}}
	\app{\lem[wit={N3,J10,Jyo}]{khecarīm}
		\rdg[wit={P11,C6,V3,V1}]{khecarī}}//}
	\sgwit{N3,C6,V3,V1,J10,Jyo}\\!}  %3.37
\end{tlg}
\endaltrecension


\begin{tlg}[hp03_037]
\tl{
\pada{\app{\lem[wit={ceteri}]{cittaṃ}
		\rdg[wit={V19}]{citte}
		\rdg[wit={N3}]{ci\,+}}
	\app{\lem[wit={ceteri}]{carati khe}
	\rdg[wit={N3},alt={\damaged}]{\skp{\damaged}}}
	\app{\lem[wit={ceteri},alt={yasmāj}]{yasmā\skp{j}}
		\rdg[wit={V3}]{yasyā}
		\rdg[wit={N3}]{+\,.āj}}j}
\pada{jihvā carati khe gatā/}\\+} % jihva J8; jihvā jihva V19; calati N3; gatāḥ C8
\tl{
\pada{\app{\lem[wit={N3,P11,V15,Jyo}]{tenaiṣā}
		\rdg[wit={V3,N19,V1,J10}]{tenaiva}
		\rdg[wit={C6,J7,Gr3a}]{teneyaṃ}} khecarī
	\app{\lem[wit={N3,P11,V15,Jyo}]{nāma}
		\rdg[wit={ceteri}]{mudrā}}}
\pada{\app{\lem[wit={N3,P11,V15,Jyo}]{mudrā}
		\rdg[wit={C6,V3,J7,Gr3a,N19,V1,J10}]{sarva}}
	siddhair namaskṛtā//}\label{III38} \NotIn{N23}
	\anm{= 4.25*4}\\!}  %3.38  % siddhai V19, siddhir C8ac
\end{tlg}

\newpage
\begin{tlg}[hp03_038]
\tl{
\pada{\app{\lem[wit={ceteri}]{khecaryā}
		\rdg[wit={V3}]{khecaryāṃ}} mudritaṃ yena} % mudritā ye tu C6
\pada{\app{\lem[wit={ceteri}]{vivaraṃ}
		\rdg[wit={K3}]{viviraṃ}
		\rdg[wit={C6,V1}]{vicaran/raṃ}}
	\app{\lem[wit={ceteri}]{lambikordhvataḥ}
		\rdg[wit={K3,C7}]{lampikordhvataḥ}
		\rdg[wit={N3},alt={\damaged}]{\skp{\damaged}}}/}\\+}
\tl{
\pada{\app{\lem[wit={C6,V3,N23,N19,V15,V1,J10}]{tasya na}% naḥ V15
		\rdg[wit={N3}]{+\,[s]ya na}
		\rdg[wit={J7,Gr3a,Jyo}]{na tasya}}
		kṣarate binduḥ} % bindu N23
\pada{kāminyā\app{\lem[wit={ceteri}]{śleṣitasya}
		\rdg[wit={Gr2,K3}]{liṅgitasya}
		\rdg[wit={C6}]{liṅgitena}} % saṃślitasya P11
		ca//}\label{III39}\\!}  %3.39
\end{tlg}

% Gr4b have the pātāla-stanza here.

%\newpage
\begin{tlg}[hp03_039]
\tl{
\pada{\app{\lem[wit={ceteri}]{calito}% +P11; patito C6
		\rdg[wit={N23}]{calitā}
		\rdg[wit={V19}]{calate}
		\rdg[wit={N3}]{calato}}'pi yadā binduḥ} % bindu K3, biṃḥ N19
\pada{\app{\lem[wit={V3,Gr2,V19,J10},alt={saṃprāptaś}]{saṃprāpta\skp{ś}}
		\rdg[wit={N3,C6,K3,C7,N19,V15,V1,Jyo}]{saṃprāpto}}%
	\app{\lem[wit={Gr2,V19},alt={ca hutāśanaṃ}]{\skm{ś }ca hutāśanam}% = VM
		\rdg[wit={K3,C7}]{pi hutāśanaṃ}
		\rdg[wit={V3,J10}]{cāgnimaṇḍalaṃ}
		\rdg[wit={C6}]{vahnimaṇḍalaṃ}% +P11,M1,M3(°le)
		\rdg[wit={J5,V15,V1,Jyo}]{yonimaṇḍalam}% +G7? illeg. G4
		\rdg[wit={N3}]{yogimaṃḍalaṃ}
		\rdg[wit={N19}]{yonimaṃgalaṃ}}/}\\+}
\tl{
\pada{\app{\lem[wit={ceteri},alt={vrajaty}]{vraja\skp{ty}}
		\rdg[wit={N3}]{vṛjaṃty}
		\rdg[wit={N23}]{jajaty}
		\rdg[wit={C7}]{vrajan}}%
	\app{\lem[wit={ceteri},alt={ūrdhvaṃ}]{\skm{ty }ūrdhvaṃ} % ṃ om. J7
		\rdg[wit={C7}]{pūrvaṃ}
		\rdg[wit={N3}]{ū\,+}}
	\app{\lem[resp=emend,postwit=\texteng{(cf.\,VM)}]{hataḥ śaktyā}
		\rdg[wit={Jyo}]{hṛtaḥ śaktyā}
		\rdg[wit={N23}]{hatāchantkā}
		\rdg[wit={C6,V3,J7,N19,V15,V1,J10}]{haṭhāc chaktyā}% havāt N19
		\rdg[wit={J5}]{haṭhāt saktyā}
		\rdg[wit={K3,C7}]{hi tacchaktyā}
		\rdg[wit={V19}]{hi tadbhuktyā}
		\rdg[wit={N3},alt={\damaged}]{\skp{\damaged}}}}\marma
\pada{\app{\lem[wit={J5,G4,C6,N19,V15,Jyo}]{nibaddho}% +J5,G4
		\rdg[wit={V1}]{nibadhno}
		\rdg[wit={Gr2,K3,C7}]{niruddho}
		\rdg[wit={V3,J10}]{nirodho}
		\rdg[wit={V19}]{viruddhe}
		\rdg[wit={N3},alt={\damaged}]{\skp{\damaged}}}
	\app{\lem[wit={N3,C6,Gr2,Gr3a,N19,V15,V1,Jyo}]{yoni}
		\rdg[wit={P11,V3,J10}]{yoga}}mudrayā//}%
\myfn{\getsiglum{C6} has this verse between \ref{III42}ab and cd.}% P11 has it at a regular place.
\\!}  %3.40
\end{tlg}

\Anm{\getsiglum{V3,J10} have \ref{III43} \textit{indhanāni} here.}

\startaltrecension\normalsize
\begin{alttlg}[hp03_039_1]
\tl{
\pada{kapālakuhare jihvā}
\pada{\app{\lem[wit={V3,V15}]{kalā}
		\rdg[wit={Gr2,N19}]{kāla}
		\rdg[wit={N3}]{kālā}
		\rdg[wit={C6}]{kṛtvā}}% +P11
	\app{\lem[wit={N3,C6,V3,V15}]{saṃdhāna}
		\rdg[wit={N19}]{saṃdhāra}
		\rdg[wit={Gr2}]{saṃhāra}}%
	\app{\lem[wit={N3,Gr2,N19,V15}]{mudrayā}
		\rdg[wit={V3}]{varjitā}}/} % varjitāḥ
%		\NotIn{Gr3a,V1,J10,Jyo}
		\sgwit{N3,C6,V3,Gr2,N19,V15}\label{III40_1}\\!}
\end{alttlg}
\endaltrecension

% In the 6-chp version it is followed by \devnote{brahmarandhragatā nityaṃ tasya siddhir na dūrataḥ}.


%\newpage
\begin{tlg}[hp03_040]
\tl{
\pada{\app{\lem[wit={ceteri}]{ūrdhva}
		\rdg[wit={N19}]{ūrdhvaṃ}}%
	\app{\lem[wit={J7,V19,C7,Jyo}]{jihvaḥ}
		\rdg[wit={N23}]{jihva}
		\rdg[wit={N3,C6,V3,K3,N19,V15,V1,J10}]{jihvā}}
	\app{\lem[wit={N3,J5,V3,J10}]{sthito}% +J5
		\rdg[wit={P11,C6,Gr2,Gr3a,V15,V1,Jyo}]{sthiro}
		\rdg[wit={N19}]{sito}} bhūtvā}
\pada{somapānaṃ
	\app{\lem[wit={ceteri}]{karoti yaḥ}
		\rdg[wit={C6}]{karoti saḥ}
		\rdg[wit={N3}]{karo\,+\,+}}/}\\+}
\tl{
\pada{\app{\lem[wit={ceteri}]{māsārdhena na}
	\rdg[wit={C6}]{māsārdhena tu}
	\rdg[wit={N3},alt={\damaged}]{\skp{\damaged}}} saṃdeho}
\pada{mṛtyuṃ jayati yogavit//}\label{III41}\\!}  %3.41
\end{tlg}

%\newpage
\begin{tlg}[hp03_041]
\tl{
\pada{nityaṃ % nitya V19
	somakalā\app{\lem[wit={ceteri}]{pūrṇaṃ}
		\rdg[wit={N23,N19}]{pūrṇa}
		\rdg[wit={J10}]{pūrṇe}}}
\pada{śarīraṃ yasya
	\app{\lem[wit={C6,Gr2,Gr3a,N19,Jyo}]{yoginaḥ}
		\rdg[wit={V3}]{yoginaṃ}
		\rdg[wit={V15,V1,J10}]{dehinaḥ}}/}\\+}
\tl{
\pada{takṣakeṇāpi % tatkṣa° N19; °nāpi C7
	\app{\lem[wit={C6,J7,V19,V15,J10,Jyo}]{daṣṭasya}
		\rdg[wit={V3,N23,V1}]{dṛṣṭasya}% +J17
		\rdg[wit={N19}]{daṃṣṭrasya}
		\rdg[wit={K3,C7}]{dagdhasya}}}
\pada{viṣaṃ tasya na
	\app{\lem[wit={ceteri}]{sarpati}
		\rdg[wit={V3}]{sparśati}
		\rdg[wit={N23}]{pīḍyate}}//}\label{III42}\myfn{%
	\getsiglum{V3} adds here:
	\devnote{tasmād idaṃ prakurvīta nityayuktaḥ samāhitaḥ}.}
		\NotIn{N3,J5}\\!}  %3.42 ; not in N3,J5,C2 but in G4,N24
\end{tlg}


\newpage
\begin{tlg}[hp03_042]
\tl{
\pada{\app{\lem[wit={ceteri}]{indhanāni}
		\rdg[wit={V3}]{yindhanāni}}
	\app{\lem[wit={ceteri}]{yathā}
		\rdg[wit={K3},alt={\om}]{\skp{\om}}} vahni}%
\pada{\app{\lem[wit={V3,V19,K3,V15,Jyo},alt={tailavartiṃ}]{\skm{s }tailavartiṃ}
		\rdg[wit={N3,C7,N19}]{tailavarti}% vartti N19
		\rdg[wit={C6,Gr2,J10}]{tailavartī}% varttī N23,C6,J10
		\rdg[wit={V1}]{tailāvṛtti}}
	\app{\lem[wit={ceteri}]{ca}
		\rdg[wit={V1}]{va}}
	\app{\lem[wit={ceteri}]{dīpakaḥ}
		\rdg[wit={V1}]{dīpikaḥ}}/}\\+} % V1 vowel sign i added later
\tl{
\pada{tathā \app{\lem[wit={ceteri}]{soma}
		\rdg[wit={N19}]{sarva}}%
	kalā\app{\lem[wit={ceteri}]{pūrṇaṃ}
		\rdg[wit={J10}]{pūrṇa}
		\rdg[wit={J7,N19}]{pūrṇo}}}
\pada{\app{\lem[wit={ceteri}]{dehī dehaṃ}% dehāṃ V3
		\rdg[wit={C6,V15}]{dehaṃ dehī}
		\rdg[wit={N3},alt={\damaged}]{\skp{\damaged}}}
	\app{\lem[wit={Gr2,Gr3a,N19,V1,Jyo}]{na muñcati}
		\rdg[wit={J10}]{na mucyati}
		\rdg[wit={V15}]{na muṃcyati}
		\rdg[wit={C6,V3}]{na mucyate}
		\rdg[wit={N3}]{+\,+\,+\,ti}}//}\label{III43}\myfn{%
	\getsiglum{Gr2,Gr3a} add here:
	\devnote{rasanāṃ veśayed ūrdhvaṃ pibet tat srāvitaṃ jalam}}\\!}  %3.43
\end{tlg}

%\startaltrecension
%\begin{tlg}[hp03_042_1]
%\pada{\app{\lem[wit={J7,Gr3a}]{rasanāṃ}
%		\rdg[wit={N23}]{rasānāṃ}}
%	\app{\lem[wit={J7,Gr3a},alt={veśayed}]{veśaye}
%		\rdg[wit={N23}]{vasayed}
%		\rdg[wit={K3}]{ūrdhvam ā°}}%
%	\app{\lem[wit={ceteri},alt={ūrdhvaṃ}]{d ūrdhvaṃ}
%		\rdg[wit={K3}]{°veśet}}}
%\pada{pibet tat \app{\lem[wit={Gr3a}]{srāvitaṃ}
%		\rdg[wit={Gr2}]{sravitaṃ}} jalam/} % ta{{sya}}chrāvitaṃ V19
%\sgwit{Gr2,Gr3a}\\!
%
%\begin{tlg}[hp03_042_2]
%\pada{tasmād idaṃ prakurvīta}
%\pada{nityayuktaḥ samāhitaḥ/}\label{III43_2} \sgwit{V3}\\!
%\endaltrecension

%\newpage
\begin{tlg}[hp03_043]
\tl{
\pada{\app{\lem[wit={ceteri}]{gomāṃsaṃ}
		\rdg[wit={J7,V19,J10}]{gomāṃsa}}
		bhakṣayen nityaṃ}
\pada{pibe%
	\app{\lem[wit={ceteri},alt={amara}]{\skm{d }amara}
		\rdg[wit={C7}]{amṛta}}%
	\app{\lem[wit={ceteri}]{vāruṇīm}
		\rdg[wit={V3,N19,V15}]{vāruṇī}}/}\\+}    % V3 bhakṣaye
\tl{
\pada{kulīnaṃ
	\app{\lem[wit={ceteri},alt={tam}]{ta\skp{m}}
		\rdg[wit={J7}]{tum}}m ahaṃ
	\app{\lem[wit={ceteri}]{manye} % ahamanye N23
		\rdg[wit={Jyo}]{manya}
		\rdg[wit={V3}]{vidyāṃ}
		\rdg[wit={J10}]{viṃdyāṃ}}}
\pada{\app{\lem[wit={N3,C6,V3,J10,Jyo}]{itare}
		\rdg[wit={V15,V1}]{tv itare}
		\rdg[wit={N19}]{cetare}
		\rdg[wit={Gr2,Gr3a}]{netarān}}
	\app{\lem[wit={ceteri}]{kulaghātakāḥ}
		\rdg[wit={Gr2,Gr3a}]{kulaghātakān}
		\rdg[wit={N3}]{kuṣvaghātakāḥ}}//}\label{III44}\\!}  %3.44
\end{tlg}

%\newpage
\begin{tlg}[hp03_044]
\tl{
\pada{gośabde\app{\lem[wit={ceteri}]{noditā jihvā}
		\rdg[wit={N23}]{nāditā jihvā}
		\rdg[wit={N3},alt={\damaged}]{\skp{\damaged}}}}
\pada{\app{\lem[wit={ceteri}]{tatpraveśo}
		\rdg[wit={N3}]{+\,[t]praveśo}}
	\app{\lem[wit={ceteri}]{hi}
		\rdg[wit={N23}]{di}}
	\app{\lem[wit={ceteri}]{tāluni}
		\rdg[wit={J10}]{tāluniṃ}}/}\\+}
\tl{
\pada{go\app{\lem[wit={ceteri}]{māṃsa}
		\rdg[wit={N19,V15,V1}]{māṃsaṃ}
		\rdg[wit={N23}]{māsaṃ}}%
	\app{\lem[wit={ceteri}]{bhakṣaṇaṃ}
		\rdg[wit={N3}]{bhakṣaṇe}}
	\app{\lem[wit={N3,P11,V3,J7,Gr3a,V1,J10,Jyo}]{tat tu}
		\rdg[wit={N23}]{\_\,rttu}
		\rdg[wit={V15}]{tac ca}
		\rdg[wit={N19}]{caitat}
		\rdg[wit={C6}]{hy etan}}}
\pada{mahāpātakanāśanam//}\\!}  %3.45
\end{tlg}


%\newpage
\begin{tlg}[hp03_045]
\tl{
\pada{jihvāpraveśasaṃbhūta}% jihva V19
\pada{\app{\lem[wit={J7,K3,N19,Jyo}]{vahninotpāditaḥ}% =P17
		\rdg[wit={V19},alt={°ditaṃ}]{vahninotpāditaṃ} % V19 om. va of vahni
		\rdg[wit={C6,C7},alt={°ditā}]{vahninotpāditā}
		\rdg[wit={N3},alt={°di\,+}]{vahninotpādi\,+}
		\rdg[wit={V3}]{vahninonnāpitā}
		\rdg[wit={J10}]{vahninottāpito}
		\rdg[wit={N23}]{vahnir utpāditaḥ}}
	\app{\lem[wit={J5,Gr2,Gr3a,N19,Jyo}]{khalu}% =P17;Gr1*
%		\rdg[wit={V17}]{daralu} % stemma point;
%		N26 vahnino[kṣ]āpitodarāt*, N9 vahni[natpi or: nāpi]tādaraṃ
		\rdg[wit={V3,J10}]{daraṃ}
		\rdg[wit={C6}]{surāḥ}
		\rdg[wit={N3},alt={\damaged}]{\skp{\damaged}}}/}\\+}
\tl{
\pada{\app{\lem[wit={C6,V3,Gr2,K3,N19,J10,Jyo}]{candrāt sravati} % śravati N19, snavati? J10
		\rdg[wit={C7}]{candraḥ sravati}
		\rdg[wit={V19}]{candrā dravati}
		\rdg[wit={N3}]{+\,+\,+\,+\,[t]i}}
	\app{\lem[wit={J7,V19,K3,J10,Jyo}]{yaḥ sāraḥ}
		\rdg[wit={N23}]{yaḥ sāra}
		\rdg[wit={N3,C6,N19}]{yat sāraṃ}
		\rdg[wit={V3},prewit=\texteng{(the same hemistich is inserted after \ref{III43})}]{yaḥ sāraṃ tasmād idam [m]akurvīta nityayuktaḥ samāhitaḥ}
		\rdg[wit={C7}]{yaḥ samyak}}}
\pada{\app{\lem[wit={ceteri}]{sā}
		\rdg[wit={K3}]{sa}}
		syā\app{\lem[wit={ceteri},alt={amaravāruṇī}]{\skm{d }amaravāruṇī}
		\rdg[wit={J10}]{aṃmavāruṇī}}//}
	\NotIn{V1,V15}\myfn{%
In \getsiglum{V1} the second half is added in the margin sec. m.:
\devnote{tasmā[tsa]rati ya[tsā]raṃ sā syād amaravāruṇī}.}\\!} % 3.48
\end{tlg}

\newpage
\begin{tlg}[hp03_046]
\tl{
\pada{\app{\lem[wit={V3,Gr3a,V15,Jyo}]{mūrdhnaḥ}
		\rdg[wit={J10}]{mūrdhneḥ}
		\rdg[wit={J7}]{mūrddhūḥ}
		\rdg[wit={J5,N19}]{mūrddhaṃ}
		\rdg[wit={N3}]{mūrddhvaḥ}
		\rdg[wit={V1}]{mūrddhva}
		\rdg[wit={N23}]{bhūrddhaḥ}
		\rdg[wit={C6}]{ūrdhvaṃ}}
ṣoḍaśa\app{\lem[wit={J5,K3,C7,V15,Jyo}]{pattrapadma}
		\rdg[wit={N3,C6,V3,J7,V19,V1,J10}]{padmapattra}
		\rdg[wit={N19}]{patrapatra}
		\rdg[wit={N23},alt={\om}]{\skp{\om}}}galitaṃ % galitaḥ N19
		prāṇād avāptaṃ
	\app{\lem[wit={ceteri},alt={haṭhād}]{haṭhā\skp{d}}
		\rdg[wit={V3}]{haṭhāṃ}}-}\\+}
\tl{
\pada{\app{\lem[wit={ceteri},alt={ūrdhvāsyo}]{\skm{d }ūrdhvāsyo} % J8 °smo?
		\rdg[wit={N23}]{ūrdhvosyo}
		\rdg[wit={C7}]{ūrdhvosya}
		\rdg[wit={V3}]{varddhāsyo}}
	\app{\lem[wit={ceteri}]{rasanāṃ}
		\rdg[wit={N19}]{rasanā}
		\rdg[wit={N23}]{ramanā}}
	\app{\lem[wit={N3,C6,J7,Gr3a,V15,Jyo}]{niyamya}
		\rdg[wit={N23,N19}]{niyasya}
		\rdg[wit={V1}]{ca yāmya}
		\rdg[wit={V3,J10}]{vidhāya}}
	\app{\lem[wit={ceteri}]{vivare}% +P11
		\rdg[wit={N23}]{vicare}
		\rdg[wit={Gr3a}]{vivaraṃ}
		\rdg[wit={C6}]{vidhivat}} % °vac chaktiṃ C6
	\app{\lem[wit={ceteri}]{śaktiṃ}
		\rdg[wit={J7}]{śaktiḥ}} parāṃ % parā J10
	\app{\lem[wit={ceteri}]{cintayet} % citayet N19
		\rdg[wit={N23}]{cintayat}
		\rdg[wit={K3,C7,Jyo}]{cintayan}}/}\\+}
\tl{
\pada{\app{\lem[wit={N3,C6,V3,C7,N19,V15,V1,Jyo}]{utkallola}
		\rdg[wit={J10}]{uttakallola}
		\rdg[wit={J7,V19}]{tatkallola}
		\rdg[wit={K3}]{tatkalola}
		\rdg[wit={N23}]{taptalola}}%
	\app{\lem[wit={N3,C6,Gr2,Gr3a,N19,V15,V1,Jyo}]{kalājalaṃ}
		\rdg[wit={V3,J10}]{jalākulaṃ}}
	\app{\lem[wit={N3,C6,J7,Gr3a,N19,V1,Jyo}]{ca}
		\rdg[wit={V3,J10}]{su}
		\rdg[wit={N23}]{ya}
		\rdg[wit={V15},alt={\om}]{\skp{\om}}}
		vimalaṃ % vimala N3, vimalā C6
	\app{\lem[wit={ceteri}]{dhārāmṛtaṃ}
		\rdg[wit={Jyo}]{dhārāmayaṃ}}\marmas yaḥ pibet}\\+} % vimalā C6
\tl{
\pada{\app{\lem[wit={ceteri}]{nirdoṣaḥ sa}
		\rdg[wit={V1}]{nirdoṣaṃ sa}
		\rdg[wit={N19}]{nirdoṣo 'sya}
		\rdg[wit={Jyo}]{nirvyādhiḥ sa}}
	mṛṇāla\app{\lem[wit={ceteri}]{komala}
		\rdg[wit={N23}]{komale}}%
	\app{\lem[wit={J5,C6,V3,J7,Gr3a,J10},alt={tanur}]{tanu\skp{r}}% +J5,N24
		\rdg[wit={N23}]{tanu}
		\rdg[wit={N3,P11,N19,V15,V1,Jyo}]{vapur}}r
		yogī ciraṃ jīvati//}% jogī V19, yoga[ṃ] C8ac
	\myfn{\getsiglum{Jyo} has a different verse order from here.}\\!}  %3.46
\end{tlg}

%\newpage
\begin{tlg}[hp03_047]
\tl{
\pada{\app{\lem[wit={ceteri}]{cumbantī}
		\rdg[wit={N23}]{vipitīṃ}}
	yadi \app{\lem[wit={ceteri}]{lambikāgram aniśaṃ jihvā}
		\rdg[wit={K3,C7}]{lampikāgram aniśaṃ jihvā}}
	\app{\lem[wit={ceteri}]{rasa} % also V3mg
		\rdg[wit={V3,J10}]{śiraḥ}}syandinī}\\+}
\tl{
\pada{\app{\lem[wit={ceteri}]{sakṣārā}
		\rdg[wit={N3,V19,N19}]{sākṣārā}
		\rdg[wit={J10}]{sakṣāra}}
	\app{\lem[wit={N3,J5}]{kaṭukātha}% = VM
		\rdg[wit={J7,Gr3a,V15,Jyo}]{kaṭukāmla} % +V3marg
		\rdg[wit={N23}]{vaṭukāmla}
%		\rdg[wit={J8}]{kaṭukāmna} % sakṣīrodakatikta P11
		\rdg[wit={V1}]{kaṭukāsa}
		\rdg[wit={V3,J10}]{kaṭukādya}
		\rdg[wit={N19}]{kaṭutikta}
		\rdg[wit={C6}]{kaṭutyakta}}
	\app{\lem[wit={ceteri}]{dugdha}
		\rdg[wit={J7}]{dugdhaṃ}
		\rdg[wit={N23}]{du}}%
	\app{\lem[wit={ceteri}]{sadṛśī}
		\rdg[wit={V19}]{sādṛśī}
		\rdg[wit={J7}]{sadṛśīṃ}
		\rdg[wit={N3,J5,V1}]{sadṛśā}
		\rdg[wit={N19,V15}]{lavaṇā}}
	\app{\lem[wit={ceteri}]{madhvājya} % also V3mg
		\rdg[wit={V3,J10}]{madhvādya}
		\rdg[wit={N19}]{vaddhājya}}%
	\app{\lem[wit={ceteri}]{tulyā} % also V3pc, talyā C6
		\rdg[wit={V3}]{tulyāṃ}
		\rdg[wit={J10}]{tulyaṃ}}%
	\app{\lem[wit={J5,Gr2,Gr3a}]{thavā}% +N24
		\rdg[wit={N3,C6,V3,N19,V1,J10,Jyo}]{tathā}
		\rdg[wit={V15}]{savā}}/} \\+}
\tl{
\pada{vyādhīnāṃ haraṇaṃ % vyādhināṃ V1
	\app{\lem[wit={ceteri}]{jarāntakaraṇaṃ} % °karaṇa V15
		\rdg[wit={V19,K3}]{jvarāntakaraṇaṃ}
		\rdg[wit={C7}]{jvarāntaḥkaraṇaṃ}
		\rdg[wit={C6}]{jarāpraśamanaṃ}}
	\app{\lem[wit={C6,V3,J7,V15,J10}]{śāstrāgamodgīraṇaṃ}
		\rdg[wit={V1}]{śastrāṃgamodgīraṇaṃ}
		\rdg[wit={N3,Jyo}]{śāstrāgamodīraṇaṃ}% śastrārthagamodīraṇaṃ J5
		\rdg[wit={N23}]{śāstrapramodīraṇaṃ}
		\rdg[wit={Gr3a,N19}]{śāstrāgamoddhāraṇaṃ}}} \\+}
\tl{
\pada{\app{\lem[wit={ceteri},alt={tasya syād}]{tasya syā\skp{d}}
		\rdg[wit={N23}]{tasyād}}d
	a\app{\lem[wit={ceteri},alt={amaratvam}]{\skp{a}maratva\skp{m}}
		\rdg[wit={N23}]{amarakṣam}
		\rdg[wit={V3}]{aramatvam}
		\rdg[wit={Gr3a}]{iha siddhir}}m
	aṣṭa\app{\lem[wit={N3,J5,P11,V1},alt={guṇavat}]{guṇava\skp{t}} % guṇava J5
		\rdg[wit={V15}]{guṇāvat}
		\rdg[wit={C6,V3,Gr2,V19,N19,J10,Jyo}]{guṇitaṃ}
		\rdg[wit={K3,C7}]{guṇitā}}%
	\app{\lem[wit={J5,P11,C6,K3,C7,V1,N19,V15,J10,Jyo},alt={siddhāṅganā}]{\skm{t }siddhāṅganā}
		\rdg[wit={N3,V3,N23,V19}]{siddhāṅgaṇā}% +J17
		\rdg[wit={J7}]{siddhāṅgānā}}%
	\app{\lem[wit={ceteri}]{karṣaṇam}
		\rdg[wit={N23}]{karṣaṇā}}//}\\!}  %3.47
\end{tlg}

%\newpage
\begin{tlg}[hp03_048]
\tl{
\pada{\app{\lem[wit={ceteri}]{ekaṃ}
	\rdg[wit={C7}]{eka}
	\rdg[wit={N23}]{evaṃ}}
\app{\lem[wit={ceteri}]{sṛṣṭi} % śṛṣṭi N3
		\rdg[wit={N19}]{dṛṣṭi}}%
	\app{\lem[wit={ceteri}]{mayaṃ} % +P11; maya N23
		\rdg[wit={C6}]{midaṃ}
		\rdg[wit={N19}]{layaṃ}} bījaṃ}
\pada{ekā mudrā \app{\lem[wit={ceteri}]{ca}% eka V15
		\rdg[wit={C7,N19}]{tu}} khecarī/}\\+}
\tl{
\pada{eko
	\app{\lem[wit={ceteri}]{devo}% +J5
		\rdg[wit={N23}]{devā}
		\rdg[wit={N3}]{nirā°}}
	\app{\lem[wit={V3,V1,Jyo}]{nirālamba}% +YCM
		\rdg[wit={J7,Gr3a}]{nirālambaś}
		\rdg[wit={N23}]{nirāśambaś}
		\rdg[wit={J5,C6,N19,V15}]{nirālambo}
		\rdg[wit={J10}]{nirālambaṃ}
		\rdg[wit={N3}]{°laṃbo deva}}}
\pada{\app{\lem[wit={N3,C6,V3,N19,V1,J10,Jyo}]{ekā}
		\rdg[wit={Gr3a}]{caikā}
		\rdg[wit={N23}]{cakā}
		\rdg[wit={J7}]{caiṣā}
		\rdg[wit={J5,V15}]{hy ekā}}vasthā manonmanī//}
	\anm{= 4.44*1}\\!}  %3.48
\end{tlg}

\newpage
\startaltrecension
\begin{alttlg}[hp03_048_1]
\tl{
\pada{\app{\lem[wit={Jyo}]{suṣiraṃ}
		\rdg[wit={J10}]{sukhiraṃ}
		\rdg[wit={V3}]{suciraṃ}} jñānajanakaṃ}
\pada{pañca\app{\lem[wit={J10,Jyo}]{srotaḥ}
		\rdg[wit={V3}]{śrotaḥ}}samanvitam/}\\+}
\tl{
\pada{\app{\lem[wit={Jyo}]{tiṣṭhate}
		\rdg[wit={V3}]{tiṣṭhaṃti}
		\rdg[wit={J10}]{tiṣṭhaṃtī}} % tiṣṭhati J10pc
		khecarī mudrā}
\pada{tasmin śūnye nirañjane//}
	\sgwit{V3,J10,Jyo} \anm{= 4.25*2}\\!} % 3.52
\end{alttlg}
\endaltrecension

\startgray
%\newpage
\begin{tlg}[hp03_049] % metre: Mandākrāntā
\tl{
\pada{\app{\lem[wit={Gr2}]{pātāle yad viśati}
		\rdg[wit={C6}]{pātālād yad viśati}
		\rdg[wit={J11}]{pātāle yadvitaya}
		\rdg[wit={V15}]{pātāle yadvitanta}
		\rdg[wit={N19}]{pātāle yadinaya}
		\rdg[wit={V1}]{pātāle yadvita}
		\rdg[wit={V3}]{yat prāleyaṃ cāpihita}
		\rdg[wit={J5}]{yat prāleyaṃ pihita}
		\rdg[wit={J10}]{yat prāleya pihita}
		\rdg[wit={Jyo}]{yat prāleyaṃ prahita}}
	\app{\lem[wit={V15,J11,Jyo}]{suṣiraṃ}
		\rdg[wit={C6}]{suśiraṃ}
		\rdg[wit={V3,Gr2,J10}]{sukhiraṃ} % sukhira N23
		\rdg[wit={J5}]{sukhire}
		\rdg[wit={N19}]{sukhīraṃ}
		\rdg[wit={V1}]{stu[v]imaṃ/me}}
	\app{meru\lem[wit={C6}]{mūle tad asti}
		\rdg[wit={J7}]{mūle yad asti}
		\rdg[wit={J5}]{mūle yad astī}
		\rdg[wit={N19,V15,J11}]{mūle tad asmin}
		\rdg[wit={N23}]{mūle pakṣasti} % merū N23
		\rdg[wit={V1}]{mūlad}
		\rdg[wit={V3}]{mūrddhyataḥthyaṃ}
		\rdg[wit={J10}]{mūrdhni sthitaṃ}
		\rdg[wit={Jyo}]{mūrdhāntarasthaṃ}}}\\+}  %
\tl{
\pada{\app{\lem[wit={C6},alt={tattvaṃ caitat}]{tattvaṃ caita\skp{t}}
		\rdg[wit={Gr2,N19}]{tadvac caitat} % tadvacaitat J7
		\rdg[wit={J11}]{tadvac caitā}
		\rdg[wit={V15}]{taddac caitat}
		\rdg[wit={V3,J10,Jyo}]{tasmiṃs tattvaṃ}% =P17
		\rdg[wit={J5}]{tasmitvaṃ}
		\rdg[wit={V1}]{asmi[ṃ]s tatvaṃ yat}}t
	pravadati % °vadaṃti J5
	\app{\lem[wit={ceteri},alt={sudhīs}]{sudhī\skp{s}}
		\rdg[wit={J11}]{sudhī}
		\rdg[wit={N23}]{sudhās}}%
	\app{\lem[wit={ceteri},alt={tan mukhaṃ}]{\skm{s }tan mukhaṃ}
	\rdg[wit={C6}]{tat sukhaṃ}}
	\app{\lem[wit={ceteri}]{nimnagānām}
		\rdg[wit={N23}]{niṣagmanāṃ}}/}\\+} % J10 gloss? korthaḥ nāḍīnāṃ
\tl{
\pada{\app{\lem[wit={ceteri}]{candrāt sāraḥ}
		\rdg[wit={V1}]{candrasāro}
		\rdg[wit={V15}]{candrā sāraḥ}
		\rdg[wit={J5,C6}]{candrāt sāraṃ}
		\rdg[wit={N19}]{candraḥ sāraḥ}}
	\app{\lem[wit={ceteri},alt={sravati/śravati}]{sravati} % śravati N19,V15,J5
		\rdg[wit={P11}]{grasati}
		\rdg[wit={N23}]{rapati}
		\rdg[wit={V1}]{[sra]vaṃtyai}}\myfn{\getsiglum{V15} jumps to Jālandharabandha (3.67) from here. For the lost part (3.50--66) \getsiglum{J14} is used instead.} % one illegible akṣara before
	\app{\lem[wit={ceteri},alt={vapuṣas}]{vapuṣa\skp{s}}
		\rdg[wit={J10}]{vapuṣes}
		\rdg[wit={V3}]{vapayuṣes}
		\rdg[wit={C6}]{vapuṣā}
		\rdg[wit={V15},alt={\om}]{\skp{\om}}}s tena
	\app{\lem[wit={ceteri},alt={mṛtyur}]{mṛtyu\skp{r}}
		\rdg[wit={V3,J10}]{mṛtyun}
		\rdg[wit={V15},alt={\om}]{\skp{\om}}}r narāṇāṃ}\\+}
\tl{
\pada{\app{\lem[wit={ceteri}]{taṃ}
		\rdg[wit={Jyo}]{tad}}
	\app{\lem[wit={ceteri},alt={badhnīyāt}]{badhnīyā\skp{t}}% yāta V3
		\rdg[wit={N23}]{cha\,\_\,yāt}
		\rdg[wit={V15},alt={\om}]{\skp{\om}}}%
	\app{\lem[wit={J5,C6,J11},alt={sukaraṇamṛdā}]{\skm{t }sukaraṇamṛdā}
		\rdg[wit={N19}]{pakaraṇamṛdā}
		\rdg[wit={V1}]{kakaraṇam amṛtaṃ}
		\rdg[wit={V3,J10}]{sukaraṇam atho}
		\rdg[wit={Jyo}]{sukaraṇam adho}
		\rdg[wit={J7}]{sukhakaram atho}
		\rdg[wit={N23}]{sukhakaraṇam artho}
		\rdg[wit={V15},alt={\om}]{\skp{\om}}}
	\app{\lem[wit={ceteri}]{nānyathā}
		\rdg[wit={N23}]{nāmarthā}
		\rdg[wit={V15},alt={\om}]{\skp{\om}}}
	\app{\lem[wit={C6,V3,Gr2,N19,J10,Jyo}]{kāya}
		\rdg[wit={J5,V1,J11}]{kārya}
		\rdg[wit={V15},alt={\om}]{\skp{\om}}}siddhiḥ//}\marma
\NotIn{N3,Gr3a}\myfn{%
\getsiglum{Gr4b} has this verse immediately after \ref{III39},
\getsiglum{N3,Gr3a} in Ch. 4 (4.25), and
%\devnote{pātālād vā vipati śikhare merumūle tadāstā,
%tatvaṃ caitat pravadati susaṃmukhaṃ nimnaśanāṃ/\\
%caṃdrāt srāvaḥ śravati vapuṣas tena mṛtyur narāṇāṃ,
%taṃ badhnīyāt svakaraṇamṛnā nānyathā kāryasiddhi//}\\
\getsiglum{Gr2} in both Ch. 3 and 4.} % but not in N2!
\anm{=\,4.25}\\!}
\end{tlg}
\endgray

%\newpage
\begin{ava}[hp03_050]
\app{\lem[wit={J7,Gr3a}]{mūlabandhaḥ}
\rdg[wit={N3,C6,V3,N19,V1,J10,Jyo}]{atha mūlabandhaḥ} % bandha V3,N19 ##
\rdg[wit={C7}]{atha mūle bandhaḥ}
\rdg[wit={N23,J11},alt={\om}]{\skp{\om}}}%
	\myfn{\getsiglum{V3,Jyo} have the Mūlabandha section after the Uḍḍiyāna. Cf. 3.6.}//
\end{ava}

\begin{tlg}[hp03_050]
\tl{
\pada{\app{\lem[wit={ceteri}]{pārṣṇi}
	\rdg[wit={N23}]{pādima}}bhāgena saṃpīḍya} % saṃpījya N23
\pada{yoni% yonīm V1
\app{\lem[wit={ceteri},alt={ākuñcayed}]{\skm{m }ākuñcaye\skp{d}}% +J5
	\rdg[wit={N3}]{ākuṃcaned}
	\rdg[wit={N23}]{ākuṃ}}%
\app{\lem[wit={ceteri},alt={gudam}]{\skm{d }gudam}
	\rdg[wit={V1,J10}]{dṛḍhaṃ}
	\rdg[wit={N23},alt={\om}]{\skp{\om}}}/}\\+}
\tl{
\pada{apānam ūrdhvam ākṛṣya}
\pada{mūlabandho%
\app{\lem[wit={ceteri}]{'yam ucyate}% +C7,VM
	\rdg[wit={K3}]{'yam īritaḥ}
	\rdg[wit={C6,V3}]{'yam iṣyate}% ##
	\rdg[wit={N3}]{mayiṣyate} % J5 jumps to the next verse: mūlabandhā(page break)di yoginaḥ.
	\rdg[wit={Jyo}]{'bhidhīyate}}//}\\!}
\end{tlg}

\begin{tlg}[hp03_051]
\tl{
\pada{\app{\lem[wit={GrB,J11,V1,Jyo},alt={adhogatim}]{adhogati\skp{m}} % better, =HR
	\rdg[wit={N3,Gr2,Gr3a,N19,J10}]{adhogatam}}% lost J5,G4
\app{\lem[wit={N3,C6,Gr2,N19,J11,J10},alt={apānaṃ vai}]{\skm{m }apānaṃ vai}
	\rdg[wit={Jyo}]{apānaṃ vā}% = vai, due to Sandhi
	\rdg[wit={V3}]{apānaṃ ca}
	\rdg[wit={Gr3a}]{apānaṃ tu}
	\rdg[wit={V1}]{apānaivam}}}
\pada{\app{\lem[wit={ceteri}]{ūrdhvagaṃ}
	\rdg[wit={N3}]{mūrddhagaṃ}
	\rdg[wit={V3}]{vidyūrdhagaṃ}} kurute
\app{\lem[wit={N3,C6,V3,J11,V1,J10,Jyo}]{balāt}
	\rdg[wit={Gr2,Gr3a,N19}]{haṭhāt}}/}\\+}
\tl{
\pada{\app{\lem[wit={ceteri}]{ākuñcanena}
	\rdg[wit={J10}]{ākuñcya tena}}
\app{\lem[wit={ceteri}]{taṃ}% +C7,P11
	\rdg[wit={V19}]{ta}
	\rdg[wit={K3}]{te}
	\rdg[wit={C6}]{tu}}
\app{\lem[wit={ceteri},alt={prāhur}]{prāhu\skp{r}}% prāhu V3,V1,J10; +J11 s.l.
	\rdg[wit={N19,J11}]{grāhyaṃ}}}%
\pada{\app{\lem[wit={ceteri},alt={mūlabandhaṃ}]{\skm{r }mūlabandhaṃ}% mūlabaṃ J11
	\rdg[wit={J10}]{mūlabandho}}
\app{\lem[wit={P11,C6,Gr2,Gr3a}]{tu}% = source
	\rdg[wit={N3,V3,N19,J11,V1,J10,Jyo}]{hi}} yoginaḥ//}\\!}
\end{tlg}

\newpage
\begin{tlg}[hp03_052]
\tl{
\pada{\app{\lem[wit={ceteri}]{gudaṃ}
	\rdg[wit={N19}]{gulpha}
	\rdg[wit={C6}]{pārṣṇi°}}
\app{\lem[wit={N3,V3,Gr3a,J10,Jyo}]{pārṣṇyā tu} % prob. V19, pāṣṇyā V3
	\rdg[wit={J7}]{pārśnī tu}
	\rdg[wit={N23}]{pādarmyāṃ tu}
	\rdg[wit={N19,J11,V1}]{pārṣṇyā ca} % pārṣṇā J11
	\rdg[wit={C6}]{°nā gudam}}
	\app{\lem[wit={ceteri}]{saṃpīḍya} % pījya N23, °pīḍye V3
	\rdg[wit={C6}]{āpīḍya}}}
\pada{\app{\lem[wit={N3,C6,V3,Gr2,V1,J10,Jyo},alt={vāyum ā°}]{vāyum ā\skp{°}} % +J11 s.l.; read pāyum?
	\rdg[wit={N19,J11}]{vāyunā}
	\rdg[wit={Gr3a}]{yonim ā°}}kuñcaye%
\app{\lem[wit={ceteri},alt={balāt}]{\skm{d }balāt}
	\rdg[wit={J7}]{balat}}/}\\+}
\tl{
\pada{vāraṃ vāraṃ % 1st vāra V1; 2nd vāra V3,J10
\app{\lem[wit={N3,C6,V3,N19,J11,V1,J10,Jyo}]{yathā}
	\rdg[wit={Gr2,Gr3a}]{tathā}}
	cordhvaṃ}
\pada{samāyāti samīraṇaḥ//}\\!}
\end{tlg}

%\newpage
\begin{tlg}[hp03_053]
\tl{
\pada{prāṇāpānau % prāṇa° V1
	\app{\lem[wit={ceteri}]{nādabindū}
	\rdg[wit={N3,V3,N19,J10}]{nādabindu}}}
\pada{mūlabandhena
\app{\lem[wit={ceteri}]{caikatām}
	\rdg[wit={C6,N19}]{caikatā}
	\rdg[wit={N23}]{cakataṃ}
	\rdg[wit={V3}]{caikataḥ}}/}\\+}
\tl{
\pada{gatvā yogasya % yagasya J10ac
\app{\lem[wit={N3,C6,V3,J7,V19,K3,J11,V1,Jyo}]{saṃsiddhiṃ}
	\rdg[wit={N23,C7,N19}]{saṃsiddhir}
	\rdg[wit={J10}]{saṃsiddhyaiḥ}}}
\pada{\app{\lem[wit={V3,Jyo}]{yacchato}% +J5?
	\rdg[wit={C6}]{yakṣyato}
	\rdg[wit={N3}]{yichato}
	\rdg[wit={Gr3a,N19}]{gacchato} % gakṣato V19
	\rdg[wit={J7}]{gacchate}
	\rdg[wit={N23}]{gacchatā}
	\rdg[wit={J11}]{gacchata°}
	\rdg[wit={V1}]{prāpnoty e°}
	\rdg[wit={J10}]{pamāta}}
\app{\lem[wit={ceteri}]{nātra}
	\rdg[wit={J11}]{°sya na}
	\rdg[wit={V1}]{°va na}
	\rdg[wit={J10}]{tra na}} saṃśayaḥ//}\\!} % V19 written in margin pr.m.
\end{tlg}


\begin{tlg}[hp03_054]
\tl{
\pada{apānaprāṇa%
\app{\lem[wit={ceteri},alt={°yor aikyaṃ}]{\skp{°}yor aikyaṃ} % +C7
	\rdg[wit={N23}]{°yor aikya}
	\rdg[wit={J10}]{°yor aikye}
	\rdg[wit={K3}]{°yoś caikyaṃ}
	\rdg[wit={J11}]{°yor aikyāt}}}
\pada{\app{\lem[wit={ceteri}]{kṣayo}
	\rdg[wit={N23,J11}]{kṣayān}} mūtrapurīṣayoḥ/}\\+}
\tl{
\pada{yuvā bhavati vṛddho'pi}
\pada{satataṃ
	mūla\app{\lem[wit={ceteri}]{bandhanāt}
	\rdg[wit={V19}]{bandhataḥ}}//}\myfn{\getsiglum{N23} adds the following verse here:
	\devnote{bandhamūlaṃ yena tena tena vighnāṃ nivāritaḥ/ ajarāmaratāṃ yāti yathā pañcamukho haraḥ//}}\\!}
\end{tlg}

%\newpage
\begin{tlg}[hp03_055]
\tl{
\pada{\app{\lem[wit={ceteri}]{apāne}
	\rdg[wit={Jyo}]{apāna}
	\rdg[wit={V3,J7}]{apānaṃ}}
\app{\lem[wit={ceteri}]{cordhvage jāte}
	\rdg[wit={V19}]{cordhvage yāte} % or yāne? V19
	\rdg[wit={J10}]{cordhvam āpāte}
	\rdg[wit={Jyo}]{ūrdhvage jāte}}}
\pada{\app{\lem[wit={P11,C6,Gr2,Gr3a,N19}]{saṃprāpte}% = GŚ keep; saṃprāprau P11
	\rdg[wit={V3}]{saṃyāte}
	\rdg[wit={N3,J5,J11,V1,J10,Jyo}]{prayāte}} % damaged G4
\app{\lem[wit={N3,V3,Jyo}]{vahnimaṇḍalaṃ}
	\rdg[wit={C6,Gr2,Gr3a,N19,J11,V1}]{vahnimaṇḍale} % °la J7ac?
	\rdg[wit={J10}]{nābhimaṇḍalaṃ}}/}\\+}
\tl{
\pada{\app{\lem[wit={ceteri}]{tadānala}
	\rdg[wit={N19}]{tadānale}
	\rdg[wit={V1}]{tathānale}
	\rdg[wit={C7,J10}]{tathānala}}śikhā dīrghā} % śiṣā V19; dīryā V1
\pada{\app{\lem[wit={N3,C6,V3,Gr2,N19}]{vardhate vāyunāhatā}% hatāḥ C6
	\rdg[wit={Gr3a}]{baṃdhane vāyunāhatā} % bandhena C7
	\rdg[wit={J10}]{kriyate vāyunāhatāḥ}
	\rdg[wit={Jyo}]{jāyate vāyunāhatā}
	\rdg[wit={J11}]{vāyunā vardhate hatā}
	\rdg[wit={V1}]{vāyunā preritā tathā}}//}\\!}
\end{tlg}


%\newpage
\begin{tlg}[hp03_056]
\tl{
\pada{\app{\lem[wit={ceteri}]{tato}
	\rdg[wit={V1}]{yātā}}
\app{\lem[wit={C6}]{yātau}% yāttau J5
	\rdg[wit={V1,Jyo}]{yāto}% +M1
	\rdg[wit={J10}]{yāte}
	\rdg[wit={N3}]{yāmau}
	\rdg[wit={J7,Gr3a,J11}]{jātau}
	\rdg[wit={N23}]{jātā}
	\rdg[wit={V3}]{jāto}
	\rdg[wit={N19}]{vahnim}}\marmas
\app{\lem[wit={J7,Gr3a,J11,V1,Jyo}]{vahnyapānau}% +P11; °panau J7
	\rdg[wit={N3}]{vahnipānau}% °pātau J5
	\rdg[wit={J10}]{vahniyonau}
	\rdg[wit={C6}]{bāhyapānau}
	\rdg[wit={N23}]{baṃdhapānau}
	\rdg[wit={V3}]{vardhapānai}
	\rdg[wit={N19}]{apānai ca}}}
\pada{\app{\lem[wit={G4,C6,N19,J11,Jyo}]{prāṇam uṣṇa}
	\rdg[wit={V3,J7}]{prāṇam uṣma}
	\rdg[wit={N23}]{prāṇamura}
	\rdg[wit={V19,C7}]{prāṇamukta}
	\rdg[wit={K3}]{prāṇamuktaṃ}%
	\rdg[wit={N3,J5,V1,J10}]{prāṇamūla}}
\app{\lem[wit={ceteri}]{svarūpakam}
	\rdg[wit={J10}]{svarūpakaḥ}
	\rdg[wit={V1}]{svarūpakau}
	\rdg[wit={C7}]{svarūpavat}}/}\\+} % °rūpaṃke N23
\tl{
\pada{\app{\lem[wit={Gr2,Gr3a,V1,Jyo}]{tenātyanta}
	\rdg[wit={N3}]{tenātyantaṃ}% tenotyataṃ J5
	\rdg[wit={V3}]{tenābhyanta}
	\rdg[wit={J10}]{tenābhyantaḥ}
	\rdg[wit={C6}]{tenāyaṃna}
	\rdg[wit={J11}]{tatotyantaṃ}
	\rdg[wit={N19}]{tailābhyaṃtaḥ}}%
\app{\lem[wit={ceteri}]{pradīptas tu} % °diptas V3
	\rdg[wit={J11,V1}]{pradīpas tu}
	\rdg[wit={N19}]{pradīpāsau}}}
\pada{\app{\lem[wit={ceteri}]{jvalano dehajas tathā} % +P11; jvalanā N23
	\rdg[wit={C6}]{jvalato dehatas tadā}
	\rdg[wit={J10}]{kuto dehakṣayas tadā}}//}\\!}
\end{tlg}

% V1 adds here vāmanāpanamū(?) at the end of the folio. The new folio begins with tena.

\newpage
\begin{tlg}[hp03_057]
\tl{
\pada{tena kuṇḍalinī suptā} % kuṃḍalanī N3
\pada{\app{\lem[wit={N3,C6,V3,Gr2,J11,V1,J10,Jyo}]{saṃtaptā} % sa pra° V3
	\rdg[wit={Gr3a,N19}]{satataṃ}}
\app{\lem[wit={N3,C6,Gr2,J11,J10,Jyo}]{saṃprabudhyate}% ddhy J7, yudhyate N23
	\rdg[wit={V1}]{saṃprabudhyati}
	\rdg[wit={V3}]{sa prabudhyate}
	\rdg[wit={K3}]{sā prabuddhyate}
	\rdg[wit={C7}]{sā prabodhyate}
	\rdg[wit={N19}]{saṃprabodhyate}
	\rdg[wit={V19}]{sānubodhyate}}/}\\+}
\tl{
\pada{daṇḍāhatā bhujaṅgīva} % hatya N23
\pada{\app{\lem[wit={N3,P11,J11,Jyo}]{niśvasya} % ,postwit={C6,(=\,M1,M3)}
	\rdg[wit={V3,V1,J10}]{viśvasya}
	\rdg[wit={K3}]{niścayam}
	\rdg[wit={C7}]{niścayād}
	\rdg[wit={Gr2,V19,N19}]{niścitam}}\marmas
\app{\lem[wit={ceteri}]{ṛjutāṃ vrajet} % ṛjvatāṃ V3
	\rdg[wit={N3}]{rujutāṃ vṛjet}
	\rdg[wit={J10}]{rijutām iyāt}}//}\\!}
\end{tlg}

%\newpage
\begin{tlg}[hp03_058]
\tl{
\pada{bilaṃ
\app{\lem[wit={N3,Gr2,V19,N19,J11,V1,Jyo}]{praviṣṭeva}
	\rdg[wit={C6}]{praviṣṭe ca}% +P11
	\rdg[wit={V3}]{praviṣṭaṃ ca}
	\rdg[wit={J10}]{praviṣṭaś ca}
	\rdg[wit={Gr3a},alt={\om}]{\skp{\om}}}
\app{\lem[wit={ceteri}]{tato}
	\rdg[wit={N23}]{to}
	\rdg[wit={Gr3a},alt={\om}]{\skp{\om}}}}
\pada{\app{\lem[wit={ceteri}]{brahma}
	\rdg[wit={N23}]{tha\,\_}
	\rdg[wit={Gr3a},alt={\om}]{\skp{\om}}}%
\app{\lem[wit={ceteri}]{nāḍyantaraṃ}
	\rdg[wit={C6,J11}]{nāḍyāntaraṃ}
	\rdg[wit={J10}]{nāḍyantare}
	\rdg[wit={Gr3a},alt={\om}]{\skp{\om}}} vrajet/}
	\lineom{ab}{Gr3a}\\+} %
\tl{
\pada{\app{\lem[wit={ceteri},alt={tasmān}]{tasmā\skp{n}}
	\rdg[wit={K3}]{tato}}%
\app{\lem[wit={ceteri}]{\skm{n }nityaṃ}
	\rdg[wit={N19}]{nityo}} mūlabandhaḥ}
\pada{kartavyo yogibhiḥ sadā//}\\!}
\end{tlg}

%\newpage
\begin{ava}[hp03_059]
\app{\lem[wit={N23}]{atha uḍḍiyānabandhaḥ}
	\rdg[wit={J10}]{atha uḍḍiyānaṃ bandhaḥ}
	\rdg[wit={N19}]{atha uḍḍīyāṇabandhaḥ}
	\rdg[wit={Jyo}]{atha uḍḍīyānabandhaḥ}
	\rdg[wit={K3}]{uḍḍīyānabandhaḥ}
	\rdg[wit={J7}]{uḍḍiyāṇaṃ bandhaḥ}
	\rdg[wit={N3}]{athoḍḍīyāṇaṃ}
	\rdg[wit={V3}]{athoḍḍiyāṇaṃ}
	\rdg[wit={C6}]{athoḍiyānaṃ}
	\rdg[wit={C7}]{athoḍḍīyānabandhaḥ}
	\rdg[wit={V1}]{athoḍyāṇabaṃdhaḥ}
	\rdg[wit={V19,J11},alt={\om}]{\skp{\om}}}/%
%	\myfn{\getsiglum{V3,Jyo} have this section before the Mūlabandha.}
\end{ava}

\begin{tlg}[hp03_059]
\tl{
\pada{\app{\lem[wit={N3,C6,V3,J7,V19,K3,Jyo}]{baddho}
	\rdg[wit={C7,N19}]{bandho}
	\rdg[wit={V1,J10}]{ūrdhvo}
	\rdg[wit={J11}]{ūrdhvā}
	\rdg[wit={N23}]{vidrā}}
\app{\lem[wit={ceteri}]{yena suṣumṇāyāṃ} % su<ṣu>mnāyāṃ K3; °yā V3
	\rdg[wit={J10}]{kṣitaḥ suṣumṇāyāḥ}}}
\pada{\app{\lem[wit={ceteri},alt={prāṇas}]{prāṇa\skp{s}}
	\rdg[wit={C6,N19,V1}]{prāṇam}}%
\app{\lem[wit={N3,J7,V19,J11,J10,Jyo},alt={tūḍḍīyate}]{\skm{s }tūḍḍīyate}
	\rdg[wit={V3}]{tūḍiyate}
	\rdg[wit={N23}]{tudīyate}
	\rdg[wit={K3}]{tūḍūyate}
	\rdg[wit={C7}]{tūyate}
	\rdg[wit={N19}]{uḍḍīyate}
	\rdg[wit={C6}]{uḍiyate}
	\rdg[wit={V1}]{uḍyayate}}
\app{\lem[wit={ceteri}]{yataḥ}
	\rdg[wit={C7}]{punaḥ}}/}\\+}
\tl{
\pada{\app{\lem[wit={ceteri},alt={tasmād}]{tasmā\skp{d}}
	\rdg[wit={J7}]{tasmātu}% sic
	\rdg[wit={J10}]{tasmāc ca}}%
\app{\lem[wit={V19,C7,Jyo},alt={uḍḍīyanākhyo},post=\texteng{\emph{m.c.}}]{\skm{d }uḍḍīyanākhyo}
	\rdg[wit={J7,J10}]{uḍḍiyānākhyo}
	\rdg[wit={V1}]{uḍḍiyāṇākhyo}
	\rdg[wit={N23}]{uddiyānākhyo}
	\rdg[wit={N19}]{uḍḍīyāṇākhyo}
	\rdg[wit={N3,J11}]{uḍḍīyanākhye}
	\rdg[wit={V3}]{uḍiyāṇākhye}
	\rdg[wit={C6}]{uḍiyānākhyaṃ}
	\rdg[wit={K3}]{uḍḍīyamāno}}\marma%
\app{\lem[wit={ceteri}]{'yaṃ}
	\rdg[wit={K3}]{sau}
	\rdg[wit={C6}]{tad}
	\rdg[wit={J10},alt={\om}]{\skp{\om}}}}
\pada{yogibhiḥ
\app{\lem[wit={ceteri}]{samudāhṛtaḥ}
	\rdg[wit={C6,V3,N19}]{samudāhṛtaṃ}}//}\\!}
\end{tlg}

\begin{tlg}[hp03_060]
\tl{
\pada{\app{\lem[wit={Gr3a,Jyo}]{uḍḍīnaṃ}
	\rdg[wit={V3}]{uḍīṇaṃ}
	\rdg[wit={C6,J11}]{uḍyānaṃ}
	\rdg[wit={N3,J7}]{uḍyāṇaṃ}
	\rdg[wit={N23,J10}]{uḍḍiyānaṃ}
	\rdg[wit={V1}]{uḍḍiyāṇaṃ}
	\rdg[wit={N19}]{uḍḍīyāṇaṃ}}
\app{\lem[wit={ceteri}]{kurute}
	\rdg[wit={J7}]{kṛyate}
	\rdg[wit={N19}]{kṛte}} yasmād}
\pada{a\app{\lem[wit={C6,V3,Gr3a,N19,V1,J10,Jyo},alt={aviśrāntaṃ}]{\skp{a}viśrāntaṃ}
	\rdg[wit={N3,J5,P11}]{aviśrāṃta}
	\rdg[wit={J7,J11}]{aviśrānto}
	\rdg[wit={N23}]{aviśrāntā}} mahākhagaḥ/}\\+} % khaga V3
\tl{
\pada{\app{\lem[wit={Gr2,J10}]{uḍḍiyānaṃ}
	\rdg[wit={Gr3a,J11,Jyo}]{uḍḍīyānaṃ}
	\rdg[wit={N3,N19,V1}]{uḍḍīyāṇaṃ}
	\rdg[wit={V3}]{uḍiyāṇaṃ}
	\rdg[wit={C6}]{uḍiyānaṃ}} 
ta\app{\lem[wit={ceteri},alt={eva}]{\skm{d }eva}
	\rdg[wit={V19}]{evaṃ}
	\rdg[wit={N19}]{evaḥ}} syā}% syā V3
\pada{\app{\lem[wit={N3,C6,V3,Gr2,N19,J11,V1,Jyo},alt={tatra}]{\skm{t }tatra}
	\rdg[wit={J10}]{kṣetra}
	\rdg[wit={Gr3a}]{mūla}}\marmas bandho
\app{\lem[wit={J5,C6,J7,J11}]{vidhīyate}
	\rdg[wit={ceteri}]{'bhidhīyate} % bhi[dh]īyate V19
	\rdg[wit={N23}]{nigadyate}}//}\\!}
\end{tlg}


%\newpage
\begin{tlg}[hp03_061]
\tl{
\pada{\app{\lem[wit={ceteri}]{udare}
	\rdg[wit={V3}]{udarāt}}
\app{\lem[wit={J5,C6,J7,C7,V1,Jyo}]{paścimaṃ}
	\rdg[wit={N3,N23,J10}]{paścima}
	\rdg[wit={V3,V19,K3,N19,J11}]{paścime}}
\app{\lem[wit={C6,Gr2,V19,C7,J11,V1,J10,Jyo}]{tānaṃ}
	\rdg[wit={N3,J5,K3,N19}]{tāṇaṃ}
	\rdg[wit={V3}]{bhāge}}}\marmas
\pada{nābhe%r nābhed J7
\app{\lem[wit={ceteri},alt={ūrdhvaṃ}]{\skm{r }ūrdhvaṃ}
	\rdg[wit={J10}]{ūrdhve}}
\app{\lem[wit={ceteri}]{ca}
	\rdg[wit={N19,J10}]{tu}} kārayet/}\\+} % °rdhva-akāra° N23
\tl{
\pada{\app{\lem[wit={Gr2,V19,J10}]{uḍḍiyāno}
	\rdg[wit={K3,C7,J11,Jyo}]{uḍḍīyāno}
	\rdg[wit={V1}]{uḍḍiyāṇo}
	\rdg[wit={N3,N19}]{uḍḍīyāṇo}
	\rdg[wit={C6}]{uḍīyāno}
	\rdg[wit={V3},alt={\om}]{\skp{\om}}} 
\app{\lem[wit={N3,C6,Gr2,N19,J11,V1,J10,Jyo},alt={asau}]{\skm{hy }asau}
	\rdg[wit={C7}]{asam}
	\rdg[wit={V19,K3}]{ayaṃ}
	\rdg[wit={V3},alt={\om}]{\skp{\om}}} bandho} % baṃdhā N23
\pada{mṛtyumātaṅga\app{\lem[wit={ceteri}]{kesarī}
	\rdg[wit={J11}]{khecarī}
	\rdg[wit={V3},alt={\om}]{\skp{\om}}}//} \lineom{cd}{V3}\\!}
\end{tlg}

\newpage
\begin{tlg}[hp03_062]
\tl{
\pada{\app{\lem[wit={Gr2,V19,J10}]{uḍḍiyānaṃ} % uddi° N23
	\rdg[wit={V1}]{uḍḍiyāṇaṃ}
	\rdg[wit={K3,C7,J11,Jyo}]{uḍḍīyānaṃ}
	\rdg[wit={N3,N19}]{uḍḍīyāṇaṃ}
	\rdg[wit={C6}]{uḍiyānaṃ}
	\rdg[wit={V3}]{uḍiyāṇaṃ}} tu
\app{\lem[wit={ceteri}]{sahajaṃ}
	\rdg[wit={J7}]{yaḥ sahate}}}
\pada{\app{\lem[wit={ceteri}]{guruṇā}
	\rdg[wit={V3}]{gurūṇāṃ}} kathitaṃ
\app{\lem[wit={N3,C6,V3,N19,J11,V1,J10,Jyo}]{sadā}
	\rdg[wit={Gr2,Gr3a}]{yathā}}\marma/}\\+}
\tl{
\pada{\app{\lem[wit={ceteri},alt={abhyased/-set}]{abhyase\skp{d}}%
%	\rdg[wit={N3,N19,V1}]{abhyased}
	\rdg[wit={N23}]{abhyāsen}
	\rdg[wit={V3}]{abhyāsāt}
	\rdg[wit={C6}]{abhyāsa°}}%
\app{\lem[wit={N3,J5},prewit=\texteng{\textless\ °tadras tu},alt={astatandras tu}]{\skm{d }astatandras tu}% =source,+M3
	\rdg[wit={N19}]{asya taṃtrasya}
	\rdg[wit={J11}]{asya taṃtraṃ tu}
	\rdg[wit={J7,Gr3a}]{tad atandras tu}
	\rdg[wit={N23}]{na taṃdras tu}
	\rdg[wit={C6}]{°taḥ svatantras tu}
	\rdg[wit={V1}]{yo hy atandras}
	\rdg[wit={V3,J10,Jyo}]{satataṃ yas tu}}\marma} % +M4, J11 s.l.
\pada{\app{\lem[wit={ceteri}]{vṛddho}
	\rdg[wit={N23}]{vṛddhā}}'pi
\app{\lem[wit={N3,J5,P11,V3,N19,V1,J10}]{taruṇo bhavet}
	\rdg[wit={C6,Gr2,Gr3a,J11,Jyo}]{taruṇāyate}}//}\\!}
\end{tlg}


\begin{tlg}[hp03_063]
\tl{
\pada{\app{\lem[alt={\ante nābher \add},nosep]{}
	\rdg[wit={C6}]{pāṭhāntaram}}%
\app{\lem[wit={ceteri},alt={nābher}]{nābhe\skp{r}}
	\rdg[wit={J7}]{nābhed}}r ūrdhva%m
\app{\lem[wit={ceteri},alt={adhaś cāpi}]{\skm{m }adhaś cāpi}
	\rdg[wit={Gr3a}]{adho vāpi}
	\rdg[wit={V1}]{adhaḥkāya}
	\rdg[wit={C6}]{avasthāpya}}}
\pada{\app{\lem[wit={C6,Gr2,Gr3a,J11,J10,Jyo}]{tānaṃ}
	\rdg[wit={N3,P11,V3,N19,V1}]{tāṇaṃ}} % tāpyaṃ
	kuryā%t
\app{\lem[wit={ceteri},alt={prayatnataḥ}]{\skm{t }prayatnataḥ}
	\rdg[wit={J10}]{ca yatnataḥ}}/}\\+}
\tl{
\pada{\app{\lem[wit={ceteri},alt={ṣaṇmāsam}]{ṣaṇmāsa\skp{m}}
	\rdg[wit={V1,J10}]{yogī sam°}}%
\app{\lem[wit={N3,P11,Gr2,J10},alt={abhyasan}]{\skm{m }abhyasa\skp{n}} %
	\rdg[wit={V3,K3,C7,N19,J11,V1,Jyo}]{abhyasen}% J5,G4??
	\rdg[wit={V19}]{abhyaseni}
	\rdg[wit={C6}]{ca samabhyān}}n mṛtyuṃ} % matyu N23
\pada{\app{\lem[wit={ceteri}]{jayaty eva na saṃśayaḥ}
	\rdg[wit={C6}]{mūlaṃ jayaty asaṃśayaḥ}}//}\\!}
\end{tlg}


%\newpage
\begin{tlg}[hp03_064]
\tl{
\pada{sati % sa[t]i V19
\app{\lem[wit={ceteri}]{vajrāsane}
	\rdg[wit={N23}]{vajrāsanau}
	\rdg[wit={N3}]{vajrāsanaṃ}} pādau\marma} % +G7,G11; jānu M3,G5
\pada{\app{\lem[wit={ceteri},post=\texteng{(dhārayad \getsiglum{J10})}]{karābhyāṃ dhārayed dṛḍham} % °ye<d>dṛ° J7,V19,V3,J11
	\rdg[wit={V1}]{karābhyā dhārayaṃ dṛḍhaṃ}
	\rdg[wit={N3}]{karābhyāṃ kāraye dṛḍhaṃ}% +J5
	\rdg[wit={N23}]{karā\,\_\,sandhāraye dṛḍhe}}/}\\+}
\tl{
\pada{gulpha\app{\lem[wit={ceteri}]{deśa}% desa V3
	\rdg[wit={N19}]{deśe}
	\rdg[wit={N3}]{deśaṃ}}%
\app{\lem[wit={N3,C6,V3,Gr2,J11,V1,J10,Jyo}]{samīpe ca}
	\rdg[wit={K3,C7,N19}]{samīpaṃ ca}
	\rdg[wit={V19}]{samīpaṃ tu}}}\marmas
\pada{\app{\lem[wit={ceteri}]{kandaṃ}
	\rdg[wit={V19}]{kaṃdhaṃ}
	\rdg[wit={C7}]{skandaṃ}}
\app{\lem[wit={C6,Gr2,Gr3a,N19,J11,Jyo}]{tatra}
	\rdg[wit={V3,J10}]{tacca}
	\rdg[wit={N3}]{tava}
	\rdg[wit={V1}]{tasya}}
\app{\lem[wit={C6,Gr2,Gr3a,N19,J11,Jyo}]{prapīḍayet}
	\rdg[wit={N3,V3,V1,J10}]{prapīḍyate}}//}%
	\myfn{In \getsiglum{Jyo} this verse appears much later as 3.114 in the printed edition.}\\!}
\end{tlg}


\begin{tlg}[hp03_065]
\tl{
\pada{\app{\lem[wit={ceteri},alt={paścimaṃ tānam}]{paścimaṃ tāna\skp{m}}% +J5
	\rdg[wit={P11,V1}]{paścimaṃ tāṇam}
	\rdg[wit={N3,V3}]{paścimatāṇam}}%
\app{\lem[wit={N3,C6,V3,J7,Gr3a,J11},alt={udare}]{\skm{m }udare}
	\rdg[wit={N23}]{udara}
	\rdg[wit={N19}]{udaraṃ}
	\rdg[wit={V1,J10}]{upari}}}
\pada{\app{\lem[wit={ceteri},alt={kārayed}]{kāraye\skp{d}}
	\rdg[wit={J10}]{pīḍayed}}%
\app{\lem[wit={N3,C6,V3,J7,J11,V1,J10},alt={dhṛdaye {\crux}gale\crux}]{\skm{d }dhṛdaye {\crux}gale\crux} % kāraye hṛdaye V3, kārayed-hṛ°? V1
	\rdg[wit={N23}]{dhṛdaye gataiḥ}
	\rdg[wit={V19}]{udare hṛdi}
	\rdg[wit={C7}]{cibukaṃ hṛdi}
	\rdg[wit={K3}]{cibukaṃ hṛdā}
	\rdg[wit={N19}]{vṛddhidaṃ śanaiḥ}}\marma/}\\+}
\tl{
\pada{śanaiḥ \app{\lem[wit={ceteri}]{śanair yathā} % J11 s.l.
	\rdg[wit={J11}]{śanair [yato]}
	\rdg[wit={N23},alt={\om}]{\skp{\om}}}
\app{\lem[wit={N3,V3,Gr3a,V1},alt={prāṇas}]{prāṇa\skp{s}}
	\rdg[wit={Gr2}]{prāṇās}% +J5
	\rdg[wit={C6,N19,J11}]{prāṇaṃ}
	\rdg[wit={J10}]{prāṇo}}}% prāṇā/os J7
\pada{\app{\lem[wit={N3,C6,J7,V19,K3,N19},alt={tunda}]{\skm{s }tunda}
	\rdg[wit={V3,V1}]{tuda}
	\rdg[wit={N23}]{taṃda}
	\rdg[wit={C7}]{tadā}
	\rdg[wit={J10}]{nāḍī} % +J11 s.l.
	\rdg[wit={J11}]{ūrddhva}}%
\app{\lem[wit={N3,P11,Gr2,V19,C7,J11,V1}]{saṃdhiṃ} % +C7
	\rdg[wit={V3,N19,J10}]{saṃdhi}
	\rdg[wit={K3}]{siṃddhiṃ}
	\rdg[wit={C6}]{siddhiṃ}}
\app{\lem[wit={N3,C6,V3,Gr2,N19,J11,V1}]{na}
	\rdg[wit={Gr3a}]{ca}
	\rdg[wit={J10}]{ni°}} gacchati\marma//} \NotIn{Jyo}\\!}
\end{tlg}

%\newpage
\begin{tlg}[hp03_066]
\tl{
\pada{sarveṣām eva bandhānām} % caiva C6
\pada{u\app{\lem[wit={ceteri},alt={uttamo}]{\skp{u}ttamo}
	\rdg[wit={N19}]{uttamaṃ}}
\app{\lem[wit={Gr2,Gr3a,J11,J10,Jyo}]{hy uḍḍiyānakaḥ}
	\rdg[wit={V1}]{hy uḍḍiyāṇakaḥ}
	\rdg[wit={N3}]{hy uḍḍīyāṇakaḥ}
	\rdg[wit={N19}]{hy uḍḍīyāṇakaṃ}
	\rdg[wit={C6}]{hy uḍiyānakaḥ}
	\rdg[wit={V3}]{hy uḍiyāṇakaḥ}}/}\\+}
\tl{
\pada{\app{\lem[wit={Gr2,V19,J10,Jyo}]{uḍḍiyāne}
	\rdg[wit={K3,C7,J11}]{uḍḍīyāne}
	\rdg[wit={N3,N19}]{uḍḍīyāṇe}
	\rdg[wit={V1}]{uḍḍiyāṇe}
	\rdg[wit={C6}]{uḍiyāne}
	\rdg[wit={V3}]{uḍiyāṇe}}
\app{\lem[wit={ceteri}]{dṛḍhe}
	\rdg[wit={Gr2,Gr3a}]{kṛte}}
	\app{\lem[wit={ceteri}]{bandhe}
	\rdg[wit={C6}]{baddhe}}} % ##? 3 x e -> a N23
\pada{\app{\lem[wit={N3,P11,V3,N19,J11,J10,Jyo}]{muktiḥ}
	\rdg[wit={V1}]{muktiṃ}
	\rdg[wit={C6,Gr2,Gr3a}]{mūlaṃ}}
\app{\lem[wit={N3,P11,V3,N19,J11,Jyo}]{svābhāvikī}
	\rdg[wit={J10}]{svābhāvakī}
	\rdg[wit={C6,Gr3a,V1}]{svābhāvikaṃ}
	\rdg[wit={J7}]{svabhāvikaṃ}
	\rdg[wit={N23}]{bhāvikaṃ}} bhavet//}\\!}
\end{tlg}


\newpage
\begin{ava}[hp03_067]
\app{\lem[wit={C6,C7,J11,V1,J10,Jyo}]{atha jālandharabandhaḥ}
	\rdg[wit={J7,K3}]{jālandharabandhaḥ}
	\rdg[wit={N23}]{atha nāśaṃdharabandhaḥ}
	\rdg[wit={N19}]{atha jālaṃdharībaṃdhaḥ}
	\rdg[wit={N3}]{atha jālāṃdharaḥ}
	\rdg[wit={V3}]{atha jālaṃdharaṃ}
	\rdg[wit={V19},alt={\om}]{\skp{\om}}}/
\end{ava}

% V15 resumes with hṛdaye in Pada a. Ca. 18 verses are omitted.
\begin{tlg}[hp03_067]
\tl{
\pada{kaṇṭham ākuñcya hṛdaye} % ākuṃci V3
\pada{sthāpaye%c% °ye Gr3a,N3,V3,J10, °yet* N19, °yed V15
\app{\lem[wit={C6,J7,Jyo},alt={cibukaṃ dṛḍham}]{\skm{c }cibukaṃ dṛḍham}
	\rdg[wit={P11,V3,Gr3a,V1,J10}]{dṛḍham icchayā}% +M1,J5,G4 ##
	\rdg[wit={N3}]{dṛḍham īchayā}
	\rdg[wit={N19}]{dṛḍham icchayet}
	\rdg[wit={V15}]{dṛḍhaniścayāt}
	\rdg[wit={N23},alt={\om}]{\skp{\om}}}/}
	\anm{\getsiglum{V15} resumes}\\+}
\tl{
\pada{bandho jālandharākhyo'yaṃ} % baṃdha J10; jālāṃdharā° V3; °ākṣo N19
\pada{\app{\lem[wit={V15}]{amṛtāvyayakārakaḥ}% Marmasthāna
	\rdg[wit={P11,V3,N19}]{amṛtavyayakārakaḥ} % kāraka V3
	\rdg[wit={N3}]{amṛtāvapakārakaḥ}
	\rdg[wit={V19}]{amṛtākṣayakārakaḥ}
	\rdg[wit={C7}]{amṛtakṣayakārakaḥ}
	\rdg[wit={K3}]{amṛtākṣarakārakaḥ}
	\rdg[wit={Gr2},post=\texteng{(mṛtaḥ \getsiglum{N23})}]{mṛtyor mṛtyuḥ paro mataḥ}
	\rdg[wit={C6}]{mṛtyumātaṃgakesarī}
	\rdg[wit={V1,J10,Jyo}]{jarāmṛtyuvināśakaḥ}}//}\label{Jaala1}\\!}
\end{tlg}


\begin{tlg}[hp03_068]
\tl{
\pada{\app{\lem[wit={N3,C6,V3,N19,V15,V1,J10,Jyo}]{badhnāti hi} % baddhāti N19
	\rdg[wit={N23}]{badhnāti ha}
	\rdg[wit={Gr3a}]{badhnātīha}
	\rdg[wit={J7}]{badhnātīhṛ}}
\app{\lem[wit={C6,J7,V19,J10,Jyo},post=\texteng{(sirā \getsiglum{Jyo})}]{śirā}
	\rdg[wit={V3,N23,K3,C7,N19,V15,V1}]{śiro}
	\rdg[wit={N3}]{śilā}}%
\app{\lem[wit={ceteri},alt={jālam}]{jāla\skp{m}}
	\rdg[wit={V3}]{jālāṃ}}}%
\pada{\app{\lem[wit={N3,C6,J7,Gr3a,N19,V15,J10,Jyo},alt={adhogāmi}]{\skm{m }adhogāmi}
	\rdg[wit={N23}]{adhogāmī}
	\rdg[wit={V3}]{madhyegāmi}
	\rdg[wit={V1}]{nādhāyāti}} nabhojalam/}\\+}
\tl{
\pada{tato jālandharo bandhaḥ} % jālāṃ° N3; °dharā V15; baṃdha V3, baṃdho C6
\pada{\app{\lem[wit={N3,C6,V3,J7,N19,V15,Jyo}]{kaṇṭha}
	\rdg[wit={N23,Gr3a,V1,J10}]{kaṇṭhe}}%
	\app{\lem[wit={ceteri}]{duḥkhaugha}}nāśanaḥ//}\\!} % nāśanaṃ V3
\end{tlg}

%\newpage
\begin{tlg}[hp03_069]
\tl{
\pada{jālandhare kṛte bandhe} % jālādhare V19, jālāṃdhare N3,V3, °dhara? V15
\pada{kaṇṭhasaṃkocalakṣaṇe/}\\+}
\tl{
\pada{na pīyūṣaṃ
\app{\lem[wit={ceteri},alt={pataty}]{pata\skp{ty}}
	\rdg[wit={V19}]{prayāty}
	\rdg[wit={N23}]{kṣaraty}}ty agnau}
\pada{na ca vāyuḥ
\app{\lem[wit={N3,C6,V3,Gr2,N19,V15,J10}]{pradhāvati}
	\rdg[wit={Gr3a,V1,Jyo}]{prakupyati}}//}\\!}
\end{tlg}


\begin{tlg}[hp03_070]
\tl{
\pada{kaṇṭha\app{\lem[wit={ceteri}]{saṃkocanenaiva}
	\rdg[wit={V1}]{saṃkocane dehe}}}
\pada{\app{\lem[wit={N3,P11,J7,Gr3a,V15,J10,Jyo}]{dve nāḍyau}% ddhe J7ac, ddhau J7pc
	\rdg[wit={N19},postwit=\texteng{\getsiglum{J7}\postcorr}]{dvau nāḍyau}% +J5
	\rdg[wit={C6}]{dvināḍyau}
	\rdg[wit={N23}]{\_\,nā\,\_}
	\rdg[wit={V1}]{nāḍyau ca}
	\rdg[wit={V3},alt={\gap}]{\skp{\gap}}}
\app{\lem[wit={N3,C6,V15,V1,J10,Jyo},alt={stambhayed}]{stambhaye\skp{d}}% °bhaye N3,J10
	\rdg[wit={Gr2,Gr3a,N19}]{stambhite}
	\rdg[wit={V3},alt={\gap}]{\skp{\gap}}}%
\app{\lem[wit={N3,J5,V1,J10,Jyo},alt={dṛḍhaṃ}]{\skm{d }dṛḍhaṃ}
	\rdg[wit={P11,Gr2,V19,C7,N19}]{dhruvam}
	\rdg[wit={K3}]{dhruve}
	\rdg[wit={V15}]{dhṛvaṃ}
	\rdg[wit={C6}]{dhuram}
	\rdg[wit={V3},alt={\gap}]{\skp{\gap}}}/}\\+}
\tl{
\pada{\app{\lem[wit={N3,C6,J7,Gr3a,N19,V15,Jyo},alt={madhyacakram}]{madhyacakra\skp{m}}
	\rdg[wit={N23}]{madhyakram}
	\rdg[wit={V3}]{madhye cakram}
	\rdg[wit={J10}]{madhyaṃ cakram}
	\rdg[wit={V1},alt={\om}]{\skp{\om}}}m idaṃ
\app{\lem[wit={ceteri}]{jñeyaṃ}
	\rdg[wit={N23}]{ya}
	\rdg[wit={V1},alt={\om}]{\skp{\om}}}}
\pada{ṣoḍaśādhārabandhanam//} \lineom{cd}{V1}\\!}
\end{tlg}


%\newpage
\begin{tlg}[hp03_071]
\tl{
\pada{bandhatrayam idaṃ śreṣṭhaṃ}
\pada{\app{\lem[wit={N3,J7,N19},alt={mahāsiddhair}]{mahāsiddhai\skp{r}} % siddhai N19
	\rdg[wit={V1,J10,Jyo}]{mahāsiddhaiś}
	\rdg[wit={N23}]{mahāsiddhe}
	\rdg[wit={C6,V15}]{mahāsiddhi}% +J5
	\rdg[wit={V3}]{mahāsīha}}%
\app{\lem[wit={N3,C6,V3,Gr2,Gr3a,N19},alt={niṣevitam}]{\skm{r }niṣevitam}% niśe° N3
	\rdg[wit={V1,J10,Jyo}]{ca sevitaṃ}% prajāyate! J5
	\rdg[wit={V15}]{pradāyakaṃ}}/}\\+}
\tl{
\pada{sarveṣāṃ
\app{\lem[wit={N3,C6,V3,J7,Jyo}]{haṭha}% +J5
	\rdg[wit={N23,N19,V15,V1,J10}]{yoga} % yoma N19
	}tantrāṇāṃ}
\pada{\app{\lem[wit={ceteri}]{sādhanaṃ}
	\rdg[wit={N23}]{sāranaṃ}} yogino viduḥ//}
	\NotIn{Gr3a}\myfn{In \getsiglum{Jyo} this verse is found after \ref{III74}.}\\!}
\end{tlg}

\newpage
\startgray
\begin{tlg}[hp03_072]
\tl{
\pada{adhastā%t
\app{\lem[wit={V1,J10},alt={kuñcanenāśu}]{\skm{t }kuñcanenāśu}
	\rdg[wit={Gr2}]{kuñcanenaiva}}}
\pada{kaṇṭha\app{\lem[wit={V1,J10}]{saṃkocane kṛte}
	\rdg[wit={Gr2}]{saṃkocanena ca}}/}\\+}
\tl{
\pada{\app{\lem[wit={V1}]{madhye}
	\rdg[wit={Gr2,J10}]{madhya}} paścimatānena}
\pada{syāt prāṇo brahmanāḍigaḥ//}
\sgwit{Gr2,V1,J10}\myfn{\getsiglum{N3,C6,V3,N19,V15,Jyo} have this verse in chp. 2.
\getsiglum{Gr2,V1,J10} have this in both chapters.
\getsiglum{Gr3a} does not have it at all.} \anm{= 2.46}\\!}
\end{tlg}
\endgray

%\newpage
\begin{tlg}[hp03_073]
\tl{
\pada{mūlasthānaṃ
\app{\lem[wit={N3,C6,V3,V15,V1,Jyo}]{samākuñcya}
	\rdg[wit={Gr2,N19}]{samākṛṣya}
	\rdg[wit={Gr3a,J10},alt={\om}]{\skp{\om}}}}
\pada{\app{\lem[wit={Gr2,V15,Jyo}]{uḍḍiyānaṃ}
	\rdg[wit={V1}]{uḍḍiyāṇaṃ}
	\rdg[wit={N3,N19}]{uḍḍīyāṇaṃ}
	\rdg[wit={C6}]{uḍiyānaṃ}
	\rdg[wit={V3}]{uḍiyāṇaṃ}
	\rdg[wit={Gr3a,J10},alt={\om}]{\skp{\om}}} tu kārayet/} \lineom{ab}{Gr3a,J10}\\+}
\tl{
\pada{\app{\lem[wit={V3,J7,Gr3a,V15,Jyo}]{iḍāṃ ca piṅgalāṃ} % piṃgulāṃ V3
	\rdg[wit={N3,C6,N23,N19}]{iḍā ca piṅgalā}
	\rdg[wit={V1}]{iḍāpiṃgalāṃ}
	\rdg[wit={J10},alt={\om}]{\skp{\om}}}
\app{\lem[wit={ceteri}]{baddhvā}
	\rdg[wit={N19}]{baddhā}
	\rdg[wit={J10},alt={\om}]{\skp{\om}}}}
\pada{vāhaye%t
\app{\lem[wit={J5,C6,Gr2,Gr3a,N19,V1},alt={paścimaṃ}]{\skm{t }paścimaṃ}
	\rdg[wit={V3}]{paścimāṃ}
	\rdg[wit={N3,V15}]{paścimā}
	\rdg[wit={J10},alt={\om}]{\skp{\om}}
	\rdg[wit={Jyo}]{paścime}}
\app{\lem[wit={ceteri}]{patham}
	\rdg[wit={Jyo}]{pathi}
	\rdg[wit={J10},alt={\om}]{\skp{\om}}}//} \lineom{cd}{J10}\\!}
\end{tlg}


\begin{tlg}[hp03_074]
\tl{
\pada{\app{\lem[wit={ceteri}]{anenaiva vidhānena}
	\rdg[wit={J10}]{brahmasthānasthito rodhaḥ}}}
\pada{\app{\lem[wit={Gr2,V15,V1,J10,Jyo}]{prayāti}
	\rdg[wit={N3,J5,C6,V3,V19,C7}]{sevayet}% ##
	\rdg[wit={K3}]{[s]e[vay]e[t]}
	\rdg[wit={N19}]{vaśayet}}
\app{\lem[wit={V3,J7,V15,V1,J10,Jyo}]{pavano layam}
	\rdg[wit={N23}]{pavano lagaṃ}
	\rdg[wit={N3,J5,C6,K3,N19}]{pavanālayam}% ##
	\rdg[wit={C7}]{pavanānalam}
	\rdg[wit={V19}]{paścimānalaṃ}}/}\\+}
\tl{
\pada{tato na jāyate
\app{\lem[wit={N3,C6,V3,J7,C7,N19,V15,V1,J10,Jyo},alt={mṛtyur}]{mṛtyu\skp{r}}
	\rdg[wit={V19,K3}]{mṛtyu}
	\rdg[wit={N23}]{mṛtyuṃ}}}
\pada{\app{\lem[wit={Gr2,K3,C7,N19,V1,Jyo},alt={jarārogādikaṃ}]{\skm{r }jarārogādikaṃ}% N24
	\rdg[wit={J5,C6,V3}]{jarārogādikā} %##
	\rdg[wit={N3}]{jarārogādikas}
	\rdg[wit={V15}]{jarāmohādikaṃ}
	\rdg[wit={V19}]{jvaro rogādikas}
	\rdg[wit={J10}]{nāsya jarādikaṃ}}
\app{\lem[wit={ceteri}]{tathā}% +N24, damaged G4
	\rdg[wit={J5}]{vyathā}
	\rdg[wit={C6,V3}]{kathā}
	\rdg[wit={N3}]{tadā}}//}\label{III74}\\!}
\end{tlg}


\begin{ava}[hp03_075]
\app{\lem[wit={N3,C6,V3,N23,C7,N19,V15,V1,J10}]{atha}
	\rdg[wit={J7,Gr3a,Jyo},alt={\om}]{\skp{\om}}}
\app{\lem[wit={C6,V3,Gr2,K3,V15,V1,J10}]{viparītakaraṇī}
	\rdg[wit={N3}]{viparītakaraṇīṃ}
	\rdg[wit={N19}]{viparītakaraṇaṃ}
	\rdg[wit={C7}]{viparītakam}
	\rdg[wit={V19,Jyo},alt={\om}]{\skp{\om}}}/
\end{ava}

\startgray
\begin{tlg}[hp03_075]
\tl{
\pada{yat kiñcit
\app{\lem[wit={C6,V3,J7,Gr3a,N19,Jyo}]{sravate} % śravate N19
	\rdg[wit={N23}]{sravanaṃ}
	\rdg[wit={N3,V15,V1,J10},alt={\om}]{\skp{\om}}}
\app{\lem[wit={C6,J7,Gr3a,Jyo}]{candrād}
	\rdg[wit={V3}]{candra}
	\rdg[wit={N19}]{caṃdrāṃn}
	\rdg[wit={N23}]{ceda<<m>>}
	\rdg[wit={N3,V15,V1,J10},alt={\om}]{\skp{\om}}}}
\pada{amṛtaṃ
\app{\lem[wit={Gr2,Gr3a}]{divyarūpi ca}% rūpa ca J5
	\rdg[wit={N19}]{divyarūpiṇaṃ}
	\rdg[wit={C6,Jyo}]{divyarūpiṇaḥ}% =N3
	\rdg[wit={V3}]{divyarūpagaḥ}
	\rdg[wit={N3,V15,V1,J10},alt={\om}]{\skp{\om}}}/}\\+}
\tl{
\pada{tat sarvaṃ grasate
\app{\lem[wit={C6,V3,Gr2,Gr3a,Jyo}]{sūryas}
	\rdg[wit={N19}]{roho}
	\rdg[wit={N3,V15,V1,J10},alt={\om}]{\skp{\om}}}}
\pada{tena \app{\lem[wit={C6,V3,J7,Gr3a,N19}]{piṇḍaṃ}
	\rdg[wit={N23}]{piḍaṃ}
	\rdg[wit={Jyo}]{piṇḍo}
	\rdg[wit={N3,V15,V1,J10},alt={\om}]{\skp{\om}}}
\app{\lem[wit={Gr2,K3,C7,N19}]{vināśi ca}
	\rdg[wit={V19}]{vinasyati}
	\rdg[wit={C6,V3}]{jarāyutaṃ}% = N3
	\rdg[wit={Jyo}]{jarāyutaḥ}
	\rdg[wit={N3,V15,V1,J10},alt={\om}]{\skp{\om}}}//}
	\NotIn{N3,V15,V1,J10}\myfn{%
	\getsiglum{Gr1r} has this pair of verses in Ch. 4:
	\devnote{% J5 similar, G4 damaged
	yat kiṃcit sravate candrād amṛtaṃ \underline{divyarūpiṇaḥ}/
	tat sarvaṃ grasate sūryas tena piṇḍaṃ \underline{jarāyutaṃ}//
	tatrāsti karaṇaṃ divyaṃ sūryasya \underline{paribandhanaṃ}/
	gurūpadeśato jñeyaṃ na tu śāstrārthakoṭibhiḥ//}
	}\\!}
\end{tlg}
%	\anm{= 4.25*0}

%%% G7 omits these two verses too. G11,M3 have them at the end of this section. !!!

%\newpage
\begin{tlg}[hp03_076]
\tl{
\pada{\app{\lem[wit={C6,V3,Gr2,V19,C7,N19,Jyo}]{tatrāsti}
	\rdg[wit={K3}]{tato sti}
	\rdg[wit={N3,V15,V1,J10},alt={\om}]{\skp{\om}}} karaṇaṃ divyaṃ} % divya C7
\pada{sūryasya mukha%
\app{\lem[wit={ceteri}]{bandhanam} % entry in Marmasthāna
	\rdg[wit={Jyo}]{vañcanam}
	\rdg[wit={N3,V15,V1,J10},alt={\om}]{\skp{\om}}}\marma/}\\+}
\tl{
\pada{gurūpadeśato % guropa° N19
\app{\lem[wit={C6,V3,J7,Gr3a,N19,Jyo}]{jñeyaṃ}
	\rdg[wit={N23}]{\_\,yaṃ}
	\rdg[wit={N3,V15,V1,J10},alt={\om}]{\skp{\om}}}}
\pada{\app{\lem[wit={C6,V3,J7,Gr3a,N19,Jyo}]{na tu}
	\rdg[wit={N23}]{rttu}
	\rdg[wit={N3,V15,V1,J10},alt={\om}]{\skp{\om}}}
śāstrārthakoṭibhiḥ//} \NotIn{N3,V15,V1,J10}\\!}%  \anm{= 4.25*0}
\end{tlg}
\endgray

\newpage
\begin{tlg}[hp03_077]
\tl{
\pada{\app{\lem[wit={ceteri},alt={ūrdhvaṃ nābhir}]{ūrdhvaṃ nābhi\skp{r}}
	\rdg[wit={N23}]{ūrdhvanābhor}
	\rdg[wit={Jyo}]{ūrdhvanābher}
	\rdg[wit={C6}]{ūrdhvaṃ nābher}}%
\app{\lem[wit={C6,V3,V19,N19,V15,J10},alt={adhas tālur}]{\skm{r }adhas tālu\skp{r}}% adhaḥstālur V19, °tālūr V15
	\rdg[wit={N3,J7,K3,V1}]{adhas tālu}
	\rdg[wit={C7}]{adhas tālum}
	\rdg[wit={N23}]{asāluktar}
	\rdg[wit={Jyo}]{adhas tālor}}}%
\pada{\app{\lem[wit={ceteri},alt={ūrdhvaṃ}]{\skm{r }ūrdhvaṃ}
	\rdg[wit={N23,V19}]{ūrdhva}
	\rdg[wit={V1}]{ūrdhvo}}
	bhānur adhaḥ śaśī/}\\+} % adho C6; śaśi V15
\tl{
\pada{\app{\lem[wit={N3,Gr2,C7,V1,Jyo}]{karaṇī viparītākhyā}% +C7
	\rdg[wit={V3,V15}]{karaṇaṃ viparītākhyaṃ}% karaṇī °kṣaṃ J5
	\rdg[wit={C6,V19,K3,N19,J10},alt={\om}]{\skp{\om}}}}
\pada{guruvākyena
\app{\lem[wit={N3,V3,N23,V15,V1,Jyo}]{labhyate}
	\rdg[wit={J7,C7}]{gamyate}
	\rdg[wit={C6,V19,K3,N19,J10},alt={\om}]{\skp{\om}}}//} \lineom{cd}{C6,V19,K3,N19,J10}\\!}
\end{tlg}


%\newpage
\begin{tlg}[hp03_078]
\tl{
\pada{karaṇī \app{\lem[wit={ceteri}]{viparītākhyā}
	\rdg[wit={C6}]{viparītākhyaṃ}% °kṣaṃ J5
	\rdg[wit={N19}]{viparītākṣaṃ}
	\rdg[wit={C7,Jyo},alt={\om}]{\skp{\om}}}}
\pada{sarvavyādhi\app{\lem[wit={ceteri}]{vināśinī}
	\rdg[wit={N19}]{vināśanī}
	\rdg[wit={C6,V3}]{vināśanaṃ}% +J5
	\rdg[wit={C7,Jyo},alt={\om}]{\skp{\om}}}/} \lineom{ab}{C7,Jyo}\\+}
\tl{
\pada{nityam abhyāsayuktasya}
\pada{jaṭharāgni\app{\lem[wit={N3,J7,N19,V15,V1}]{vivardhanī}
	\rdg[wit={N23,Gr3a,J10,Jyo}]{vivardhinī}
	\rdg[wit={C6,V3}]{vivardhanaṃ}}//}\\!} % +J5
\end{tlg}


\begin{tlg}[hp03_079]
\tl{
\pada{āhāro bahulas tasya} % bahu<la>s C7
\pada{saṃpādyaḥ sādhakasya % °pādya N3
\app{\lem[wit={ceteri}]{tu}
	\rdg[wit={N23,K3,Jyo}]{ca}}/}\\+}
\tl{
\pada{\app{\lem[wit={N3,C6,V3,V1,J10,Jyo}]{alpāhāro}
	\rdg[wit={Gr2,V19,C7,N19,V15}]{anāhāro}
	\rdg[wit={K3}]{anāhāre}}
\app{\lem[wit={ceteri},alt={yadi bhaved}]{yadi bhave\skp{d}} % bhuved? C7
	\rdg[wit={J10}]{nirāhāraḥ}}}%
\pada{\app{\lem[wit={N3,V3,Gr3a,V15},alt={agnir dehaṃ}]{\skm{d }agnir dehaṃ}
	\rdg[wit={P11,Gr2,N19}]{agnidehaṃ}
	\rdg[wit={V1}]{deham agnir}
	\rdg[wit={C6}]{agnidāho}
	\rdg[wit={Jyo}]{agnir daha°}
	\rdg[wit={J10}]{kṣudhālasya}}
\app{\lem[wit={ceteri},alt={dahet}]{dahe\skp{t}} % dahe V3,N19
	\rdg[wit={P11,V15}]{haret}
	\rdg[wit={C6}]{bhavet}
	\rdg[wit={Jyo}]{°ti tat}
	\rdg[wit={J10}]{vaśe}}%
\app{\lem[wit={ceteri},alt={kṣaṇāt}]{\skm{t }kṣaṇāt}% +J5
	\rdg[wit={N3}]{kramāt}% =HR; ##?
	\rdg[wit={J7}]{tataḥ}
	\rdg[wit={J10}]{bhavet}}//}\\!}
\end{tlg}


\begin{tlg}[hp03_080]
\tl{
\pada{adhaḥ\app{\lem[wit={C6,Jyo}]{śirāś cordhva}
	\rdg[wit={N3,V3,V19,N19,V15,V1,J10}]{śiraś cordhva} % corddhaṃ V3, cordha V19
	\rdg[wit={J7,K3}]{śirā ūrdhva} % arddha J10; śiro J7pc, śirāḥ K3
	\rdg[wit={N23}]{śīrā ūrdhva}
	\rdg[wit={C7}]{śira ūrdhva}}%
\app{\lem[wit={N3,C6,Gr2,Gr3a,V15,Jyo}]{pādaḥ}
	\rdg[wit={V3,N19}]{pāda}
	\rdg[wit={V1,J10}]{pādau}}}
\pada{\app{\lem[wit={ceteri},alt={kṣaṇaṃ syāt}]{kṣaṇaṃ syā\skp{t}} % kṣaṇa N19,N3, kṣaṇaḥ C6
	\rdg[wit={V19}]{kṣīṇaṃ syāt}
	\rdg[wit={J10}]{lakṣaṇaṃ}}t prathame dine/}\\+}
\tl{
\pada{\app{\lem[wit={N3,C6,V3,V15,V1,J10,Jyo}]{kṣaṇāc ca}
	\rdg[wit={Gr2}]{kṣaṇāt tu}
	\rdg[wit={C7,N19}]{kṣaṇārdhaṃ} % ṃ om. N19
	\rdg[wit={K3}]{kṣaṇārdhe}
	\rdg[wit={V19},alt={\om}]{\skp{\om}}} 
	kiṃci%d
\app{\lem[wit={ceteri},alt={adhikam}]{\skm{d }adhika\skp{m}}
	\rdg[wit={N23}]{apika}
	\rdg[wit={V19},alt={\om}]{\skp{\om}}}}% kṣaṇārtu N23, apika N23 >> Śāradā?
\pada{\app{\lem[wit={ceteri},alt={abhyasec ca}]{\skm{m }abhyasec ca}
	\rdg[wit={J7}]{abhyasetva}
	\rdg[wit={N23}]{bhyarccayec ca}
	\rdg[wit={V19},alt={\om}]{\skp{\om}}} dine dine//} \lineom{cd}{V19}\\!}
\end{tlg}


\begin{tlg}[hp03_081]
\tl{
\pada{\app{\lem[wit={N3,C6,V3,V1}]{valiś ca}
	\rdg[wit={N23,Gr3a,N19,V15,J10,Jyo}]{valitaṃ}
	\rdg[wit={J7}]{calitaṃ}}
	\app{\lem[wit={ceteri}]{palitaṃ}
	\rdg[wit={C6}]{palitaś}} caiva}
\marma\pada{\app{\lem[wit={Jyo}]{ṣaṇmāsordhvaṃ na}% Marmasthāna
	\rdg[wit={N3,C6,V15}]{ṣaṇmāsārdhān na}% +M1,G5,6
	\rdg[wit={V3}]{ṣaṇmāsārdhaṃ na}
	\rdg[wit={Gr2,Gr3a}]{ṣaṇmāsārdhena} % +J5
	\rdg[wit={N19}]{ṣaṇmāsārdhe ca}
	\rdg[wit={V1,J10}]{ṣaṇmāsāt tu na}}
\app{\lem[wit={N3,C6,V3,V15,V1,J10,Jyo}]{dṛśyate}% +M1,G5,6
	\rdg[wit={Gr2,Gr3a,N19}]{naśyati}}/}\\+}
\tl{
\pada{\app{\lem[wit={ceteri}]{yāmamātraṃ tu}% jāma V19; mātras tu N19
	\rdg[wit={V15}]{yāmamātraṃ ca}
	\rdg[wit={C7}]{yo māsatraya}
	\rdg[wit={J10}]{māsatrayaṃ tu}} yo
\app{\lem[wit={ceteri},alt={nityam}]{nitya\skp{m}}
	\rdg[wit={N23}]{gnibhyam}}}%
\pada{\app{\lem[wit={ceteri},alt={abhyaset}]{\skm{m }abhyase\skp{t}}
	\rdg[wit={V19}]{aset}}t sa
\app{\lem[wit={ceteri}]{tu}
	\rdg[wit={J7}]{su}
	\rdg[wit={N19}]{ca}}
\app{\lem[wit={ceteri}]{kālajit}% kārajit N3ac
	\rdg[wit={N19}]{kālavit}}//}\\!}
\end{tlg}


\begin{altava}[hp03_082]
%\grau{
atratyā vajrolī
\app{\lem[wit={V19}]{granthānte likhitā}% liṣitā V19
	\rdg[wit={K3}]{granthāntare likhitā vartate}
	\rdg[wit={C7}]{granthāntare tu likhitāsīt}}/
\app{\lem[wit={K3,C7}]{kramaprāptāpy atra tyaktā} % oṃ kramaprāptyāpy  C7
	\rdg[wit={V19},alt={\om}]{\skp{\om}}}/
\app{\lem[wit={K3}]{asādhāraṇa}
	\rdg[wit={V19}]{asādhāraṇaṃ}
	\rdg[wit={C7}]{asāraṇa}}prāṇyanuṣṭheyatvāt tasyāḥ/ % prāṇuṣṭheyatvāt C7
	\sgwit{Gr3a}%
	\myfn{In \getsiglum{Gr3a} the Vajrolī section is found at the end of the work.}
%}
\end{altava}

\newpage
\begin{ava}[hp03_082]
atha vajrolī/\myfn{In \getsiglum{C6,J10} this header is found after \emph{vinā} of the next line.}
\end{ava}


\begin{tlg}[hp03_082]
\tl{
\pada{svecchayā vartamāno'pi}
\pada{\app{\lem[wit={C6,V3,J7,V19,V1,J10,Jyo},alt={yogoktair}]{yogoktai\skp{r}}
		\rdg[wit={N23}]{yogokair}
		\rdg[wit={N19}]{yogoktar}
		\rdg[wit={J5,V15}]{yogokta}% ## = DYŚ
		\rdg[wit={N3}]{yogoktaṃ}
		\rdg[wit={C7}]{niyamair}}%
	\app{\lem[wit={ceteri},alt={niyamair vinā}]{\skm{r }niyamair vinā} % V3 niyamai
		\rdg[wit={C7}]{vividhais tathā}}/}\\+}
\tl{
\pada{\app{\lem[wit={C6,V19,V15,V1,J10,Jyo}]{vajrolīṃ yo}
		\rdg[wit={V3,Gr2,C7,N19}]{vajrolī yo}
		\rdg[wit={N3}]{vajrālī yo}} % yo (l.br.) yo J10
	\app{\lem[wit={ceteri}]{vijānāti}% +K4
		\rdg[wit={Gr2}]{bhijānāti}}}
\pada{sa yogī \app{\lem[wit={ceteri}]{siddhibhājanam}
		\rdg[wit={N23},alt={°bhājanaḥ}]{siddhibhājanaḥ}
		\rdg[wit={J10}]{siddhimān bhavet}}\marma//}\\!}
\end{tlg}

\begin{tlg}[hp03_083]
\tl{
\pada{tatra % tava N3
	\app{\lem[wit={ceteri}]{vastu}
		\rdg[wit={N3}]{castu}
		\rdg[wit={N19}]{bheda}}dvayaṃ
	\app{\lem[wit={ceteri},alt={vakṣ(y)e}]{vakṣye}%
%		\rdg[wit={V3,V19}]{vakṣe}
		\rdg[wit={J7}]{manye}
		\rdg[wit={N23}]{api}}} % dvayam api N23
\pada{durlabhaṃ yasya kasya
	\app{\lem[wit={ceteri}]{cit}
		\rdg[wit={V15}]{tu}}/}\\+}
\tl{
\pada{kṣīraṃ \app{\lem[wit={N3,P11,V3,V19,C7,N19,V15,V1,Jyo}]{caikaṃ}% vaikaṃ C7
		\rdg[wit={J10}]{caiva}
		\rdg[wit={C6,Gr2}]{ekaṃ}}
		dvitīyaṃ tu} % °tiyaṃ V3
\pada{nārī  % nāḍī C6
	\app{\lem[wit={ceteri}]{ca}
		\rdg[wit={C7}]{tu}} vaśavartinī//}\\!} % vaśi°? N19
\end{tlg}


\begin{tlg}[hp03_084]
\tl{
\pada{\app{\lem[wit={N3,C6,J7,N19,V15,Jyo}]{mehanena}
		\rdg[wit={N23}]{mehanaiva}
		\rdg[wit={V19}]{mohanena}
		\rdg[wit={C7}]{mohanenā}
		\rdg[wit={V3}]{meḍhrenena}
		\rdg[wit={V1},post={\unm}]{meḍhreṇa}
		\rdg[wit={J10}]{mahānibhaṃ}}} % mehanena J10pc1, meḍhra J10pc2
	\app{\lem[wit={ceteri}]{śanaiḥ}
		\rdg[wit={V19}]{sadā}} samya% kasyag V15
\pada{\app{\lem[wit={N3,C6,V3,N23,V19,C7,N19,V15,Jyo},alt={ūrdhvākuñcanam}]{\skm{g }ūrdhvākuñcana\skp{m}} % ā cancelled in N4?
		\rdg[wit={J7}]{ūrdhva kiṃcanam}
		\rdg[wit={J10}]{kṛtvā kuñcanam}
		\rdg[wit={V1}]{gudākuñcanam}}m abhyaset/}\\+}
\tl{
\pada{puruṣo \app{\lem[wit={ceteri}]{vāpi nārī vā}% vāti N19
		\rdg[wit={C7}]{vāpi vā nārī}
		\rdg[wit={Jyo}]{'py atha vā nārī}}}
\pada{\app{\lem[wit={N3,C6,V3,N23,C7,N19,V15,J10,Jyo}]{vajrolī} % °li V15
		\rdg[wit={V19,V1}]{vajrolīṃ}
		\rdg[wit={J7}]{vajrolīḥ}}%
	\app{\lem[wit={ceteri}]{siddhim āpnuyāt}
		\rdg[wit={J7}]{siddhibhājanam}
		\rdg[wit={N23}]{siddhibhājanaḥ}}//}\\!}  %3.81
\end{tlg}

% om. N23
\begin{tlg}[hp03_085]
\tl{
\pada{\app{\lem[wit={N3,P11,V3,V19,N19,V15,Jyo}]{yatnataḥ} % yantrataḥ C7
		\rdg[wit={J7,V1,J10}]{prayatnataḥ}
		\rdg[wit={C6}]{prayatnāt}} % + J10pc
	\app{\lem[wit={N3,C6,V3,V19,C7}]{śaranālena}% sara° C6
		\rdg[wit={N19}]{śalanolena}
		\rdg[wit={V15}]{śatanārīṇāṃ}
		\rdg[wit={Jyo}]{śastanālena}
		\rdg[wit={J7,V1,J10}]{śironāle}}} % + J10pc
\pada{\app{\lem[wit={N3,C7,N19,V1,Jyo}]{phūtkāraṃ} % + J10pc
		\rdg[wit={V3}]{phutkāraṃ}
		\rdg[wit={V19,V15}]{pūtkāraṃ}
		\rdg[wit={J7,J10}]{phūtkāraḥ}
		\rdg[wit={C6}]{sphūtkāraṃ}}
	\app{\lem[wit={ceteri}]{vajra} % + J10pc
		\rdg[wit={J7,J10}]{kaṃbu}}%
	\app{\lem[wit={C6,V3,N19,V15,Jyo}]{kandare}
		\rdg[wit={N3,J7,V19,C7,V1,J10}]{kandhare}}/}\\+}
\tl{
\pada{śanaiḥ \app{\lem[wit={ceteri}]{śanaiḥ}
		\rdg[wit={J10}]{śanaḥ}}
	\app{\lem[wit={ceteri}]{prakurvīta}% +P11
		\rdg[wit={C6,J10}]{prakurvaṃti}}}
\pada{vāyusaṃcārakāraṇāt//} \NotIn{N23}\\!}  % vāyu{{ḥ}} C7; vāyuma° J10ac, vāyoḥ C6
\end{tlg}


\begin{tlg}[hp03_086]
\tl{
\pada{\app{\lem[wit={C6,Gr2,V19,C7,N19,V15}]{nāryā}
		\rdg[wit={J5,Jyo}]{nārī}
		\rdg[wit={N3}]{māryā}
		\rdg[wit={V3,V1,J10}]{bhāryā}}
	\app{\lem[wit={ceteri}]{bhage}
		\rdg[wit={N3}]{bhāge}}
	\app{\lem[wit={ceteri},alt={patad}]{pata\skp{d}} % patat V1
		\rdg[wit={J7}]{pated}
		\rdg[wit={N19}]{ca tad}}%
	\app{\lem[wit={N3,C6,Gr2,V19,C7,N19,V15,Jyo},alt={bindum}]{\skm{d}bindu\skp{m}} % biṃduṃm N19
		\rdg[wit={V3}]{bindhuḥm}
		\rdg[wit={V1,J10}]{bindur}}}%
\pada{m abhyāsenordhva%m
	\app{\lem[wit={ceteri},alt={āharet}]{\skm{m }āharet}
		\rdg[wit={C7}]{āruhet}}/}\\+}
\tl{
\pada{calitaṃ
	\app{\lem[wit={N3,J5}]{ca svakaṃ}% ca svayaṃ C2
		\rdg[wit={P11,C6,Gr2,N19}]{tu svakaṃ}% +G4
		\rdg[wit={V3}]{tu sukaṃ}
		\rdg[wit={V15,Jyo}]{ca nijaṃ}
		\rdg[wit={V1}]{patitaṃ}
		\rdg[wit={J10}]{calitaṃ}
		\rdg[wit={V19,C7},alt={\om}]{\skp{\om}}} bindu}%m
\pada{\app{\lem[wit={ceteri},alt={ūrdhvam ākṛṣya rakṣayet}]{\skm{m }ūrdhvam ākṛṣya rakṣayet}
		\rdg[wit={N3}]{ūrdhvam ākṛ\,+\,+\,+\,+}
		\rdg[wit={V15}]{ūrdhvam āhṛtya rakṣayet}
		\rdg[wit={N19}]{abhyāsenordhvam āharet}
		\rdg[wit={V19,C7},alt={\om}]{\skp{\om}}}//}
	\lineom{cd}{Gr3a}\\!}
\end{tlg}

\newpage
\begin{tlg}[hp03_087]
\tl{
\pada{evaṃ
	\app{\lem[wit={V1,J10}]{rakṣati yo}
		\rdg[wit={J5,C6,V3,Gr2,N19}]{tu rakṣayed} % ## +Gr1
		\rdg[wit={V19,C7,Jyo}]{saṃrakṣayed}
		\rdg[wit={V15}]{surakṣayed}}
		binduṃ} % biṃdu V3,N4
\pada{mṛtyuṃ jayati yogavit/}
	\anm{\ref{VuIII88}--\ref{VuIII116}a lost \getsiglum{N3}}\\+}
\tl{
\pada{maraṇaṃ \app{\lem[wit={ceteri}]{bindu}
		\rdg[wit={N19}]{vida}
		\rdg[wit={V19},alt={\om}]{\skp{\om}}}pātena}
\pada{\app{\lem[wit={ceteri}]{jīvitaṃ} % +J5,N24; jīvituṃ P11
		\rdg[wit={C6,J7,Jyo}]{jīvanaṃ}
		\rdg[wit={N23}]{jī}
		\rdg[wit={V19},alt={\om}]{\skp{\om}}}
	\app{\lem[wit={ceteri}]{bindudhāraṇāt}
		\rdg[wit={V15}]{bindurakṣaṇāt}
		\rdg[wit={N23,V19},alt={\om}]{\skp{\om}}}//}\label{VuIII88}
	\lineom{cd}{V19}\\!}
\end{tlg}

\begin{tlg}[hp03_088]
\tl{
\pada{\app{\lem[resp=emend,alt={sugandhir}]{sugandhi\skp{r}}% +L2
		\rdg[wit={J5,GrB,Gr2,V19,V15}]{sugandhi}% +J5, yugaṃndhi N23
		\rdg[wit={N19,Jyo}]{sugandho}
		\rdg[wit={C7,V1,J10},alt={\om}]{\skp{\om}}}r yogino
	\app{\lem[wit={C6,Gr2,N19}]{deho}
		\rdg[wit={V19,V15,Jyo}]{dehe}
		\rdg[wit={P11,V3}]{dehaṃ}% +J5
		\rdg[wit={C7,V1,J10},alt={\om}]{\skp{\om}}}}
\pada{jāyate bindu\app{\lem[wit={V3,N23,J7,V19,N19,Jyo}]{dhāraṇāt}% +P11
		\rdg[wit={C6,V15}]{rakṣaṇāt}
		\rdg[wit={C7,V1,J10},alt={\om}]{\skp{\om}}}/}\myfn{\getsiglum{V15} has this hemistich after the first half of the next verse.}
		\lineom{ab}{C7,V1,J10}\\+} % Haplography?
\tl{
%		\sgwit{GrB,J7,V19,N19,V15,Jyo}
\pada{yāva\app{\lem[wit={N23,C7,J10,Jyo},alt={binduḥ}]{\skm{d }binduḥ}
		\rdg[wit={GrB,J7,V19,N19,V15,V1}]{bindu}}
	\app{\lem[wit={J5,Gr2,V19,N19,V1,J10,Jyo}]{sthiro}% +J5, kṣīro G4?
		\rdg[wit={GrB,C7,V15}]{sthito}}
	\app{\lem[wit={ceteri}]{dehe}
		\rdg[wit={Gr2}]{deho}}}
\pada{tāva\app{\lem[wit={GrB,V19,C7,V1,J10},alt={mṛtyubhayaṃ kutaḥ}]{\skm{n }mṛtyubhayaṃ kutaḥ}% +G4,P11
		\rdg[wit={Gr2,N19,Jyo}]{kālabhayaṃ kutaḥ}% +J5
		\rdg[wit={V15}]{jīvanam ucyate}}//}\\!}
\end{tlg}

\begin{tlg}[hp03_089]
\tl{
\pada{\app{\lem[wit={ceteri}]{cittāyattaṃ}
		\rdg[wit={N23}]{cittamattaṃ}
		\rdg[wit={J5}]{manomayaṃ}
		\rdg[wit={C6,V3}]{manodhīnaṃ}}
		nṛṇāṃ % bhavet C6
	\app{\lem[wit={ceteri}]{śukraṃ}
		\rdg[wit={V3}]{śuklaṃ}}}
\pada{\app{\lem[wit={ceteri}]{śukrāyattaṃ}
		\rdg[wit={V3}]{śuklāyataṃ}
		\rdg[wit={C6}]{śukrādhīnaṃ}}
	\app{\lem[wit={V3,N19,V1,J10}]{hi}% +P11
		\rdg[wit={J5,C6,Gr2,V19}]{tu}% +J5,N24
		\rdg[wit={C7,V15,Jyo}]{ca}}
	\app{\lem[wit={ceteri}]{jīvitam} % +N23,P11
		\rdg[wit={C6,J7}]{jīvanaṃ}}/}\\+}
\tl{
\pada{tasmāc chukraṃ
	\app{\lem[wit={ceteri},alt={manaś}]{mana\skp{ś}} % + J10pc
		\rdg[wit={J10}]{rajaś}
		\rdg[wit={C7}]{rakṣa°}}% + J7pc
	\app{\lem[wit={ceteri},alt={caiva}]{\skm{ś }caiva}
		\rdg[wit={V1}]{caivaṃ}
		\rdg[wit={C7}]{°ṇīyaṃ}}}
\pada{\app{\lem[wit={ceteri}]{rakṣaṇīyaṃ}% °yaḥ C6
	\rdg[wit={C7}]{yogibhiś ca}} prayatnataḥ//}\\!}
\end{tlg}


\begin{tlg}[hp03_090]
\tl{
\pada{\app{\lem[wit={C6,V3,Gr2,N19,V15,Jyo}]{ṛtumatyā} % rutu° N23,V3
		\rdg[wit={V19,C7,V1,J10}]{bindumadhye}}
	\app{rajo\lem[wit={C6,Gr2,V19,C7,N19,V15,V1,Jyo}]{'py evaṃ} % + J10pc
		\rdg[wit={V3}]{thevaṃ}
		\rdg[wit={J10}]{py eva}}}
\pada{\app{\lem[wit={Gr2}]{striyā}
		\rdg[wit={V19,N19,V15,V1,J10}]{bījaṃ}% ##
		\rdg[wit={J5}]{vīryaṃ}
		\rdg[wit={C7}]{jīvaṃ}
		\rdg[wit={Jyo}]{nijaṃ}% + nija/nijaṃ Gr7
		\rdg[wit={V3}]{jayaṃ}
		\rdg[wit={C6}]{biṃduṃ}}\marmas
	\app{\lem[wit={ceteri}]{binduṃ}
		\rdg[wit={V3,J10}]{bindu}
		\rdg[wit={C6}]{rakṣe}}
	\app{\lem[wit={ceteri}]{ca}
		\rdg[wit={C6,N19,V1}]{tu}
		\rdg[wit={C7}]{pra°}}
	\app{\lem[wit={ceteri}]{rakṣayet}
		\rdg[wit={V3}]{rakṣayan}
		\rdg[wit={V19}]{taṃnnayet}
		\rdg[wit={C7}]{°pālayet}
		\rdg[wit={C6}]{yogavit}}/}\\+}
\tl{
\pada{\app{\lem[wit={ceteri}]{meḍhreṇā} % meṃḍhre° N19
		\rdg[wit={V19,C7,V15}]{meḍhreṇa}
		\rdg[wit={N23}]{meḍhrā}
		\rdg[wit={J10}]{meḍhrām ā}}% $$ J10 is partly based on Gr2?
	\app{\lem[wit={ceteri},alt={karṣayed}]{karṣaye\skp{d}}
		\rdg[wit={V3}]{karṣayad}
		\rdg[wit={J10}]{kuṃcayed}}d ūrdhvaṃ}
\pada{samyagabhyāsa%
	\app{\lem[wit={Gr2,V19,C7,V1}]{yogataḥ}
		\rdg[wit={G4,P11,V3,N19,V15}]{yogavān}% ##
		\rdg[wit={J10,Jyo}]{yogavit}
		\rdg[wit={J5,C6}]{pāṭavāt}}//}\\!}
\end{tlg}

\startgray
\begin{tlg}[hp03_091]
\tl{
\pada{ayaṃ yogaḥ puṇyavatāṃ}
\pada{\app{\lem[wit={ceteri}]{dhanyānāṃ}
		\rdg[wit={Jyo}]{dhīrāṇāṃ}}
	tattva\app{\lem[wit={C6,J7,V19,C7,V15,V1}]{śālinām} % tattva in V19 unclear
		\rdg[wit={V3,N19}]{śālinaṃ}
		\rdg[wit={N23}]{sattināṃ}
		\rdg[wit={J10,Jyo}]{darśinām}}/} \lineom{ab}{Gr1r}\\+}
\tl{
\pada{nirmatsarāṇāṃ
	\app{\lem[wit={P11,V3,N23,V19,N19,V15,V1}]{sidhyeta} % sidhyaita N23
		\rdg[wit={J7}]{siddheta}
		\rdg[wit={Jyo},post=\texteng{(but sidhyeta in mss?)}]{vai sidhyen}
		\rdg[wit={J10}]{siddhet}
		\rdg[wit={C6}]{siddhānāṃ}}}
\pada{na tu matsara\app{\lem[wit={C6,V3,Gr2,V19,V15,V1,Jyo}]{śālinām}
		\rdg[wit={N19}]{śālinaṃ}
		\rdg[wit={J10}]{śīlinām}}//} \lineom{cd}{Gr1r,C7}\myfn{%
		In \getsiglum{V15} Pāda b and d are transposed; \getsiglum{Jyo} has this verse at the end of the Sahajolī section.}\\!}
\end{tlg}
\endgray

\newpage
\begin{altava}[hp03_092]
\app{\lem[wit={J7,J10}]{atha sahajolī}% Jyo-mss
	\rdg[wit={Jyo}]{atha sahajoliḥ}}/
	\sgwit{J7,J10,Jyo}
\end{altava}


\begin{tlg}[hp03_092]
\tl{
\pada{\app{\lem[wit={C6,V19,C7,V1,J10}]{sahajolī}
		\rdg[wit={V3,Gr2,N19,V15,Jyo}]{sahajoliś}} % Jyo-ed
%		\rdg[wit={N4}]{sahajolīś}
	\app{\lem[wit={C6,V19,C7}]{cāmarolī}
%		\rdg[wit={N4}]{cāmarolīr}
		\rdg[wit={V3,N19}]{cāmaroli}
		\rdg[wit={V15,Jyo}]{cāmarolir}
		\rdg[wit={J10}]{vāmarolī}
		\rdg[wit={V1}]{cāmarolī ca}
		\rdg[wit={Gr2}]{cāmaroliś ca}}}
\pada{\app{\lem[wit={ceteri}]{vajrolyā} % vajrāḷyā V15
		\rdg[wit={V19,C7}]{vajrolyante}
		\rdg[wit={C6}]{vajrolī}}
	\app{\lem[wit={C6,V3,Gr2,N19,V15,V1}]{eva bhedataḥ}
		\rdg[wit={J10}]{ekabhedataḥ} % eka -> nāma J10pc
		\rdg[wit={Jyo}]{bheda ekataḥ}
		\rdg[wit={V19}]{prakīrtitā}
		\rdg[wit={C7}]{pracodyate}}/}\\+}
\tl{
\grau{\pada{\app{\lem[wit={J7,V19,N19,V15,V1,J10},alt={jaleṣu bhasma}]{jaleṣu\marmas bhasma}
		\rdg[wit={Jyo}]{jale subhasma}
		\rdg[wit={C7}]{jale bhasmani}}
	\app{\lem[wit={V19,C7,N19,V15,V1,J10,Jyo}]{nikṣipya}
		\rdg[wit={J7}]{niḥkṣipya}}}
\pada{\app{\lem[wit={J7,V19,C7,V15,V1,J10,Jyo}]{dagdha}
		\rdg[wit={N19}]{dagdhaṃ}}gomaya%
	\app{\lem[wit={J7,V19,N19,V15,V1,J10,Jyo}]{sambhavaṃ}
		\rdg[wit={C7}]{sambhave}}//}
		\lineom{cd}{Gr1r,C6,V3,N23}}\\!}
\end{tlg}


\begin{tlg}[hp03_093]
\tl{
\pada{\app{\lem[wit={ceteri},alt={vajrolīmaithunād}]{vajrolīmaithunā\skp{d}}
		\rdg[wit={V15}]{vajroḷimithunād}}d ūrdhvaṃ}
\pada{strī\app{\lem[wit={J7,N19,V1,J10,Jyo}]{puṃsoḥ}
		\rdg[wit={V3}]{puṃso}
		\rdg[wit={N23}]{puṃsā}
		\rdg[wit={V15}]{puṃsau}
		\rdg[wit={C6,V19,C7}]{puṃsoś}}
	\app{\lem[wit={Gr2,N19,V15,V1,J10,Jyo}]{svāṅga}
		\rdg[wit={V3}]{svāṃgu}
		\rdg[wit={C6,V19,C7}]{cāṃga}}lepanam/}\\+}  %lāpanaṃ C6
\tl{
\pada{\app{\lem[wit={ceteri}]{āsīnayoḥ} % °yot N19
		\rdg[wit={V15}]{anenaiva}}
	\app{\lem[wit={ceteri}]{sukhenaiva}
		\rdg[wit={J10}]{mukhenaiva}}}
\pada{mukta%
	\app{\lem[wit={C6,Gr2,C7,V15,V1,Jyo}]{vyāpārayoḥ}
		\rdg[wit={N19,J10}]{vyāpārayo}
		\rdg[wit={V3}]{vyāpāramo}
		\rdg[wit={V19}]{vyāpārala°}}
	\app{\lem[wit={C6,V3,V19,C7,N19,V15,V1,J10}]{kṣaṇam}
		\rdg[wit={Gr2,Jyo}]{kṣaṇāt}}//}\\!}
\end{tlg}

\begin{tlg}[hp03_094]
\tl{
\pada{\app{\lem[wit={V3,N23,N19,V15,Jyo},alt={sahajolir}]{sahajoli\skp{r}}
		\rdg[wit={C6,J7,V19,C7,V1,J10}]{sahajolī}% sahayolīr V19; °jolīr P11
		}r iyaṃ proktā}
\pada{\app{\lem[wit={P11,V3,Jyo}]{śraddheyā}% = DYŚ
		\rdg[wit={Gr1r,C6,V19,C7,V1}]{śraddhayā}% +M1,M3,G7
		\rdg[wit={J10}]{sādhyeyā}
		\rdg[wit={V15}]{siddhaye}
		\rdg[wit={Gr2,N19}]{sevyate}} % +C2
		yogibhiḥ sadā/}\\+}
\tl{
\grau{\pada{ayaṃ śubhakaro yogo} % From hier 2.5 verses om. in N23
\pada{\app{\lem[wit={V3,J7,V15,J10}]{bhoge}
		\rdg[wit={C6,N19,V1,Jyo}]{bhoga}
		\rdg[wit={C7}]{yoga}
		\rdg[wit={V19},alt={\gap}]{\skp{\gap}}}
	\app{\lem[wit={V3,J7,V15,J10}]{bhukte}
		\rdg[wit={Jyo}]{yukto}
		\rdg[wit={N19}]{mukte}
		\rdg[wit={V19,C7,V1}]{mukti}
		\rdg[wit={C6}]{yoge}}%
	\app{\lem[wit={C6,V3,J7,N19,V15,J10,Jyo}]{'pi muktidaḥ}
		\rdg[wit={C7,V1}]{vimuktidaḥ}
		\rdg[wit={V19}]{pradāyakaḥ}}//}\label{III94}
	\lineom{cd}{Gr1r,N23} \anm{cf. \ref{III101}cd}}\\!}  % M1 omits too, but M3 has it.
\end{tlg}

%\newpage
\begin{altava}[hp03_095]
\app{\lem[wit={J7,J10}]{atha amarolī}
	\rdg[wit={Jyo}]{athāmarolī}
	\rdg[wit={V15}]{āthamāroḷi}
	\rdg[wit={V19,C7}]{tatrāmarolī}}/ \sgwit{J7,Gr3a,V15,J10,Jyo}
\end{altava}


\begin{tlg}[hp03_095]
\tl{
\pada{\app{\lem[wit={V3,V19,C7,V15,V1,Jyo},alt={pittolbaṇatvāt}]{pittolbaṇatvā\skp{t}} % °lvana° V19
		\rdg[wit={C6}]{pītvā aṇut}
		\rdg[wit={N19}]{virttaṇatvāḍyat}
		\rdg[wit={J10}]{vihāya nityāṃ}
		\rdg[wit={J7}]{vihāya nīv\,..\,ḥ}}%
	\app{\lem[wit={C7,V1,Jyo},alt={prathamāmbu}]{\skm{t }prathamāmbu}
		\rdg[wit={C6,N19,V15,J10}]{prathamāṃ ca}
		\rdg[wit={J7}]{prathamaṃ ca}
		\rdg[wit={V3}]{prathamaṃ vi}
		\rdg[wit={V19},post={\unm}]{prathamāṃ}}%
	\app{\lem[wit={ceteri}]{dhārāṃ}
		\rdg[wit={V19},alt={\om}]{\skp{\om}}}}\\+}
\tl{
\pada{vihāya
	\app{\lem[wit={V19,V15,V1,J10,Jyo}]{niḥsāratayāntya}
		\rdg[wit={C6}]{niḥsāratapāṃśu}
		\rdg[wit={C7}]{niḥsārabhayāntya}
		\rdg[wit={J7}]{niḥsāralayāṃtya}
		\rdg[wit={V3}]{niḥsārayāṃtya}
		\rdg[wit={N19}]{niḥsmāratayāṃtya}}dhārām/}\\+}
\tl{
\pada{\app{\lem[wit={ceteri}]{niṣevyate}
		\rdg[wit={C6}]{niṣevite}
		\rdg[wit={V1}]{niḥsevyate}
		\rdg[wit={V3}]{nikhyevyate}}
	śītalamadhya\app{\lem[wit={V3,N19,V15,J10,Jyo}]{dhārā}
		\rdg[wit={C6,J7,C7,V1}]{dhārāṃ}
		\rdg[wit={V19}]{dhārāḥ}}}\\+}
\tl{
\pada{\app{\lem[wit={V3,V19,C7,N19}]{kāpālikaiḥ}
		\rdg[wit={J7,V15,V1,J10}]{kapālikaiḥ}
		\rdg[wit={C6}]{kapālakaiḥ}
		\rdg[wit={Jyo}]{kāpālike}}
	\app{\lem[wit={C6,V3,V1},alt={khaṇḍamatair}]{khaṇḍamatai\skp{r}}
		\rdg[wit={N19}]{khaṃḍamitair}
		\rdg[wit={V15,Jyo}]{khaṃḍamate}
		\rdg[wit={V19,C7}]{kaṃṭhamaṭhair}
		\rdg[wit={J7,J10}]{kuṃṭhamatair}}% kaṃṭha J10pc
	\app{\lem[wit={V19,C7,N19},alt={amaryāḥ}]{\skm{r }amaryāḥ}
		\rdg[wit={C6}]{amaryā}
		\rdg[wit={V3}]{aryā}
		\rdg[wit={J10}]{amedhyā}
		\rdg[wit={V1}]{amedhya}
		\rdg[wit={J7}]{amedhyāṃ}
		\rdg[wit={Jyo}]{'marolī}
		\rdg[wit={V15}]{'maroḷi}}//} \NotIn{N23}% M3 omits too
	\myfn{\getsiglum{J7} seems to have supplied this verse and the next one from a ms belonging to the {\textepsilon}-group.}\\!}
\end{tlg}

\newpage
\begin{tlg}[hp03_096]
\tl{
\pada{\app{\lem[wit={J7,V19,C7,J10,Jyo}]{amarīṃ}
		\rdg[wit={V3,N19,V15,V1}]{amarī}
		\rdg[wit={C6}]{amariṃ}}
	\app{\lem[resp=emend,alt={yat}]{ya\skp{t}}
		\rdg[wit={C6,V3,V19,C7,N19,V15,V1,Jyo}]{yaḥ} % + j10pc
		\rdg[wit={J7,J10}]{yo}}%
	\app{\lem[wit={ceteri},alt={piben}]{\skm{t }pibe\skp{n}} % paven V19
		\rdg[wit={C7}]{piban}}n nityaṃ}
\pada{\app{\lem[wit={G4,C6,V3,V19,N19,V15,Jyo},post=\texteng{(naśyaṃ \getsiglum{G4,N19,V15})}]{nasyaṃ kurvan}% kurvana V3, naśyaṃ kurvan G4,
		\rdg[wit={C7}]{na saṃkurvan}
		\rdg[wit={J5,V1}]{nasyaṃ kuryād}% nasya J5, naśyaṃ V1
		\rdg[wit={J7}]{tasya kuryā}
		\rdg[wit={J10}]{tasthaṃ kuryād}}\marmas % nāśāraṃdhrā J10pc
	dine dine/}\\+}
\tl{
\pada{\app{\lem[wit={V19,C7}]{vajrolīṃ cā}
		\rdg[wit={V3,N19,V15,V1}]{vajrolī cā}
		\rdg[wit={J7,J10,Jyo}]{vajrolīm a}
		\rdg[wit={C6}]{vajrolī ka}}%
	\app{\lem[wit={V3},alt={°bhyaset seyam}]{bhyaset seya\skp{m}}
		\rdg[wit={V19,C7,V15}]{bhyasec ceyam}
		\rdg[wit={N19}]{bhyasec ceya} % ceya ama°
		\rdg[wit={V1}]{bhyasen nityaṃ} % nityaṃ a°
		\rdg[wit={J7}]{bhyaset satve}
		\rdg[wit={J10}]{bhyasec chattve} % bhyased eva J10pc
		\rdg[wit={Jyo}]{bhyaset samyak}
		\rdg[wit={C6}]{thyate seyam}}}%
\pada{\app{\lem[wit={ceteri},alt={amarolīti}]{\skm{m }amarolīti}
		\rdg[wit={Jyo}]{sāmarolīti}
		\rdg[wit={V15}]{amaroḷīṃ tu}}
	\app{\lem[wit={ceteri}]{kathyate}
		\rdg[wit={V15}]{kalpayet}
		\rdg[wit={J10}]{kasyate}}//}\myfn{%
	In \getsiglum{Jyo} the verse \ref{VuIII98} is found after this}
	\NotIn{N23}\label{III96}\\!} % M3 omits too
\end{tlg}


\startgray
\begin{tlg}[hp03_097]
\tl{
\pada{\app{\lem[wit={N23,V19,C7,N19,V15,J10,Jyo}]{puṃso}
		\rdg[wit={C6,J7}]{puṃsor}
		\rdg[wit={V3,V1}]{puṃsāṃ}}
	\app{\lem[wit={ceteri}]{binduṃ}
		\rdg[wit={V3,N19,V15,J10}]{bindu}}
	\app{\lem[wit={Gr2}]{samākṛṣya}
		\rdg[wit={ceteri}]{samākuñcya}}}\label{III99}
\pada{samyagabhyāsa% N19 om. sa
	\app{\lem[wit={C6,V19,C7,V15,Jyo}]{pāṭavāt}
		\rdg[wit={V3,Gr2,N19,J10}]{pāṭavān}
		\rdg[wit={V1}]{pāravān}}/}\\+}  %3.97
\tl{
\pada{yadi nārī rajo rakṣe}%d % rakṣe N19
\pada{\app{\lem[wit={V3,J7,N19,V1,Jyo},alt={vajrolyā}]{\skm{d }vajrolyā}
		\rdg[wit={V19}]{vajrolyāṃ}
		\rdg[wit={C7}]{vajrolya}
		\rdg[wit={C6}]{vajrolī}
		\rdg[wit={V15}]{vajroḷi}
		\rdg[wit={J10}]{saṃyoge}
		\rdg[wit={N23},alt={\om},post=\texteng{(\ref{III97}d--\ref{III101}a om. prob. by eye-skip)}]{\skp{\om}}}
	\app{\lem[wit={V3}]{sā hi}
		\rdg[wit={C6,J7,N19}]{saha}
		\rdg[wit={V19,V15,V1,Jyo}]{sāpi}
		\rdg[wit={C7}]{syāpi}
		\rdg[wit={J10}]{cāpi}
		\rdg[wit={N23},alt={\om}]{\skp{\om}}}
		yoginī//}\label{III97} \NotIn{Gr1r} \\!}
\end{tlg}
	%\anm{\getsiglum{N23} om. 97d--100c by haplogr.?}

\begin{tlg}[hp03_098]
\tl{
\pada{tasyāḥ kiṃcid rajo nāśaṃ} % tasyā N4
\pada{na gacchati % gakṣati; gacchaṃti N19
		na saṃśayaḥ/}\\+}
\tl{
\pada{\app{\lem[wit={C6,V19,C7,V15,V1,J10,Jyo}]{tasyāḥ}
		\rdg[wit={N19}]{yasyāḥ}
		\rdg[wit={V3}]{asyāḥ}
		\rdg[wit={Gr2},alt={\om}]{\skp{\om}}}
	\app{\lem[wit={ceteri}]{śarīre}
		\rdg[wit={C7,V15}]{śarīra}
		\rdg[wit={Gr2},alt={\om}]{\skp{\om}}}
	\app{\lem[wit={C6,V3,C7,N19,V15,V1}]{nādas tu}
		\rdg[wit={J10}]{nādas tat}
		\rdg[wit={V19}]{nādātmā}
		\rdg[wit={Jyo}]{nādaś ca}
		\rdg[wit={Gr2},alt={\om}]{\skp{\om}}}}
\pada{\app{\lem[wit={V3,V19,C7,N19,V15,V1,Jyo}]{bindutām eva}
		\rdg[wit={J10}]{bindus tam eva}
		\rdg[wit={C6}]{vyaṃjatām eva}
		\rdg[wit={Gr2},alt={\om}]{\skp{\om}}} gacchati//}
	\NotIn{Gr1r}	\lineom{cd}{J7}\\!}
\end{tlg}

%\newpage
\begin{tlg}[hp03_099]
\tl{
\pada{sa bindus tad rajaś caiva} % svabiṃdu? V15
\pada{\app{\lem[wit={ceteri}]{ekī}
		\rdg[wit={C7}]{hy ekī}
		\rdg[wit={N23},alt={\om}]{\skp{\om}}}%
	\app{\lem[wit={C6,V3,J7,V19,C7,V15,Jyo}]{bhūya}
		\rdg[wit={N19,J10}]{bhūyaḥ}
		\rdg[wit={V1}]{bhūtaḥ}
		\rdg[wit={N23},alt={\om}]{\skp{\om}}}
	\app{\lem[wit={C7}]{svadehajam}
		\rdg[wit={V19}]{sadehajaṃ}
		\rdg[wit={C6,J10}]{svadehajaiḥ}
		\rdg[wit={V3,J7,N19,V15,V1}]{svadehajau}
		\rdg[wit={Jyo}]{svadehagau}
		\rdg[wit={N23},alt={\om}]{\skp{\om}}}\marma/}\\+}
\tl{
\pada{\app{\lem[wit={C6,V3,N19,V15,V1,J10}]{vajrolyā}
		\rdg[wit={J7,V19,C7,Jyo}]{vajrolya}
		\rdg[wit={N23},alt={\om}]{\skp{\om}}}bhyāsayogena} % +N4
\pada{sarva% sarvāṃ C6
	\app{\lem[wit={C7,J10}]{siddhiḥ}
		\rdg[wit={V3,V1}]{siddhi}
		\rdg[wit={C6,J7,V19,N19,V15,Jyo}]{siddhiṃ}
		\rdg[wit={N23},alt={\om}]{\skp{\om}}} % +J10pc,M1
	\app{\lem[wit={V3,V1,J10}]{prajāyate}% = DYŚ
		\rdg[wit={J7,V19,C7}]{prakurvate}
		\rdg[wit={N19,V15}]{prakurvataḥ}% +M1,P11 ##
		\rdg[wit={Jyo}]{prayacchataḥ}
		\rdg[wit={C6}]{prayacchati}
		\rdg[wit={N23},alt={\om}]{\skp{\om}}}//}
	\NotIn{Gr1r}\\!}
\end{tlg}
\endgray


\begin{tlg}[hp03_100]
\tl{
\pada{\app{\lem[wit={Gr1r},postwit=\texteng{(rakṣed ākuṃbhanonordhaṃ \getsiglum{J5}, rakṣaṇe kuṃcanenorddhva \getsiglum{N24}, \textit{damaged} \getsiglum{G4})}]{rakṣed ākuñcanenordhvaṃ}
	\rdg[wit={Jyo}]{rakṣed ākuñcanād ūrdhvaṃ}
	\rdg[wit={J7}]{mehenākuñcanād ūrdhvaṃ}
	\rdg[wit={J10}]{meḍhrām ākuṃcanād ūrdhvaṃ }}}
\pada{\app{\lem[wit={Jyo}]{yā rajaḥ sā hi yoginī}
	\rdg[wit={J5}]{yā rajaḥ saha yoginī}
	\rdg[wit={J7,J10}]{rajasāpi hi yoginaḥ}}/} \sgwit{Gr1r,J7,J10,Jyo}\myfn{\getsiglum{J7} has this hemistich between \ref{III96} and \ref{III99}.}\\+}
\tl{
% J5: rakṣed ākuṃbhanonordhaṃ yā rajaḥ saha yoginī
% N24: rakṣaṇe kuṃcanenorddhva yo radhaḥ saha yogavit
% G4: (damaged)
\pada{\app{\lem[wit={ceteri}]{atītānāgataṃ}
		\rdg[wit={C6}]{atītānāgate}
		\rdg[wit={V15}]{atītānāgatiṃ}
		\rdg[wit={N19}]{atītānāṃ gatiṃ}
		\rdg[wit={N23},alt={\om}]{\skp{\om}}} vetti}
\pada{\app{\lem[wit={ceteri}]{khecarī ca}% +J10pc
		\rdg[wit={V19},post=\texteng{(one syllable missing)}]{khecarī}
		\rdg[wit={C7}]{khecarīṃ la°}
		\rdg[wit={J10}]{khecaraś ca}
		\rdg[wit={N23},alt={\om}]{\skp{\om}}}
	\app{\lem[wit={ceteri}]{bhaved dhruvam}
		\rdg[wit={C7}]{°bhate dhruvam}
		\rdg[wit={J7}]{prajāyate}
		\rdg[wit={N23},alt={\om}]{\skp{\om}}}//}\\!}
\end{tlg}


\newpage
\begin{tlg}[hp03_101]
\tl{
\pada{dehasiddhiṃ % dehe C6
	\app{\lem[wit={ceteri}]{ca}
		\rdg[wit={V1}]{tu}
		\rdg[wit={N23},alt={\om}]{\skp{\om}}}
	\app{\lem[wit={ceteri}]{labhate}
		\rdg[wit={C6}]{labhyeta}
		\rdg[wit={N23},alt={\om}]{\skp{\om}}}}
\pada{\app{\lem[wit={J7,V19,C7,Jyo}]{vajrolyabhyāsa}
		\rdg[wit={C6,V3,N23,N19,V15,V1,J10}]{vajrolyābhyāsa}% +N4; ā N23pc
		}yogataḥ/}\\+} % +N24, yogavit G4, om. J5
\tl{
\pada{ayaṃ \app{\lem[wit={G4,N24}]{śubhakaro}
	\rdg[wit={Jyo}]{puṇyakaro}} yogo}
\pada{\app{\lem[wit={Jyo}]{bhoge bhukte'pi muktidaḥ}
	\rdg[wit={G4}]{bhāvamukthivimukthidaḥ}
	\rdg[wit={N24}]{bhyāsyayuktasya muktida}}//}
	\anm{cd in \getsiglum{G4,N24,Jyo}; cf. \ref{III94}cd}\\!}
\end{tlg}


\startaltrecension\normalsize
\begin{alttlg}[hp03_101_1]
\tl{
\pada{\app{\lem[wit={V19,C7,N19,V15,V1,J10}]{tasmād ayaṃ}
		\rdg[wit={C6,V3}]{yasmād ayaṃ}}
	\app{\lem[wit={C6,V3,V19,C7,N19,V15}]{sādhakāya}
		\rdg[wit={V1}]{sādhako'yaṃ}
		\rdg[wit={J10}]{sādhakānāṃ}}}
\pada{\app{\lem[wit={V3,J10}]{bhoge}
		\rdg[wit={C6,C7,N19,V15,V1}]{bhoga}
		\rdg[wit={V19}]{yoga}}
	\app{\lem[resp=emend]{bhukte}
		\rdg[wit={V3},postwit=\texteng{\getsiglum{V19}\postcorr?}]{bhukti}
		\rdg[wit={N19}]{mukte}
		\rdg[wit={V19,C7,V1,J10}]{mukti}
		\rdg[wit={V15}]{yukto}
		\rdg[wit={C6}]{yoge}}%
	\app{\lem[wit={C6,V3,N19,V15}]{'pi muktidaḥ} % <<pi>> V3
		\rdg[wit={V19,C7,V1,J10}]{vimuktidaḥ}}\marma//}
	\lineom{ab}{Gr1r,Gr2,Jyo}\label{III101}\\+}
\tl{
\pada{tasmāt puṇyavatā%m 
	\app{\lem[wit={C6,Gr2,J10},alt={eva}]{\skm{m }eva}
		\rdg[wit={V3,V19,C7,N19,V15,V1}]{evam}}}
\pada{\app{\lem[wit={C6,V3,Gr2,N19,V15,V1}]{ayaṃ yogaḥ}
		\rdg[wit={V19,C7}]{eṣa yogaḥ}
		\rdg[wit={J10}]{yogo'yaṃ}}
	\app{\lem[wit={C6,V3,Gr2,V19,C7,N19,V15,V1}]{prasidhyati}
		\rdg[wit={J10}]{saṃprasidhyati}}}//%
	\myfn{\label{N23ch3end}\getsiglum{N23} has a sub-colophon marking the end of Chap. 3 after this verse (the 100th!). Chap. 4 contains only 29 verses, which are the remaining verses of the usual Chap. 3. Chap. 5 corresponds to the usual Chap. 4.}
%\myfn{\getsiglum{V19} adds here: \devnote{iti haṭhayogapradīpikāyāṃ paṃcama upadeśaḥ// 5 // samāptoyaṃ graṃthaḥ// saṃvat 1707 jyeṣṭha kṛṣṇa 4 bhṛgau liṣitam idaṃ// // śubhaṃ// //}; \getsiglum{C7} \devnote{iti śrīmadātmārāmaviracitāyāṃ pañcamoyam upadeśaḥ// 5 // śubham astu sarvajagatām//}; \getsiglum{P23} \devnote{iyaṃ vajrolī trayodaśe patre śakticālanāt pūrvaṃ jñātavyā// //iti śrīātmārāmamunīṃdraviracitāyāṃ haṭhadīpikāyaṃ(!) paṃcamopadeśaḥ// 5 //}}
	\lineom{cd}{Gr1r,Jyo}\\!}
\end{alttlg}
\endaltrecension


\begin{ava}[hp03_102]
\app{\lem[wit={C6,V3,N19,V1,J10,Jyo}]{atha}
\rdg[wit={Gr2,Gr3a,V15},alt={\om}]{\skp{\om}}}
\app{\lem[wit={C6,V3,J7,Gr3a,N19,V1,Jyo}]{śakticālanam}
	\rdg[wit={N23}]{śaktiyānaṃ}
	\rdg[wit={J10}]{śakti}
	\rdg[wit={V15},alt={\om}]{\skp{\om}}}/
\end{ava}

\startgray
\begin{tlg}[hp03_102]
\tl{
\pada{\app{\lem[wit={C6,V3,Gr3a,V1,J10,Jyo}]{kuṭilāṅgī}
	\rdg[wit={J7,N19,V15}]{kuṃḍalāṅgī}
	\rdg[wit={N23}]{kundalīgī}} kuṇḍalinī} % kuḍalinī V15
\pada{bhujaṅgī
\app{\lem[wit={C6,V3,J7,N19,V15,V1,J10,Jyo}]{śaktir īśvarī}
	\rdg[wit={V19,K3}]{śaktir aiśvarī}
	\rdg[wit={N23}]{śaktir asvarī}
	\rdg[wit={C7}]{śaktivardhinī}}/}\\+}
\tl{
\pada{\app{\lem[wit={C6,V3,Gr2,N19,V15,V1,J10,Jyo},alt={kuṇḍaly}]{kuṇḍaly\skm{a}} % kuṃḍalasaṃdhanī N23
	\rdg[wit={Gr3a}]{kuṭily}}%
\app{\lem[wit={ceteri},alt={arundhatī}]{\skp{a}rundhatī}
	\rdg[wit={V1}]{ā[ku]ṃḍalī}
	\rdg[wit={J10}]{āceti ruṃ°}}
\app{\lem[wit={V1}]{ceti} % c[e]ti V1
	\rdg[wit={V3}]{veti}
	\rdg[wit={N19}]{vati}
	\rdg[wit={V15}]{caiva}
	\rdg[wit={C6,Jyo}]{caite}
	\rdg[wit={Gr2,V19,C7}]{devī}
	\rdg[wit={K3}]{dīvī}
	\rdg[wit={J10}]{dhaṃti}}}
\pada{\app{\lem[wit={ceteri}]{śabdāḥ paryāyavācakāḥ} % śabdā V3; °vācakā V3
	\rdg[wit={V19,C7}]{śabdaḥ paryāyavācakaḥ}}//} \NotIn{Gr1r}\\!}
\end{tlg}

\begin{tlg}[hp03_103]
\tl{
\pada{\app{\lem[wit={ceteri},alt={udghāṭayet}]{udghāṭaye\skp{t}}
	\rdg[wit={N19}]{udghāṭayati}}%
\app{\lem[wit={ceteri},alt={kapāṭaṃ}]{\skm{t }kapāṭaṃ}
	\rdg[wit={C7}]{kapālaṃ}}
\app{\lem[wit={ceteri}]{tu}
	\rdg[wit={N19},alt={\om}]{\skp{\om}}}}
\pada{yathā
\app{\lem[wit={ceteri}]{kuñcikayā}% kuṃcīkayā V3
	\rdg[wit={C6}]{kaṃcukayā}} haṭhāt/}\\+}
\tl{
\pada{kuṇḍalinyā tathā yogī}
\pada{mokṣadvāraṃ
\app{\lem[wit={ceteri}]{vibhedayet}
	\rdg[wit={N23}]{prabhedayet}
	\rdg[wit={J7}]{nirodhayet}}//}\myfn{This verse and the next one are transposed in \getsiglum{N19}.} \NotIn{Gr1r}\\!}
\end{tlg}

\begin{tlg}[hp03_104]
\tl{
\pada{yena \app{\lem[wit={C6,V3,N19,V15,V1,J10,Jyo}]{mārgeṇa}
	\rdg[wit={Gr2,Gr3a}]{dvāreṇa}} gantavyaṃ}
\pada{brahmasthānaṃ nirāmayam/}\\+} % °khyānaṃ N23, sthāna N19
\tl{
\pada{mukhe\app{\lem[wit={ceteri},alt={ācchādya}]{\skm{n}ācchādya} % sukhenā° K3
	\rdg[wit={V19}]{ākṣādya/ājñādya}
	\rdg[wit={N19}]{āvādya}}
\app{\lem[wit={N23,C7,J10}]{taddvāraṃ}
	\rdg[wit={V3,J7,N19,V15,V1,Jyo}]{tadvāraṃ}
	\rdg[wit={C6}]{taṃ dvāraṃ}
	\rdg[wit={V19}]{nadvāraṃ}
	\rdg[wit={K3}]{tedvāraṃ}}}
\pada{prasuptā parameśvarī//} \NotIn{Gr1r}\\!}
\end{tlg}

\newpage
\begin{tlg}[hp03_105]
\tl{
\pada{\app{\lem[wit={Gr2,N19,V15}]{kandordhvaṃ}
	\rdg[wit={V19,C7,V1,J10,Jyo},post=\texteng{(kandho° \getsiglum{V19}\antecorr)}]{kandordhve}
	\rdg[wit={V3}]{kandorddha}
	\rdg[wit={C6}]{kaṃṭhorddhaṃ}
	\rdg[wit={K3}]{kuṇḍovvo}} kuṇḍalī
\app{\lem[wit={ceteri},alt={śaktiḥ/śaktir}]{śaktiḥ}
	\rdg[wit={V3}]{śakti}}}
\pada{\app{\lem[wit={C6,V3,V15,V1,J10,Jyo}]{suptā}
	\rdg[wit={Gr2,K3,N19}]{buddhā}
	\rdg[wit={V19,C7}]{baddhā}} mokṣāya yoginām/}\\+} % yoginaṃ N23
\tl{
\pada{bandhanāya ca
\app{\lem[wit={ceteri}]{mūḍhānāṃ}
	\rdg[wit={J7}]{mūrkhāṇāṃ}}}
\pada{yas tāṃ vetti sa yogavit//} \NotIn{Gr1r}\\!} % vartti N23; taṃ J10ac
\end{tlg}

%\newpage
\begin{tlg}[hp03_106]
\tl{
\pada{\app{\lem[wit={ceteri}]{ambhodhi}
		\rdg[wit={Gr1r,Jyo},alt={\om}]{\skp{\om}}}%
	\app{\lem[wit={V3,Gr2,V15,V1,J10},alt={śailadvīpānām}]{śailadvīpānā\skp{m}} % aṃbhodha N23
		\rdg[wit={C6}]{śailordvagānām}
		\rdg[wit={N19}]{plauladvīpānām}
		\rdg[wit={Gr3a}]{dvīpaśailānām}
		\rdg[wit={Gr1r,Jyo},alt={\om}]{\skp{\om}}}}% dvepa K3
\pada{\app{\lem[wit={ceteri},alt={ādhāraḥ}]{\skm{m }ādhāraḥ}
		\rdg[wit={J7}]{ādharaḥ}
		\rdg[wit={N19}]{ādhāraṃ}
		\rdg[wit={Gr1r,Jyo},alt={\om}]{\skp{\om}}} śeṣakuṇḍalī/}
		\lineom{ab}{Gr1r,Jyo}\\+}
\tl{
\pada{aśeṣayoga\app{\lem[wit={ceteri},alt={tantrāṇām}]{tantrāṇā\skp{m}} % +J10pc
	\rdg[wit={J10}]{jagatām}
	\rdg[wit={Gr1r,V1,Jyo},alt={\om}]{\skp{\om}}}}%
\pada{m ādhāraḥ
\app{\lem[wit={ceteri}]{kuṇḍalī tathā}
	\rdg[wit={V19}]{kuṇḍalī yathā}
	\rdg[wit={V15}]{śeṣakuṇḍalī}
	\rdg[wit={Gr1r,V1,Jyo},alt={\om}]{\skp{\om}}}//}
	\lineom{cd}{Gr1r,V1,Jyo} \anm{cf. 3.1}\\!}
\end{tlg}

\begin{tlg}[hp03_107]
\tl{
\pada{kuṇḍalī
\app{\lem[wit={V3,Gr2,Gr3a,N19,Jyo}]{kuṭilākārā}
	\rdg[wit={V15}]{kuṃḍilākārā}
	\rdg[wit={V1}]{kuṃḍalākārā}
	\rdg[wit={J10}]{kuṭilākarī}}}
\pada{sarpavat parikīrtitā/}\\+}
\tl{
\pada{sā śaktiś cālitā yena} % ye<<na>> J7
\pada{sa mukto nātra saṃśayaḥ//} \NotIn{Gr1r} \anm{= 4.77*1}\\!}
\end{tlg}
\endgray

%\newpage
\begin{tlg}[hp03_108]
\tl{
\pada{gaṅgāyamunayor madhye} % jamunāyor V3,J10
\pada{\app{\lem[wit={ceteri}]{bālaraṇḍā}% vā in mss; bā V15,V1,K3
	\rdg[wit={Jyo}]{bālaraṇḍāṃ}}
\app{\lem[wit={ceteri}]{tapasvinī}
	\rdg[wit={N19}]{tapaśvinī}
	\rdg[wit={V19}]{tapaścānī}
	\rdg[wit={Jyo}]{tapasvinīm}
	\rdg[wit={P11,C6}]{sarasvatī}}/}\\+}
\tl{
\pada{balātkāreṇa gṛhṇīyāt} % gṛhnī° Gr2,V19,N19
\pada{tad viṣṇoḥ paramaṃ padam//}\\!} % padamaṃ N23
\end{tlg}

\startaltrecension
\begin{alttlg}[hp03_108_1]
\tl{
\pada{iḍā bhagavatī gaṅgā}
\pada{piṅgalā yamunā nadī/}\\+} % jamunā V3
\tl{
\pada{iḍāpiṅgalayor madhye} % ilā V1; piṃgalāyor V3,J10
\pada{bālaraṇḍā
	\app{\lem[wit={C6,V3,J7,V1,J10}]{sarasvatī}
	\rdg[wit={Jyo}]{ca kuṇḍalī}}//}
\sgwit{C6,V3,J7,V1,J10,Jyo}\\!}
\end{alttlg}
%\NotIn{P11,N23,Gr3a,N19,V15}
\endaltrecension


\begin{tlg}[hp03_109]
\tl{
\pada{\app{\lem[wit={ceteri}]{pucchaṃ}
	\rdg[wit={K3,J10,Jyo}]{pucche}}
\app{\lem[wit={C6,V3,Gr2,Gr3a,N19,J10,Jyo}]{pragṛhya}
	\rdg[wit={V15}]{nigṛhya}
	\rdg[wit={V1}]{gṛhya}}
\app{\lem[wit={C6,J7,Gr3a}]{bhujagīṃ}
	\rdg[wit={V3,N23}]{bhujaṃgī}
	\rdg[wit={J10}]{bhujaṃgīṃ}
	\rdg[wit={V1}]{bhujaṃgīva}
	\rdg[wit={N19},alt={\illeg}]{\skp{\illeg}}}}
\pada{suptā%m % \illeg N19
\app{\lem[wit={C6,V3,J7,V19,K3,V15,J10},alt={udbodhayed}]{\skm{m }udbodhaye\skp{d}}
	\rdg[wit={V1,Jyo}]{udbodhayec}
	\rdg[wit={C7}]{uddyotayed}
	\rdg[wit={N23}]{udrodhyamed}
	\rdg[wit={N19},alt={\illeg}]{\skp{\illeg}}}%
\app{\lem[wit={P11,Gr2},alt={abhīḥ}]{\skm{d }abhīḥ}% abhī P11
	\rdg[wit={V15}]{abhiḥ}
	\rdg[wit={Gr3a}]{api}
	\rdg[wit={V3,V1,J10,Jyo}]{ca tām}
	\rdg[wit={C6}]{balāt}
	\rdg[wit={N19},alt={\illeg}]{\skp{\illeg}}}/}\\+}
\tl{
\pada{nidrāṃ vihāya sā
\app{\lem[wit={ceteri}]{ṛjvī}
	\rdg[wit={J7}]{ṛjvīṃ} % ṃ not m
	\rdg[wit={V3}]{rujvī}
	\rdg[wit={N19}]{rajvī}
	\rdg[wit={Jyo}]{śaktir}}}
\pada{\app{\lem[wit={K3,C7,V15,V1,J10,Jyo},alt={ūrdhvam/ūrddham}]{ūrdhva\skp{m}}
	\rdg[wit={N23}]{urddham}
	\rdg[wit={C6,V3,V19}]{mūrddham}
	\rdg[wit={N19}]{kurddham}}%
\app{\lem[wit={ceteri},alt={uttiṣṭhate}]{\skm{m }uttiṣṭhate}
	\rdg[wit={N19}]{ākṛṣyate}}
\app{\lem[wit={ceteri}]{haṭhāt}% +P11
	\rdg[wit={C6}]{kṣaṇāt} % +V23
	}//}\\!}
\end{tlg}

\newpage
\startgray
\begin{tlg}[hp03_110]
\tl{
\pada{\app{\lem[wit={C6,Gr2,Gr3a,N19}]{paristhitā caiva}
	\rdg[wit={V15}]{paristhitasyaiva}
	\rdg[wit={V1}]{paristhitā [sai]va}
	\rdg[wit={V3}]{pṛṣṭisthitasyaiva}
	\rdg[wit={J10}]{avasthitasya}
	\rdg[wit={Jyo}]{avasthitā caiva}}
\app{\lem[wit={ceteri},post=\texteng{(kaṇā° \getsiglum{N23})}]{phaṇāvatī sā}
	\rdg[wit={C7}]{phaṇāvatīva sā}
	\rdg[wit={J10}]{phaṇāryayāṃtīyaṃ}}}\\+}
\tl{
\pada{\app{\lem[wit={ceteri}]{prātaś ca sāyaṃ}
	\rdg[wit={V15}]{prātas tu sāyaṃ}
	\rdg[wit={K3}]{sāyaṃ ca prātaḥ}}
	praharārdha\app{\lem[wit={ceteri}]{mātraṃ}
	\rdg[wit={V3}]{rātraṃ}}/}\\+}
\tl{
\pada{\app{\lem[wit={ceteri}]{prapūrya}
	\rdg[wit={N23}]{prapūrvva}
	\rdg[wit={V1}]{prasūrya}
	\rdg[wit={C6,J10}]{prasārya}}
\app{\lem[wit={P11,V3,Gr2,N19,V15,V1,Jyo},alt={sūryāt}]{sūryā\skp{t}}
	\rdg[wit={V19}]{sauryā}
	\rdg[wit={C7}]{saudhā}
	\rdg[wit={C6}]{sācāryya}
	\rdg[wit={K3}]{tesau}
	\rdg[wit={J10}]{ryāṣṇut}}%
\app{\lem[wit={ceteri},alt={paridhāna}]{\skm{t }paridhāna} % pa[ri] .. na V1
	\rdg[wit={V3}]{paridhāya}
	\rdg[wit={P11}]{mavidhāna}
	\rdg[wit={C6}]{vidhāna}}%
\app{\lem[wit={C6,J10,Jyo}]{yuktyā}% +P11?
	\rdg[wit={V3,Gr2,N19,V15,V1}]{yuktā}
	\rdg[wit={Gr3a}]{muktā}}}\\+}
\tl{
\pada{pragṛhya
\app{\lem[wit={C6,Jyo}]{nityaṃ paricālanīyā}
	\rdg[wit={V3}]{niryāt paricālanīyā}
	\rdg[wit={N19}]{niryāt paricālanīyāt}
	\rdg[wit={V15},post={\unm}]{niryātya paricālanīyā}
	\rdg[wit={N23}]{niyāt* pavicālinī sā}
	\rdg[wit={J7,Gr3a}]{niryāty avicālinī sā}
	\rdg[wit={V1}]{teyā paricālanīy[ai]}
	\rdg[wit={J10}]{paricālanīyā}}//}\marma
	\NotIn{Gr1r}\\!}
\end{tlg}

%\newpage
\begin{tlg}[hp03_111]
\tl{
\pada{\app{\lem[wit={C6,Gr2,Gr3a,J10}]{vitastipramitaṃ dīrghaṃ} % vitasthi N23
	\rdg[wit={V3,N19}]{vitastipramitaṃ dairghyaṃ}
	\rdg[wit={V15,V1}]{vitastipramita-dairghyaṃ}
	\rdg[wit={Jyo}]{ūrdhvaṃ vitastimātraṃ tu}}}
\pada{\app{\lem[wit={J7,N19}]{vistāre}
	\rdg[wit={C6,V3,N23,Gr3a,V15,V1,J10,Jyo}]{vistāraṃ}} caturaṅgulam/}\\+}
\tl{
\pada{\app{\lem[wit={ceteri}]{mṛdulaṃ}
	\rdg[wit={V19}]{mṛlaṃ}}
\app{\lem[wit={ceteri}]{dhavalaṃ}
	\rdg[wit={C7}]{pavanaṃ}} proktaṃ}
\pada{\app{\lem[wit={V15,V1,J10}]{veṣṭanāmbara}
	\rdg[wit={J7}]{veṣṭanāṃvala}
	\rdg[wit={N23,N19}]{vaṣṭanāṃcara}
	\rdg[wit={C6}]{vaṣṭanāṃba}
	\rdg[wit={N19}]{vaṣṭanāṃ}
	\rdg[wit={V3}]{veṣṭatāṃvara}
	\rdg[wit={Jyo}]{veṣṭitāmbara}
	\rdg[wit={Gr3a}]{veṣṭanādhāra}}lakṣaṇam//} \anm{=\,3.33*5} \NotIn{Gr1r}\\!}
\end{tlg}
\endgray

%\newpage
\begin{tlg}[hp03_112]
\tl{%
\myfn{\getsiglum{Jyo} has 3.64 before this verse.}%
\pada{\app{\lem[wit={ceteri}]{vajrāsana}
	\rdg[wit={C6,Jyo}]{vajrāsane}}sthito yogī}
\pada{cālayitvā
\app{\lem[wit={C6,V3,Gr2,Gr3a,N19}]{tu}
	\rdg[wit={V15,J10,Jyo}]{ca}% +J5
	\rdg[wit={V1},alt={\om}]{\skp{\om}}} kuṇḍalīm/}\\+}% °lī N23,N19,C6,V3
\tl{
\pada{\app{\lem[alt={\ante kuryād \add},nosep]{}
	\rdg[wit={N23,K3,C7},postwit=\texteng{(as header \getsiglum{C7})}]{sūryabhedāt}}
\app{\lem[wit={C6,V3,V1,J10,Jyo},alt={kuryād}]{kuryā\skp{d}}
	\rdg[wit={Gr2,V19,C7,N19,V15}]{sūryād}
	\rdg[wit={K3}]{tathā}}%
\app{\lem[wit={ceteri},alt={anantaraṃ}]{\skm{d }anantaraṃ}
	\rdg[wit={N23}]{vanara}
	\rdg[wit={K3}]{sūryāt}}
\app{\lem[wit={V15,J10}]{bhastrīṃ}
	\rdg[wit={N23,Gr3a}]{bhastrī}% +J5
	\rdg[wit={J7}]{bhasrī}
	\rdg[wit={V3,N19}]{bhastri}
	\rdg[wit={C6,Jyo}]{bhastrāṃ}
	\rdg[wit={V1},alt={\illeg}]{\skp{\illeg}}
	\rdg[wit={K3},alt={\om}]{\skp{\om}}}}
\pada{\app{\lem[wit={ceteri}]{kuṇḍalīm āśu bodhayet} % āsu V3
	\rdg[wit={K3},alt={\om}]{\skp{\om}}}//}\\!}
\end{tlg}

%\newpage
\begin{tlg}[hp03_113]
\tl{
\pada{\app{\lem[wit={ceteri},alt={bhānor}]{bhāno\skp{r}}
	\rdg[wit={K3},alt={\om}]{\skp{\om}}}%
\app{\lem[wit={ceteri},alt={ākuñcanaṃ kuryāt}]{\skm{r }ākuñcanaṃ kuryā\skp{t}}
	\rdg[wit={V19}]{ākuñcanaṃ pu(?)ryāt}
	\rdg[wit={V1}]{ākuṃcanaivaṃ}% āku(ṃ)nacaivaṃ V1
	\rdg[wit={J10}]{ākuñcanenaiva}
	\rdg[wit={K3},alt={\om}]{\skp{\om}}}t}
\pada{kuṇḍalīṃ % kuṃḍalī Gr2,V15,N3,V3, kuṇḍalīś K3;
\app{\lem[wit={ceteri},alt={cālayet}]{cālaye\skp{t}} % cālayat N23
	\rdg[wit={N3}]{bodhayet}}%
\app{\lem[wit={ceteri},alt={tataḥ}]{\skm{t }tataḥ}
	\rdg[wit={J10}]{tadā}}/}\\+}
\tl{
\pada{\app{\lem[wit={ceteri}]{mṛtyu}
	\rdg[wit={J10}]{mṛtyor}}%
\app{\lem[wit={ceteri}]{vaktra}
	\rdg[wit={V3}]{vaktraṃ}}gatasyāpi}
\pada{tasya mṛtyubhayaṃ kutaḥ//}\label{VuIII116}\\!}
\end{tlg}

\newpage
\startgray
\begin{tlg}[hp03_114]
\tl{
\pada{nāsā\app{\lem[wit={V3,Gr2,N19,V15,J10,Jyo}]{dakṣiṇamārgavāhi} % nāśā N19; +P11
	\rdg[wit={C6}]{dakṣiṇavāhimārga}
	\rdg[wit={K3,C7}]{dakṣiṇavartmavāhi}
	\rdg[wit={V19}]{paścimavartmavāhi}
	\rdg[wit={V1}]{da\,..\,ṇa[vā]\,..\,mārgeṇa}}%
\app{\lem[wit={C6,J7,Gr3a}]{pavano}
	\rdg[wit={V15}]{pavanot}
	\rdg[wit={P11,V3,N19,V1,J10,Jyo}]{pavanāt}
	\rdg[wit={N23}]{pavana}}
\app{\lem[wit={N23}]{prāṇe}
	\rdg[wit={P11,V3,N19,V15,V1,J10,Jyo}]{prāṇo}
	\rdg[wit={C6,J7,K3,C7}]{ghrāṇe}
	\rdg[wit={V19}]{ghrāṇo}}%
\app{\lem[resp=emend]{'tidīrghīkṛte}
	\rdg[wit={V3,K3,V15,V1,Jyo}]{'tidīrghīkṛtaś}
	\rdg[wit={J7}]{'tidīrghīkṛteś}
	\rdg[wit={N19,J10}]{tidīrghākṛtiś}
	\rdg[wit={N23}]{tidīrghākṛtaś}
	\rdg[wit={V19},post=\texteng{(°kṛtaś \emph{pc}?)}]{tirghīkṛtiś}
	\rdg[wit={C6}]{na dīrghīkṛtaḥ}% pi P11
	\rdg[wit={C7}]{ca dīrghīkṛtaś}}}\\+}
\tl{
\pada{\app{\lem[wit={C6,V15,V1,J10}]{candrāmbhaḥ} % caṃdrāṃbhaḥ; 2nd ṃ unclear V1
	\rdg[wit={Gr2,Jyo}]{candrābhaḥ}
	\rdg[wit={Gr3a}]{candrāṃtaḥ}
	\rdg[wit={V3}]{caṃdrāṃgāt}
	\rdg[wit={N19}]{caṃdrād[vā]}}%
\app{\lem[wit={C6,Gr2,N19,J10,Jyo}]{paripūritāmṛtatanuḥ}
	\rdg[wit={V15}]{paripūrṇatāmṛtatanuḥ}
	\rdg[wit={V3}]{paripūritāmṛtyutanuḥ}
	\rdg[wit={V1}]{paripūritā\,..\,..\,..\,..}
	\rdg[wit={Gr3a}]{paripūrya pūritatanuḥ}}
\app{\lem[wit={ceteri},alt={prāg}]{prā\skp{g}}% prāghgh° J7,V15; prāk V1
	\rdg[wit={C6,V19}]{prā}}g ghaṇṭikāyā%s % °kāyā N23; ghaṇṭi illeg. V1
\app{\lem[wit={C6,V3,N23,V19,K3,N19,J10},alt={tathā}]{\skm{s }tathā}
	\rdg[wit={C7,Jyo}]{tataḥ}
	\rdg[wit={J7}]{tadā}
	\rdg[wit={V15}]{sadā}
	\rdg[wit={V1},alt={\illeg}]{\skp{\illeg}}}/}\\+}
\tl{
\pada{\app{\lem[resp=emend,postwit=\texteng{(=\,Amaraughaśāsana)},alt={siñcan}]{siñca\skp{n}}
	\rdg[wit={N19,V15}]{chindan}% ##
	\rdg[wit={V3,J10}]{chinnat}
	\rdg[wit={Jyo}]{chittvā}
	\rdg[wit={C6}]{chaṃdaḥ}
	\rdg[wit={J7,Gr3a}]{bhindan}% siñcan! Amaraughaśāsana
	\rdg[wit={N23}]{bhidan}
	\rdg[wit={V1},alt={\illeg}]{\skp{\illeg}}}n
	kālaviśāla% kāla illeg. V1
\app{\lem[wit={ceteri}]{vahni}
	\rdg[wit={V15}]{pāśa}
	\rdg[wit={N23},alt={\om}]{\skp{\om}}}%
\app{\lem[wit={V3,J7,K3,V1},alt={vaśagān}]{vaśagā\skp{n}}
	\rdg[wit={J10}]{vaśagāt}
	\rdg[wit={V19,C7,V15}]{vaśagā}
	\rdg[wit={N19}]{vaśanān}
	\rdg[wit={Jyo}]{vaśagaṃ}
	\rdg[wit={C6}]{pavanān}}%
\app{\lem[wit={ceteri},alt={bhrū}]{\skm{n }bhrū}
	\rdg[wit={V15}]{bhū}
	\rdg[wit={N23}]{tū}
	\rdg[wit={V3}]{bhṛṃ}
	\rdg[wit={J10}]{prāg}}randhranāḍī% raṃdhranā_n N23
\app{\lem[wit={ceteri},alt={gaṇān/gaṇāṃs}]{gaṇāṃ\skp{s}} % gaṇān* J7,P23; gaṇāṃs Gr3
	\rdg[wit={J10}]{gaṇāt}
	\rdg[wit={Jyo}]{gataṃ}}}-\\+}
\tl{
\pada{\app{\lem[wit={C6,N19,V15,J10,Jyo},alt={tat}]{\skm{s }ta\skp{t}}
	\rdg[wit={P11,V3,Gr2,Gr3a,V1}]{taṃ}}%
	t kāyaṃ kurute punar navataraṃ % kāryaṃ, navattaraṃ J10; °tara V3
\app{\lem[wit={C6,V3,J7,V19,K3}]{jīrṇa}
	\rdg[wit={C7,N19}]{jīrṇaṃ} % chiṃjīrṇaṃ N19
	\rdg[wit={J10,Jyo}]{chinna}
	\rdg[wit={V15}]{chinnaṃ}
	\rdg[wit={V1}]{kṛnta}
	\rdg[wit={N23}]{bhasma}}drumaskandhavat//}\marma
	\NotIn{N3}\myfn{In \getsiglum{Jyo} this verse is found after \ref{VuIII121} together with the next one and has no commentary.}
	\\!}
\end{tlg}

%\newpage
\begin{tlg}[hp03_115]
\tl{
\pada{kuṇḍalīṃ % lī Gr2,N19,V15,V3, °liṃ J10
cālayi\app{\lem[wit={J7,Gr3a,N19,Jyo},alt={°tvā tu}]{\skp{°}tvā tu} % cāla-i-tvā V19
	\rdg[wit={N23}]{°tvācca}
	\rdg[wit={C6,V15,J10}]{°tvātha}
	\rdg[wit={V3}]{°tvādhaḥ}
	\rdg[wit={V1},alt={\illeg}]{\skp{\illeg}}}}
\pada{\app{\lem[wit={K3,C7,V15,V1}]{kuryād bhastrīṃ}
	\rdg[wit={V3,V19,N19}]{kuryād bhastrī} % kuryā bhastri V3
	\rdg[wit={J10}]{kuryād bhastrāṃ}
	\rdg[wit={Gr2}]{bhasrī kuryād}
	\rdg[wit={C6,Jyo}]{bhastrāṃ kuryād}}
	viśeṣataḥ/}\\+}
\tl{
\pada{eva%m
\app{\lem[wit={J10,Jyo},alt={abhyasyato}]{\skm{m }abhyasyato}
	\rdg[wit={V3}]{abhyasyatā}
	\rdg[wit={C7,V15}]{abhyasato}
	\rdg[wit={C6,Gr2,V19,K3,N19}]{abhyāsato}
	\rdg[wit={V1}]{..\,..\,syat.}} nityaṃ}
\pada{\app{\lem[wit={ceteri}]{yaminaḥ śaṅkate yamaḥ}
	\rdg[wit={Jyo}]{yamino yamabhīḥ kutaḥ}}//} \NotIn{N3}\\!}
\end{tlg}

%\newpage
\begin{tlg}[hp03_116]
\tl{
\pada{\app{\lem[wit={V3,Gr2,K3,N19,V1},alt={tadābhyaset}]{tadābhyase\skp{t}}
	\rdg[wit={J10}]{tadābhyasyet}
	\rdg[wit={C6,V19,V15}]{tad abhyaset}
	\rdg[wit={C7}]{tam abhyaset}}%
\app{\lem[wit={ceteri},alt={sūryabhedam}]{\skm{t }sūryabheda\skp{m}} % °bhedan*m J10
	\rdg[wit={V15}]{sūryabhede}
	\rdg[wit={C7}]{sūryabījam}}}%
\pada{\app{\lem[wit={C6,V3,Gr2,K3,C7,V1,J10},alt={ujjāyīṃ}]{\skm{m }ujjāyīṃ} % °yī N23,V3
	\rdg[wit={N19}]{ujjāī}
	\rdg[wit={V15}]{ujjāyāṃ}
	\rdg[wit={V19}]{ujrākhyām}}
\app{\lem[wit={ceteri}]{cāpi}
	\rdg[wit={V15}]{vāpi}
	\rdg[wit={V1}]{[vā]\,..}
	\rdg[wit={V19}]{api}} śītalīm/}\\+} % śītalī N23,N19,V15; sītalī V3
\tl{
\pada{evam
abhyāsa\app{\lem[wit={ceteri}]{yuktasya}
	\rdg[wit={J10}]{yogena}}}
\pada{\app{\lem[wit={Gr2,Gr3a}]{yamas tu}
	\rdg[wit={V15}]{śramas tu}
	\rdg[wit={N19,V1}]{śamino}% C6pc
	\rdg[wit={C6,V3,J10}]{śamano}}
\app{\lem[wit={ceteri}]{yaminaḥ}
	\rdg[wit={V3}]{yaminaṃ}} kutaḥ//} % kva ca C6
	\NotIn{N3,Jyo}\\!}
\end{tlg}
\endgray

%\newpage
\begin{tlg}[hp03_117]
\tl{
\pada{muhūrtadvayaparyantaṃ} % mahūrtta V3; paryaṃta V15
\pada{\app{\lem[wit={Gr2,Gr3a}]{nirbharaṃ}
	\rdg[wit={N3,C6,V3,V15}]{nirbhayaś}
	\rdg[wit={N19}]{nirbhayaṃś}
	\rdg[wit={V1,J10,Jyo}]{nirbhayaṃ}} % +M1,G7 ##? ṃ unsichtbar V1
\app{\lem[wit={N3,C6,V3,Gr2,N19,V15,V1,Jyo}]{cālanād asau}
	\rdg[wit={Gr3a}]{calanād asau}
	\rdg[wit={J10}]{vā diśodiśa}}/}\\+}
\tl{
\pada{ūrdhva%m % ūrdham V3
\app{\lem[wit={ceteri},alt={ākṛṣyate}]{\skm{m }ākṛṣyate}
	\rdg[wit={V15}]{ākṛte}
	\rdg[wit={Gr3a},alt={\om}]{\skp{\om}}} kiṃcit}
\pada{\app{\lem[wit={C6,Gr2,V15}]{suṣumṇāgatakuṇḍalī}
	\rdg[wit={J5,N19}]{suṣumnā kuṇḍalīgatā}% +J5,M1,M3
	\rdg[wit={N3}]{suṣumnā kuṃḍalīgataḥ}% +G7
	\rdg[wit={P11}]{suṣumṇāṃ kuṃḍalīgatāṃ}
	\rdg[wit={V3}]{suṣumnāṃ kuṇḍalī gatā}
	\rdg[wit={Jyo}]{suṣumnāyāṃ samudgatā}
	\rdg[wit={J10}]{suṣumṇāyāḥ samuddhṛtaḥ}
	\rdg[wit={Gr3a,V1},alt={\om}]{\skp{\om}}}//} \lineom{cd}{Gr3a}\\!}
\end{tlg}

\newpage
\begin{tlg}[hp03_118]
\tl{
\pada{\app{\lem[wit={N3,C6,V3,Gr2,N19,V15,J10,Jyo}]{tena kuṇḍalinī tasyāḥ} % kuṃḍalanī N3, kuṇḍalīnī N23; tasyā V3
	\rdg[wit={V1},alt={\om}]{\skp{\om}}}}
\pada{suṣumṇāyāḥ % °yā N3,C6,V3
\app{\lem[wit={V3,Gr2,N19,V15,V1,J10}]{samuddhṛtā} % tāḥ N19, samudhṛtā J5
	\rdg[wit={N3}]{samudbhutā}
	\rdg[wit={P11,C6,Jyo}]{mukhaṃ dhruvam}}/}\\+} % = GŚ
\tl{
\pada{\app{\lem[wit={ceteri}]{jahāti}
	\rdg[wit={J10}]{na yāti}} tasmāt prāṇo'yaṃ}
\pada{suṣumṇāṃ vrajati % suṣumṇā C6
\app{\lem[wit={V15,V1,Jyo}]{svataḥ}
	\rdg[wit={N3,P11,V3,N19}]{svanaḥ} % svana V3
	\rdg[wit={C6,Gr2}]{svayam}
	\rdg[wit={J10}]{niścalaḥ}}//} \NotIn{Gr3a}\\!}
\end{tlg}

%\newpage
\begin{tlg}[hp03_119]
\tl{
\pada{\app{\lem[wit={N3,C6,V3,J7,N19,V15,V1,J10,Jyo},alt={tasmāt}]{tasmā\skp{t}}
	\rdg[wit={N23}]{kasmāt}}t saṃcālayen nityaṃ} % saṃcārayen N3ac
\pada{\app{\lem[wit={C6,V3},alt={śabdagarbhām}]{śabdagarbhā\skp{m}}% gabhāṃ C6
	\rdg[wit={N3}]{śabdagaṃdhām}
	\rdg[wit={Gr2,N19,V15}]{śambhugarbhām}
	\rdg[wit={Jyo}]{sukhasuptām}
	\rdg[wit={J10}]{suṣasuptām}
	\rdg[wit={V1},alt={\illeg}]{\skp{\illeg}}}\marma%
\app{\lem[wit={ceteri},alt={arundhatīm}]{\skm{m }arundhatīm} % °tī N23,N19,N3, ddhaṃtī V3
	\rdg[wit={C6}]{sarasvatīṃ}}/}\\+}
\tl{
\pada{\app{\lem[wit={N3,C6,V3,V15,Jyo}]{tasyāḥ}
	\rdg[wit={J10}]{tasyāṃ}
	\rdg[wit={Gr2,N19}]{yasyāḥ}
	\rdg[wit={V1}]{[ya]\,..}}
\app{\lem[wit={N3,C6,V3,Jyo}]{saṃcālanenaiva}% sacāla° N3
	\rdg[wit={Gr2,V15}]{saṃcālanenāśu}
	\rdg[wit={N19,J10}]{saṃcālayenāśu}
	\rdg[wit={V1}]{..\,..\,lanen.\,..}}}
\pada{yogī \app{\lem[wit={N3,C6,J7,N19,V15,V1,Jyo},alt={rogaiḥ/rogair}]{rogaiḥ}% raugaiḥ C6
%	\rdg[wit={J7,N19,V15}]{rogair}
	\rdg[wit={N23}]{[r]. .air}
	\rdg[wit={V3}]{rogoḥ}
	\rdg[wit={J10}]{rogāt}}
\app{\lem[wit={N3,C6,V3,V1,J10,Jyo}]{pramucyate}
	\rdg[wit={Gr2,N19,V15}]{vimucyate}}//} \NotIn{Gr3a}\\!}
\end{tlg}

\begin{tlg}[hp03_120]
\tl{
\pada{yena \app{\lem[wit={C6,V3,N23,V15,V1,J10,Jyo}]{saṃcālitā}
	\rdg[wit={N19}]{saṃcalitā}
	\rdg[wit={N3}]{saṃcalatā}
	\rdg[wit={J7}]{sa cālitā}} śaktiḥ} % śakti V3
\pada{sa yogī \app{\lem[wit={N3,C6,V3,N19,V15,V1,J10,Jyo}]{siddhi}
	\rdg[wit={Gr2}]{mukti}}% .. ddhi V1 ##
\app{\lem[wit={ceteri}]{bhājanam}
	\rdg[wit={C6}]{bhājanaḥ}
	\rdg[wit={V1}]{..\,janaḥ}}/}\\+}
\tl{
\pada{kim atra bahunoktena}
\pada{kālaṃ \app{\lem[wit={ceteri}]{jayati}
	\rdg[wit={J10}]{vrajati}} līlayā//} \NotIn{Gr3a}\\!} % jayalati? V1
\end{tlg}


%\newpage
\startgray
\begin{tlg}[hp03_121]
\tl{
\pada{\app{\lem[wit={Gr3a,V1,Jyo}]{brahmacaryaratasyaiva}
	\rdg[wit={J7}]{brahmacarye ca tasyaiva}
	\rdg[wit={N23}]{brahmacatasyaiva}
	\rdg[wit={N19}]{brahmacaryarataś caiva}
	\rdg[wit={V3,V15}]{brahmacaryavratasyaiva}
	\rdg[wit={C6}]{brahmacaryavrataṃ}
	\rdg[wit={J10}]{brahmadharmaratasyaiva}}}
\pada{nityaṃ
\app{\lem[wit={J7,C7,Jyo}]{hitamitāśinaḥ}% °śanaḥ C7ac
	\rdg[wit={V3,N23,V19,N19}]{hitamitāśanaḥ} % °sanaḥ V19
	\rdg[wit={C6}]{hitamitāśanaṃ}
	\rdg[wit={V15}]{hitamitāśanaiḥ}
	\rdg[wit={K3,J10},post=\texteng{(°śanaḥ \getsiglum{K3}\postcorr)}]{mitahitāśinaḥ}
	\rdg[wit={V1},alt={\illeg}]{\skp{\illeg}}}/}\\+}
\tl{
\pada{\app{\lem[wit={C6,J7,K3,C7,N19,V15,Jyo}]{maṇḍalād}% written often maṇḍalāt*
	\rdg[wit={V3,N23,J10}]{maṃḍalā}
	\rdg[wit={V19}]{maṃḍalī}
	\rdg[wit={V1},alt={\illeg}]{\skp{\illeg}}} dṛśyate siddhiḥ} % siddhiṃ N19; siddhi V3
\pada{\app{\lem[wit={C6,J7,V15,Jyo}]{kuṇḍalya}
	\rdg[wit={V3,V19,C7,N19,J10}]{kuṇḍalyā}
	\rdg[wit={K3}]{kuṇḍalā}
	\rdg[wit={N23}]{kuṇḍali}
	\rdg[wit={V1},alt={\illeg}]{\skp{\illeg}}}bhyāsa%
\app{\lem[wit={C6,Gr2,Gr3a,V1,J10}]{yogataḥ}% V1 uncertain
	\rdg[wit={V3,N19,V15,Jyo}]{yoginaḥ}}//}\label{VuIII121}
	\NotIn{N3}\\!}
\end{tlg}
\endgray


%\newpage
\begin{tlg}[hp03_122]
\tl{
\pada{\app{\lem[wit={N3,C6,V3,Gr2,N19,V15,V1}]{abhyāsa}
	\rdg[wit={Jyo}]{abhyāsān}
	\rdg[wit={J10}]{abhyāsā}}%
\app{\lem[wit={C6,V3,V15,Jyo}]{niḥsṛtāṃ}
	\rdg[wit={V1}]{niḥsṛtā}
	\rdg[wit={J10}]{niḥśritāṃ}
	\rdg[wit={N3}]{nisṛtā}
	\rdg[wit={N19}]{nibhṛtāṃ}
	\rdg[wit={Gr2}]{sahitaṃ}}
\app{\lem[wit={N3,C6,N19,J10,Jyo}]{cāndrīṃ}
	\rdg[wit={V3,V15,V1}]{cāndrī}
	\rdg[wit={Gr2}]{candraṃ}}}
\pada{vibhūtyā saha % vibhūbhyā N23
\app{\lem[wit={N19,V15,V1,J10,Jyo}]{miśrayet}
	\rdg[wit={C6,V3}]{miśritāṃ}
	\rdg[wit={N3}]{mīśritaṃ}% miśritaṃ J5
	\rdg[wit={N23}]{micchayet}
	\rdg[wit={J7}]{mūrchayet}}/}\\+}
\tl{
\pada{\app{\lem[wit={C6,N19,V15}]{taddhāraṇaṃ}
	\rdg[wit={Gr2}]{taddhāraṇā}
	\rdg[wit={V3}]{tadvāraṇaṃ}
	\rdg[wit={V1}]{tad[v/dh].\,..\,..}
	\rdg[wit={N3}]{tad dhārayed}% +J5
	\rdg[wit={J10}]{tāṃ dhārayed}
	\rdg[wit={Jyo}]{dhārayed}}
\app{\lem[wit={C6,V3,N19,V15}]{tūttamāṅge}
	\rdg[wit={J7}]{cottamāṅge}
	\rdg[wit={N23}]{cottamāṃga}
	\rdg[wit={N3,J10}]{uttamāṅge}% +J5
	\rdg[wit={Jyo}]{uttamāṅgeṣu}
	\rdg[wit={V1},alt={\illeg}]{\skp{\illeg}}}}
\pada{\app{\lem[wit={ceteri}]{divya}
	\rdg[wit={G4,C6,V3}]{dīrgha}}% +G7
\app{\lem[wit={N3,C6,V3,N19,V15,V1}]{dṛṣṭipradāyakam}% +J5
	\rdg[wit={Gr2}]{dṛṣṭipradāyinī}
	\rdg[wit={J10}]{dṛṣṭipradāyinīṃ}
	\rdg[wit={V1}]{dṛṣṭiḥ prajāyate}}//}\label{VuIII98} \NotIn{Gr3a,Jyo}\myfn{\getsiglum{Jyo} has this verse in the Vajrolī section, immediately after \ref{III96}.}\\!}
\end{tlg}


\newpage
\startgray
\begin{tlg}[hp03_123]
\tl{
\pada{\app{\lem[wit={N23,V19,C7,Jyo}]{dvā}
	\rdg[wit={C6,J7,K3,N19,V15,V1,J10}]{dvi}}saptatisahasrāṇāṃ} % sahasrānā V19
\pada{nāḍīnāṃ mala\app{\lem[wit={J10,Jyo}]{śodhane}% +M1
	\rdg[wit={C6,Gr2,Gr3a,N19,V15,V1}]{śodhanam}}/}\\+}
\tl{
\pada{\app{\lem[wit={Gr3a,V15,Jyo}]{kutaḥ}
	\rdg[wit={N19}]{kṛta}
	\rdg[wit={J7}]{gudaḥ}
	\rdg[wit={V1,J10}]{guda}
	\rdg[wit={C6}]{aṃtaḥ}
	\rdg[wit={N23},alt={\om}]{\skp{\om}}}
\app{\lem[wit={J7,Gr3a,V15,J10,Jyo}]{prakṣālanopāyaḥ}
	\rdg[wit={N19,V1},alt={°pāyaṃ}]{prakṣālanopāyaṃ}
	\rdg[wit={C6}]{prakṣālano vāyuḥ}
	\rdg[wit={N23},alt={\om}]{\skp{\om}}}}
\pada{\app{\lem[wit={J7}]{kuṇḍalyabhyāsato vinā}
	\rdg[wit={N23,Gr3a}]{kuṇḍalyābhyāsato vinā}
	\rdg[wit={C6,N19,V15}]{kuṇḍalyabhyāsanād ṛte}
	\rdg[wit={Jyo}]{kuṇḍalyabhyasanād ṛte}
	\rdg[wit={J10}]{kuṇḍalyabhyāsa iṣyate}
	\rdg[wit={V1}]{ku\,..\,..\,[bhyā]\,..\,[mā]\,..\,..}}//} \NotIn{N3,V3}\\!}
\end{tlg}
\endgray

%\newpage
\begin{postmula}[hp03_123]
iti śakticālanam/ \sgwit{N19,V15,V1,J10}
\end{postmula}

%\newpage
\begin{tlg}[hp03_124]
\tl{
\pada{iti mudrā
\app{\lem[wit={ceteri}]{daśa}
	\rdg[wit={N3}]{dabhā}
	\rdg[wit={Gr3a}]{nava}} proktā}
\pada{ādināthena śambhunā/}\\+}
\tl{
\pada{\app{\lem[wit={N3,K3,C7,Jyo}]{ekaikā tāsu}
	\rdg[wit={N19}]{ekaikatāsu}
	\rdg[wit={V19}]{ekaiva tāsu}
	\rdg[wit={Gr2},post={\unm}]{ekaikāpi su°}
	\rdg[wit={V15}]{karaṇe sarva}
	\rdg[wit={J10}]{kāraṇe sarva}
	\rdg[wit={C6,V3}]{kāraṇaṃ sarva}
	\rdg[wit={V1}]{ka\,..\,..\,sarva}}
\app{\lem[wit={N3,Gr2,Gr3a,N19,Jyo}]{yamināṃ}
	\rdg[wit={V15,V1,J10}]{siddhānām}
	\rdg[wit={C6,V3}]{siddhīnām}}}
\pada{\app{\lem[wit={N3,Gr2,Gr3a,N19,Jyo}]{mahāsiddhipradāyinī}
	\rdg[wit={C6,V3,V15,V1,J10},postwit=\texteng{(\getsiglum{V1} partly illegible)}]{ekaikāpi kṣamaiva sā}
%	\rdg[wit={V1}]{e[k]. .. [p]i [kṣamai] .. ..}
	}//}\label{III124}\myfn{Verse order of \getsiglum{Jyo}: \ref{III127} \rightarrow\ \ref{III128} \rightarrow\ \ref{III125} \rightarrow\ \ref{III126} \rightarrow\ \ref{III124}}\\!}
\end{tlg}

\begin{tlg}[hp03_125]
\tl{
\pada{rājayogaṃ vinā % °yoge N23, yoga N19
\app{\lem[wit={ceteri}]{pṛthvī}
	\rdg[wit={J10}]{pṛthvīṃ}
	\rdg[wit={V15}]{siddhī}
	\rdg[wit={N19}]{vṛddhir}}}
\pada{rājayogaṃ vinā % yoga N19
\app{\lem[wit={ceteri}]{niśā}
	\rdg[wit={J10}]{niśāṃ}
	\rdg[wit={N23}]{nyathā}}/}\\+}
\tl{
\pada{rājayogaṃ vinā mudrā} % yoga N19
\pada{vicitrāpi na
\app{\lem[wit={ceteri}]{rājate}
	\rdg[wit={C6,Jyo}]{śobhate}}//}\label{III125}\\!} % rojate N23
\end{tlg}

%\newpage
\begin{tlg}[hp03_126]
\tl{
\pada{\app{\lem[wit={ceteri}]{mārutasya vidhiṃ}% vidhi J7
	\rdg[wit={V15,V1,J10}]{mārutābhyasanaṃ}}
\app{\lem[wit={ceteri}]{sarvaṃ}% sarva J7
	\rdg[wit={C6}]{sarvāṃ}
	\rdg[wit={N3}]{sarve}
	\rdg[wit={K3,C7}]{siddhiṃ}
	\rdg[wit={J10}]{kiṃcin}}}
\pada{manoyuktaṃ
\app{\lem[wit={ceteri}]{samabhyaset}
	\rdg[wit={V1,J10}]{samācaret}}/}\\+} % yukta N23
\tl{
\pada{itaratra na kartavyā} % italantra na N23; ityatatra C6; kartavyaṃ N23
\pada{manovṛtti%r % manārvvartti N23
\app{\lem[wit={N3,C6,Gr2,V19,V15,J10,Jyo},alt={manīṣiṇā}]{\skm{r }manīṣiṇā}
	\rdg[wit={V3,K3,C7,N19}]{manīṣiṇām}
	\rdg[wit={V1}]{..\,[nī]\,..\,ṇ.}}//}\label{III126}\\!}
\end{tlg}

%\newpage
\begin{tlg}[hp03_127]
\tl{
\pada{\app{\lem[wit={N3,C6,V3,J7,Gr3a}]{khilāpi}% +P11,C6 >> Marmasthana
	\rdg[wit={N23}]{sthirāpi}
	\rdg[wit={N19,V15}]{calāpi}
	\rdg[wit={Jyo}]{iyaṃ tu}
	\rdg[wit={V1,J10}]{vināpi}}\marmas
\app{\lem[wit={ceteri}]{madhyamā}
	\rdg[wit={J10}]{madhyamāṃ}
	\rdg[wit={V1},alt={\illeg}]{\skp{\illeg}}}
\app{\lem[wit={ceteri}]{nāḍī}
	\rdg[wit={V1}]{..\,ḍīṃ}}}
\pada{dṛḍhābhyāsena % driḍhā° J10
\app{\lem[wit={ceteri}]{yoginām}
	\rdg[wit={C6}]{yoginā}
	\rdg[wit={V3}]{yoginaṃ}
	\rdg[wit={J10}]{yoginaḥ}}/}\\+}
\tl{
\pada{\app{\lem[wit={N3,C6,V3,V19,K3,J10,Jyo}]{āsana} % asana N3
	\rdg[wit={Gr2,C7,N19,V15,V1}]{āsanaṃ}}prāṇa%
\app{\lem[wit={N3,N23,N19,V15,V1,Jyo}]{saṃyāma}
	\rdg[wit={V3}]{saṃyama}
	\rdg[wit={C6,K3,C7}]{saṃyāmair}
	\rdg[wit={J7,V19}]{saṃyāmai}
	\rdg[wit={J10}]{saṃyamair}}}%
\pada{mudrābhiḥ % °bhi V3, °niḥ C6
\app{\lem[wit={ceteri}]{saralā}
	\rdg[wit={V19}]{na calā}
	\rdg[wit={V15}]{sabalā}
	\rdg[wit={N19}]{śavalā}} bhavet//}\label{III127}\\!}
\end{tlg}

\newpage
\begin{tlg}[hp03_128]
\tl{
\pada{\app{\lem[wit={N3}]{upāsane}% +J5
	\rdg[wit={Gr2}]{upāsanaṃ}
	\rdg[wit={V19,C7}]{upāsana}
	\rdg[wit={K3}]{tathāsana}
	\rdg[wit={V1}]{abhyāse\,..}
	\rdg[wit={C6,V3,V15}]{abhyāseṣu}% P11
	\rdg[wit={J10}]{abhyāsena}
	\rdg[wit={Jyo}]{abhyāse tu}}
\app{\lem[wit={ceteri}]{vinidrāṇāṃ} % °ṇā N3
	\rdg[wit={J10}]{hi mudrāṇāṃ}}}
\pada{\app{\lem[wit={Gr2,Gr3a}]{rājayogaḥ}
	\rdg[wit={N3,J5,G4}]{rājayoga}% +G4
	\rdg[wit={V1}]{anuddhṛta}
	\rdg[wit={V15}]{anuddhata}
	\rdg[wit={C6}]{anudbhūta}
	\rdg[wit={P11}]{anudruta}
	\rdg[wit={V3}]{manudṛta}
	\rdg[wit={Jyo}]{mano dhṛtvā}
	\rdg[wit={J10}]{tad udeti}}
\app{\lem[wit={J7}]{samudrakaḥ}
	\rdg[wit={N3,G4}]{samudravat}
	\rdg[wit={J5}]{samudbhavān}
	\rdg[wit={N23}]{samūcakaḥ}
	\rdg[wit={V19}]{samāhnakaḥ}
	\rdg[wit={C7}]{samahnakaḥ}
	\rdg[wit={K3}]{samāhakaḥ}
	\rdg[wit={C6,V15,V1}]{samādhināṃ}
	\rdg[wit={J10,Jyo}]{samādhinā}
	\rdg[wit={P11,V3}]{samādhiṣu}}/}\marma\\+}
\tl{
\pada{\app{\lem[wit={N3,C6,V3,Gr2,Gr3a,V15,Jyo}]{rudrāṇī}
	\rdg[wit={V1,J10}]{mudrāṇāṃ}}
\app{\lem[wit={Gr2,Gr3a}]{sā}
	\rdg[wit={N3,C6,V3,V15,V1,J10}]{cā}
	\rdg[wit={Jyo}]{vā}} parā mudrā}
\pada{\app{\lem[wit={ceteri}]{bhadrāṃ} % bhadrā V3, mudrā(ṃ) P11,C6
	\rdg[wit={N23}]{bhavāṃ}
	\rdg[wit={N3}]{sadā}} siddhiṃ % siddhi V19,N3,J10
\app{\lem[wit={ceteri}]{prayacchati}
	\rdg[wit={V19}]{prayakṣati}}//}\label{III128} \NotIn{N19}\\!}
\end{tlg}


\startgray
\begin{tlg}[hp03_129]
\tl{
\pada{\app{\lem[wit={ceteri}]{upadeśaṃ}% °dehaṃ C6
	\rdg[wit={V1}]{upadeśe}
	\rdg[wit={N19}]{upadeśo}} hi mudrāṇāṃ}
\pada{yo \app{\lem[wit={C6,V3,J7,Gr3a,N19,J10}]{dhatte}% dadyāt J7pc
	\rdg[wit={V15,Jyo}]{datte}
	\rdg[wit={N23}]{dartte}
	\rdg[wit={V1}]{..\,[tte]}}
\app{\lem[wit={P11,V3,Gr3a,V1,J10,Jyo}]{sāṃpradāyikam}
	\rdg[wit={V15},alt={°yikāṃ}]{sāṃpradāyikāṃ}
	\rdg[wit={Gr2},alt={°yikaḥ}]{sāṃpradāyikaḥ}
	\rdg[wit={N19},alt={°yakaṃ}]{sāṃpradāyakaṃ}
	\rdg[wit={C6}]{sāṃpradāyakaḥ}}/}\\+}
\tl{
\pada{sa \app{\lem[wit={P11,J7,Gr3a,N19,V1}]{evāstu}
	\rdg[wit={V3}]{evastu}
	\rdg[wit={V15,J10,Jyo}]{eva śrī}
	\rdg[wit={N23}]{evavāca}
	\rdg[wit={C6}]{vāstava}} guruḥ svāmī} % guru N23,V1,J10, <gu>ruḥ V19
\pada{sākṣādīśvara % sākhyād N23, sakṣād V19; eṣa N19
\app{\lem[wit={ceteri}]{eva}
	\rdg[wit={N19}]{eṣa}}
\app{\lem[wit={ceteri}]{saḥ} % sa V3
	\rdg[wit={N23}]{ca}}//} \NotIn{N3}\\!}
\end{tlg}

\begin{tlg}[hp03_130]
\tl{
\pada{tasya vākyaparo % °parā N23
\app{\lem[wit={Gr2,Gr3a,V15,Jyo}]{bhūtvā}
	\rdg[wit={C6,V3,N19,J10}]{nityaṃ}}} % nitya V3; mudrāṃ M1,M3,G7
\pada{\app{\lem[wit={Gr3a,V15}]{yo'bhyasyati}% +P11,M1,M3
	\rdg[wit={C6}]{yo bhyasati}
	\rdg[wit={N23}]{yo bhyaset su°}
	\rdg[wit={J7}]{yo bhyaseta}
	\rdg[wit={N19}]{yo bhyasena}% yobhyāsena G7
	\rdg[wit={V3}]{yomabhyaset}
	\rdg[wit={J10}]{athābhyāsa}
	\rdg[wit={Jyo}]{mudrābhyāse}} samāhitaḥ/}\\+} % °hita V3
\tl{
\pada{aṇimādi\app{\lem[wit={ceteri}]{guṇaiśvaryaṃ}
	\rdg[wit={V15,Jyo}]{guṇaiḥ sārdhaṃ}}} % guṇai V15
\pada{\app{\lem[wit={ceteri}]{jāyate}% jayate G7
	\rdg[wit={J10,Jyo}]{labhate}}
kāla\app{\lem[wit={Gr3a,Jyo}]{vañcanam}% M1,M3
	\rdg[wit={Gr2}]{vañcanāt}
	\rdg[wit={C6,V3,N19,V15,J10}]{vañcakaḥ}}//} % ## G7
	\NotIn{N3,V1}\\!}
\end{tlg}
\endgray

%\newpage
\begin{col}[hp03_col]
\app{\lem[wit={N23,V1}]{iti svātmārāma}
	\rdg[wit={V3}]{iti śrīsvātmārāma}
	\rdg[wit={N3}]{ti śrīsadgurusvātmārāma}
	\rdg[wit={J10}]{ity ātmārāma}
	\rdg[wit={J7,N19,V15},post=\texteng{(ciṃtāmaṇinā \getsiglum{V15})}]{iti śrīsahajānaṃdasaṃtānaciṃtāmaṇisvātmārāma}
	\rdg[wit={C6,Gr3a}]{iti}}%
\app{\lem[wit={V3,Gr2,J10}]{yogīndra} % yogiṃdra V3
	\rdg[wit={N3}]{yogeṃdra}
	\rdg[wit={N19,V15,V1}]{yoginā}
	\rdg[wit={C6,Gr3a},alt={\om}]{\skp{\om}}}%
\app{\lem[wit={N3,V3,Gr2,N19,V15,V1,J10}]{viracitāyāṃ}
	\rdg[wit={C6,Gr3a},alt={\om}]{\skp{\om}}}
\app{\lem[wit={N3,V3,J7,C7,N19,V15,V1,J10}]{haṭhapradīpikāyāṃ}
	\rdg[wit={C6,K3}]{śrīhaṭhapradīpikāyāṃ}
	\rdg[wit={V19}]{haṭhayogavidyāyāṃ}
	\rdg[wit={N23},alt={\om}]{\skp{\om}}}
\app{\lem[alt={\ante tṛtīyo° \add},nosep]{}
	\rdg[wit={V15}]{mudrāvidhānaṃ}}%
\app{\lem[wit={N3,C6,V3,J7,N19,V15}]{tṛtīyopadeśaḥ} % tṛtiyo V3
	\rdg[wit={V19}]{tṛtīya upadeśaḥ}
	\rdg[wit={K3,C7}]{tṛtīyoyam upadeśaḥ}
	\rdg[wit={V1,J10}]{tṛtīyo dhyāyaḥ}
	\rdg[wit={N23},postwit=\texteng{(cf. fn. \footref{N23ch3end})}]{caturthopadeśa}}// 3 //
\end{col}

\end{ekdosis}
\end{otherlanguage}
\newpage

\bigskip
\bigskip
%\section*{List of sigla}

% N23,J7, V19,K3,C7, P15(up to 13a),N19,V15, V1,N3,V3,J10,Jyo
% N26,N9,J6 (for the Khecaryabhyāsakrama only)

\small
\begin{tabular}{lllp{8cm}}
\multicolumn{4}{l}{\textbf{List of Sigla}} \\
\\
\getsiglum{N3} & N3 & Gr1 & one folio is missing (\ref{VuIII88}--\ref{VuIII116}a)
[\getsiglum{Gr1r} is a reconstruction of Gr1 from the other mss of the group for the missing part of \getsiglum{N3}]\\
\getsiglum{J5} & J5 & Gr1 & consulted sporadically\\
\getsiglum{G4} & G4 & Gr1 & consulted sporadically\\
\getsiglum{P11} & P11 & Gr4b & consulted only when the reading of \getsiglum{C6} is unusual from the stemmatic point of view\\
\getsiglum{C6} & C6 & Gr4b & contaminated with Gr3?\\
\getsiglum{V3} & V3 & Gr6b\\
\getsiglum{N23} & N23 & Gr2\\
\getsiglum{J7} & J7 & Gr2\\
\getsiglum{V19} & V19 & Gr3\\
\getsiglum{K3} & K3 & Gr3 & the Vajrolī section is lost\\
\getsiglum{C7} & C7 & Gr3 & one folio is missing (3.11d--3.19c)\\
\getsiglum{J6} & J6 & Gr6a & collated only for 3.32*1--33*19\\
\getsiglum{P15} & P15 & Gr4c & lost after 3.13a\\
\getsiglum{N19} & N19 & Gr4c\\
\getsiglum{V15} & V15 & Gr4c & 3.49c--3.67 omitted; contaminated with Gr3?\\
\getsiglum{J11} & J11 & Gr4c & collated only for 3.49--66 as substitute for \getsiglum{V15}\\
\getsiglum{N26} & N26 & Gr6c & collated only for 3.32*1--33*19\\
\getsiglum{V1} & V1 & Gr4d & \\
\getsiglum{J10} & J10 & Gr4d\\
\getsiglum{N9} & N9 & Gr6d & collated only for 3.32*1--33*19\\
\getsiglum{Jyo} & Jyo & Gr4a &  Brahmānanda's version, based on the edition 1972 \\
\end{tabular}

\end{document}


